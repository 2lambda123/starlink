\begin{htmlonly}
\documentclass[dvips,a4paper]{article}
\usepackage{html,htmllist,makeidx}

%this font is used in  features.tex
\font\wncyr = wncyr10


\begin{htmlonly}
 \usepackage[dvips]{graphicx}
 \usepackage[dvips]{color}
 \usepackage[dvips,leftbars]{changebar}
 \newcommand{\FoilTeX}{\env{FoilTeX}}
%
 \def\url#1{\htmladdnormallink{#1}{#1}}
 \def\Email#1{\htmladdnormallink{<#1>}{mailto:#1}}
 \def\path#1{\texttt{#1}}
% \def\urldef#1#2#3{\def#1{#2{#3}}}
 \def\onlinedoc{\url{http://www-dsed.llnl.gov/files/programs/unix/latex2html/manual/}}
 \def\patches{\url{http://www-dsed.llnl.gov/files/programs/unix/latex2html/}}
 \def\sourceA{%
  \url{http://www-dsed.llnl.gov/files/programs/unix/latex2html/sources/}}
% \def\sourceB{\url{ftp://ftp.mpn.com/pub/nikos/latex2html-98.1.tar.gz}}
 \def\sourceB{\CTANtug{\CTANA}}
 \def\sourceC{\url{http://ftp.rzg.mpg.de/pub/software/latex2html/sources/}}
 \def\CVSsite{\url{http://cdc-server.cdc.informatik.tu-darmstadt.de/\~{}latex2html/}}
 \def\CVSrepos{\url{http://cdc-server.cdc.informatik.tu-darmstadt.de/\~{}latex2html/user/}}
 \def\CVSlatest{\url{http://cdc-server.cdc.informatik.tu-darmstadt.de/\~{}latex2html/l2h-latest.tar.gz}}
 \newcommand{\CTANtug}[1]{\path{http://ctan.tug.org/ctan/#1}}
 \def\CTANA{tex-archive/support/latex2html}
 \def\tugURL{\url{http://www.tug.org}}
 \def\danteURL{\url{http://www.dante.de}}
 \def\ListURL{\url{http://www.rosat.mpe-garching.mpg.de/mailing-lists/LaTeX2HTML/}}
%
 \def\glossary#1{\index{#1@\texttt{#1} \label{III#1}\htmlref{(G)}{GGG#1}}}
 \def\Glossary#1#2{\index{#1@{#2} \label{III#1}\htmlref{(G)}{GGG#1}}}
%
% hack to suppress changebar entries read from .aux file  !!! bug in changebar.perl !!! 
 \makeatletter
 \def\cb@barpoint#1#2#3{}
 \makeatother
%
\newcommand{\sameas}[1]{\textcolor{red}{Same as setting: #1}}
\newcommand{\onlinedocRM}{\url{http://www-math.mpce.mq.edu.au/\~{}ross/latex/manual/manual.html}}
\newcommand{\EXcolors}{\url{http://www-math.mpce.mq.edu.au/\~{}ross/latex/crayola/crayola.html}}
%
\newcommand{\Lc}[1]{\texttt{\char92#1}} % LaTeX command
\newcommand{\Tc}[1]{\texttt{\char92#1}} % TeX command
\newcommand{\Cs}[1]{ \texttt{-#1} }   % command-line switch
\newcommand{\Ve}[1]{\index{#1@\texttt{#1}}\texttt{#1}} % version extension
\newcommand{\gsl}[1]{#1}
\newcommand{\indexentry}[2]{\item #1 #2}

%
\internal{}%
\internal{O}%
\internal{S}%
\internal{E}%
\internal{M}%
\internal{H}%
\internal{P}%

\newcommand{\texdev}[1]{\htmladdnormallink
 {\texdevURL/#1}{\texdevURL/#1}}
\newcommand{\ctan}{http://ctan.tug.org/ctan}
\newcommand{\ctanTUG}{\htmladdnormallinkfoot
 {TUG's searchable CTAN site}{\ctan}}
\newcommand{\ctanURL}[1]{\htmladdnormallink
 {CTAN:\texttt{.../#1}}{\ctan/tex-archive/#1}}
\newcommand{\ctanURLbr}[1]{\htmladdnormallink
 {CTAN:\texttt{.../#1}}{\ctan/tex-archive/#1}}
\newcommand{\ctanTUGurl}[2]{\ctanURL{#2}}
\newcommand{\indichtml}{l2h/indic/IndicHTML/}

\end{htmlonly}

\newcommand{\texdevURL}{http://www-texdev.mpce.mq.edu.au/}
\newcommand{\Unicode}{\htmladdnormallinkfoot
 {Unicode}{http://www.unicode.org/}}
\newcommand{\IndicHTML}{\htmladdnormallinkfoot
 {Indic\TeX/HTML}{\texdevURL{\indichtml}}}

%begin{latexonly}
\usepackage{array}

\newcommand{\Cs}[1]{{\upshape`\,\texttt{-#1}\,'}}
\newcommand{\Ve}[1]{\index{#1@\texttt{#1}}{\upshape`\,\texttt{#1}\,'}}
\newcommand{\Lc}[1]{{\upshape\ttfamily\char92#1}}
\newcommand{\Tc}[1]{{\upshape\ttfamily\char92#1}}

\newcommand{\ctanTUG}[1]{TUG's searchable CTAN site\footnote{\ctanTUGurl}}
\newcommand{\ctanURL}{CTAN: \penalty-200\ctanurl}
\newcommand{\ctanURLbr}{CTAN: \newline\ctanurl}
\urldef\texdev\url{http://www-texdev.mpce.mq.edu.au/}
\urldef\ctanurl\url{.../tex-archive/}
\urldef\ctanTUGurl\url{http://ctan.tug.org/ctan/tex-archive/}
\urldef\indichtml\url{l2h/indic/IndicHTML/}

\def\mathsmiley{\smiley}
%end{latexonly}

% thanks to \KrisRose for this macro:
\def\smiley{\hbox{\rlap{$\bigcirc$}\kern1.3pt$\scriptstyle\ddot\smile$}}

\begin{imagesonly}
\def\mathsmiley{\smiley}
\end{imagesonly}


\newcommand{\Lcs}[1]{{\upshape\ttfamily\char92#1}}

\renewcommand{\thefootnote}{\arabic{footnote}}

%\newcommand{\godown}[1]{{\htmlref
% {\htmladdimg[left BORDER=0]{../psfiles/dn.gif}}{#1}}}
\newcommand{\godown}[1]{}

%\newcommand{\goback}[1]{{\htmlref
% {\htmladdimg[left BORDER=0]{../psfiles/up.gif}}{#1}}}
\newcommand{\goback}[1]{}


\newcommand{\WiiiC}{\htmladdnormallinkfoot{World Wide Web Consortium}{http://www.w3c.org/}}
\newcommand{\latextohtmlNG}{\latextohtml-NG}
\newcommand{\maillist}{\htmladdnormallinkfoot{\latextohtml{} mailing list}%
{http://cbl.leeds.ac.uk/nikos/tex2html/doc/mail/mail.html}}
%
\newcommand{\HTMLiii}{\textup{\texttt{HTML} 3.2}}%
\newcommand{\HTMLIII}{\htmladdnormallink{\HTMLiii}%
{http://www.w3.org/TR/REC-html32.html}}%

\newcommand{\HTMLiv}{\textup{\texttt{HTML} 4.0}}%
\newcommand{\HTMLIV}{\htmladdnormallink{\HTMLiv}%
{http://www.w3.org/TR/REC-html40/}}%

%\newcommand{\latextohtml}{\textup{\LaTeX 2{\ttfamily HTML}}}%
\newcommand{\Perl}{\htmlref{\textsl{Perl}}{GGGPerl}}%  
\newcommand{\PS}{\htmlref{\textup{Post\-Script}}{GGGpostscript}\Glossary{PostScript}{PostScript}{}}%
\newcommand{\MF}{\htmlref{\textsl{Metafont}}{GGGMetafont}\Glossary{Metafont}{\textsl{Metafont}}{}}%
\newcommand{\fn}[1]{\htmlref{\texttt{#1}}{GGG#1}\glossary{#1}}%  file names, with link to glossary
\newcommand{\gn}[1]{\texttt{#1}\label{GGG#1}\htmlref{\^{}}{III#1}}%  file names, labelled within glossary
%\newcommand{\appl}[1]{\htmlref{\textsl{#1}}{GGG#1}\Glossary{#1}{\textsl{#1}}{}}%  application software names
\newcommand{\appl}[1]{\htmlref{\textsl{#1}}{GGG#1}\Glossary{#1}{\gsl{#1}}{}}%  application software names
%
\newcommand{\env}[1]{{\upshape\sffamily #1}}%  LaTeX environment and package names
\newcommand{\HTMLtag}[1]{\path{<#1>}}%  HTML tag
\newcommand{\Meta}[1]{\texttt{\upshape<\textit{#1}>}}%  Meta string

%
% developer names and addresses:
%
\newcommand{\NikosDrakos}{\index{Nikos Drakos}%\Email{nikos@cbl.leeds.ac.uk}
\htmladdnormallink{Nikos Drakos}{http://www.cbl.leeds.ac.uk/nikos/personal.html}}
%
\newcommand{\RossMoore}{\index{Ross Moore}%\Email{ross@mpce.mq.edu.au}
\htmladdnormallink{Ross Moore}{http://www.mpce.mq.edu.au/\~{}ross/}}
\newcommand{\Macquarie}{\htmladdnormallink
{Macquarie University}{http://www.mq.edu.au/}}
%
\newcommand{\Hennecke}{\index{Marcus Hennecke}%\Email{hennecke@dbag.ulm.DaimlerBenz.COM}
\htmladdnormallink{Marcus Hennecke}{http://www.crc.ricoh.com/\~{}marcush/}}
%
\newcommand{\Noworolski}{\index{Mark Noworolski}%\Email{jmn@eecs.berkeley.edu}
\htmladdnormallink{Mark Noworolski}{http://www-power.eecs.berkeley.edu/\~{}jmn/}}
%
\newcommand{\Isani}{\index{Sidik Isani}%\Email{isani@cfht.hawaii.edu}
\htmladdnormallink{Sidik Isani}{http://www.cfht.hawaii.edu/\~{}isani/si.html}}
%
\newcommand{\Goossens}{\index{Michel Goossens}%\Email{goossens@cern.ch}
\htmladdnormallink{Michel Goossens}{http://wwwcn1.cern.ch/\~{}goossens/}}
%
\newcommand{\Wilck}{\index{Martin Wilck}%\Email{martin@tropos.de}
\htmladdnormallink{Martin Wilck}{http://www.tropos.de/personal/wilck.html}}
%
\newcommand{\PatrickDaly}{\index{Patrick Daly}%\Email{daly@linmpi.dnet.gwdg.de}
\htmladdnormallink{Patrick Daly}{mailto:daly@linmpi.dnet.gwdg.de}}
%
\newcommand{\HerbSwan}{\index{Herb Swan}%\Email{herb.swan@perc.Arco.COM}
%\htmladdnormallink{Herb Swan}{mailto:herb.swan@perc.Arco.COM}}
\htmladdnormallink{Herb Swan}{mailto:lanhws@expl.aai.arco.com}}
%
\newcommand{\Lippmann}{\index{Jens Lippmann}%\Email{lippmann@cdc.informatik.tu-darmstadt.de}
\htmladdnormallink{Jens Lippmann}{http://www-jb.cs.uni-sb.de/\~{}www/people/lippmann}}
%
\newcommand{\Rouchal}{\index{Marek Rouchal}%\Email{marek@hl.siemens.de}
\htmladdnormallink{Marek Rouchal}{mailto:marek@hl.siemens.de}}
%
\newcommand{\Bohnet}{\index{Achim Bohnet}%\Email{ach@rosat.mpe-garching.mpg.de}
\htmladdnormallink{Achim Bohnet}{mailto:ach@rosat.mpe-garching.mpg.de}}
%
\newcommand{\Nelson}{\index{Scott Nelson}%\Email{nelson18@llnl.gov}
\htmladdnormallink{Scott Nelson}{mailto:nelson18@llnl.gov}}
%
\newcommand{\AxelRamge}{\index{Axel Ramge}%\Email{axel@ramge.de}
\htmladdnormallink{Axel Ramge}{mailto:axel@ramge.de}}
%
\newcommand{\Popineau}{\index{Fabrice Popineau}%\Email{popineau@esemetz.ese-metz.fr}
\htmladdnormallink{Fabrice Popineau}{mailto:popineau@esemetz.ese-metz.fr}}
%
\newcommand{\Wortmann}{\index{Uli Wortmann}%\Email{uli12@bonk.ethz.ch}
\htmladdnormallink{Uli Wortmann}{mailto:uli12@bonk.ethz.ch}}
%
\newcommand{\Veytsman}{\index{Boris Veytsman}%\Email{boris@plmsc.psu.edu}
\htmladdnormallink{Boris Veytsman}{mailto:boris@plmsc.psu.edu}}
%
\newcommand{\Taupin}{\index{Daniel Taupin}%\Email{taupin@lps.u-psud.fr}
\htmladdnormallink{Daniel Taupin}{mailto:taupin@lps.u-psud.fr}}
%
%\endinput

%
% (La)TeX related URLs
%
\newcommand{\CSEP}{\index{Computer Science Education Project!CSEP}%
\htmladdnormallinkfoot{Computer Science Education Project}%
{http://csep1.phy.ornl.gov/csep.html}} 
%
\newcommand{\CBLU}{\index{Computer Based Learning Unit!CBLU}%
\htmladdnormallinkfoot{Computer Based Learning Unit}%
{http://cbl.leeds.ac.uk/\~{}www/home.html}}
%
\newcommand{\CERN}{\index{CERN}%
\htmladdnormallink{CERN}{http://wwwcn.cern.ch/Welcome.html}} 
%
\newcommand{\KrisRose}{\index{Kristoffer Rose}%\Email{krisrose@brics.dk}
\htmladdnormallink{Kristoffer Rose}{http://www.brics.dk/\~{}krisrose/}}
%
\newcommand{\XypicDK}{\index{Xy-pic@\protect\Xy-pic graphics package}%
\index{Xy-pic@\protect\Xy-pic graphics package!home page}%
\htmladdnormallink{Xy-pic}{http://www.brics.dk/\~{}krisrose/Xy-pic.html}}
\newcommand{\XypicAUS}{\index{Xy-pic@\protect\Xy-pic graphics package!home page, down under}%
\htmladdnormallink{Ross Moore}{http://www.mpce.mq.edu.au/\~{}ross/Xy-pic.html}}
%
\newcommand{\LiPS}{\index{LiPS Design Team}%
\htmladdnormallink{LiPS Design Team}{http://www-jb.cs.uni-sb.de/LiPS/node2.html}}
\newcommand{\FIDarmstadt}{\index{Fachbereich Informatik, Darmstadt}%
\htmladdnormallink{Fachbereich Informatik}{http://www.informatik.tu-darmstadt.de/}}
\newcommand{\Darmstadt}{\index{Darmstadt!Fachbereich Informatik}%
\htmladdnormallink{Darmstadt}{http://www.tu-darmstadt.de/Welcome.de.html}}
%
\newcommand{\DANTE}{\index{DANTE}%
\htmladdnormallink{DANTE e.V.}{http://www.dante.de/}}
\newcommand{\Praesidium}{\index{DANTE!Praesidium}%
\htmladdnormallink{Praesidium}{http://www.dante.de/dante/Organe.html}}
\newcommand{\LaTeXiii}{\index{LaTeX@\LaTeX!LaTeX3@\LaTeX{3}}%
\htmladdnormallink{\LaTeX{3}}{http://www.tex.ac.uk/CTAN/latex/latex3}}
%
\newcommand{\Engberg}{\index{Uffe Engberg}%
\htmladdnormallinkfoot{Uffe Engberg}{http://www.brics.dk/\~{}engberg}} 

\newcommand{\texdevINDIC}{\texdev\indic}
%\endinput


%% earlier contributions from...
\newcommand{\AndrewCole}{\index{Computer Based Learning Unit!Andrew Cole}%
\htmladdnormallink{Andrew Cole}{http://www.cbl.leeds.ac.uk/ajcole/personal.html}}
%
\newcommand{\AnaPaiva}{\index{Computer Based Learning Unit!Ana Maria Paiva}%
\htmladdnormallink{Ana Maria Paiva}{http://www.cbl.leeds.ac.uk/amp/personal.html}}
%
\newcommand{\RodWilliams}{\index{Computer Based Learning Unit!Roderick Williams}%
\htmladdnormallink{Roderick Williams}{http://www.cbl.leeds.ac.uk/rodw/personal.html}}
%
\newcommand{\JamilSawar}{\index{Computer Based Learning Unit!Jamil Sawar}%
\htmladdnormallink{Jamil Sawar}{http://www.cbl.leeds.ac.uk/sawar/personal.html}}

\newcommand{\RobertThau}{\index{Robert S. Thau}%
\htmladdnormallink{Robert S. Thau}{mailto: rst@edu.mit.ai}}

\endinput




%
\newcommand{\RobertThau}{\index{Robert S. Thau}%
\htmladdnormallink{Robert S. Thau}{mailto: rst@edu.mit.ai}}


%
% Ian Foster \Email{itf@mcs.anl.gov}
% Bob Olson \Email{olson@mcs.anl.gov}
% Verena Umar \Email{verena@edu.vanderbilt.cas.compsci}
% Axel Belinfante \Email{Axel.Belinfante@cs.utwente.nl}
% Todd Little \Email{little@com.dec.enet.nuts2u}
% Franz Vojik \Email{vojik@de.tu-muenchen.informatik}
% Eric Carroll \Email{eric@ca.utoronto.utcc.enfm}
% Roderick Williams \Email{rodw@cbl.leeds.ac.uk}
% Robert Cailliau \Email{cailliau@cernnext.cern.ch}
% Toni Lantunen at CERN
% Ana Maria Paiva, Jamil Sawar, Andrew Cole at CBLU Leeds
% Phillip Conrad (Perfect Byte, Inc.)
% L. Peter Deutsch (

%
\begin{htmlonly}
\documentclass[dvips,a4paper]{article}
\usepackage{html,htmllist,makeidx}

%this font is used in  features.tex
\font\wncyr = wncyr10


\begin{htmlonly}
 \usepackage[dvips]{graphicx}
 \usepackage[dvips]{color}
 \usepackage[dvips,leftbars]{changebar}
 \newcommand{\FoilTeX}{\env{FoilTeX}}
%
 \def\url#1{\htmladdnormallink{#1}{#1}}
 \def\Email#1{\htmladdnormallink{<#1>}{mailto:#1}}
 \def\path#1{\texttt{#1}}
% \def\urldef#1#2#3{\def#1{#2{#3}}}
 \def\onlinedoc{\url{http://www-dsed.llnl.gov/files/programs/unix/latex2html/manual/}}
 \def\patches{\url{http://www-dsed.llnl.gov/files/programs/unix/latex2html/}}
 \def\sourceA{%
  \url{http://www-dsed.llnl.gov/files/programs/unix/latex2html/sources/}}
% \def\sourceB{\url{ftp://ftp.mpn.com/pub/nikos/latex2html-98.1.tar.gz}}
 \def\sourceB{\CTANtug{\CTANA}}
 \def\sourceC{\url{http://ftp.rzg.mpg.de/pub/software/latex2html/sources/}}
 \def\CVSsite{\url{http://cdc-server.cdc.informatik.tu-darmstadt.de/\~{}latex2html/}}
 \def\CVSrepos{\url{http://cdc-server.cdc.informatik.tu-darmstadt.de/\~{}latex2html/user/}}
 \def\CVSlatest{\url{http://cdc-server.cdc.informatik.tu-darmstadt.de/\~{}latex2html/l2h-latest.tar.gz}}
 \newcommand{\CTANtug}[1]{\path{http://ctan.tug.org/ctan/#1}}
 \def\CTANA{tex-archive/support/latex2html}
 \def\tugURL{\url{http://www.tug.org}}
 \def\danteURL{\url{http://www.dante.de}}
 \def\ListURL{\url{http://www.rosat.mpe-garching.mpg.de/mailing-lists/LaTeX2HTML/}}
%
 \def\glossary#1{\index{#1@\texttt{#1} \label{III#1}\htmlref{(G)}{GGG#1}}}
 \def\Glossary#1#2{\index{#1@{#2} \label{III#1}\htmlref{(G)}{GGG#1}}}
%
% hack to suppress changebar entries read from .aux file  !!! bug in changebar.perl !!! 
 \makeatletter
 \def\cb@barpoint#1#2#3{}
 \makeatother
%
\newcommand{\sameas}[1]{\textcolor{red}{Same as setting: #1}}
\newcommand{\onlinedocRM}{\url{http://www-math.mpce.mq.edu.au/\~{}ross/latex/manual/manual.html}}
\newcommand{\EXcolors}{\url{http://www-math.mpce.mq.edu.au/\~{}ross/latex/crayola/crayola.html}}
%
\newcommand{\Lc}[1]{\texttt{\char92#1}} % LaTeX command
\newcommand{\Tc}[1]{\texttt{\char92#1}} % TeX command
\newcommand{\Cs}[1]{ \texttt{-#1} }   % command-line switch
\newcommand{\Ve}[1]{\index{#1@\texttt{#1}}\texttt{#1}} % version extension
\newcommand{\gsl}[1]{#1}
\newcommand{\indexentry}[2]{\item #1 #2}

%
\internal{}%
\internal{O}%
\internal{S}%
\internal{E}%
\internal{M}%
\internal{H}%
\internal{P}%

\newcommand{\texdev}[1]{\htmladdnormallink
 {\texdevURL/#1}{\texdevURL/#1}}
\newcommand{\ctan}{http://ctan.tug.org/ctan}
\newcommand{\ctanTUG}{\htmladdnormallinkfoot
 {TUG's searchable CTAN site}{\ctan}}
\newcommand{\ctanURL}[1]{\htmladdnormallink
 {CTAN:\texttt{.../#1}}{\ctan/tex-archive/#1}}
\newcommand{\ctanURLbr}[1]{\htmladdnormallink
 {CTAN:\texttt{.../#1}}{\ctan/tex-archive/#1}}
\newcommand{\ctanTUGurl}[2]{\ctanURL{#2}}
\newcommand{\indichtml}{l2h/indic/IndicHTML/}

\end{htmlonly}

\newcommand{\texdevURL}{http://www-texdev.mpce.mq.edu.au/}
\newcommand{\Unicode}{\htmladdnormallinkfoot
 {Unicode}{http://www.unicode.org/}}
\newcommand{\IndicHTML}{\htmladdnormallinkfoot
 {Indic\TeX/HTML}{\texdevURL{\indichtml}}}

%begin{latexonly}
\usepackage{array}

\newcommand{\Cs}[1]{{\upshape`\,\texttt{-#1}\,'}}
\newcommand{\Ve}[1]{\index{#1@\texttt{#1}}{\upshape`\,\texttt{#1}\,'}}
\newcommand{\Lc}[1]{{\upshape\ttfamily\char92#1}}
\newcommand{\Tc}[1]{{\upshape\ttfamily\char92#1}}

\newcommand{\ctanTUG}[1]{TUG's searchable CTAN site\footnote{\ctanTUGurl}}
\newcommand{\ctanURL}{CTAN: \penalty-200\ctanurl}
\newcommand{\ctanURLbr}{CTAN: \newline\ctanurl}
\urldef\texdev\url{http://www-texdev.mpce.mq.edu.au/}
\urldef\ctanurl\url{.../tex-archive/}
\urldef\ctanTUGurl\url{http://ctan.tug.org/ctan/tex-archive/}
\urldef\indichtml\url{l2h/indic/IndicHTML/}

\def\mathsmiley{\smiley}
%end{latexonly}

% thanks to \KrisRose for this macro:
\def\smiley{\hbox{\rlap{$\bigcirc$}\kern1.3pt$\scriptstyle\ddot\smile$}}

\begin{imagesonly}
\def\mathsmiley{\smiley}
\end{imagesonly}


\newcommand{\Lcs}[1]{{\upshape\ttfamily\char92#1}}

\renewcommand{\thefootnote}{\arabic{footnote}}

%\newcommand{\godown}[1]{{\htmlref
% {\htmladdimg[left BORDER=0]{../psfiles/dn.gif}}{#1}}}
\newcommand{\godown}[1]{}

%\newcommand{\goback}[1]{{\htmlref
% {\htmladdimg[left BORDER=0]{../psfiles/up.gif}}{#1}}}
\newcommand{\goback}[1]{}


\newcommand{\WiiiC}{\htmladdnormallinkfoot{World Wide Web Consortium}{http://www.w3c.org/}}
\newcommand{\latextohtmlNG}{\latextohtml-NG}
\newcommand{\maillist}{\htmladdnormallinkfoot{\latextohtml{} mailing list}%
{http://cbl.leeds.ac.uk/nikos/tex2html/doc/mail/mail.html}}
%
\newcommand{\HTMLiii}{\textup{\texttt{HTML} 3.2}}%
\newcommand{\HTMLIII}{\htmladdnormallink{\HTMLiii}%
{http://www.w3.org/TR/REC-html32.html}}%

\newcommand{\HTMLiv}{\textup{\texttt{HTML} 4.0}}%
\newcommand{\HTMLIV}{\htmladdnormallink{\HTMLiv}%
{http://www.w3.org/TR/REC-html40/}}%

%\newcommand{\latextohtml}{\textup{\LaTeX 2{\ttfamily HTML}}}%
\newcommand{\Perl}{\htmlref{\textsl{Perl}}{GGGPerl}}%  
\newcommand{\PS}{\htmlref{\textup{Post\-Script}}{GGGpostscript}\Glossary{PostScript}{PostScript}{}}%
\newcommand{\MF}{\htmlref{\textsl{Metafont}}{GGGMetafont}\Glossary{Metafont}{\textsl{Metafont}}{}}%
\newcommand{\fn}[1]{\htmlref{\texttt{#1}}{GGG#1}\glossary{#1}}%  file names, with link to glossary
\newcommand{\gn}[1]{\texttt{#1}\label{GGG#1}\htmlref{\^{}}{III#1}}%  file names, labelled within glossary
%\newcommand{\appl}[1]{\htmlref{\textsl{#1}}{GGG#1}\Glossary{#1}{\textsl{#1}}{}}%  application software names
\newcommand{\appl}[1]{\htmlref{\textsl{#1}}{GGG#1}\Glossary{#1}{\gsl{#1}}{}}%  application software names
%
\newcommand{\env}[1]{{\upshape\sffamily #1}}%  LaTeX environment and package names
\newcommand{\HTMLtag}[1]{\path{<#1>}}%  HTML tag
\newcommand{\Meta}[1]{\texttt{\upshape<\textit{#1}>}}%  Meta string

%
% developer names and addresses:
%
\newcommand{\NikosDrakos}{\index{Nikos Drakos}%\Email{nikos@cbl.leeds.ac.uk}
\htmladdnormallink{Nikos Drakos}{http://www.cbl.leeds.ac.uk/nikos/personal.html}}
%
\newcommand{\RossMoore}{\index{Ross Moore}%\Email{ross@mpce.mq.edu.au}
\htmladdnormallink{Ross Moore}{http://www.mpce.mq.edu.au/\~{}ross/}}
\newcommand{\Macquarie}{\htmladdnormallink
{Macquarie University}{http://www.mq.edu.au/}}
%
\newcommand{\Hennecke}{\index{Marcus Hennecke}%\Email{hennecke@dbag.ulm.DaimlerBenz.COM}
\htmladdnormallink{Marcus Hennecke}{http://www.crc.ricoh.com/\~{}marcush/}}
%
\newcommand{\Noworolski}{\index{Mark Noworolski}%\Email{jmn@eecs.berkeley.edu}
\htmladdnormallink{Mark Noworolski}{http://www-power.eecs.berkeley.edu/\~{}jmn/}}
%
\newcommand{\Isani}{\index{Sidik Isani}%\Email{isani@cfht.hawaii.edu}
\htmladdnormallink{Sidik Isani}{http://www.cfht.hawaii.edu/\~{}isani/si.html}}
%
\newcommand{\Goossens}{\index{Michel Goossens}%\Email{goossens@cern.ch}
\htmladdnormallink{Michel Goossens}{http://wwwcn1.cern.ch/\~{}goossens/}}
%
\newcommand{\Wilck}{\index{Martin Wilck}%\Email{martin@tropos.de}
\htmladdnormallink{Martin Wilck}{http://www.tropos.de/personal/wilck.html}}
%
\newcommand{\PatrickDaly}{\index{Patrick Daly}%\Email{daly@linmpi.dnet.gwdg.de}
\htmladdnormallink{Patrick Daly}{mailto:daly@linmpi.dnet.gwdg.de}}
%
\newcommand{\HerbSwan}{\index{Herb Swan}%\Email{herb.swan@perc.Arco.COM}
%\htmladdnormallink{Herb Swan}{mailto:herb.swan@perc.Arco.COM}}
\htmladdnormallink{Herb Swan}{mailto:lanhws@expl.aai.arco.com}}
%
\newcommand{\Lippmann}{\index{Jens Lippmann}%\Email{lippmann@cdc.informatik.tu-darmstadt.de}
\htmladdnormallink{Jens Lippmann}{http://www-jb.cs.uni-sb.de/\~{}www/people/lippmann}}
%
\newcommand{\Rouchal}{\index{Marek Rouchal}%\Email{marek@hl.siemens.de}
\htmladdnormallink{Marek Rouchal}{mailto:marek@hl.siemens.de}}
%
\newcommand{\Bohnet}{\index{Achim Bohnet}%\Email{ach@rosat.mpe-garching.mpg.de}
\htmladdnormallink{Achim Bohnet}{mailto:ach@rosat.mpe-garching.mpg.de}}
%
\newcommand{\Nelson}{\index{Scott Nelson}%\Email{nelson18@llnl.gov}
\htmladdnormallink{Scott Nelson}{mailto:nelson18@llnl.gov}}
%
\newcommand{\AxelRamge}{\index{Axel Ramge}%\Email{axel@ramge.de}
\htmladdnormallink{Axel Ramge}{mailto:axel@ramge.de}}
%
\newcommand{\Popineau}{\index{Fabrice Popineau}%\Email{popineau@esemetz.ese-metz.fr}
\htmladdnormallink{Fabrice Popineau}{mailto:popineau@esemetz.ese-metz.fr}}
%
\newcommand{\Wortmann}{\index{Uli Wortmann}%\Email{uli12@bonk.ethz.ch}
\htmladdnormallink{Uli Wortmann}{mailto:uli12@bonk.ethz.ch}}
%
\newcommand{\Veytsman}{\index{Boris Veytsman}%\Email{boris@plmsc.psu.edu}
\htmladdnormallink{Boris Veytsman}{mailto:boris@plmsc.psu.edu}}
%
\newcommand{\Taupin}{\index{Daniel Taupin}%\Email{taupin@lps.u-psud.fr}
\htmladdnormallink{Daniel Taupin}{mailto:taupin@lps.u-psud.fr}}
%
%\endinput

%
% (La)TeX related URLs
%
\newcommand{\CSEP}{\index{Computer Science Education Project!CSEP}%
\htmladdnormallinkfoot{Computer Science Education Project}%
{http://csep1.phy.ornl.gov/csep.html}} 
%
\newcommand{\CBLU}{\index{Computer Based Learning Unit!CBLU}%
\htmladdnormallinkfoot{Computer Based Learning Unit}%
{http://cbl.leeds.ac.uk/\~{}www/home.html}}
%
\newcommand{\CERN}{\index{CERN}%
\htmladdnormallink{CERN}{http://wwwcn.cern.ch/Welcome.html}} 
%
\newcommand{\KrisRose}{\index{Kristoffer Rose}%\Email{krisrose@brics.dk}
\htmladdnormallink{Kristoffer Rose}{http://www.brics.dk/\~{}krisrose/}}
%
\newcommand{\XypicDK}{\index{Xy-pic@\protect\Xy-pic graphics package}%
\index{Xy-pic@\protect\Xy-pic graphics package!home page}%
\htmladdnormallink{Xy-pic}{http://www.brics.dk/\~{}krisrose/Xy-pic.html}}
\newcommand{\XypicAUS}{\index{Xy-pic@\protect\Xy-pic graphics package!home page, down under}%
\htmladdnormallink{Ross Moore}{http://www.mpce.mq.edu.au/\~{}ross/Xy-pic.html}}
%
\newcommand{\LiPS}{\index{LiPS Design Team}%
\htmladdnormallink{LiPS Design Team}{http://www-jb.cs.uni-sb.de/LiPS/node2.html}}
\newcommand{\FIDarmstadt}{\index{Fachbereich Informatik, Darmstadt}%
\htmladdnormallink{Fachbereich Informatik}{http://www.informatik.tu-darmstadt.de/}}
\newcommand{\Darmstadt}{\index{Darmstadt!Fachbereich Informatik}%
\htmladdnormallink{Darmstadt}{http://www.tu-darmstadt.de/Welcome.de.html}}
%
\newcommand{\DANTE}{\index{DANTE}%
\htmladdnormallink{DANTE e.V.}{http://www.dante.de/}}
\newcommand{\Praesidium}{\index{DANTE!Praesidium}%
\htmladdnormallink{Praesidium}{http://www.dante.de/dante/Organe.html}}
\newcommand{\LaTeXiii}{\index{LaTeX@\LaTeX!LaTeX3@\LaTeX{3}}%
\htmladdnormallink{\LaTeX{3}}{http://www.tex.ac.uk/CTAN/latex/latex3}}
%
\newcommand{\Engberg}{\index{Uffe Engberg}%
\htmladdnormallinkfoot{Uffe Engberg}{http://www.brics.dk/\~{}engberg}} 

\newcommand{\texdevINDIC}{\texdev\indic}
%\endinput


%% earlier contributions from...
\newcommand{\AndrewCole}{\index{Computer Based Learning Unit!Andrew Cole}%
\htmladdnormallink{Andrew Cole}{http://www.cbl.leeds.ac.uk/ajcole/personal.html}}
%
\newcommand{\AnaPaiva}{\index{Computer Based Learning Unit!Ana Maria Paiva}%
\htmladdnormallink{Ana Maria Paiva}{http://www.cbl.leeds.ac.uk/amp/personal.html}}
%
\newcommand{\RodWilliams}{\index{Computer Based Learning Unit!Roderick Williams}%
\htmladdnormallink{Roderick Williams}{http://www.cbl.leeds.ac.uk/rodw/personal.html}}
%
\newcommand{\JamilSawar}{\index{Computer Based Learning Unit!Jamil Sawar}%
\htmladdnormallink{Jamil Sawar}{http://www.cbl.leeds.ac.uk/sawar/personal.html}}

\newcommand{\RobertThau}{\index{Robert S. Thau}%
\htmladdnormallink{Robert S. Thau}{mailto: rst@edu.mit.ai}}

\endinput




%
\newcommand{\RobertThau}{\index{Robert S. Thau}%
\htmladdnormallink{Robert S. Thau}{mailto: rst@edu.mit.ai}}


%
% Ian Foster \Email{itf@mcs.anl.gov}
% Bob Olson \Email{olson@mcs.anl.gov}
% Verena Umar \Email{verena@edu.vanderbilt.cas.compsci}
% Axel Belinfante \Email{Axel.Belinfante@cs.utwente.nl}
% Todd Little \Email{little@com.dec.enet.nuts2u}
% Franz Vojik \Email{vojik@de.tu-muenchen.informatik}
% Eric Carroll \Email{eric@ca.utoronto.utcc.enfm}
% Roderick Williams \Email{rodw@cbl.leeds.ac.uk}
% Robert Cailliau \Email{cailliau@cernnext.cern.ch}
% Toni Lantunen at CERN
% Ana Maria Paiva, Jamil Sawar, Andrew Cole at CBLU Leeds
% Phillip Conrad (Perfect Byte, Inc.)
% L. Peter Deutsch (

%
\begin{htmlonly}
\documentclass[dvips,a4paper]{article}
\usepackage{html,htmllist,makeidx}

%this font is used in  features.tex
\font\wncyr = wncyr10


\begin{htmlonly}
 \usepackage[dvips]{graphicx}
 \usepackage[dvips]{color}
 \usepackage[dvips,leftbars]{changebar}
 \newcommand{\FoilTeX}{\env{FoilTeX}}
%
 \def\url#1{\htmladdnormallink{#1}{#1}}
 \def\Email#1{\htmladdnormallink{<#1>}{mailto:#1}}
 \def\path#1{\texttt{#1}}
% \def\urldef#1#2#3{\def#1{#2{#3}}}
 \def\onlinedoc{\url{http://www-dsed.llnl.gov/files/programs/unix/latex2html/manual/}}
 \def\patches{\url{http://www-dsed.llnl.gov/files/programs/unix/latex2html/}}
 \def\sourceA{%
  \url{http://www-dsed.llnl.gov/files/programs/unix/latex2html/sources/}}
% \def\sourceB{\url{ftp://ftp.mpn.com/pub/nikos/latex2html-98.1.tar.gz}}
 \def\sourceB{\CTANtug{\CTANA}}
 \def\sourceC{\url{http://ftp.rzg.mpg.de/pub/software/latex2html/sources/}}
 \def\CVSsite{\url{http://cdc-server.cdc.informatik.tu-darmstadt.de/\~{}latex2html/}}
 \def\CVSrepos{\url{http://cdc-server.cdc.informatik.tu-darmstadt.de/\~{}latex2html/user/}}
 \def\CVSlatest{\url{http://cdc-server.cdc.informatik.tu-darmstadt.de/\~{}latex2html/l2h-latest.tar.gz}}
 \newcommand{\CTANtug}[1]{\path{http://ctan.tug.org/ctan/#1}}
 \def\CTANA{tex-archive/support/latex2html}
 \def\tugURL{\url{http://www.tug.org}}
 \def\danteURL{\url{http://www.dante.de}}
 \def\ListURL{\url{http://www.rosat.mpe-garching.mpg.de/mailing-lists/LaTeX2HTML/}}
%
 \def\glossary#1{\index{#1@\texttt{#1} \label{III#1}\htmlref{(G)}{GGG#1}}}
 \def\Glossary#1#2{\index{#1@{#2} \label{III#1}\htmlref{(G)}{GGG#1}}}
%
% hack to suppress changebar entries read from .aux file  !!! bug in changebar.perl !!! 
 \makeatletter
 \def\cb@barpoint#1#2#3{}
 \makeatother
%
\newcommand{\sameas}[1]{\textcolor{red}{Same as setting: #1}}
\newcommand{\onlinedocRM}{\url{http://www-math.mpce.mq.edu.au/\~{}ross/latex/manual/manual.html}}
\newcommand{\EXcolors}{\url{http://www-math.mpce.mq.edu.au/\~{}ross/latex/crayola/crayola.html}}
%
\newcommand{\Lc}[1]{\texttt{\char92#1}} % LaTeX command
\newcommand{\Tc}[1]{\texttt{\char92#1}} % TeX command
\newcommand{\Cs}[1]{ \texttt{-#1} }   % command-line switch
\newcommand{\Ve}[1]{\index{#1@\texttt{#1}}\texttt{#1}} % version extension
\newcommand{\gsl}[1]{#1}
\newcommand{\indexentry}[2]{\item #1 #2}

%
\internal{}%
\internal{O}%
\internal{S}%
\internal{E}%
\internal{M}%
\internal{H}%
\internal{P}%

\newcommand{\texdev}[1]{\htmladdnormallink
 {\texdevURL/#1}{\texdevURL/#1}}
\newcommand{\ctan}{http://ctan.tug.org/ctan}
\newcommand{\ctanTUG}{\htmladdnormallinkfoot
 {TUG's searchable CTAN site}{\ctan}}
\newcommand{\ctanURL}[1]{\htmladdnormallink
 {CTAN:\texttt{.../#1}}{\ctan/tex-archive/#1}}
\newcommand{\ctanURLbr}[1]{\htmladdnormallink
 {CTAN:\texttt{.../#1}}{\ctan/tex-archive/#1}}
\newcommand{\ctanTUGurl}[2]{\ctanURL{#2}}
\newcommand{\indichtml}{l2h/indic/IndicHTML/}

\end{htmlonly}

\newcommand{\texdevURL}{http://www-texdev.mpce.mq.edu.au/}
\newcommand{\Unicode}{\htmladdnormallinkfoot
 {Unicode}{http://www.unicode.org/}}
\newcommand{\IndicHTML}{\htmladdnormallinkfoot
 {Indic\TeX/HTML}{\texdevURL{\indichtml}}}

%begin{latexonly}
\usepackage{array}

\newcommand{\Cs}[1]{{\upshape`\,\texttt{-#1}\,'}}
\newcommand{\Ve}[1]{\index{#1@\texttt{#1}}{\upshape`\,\texttt{#1}\,'}}
\newcommand{\Lc}[1]{{\upshape\ttfamily\char92#1}}
\newcommand{\Tc}[1]{{\upshape\ttfamily\char92#1}}

\newcommand{\ctanTUG}[1]{TUG's searchable CTAN site\footnote{\ctanTUGurl}}
\newcommand{\ctanURL}{CTAN: \penalty-200\ctanurl}
\newcommand{\ctanURLbr}{CTAN: \newline\ctanurl}
\urldef\texdev\url{http://www-texdev.mpce.mq.edu.au/}
\urldef\ctanurl\url{.../tex-archive/}
\urldef\ctanTUGurl\url{http://ctan.tug.org/ctan/tex-archive/}
\urldef\indichtml\url{l2h/indic/IndicHTML/}

\def\mathsmiley{\smiley}
%end{latexonly}

% thanks to \KrisRose for this macro:
\def\smiley{\hbox{\rlap{$\bigcirc$}\kern1.3pt$\scriptstyle\ddot\smile$}}

\begin{imagesonly}
\def\mathsmiley{\smiley}
\end{imagesonly}


\newcommand{\Lcs}[1]{{\upshape\ttfamily\char92#1}}

\renewcommand{\thefootnote}{\arabic{footnote}}

%\newcommand{\godown}[1]{{\htmlref
% {\htmladdimg[left BORDER=0]{../psfiles/dn.gif}}{#1}}}
\newcommand{\godown}[1]{}

%\newcommand{\goback}[1]{{\htmlref
% {\htmladdimg[left BORDER=0]{../psfiles/up.gif}}{#1}}}
\newcommand{\goback}[1]{}


\newcommand{\WiiiC}{\htmladdnormallinkfoot{World Wide Web Consortium}{http://www.w3c.org/}}
\newcommand{\latextohtmlNG}{\latextohtml-NG}
\newcommand{\maillist}{\htmladdnormallinkfoot{\latextohtml{} mailing list}%
{http://cbl.leeds.ac.uk/nikos/tex2html/doc/mail/mail.html}}
%
\newcommand{\HTMLiii}{\textup{\texttt{HTML} 3.2}}%
\newcommand{\HTMLIII}{\htmladdnormallink{\HTMLiii}%
{http://www.w3.org/TR/REC-html32.html}}%

\newcommand{\HTMLiv}{\textup{\texttt{HTML} 4.0}}%
\newcommand{\HTMLIV}{\htmladdnormallink{\HTMLiv}%
{http://www.w3.org/TR/REC-html40/}}%

%\newcommand{\latextohtml}{\textup{\LaTeX 2{\ttfamily HTML}}}%
\newcommand{\Perl}{\htmlref{\textsl{Perl}}{GGGPerl}}%  
\newcommand{\PS}{\htmlref{\textup{Post\-Script}}{GGGpostscript}\Glossary{PostScript}{PostScript}{}}%
\newcommand{\MF}{\htmlref{\textsl{Metafont}}{GGGMetafont}\Glossary{Metafont}{\textsl{Metafont}}{}}%
\newcommand{\fn}[1]{\htmlref{\texttt{#1}}{GGG#1}\glossary{#1}}%  file names, with link to glossary
\newcommand{\gn}[1]{\texttt{#1}\label{GGG#1}\htmlref{\^{}}{III#1}}%  file names, labelled within glossary
%\newcommand{\appl}[1]{\htmlref{\textsl{#1}}{GGG#1}\Glossary{#1}{\textsl{#1}}{}}%  application software names
\newcommand{\appl}[1]{\htmlref{\textsl{#1}}{GGG#1}\Glossary{#1}{\gsl{#1}}{}}%  application software names
%
\newcommand{\env}[1]{{\upshape\sffamily #1}}%  LaTeX environment and package names
\newcommand{\HTMLtag}[1]{\path{<#1>}}%  HTML tag
\newcommand{\Meta}[1]{\texttt{\upshape<\textit{#1}>}}%  Meta string

%
% developer names and addresses:
%
\newcommand{\NikosDrakos}{\index{Nikos Drakos}%\Email{nikos@cbl.leeds.ac.uk}
\htmladdnormallink{Nikos Drakos}{http://www.cbl.leeds.ac.uk/nikos/personal.html}}
%
\newcommand{\RossMoore}{\index{Ross Moore}%\Email{ross@mpce.mq.edu.au}
\htmladdnormallink{Ross Moore}{http://www.mpce.mq.edu.au/\~{}ross/}}
\newcommand{\Macquarie}{\htmladdnormallink
{Macquarie University}{http://www.mq.edu.au/}}
%
\newcommand{\Hennecke}{\index{Marcus Hennecke}%\Email{hennecke@dbag.ulm.DaimlerBenz.COM}
\htmladdnormallink{Marcus Hennecke}{http://www.crc.ricoh.com/\~{}marcush/}}
%
\newcommand{\Noworolski}{\index{Mark Noworolski}%\Email{jmn@eecs.berkeley.edu}
\htmladdnormallink{Mark Noworolski}{http://www-power.eecs.berkeley.edu/\~{}jmn/}}
%
\newcommand{\Isani}{\index{Sidik Isani}%\Email{isani@cfht.hawaii.edu}
\htmladdnormallink{Sidik Isani}{http://www.cfht.hawaii.edu/\~{}isani/si.html}}
%
\newcommand{\Goossens}{\index{Michel Goossens}%\Email{goossens@cern.ch}
\htmladdnormallink{Michel Goossens}{http://wwwcn1.cern.ch/\~{}goossens/}}
%
\newcommand{\Wilck}{\index{Martin Wilck}%\Email{martin@tropos.de}
\htmladdnormallink{Martin Wilck}{http://www.tropos.de/personal/wilck.html}}
%
\newcommand{\PatrickDaly}{\index{Patrick Daly}%\Email{daly@linmpi.dnet.gwdg.de}
\htmladdnormallink{Patrick Daly}{mailto:daly@linmpi.dnet.gwdg.de}}
%
\newcommand{\HerbSwan}{\index{Herb Swan}%\Email{herb.swan@perc.Arco.COM}
%\htmladdnormallink{Herb Swan}{mailto:herb.swan@perc.Arco.COM}}
\htmladdnormallink{Herb Swan}{mailto:lanhws@expl.aai.arco.com}}
%
\newcommand{\Lippmann}{\index{Jens Lippmann}%\Email{lippmann@cdc.informatik.tu-darmstadt.de}
\htmladdnormallink{Jens Lippmann}{http://www-jb.cs.uni-sb.de/\~{}www/people/lippmann}}
%
\newcommand{\Rouchal}{\index{Marek Rouchal}%\Email{marek@hl.siemens.de}
\htmladdnormallink{Marek Rouchal}{mailto:marek@hl.siemens.de}}
%
\newcommand{\Bohnet}{\index{Achim Bohnet}%\Email{ach@rosat.mpe-garching.mpg.de}
\htmladdnormallink{Achim Bohnet}{mailto:ach@rosat.mpe-garching.mpg.de}}
%
\newcommand{\Nelson}{\index{Scott Nelson}%\Email{nelson18@llnl.gov}
\htmladdnormallink{Scott Nelson}{mailto:nelson18@llnl.gov}}
%
\newcommand{\AxelRamge}{\index{Axel Ramge}%\Email{axel@ramge.de}
\htmladdnormallink{Axel Ramge}{mailto:axel@ramge.de}}
%
\newcommand{\Popineau}{\index{Fabrice Popineau}%\Email{popineau@esemetz.ese-metz.fr}
\htmladdnormallink{Fabrice Popineau}{mailto:popineau@esemetz.ese-metz.fr}}
%
\newcommand{\Wortmann}{\index{Uli Wortmann}%\Email{uli12@bonk.ethz.ch}
\htmladdnormallink{Uli Wortmann}{mailto:uli12@bonk.ethz.ch}}
%
\newcommand{\Veytsman}{\index{Boris Veytsman}%\Email{boris@plmsc.psu.edu}
\htmladdnormallink{Boris Veytsman}{mailto:boris@plmsc.psu.edu}}
%
\newcommand{\Taupin}{\index{Daniel Taupin}%\Email{taupin@lps.u-psud.fr}
\htmladdnormallink{Daniel Taupin}{mailto:taupin@lps.u-psud.fr}}
%
%\endinput

%
% (La)TeX related URLs
%
\newcommand{\CSEP}{\index{Computer Science Education Project!CSEP}%
\htmladdnormallinkfoot{Computer Science Education Project}%
{http://csep1.phy.ornl.gov/csep.html}} 
%
\newcommand{\CBLU}{\index{Computer Based Learning Unit!CBLU}%
\htmladdnormallinkfoot{Computer Based Learning Unit}%
{http://cbl.leeds.ac.uk/\~{}www/home.html}}
%
\newcommand{\CERN}{\index{CERN}%
\htmladdnormallink{CERN}{http://wwwcn.cern.ch/Welcome.html}} 
%
\newcommand{\KrisRose}{\index{Kristoffer Rose}%\Email{krisrose@brics.dk}
\htmladdnormallink{Kristoffer Rose}{http://www.brics.dk/\~{}krisrose/}}
%
\newcommand{\XypicDK}{\index{Xy-pic@\protect\Xy-pic graphics package}%
\index{Xy-pic@\protect\Xy-pic graphics package!home page}%
\htmladdnormallink{Xy-pic}{http://www.brics.dk/\~{}krisrose/Xy-pic.html}}
\newcommand{\XypicAUS}{\index{Xy-pic@\protect\Xy-pic graphics package!home page, down under}%
\htmladdnormallink{Ross Moore}{http://www.mpce.mq.edu.au/\~{}ross/Xy-pic.html}}
%
\newcommand{\LiPS}{\index{LiPS Design Team}%
\htmladdnormallink{LiPS Design Team}{http://www-jb.cs.uni-sb.de/LiPS/node2.html}}
\newcommand{\FIDarmstadt}{\index{Fachbereich Informatik, Darmstadt}%
\htmladdnormallink{Fachbereich Informatik}{http://www.informatik.tu-darmstadt.de/}}
\newcommand{\Darmstadt}{\index{Darmstadt!Fachbereich Informatik}%
\htmladdnormallink{Darmstadt}{http://www.tu-darmstadt.de/Welcome.de.html}}
%
\newcommand{\DANTE}{\index{DANTE}%
\htmladdnormallink{DANTE e.V.}{http://www.dante.de/}}
\newcommand{\Praesidium}{\index{DANTE!Praesidium}%
\htmladdnormallink{Praesidium}{http://www.dante.de/dante/Organe.html}}
\newcommand{\LaTeXiii}{\index{LaTeX@\LaTeX!LaTeX3@\LaTeX{3}}%
\htmladdnormallink{\LaTeX{3}}{http://www.tex.ac.uk/CTAN/latex/latex3}}
%
\newcommand{\Engberg}{\index{Uffe Engberg}%
\htmladdnormallinkfoot{Uffe Engberg}{http://www.brics.dk/\~{}engberg}} 

\newcommand{\texdevINDIC}{\texdev\indic}
%\endinput


%% earlier contributions from...
\newcommand{\AndrewCole}{\index{Computer Based Learning Unit!Andrew Cole}%
\htmladdnormallink{Andrew Cole}{http://www.cbl.leeds.ac.uk/ajcole/personal.html}}
%
\newcommand{\AnaPaiva}{\index{Computer Based Learning Unit!Ana Maria Paiva}%
\htmladdnormallink{Ana Maria Paiva}{http://www.cbl.leeds.ac.uk/amp/personal.html}}
%
\newcommand{\RodWilliams}{\index{Computer Based Learning Unit!Roderick Williams}%
\htmladdnormallink{Roderick Williams}{http://www.cbl.leeds.ac.uk/rodw/personal.html}}
%
\newcommand{\JamilSawar}{\index{Computer Based Learning Unit!Jamil Sawar}%
\htmladdnormallink{Jamil Sawar}{http://www.cbl.leeds.ac.uk/sawar/personal.html}}

\newcommand{\RobertThau}{\index{Robert S. Thau}%
\htmladdnormallink{Robert S. Thau}{mailto: rst@edu.mit.ai}}

\endinput




%
\newcommand{\RobertThau}{\index{Robert S. Thau}%
\htmladdnormallink{Robert S. Thau}{mailto: rst@edu.mit.ai}}


%
% Ian Foster \Email{itf@mcs.anl.gov}
% Bob Olson \Email{olson@mcs.anl.gov}
% Verena Umar \Email{verena@edu.vanderbilt.cas.compsci}
% Axel Belinfante \Email{Axel.Belinfante@cs.utwente.nl}
% Todd Little \Email{little@com.dec.enet.nuts2u}
% Franz Vojik \Email{vojik@de.tu-muenchen.informatik}
% Eric Carroll \Email{eric@ca.utoronto.utcc.enfm}
% Roderick Williams \Email{rodw@cbl.leeds.ac.uk}
% Robert Cailliau \Email{cailliau@cernnext.cern.ch}
% Toni Lantunen at CERN
% Ana Maria Paiva, Jamil Sawar, Andrew Cole at CBLU Leeds
% Phillip Conrad (Perfect Byte, Inc.)
% L. Peter Deutsch (

%
\begin{htmlonly}
\documentclass[dvips,a4paper]{article}
\usepackage{html,htmllist,makeidx}
\input{manhtml.tex}
%
\input{problems.ptr}    % Input counters and section
\end{htmlonly}
\internal{}%
\startdocument
%
%\section{Known Problems\label{sec:prb}\index{problems|(}}%
\label{sec:prb}\index{problems|(}%
\tableofchildlinks*\htmlrule\index{bugs}\html{\\}%
\noindent
Here are some of the problems that were known to exist
with previous versions of \latextohtml.
Most of those that were real errors are either fixed completely
in the current version (\textsc{v98.1}), or are much less likely to occur
within correct \LaTeX{} source. (Some are not really errors but
indications of poor style in the user's choices among
various ways to organise their source code.)\html{\\}
Several are indeed limitations inherent in the way
\latextohtml{} currently performs its processing.
%
\begin{htmllist}
\htmlitemmark{PurpleBall}
\item [Correctness and Efficiency:\index{efficiency} ]
The translator cannot be guaranteed to perform as expected.
Several aspects of the implementation need
optimisation and improvement. Apart from possible bugs the translator
may place heavy demands on your resources.
\cbversion{97.1}\begin{changebar}
%begin{latexonly}
\textit
%end{latexonly}
{The current version works much more efficiently than previous
versions; many subtle bugs have been identified and eliminated.}
\end{changebar}
\cbversion{98.1}\begin{changebar}
%begin{latexonly}
\textit
%end{latexonly}
{The process of command substitution has been improved significantly,
resulting in memory savings and faster translation.}
\end{changebar}

\item [Unrecognised Commands and Environments:\index{unrecognised commands} ]
Unrecognised commands are ignored and any arguments are left in the text.
Unrecognised environments are passed to \LaTeX{}  and the result is
included in the document as one or more inlined images.
\cbversion{97.1}\begin{changebar}
%begin{latexonly}
\newline\textit
%end{latexonly}
{There are very few standard \LaTeX{} commands that are not recognised.
Many common \TeX{} commands are recognised also,
even though not explicitly mentioned
in the \LaTeX{} \hypercite{blue book}{blue book}{}{lamp:latex}.
Any aberrant commands should be reported to the \latextohtml{}
\hyperref{mailing list}{mailing list, see Section~}{}{support}.}
\end{changebar}


\item [Cross-references:\index{cross-references} ]
References in environments that are passed to \LaTeX{}  for processing
(e.g. a \Lc{cite}, or a \Lc{ref} command),
are not processed correctly.
\Lc{label} commands are handled correctly.
\cbversion{97.1}\begin{changebar}
%begin{latexonly}
\newline\textit
%end{latexonly}
{All citation, reference and label commands should work correctly now.
Report any problems to the \latextohtml{}
\hyperref{mailing list}{mailing list, see Section~}{}{support}.}
\end{changebar}

\item[Order-Sensitive Commands: ]
Commands which affect global parameters during the translation,
and are sensitive to the order in which they are processed may
not be handled correctly. In particular, counter manipulation
(e.g. \Lc{newcounter}, \Lc{setcounter}, \Lc{stepcounter}, etc.)
commands may cause problems.
\cbversion{97.1}\begin{changebar}
%begin{latexonly}
\newline\textit
%end{latexonly}
{Counter commands now work correctly; dependencies
are also implemented.}
\end{changebar}

\index{index}
\item [Index: ]
The translator generates its own index by saving the arguments  of
the \Lc{index} command. The contents of the \env{theindex}
environment are ignored.
\cbversion{97.1}\begin{changebar}
%begin{latexonly}
\newline\textit
%end{latexonly}
{This remains true. When using the \env{makeidx} package,
very sophisticated Indexes can be built automatically.
The Index for this manual is a good example.}
\end{changebar}

\item[New Definitions:\index{extensions!definitions} ]
New definitions (\Lc{newcommand}, \Lc{newenvironment},
\Lc{newtheorem} and \Lc{def}),
will not work as expected if they are defined more than once.
Only the last definition will be used throughout the document.
\cbversion{97.1}\begin{changebar}
%begin{latexonly}
\newline\textit
%end{latexonly}
{This remains true. Stylistically it is bad to declare new
environments or theorems outside of the document preamble,
so these should cause no problems anyway.\html{\\}
Changes to commands using \Lc{def} or \Lc{renewcommand}
should usually be made only locally, within special environments,
to set a needed parameter;
e.g. a basic length in a \env{picture} environment.
But when such environments force an image to be generated,
then \LaTeX{} will make the correct redefinition.}
\end{changebar}

\item [Scope of declarations and environments: ]
If the scope of a declaration or environment crosses section
boundaries, then the output may not be as expected, because each
section is processed independently.
\cbversion{97.1}\begin{changebar}
%begin{latexonly}
\newline\textit
%end{latexonly}
{This is inherent to the way \latextohtml{} does its processing.
It will not be fixed until later versions change this strategy;
e.g. when \latextohtmlNG\ becomes fully integrated.}
\end{changebar}

\item [Math-mode font-size changes: ]  Math-mode font changes
made outside the math-mode are not honoured.  Thus the two equations
in \verb|$a_b$| and \verb|{\LARGE $a_b$}|
would come out looking the same.  The trick is to write
\verb|$a_b| and \verb|$\mbox{\LARGE $a_b$}$|.
\cbversion{97.1}\begin{changebar}
%begin{latexonly}
\newline\textit
%end{latexonly}
{This remains. The work-around is effective.}
\end{changebar}
\end{htmllist}



\subsection{Troubleshooting}%
%\section{Troubleshooting}%
\index{debugging}\index{problems}\index{fixes}%
Here are some curable symptoms:

\begin{htmllist}
\htmlitemmark{BlueBall}
\item [Cannot run any of the \Perl{} programs: ]
If your \Perl{} installation is such that \Perl{} programs are not allowed
to run as shell scripts you may be unable to run
\fn{latex2html}, \fn{texexpand}, \fn{pstoimg} and  \fn{install-test}.
In this case change the first line in each of these
programs from
%begin{latexonly}
\begin{small}
%end{latexonly}
\begin{verbatim}
#!/usr/local/bin/perl
\end{verbatim}
%begin{latexonly}
\end{small}%
%end{latexonly}
to
%begin{latexonly}
\begin{small}%
%end{latexonly}
\begin{verbatim}
: # *-*-perl-*-*
    eval 'exec perl -S  $0 "$@"'
    if $running_under_some_shell;
\end{verbatim}
%begin{latexonly}
\end{small}
%end{latexonly}

\item [The \fn{install-test} script gives uninformative error messages: ]
If, for any reason, you have trouble running \fn{install-test},
do not despair. Most of what it does is to do with checking
your installation rather than actually installing anything.
To do a \textit{manual installation} just change the variable
\fn{\$LATEX2HTMLDIR} in the beginning of the file \fn{latex2html}
to point to the directory where the \latextohtml{} files can be found.

Also, make sure that the files
\fn{pstoimg}, \fn{texexpand} and \fn{latex2html} are executable;
if necessary use the Unix \fn{chmod} command to make them
executable.

\item [It just stops. ] Check the style
files that you are using. It is possible that you are using
a style file which contains raw \TeX{} commands. In such a case
start \latextohtml{} with the option
\Cs{dont\_include \Meta{style-file name}}.
Alternatively, add the name of the style to the variable
\fn{\$DONT\_INCLUDE} in your
\fn{\$HOME/.latex2html-init} file. If you don't have such a file then
create one and add the lines:
\begin{quote}
%begin{latexonly}
\begin{small}
%end{latexonly}
\verb|$DONT_INCLUDE = "$DONT_INCLUDE" . ":|\Meta{style file name}\verb|";|\\
\verb|1;        # This must be the last line|
%begin{latexonly}
\end{small}
%end{latexonly}
\end{quote}

Another reason why \latextohtml{} might stop is that the \LaTeX{} source
file itself contains raw \TeX{} commands. In this case you may
put such commands inside a \env{latexonly}
\hyperref{environment}{environment (see Section }{)}{sec:latexonly}.

\cbversion{97.1}\begin{changebar}
The \fn{\$VERBOSITY} \htmlref{variable}{verbosity}{} can be used
to create tracing messages, which may help to locate which command
or environment was being processed when everything stopped.
\end{changebar}

\index{unending loop}\index{looping endlessly}
\cbversion{97.1}\begin{changebar}
\item [It appears to be doing nothing.\label{looping}]
Perhaps the processor has fallen into an unending loop.
Usually there will be a bad definition, or other faulty source code,
which has caused this.
The \fn{\$VERBOSITY} \hyperref[page]{variable}{variable (see page~}{)}{cs_verbositylevel}%
can be set to generate tracing messages, which may help to locate which command
or environment is being processed repeatedly.
Try setting a value of `\texttt{3}'; e.g. using the commandline switch \Cs{verbosity 3}.
This will print command and environment names, as thaey are processed.
It should soon become apparent where any such looping occurs.

\item [It just fills the endlessly with dots.]
No `perhaps' here; the processor has definitely fallen into an unending loop.
See the previous item for how to detect which command or environment
is causing the problem.
\end{changebar}

\cbversion{98.1}\begin{changebar}
%
\item [\Perl{} cannot parse the \fn{latex2html} script: ]
If \Perl{} refuses to start \latextohtml{} and issues errors, your
\Perl{} version is not up to date.
Update your \Perl{} to 5.003 or later. You can check which version of
\Perl{} you are using by invoking \Perl{} with the \Cs{v} option.

If \Perl{} issues errors during runtime, this is most probably related
to bugs within \latextohtml{} or one of its modules.
In this case you will need help from the developers or experienced users;
this can be obtained via the discussion list.


\item [It crashes (dumps core) as soon as it starts\label{perl}: ]
Update your \Perl{} to 5.003 or later.
\end{changebar}


\cbversion{98.1}\begin{changebar}
\item [It does not show any of your images: ]
  You can't run \latextohtml{} in a subdirectory that contains a dot
  within the directory name, such as {\tt latex2html-98.1},
  or in name of any higher directory.
  This is because \fn{dvips}'s \texttt{-o} option will change 98.1 into 98.001
  and use that as the resulting output file, instead of image001~.
  The \PS\ files will be placed higher up in the directory tree.

For instance, if \texttt{ pwd } returns something like:
\begin{verbatim}
/usr/people/nelson/files/latex2html-98.1/work/tests
\end{verbatim}
and you run \latextohtml, then \fn{dvips} will generate
image output files here:
\begin{verbatim}
/usr/people/nelson/files
\end{verbatim}
  called \texttt{latex2html-98.001}, \texttt{latex2html-98.002}, \dots
  instead of \texttt{image001}, \texttt{image002}, \texttt{image003}, \dots
  in the subdirectory where your \texttt{.html} files were created.
  As a result the images will not show in your documents.

  If you are getting `File Not Found' errors,
  then reprocess your job using the \Cs{debug} switch,
  to see what options are passed to \fn{dvips}.
  If there is a `.' in some parts of any directory name,
  then look above that directory
  to see if files are being generated there.

\medskip

  One obvious fix is to rename the offending directory
  to remove the `.' from its name.

  If that is not possible, then define an alternative location
  for image generation to take place; set \texttt{\$TMP} to contain
  the name for this location. Typically \texttt{\$TMP = '/usr/tmp';}~.
  (This use of \texttt{\$TMP} is a good thing to do anyway,
  especially if your Unix account is subject to quota limitations.)

\item [It stops after having run \LaTeX, displaying a message about \fn{dvips}: ]
  See the previous item for the reason and `fix'.
  This version of \latextohtml{} detects the problem
  and will not allow images to be generated in the wrong place.
\end{changebar}

\item [\fn{dvips} complains about incorrect arguments:\label{dvips} ]
Please use a version which supports the command-line options \Cs{M}, \Cs{S},
\Cs{o} and \Cs{i}. ``Recent'' versions, at least after 5.516, do support them.


\item [It gives an ``\texttt{Out of memory}'' message and dies: ]
Try splitting your source file into more than one file,
using the \LaTeX{} commands \Lc{input} or \Lc{include}.
Also, try using the \Cs{no\_images} option.

\cbversion{97.1}\begin{changebar}
Perhaps the processor has fallen into an infinite loop.
Usually there will be a bad definition, or other faulty source code,
which has caused this.
See an \htmlref{earlier problem}{looping} for
how to set the \fn{\$VERBOSITY} \htmlref{variable}{verbosity} to
help locate the bad code leading to this memory exhaustion.
\end{changebar}

As a last resort you may consider increasing the virtual memory
(swap space) of your machine. As an indication
of what you might be able to do on your machine,
a very long book (about 1000 printed pages) required about
24MB of RAM and over 150MB of swap space to convert on a local Sun Sparc ELC
running SunOS 4.1.3.
\cbversion{97.1}\begin{changebar}
%begin{latexonly}
\newline\textit
%end{latexonly}%
{Much of this memory would have been consumed during image-generation.
This part of the processing is much more efficient in \textsc{v97.1}\,.}
\end{changebar}%


\cbversion{98.1}\begin{changebar}
\item [\fn{install-test} issues ``dbm'' related error messages: ]\latex{~\\}%
\latextohtml{} requires a DataBase Management system (\appl{NDBM},
\appl{GDBM}, or \appl{SDBM}) in order to run.
This is usually part of each Unix-like operating system and
\appl{SDBM} is part of \Perl{ 5}, but sometimes this is either missing
on your operating system or left out in a binary \Perl{} distribution.
\htmladdnormallinkfoot{Use \appl{Archie}}{http://www.pvv.unit.no/archie/}
to find one or (better) update to a complete \Perl{ 5} distribution.

\item [\fn{latex2html} issues ``dbm'' related error messages: ]
If you get warnings like
\begin{verbatim}
ndbm store returned -1, errno 28, key "xyz" at latex2html line 123
\end{verbatim}
this is related to an overflow of \latextohtml{} internals.
You will need help from the list, here.
\smallskip

If you get real error messages which cause \latextohtml{} to abort,
run \fn{install-test} to check if your DataBase management works.
You will probably need to re-install \Perl{ 5} (see above topic).
\end{changebar}


\cbversion{97.1}\begin{changebar}
This can happen when an image is being created from a large piece
of \LaTeX{} source code. The image-reuse mechanism uses the code itself
to construct a database key. If too long, the key is invalid and may
crash \appl{DBM} or \appl{NDBM}. (In fact this error should no longer
occur in \textsc{v97.1}\,, so please advise the \latextohtml{} developers if
you get error messages of this kind.)\html{\\}
The message should contain the name of environment which caused the problem,
along with an identifying number; e.g. \texttt{eqnarray268}.
To find which exact piece of code this represents, run \latextohtml{}
again, using the \Cs{debug} switch. Then look at the files
in the \texttt{TMP} subdirectory of the working directory
named \texttt{TMP/part001}, \texttt{TMP/part002}, etc.
Use the unix \fn{grep} command: \texttt{ grep 268 }\Meta{dir}\texttt{/TMP/part* }
to find that number in these files.
This should enable you to locate exactly where the problem occurs.

One solution may be to wrap the whole environment within
\Lc{begin\char123makeimage\char125} and \Lc{end\char123makeimage\char125}.
This will still cause the image to be created,
but uses just the environment name and number as the database key.
\end{changebar}%



\item [The \Lc{verb"ABC"} command doesn't work: ]
This is an unfortunate bug which can be avoided by using
any character other than quotes; e.g. \Lc{verb+ABC+}.

\item [Cannot get the ``tilde'' (\~{}) to show: \hfill]
The trick here is to use \verb|\~{}|.\html{\\}
Alternatively try using something like:
\url{\htmladdnormallink{mylink}{http://host/\%7Eme/path/file.html}}.

\textbf{Warning:} Some browsers may not be able to interpret the \verb|%7E|
as a ``tilde'' character.

\cbversion{98.1}\begin{changebar}
Try using \Lc{char126}~.
In any case tildes within \Lc{htmladdnormallink} and similar commands
are now handled correctly.
\end{changebar}

\item [Macro definitions don't work correctly: ]
As mentioned in other places, not all plain \TeX{} \Lc{def}-initions
can be converted.
But you may also have problems even when using \LaTeX{} definitions
(with \Lc{newcommand} and \Lc{newenvironment}) if such definitions
make use of sectioning or \Lc{verbatim} commands. These are
handled in a special way by \latextohtml{} and cannot be used in
macro definitions.

In general the macro handling mechanism is inefficient and very
fragile. Avoid using macros if possible.
\cbversion{97.1}\begin{changebar}%
A greater range of macros definitions can now be handled,
especially if appropriate declarations are added to an initialization file.
\end{changebar}%


\item [\Lc{input} commands: ]
There is a bug in the expansion of \Lc{input} commands which causes
a problem
when more than one \Lc{input} command appears on the same line.
There is no quick fix other than suggesting that you put each
\Lc{input} command on a line by itself,
in the \LaTeX{} source files.

\item [\Lc{input} commands in \env{verbatim} environments: ]
\cbversion{98.1}\begin{changebar}
These should no longer cause problems (actually since 97.1).
\Lc{input} commands are also handled correctly within comment environments
declared using \Lc{excludecomment}
and when loading multiple \texttt{.aux} files,
due to use of the \Lc{include} command.
Alternatively you might want to use
either the \env{verbatim} or the \env{verbatimfiles} package.
\end{changebar}

\item [Optional arguments in \env{description} environments: ]
If~you have optional arguments for the \Lc{item} command in
a \env{description} environment containing nested ``\texttt{]}'' characters then
these may not show up correctly. To avoid the problem enclose them
in \verb|{}|s;\\ e.g. \ \verb+\item[{[nested [angle [brackets] are ok]]}]+


\item [\latextohtml{} behaves differently even when you run it on the
same file: ]~\\
If~you notice any strange side-effects from previous runs of \latextohtml{},
try using the option \Cs{no\_reuse} and choose \texttt{(d)} when prompted.
This will clear any intermediate files generated during previous runs.
Note that this option will disable the image-reuse mechanism.

\item [Cannot convert \PS\ images which are included
in the \LaTeX{} file: ]~\\
It~is likely that the macros you are using for including \PS\
files (e.g. \Lc{epsffile}) are not understood by \latextohtml.
To avoid this problem enclose them in an environment which will
be passed to \LaTeX{} anyway; e.g.
\begin{quote}
%begin{latexonly}
\begin{small}
%end{latexonly}
\Lc{begin}\verb|{figure}|\\
\Lc{epsffile}\verb|{|\Meta{PostScript file name}\verb|}|\\
\Lc{end}\verb|{figure}|
%begin{latexonly}
\end{small}
%end{latexonly}
\end{quote}

Another reason why this might happen is that your shell environment variable
\fn{TEXINPUTS} may be undefined. This is not always
fatal but if you have problems you can use full
path-names for included \PS\ files (even when the \PS\
files are in the same directory as the \LaTeX{} source file).
Alternatively try setting \fn{TEXINPUTS} to `\texttt{.::}'.
With some \TeX{} and \LaTeX{} installations setting \fn{TEXINPUTS} to
`\texttt{.::}' may cause problems in the normal operation of \LaTeX.
If you get errors such as \LaTeX{} complaining that it can no longer find
any style files then you must set \fn{TEXINPUTS} to
\verb|"<path to your LaTeX installation>:."|
if you want to use both \LaTeX{} and \latextohtml{}.

\item [Some of the inlined images are in the wrong places: ]
There are several known ways that this may occur.
\begin{itemize}
\item
Perhaps one of the inlined images is more than a page (paper page) long.
This is sometimes the case with very large tables
or large \PS\ images. In this case you can try specifying
a larger paper size (e.g. `\texttt{a4}', `\texttt{a3}' or even `\texttt{a0}')
instead of the default (`\texttt{a5}')
using the \latextohtml{} variable \fn{\$PAPERSIZE}
in the file \fn{latex2html.config}.

\smallskip\noindent
\textit{This error should no longer occur, with \textsc{v97.1}\,.
Please report it on the mailing-list, if it does.}

\item
More likely is that some inappropriate \LaTeX{} code
has caused an error, resulting in an extra page (or pages) being generated.
Examine the \fn{images.log} file, to see if it reports any \LaTeX{} errors.

\item
A much rarer reason is that by default the \fn{dvips} program
reverses the \PS\ pages it generates. If your \fn{dvips} program
behaves in this way try changing the line: \html{\\}
\verb|$DVIPS = "dvips";| \html{\\}
to: \html{\\}
\verb|$DVIPS = "dvips -r0";| \html{\\}
within the file \fn{latex2html.config}.

\item
Yet another reason for images appearing out of place, especially while
developing a document, is that the browser's image cache is providing
out-of-date versions rather than getting the latest version afresh.
When this occurs there will often be images stretched or shrunk
to fit the wrong sized imaging area; this symptom is browser-dependent.
Flushing the cache, then reloading the \texttt{HTML} document, should
clear up the problem.
\end{itemize}


\item [\textbf{Unacceptable quality of converted images: }]
Try changing the size of the image
(see \hyperref{image conversion}{Section~}{}{imgcon}).

\item [The bibliographic references are missing: ]
Run \fn{latex} and then \fn{bibtex} on the original source file in
order to generate a \texttt{.bbl } file.
\latextohtml{} may need the \texttt{.bbl } file in order to generate
the references.


\item [The labels of figures, tables or equations are wrong: ]
This can happen if you have used any figures, tables, equations or
any counters inside conditional text; i.e. in a \env{latexonly}
or a \env{htmlonly} environment.


\item [Problems after changing the configuration files: ]
Please make sure that the last line in the configuration files
(i.e. \fn{.latex2html-init} and \fn{latex2html.config}) is:
\begin{small}
\begin{verbatim}
1;      # This is the last line
\end{verbatim}
\end{small}
This is a \Perl{} quirk.

\item [Problems when producing the \texttt{.dvi} version: \label{htmlsty}]
If you are using any of the new \LaTeX{} commands which are defined in
the \fn{html.sty} file make sure that \fn{html.sty} is included;
e.g. as one of the optional arguments to the \Lc{documentclass} command.

Of course you also have to make sure that \LaTeX{} knows where the \fn{html.sty}
file is, either by putting it in the same place as the other style-files on
your system, or by changing your \fn{TEXINPUTS} shell environment variable
\footnote{If you don't know how to do either of these things, copy (or
link) \fn{html.sty} to the directory of your \LaTeX{} document.}.

\item [Some of the fonts are translated incorrectly: ]
There is a fault in way the \LaTeX{} scoping rules have been
interpreted in \latextohtml. Consider this:
\begin{small}
\begin{verbatim}
\ttfamily fixed-width font.
\begin{something}
nothing here
\end{something}
default font.
\end{verbatim}
\end{small}
When processed by \LaTeX, the effect of the \Lc{tt} command is
delimited by the beginning of the environment ``something'',
so that ``default font'' will appear in the default font.
But \latextohtml{} will not recognise
``something'' as a delimiter and ``default font'' will appear in the
wrong font.

To avoid this problem (until it is fixed) you may delimit the scope of
some commands explicitly using \verb|{}|'s; i.e.
\begin{small}
\begin{verbatim}
\texttt{fixed-width font}.
\begin{something}
nothing here
\end{something}
default font.
\end{verbatim}
\end{small}

\cbversion{98.1}\begin{changebar}
Nesting of font changing commands is now handled correctly.
Such problems should no longer occur, though it always helps
to use explicitly delimited `pseudo'-environments; e.g.
\verb|{\bf ... }|, or \LaTeX's commands requiring an explicit argument,
such as \Lc{textbf}.
\end{changebar}

\item [Cannot get it to generate inlined images: ]
Run \latextohtml{} on your document, using the \Cs{debug} switch.
Look in the directory where the \texttt{HTML} files are generated
for two files named \fn{images.tex} and \fn{images.log}.
Examine their contents.
Do you notice anything unusual in them?

Copy \fn{images.tex} into the directory
of your original \LaTeX{} file and run \fn{latex} on \fn{images.tex}.
Can you see any errors in \fn{images.log}?
If so, can you fix \fn{images.tex} to get rid of the errors?
After fixing \fn{images.tex}
you can put it back in the directory of \texttt{HTML},
then run \latextohtml{} on the original document
using the \Cs{images\_only} switch.

However if you make changes or additions to the original source
then the same problems may occur again.
Thus it is better to understand why the changes were required
and alter your document's source code appropriately.

\medskip

If you get into a mess delete all the image files and run \latextohtml{}
again. Often it is sufficient to just delete the file \fn{images.pl}.

If you \emph{still} get into a mess, try running \latextohtml{} with the options
\Cs{no\_reuse} and \Cs{no\_images}; e.g.
\begin{small}
\begin{verbatim}
cblipca% latex2html -no_reuse -no_images test.tex
This is LaTeX2HTML Version 95 (Tue Nov 29 1994) by Nikos Drakos,
Computer Based Learning Unit, University of Leeds.

OPENING /tmp_mnt/home/cblelca/nikos/tmp/test.tex
Cannot create directory /usr/cblelca/nikos/tmp/test: File exists
(r) Reuse the images in the old directory OR
(d) *** DELETE *** /usr/cblelca/nikos/tmp/test AND ITS CONTENTS OR
(q) Quit ?
:d

Reading ...
Processing macros ....+.
Reading test.aux ......................
Translating ...0/1........1/1.....
Writing image file ...

Doing section links .....

*********** WARNINGS ***********

If you are having problems displaying the correct images with Mosaic,
try selecting "Flush Image Cache" from "Options" in the menu-bar
and then reload the HTML file.

Done.
\end{verbatim}
\end{small}
Then try to have a look in the file  \fn{images.tex}
(as described earlier) and perhaps fix it.
Once you are happy that \fn{images.tex} is OK, run \latextohtml{}
again with the option \Cs{images\_only}.

Some problems in displaying the correct inlined images,
may be due to the image caching mechanisms of your browser.
With some browsers a simple ``\textbf{Reload Current Document}'' will be enough
to refresh the images but with others (e.g. \appl{Mosaic}) you may need
to request for the cache to be refreshed. With \appl{Mosaic} try
selecting ``\textbf{Flush Image Cache}'' from ``\textbf{Options}'' in the menu-bar
and then reload the \texttt{HTML} file.


\item [It cannot do slides, memos, etc. ]
If you use Sli\TeX{} you can go a long way just by replacing
the \verb|{slides}| argument of the \Lc{documentclass} command with
something like \verb|{article}| just before using \latextohtml.
One problem may be that all your slides will end up in the  same \texttt{HTML}
file.
If you use \fn{lslide.sty} you may get much better results
(\htmladdnormallinkfoot{use \appl{Archie}}{http://www.pvv.unit.no/archie/}
to find this or any other style files).
\end{htmllist}
\index{problems|)}

    % Input counters and section
\end{htmlonly}
\internal{}%
\startdocument
%
%\section{Known Problems\label{sec:prb}\index{problems|(}}%
\label{sec:prb}\index{problems|(}%
\tableofchildlinks*\htmlrule\index{bugs}\html{\\}%
\noindent
Here are some of the problems that were known to exist
with previous versions of \latextohtml.
Most of those that were real errors are either fixed completely
in the current version (\textsc{v98.1}), or are much less likely to occur
within correct \LaTeX{} source. (Some are not really errors but
indications of poor style in the user's choices among
various ways to organise their source code.)\html{\\}
Several are indeed limitations inherent in the way
\latextohtml{} currently performs its processing.
%
\begin{htmllist}
\htmlitemmark{PurpleBall}
\item [Correctness and Efficiency:\index{efficiency} ]
The translator cannot be guaranteed to perform as expected.
Several aspects of the implementation need
optimisation and improvement. Apart from possible bugs the translator
may place heavy demands on your resources.
\cbversion{97.1}\begin{changebar}
%begin{latexonly}
\textit
%end{latexonly}
{The current version works much more efficiently than previous
versions; many subtle bugs have been identified and eliminated.}
\end{changebar}
\cbversion{98.1}\begin{changebar}
%begin{latexonly}
\textit
%end{latexonly}
{The process of command substitution has been improved significantly,
resulting in memory savings and faster translation.}
\end{changebar}

\item [Unrecognised Commands and Environments:\index{unrecognised commands} ]
Unrecognised commands are ignored and any arguments are left in the text.
Unrecognised environments are passed to \LaTeX{}  and the result is
included in the document as one or more inlined images.
\cbversion{97.1}\begin{changebar}
%begin{latexonly}
\newline\textit
%end{latexonly}
{There are very few standard \LaTeX{} commands that are not recognised.
Many common \TeX{} commands are recognised also,
even though not explicitly mentioned
in the \LaTeX{} \hypercite{blue book}{blue book}{}{lamp:latex}.
Any aberrant commands should be reported to the \latextohtml{}
\hyperref{mailing list}{mailing list, see Section~}{}{support}.}
\end{changebar}


\item [Cross-references:\index{cross-references} ]
References in environments that are passed to \LaTeX{}  for processing
(e.g. a \Lc{cite}, or a \Lc{ref} command),
are not processed correctly.
\Lc{label} commands are handled correctly.
\cbversion{97.1}\begin{changebar}
%begin{latexonly}
\newline\textit
%end{latexonly}
{All citation, reference and label commands should work correctly now.
Report any problems to the \latextohtml{}
\hyperref{mailing list}{mailing list, see Section~}{}{support}.}
\end{changebar}

\item[Order-Sensitive Commands: ]
Commands which affect global parameters during the translation,
and are sensitive to the order in which they are processed may
not be handled correctly. In particular, counter manipulation
(e.g. \Lc{newcounter}, \Lc{setcounter}, \Lc{stepcounter}, etc.)
commands may cause problems.
\cbversion{97.1}\begin{changebar}
%begin{latexonly}
\newline\textit
%end{latexonly}
{Counter commands now work correctly; dependencies
are also implemented.}
\end{changebar}

\index{index}
\item [Index: ]
The translator generates its own index by saving the arguments  of
the \Lc{index} command. The contents of the \env{theindex}
environment are ignored.
\cbversion{97.1}\begin{changebar}
%begin{latexonly}
\newline\textit
%end{latexonly}
{This remains true. When using the \env{makeidx} package,
very sophisticated Indexes can be built automatically.
The Index for this manual is a good example.}
\end{changebar}

\item[New Definitions:\index{extensions!definitions} ]
New definitions (\Lc{newcommand}, \Lc{newenvironment},
\Lc{newtheorem} and \Lc{def}),
will not work as expected if they are defined more than once.
Only the last definition will be used throughout the document.
\cbversion{97.1}\begin{changebar}
%begin{latexonly}
\newline\textit
%end{latexonly}
{This remains true. Stylistically it is bad to declare new
environments or theorems outside of the document preamble,
so these should cause no problems anyway.\html{\\}
Changes to commands using \Lc{def} or \Lc{renewcommand}
should usually be made only locally, within special environments,
to set a needed parameter;
e.g. a basic length in a \env{picture} environment.
But when such environments force an image to be generated,
then \LaTeX{} will make the correct redefinition.}
\end{changebar}

\item [Scope of declarations and environments: ]
If the scope of a declaration or environment crosses section
boundaries, then the output may not be as expected, because each
section is processed independently.
\cbversion{97.1}\begin{changebar}
%begin{latexonly}
\newline\textit
%end{latexonly}
{This is inherent to the way \latextohtml{} does its processing.
It will not be fixed until later versions change this strategy;
e.g. when \latextohtmlNG\ becomes fully integrated.}
\end{changebar}

\item [Math-mode font-size changes: ]  Math-mode font changes
made outside the math-mode are not honoured.  Thus the two equations
in \verb|$a_b$| and \verb|{\LARGE $a_b$}|
would come out looking the same.  The trick is to write
\verb|$a_b| and \verb|$\mbox{\LARGE $a_b$}$|.
\cbversion{97.1}\begin{changebar}
%begin{latexonly}
\newline\textit
%end{latexonly}
{This remains. The work-around is effective.}
\end{changebar}
\end{htmllist}



\subsection{Troubleshooting}%
%\section{Troubleshooting}%
\index{debugging}\index{problems}\index{fixes}%
Here are some curable symptoms:

\begin{htmllist}
\htmlitemmark{BlueBall}
\item [Cannot run any of the \Perl{} programs: ]
If your \Perl{} installation is such that \Perl{} programs are not allowed
to run as shell scripts you may be unable to run
\fn{latex2html}, \fn{texexpand}, \fn{pstoimg} and  \fn{install-test}.
In this case change the first line in each of these
programs from
%begin{latexonly}
\begin{small}
%end{latexonly}
\begin{verbatim}
#!/usr/local/bin/perl
\end{verbatim}
%begin{latexonly}
\end{small}%
%end{latexonly}
to
%begin{latexonly}
\begin{small}%
%end{latexonly}
\begin{verbatim}
: # *-*-perl-*-*
    eval 'exec perl -S  $0 "$@"'
    if $running_under_some_shell;
\end{verbatim}
%begin{latexonly}
\end{small}
%end{latexonly}

\item [The \fn{install-test} script gives uninformative error messages: ]
If, for any reason, you have trouble running \fn{install-test},
do not despair. Most of what it does is to do with checking
your installation rather than actually installing anything.
To do a \textit{manual installation} just change the variable
\fn{\$LATEX2HTMLDIR} in the beginning of the file \fn{latex2html}
to point to the directory where the \latextohtml{} files can be found.

Also, make sure that the files
\fn{pstoimg}, \fn{texexpand} and \fn{latex2html} are executable;
if necessary use the Unix \fn{chmod} command to make them
executable.

\item [It just stops. ] Check the style
files that you are using. It is possible that you are using
a style file which contains raw \TeX{} commands. In such a case
start \latextohtml{} with the option
\Cs{dont\_include \Meta{style-file name}}.
Alternatively, add the name of the style to the variable
\fn{\$DONT\_INCLUDE} in your
\fn{\$HOME/.latex2html-init} file. If you don't have such a file then
create one and add the lines:
\begin{quote}
%begin{latexonly}
\begin{small}
%end{latexonly}
\verb|$DONT_INCLUDE = "$DONT_INCLUDE" . ":|\Meta{style file name}\verb|";|\\
\verb|1;        # This must be the last line|
%begin{latexonly}
\end{small}
%end{latexonly}
\end{quote}

Another reason why \latextohtml{} might stop is that the \LaTeX{} source
file itself contains raw \TeX{} commands. In this case you may
put such commands inside a \env{latexonly}
\hyperref{environment}{environment (see Section }{)}{sec:latexonly}.

\cbversion{97.1}\begin{changebar}
The \fn{\$VERBOSITY} \htmlref{variable}{verbosity}{} can be used
to create tracing messages, which may help to locate which command
or environment was being processed when everything stopped.
\end{changebar}

\index{unending loop}\index{looping endlessly}
\cbversion{97.1}\begin{changebar}
\item [It appears to be doing nothing.\label{looping}]
Perhaps the processor has fallen into an unending loop.
Usually there will be a bad definition, or other faulty source code,
which has caused this.
The \fn{\$VERBOSITY} \hyperref[page]{variable}{variable (see page~}{)}{cs_verbositylevel}%
can be set to generate tracing messages, which may help to locate which command
or environment is being processed repeatedly.
Try setting a value of `\texttt{3}'; e.g. using the commandline switch \Cs{verbosity 3}.
This will print command and environment names, as thaey are processed.
It should soon become apparent where any such looping occurs.

\item [It just fills the endlessly with dots.]
No `perhaps' here; the processor has definitely fallen into an unending loop.
See the previous item for how to detect which command or environment
is causing the problem.
\end{changebar}

\cbversion{98.1}\begin{changebar}
%
\item [\Perl{} cannot parse the \fn{latex2html} script: ]
If \Perl{} refuses to start \latextohtml{} and issues errors, your
\Perl{} version is not up to date.
Update your \Perl{} to 5.003 or later. You can check which version of
\Perl{} you are using by invoking \Perl{} with the \Cs{v} option.

If \Perl{} issues errors during runtime, this is most probably related
to bugs within \latextohtml{} or one of its modules.
In this case you will need help from the developers or experienced users;
this can be obtained via the discussion list.


\item [It crashes (dumps core) as soon as it starts\label{perl}: ]
Update your \Perl{} to 5.003 or later.
\end{changebar}


\cbversion{98.1}\begin{changebar}
\item [It does not show any of your images: ]
  You can't run \latextohtml{} in a subdirectory that contains a dot
  within the directory name, such as {\tt latex2html-98.1},
  or in name of any higher directory.
  This is because \fn{dvips}'s \texttt{-o} option will change 98.1 into 98.001
  and use that as the resulting output file, instead of image001~.
  The \PS\ files will be placed higher up in the directory tree.

For instance, if \texttt{ pwd } returns something like:
\begin{verbatim}
/usr/people/nelson/files/latex2html-98.1/work/tests
\end{verbatim}
and you run \latextohtml, then \fn{dvips} will generate
image output files here:
\begin{verbatim}
/usr/people/nelson/files
\end{verbatim}
  called \texttt{latex2html-98.001}, \texttt{latex2html-98.002}, \dots
  instead of \texttt{image001}, \texttt{image002}, \texttt{image003}, \dots
  in the subdirectory where your \texttt{.html} files were created.
  As a result the images will not show in your documents.

  If you are getting `File Not Found' errors,
  then reprocess your job using the \Cs{debug} switch,
  to see what options are passed to \fn{dvips}.
  If there is a `.' in some parts of any directory name,
  then look above that directory
  to see if files are being generated there.

\medskip

  One obvious fix is to rename the offending directory
  to remove the `.' from its name.

  If that is not possible, then define an alternative location
  for image generation to take place; set \texttt{\$TMP} to contain
  the name for this location. Typically \texttt{\$TMP = '/usr/tmp';}~.
  (This use of \texttt{\$TMP} is a good thing to do anyway,
  especially if your Unix account is subject to quota limitations.)

\item [It stops after having run \LaTeX, displaying a message about \fn{dvips}: ]
  See the previous item for the reason and `fix'.
  This version of \latextohtml{} detects the problem
  and will not allow images to be generated in the wrong place.
\end{changebar}

\item [\fn{dvips} complains about incorrect arguments:\label{dvips} ]
Please use a version which supports the command-line options \Cs{M}, \Cs{S},
\Cs{o} and \Cs{i}. ``Recent'' versions, at least after 5.516, do support them.


\item [It gives an ``\texttt{Out of memory}'' message and dies: ]
Try splitting your source file into more than one file,
using the \LaTeX{} commands \Lc{input} or \Lc{include}.
Also, try using the \Cs{no\_images} option.

\cbversion{97.1}\begin{changebar}
Perhaps the processor has fallen into an infinite loop.
Usually there will be a bad definition, or other faulty source code,
which has caused this.
See an \htmlref{earlier problem}{looping} for
how to set the \fn{\$VERBOSITY} \htmlref{variable}{verbosity} to
help locate the bad code leading to this memory exhaustion.
\end{changebar}

As a last resort you may consider increasing the virtual memory
(swap space) of your machine. As an indication
of what you might be able to do on your machine,
a very long book (about 1000 printed pages) required about
24MB of RAM and over 150MB of swap space to convert on a local Sun Sparc ELC
running SunOS 4.1.3.
\cbversion{97.1}\begin{changebar}
%begin{latexonly}
\newline\textit
%end{latexonly}%
{Much of this memory would have been consumed during image-generation.
This part of the processing is much more efficient in \textsc{v97.1}\,.}
\end{changebar}%


\cbversion{98.1}\begin{changebar}
\item [\fn{install-test} issues ``dbm'' related error messages: ]\latex{~\\}%
\latextohtml{} requires a DataBase Management system (\appl{NDBM},
\appl{GDBM}, or \appl{SDBM}) in order to run.
This is usually part of each Unix-like operating system and
\appl{SDBM} is part of \Perl{ 5}, but sometimes this is either missing
on your operating system or left out in a binary \Perl{} distribution.
\htmladdnormallinkfoot{Use \appl{Archie}}{http://www.pvv.unit.no/archie/}
to find one or (better) update to a complete \Perl{ 5} distribution.

\item [\fn{latex2html} issues ``dbm'' related error messages: ]
If you get warnings like
\begin{verbatim}
ndbm store returned -1, errno 28, key "xyz" at latex2html line 123
\end{verbatim}
this is related to an overflow of \latextohtml{} internals.
You will need help from the list, here.
\smallskip

If you get real error messages which cause \latextohtml{} to abort,
run \fn{install-test} to check if your DataBase management works.
You will probably need to re-install \Perl{ 5} (see above topic).
\end{changebar}


\cbversion{97.1}\begin{changebar}
This can happen when an image is being created from a large piece
of \LaTeX{} source code. The image-reuse mechanism uses the code itself
to construct a database key. If too long, the key is invalid and may
crash \appl{DBM} or \appl{NDBM}. (In fact this error should no longer
occur in \textsc{v97.1}\,, so please advise the \latextohtml{} developers if
you get error messages of this kind.)\html{\\}
The message should contain the name of environment which caused the problem,
along with an identifying number; e.g. \texttt{eqnarray268}.
To find which exact piece of code this represents, run \latextohtml{}
again, using the \Cs{debug} switch. Then look at the files
in the \texttt{TMP} subdirectory of the working directory
named \texttt{TMP/part001}, \texttt{TMP/part002}, etc.
Use the unix \fn{grep} command: \texttt{ grep 268 }\Meta{dir}\texttt{/TMP/part* }
to find that number in these files.
This should enable you to locate exactly where the problem occurs.

One solution may be to wrap the whole environment within
\Lc{begin\char123makeimage\char125} and \Lc{end\char123makeimage\char125}.
This will still cause the image to be created,
but uses just the environment name and number as the database key.
\end{changebar}%



\item [The \Lc{verb"ABC"} command doesn't work: ]
This is an unfortunate bug which can be avoided by using
any character other than quotes; e.g. \Lc{verb+ABC+}.

\item [Cannot get the ``tilde'' (\~{}) to show: \hfill]
The trick here is to use \verb|\~{}|.\html{\\}
Alternatively try using something like:
\url{\htmladdnormallink{mylink}{http://host/\%7Eme/path/file.html}}.

\textbf{Warning:} Some browsers may not be able to interpret the \verb|%7E|
as a ``tilde'' character.

\cbversion{98.1}\begin{changebar}
Try using \Lc{char126}~.
In any case tildes within \Lc{htmladdnormallink} and similar commands
are now handled correctly.
\end{changebar}

\item [Macro definitions don't work correctly: ]
As mentioned in other places, not all plain \TeX{} \Lc{def}-initions
can be converted.
But you may also have problems even when using \LaTeX{} definitions
(with \Lc{newcommand} and \Lc{newenvironment}) if such definitions
make use of sectioning or \Lc{verbatim} commands. These are
handled in a special way by \latextohtml{} and cannot be used in
macro definitions.

In general the macro handling mechanism is inefficient and very
fragile. Avoid using macros if possible.
\cbversion{97.1}\begin{changebar}%
A greater range of macros definitions can now be handled,
especially if appropriate declarations are added to an initialization file.
\end{changebar}%


\item [\Lc{input} commands: ]
There is a bug in the expansion of \Lc{input} commands which causes
a problem
when more than one \Lc{input} command appears on the same line.
There is no quick fix other than suggesting that you put each
\Lc{input} command on a line by itself,
in the \LaTeX{} source files.

\item [\Lc{input} commands in \env{verbatim} environments: ]
\cbversion{98.1}\begin{changebar}
These should no longer cause problems (actually since 97.1).
\Lc{input} commands are also handled correctly within comment environments
declared using \Lc{excludecomment}
and when loading multiple \texttt{.aux} files,
due to use of the \Lc{include} command.
Alternatively you might want to use
either the \env{verbatim} or the \env{verbatimfiles} package.
\end{changebar}

\item [Optional arguments in \env{description} environments: ]
If~you have optional arguments for the \Lc{item} command in
a \env{description} environment containing nested ``\texttt{]}'' characters then
these may not show up correctly. To avoid the problem enclose them
in \verb|{}|s;\\ e.g. \ \verb+\item[{[nested [angle [brackets] are ok]]}]+


\item [\latextohtml{} behaves differently even when you run it on the
same file: ]~\\
If~you notice any strange side-effects from previous runs of \latextohtml{},
try using the option \Cs{no\_reuse} and choose \texttt{(d)} when prompted.
This will clear any intermediate files generated during previous runs.
Note that this option will disable the image-reuse mechanism.

\item [Cannot convert \PS\ images which are included
in the \LaTeX{} file: ]~\\
It~is likely that the macros you are using for including \PS\
files (e.g. \Lc{epsffile}) are not understood by \latextohtml.
To avoid this problem enclose them in an environment which will
be passed to \LaTeX{} anyway; e.g.
\begin{quote}
%begin{latexonly}
\begin{small}
%end{latexonly}
\Lc{begin}\verb|{figure}|\\
\Lc{epsffile}\verb|{|\Meta{PostScript file name}\verb|}|\\
\Lc{end}\verb|{figure}|
%begin{latexonly}
\end{small}
%end{latexonly}
\end{quote}

Another reason why this might happen is that your shell environment variable
\fn{TEXINPUTS} may be undefined. This is not always
fatal but if you have problems you can use full
path-names for included \PS\ files (even when the \PS\
files are in the same directory as the \LaTeX{} source file).
Alternatively try setting \fn{TEXINPUTS} to `\texttt{.::}'.
With some \TeX{} and \LaTeX{} installations setting \fn{TEXINPUTS} to
`\texttt{.::}' may cause problems in the normal operation of \LaTeX.
If you get errors such as \LaTeX{} complaining that it can no longer find
any style files then you must set \fn{TEXINPUTS} to
\verb|"<path to your LaTeX installation>:."|
if you want to use both \LaTeX{} and \latextohtml{}.

\item [Some of the inlined images are in the wrong places: ]
There are several known ways that this may occur.
\begin{itemize}
\item
Perhaps one of the inlined images is more than a page (paper page) long.
This is sometimes the case with very large tables
or large \PS\ images. In this case you can try specifying
a larger paper size (e.g. `\texttt{a4}', `\texttt{a3}' or even `\texttt{a0}')
instead of the default (`\texttt{a5}')
using the \latextohtml{} variable \fn{\$PAPERSIZE}
in the file \fn{latex2html.config}.

\smallskip\noindent
\textit{This error should no longer occur, with \textsc{v97.1}\,.
Please report it on the mailing-list, if it does.}

\item
More likely is that some inappropriate \LaTeX{} code
has caused an error, resulting in an extra page (or pages) being generated.
Examine the \fn{images.log} file, to see if it reports any \LaTeX{} errors.

\item
A much rarer reason is that by default the \fn{dvips} program
reverses the \PS\ pages it generates. If your \fn{dvips} program
behaves in this way try changing the line: \html{\\}
\verb|$DVIPS = "dvips";| \html{\\}
to: \html{\\}
\verb|$DVIPS = "dvips -r0";| \html{\\}
within the file \fn{latex2html.config}.

\item
Yet another reason for images appearing out of place, especially while
developing a document, is that the browser's image cache is providing
out-of-date versions rather than getting the latest version afresh.
When this occurs there will often be images stretched or shrunk
to fit the wrong sized imaging area; this symptom is browser-dependent.
Flushing the cache, then reloading the \texttt{HTML} document, should
clear up the problem.
\end{itemize}


\item [\textbf{Unacceptable quality of converted images: }]
Try changing the size of the image
(see \hyperref{image conversion}{Section~}{}{imgcon}).

\item [The bibliographic references are missing: ]
Run \fn{latex} and then \fn{bibtex} on the original source file in
order to generate a \texttt{.bbl } file.
\latextohtml{} may need the \texttt{.bbl } file in order to generate
the references.


\item [The labels of figures, tables or equations are wrong: ]
This can happen if you have used any figures, tables, equations or
any counters inside conditional text; i.e. in a \env{latexonly}
or a \env{htmlonly} environment.


\item [Problems after changing the configuration files: ]
Please make sure that the last line in the configuration files
(i.e. \fn{.latex2html-init} and \fn{latex2html.config}) is:
\begin{small}
\begin{verbatim}
1;      # This is the last line
\end{verbatim}
\end{small}
This is a \Perl{} quirk.

\item [Problems when producing the \texttt{.dvi} version: \label{htmlsty}]
If you are using any of the new \LaTeX{} commands which are defined in
the \fn{html.sty} file make sure that \fn{html.sty} is included;
e.g. as one of the optional arguments to the \Lc{documentclass} command.

Of course you also have to make sure that \LaTeX{} knows where the \fn{html.sty}
file is, either by putting it in the same place as the other style-files on
your system, or by changing your \fn{TEXINPUTS} shell environment variable
\footnote{If you don't know how to do either of these things, copy (or
link) \fn{html.sty} to the directory of your \LaTeX{} document.}.

\item [Some of the fonts are translated incorrectly: ]
There is a fault in way the \LaTeX{} scoping rules have been
interpreted in \latextohtml. Consider this:
\begin{small}
\begin{verbatim}
\ttfamily fixed-width font.
\begin{something}
nothing here
\end{something}
default font.
\end{verbatim}
\end{small}
When processed by \LaTeX, the effect of the \Lc{tt} command is
delimited by the beginning of the environment ``something'',
so that ``default font'' will appear in the default font.
But \latextohtml{} will not recognise
``something'' as a delimiter and ``default font'' will appear in the
wrong font.

To avoid this problem (until it is fixed) you may delimit the scope of
some commands explicitly using \verb|{}|'s; i.e.
\begin{small}
\begin{verbatim}
\texttt{fixed-width font}.
\begin{something}
nothing here
\end{something}
default font.
\end{verbatim}
\end{small}

\cbversion{98.1}\begin{changebar}
Nesting of font changing commands is now handled correctly.
Such problems should no longer occur, though it always helps
to use explicitly delimited `pseudo'-environments; e.g.
\verb|{\bf ... }|, or \LaTeX's commands requiring an explicit argument,
such as \Lc{textbf}.
\end{changebar}

\item [Cannot get it to generate inlined images: ]
Run \latextohtml{} on your document, using the \Cs{debug} switch.
Look in the directory where the \texttt{HTML} files are generated
for two files named \fn{images.tex} and \fn{images.log}.
Examine their contents.
Do you notice anything unusual in them?

Copy \fn{images.tex} into the directory
of your original \LaTeX{} file and run \fn{latex} on \fn{images.tex}.
Can you see any errors in \fn{images.log}?
If so, can you fix \fn{images.tex} to get rid of the errors?
After fixing \fn{images.tex}
you can put it back in the directory of \texttt{HTML},
then run \latextohtml{} on the original document
using the \Cs{images\_only} switch.

However if you make changes or additions to the original source
then the same problems may occur again.
Thus it is better to understand why the changes were required
and alter your document's source code appropriately.

\medskip

If you get into a mess delete all the image files and run \latextohtml{}
again. Often it is sufficient to just delete the file \fn{images.pl}.

If you \emph{still} get into a mess, try running \latextohtml{} with the options
\Cs{no\_reuse} and \Cs{no\_images}; e.g.
\begin{small}
\begin{verbatim}
cblipca% latex2html -no_reuse -no_images test.tex
This is LaTeX2HTML Version 95 (Tue Nov 29 1994) by Nikos Drakos,
Computer Based Learning Unit, University of Leeds.

OPENING /tmp_mnt/home/cblelca/nikos/tmp/test.tex
Cannot create directory /usr/cblelca/nikos/tmp/test: File exists
(r) Reuse the images in the old directory OR
(d) *** DELETE *** /usr/cblelca/nikos/tmp/test AND ITS CONTENTS OR
(q) Quit ?
:d

Reading ...
Processing macros ....+.
Reading test.aux ......................
Translating ...0/1........1/1.....
Writing image file ...

Doing section links .....

*********** WARNINGS ***********

If you are having problems displaying the correct images with Mosaic,
try selecting "Flush Image Cache" from "Options" in the menu-bar
and then reload the HTML file.

Done.
\end{verbatim}
\end{small}
Then try to have a look in the file  \fn{images.tex}
(as described earlier) and perhaps fix it.
Once you are happy that \fn{images.tex} is OK, run \latextohtml{}
again with the option \Cs{images\_only}.

Some problems in displaying the correct inlined images,
may be due to the image caching mechanisms of your browser.
With some browsers a simple ``\textbf{Reload Current Document}'' will be enough
to refresh the images but with others (e.g. \appl{Mosaic}) you may need
to request for the cache to be refreshed. With \appl{Mosaic} try
selecting ``\textbf{Flush Image Cache}'' from ``\textbf{Options}'' in the menu-bar
and then reload the \texttt{HTML} file.


\item [It cannot do slides, memos, etc. ]
If you use Sli\TeX{} you can go a long way just by replacing
the \verb|{slides}| argument of the \Lc{documentclass} command with
something like \verb|{article}| just before using \latextohtml.
One problem may be that all your slides will end up in the  same \texttt{HTML}
file.
If you use \fn{lslide.sty} you may get much better results
(\htmladdnormallinkfoot{use \appl{Archie}}{http://www.pvv.unit.no/archie/}
to find this or any other style files).
\end{htmllist}
\index{problems|)}

    % Input counters and section
\end{htmlonly}
\internal{}%
\startdocument
%
%\section{Known Problems\label{sec:prb}\index{problems|(}}%
\label{sec:prb}\index{problems|(}%
\tableofchildlinks*\htmlrule\index{bugs}\html{\\}%
\noindent
Here are some of the problems that were known to exist
with previous versions of \latextohtml.
Most of those that were real errors are either fixed completely
in the current version (\textsc{v98.1}), or are much less likely to occur
within correct \LaTeX{} source. (Some are not really errors but
indications of poor style in the user's choices among
various ways to organise their source code.)\html{\\}
Several are indeed limitations inherent in the way
\latextohtml{} currently performs its processing.
%
\begin{htmllist}
\htmlitemmark{PurpleBall}
\item [Correctness and Efficiency:\index{efficiency} ]
The translator cannot be guaranteed to perform as expected.
Several aspects of the implementation need
optimisation and improvement. Apart from possible bugs the translator
may place heavy demands on your resources.
\cbversion{97.1}\begin{changebar}
%begin{latexonly}
\textit
%end{latexonly}
{The current version works much more efficiently than previous
versions; many subtle bugs have been identified and eliminated.}
\end{changebar}
\cbversion{98.1}\begin{changebar}
%begin{latexonly}
\textit
%end{latexonly}
{The process of command substitution has been improved significantly,
resulting in memory savings and faster translation.}
\end{changebar}

\item [Unrecognised Commands and Environments:\index{unrecognised commands} ]
Unrecognised commands are ignored and any arguments are left in the text.
Unrecognised environments are passed to \LaTeX{}  and the result is
included in the document as one or more inlined images.
\cbversion{97.1}\begin{changebar}
%begin{latexonly}
\newline\textit
%end{latexonly}
{There are very few standard \LaTeX{} commands that are not recognised.
Many common \TeX{} commands are recognised also,
even though not explicitly mentioned
in the \LaTeX{} \hypercite{blue book}{blue book}{}{lamp:latex}.
Any aberrant commands should be reported to the \latextohtml{}
\hyperref{mailing list}{mailing list, see Section~}{}{support}.}
\end{changebar}


\item [Cross-references:\index{cross-references} ]
References in environments that are passed to \LaTeX{}  for processing
(e.g. a \Lc{cite}, or a \Lc{ref} command),
are not processed correctly.
\Lc{label} commands are handled correctly.
\cbversion{97.1}\begin{changebar}
%begin{latexonly}
\newline\textit
%end{latexonly}
{All citation, reference and label commands should work correctly now.
Report any problems to the \latextohtml{}
\hyperref{mailing list}{mailing list, see Section~}{}{support}.}
\end{changebar}

\item[Order-Sensitive Commands: ]
Commands which affect global parameters during the translation,
and are sensitive to the order in which they are processed may
not be handled correctly. In particular, counter manipulation
(e.g. \Lc{newcounter}, \Lc{setcounter}, \Lc{stepcounter}, etc.)
commands may cause problems.
\cbversion{97.1}\begin{changebar}
%begin{latexonly}
\newline\textit
%end{latexonly}
{Counter commands now work correctly; dependencies
are also implemented.}
\end{changebar}

\index{index}
\item [Index: ]
The translator generates its own index by saving the arguments  of
the \Lc{index} command. The contents of the \env{theindex}
environment are ignored.
\cbversion{97.1}\begin{changebar}
%begin{latexonly}
\newline\textit
%end{latexonly}
{This remains true. When using the \env{makeidx} package,
very sophisticated Indexes can be built automatically.
The Index for this manual is a good example.}
\end{changebar}

\item[New Definitions:\index{extensions!definitions} ]
New definitions (\Lc{newcommand}, \Lc{newenvironment},
\Lc{newtheorem} and \Lc{def}),
will not work as expected if they are defined more than once.
Only the last definition will be used throughout the document.
\cbversion{97.1}\begin{changebar}
%begin{latexonly}
\newline\textit
%end{latexonly}
{This remains true. Stylistically it is bad to declare new
environments or theorems outside of the document preamble,
so these should cause no problems anyway.\html{\\}
Changes to commands using \Lc{def} or \Lc{renewcommand}
should usually be made only locally, within special environments,
to set a needed parameter;
e.g. a basic length in a \env{picture} environment.
But when such environments force an image to be generated,
then \LaTeX{} will make the correct redefinition.}
\end{changebar}

\item [Scope of declarations and environments: ]
If the scope of a declaration or environment crosses section
boundaries, then the output may not be as expected, because each
section is processed independently.
\cbversion{97.1}\begin{changebar}
%begin{latexonly}
\newline\textit
%end{latexonly}
{This is inherent to the way \latextohtml{} does its processing.
It will not be fixed until later versions change this strategy;
e.g. when \latextohtmlNG\ becomes fully integrated.}
\end{changebar}

\item [Math-mode font-size changes: ]  Math-mode font changes
made outside the math-mode are not honoured.  Thus the two equations
in \verb|$a_b$| and \verb|{\LARGE $a_b$}|
would come out looking the same.  The trick is to write
\verb|$a_b| and \verb|$\mbox{\LARGE $a_b$}$|.
\cbversion{97.1}\begin{changebar}
%begin{latexonly}
\newline\textit
%end{latexonly}
{This remains. The work-around is effective.}
\end{changebar}
\end{htmllist}



\subsection{Troubleshooting}%
%\section{Troubleshooting}%
\index{debugging}\index{problems}\index{fixes}%
Here are some curable symptoms:

\begin{htmllist}
\htmlitemmark{BlueBall}
\item [Cannot run any of the \Perl{} programs: ]
If your \Perl{} installation is such that \Perl{} programs are not allowed
to run as shell scripts you may be unable to run
\fn{latex2html}, \fn{texexpand}, \fn{pstoimg} and  \fn{install-test}.
In this case change the first line in each of these
programs from
%begin{latexonly}
\begin{small}
%end{latexonly}
\begin{verbatim}
#!/usr/local/bin/perl
\end{verbatim}
%begin{latexonly}
\end{small}%
%end{latexonly}
to
%begin{latexonly}
\begin{small}%
%end{latexonly}
\begin{verbatim}
: # *-*-perl-*-*
    eval 'exec perl -S  $0 "$@"'
    if $running_under_some_shell;
\end{verbatim}
%begin{latexonly}
\end{small}
%end{latexonly}

\item [The \fn{install-test} script gives uninformative error messages: ]
If, for any reason, you have trouble running \fn{install-test},
do not despair. Most of what it does is to do with checking
your installation rather than actually installing anything.
To do a \textit{manual installation} just change the variable
\fn{\$LATEX2HTMLDIR} in the beginning of the file \fn{latex2html}
to point to the directory where the \latextohtml{} files can be found.

Also, make sure that the files
\fn{pstoimg}, \fn{texexpand} and \fn{latex2html} are executable;
if necessary use the Unix \fn{chmod} command to make them
executable.

\item [It just stops. ] Check the style
files that you are using. It is possible that you are using
a style file which contains raw \TeX{} commands. In such a case
start \latextohtml{} with the option
\Cs{dont\_include \Meta{style-file name}}.
Alternatively, add the name of the style to the variable
\fn{\$DONT\_INCLUDE} in your
\fn{\$HOME/.latex2html-init} file. If you don't have such a file then
create one and add the lines:
\begin{quote}
%begin{latexonly}
\begin{small}
%end{latexonly}
\verb|$DONT_INCLUDE = "$DONT_INCLUDE" . ":|\Meta{style file name}\verb|";|\\
\verb|1;        # This must be the last line|
%begin{latexonly}
\end{small}
%end{latexonly}
\end{quote}

Another reason why \latextohtml{} might stop is that the \LaTeX{} source
file itself contains raw \TeX{} commands. In this case you may
put such commands inside a \env{latexonly}
\hyperref{environment}{environment (see Section }{)}{sec:latexonly}.

\cbversion{97.1}\begin{changebar}
The \fn{\$VERBOSITY} \htmlref{variable}{verbosity}{} can be used
to create tracing messages, which may help to locate which command
or environment was being processed when everything stopped.
\end{changebar}

\index{unending loop}\index{looping endlessly}
\cbversion{97.1}\begin{changebar}
\item [It appears to be doing nothing.\label{looping}]
Perhaps the processor has fallen into an unending loop.
Usually there will be a bad definition, or other faulty source code,
which has caused this.
The \fn{\$VERBOSITY} \hyperref[page]{variable}{variable (see page~}{)}{cs_verbositylevel}%
can be set to generate tracing messages, which may help to locate which command
or environment is being processed repeatedly.
Try setting a value of `\texttt{3}'; e.g. using the commandline switch \Cs{verbosity 3}.
This will print command and environment names, as thaey are processed.
It should soon become apparent where any such looping occurs.

\item [It just fills the endlessly with dots.]
No `perhaps' here; the processor has definitely fallen into an unending loop.
See the previous item for how to detect which command or environment
is causing the problem.
\end{changebar}

\cbversion{98.1}\begin{changebar}
%
\item [\Perl{} cannot parse the \fn{latex2html} script: ]
If \Perl{} refuses to start \latextohtml{} and issues errors, your
\Perl{} version is not up to date.
Update your \Perl{} to 5.003 or later. You can check which version of
\Perl{} you are using by invoking \Perl{} with the \Cs{v} option.

If \Perl{} issues errors during runtime, this is most probably related
to bugs within \latextohtml{} or one of its modules.
In this case you will need help from the developers or experienced users;
this can be obtained via the discussion list.


\item [It crashes (dumps core) as soon as it starts\label{perl}: ]
Update your \Perl{} to 5.003 or later.
\end{changebar}


\cbversion{98.1}\begin{changebar}
\item [It does not show any of your images: ]
  You can't run \latextohtml{} in a subdirectory that contains a dot
  within the directory name, such as {\tt latex2html-98.1},
  or in name of any higher directory.
  This is because \fn{dvips}'s \texttt{-o} option will change 98.1 into 98.001
  and use that as the resulting output file, instead of image001~.
  The \PS\ files will be placed higher up in the directory tree.

For instance, if \texttt{ pwd } returns something like:
\begin{verbatim}
/usr/people/nelson/files/latex2html-98.1/work/tests
\end{verbatim}
and you run \latextohtml, then \fn{dvips} will generate
image output files here:
\begin{verbatim}
/usr/people/nelson/files
\end{verbatim}
  called \texttt{latex2html-98.001}, \texttt{latex2html-98.002}, \dots
  instead of \texttt{image001}, \texttt{image002}, \texttt{image003}, \dots
  in the subdirectory where your \texttt{.html} files were created.
  As a result the images will not show in your documents.

  If you are getting `File Not Found' errors,
  then reprocess your job using the \Cs{debug} switch,
  to see what options are passed to \fn{dvips}.
  If there is a `.' in some parts of any directory name,
  then look above that directory
  to see if files are being generated there.

\medskip

  One obvious fix is to rename the offending directory
  to remove the `.' from its name.

  If that is not possible, then define an alternative location
  for image generation to take place; set \texttt{\$TMP} to contain
  the name for this location. Typically \texttt{\$TMP = '/usr/tmp';}~.
  (This use of \texttt{\$TMP} is a good thing to do anyway,
  especially if your Unix account is subject to quota limitations.)

\item [It stops after having run \LaTeX, displaying a message about \fn{dvips}: ]
  See the previous item for the reason and `fix'.
  This version of \latextohtml{} detects the problem
  and will not allow images to be generated in the wrong place.
\end{changebar}

\item [\fn{dvips} complains about incorrect arguments:\label{dvips} ]
Please use a version which supports the command-line options \Cs{M}, \Cs{S},
\Cs{o} and \Cs{i}. ``Recent'' versions, at least after 5.516, do support them.


\item [It gives an ``\texttt{Out of memory}'' message and dies: ]
Try splitting your source file into more than one file,
using the \LaTeX{} commands \Lc{input} or \Lc{include}.
Also, try using the \Cs{no\_images} option.

\cbversion{97.1}\begin{changebar}
Perhaps the processor has fallen into an infinite loop.
Usually there will be a bad definition, or other faulty source code,
which has caused this.
See an \htmlref{earlier problem}{looping} for
how to set the \fn{\$VERBOSITY} \htmlref{variable}{verbosity} to
help locate the bad code leading to this memory exhaustion.
\end{changebar}

As a last resort you may consider increasing the virtual memory
(swap space) of your machine. As an indication
of what you might be able to do on your machine,
a very long book (about 1000 printed pages) required about
24MB of RAM and over 150MB of swap space to convert on a local Sun Sparc ELC
running SunOS 4.1.3.
\cbversion{97.1}\begin{changebar}
%begin{latexonly}
\newline\textit
%end{latexonly}%
{Much of this memory would have been consumed during image-generation.
This part of the processing is much more efficient in \textsc{v97.1}\,.}
\end{changebar}%


\cbversion{98.1}\begin{changebar}
\item [\fn{install-test} issues ``dbm'' related error messages: ]\latex{~\\}%
\latextohtml{} requires a DataBase Management system (\appl{NDBM},
\appl{GDBM}, or \appl{SDBM}) in order to run.
This is usually part of each Unix-like operating system and
\appl{SDBM} is part of \Perl{ 5}, but sometimes this is either missing
on your operating system or left out in a binary \Perl{} distribution.
\htmladdnormallinkfoot{Use \appl{Archie}}{http://www.pvv.unit.no/archie/}
to find one or (better) update to a complete \Perl{ 5} distribution.

\item [\fn{latex2html} issues ``dbm'' related error messages: ]
If you get warnings like
\begin{verbatim}
ndbm store returned -1, errno 28, key "xyz" at latex2html line 123
\end{verbatim}
this is related to an overflow of \latextohtml{} internals.
You will need help from the list, here.
\smallskip

If you get real error messages which cause \latextohtml{} to abort,
run \fn{install-test} to check if your DataBase management works.
You will probably need to re-install \Perl{ 5} (see above topic).
\end{changebar}


\cbversion{97.1}\begin{changebar}
This can happen when an image is being created from a large piece
of \LaTeX{} source code. The image-reuse mechanism uses the code itself
to construct a database key. If too long, the key is invalid and may
crash \appl{DBM} or \appl{NDBM}. (In fact this error should no longer
occur in \textsc{v97.1}\,, so please advise the \latextohtml{} developers if
you get error messages of this kind.)\html{\\}
The message should contain the name of environment which caused the problem,
along with an identifying number; e.g. \texttt{eqnarray268}.
To find which exact piece of code this represents, run \latextohtml{}
again, using the \Cs{debug} switch. Then look at the files
in the \texttt{TMP} subdirectory of the working directory
named \texttt{TMP/part001}, \texttt{TMP/part002}, etc.
Use the unix \fn{grep} command: \texttt{ grep 268 }\Meta{dir}\texttt{/TMP/part* }
to find that number in these files.
This should enable you to locate exactly where the problem occurs.

One solution may be to wrap the whole environment within
\Lc{begin\char123makeimage\char125} and \Lc{end\char123makeimage\char125}.
This will still cause the image to be created,
but uses just the environment name and number as the database key.
\end{changebar}%



\item [The \Lc{verb"ABC"} command doesn't work: ]
This is an unfortunate bug which can be avoided by using
any character other than quotes; e.g. \Lc{verb+ABC+}.

\item [Cannot get the ``tilde'' (\~{}) to show: \hfill]
The trick here is to use \verb|\~{}|.\html{\\}
Alternatively try using something like:
\url{\htmladdnormallink{mylink}{http://host/\%7Eme/path/file.html}}.

\textbf{Warning:} Some browsers may not be able to interpret the \verb|%7E|
as a ``tilde'' character.

\cbversion{98.1}\begin{changebar}
Try using \Lc{char126}~.
In any case tildes within \Lc{htmladdnormallink} and similar commands
are now handled correctly.
\end{changebar}

\item [Macro definitions don't work correctly: ]
As mentioned in other places, not all plain \TeX{} \Lc{def}-initions
can be converted.
But you may also have problems even when using \LaTeX{} definitions
(with \Lc{newcommand} and \Lc{newenvironment}) if such definitions
make use of sectioning or \Lc{verbatim} commands. These are
handled in a special way by \latextohtml{} and cannot be used in
macro definitions.

In general the macro handling mechanism is inefficient and very
fragile. Avoid using macros if possible.
\cbversion{97.1}\begin{changebar}%
A greater range of macros definitions can now be handled,
especially if appropriate declarations are added to an initialization file.
\end{changebar}%


\item [\Lc{input} commands: ]
There is a bug in the expansion of \Lc{input} commands which causes
a problem
when more than one \Lc{input} command appears on the same line.
There is no quick fix other than suggesting that you put each
\Lc{input} command on a line by itself,
in the \LaTeX{} source files.

\item [\Lc{input} commands in \env{verbatim} environments: ]
\cbversion{98.1}\begin{changebar}
These should no longer cause problems (actually since 97.1).
\Lc{input} commands are also handled correctly within comment environments
declared using \Lc{excludecomment}
and when loading multiple \texttt{.aux} files,
due to use of the \Lc{include} command.
Alternatively you might want to use
either the \env{verbatim} or the \env{verbatimfiles} package.
\end{changebar}

\item [Optional arguments in \env{description} environments: ]
If~you have optional arguments for the \Lc{item} command in
a \env{description} environment containing nested ``\texttt{]}'' characters then
these may not show up correctly. To avoid the problem enclose them
in \verb|{}|s;\\ e.g. \ \verb+\item[{[nested [angle [brackets] are ok]]}]+


\item [\latextohtml{} behaves differently even when you run it on the
same file: ]~\\
If~you notice any strange side-effects from previous runs of \latextohtml{},
try using the option \Cs{no\_reuse} and choose \texttt{(d)} when prompted.
This will clear any intermediate files generated during previous runs.
Note that this option will disable the image-reuse mechanism.

\item [Cannot convert \PS\ images which are included
in the \LaTeX{} file: ]~\\
It~is likely that the macros you are using for including \PS\
files (e.g. \Lc{epsffile}) are not understood by \latextohtml.
To avoid this problem enclose them in an environment which will
be passed to \LaTeX{} anyway; e.g.
\begin{quote}
%begin{latexonly}
\begin{small}
%end{latexonly}
\Lc{begin}\verb|{figure}|\\
\Lc{epsffile}\verb|{|\Meta{PostScript file name}\verb|}|\\
\Lc{end}\verb|{figure}|
%begin{latexonly}
\end{small}
%end{latexonly}
\end{quote}

Another reason why this might happen is that your shell environment variable
\fn{TEXINPUTS} may be undefined. This is not always
fatal but if you have problems you can use full
path-names for included \PS\ files (even when the \PS\
files are in the same directory as the \LaTeX{} source file).
Alternatively try setting \fn{TEXINPUTS} to `\texttt{.::}'.
With some \TeX{} and \LaTeX{} installations setting \fn{TEXINPUTS} to
`\texttt{.::}' may cause problems in the normal operation of \LaTeX.
If you get errors such as \LaTeX{} complaining that it can no longer find
any style files then you must set \fn{TEXINPUTS} to
\verb|"<path to your LaTeX installation>:."|
if you want to use both \LaTeX{} and \latextohtml{}.

\item [Some of the inlined images are in the wrong places: ]
There are several known ways that this may occur.
\begin{itemize}
\item
Perhaps one of the inlined images is more than a page (paper page) long.
This is sometimes the case with very large tables
or large \PS\ images. In this case you can try specifying
a larger paper size (e.g. `\texttt{a4}', `\texttt{a3}' or even `\texttt{a0}')
instead of the default (`\texttt{a5}')
using the \latextohtml{} variable \fn{\$PAPERSIZE}
in the file \fn{latex2html.config}.

\smallskip\noindent
\textit{This error should no longer occur, with \textsc{v97.1}\,.
Please report it on the mailing-list, if it does.}

\item
More likely is that some inappropriate \LaTeX{} code
has caused an error, resulting in an extra page (or pages) being generated.
Examine the \fn{images.log} file, to see if it reports any \LaTeX{} errors.

\item
A much rarer reason is that by default the \fn{dvips} program
reverses the \PS\ pages it generates. If your \fn{dvips} program
behaves in this way try changing the line: \html{\\}
\verb|$DVIPS = "dvips";| \html{\\}
to: \html{\\}
\verb|$DVIPS = "dvips -r0";| \html{\\}
within the file \fn{latex2html.config}.

\item
Yet another reason for images appearing out of place, especially while
developing a document, is that the browser's image cache is providing
out-of-date versions rather than getting the latest version afresh.
When this occurs there will often be images stretched or shrunk
to fit the wrong sized imaging area; this symptom is browser-dependent.
Flushing the cache, then reloading the \texttt{HTML} document, should
clear up the problem.
\end{itemize}


\item [\textbf{Unacceptable quality of converted images: }]
Try changing the size of the image
(see \hyperref{image conversion}{Section~}{}{imgcon}).

\item [The bibliographic references are missing: ]
Run \fn{latex} and then \fn{bibtex} on the original source file in
order to generate a \texttt{.bbl } file.
\latextohtml{} may need the \texttt{.bbl } file in order to generate
the references.


\item [The labels of figures, tables or equations are wrong: ]
This can happen if you have used any figures, tables, equations or
any counters inside conditional text; i.e. in a \env{latexonly}
or a \env{htmlonly} environment.


\item [Problems after changing the configuration files: ]
Please make sure that the last line in the configuration files
(i.e. \fn{.latex2html-init} and \fn{latex2html.config}) is:
\begin{small}
\begin{verbatim}
1;      # This is the last line
\end{verbatim}
\end{small}
This is a \Perl{} quirk.

\item [Problems when producing the \texttt{.dvi} version: \label{htmlsty}]
If you are using any of the new \LaTeX{} commands which are defined in
the \fn{html.sty} file make sure that \fn{html.sty} is included;
e.g. as one of the optional arguments to the \Lc{documentclass} command.

Of course you also have to make sure that \LaTeX{} knows where the \fn{html.sty}
file is, either by putting it in the same place as the other style-files on
your system, or by changing your \fn{TEXINPUTS} shell environment variable
\footnote{If you don't know how to do either of these things, copy (or
link) \fn{html.sty} to the directory of your \LaTeX{} document.}.

\item [Some of the fonts are translated incorrectly: ]
There is a fault in way the \LaTeX{} scoping rules have been
interpreted in \latextohtml. Consider this:
\begin{small}
\begin{verbatim}
\ttfamily fixed-width font.
\begin{something}
nothing here
\end{something}
default font.
\end{verbatim}
\end{small}
When processed by \LaTeX, the effect of the \Lc{tt} command is
delimited by the beginning of the environment ``something'',
so that ``default font'' will appear in the default font.
But \latextohtml{} will not recognise
``something'' as a delimiter and ``default font'' will appear in the
wrong font.

To avoid this problem (until it is fixed) you may delimit the scope of
some commands explicitly using \verb|{}|'s; i.e.
\begin{small}
\begin{verbatim}
\texttt{fixed-width font}.
\begin{something}
nothing here
\end{something}
default font.
\end{verbatim}
\end{small}

\cbversion{98.1}\begin{changebar}
Nesting of font changing commands is now handled correctly.
Such problems should no longer occur, though it always helps
to use explicitly delimited `pseudo'-environments; e.g.
\verb|{\bf ... }|, or \LaTeX's commands requiring an explicit argument,
such as \Lc{textbf}.
\end{changebar}

\item [Cannot get it to generate inlined images: ]
Run \latextohtml{} on your document, using the \Cs{debug} switch.
Look in the directory where the \texttt{HTML} files are generated
for two files named \fn{images.tex} and \fn{images.log}.
Examine their contents.
Do you notice anything unusual in them?

Copy \fn{images.tex} into the directory
of your original \LaTeX{} file and run \fn{latex} on \fn{images.tex}.
Can you see any errors in \fn{images.log}?
If so, can you fix \fn{images.tex} to get rid of the errors?
After fixing \fn{images.tex}
you can put it back in the directory of \texttt{HTML},
then run \latextohtml{} on the original document
using the \Cs{images\_only} switch.

However if you make changes or additions to the original source
then the same problems may occur again.
Thus it is better to understand why the changes were required
and alter your document's source code appropriately.

\medskip

If you get into a mess delete all the image files and run \latextohtml{}
again. Often it is sufficient to just delete the file \fn{images.pl}.

If you \emph{still} get into a mess, try running \latextohtml{} with the options
\Cs{no\_reuse} and \Cs{no\_images}; e.g.
\begin{small}
\begin{verbatim}
cblipca% latex2html -no_reuse -no_images test.tex
This is LaTeX2HTML Version 95 (Tue Nov 29 1994) by Nikos Drakos,
Computer Based Learning Unit, University of Leeds.

OPENING /tmp_mnt/home/cblelca/nikos/tmp/test.tex
Cannot create directory /usr/cblelca/nikos/tmp/test: File exists
(r) Reuse the images in the old directory OR
(d) *** DELETE *** /usr/cblelca/nikos/tmp/test AND ITS CONTENTS OR
(q) Quit ?
:d

Reading ...
Processing macros ....+.
Reading test.aux ......................
Translating ...0/1........1/1.....
Writing image file ...

Doing section links .....

*********** WARNINGS ***********

If you are having problems displaying the correct images with Mosaic,
try selecting "Flush Image Cache" from "Options" in the menu-bar
and then reload the HTML file.

Done.
\end{verbatim}
\end{small}
Then try to have a look in the file  \fn{images.tex}
(as described earlier) and perhaps fix it.
Once you are happy that \fn{images.tex} is OK, run \latextohtml{}
again with the option \Cs{images\_only}.

Some problems in displaying the correct inlined images,
may be due to the image caching mechanisms of your browser.
With some browsers a simple ``\textbf{Reload Current Document}'' will be enough
to refresh the images but with others (e.g. \appl{Mosaic}) you may need
to request for the cache to be refreshed. With \appl{Mosaic} try
selecting ``\textbf{Flush Image Cache}'' from ``\textbf{Options}'' in the menu-bar
and then reload the \texttt{HTML} file.


\item [It cannot do slides, memos, etc. ]
If you use Sli\TeX{} you can go a long way just by replacing
the \verb|{slides}| argument of the \Lc{documentclass} command with
something like \verb|{article}| just before using \latextohtml.
One problem may be that all your slides will end up in the  same \texttt{HTML}
file.
If you use \fn{lslide.sty} you may get much better results
(\htmladdnormallinkfoot{use \appl{Archie}}{http://www.pvv.unit.no/archie/}
to find this or any other style files).
\end{htmllist}
\index{problems|)}

    % Input counters and section
\end{htmlonly}
\internal{}%
\startdocument
%
%\section{Known Problems\label{sec:prb}\index{problems|(}}%
\label{sec:prb}\index{problems|(}%
\tableofchildlinks*\htmlrule\index{bugs}\html{\\}%
\noindent
Here are some of the problems that were known to exist
with previous versions of \latextohtml. 
Most of those that were real errors are either fixed completely
in more recent versions (\textsc{v98.1} and later),
or are much less likely to occur within correct \LaTeX{} source. 
(Some are not really errors but indications of poor style
in the user's choices among
various ways to organise their source code.)\html{\\} 
Several are indeed limitations inherent in the way 
\latextohtml{} currently performs its processing.
%
\begin{htmllist}
\htmlitemmark{PurpleBall}
\item [Unrecognised Commands and Environments:\index{unrecognised commands} ]
Unrecognised commands are ignored and any arguments are left in the text. 
Unrecognised environments are passed to \LaTeX{}  and the result is
included in the document as one or more inlined images.
%begin{latexonly}
\newline\textit
%end{latexonly}
{There are very few standard \LaTeX{} commands that are not recognised.
Many common \TeX{} commands are recognised also,
even though not explicitly mentioned 
in the \LaTeX{} \hypercite{blue book}{blue book}{}{lamp:latex}.
Any aberrant commands should be reported to the \latextohtml{}
\hyperref{mailing list}{mailing list, see Section~}{}{support}.}


\index{index}
\item [Index: ]
The translator generates its own index by saving the arguments  of 
the \Lc{index} command. The contents of the \env{theindex}
environment are ignored.
When using the \env{makeidx} package,
very sophisticated Indexes can be built automatically.
The Index for this manual is a good example.


\item[New Definitions:\index{extensions!definitions} ]
New definitions (\Lc{newcommand}, \Lc{newenvironment}, 
\Lc{newtheorem} and \Lc{def}),
will not work as expected if they are defined more than once.
Only the last definition will be used throughout the document.
Stylistically it is bad to declare new
environments or theorems outside of the document preamble,
so this should not cause any real problems.\html{\\}
Changes to commands using \Lc{def} or \Lc{renewcommand}
should usually be made only locally, within special environments,
to set a needed parameter; 
e.g. a basic length in a \env{picture} environment.
But when such environments force an image to be generated,
then \LaTeX{} will make the correct redefinition.

\item [Scope of declarations and environments: ]
If the scope of a declaration or environment crosses section
boundaries, then the output may not be as expected, because each
section is processed independently.
\newline
\textit{This is inherent to the way \latextohtml{} does its processing. 
It will not be different unless later versions change this strategy;
(e.g. if \latextohtmlNG\ ever becomes fully integrated.)}

\item [Math-mode font-size changes: ]  Math-mode font changes
made outside the math-mode are not honoured.  Thus the two equations
in \verb|$a_b$| and \verb|{\LARGE $a_b$}|
would come out looking the same.  The trick is to write
\verb|$a_b| and \verb|$\mbox{\LARGE $a_b$}$|.
\end{htmllist}



\subsection{Troubleshooting}%
%\section{Troubleshooting}%
\index{debugging}\index{problems}\index{fixes}%
Here are some curable symptoms:

\begin{htmllist}
\htmlitemmark{BlueBall}
\item [Cannot run any of the \Perl{} programs: ]
If your \Perl{} installation is such that \Perl{} programs are not allowed 
to run as shell scripts you may be unable to run  
\fn{latex2html}, \fn{texexpand}, \fn{pstoimg} and  \fn{install-test}. 
In this case change the first line in each of these
programs from
%begin{latexonly}
\begin{small}
%end{latexonly}
\begin{verbatim}
#!/usr/local/bin/perl
\end{verbatim}
%begin{latexonly}
\end{small}%
%end{latexonly}
to
%begin{latexonly}
\begin{small}%
%end{latexonly}
\begin{verbatim}
: # *-*-perl-*-*
    eval 'exec perl -S  $0 "$@"'
    if $running_under_some_shell; 
\end{verbatim}
%begin{latexonly}
\end{small}
%end{latexonly}

\item [The \fn{install-test} script gives uninformative error messages: ]
If, for any reason, you have trouble running \fn{install-test},
do not despair. Most of what it does is to do with checking
your installation rather than actually installing anything.
To do a \textit{manual installation} just change the variable
\fn{\$LATEX2HTMLDIR} in the beginning of the file \fn{latex2html}
to point to the directory where the \latextohtml{} files can be found.

Also, make sure that the files
\fn{pstoimg}, \fn{texexpand} and \fn{latex2html} are executable;
if necessary use the Unix \fn{chmod} command to make them 
executable.

\item [It just stops. ] Check the style
files that you are using. It is possible that you are using
a style file which contains raw \TeX{} commands. In such a case
start \latextohtml{} with the option 
\Cs{dont\_include \Meta{style-file name}}. 
Alternatively, add the name of the style to the variable 
\fn{\$DONT\_INCLUDE} in your
\fn{\$HOME/.latex2html-init} file. If you don't have such a file then
create one and add the lines:
\begin{quote}
%begin{latexonly}
\begin{small}
%end{latexonly}
\verb|$DONT_INCLUDE = "$DONT_INCLUDE" . ":|\Meta{style file name}\verb|";|\\
\verb|1;        # This must be the last line|
%begin{latexonly}
\end{small}
%end{latexonly}
\end{quote}

Another reason why \latextohtml{} might stop is that the \LaTeX{} source
file itself contains raw \TeX{} commands. In this case you may 
put such commands inside a \env{latexonly}
\hyperref{environment}{environment (see Section }{)}{sec:latexonly}.

The \fn{\$VERBOSITY} \htmlref{variable}{verbosity}{} can be used 
to create tracing messages, which may help to locate which command 
or environment was being processed when everything stopped.

\index{unending loop}\index{looping endlessly}
\item [It appears to be doing nothing.\label{looping}]
Perhaps the processor has fallen into an unending loop.
Usually there will be a bad definition, or other faulty source code,
which has caused this.
The \fn{\$VERBOSITY} \hyperref[page]{variable}{variable (see page~}{)}{cs_verbositylevel}%
can be set to generate tracing messages, which may help to locate which command 
or environment is being processed repeatedly.
Try setting a value of `\texttt{3}'; e.g. using the commandline switch \Cs{verbosity 3}.
This will print command and environment names, as thaey are processed.
It should soon become apparent where any such looping occurs.

\item [It just fills the endlessly with dots.]
No `perhaps' here; the processor has definitely fallen into an unending loop.
See the previous item for how to detect which command or environment
is causing the problem.

\cbversion{98.1}\begin{changebar}
%
\item [\Perl{} cannot parse the \fn{latex2html} script: ]
If \Perl{} refuses to start \latextohtml{} and issues errors, your
\Perl{} version is not up to date.
Update your \Perl{} to 5.003 or later. You can check which version of
\Perl{} you are using by invoking \Perl{} with the \Cs{v} option.

If \Perl{} issues errors during runtime, this is most probably related
to bugs within \latextohtml{} or one of its modules.
In this case you will need help from the developers or experienced users;
this can be obtained via the discussion list.


\item [It crashes (dumps core) as soon as it starts\label{perl}: ]
Update your \Perl{} to 5.003 or later.
\end{changebar}


\cbversion{98.1}\begin{changebar}
\item [It does not show any of your images: ]
  You can't run \latextohtml{} in a subdirectory that contains a dot
  within the directory name, such as {\tt latex2html-98.1},
  or in name of any higher directory.
  This is because \fn{dvips}'s \texttt{-o} option will change 98.1 into 98.001
  and use that as the resulting output file, instead of image001~.
  The \PS\ files will be placed higher up in the directory tree.

For instance, if \texttt{ pwd } returns something like:
\begin{verbatim}
/usr/people/nelson/files/latex2html-98.1/work/tests
\end{verbatim}
and you run \latextohtml, then \fn{dvips} will generate 
image output files here:
\begin{verbatim}
/usr/people/nelson/files
\end{verbatim}
  called \texttt{latex2html-98.001}, \texttt{latex2html-98.002}, \dots
  instead of \texttt{image001}, \texttt{image002}, \texttt{image003}, \dots
  in the subdirectory where your \texttt{.html} files were created.
  As a result the images will not show in your documents.

  If you are getting `File Not Found' errors,
  then reprocess your job using the \Cs{debug} switch,
  to see what options are passed to \fn{dvips}.
  If there is a `.' in some parts of any directory name, 
  then look above that directory
  to see if files are being generated there.

\medskip

  One obvious fix is to rename the offending directory
  to remove the `.' from its name.

  If that is not possible, then define an alternative location
  for image generation to take place; set \texttt{\$TMP} to contain
  the name for this location. Typically \texttt{\$TMP = '/usr/tmp';}~.
  (This use of \texttt{\$TMP} is a good thing to do anyway,
  especially if your Unix account is subject to quota limitations.)

\item [It stops after having run \LaTeX, displaying a message about \fn{dvips}: ]
  See the previous item for the reason and `fix'. 
  This version of \latextohtml{} detects the problem 
  and will not allow images to be generated in the wrong place.
\end{changebar}

\item [\fn{dvips} complains about incorrect arguments:\label{dvips} ]
Please use a version which supports the command-line options \Cs{M}, \Cs{S},
\Cs{o} and \Cs{i}. ``Recent'' versions, at least after 5.516, do support them.


\item [It gives an ``\texttt{Out of memory}'' message and dies: ] 
Try splitting your source file into more than one file,
using the \LaTeX{} commands \Lc{input} or \Lc{include}. 
Also, try using the \Cs{no\_images} option.

Perhaps the processor has fallen into an infinite loop.
Usually there will be a bad definition, or other faulty source code,
which has caused this. 
See an \htmlref{earlier problem}{looping} for
how to set the \fn{\$VERBOSITY} \htmlref{variable}{verbosity} to 
help locate the bad code leading to this memory exhaustion.

As a last resort you may consider increasing the virtual memory
(swap space) of your machine.


\cbversion{98.1}\begin{changebar}
\item [\fn{install-test} issues ``dbm'' related error messages: ]\latex{~\\}%
\latextohtml{} requires a DataBase Management system (\appl{NDBM},
\appl{GDBM}, or \appl{SDBM}) in order to run.
This is usually part of each Unix-like operating system and
\appl{SDBM} is part of \Perl{ 5}, but sometimes this is either missing
on your operating system or left out in a binary \Perl{} distribution.
\htmladdnormallinkfoot{Use \appl{Archie}}{http://www.pvv.unit.no/archie/} 
to find one or (better) update to a complete \Perl{ 5} distribution.

\item [\fn{latex2html} issues ``dbm'' related error messages: ]
If you get warnings like
\begin{verbatim}
ndbm store returned -1, errno 28, key "xyz" at latex2html line 123
\end{verbatim}
this is related to an overflow of \latextohtml{} internals.
You will need help from the list, here.
\smallskip

If you get real error messages which cause \latextohtml{} to abort,
run \fn{install-test} to check if your DataBase management works.
You will probably need to re-install \Perl{ 5} (see above topic).
\end{changebar}


This can happen when an image is being created from a large piece
of \LaTeX{} source code. The image-reuse mechanism uses the code itself
to construct a database key. If too long, the key is invalid and may
crash \appl{DBM} or \appl{NDBM}. (In fact this error should no longer
occur in \textsc{v97.1}\,, so please advise the \latextohtml{} developers if
you get error messages of this kind.)\html{\\}
The message should contain the name of environment which caused the problem,
along with an identifying number; e.g. \texttt{eqnarray268}.
To find which exact piece of code this represents, run \latextohtml{}
again, using the \Cs{debug} switch. Then look at the files
in the \texttt{TMP} subdirectory of the working directory
named \texttt{TMP/part001}, \texttt{TMP/part002}, etc.
Use the unix \fn{grep} command: \texttt{ grep 268 }\Meta{dir}\texttt{/TMP/part* }
to find that number in these files. 
This should enable you to locate exactly where the problem occurs.

One solution may be to wrap the whole environment within
\Lc{begin\char123makeimage\char125} and \Lc{end\char123makeimage\char125}.
This will still cause the image to be created, 
but uses just the environment name and number as the database key.



\item [The \Lc{verb"ABC"} command doesn't work: ]
This is an unfortunate bug which can be avoided by using 
any character other than quotes; e.g. \Lc{verb+ABC+}.

\item [Cannot get the ``tilde'' (\~{}) to show: \hfill]
The trick here is to use \verb|\~{}|.\html{\\}
Alternatively try using something like:
\url{\htmladdnormallink{mylink}{http://host/\%7Eme/path/file.html}}. 

\textbf{Warning:} Some browsers may not be able to interpret the \verb|%7E|
as a ``tilde'' character.

Try using \Lc{char126}~.
In any case tildes within \Lc{htmladdnormallink} and similar commands
should be handled correctly.

\item [Macro definitions don't work correctly: ]
As mentioned in other places, not all plain \TeX{} \Lc{def}-initions 
can be converted.
But you may also have problems even when using \LaTeX{} definitions
(with \Lc{newcommand} and \Lc{newenvironment}) if such definitions
make use of sectioning or \Lc{verbatim} commands. These are 
handled in a special way by \latextohtml{} and cannot be used in
macro definitions. 


\item [\Lc{input} commands: ]
There is a bug in the expansion of \Lc{input} commands which may cause
a problem when more than one \Lc{input} command appears on the same line.
There is no quick fix other than suggesting that you put each
\Lc{input} command on a line by itself, in the \LaTeX{} source files.

\item [\Lc{input} commands in \env{verbatim} environments: ]
\cbversion{98.1}\begin{changebar}
These should no longer cause problems (actually since 97.1).
\Lc{input} commands are also handled correctly within comment environments
declared using \Lc{excludecomment} 
and when loading multiple \texttt{.aux} files, 
due to use of the \Lc{include} command.
Alternatively you might want to use
either the \env{verbatim} or the \env{verbatimfiles} package.
\end{changebar}

\item [Optional arguments in \env{description} environments: ]
If~you have optional arguments for the \Lc{item} command in 
a \env{description} environment containing nested ``\texttt{]}'' characters then 
these may not show up correctly. To avoid the problem enclose them
in \verb|{}|s;\\ e.g. \ \verb+\item[{[nested [angle [brackets] are ok]]}]+


\item [\latextohtml{} behaves differently even when you run it on the
same file: ]~\\
If~you notice any strange side-effects from previous runs of \latextohtml{}, 
try using the option \Cs{no\_reuse} and choose \texttt{(d)} when prompted. 
This will clear any intermediate files generated during previous runs.
Note that this option will disable the image-reuse mechanism.

\item [Cannot convert \PS\ images which are included
in the \LaTeX{} file: ]~\\
It~is likely that the macros you are using for including \PS\ 
files (e.g. \Lc{epsffile}) are not understood by \latextohtml.
To avoid this problem enclose them in an environment which will
be passed to \LaTeX{} anyway; e.g.
\begin{quote}
%begin{latexonly}
\begin{small}
%end{latexonly}
\Lc{begin}\verb|{figure}|\\
\Lc{epsffile}\verb|{|\Meta{PostScript file name}\verb|}|\\
\Lc{end}\verb|{figure}|
%begin{latexonly}
\end{small}
%end{latexonly}
\end{quote}

Another reason why this might happen is that your shell environment variable 
\fn{TEXINPUTS} may be undefined. This is not always 
fatal but if you have problems you can use full
path-names for included \PS\ files (even when the \PS\
files are in the same directory as the \LaTeX{} source file).
Alternatively try setting \fn{TEXINPUTS} to `\texttt{.::}'. 
With some \TeX{} and \LaTeX{} installations setting \fn{TEXINPUTS} to 
`\texttt{.::}' may cause problems in the normal operation of \LaTeX.
If you get errors such as \LaTeX{} complaining that it can no longer find
any style files then you must set \fn{TEXINPUTS} to 
\verb|"<path to your LaTeX installation>:."|
if you want to use both \LaTeX{} and \latextohtml{}.

\item [Some of the inlined images are in the wrong places: ]
There are several known ways that this may occur.
\begin{itemize}
\item
Perhaps one of the inlined images is more than a page (paper page) long. 
This is sometimes the case with very large tables
or large \PS\ images. In this case you can try specifying 
a larger paper size (e.g. `\texttt{a4}', `\texttt{a3}' or even `\texttt{a0}') 
instead of the default (`\texttt{a5}') 
using the \latextohtml{} variable \fn{\$PAPERSIZE} 
in the file \fn{latex2html.config}.

\smallskip\noindent
\textit{This error should no longer occur, with \textsc{v97.1}\,.
Please report it on the mailing-list, if it does.}

\item
More likely is that some inappropriate \LaTeX{} code 
has caused an error, resulting in an extra page (or pages) being generated.
Examine the \fn{images.log} file, to see if it reports any \LaTeX{} errors.

\item
A much rarer reason is that by default the \fn{dvips} program
reverses the \PS\ pages it generates. If your \fn{dvips} program
behaves in this way try changing the line: \html{\\}
\verb|$DVIPS = "dvips";| \html{\\}
to: \html{\\}
\verb|$DVIPS = "dvips -r0";| \html{\\}
within the file \fn{latex2html.config}.

\item
Yet another reason for images appearing out of place, especially while
developing a document, is that the browser's image cache is providing
out-of-date versions rather than getting the latest version afresh.
When this occurs there will often be images stretched or shrunk
to fit the wrong sized imaging area; this symptom is browser-dependent.
Flushing the cache, then reloading the \texttt{HTML} document, should
clear up the problem. 
\end{itemize}


\item [\textbf{Unacceptable quality of converted images: }]
Try changing the size of the image 
(see \hyperref{image conversion}{Section~}{}{imgcon}).

\item [The bibliographic references are missing: ]
Run \fn{latex} and then \fn{bibtex} on the original source file in
order to generate a \texttt{.bbl } file. 
\latextohtml{} may need the \texttt{.bbl } file in order to generate 
the references.


\item [The labels of figures, tables or equations are wrong: ]
This can happen if you have used any figures, tables, equations or
any counters inside conditional text; i.e. in a \env{latexonly} 
or a \env{htmlonly} environment. 


\item [Problems after changing the configuration files: ]
Please make sure that the last line in the configuration files 
(i.e. \fn{.latex2html-init} and \fn{latex2html.config}) is:
\begin{small}
\begin{verbatim}
1;      # This is the last line
\end{verbatim}
\end{small}
This is a \Perl{} quirk.

\item [Problems when producing the \texttt{.dvi} version: \label{htmlsty}]
If you are using any of the new \LaTeX{} commands which are defined in 
the \fn{html.sty} file make sure that \fn{html.sty} is included;
e.g. as one of the optional arguments to the \Lc{documentclass} command.

Of course you also have to make sure that \LaTeX{} knows where the \fn{html.sty}
file is, either by putting it in the same place as the other style-files on
your system, or by changing your \fn{TEXINPUTS} shell environment variable
\footnote{If you don't know how to do either of these things, copy (or
link) \fn{html.sty} to the directory of your \LaTeX{} document.}.

\item [Some of the fonts are translated incorrectly: ]
There is a fault in way the \LaTeX{} scoping rules have been 
interpreted in \latextohtml. Consider this:
\begin{small}
\begin{verbatim}
\ttfamily fixed-width font.
\begin{something}
nothing here
\end{something}
default font.
\end{verbatim}
\end{small}
When processed by \LaTeX, the effect of the \Lc{tt} command is
delimited by the beginning of the environment ``something'',
so that ``default font'' will appear in the default font. 
But \latextohtml{} will not recognise
``something'' as a delimiter and ``default font'' will appear in the
wrong font. 

To avoid this problem (until it is fixed) you may delimit the scope of
some commands explicitly using \verb|{}|'s; i.e.
\begin{small}
\begin{verbatim}
\texttt{fixed-width font}.
\begin{something}
nothing here
\end{something}
default font.
\end{verbatim}
\end{small}

\cbversion{98.1}\begin{changebar}
Nesting of font changing commands is now handled correctly.
Such problems should no longer occur, though it always helps
to use explicitly delimited `pseudo'-environments; e.g.
\verb|{\bf ... }|, or \LaTeX's commands requiring an explicit argument,
such as \Lc{textbf}.
\end{changebar}

\item [Cannot get it to generate inlined images: ]
Run \latextohtml{} on your document, using the \Cs{debug} switch.
Look in the directory where the \texttt{HTML} files are generated
for two files named \fn{images.tex} and \fn{images.log}.
Examine their contents.
Do you notice anything unusual in them?

Copy \fn{images.tex} into the directory 
of your original \LaTeX{} file and run \fn{latex} on \fn{images.tex}.
Can you see any errors in \fn{images.log}?
If so, can you fix \fn{images.tex} to get rid of the errors? 
After fixing \fn{images.tex}
you can put it back in the directory of \texttt{HTML},
then run \latextohtml{} on the original document 
using the \Cs{images\_only} switch. 

However if you make changes or additions to the original source
then the same problems may occur again. 
Thus it is better to understand why the changes were required 
and alter your document's source code appropriately.

\medskip

If you get into a mess delete all the image files and run \latextohtml{}
again. Often it is sufficient to just delete the file \fn{images.pl}.

If you \emph{still} get into a mess, try running \latextohtml{} with the options
\Cs{no\_reuse} and \Cs{no\_images}; e.g.
\begin{small}
\begin{verbatim}
cblipca% latex2html -no_reuse -no_images test.tex
This is LaTeX2HTML Version 95 (Tue Nov 29 1994) by Nikos Drakos, 
Computer Based Learning Unit, University of Leeds.

OPENING /tmp_mnt/home/cblelca/nikos/tmp/test.tex 
Cannot create directory /usr/cblelca/nikos/tmp/test: File exists
(r) Reuse the images in the old directory OR
(d) *** DELETE *** /usr/cblelca/nikos/tmp/test AND ITS CONTENTS OR
(q) Quit ?
:d

Reading ...
Processing macros ....+.
Reading test.aux ......................
Translating ...0/1........1/1.....
Writing image file ...

Doing section links .....

*********** WARNINGS ***********

If you are having problems displaying the correct images with Mosaic,
try selecting "Flush Image Cache" from "Options" in the menu-bar 
and then reload the HTML file.

Done.
\end{verbatim}
\end{small}
Then try to have a look in the file  \fn{images.tex} 
(as described earlier) and perhaps fix it.
Once you are happy that \fn{images.tex} is OK, run \latextohtml{}
again with the option \Cs{images\_only}.

Some problems in displaying the correct inlined images,
may be due to the image caching mechanisms of your browser.
With some browsers a simple ``\textbf{Reload Current Document}'' will be enough
to refresh the images but with others (e.g. \appl{Mosaic}) you may need
to request for the cache to be refreshed. With \appl{Mosaic} try 
selecting ``\textbf{Flush Image Cache}'' from ``\textbf{Options}'' in the menu-bar 
and then reload the \texttt{HTML} file.


\item [It cannot do slides, memos, etc. ] 
If you use Sli\TeX{} you can go a long way just by replacing 
the \verb|{slides}| argument of the \Lc{documentclass} command with 
something like \verb|{article}| just before using \latextohtml.
One problem may be that all your slides will end up in the  same \texttt{HTML} 
file.
If you use \fn{lslide.sty} you may get much better results 
(\htmladdnormallinkfoot{use \appl{Archie}}{http://www.pvv.unit.no/archie/} 
to find this or any other style files).
\end{htmllist}
\index{problems|)}

