\documentclass[11pt,a4paper]{article}
% $Id: hthtml.tex,v 1.1 1996/12/21 19:56:26 JCL Exp $
\usepackage{html}
\usepackage{hthtml}
\title{\htmladdnormallinkfoot{hthtml}
  {http://www.cs.tu-bs.de/\~{}krinke/hthtml/index.html} 
  - an extended interface to LaTeX2HTML}
\author{\htmladdnormallinkfoot{Jens Krinke}%
  {http://www.cs.tu-bs.de/\~{}krinke/}\\
  \htmailto{j.krinke@tu-bs.de}\\
  \htmladdnormallinkfoot{TU Braunschweig}{http://www.tu-bs.de/}}
\htmetainfo{keywords}{latex2html, hthtml}
\htsetvar{NO_NAVIGATION}{1}
\htsetvar{ADDRESS}
{'<A HREF="http://www.cs.tu-bs.de/~krinke/hthtml/index.html">Original 
site</A><P><HR>' . \$ADDRESS}
\begin{document}
\maketitle
\tableofcontents
\section{Introduction}
The use of \texttt{html.sty} makes the use of some special
\LaTeX-commands possible, which supports the use of HTML-features in
LaTeX and give different results in the generated DVI-file and the
translated HTML-file.  However, more support can be added, which makes
it easier to write documents that are looking similar in DVI and HTML
format. 

\section{Features of \texttt{hthtml}}

\subsection{A frequently requested feature}

Every now and the somebody asks how to use `\verb|_|' and
`\verb|~|' in the \texttt{htmladdnormallink}-commands.  This is
not supported, one has to use \verb|\_| and \verb|\~{}|.  

The following commands are able to process `\verb|_|' and `\verb|~|'
directly:
\begin{enumerate}
\item \verb|\htlink{|\emph{text}\verb|}{|\emph{url}\verb|}|: Sets
  \emph{text} with a link to \emph{url} in HTML and \emph{text} with a
  footnote of \emph{url} in typewriter-style in DVI. This is an
  \htlink{example}{http://foo.bar/~gnus_and_gnats}.

This is made from:
\begin{verbatim}
\htlink{example}{http://foo.bar/~gnus_and_gnats}
\end{verbatim}

\item \verb|\hturl{|\emph{url}\verb|}|: Sets \emph{url} as text with a
  link to \emph{url} in HTML and \emph{url} in typewriter-style in
  DVI. Example: \hturl{http://foo.bar/~gnus_and_gnats}.

This is made from:
\begin{verbatim}
\hturl{http://foo.bar/~gnus_and_gnats}
\end{verbatim}

\item \verb|\htmailto{|\emph{mail-address}\verb|}|: Sets a mail address
  with a mailto-url pointing to it in HTML and just the address in
  DVI.  Example: \htmailto{foo@bar}.

This is made from:
\begin{verbatim}
\htmailto{foo@bar}
\end{verbatim}

\end{enumerate}

\verb|htlink| and \verb|hturl| aren't useable inside commands like
\verb|\author| or \verb|\title|, where one has to use the old
\verb|\htmladdnormallinkfoot| command.  This command is redefined in
\emph{hthtml} to set the footnote in typewriter-style. 

\subsection{A patched description environment}

The use of \verb|\htmladdnormallinkfoot| and \verb|htlink| is
impossible in the item-elements of a description environment because
it is impossible to use footnotes there.  The \verb|htdescription|
environment is a different implementation of the \verb|description|
environment which allows footnote and the other mentioned commands.

\begin{htdescription}
\item[foo] is not described here.
\item[Another \htlink{example}{http://foo.bar/~gnus_and_gnats} here.]
  Look at the footnote!
\item[bar] is described somewhere else.
\end{htdescription}

This is made from:
\begin{verbatim}
\begin{htdescription}
\item[foo] is not described here.
\item[Another \htlink{example}{http://foo.bar/~gnus_and_gnats} here.]
  Look at the footnote!
\item[bar] is described somewhere else.
\end{htdescription}
\end{verbatim}

\subsection{Some nice features}

\begin{itemize}
\item \verb|\htaddress{|\emph{text}\verb|}|:  Sets \emph{text} with
  \verb|<ADDRESS>| in HTML.  In DVI, \emph{text} is set in italics on
  a line by its own.  This looks like most browsers are rendering
  \verb|<ADDRESS>|. Example: \htaddress{gnus \& gnats\\
    \hturl{http://foo.bar/~gnus_and_gnats}\\
    \htmailto{foo@bar}} (with embedded urls).

This is made from:
\begin{verbatim}
\htaddress{gnus \& gnats\\
\hturl{http://foo.bar/~gnus_and_gnats}\\
\htmailto{foo@bar}}
\end{verbatim}

\item \verb|\htsetvar{|\emph{variable}\verb|}{|\emph{value}\verb|}|:
  Sets a variable of latex2html to a different value.  Has no other
  effect and can be used to set configuration variables in the
  \TeX-source. 

Look at this Example, which is part of this document:
\begin{verbatim}
\htsetvar{NO_NAVIGATION}{1}
\htsetvar{ADDRESS}
{'<A HREF="http://www.cs.tu-bs.de/~krinke/hthtml/index.html">Original 
site</A><P><HR>' . \$ADDRESS}
\end{verbatim}

This influences the look of the HTML-Version of this document:  There
is no navigation panel and the url of the original hthtml site is
included in the address.\par

\item \verb|\htmetainfo{|\emph{field}\verb|}{|\emph{value}\verb|}|:
  An entry of the form \verb|| is appended to the information in the
  head of the HTML-document.


Look at this Example:
\begin{verbatim}
\htmetainfo{keywords}{latex2html, hthtml}
\end{verbatim}

    sets the following in the head-section of the HTML-document.
\begin{verbatim}
<META NAME="keywords" CONTENT="latex2html, hthtml">
\end{verbatim}

    This command has to be placed in the document's head.
\end{itemize}

\section{Availability}

The complete \emph{hthtml}-package is available at the author's
\htlink{homepage}{http://www.cs.tu-bs.de/~krinke/} where the
\htlink{original hthtml-page}
{http://www.cs.tu-bs.de/~krinke/hthtml/index.html} is located.  It is
available as \htlink{gzip'ed tar-file}{http://www.cs.tu-bs.de/~krinke/hthtml/hthtml.tar.gz}.

\end{document}
