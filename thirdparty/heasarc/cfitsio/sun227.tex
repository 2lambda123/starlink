\documentclass[twoside,11pt,nolof]{starlink}
% -----------------------------------------------------------------------------
\stardoccategory    {Starlink User Note}
\stardocinitials    {SUN}
\stardocsource      {sun\stardocnumber}
\stardocnumber      {227.5}
\stardocauthors     {S. E. Rankin, \, M. J. Bly}
\stardocdate        {3rd March 2009}
\stardoctitle       {CFITSIO --- Disk FITS Input/Output Functions}
\stardocversion     {v3.13}
\stardocabstract  {
The CFITSIO package is a series of C functions that provide easy access
to FITS files on disk.  It was written by William D. Pence of HEASARC
(High Energy Astrophysics Science Archive Research Center) at the
Goddard Space Flight Center, USA.  Starlink is grateful to William
Pence for giving his permission for CFITSIO to be released on Starlink.}
\stardocname  {\stardocinitials /\stardocnumber}
% -----------------------------------------------------------------------------
\begin{document}
\scfrontmatter

\section{\xlabel{linking_the_cfitsio_library}Linking with the CFITSIO Library}
\label{linking_the_cfitsio_library}

CFITSIO is a series of C functions, built into a object library.
All C programs using the CFITSIO library will need to include the
\texttt{fitsio.h} include file at the beginning of the program, thus:

\begin{terminalv}
/* Example C program using CFITSIO */
#include <stdio.h>
#include "fitsio.h"
.
.
\end{terminalv}

To compile and link a C program with the CFITSIO library, include
\texttt{/star/include} in the include file search directives and
include the command \texttt{`cfitsio\_link`} in the list of
libraries.  For example (assuming you have \texttt{/star/bin} in your
\texttt{PATH}):

\begin{terminalv}
% cc -I/star/include -O myprog.c -o myprog -L/star/lib `cfitsio_link` -lm
\end{terminalv}

On Solaris systems you will need to add \texttt{-lsocket} and \texttt{-lnsl} 
to the list of libraries after \texttt{-lm}.

The CFITSIO library can also be used from Fortran programs because each
function has a Fortran wrapper routine that can be called from Fortran.

To link a Fortran program with the CFITSIO library, include the command
\texttt{`cfitsio\_link`} in your list of libraries, as for C programs.
For example:

\begin{terminalv}
% f77 -O myprog.f -o myprog -L/star/lib `cfitsio_link`
\end{terminalv}

On Solaris systems you will need to add \texttt{-lsocket} and \texttt{-lnsl}
to the list of libraries after \texttt{`cfitsio\_link`}.

\section{\xlabel{more_information}More Information}
\label{more_information}

This Starlink User Note is a brief introduction to the CFITSIO package.

For more detailed information on using CFITSIO from C there is a User's 
Guide: 

\begin{quote}
\textit{``CFITSIO User's Guide -- An Interface to FITS Format Files for 
C Programmers, v3.0 (MUD/166)''}
\end{quote}

For more detailed information on using CFISTIO from Fortran, there is a 
User's Guide:

\begin{quote}
\textit{``FITSIO User's Guide -- A Subroutine Interface to FITS Format 
Files for Fortran Programmers, v3/0 (MUD/167)''}
\end{quote}

There is also a Quick Start Guide:

\begin{quote}
\textit{``CFITSIO Quick Start Guide (MUD/162)''}
\end{quote}

These documents by William Pence have been issued as Starlink
Miscellaneous User Documents:  MUD/166, MUD/167 and MUD/162 respectively.
The sources are also available in the documents directory
(\texttt{/star/docs} on Starlink systems) as \texttt{mud166.tex}, 
\texttt{mud167.tex} and \texttt{mud162.tex}.

\section{\xlabel{release_information}Release Information}
\label{release_information}

The latest information about the CFITSIO release and bug fixes, changes 
and enhancements may be found in file:

\begin{quote}
\texttt{/star/share/cfitsio/cfitsio\_changes.txt}
\end{quote}

Users are advised to read the release information file at each new
release.

% ? End of main text
\end{document}

