\documentstyle[11pt]{article} 
\pagestyle{myheadings}

%------------------------------------------------------------------------------
\newcommand{\stardoccategory}  {Starlink Guide}
\newcommand{\stardocinitials}  {SG}
\newcommand{\stardocnumber}    {3.2}
\newcommand{\stardocauthors}   {Paul Rees, Jack Giddings \& Dave Mills}
\newcommand{\stardocdate}      {6 October 1993}
\newcommand{\stardoctitle}     {IUEDR --- Reference Manual}
%------------------------------------------------------------------------------

\newcommand{\stardocname}{\stardocinitials /\stardocnumber}
\markright{\stardocname}
\setlength{\textwidth}{160mm}
\setlength{\textheight}{230mm}
\setlength{\topmargin}{-2mm}
\setlength{\oddsidemargin}{0mm}
\setlength{\evensidemargin}{0mm}
\setlength{\parindent}{0mm}
\setlength{\parskip}{\medskipamount}
\setlength{\unitlength}{1mm}

\begin{document}
\thispagestyle{empty}
SCIENCE \& ENGINEERING RESEARCH COUNCIL \hfill \stardocname\\
RUTHERFORD APPLETON LABORATORY\\
{\large\bf Starlink Project\\}
{\large\bf \stardoccategory\ \stardocnumber}
\begin{flushright}
\stardocauthors\\
\stardocdate
\end{flushright}
\vspace{-4mm}
\rule{\textwidth}{0.5mm}
\vspace{5mm}
\begin{center}
{\Large\bf \stardoctitle}
\end{center}
\vspace{5mm}

\tableofcontents
\markright{\stardocname}

\begin{center}
\vspace {10mm}
{\bf NOTE}
\end {center}

This document replaces the IUEDR global parameter manual,  {\tt
IUEDR\_DOC:GLOBAL}, and the IUEDR command manual information provided
in the directory {\tt IUEDR\_MAN}.

\newpage
\section {Introduction}

This Manual describes the commands and parameters used by IUEDR.  It
is intended as a reference aid for people using IUEDR. If you are new
to IUE data reduction, you should read the IUEDR User Guide (SUN/37)
before proceeding any further. 

IUEDR functions fall into a number of specific categories:

\begin {itemize}
\item IUE tape inspection and file reading
\item Data display and manipulation
\item Spectrum extraction and calibration
\item Extraction product inspection and manipulation
\item Extraction product output
\end {itemize}

These functions are controlled by over fifty commands, with  nearly
one hundred global parameters within IUEDR.  Sensible defaults for the
most important parameters are set at the beginning of each IUEDR
session by a startup comand file. The default parameters are given in
the Appendix. There follows a summary of the commands in each of the
categories described above. 

\subsection {IUE tape inspection and file reading}

\begin {description}
\item [KILL IUEDR] Exit IUEDR and return to ICL. (type EXIT in the non-ICL 
UNIX version).
\item [LISTIUE] Analyse the contents of one or more IUE tape files.
\item [MTMOVE] Move to the start of a tape file.
\item [MTREW] Rewind to the start of the tape.
\item [MTSHOW] Show the current tape position.
\item [MTSKIPEOV] Skip over the end-of-volume mark.
\item [MTSKIPF] Skip over NSKIP tape marks.
\item [READIUE] Read a RAW, GPHOT or PHOT IUE image from the tape/file.
\item [READSIPS] Read the MELO or MEHI IUESIPS product from the tape/file.
\item [SAVE] Write new versions for any files that have had their contents 
updated during the current IUEDR session.
\end {description}

\subsubsection{VMS version}

An ICL abort (Control-C or Control-Y) may have the effect of losing
data not saved during the IUEDR session.

\subsubsection{UNIX version}

Interrupting the UNIX IUEDR (pre-ICL) will put it into the
`background'. The task may be reattatched by typing `{\tt fg}'.

A log of all commands typed during an IUEDR session and the response
printed at the terminal can be found in {\tt SESSION.LIS}.  This is
particularly useful when investigating the contents of IUE tapes. 

\subsection {Data display and manipulation}

\begin {description}

\item [CULIMITS] Delineate the graphical display limits using the
graphics cursor. 

\item [CURSOR] Determine dipslay coordinates using the graphics cursor
and print them at the terminal. 

\item [DRIMAGE] Display an IUE image on a suitable graphics
workstation.

\item [EDIMAGE] Edit the image data quality using the graphics cursor.

\item [MODIMAGE] Modify image pixel intensities interactively.

\item [SHOW] Print information relating to the current dataset at the
terminal.

\item [ERASE] Erase the display screen of the current graphics
workstation.

\end {description}

Image displays are colour coded to provide data quality information.
The  colour codes used by IUEDR are as follows:

\begin {quote}
\begin {description}
\item GREEN -- pixels affected by reseau marks
\item RED -- pixels which are saturated (DN=255)
\item ORANGE -- pixels affected by ITF truncation
\item YELLOW -- pixels marked bad by the user
\end {description}
\end {quote}

When using a mouse or tracker-ball with the graphics cursor, the
cursor ``hit'' buttons are normally numbered in increasing order from
left to right. 

\subsection {Spectrum extraction and calibration}

\begin {description}

\item [BARKER] Correct the extracted data for echelle ripple using a 
method based upon that of Barker (1984).

\item [CGSHIFT] Determine the spectrum template shift using the cursor
on  a SCAN plot.

\item [LBLS] Extract a line-by-line-spectrum array from the image.

\item [NEWABS] Associate a new absolute flux calibration with the
current  dataset.

\item [NEWCUT] Associate new echelle order wavelength limits with the
current  dataset.

\item [NEWDISP] Associate new spectrograph dispersion data with the
current  dataset.

\item [NEWFID] Associate new fiducial positions with the current
dataset.

\item [NEWRIP] Associate new ripple calibration data with the current 
dataset.

\item [NEWTEM] Associate new spectrum centroid template data with the 
current dataset.

\item [SCAN] Perform a scan of the image data perpendicular to the 
spectrograph dispersion.

\item [SETA] Set dataset parameters which are aperture specific.

\item [SETD] Set dataset parameters which are independent of order and 
aperture.

\item [SETM] Set dataset parameters which are order specific.

\item [TRAK] Extract a spectrum from the image.

\end {description}

\subsection {Extraction product inspection and manipulation}

\begin {description}

\item [EDMEAN] Edit the mean extracted spectrum using the graphics
cursor.

\item [EDSPEC] Edit the net extracted spectrum using the graphics
cursor.

\item [MAP] Map and merge extracted spectrum components to produce a
mean  spectrum.

\item [PLCEN] Plot the smoothed spectrum centroid shifts.

\item [PLFLUX] Plot the calibrated flux spectrum.

\item [PLGRS] Plot the pseudo-gross and background resulting from the 
spectrum extraction.

\item [PLMEAN] Plot the mean spectrum.

\item [PLNET] Plot the uncalibrated net spectrum.

\item [PLSCAN] Plot the image scan perpendicular to the dispersion.

\item [SGS] Print the names of the available SGS graphics devices at
the  terminal.

\end {description}

Plots of extracted IUE spectra and image scans include data quality
information for bad data. The data quality codes used by IUEDR are as 
follows:

\begin {quote}
\begin {description}
\item 1 -- affected by extrapolated ITF
\item 2 -- affected by microphonics
\item 3 -- affected by noise spike
\item 4 -- affected by bright point (or user)
\item 5 -- affected by reseau mark
\item 6 -- affected by ITF truncation
\item 7 -- affected by saturation
\item U -- affected by user edit
\end {description}
\end {quote}

\subsection {Extraction product output}

\begin {description}

\item [OUTEM] Output the current spectrum template data to a formatted
data  file.

\item [OUTLBLS] Output the current LBLS array to a binary data file.

\item [OUTMEAN] Output the current mean spectrum to a DIPSO ``SP''
format  data file.\footnote{The DIPSO SP0 format has been changed from
unformatted into a NDF format which allows inter-machine operation.}

\item [OUTNET] Output the current net spectrum to a DIPSO ``SP''
format data file. 

\item [OUTRAK] Output the current uncalibrated spectrum to a ``TRAK'' 
formatted data file. 

\item [OUTSCAN] Output the current scan data to a DIPSO ``SP'' format
data  file. 

\item [OUTSPEC] Output the current aperture (LORES) or order (HIRES)
spectrum to a DIPSO ``SP'' format data file.  

\item [PRGRS] Print the current extracted aperture or order spectrum
in tabular form.  

\item [PRLBLS] Print the current LBLS array in tabular form. 

\item [PRMEAN] Print the current mean spectrum in tabular form. 

\item [PRSCAN] Print the intensities of the current image scan in
tabular  form. 

\item [PRSPEC] Print the current aperture or order spectrum in tabular 
form. 

\end {description} 

\newpage

\section {User Interface}

\subsection {IUEDR Prompts}

The IUEDR user interface has been replaced by the ICL user interface.
This provides more functionality and on-line help to the user.

Type HELP at the ICL prompt for details (See also SG/4).

The UNIX version of ICL is not yet available. Therefore, the UNIX
IUEDR user interface consists of a dedicated command line (prompt
\verb+iuedr>+) which accepts only IUEDR commands (and optional
parameter specifications).

\subsection {Response to Command Prompts}

Instructions to IUEDR are given as  command lines.

Command lines begin with a command and an optional list of parameter
assignments.

Command lines and parameter lists can be continued onto a subsequent
line by finishing the line with a tilde: \(\tilde{ }\). This indicates
to IUEDR that further parameters are to be specified, {\em e.g.}

\begin{verbatim}
      ICL> READIUE ~
      ICL>  DRIVE=TAPE FILE=1
\end{verbatim}

Usually IUEDR will only prompt for parameters required by commands if
they have no currently defined value. However, some parameters are
either cancelled during the execution of a command or are set so that
the user is always prompted for a value. A command can be forced to
prompt for all required parameter values thus:

\begin{verbatim}
      ICL> READIUE PROMPT
\end{verbatim}

To force a command to prompt for parameter values and give additional
HELP information, you can reply `{\'tt ?}' in response to a parameter
prompt:

\begin{verbatim}
      ICL> READIUE PROMPT
      DRIVE [MUA0]: ?
\end{verbatim}

To route command output normally printed at the terminal to a log
file, the syntax:

\begin{verbatim}
      ICL> REPORT OUTPUT.DAT
\end{verbatim}

is used. This will log all output generated by the IUEDR commands to
the  file {\tt OUTPUT.DAT}. 

Such logfiles can be examined using the REPFILE command.

\subsection {Response to Parameter prompts} 

Help about a parameter can
be obtained by responding to the parameter prompt with a question
mark, {\em e.g.}

\begin{verbatim}
      DATASET(SWP19986) ?
\end{verbatim}

Help information will then be printed at the terminal and the prompt
repeated.

Sometimes an undefined parameter value is interpreted by a command in
a specific way ({\em e.g.} autoscaling within plotting commands).  A
parameter can be set undefined by responding to the prompt with an
exclaimation mark, {\em e.g.}

\begin{verbatim}
      XL(1150,1950) !
\end{verbatim}

A command may be aborted by responding to a parameter prompt with a 
double exclaimation mark, {\em e.g.}

\begin{verbatim}
      XL(1150,1950) !!
\end{verbatim}

\subsection {Getting HELP}

\subsubsection{VMS version}

Type HELP IUEDR at the ICL command line prompt. You may optionally
append a detailed description of the topic on which help is required.

\subsubsection{UNIX version}

Type HELP at the IUEDR command line prompt. You may optionally append
a detailed description of the topic on which help is required.


\subsection {VMS version - Access to DCL}

Access to DCL commands from within an ICL session is accomplished  by
pre-pending the command with a dollar sign, {\em e.g.}

\begin{verbatim}
      ICL> $ some-DCL-command
\end{verbatim}

Executing DCL commands in this way will end by returning to the ICL
command prompt once the DCL command has finished execution.

If you want an extended DCL session then you may type SPAWN at the ICL
prompt. The DCL command LOGOUT is used to return to ICL from DCL, {\em
e.g.}

\begin{verbatim}
      $ LOGOUT
\end{verbatim}

\subsection {IUEDR in Script and Batch Modes}

NOTE: The previous Script and Batch facilities have been superseded by
those provided within ICL. 

It is possible to run IUEDR in ``script'' mode, where command and
parameter input originates from a file instead of the terminal.  The
file name from which the command input is to be taken is pre-pended
with a LOAD command, {\tt e.g.}

\begin{verbatim}
ICL> LOAD SCRIPT
\end{verbatim}

This will result in the command input being taken from the file
SCRIPT.ICL and the text output being written to the screen and default
logfile.

When ICL starts up it look for a login script in LOGIN.ICL thus to 
run a BATCH job you need to place your commands in this file and then 
submit a command file consisting of the following:

\begin{verbatim}
      $ ADAMSTART
      $ SET DEFAULT somedisk:[some-dir]
      $ ICL
      $ EXIT
\end{verbatim}

NOTE: remember to include EXIT as the last ICL command in your login
script.

In addition to IUEDR commands, the script may contain any valid ICL
command and also commands for any other loaded package.

\subsection{UNIX version}

ICL is not yet available on all UNIX platforms. If you are running the
non-ICL IUEDR monolith then you can create a batch job by creating an 
ascii file containing your commands and their associated parameters.
You can then run this in the background by typing

\begin{verbatim}
      ascii-file-name &
\end{verbatim}

If you want the output to go to a logfile rather than the screen you
can type: 

\begin{verbatim}
      ascii-file-name > log-file-name &
\end{verbatim}

This strategy runs each IUEDR command as a seperate program and is
thus  rather inefficient as is will need to keep re-reading the data
files for each command.

\subsection {IUEDR Data Files}

NOTE: The IUEDR data files have changed format in order to allow 
inter-machine operation. However, IUEDR will still READ the OLD format
files IF they are present (this is only useful on the VAX as all OLD
format files will have been created on VAXen). If you have old format
files then you should use IUEDR to convert them to the new format by
doing the following, 

\begin{verbatim}
      ICL> TRAK DATASET=cam12345
      ICL> $ RENAME cam12345.UE* cam12345\_old.UE*
      ICL> $ SAVE
\end{verbatim}

During the reduction of an IUE dataset IUEDR will create a number of
binary data files. Their filenames are as follows:

\begin {description}
\item \verb+<dataset>.UEC+ --- calibration file
\item \verb+<dataset>_UED.SDF+ --- image data and quality file
\item \verb+<dataset>_UES.SDF+ --- uncalibrated spectrum file
\item \verb+<dataset>_UEM.SDF+ --- calibrated mean spectrum file
\end {description}

where \verb+<dataset>+ refers to the IUEDR DATASET parameter.

The {\tt .SDF} files are STARLINK NDF format files and can be read and
processed by any of the standard  packages ({\em e.g.}, KAPPA,FIGARO).
These files are in a machine indepedent format and can be freely
copied between any of the platforms which STARLINK supports. They may
be quickly examined using the ICL TRACE and NDFTRACE commands.

These files are in addition to the IUEDR log file and files generated
by the  IUEDR output commands. They should only be deleted once the
data reduction is complete and output  spectra obtained.

\newpage

\section{Changes to latest version}

This section describes the main changes that have been made to IUEDR
during its conversion to an ADAM based application which runs on both
VAX and UNIX based systems.

If you are a seasoned IUEDR user then you should study this section
especially carefully.

The most significant change from the scientific point of view is that 
the precision of all floating point calculations has been upgraded to
DOUBLE PRECISION. This was done after it was noticed that for high
resolution extraction the output spectra were subject to rounding
noise at the 1\% level.

The format of the calibration file ({\tt .UEC}) created by IUEDR has
been  changed to make it more readable. Old format files can still be 
read, the format being automatically determined by the program.

The functionality of the package has been enhanced to allow image data
to be read directly from disk, and command procedures are provided to 
automatically retreive images from the IUE archives.

The general operation of IUEDR, and all the command and parameter
names, are identical to those used in previous versions.

\subsection{UNIX version}

This release of IUEDR is written to run from the ICL command language.
However, this is not yet available on all UNIX platforms. In these
cases IUEDR runs as a single task which prompts the user for IUEDR
specific commands.

To start the IUEDR monolith task type 

\begin{verbatim}
      % iuedr3
\end{verbatim}

at the system prompt. The monolith then prompts repeatedly for IUEDR
commands until exit. To exit from this task, type 

\begin{verbatim}
      iuedr> exit
\end{verbatim}

To access the on-line help, type:

\begin{verbatim}
      iuedr> help
\end{verbatim}

To abort a task type:

\begin{verbatim}
      iuedr> ctrl-c
\end{verbatim}

In addition to running as a monolithic task each IUEDR command is also
available from the UNIX command line. Thus you can type:

\begin{verbatim}
      % trak
\end{verbatim}

at the system prompt and the IUEDR TRAK command will be invoked.

{\em However, it is important to remember that in this case the
program is re-started for each individual command and any previous
context is lost leading to the inefficient re-reading of files.}

\subsection{Log files}

The creation of log files recording the output of IUEDR is different 
depending upon whether you are running under ICL or not (not available
for all UNIX platforms at the time of writing).

\subsubsection{From ICL}

ICL has its own message logging and reporting functions and
information on these can be found using the HELP command.
 
For convenience IUEDR defines the alias HISTORY as a shorthand command
to provide a replay of the session log. By default this log is stored
in {\tt SESSION.DAT} and may be examined using the ICL REPFILE
command.

To direct the logging to a different file the ICL command: 
 
\begin{verbatim}
      iuedr> report filename
\end{verbatim}

may be used.

\subsubsection{From UNIX}

Until ICL arrives on your UNIX platform the easiest way to create a
log of a sequence of IUEDR commands is to place the commands in an
ascii file (using an editor of your choice) and then type:

\begin{verbatim}
      % filename > log-filename
\end{verbatim}

at the system prompt.

If you only need to log the output of a single command then you can
simply type:

\begin{verbatim}
      iuedr-command > log-filename
\end{verbatim}

\subsection{IUEDR command files}

The .CMD style of IUEDR command files is not directly supported by
ICL, and neither is the associated input/output redirection using
\(<\) and \(>\).

ICL command procedures provide this functionality in the version 3.

It is very easy to convert a {\tt .CMD} file into an ICL command procedure.
{\em e.g.} a {\tt DEMO.CMD} procedure:

\begin{verbatim}
      DATASET=SWP03196
      SHOW
      TRAK APERTURE=LAP
      SHOW V=S
\end{verbatim}

would become ICL procedure DEMO.ICL, thus:

\begin{verbatim}
      SHOW DATASET=SWP03196
      TRAK APERTURE=LAP
      SHOW V=S
\end{verbatim}

The only changes which need be made are to move any parameter
specifications (eg DATASET=) onto the same line as they  command they
apply to.

ICL command procedures are run by typing:

\begin{verbatim}
      ICL> LOAD procedure-name
\end{verbatim}

ICL procedures may contain any valid ICL commands as well as the IUEDR
provided command set. In addition you may  include commands provided
by other monoliths ({\em e.g.} KAPPA,PONGO) to access functions not
integrated into IUEDR.

{\em If ICL is not yet available on your machine then you may simply 
type the name of your command file at the system prompt and the
IUEDR commands will be executed by the system shell.}

\subsection{Interaction with DIPSO}

This release of IUEDR is accompanied with a UNIX release of DIPSO. The
format of the default DIPSO spectrum format SP0 files has been 
changed to use the STARLINK NDF data format. This means that these
files can be read by any standard STARLINK package.

The IUEDR/DIPSO user should notice no difference, as both IUEDR and
DIPSO understand the new format. However, if you have old VMS based
{\tt .UES}/{\tt .UEM} files generated by previous versions of IUEDR,
it is possible to read them into the new version of DIPSO (on VMS
ONLY) by using the OSP0RD command instead of SP0RD.

\subsection{DRIVE parameter options}

The use of the DRIVE parameter has been enhanced to allow
specification of disk files containing IUE datasets. This is intended
for use with  files obtained from online archives (RAL and NASA).

The sytax is to provide the full filename and extension in response to 
the DRIVE prompt:

\begin{verbatim}
      DRIVE> SWP12345.RAW
\end{verbatim}

Two command procedures are provided to help automate the retreival of
datasets from the archives: NDADSA\_FETCH and RLVAD\_FETCH retreive
datasets from their respective archive sites. To use the procedures
type:

\begin{verbatim}
      NDADSA_FETCH parmameter-1 [parameter-2]
\end{verbatim}

The first parameter may be either a dataset  identifier ({\em e.g.}
SWP12345) or the name of a file containing a list of dataset
identifiers (one per line).

The second parameter is optional and specifies the type of dataset
required. The default value is RAW, meaning raw data image. The set of
available types is:

\begin{itemize}
\item RAW --- raw data image
\item MELO --- extracted lo-res spectrum
\item ELBL --- extracted line-by-line hi-res spectra
\end{itemize}

For example:

\begin{verbatim}
      NDADSA_FETCH images.lis ELBL
\end{verbatim}

would attempt to retreive a set of ELBL files whose identifiers are
listed on the file `images.lis'

The process can take some time depending upon the network load and the
number of datasets requested. It may be convenient to let the job run
in the background by typing:

\begin{verbatim}
      SPAWN/NOWAIT @IUEDR_DIR:NDADSA_FETCH images.lis ELBL
\end{verbatim}

{\em Automated retreival is not yet available from UNIX machines.}

\subsection{VAX-UNIX IUEDR image file exchange}

An IUEDR image file is one of RAW,PHOT, or GPHOT type and consist of
768 by 768 pixels each stored in a 1 or 2-byte integer.

The transfer of files between VAX and UNIX systems is complicated by
the sophistication of the VAX file system. Under VMS the system
records a complex description of the precise format of all the  files
(and stores it in  the directory entry). Under UNIX this information
has to be provided by the user of the file when it is  opened. Because
of this difference it is sometimes necessary to use the following
format conversion utilities.

\subsubsection{VAX to UNIX}

If you wish to transfer IUE image data from a VAX onto a UNIX machine
in order to use the UNIX IUEDR then the transfer should be done using
FTP (in BINARY mode).

If you intend to copy the file using some other method (eg via NFS)
then you should first use the command:

\begin{verbatim}
      UNIX_FORMAT image-name
\end{verbatim}

to ensure the file is properly transferred.

Note that this also applies if you wish to just {\em access} the file
via NFS without explicitly transferring it.

\subsubsection{UNIX to VAX}

If you wish to transfer IUE image data from a UNIX machine onto a VAX
in order to use the VAX IUEDR then the transfer should be done using
FTP (in BINARY mode) and the command:

\begin{verbatim}
      VAX_FORMAT  image-name
\end{verbatim}

should then be used to ensure the file has the correct format.

If you use some other method of transferring the file (eg NFS) then 
the above command is {\em still} required.

\subsubsection{What will work?}

In general the following two commands will allow you to use any disk
based IUE image with any machine running IUEDR:

\begin{itemize}
\item {\tt VAX\_FORMAT} sets the file format as required by VAX IUEDR
\item {\tt UNIX\_FORMAT} sets the file format as required by UNIX IUEDR
\end{itemize}

Both commands operate only on the VAX.

The following illustrate some of the situations in which the format
conversions are used. A YES indicates that the strategy will work as
is, and  a NO that it will not, unless the remedial action described
in the bracketed text, is applied.

\begin{itemize}
\item
Copy file from archive onto VAX
\begin{itemize}
\item{YES using VAX IUEDR}
\item{NO using UNIX IUEDR and accessing file via NFS (use UNIX\_FORMAT filename).}
\end{itemize}

\item
FTP file from VAX to a UNIX system using BINARY transfer
\begin{itemize}
\item{YES using UNIX IUEDR}
\item{UNKNOWN using VAX IUEDR and accessing file via NFS)}
\end{itemize}

\item
FTP file from UNIX to a VAX system using BINARY transfer
\begin{itemize}
\item{NO using VAX IUEDR (use VAX\_FORMAT filename command).}
\item{YES using UNIX IUEDR and accessing file via NFS)}
\end{itemize}

\item
NFS copy file from VAX to a UNIX system after using UNIX\_FORMAT filename
\begin{itemize}
\item{YES using UNIX IUEDR}
\item{UNKNOWN using VAX IUEDR and accessing file via NFS)}
\end{itemize}

\end{itemize}

\subsection{Accessing data via NFS}

UNIX machines commonly provide disk sharing amongst remote machines
using the NFS protocol. 

For example your data frame may reside on a DECstation local  disk
called `iuedata' in the Rutherford cluster on machine `adam4'. In
order to get IUEDR to read it directly you could enter the following
in response to the DRIVE prompt:

\begin{verbatim}
      DRIVE> /adam4/iuedata/swp12345.raw
\end{verbatim}

To see which disks you have NFS access to you should use the {\tt
\%~df } command. In general any disks whose entry does not start with
a `/dev' are being served by a remote machine.


{\em IMPORTANT NOTE}

IUEDR allows you to use this method of data access with the following
proviso.

If the data resides on a VAX served disk then you must first  convert
its directory entry using the following command:

\begin{verbatim}
      UNIX_FORMAT  image-name
\end{verbatim}

This command does not change the data in any way. It merely alters the
description of the file format as stored in the VAX directory.

If at some later stage you wish to use the VAX version of IUEDR on the
same data file it first be necessary to use the command:

\begin{verbatim}
      VAX_FORMAT  image-name
\end{verbatim}

to convert back.
	
\subsection{Using on-line HELP}

IUEDR now has a selection of methods for obtaining on-line help.

\begin{itemize}
\item
From the VAX command line you may type:

\begin{verbatim}
      $ HELP IUEDR
\end{verbatim}
\item
From the ICL command line you may type:

\begin{verbatim}
      ICL> HELP IUEDR
\end{verbatim}

\item
From the IUEDR prompt on a UNIX machine you may type:

\begin{verbatim}
      iuedr> help
\end{verbatim}

\item
From any IUEDR parameter prompt on any machine you may type:

\begin{verbatim}
      ?            for help on that parameter
      ??           for help and to browse the help library
\end{verbatim}

\end{itemize}

\subsection{Specifying vector parameters}

Some IUEDR parameters (eg. XP, YP) require the specification of a pair
of numbers defining the limits of a range of values (eg pixels).

The method of setting such values has changed to the ADAM style:

\begin{verbatim}
	XP=[100,300]
\end{verbatim}

Note that the square brackets are only necessary when vector
parameters are specified on the command line. They are not required
when IUEDR prompts the user for a vector parameter.

\newpage

\section {Commands}

This section contains a detailed description of each of the commands
used by IUEDR.
 
\begin {description}

\item [BARKER] The spectrum data in DATASET are corrected for echelle 
ripple in the range ORDERS using a the method based upon that of
Barker (1984).

\begin {description}
\item Parameters:

\begin {description}
\item DATASET -- Dataset name.
\item ORDERS -- This delineates a range of echelle orders.
\end {description}

\item Description:

The spectrum data in DATASET are corrected for residual echelle ripple
using  the method described by Barker (1984. Astronomical Journal,
{\underline 89},  899). Orders in the range ORDERS are used in the
ripple correction optimisation.  Note that this optimisation method is
only applicable for SWP spectra. \end {description}

\item [CGSHIFT] Determine spectrum template shift using the cursor on
a  scan plot.

\begin {description}
\item Parameters:

\begin {description}
\item DATASET -- Dataset name.
\item APERTURE -- Aperture name (SAP or LAP).
\item ORDERS -- This delineates a range of echelle orders.
\item DEVICE -- GKS/SGS graphics device name.
\end {description}

\item Description:

This command allows the graphics cursor to be used to provide
information about spectrum template registration shifts.

A plot of the current spectrum scan must be available. A cycle
consisting of hitting either the ``1'' or ``2'' cursor keys is used to
mark the position of the spectrum. Each hit is used to calculate a
linear geometric shift of the spectrum template relative to the image.
The cycle is terminated by hitting any other key.

For LORES, when the cycle is complete the last geometric shift
determined is adopted and the scan is revoked.

For HIRES, each cursor hit is automatically associated with an echelle
order in the range defined by the ORDERS parameter. The last shift is
again adopted, but the scan is available for further display or
measurement. 

\end {description}

\item [CULIMITS] This command allows the display limits to be
delineated  by the cursor and the appropriate parameters set
accordingly.

\begin {description}
\item Parameters:

\begin {description}
\item DEVICE -- GKS/SGS graphics device name.
\item XL -- $x$-axis plotting limits, [0,0] means auto-scale.
\item YL -- $y$-axis plotting limits, [0,0] means auto-scale.
\item XP -- $x$-axis pixel limits, undefined means full extent.
\item YP -- $y$-axis pixel limits, undefined means full extent.
\end {description}

\item Description:

This command uses the cursor to delineate part of a current display,
graph or image, to be displayed in some subsequent command ({\em e.g.}
PLFLUX,  DRIMAGE, ...). 

The two cursor positions should be at the corners of the required
rectangular subset. The relation between cursor position sequences and
axis reversals for graphs is:

\begin {quote}
\begin {tabbing}
Positionx1xxxx\=Positionx2xxxx\=$x$-reversedxxx\=$y$-reversed\=\kill
\\
{\bf Position 1}\>{\bf Position 2}\>{\bf $x$-reversed}\>{\bf 
$y$-reversed}\\
\\
bottom/left\>top/right\>NO\>NO\\
bottom/right\>top/left\>YES\>NO\\
top/left\>bottom/right\>NO\>YES\\
top/right\>bottom/left\>YES\>YES\\
\end {tabbing}
\end {quote}

The XL and YL values are changed accordingly.

In the case of an image display, the XP and YP parameter values
are changed. The image will ALWAYS be drawn without axis reversals.
\end {description}

\item [CURSOR]
Find display coordinates using the cursor and print them at the 
terminal.
                                                                
\begin {description}
\item Parameters:

\begin {description}
\item None.
\end {description}

\item Description:

This command uses the graphics cursor to find coordinates on a
displayed graph or image. 

Either of the cursor hit keys ``1'' and ``2'' can be used repeatedly, with
any other key being used to terminate.
The coordinates for each cursor hit are printed on the terminal and 
correspond to scale of the axes on the current diagram (e.g.
(wavelength, flux) ).
If meaningful, additional coordinate information is also printed.
\end {description}

\item [DRIMAGE]
Display an IUE image on an suitable graphics workstation.

\begin {description}
\item Parameters:

\begin {description}
\item DATASET -- Dataset name.
\item DEVICE -- GKS/SGS graphics device name.
\item XP -- $x$-axis pixel limits, undefined means full extent.
\item YP -- $y$-axis pixel limits, undefined means full extent.
\item ZL -- Data limits for image display, undefined means full range.
\item FLAG -- Whether data quality for faulty pixels are displayed.
\end {description}

\item Description:

This command displays the image specified by the DATASET
parameter on the device specified by the DEVICE parameter.

The part of the image displayed is specified by the XP and YP
parameter values.
If unspecified, XP and YP default to the entire image extent, i.e.

\begin {quote}
      XP = (1,768), YP = (1,768).
\end {quote}

If the values of XP or YP are specified in decreasing order, the
image will NOT be reversed along the appropriate axis.

The range of data values displayed as a grey scale is limited
by the two values of the ZL parameter.
Data values at or below ZL(1) will appear BLACK, those at ZL(2) will
appear WHITE and those above ZL(2) will appear blue.
If the ZL values are given in decreasing order, then high data
values will be represented by low (dark) display intensities,
and vice-versa.
If the values are undefined, then the full intensity range of the
image will be used.
The full intensity range of the image can be found using the command

\begin {quote}
\begin{verbatim}
> SHOW V=I
\end{verbatim}
\end {quote}

If a pixel is affected by more than one of the above faults, then
the first in the list is adopted for display.

\begin {description}
\item GREEN -- pixels affected by reseau marks
\item RED -- pixels which are saturated (DN=255)
\item ORANGE -- pixels affected by ITF truncation
\item YELLOW -- pixels marked bad by the user
\end {description}

If a pixel is affected by more than one of the above faults, then
the first in the list is adopted for display.
\end {description}

\item [EDIMAGE]
Edit the image data quality using the graphics cursor.

\begin {description}
\item Parameters:

\begin {description}
\item DATASET -- Dataset name.
\item DEVICE -- GKS/SGS graphics device name.
\end {description}

\item Description:

This command uses the image display cursor to mark pixels and
regions of the current image that are ``bad'' or ``good''.
The image should have previously been displayed using the
DRIMAGE command.
So that faulty pixels can be seen, the FLAG=TRUE option in DRIMAGE should be
used.

The image display is specified by the DEVICE parameter and the associated
dataset by the DATASET parameter.

The following cursor hit sequences can be used in a cycle:

\begin {description}
\item 1 then 1 -- marks all pixels in the rectangle GOOD
\item 2 then 2 -- marks all points in the rectangle BAD
\item 1 -- marks the nearest pixel GOOD
\item 2 -- marks the nearest pixel BAD
\item 3 -- causes the cursor cycle to terminate
\end {description}

The pixels or ranges changed are printed on the terminal.
The term ``rectangle'' is used above to indicate a rectangular
set of pixels delineated by the two cursor positions.
Thus, for the first hit, the cursor can be positioned at the
bottom left corner, and for the second at the top right corner.

Only the user-defined data quality bit can be changed by this
command.
Initially, all faulty pixels have this bit set BAD, so that
spectrum extraction (say) can ignore these where appropriate.
However, the user-defined data quality can also be set GOOD.

See the IUEDR User Guide for further information on data quality.
\end {description}

\item [EDMEAN]
Edit the mean extracted spectrum using the graphics cursor.

\begin {description}
\item Parameters:

\begin {description}
\item DATASET -- Dataset name.
\item DEVICE -- GKS/SGS graphics device name.
\end {description}

\item Description:

This command uses the graphics cursor to mark points and
regions of the mean spectrum that are ``bad'' or ``good''.

The following cursor hit sequences can be used in a cycle:

\begin {description}
\item 1 then 1 -- marks all points in the $x$-range GOOD
\item 2 then 2 -- marks all points in the $x$-range BAD
\item 1 -- marks the nearest point in $x$-direction GOOD
\item 2 -- marks the nearest point in $x$-direction BAD
\item 3 -- causes the cursor cycle to terminate
\end {description}

The points or ranges changed are printed on the terminal.

See the IUEDR User Guide for further information on data quality.
\end {description}

\item [EDSPEC]
Edit the net extracted spectrum using the graphics cursor.

\begin {description}
\item Parameters:

\begin {description}
\item DATASET -- Dataset name.
\item ORDER -- Echelle order number.
\item APERTURE -- Aperture name (SAP or LAP).
\item DEVICE -- GKS/SGS graphics device name.
\end {description}

\item Description:

This command uses the graphics cursor to mark points and
regions of the current net spectrum that are ``bad'' or ``good''.
A plot of the APERTURE or ORDER spectrum is required before this
command can be used.

The following cursor hit sequences can be used in a cycle:

\begin {description}
\item 1 then 1 -- marks all points in the $x$-range GOOD
\item 2 then 2 -- marks all points in the $x$-range BAD
\item 1 -- marks the nearest point in $x$-direction GOOD
\item 2 -- marks the nearest point in $x$-direction BAD
\item 3 -- causes the cursor cycle to terminate
\end {description}

The points or ranges changed are printed on the terminal.

Only the user-defined data quality bit can be changed by this
command.
Initially, all faulty points have this bit set BAD (e.g. by TRAK).
However, whether they are considered bad (e.g. when plotting
or creating output files) is determined by the user-defined
bit, which can be changed at will.

See the IUEDR User Guide for further information on data quality.
\end {description}

\item [ERASE]
Erase the display screen of the graphics device.

\begin {description}
\item Parameters:

\begin {description}
\item DEVICE -- GKS/SGS graphics device name.
\end {description}

\item Description:

The display screen of the specified graphics device is erased.
\end {description}

\item [EXIT]
Quit IUEDR and return to DCL.

\begin {description}
\item Parameters:

\begin {description}
\item None.
\end {description}

\item Description:

This command quits IUEDR and returns to DCL.
Any files that require new versions will be written by this
command.
This command is a synonym for the QUIT command.
\end {description}

\item [LBLS]
Extracts a line-by-line-spectrum (LBLS) array from the 
image.

\begin {description}
\item Parameters:

\begin {description}
\item DATASET -- Dataset name.
\item ORDER -- Echelle order number.
\item APERTURE -- Aperture name (SAP or LAP).
\item GSAMP -- Spectrum grid sampling rate (geometric pixels).
\item CUTWV -- Whether wavelength cutoff data used for extraction grid.
\item CENTM -- Whether pre-existing centroid template is used.
\item RL -- Limits across spectrum for LBLS array (pixels).
\item RSAMP -- Radial coordinate sampling rate for LBLS grid (pixels).
\end {description}

\item Description:

This command creates a line-by-line-spectrum (LBLS) array from the
image defined by DATASET.
The array consists of intensities F(IR,IW) for a grid of wavelengths,
W(IW), and radial coordinates, R(IR).
The wavelength grid, W, is determined in a similar way to the TRAK command,
using the CUTWV (HIRES) and GSAMP (HIRES/LORES) parameters.

The radial coordinates are distances from the centre of the spectrum,
derived from the template data,
along a line perpendicular to the dispersion direction and
measured in geometric pixels.
The radial grid, R, is determined by the RL and RSAMP parameters.

The value of each pixel in the array corresponds to the surface
over the image of a rectangle centred on its (R,W) coordinates,
and extents

\begin {quote}
(R(IR)-DR/2, R(IR)+DR/2)
\end {quote}
and

\begin {quote}
(W(IW)-DW/2, W(IW)+DW/2).
\end {quote}
DR is the distance between R values, and DW is the wavelength step
between W values.

This surface integral is scaled along the W direction to
correspond to an interval of 1.414 geometric pixels.
The reason for this is to make LBLS intensities consistent with 
those produced by the TRAK command.
For a particular wavelength, W(IW), the sum of LBLS intensities
after removal of background should correspond to the net
flux as measured by TRAK.
\end {description}

\item [LISTIUE]
Analyse the contents of one or more IUE tape files.

\begin {description}
\item Parameters:

\begin {description}
\item DRIVE -- Tape drive logical name/file.
\item FILE -- Tape file number.
\item NFILE -- Number of tape files to be processed.
\item NLINE -- Number of IUE header lines printed.
\item SKIPNEXT -- Whether skip to next file.
\end {description}

\item Description:

This performs an analysis of NFILE IUE tape files, starting at
the file specified by the FILE parameter.
NFILE=-1 means list all files until the end of the tape.
NLINE=-1 means print all lines in file header.
\end {description}

\item [MAP]
Map and merge the extracted spectrum components to produce a mean spectrum.

\begin {description}
\item Parameters:

\begin {description}
\item DATASET -- Dataset name.
\item ORDERS -- This delineates a range of echelle orders.
\item RM -- Whether mean spectrum is reset before averaging.
\item ML -- Wavelength grid limits for mean spectrum.
\item MSAMP -- Wavelength sampling rate for mean spectrum grid.
\item FILLGAP -- Whether gaps can be filled within order.
\item COVERGAP -- Whether gaps can be filled by covering orders.
\end {description}

\item Description:

This command can be used to produce a mean spectrum with contributions
from several echelle orders (HIRES), or from several apertures (LORES).

If RM=TRUE, or if there is no existing mean spectrum, then an
evenly spaced wavelength grid is constructed between the
limits specified by the ML parameter using the sampling rate
specified by the MSAMP parameter.

If RM=FALSE and there IS an existing mean spectrum, then the wavelength
grid AND CONTENTS are retained.
New components will be averaged with what is already there.

In the case of HIRES, the ORDERS parameter is used to delimit the range
of echelle orders that are allowed to contribute to the mean.

In the case of LORES, only a single aperture specified by the 
APERTURE parameter is mapped at a given time.
\end {description}

\item [MODIMAGE]
Modifies image pixel intensities interactively.

\begin {description}
\item Parameters:

\begin {description}
\item DATASET -- Dataset name.
\item DEVICE -- GKS/SGS graphics device name.
\item FN -- Replacement Flux Number for pixel.
\end {description}

\item Description:

This command uses the image display cursor to modify image data.
The image should already have been displayed using the DRIMAGE command.

The following cursor sequences are adopted:

\begin {description}
\item 1 then 2 -- copy intensity of first picked pixel to the second
\item 2 -- prompt for replacement pixel intensity
\item 3 -- finish
\end {description}

If the data or data qualities change after a session, then the file is
saved on disk.

The assumption is made that the current image displayed corresponds
to the current dataset!
\end {description}

\item [MTMOVE]
Move to the start of a tape file.

\begin {description}
\item Parameters:

\begin {description}
\item DRIVE -- Tape drive logical name.
\item FILE -- Tape file number.
\end {description}

\item Description:

Move to the start of the file specified by the FILE parameter on the 
tape specified by the DRIVE parameter.
\end {description}

\item [MTREW]
Rewind to the start of the tape.

\begin {description}
\item Parameters:

\begin {description}
\item DRIVE -- Tape drive logical name.
\end {description}

\item Description:

This command rewinds the tape specified by the DRIVE parameter.
The FILE parameter is also set to 1 by this command.
\end {description}

\item [MTSHOW]
Show the current tape position.

\begin {description}
\item Parameters:

\begin {description}
\item DRIVE -- Tape drive logical name.
\end {description}

\item Description:

This command displays the current tape position.
This includes the file number and the block position relative to either 
the start or the end of the file.

Note that the actual file position may differ from the
value of the FILE parameter.
\end {description}

\item [MTSKIPEOV]
Skip over end-of-volume (EOV) mark.

\begin {description}
\item Parameters:

\begin {description}
\item DRIVE -- Tape drive logical name.
\end {description}

\item Description:

This command skips over an end-of-volume (EOV) mark on the tape specified 
by the DRIVE parameter.
An EOV condition is where there are two consecutive tape marks.
When attempting to skip across an EOV, an error will be reported
and the tape left positioned between the two marks.
Subsequent attempts to skip forward will fail and
only this command can be used to move forward beyond the
second tape mark.
\end {description}

\item [MTSKIPF]
Skip over NSKIP tape marks.

\begin {description}
\item Parameters:

\begin {description}
\item DRIVE -- Tape drive logical name.
\item NSKIP -- Number of tape marks to be skipped over.
\end {description}

\item Description:

This command skips over NSKIP tape marks on the tape specified
by the DRIVE parameter.
If NSKIP is negative this means that tape marks are skipped in the
reverse direction, i.e. towards the start of the tape.
\end {description}

\item [NEWABS]
Associate a new absolute flux calibration with the current dataset.

\begin {description}
\item Parameters:

\begin {description}
\item DATASET -- Dataset name.
\item ABSFILE -- Name of file containing absolute flux calibration.
\footnote{If you are on a UNIX machine then you should ensure that the
correct case if used (upper/lower) or the file will not be found.}
\end {description}

\item Description:

This command reads the absolute flux calibration from a text file
specified by the ABSFILE parameter and stores it in the dataset
specified by DATASET.

The file type is assumed to be ``.ABS'' and should not be
specified as part of the ABSFILE parameter.

The calibration of any current spectrum is automatically updated.
\end {description}

\item [NEWCUT]
Associate new echelle order wavelength limits with the current
dataset.

\begin {description}
\item Parameters:

\begin {description}
\item DATASET -- Dataset name.
\item CUTFILE -- Name of file containing echelle order wavelength limits.
\end {description}

\item Description:

This command reads the echelle order wavelength limits from a text file
specified by the CUTFILE parameter and stores them in the dataset
specified by DATASET.

The file type is assumed to be ``.CUT'' and should not be
specified as part of the CUTFILE parameter.

The calibration of any current spectrum is automatically updated.
\end {description}

\item [NEWDISP]
Associate new spectrograph dispersion data with the current dataset.

\begin {description}
\item Parameters:

\begin {description}
\item DATASET -- Dataset name.
\item DISPFILE -- Name of file containing dispersion data.
\end {description}

\item Description:

This command reads spectrograph dispersion data from a text file
specified by the DISPFILE parameter and stores them in the dataset
specified by DATASET.

The file type is assumed to be ``.DSP'' and should not be specified
as part of the DISPFILE parameter.
\end {description}

\item [NEWFID]
Read IUE fiducial positions into DATASET from text file with
name specified by FIDFILE.

\begin {description}
\item Parameters:

\begin {description}
\item DATASET -- Dataset name.
\item FIDFILE -- Name of file containing fiducial positions.
\item NGEOM -- Number of Chebyshev terms used to represent geometry.
\end {description}

\item Description:

This command reads IUE fiducial positions from a text file
specified by the FIDFILE parameter and stores them in the dataset
specified by DATASET.

The file type is assumed to be ``.FID'' and should not be specified
as part of the FIDFILE parameter.

The image data quality and geometry representation are updated to
account for any changes that these fidicual positions imply.
In the case of datasets containing image distortion, the NGEOM
parameter is used to specify the number of terms used for the Chebyshev 
representation along each axis.
\end {description}

\item [NEWRIP]
Read echelle ripple calibration into DATASET from text file 
with name specified by RIPFILE.

\begin {description}
\item Parameters:

\begin {description}
\item DATASET -- Dataset name.
\item RIPFILE -- Name of file containing echelle ripple calibration.
\end {description}

\item Description:

This command reads an echelle ripple calibration from a text file
specified by the RIPFILE parameter and stores it in the dataset
specified by DATASET.

The file type is assumed to be ``.RIP'' and should not be specified 
as part of the RIPFILE parameter.

The calibration of any current spectrum is automatically updated.
\end {description}

\item [NEWTEM]
Read spectrum centroid template data into DATASET from text 
file with name specified by TEMFILE.

\begin {description}
\item Parameters:

\begin {description}
\item DATASET -- Dataset name.
\item TEMFILE -- Name of file containing spectrum template data.
\end {description}

\item Description:

This command reads the spectrum centroid template data into DATASET from 
a text file with name specified by TEMFILE.

The file type is assumed to be ``.TEM'' and should not be specified
as part of the TEMFILE.
\end {description}

\item [OUTEM]
Output the current spectrum template data to a formatted data file.

\begin {description}
\item Parameters:

\begin {description}
\item DATASET -- Dataset name.
\item TEMFILE -- Name of file containing spectrum template data.
\end {description}

\item Description:

This command outputs the templates stored with the current
dataset to a separate text file. 
If not specified, the file name is constructed as:

\begin {quote}
$<$CAMERA$>$HI$<$APERTURE$>$.TEM
\end {quote}
or

\begin {quote}
$<$CAMERA$>$LO.TEM
\end {quote}

for the HIRES and LORES cases respectively.
\end {description}

\item [OUTLBLS]
Output the current LBLS array to a binary data file.

\begin {description}
\item Parameters:

\begin {description}
\item DATASET -- Dataset name.
\item OUTFILE -- Name of output file.
\end {description}

\item Description:

This command outputs the current LBLS array 
to a file.
If not specified by the OUTFILE parameter, the file name is constructed as:

\begin {quote}
$<$CAMERA$><$IMAGE$>$R.DAT
\end {quote}
The format of this file is described by the Fortran 77 routine, RDLBLS, which
can be found in the file:

\begin {quote}
IUEDR\_USER:RDLBLS.FOR
\end {quote}
The directory IUEDR\_USER: also contains a test
program for using RDLBLS and other helpful items.
\end {description}

\item [OUTMEAN]
Output current mean spectrum to a DIPSO ``SP'' format data file.

\begin {description}
\item Parameters:

\begin {description}
\item DATASET -- Dataset name.
\item OUTFILE -- Name of output file.
\item SPECTYPE -- DIPSO ``SP'' file type (0, 1 or 2).
\end {description}

\item Description:

This command outputs the mean spectrum associated with 
DATASET to a file that can be read into DIPSO (i.e. SUN/50).

This file is created with type specified
by the SPECTYPE parameter (see SUN/50 for ``SP'' options). 
If not specified, the file name is constructed as:

\begin {quote}
$<$CAMERA$><$IMAGE$>$M.DAT
\end {quote}
In DIPSO ``SP'' format, bad points are indicated by having zero intensities.
In determining which points in the output file are to be marked
``bad'', the user-defined data quality bit is used.
Since this bit can be arbitrarily edited,
faulty data values can be written to the output file 
without subsequent information being retained.
\end {description}

\item [OUTNET]
Output the current net spectrum to a DIPSO  ``SP'' format data file.

\begin {description}
\item Parameters:

\begin {description}
\item DATASET -- Dataset name.
\item APERTURE -- Aperture name (SAP or LAP).
\item ORDER -- Echelle order number.
\item OUTFILE -- Name of output file.
\item SPECTYPE -- DIPSO ``SP'' file type (0, 1 or 2).
\end {description}

\item Description:

This command outputs the net spectrum associated with ORDER or APERTURE
and DATASET to a file that can be read into DIPSO (i.e. SUN/50).

The file is created with type specified
by the SPECTYPE parameter (see SUN/50 for ``SP'' options).
If not specified, the file name is constructed as:

\begin {quote}
$<$CAMERA$><$IMAGE$>.<$APERTURE$>$
\end {quote}
in the case of LORES and

\begin {quote}
$<$CAMERA$><$IMAGE$>.<$ORDER$>$
\end {quote}
in the case of HIRES.
Here, $<$APERTURE$>$ is the aperture name (SAP or LAP), or index, and
$<$ORDER$>$ is the echelle order number.

In DIPSO ``SP'' format, bad points are indicated by having zero intensities.
In determining which points in the output file are to be marked
``bad'', the user-defined data quality bit is used.
Since this bit can be arbitrarily edited,
faulty data values can be written to the output file 
without subsequent information being retained.
\end {description}

\item [OUTRAK]
Output the current uncalibrated spectrum to a ``TRAK'' formatted data file.

\begin {description}
\item Parameter:

\begin {description}
\item DATASET -- Dataset name.
\item OUTFILE -- Name of output file.
\end {description}

\item Description:

This command outputs the uncalibrated spectrum associated with
DATASET to a formatted file that is compatible with output
from the old ``TRAK'' program.
The default file name is of the form:

\begin {quote}
$<$CAMERA$><$IMAGE$>$.TRK
\end {quote}
The main difference from an actual ``TRAK'' file is that the background
level is uniformly zero, so that GROSS=NET.
\end {description}

\item [OUTSCAN]
Output the current scan data to a DIPSO  ``SP'' format data file.

\begin {description}
\item Parameter:

\begin {description}
\item DATASET -- Dataset name.
\item OUTFILE -- Name of output file.
\item SPECTYPE -- DIPSO ``SP'' file type (0, 1 or 2).
\end {description}

\item Description:

This command outputs the current scan associated with 
DATASET to a file which can be read into DIPSO (i.e. SUN/50).

The file is created with type specified
by the SPECTYPE parameter (see SUN/50 for ``SP'' options).
If not specified, the file name is constructed as:

\begin {quote}
$<$CAMERA$><$IMAGE$>$S.DAT
\end {quote}
In DIPSO ``SP'' format, bad points are indicated by having zero intensities.
In determining which points in the output file are to be marked
``bad'', the user-defined data quality bit is used.
Since this bit can be arbitrarily edited,
faulty data values can be written to the output file 
without subsequent information being retained.
\end {description}

\item [OUTSPEC]
Output the current aperture (LORES) or order (HIRES) spectrum 
to a DIPSO ``SP'' format data file.

\begin {description}
\item Parameters:

\begin {description}
\item DATASET -- Dataset name.
\item APERTURE -- Aperture name (SAP or LAP).
\item ORDER -- Echelle order number.
\item OUTFILE -- Name of output file.
\item SPECTYPE -- DIPSO ``SP'' file type (0, 1 or 2).
\end {description}

\item Description:

This command outputs the spectrum associated with ORDER or APERTURE
and DATASET to a file which can be read into DIPSO (i.e. SUN/50).

The file is created with type specified
by the SPECTYPE parameter (see SUN/50 for ``SP'' options)
If not specified, the file name is constructed as:

\begin {quote}
$<$CAMERA$><$IMAGE$>.<$APERTURE$>$
\end {quote}
in the case of LORES and

\begin {quote}
$<$CAMERA$><$IMAGE$>.<$ORDER$>$
\end {quote}
in the case of HIRES.
Here, $<$APERTURE$>$ is the aperture name (SAP or LAP), or index, and
$<$ORDER$>$ is the echelle order number.

In DIPSO ``SP'' format, bad points are indicated by having zero intensities.
In determining which points in the output file are to be marked
``bad'', the user-defined data quality bit is used.
Since this bit can be arbitrarily edited,
faulty data values can be written to the output file
without subsequent information being retained.
\end {description}

\item [PLCEN]
Plot smoothed centroid shifts.

\begin {description}
\item Parameters:

\begin {description}
\item DATASET -- Dataset name.
\item ORDER -- Echelle order number.
\item APERTURE -- Aperture name (SAP or LAP).
\item RS -- Whether display is reset before plotting.
\item DEVICE -- GKS/SGS graphics device name.
\item ZONE -- Zone to be used for plotting.
\item LINE -- Plotting line style (SOLID, DASH, DOTDASH, DOT).
\item LINEROT -- Whether line style is changed after next plot.
\item COL -- Plotting line colour (1, 2, 3, ..., 10).
\item COLROT -- Whether line colour is changed after next plot.
\item XL -- $x$-axis plotting limits, [0,0] means auto-scale.
\item YL -- $y$-axis plotting limits, [0,0] means auto-scale.
\end {description}

\item Description:

This command plots the smoothed centroid shifts produced during
the most recent spectrum extraction from DATASET
on the graphics device and zone specified by the DEVICE and
ZONE parameters respectively.

In the case of a LORES spectrum, if there is more than a single
aperture available, then the APERTURE parameter needs to be specified.

In the case of a HIRES spectrum, if there is more than a single
echelle order, then the ORDER parameter needs to be specified.

The RS parameter specifies whether a new plot is started, or whether
the data can be plotted over an existing plot.

The LINE and LINEROT parameters determine the line style which will be
used for plotting.

The COL and COLROT parameters determine the line colour which will be used
for plotting if the DEVICE supports colour graphics.

The diagram limits are specified by the XL and YL parameter values.
If XL and YL have values

\begin {quote}
XL=[0,0], YL=[0,0],
\end {quote}
then the plot limits along each axis are determined so that the whole
spectrum is visible.
If the values of XL or YL are specified in decreasing order, then
the coordinates will be reversed along the appropriate axis.
\end {description}

\item [PLFLUX]
Plot calibrated flux spectrum.

\begin {description}
\item Parameters:

\begin {description}
\item DATASET -- Dataset name.
\item ORDER -- Echelle order number.
\item APERTURE -- Aperture name (SAP or LAP).
\item RS -- Whether display is reset before plotting.
\item DEVICE -- GKS/SGS graphics device name.
\item ZONE -- Zone to be used for plotting.
\item LINE -- Plotting line style (SOLID, DASH, DOTDASH, DOT).
\item LINEROT -- Whether line style is changed after next plot.
\item COL -- Plotting line colour (1, 2, 3, ..., 10).
\item COLROT -- Whether line colour is changed after next plot.
\item HIST -- Whether lines are drawn as histograms.
\item QUAL -- Whether data quality information is plotted.
\item XL -- $x$-axis plotting limits, [0,0] means auto-scale.
\item YL -- $y$-axis plotting limits, [0,0] means auto-scale.
\end {description}

\item Description:

This command plots the calibrated flux spectrum from DATASET
on the graphics device and zone specified by the DEVICE and 
ZONE parameters respectively.

In the case of a LORES spectrum, if there is more than a single
aperture available, then the APERTURE parameter needs to be specified.

In the case of a HIRES spectrum, if there is more than a single
echelle order, then the ORDER parameter needs to be specified.

The RS parameter specifies whether a new plot is started, or whether
the data can be plotted over an existing plot.

The HIST parameter determines whether the line is drawn as a histogram
rather than a continuous polyline.

The LINE and LINEROT parameters determine the line style which will be
used for plotting.

The COL and COLROT parameters determine the line colour which will be used
for plotting if the DEVICE supports colour graphics.

The diagram limits are specified by the XL and YL parameter values.
If XL and YL have values

\begin {quote}
XL=[0,0], YL=[0,0],
\end {quote}
then the plot limits along each axis are determined so that the whole
spectrum is visible.
If the values of XL or YL are specified in decreasing order, then
the coordinates will be reversed along the appropriate axis.

The QUAL parameter indicates whether faulty points are flagged with
their data quality codes (see Section 1).

If a point is affected by more than one of the above faults, then
the highest code is plotted.
Points marked bad by user edits are only indicated if they are otherwise
fault-free.
\end {description}

\item [PLGRS]
Plot pseudo-gross and background from spectrum extraction.

\begin {description}
\item Parameters:

\begin {description}
\item DATASET -- Dataset name.
\item ORDER -- Echelle order number.
\item APERTURE -- Aperture name (SAP or LAP).
\item RS -- Whether display is reset before plotting.
\item DEVICE -- GKS/SGS graphics device name.
\item ZONE -- Zone to be used for plotting.
\item LINE -- Plotting line style (SOLID, DASH, DOTDASH, DOT).
\item LINEROT -- Whether line style is changed after next plot
\item COL -- Plotting line colour (1, 2, 3, ..., 10).
\item COLROT -- Whether line colour is changed after next plot.
\item HIST -- Whether lines are drawn as histograms.
\item QUAL -- Whether data quality information is plotted.
\item XL -- $x$-axis plotting limits, [0,0] means auto-scale.
\item YL -- $y$-axis plotting limits, [0,0] means auto-scale.
\end {description}

\item Description:

This command plots the pseudo-gross and smooth background produced during
the most recent spectrum extraction from DATASET
on the graphics device and zone specified by the DEVICE and
ZONE parameters respectively.

The pseudo-gross is constructed by taking the net spectrum and adding
the smooth background multiplied by the width of the object channel.
The smooth background plotted is also for the object channel width.

In the case of a LORES spectrum, if there is more than a single
aperture available, then the APERTURE parameter needs to be specified.

In the case of a HIRES spectrum, if there is more than a single
echelle order, then the ORDER parameter needs to be specified.

The RS parameter specifies whether a new plot is started, or whether
the data can be plotted over an existing plot.

The HIST parameter determines whether the line is drawn as a histogram
rather than a continuous polyline.

The LINE and LINEROT parameters determine the line style which will be
used for plotting.

The COL and COLROT parameters determine the line colour which will be used
for plotting if the DEVICE supports colour graphics.

The diagram limits are specified by the XL and YL parameter values.
If XL and YL have values

\begin {quote}
XL=[0,0], YL=[0,0],
\end {quote}
then the plot limits along each axis are determined so that the whole
spectrum is visible.
If the values of XL or YL are specified in decreasing order, then
the coordinates will be reversed along the appropriate axis.

The QUAL parameter indicates whether faulty points are flagged with
their data quality codes (see Section 1).

If a point is affected by more than one of the above faults, then
the highest code is plotted.
Points marked bad by user edits are only indicated if they are otherwise
fault-free.
\end {description}

\item [PLMEAN]
Plot mean spectrum.

\begin {description}
\item Parameters:

\begin {description}
\item DATASET -- Dataset name.
\item RS -- Whether display is reset before plotting.
\item DEVICE -- GKS/SGS graphics device name.
\item ZONE -- Zone to be used for plotting.
\item LINE -- Plotting line style (SOLID, DASH, DOTDASH, DOT).
\item LINEROT -- Whether line style is changed after next plot
\item COL -- Plotting line colour (1, 2, 3, ..., 10).
\item COLROT -- Whether line colour is changed after next plot.
\item HIST -- Whether lines are drawn as histograms.
\item QUAL -- Whether data quality information is plotted.
\item XL -- $x$-axis plotting limits, [0,0] means auto-scale.
\item YL -- $y$-axis plotting limits, [0,0] means auto-scale.
\end {description}

\item Description:

This command plots the mean spectrum associated with DATASET on the
graphics device and zone specified by the DEVICE and ZONE parameters 
respectively.

The RS parameter specifies whether a new plot is started, or whether
the data can be plotted over an existing plot.

The HIST parameter determines whether the line is drawn as a histogram
rather than a continuous polyline.

The LINE and LINEROT parameters determine the line style which will be
used for plotting.

The COL and COLROT parameters determine the line colour which will be used
for plotting if the DEVICE supports colour graphics.

The diagram limits are specified by the XL and YL parameter values.
If XL and YL have values

\begin {quote}
XL=[0,0], YL=[0,0],
\end {quote}
then the plot limits along each axis are determined so that the whole
spectrum is visible.
If the values of XL or YL are specified in decreasing order, then
the coordinates will be reversed along the appropriate axis.

The QUAL parameter indicates whether faulty points are flagged with
their data quality codes (see Section 1).

If a point is affected by more than one of the above faults, then
the highest code is plotted.
Points marked bad by user edits are only indicated if they are otherwise
fault-free.
\end {description}

\item [PLNET]
Plot uncalibrated net spectrum.

\begin {description}
\item Parameters:

\begin {description}
\item DATASET -- Dataset name.
\item ORDER -- Echelle order number.
\item APERTURE -- Aperture name (SAP or LAP).
\item RS -- Whether display is reset before plotting.
\item DEVICE -- GKS/SGS graphics device name.
\item ZONE -- Zone to be used for plotting.
\item LINE -- Plotting line style (SOLID, DASH, DOTDASH, DOT).
\item LINEROT -- Whether line style is changed after next plot
\item COL -- Plotting line colour (1, 2, 3, ..., 10).
\item COLROT -- Whether line colour is changed after next plot.
\item HIST -- Whether lines are drawn as histograms.
\item QUAL -- Whether data quality information is plotted.
\item XL -- $x$-axis plotting limits, [0,0] means auto-scale.
\item YL -- $y$-axis plotting limits, [0,0] means auto-scale.
\end {description}

\item Description:

This command plots the uncalibrated net spectrum specified by the DATASET
parameter on the graphics device and zone specified by the DEVICE and 
ZONE parameters respectively.

In the case of a LORES spectrum, if there is more than a single
aperture available, then the APERTURE parameter needs to be specified.

In the case of a HIRES spectrum, if there is more than a single
echelle order, then the ORDER parameter needs to be specified.

The RS parameter specifies whether a new plot is started, or whether
the data can be plotted over an existing plot.

The HIST parameter determines whether the line is drawn as a histogram
rather than a continuous polyline.

The LINE and LINEROT parameters determine the line style which will be
used for plotting.

The COL and COLROT parameters determine the line colour which will be used
for plotting if the DEVICE supports colour graphics.

The diagram limits are specified by the XL and YL parameter values.
If XL and YL have values

\begin {quote}
XL=[0,0], YL=[0,0],
\end {quote}
then the plot limits along each axis are determined so that the whole
spectrum is visible.
If the values of XL or YL are specified in decreasing order, then
the coordinates will be reversed along the appropriate axis.

The QUAL parameter indicates whether faulty points are flagged with
their data quality codes (see Section 1).

If a point is affected by more than one of the above faults, then
the highest code is plotted.
Points marked bad by user edits are only indicated if they are otherwise
fault-free.
\end {description}

\item [PLSCAN]
Plot scan perpendicular to dispersion.

\begin {description}
\item Parameters:

\begin {description}
\item DATASET -- Dataset name.
\item RS -- Whether display is reset before plotting.
\item DEVICE -- GKS/SGS graphics device name.
\item ZONE -- Zone to be used for plotting.
\item LINE -- Plotting line style (SOLID, DASH, DOTDASH, DOT).
\item LINEROT -- Whether line style is changed after next plot.
\item COL -- Plotting line colour (1, 2, 3, ..., 10).
\item COLROT -- Whether line colour is changed after next plot.
\item QUAL -- Whether data quality information is plotted.
\item XL -- $x$-axis plotting limits, [0,0] means auto-scale.
\item YL -- $y$-axis plotting limits, [0,0] means auto-scale.
\end {description}

\item Description:

This command plots the most recent scan perpendicular to dispersion 
associated with DATASET on the graphics device and zone specified by
the DEVICE and ZONE parameters respectively.

The RS parameter specifies whether a new plot is started, or whether
the data can be plotted over an existing plot.

The LINE and LINEROT parameters determine the line style which will be
used for plotting.

The COL and COLROT parameters determine the line colour which will be used
for plotting if the DEVICE supports colour graphics.

The diagram limits are specified by the XL and YL parameter values.
If XL and YL have values

\begin {quote}
XL=[0,0], YL=[0,0],
\end {quote}
then the plot limits along each axis are determined so that the whole
spectrum is visible.
If the values of XL or YL are specified in decreasing order, then
the coordinates will be reversed along the appropriate axis.

The QUAL parameter indicates whether faulty points are flagged with
their data quality codes (see Section 1).

If a point is affected by more than one of the above faults, then
the highest code is plotted.
Points marked bad by user edits are only indicated if they are otherwise
fault-free.
\end {description}

\item [PRGRS]
Print the current extracted aperture or order spectrum in tabular 
form.

\begin {description}
\item Parameters:

\begin {description}
\item DATASET -- Dataset name.
\item APERTURE -- Aperture name (SAP or LAP).
\end {description}

\item Description:

This command prints the recently extracted spectrum associated
with ORDER or APERTURE
and DATASET in tabular form.
The table consists of wavelengths, ``gross'', smooth background, net
and calibrated fluxes, along
with any data quality information.
The ``gross'' and smooth background correspond to an image sample
with width specified by the adopted extraction slit.

The output from this command should be diverted to a file, since
it is likely to be too voluminous to read at the terminal.
\end {description}

\item [PRLBLS]
Print the current LBLS array in tabular form.

\begin {description}
\item Parameters:

\begin {description}
\item DATASET -- Dataset name.
\end {description}

\item Description:

This command prints the current LBLS array in tabular form.

Each line of the main table consists of a wavelength and a set of
mapped image intensities (FN), corresponding to cells at distances, R (pixels),
from spectrum centre.

Any array cells which are affected by ``bad'' image pixels (e.g. reseaux,
saturation, etc.) have data quality values printed below them, the
meaning of which is given at the start of the output.

The output from this command should be diverted to a file, since
it is likely to be too voluminous to read at the terminal.
\end {description}

\item [PRMEAN]
Print the current mean spectrum in tabular form.

\begin {description}
\item Parameters:

\begin {description}
\item DATASET -- Dataset name.
\end {description}

\item Description:

This command prints the mean spectrum associated with DATASET in tabular form.

The table consists of wavelengths and calibrated fluxes, along
with any data quality information.

The output from this command should be diverted to a file, since
it is likely to be too voluminous to read at the terminal.
\end {description}

\item [PRSCAN]
Print the intensities of the current image scan in tabular form.

\begin {description}
\item Parameters:

\begin {description}
\item DATASET -- Dataset name.
\end {description}

\item Description:

This command prints the scan associated with DATASET in tabular form.
The table consists of wavelengths and net fluxes, along
with any data quality information.

The output from this command should be diverted to a file, since
it is likely to be too voluminous to read at the terminal.
\end {description}

\item [PRSPEC]
Print the current aperture or order spectrum in tabular form.

\begin {description}
\item Parameters:

\begin {description}
\item DATASET -- Dataset name.
\item APERTURE -- Aperture name (SAP or LAP).
\item ORDER -- Echelle order number.
\end {description}

\item Description:

This command prints the spectrum associated with ORDER or APERTURE
and DATASET in tabular form.

The table consists of wavelengths, net and calibrated fluxes, along
with any data quality information.

The output from this command should be diverted to a file, since
it is likely to be too voluminous to read at the terminal.
\end {description}

\item [QUIT]
Quit IUEDR and return to DCL.

\begin {description}
\item Parameters:

\begin {description}
\item None.
\end {description}

\item Description:

This command quits IUEDR and returns to DCL.
Any files that require new versions will be written by this
command.
This command is a synonym for the EXIT command.
\end {description}

\item [READIUE]
Read a RAW, GPHOT or PHOT IUE image from tape.

\begin {description}
\item Parameters:

\begin {description}
\item DRIVE -- Tape drive logical name/file name.
\item FILE -- Tape file number.
\item NLINE -- Number of IUE header lines printed.
\item DATASET -- Dataset name.
\item TYPE -- Dataset type (RAW, PHOT, GPHOT).
\item OBJECT -- Object identification text.
\item CAMERA -- Camera name (LWP, LWR, SWP).
\item IMAGE -- Image number.
\item APERTURES -- Aperture name.
\item RESOLUTION -- Spectrograph resolution mode (LORES or HIRES).
\item EXPOSURES -- Spectrum exposure time(s) (seconds).
\item THDA -- IUE camera temperature (C).
\item ITFMAX -- Pixel value on tape for ITF saturation.
\item BADITF -- Whether bad LORES SWP ITF requires correction.
\item YEAR -- Year number (A.D.).
\item MONTH -- Month number (1-12).
\item DAY -- Day number in Month.
\item NGEOM -- Number of Chebyshev terms used to represent geometry.
\item ITF -- This is the ITF generation used in the image calibration.
\end {description}

\item Description:

This command reads an IUE dataset (RAW, GPHOT or PHOT) from tape.
The DATASET parameter determines the names of VMS files that
will contain the various data components (e.g. Calibration, Image,
Image Quality).

The ITFMAX and NGEOM parameters are only prompted for if the image
data are geometrically and photometrically calibrated.
\end {description}

\item [READSIPS]
Reads MELO/MEHI from IUESIPS\#1 or IUESIPS\#2 tape/file name.

\begin {description}
\item Parameters:

\begin {description}
\item DRIVE -- Tape drive logical name.
\item FILE -- Tape file number.
\item NLINE -- Number of IUE header lines printed.
\item DATASET -- Dataset name.
\item OBJECT -- Object identification text.
\item CAMERA -- Camera name (LWP, LWR, SWP).
\item IMAGE -- Image number.
\item APERTURES -- Aperture name.
\item EXPOSURES -- Spectrum exposure time(s) (seconds).
\item THDA -- IUE camera temperature (C).
\item YEAR -- Year number (A.D.).
\item MONTH -- Month number (1-12).
\item DAY -- Day number in Month.
\end {description}

\item Description:

This command reads the MELO/MEHI product from IUESIPS\#1 or IUESIPS\#2
tape. Operation is much like READIUE, except that some parameters and
associated information are not needed. Only calibration ({\tt .UEC})
and spectrum ({\tt \_UES.SDF}) files are created. The values for
certain parameters may be obtained from the tape, in which case you
will not be prompted for them. 

\end {description}

\item [SAVE]
Write new versions for any files that have had their contents
changed during the current session.

\begin {description}
\item Parameters:

\begin {description}
\item None.
\end {description}

\item Description:

This command forces new versions to be written for dataset files
that have had their contents changed during the current session.
If there are no outstanding files then this command does nothing.
\end {description}

\item [SCAN]
This command performs a scan perpendicular to spectrograph 
dispersion.

\begin {description}
\item Parameters:

\begin {description}
\item DATASET -- Dataset name.
\item ORDERS -- This delineates a range of echelle orders.
\item SCANDIST -- Distance of HIRES scan from faceplate centre (geometric 
pixels).
\item SCANAV -- Averaging filter FWHM for image scan (geometric pixels).
\item SCANWV -- Central wavelength for LORES image scan (\AA).
\end {description}

\item Description:

This command performs a scan perpendicular to spectrograph dispersion.
The scan is performed by folding pixels with a triangle function
with FWHM of SCANAV geometric pixels along the dispersion direction.

In the case of HIRES, the SCANDIST parameter determines the distance
of the scan from the faceplate centre.

In the case of LORES, the SCANWV parameter determines the central wavelength
of the scan in Angstroms.

The algorithm used to produce scan intensities is not very good
and so quantitative results should not be sought from this command.
Its sole intention lies in providing data for aligning the spectrum.
\end {description}

\item [SETA]
Set dataset parameters that are APERTURE specific.

\begin {description}
\item Parameters:

\begin {description}
\item DATASET -- Dataset name.
\item APERTURE -- Aperture name (SAP or LAP).
\item EXPOSURE -- Spectrum exposure time (seconds).
\item FSCALE -- Flux scale factor.
\item WSHIFT -- Constant wavelength shift (\AA).
\item VSHIFT -- Velocity shift of detector relative to Sun (km/s).
\item ESHIFT -- Global echelle wavelength shift.
\item GSHIFT -- Global shift of spectrum on image (geometric pixels).
\end {description}

\item Description:

This command allows changes to be made to dataset values which are
specific to the specified APERTURE.
Items for which parameters are not specified retain their current
values.
\end {description}

\item [SETD]
Set dataset parameters which are independent of ORDER/APERTURE.

\begin {description}
\item Parameters:

\begin {description}
\item DATASET -- Dataset name.
\item OBJECT -- Object identification text.
\item THDA -- IUE camera temperature (C).
\item FIDSIZE -- Half width of fiducials (pixels).
\item BADITF -- Whether bad LORES ITF requires correction.
\item NGEOM -- Number of Chebyshev terms used to represent geometry.
\item RIPK -- Echelle ripple constant (\AA).
\item RIPA -- Ripple function scale factor.
\item XCUT -- Global echelle wavelength clipping.
\item HALTYPE -- The type of halation (order-overlap) correction used.
\item HALC -- Halation correction constant (fraction of continuum).
\item HALWC -- Wavelength for which the halation correction HALC is defined 
(\AA).
\item HALW0 -- Wavelength at which halation correction is zero (\AA).
\item HALAV -- Averaging FWHM for halation correction (gometric pixels).
\end {description}

\item Description:

This command allows changes to be made to dataset values which are
independent of any specific APERTURE/ORDER.
Items for which parameters are not specified retain their current
values.
\end {description}

\item [SETM]
Set dataset parameters that are ORDER specific.

\begin {description}
\item Parameters:

\begin {description}
\item DATASET -- Dataset name.
\item ORDER -- Echelle order number.
\item RIPK -- Echelle ripple constant (\AA).
\item RIPA -- Ripple function scale factor.
\item RIPC -- Ripple function correction polynomial.
\item WCUT -- Wavelength limits for echelle order (\AA).
\end {description}

\item Description:

This command allows changes to be made to dataset values which are
specific to the specified ORDER.
Items for which parameters are not specified retain their current
values.
\end {description}

\item [SGS]
Write names of available SGS devices at the terminal.

\begin {description}
\item Parameters:

\begin {description}
\item None.
\end {description}

\item Description:

This commands writes a list of availavle SGS device names at the terminal. See
SUN/85 for details of the SGS graphics system.
\end {description}

\item [SHOW]
Print dataset values.

\begin {description}
\item Parameters:

\begin {description}
\item DATASET -- Dataset name.
\item V -- List of items to be printed.
\end {description}

\item Description:

This command shows the values of parameters in the dataset specified
by the DATA\-SET parameter. The items to be printed are specified by the
V parameter, which is a string containing any of the following
characters:

\begin {description}
\item H -- Header and file information
\item I -- Image details
\item F -- Fiducials
\item G -- Geometry
\item D -- Dispersion
\item C -- Centroid templates
\item R -- Echelle Ripple and halation
\item A -- Absolute calibration
\item S -- Raw Spectrum
\item M -- Mean spectrum
\item * -- All of the above
\item Q -- Image data Quality summary
\end {description}

The ``*'' character needs to be placed within inverted commas. 

The V parameter is cancelled afterwards.
\end {description}

\item [TRAK]
Extract spectrum from image.

\begin {description}
\item Parameters:

\begin {description}
\item DATASET -- Dataset name.
\item ORDER -- Echelle order number.
\item APERTURE -- Aperture name (SAP or LAP).
\item NORDER -- Number of echelle orders to be processed.
\item AUTOSLIT -- Whether GSLIT, BDIST and BSLIT are determined automatically.
\item GSLIT -- Object channel limits (geometric pixels).
\item BSLIT -- Background channel half widths (geometric pixels).
\item BDIST -- Distances of background channels from centre (geometric pixels).
\item GSAMP -- Spectrum grid sampling rate (geometric pixels).
\item CUTWV -- Whether wavelength cutoff data used for extraction grid.
\item BKGIT -- Number of background smoothing iterations.
\item BKGAV -- Background averaging filter FWHM (geometric pixels).
\item BKGSD -- Discrimination level for background pixels (s.d.).
\item CENTM -- Whether pre-existing centroid template is used.
\item CENSH -- Whether the spectrum template is just shifted linearly.
\item CENSV -- Whether the spectrum template is saved in the dataset.
\item CENIT -- Number of centroid tracking iterations.
\item CENAV -- Centroid averaging filter FWHM (geometric pixels).
\item CENSD -- Significance level for signal to be used for centroids
(s.d.).
\item EXTENDED -- Whether the object is not a point source.
\item CONTINUUM -- Whether the object spectrum is expected to have a 
``continuum''.
\end {description}

\item Description:

This command extracts a spectrum from an image.
It does this by defining an evenly spaced wavelength grid along the
spectrum, and mapping pixel intensities onto this grid in object
and background channels.
The background pixels are used to form a smooth background spectrum.
The object pixels (less smooth background) are used to form the
integrated net signal for the object.

In the LORES case, the spectrum specified by the APERTURE parameter
is extracted.

In the HIRES case, the first echelle order to be extracted is
specified by the ORDER.
Up to NORDER orders are extracted, with ORDER being
decremented each time.

The wavelength grid is defined by the region of the dispersion line
contained in the image subset (faceplate).
The grid spacing is set by the GSAMP parameter value which is
the sample step in geometric pixels.
The wavelength limits can optionally be constrained within the
echelle cutoff values by specifying CUTWV=TRUE.

The object and background channel widths and positions are determined
automatically if AUTOSLIT=TRUE.
Otherwise, the object channel is specified by the values of the
GSLIT parameter, whilst the background channel positions and
widths are determined by the BDIST and BSLIT parameter values
respectively.

The EXTENDED and CONTINUUM parameters allow more precise control
over slit determinations (see the IUEDR User Guide for details).

The background spectrum is smoothed with a triangle function with a 
FWHM given in geometric pixels by the BKGAV parameter.
Once the background channel spectra have been obtained, they are
extracted a further BKGIT times.
Prior to each additional background extraction pixels which are
outside BKGSD local standard deviations are rejected.

The object spectrum is obtained by integrating pixel intensities
(less smooth background) within the object channel.
Once the object spectrum has been obtained it is extracted an
additional CENIT times, the centroid positions
from the previous extraction being used to ``follow'' the
spectrum each time.
The centroid spectrum (template) is smoothed by folding with a
triangle function, FWHM given in geometric
pixels by the CENAV parameter.
Wavelengths with flux levels below CENSD standard deviations
above background are not used in determining the centroid spectrum.

By default, the initial spectrum template is given by the dispersion
relations and the geometric shifts determined using the CGSHIFT.
However, if CENTM=TRUE, then a pre-defined template associated with
the dataset may be used as a start guess.
If CENSH=TRUE, then this template can be shifted linearly to match the
image (i.e. without changing its shape).
If CENSV=TRUE, then the final centroid spectrum is used to update the
spectrum template in the dataset.

The net flux associated with a wavelength point in the final extracted
spectrum is defined as the integral of pixel intensities over a rectangle
with dimensions given by the object channel width and the wavelength
interval.
These fluxes are scaled so that they correspond to an interval
along the wavelength direction of 1.414 geometric pixels.
This is so that the standard IUESIPS calibrations can be applied
regardless of what actual sample rate has been employed.
The integral is performed by using linear interpolation of pixel intensities.

For the extraction of large amounts of data, e.g. whole HIRES spectra,
it is advisable to run IUEDR in batch mode. Typically, the extraction
of a complete HIRES spectrum will take 23 minutes on a VAX 11/780 (approx
1 minute CPU on a DECstation 5000/125).
\end {description}
\end {description}
\newpage
%------------------------------------------------------------------------------

\section {Parameters}

There follows a detailed description for each of the parameters used by
IUEDR commands. 
The description for a particular parameter applies in any
command which uses it. 
Some parameters have default values which at initialised on invoking
IUEDR.
These default parameters are indicated in this section and their
default values are given in the Appendix. 

This release of IUEDR uses the ADAM parameter system. In this system
all the parameters and their usage is described in an interface file 
(See SG/4 for more detail). It is possible to override the default
interface file with your own local version. This permits you to tailor
the precise behavior of each parameter according to requirements.


\begin {description}

\item [ABSFILE=string]
This is the name of a VMS file containing an absolute flux calibration.
A file type of ``.ABS'' is assumed and should not be specified
explicitly.
If the file name contains a directory specification, then it should be
enclosed in quotes.
\footnote{If you are on a UNIX machine then you should ensure that the
correct case if used (upper/lower) or the file will not be found.}

\item [APERTURE=name]
This is the name of an individual aperture.
The following names have defined meanings:

\begin {quote}
\begin {description}
\item SAP -- IUE small aperture
\item LAP -- IUE large aperture
\end {description}
\end {quote}

Other apertures may also be defined.

\item [APERTURES=name]
This specifies an aperture or group of apertures.
The following names have defined meanings:

\begin {quote}
\begin {description}
\item SAP -- IUE small aperture
\item LAP -- IUE large aperture
\item BAP -- IUE both apertures (i.e. SAP and LAP together)
\end {description}
\end {quote}

\item [AUTOSLIT=boolean]
This determines whether the extraction slit is determined automatically
by the command.
When AUTOSLIT=TRUE the GSLIT, BDIST and BSLIT parameter values are
determined automatically, based on the IUE camera, resolution, aperture,
and on the values of the EXTENDED and CONTINUUM parameters.
This mode of operation is probably the best for point source objects.

This parameter has a default value of ``TRUE''.

\item [BADITF=boolean]
This parameter determines whether a correction is made to the pixel
intensities to account for errors during Ground Station ITF
calibration.
(Note that the best scientific results would be obtained 
from reprocessed data which can be obtained on request.)
The following case is handled:

\begin {quote}
\begin {description}
\item SWP,LORES -- correction of 2nd (faulty) ITF
\end {description}
\end {quote}

\item [BDIST=(number\{,number\})]
This is a pair of numbers delineating the background spectrum channel
positions during spectrum extraction.
The distances are measured in geometric pixels from the spectrum centre.

Negative distances mean ``to the left of centre'', and positive distances
mean ``to the right of centre''.

If only one value is defined, then this is taken as meaning
that the channels are positioned symmetrically
about centre.

The spectrum ``centre'' is determined by the dispersion relations, and
modified by any prevailing centroid shifts.

\item [BKGAV=number]
This is the FWHM of a triangle function filter used in folding the
pixel intensities to form the smooth background spectrum.
It is measured in geometric pixels.

This parameter has a default value of ``30.0''.

\item [BKGIT=number]
This is the number of background smoothing iterations performed during
spectrum extraction.

\begin {description}
\item BKGIT=0 means that the background is taken as the result of the
first pass of a triangle function filter with a FWHM defined by
the BKGAV parameter.

\item BKGIT=1 means that, after producing the initial estimate for the
smooth background, pixels discrepant by more that BKGSD standard
deviations are marked as ``spikes''.
The smooth background is then re-evaluated, missing out these marked
pixels.
\end {description}

Higher values of BKGIT are possible, but seldom necessary.

This parameter has a default value of ``1''.

\item [BKGSD=number]
This is the discrimination level, measured in standard deviations,
beyond which background pixels are marked as ``spikes''.
It is not used for BKGIT=0.

This parameter has a default value of ``2.0''.

\item [BLOCK=number] This is the tape block number. The first block in
a file is BLOCK=1.

This parameter has a default value of ``1''.

\item [BSLIT=(number\{,number\})]
This defines the half width of each background channel, measured
in geometric pixels.
A single value means that both channels have the same width.

\item [CAMERA=name]
This is the camera name.
The following are defined:

\begin {quote}
\begin {description}
\item LWP -- IUE long wavelength prime
\item LWR -- IUE long wavelength redundant
\item SWP -- IUE short wavelength prime
\end {description}
\end {quote}

\item [CENAV=number]
This is the FWHM of a triangle function filter used in folding the
pixel intensities to form the smooth spectrum centroid positions.
It is measured in geometric pixels.

This parameter has a default value of ``30.0''.

\item [CENIT=number]
This is the number of spectrum centroid tracking iterations performed during
spectrum extraction.

\begin {description}
\item CENIT=0 means that the spectrum position is taken directly from the
dispersion relations.

\item CENIT=1 means that the spectrum position is first taken from the
dispersion relations, but is modified to force it to
follow the spectrum centroid.
\end {description}

Higher values of CENIT are possible, but seldom necessary:  it either
works or fails.

This parameter has a default value of ``1''.

\item [CENSD=number]
This is the discrimination level, measured in standard deviations,
below which object signal is not considered significant enough
to be used to determine the centroid position.
It is not used for CENIT=0.

This parameter has a default value of ``4.0''.

\item [CENSH=boolean]
This indicates whether the spectrum signal produces a single linear
shift to the initial template.

This can be used in cases where the object signal is too weak
to provide a detailed centroid determination by moving a pre-existing
template shape into the right position.

This parameter has a default value of ``FALSE''.

\item [CENSV=boolean]
This indicates whether the spectrum template, as refined by the
object centroid during spectrum extraction, is saved in the calibration
dataset.

The primary use of this facility is in determining templates from,
say, the whole spectrum using TRAK, and subsequently using these
with LBLS, or another spectrum.

This parameter has a default value of ``FALSE''.

\item [CENTM=boolean]
This indicates whether a centroid template from the calibration dataset
is used as a start in defining the precise position of the spectrum
signal on the image.

This parameter has a default value of ``FALSE''.

\item [COL=number]
This specifies the line colour to be used for the next curve to be
plotted.
It can be an integer in the range 1 to 10, and the corresponding
colours are as follows:

\begin {quote}
\begin {description}
\item 1 -- Yellow
\item 2 -- Green
\item 3 -- Red
\item 4 -- Blue
\item 5 -- Pink
\item 6 -- Violet
\item 7 -- Turquoise
\item 8 -- Orange
\item 9 -- Light green
\item 10 -- Olive
\end {description}
\end {quote}

Lines will only appear with different colours it the device supports colour 
graphics, on other devices COL is ignored.

\item [COLROT=boolean]
This indicates whether the line colour is to be changed after the
next plot.
The initial line has colour index 1 (YELLOW), unless specified explicitly
using the COL parameter.
The sequence of colour indices goes (1, 2, 3, ..., 10, 1, 2, ...).

In commands where more than one line is plotted, COLROT determines
whether these lines have different colours.

Lines will only appear with different colours if the device supports
colour graphics; on other devices COLROT is harmless.

This parameter has a default value of ``TRUE''.

\item [CONTINUUM=boolean]
This indicates whether the object spectrum is expected to contain a
significant continuum.
It is used in conjunction with the EXTENDED parameter in determining
the positions and widths of object and background channels for
spectrum extraction from HIRES datasets.

This parameter has a default value of ``TRUE''.

\item [COVERGAP=boolean]
If after mapping an order/aperture, a grid point is marked as unusable,
then this parameter determines whether other orders/apertures
can be allowed to contribute to this grid point.

This parameter has a default value of ``FALSE''.

\item [CUTFILE=string]
This is the name of a file containing an echelle order
wavelength limits.
A file type of ``.CUT'' is assumed and should not be specified
explicitly.
If the file name contains a directory specification, then it should be
enclosed in quotes.
\footnote{If you are on a UNIX machine then you should ensure that the
correct case if used (upper/lower) or the file will not be found.}

\item [CUTWV=boolean]
This indicates whether any available echelle order wavelength cutoff
limits are to be used for the spectrum extraction wavelength grid
limits.
Highly recommended, provided that you are happy with these wavelength limits.

This parameter has a default value of ``TRUE''.

\item [DATASET=filename]
This is the root name of the files containing the dataset.
The file type (e.g. ``.SDF'') should not be given in the DATASET
name.
If the file name contains a directory specification, then it
should be enclosed in quotes.
\footnote{If you are on a UNIX machine then you should ensure that the
correct case if used (upper/lower) or the file will not be found.}

Note that the actual filename have an additional 4 characters appended
to the name to define their contents (eg, LWP12345\_UES - spectrum).

\item [DAY=number]
This is the day number, measured from the start of the month, used
for constructing dates. 
The DAY, MONTH and YEAR parameters refer to the date the IUE
observations were made and are important to the calibration of the
data.

\item [DEVICE=name]
This is the GKS/SGS graphics device.
A list of possible GKS workstations may be found in SUN/83.
A list of SGS workstation names available at your 
site may be obtained either by a null response to the DEVICE parameter 
prompt, i.e. ``!'', or by using the IUEDR Command SGS.

\item [DISPFILE=string]
This is the name of a file containing dispersion data.
A file type of ``.DSP'' is assumed and should not be specified
explicitly.
If the file name contains a directory specification, then it should be
enclosed in quotes.
\footnote{If you are on a UNIX machine then you should ensure that the
correct case if used (upper/lower) or the file will not be found.}

\item [DRIVE=name] This is the name of the tape drive. Feasible values
are `TAPE', `MUx0',`MSx0' and `MTx1'. Where `x' is A,B,C,D
etc.

On VMS these will be logical names pointing to the actual device.

On UNIX they will be environment variables which translate to the
name of the appropriate device driver in the /dev directory.

This parameter has a default value of ``TAPE''.

This version of IUEDR additionally supports the direct reading of 
IUEDR data from disk files which have the same format as those on
GO format tapes. 

In order to read from directly from such a file (probably grabbed from
an on-line archive such as NDADSA), you specify its name directly in
response to the DRIVE parameter prompt.

If the file is not in the current directory then you must provide 
the full pathname (remember UNIX systems are CaSe-sensitive).
Access to files on NFS (Network Filing System) drives is also
supported (See Using NFS).

\item [ESHIFT=number]
This is a global wavelength shift applied to the wavelengths in
echelle spectral orders.
It is measured in Angstroms, and affects the spectrum wavelengths as follows:

\begin {equation}
\lambda _{new} = \lambda _{old} + \frac{ESHIFT}{ORDER}
\end {equation}

This is designed to account for wavelength errors that result
from a global linear shift of the spectrum format on the
image.

\item [EXPOSURE=number]
This is the exposure time associated with the spectrum, measured in seconds.
If there is more than one aperture, then this time applies
to that specified by the APERTURE parameter.

\item [EXPOSURES=(number\{,number\})]
This is one or more exposure times associated with the 
spectrum, measured in seconds.
There is an exposure time for each aperture defined.

\item [EXTENDED=boolean]
This indicates whether the object spectrum is expected to be extended,
rather than a point source.
It is used in conjunction with the CONTINUUM parameter in determining
the positions and widths of the object and background channels used
for spectrum extraction from HIRES datasets.

This parameter has a default value of ``FALSE''.

\item [FIDFILE=string]
This is the name of a file containing fiducial positions.
A file type of ``.FID'' is assumed and should not be specified
explicitly.
If the file name contains a directory specification, then it should be
enclosed in quotes.

\item [FIDSIZE=number]
This is the half width of a fiducial measured in pixel units. The fiducials 
are considered to be square.

\item [FILE=number]
This is the tape file number.
The first file on a tape would be FILE=1.
One case which may present some problems is
that of a tape with a an end-of-volume (EOV) mark in the middle
and with valuable data beyond.
An EOV is two consecutive tape marks (sometimes called ``file marks'').
A file is defined here as the information between two tape marks.
So if the number for a real file before EOV is FILEN, then
the number of the next real file following the EOV is (FILEN+2).

\item [FILLGAP=boolean]
If a grid point in the mean spectrum would have had a contribution
from a bad data point, this parameter determines whether that
grid point is marked as unusable within the context
of the order or aperture being mapped.
If the grid point is marked as unusable in this way then other
good points cannot contribute to it.

This parameter has a default value of ``FALSE''.

\item [FLAG=boolean]
This specifies whether the data quality information is displayed
along with the image.
If so, then faulty pixels will be marked with a colour according to
the following scheme:

\begin {quote}
\begin {description}
\item GREEN -- pixels affected by reseau marks
\item RED -- pixels which are saturated (DN=255)
\item ORANGE -- pixels affacted by ITF truncation
\item YELLOW -- pixels marked bad by the user
\end {description}
\end {quote}

If a pixel is affected by more than one of the above faults, then
the first in the list is adopted for display.
Hence, user edits are only shown where no other fault is present.

This option would normally only be used when assessing the quality
of faulty pixels, possibly with a view to using them, i.e. marking them
``good'' with a cursor editor.

This parameter has a default value of ``TRUE''.

\item [FN=number]
This parameter is the replacement Flux Number for a pixel changed
explicitly by the user.

\item [FSCALE=number]
This is an arbitrary scale factor applied to spectrum fluxes.
It affects fluxes as follows:

\begin {equation}
{\cal F}_{new} = {\cal F}_{new} \times FSCALE
\end {equation}

It finds application in accounting for grey attentuation, or obscuration
of object signal through a narrow aperture.

\item [GSAMP=number]
This is the sampling rate used for spectrum extraction.
It is measured in geometric pixels.
GSAMP=1.414 corresponds to the original IUESIPS (\#1) sampling
rate, while GSAMP=0.707 corresponds to the new IUESIPS (\#2) sampling
rate.
Other values can be chosen.

This parameter has a default value of ``1.414''.

\item [GSHIFT=(number,number)]
This is a global constant shift of the spectrum format on the image, 
$(dx,dx)$, where the geometric coordinates, $(x,y)$ of a spectrum position
are

\begin {equation}
x_{new} = x_{old} + dx
\end {equation}

and

\begin {equation}
y_{new} = y_{old} + dy
\end {equation}

\item [GSLIT=(number\{,number\})]
This is a pair of numbers delineating the object spectrum channel
during spectrum extraction.
The distances are measured in geometric pixels.

Negative distances mean ``to the left of centre'', and positive distances
mean ``to the right of centre''.

Object channels that do not cover the actual object signal on the
image will not be meaningful when centroid tracking is employed.

If only one value is defined, then this is taken as representing
a channel that is symmetrical about the spectrum centre.

The spectrum ``centre'' is determined by the dispersion relations 
modified by any prevailing centroid shifts.

\item [HALAV=number]
This is the FWHM of a triangle function used for smoothing the
net spectrum for the HALTYPE=POWER halation correction technique.

\item [HALC=number]
This is the Halation correction constant used for HALTYPE=POWER
cases, and defined at wavelength HALWC.
The value of the correction constant
corresponds roughly to the measured depression of a broad
zero intensity absorption below ``zero'', in units
of the continuum in adjacent orders.
The ``constant'', $C$, varies with wavelength as follows:

\begin {equation}
C_\lambda = \frac {HALC \times (\lambda - HALW0)}{(HALWC - HALW0)}
\end {equation}

See the HALTYPE, HALWC, HALW0, HALAV parameters.

\item [HALTYPE=name]
This is the type of Halation or order-overlap correction applied to the
flux spectrum.
It can take the value

\begin {quote}
\begin {description}
\item POWER -- correction based on power-law PSF decay
\end {description}
\end {quote}

\item [HALW0=number]
This is the wavelength, measured in Angstroms, at which the
halation correction is zero.

See the HALTYPE, HALC and HALWC parameters.

\item [HALWC=number]
This is the wavelength, measured in Angstroms, at which the
halation correction is HALC.

See the HALTYPE, HALC and HALW0 parameters.

\item [HIST=boolean]
This determines whether lines are plotted as histograms rather than
continuous lines (polylines).

This parameter has a default value of ``TRUE''.

\item [IMAGE=number]
This is the Image Sequence Number.

\item [ITF=number]
This is the ITF generation used in the photometric calibration of the 
image. This information is needed for the correct absolute flux calibration 
of the resulting spectra. Possible values for each camera are as follows:

\begin {quote}
\begin {description}
\item SWP -- 2
\item LWR -- 1 and 2
\item LWP -- 1 and 2
\end {description}
\end {quote}

The appropriate value can be determined from inspection of the
IUE header text for the GPHOT/PHOT file using the table
of numbers following the line beginning ``PCF C/**''.
Here are the ITF values associated with various forms of this
table:

\begin {quote}
\begin {tabbing}
xx\=1800xx\=3700xx\=5600xx\=...xx\=30000xxx\=
(Corrected, 3rd SWP ITF)xx\=ITFxx\kill
\>\>{\bf TABLE}\>\>\>\>{\bf IDENTIFICATION}\>{\bf ITF}\\
\\
0\>1800\>3700\>5600\>...\>30000\>Preliminary LWR ITF\>ITF0\\
0\>2303\>4069\>8008\>...\>42032\>2nd LWR ITF\>ITF1\\
0\>2300\>3969\>6062\>...\>32973\>1st LWP ITF\>ITF1\\
0\>2723\>5429\>8145\>...\>38389\>2nd LWP ITF\>ITF2\\
0\>1800\>3600\>5500\>...\>\>Preliminary SWP ITF\>ITF0\\
0\>1753\>3461\>6936\>...\>28674\>Faulty, 2nd SWP ITF\>ITF1\\
0\>1684\>3374\>6873\>...\>28500\>Corrected, 3rd SWP ITF\>ITF2\\
\end {tabbing}
\end {quote}

If the ITF table used has no corresponding absolute flux calibration within
IUEDR, e.g. LWR ITF0 or SWP ITF0, you are advised to contact the IUE Project.
Although the BADITF parameter is available for data calibrated using SWP ITF1, 
it is advisable to have these data reprocessed by the IUE Project. 

\item [ITFMAX=number]
This is the pixel value on tape corresponding to ITF saturation.
Its value is fixed for a given ITF table.
The value of ITFMAX is only needed for IUE images of type ``GPHOT''.
The appropriate value can be determined from inspection of the
IUE header text for the GPHOT file using the table
of numbers following the line beginning ``PCF C/**''.

\begin {quote}
\begin {tabbing}
ITFMAXxxx\=0xx\=1800xx\=3700xx\=5600xx\=...xx\=30000xxx\=
(Corrected, 3rd SWP ITF)\kill
{\bf ITFMAX}\>\>\>{\bf TABLE}\>\>\>\>{\bf IDENTIFICATION}\\
\\
20000\>0\>1800\>3700\>5600\>...\>30000\>Preliminary LWR ITF\\
27220\>0\>2303\>4069\>8008\>...\>42032\>2nd LWR ITF\\
19983\>0\>1800\>3600\>5500\>...\>\>Preliminary SWP ITF\\
19740\>0\>1753\>3461\>6936\>...\>28674\>Faulty, 2nd SWP ITF\\
19632\>0\>1684\>3374\>6873\>...\>28500\>Corrected, 3rd SWP ITF\\
\end {tabbing}
\end {quote}

\item [LINE=name]
This specifies the line style to be used for the next curve to be
plotted.
It can be one of the following:

\begin {quote}
\begin {description}
\item SOLID -- solid (continuous) line
\item DASH -- dashed line
\item DOTDASH -- dot-dash line
\item DOT -- dotted line
\end {description}
\end {quote}

The order of these is that invoked when automatic line style rotation
is in effect (see the LINEROT parameter).

\item [LINEROT=boolean]
This indicates whether the line style is to be changed after the
next plot.
The initial line style is SOLID, unless specified explicitly
using the LINE parameter.
The sequence of line styles goes (SOLID, DASH, DOTDASH, DOT, SOLID,
DASH, ...).

In commands where more than one line is plotted, LINEROT determines
whether these lines have different styles.

This parameter has a default value of ``FALSE''.

\item [ML=(number,number)]
This is a pair of wavelength values defining the start and end of
the mean spectrum grid.
The grid will consist of evenly spaced vacuum wavelengths between these
values.

\item [MONTH=number]
This is the month number, measured from the start of the Year,
used in constructing dates.

\item [MSAMP=number]
This is the vacuum wavelength sampling rate for the mean spectrum
grid.
If it does not fit an integral number of times into the grid limits,
then the latter are adjusted to fit.

\item [NFILE=number]
This is the number of tape files to be processed.
A value of ``-1'' means all files until the end.

This parameter has a default value of ``1''.

\item [NGEOM=number]
This is the number of Chebyshev terms used to represent the
geometrical distortion.
The same value is used for each axis direction.

\item [NLINE=number]
This is the number of IUE header lines printed.
A value of ``-1'' means the entire header is printed.

This parameter has a default value of ``10''.

\item [NORDER=number]
This is the number of echelle orders to be processed by a command.

This parameter has a default value of ``0''.

\item [NSKIP=number]
This is the number of tape marks to be skipped.

This parameter has a default value of ``1''.

\item [OBJECT=string]
This is a string containing text to identify the ``object'' of the
observation.
It can also contain information about the observation
(e.g. camera, image, ...) if required.
The maximum allowed length of the string is 40 characters.

\item [ORDER=number]
This is the echelle order number.

\item [ORDERS=(number\{,number\})]
This is a pair of echelle order numbers delineating a range.
If only a single value is specified, then the range consists of that
order only.
The sequence of the two numbers is not significant.
The useful maximum range for each camera is as follows:

\begin {quote}
\begin {description}
\item SWP -- orders 66 to 125
\item LWR -- orders 72 to 125
\item LWP -- orders 72 to 125
\end {description}
\end {quote}

\item [OUTFILE=string]
This is the name of a file to receive the output spectrum.
\footnote{If you are on a UNIX machine then you should ensure that the
correct case if used (upper/lower) or the file will not be found.}
This release of IUEDR uses the STARLINK NDF format for all output
spectra. This means that all standard STARLINK packages can be used
to plot/display/analyse the spectra, in particular some of the
facilites of KAPPA and FIGARO may prove useful to the general user.

\item [PRINT=number]
This is determines how much printable information is produced.
Higher values of PRINT produce more printout.

\item [QUAL=boolean]
This specifies whether the data quality information is plotted
along with the data.
If so, then faulty points will be marked with their data quality
severity code, which is one from:

\begin {quote}
\begin {description}
\item 1 -- affected by extrapolated ITF
\item 2 -- affected by microphonics
\item 3 -- affected by spike
\item 4 -- affected by bright point (or user)
\item 5 -- affected by reseau mark
\item 6 -- affected by ITF truncation
\item 7 -- affected by saturation
\item U -- affected by user edit
\end {description}
\end {quote}

User edits are only shown where no other fault is present.

This option would normally only be used when assessing the quality
of faulty points, possibly with a view to using them, i.e. marking them
``good'' with a cursor editor.

This parameter has a default value of ``TRUE''.

\item [RESOLUTION=name]
This is the spectrograph resolution mode.
The following modes are defined:

\begin {quote}
\begin {description}
\item LORES -- IUE Low Resolution
\item HIRES -- IUE High Resolution (echelle mode)
\end {description}
\end {quote}

\item [RIPA=number]
This is an empirical scale factor that can be used to modify the
echelle ripple function.
The normal value is ``1''.
The primary component of the ripple function is

\begin {equation}
SCALE = (\frac{\sin x}{x})^2
\end {equation}

where

\begin {equation}
x = \frac {\pi \times RIPA \times (\lambda - \lambda_c) \times ORDER}
{\lambda_c}
\end {equation}

and

\begin {equation}
\lambda_c = \frac {RIPK}{ORDER}
\end {equation}

The net spectrum is divided by SCALE above.
Empirical values of RIPA can be used to optimise the ripple correction.

See the RIPK and RIPC parameter description.

\item [RIPC=(number\{,number\})]
This is a polyomial in ``$x$'' used to modify the standard echelle
ripple calibration function.
The calibration is given by

\begin {equation}
SCALE = (\frac {\sin x}{x})^2 \times (RIPC(1) + RIPC(2) \times x 
+ RIPC(3) \times x^2 ...)
\end {equation}

where

\begin {equation}
x = \frac {\pi \times RIPA \times (\lambda - \lambda_c) \times ORDER}
{\lambda_c}
\end {equation}

and

\begin {equation}
\lambda_c = \frac {RIPK}{ORDER}
\end {equation}

The net spectrum is divided by SCALE above.

See the RIPK and RIPA parameter descriptions.

\item [RIPFILE=string]
This is the name of a file containing an echelle ripple calibration.
A file type of ``.RIP'' is assumed and should not be specified
explicitly.
If the file name contains a directory specification, then it should be
enclosed in quotes.
\footnote{If you are on a UNIX machine then you should ensure that the
correct case if used (upper/lower) or the file will not be found.}

\item [RIPK=(number\{,number\})]
This is the echelle ripple constant measured in Angstroms.
It corresponds to the central wavelength of echelle order number 1.
The central wavelength of an arbitrary ORDER is

\begin {equation}
\lambda_c = \frac {RIPK}{ORDER}
\end {equation}

Where this parameter is used for an entire HIRES dataset, the
parameter can have more than one value, and represent a polynomial
in ORDER

\begin {equation}
\lambda_c 
= \frac {(RIPK(1) + RIPK(2) \times ORDER + RIPK(3) \times ORDER^2 + ...)}
{ORDER}
\end {equation}

\item [RL=(number,number)]
This is a pair of radial coordinate values defining the start and end of
the radial grid in an LBLS array.
These radial coordinates are measured in geometric pixels, and run
perpendicular to the dispersion direction.
A coordinate value of ``0.0'' corresponds to the centre of the spectrum.

Values (0.0,0.0) indicate that internal defaults are to be adopted.
A single value is reflected symmetrically about 0.0.

\item [RM=boolean]
This determines whether the mean spectrum is reset before a mapping
takes place.
If the spectrum is not reset, then the spectra being mapped will be
averaged with the existing mean spectrum.

This parameter has a default value of ``TRUE''.

\item [RS=boolean]
This determines whether the display screen is reset before plotting.

This parameter has a default value of ``TRUE''.

\item [RSAMP=number]
This is the sample spacing used for the radial grid in the LBLS array.
If it does not fit an integral number of times into the grid limits, RL,
then the latter are adjusted to fit.

Suggested values range from 0.707 to 1.414 pixels, the latter
corresponding to the IUESIPS LBLS grid.

This parameter has a default value of ``1.414''.

\item [SCANAV=number]
This is the FWHM of a triangle function with which pixels are folded
during the generation of a scan across the image perpendicular to
spectrograph dispersion.
It is measured in geometric pixels.

This parameter has a default value of ``5''.

\item [SCANDIST=number]
This is the distance of a scan across a HIRES image from the faceplate
centre.
It is measured in geometric pixels.

\item [SCANWV=number]
This is the central wavelength for a scan of a LORES image
perpendicular to spectrograph dispersion.
It is measured in Angstroms in vacuo.

\item [SKIPNEXT=boolean]
This determines whether the tape is positioned at the start of the
next file after processing.
If only the start of a file is being processed, then by setting
SKIPNEXT=FALSE time can be saved.

This parameter has a default value of ``FALSE''.

\item [SPECTYPE=number]
This is the type of file, in the DIPSO ``SP'' format terminology, to be
created. The following values are allowed: 

\begin {quote}
\begin {description}
\item 0 -- unformatted (binary) file
\item 1 -- fixed format file
\item 2 -- free format file
\end {description}
\end {quote}

It is recommended that datasets with many points be written with
SPECTYPE=0, which is more efficient in disk space and time spent
reading and writing.
A description of the format of each of these file types can be found
in the documentation for the DIPSO package, SUN/50.

This parameter has a default value of ``0''.

\item [TEMFILE=string]
This is the name of a file containing the standard spectrum template
data.
A  file type of ``.TEM'' is assumed and should not be specified
explicitly.
If the file name contains a directory specification, then it should be
enclosed in quotes.
\footnote{If you are on a UNIX machine then you should ensure that the
correct case if used (upper/lower) or the file will not be found.}

\item [THDA=number]
This is the IUE camera temperature, measured in degrees Centigrade.
It is used for such things as adjustments to fiducial positions
and spectrograph dispersion relations.
A value of 0.0 implies that no THDA value is available, the program
will then  use a suitable mean THDA for the camera being used.
Values for the THDA can be found in the IUE header text of the final
spectrum file on the Guest Observer tape for IUESIPS\#2 ---
THDA values derived from spectrum motion are best.

\item [TYPE=name]
This is the type of dataset.
Defined values are as follows:

\begin {quote}
\begin {description}
\item RAW -- IUE raw image
\item GPHOT -- IUE GPHOT image (geometric and photometric)
\item PHOT -- IUE PHOT image (photometric only)
\end {description}
\end {quote}

Types PHOT and GPHOT are not automatically distinguishable from IUE
Guest Observer tape contents.

\item [V=string]
This is a string defining a list of items and includes
any of the following characters:

\begin {quote}
\begin {description}
\item H -- header and file information
\item I -- image details
\item F -- fiducials
\item G -- geometry
\item D -- dispersion
\item C -- centroid templates
\item R -- echelle ripple and halation
\item A -- absolute calibration
\item S -- raw spectrum
\item M -- mean spectrum
\item * -- all of the above
\item Q -- image data quality summary
\end {description}
\end {quote}

\item [VSHIFT=number]
This is the radial velocity of the detector relative to the Sun.
It is measured in km/s and affects the calibrated wavelength
scale as follows:

\begin {equation}
\lambda_{true} = \frac {\lambda_{obs}}{(1 + \frac {VSHIFT}{c})}
\end {equation}

where $c$ is the velocity of light in km/s.

\item [WCUT=(number,number)]
This is one of the mechanisms that can be used to delimit the
parts of echelle orders that are calibrated for ripple response.
The two values of this parameter are the start and end wavelengths
for a specific ORDER.

Apart from poor ripple calibration, the order ends can also be affected
by the parts of the camera faceplate that are retained in the image.

\item [WSHIFT=number]
This is a constant wavelength shift applied to spectrum wavelengths.
It is measured in Angstroms and affects the spectrum wavelengths as follows:

\begin {equation}
\lambda_{new} = \lambda_{old} + WSHIFT
\end {equation}

This is only used for LORES spectra.

\item [XCUT=(number,number)]
This is one of the mechanisms that can be used to delimit the
ends of echelle orders that are calibrated for ripple response.
The two values of this parameter are the start and end ``$x$'' coordinates
of the order, where

\begin {equation}
x = \frac {\pi \times RIPA \times (\lambda - \lambda_c) \times ORDER}
{\lambda_c}
\end {equation}

and

\begin {equation}
\lambda_c = \frac {RIPK}{ORDER}
\end {equation}

The nature of the standard ripple function is that $x$ is only
formally meaningful in the range ($-\pi$,$+\pi$).

See the RIPK, RIPC and RIPA parameter descriptions.

\item [XL=(number,number)]
This specifies the data limits used for plotting in the $x$-direction.
This parameter is only read if the display has been reset, and
the axes are being redrawn.
If both values are the same (e.g. [0,0]),
then the data limits in the $x$-direction will be determined from the
data being plotted.

\item [XP=(number,number)]
This specifies the pixel limits along the $x$-direction used for
image display.
Values in decreasing order will cause the image to be inverted
along the $x$-direction.
If the values are undefined, the pixel limits will default to include
the whole extent of the image along the $x$-direction.

\item [YEAR=number]
This is the year (A.D.) used in constructing dates.

\item [YL=(number,number)]
This specifies the data limits used for plotting in the $y$-direction.
This parameter is only read if the display has been reset, and
the axes are being redrawn.
If both values are the same (e.g. [0,0]),
then the data limits in the $y$-direction will be determined from the
data being plotted.

\item [YP=(number,number)]
This specifies the pixel limits along the $y$-direction used for
image display.
Values in decreasing order will cause the image to be inverted
along the $y$-direction.
If the values are undefined, the pixel limits will default to include
the whole extent of the image along the $y$-direction. 

\item [ZL=(number,number)]
This specifies the data limits used for display of images.
If the values are given in decreasing order, then high data
values will be represented by low (dark) display intensities,
and vice-versa.
If the values are undefined, then the full intensity range of the
image will be used.

Data values which fall at or below the lowest display intensity are drawn
BLACK, those which are at the highest display intensity are drawn
WHITE and those which are above the highest display intensity are
drawn BLUE.

\item [ZONE=number]
This specifies the zone to be used for plotting.
The zone numbers range
from 0 to 8 and correspond to those defined by the TZONE command in 
DIPSO (i.e. SUN/50), e.g.

\begin {quote}
\begin {description}
\item 0 -- entire display surface
\item 1 -- top left quarter
\item 2 -- top right quarter
\item 3 -- bottom left quarter
\item 4 -- bottom right quarter
\item 5 -- top half
\item 6 -- bottom half
\item 7 -- left half
\item 8 -- right half
\end {description}
\end {quote}

This parameter has a default value of ``0''.

\end {description}
\newpage
%------------------------------------------------------------------------------

\appendix
\section {Parameter Defaults}

The following parameters have default values:

\begin {description}
\item [AUTOSLIT=TRUE] Whether GSLIT, BDIST and BSLIT are determined 
automatically.
\item [BKGAV=30.0] Background averaging filter FWHM (geometric pixels).
\item [BKGIT=1] Number of background smoothing iterations.
\item [BKGSD=2.0] Discrimination level for background pixels (standard 
deviations).
\item [BLOCK=1] Tape block number.
\item [CENAV=30.0] Centroid averaging filter FWHM (geometric pixels).
\item [CENIT=1] Number of centroid tracking iterations.
\item [CENSD=4.0] Discrimination level for signal to be used for centroids 
(standard deviations).
\item [CENSH=FALSE] Whether the spectrum template is just shifted linearly.
\item [CENSV=FALSE] Whether the spectrum template is saved in the dataset.
\item [CENTM=FALSE] Whether an existing centroid template is used.
\item [COLROT=TRUE] Whether the line colour is changed after the next plot.
\item [CONTINUUM=TRUE] Whether the object spectrum is expected to have a 
``continuum''.
\item [COVERGAP=FALSE] Whether gaps can be filled by covering orders.
\item [CUTWV=TRUE] Whether wavelength cutoff data is to be used for the 
extraction grid.
\item [DRIVE=TAPE] Tape drive logical name.
\item [EXTENDED=FALSE] Whether the object spectrum is expected to be 
extended.
\item [FILLGAP=FALSE] Whether gaps can be filled within an order.
\item [FLAG=TRUE] Whether data quality for faulty pixels is displayed.
\item [GSAMP=1.414] Spectrum grid sampling rate (geometric pixels).
\item [HIST=TRUE] Whether lines are to be drawn as histograms during 
plotting.
\item [LINEROT=FALSE] Whether line style is to be changed after the next 
plot.
\item [NFILE=1] Number of tape files to be processed.
\item [NLINE=10] Number of IUE header lines to be printed.
\item [NORDER=0] Number of echelle orders to be processed.
\item [NSKIP=1] Number of tape marks to be skipped over.
\item [QUAL=TRUE] Whether data quality information is plotted.
\item [RM=TRUE] Whether the mean spectrum is reset before averaging.
\item [RS=TRUE] Whether the display is reset before plotting.
\item [RSAMP=1.414] Radial coordinate sampling rate for LBLS grid (pixels).
\item [SCANAV=5] Averaging filter FWHM for image scan (goemetric pixels).
\item [SKIPNEXT=FALSE] Whether skip to next tape file.
\item [SPECTYPE=0] DIPSO ``SP'' format file type (0, 1 or 2).
\item [ZONE=0] Graphics zone to be used for plotting.
\end {description}
\end {document}

