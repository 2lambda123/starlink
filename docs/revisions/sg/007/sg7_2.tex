\documentstyle[11pt]{article} 
\pagestyle{myheadings}

%------------------------------------------------------------------------------
\newcommand{\stardoccategory}  {Starlink Guide}
\newcommand{\stardocinitials}  {SG}
\newcommand{\stardocnumber}    {7.2}
\newcommand{\stardocauthors}   {R W Tweedy}
\newcommand{\stardocdate}      {9 March 1992}
\newcommand{\stardoctitle}     {IUE Analysis --- A Tutorial}
%------------------------------------------------------------------------------

\newcommand{\stardocname}{\stardocinitials /\stardocnumber}
\newcommand{\numcir}[1]{\mbox{\hspace{3ex}$\bigcirc$\hspace{-1.7ex}{\small #1}}}
\newcommand{\lsk}{\raisebox{-0.4ex}{\rm *}}
\renewcommand{\_}{{\tt\char'137}}     % re-centres the underscore
\markright{\stardocname}
\setlength{\textwidth}{160mm}
\setlength{\textheight}{230mm}
\setlength{\topmargin}{-2mm}
\setlength{\oddsidemargin}{0mm}
\setlength{\evensidemargin}{0mm}
\setlength{\parindent}{0mm}
\setlength{\parskip}{\medskipamount}
\setlength{\unitlength}{1mm}

\begin{document}
\thispagestyle{empty}
SCIENCE \& ENGINEERING RESEARCH COUNCIL \hfill \stardocname\\
RUTHERFORD APPLETON LABORATORY\\
{\large\bf Starlink Project\\}
{\large\bf \stardoccategory\ \stardocnumber}
\begin{flushright}
\stardocauthors\\
\stardocdate
\end{flushright}
\vspace{-4mm}
\rule{\textwidth}{0.5mm}
\vspace{5mm}
\begin{center}
{\Large\bf \stardoctitle}
\end{center}
\vspace{5mm}

\setlength{\parskip}{0mm}
\tableofcontents
\setlength{\parskip}{\medskipamount}
\markright{\stardocname}

\newpage

\section{Introduction}

The techniques for analysing IUE data have evolved over the fourteen 
years of the satellite's existence, and are consequently well-developed. 
However, it remains difficult for someone new to the area to become familiar
with the procedures. This document is designed for someone acquiring their
first IUE tape, and gives a step-by-step guide from mounting the tape to 
doing simple analysis of the extracted spectrum. 

\section{The IUE satellite}

The {\it International Ultraviolet Explorer}\, satellite (IUE) was launched in
January 1978 as a result of a three-agency collaboration between SERC, ESA,
and NASA. It was put into a 24-hour elliptical orbit so that while it is 
permanently accessible to the Goddard Space Flight Center, it can also be
operated for eight hours a day by the Vilspa ground-station at Villafranca 
del Castillo, near Madrid. This is the ESA shift; the remaining 16 hours are
divided into two shifts, US1 and US2, with the latter spanning the period 
characterised by a high particle background. 

IUE observes in the range 1150-2000{\AA}, covered by the Short Wavelength Prime
camera (SWP), and between 1900 and 3200{\AA}, which involves the Long 
Wavelength Prime and Redundant cameras (LWP and LWR). (The SWR has a faulty 
readout section, so consequently has never been used since the initial 
commissioning period). Two types of spectrograph exist: one uses a single 
spherical grating, producing low dispersion spectra with a resolution of
$\sim$6{\AA}; the other involves an echelle grating as well, providing about 
60 spectral orders, and a resolution $\sim$0.1{\AA}. Only the low-dispersion 
grating provides flux calibration because at the short-wavelength 
end the orders from the echelle spectrum begin to overlap, so that the 
background cannot be properly defined. 

During an observation, the spectrum appears as an image on the target of the 
television cameras. There is a limit to the amount of charge that can be 
accumulated there without saturation occurring, which means that the optimum
signal-to-noise is about 20. This is quite difficult to 
achieve, largely because of the varying sensitivity of the cameras at different
wavelengths, and figures of around 10 are more typical. Consequently, even 
in high-dispersion only the centroid and equivalent width of a 
narrow absorption line can be used reliably, unless several datasets are 
used and carefully co-added. 

Full details of the IUE satellite just after launch are given in 
Boggess, A. et al., 1978, {\it Nature}, {\bf 275}, 372, with the subsequent
series of papers giving an overview of the science that can be achieved with 
it. There are frequent symposia reviewing recent research, including 
{\em `Exploring the Universe with the IUE satellite'}, ed.\ Y.Kondo,
publ.\ D.Reidel, 1987, and {\em `Evolution in Astrophysics: IUE Astronomy in
the era of new space missions'}, ed.\ E.J.Rolfe, ESA SP-310.

\section{Reading the tape and displaying an image}

A tape from a single observation contains four files. The first contains the 
raw data, and is of little practical use, but the second contains the 
photometrically calibrated spectrum (the PHOT file), which is used with the
Starlink IUE data-reduction package IUEDR (the {\it Starlink Guide no.3}
gives a complete description of the commands available). The third and fourth 
are the extracted data using the standard NASA program IUESIPS for both 
high and low dispersion, irrespective of whether both modes were used in the
observation. Although this represents a quick way to look at the data, it is
scientifically unusable in high-dispersion. In low-dispersion it is usually 
preferable to do the extraction in a more controlled way, and thus account
for odd features (such as cosmic ray hits, or broad emission lines). 

\subsection{Reading the tape headers}

The tape should be mounted in the normal way, both on the device itself and 
within VMS. Various logicals need to be set up, so this should be done using
\begin{verbatim}
    $ IUEDRSTART
\end{verbatim}
which then gives some brief information. Entering IUEDR can then be done
\begin{verbatim}
    $ IUEDR
\end{verbatim}
which, after a ``welcome'' message, gives a `$>$' prompt. There is a moderately
useful HELP facility, if rather slow. Each time IUEDR is run, a SESSION.LIS
file is created which records all the input from the keyboard and output from 
the terminal. The command for reading the header is
LISTIUE, but unless its default parameters are changed this 
will give just the initial ten lines of the first file. The most useful
approach is to read all the headers, which can be done using 
\begin{verbatim}
    > LISTIUE ?
\end{verbatim}
--- the question mark may be used with all commands in IUEDR to request all the 
prompts. Typical responses might be 
\begin{quote}
\begin{verbatim}
? DRIVE(TAPE)=MSA0
? FILE=1                    *** i.e. the first file to be read.
? NLINE(10)=-1              *** all lines to be listed.
? SKIPNEXT(F)=<CR>          *** don't skip next file.
? NFILE(1)=-1               *** all files to be listed.
\end{verbatim}
\end{quote}
--- the ``--1'' response is unique to this command. Much of what is produced is
rubbish, and once LISTIUE has completed it is probably best
to {\tt QUIT} from IUEDR and delete large sections from the 
SESSION.LIS file before sending to a printer. While the first ten lines are
the most useful, important information is included at the tail of the IUESIPS
files, which is why it is necessary to extract the full header information. 

\subsection{Reading the data}

Re-entering IUEDR, data may be extracted from the tape by using READIUE. 
Since this package is about ten years old, there are some prompts that 
would not be necessary in a more automated system, and responses are 
often not obvious. An example of running it is shown below. 
\begin{quote}
\begin{verbatim}  
> readiue ?
? DRIVE(msa0)=mua1     
? FILE=3
  Tape is positioned at the start of file 3
? NLINE(10)=<CR>            *** typical header shown below; 10 lines.
                           895 895 7681536   1 2 013038776   +101     1  C
    9152*   2*IUESOC  *   *   *20,400*      *   *  * * * * * *     *  2  C
  SWP 38776, LT-5, 340 MIN EXPO, HI DISP, LARGE APERTURE              3  C
                                                                      4  C
                                                                      5  C
                                                                      6  C
  OBSERVER: CHENG    ID: PNMWF    12 MAY 1990 DAY 132                 7  C
                                                                      8  C
                                                                      9  C
  90132142546* 10  * 218 *OPS2PR12*142042 FES CTS 98 1 0 2560      * 10  C
  VICAR Image is 768 records,  each consisting 1536 bytes
  This is either a PHOT or a GPHOT Image
? TYPE=phot
 Assumed to be PHOT Image.
? DATASET=lt5              *** i.e. the name for the output files.
? CAMERA=swp               *** or LWP or LWR
? IMAGE=38776              *** the number of the image
? RESOLUTION=hires         *** or lores
? APERTURES=lap            *** or sap for small aperture, or bap for both
? EXPOSURES=20400          *** exposure time in seconds; line 2 of header
? YEAR=1990
? MONTH=5
? DAY=12
? OBJECT="LT-5 : 340 mins" *** title for the output files. 
? THDA=7.83                *** temperature of the camera faceplate
? NGEOM(5)=<CR>            *** number of Chebyshev polynomial co-efficients
 No Absolute Calibration.
 No Spectrum Template Data.
 Reading Image from Tape.
 Calibrating Geometry.
 Writing lt5.UEC (Calibration File).
 Writing lt5.UED (Image File).
 Writing lt5.UEQ (Image Quality File).
\end{verbatim}
\end{quote}
Most of the tapes will include PHOT files, but prior to about 1981, geometric
corrections were applied at the same time as the photometric ones.
Consequently, GPHOT files appear instead, which require a different procedure. 
The program will ask for {\tt ITFMAX} which is obtainable by using the 
header information, combined with information in the HELP facility: type
``?'' at the {\tt ITFMAX} prompt. It has been found that correcting for 
the geometric distortions in the image is best done separately, which
is why the Chebyshev polynomial co-efficients are required. There is rarely
any need to change from the default. The {\tt THDA} parameter needs to be
obtained from the IUESIPS output header information. Two values are given, 
one derived from reseau motion, the other from spectrum motion; it is 
believed that the former gives the more accurate result where there is a
difference. 

At present the only image display devices that can be used are the IKONs. 
The next release of IUEDR should include Vax workstations amongst a number
of other improvements, but it is completely uncertain when this will appear. 
It is quite possible that there will be no modifications until the whole
package is overhauled. There are two reasons why this is expected: the 
graphics packages are still Fortran-based, and thus agonisingly slow compared
to, say, ASTERIX or Q; and the basic techniques of IUE data-reduction are 
being transformed for the creation of the final archive. 

For low-resolution data there are two extra prompts: `{\tt ? ITF}' requires
the answer `2' for SWP spectra, and `1' or `2' for those from the LWP or LWR.
`{\tt ? BADITF}' should almost always be answered `YES'; this 
counter-intuitive reply requests that bad data points will be quality-flagged.

Displaying the image uses {\tt DRIMAGE}, or {\tt DRI} for short. 
\begin{quote}
\begin{verbatim}
 > dri ?
 ? DATASET(lt5)=<CR>
 ? DEVICE=ikon
 ? FLAG(T)=<CR>
 ? XP=!              *** exclamation marks are used to request the
 ? YP=!                  default if graphics are being used. 
 ? ZL=!              *** limits on the intensity displayed. The default 
                         is about 0 to 10000. 
\end{verbatim}
\end{quote}
Changing the limits can be done using {\tt CULIM}, and then once the image
has been re-displayed {\tt CURSOR} will allow wavelengths and other information
to be read directly. 
\begin{quote}
\begin{verbatim}
 > culim
 > dri
 > cursor
  S(PIXEL)    L(PIXEL)     Inten.(FN)   R(PIXEL)     Wave(A)      Order
  279.89      264.732      2146.        -3.3937      1528.42      90
  275.23      272.225      1896.        -1.32646     1545.71      89
  200.673     183.81       1922.        -3.15636     1540.92      89
\end{verbatim}
\end{quote}
Within all but those with the shortest exposures, there will be defects
in the image that need to be quality-flagged, and which cannot be done
automatically. This mostly involves cosmic-ray hits, which are usually 
distinguished from, say, emission lines by affecting far fewer pixels ---
as well as often occupying the inter-order region. A typical emission line
is the geocoronal Lyman $\alpha$ at 1215{\AA}, visible in orders 113 and 114, 
which significantly contaminates all large-aperture spectra. 
The other main problem that is found occasionally is the 
result of a telemetry error when the spectrum was read down from the 
satellite, so that a strip of data may be missing. Quality-flagging may be done 
using the command {\tt EDIMAGE}, or {\tt EDIM}, and using the mouse. It will 
be necessary to expand the image to a size that makes it easy to distinguish
individual pixels --- usually $200\times 200$ is a good compromise between
compactness and usability. Occasionally an image will be strewn with cosmic-ray
hits, and unfortunately there is no substitute for going through the image 
systematically. In this case, rather than using {\tt DRI}, {\tt CULIM},
and {\tt EDIM}, it is probably simpler to ignore the {\tt CULIM} stage and
use the {\tt XL} and {\tt YL} options in {\tt DRI}. {\tt EDIM} works by 
changing the quality of all pixels within a box defined by the user using
cursor hits at opposite corners of the box required. The middle button of 
the mouse marks the pixels ``bad'', whereas the left marks them ``good'' ---
which can thus be used to rectify mistakes. The third button exits from 
{\tt EDIM}. 

Pixels defined ``bad'' in this way will be coloured yellow. Other colours may
be present on the image: red signifies overexposed pixels, and blue those 
that fall outside the intensity range for plotting. There will also be a
grid of green pixels, which identify those that are affected by reseau marks.
These are a set of points within the camera on IUE which provide a reference
frame against which geometrical distortions in the camera target are measured. 
If an archival image is used from before 1981, and the {\tt PHOT} option 
was incorrectly selected instead of {\tt GPHOT}, these green pixels will 
fail to cover the reseau marks. It will then be necessary to re-run 
{\tt READIUE}. 

\section{Extracting a low-resolution spectrum}

\subsection{Point sources}

Extracting a low-resolution spectrum for a point-source is quite simple, 
since there is just a single order, 
and nothing more need be done than just typing the
command {\tt TRAK}. As well as extracting the spectrum, it does a background
subtraction from the surrounding region of the image. Ideally, the output
pixel size should be changed from the default, but rather than going through
the full array of parameters it can be done on a single line: {\tt TRAK
GSAMP=0.707} --- the default is 1.414. There is a fair amount of output 
to the terminal, most of which is unnecessary. The resultant spectrum will 
be both flux and wavelength calibrated, with maximum signal-to-noise of
$\sim20$ and resolution of 6\AA. 

Plotting the spectrum may be done with {\tt PLFLUX}, which will work 
unless the axes have been changed while using {\tt PLSCAN} (see below),
in which case `{\tt PLFLUX ?}' is essential so that the plotting 
limits can be changed to something sensible. However, a more useful 
procedure is to output it to DIPSO using {\tt OUTSPEC ?} --- 
hitting the defaults
except with `{\tt ? OUTFILE}', which should probably be something less opaque
than the original image number. Essential details of DIPSO will be given
below. 

CAUTION: never hit $<$CTRL$>$-C, $<$CTRL$>$-Y, or $<$CTRL$>$-Z, which will 
cause the loss of the data being analysed. To quit the program, 
type {\tt QUIT} or {\tt Q}. 

\subsection{Extended sources}

With extended sources and those without continua, such as high-excitation 
planetary nebulae, the extraction is more complicated, but the responses
to the prompts are straightforward. An example of the responses is given below.
\begin{quote}
\begin{verbatim}
> trak ?
? DATASET(lt5)=<CR>
? GSAMP(1.414)=0.707
? CUTWV(T)=<CR>         *** this provides the standard wavelength limits
? CENTM(F)=<CR>         *** these parameters control the centroiding, and
? CENSH(F)=<CR>             rarely need to be altered. 
? CENSV(F)=<CR>
? CENIT(1)=<CR>
? CENAV(30.)=<CR>
? CENSD(4.)=<CR>
? BKGIT(1)=<CR>         *** these three control the background procedure.
? BKGAV(30.)=<CR>
? BKGSD(2.)=<CR>
? EXTENDED(F)=T   
? CONTINUUM(T)=F 
? AUTOSLIT(T)=<CR>      *** default source and background channels requested
\end{verbatim}
\end{quote}
\subsection{Unusual extractions}

Occasionally a non-standard extraction is necessary. For example, two spectra 
may have been taken side-by-side in the same aperture and on the same image. 
This is done in order to reduce the overheads during the observing, which arise
from the time 
required to read down an image and prepare for the next one. Another reason may 
be that cosmic ray hits may affect a critical part of the spectrum, but there
is enough remaining that is usable. In both cases the {\tt AUTOSLIT} parameter
needs to be set to false, since the automatic extraction slit --- which is 
defined by the camera, resolution and aperture, as well as whether the source 
is extended and has a continuum --- is no longer appropriate. 

If the object is well centred in the aperture, it will be located at the
position expected from the dispersion constants of the spectrograph. For
a point-source in low-dispersion it is not necessary to check this, but for
unusual extractions it is vital. This is done by using {\tt CGSHIFT} on a 
cut across the spectrum, provided by {\tt SCAN} and plotted using {\tt PLSCAN}.
{\tt CGSHIFT} requires the location of the centroid of the spectrum by hand, 
which thus defines the zero-line. 

Three parameters follow once {\tt AUTOSLIT} is set to {\tt F}, 
whose values determine the new extraction. In each case
a pair of numbers is required unless one is the symmetric pair of the other, 
in which case only one is needed. {\tt GSLIT} defines the limit of the spectrum
channel, so {\tt GSLIT=5} is the equivalent of {\tt GSLIT=-5,5}, but
{\tt GSLIT=0,5} would be an asymmetric extraction.
{\tt BSLIT} sets the half-width of 
each background channel, so setting one of them to 0 enables only one to be
used, which is useful for the two spectra within the same aperture. {\tt BDIST}
specifies the distance of each background channel from the centre.

The flux calibration from this new extraction is not necessarily correct
and should be checked, where possible, by comparing with the spectrum 
obtained with the default parameters. Multiplying the new fluxes by a 
constant may suffice to provide sensible values. 

\section{Analysing the spectrum with DIPSO}

The Starlink User Note SUN/50 is an excellent description of DIPSO, 
and also provides a tutorial for first-time users. Much of the information 
is also contained in an on-line HELP facility. However, brief details are 
given here in order that basic measurements may be made. 

The package is entered simply by typing 
\begin{verbatim}
    $ DIPSO
\end{verbatim}
which gives a brief ``welcome'' and, like IUEDR, changes the prompt to `$>$'. 
Spectra produced by {\tt OUTMEAN} within IUEDR can be read in using 
{\tt SP0RD}, where the 0 represents the format selected within {\tt OUTMEAN}. 
It is useful to push the data onto the stack so that it can easily be
reclaimed if several have been read in, or if various manipulations, like
flattening the continuum or smoothing, have been performed. A typical start
to the session might be 
\begin{quote}
\begin{verbatim}
> sp0rd ngc7293_neb96
> push                           
> sp0rd ngc7293_neb10, push   *** several commands may be used on one line
> sl                          *** ``sl'' = Stack List
      I    N     X1         X2              TITLE
      1   611   1150.      1969.     SWP 6157: 96 arcsec from CS
      2   639   1150.      1969.     SWP 42072: 10 arcsec from CS
\end{verbatim}
\end{quote}
The command line {\tt DEV MG100, PM 2} then selects the plotting and plots
up the second stack entry. Had the number 2 not been added, the data in the
current arrays --- usually the last dataset accessed --- would be plotted 
instead. Measuring X and/or Y values may be done using {\tt XV}, {\tt YV},
or {\tt XYV}, using the graphics cursor. Exitting from either of these routines
can be done by hitting the cursor twice at the same point. 

Crude measurements of the flux and equivalent widths may be performed using 
{\tt FLUX} and {\tt EW} respectively. In each case, a linear continuum is 
defined between the two cursor hits delineating the range over which the 
measurement is made. However, a more sophisticated method is to flatten 
the continuum, by fitting a polynomial to it and then dividing it out. 
This uses {\tt CREGS}, {\tt PF}, and {\tt ADIV}, and if {\tt EW} is used 
immediately afterwards, an error estimate is given. A more detailed discussion
is beyond the scope of this document. 

Saving the stack data can be done with {\tt SAVE <filename>}. It can be 
restored in a new session with the command {\tt RESTORE <filename>}. 

CAUTION: as with IUEDR, $<$CTRL$>$-Y and $<$CTRL$>$-Z should never be used
since all of the data analysed in the current session will be lost.
Only in desperate
circumstances should $<$CTRL$>$-C be hit, which will cause the stack to be 
saved in CRASH.STK, but the current arrays will be lost. 

\section{Calibrating spectra}

Low-resolution spectra are both flux- and wavelength-calibrated. However, while 
the standard calibration provided by IUEDR is adequate for many purposes, 
there are problems. In all but the shortest exposures, there will be an 
emission line at 1215.7{\AA} from geocoronal Lyman $\alpha$, but in some 
spectra it may appear anything up to 8\AA\ either side of this wavelength. 
Given that the emission is extended, this cannot be due to a velocity shift
artificially induced by the object not being correctly centred in the aperture. 
One solution is to apply a single wavelength shift to the whole spectrum, 
although the correct procedure may not be so simple. The reason is that the 
calibration lamp on IUE has few emission lines visible in low-resolution mode
shortward of 1480{\AA}, and none below 1380{\AA} --- so the 8{\AA} error may be 
because of a failure to correctly extrapolate the wavelength scale towards
1200{\AA}. Applying the constant shift may be done in DIPSO by using
{\tt XV} to ascertain its value, followed by {\tt XADD} or {\tt XSUB}. 

The flux calibration provided by IUEDR takes into account the temperature
dependence (with the THDA parameter) and the gradual loss of sensitivity 
over time (using the date given in {\tt READIUE}). However, it has recently
been found that there are errors $\sim 20$\% compared to model continua of
white dwarfs, so that a further wavelength-dependent correction needs to be 
applied (Finley, D.S., Basri, G., Bowyer, S., 1990, {\it Astrophys. J.,} 
{\bf 359}, 483). These have been provided for all three fully-operative 
cameras (SWP, LWP and LWR), and are available in the directory SG7DIR. 
To use on a given low-resolution dataset which has already been pushed onto 
the stack, place the correction spectrum in the current arrays (either by 
reading the file from the directory or using {\tt POP}), and then use {\tt
ADIV}. 

\section{Lines found in low-resolution spectra}

There are only a few lines that are strong enough to be detected in 
low-resolution data. They are summarised in Table 1, with laboratory 
wavelengths used. For high-resolution data, it is best to use a more detailed 
line-list, such as Howarth, I.D. and Phillips, A.P., 1986, {\it Mon. Not. R. 
astr. Soc.}, {\bf 222}, 809, or Morton, D.C., and Smith, W.H., 1973, 
{\it Astrophys. J. Suppl.}, {\bf 26}, 333, although while both are valuable 
sources of information, neither are in any sense complete. The information on 
AGNs is taken from Francis, P.J. et al., 1991, {\it Astrophys. J.,}, {\bf 373}, 
465. 


\begin{center}
\begin{tabular}{lll}
\multicolumn{3}{c}{\bf Table 1} \\ 
\multicolumn{3}{c}{Lines found in low-resolution IUE spectra.} \\ \hline
Wavelength & Ion & Comments \\ \hline
1216   & H I & Geocoronal emission + everything else! \\
1241   & N V & Doublet at 1238.8 and 1242.8. Phot, C/S, PC, AGN. \\
1260   & Si II & I/S \\
1300   & Si III & Sextuplet. Phot. \\
1335   & C II  & Interstellar lines at 1334.5 and 1335.7. \\
1371   & O V   & Phot, PC \\
1399   & Si IV & Doublet at 1393.8 and 1402.8. Phot, C/S, AGN. \\
1549   & C IV & Doublet at 1548.2, 1550.7. Phot, C/S, AGN, PN, PC. \\
1640   & He II & Balmer $\alpha$. Phot, AGN, PN, PC.  \\
1663   & O III] & PN, AGN.\\
1718   & N IV & PN, PC. \\
1750   & N III] & PN. \\
1858   & Al III & AGN. \\
1908   & C III] & PN, AGN. \\ 
2326   & C II]  & PN, AGN. \\
2423  & [Ne IV] & PN, AGN. \\
2798  & Mg II & I/S, PN, AGN. \\
\multicolumn{2}{c}{Fe II blends in AGNs:} & \\
Start & Finish & \\
1610 & 2210 & \\
2210 & 2730 & \\
2960 & 4040 & \\ \hline
\end{tabular}
\end{center}
{\bf Notes on typical occurrence:} 
\begin{itemize}
\item Phot =  Photospheres of hot stars, e.g.\ white dwarfs. 
\item C/S = circumstellar material around hot stars.
\item I/S = interstellar medium. 
\item PN = planetary nebulae (emission lines). 
\item PC = P Cygni profiles in PNs.
\item AGN = Active Galactic Nuclei (emission lines). 
\end{itemize}


\section{Extracting high-resolution data}

Spectra from the IUE echelle are complicated not only because of the number
of orders ($\sim 60$ instead of 1), but because at the short-wavelength end
the wings of these orders overlap. Consequently, there is no absolute flux
calibration, and providing an approximate one is something of a black art. 

{\tt TRAK} is generally able to find the centroid of a low-dispersion 
spectrum without difficulty, but without guidance in high-dispersion the 
performance is unreliable. Consequently, it is necessary to {\tt SCAN} the
spectrum, and use {\tt CGSHIFT} to locate the peaks of one of the 
orders. It is best to do this at the short-wavelength end, and {\tt CULIM}
should probably be used with {\tt PLSCAN} to expand the plot to a useful scale. 
One flaw with {\tt CGSHIFT} is that the shift used subsequently by {\tt TRAK}
is the last one obtained: it does not take an average obtained from several 
orders. Consequently, the best approach is to click the cursor on several 
orders and find the one that is most representative. 

To account for the order overlap, a first-order correction needs to be applied. 
The usual assumption is that the background region is contaminated by 
contributions from the two adjacent orders. At $\sim$ 1400{\AA} the value of 
this halation correction, HALC, is zero, which is included within the default
parameters of IUEDR --- it is not necessary to change this. 
It rises linearly toward shorter wavelengths, and so needs to 
be defined at one other point. 
This is possible whenever the Lyman $\alpha$ trough at 1215{\AA} 
is fully saturated, so that at the centre the flux is zero. Therefore, the
correct value of HALC at, say, 1200{\AA} (the default), is the one which 
lifts this feature to the zero flux level. It varies according to the 
individual image from about 0.1 to 0.3, but for B stars the average is 
around 0.12, for O stars it is about 0.15, and for white dwarfs about 0.20:
these are useful as initial values, but it will be necessary to iterate 
manually to find the optimum.

Consequently, {\tt TRAK} should initially be used only on the orders 
containing the Lyman $\alpha$ feature, combined with the two adjacent ones, 
until a satisfactory value of HALC is obtained. It is essential that this
is done, otherwise the equivalent width measurements out to 1400{\AA} will
be worthless. A typical start to this cycle is given below. 
\begin{quote}
\begin{verbatim}
 > scan ?
 ? DATASET(lt5)=<CR> 
 ? SCANDIST(0.)=<CR> 
 ? SCANAV(5.)=<CR> 
 ? ORDERS=125,66   *** the SWP orders are from 66 to 125. 
  Scan grid (-255.7,349.3, 0.5), offset 0.0, half-width 5.0
  Includes echelle orders (66,125).
 > pls             *** all orders from 66 to 125 displayed
 > culim           *** this was used to locate the peaks at the short 
 > pls                 wavelength end. 
 > cgs
  (nearest_order, R, W) = (108,-1.51,1279.367)
  Relative geometric shift (-1.18, 0.94)      *** these are the relevant 
  Absolute geometric shift (-1.18, 0.94)          figures
  (nearest_order, R, W) = (106, -0.99,1303.489)
  Relative geometric shift ( -0.78, 0.62)
  Absolute geometric shift ( -0.78, 0.62)
  (nearest_order, R, W) = (105,-1.83,1315.894)
  Relative geometric shift (-1.42,1.14)
  Absolute geometric shift (-1.42,1.14)
  (nearest_order, R, W) = (108,-1.51,1279.367)
  Relative geometric shift (-1.18, 0.94)   *** this is the most typical 
  Absolute geometric shift (-1.18, 0.94)       value. A better exposed 
  Last Shift Retained.                         image would have a smaller   
                                               range of shifts.
 > setd halc=0.21                          *** sets the halation correction
                                               to the default value of 0.21
 > trak ?
 ? DATASET(lt5)=<CR> 
 ? GSAMP(1.414)=0.707
 ? CUTWV(T)=<CR> 
 ? CENTM(F)=<CR> 
 ? CENSH(F)=<CR> 
 ? CENSV(F)=<CR> 
 ? CENIT(1)=<CR> 
 ? CENAV(30.)=<CR> 
 ? CENSD(4.)=<CR> 
 ? BKGIT(1)=<CR> 
 ? BKGAV(30.)=<CR> 
 ? BKGSD(2.)=<CR> 
 ? EXTENDED(F)=<CR> 
 ? AUTOSLIT(T)=<CR> 
 ? ORDER=115   *** the 4 orders 115 to 112 include Lyman Alpha (113 and 114)
 ? NORDER(0)=4     plus the two adjacent orders. Going from 115 to 112
                   provides the correct wavelength sequence. 
  Will extract echelle orders (115,112).
  Sample width  0.71 pixels ( 0.50 that of IUESIPS#1).
  Background folding FWHM 30.0 pixels, evaluated 2 times,
  with pixels outside 2.0 s.d. rejected.
  Will base initial templates on dispersion constants.
  Centroid folding FWHM 30.0 pixels, evaluated 2 times,
  with signal above 4.0 s.d. used for tracking.
  Point source Object.
  Slit determined automatically.

  Echelle Order 115
  Channels:  Object (-2.6,2.6),  Backgrounds (-3.6,-2.6),(2.6,3.6).
  Wavelength grid (1193.007,1205.005, 0.022) based on cutoff limits.
                                Background                  Object
                              Left       Right         Net       Shift
  Good pixels used             361         368        2212        2212
  Bad pixels used                0           0          13          13
  Pixels not used               16          16           0           0
  Rejected pixels                9          16           0           0
  Mean value                 697.6       635.4      1554.6       0.260
  RMS variation                0.2         0.2      1442.2       0.065
  Evaluations                    4           4           2           2
  :
  :
  :
  Echelle Order 112
  Channels:  Object (-2.7,2.7),  Backgrounds (-3.8,-2.8),(2.8,3.8).
  Wavelength grid (1224.492,1237.506, 0.022) based on cutoff limits.
                                Background                  Object
                              Left       Right         Net       Shift
  Good pixels used             386         376        2419        2419
  Bad pixels used                0           0          24          24
  Pixels not used               16          20           0           0
  Rejected pixels               16          18           0           0
  Mean value                 710.1       844.0      2659.5       0.079
  RMS variation                0.2         0.2      2572.2       0.081
  Evaluations                    4           4           2           2
  Changing ORDER and NORDER parameters.
  Spectrum Extraction Completed.
\end{verbatim}
\end{quote}
The extracted orders, if saved (either be typing {\tt SAVE} or by quitting
the program) will appear on a .UES file. It will appear order-by-order, and it 
is generally more useful to amalgamate them into a single spectrum. First of 
all, however, it is necessary to correct the data for a ripple that originates 
in the echelle grating itself. This is done by using the eponymous routine 
{\tt BARKER}, which is fully automatic (see Barker, P.K., 1984, {\it Astron. 
J.}, {\bf 89}, 899). The only parameter is {\tt ORDERS}, which should be 
answered in reverse fashion, such as {\tt 115, 112}, or {\tt 125, 66}. 

Converting the individual orders on a single scale involves the
routine {\tt MAP}, which takes into account the common-wavelength region of
adjacent orders (for example, the geocoronal Lyman $\alpha$ feature appears in
both order 113 and 114, which can be seen on any image). 
\begin{quote}
\begin{verbatim}
 > map ?
 ? DATASET(lt5)=<CR>          
 ? RM(T)=<CR>               *** i.e. that a new mean spectrum is created
  New Spectrum.                 rather than one averaged with the old. 
 ? ORDERS(115,112)=<CR>
  Orders in the range (112,115) can be used.
 ? ML=1193, 1237            *** wavelength limits. 
 ? MSAMP( 0.0218142)=0.04   *** the bin size. 0.04A is the best for all 
                                wavelengths, as the default is only
                                appropriate for the first order selected. 
  Wavelength grid is (1193.000,1237.000, 0.040).
 ? FILLGAP(F)=<CR>
 ? COVERGAP(F)=<CR>
  Gaps are not filled.
  Using order 112
  Using order 113
  Using order 114
  Using order 115
\end{verbatim}
\end{quote}
The output spectrum can then be examined using {\tt PLMEAN}, or {\tt PLM} for
short. If the zero line passes through the base of the Lyman $\alpha$ trough, 
then the value of HALC is acceptable; otherwise it will have to be increased
if the spectrum needs to be lifted, or lowered otherwise. If HALC does need 
to be altered, then the {\tt SETD}, {\tt TRAK}, {\tt BARKER}, and {\tt MAP}
commands will have to be repeated until a satisfactory value is found. 
Occasionally, the Lyman $\alpha$ is not saturated, which is obvious since the
profile is not flat at the base; a default value will have to be assumed. This
is not satisfactory, but is the best that can be done. 

Once HALC has been found, {\tt TRAK} can be run on the whole spectrum. 
Thus, {\tt ORDERS} should be set to 125 and {\tt NORDER} to 60, although if
120 and 52 are used little useful information will be lost. It will tend
to take about half-an-hour to run, but needs no interaction.
BEWARE: do not leave the session so that you are automatically logged 
out after the terminal has been inactive for half-an-hour, since the 
{\tt TRAK}ked spectrum will be lost. 

As before, {\tt BARKER} and {\tt MAP} should be run on the {\tt TRAK}ked 
spectrum. Although this full spectrum can be examined in IUEDR, it is much 
more convenient to output it to DIPSO using {\tt OUTMEAN}.
\begin{quote}
\begin{verbatim}
 > outmean ?
 ? DATASET=lt5
  Reading lt5.UEC (Calibration File).
  Reading lt5.UEM (Mapped Spectrum File).
 ? OUTFILE(swp38776M.DAT)=lt5          *** name of the output file
 ? SPECTYPE(0)=<CR>                    *** can be read into DIPSO using
                                           > sp0rd <filename>
  Writing Binary SPECTRUM File (lt5).         
\end{verbatim}
\end{quote}

\section{Examining a full IUE spectrum with DIPSO}

There is no single command within DIPSO for displaying a full IUE 
high-resolution spectrum on one page of output, but a procedure has been 
written to do this. However, it is necessary to normalise the spectrum, 
since the scale plotted is from 0 to 1 only. A typical command sequence 
might be:
\begin{quote}
\begin{verbatim}
> sp0rd lt5         *** DO NOT ``push'' data onto stack!
> dev mg100         
> xr 1200 1400      *** for a hot star, this interval encompasses the peak 
                        of the spectrum. 
> pm                *** plots data located in current arrays, i.e. what has
                        just been read in. Suppose the maximum is 6.5...
> ydiv 7            *** resets scale so maximum is less than 1
> push 
> @sg7dir:page *** plots out whole spectrum as 8 100A strips on a
                        single sheet of A4. The graphics device has been
                        set to CANP within the procedure itself. 
> q                 *** quit session
$ print/que=sys_laser canon.dat
                    *** the plot file is not automatically queued. 
\end{verbatim}
\end{quote}
Once in DIPSO, line centroids and equivalent widths may be computed 
easily, but the details are beyond the scope of this document. 

\section{The IUE log and de-archiving images}

There is a log of all the IUE observations that is produced periodically, and 
an online version is kept on the STADAT node. Examining this is done by 
logging onto the captive account on this machine, username IUE, and using the 
{\tt IUELOG option}, following the instructions given. In addition to
the usual information (like co-ordinates, object name, exposure time etc.)
each spectrum is assigned a class between 01 and 99, so it is possible to 
obtain a subset of the IUE log for one particular class of objects. The 
option {\tt CLASSKEY} will provide the code used. 

Two methods exist for examining spectra in the archive. Low-resolution data 
are available on-line, also on STADAT, but it is necessary to use the LEI 
account. The command needed is QUEST, which is described fully in 
SUN/20. It is often necessary to analyse the original images, though, 
and this is unavoidable for high-resolution data. For this, {\tt IUEDEARCH}
should be used, described in SUN/58. 

\section{Future developments}

The method of obtaining spectra from images is currently being overhauled, so 
that the whole archive can be re-processed. The new developments have recently 
been reviewed in the {\em `Evolution in Astrophysics'} proceedings (see above),
and are likely to result in an increase in the signal-to-noise ratio of $\sim 
50$\%. However, it is unlikely that the results of this will be generally 
available for a couple of years. 

\section*{Acknowledgements.}

I would like to thank Dave Finley, of the University of California at
Berkeley, for making the low-resolution correction spectra generally 
available on {\it Starlink}. 

\end{document}
