\documentstyle[11pt]{article} 
\pagestyle{myheadings}

%------------------------------------------------------------------------------
\newcommand{\stardoccategory}  {Starlink User Note}
\newcommand{\stardocinitials}  {SUN}
\newcommand{\stardocnumber}    {45.4}
\newcommand{\stardocauthors}   {Nicholas Eaton}
\newcommand{\stardocdate}      {19 March 1992}
\newcommand{\stardoctitle}     {PHOTOM --- An aperture photometry routine}
%------------------------------------------------------------------------------

\newcommand{\stardocname}{\stardocinitials /\stardocnumber}
\renewcommand{\_}{{\tt\char'137}}     % re-centres the underscore
\markright{\stardocname}
\setlength{\textwidth}{160mm}
\setlength{\textheight}{230mm}
\setlength{\topmargin}{-2mm}
\setlength{\oddsidemargin}{0mm}
\setlength{\evensidemargin}{0mm}
\setlength{\parindent}{0mm}
\setlength{\parskip}{\medskipamount}
\setlength{\unitlength}{1mm}

%------------------------------------------------------------------------------
% Add any \newcommand or \newenvironment commands here
%------------------------------------------------------------------------------

\begin{document}
\thispagestyle{empty}
SCIENCE \& ENGINEERING RESEARCH COUNCIL \hfill \stardocname\\
RUTHERFORD APPLETON LABORATORY\\
{\large\bf Starlink Project\\}
{\large\bf \stardoccategory\ \stardocnumber}
\begin{flushright}
\stardocauthors\\
\stardocdate
\end{flushright}
\vspace{-4mm}
\rule{\textwidth}{0.5mm}
\vspace{5mm}
\begin{center}
{\Large\bf \stardoctitle}
\end{center}
\vspace{5mm}

%------------------------------------------------------------------------------
%  Add this part if you want a table of contents
  \setlength{\parskip}{0mm}
  \tableofcontents
  \setlength{\parskip}{\medskipamount}
  \markright{\stardocname}
%------------------------------------------------------------------------------


\section{INTRODUCTION}

PHOTOM is an ADAM task for performing aperture photometry. It has two
basic modes of operation; using an interactive display to specify the
positions for the measurements, or obtaining those positions from a file.
The aperture is circular or elliptical and the size and shape can be
varied interactively on the display, or by entering values from the
keyboard or parameter system. The background sky level can be sampled
interactively by manual positioning of the aperture, or automatically
from an annulus surrounding the object.


\section{RUNNING PHOTOM}

PHOTOM accepts data files in NDF format (~SGP/38~).
If the measurements are to be made in interactive mode, the data must
be first displayed using a suitable routine, which writes to the AGI
graphics database. All of the display routines in KAPPA are suitable
for use with PHOTOM. PHOTOM then uses the database to position the
aperture on the display. For reference PHOTOM uses the last 'DATA'
picture in the database to establish its coordinate system; see SUN/48
for more details on the graphics database.

To run PHOTOM from within the ADAM command language give the task name
thus :-
\begin{quote}
\begin{verbatim}
     ICL> PHOTOM
\end{verbatim}
\end{quote}

The task can also be run from DCL with the following command
\begin{quote}
\begin{verbatim}
     $ ADAMSTART
     $ PHOTOM
\end{verbatim}
\end{quote}

The first requested parameter is the name of the file containing the
data object. This data object remains with the task until PHOTOM is
quitted. If another data object needs to be measured then PHOTOM has
to be re-run.

The routine creates some temporary disk work space. An error will occur
if there is not enough room on the disk to accommodate this.


\section{RUNNING PHOTOM INTERACTIVELY}

If an interactive menu option (~I~-~interactive shape or M~-~interactive
measurement~) is selected then the name of the display device to be used
is requested the first time one of these options is chosen.
PHOTOM is mainly intended to be run interactively on an image display
with an overlay plane. On such a device the image will be displayed on
the base plane and PHOTOM will be run on the overlay. The use of an
overlay plane gives the greatest flexibility to the program, since
regions of the overlay can be erased without affecting the image.
Also PHOTOM can be called several times without having to erase the
underlying image. Of course if a different data set is to be measured,
then this has to be displayed first.

It is possible to run PHOTOM interactively on a standard graphics terminal
e.g.~Tektronix. Clearly displaying the image as a greyscale on such a
terminal is not possible, but KAPPA has a number of contouring routines
that can be used instead. After running the contouring package PHOTOM is
started as normal. When the graphics device is requested by PHOTOM, the
same name as used for the contour plot should be given.

The disadvantage of using graphics terminals is that the screen erasing
is inhibited to ensure that parts of the image do not get erased, and thus
apertures drawn by PHOTOM become permanent parts of the picture. This will
also occur if an image display is used and the base plane, and not the
overlay, is selected for PHOTOM.

On an image display the interaction is controlled by the tracker ball or
mouse. The cursor position is under the control of the mouse movement and
the mouse buttons select the operation to be performed, as indicated by
the menu boxes on the screen.
On a graphics terminal these functions will normally be controlled
from the keyboard. For example the arrow keys may move the cursor. The
button selections are usually under the control of the numeric keys above
the QWERTY keyboard. The three boxes shown on the display need responses
of 1, 2 and 0 reading from left to right, i.e. the function RETURN TO
KEYBOARD is selected by pressing key 0 and the operation in the left hand
box is selected by pressing key 1. Other keys on the keyboard may
also return one or more of these values e.g. the QWERTY keys normally
return 0.


\section{RUNNING PHOTOM IN BATCH}

If there are a large number of objects to be measured non-interactively
then it may be preferable to run the job in batch. This can be done
either with a DCL command procedure or an ICL command procedure.
The DCL command procedure contains the command to run the program and
the entries required by the prompts. It will be necessary to run the
program interactively first to verify the order of the prompts.
\begin{quote}
\begin{verbatim}
     $ ADAMSTART
     $ PHOTOM
     FRAME.SDF
     N
     5.0
     0.0
     0.0
     F
     POSITIONS.DAT
     E
\end{verbatim}
\end{quote}
In this example the image data is assumed to be in a file FRAME.SDF in
the default directory. The size and shape of the aperture is set using
the non-interactive command (~N~) and the command (~F~) instructs the
program to take the initial positions from the file POSITIONS.DAT. The
(~E~) command ends the program. All other parameters are taken as the
run-time defaults.


\section{MENU OPTIONS}

PHOTOM is a menu driven application. It is a single executable image so
the user has to exit from the routine to perform any other tasks. The
menu has been designed around single character entries, which hopefully
have easily remembered mnemonics. Many of the options have counterparts
in the ADAM parameter system, and so can be controlled outside the task
by the environment. The options are selected by inputting a single
letter corresponding to the chosen operation at the COMMAND prompt.
The first letter entered is the one used by the program; any
other letters following this on the same command line are ignored.
For instance 'M', 'Measure' and 'magic' all initiate the interactive
measurements. The menu options are as follows:-

\subsection{A --- Annulus}

This is a toggle switch which alters the way in which the background
level is measured. There are two methods available. The first is
interactive, and uses an aperture identical in size and shape to the
object aperture. The aperture is positioned manually to select the
region of sky to measure. The message 'Interactive aperture in use'
will signify that this has been chosen.

The alternative is to use an aperture which is a concentric annulus
around the object aperture. The size of the aperture is specified by
the INNER and OUTER parameters. In this mode the sky is measured
automatically every time a measurement is made. The message
'Concentric aperture in use' signifies this choice.

When starting PHOTOM one of these modes will be chosen as the default.
This choice is controlled by the CONCEN parameter in the interface file.
If the positions of the objects is entered by a file of positions
(~command F~), then the background is automatically taken with the
concentric annulus, whatever the default value of CONCEN.

When using the interactive aperture, several sky areas can be sampled
to improve the estimate of the sky. The sky estimates in each aperture
are simply summed, and the mean of these is used when the object is
measured.

\subsection{C --- Centroid}

This is a toggle switch which alters whether the object is centered in
the aperture before doing the measurement. The centroiding algorithm is
the same as the one used by KAPPA. The degree of centroiding is
controlled by the parameters SEARCH, POSITIVE, MAXSHIFT, MAXITER and
TOLER. These cannot be changed from within the program, and if
alternative values are required they should be specified at runtime
using the keyword facility, or the SET command (~see section 4~)

The choice of mode is indicated by the messages 'Centroiding in stellar
aperture', or 'No centroiding'. When starting PHOTOM one of these modes
will be chosen as the default. This choice is controlled by the CENTRO
parameter in the interface file.

Unless the field under investigation is very crowded, or there are other
special conditions, it is probably best to leave the centroiding option
on all the time. This is more important if the measurements are being
made in non-interactive mode (~command F~); unless the user is certain
that the positions supplied are as required.

\subsection{E --- Exit}

This command terminates the current PHOTOM session.

\subsection{F --- File of positions}

This command causes the measurements to be done automatically. A file
containing the positions is requested via the parameter system, and the
photometry is performed with the current setup of aperture parameters.

The name of the file containing the positions is requested through the
POSFILE parameter. If the file cannot be found or is not in a suitable
format the user is returned to the command level. 

The file of positions should specify an index number and the x and y
positions in pixel coordinates. For every x and y pair in the file an
aperture measurement is made, sampling the sky with the concentric
annulus, whose size is specified by the current values of the INNER and
OUTER parameters. Centroiding in the object aperture is, or is not done,
depending on the current value of the CENTRO parameter, which is selected
with command C. When the input file is exhausted the user is returned
to the command level.

Results of the measurements are shown on the terminal as well as output
to a file pointed to by the RESFILE parameter. Results are identified by
the index number associated with the x and y position in the input file.

A previous results file can be used as the input file of positions. Just
typing the name without a version number in response to the file name will
result in an error, since the program has opened a new results file which
is trying to be read. To get round this give the explicit version number
of the results file to be used, or use version number -1 to get the
previous file of results.

The format of the input and the results files are given in appendix A.

\subsection{H --- Help}

This displays a brief line of help for each command. For more extensive
help refer to this manual.

\subsection{I --- Interactive shape}

This allows the size and shape of the cursor to be adjusted interactively
on the screen to best suit the needs of the particular frame. If the name
of the display device has not already been requested, it will be now.
PHOTOM usually works in an overlay plane of the image displays, so that
selected areas can be erased without affecting the underlying image.

The size and shape of the aperture is governed by three parameters, defining
an ellipse. The semi-major axis and eccentricity are as usually defined for
an ellipse. An eccentricity of 0 gives a circular aperture with a radius
equal to the semi-major axis. The orientation of the ellipse is given in
degrees and specifies the orientation of the semi-major axis of the ellipse
anti-clockwise with respect to the vertical axis of the screen. If the
data is displayed in the normal sense then the orientation corresponds to
a position angle.

The selection of the parameter values is controlled by two buttons on the
trackerball/mouse. The boxes on the screen are representative of the
buttons on the mouse. Pressing the first button changes the value of the
particular parameter named on the screen. The current value of that
parameter is also shown. The parameters have a limited number of preset
values which can be cycled through by repeated presses of the first
button. If none of the set-up parameter values is suitable, then the
values can be entered from the keyboard using the non-interactive shape
command N. The second button on the mouse selects which of the parameters
SEMI-MAJOR, ECCENTRICITY or ORIENTATION is to be under the control of the
first button. Repeated presses of the second button will cycle the names.
Pressing the second button does not change the parameter values, so it
can be used to check the current values of the parameters (~this can also
be done from the keyboard with the V command~). The third button returns
the control to the keyboard. Control will remain with the mouse until this
button is pressed. The third button really corresponds to the
break key, and so on ARGS trackerballs is the fourth (~red~) button.

When a parameter value is changed, the aperture displayed at the cursor
position is updated to reflect this. The aperture is also redisplayed
whenever the second key is pressed. As this does not change any of the
parameter values, then moving the cursor and pressing the second button
can be used to verify the shape of the cursor is suitable at different
parts of the frame.

The initial values of the three parameters, shown on the display, are
chosen to reflect the current values. The semi-major axis is set at the
current value, and the available choices enables the size to be doubled
or halved. The possible values of the eccentricity and orientation are
limited, in this interactive selection, to a number of preset values.
The value initially displayed is taken to be the member of the preset
table which is closest to, but lower than, the current value of that
parameter.

\subsection{M --- Measure}

This performs interactive measurements of objects individually selected
from the displayed frame. The size and shape of the cursor should be
set-up in advance using the I or N commands.

If the name of the display device has not already been requested, it will
be now. PHOTOM usually works in an overlay plane of the image displays, so
that selected areas can be erased without affecting the underlying image.

There are two basic methods depending on the choice of whether the
background is sampled from an annulus around the object aperture, or from
a separately chosen area of sky (~see command A~). The two can be
distinguished at this stage from the on-screen display. In the case of
the automatic sky measurement the two outer boxes labelled STAR and
RETURN~TO~KEYBOARD are annotated, with the middle box empty. In the case
of the manual sky measurement the middle box is labelled with SKY.

To perform the measurements with the automatic sampling of the sky, the
cursor is positioned over the chosen object and the first button is
pressed. On the screen an aperture is displayed where the measurement
was made. If centroiding is being done, (~command C~) then the displayed
aperture may not be centered on the cursor position. The results of the
measurement are printed on the terminal screen and recorded in the
results file. Measurements can be continued until the RETURN~TO~KEYBOARD
button is pressed.

When using manual selection of the background the second button comes into
play. Pressing this button records the sky estimate in an aperture
identical in size and shape to the object aperture, at the position
specified by the cursor. On the screen an aperture is displayed at that
position. No centroiding is done in this aperture, even if the centroiding
option is on. When the measurement of the object is made by pressing the
first button, the most recent value of the sky is used. This means that
the sky has to be sampled BEFORE the measurement of the object. If the
background needs to be sampled in several places around an object, to
minimise the noise or take account of a sloping background, then the
SKY button can be pressed a number of times in order to get the mean
of the sky measurements. The calculation of the mean is only cleared when
the STAR button is pressed, so if a mistake has been made in estimating
the mean of the skies then a measurement has to be made with the STAR
button, and a note made that the measurement was in error, before going
back to the estimation of the sky. Control stays with the mouse until
the RETURN~TO~KEYBOARD button is pressed.

The results of the measurements are displayed on the terminal and sent
to a file named pointed to by the RESFILE parameter.

\subsection{N --- Non-interactive shape}

The size and shape of the aperture can be specified from the keyboard
by entering values for the semi-major axis, the eccentricity and the
orientation of the ellipse defining the aperture.

An eccentricity of 0.0 gives a circular aperture with a radius equal to
the semi-major axis. The orientation of the ellipse is given in degrees
and specifies the orientation of the semi-major axis of the ellipse
anti-clockwise with respect to the vertical axis of the pixel array. If the
array is considered in the normal sense, with the first pixel in the bottom
left corner, then the orientation corresponds to a position angle.


\subsection{O --- Options}

This allows the user to change the values of some of the parameters
specified in the interface file from the keyboard.

The INNER and OUTER parameters define the size of the annulus to be
used in the automatic sampling of the sky. The annulus has the same
elliptical shape as the object aperture, but is larger by the factors
given by INNER and OUTER. These two parameters are given in terms of
multiplicative factors of the semi-major axis of the object aperture.
Thus an INNER radius of 1 means that the sky annulus starts where the
object aperture ends. The annulus thus grows and shrinks with changes
to the object aperture.

The PADU parameter defines the number of photons for each interval of
the data. Multiplying the data value in each pixel by PADU gives the
number of photons recorded (~after correcting for BIASLE~). If this
parameter is unknown then leave it at 1.

The SKYMAG parameter specifies the magnitude to be given to the sky
when calculating the magnitude of the object. The magnitude of the
object is calculated from
$mag = SKYMAG - 2.5 \log_{10} ( signal )$
where signal is the brightness of the object minus sky in photons.

The BIASLE parameter gives the level in data units of any offset in
the bias level per pixel. This is needed if there is any non-photon
source of background, and proper photon statistics are required. If
this parameter is unknown then leave it at 0.

The SATURE parameter is a user supplied value giving a saturation
level for the image in data units. If there are any pixels in the
object aperture with values greater than the saturation level then
this is indicated by an error code 'S' in the final column of the
output table. The object magnitude is calculated with the saturated
pixel {\bf included} in the result.

\subsection{P --- Photon statistics}

This is used to choose between the different ways in which the errors
are calculated. There are three possible choices selected by the integers
1 to 3 which have the following bindings :-
\begin{tabbing}
xxx\= \kill
1 \> Errors from photon statistics. \\
2 \> Errors from variations in the sky aperture. \\
3 \> Errors from data variance. \\
\end{tabbing}
The first works out the errors from photon statistics in the sky and
signal apertures. This requires the user to know and set-up the
parameters PADU and BIASLE which convert the data values to numbers of
photons. The message 'Errors from photon statistics' will signify that
this has been chosen.

The second method of calculating the errors is from the measured
variance in the sky aperture. This method assumes that the measured
variance is due to photon statistics and scales the measurement in
the object aperture accordingly. This method still requires the
parameter PADU to be known, but does not need the BIASLE parameter to
be known. The message 'Errors from sky variance' will signify that
this has been chosen. If neither the PADU or BIASLE parameters are
known, then it is best to use this method to indicate the reliability
of the measurements, but not to take the quoted error values as
absolute since this method will be wrong by a factor $\sqrt{PADU}$,
where $PADU$ is the unknown conversion factor.

The third method of calculating the errors is from the data variance
component of the NDF. This method of calculating the errors also
requires the parameter PADU to be known. The message 'Errors from
data variance' will signify that this has been chosen. A variance
component may not always be present in the NDF along with the data
array, and if this is the case then PHOTOM will issue the warning
'Data does not have a variance component' if this method is selected. 

Appendix B gives a full discussion of the calculation of the errors
assuming photon statistics.

\subsection{S --- Sky}

This is used to choose between the different methods of estimating the
background level in the sky aperture. There are four possible choices
selected by the integers 1 to 4 which have the following bindings :-
\begin{tabbing}
xxx\= \kill
1 \> Simple mean. \\
2 \> Mean with 2 sigma rejection. \\
3 \> Mode. \\
4 \> User supplied value.
\end{tabbing}
The simple mean uses all the values in the sky aperture for its estimate
except for those that have the `magic' bad values. The mean with 2 sigma
rejection excludes all those points which are more than 2 standard
deviations from the mean. Because one or more wayward outliers can
affect the size of the standard deviation, the mean and standard deviation
are recalculated after each stage of clipping up to a maximum of three
times.
The mode is superficially calculated from the empirical relation
$mode = 3 * median - 2 * mean$, but because this can be fooled by
excessive skewness in the histogram there are rejection and averaging
schemes in the algorithm to ensure stability. The final option is to
supply a value for the sky value which is used as a constant value for
subsequent measurements. This value is used until either a new value is
chosen or one of the other methods of estimation is selected. The sky
variance is also requested so that if the errors are calculated from the
sky variance (~command~P~) then a realisitic error can be assigend. Both
the sky value and variance should be given in data units.

When using a concentric background aperture it is recommended that the
mode or mean with $2\sigma$ rejection is used as these offer protection
against contamination from other objects in the sky aperture.

\subsection{V --- Values}

This summarises the current settings of the significant parameters on
the terminal.


\section{DEFAULTED PARAMETERS}

A number of parameters defined in the interface file cannot be changed
within the program. They will normally take the default values given in
the interface file, but if required they can be changed before running
the program. A number of these may be of use to some users and they are
described below.

\subsection{RESFILE}

This specifies the name of the results file which makes a permanent record
of the measurements.

\subsection{Centroiding}

There are a number of parameters that control the centroiding algorithm.
SEARCH defines the size of the search box to be used in locating the
centroid in pixels. MXITER defines the maximum number of iteration steps.
MAXSHIFT gives the maximum allowable shift in pixels between the initial,
rough, position and the calculated centroid. TOLER defines the position
accuracy in pixels that will terminate the centroiding iterations.

\subsection{ETIME}

This parameter gives the path to an HDS element in the image data file
that contains the exposure time of the frame. If a valid real number is
found in this structure then this is used to scale the results compared
to an exposure time of 1.
$mag = SKYMAG - 2.5 \log_{10} ( signal / exp\_time )$
This affects the output values for the measured signal in the object and
the resultant magnitude, but it does not change the reported value for
the sky in each pixel, or the error in the magnitude.
If this parameter is not specified or the structure does not contain a
valid number then an exposure time of 1 is used.
If the component containing the exposure time is several layers down in
the HDS file then the pathname should specify the names of each of the
levels separated by fullstops, for example MORE.EXP\_TIME indicates that
the exposure time will be found in a component EXP\_TIME which is a
primitive component of the structure MORE.

\subsection{USEMASK}

This parameter is a logical flag which indicates whether a mask is to be
used when estimating the background. The purpose of the mask is to block
out contaminating objects from the background aperture. In this way bright
stars can be excluded from the estimation of the sky, which would
otherwise introduce contamination. Note that the sky estimators that
perform clipping of the pixel histogram, the mode and the mean with
$2\sigma$ rejection, also exclude contaminating pixels, but using the mask
along with the mean estimator allows this to be done in a controlled way.

If the USEMASK flag is true then a file containing a list of positions is
requested (~MASKFILE~). The format of the file is the same as for inputting
a list of positions to measure (~command~F~), namely an index number
followed by an x and y position. The given coordinates define the
centres of circles and any pixel with its centre within a circle will be
excluded from the sky estimation. The radius of the masking circle is
another requested parameter (~MASKRAD~) that can be changed by the user.

The mask only affects pixels in the background aperture, it does not
exclude any pixels from the measurement aperture. This means that
identical lists can be used to create the mask and to provide a source
for measurement. The output from an automatic object finding package
could be used in this way.

\section{ALTERING PARAMETERS}

When PHOTOM is started, the initial selection of most of the parameters
is taken from the previous run of the routine. The parameter values are
stored in the GLOBAL.SDF or PHOTOM.SDF file in the ADAM\_USER directory
at the end of a run. The current values can be examined using the TRACE
facility. To clear these values, and to revert to the start-up defaults
the GLOBAL.SDF and PHOTOM.SDF files have to be deleted. A new empty
GLOBAL.SDF file has to then be created by giving the ADAMSTART command.
If the file is not present when PHOTOM is run then an HDS error 'File
not found' will be given at the end of the run. This means that the
current parameter values have not been stored.

The starting values of the parameters can also be specified within the
ADAM command language. The keyword facility allows the parameters to be
given on the command line (~see AED/3, the Interface Module Reference
Manual~). The keywords have the same name as the parameters; for example
the search box for the centroiding can be changed using the command
\begin{quote}
\begin{verbatim}
     ICL> PHOTOM SEARCH=5
\end{verbatim}
\end{quote}
The parameter can also be changed with the SET command.
\begin{quote}
\begin{verbatim}
     ICL> SEND PHOTOM_DIR:PHOTOM SET SEARCH 5
     ICL> PHOTOM
\end{verbatim}
\end{quote}

From DCL the keywords can be included on the command line.
\begin{quote}
\begin{verbatim}
     $ PHOTOM SEARCH=5
\end{verbatim}
\end{quote}

\section{ACKNOWLEGEMENTS}

Thanks go to Nial Tanvir for his helpful comments on the sky background
estimation section in Appendix~C, and a number of people for suggestions
for improvements to PHOTOM.


\appendix
\newpage
\section{FORMAT OF ASSOCIATED FILES}

At present the file containing the positions of objects
to be measured, given by the POSFILE parameter (~command F~), is read in in
free format. In future this file will be accessed with a SCAR descriptor
file. The first three columns of the input file have to contain an
index number (~INTEGER~), and the x and y positions (~REAL~) in that order.
The index number is passed to the output to assist in identification of the
objects.

The output on the screen contains column headers to indicate the content of
each column of the results. These column headers do not appear in the
output file given by the RESFILE parameter in order that this file can be
accessed by the SCAR database routines. A file DSCFPHOTOM.DAT in the
PHOTOM\_DIR directory can be copied to supply the descriptor information
required by SCAR. Note that SCAR cannot access catalogues with names
containing more than five letters, and since PHOTOM contains six letters
the results file and DSCFPHOTOM.DAT will have to be changed to shorter
names, for example PHOT.DAT and DSCFPHOT.DAT, to work with SCAR.

There are eleven columns in the output file containing the following
information :-

\begin{tabbing}
xxxxxxxx \= xxxxxxxxxx \= \kill
Column \> Name  \> Description \\
 1 \> INDEX \> Index number of star.\\
 2 \> XPOS \> X position of centre of aperture in pixels.\\
 3 \> YPOS \> Y position of centre of aperture in pixels.\\
 4 \> MAG \> Magnitude of star.\\
 5 \> MAGERR \> Error in magnitude.\\
 6 \> SKY \> Sky value in photons per pixel.\\
 7 \> SIGNAL \> Total number of photons in aperture due to star.\\
 8 \> CODE \> Error code flag.\\
 9 \> SEMIM \> Semi-major axis of aperture.\\
 10 \> ECCEN \> Eccentricity of aperture.\\
 11 \> ANGLE \> Orientation of aperture in degrees.\\
\end{tabbing}

The magnitude (~MAG~) is calculated from
$mag = SKYMAG - 2.5 \log_{10} ( signal )$.

The error in the magnitude (~MAGERR~) is estimated using one of the
methods expounded in appendix B.

The error CODE can take on three possible values
\begin{tabbing}
xxxx\=\kill
B \> One or more pixels in the object aperture is bad.\\
S \> One or more pixels in the object aperture is above the saturation level.\\
E \> The object aperture intersects the edge of the data array.\\
\end{tabbing}

If a bad pixel occurs in the object aperture then the pixel is not
included in the calculation of the object signal. The bad pixel is not
replaced by an estimate. If a saturated pixel occurs in the object
aperture then it is included in the calculation of the object signal. 
If the aperture intersects the edge of the data array, the object signal
is calculated for the reduced area of the aperture.

Only one of these code letters is displayed, even if more than one of the
conditions has occurred. The codes are in a increasing hierarchy 'B', 'S',
'E', such that 'S' overrides 'B', and 'E' overrides 'S'.

\newpage
\section{CALCULATION OF THE ERRORS}

The errors are calculated in one of three ways, as discussed in the section
on command P. The first method assumes true photon statistics and the
error is calculated from the following definitions :-\\

Number of pixels in object aperture  $= a_o$\\
Number of pixels in sky aperture     $= a_s$\\
Sum of data in object aperture       $= D_o$\\
Sum of data in sky aperture          $= D_s$\\
Offset in one pixel                  $= BIASLE$\\
Number of photons per data unit      $= PADU$\\

The contribution of the sky in the object aperture can now be calculated\\

Number of photons in object aperture $= P_o = PADU * ( D_o - a_o * BIASLE )$\\
Number of photons in sky aperture    $= P_s = PADU * ( D_s - a_s * BIASLE )$\\
Number of photons in object aperture due to sky
$= P_{so} = PADU * ( D_s - a_s * BIASLE ) * ( a_o / a_s )$\\

The signal due to the object is the difference of the total number of
photons in the object aperture minus the number due to the sky\\

Object signal $= S_o = P_o - P_{so} = PADU * ( D_o - D_s * ( a_o / a_s ) )$\\

The error on the object signal is the quadratic sum of the errors on the
individual measurements. Using $\varepsilon$ to signify the error
\[\varepsilon( S_o )^2 \sim \varepsilon( D_o )^2 + \varepsilon( D_s )^2 *
( a_o / a_s )^2\]
Assuming the errors are solely from photon statistics then the error on
the signal is
\[\varepsilon( S_o )^2 = \varepsilon( P_o )^2 + \varepsilon( P_s )^2 *
( a_o / a_s )^2\]

The error from photon counting is the square root of the number of photons
$\varepsilon( P_o ) = \sqrt{ P_o }$ and $\varepsilon( P_s ) = \sqrt{ P_s }$

Therefore
\begin{equation}\underline{
\varepsilon( S_o )^2 = P_o + P_s * ( a_o^2 / a_s^2 ) 
                      = PADU * ( D_o + D_s *( a_o^2 / a_s^2 ) -
                        BIASLE * a_o ( 1 + a_o / a_s ) )}\end{equation}

\newpage
The second method of calculating the errors assumes that the variance in
the sky aperture corresponds to the photon noise. This allows the photon
errors to be calculated without knowing BIASLE. One additional definition
has to be given\\

Standard deviation in sky aperture per pixel in data units $=\sigma_s$\\

If the photon error $\sqrt{P_s}$ is equated to the standard deviation
$PADU * \sigma_s$ then the total number of photons in the sky aperture is
given by
\[P_s = a_s * PADU^2 * \sigma_s^2\]

The offset in the sky aperture can now be calculated
\[BIASLE = ( D_s / a_s ) - ( PADU * \sigma_s^2 )\]

Substituting this into the calculation of the error gives
\[\varepsilon( S_o )^2 = PADU * ( D_o - D_s * ( a_o / a_s ) +
                         PADU * \sigma_s^2 * a_o * ( 1 + a_o / a_s ) )\]

or
\begin{equation}\underline{
\varepsilon( S_o )^2 = S_o + PADU^2 * \sigma_s^2 * a_o * ( 1 + a_o / a_s ) )
}\end{equation}

The third method of calculating the errors sums the data variances from
the variance component of the NDF. Two additional definitions
have to be given\\

Sum of variance in object aperture       $= V_o$\\
Sum of variance in sky aperture          $= V_s$\\

The error is then calculated from
\begin{equation}\underline{
\varepsilon( S_o )^2 = PADU^2 * ( V_o + V_s * ( a_o / a_s )^2 )
}\end{equation}

The magnitude error is calculated from differentiating the magnitude equation
\[m=-2.5\log_{10}S_o\]

thus
\[\underline{dm=\frac{-2.5}{\ln10}*\frac{\varepsilon(S_o)}{S_o}}\]

\newpage
\section{TECHNIQUES OF APERTURE PHOTOMETRY}

In principle aperture photometry of digitised data is a straightforward
procedure. Put down a computer generated aperture over the grid of data
and add up the counts within the aperture. In astronomical applications the
usual purpose of aperture photometry is to measure the brightness of an
object without including possible contributions from contaminating sources
such as bias levels, sky, defects or other stars and galaxies.
Some if not all of these contaminants will always be present in a finite
sized aperture and so this `background' has to be accounted for.
If it were possible the best place to estimate this background would be
behind the object, i.e. in the object aperture with the object not there.
As this is usually not possible to achieve, except for the case of moving
objects or supernovae, the usual method is to estimate the background
from other regions close to the object.

Estimating the contribution in this background is not always straightforward.
In the simplest case the histogram of pixel values in the background will
have an approximately gaussian distribution, due to random fluctuations,
and the best estimator is a simple mean.
It is however common for real astronomical situations to be less
straightforward than this.
Other contributors are likely to be present, such as non-random noise,
bad pixels, cosmic rays, and the presence of other objects in the background.
Even when the possible contaminating objects are very faint compared to the
object to be measured the histogram of pixel values can be sufficiently
skewed to result in the mean giving an estimate of the sky too poor for high
precision photometry.
In this case it is usual to use some sort of clipping ( or filtering ) to
remove the effects of such contamination.

There are conflicting interests at work here. On the one hand it is
desirable to use a large background aperture to get good statistics, but
on the other hand this increases the probability of introducing extra
contamination or of sampling areas which are uncharacterisitic of the
background near the object.

Of course if there is contamination in the background aperture then there
will almost certainly be some present in the object aperture. The effects
of this on the measurement depend on the density of objects and the ratio
of the sizes of the object and sky apertures. If the sky aperture is
larger than the object aperture and the objects are randomly distributed,
then there is a greater chance of having more contaminating objects in the
sky aperture than in the object aperture.
If the density of objects is large then the proportion in each aperture will
be similar, but when the number of objects is small then the chances of
the larger aperture being disproportionately endowed will increase. This
is best seen by considering the limiting case of one contaminating object
and its most probable location. In astronomical situations the smaller
populations tend to be the brighter objects and thus have an even greater
effect on the results.

An ideal filter would therefore include the many faint objects which
inhabit each aperture equally, but exclude the rare bright objects, which
are more likely to occur in the larger aperture. Unfortunately in reality
the spectra of densities and brightnesses are continuous and a clean
rejection scheme is difficult to construct.

The best scheme would seem to be to take many independent samples
using the same sized aperture as for the object. A new population is made
from the mean sky value in each aperture, and the peak of the histogram of
values, here called the mode, is used as the sky value. The mode is a
maximum likelihood estimator and this scheme ensures that the most likely
sky value averaged over the aperture size is used.
The problem in this case
is to get sufficient independent samples close to the object aperture and
therefore individual pixel values are usually used to construct the
sample population.

What these arguments are leading to is that there is no unique answer to
the question which is the best estimate for the sky; it depends on the
circumstances.
PHOTOM offers a choice of three estimators, a simple mean, a mean with
$2\sigma$ clipping and the mode.
When running jobs non-interactively it is best to use one of the estimators
that performs clipping; either the mode or the mean with $2\sigma$ rejection.
In the presence of positive contamination the mode will, in general,
provide the most rejection, the 2 sigma clipping the next and the mean
will provide no rejection. If the user suspects that one of the
estimators may be giving wrong results then they should try one of the
others.

\subsection{PHOTOPT - examining PHOTOM's performance}

PHOTOPT is an auxiliary package to examine the performance of the various
sky estimators used by PHOTOM. As indicated in the last section the
best choice of estimator depends on the circumstances.
The idea behind PHOTOPT is to put down an aperture on a random piece of sky,
estimate the background from a concentric aperture using one of the sky
estimators offered by PHOTOM, and subtract this estimated sky from the
flux in the central aperture. If the sky estimator correctly estimates
the sky then the difference of the two will be zero. Since an automatic
procedure cannot be certain of selecting a representative piece of sky,
the procedure is repeated a number of times to increase the statistics.

PHOTOPT selects regularly spaced points within an image, puts down an
aperture at each of the points and calculates the difference between
the central aperture and the sky estimate from the surrounding annulus.
Each of the three sky estimators offered by PHOTOM are tried in turn on
the same set of points so that a comparison can be made between the three
methods. The output is in the form of two graphs for each estimator. The
first shows the difference between the central aperture and the sky
estimator for each of the samples. The sample number forms the x-axis
and the difference (~object - sky~) given in photons per pixel forms
the y-axis. The second plot shows a histogram of the differences for
the samples. The differences are binned into suitable intervals and form
the x-axis of the histogram, with the y-axis giving the number of samples
in each difference bin.

PHOTOPT has a number of parameters in common with PHOTOM; the aperture
shape parameters, SEMIM, ECCEN and ANGLE, the scaling factors PADU and
SATURE and the background annulus size INNER and OUTER. The two
parameters unique to PHOTOPT are NP, the number of points to sample, up
to a maximum of 100, and the RANGE, which defines the bounds of the
data value used in the plot. The number of points sampled may not exactly
equal the number requested as the program automatically positions the
points to be on a rectangular grid which evenly covers the whole data array.
The program also ensures that the density of points does not result in
the central apertures overlapping.

PHOTOPT can be run from DCL or ICL. From ICL use :
\begin{quote}
\begin{verbatim}
     ICL> PHOTOPT
\end{verbatim}
\end{quote}
From DCL use (assuming ADAMSTART has been executed) :
\begin{quote}
\begin{verbatim}
     $ PHOTOPT
\end{verbatim}
\end{quote}

\end{document}
