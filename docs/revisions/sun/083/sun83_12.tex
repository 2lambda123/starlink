%   This document contains diagrams that are included with the PostScript 
%   specific \special command. The diagrams are generated by running the 
%   program SUN83-C. The five GKS_72.PS files produced must then be renamed to
%   SUN83-C1.PS to SUN83-C5.PS.
%
\documentstyle[11pt]{article} 
\pagestyle{myheadings}

%------------------------------------------------------------------------------
\newcommand{\stardoccategory}  {Starlink User Note}
\newcommand{\stardocinitials}  {SUN}
\newcommand{\stardocnumber}    {83.12}
\newcommand{\stardocauthors}   {D L Terrett}
\newcommand{\stardocdate}      {11 December 1991}
\newcommand{\stardoctitle}     {GKS --- Graphical Kernel System (7.2)}
%------------------------------------------------------------------------------

\newcommand{\stardocname}{\stardocinitials /\stardocnumber}
\renewcommand{\_}{{\tt\char'137}}     % re-centres the underscore
\markright{\stardocname}
\setlength{\textwidth}{160mm}
\setlength{\textheight}{230mm}
\setlength{\topmargin}{-2mm}
\setlength{\oddsidemargin}{0mm}
\setlength{\evensidemargin}{0mm}
\setlength{\parindent}{0mm}
\setlength{\parskip}{\medskipamount}
\setlength{\unitlength}{1mm}

%------------------------------------------------------------------------------
% Add any \newcommand or \newenvironment commands here
%------------------------------------------------------------------------------

\begin{document}
\thispagestyle{empty}
SCIENCE \& ENGINEERING RESEARCH COUNCIL \hfill \stardocname\\
RUTHERFORD APPLETON LABORATORY\\
{\large\bf Starlink Project\\}
{\large\bf \stardoccategory\ \stardocnumber}
\begin{flushright}
\stardocauthors\\
\stardocdate
\end{flushright}
\vspace{-4mm}
\rule{\textwidth}{0.5mm}
\vspace{5mm}
\begin{center}
{\Large\bf \stardoctitle}
\end{center}
\vspace{5mm}

%------------------------------------------------------------------------------
%  Add this part if you want a table of contents
%  \setlength{\parskip}{0mm}
%  \tableofcontents
%  \setlength{\parskip}{\medskipamount}
%  \markright{\stardocname}
%------------------------------------------------------------------------------

\begin{quote}\bf
The RAL/ICL GKS is proprietary software and unauthorized copies must not be
made.
Further information is available from the Starlink project office.
\end{quote}

\section{Introduction}
This note describes the release on Starlink of the RAL/ICL GKS~7.2 graphics
package.

GKS has two major advantages over other graphics packages:
\begin{itemize}                   

\item It is an international standard and real portability of graphics software
is now possible. An illustration of this is that the NCAR
package---an extensive library of scientific graphics utilities produced by the
National Center for Atmospheric Research in the USA---runs on the RAL/ICL GKS
without any modification, despite the fact that it was developed using a
completely independent GKS implementation. Similarly, both SGS and PGPLOT,
packages developed within or in collaboration with Starlink, have been run
using the DEC GKS on a VAXstation. 

\item No other package provides the ability to write device independent graphics
programs to the extent that GKS does.
Only with GKS can you write a program that will run, and produce good pictures,
on all devices supported by the implementation, including devices not supported
at the time the program was written.
Programs can nonetheless still fully exploit all the facilities offered by the
hardware.
\end{itemize}
The RAL/ICL GKS implementation has several advantages of its own in addition:
\begin{itemize}
\item A substantial support commitment is being made by the SERC.
\item The implementation will be in use on several hundred computers, of many
different types, throughout the UK academic community.
This means that the software receives a large amount of testing under a wide
variety of conditions and has matured into a highly reliable package.
\item It contains features which are not all found in any other single package:
\begin{itemize}
\item Area fill with several fill styles on {\em all\/} devices.
\item Cell array supported on dot matrix printers.
\item Multiple fonts.
\item Selective clearing of display surface.
\end{itemize}
\end{itemize}
\section{Documentation}
The RAL/ICL GKS implementation is described in the RAL GKS Guide (obtainable
from your Starlink site manager or Starlink user support at RAL).
Appendix~\ref{workstations} of this note contains workstation specific 
information for devices
for which Starlink has written device handlers.
Where this note and the RAL GKS guide differ this note is correct.

Writers of applications programs may prefer to do low level graphics by means
of the SGS package rather than by using GKS directly.
SGS is described in SUN/85 (SGS Users Manual).
Even where a program is pure GKS (perhaps because it has been imported from
another system) you are recommended to take advantage of the workstation name
scheme provided by GNS (SUN/57).

\section{Workstation Types}                        
The workstation types for all the drivers distributed by Starlink are 
listed in Tables \ref{terminals}--\ref{metafiles}.
Not all of them may be installed at every site and additional local devices may
have been added; consult your site manager for more information.

The support categories are as follows:
\begin{itemize}
\item S---Supported by Starlink. See Appendix~\ref{workstations} of 
this document for more
information.
\item O---Obsolete device, limited support only. See 
Appendix~\ref{workstations}.
\item C---Supported by RAL Informatics; more details can be found in 
Appendix~\ref{workstations}
of the RAL GKS Guide. 
\item A---Supported by the AAO on a best efforts basis. See 
Appendix~\ref{workstations}.
\item U---Not supported; no documentation available.
\item T---\TeX\ specific workstation. See SUN/9.
\end{itemize}

Where Appendix~\ref{workstations} contains more information about the 
workstation a page
number is given.

\begin{table}\caption{Graphics Terminals}\label{terminals}
\[\begin{tabular}{|l|c|c|c|}\hline        
\multicolumn{1}{|c|}{Description} &Workstation type &Support &Page No.\\\hline
Sigmex R5664 4 plane colour         & 101 &C &-\\
Sigmex 5674 'A' series, 1 plane     & 102 &C &-\\
Sigmex 5674 'B' series, 1 plane     & 103 &C &-\\
Sigmex 5674 'A' series, 4 plane     & 104 &C &-\\
Sigmex 5674 'B' series, 4 plane     & 105 &C &-\\
Sigmex 5674 4 plane colour          & 106 &C &-\\
Sigmex 5688 8 plane                 & 107 &C &-\\
Sigmex 5472 high resolution 1 plane & 109 &C &-\\
Sigmex 5472 high resolution 4 plane & 110 &C &-\\
Tektronix 4010                      & 201 &C &-\\
Tektronix 4014                      & 203 &C &-\\
HP 2648A                            & 400 &A &\pageref{hpgt}\\  
Cifer 2634                          & 800 &O &\pageref{cifgt}\\
Cifer T5                            & 801 &S &\pageref{cifgt}\\
Pericom 7800                        & 825 &S &\pageref{pergt}\\
Pericom Graphpack/MG Series         & 827 &S &\pageref{pergt}\\
Lear Siegler ADM-3                  & 830 &O &\pageref{admgt}\\
GraphOn 235                         & 845 &S &\pageref{gragt}\\
BBC micro + Termulator chip         & \makebox[0pt][r]{1}500&O
&\pageref{bbc}\\    
\hline\end{tabular}\]\end{table}

\begin{table}\caption{Image Displays}
\[\begin{tabular}{|l|c|c|c|}\hline
\multicolumn{1}{|c|}{Description} &Workstation type &Support &Page No.\\\hline
Sigmex ARGS 7000                       &  160 &S &\pageref{args}\\
Sigmex ARGS 7000 overlay               &  161 &S &\pageref{args}\\
Sigmex ARGS 7000 + VT terminal         &  162 &S &\pageref{args}\\
Sigmex ARGS 7000 overlay + VT terminal &  163 &S &\pageref{args}\\
Digisolve Ikon                         & \makebox[0pt][r]{3}200 &S
&\pageref{ikon}\\
Digisolve Ikon overlay                 & \makebox[0pt][r]{3}201 &S
&\pageref{ikon}\\
Digisolve Ikon + VT terminal           & \makebox[0pt][r]{3}202 &S
&\pageref{ikon}\\
Digisolve Ikon overlay + VT terminal   & \makebox[0pt][r]{3}203 &S
&\pageref{ikon}\\
\hline\end{tabular}\]\end{table}

\begin{table}\caption{Laser Printers}
\[\begin{tabular}{|l|c|c|c|}\hline
\multicolumn{1}{|c|}{Description} &Workstation type &Support &Page No.\\\hline
QMS Lasergrafix 800 (portrait---s/w line-types)  &2000 &A
&\pageref{qms}\\
QMS Lasergrafix 800 (landscape---s/w line-types) &2001 &A
&\pageref{qms}\\
QMS Lasergrafix 800 (portrait---h/w line-types)  &2010 &A
&\pageref{qms}\\
QMS Lasergrafix 800 (landscape---h/w line-types) &2011 &A
&\pageref{qms}\\
Canon LBP-8 A2/II (landscape)                &2600 &S &\pageref{can}\\
Canon LBP-8 A2/II (portrait)                 &2601 &S &\pageref{can}\\
Canon LBP-8 for \TeX\ (landscape)            &2610 &T &\pageref{can}\\
Canon LBP-8 for \TeX\ (portrait)             &2611 &T &\pageref{can}\\
72 dpi Postscript (portrait)                 &2700 &U &\pageref{ps}\\
72 dpi Postscript (landscape)                &2701 &U &\pageref{ps}\\
72 dpi EPSF Postscript (portrait)            &2702 &U &\pageref{ps}\\
72 dpi EPSF Postscript (landscape)           &2703 &U &\pageref{ps}\\
300 dpi Postscript (portrait)                &2704 &S &\pageref{ps}\\
300 dpi Postscript (landscape)               &2705 &S &\pageref{ps}\\
300 dpi EPSF Postscript (portrait)           &2706 &S &\pageref{ps}\\
300 dpi EPSF Postscript (landscape)          &2707 &S &\pageref{ps}\\
DEC LN03 (low resolution)                    &3700 &U &-\\
DEC LN03 (high resolution)                   &3701 &U &-\\
\hline\end{tabular}\]\end{table}

\begin{table}\caption{Ink Jet Printers}
\[\begin{tabular}{|l|c|c|c|}\hline
\multicolumn{1}{|c|}{Description} &Workstation type &Support &Page No.\\\hline
DEC LJ250 (portrait)                &2708 &S &\pageref{inkj}\\
DEC LJ250 (landscape)               &2709 &S &\pageref{inkj}\\
\hline\end{tabular}\]\end{table}

\begin{table}\caption{Workstations}
\[\begin{tabular}{|l|c|c|c|}\hline
\multicolumn{1}{|c|}{Description} &Workstation type &Support &Page No.\\\hline
VAXstation (VWS) 1 plane monochrome   &1740 &S &\pageref{vws}\\
VAXstation (VWS) 4 plane intensity    &1741 &S &\pageref{vws}\\
VAXstation (VWS) 8 plane colour       &1742 &S &\pageref{vws}\\
VAXstation (VWS) 8 plane intensity    &1743 &S &\pageref{vws}\\
X-Windows                             &3800 &S &\pageref{xwin}\\
X-Windows                             &3801 &S &\pageref{xwin}\\
X-Windows                             &3802 &S &\pageref{xwin}\\
X-Windows                             &3803 &S &\pageref{xwin}\\
\hline\end{tabular}\]\end{table}

\begin{table}\caption{Dot Matrix Printers}
\[\begin{tabular}{|l|c|c|c|}\hline
\multicolumn{1}{|c|}{Description} &Workstation type &Support &Page No.\\\hline
Versatec 1200A/V80 (fan fold)         &1100 &O &\pageref{vers}\\
Versatec 1200A/V80 (roll)             &1101 &O &\pageref{vers}\\
Canon 1080 (Epson compatible) printer &1400 &U &-\\
Printronix P300 (single page)         &1200 &S &\pageref{prtx}\\
Printronix P300 (square)              &1201 &S &\pageref{prtx}\\
\hline\end{tabular}\]\end{table}

\begin{table}\caption{Pen Plotters}
\[\begin{tabular}{|l|c|c|c|}\hline
\multicolumn{1}{|c|}{Description} &Workstation type &Support &Page No.\\\hline
HPGL plotter A3 paper             &410  &U &-\\
HPGL plotter A4 paper             &411  &U &-\\
HP 7550 A3 paper                  &430  &U &-\\
HP 7550 A4 paper                  &431  &U &-\\
Calcomp 81 sheet A4               &700  &C &-\\
Calcomp 81 sheet A3               &701  &C &-\\
Calcomp 81 roll A4                &702  &C &-\\
Calcomp 81 roll A3                &703  &C &-\\
Calcomp 81 roll                   &704  &C &-\\
Zeta 8                            &\makebox[0pt][r]{1}000 &S
&\pageref{zeta}\\
Zeta 8 (long workstation)         &\makebox[0pt][r]{1}001 &S
&\pageref{zeta}\\
Complot Houston EDP3              &\makebox[0pt][r]{3}900 &S&-\\
\hline\end{tabular}\]\end{table}

\begin{table}\caption{Metafile Workstations}\label{metafiles}
\[\begin{tabular}{|l|c|c|}\hline
\multicolumn{1}{|c|}{Description} &Workstation type &Support\\\hline
Metafile input (Annex E)  &10  &C\\
Metafile output (Annex E) &50  &C\\
\hline\end{tabular}\]\end{table}

\section{Connection Identifiers}

For workstations which are terminals, specifying a connection identifier
of 0 always selects the command terminal (i.e.\ the terminal used to log
on to the system); for other devices, 0 will select a `default' device (usually
via a logical name) or file name. Appendix~\ref{workstations} gives the 
default device name
for all such devices.

If the connection identifier is not 0 then the device or file name is
obtained by translating the logical name {\tt GKS\_\em{w}\_{\em c}}, where {\em
w} is the workstation type and {\em c} is the connection identifier. Note that
{\tt GKS\_\em{w}\_{\em c}} is a valid file name and so on workstations that
write files the workstation may be opened successfully even if no translation
exists, and a file called {\tt GKS\_{\em w}\_{\em c}.DAT} produced. 

\section{Compiling and Linking GKS programs}
Before compiling or linking a program that uses GKS you must first execute the
command:
\begin{quote}\tt
\$ GKS\_DEV
\end{quote}
(this command is defined by executing {\tt SSC:LOGIN.COM}).

Programs are then linked with GKS by:
\begin{quote}\tt
\$ LINK objmodule,GKS\_LINK/OPT
\end{quote}

\section{GKS Error Handler}
The error handling routine that is used by default reports errors via the
Starlink Error Reporting System (SUN/104) and the error channel parameter
passed to {\tt GOPWK} is ignored.

Using the ERR package for error reporting not only makes GKS's error reporting
the same as other Starlink subroutine libraries, but also allows programs
to detect that GKS has reported an error without having to supply their
own error reporting routine. The following example show how this is
done:
\begin{verbatim}
      INTEGER LASTER

*   Include GKS error codes
      INCLUDE 'GKS_ERR'

*   Call GKS routines....
....

*   Call err to see if an error has been reported
      CALL ERR_STAT(LASTER)

*   See if it was a GKS error
      IF ( LASTER .EQ. GKS__ERRROR ) THEN

*      Yes...
...

      ELSE

*      No...
...
      END IF
...
\end{verbatim}

If your program provides its own version of the GKS error handling routine
({\tt GERHND}) then the program must be linked with the object module library
containing the user callable routines instead of the shareable library.
\begin{quote}\tt
\$ LINK objmodule,GKS\_DIR:GKSLIBG/LIBRARY,SYS\$INPUT/OPT\\
GKS\_WS\_IMAGE/SHARE
\end{quote}
The standard error logging routine that your error handler is allowed to call
does use the error channel for outputting messages.

\section{External Names}
All the routines and common blocks visible to the linker, other than the 
names specified in the GKS standard, begin with the letters GK.

\section{Fonts}
In addition to the default font (font number 1) the following software fonts
are available on all workstations:

\[\begin{tabular}{|c|l|}\hline
Font number &\multicolumn{1}{|c|}{Description}\\\hline
 101 &Roman, Medium, Sans serif\\
 102 &Roman, Bold, Sans serif\\
 103 &Greek, Medium, Sans serif\\
 104 &Roman, Medium, Seriffed\\
 105 &Roman, Italic, Seriffed\\
 106 &Roman, Bold, Seriffed\\
 107 &Roman, Bold Italic, Seriffed\\
 110 &Greek, Medium, Seriffed\\
 115 &Mathematical\\\hline
\end{tabular}\]
These fonts are illustrated in Appendix~\ref{fonts}.
\section{GKS standards}
GKS is an international standard and yet several ``versions'' exist.
This section is intended to clear up any confusion that may exist.

During the development of the standard, GKS evolved a great deal.
Each revision of the standard document was allocated a number and versions of
the draft standard are referred to by these numbers.
These documents are drafts; there is only one GKS standard; the document finally
adopted by ISO as an international standard.

The first draft that is of more than historical interest is version~6.2, because
this version was implemented by the Technische Hochshule Darmstadt and obtained
at an early date by Starlink.
At that time there was no draft FORTRAN binding (the subroutine names and
argument lists) and the names were invented by Darmstadt.

The RAL/ICL GKS is an implementation of the 7.2 draft and differs in only minor
detail from the version adopted as an ISO standard (version 7.4).
The subroutine names conform to the draft standard FORTRAN binding (also
version~7.2).

A version that conforms to ISO standard is in preparation; the important
differences between 7.2 and 7.4 are:
\begin{itemize}
\item Implementation dependent fonts and hatch styles have negative numbers.
\item Cell Array (GCA) has extra arguments that allow subsets of arrays
to be plotted.
\end{itemize}
There will also be changes to the way connection identifiers are used to map
onto particular physical devices.

When 7.4 is introduced some programs will need minor modification and all
programs will have to be re-linked.

\section{Reporting Bugs and Problems}
Problems and suspected bugs should be reported to {\tt RLVAD::STAR} or
{\tt STAR@UK.\linebreak[0]AC.\linebreak[0]RUTHERFORD.\linebreak[0]STARLINK}.
Bugs will be verified and then, if the problem is not in a Starlink written
device driver, reported to Computing Services Division graphics section.

\section{Screen clear suppression}
When a GKS workstation is opened, the display surface is cleared.
For some data reduction system architectures this is inappropriate as it
prevents one applications program from adding to or interacting with a picture
drawn by another.
To circumvent this difficulty, Starlink has implemented an escape function which
suppresses the screen clearing.
This facility should only be used where absolutely necessary as it is unique to
Starlink's GKS and is only available on devices with device handlers written
or modified by Starlink.
It is only available for interactive devices where other techniques such as the
use of metafiles to redraw pictures are too slow to be useful.

To enable the suppression of screen clearing on those devices that support it,
the GKS routine GESC should be called with escape function identifier 1000 and
an ASCII value of 1 as the first character of the data record. The feature is
disabled by 0 in the first character.

An SGS function to access this facility is available and this should be used
wherever possible because the details of the parameters may change in the future
as function identifier values are registered with ISO.

\newpage\appendix
\section{Workstation Specific Information}\label{workstations}

\subsection{Sigmex ARGS 7000 Image display}
\label{args}
\subsubsection{Workstation types}

\[\begin{tabular}{|l|l|}\hline
160 &ARGS 7000 8 plane image display\\
161 &ARGS 7000 3 plane image overlay\\
162 &ARGS 7000 8 plane image display + VT terminal\\
163 &ARGS 7000 3 plane image overlay + VT terminal\\\hline
\end{tabular}\]                           

Opening the overlay workstation does not reset the ARGS in order to preserve
the contents of the image planes.
This means that if the ARGS has been left in a peculiar state by a previous
program the workstation may not behave correctly.

The default device name is {\tt ARGS\_DEFAULT}.

\subsubsection{Input devices}

\begin{description}

\item[Locator] A cross hair cursor.
Press any trackerball button except the right hand (red) button to terminate
the interaction.
A break is indicated by pressing the right hand (red) button.

On workstation types 162 and 163 the cursor can also be moved with the keypad
keys on a DEC VT type terminal. Keys 8, 6, 4 and 2 move the cursor up, right
left and down respectively and keys 7, 9, 1, and 3 move the cursor diagonally.
The speed of cursor movement is controlled by keys \fbox{PF1} (slowest) to
\fbox{PF4} (fastest). The interaction can be terminated by pressing any
alphanumeric key or a trackerball button. \fbox{Control}+Z indicates a break. 

Pressing a keypad key many times in quick succession (or allowing a key to
auto-repeat) can cause the interaction to terminate incorrectly and should
be avoided.

\item[Stroke] A cross hair cursor.
Each point is indicated by pressing one of the two middle (white) buttons;
the sequence of points is terminated by pressing the left hand (green) button.
A break is indicated by pressing the right hand (red) button.

\item[Choice] The trackerball buttons.
From left to right the buttons are 1, 2, 3, break.

\end{description}

\subsubsection{Deficiencies}
\begin{itemize}
\item Pick device not implemented.
\item Hardware text not implemented.
\item Initialization of input devices except locator position not implemented.
\item Read pixel and read pixel array not implemented
\end{itemize}                    

\subsection{Hewlett-Packard HP 2648A Graphics Terminal}
\label{hpgt}
\subsubsection{Workstation types}
\[\begin{tabular}{|l|l|}\hline
400 &HP 2648A\\\hline
\end{tabular}\]

\subsubsection{Operation}
On open workstation the terminal switches on the graphics screen and does not
interfere with the alpha screen. The alpha screen is only switched on if the
driver requires user input from the keyboard. On close workstation, neither
screen is interfered with. Both screens may be switched on and off at will
whilst plotting.

In order to make input functions work, keyboard switches G and H must be open.
This disables the DC1 / DC2 handshake.

It is possible to fit an extra board that contains extra buffer memory and
implements Control S / Control Q flow control. If this board is fitted then it
should be possible to run at 9600 baud. If it is not fitted then characters may
get lost (this will show up as missing vectors) and the terminal may have to be
run at 4800 baud.

\subsubsection{Input devices}
\begin{description}
\item[Locator] A cross hair cursor.
Press any key to indicate a point.
The break character is Control Z.
\item[Stroke] A rubber band cursor.
Press any key except \fbox{Return} to indicate a point.
The \fbox{Return} key signals the end of the entire stroke input.
The break character is \fbox{Control}+Z.
\item[Valuator] A value typed on the keyboard.
Anything but a valid floating point number is a break.
The prompt and echo will appear on the alpha screen.
\item[Choice] A single digit typed on the keyboard.
Any other key is a break.
The prompt and echo will appear on the alpha screen.
\end{description}

\subsubsection{Deficiencies}
\begin{itemize}
\item String and Pick devices not implemented.
\item Initialization of input devices not implemented.
\item Many hardware terminal features are not utilised and the driver is
      thus relatively slow.
\end{itemize}

\subsection{Cifer Graphics Terminals}
\label{cifgt}
\subsubsection{Workstation types}
\[\begin{tabular}{|l|l|}\hline
800 &Cifer 2634\\
801 &Cifer T5\\\hline
\end{tabular}\]

\subsubsection{Operation}

\paragraph{2634}
On open workstation the terminal switches to graphics mode and only the
graphics screen is displayed.
The terminal remains in graphics mode until the workstation is closed when it is
switched back to alpha mode.
\paragraph{T5}
On Open Workstation both the Text and Graphics screens are displayed.
To view the graphics screen only, press \fbox{SHIFT} + \fbox{F19}, to view 
the alpha screen press \fbox{SHIFT} + \fbox{F18}.
To view both, press \fbox{SHIFT} + \fbox{F20}.
The screen being viewed can be changed while plotting is in progress.
\subsubsection{Input devices}
\begin{description}
\item[Locator] A cross hair cursor.         
Press any key to indicate a point.
The break character is \fbox{Control}+Z.
\item[Stroke] A cross hair cursor.
Press any key except \fbox{Return} to indicate a point.
The \fbox{Return} key signals the end of the entire stroke input.
The break character is \fbox{Control}+Z.
\item[Valuator] A value typed on the keyboard.
Anything but a valid floating point number is a break.
The prompt and echo will appear on the alpha screen.
\item[Choice] A single digit typed on the keyboard.
Any other key is a break.
The prompt and echo will appear on the alpha screen.
\end{description}
\subsubsection{Deficiencies}
\begin{itemize}
\item String and Pick devices not implemented.
\item Initialization of input devices not implemented.
\end{itemize}

\subsection{Pericom Graphics Terminals}
\label{pergt}
\subsubsection{Workstation types}
\[\begin{tabular}{|l|l|}\hline
825 &Pericom 7800\\
827 &Pericom Graphpack\\
827 &Pericom MG series\\\hline
\end{tabular}\]

The Graphpack terminals have two blue keys and four white keys at the right hand
side of the keyboard; on a 7800 terminal these keys are grey.
The important difference between the two models is that the Graphpack is capable
of displaying both the alpha and graphics screens simultaneously while the 7800
is not.

\subsubsection{Operation}
\paragraph{7800}
On open workstation the terminal switches to graphics mode and only the graphics
screen is displayed.
The terminal remains in graphics mode until the workstation is closed when it is
switched back to alpha mode; this causes the graphics screen to become
invisible.
The graphics screen can be viewed by pressing the \fbox{SHIFT}+\fbox{GRAPH},
and the alpha screen viewed by pressing \fbox{Control}+\fbox{GRAPH}.
The terminal must not be switched into alpha mode during plotting.
\paragraph{Graphpack}
On Open Workstation both the Text and Graphics screens are displayed.
To view the graphics screen only press the \fbox{GRAPH} key, to view 
the alpha screen press \fbox{VDU}.
To view both, press both keys simultaneously and release the key corresponding
to the mode you want the terminal in, second.
The mode of the terminal can be changed while plotting provided that the
terminal is not actually executing graphics commands.

The following terminal setups must be set:
\[\begin{tabular}{lll}
SETUP G &block r &xx01\\
        &block v &0001\\
\end{tabular}\]
The equivalent on an MG series are:
\[\begin{tabular}{ll}
Graphics General   &M) CR Status term\\
Graphics General   &L) ESC=ESC\\
Graphics Modes     &D) Pericom 4014 graphics\\
Graphics Directory &H) GS/CAN Sets Term. Only\\
\end{tabular}\]

\subsubsection{Input devices}
\begin{description}
\item[Locator] A cross hair cursor.
Press any key to indicate a point.
The break character is \fbox{Control}+Z.
\item[Stroke] A cross hair cursor.
Press any key except \fbox{Return} to indicate a point.
The \fbox{Return} key signals the end of the entire stroke input.
The break character is \fbox{Control}+Z.
\item[Valuator] A value typed on the keyboard.
Anything but a valid floating point number is a break.
The prompt and echo will appear on the alpha screen.
\item[Choice] A single digit typed on the keyboard.
Any other key is a break.
The prompt and echo will appear on the alpha screen.
\end{description}

\subsubsection{Deficiencies}
\begin{itemize}
\item String and Pick devices not implemented.
\item Initialization of input devices not implemented.
\item Hardware dotted lines are not used because of a firmware bug in the
terminal.
\end{itemize}

\subsection{Lear-Siegler ADM-3}
\label{admgt}
\subsubsection{Workstation types}
\[\begin{tabular}{|l|l|}\hline
830 &ADM-3\\\hline
\end{tabular}\]

\subsubsection{Input devices}
\begin{description}
\item [Valuator] A value typed on the keyboard.
Anything but a valid floating point number is a break.
The prompt and echo will appear on the alpha screen.
\item[Choice] A single digit typed on the keyboard.
Any other key is a break.
The prompt and echo will appear on the alpha screen.
\end{description}

\subsubsection{Deficiencies}
\begin{itemize}
\item String device not implemented.
\end{itemize}
\subsection{GraphOn 235 Graphics Terminal}
\label{gragt}
\subsubsection{Workstation types}
\[\begin{tabular}{|l|l|}\hline
845 &GraphOn 235\\\hline
\end{tabular}\]

\subsubsection{Operation}
On Open Workstation both the Text and Graphics screens are displayed.

\subsubsection{Input devices}
\begin{description}
\item[Locator] A cross hair cursor.
Press any key to indicate a point.
The break character is \fbox{Control}+Z.
\item[Stroke] A cross hair cursor.
Press any key except \fbox{Return} to indicate a point.
The \fbox{Return} key signals the end of the entire stroke input.
The break character is \fbox{Control}+Z.
\item[Valuator] A value typed on the keyboard.
Anything but a valid floating point number is a break.
The prompt and echo will appear on the alpha screen.
\item[Choice] A single digit typed on the keyboard.
Any other key is a break.
The prompt and echo will appear on the alpha screen.
\end{description}

\subsubsection{Deficiencies}
\begin{itemize}
\item String and Pick devices not implemented.
\item Initialization of input devices not implemented.
terminal.
\end{itemize}

\subsection{Zeta 8 pen plotter}
\label{zeta}
\subsubsection{Workstation types}
\[\begin{tabular}{|l|l|}\hline
1000 &4 by 3 aspect ratio workstation\\ 
1001 &0.8 metre long workstation\\\hline
\end{tabular}\]

The default file name is {\tt ZETA.DAT}.

\subsubsection{Operation}
The output from a program that uses this workstation is Zeta plot commands.
These are normally stored in a file which must be copied to the plotter in
a site dependent way; probably one of the following:
\begin{itemize}
\item Printing the file on a queue with a {\tt PRINT} command.
\item Copying the file to the terminal with the plotter connected with a
{\tt COPY} command.
\item Logging onto the terminal connected to the plotter and typing the file.
\end{itemize}
Before sending the file you should ensure that the pen holder is loaded with
the following coloured pens, starting from the left hand end:

\begin{center}
BLACK RED GREEN BLUE YELLOW PURPLE BROWN ORANGE
\end{center}
           
\subsubsection{Deficiencies}
\begin{itemize}
\item Hardware text not implemented.
\end{itemize}

\subsection{Versatec 1200A and V80 printer plotter}
\label{vers}
\subsubsection{Workstation types}
\[\begin{tabular}{|l|l|}\hline
1100 &Fan fold single page\\
1101 &Roll paper. Square workstation\\\hline
\end{tabular}\]

The default file name is {\tt VERSATEC.BIT}

\subsubsection{Operation}
The output from a program that uses these workstations is a file containing a
bit map.
This file must be spooled to the Versatec printer with the command:
\begin{quote}\tt
\$ PRINT/QUEUE=SYS\_VERSATEC/PASSALL {\em filename}
\end{quote}
The file is potentially very large; approximately 1000 blocks per frame.

\subsection{Printronix P300 printer}
\label{prtx}
\subsubsection{Workstation types}
\[\begin{tabular}{|l|l|}\hline
1200 &Single page\\
1201 &Square workstation\\\hline
\end{tabular}\]

The default file name is {\tt PRINTRONIX.BIT}.

\subsubsection{Operation}
The output from a program that uses these workstations is a file containing a
bit map.
This file must be spooled to the Printronix printer with the command:
\begin{quote}\tt
\$ PRINT/QUEUE=SYS\_PRINTRONIX/PASSALL {\em filename}
\end{quote}

The file is potentially very large; approximately 700 blocks per frame.

\subsection{BBC micro with Termulator chip}
\label{bbc}
\subsubsection{Workstation types}
\[\begin{tabular}{|l|l|}\hline
1500 &BBC micro\\\hline
\end{tabular}\]

\subsubsection{Operation}
On Open workstation the terminal is switched to graphics mode and the bottom
three lines of the screen reserved for text.
The terminal is switched back to VT100 emulation mode (which also clears the
screen) by pressing \fbox{Control}+\fbox{f2}.

\subsubsection{Input devices}
\begin{description}
\item[Locator] A cross hair cursor.
Press any key to indicate a point.
The break character is \fbox{Control}+Z.
\item[Stroke] A cross hair cursor.
Press any key except \fbox{Return} to indicate a point.
The \fbox{Return} key signals the end of the entire stroke input.
The break character is \fbox{Control}+Z.
\item[Valuator] A value typed on the keyboard.
Anything but a valid floating point number is a break.
The prompt and echo will appear on the text portion of the screen.
\item[Choice] A single digit typed on the keyboard.
Any other key is a break.
The prompt and echo will appear on the text portion of the screen.
\end{description}

\subsubsection{Deficiencies}
\begin{itemize}
\item String and Pick devices not implemented.
\item Hardware text not implemented.
\item Initialization of input devices not implemented.
\end{itemize}

\subsection{VAXstations running VWS}
\label{vws}
\subsubsection{Workstation types}
\[\begin{tabular}{|l|l|}\hline
1740 &1 plane monochrome\\
1741 &4 plane intensity\\
1742 &8 plane colour\\
1743 &8 plane intensity\\\hline
\end{tabular}\]

\subsubsection{Operation}
When the workstation is opened a new window is created. Up to four windows may
be created by using different connection identifiers in each call to ``Open
Workstation'' ({\tt GOPWK}). The default window 
size is 768 by 640 pixels but the size and position of the window
can be controlled in one of two ways:
\begin{itemize}
\item Setting the values of the following logical names:
\[\tt\begin{tabular}{l}
GUIS\_XORIG\_n\\
GUIS\_YORIG\_n\\
GUIS\_XSIZE\_n\\
GUIS\_YSIZE\_n\\
\end{tabular}\]
\item Creating a text file containing four lines giving the values of:
\[\begin{tabular}{l}
x origin\\
y origin\\
x size\\
y size\\
\end{tabular}\]
and assigning  the name of the file to the logical name {\tt GUIS\_OPT\_n}.
\end{itemize}
where {\tt n} is the connection identifier (in the range 0--9).
The window size is given in pixels and the origin in cm. The maximum size of
window allowed is 996 by 819 pixels; larger values are ignored, as are the
origin settings if they would result in all or part of the window lying outside
the boundaries of the screen.

If the size of the window is altered in this way the behaviour of "Inquire
Maximum Display Surface" ({\tt GQMDS}) deviates from the GKS standard as
the actual size of the window is not known until the workstation is open and
depends on the connection identifier.

\subsubsection{Input devices}
\begin{description}
\item[Locator] A cross hair cursor.
Press the left hand or centre mouse button to indicate a point. A break is
indicated by pressing the right hand mouse button.
\item[Valuator] A value typed on the keyboard.
Anything but a valid floating point number is a break.
\item[Choice] A single digit typed on the keyboard.
Any other key is a break.
\item[String] A string of characters typed on the keyboard.
\end{description}
The keyboard must be attached to the GKS window for input to be accepted from
the keyboard.

\subsubsection{Hardcopy}

Hardcopy from a GKS Vaxstation window can be produced on an inkjet
printer.  Both the DEC {\it Companion Colour Printer} (LJ250/LJ252) and
the HP {\it Paintjet} are supported. The software is designed to work on 
any model of Vaxstation but has only been tested 
on 8-plane colour versions so far. 

To enable hard-copy output the logical name GUIS\_HCOPY must be defined
and it should be set to the name of the appropriate printer queue, e.g. 

{\tt \$ DEFINE GUIS\_HCOPY SYS\_INKJET}
\\
If this logical name is defined when the GKS window is closed a terminal
prompt is issued asking for a single key-stroke reply.  Press
carriage-return if no hard-copy is required; to copy the whole window
press W; to copy a section of the window press S.  In this case the
cursor will change to a cross-hair: move it to one corner of the desired
area with the mouse, drag it to the other corner with any mouse button
depressed and then release the button.  A rubber outline will show the
area to be used. If the selected area consists of fewer than 100 pixels
then it is assumed that the user did not want hard-copy after all. 
Otherwise the colour table and screen memory are read and converted to
HP PCL codes using an ordered-dither algorithm.
Pure white and black in the first 16 colour
indices are inverted in an attempt to produce black annotations on white
paper. A magnification factor of 2, 3 or 4 will be applied if the
resulting output will fit on standard page of 8 by 10 inches. At the
foot of the page a line is printed giving the user-name, image-name, and
date and time. The output goes on a file {\tt INKJET.DAT} on {\tt
SYS\$SCRATCH} which is sent to the specified queue when processing is
complete; this is deleted after printout. 

\subsubsection{Deficiencies}
\begin{itemize}
\item Stroke and Pick devices not implemented.
\item Initialization of input devices not implemented.
\end{itemize}

\subsection{QMS Lasergrafix 800 Laser printer}
\label{qms}
\subsubsection{Workstation types}
\[\begin{tabular}{|l|l|}\hline
2000 &portrait orientation  (software line-types)\\
2001 &landscape orientation (software line-types)\\
2010 &portrait orientation  (hardware line-types)\\
2011 &landscape orientation (hardware line-types)\\\hline
\end{tabular}\]

There is apparently a bug in QUIC (the QMS plotting language) firmware versions
{\tt <} 1.39 that can occasionally cause the printer to hang when drawing
hardware line-types. If you are using workstation 2010 or 2011 and you observe
this problem, try workstation 2000 or 2001. Obviously, it is is preferable
to use workstations 2010 and 2011 if there are no problems, because the
plotting files may be considerably smaller.

The default file name is {\tt QMS.DAT}.

\subsubsection{Operation}
The output from a program that uses these workstations is a file containing a
plot commands. This file must be spooled to the QMS printer with the command:
\begin{quote}\tt
\$ PRINT/QUEUE=SYS\_LASER/PASSALL  {\em filename}
\end{quote}

If no queue {\tt SYS\_LASER} is set up on your system, ask your system manager
for the appropriate queue to use.

The file is potentially very large if the Cell Array primitive is used.
Note that cell arrays are written using 8-bit characters, so the most
significant bit must not be discarded by VMS or by the laser printer.

\subsubsection{Deficiencies}
\begin{itemize}
\item Various hardware features, for example area fill, are not used.
\item There is an occasional problem with the use of the cell array primitive
in landscape mode. It works but the result can have horizontal black trails.
The effect appears to be sensitive to exactly how the cell array is positioned
on the display surface and is believed to be due to a firmware bug.
\end{itemize}

\subsection{Canon LBP-8 Laser printer}
\label{can}
\subsubsection{Workstation types}
\[\begin{tabular}{|l|l|}\hline
2600 &landscape orientation\\
2601 &portrait orientation\\\hline
\end{tabular}\]

The default file name is {\tt CANON.DAT}.

\subsubsection{Operation}
The output from a program that uses these workstations is a file containing a
plot commands.
This file must be spooled to the Canon printer with the command:
\begin{quote}\tt
\$ PRINT/QUEUE=SYS\_LASER/PASSALL  {\em filename}
\end{quote}

The file is potentially very large if the Cell Array primitive is used.

\subsection{Postscript}
\label{ps}
\subsubsection{Workstation Types}
\[\begin{tabular}{|l|l|}\hline
2700 &72 dpi portrait orientation\\
2701 &72 dpi landscape orientation\\
2702 &72 dpi EPSF portrait orientation\\
2703 &72 dpi EPSF landscape orientation\\
2704 &300 dpi portrait orientation\\
2705 &300 dpi landscape orientation\\
2706 &300 dpi EPSF portrait orientation\\
2707 &300 dpi EPSF landscape orientation\\\hline
\end{tabular}\]

The default file name is {\tt GKS\_72.PS}.

\subsubsection{Operation}
The output from a program that uses these workstations is a file containing
postscript commands. The non-EPSF (Encapsulated Postscript Format) 
workstations produce a file that is designed printed directly on a Postscript
printer.

This file must be spooled to the printer with a command such as:
\begin{quote}\tt
\$ PRINT/QUEUE=SYS\_POST/FORM=POST  {\em filename}
\end{quote}
Check local documention for the correct command for printing postscript files
on your system.

The EPSF workstations produce postscript files that can be merged with other
postscript output; for example, inserted into TeX documents with the
\verb+\special{}+ command and processed with {\tt DVIPS} - see SUN/9.

\subsection{DEC LJ250 InkJet Printer}
\label{inkj}
\subsubsection{Workstation Types}
\[\begin{tabular}{|l|l|}\hline
2708 &portrait orientation\\
2709 &landscape orientation\\\hline
\end{tabular}\]

The default file name is {\tt GKS\_72.PS}.

\subsubsection{Operation}
The output from a program that uses these workstations is a file containing
colour postscript commands.

This file must be spooled to the printer with a command such as:
\begin{quote}\tt
\$ PSPRINT/QUEUE=SYS\_INKJET {\em filename}
\end{quote}

\subsection{Digisolve Ikon Image display}
\label{ikon}
\subsubsection{Workstation types}
\[\begin{tabular}{|l|l|}\hline
3200 &Ikon 8 plane image display\\
3201 &Ikon 8 plane overlay\\
3202 &Ikon 8 plane image display + VT terminal\\
3203 &Ikon 8 plane overlay + VT terminal\\\hline
\end{tabular}\]

The default device name is {\tt IKON\_DEFAULT}.

Opening the overlay workstation does not reset the Ikon in order to preserve
the contents of the image planes.
This means that if the Ikon has been left in a peculiar state by a previous
program the workstation may not behave correctly.
Apart from the background on the overlay workstation being transparent the
two workstations are identical.

\subsubsection{Input devices}
\begin{description}

\item[Locator] A cross hair cursor.
\begin{description}
\item[Ikon only workstations] The cursor is moved with the mouse.
Press any mouse button except the right hand button to terminate 
the interaction. A break is indicated by pressing the right hand button.
\item[Ikon + VT terminal workstations]
The cursor is moved with
the keypad keys on a DEC VT type terminal. Keys 8, 6, 4 and 2 move the cursor
up, right left and down respectively and keys 7, 9, 1, and 3 move the cursor
diagonally. The speed of cursor movement is controlled by keys \fbox{PF1}
(slowest)
to \fbox{PF4} (fastest). The interaction can be terminated by pressing any
alphanumeric key. \fbox{Control}+Z indicates a break. The mouse buttons cannot
be used.

Pressing a keypad key many times in quick succession (or allowing a key to
auto-repeat) can cause the interaction to terminate incorrectly and should
be avoided.

\end{description}
             
\item[Stroke] A cross hair cursor.
Each point is indicated by pressing the middle mouse button; the sequence
is terminated by pressing the left hand button.
A break is indicated by pressing the right hand button.
\item[Choice] The mouse buttons.
From left to right the buttons are 1, 2, break.
\end{description}

\subsubsection{Deficiencies}
\begin{itemize}
\item Pick device not implemented.
\item Hardware text not implemented.
\item Initialization of input devices (except locator position) not implemented.
\item Read pixel and read pixel array not implemented
\end{itemize}

\subsection{X-Windows Server}
\label{xwin}
\subsubsection{Workstation types}
\[\begin{tabular}{|l|l|}\hline
3800 &X Windows server\\
3801 &X Windows server\\
3802 &X Windows server\\
3803 &X Windows server\\\hline
\end{tabular}\]

\subsubsection{Operation}
When an X workstation is opened the device handler connects to the
default X display (defined by the {\tt SET DISPLAY} command) 
and searches the display for a GWM (Graphics Window Manager) window 
with the name {\tt GKS\_{\em nnnn}}, after attempting logical name
translation, where {\em nnnn} is the workstation type. If the window is
not found it is created. The connection
identifier is ignored and only one workstation of a particular
workstation type can be open at one time.

The size, number of colours and other properties of the window can be
controlled either by using the X resources database or by creating the
window with the {\tt xmake} command as described in SUN/130. The colours
of colour table entries 0 and 1 are defined by the background and
foreground colours of the window.

Changing the size of the window after the window has been created does
not alter the size of the plotting area.

When the workstation is closed, the window remains on the display and
opening another workstation with the same workstation type will
reconnect to the same window.

\subsubsection{Input devices}
\begin{description}
\item[Locator] A cross hair cursor.
Press any mouse button except the right hand button to indicate a point. A 
break is indicated by pressing the right hand mouse button.
\item[Choice] A single digit typed on the keyboard.
Any other key is a break.
\end{description}

\subsubsection{Deficiencies}
\begin{itemize}
\item Pick device not implemented.
\item String device not implemented.
\item Valuator device not implemented.
\item Hardware text not implemented.
\item Initialization of input devices (except locator position) not implemented.
\item Read pixel and read pixel array not implemented
\end{itemize}

\newpage\section{Logical Names}\label{logicals}
\begin{figure}[h]
\caption{Logical names required to run a GKS program}
\[\begin{tabular}{|l|l|l|}\hline
\multicolumn{1}{|c|}{Name}&\multicolumn{1}{c|}{Usual file name}&
\multicolumn{1}{c|}{Description}\\\hline
\tt GKS\_IMAGE&\tt GKS\_DIR:GKS\_IMAGE.EXE&GKS Shareable image\\
\tt GKS\_WS\_IMAGE&\tt GKS\_DIR:GKS\_WS\_IMAGE.EXE&Workstation Shareable image\\
\tt GKS\_WDT&\tt GKS\_DIR:GKSWDT.DAM&Workstation description file\\
\tt GKS\_EMF&\tt GKS\_DIR:GKSEMF.DAM&Error message texts\\
\tt GKS\_FONTS&\tt GKS\_DIR:GKSDBS.DAM&Character font defintions\\\hline
\end{tabular}\]
\end{figure} 

\begin{figure}[h]
\caption{Logical names required to develop a GKS program}
\[\begin{tabular}{|l|l|l|}\hline
\multicolumn{1}{|c|}{Name}&\multicolumn{1}{c|}{Usual file name}&
\multicolumn{1}{c|}{Description}\\\hline
\tt GKS\_LINK&\tt GKS\_DIR:GKS\_LINK.OPT&Linker options file\\
\tt GKS\_PAR&\tt GKS\_DIR:GKS.PAR &GKS enumnerated type definitions\\
\tt GKS\_EPAR&\tt GKS\_DIR:GKSE.PAR&GKS routine indentifier defintions\\
\tt GKS\_ERR&\tt GKS\_DIR:GKS\_ERR.FOR&Starlink GKS Error codes\\\hline
\end{tabular}\]
\end{figure} 
\newpage\section{Font Tables}\label{fonts}

\[\begin{picture}(160,220)
\put(0,220){\special{include sun83-c1.ps}}
\end{picture}\]

\setlength{\unitlength}{1mm}
\[\begin{picture}(160,220)
\put(0,220){\special{include sun83-c2.ps}}
\end{picture}\]

\setlength{\unitlength}{1mm}
\[\begin{picture}(160,220)
\put(0,220){\special{include sun83-c3.ps}}
\end{picture}\]

\setlength{\unitlength}{1mm}
\[\begin{picture}(160,220)
\put(0,220){\special{include sun83-c4.ps}}
\end{picture}\]

\setlength{\unitlength}{1mm}
\[\begin{picture}(160,220)
\put(0,220){\special{include sun83-c5.ps}}
\end{picture}\]

\end{document}
