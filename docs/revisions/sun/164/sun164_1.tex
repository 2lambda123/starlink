\documentstyle[11pt]{article} 
\pagestyle{myheadings}

%------------------------------------------------------------------------------
\newcommand{\stardoccategory}  {Starlink User Note}
\newcommand{\stardocinitials}  {SUN}
\newcommand{\stardocnumber}    {164.1}
\newcommand{\stardocauthors}   {D L Terrett}
\newcommand{\stardocdate}      {25 March 1993}
\newcommand{\stardoctitle}     {PSMERGE - Encapsulated PostScript Handling
Utility}
%------------------------------------------------------------------------------

\newcommand{\stardocname}{\stardocinitials /\stardocnumber}
\renewcommand{\_}{{\tt\char'137}}     % re-centres the underscore
\markright{\stardocname}
\setlength{\textwidth}{160mm}
\setlength{\textheight}{230mm}
\setlength{\topmargin}{-2mm}
\setlength{\oddsidemargin}{0mm}
\setlength{\evensidemargin}{0mm}
\setlength{\parindent}{0mm}
\setlength{\parskip}{\medskipamount}
\setlength{\unitlength}{1mm}

%------------------------------------------------------------------------------
% Add any \newcommand or \newenvironment commands here
%------------------------------------------------------------------------------

\begin{document}
\thispagestyle{empty}
SCIENCE \& ENGINEERING RESEARCH COUNCIL \hfill \stardocname\\
RUTHERFORD APPLETON LABORATORY\\
{\large\bf Starlink Project\\}
{\large\bf \stardoccategory\ \stardocnumber}
\begin{flushright}
\stardocauthors\\
\stardocdate
\end{flushright}
\vspace{-4mm}
\rule{\textwidth}{0.5mm}
\vspace{5mm}
\begin{center}
{\Large\bf \stardoctitle}
\end{center}
\vspace{5mm}

%------------------------------------------------------------------------------
%  Add this part if you want a table of contents
%  \setlength{\parskip}{0mm}
%  \tableofcontents
%  \setlength{\parskip}{\medskipamount}
%  \markright{\stardocname}
%------------------------------------------------------------------------------

An Encapsulated PostScript File (EPSF) is a PostScript file structured so that
it can be incorporated or included into another PostScript file (so that for
example a diagram created with a graphics application can be inserted into a
text document created with a word processor).

{\tt psmerge} is a utility program for merging one or more Encapsulated
PostScript Files into a single PostScript file. The input files can be
individually rotated, scaled and shifted. The output file can either be
Encapsulated PostScript or ``normal'' PostScript suitable for sending to a
printer. This allows:
\begin{itemize}
\item Pictures created by different applications to be overlayed.
\item Multiple pictures to be pasted onto a single page.
\item Pictures scaled before insertion into a \TeX document.
\end{itemize}

{\tt psmerge} always writes its output to the standard output channel ({\tt
SYS\$OUTPUT} on  VMS). On UNIX systems this can be piped to a file or even
straight to a printer queue; e.g.
\begin{quote}\tt
psmerge picture.ps > printme.ps
\end{quote}
On VMS the {\tt /OUTPUT} qualifier can be used to direct the output to a file;
e.g.
\begin{quote}\tt
PSMERGE/OUTPUT=PRINTME.PS PICTURE.PS
\end{quote}
On VMS the {\tt PSMERGE} command runs a command procedure so the {\tt /OUTPUT}
qualifier must come immediately after the command and before any arguments or
command options. Also, DCL command procedure are restricted to a
maximum of 8 parameters, this  means that if you want to supply more than 8
command parameters you must enclose the entire list in quotes ({\tt "}) so that
they appear as a single argument to the command procedure.

The examples above simply convert an EPS file to a normal PostScript file, the
only difference being that the normal PostScript file shifts the picture origin
to the bottom left corner of the printable area on the printer. This is useful
because if an EPS file is printed without conversion, the picture origin will be
at the bottom left corner of the paper and on most printers this means that the
bottom and left edges will lie outside the printable area and will be invisible.

The principle use of {\tt psmerge} is to combine more than one EPS File into a
single picture. For example:
\begin{quote}\tt
psmerge pic1.ps pic2.ps > printme.ps
\end{quote}
will overlay the two pictures with their origins at the same place on the
paper. For line graphics and text the order of the files doesn't make any
difference but if either picture contains filled areas or grey scale images it
does. Filled areas and grey scales are opaque and will obliterate anything
already plotted in the same region of the page. Therefore, for example, to
overlay a contour map on a grey scale image, the file containing the contour
map must appear after the one containing the image.

Pictures can be scaled, translated (shifted) and rotated. For example:
\begin{quote}\tt
psmerge -s0.5x0.5 -t72x144 -r10 pic1.ps > printme.ps
\end{quote}
will shrink the picture to half its normal size, move the origin 1 inch to the
right and 2 inches upwards and rotate it 10 degrees anti-clockwise. The
transformations are applied in the order that they appear on the command line
and are cumulative, so that, for example, {\tt -r45 -r45} is equivalent to {\tt
-r90} but {\tt -t72x72 -s2x2} does not have the same effect as {\tt -s2x2
-t72x72}. Translations are in units of 1/72 of an inch\footnote{1/72 of an
inch is sometimes referred to as a PostScript point and is the default unit for
PostScript commands.} and rotations are in degrees anti-clockwise.

After each EPS file is processed the scaling factors are reset to unity and the
translation and rotations reset to zero. The following example takes two A4
landscape pictures and pastes them one above the other on a single A4 page in
portrait orientation:
\begin{quote}\tt
psmerge -s0.707x0.707 -r90 -t550x396 pic1.ps \verb+\+\\
\hspace*{1cm}-s0.707x0.707 -r90 -t550x0 pic2.ps > printme.ps
\end{quote}

If {\tt -e} appears on the command line anywhere, the output file will be
another EPS file rather than normal PostScript. This allows files to be
processed by psmerge more than once (in a pipeline on UNIX) using the special
file name {\tt -} (hyphen) which means ``use the standard input as the input
file''. Therefore, the same effect as the previous example can be achieved with
the following command:
\begin{quote}\tt
psmerge -e -t0x560 pic1.ps pic2.ps | \verb+\+\\
\hspace*{1cm}psmerge -s0.707x0.707 -r90 -t550x0 - > printme.ps
\end{quote}
Instead of scaling each picture and positioning it on the page independently,
the two pictures are combined into one and then this single picture rotated and
scaled.

On VMS the operation has to be performed in two steps with the output from
the first stored in a temporary file:
\begin{quote}\tt
psmerge/output=temp.ps  -e -t0x560 pic1.ps pic2.ps \\
psmerge/output=printme.ps -s0.707x0.707 -r90 -t550x0 temp.ps
\end{quote}
\end{document}
