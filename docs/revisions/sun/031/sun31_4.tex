\documentstyle[11pt]{article} 
\pagestyle{headings}

%------------------------------------------------------------------------------
\newcommand{\stardoccategory}  {Starlink User Note}
\newcommand{\stardocinitials}  {SUN}
\newcommand{\stardocnumber}    {31.4}
\newcommand{\stardocauthors}   {R.F.~Warren-Smith \& \\
                                A.J.~Chipperfield}
\newcommand{\stardocdate}      {25 March 1993}
\newcommand{\stardoctitle}     {REF \\ [1ex]
                                Routines for Handling References\\
                                to HDS Objects}
\newcommand{\stardocversion}   {Version 1.1}
\newcommand{\stardocmanual}    {Programmer's Manual}
%------------------------------------------------------------------------------

\newcommand{\stardocname}{\stardocinitials /\stardocnumber}
\markright{\stardocname}
\setlength{\textwidth}{160mm}
\setlength{\textheight}{230mm}
\setlength{\topmargin}{-2mm}
\setlength{\oddsidemargin}{0mm}
\setlength{\evensidemargin}{0mm}
\setlength{\parindent}{0mm}
\setlength{\parskip}{\medskipamount}
\setlength{\unitlength}{1mm}

\renewcommand{\_}{{\tt\char'137}}
\renewcommand{\thepage}{\roman{page}}

\begin{document}

\thispagestyle{empty}
SCIENCE \& ENGINEERING RESEARCH COUNCIL \hfill \stardocname\\
RUTHERFORD APPLETON LABORATORY\\
{\large\bf Starlink Project\\}
{\large\bf \stardoccategory\ \stardocnumber}
\begin{flushright}
\stardocauthors\\
\stardocdate
\end{flushright}
\vspace{-4mm}
\rule{\textwidth}{0.5mm}
\vspace{5mm}
\begin{center}
{\Huge\bf  \stardoctitle \\ [2.5ex]}
{\LARGE\bf \stardocversion \\ [4ex]}
{\Huge\bf  \stardocmanual}
\end{center}
\vspace{15mm}

%------------------------------------------------------------------------------
%  Package Description
\begin{center}
{\Large\bf Description}
\end{center}

It is sometimes useful to use the Hierarchical Data System HDS (SUN/92) to
store {\em references } or {\em pointers} to other HDS objects. For instance,
this allows the same data object to be used in several places without the need
to have more than one copy. The REF library is provided to facilitate this data
object {\em referencing} process and the subsequent accessing of objects which
have been referenced in this way.

%------------------------------------------------------------------------------
%  Add this part if you want a table of contents
\newpage
\markright{\stardocname}
\null\vspace {5mm}
\begin {center}
\rule{80mm}{0.5mm} \\ [1ex]
{\Large\bf \stardoctitle \\ [2.5ex]
           \stardocversion} \\ [2ex]
\rule{80mm}{0.5mm}
\end{center}

\setlength{\parskip}{0mm}
\tableofcontents
\setlength{\parskip}{\medskipamount}

%------------------------------------------------------------------------------
%  Introduction page
\newpage
\markright{\stardocname}
\renewcommand{\thepage}{\arabic{page}}
\setcounter{page}{1}

\null\vspace {5mm}
\begin {center}
\rule{80mm}{0.5mm} \\ [1ex]
{\Large\bf \stardoctitle \\ [2.5ex]
           \stardocversion} \\ [2ex]
\rule{80mm}{0.5mm}
\end{center}
\vspace{30mm}

\section{INTRODUCTION}
This package enables the user to store references to HDS objects in special HDS
reference objects.
Although it would be possible for users to concoct their own scheme, the use of
this package will assist in portability and will in any case avoid re-inventing
the wheel.

\section{FACILITIES}

The package allows reference objects to be created and written and it allows
locators to referenced objects to be obtained.

The referenced object may be defined as {\em internal\/} in which case it is
assumed to be within the same container file as the reference object itself,
even if the reference object is copied to another container file.
In that case the reference must point to an object which has the same pathname
within the new file as it had in the old one.
References which are not {\em internal\/} will point to a named container file.

Reference objects may be copied and erased using DAT\_COPY and DAT\_ERASE\@.
Care must be taken when copying reference objects or referenced objects;
otherwise the reference may no longer point to the referenced object.

Referenced objects must exist at the time the reference is made or used.

The following subroutines are available:
\begin{description}
\begin{description}
\item [REF\_CRPUT] --- Create a reference object and put a reference in it.
\item [REF\_FIND] --- Obtain locator to an object (possibly {\em via} a
reference).
\item [REF\_GET] --- Obtain a locator to a referenced object.
\item [REF\_NEW] --- Create an empty reference object.
\item [REF\_PUT] --- Put a reference into a reference object.
\item [REF\_ANNUL] --- Annul a locator which may have been obtained {\em via} a 
reference.
\end{description}
\end{description}

\section{USING THE PACKAGE}

Two main uses for this package are foreseen:
\begin{enumerate}
\item To maintain a catalogue of HDS objects.
\item To avoid duplicating a large dataset.
\end{enumerate}
As an example of the second case, suppose that a large dataset is logically
required to form part of a number of other datasets.
To avoid duplicating the common dataset, the others may contain a reference to
it.

For example:
\begin{quote}
\begin{verbatim}
   Name                type                  Comments

DATA                DATA_SETS
  .SET1             SPECTRUM
     .AXIS1         _REAL(1024)          Actual axis data
     .DATA_ARRAY    _REAL(1024)
  .SET2             SPECTRUM
     .AXIS1         REFERENCE_OBJ        Reference to DATA.SET1.AXIS1
     .DATA_ARRAY    _REAL(1024)
  .SET3             SPECTRUM
     .AXIS1         REFERENCE_OBJ        Reference to DATA.SET1.AXIS1
     .DATA_ARRAY    _REAL(1024)  
   -
  etc.
\end{verbatim}
\end{quote}
Then a piece of code which handles structures of type SPECTRUM, which would
normally contain the axis data in .AXIS1 (as SET1 does), could be modified as
follows to handle an object .AXIS1 containing either the actual axis data or
a reference to the object which does contain the actual axis data.
\begin{quote}
\begin{verbatim}
*    LOC1 is a locator associated with a SPECTRUM object
*    Obtain locator to AXIS data
      CALL DAT_FIND(LOC1, 'AXIS1', LOC2, STATUS)
*    Modification to allow AXIS1 to be a reference object
*    Check type of object
      CALL DAT_TYPE(LOC2, TYPE, STATUS)
      IF (TYPE .EQ. 'REFERENCE_OBJ') THEN
          CALL REF_GET(LOC2, 'READ', LOC3, STATUS)
          CALL DAT_ANNUL(LOC2, STATUS)
          CALL DAT_CLONE(LOC3, LOC2, STATUS)
          CALL DAT_ANNUL(LOC3, STATUS)
      ENDIF
*    End of modification
*    LOC2 now locates the axis data wherever it is.
\end{verbatim}
\end{quote}
This code has been packaged into the subroutine {\bf REF\_FIND} which can be
used instead of DAT\_FIND in cases where the component requested may be a
reference object.

When a locator which has been obtained in this way is finished with, it should
be annulled using REF\_ANNUL rather than DAT\_ANNUL. 
This is so that, if the locator was obtained {\em via} a reference, the
HDS\_OPEN for the container file may be matched by an HDS\_CLOSE.
{\em Note that this should only be done when any other locators derived from
the locator to the referenced object are also finished with.}

\section{IMPLEMENTATION}

The way in which the package is implemented is described here for interest.
Programmers should not make use of this information; otherwise portability
is compromised.

A reference object is an HDS structure of type \verb%REFERENCE_OBJ% with
two components, {\bf FILE} and {\bf PATH}, of type \verb%_CHAR*(REF__SZREF)%\@.
{\bf REF\_\_SZREF} is defined in the {\bf REF\_PAR} include file.
\begin{description}
\item[FILE] contains the name of the container file for the referenced object.
This is set to spaces if the reference is {\em internal}\/.
\item[PATH] contains the pathname of the referenced object (as supplied by
HDS\_TRACE)\@.
The name of the top level component of the pathname will not be used in finding
the locator for the referenced object.
This fact allows structures containing internal references to be copied but
the path below the top level must still lead to an appropriate object.
\end{description}
Locators obtained {\em via} a reference are flagged as such by being linked to
the  group \$\$REFERENCED\$ using the subroutine HDS\_LINK.
This fact is used by REF\_ANNUL in determining whether or not HDS\_CLOSE should
be called for the container file of the object specified by the locator
argument.
Note that the effect of calling HDS\_CLOSE is to counter the HDS\_OPEN done 
in obtaining a locator to the referenced object.
The container file will only be physically closed if the container file
reference count goes to zero. 

\section{ERROR HANDLING}

The REF routines adhere throughout to the Starlink error-handling strategy
described in SUN/104. Most of the routines therefore carry an integer inherited
status argument called STATUS and will return without action unless this is set
to the value SAI\_\_OK\footnote{The symbolic constant SAI\_\_OK is defined in
the include file SAE\_PAR.} when they are invoked. When necessary, error
reports are made through the EMS\_ routines in the manner described in SSN/4.
This gives complete compatibility with the use of ERR\_ and MSG\_ routines in
applications (SUN/104).

\section{COMPILING AND LINKING}

\subsection{UNIX Systems}

Before compiling applications which use the REF library on UNIX systems, you
should normally ``log in'' for REF software development with the following
shell command:

\begin{verbatim}
   % ref_dev
\end{verbatim}

This will create links in your current working directory which refer to the REF
include files. You may then refer to these files using their standard (upper
case) names without having to know where they actually reside. These links will
persist, but may be removed at any time, either explicitly or with the command:

\begin{verbatim}
   % ref_dev remove
\end{verbatim}

If you do not ``log in'' in this way, then references to REF include files
should be in lower case and must contain an absolute pathname identifying the
Starlink include file directory, thus:

\begin{verbatim}
      INCLUDE '/star/include/ref_par'
\end{verbatim}

The former method is recommended.

Applications which use the ADAM programming environment (SG/4) may be linked
with the REF library by specifying `ref\_link\_adam` on the appropriate command
line. Thus, for instance, an ADAM A-task which calls REF routines might be
linked as follows:

\begin{verbatim}
   % alink adamprog.f `ref_link_adam`
\end{verbatim}

(note the use of backward quote characters, which are required).

``Stand-alone'' ({\em i.e.}\ non-ADAM) applications which use the REF library
may be linked by specifying `ref\_link` on the compiler command line. Thus, to
compile and link a stand-alone application called `prog', the following might
be used:

\begin{verbatim}
   % f77 prog.f `ref_link` -o prog
\end{verbatim}

\subsection{VMS Systems}

Users of the ADAM programming environment (SG/4) on VMS systems need take no
special steps in order to access routines from the REF library. The normal ADAM
startup commands:

\begin{verbatim}
   $ ADAMSTART
   $ ADAMDEV
\end{verbatim}

will ensure that all necessary definitions are made, and the standard ADAM link
commands will automatically access the appropriate version of the REF library.
Thus, for instance, an ADAM A-task which calls REF routines might be linked
simply as follows:

\begin{verbatim}
   $ FORTRAN ADAMPROG
   $ ALINK ADAMPROG
\end{verbatim}

For non-ADAM users, the initialisation command:

\begin{verbatim}
   $ REF_DEV
\end{verbatim}

must first be executed before compiling or linking an application which calls
REF routines. This will define all the logical names required for accessing
include files and linker options files.

The preferred method of linking ``stand-alone'' ({\em i.e.}\ non-ADAM)
applications which use the REF library is {\em via} the standard STAR\_LINK
options file. Thus, to compile and link a stand-alone application, the
following might be used:

\begin{verbatim}
   $ FORTRAN PROG
   $ LINK PROG,STAR_LINK/OPT
\end{verbatim}

If, for any reason, you wish to link explicitly with the REF library, rather
than {\em via} the STAR\_LINK library, then the options file REF\_LINK may be
used instead, thus:

\begin{verbatim}
   $ LINK PROG,REF_LINK/OPT
\end{verbatim}

\appendix
\newpage
\section{ROUTINE DESCRIPTIONS}
%+
%  Name:
%     LAYOUT.TEX

%  Purpose:
%     Define Latex commands for laying out documentation produced by PROLAT.

%  Language:
%     Latex

%  Type of Module:
%     Data file for use by the PROLAT application.

%  Description:
%     This file defines Latex commands which allow routine documentation
%     produced by the SST application PROLAT to be processed by Latex. The
%     contents of this file should be included in the source presented to
%     Latex in front of any output from PROLAT. By default, this is done
%     automatically by PROLAT.

%  Notes:
%     The definitions in this file should be included explicitly in any file
%     which requires them. The \include directive should not be used, as it
%     may not then be possible to process the resulting document with Latex
%     at a later date if changes to this definitions file become necessary.

%  Authors:
%     RFWS: R.F. Warren-Smith (STARLINK)

%  History:
%     10-SEP-1990 (RFWS):
%        Original version.
%     10-SEP-1990 (RFWS):
%        Added the implementation status section.
%     12-SEP-1990 (RFWS):
%        Added support for the usage section and adjusted various spacings.
%     10-DEC-1991 (RFWS):
%        Refer to font files in lower case for UNIX compatibility.
%     {enter_further_changes_here}

%  Bugs:
%     {note_any_bugs_here}

%-

%  Define length variables.
\newlength{\sstbannerlength}
\newlength{\sstcaptionlength}

%  Define a \tt font of the required size.
\font\ssttt=cmtt10 scaled 1095

%  Define a command to produce a routine header, including its name,
%  a purpose description and the rest of the routine's documentation.
\newcommand{\sstroutine}[3]{
   \goodbreak
   \rule{\textwidth}{0.5mm}
   \vspace{-7ex}
   \newline
   \settowidth{\sstbannerlength}{{\Large {\bf #1}}}
   \setlength{\sstcaptionlength}{\textwidth}
   \addtolength{\sstbannerlength}{0.5em} 
   \addtolength{\sstcaptionlength}{-2.0\sstbannerlength}
   \addtolength{\sstcaptionlength}{-4.45pt}
   \parbox[t]{\sstbannerlength}{\flushleft{\Large {\bf #1}}}
   \parbox[t]{\sstcaptionlength}{\center{\Large #2}}
   \parbox[t]{\sstbannerlength}{\flushright{\Large {\bf #1}}}
   \begin{description}
      #3
   \end{description}
}

%  Format the description section.
\newcommand{\sstdescription}[1]{\item[Description:] #1}

%  Format the usage section.
\newcommand{\sstusage}[1]{\item[Usage:] \mbox{} \\[1.3ex] {\ssttt #1}}

%  Format the invocation section.
\newcommand{\sstinvocation}[1]{\item[Invocation:]\hspace{0.4em}{\tt #1}}

%  Format the arguments section.
\newcommand{\sstarguments}[1]{
   \item[Arguments:] \mbox{} \\
   \vspace{-3.5ex}
   \begin{description}
      #1
   \end{description}
}

%  Format the returned value section (for a function).
\newcommand{\sstreturnedvalue}[1]{
   \item[Returned Value:] \mbox{} \\
   \vspace{-3.5ex}
   \begin{description}
      #1
   \end{description}
}

%  Format the parameters section (for an application).
\newcommand{\sstparameters}[1]{
   \item[Parameters:] \mbox{} \\
   \vspace{-3.5ex}
   \begin{description}
      #1
   \end{description}
}

%  Format the examples section.
\newcommand{\sstexamples}[1]{
   \item[Examples:] \mbox{} \\
   \vspace{-3.5ex}
   \begin{description}
      #1
   \end{description}
}

%  Define the format of a subsection in a normal section.
\newcommand{\sstsubsection}[1]{\item[{#1}] \mbox{} \\}

%  Define the format of a subsection in the examples section.
\newcommand{\sstexamplesubsection}[1]{\item[{\ssttt #1}] \mbox{} \\}

%  Format the notes section.
\newcommand{\sstnotes}[1]{\item[Notes:] \mbox{} \\[1.3ex] #1}

%  Provide a general-purpose format for additional (DIY) sections.
\newcommand{\sstdiytopic}[2]{\item[{\hspace{-0.35em}#1\hspace{-0.35em}:}] \mbox{} \\[1.3ex] #2}

%  Format the implementation status section.
\newcommand{\sstimplementationstatus}[1]{
   \item[{Implementation Status:}] \mbox{} \\[1.3ex] #1}

%  Format the bugs section.
\newcommand{\sstbugs}[1]{\item[Bugs:] #1}

%  Format a list of items while in paragraph mode.
\newcommand{\sstitemlist}[1]{
  \mbox{} \\
  \vspace{-3.5ex}
  \begin{itemize}
     #1
  \end{itemize}
}

%  Define the format of an item.
\newcommand{\sstitem}{\item}

%  End of LAYOUT.TEX layout definitions.
%.
\small
\sstroutine{
   REF\_ANNUL
}{
   Annul a locator to a referenced object
}{
   \sstdescription{
      This routine annuls the locator and, if the locator was linked to
      group \$\$REFERENCED\$, issues HDS\_CLOSE for the container file of
      the object.
   }
   \sstinvocation{
      CALL REF\_ANNUL( LOC, STATUS )
   }
   \sstarguments{
      \sstsubsection{
         LOC = CHARACTER $*$ ( DAT\_\_SZLOC ) (Given and Returned)
      }{
         Locator to be annulled.
      }
      \sstsubsection{
         STATUS = INTEGER (Given and Returned)
      }{
         Inherited global status.
      }
   }
   \sstnotes{
      This routine attempts to execute even if STATUS is set on entry,
      although no further error report will be made if it subsequently
      fails under these circumstances. In particular, it will fail if
      the locator supplied is not initially valid, but this will only
      be reported if STATUS is set to SAI\_\_OK on entry.
   }
}
\sstroutine{
   REF\_CRPUT
}{
   Create and write a reference object
}{
   \sstdescription{
      This routine creates a reference object as a component of a
      specified structure and writes a reference to an HDS object in
      it.  If the specified component already exists and is a reference
      object, it will be used. If it is not a reference object, an
      error is reported.  The reference may be described as {\tt "}internal{\tt "}
      which means that the referenced object is in the same container
      file as the reference object.
   }
   \sstinvocation{
      CALL REF\_CRPUT( ELOC, CNAME, LOC, INTERN, STATUS )
   }
   \sstarguments{
      \sstsubsection{
         ELOC = CHARACTER $*$ ( DAT\_\_SZLOC ) (Given)
      }{
         A locator associated with the structure which is to contain
         the reference object.
      }
      \sstsubsection{
         CNAME = CHARACTER $*$ ( DAT\_\_SZNAM ) (Given)
      }{
         The component name of the reference object to be created.
      }
      \sstsubsection{
         LOC = CHARACTER $*$ ( $*$ ) (Given)
      }{
         A locator associated with the object to be referenced.
      }
      \sstsubsection{
         INTERN = LOGICAL (Given)
      }{
         Whether or not the referenced object is {\tt "}internal{\tt "}.  Set this
         to .TRUE. if the reference is {\tt "}internal{\tt "} and to .FALSE. if it
         is not.
      }
      \sstsubsection{
         STATUS = INTEGER (Given and Returned)
      }{
         Inherited global status.
      }
   }
}
\sstroutine{
   REF\_FIND
}{
   Get locator to data object (via reference if necessary)
}{
   \sstdescription{
      This routine gets a locator to a component of a specified
      structure or, if the component is a reference object, it gets a
      locator to the object referenced. Any locator obtained in this
      way should be annulled, when finished with, by REF\_ANNUL so that
      the top-level object will also be closed if the locator was
      obtained via a reference.
   }
   \sstinvocation{
      CALL REF\_FIND( ELOC, CNAME, MODE, LOC, STATUS )
   }
   \sstarguments{
      \sstsubsection{
         ELOC = CHARACTER $*$ ( $*$ ) (Given)
      }{
         The locator of a structure.
      }
      \sstsubsection{
         CNAME = CHARACTER $*$ ( $*$ ) (Given)
      }{
         The name of the component of the specified structure.
      }
      \sstsubsection{
         MODE = CHARACTER $*$ ( $*$ ) (Given)
      }{
         Mode of access required to the object. ({\tt '}READ{\tt '}, {\tt '}WRITE{\tt '} or
         {\tt '}UPDATE{\tt '}). This is specified so that the container file of any
         referenced object can be opened in the correct mode.
      }
      \sstsubsection{
         LOC = CHARACTER $*$ ( DAT\_\_SZLOC ) (Returned)
      }{
         A locator associated with the object found.
      }
      \sstsubsection{
         STATUS = INTEGER (given and Returned)
      }{
         Inherited global status.
      }
   }
}
\sstroutine{
   REF\_GET
}{
   Get locator to referenced data object
}{
   \sstdescription{
      This routine gets a locator to an HDS object referenced in a
      reference object and links it to the group \$\$REFERENCED\$.  Any
      locator obtained in this way should be annulled, when finished
      with, by REF\_ANNUL so that the top-level object will also be
      closed.
   }
   \sstinvocation{
      CALL REF\_GET( ELOC, MODE, LOC, STATUS )
   }
   \sstarguments{
      \sstsubsection{
         ELOC = CHARACTER $*$ ( DAT\_\_SZLOC ) (Given)
      }{
         A locator associated with the reference object
      }
      \sstsubsection{
         MODE = CHARACTER $*$ ( $*$ ) (Given)
      }{
         Mode of access required to the object. ({\tt '}READ{\tt '}, {\tt '}WRITE{\tt '} or
         {\tt '}UPDATE{\tt '}). This is specified so that the container file of any
         referenced object can be opened in the correct mode.
      }
      \sstsubsection{
         LOC = CHARACTER $*$ ( DAT\_\_SZLOC ) (Returned)
      }{
         A locator pointing to the object referenced.
      }
      \sstsubsection{
         STATUS = INTEGER (Given and Returned)
      }{
         Inherited global status.
      }
   }
}
\newpage
\sstroutine{
   REF\_NEW
}{
   Create a new reference object
}{
   \sstdescription{
      This routine creates a reference object as a component of a
      specified structure.  If the component already exists, an error
      is reported.
   }
   \sstinvocation{
      CALL REF\_NEW( ELOC, CNAME, STATUS )
   }
   \sstarguments{
      \sstsubsection{
         ELOC = CHARACTER $*$ ( DAT\_\_SZLOC ) (Given)
      }{
         A locator associated with the structure which is to contain
         the reference object.
      }
      \sstsubsection{
         CNAME = CHARACTER $*$ ( DAT\_\_SZNAM ) (Given)
      }{
         The name of the component to be created in the structure
         located by ELOC.
      }
      \sstsubsection{
         STATUS = INTEGER (Given and Returned)
      }{
         Inherited global status.
      }
   }
}
\sstroutine{
   REF\_PUT
}{
   Write a reference into a reference object
}{
   \sstdescription{
      This routine writes a reference to an HDS object into an existing
      reference structure.  An error is reported if an attempt is made
      to write a reference into an object which is not a reference
      object.  The reference may be described as {\tt "}internal{\tt "} which means
      that the referenced object is in the same container file as the
      reference object.
   }
   \sstinvocation{
      CALL REF\_PUT( ELOC, LOC, INTERN, STATUS )
   }
   \sstarguments{
      \sstsubsection{
         ELOC = CHARACTER $*$ ( $*$ ) (Given)
      }{
         A locator associated with the reference object.
      }
      \sstsubsection{
         LOC = CHARACTER $*$ ( $*$ ) (Given)
      }{
         A locator associated with the object to be referenced.
      }
      \sstsubsection{
         INTERN = LOGICAL (Given)
      }{
         Whether or not the referenced object is {\tt "}internal{\tt "}.  Set this
         to .TRUE. if the reference is {\tt "}internal{\tt "} and to .FALSE. if it
         is not.
      }
      \sstsubsection{
         STATUS = INTEGER (Given and Returned)
      }{
         Inherited global status.
      }
   }
}
\normalsize
\newpage
\section{ADAM/STAND-ALONE DIFFERENCES}

Note that when using the stand-alone version of the REF library, it is
currently necessary to ensure that HDS\_START is called to activate HDS prior
to making calls to any REF routines. This requirement will be removed in
future, and is currently not required with the ADAM version.

\section{MACHINE-DEPENDENT FEATURES}

The REF library contains no explicit use of machine-dependent features, so its
behaviour should be the same on all platforms on which it is implemented.

However, external references to HDS objects (those not identified as
``internal'' to the REF library) will contain explicit file names, so true
portability of data (in the manner provided by HDS) cannot be expected with REF
when working with operating systems which have different file naming
conventions. If complete data portability is required, then use of the REF
library should be restricted to internal references only.

\section{SOFTWARE DEPENDENCIES}

The REF library explicitly depends on the following other Starlink packages:

\begin{quote}
\begin{tabular}{ll}
HDS & Hierarchical data system (SUN/92)\\
EMS & Error message service (SSN/4)
\end{tabular}
\end{quote}

Note that these packages may also depend on other sub-packages. Please consult
the relevant documentation for details.

\newpage
\section{CHANGES AND NEW FEATURES IN V1.1}

The following describes the most significant changes which have occurred in the
REF library since the previous version (V1.0).

\begin{enumerate}

\item
{\large \bf WARNING:} A work-around has been installed in this version of the
REF library so that it will continue to work after forthcoming changes to HDS
are implemented. All programmers who use the REF library on VMS systems {\large
\bf should re-link their software now} in order to make use of the new
shareable libraries provided in this version. This will protect against future
changes.

Software which continues to use the old (non-shareable) version of the REF
library may not be compatible with future versions of HDS.

\item
The VMS and UNIX implementations of REF have been rationalised and are now
generated from the same source files.

\item
The VMS implementation now makes use of shareable libraries and conforms to the
Starlink subroutine library naming conventions described in SSN/8.

\item
The UNIX implementation of REF has been documented, and the new {\tt ref\_dev}
command has been provided.

\item
The documentation has been revised and now uses a larger font.

\end{enumerate}
\end{document}
