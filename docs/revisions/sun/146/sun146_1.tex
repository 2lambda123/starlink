\documentstyle[11pt]{article} 
\pagestyle{myheadings}

%------------------------------------------------------------------------------
\newcommand{\stardoccategory}  {Starlink User Note}
\newcommand{\stardocinitials}  {SUN}
\newcommand{\stardocnumber}    {146.1}
\newcommand{\stardocauthors}   {Julian Osborne}
\newcommand{\stardocdate}      {3 April 1992}
\newcommand{\stardoctitle}     {OBSERVE --- Check Star Observability}    
%------------------------------------------------------------------------------

\newcommand{\stardocname}{\stardocinitials /\stardocnumber}
\renewcommand{\_}{{\tt\char'137}}     % re-centres the underscore
\markright{\stardocname}
\setlength{\textwidth}{160mm}
\setlength{\textheight}{230mm}
\setlength{\topmargin}{-2mm}
\setlength{\oddsidemargin}{0mm}
\setlength{\evensidemargin}{0mm}
\setlength{\parindent}{0mm}
\setlength{\parskip}{\medskipamount}
\setlength{\unitlength}{1mm}

%------------------------------------------------------------------------------
% Add any \newcommand or \newenvironment commands here
%------------------------------------------------------------------------------

\begin{document}
\thispagestyle{empty}
SCIENCE \& ENGINEERING RESEARCH COUNCIL \hfill \stardocname\\
RUTHERFORD APPLETON LABORATORY\\
{\large\bf Starlink Project\\}
{\large\bf \stardoccategory\ \stardocnumber}
\begin{flushright}
\stardocauthors\\
\stardocdate
\end{flushright}
\vspace{-4mm}
\rule{\textwidth}{0.5mm}
\vspace{5mm}
\begin{center}
{\Large\bf \stardoctitle}
\end{center}
\vspace{5mm}

%------------------------------------------------------------------------------
%  Add this part if you want a table of contents
%  \setlength{\parskip}{0mm}
%  \tableofcontents
%  \setlength{\parskip}{\medskipamount}
%  \markright{\stardocname}
%------------------------------------------------------------------------------


\section{Introduction}

The programme OBSERVE is designed to allow you to get a quick overview of the 
observability of a star through the year from the geographical location 
selected. You will be prompted for the RA \& Dec of the star, the observatory
(you can also specify arbitrary locations), and the year. On the selected GKS 
graphics device a plot will be drawn that tells you, for all dates of the year,
the star rising \& setting times and the times that it is 30$^{\circ}$ above
the  horizon, the times of astronomical twilight, the phase and rising and
setting  times of the Moon as well as its distance from the star in question.

The algorithms used in calculating these times, positions and phases come from 
the book {\em `Practical Astronomy with your calculator'}  by P. Duffett-Smith 
(3$^{\rm rd}$ ed. CUP 1988), and are used with permission. These algorithms are 
not designed for high precision calculations, a fact which is to some extent 
conveyed by the programme output. Nevertheless, I have verified that the 
routines are good for their purpose.

As mentioned above, all standard Starlink graphics devices are supported.
However, the graphical output created by OBSERVE is  sufficiently complicated
and the device drivers have sufficiently different effects that plotting to a
laser printer is almost certainly the best option. If a GKS device such as {\tt
CANL} is chosen, then the following command will print the output file:

\begin{verbatim}
$ PRINT/PASSALL/QUEUE=SYS_LASER CANON.DAT
\end{verbatim}

The programme OBSERVE was originally created by Manfred Gottwald, then of the 
EXOSAT Observatory. I have increased the quantity of information displayed 
and increased the list of standard observatories, and now take responsibility 
for this software.

\section{Description of the Output}

OBSERVE draws one screen of graphics per star. This consists of a header, a 
large plot, and some lines of text explanation. The plot is formatted with 
date through the year shown horizontally and time through the day  shown
vertically. The  left-hand scale shows UT and the right-hand scale shows local
time with midnight at the centre. Figure \ref{fig:sum} shows a summary of the
symbols used on a plot.

\subsection{The start and end of the night}

Wavy horizontal solid lines show the start and end of the  astronomical night.
Symbol {\sf X}  shows the start of the night ({\em i.e.}~the end of
day-to-night astronomical  twilight), and  symbol {\sf V} shows the end of the
night ({\em i.e.}~the  start of night-to-day astronomical twilight). For
extreme  northerly and southerly geographical locations these lines merge to
become a closed shape as there is no astronomical night during polar summer.

\subsection{The rising and setting of the star}

Straight diagonal solid lines marked with symbols  show the times of star
rising ($\uparrow$) and setting ($\downarrow$). Parallel lines show the times
at  which the star rises above ($\circ$) and sets below ($\bullet$)
30$^{\circ}$ elevation, a typical minimum elevation for useful observing.

\subsection{The Moon}

The remaining lines in the plot refer to the Moon. The great-circle 
distance of the Moon 
from the star is shown (in hours) by a dashed sinewave in the top half of
the plot. Read the distance off the right-hand scale
and multiply by 15 to get the distance in degrees. 

The rising and setting  of the Moon is shown by vertical solid lines of
variable width. A line is  plotted for every other day, it starts at the time
of Moonrise and ends at  the time of Moonset. The width of the Moon lines are
related to the phase of  the Moon for that date. The thickest lines are plotted
when the Moon is  75\%-100\% illuminated, thinner lines are plotted for
50\%-75\% and 25\%-50\% illumination. No line is plotted when the fractional
illumination is less than  25\%.

\subsection{How to look at the plot}

The region of visibility of the star typically falls halfway down the plot
(between the wavy lines), and is bounded on left and right by diagonal lines
with empty and filled  circles superimposed. Depending on the nature of the
observation planned, you may then need  to look for  dates and/or times clear
of the vertical Moon lines.

\begin{figure}\caption{Summary of OBSERVE Plot Symbols}
\label{fig:sum}
\begin{center}
\begin{tabular}{|c|l|}
\hline
\large
---{\sf X}--- & start of astronomical night \\
---{\sf V}--- & end of astronomical night \\
---$\uparrow$--- & star rises \\
---$\downarrow$--- & star sets \\
---$\circ$--- & star rises above 30$^o$ elevation\\
---$\bullet$--- & star falls below 30$^o$ elevation\\
- - - - & great circle Moon-star distance \\
\rule{0.1mm}{4mm} & 25\%--50\% illuminated Moon above horizon\\
\rule{0.3mm}{4mm} & 50\%--75\% illuminated Moon above horizon\\
\rule{0.5mm}{4mm} & 75\%--100\% illuminated Moon above horizon\\
\hline
\end{tabular}
\end{center}
\end{figure}

\newpage
\section{An Example}

Suppose that you want to check when the bright star Mira can be observed  from
La Palma.  You should enter the command {\tt OBSERVE} and respond to the
questions as  shown:

\begin{small}
\begin{verbatim}
      $ observe                                                            
      OBSERVE 1.2                                                          
      * Enter observatory                                                  
      1 - Calar Alto/Spain         2 - ESO (La Silla)                      
      3 - La Palma                 4 - Hawaii (Mauna Kea)                  
      5 - AAT                      6 - SAAO                                
      7 - MSSSO                    8 - Kit Peak                            
      9 - McDonald/Texas          10 - Haute-Provence                      
      0 - Other        -----> 3                                            
      enter star name > Mira                                               
      enter star RA (hh mm ss.s) > 02 16 48.                               
      enter star DEC (sdd mm ss.s) > -03 12 0                              
      enter year of observation > 1992                                     
      Graphics device/type (? to see list): ?                              
      PGPLOT v4.9D Copyright 1990 California Institute of Technology   
      MFILE_OUTPUT      TEK_4010          TEK_4014          7800       
      MG100             GRAPHON           ZETA_8            ZETA_8L    
      PRINTRONIX_F      PRINTRONIX_S      CANL              CANP       
      CANON1_L          CANON1_P          CANON2_L          CANON2_P   
      CANON_LTEX        CANON_PTEX        PS_P              PS_L       
      PS_PTEX           PS_LTEX           LJ250_P           LJ250_L    
      IKON              IKON_OVERLAY      IKON_VT           IKON_OVERLAY_V 
      X                 X2                X3                X4             
      Graphics device/type (? to see list):                                
      canl                                                                 
                                                                           
       Visibility graph plotted                                            
                                                                           
      Change:  Source, Observatory & Source, Quit (S/O/Q)? > q             
                                                                           
      $ print/passall/queue=sys_laser canon.dat                           
\end{verbatim}                                                            
\end{small}

From the description of the graph in section 2, you can see in this example 
that Mira is  observable between mid July and the end of February from La
Palma.  The longest observations  are possible between approximately October 10
and late November, because  between these dates daylight does not intrude when
the star is above 30$^{\circ}$ elevation. Good dates for getting a long
observation would be late October and late November, because the Moon is  close
to new and is maximally distant from your star. For example, around  October 26
1992 Mira would be visible from La Palma between 19:50 UT and 06:00 UT. The
plot shows that UT is the same as local time at La Palma.

A hardcopy print of the plot produced by the example is appended.

% To get the hardcopy for appending to this SUN, follow the example contained
% herein, allowing for use of either Canon or PostScript printers. Print
% the result, and you have a copy of the last page of this sun. Neither a
% PostScript or Canon data file for the image are stored on-line.

\end{document}
