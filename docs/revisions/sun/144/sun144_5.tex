\documentstyle[11pt]{article} 
\pagestyle{myheadings}

%------------------------------------------------------------------------------
\newcommand{\stardoccategory}  {Starlink User Note}
\newcommand{\stardocinitials}  {SUN}
\newcommand{\stardocnumber}    {144.5}
\newcommand{\stardocauthors}   {A J Chipperfield}
\newcommand{\stardocdate}      {18 February 1994}
\newcommand{\stardoctitle}     {ADAM --- Unix Version}
%------------------------------------------------------------------------------

\newcommand{\stardocname}{\stardocinitials /\stardocnumber}
\renewcommand{\_}{{\tt\char'137}}     % re-centres the underscore
\markright{\stardocname}
\setlength{\textwidth}{160mm}
\setlength{\textheight}{230mm}
\setlength{\topmargin}{-2mm}
\setlength{\oddsidemargin}{0mm}
\setlength{\evensidemargin}{0mm}
\setlength{\parindent}{0mm}
\setlength{\parskip}{\medskipamount}
\setlength{\unitlength}{1mm}

%------------------------------------------------------------------------------
% Add any \newcommand or \newenvironment commands here
%------------------------------------------------------------------------------

\begin{document}
\thispagestyle{empty}
SCIENCE \& ENGINEERING RESEARCH COUNCIL \hfill \stardocname\\
RUTHERFORD APPLETON LABORATORY\\
{\large\bf Starlink Project\\}
{\large\bf \stardoccategory\ \stardocnumber}
\begin{flushright}
\stardocauthors\\
\stardocdate
\end{flushright}
\vspace{-4mm}
\rule{\textwidth}{0.5mm}
\vspace{5mm}
\begin{center}
{\Large\bf \stardoctitle}
\end{center}
\vspace{5mm}

%------------------------------------------------------------------------------

%  Package Description
\begin{center}
{\Large\bf Description}
\end{center}

This document describes the use of ADAM on Unix and points to some 
differences between the Unix and VMS systems.

It is assumed that the reader is familiar with the concepts of ADAM 
programming and that the Starlink software is installed in the standard way.

%------------------------------------------------------------------------------
%\vspace{2cm}
%------------------------------------------------------------------------------
%  Add this part if you want a table of contents
  \setlength{\parskip}{0mm}
  \tableofcontents
  \setlength{\parskip}{\medskipamount}
  \markright{\stardocname}
%------------------------------------------------------------------------------

\newpage
\section{Introduction}
\label{intro}
ADAM analysis application programs (A-tasks) or instrumentation application
programs (I-tasks) may be developed and run on the supported Unix platforms. 
Single A-tasks may also be combined into A-task monoliths for efficiency.

Usually only minor changes to code  are required to port ADAM tasks from VMS
to Unix and the ported code will continue to work for VMS.
As far as possible, programs will behave the same under Unix as they did with
VMS.

A list of differences and deficiencies is given in Appendix~\ref{diffs}.

For details of the concepts of ADAM programming, see Starlink Guide 4, 
"ADAM -- The Starlink Software Environment" for A-tasks and Starlink User Note
134, "ADAM -- Guide to Writing Instrumentation Tasks" for I-tasks.

\section{Setting Up}
To prepare for basic use of ADAM under Unix it is simply necessary to create 
the directory {\bf adam} in the directory specified by the environment 
variable {\bf HOME}. 
ADAM programs may then be run. 
There is no equivalent of the {\bf ADAMSTART}
command of VMS ADAM.

\section{Compiling and Linking}
\subsection{Include Files}
\label{incs}
All public include files, such as {\bf {\em pkg}\_par} and
{\bf {\em pkg}\_err} files, will be found in {\bf /star/include}.
The filenames are in lower case but they should be specified in the program 
in the standard Starlink way. That is, for example:
\begin{quote} \begin{verbatim}
INCLUDE 'SAE_PAR'
INCLUDE 'PAR_ERR'
\end{verbatim} \end{quote}
Links are then set up to the public include files in /star/include using the
appropriate {\em pkg}\_dev script. (SAE\_PAR is defined by {\bf star\_dev}.)

{\em Note that {\bf sae\_par} on Unix does not include {\bf par\_par} or 
{\bf dat\_par} therefore they must be added to any code where they are 
required.}

To compile and link ADAM programs, it is necessary to add {\bf /star/bin} to 
your PATH environment variable. This is done if you have `sourced'
{\bf /star/etc/login} to set up for Starlink software generally. 
There is no equivalent to the {\bf ADAM\_DEV} command of VMS ADAM.

\subsection{The alink and ilink Scripts}
\label{link_scripts}
Two shell scripts are provided to link ADAM tasks.
{\bf alink} is used to link A-tasks and A-task monoliths, and {\bf ilink} to 
link I-tasks.
The executable images produced for A-tasks and monoliths may be run either
from ICL or directly from the shell. 

The following description uses {\bf alink} in examples; {\bf ilink} is used in
the same way.

To link an ADAM program, ensure that {\bf /star/bin} is added to your 
environment variable PATH, then type:
\begin{quote} 
\verb!% alink! [\verb!-xdbx!] {\em file} [{\em arguments}]
\end{quote}
Where:
\begin{description}
\item[\verb!-xdbx!] optional (but must be the first argument if used), 
overcomes some problems with debugging ADAM tasks, notably with {\bf xdbx} 
and {\bf ups} on SunOS, by supplying dummy source files for required ADAM 
system routines. It also has the effect of inserting a \verb!-g! option into 
the argument list. For more details, see Appendix \ref{link_det}.
\item[{\em file}] specifies a filename of the form 
{\em path}/{\em name}{\bf .f} or {\em path}/{\em name}{\bf .o} or 
{\em path}/{\em name}. The {\em path}/ component is optional but
{\em name} must be the name of the program's main subroutine and will
be the name of the executable file produced in the current working directory.
If the filename ends in {\bf .f}, the specified file will be compiled; if it
ends neither in {\bf .f} nor {\bf .o}, {\bf .o} is appended.

Note that after a {\bf .f} file has been compiled, on some platforms the 
{\bf .o} file will be retained in the current working directory so that 
subsequent {\bf alink}s with an unchanged {\bf .f} file may specify the name
with no extension ( or {\bf .o}).
\item[{\em arguments}] optional, is any additional arguments (library 
specifications, compiler options {\em etc.}) legal for the Fortran 
compiler.
The list of Starlink libraries automatically included in the link is given in 
Appendix~\ref{libs}; the method of including other Starlink libraries in ADAM
programs will be specified in the relevant Starlink User Note.

In most cases it will be:
\begin{quote}
\verb!% alink! {\em file} \verb!`!{\em pkg}\verb!_link_adam`!
\end{quote}
where {\em pkg} is the specific software item name {\em e.g.} {\bf ndf}.
\end{description}

Appendix~\ref{link_det} gives details of the command used within 
{\bf alink/ilink} to compile and/or link the tasks. 
This may assist users who wish to alter the standard behaviour.

\subsection{Interface Files}
Generally, the only things which will need changing in interface files are
file and device names. Case is significant on Unix and obviously the format of 
filenames is different.

Interface files may be compiled by the {\bf compifl} program which is available
in {\bf /star/bin}. Compiled interface files ({\bf .ifc}s) must be produced
on the platform on which they are to be used.

\subsection{Monoliths}
\label{monoliths}
The top-level routine for Unix A-task monoliths differs slightly from the 
old-style VMS monoliths but the new style may be used for monoliths to be run 
from ICL on Unix or VMS, or from the Unix shell.

New-style monoliths are linked using {\bf alink}.

The top-level routine will be of the form:
\begin{small}
\begin{verbatim}
          SUBROUTINE TEST( STATUS )
          INCLUDE 'SAE_PAR'
          INCLUDE 'PAR_PAR'

          INTEGER STATUS

          CHARACTER*(PAR__SZNAM) NAME

          IF (STATUS.NE.SAI__OK) RETURN

*       Get the action name
          CALL TASK_GET_NAME( NAME, STATUS )

*       Call the appropriate action routine
          IF (NAME.EQ.'TEST1') THEN
            CALL TEST1(STATUS)
          ELSE IF (NAME.EQ.'TEST2') THEN
            CALL TEST2(STATUS)
          ELSE IF (NAME.EQ.'TEST3') THEN
            CALL TEST3(STATUS)
          END IF
          END
\end{verbatim}
\end{small}

To run such a monolith from a Unix shell, link the required action name to the
monolith, then execute the linkname (possibly via an alias). For example:

\begin{quote} \begin{verbatim}
% ln -s $KAPPA_DIR/kappa add
% add
\end{verbatim} \end{quote}

Separate interface files are required for each action run from the shell --
a monolithic interface file is required for monoliths run from ICL.

\section{Help Files}
\label{hlp}
The Starlink portable help system is used to provide help if ? or ?? is typed 
in response to a parameter prompt.
Instructions for creating help libraries and navigating through them may be 
found in SUN/124.

Interface file entries specifying help libraries may be given as standard
pathnames {\em e.g.}
\begin{quote} 
\begin{verbatim}
helplib /star/help/kappa/kappa.shl
\end{verbatim}
or
\begin{verbatim}
help %/star/help/kappa/kappa ADD Parameters IN1
\end{verbatim} 
\end{quote}
Note that the file extension {\bf .shl} is optional.

The system will also accept environment variables and \verb!~! in the help 
library name, {\em e.g.}
\begin{quote} 
\verb!$KAPPA_DIR/kappahelp.shl ! or \verb! $KAPPA_HELP ! or
\verb! ~/help/myprog!
\end{quote}

As an interim measure, so that the same interface file entry will work with
both Unix and VMS, the VMS form:
\begin{quote} 
\verb!KAPPA_DIR:kappahelp.shl ! or \verb! KAPPA_HELP:!
\end{quote}
will be accepted.

If the VMS form is used, the environment variable name will be forced to upper
case, and the filename to lower case for interpretation. 
Unix-style specifications will be interpreted as given.

\section{Use of Environment Variables}
\label{envars}
For more complex operations, the following environment variables may be used:
\begin{description}
\item[HOME] Is expected to specify the user's home directory
\item[ADAM\_USER] If the user does not wish to use directory {\bf \$HOME/adam}
to hold the program's parameter file and the global parameter file,
{\bf ADAM\_USER} must be set to define the desired directory.
Whichever directory is used, it must exist but the files will be created if 
necessary.
The global parameter file is named GLOBAL.sdf.
\item[ADAM\_IFL] Specifies a search path of directories in which the system is 
to look for interface files. 
If the variable is undefined, or the search is unsuccessful, the directory in
which the executable was found is assumed.
A file with the same name as the executable and with extension {\bf .ifc} or 
{\bf .ifl} (lower case) is sought.
As with the VMS system, {\bf .ifc} will be used in preference to {\bf .ifl}.
\item[PATH] In addition to its use by the system to find the required
executable file, the environment variable {\bf PATH} is used by the parameter
system to find the pathname of the file being executed if it was invoked by 
simply typing its name (not its pathname).
This is needed to discover the directory in which to look for the interface 
file if the ADAM\_IFL search is unsuccessful.
This process means that the use of links may cause confusion -- the name and
directory of the link will be used.
\item[HDS\_SHELL] The interpretation of names given as values for parameters
accessed via PAR or DAT routines will be handled by HDS.
The environment variable HDS\_SHELL (see SUN/92) will be effective.
If it is not set when the application starts, interpretation with SHELL=2
({\bf tcsh} or failing that {\bf csh} or failing that {\bf sh}) is selected 
-- thus environment variables and `\verb!~!' are usually expanded.
Note that parameter system syntax will usually prevent the use of more general 
shell expressions as names.
\item[ICL Environment Variables] See Appendix \ref{iclvars} for details of the
environment variables used by ICL.
\end{description}

\newpage
\appendix
\section{Example Session}
The following session shows the process of compiling, linking and running an
example program, derived from SUN/101, on the Sun.

\begin{quote} \begin{verbatim}
% source /star/etc/login
% mkdir ~/adam
%
% ls
repdim2.f       repdim2.ifl     example.sdf
%
% cat repdim2.f
*   Program to report the dimensions of an NDF.
*   The CHR package is used to produce a nice output message.
*   See SUN/101, section 11.

      SUBROUTINE REPDIM2 (STATUS)                                
      IMPLICIT NONE
      INCLUDE 'SAE_PAR'                                       
      INTEGER DIM(10), I, NCHAR, NDF1, NDIM, STATUS
      CHARACTER*100 STRING

*   Check inherited global status.
      IF (STATUS.NE.SAI__OK) RETURN

*   Begin an NDF context.    
      CALL NDF_BEGIN                                          
                                                              
*   Get the name of the input NDF file and associate an NDF 
*   identifier with it.
      CALL NDF_ASSOC ('INPUT', 'READ', NDF1, STATUS)
                                                              
*   Enquire the dimension sizes of the NDF.
      CALL NDF_DIM (NDF1, 10, DIM, NDIM, STATUS)

*   Set the token 'NDIM' with the value NDIM.
      CALL MSG_SETI ('NDIM', NDIM)

*   Report the message.
      CALL MSG_OUT (' ', 'No. of dimensions is ^NDIM', STATUS) 

*   Report the dimensions.
      NCHAR = 0
      CALL CHR_PUTC ('Array dimensions are ', STRING, NCHAR)
      DO  I = 1, NDIM
*      Add a `x' between the dimensions if there are more than one.
         IF (I.GT.1) CALL CHR_PUTC (' x ', STRING, NCHAR)
*      Add the next dimension to the string.
         CALL CHR_PUTI (DIM(I), STRING, NCHAR)
      ENDDO
      CALL MSG_OUT ('  ', STRING(1:NCHAR), STATUS) 

*   End the NDF context.                                       
      CALL NDF_END (STATUS)                                  
      END                 
%
% cat repdim2.ifl
interface REPDIM2
   parameter      INPUT
      position    1 
      prompt      'Input NDF structure'
      default     example
      association '->global.ndf'
   endparameter
endinterface
%
% star_dev
% alink repdim2.f `ndf_link_adam`
f77 -g -c repdim2.f
repdim2.f:
        repdim2:
dtask_applic.f:
        dtask_applic:
%
% repdim2
INPUT - Input NDF structure /@EXAMPLE/ >
No. of dimensions is 1
Array dimensions are 856
%
% ls ~/adam
repdim2.sdf     GLOBAL.sdf
%  
% compifl repdim2
!! COMPIFL: Successful completion
%
% ls
example.sdf   repdim2.f     repdim2.ifl
repdim2       repdim2.ifc   repdim2.o
%

\end{verbatim} \end{quote}
Note that whether or not the file repdim2.o is retained depends upon the
compiler used.

\newpage
\section{Link Script Details}
\label{link_det}
The link scripts firstly have to create a subroutine, DTASK\_APPLIC, which is
which is called by the ADAM fixed part and in turn calls the user's top-level
routine.
The difference between {\bf alink} and {\bf ilink} is just that the template
DTASK\_APPLIC for {\bf alink} contains a call to close down the parameter
system after each invocation of the task.

During installation, the actual compile/link command within {\bf alink/ilink} 
is edited depending upon the platform. The template command is:
\begin{quote} \begin{verbatim}
F77 $FFLAGS -o $PROGNAME \
$XDBX \
/star/lib/dtask_main.o \
dtask_applic.f \
-Bstatic \
$ARGS \
-L/star/lib \
-lhdspar_adam \
-lpar_adam \
`dtask_link_adam` \
-Bdynamic
\end{verbatim} \end{quote}
Notes:
\begin{itemize}
\item F77 is replaced by whatever the FC environment variable is when 
{\bf alink/ilink} is installed.
\item Environment variable FFLAGS may be used to specify any options which
must be included at the start of the command line.
\item \verb!$XDBX! is set to \verb!-g! if the {bf -xdbx} argument is given.
\item The \verb%-Bstatic% and \verb%-Bdynamic% options are retained, removed or
modified, depending upon the platform, during installation to ensure static 
linking of the Starlink libraries but not the system libraries.
\item \verb%$PROGNAME% is the basename of the first argument of 
{\bf alink/ilink} with any {\bf .f} or {\bf .o} suffix removed.
\item \verb%$ARGS% is the first argument of {\bf alink/ilink}, with {\bf .o}
appended if the extension was not {\bf .f} or {\bf .o}, followed by the 
remaining arguments unchanged.
\item To speed up the link, {\em pkg}\_link\_adam scripts are only used
selectively. 
{\bf dtask\_link\_adam} refers to dtask, task and err libraries directly then
invokes {\bf subpar\_link\_adam} which references the necessary
libraries directly, apart from hds, hlp and psx whose link\_adam scripts are
invoked.
\item Other adjustments are made during installation if the system is not being
installed in /star. In particular, a -L option is added to include the newly
installed libraries before those in /star/lib.
\end{itemize}
The {\bf -xdbx} argument is provided to overcome some awkward problems which
can arise when debugging ADAM applications. Usually it is sufficient to
include \verb!-g! in the arguments of {\bf alink/ilink} but sometimes, notably 
when using {\bf xdbx} and {\bf ups} on SunOS, the debuggers do not behave
sensibly if required source files are missing so the \verb!-xdbx! argument
should be used instead.
The effects of the argument are:
\begin{itemize}
\item A dummy source file for {\bf dtask\_main}, the main routine of every ADAM
task, is created in the working directory. 
The file contains an explanatory message to the user and the name of the 
user's top-level subroutine (which may be helpful in selecting a breakpoint).
\item The source file of DTASK\_APPLIC is not deleted from the working 
directory.
\item {\verb!-g!} is inserted in the compile/link command.
\end{itemize}
\newpage
\section{Available Libraries}
\label{libs}
\subsection*{ADAM System Libraries}
\begin{verse}
HDSPAR (DAT\_ASSOC {\em etc.}). This library is named DATPAR in the VMS
release \\
SUBPAR \\
PARSECON \\
STRING \\
LEX \\
DTASK \\
TASK \\
MISC (Miscellaneous routines required for Unix until further developments) \\
ADAM \\
MESSYS \\
MSP 
\end{verse}

\subsection*{Starlink Libraries Searched Automatically}
The following separate Starlink libraries will be searched automatically by
the ADAM link. The libraries used for ADAM may differ from the standalone
versions (see relevant documents for details).

\begin{verse}
PAR \\
ERR/MSG/EMS \\
HDS \\
CHR \\
PSX \\
HLP
\end{verse}

\subsection*{Libraries Not Searched Automatically}
\label{nolibs}
The following libraries, which are searched automatically by {\bf alink} on
VMS, must be included as optional arguments on Unix.

\begin{verse}
AGI \\
ARY \\
CNF \\
GKS (includes GKSPAR) \\
GNS \\
IDI \\
NDF \\
SGS (includes SGSPAR) \\
PGPLOT (includes PGPPAR) \\
\end{verse}

\section{ICL for Unix}
\label{icl}
\subsection{ICL Startup}
ICL is started from a Unix shell by a command of the form:
\begin{quote} 
\verb!% icl! [{\em ICL\_options}\/] [{\em command\_filenames...}]
\end{quote}
Where:
\begin{description}
\item[{\em ICL\_options}] (optional) have not yet been fully developed 
and should not be used without advice. ICL options start
with `--' and must appear before any command-file names.
\item[{\em command\_filenames...}] (optional) are the names of any files 
containing ICL commands which are to be obeyed by ICL before the ICL prompt 
appears.
They are loaded {\em after} any ICL login files (see Section \ref{iclvars}).
A file extension of \verb!.icl! is assumed if no extension is specified. 
All filenames must appear after any ICL options.
\end{description}
Example:
\begin{quote} \begin{verbatim}
% icl -io test_io comfile1 comfile2
\end{verbatim} \end{quote}
which would set the value of the ICL option \verb!-io! to \verb!test_io!
and the ICL command files \verb!comfile1.icl! and \verb!comfile2.icl! would be 
automatically loaded into ICL before the ICL prompt was output.

\subsection{ICL Environment Variables}
\label{iclvars}
ICL's operation can be controlled by several (optional) environment
variables. The variables are:
\begin{description}
\item[ICL\_LOGIN\_SYS, ICL\_LOGIN\_LOCAL and ICL\_LOGIN\_USER]
If set, these environment variables specify ICL command files to be obeyed, 
in the above order, by ICL before the ICL prompt appears.
A default file extension of {\bf .icl} is assumed. For example:
\begin{quote} \begin{verbatim}
% setenv ICL_LOGIN ~/myprocs
\end{verbatim} \end{quote}
will cause file \verb!myprocs.icl! in the user's home directory to be loaded
as ICL starts up.
\item[EDITOR] If set, this will override the ICL default editor (vi). 
For example:
\begin{quote} \begin{verbatim}
% setenv EDITOR tpu
\end{verbatim} \end{quote}
\item[ICL\_HELPFILE] If set, this will override 
\verb!/star/help/icl/iclhelpfile! as the default ICL helpfile.
\end{description}

\section{Summary of Differences and Deficiencies}
\label{diffs}
\begin{itemize}
\item There is no {\bf ADAMSTART} command. Directory {\bf \$HOME/adam} must be
created explicitly unless environment variable {\bf ADAM\_USER} is used to point
elsewhere. The global parameter file {\bf GLOBAL.sdf} will be created 
automatically if required.
\item {\bf DAT\_PAR} must be INCLUDEd explicitly in any code which requires 
the symbolic constants for HDS. Similarly {\bf PAR\_PAR} for the PAR constants.
(see Section \ref{incs}).
\item Some libraries ({\em e.g.}\ NDF) which are automatically included by 
{\bf alink} on VMS are not included on Unix (see Appendix \ref{libs}).
\item Compiled interface files ({\bf .ifc}s) will only work for the platform
type on which they were created.
\item Monolith top-level routines must be written in the new style for use on
Unix. Monoliths are linked using {\bf alink}.
\item ICL shareable monoliths will not be available on Unix.
\item Editing of prompt suggested values, using the TAB key, is
not available.
\item When running tasks directly from a shell, the normal Unix rules for the 
use of metacharacters on the command line will apply.
Characters to be particularly wary of are `\verb%"%', `\verb%\%' and 
`\verb%$%'.
\item ADAM tasks will usually default to interpreting environment variables
and `\verb!~!' in filenames used with HDS. This is not necessarily true of 
names used with other ADAM facilities such as FIO and may be overridden by 
user or programmer action for HDS filenames.
\end{itemize}

\section{Document Changes}
This document has been updated to reflect:
\begin{itemize}
\item The availability of ICL for Unix. (Appendix \ref{icl}.)
\item The use of HDS to handle expansion of filename for PAR and DAT.
(Section \ref{envars} and Appendix \ref{diffs}.)
\item The {\bf -xdbx} argument for {\bf alink/ilink}.
(Section \ref{link_scripts} and Appendix \ref{link_det}.)
\end{itemize}
\end{document}
                                         
