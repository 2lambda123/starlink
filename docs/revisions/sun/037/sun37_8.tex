\documentstyle[11pt]{article} 
\pagestyle{myheadings}

%------------------------------------------------------------------------------
\newcommand{\stardoccategory}  {Starlink User Note}
\newcommand{\stardocinitials}  {SUN}
\newcommand{\stardocnumber}    {37.8}
\newcommand{\stardocauthors}   {Jack Giddings, Paul Rees \& Dave Mills}
\newcommand{\stardocdate}      {6 October 1993}
\newcommand{\stardoctitle}     {IUEDR --- IUE Data Reduction package (3.0)}
%------------------------------------------------------------------------------

\newcommand{\stardocname}{\stardocinitials /\stardocnumber}
\markright{\stardocname}
\setlength{\textwidth}{160mm}
\setlength{\textheight}{230mm}
\setlength{\topmargin}{-2mm}
\setlength{\oddsidemargin}{0mm}
\setlength{\evensidemargin}{0mm}
\setlength{\parindent}{0mm}
\setlength{\parskip}{\medskipamount}
\setlength{\unitlength}{1mm}

\begin{document}
\thispagestyle{empty}
SCIENCE \& ENGINEERING RESEARCH COUNCIL \hfill \stardocname\\
RUTHERFORD APPLETON LABORATORY\\
{\large\bf Starlink Project\\}
{\large\bf \stardoccategory\ \stardocnumber}
\begin{flushright}
\stardocauthors\\
\stardocdate
\end{flushright}
\vspace{-4mm}
\rule{\textwidth}{0.5mm}
\vspace{5mm}
\begin{center}
{\Large\bf \stardoctitle}
\end{center}
\vspace{5mm}

%------------------------------------------------------------------------------
%  Add this part if you want a table of contents
\setlength{\parskip}{0mm}
\tableofcontents
\setlength{\parskip}{\medskipamount}
\markright{\stardocname}
%------------------------------------------------------------------------------

\newpage
\section {Introduction}

IUEDR is a program that provides facilities for the reduction of IUE
data. It addresses the problem of working from the IUE Guest Observer
Tape through to a calibrated spectrum that can be used in scientific
analysis. In this respect, it aims to be a ``complete'' system for IUE
data reduction.

\section {Summary of Facilities}

Here is a brief summary of what can be done using IUEDR. It should
help you to decide whether it can help with your IUE work.

\begin {itemize}

\item TAPE ANALYSIS --- The contents of IUE tapes can be examined
interactively to find what images are present, and so to plan the data
reduction.

\item READING IUE IMAGES --- RAW, GPHOT and PHOT images can be read
from IUE tapes (or disk files in  GO format) into IUEDR files stored
on disk. These files are provided with default calibrations.

\item SPECTRUM EXTRACTION --- This uses techniques that are an
enhancement of those present in the TRAK program (Giddings, 1982).
Spectra taken using either resolution mode (HIRES or LORES) can be
extracted from GPHOT or PHOT images (the latter being the more recent
style photometric images that retain geometric distortion). It is
possible to correct photometric LORES images obtained with the SWP
camera for defects in the original ITF calibration. 

\item LBLS --- Create a line-by-line-spectrum, corresponding to the
IUESIPS product.

\item SPECTRUM CALIBRATION --- Fully calibrated spectra can be
produced. This includes various forms of wavelength corrections,
absolute flux calibration, and echelle ripple correction for HIRES
spectra. There is also a semi-empirical correction for the HIRES
(order-overlap) background problem. 

\item GRAPHICAL DISPLAY --- Graphical display facilities are provided
to aid spectrum extraction and calibration operations. Any GKS
workstation type can be used (see SUN/83 for details of GKS on 
Starlink).

\item IMAGE DISPLAY --- The image can be displayed on any available
GKS workstation which has colour image display capability. Various
cursor operations can be performed including the marking of bad
pixels, image modification and feature identification.

\item READING IUESIPS EXTRACTED SPECTRA --- Extracted spectra from
IUESIPS \#1 and \#2, both at low and high resolution (MELO and MEHI)
can also be read from tape (or disk files in GO format).

\item SPECTRUM AVERAGING --- It is possible to combine the spectra
from groups of echelle orders (HIRES) or from different apertures
(LORES) by mapping and averaging them onto an evenly spaced wavelength
grid.

\item OUTPUT PRODUCTS --- Both individual extracted spectra
(orders/apertures) and combined spectra can be output to files which
may be read by DIPSO (SUN/50). The format of DIPSO type SP0 files has
changed to use the STARLINK NDF facilities. This allows the use of all
standard STARLINK packages to  analyse the output products. Files can
also be created in the same format as the TRAK program.

\end {itemize}

The user interface takes the form of a ``dialogue'' consisting of 
prompts supplied by the program to ask for the information it needs
and  commands typed  at the terminal. Batch mode operation is also
provided via the use of ICL command procedures.

\section {Getting Started}

\subsection{VMS machines}

In order to use IUEDR your {\tt LOGIN.COM} file, which is executed each time
you login to the VAX, should contain the following lines:

\begin{verbatim}
      $ @SSC:LOGIN       ! Make Starlink Software available
      $ IUEDRSTART       ! Make IUEDR available}
\end{verbatim}

The lines must begin with a ``\$'' character when they are in a
``{\tt .COM}'' file. The first line is mandatory if you wish to use Starlink
Software. The IUEDRSTART command defines a number of Symbols and
Logical Names that are needed to run IUEDR.

The next thing is to determine whether you have enough ``Page File
Quota''. This determines the maximum size of program you are allowed
to run. IUEDR is quite large, and you will need a Quota of around
30000 pages. You can find out your current quotas using the DCL
command:

\begin{verbatim}
      $ SHOW PROCESS/ALL
\end{verbatim}

If you do not have enough Page File Quota, see your Site Manager about
an increase.

You are now ready to start the interactive command interpreter (ICL). 
Type:

\begin{verbatim}
      $ ICL
\end{verbatim}

The `\verb+ICL>+' prompt will appear. The final step is to load the IUEDR
package:

\begin{verbatim}
      ICL> LOAD IUEDR_DIR:IUEDR3
\end{verbatim}

Once these preliminaries are out of the way, all the IUEDR commands
are available, as well as all the facilities of ICL and any other ADAM
packages you have loaded.

IUEDR will not actually start until you use one its commands. Then it
will begin by printing some information and then prompt for any
command parameters

To remove the IUEDR package you can type  KILL IUEDR at the ICL
prompt,  or simply EXIT to leave ICL completely.

\subsection{UNIX machines}

The ICL command interpreter is not yet available for UNIX machines.
When it becomes available the method of starting IUEDR will be very
similar to that decribed above for VMS machines.

Currently however, you setup for IUEDR use by typing:

\begin{verbatim}
      % iuedrstart
\end{verbatim}
at the system shell prompt {\tt \%}.

This sets up all the IUEDR commands and any necessary environment
variables. The IUEDR command may then by used in two ways:

\begin{itemize}
\item {Type the command at the shell prompt}
\item {Type {\tt iuedr} at the shell prompt}
\end{itemize}

The first mechanism is only useful when you want to use a single IUEDR
command ({\em e.g.} LISTIUE).

The second mechanism runs the IUEDR monolith program which consists of
a simple loop which accepts IUEDR commands until EXIT is typed.

\section {Documentation} 

Documentation is stored in a mixture of text and \LaTeX\ files. The
\LaTeX\ files are STARLINK documents and your site manager should be
able to provide paper copies.

The text files may be found in {\tt IUEDR\_DOC:} ({\tt \$IUEDR\_DOC}
on UNIX machines).

The primary document that you should read is the IUEDR User Guide
which provides an introduction to IUEDR, along with much practical
advice. A printed copy of this should be available at your Site. If it
isn't, or if you are a remote user, it can be printed by:

\begin{verbatim}
      $ PRINT IUEDR_DOC:GUIDE.DOC
\end{verbatim}

NB: the User Guide applies to version 1.1. Since the User Guide is
rather LARGE, please don't print personal copies unless it is really
necessary. You are also strongly  advised to consult the release
notes, in particular {\tt IUEDR\_DOC:VER20.DOC} and {\tt VER30.DOC},
for details of changes and extensions made  in later versions of
IUEDR, {\em e.g.}

\begin{verbatim}
      $ PRINT IUEDR\_DOC:VER12.DOC
\end{verbatim}

or, to view the text at the terminal,

\begin{verbatim}
      $ TYPE IUEDR\_DOC:VER12.DOC
\end{verbatim}

The USER GUIDE describes other IUEDR documents including

\begin {itemize}

\item IUEDR REFERENCE MANUAL (SG/3)--- A formal description of each
command and its parameters can be found in Starlink Guide 3 (SG/3).
Your site manager should be able to provide a copy.

\item IUEDR PROBLEMS --- A list of the main problems (deficiencies)
associated with existing IUEDR facilities. The problems are described
in file {\tt IUEDR\_DOC:PROBLEMS.DOC}

\item IUEDR VERSIONS --- A chronological list of changes made to IUEDR
and its documentation. The different versions are described in the
following files:

\begin {itemize}
\item {\tt IUEDR\_DOC:VER10.DOC} --- Changes made between pre-release and 1.0
\item {\tt IUEDR\_DOC:VER11.DOC} --- Changes made between 1.0 and 1.1
\item {\tt IUEDR\_DOC:VER12.DOC} --- Changes made between 1.1 and 1.2
\item {\tt IUEDR\_DOC:VER13.DOC} --- Changes made between 1.2 and 1.3
\item {\tt IUEDR\_DOC:VER14.DOC} --- Changes made between 1.3 and 1.4
\item {\tt IUEDR\_DOC:VER20.DOC} --- Changes made between 1.4 and 2.0
\item {\tt IUEDR\_DOC:VER30.DOC} --- Changes made between 2.0 and 3.0
\end {itemize}

The files {\tt VER12.DOC} and {\tt VER13.DOC} contain information about new
features which  are NOT described in the User Guide, but the Manual
and HELP are updated.

\end {itemize}

\section {Support}

IUEDR was written at UCL by Dr Jack Giddings, however Dave Mills at
UCL is now  responsible for supporting IUEDR.  The available
documentation should provide sufficient information for you to  use
IUEDR effectively. However, if you find something difficult to
understand,  please ask for help. There are many experienced users at
UCL, and if Dave cannot answer your enquiry himself, he can consult
with people with  considerable experience at UCL.

If the program appears to produce incorrect results, then it is worth
reading  through the IUEDR documentation carefully before reporting a
``bug''. However, if the program fails (crashes), then this should be
reported immediately. Your report should include the names of any
files that you were using at the time of the error, and the log file
produced during the session ({\tt SESSION.LIS}). Keep the  data and
the log file  somewhere safe and give their whereabouts in the error
report. Then they can be inspected remotely over the Network if
necessary. It is very helpful in tracing programming errors to have as
much information to hand as possible, and you are urged to give as
full an account of the error as you can. 

The preferred mechanism for correspondence is the EMAIL system.
Telephone enquiries can be very hard to understand, and are often
difficult to arrange at UCL. Please send enquiries and bug reports to
UK.AC.UCL.STAR::IUEDR. Your enquiry could help to provide improved
documentation which would benefit other Starlink users.

\end {document}

