\documentstyle[11pt]{article} 
\pagestyle{myheadings}

%------------------------------------------------------------------------------
\newcommand{\stardoccategory}  {Starlink User Note}
\newcommand{\stardocinitials}  {SUN}
\newcommand{\stardocnumber}    {120.2}
\newcommand{\stardocauthors}   {A R Wood}
\newcommand{\stardocdate}      {18 August 1992}
\newcommand{\stardoctitle}     {CATPAC --- Catalogue Applications Package}
%------------------------------------------------------------------------------

\newcommand{\stardocname}{\stardocinitials /\stardocnumber}
\markright{\stardocname}
\setlength{\textwidth}{160mm}
\setlength{\textheight}{230mm}
\setlength{\topmargin}{-2mm}
\setlength{\oddsidemargin}{0mm}
\setlength{\evensidemargin}{0mm}
\setlength{\parindent}{0mm}
\setlength{\parskip}{\medskipamount}
\setlength{\unitlength}{1mm}

\begin{document}
\thispagestyle{empty}
SCIENCE \& ENGINEERING RESEARCH COUNCIL \hfill \stardocname\\
RUTHERFORD APPLETON LABORATORY\\
{\large\bf Starlink Project\\}
{\large\bf \stardoccategory\ \stardocnumber}
\begin{flushright}
\stardocauthors\\
\stardocdate
\end{flushright}
\vspace{-4mm}
\rule{\textwidth}{0.5mm}
\vspace{5mm}
\begin{center}
{\Large\bf \stardoctitle}
\end{center}
\vspace{5mm}

\setlength{\parskip}{0mm}
\tableofcontents
\setlength{\parskip}{\medskipamount}
\markright{\stardocname}

\newpage
\section {Introduction}

CATPAC is the STARLINK Catalogue and Table Package. This document describes the
CATPAC applications. These  include applications for inputing, processing and
reporting tabular data including astronomical catalogues. The applications all
fall into  the following categories:

\begin{itemize}
\item Reporting catalogues.
\item Creating catalogues.
\item Deleting catalogues.
\item Manipulating information about catalogues.
\item Manipulating data in catalogues.
\end{itemize}

Appendix~\ref{ap:summary} lists, in alphabetical order, the commands and 
their functions.

\section {Catalogue terms and ideas}

\subsection {Parameter and field information}

A catalogue has information that applies to the  whole catalogue. For an
astronomical catalogue this may well include items such as:
 
\begin{description}
\begin{description}
\item [CATTITLE] --- Title of the catalogue.
\item [IDENTIFIER] --- Name by which the catalogue is known to the system.
\item [NUMENTS] --- Number of entries in the catalogue.
\item [AUTHOR] --- Authors name.
\item [WAVLENGTH] --- Catalogue wavelength keyword for librarian.
\item [OBJECT] --- Type of objects in catalogue STAR, GALAXY {\em etc}.
\item [EPOCH] --- Date of position measurements.
\item [EQUINOX] --- Equinox of coordinates.
\item [NOTES] --- Notes on the catalogue.
\end{description}
\end{description}

Traditionally these information items have been known as {\bf PARAMETERS} of 
the catalogue. These parameters should not be confused with the {\small ADAM}
parameter system.

Each parameter requires:

\begin{description}
\begin{description}
\item [NAME] --- Name identifying this parameter.
\item [FORMAT] --- The format/type of the value.
\item [VALUE] --- Value of the parameter.
\item [COMMENT] --- Comment.
\end{description}
\end{description}

For example:
\begin{small}
\begin{verbatim}
      NAME     FORMAT       VALUE                      COMMENT
      ----     ------       -----                      -------

     AUTHOR      A16      J. Bloggs          J. Bloggs, F. Smith and R. Brown
     NUMENTS      I4          3432           Number of entries in the catalogue
\end{verbatim}
\end{small}

A catalogue is organised in rows and columns. In an astronomical  catalogue 
columns are called 
{\bf FIELDS} and rows are called {\bf ENTRIES}. Each field has information 
associated with it.

Each field requires :
 
\begin{description}
\begin{description}
\item [NAME] --- Field Name. 

The name identifies a particular field (column) in the 
catalogue and therefore must be unique in this catalogue.

\item [FORMAT] --- Format of the field.

The format indicates how this field should be displayed. The Field Formats
section of this document give further details of the supported formats.

\item [UNITS] --- Units of the field.

\item [NULL VALUE] --- Null value of the field.

Catalogues sometimes contain objects where a value for a  field is unavailable.
In these cases the authors may choose to insert a `magic value' or NULL  VALUE
in its place in the catalogue. Applications can then take appropriate action if
a null value is encountered. 

\item [COMMENT] --- Comment on the field.

\end{description}
\end{description}

For example:

\begin{small}
\begin{verbatim}
      NAME     FORMAT     UNITS       NULL VALUE              COMMENT
      ----     ------     -----       ----------              -------

     FLUX       F8.4     JANSKY        999.9999             Flux Density   
     RADIUS     I4       ARCSEC        9999                 Radius Vector
\end{verbatim}
\end{small}

\subsection {Field Formats}
 
The field format determines how the values of that field are to be displayed.
The report application CATREPORT supports all FORTRAN formats plus all the 
common sexagesimal formats (SDD MM SS, SDD:MM:SS, DEGREES, HH MM SS 
{\em etc.})

You may of course chose to write your own applications that allow different 
field formats. See SUN/119 if you are interested in writing your own
applications.

\subsection {Restrictions in this implementation of CATPAC}

Personal catalogues are created, manipulated or deleted in your current
directory. Users are strongly advised to create a single directory in which to
carry out their catalogue handling work.

The current implementation uses the following parameter names:

\begin{small}
\begin{verbatim}
      TITLE
      MEDIUM
      ACCESSMODE
      RECORDSIZE
      BLOCKSIZE
      FILENUMBER
      LOCAL_INDEX
      GLOBAL_INDEX
      START_RECORD
      NRECORDS
\end{verbatim}
\end{small}

and the following field names:

\begin{small}
\begin{verbatim}
      NUMBER
      POINTER    
      CATALOGUE
\end{verbatim}
\end{small}

within the system. They should not be used by any application.

\section{Getting started}

\subsection{Quotas}

You may need increased quotas to run {\small ADAM} applications and {\small
CATPAC} in particular.  See Appendix~\ref{ap:quotas} for details. You can check
your quotas from {\small DCL} with:

\begin{verbatim}
      $ SHOW PROCESS/QUOTA
\end{verbatim}

Note that the {\tt \$} is the standard VMS prompt which you do not type. If any
additional quotas are required see your computer manager. 

\subsection{ADAM startup}

Before you can run an {\small ADAM} application you must enter:

\begin{verbatim}
      $ ADAMSTART
\end{verbatim}

to define some logical names and symbols. You need only do this once per
terminal session.  Regular {\small CATPAC} users may wish to put this command
into their {\tt LOGIN.COM}.

\subsection{Running CATPAC}

{\small CATPAC} comes in two forms.  One is a monolith that is activated
within the {\small ADAM} command language---{\small ICL}.  The other
comprises individual applications and is run from {\small DCL}.  The
monolith has the advantage that after the initial pause while it loads,
all its constituent applications are available immediately.  There is no
pause while each new application is fired up.  This is fine when you
wish to run several applications and/or the same application several
times. However, for single tasks it is more convenient to run from
{\small DCL}.  You may simply prefer the familiar {\small DCL} to
{\small ICL}, though {\small DCL} commands, including editing, are
accessible from {\small ICL} via a {\tt \$} prefix. 

To run {\small CATPAC} from {\small DCL} just enter the command:

\begin{verbatim}
      $ CATPAC
\end{verbatim}

This executes a procedure setting up symbols for {\small CATPAC}'s command
names, and defines some logical names to make help information available. Then
you'll be able to mix {\small CATPAC} commands with the familiar {\small DCL}
ones.

To run the {\small CATPAC} monolith is almost the same except you must be
within the {\small ADAM} command language. This requires just one extra
command, namely:

\begin{verbatim}
      $ ICL
\end{verbatim}

You will see any messages produced by system and user procedures, followed by
the {\tt ICL>} prompt. Again there is a procedure for making the commands known
to the command language, and not unexpectedly, it too is:

\begin{verbatim}
      ICL> CATPAC
\end{verbatim}

Then you are ready to go.  In either case  you'll see message from CATPAC
telling you which version is ready for use.

So what applications are now available?  Appendix~\ref{ap:summary} lists
in alphabetical order the commands and their functions, and
Appendix~\ref{ap:classified} is a classified list of the same commands. 

\subsection{Issuing Commands}

To run an application you then can just give its name---you will be
prompted for any required parameters. Alternatively, you may enter
parameter values on the command line specified by position or by
keyword.  More on this in Section~\ref{se:param}.

Commands are interpreted in a case-independent way.
They may be abbreviated provided they are unambiguous strings with
at least four characters.  Commands shorter than five characters,
therefore, cannot be shortened. So:
\begin{verbatim}
      ICL> CORR
      ICL> CORRE
      ICL> CorR
      ICL> CORRELATE
\end{verbatim}
would all run CORRELATE.

Note if other packages are active there may be command-name clashes, Issuing
such a command will run that command in the package last activated.  You can
ensure receiving the {\small CATPAC} command by inserting a {\tt CAT\_} prefix
before the command name.  For example:

\begin{verbatim}
      $ CAT_CORRELATE
\end{verbatim}

will execute {\small CAT}'s CORRELATE application.

\begin{quote}
{\large {\bf Since the {\normalsize{\bf CAT}} commands are the same in both
{\normalsize{\bf DCL}} and {\normalsize{\bf ICL}}, the {\tt \$} and {\tt ICL>}
prompts in the examples and description below are interchangeable unless noted
otherwise.}}
\end{quote}

\subsection{Obtaining Help}

\subsubsection{Entering the Help System}

An introduction is given in the CATPAC topic via:

\begin{verbatim}
      ICL> HELP CATPAC
\end{verbatim}

The behaviour is different for {\small DCL} and {\small ICL} because of a
restriction imposed by VMS help libraries that is circumvented in {\small ICL}. 
From {\small ICL} {\tt HELP CATPAC} puts you into the top level of the help
library. Type {\tt ?} to get a list of  help topics.  These are mostly the
commands for running applications, but they also include global information on
matters such as parameters.

\begin{verbatim}
      ICL> HELP PACKAGES CATPAC
\end{verbatim}

gives a summary of purpose of the package. (This is part of an index
of {\small ADAM} packages.)

Whereas from {\small DCL}:

\begin{verbatim}
      $ HELP CATPAC
\end{verbatim}

combines the two help items described above.  Note this locates you at the
second level of the help hierarchy because the highest is {\tt "Help"}, not
{\tt "CATPAC"}; you'll have to hit a carriage return ({\tt <CR>}) to climb a
level to see the list of help on commands and general information. 

If you have commenced running an application you can still access the help
library at prompts for parameters.  It is primarily there to provide
information about the parameter being prompted. See Section~\ref{se:parhelp}
for details. 

\subsubsection{Navigating Help Hierarchies}

The help systems may be navigated in the normal ways for a VMS help library,
except that from {\small ICL} {\tt HELP} does not recognise all the navigation
commands. For those not familiar with VMS help libraries the topic {\tt
Using\_help} has details. You can exit from the help by typing {\tt CTRL/Z}
(that is, pressing the CONTROL and Z keys simultaneously) in response to any
prompt (except from {\small ICL} where it must be a {\tt subtopic?} or {\tt
topic?} prompt).  A series of carriage returns has the same effect. 

\subsubsection{Help on CATPAC commands}

Help on an individual {\small CATPAC} application is simply achieved by
entering {\tt HELP} followed by the command name, for example: 

\begin{verbatim}
      $ HELP CORRELATE
\end{verbatim}

will give the description and usage of the CORRELATE command.  There are
subtopics which contain details of the parameters, including defaults, and
valid ranges; examples; notes expanding on the description; implementation
status; and occasionally timing. For example:

\begin{verbatim}
      ICL> HELP CORR PARAM ALLFLDS
\end{verbatim}

gives details of parameter {\tt ALLFLDS} in all applications prefixed by {\tt
CORR}. 

The instruction:

\begin{verbatim}
      ICL> HELP CLASSIFIED
\end{verbatim}

displays a list of subject areas as subtopics.  Each subtopic lists and gives
the function of each {\small CATPAC} application in that classification. There
is also an alphabetic list which can be obtained directly via the command:

\begin{verbatim}
      ICL> HELP SUMMARY
\end{verbatim}

It is expected that the {\small ICL} help system will be improved via
the fast and more-flexible {\small HELP} package (SUN/124).  More
documentation, including a cookbook and beginners guide are planned for
later versions.

\subsubsection{Command clashes}

Note if other packages are active, there may be duplicated commands, {\it
e.g.}\ GLOBALS is in {\small CATPAC} and {\small KAPPA}.  Help on the most
recently activated will be given.  From {\small ICL} you can ensure receiving
the {\small CATPAC} help by inserting a {\tt CAT\_} prefix before the command
name.  For example:

\begin{verbatim}
      ICL> HELP CAT_GLOBALS
\end{verbatim}

will describe {\small CATPAC}'s GLOBALS application.  From {\small DCL}
go via {\tt \$ HELP CATPAC} then select the GLOBALS topic.

\subsection{Exiting an Application}

In normal circumstances when you've finished using {\small CATPAC} nothing need
be done from {\small DCL}, but to end an {\small ICL} session, enter the {\tt
EXIT} command to return to {\small DCL}.

What if you've done something wrong, like entering the wrong value for a
parameter?  If there are further prompts you can enter the abort code {\tt !!}
to exit the application.  This is recommended even from {\small DCL} because
certain files may become corrupted if you use the crude {\tt CTRL/Y}. If,
however, processing of the data has begun in the application {\tt CNTL/C}
should be hit.  From {\small ICL} this ought to return you to a prompt, but the
processing will continue.  Then you can stop the running process via:

\begin{verbatim} 
      ICL> KILL CATPAC_DIR:CATPAC 
\end{verbatim} 

{\small CATPAC} will be loaded again once you enter a {\small CATPAC} command.
If several attempts with {\tt CTRL/C} fail to return you to an {\small ICL}
prompt then it's time for the heavy artillery---{\tt CTRL/Y}.  Once back to
{\small DCL} enter {\tt CONTINUE} to return to {\small ICL} where you left off,
and then kill the process as described above. 

\section{Parameters}
\label{se:param}

{\small CATPAC} is a command-driven package.  Commands have {\em parameters\/}
by which you can qualify their behaviour. Parameters are obtained in response
to prompts or supplied on a command line.

For convenience, the main aspects of the {\small ADAM} parameter system
as seen by a user of {\small CATPAC} are described below.  Though note
that most of what follows is applicable to any {\small ADAM}
application.

\subsection{Defaults}
\label{se:defaults}

Command-line values are used mostly for those parameters that are normally
defaulted by the application.   Defaulted parameters enable applications to
have many options, say for controlling the appearance of some graphical output,
without making routine operations tedious because of a large number of prompts.
The values of normally defaulted parameters can be found by obtaining online
help on a specific parameter.  They are enclosed in square brackets at the end
of the parameter description.

\begin{verbatim}
      ICL> HELP CORRELATE PARAM *
\end{verbatim}

gives details of all parameters in application CORRELATE. Applications show the
defaults at the end of each  parameter description enclosed in square
brackets---the same as in Appendix~\ref{ap:full}. If you want to override one
of these defaults, then you must specify the parameter's value on the command
line. 

When you are prompted you will usually be given a suggested default value in
{\tt / /} delimiters.  You can choose to accept the default by pressing
carriage return.  For example, 5 is the suggested value below:

\begin{verbatim}
      FREQUENCY - Sample frequency N.  /5/ >
\end{verbatim}
Alternatively, enter a different value
\begin{verbatim}
      FREQUENCY - Sample frequency N. /5/ > 10
\end{verbatim}
to override the default. 
Some defaults begin with an {\tt @}.
\begin{verbatim}
      DATAFILE - Name of file containing the data /@TESTDATA.DAT/ > 
\end{verbatim}

These are associated with files (ASCII, HDS) and devices (graphics,
tape). If you want to override the default given, you do not have to
prefix your value with an {\tt @}, {\it e.g.}

\begin{verbatim}
      DATAFILE - Name of file containing the data /@TESTDATA.DAT/ > NEWDATA.DAT
\end{verbatim}

The default value can be edited to save typing by first pressing the {\tt
<TAB>} key. The editor behaves like the {\small DCL} command-line editor.
Defaults may change as data are processed by {\small CATPAC}. Often the current
(last) value of the parameter will be substituted, or a dynamic value is
suggested depending on the values of other parameters.  Current values of
CATPAC parameters are stored in the HDS file {\tt ADAM\_USER:CATPAC.SDF}, and
so they persist between {\small CATPAC} sessions. This file should not be
deleted unless {\small CATPAC} parameters are to be completely reset. 

\subsection{Globals}

{\small CATPAC} stores a number of global parameters that are used as defaults
to reduce typing in response to prompts. Global means that they are shared
between applications.  The most common is the last INPUT catalogue. If you just
press {\tt <CR>} to the prompt, the global value is unchanged. Only when you
modify the parameter and the  application completes without error is the global
value updated, and so becomes the suggested default value for the next prompt
for an INPUT catalogue name.

All global parameters are stored in HDS file {\tt ADAM\_USER:GLOBAL.SDF}.
The full list is:

\begin{description}
\begin{description}
\item [GLOBAL.INPUT] --- Current input catalogue
\end{description}
\end{description}

{\small CATPAC} uses the last INPUT catalogue as the
suggested default value for the next prompt for an INPUT. 

The values of of all global parameters may be inspected via the
{\tt GLOBALS} command:
\begin{verbatim}
      ICL> GLOBALS
      The current INPUT catalogue is       : TEST
\end{verbatim}

\subsection{Strings}
\label{se:parstring}
Notice that the apostrophes around strings given in response to prompts
for a character parameter can be omitted. However, on the command
line quotes or double quotes are needed if the string contains spaces,
otherwise the second and subsequent words could be treated as
separate positional parameters.

Responses to prompts are case insensitive. 

\subsection{Arrays}
If a parameter requires an array of values, the series
should be in brackets separated by commas or spaces.  For example:
\begin{verbatim}
      FLDNAMES - List of field names > [STARNAME,RA,DEC]
\end{verbatim}
would input three values: STARNAME, RA and DEC into the character parameter
FLDNAMES.  If the application is expecting an exact number of values
you will be reprompted, either for all the values if you give too many,
or the remaining values if you supply too few.  There is one exception
where you can omit the brackets---a fairly common one---and that is in
response to a prompt for a one-dimensional array as above.

\subsection{Abort and Null}

Responding to a prompt with a null value {\tt !} will not necessarily
cause the application to abort, but it can force a suitable default to
be used, where this is the most-sensible action. A further use in CATPAC
is to terminate the interactive loop in the POLYGON application.

Responding to a prompt with an abort request {\tt !!} will abort the
application.  This process includes the various tidying operations
such as the unmapping and closing of files.  Any other method of
stopping an application prematurely can leave files mapped or corrupted. 

\subsection{Help}
\label{se:parhelp}

To get help about a parameter enter {\tt ?}.  Usually, this will give
access to the help-library
information for that parameter, for example:
\begin{verbatim}
      DATAFILE - Name of the ASCII data file.
 
      ASCIITOCAT
 
        Parameters
 
          DATAFILE = _CHAR (Read)
             Name of the file containing the tabular ascii data.
 
      DATAFILE - Name of the ASCII data file.
\end{verbatim}

and then immediately reprompt you for the parameter.  There are occasions when
information about the parameter is insufficient; you may require to examine the
examples or the description of related parameters.  This can be achieved via
entering {\tt ??} to the prompt. You can then delve anywhere in the help
information.  When you exit the help system you're reprompted for the
parameter.

\subsection{Logical names}

From {\small DCL}, ordinary process logical names may be used as normal.
However, logical names to be used by applications (including {\small CATPAC})
from within {\small ICL} must be defined with the /JOB qualifier. Thus if your
ascii data files are stored in {\tt DISK\$USER1:[CATALOGUES.ASCII]}, with alias
ASCIIDIR, then:

\begin{verbatim}
      $ DEFINE/JOB ASCIIDIR DISK$USER1:[CATALOGUES.ASCII]
\end{verbatim}

will enable you to respond to a prompt thus:

\begin{small}
\begin{verbatim}
DATAFILE - Name of file containing the data /@TESTDATA.DAT/ > ASCIIDIR:NEWDATA.DAT
\end{verbatim}
\end{small}

Regrettably, at present {\small ICL} cannot change default directories
for {\small CATPAC}.  The DEFAULT command only applies to {\small ICL}
and the {\small DCL} process.  Therefore, if you need to access files in
several directories with the minimum of typing, you should define some
job logical names as described above.  If you require many then you will
need your JTQUOTA increased.  See your site manager if you run into
difficulties.  It is advisable to set default to the directory
containing your data files before entering {\small ICL}.

\subsection{Specifying Parameter Values on Command Lines}
\label{se:cmdlindef}
Parameters may be assigned values on the command line. This is useful
for running {\small CATPAC} in batch mode and in procedures, and for
specifying the values of parameters that would otherwise be defaulted. A
command-line parameter will prevent prompting for that parameter unless
there is an error with the given value, say giving an alphabetic
character string for a floating-point value. 

There are two ways in which parameter values may be given on the
command line: by keyword and by position. The two forms may be
mixed with care. The parser looks for positional parameters then
keywords, so you can have some positional values followed by keyword
values.  See some of the examples presented in Appendix~\ref{ap:full}.

Keywords may appear in any order.
Here is an example of command-line defaults using keywords: 
\begin{verbatim}
      ICL> CATREPORT TEST ALLFLDS=F FLDNAMES=[RA,DEC]
\end{verbatim}

To obtain a false value for a logical parameter
you add a {\tt NO} prefix to keyword, for example:
\begin{verbatim}
      ICL> CATREPORT NOALLFLDS
\end{verbatim}
would be equivalent to {\tt ALLFLDS=F}.

To obtain a true value for a logical parameter use just the keyword, 
for example:
\begin{verbatim}
      ICL> CATREPORT ALLFLDS
\end{verbatim}
would be equivalent to {\tt ALLFLDS=T}.

Alternatively, you can specify command-line values by position:
\begin{verbatim}
      $ SAMPLE TEST SAMPTEST 5
\end{verbatim}
The application samples the catalogue {\tt TEST} at a frequency of 5 creating 
a new catalogue {\tt SAMPTEST}. Note
trailing parameters may missed but
intermediate ones may not.  Some applications have many parameters and
it would be tedious not only to enter all the intermediate values
between the ones you want to define, but also to remember them all. The
position of a parameter can be found in the {\tt Usage} heading
in Appendix~\ref{ap:full} or the help for the application.

Another consideration is that some parameters do not have defined positions
because they are normally defaulted. Thus the keyword technique is
recommended for most parameters.  See Section~\ref{se:custom} if you
want to abbreviate some command strings to reduce typing.

Sometimes specifying a parameter on the command line induces different
behaviour, usually to inhibit a loop for procedures, or to eliminate
unnecessary processing.
For instance:

\begin{verbatim}
      $ SAMPLE TEST SAMPTEST 5 REJECT=T OUTREJECT=TESTREJECTS
\end{verbatim}

will cause a second catalogue to be created {\tt TESTREJECTS} containing
those entries rejected from the sample.

\subsection{Special Keywords: ACCEPT, PROMPT, RESET}
\label{se:iclkey}

Another way in which prompts and default values can be controlled is by use of
the keywords ACCEPT, PROMPT and RESET.

The RESET keyword causes the default value of all parameters (apart from those
already specified before it on the command line) to be set to the original
values specified by the application or its interface file.  In other words
global and current values are ignored.

The PROMPT keyword forces a prompt to appear for every application parameter. 
This can be useful if you cannot remember the name of a defaulted parameter or
there would be too much to type on the command line.  However, it may prove
tedious for certain {\small KAPPA} applications that have tens of parameters,
most of which you normally never see.  You can abort if it becomes too boring.

The ACCEPT keyword forces the parameter system to accept the {\em suggested}
default values for every application parameter.  In other words, those
parameters which would normally be prompted with a value between `/ /'
delimiters take the value between those delimiters, {\it e.g.}  INPUT we saw in
Section~\ref{se:defaults} would take the value {\tt TESTDATA.DAT}.  Parameters
that are normally defaulted behave as normal.  The ACCEPT keyword needs to be
used with care because not every parameter has a default, and therefore must be
given on the command line for the application to work properly. For example,
CATSEARCH must have a value specified for parameter OUTPUT, the name of the
output catalogue. If we run the application like this:

\begin{verbatim}
      ICL> CATSEARCH ACCEPT
\end{verbatim}

it would fail in the sense that it would still have to prompt for a value---it
does not know where to write the output catalogue. However, if we run CATSEARCH
like this:

\begin{verbatim}
      ICL> CATSEARCH OUTPUT=STARS 
\end{verbatim}

it would generate an output catalogue using default values for all the
parameters except OUTPUT. Another point to be wary of is that some applications
have loops, {\it e.g.}\ ASCIITOCAT, POLYGON, and if you use the ACCEPT keyword
it will only operate the first time the application gets a parameter value. 

\section {Demonstration}

A demonstration of the CATPAC Applications has been included in this release.
The demonstration is self describing and takes the form of ICL command file.
To run the demonstration at the ICL prompt type:

\begin{verbatim}
      ICL> LOAD CATPAC_DIR:CATPACDEMO
\end{verbatim}

Most of your questions about the CATPAC package will be answered by working 
through this demonstration.

\section{Custom CATPAC}
\label{se:custom}

If you don't like {\small CATPAC}'s parameter defaults, or its choice of what
parameters get prompted and what get defaulted, then you can copy the interface
file ({\tt CATPAC\_DIR:CATPAC.IFL}) to your work directory and make the
required modifications, and then recompile it.  Then when you run CATPAC from
that directory, your version of the interface file is accessed, and not the
released one.  Of course, once you have done this the documentation in
Appendix~\ref{ap:full} will no longer be correct.  See Sections~8.3 and 8.4 of
SG/4 on the meanings and possible values of the fieldnames, and how to
recompile the interface file.

There is an easier method of tailoring {\small CATPAC} to your  requirements.
If you frequently use certain commands, especially those with a long list of
keywords and fixed values, you can define some symbols for the commands so that
each time you activate {\small ICL} these abbreviations will be available to
you.  This is achieved via an {\small ICL} {\em login file}. It works in the
same way as the DCL {\tt LOGIN.COM}.  What you should do is to create a {\tt
LOGIN.ICL} in a convenient  directory, and put the following definition in your
DCL {\tt LOGIN.COM} file:

\begin{verbatim}
      $ DEFINE ICL_LOGIN DISK$USER:[XYZ.ABC]LOGIN.ICL
\end{verbatim}

where {\tt DISK\$USER:[XYZ.ABC]} needs to be replaced by the actual  directory
used.  It is possible to have several {\small ICL} login files, each for
different work in different directories.  Now to abbreviate a command you put a
DEFSTRING entry into the {\small ICL} login file. For example:

\begin{verbatim}
      DEFSTRING MYR{EPORT} CATREPORT ALLFLDS=N FLDNAMES=[RA,DEC]
\end{verbatim}
defines {\tt MYR} or {\tt MYRE} or {\tt MYREPORT} to run CATREPORT without
selecting only the RA and DEC fields.

\newpage
\appendix
\section{An Alphabetical Summary of CATPAC Commands}
\label{ap:summary}

\begin{description}
\item [ADDPARAM]:
 Adds a parameter to a catalogue.
\item [ASCIITOCAT]:
 Create a catalogue from an ascii file.
\item [CALCFLD]:
 Calculate a new field.
\item [CATRENAME]:
 Rename a catalogue.
\item [CATREPORT]:
 Produce a catalogue report.
\item [CATSEARCH]:
 Select entries from a catalogue.
\item [CATSORT]:
 Sort the entries in a catalogue.
\item [COPYCAT]:
 Make a copy of a catalogue.
\item [CORRELATE]:
 Non-parametric correlation between two fields.
\item [DELCAT]:
 Deletes a catalogue from the system.
\item [DELPARAM]:
 Deletes a parameter from a catalogue.
\item [DELSORT]:
 Deletes the sort information associated with a catalogue.
\item [ENTRIES]:
 Find the number of entries in a catalogue.
\item [FIELDINFO]:
 Find specific information about a field in the catalogue.
\item [FIELDS]:
 Finds the number and names of fields in a catalogue.
\item [FK425]:
 Create a new catalogue in the FK5 coordinate system. (See SLALIB)
\item [FK45Z]:
 Create a new catalogue in the FK5 coordinate system. (See SLALIB)
\item [FK524]:
 Create a new catalogue in the FK4 coordinate system. (See SLALIB)
\item [FK54Z]:
 Create a new catalogue in the FK4 coordinate system. (See SLALIB)
\item [GLOBALS]:
 Displays the values of the CATPAC global parameters.
\item [JOIN]:
 Join two catalogues.
\item [LINCOR]:
 Linear correlation between fields.
\item [LISTIN]:
 Create a catalogue from a free format ascii file.
\item [LITTLEBIG]:
 Extracts entries with largest or smallest values of a given field.
\item [MERGE]:
 Merge two catalogues.
\item [PARAMINFO]:
 Find specific information about a parameter in the catalogue.
\item [PARAMS]:
 Finds the number and names of parameters in a catalogue.
\item [POLYGON]:
 Create a polygon definition for use with WITHIN.
\item [PROPERM]:
 Create a new catalogue preforming proper motion corrections (See SLALIB)
\item [REJECT]:
 Select rejected entries from a catalogue.
\item [SAMPLE]:
 Select every Nth entry from a catalogue.
\item [SELECTFLDS]:
 Select fields from a catalogue.
\item [SORTFLDS]:
 Get the sort information associated with a catalogue.
\item [UPDATE]:
 Update a field in a catalogue.
\item [UPFIELD]:
 Update the information associated with a field in a catalogue.
\item [UPPARAM]:
 Update the information associated with a parameter in a catalogue.
\item [WITHIN]:
 Select entries within a polygon.
\end{description}

\newpage
\section{Classified CATPAC commands}
\label{ap:classified}

{\small CATPAC} applications may be classified in terms of their
functions as follows:

\begin{description}

\item [Reporting Catalogues] ---

\begin{description}

\item [Reporting Catalogues.]:

\begin {description}

\item [CATREPORT]:
 Produce a catalogue report.

\end{description}
\end{description}

\item [Creating Catalogues] ---

\begin{description}

\item [Creating a catalogue from data]:

\begin {description}

\item [ASCIITOCAT]:
 Create a catalogue from an ASCII file.

\item [LISTIN]:
 Create a catalogue from a free format ASCII file.

\item [POLYGON]:
 Create a polygon definition for use with WITHIN.

\end{description}

\item [Creating a catalogue from another catalogue.] :

\begin{description}

\item [CALCFLD]:
 Create a new catalogue with an extra calculated field.

\item [CATRENAME]:
 Rename a catalogue.

\item [CATSEARCH]:
 Select entries from a catalogue.

\item [CATSORT]:
 Sort the entries in a catalogue.

\item [COPYCAT]:
 Make a copy of a catalogue.

\item [FK425]:
 Create a new catalogue in the FK5 coordinate system. (See SLALIB)

\item [FK45Z]:
 Create a new catalogue in the FK5 coordinate system. (See SLALIB)

\item [FK524]:
 Create a new catalogue in the FK4 coordinate system. (See SLALIB)

\item [FK54Z]:
 Create a new catalogue in the FK4 coordinate system. (See SLALIB)

\item [JOIN]:
 Join two catalogues.

\item [LITTLEBIG]:
 Extracts entries with largest or smallest values of a given field.

\item [MERGE]:
 Merge two catalogues.

\item [PROPERM]:
 Create a new catalogue preforming proper motion corrections (See SLALIB)

\item [REJECT]:
 Select rejected entries from a catalogue.

\item [SAMPLE]:
 Select every Nth entry from a catalogue.

\item [SELECTFLDS]:
 Select fields from a catalogue.

\item [WITHIN]:
 Select entries within a polygon.

\end{description}

\end{description}

\item [Deleting Catalogues] ---

\begin{description}

\item [Deleting catalogues.]:

\begin {description}

\item [DELCAT]:
 Deletes a catalogue from the system.

\end{description}

\end{description}

\item [Manipulating information about Catalogues] ---

\begin{description}

\item [Examining the information about the catalogue.]:

\begin {description}

\item [ENTRIES]:
 Find the number of entries in a catalogue.

\item [FIELDINFO]:
 Find specific information about a field in the catalogue.

\item [FIELDS]:
 Finds the number and names of fields in a catalogue.

\item [PARAMINFO]:
 Find specific information about a parameter in the catalogue.

\item [PARAMS]:
 Finds the number and names of parameters in a catalogue.

\item [SORTFLDS]:
 Get the sort information associated with a catalogue.

\end{description}

\item [Changing the information about the catalogue.] :

\begin{description}

\item [ADDPARAM]:
 Adds a parameter to a catalogue.

\item [DELPARAM]:
 Deletes a parameter from a catalogue.

\item [DELSORT]:
 Deletes the sort information associated with a catalogue.

\item [UPFIELD]:
 Update the information associated with a field in a catalogue.

\item [UPPARAM]:
 Update the information associated with a parameter in a catalogue.

\end{description}

\end{description}

\item [Manipulating data in a catalogue] ---

\begin{description}

\item [Correlation Analysis.]:

\begin{description}

\item [CORRELATE]:
 Non-parametric correlation between two fields.

\item [LINCOR]:
 Linear correlation between fields.

\end{description}

\item [UPDATE]:
 Update a field in a catalogue.

\end{description}

\item [Inquiries and Status] ---

\begin{description}

\begin{description}

\item [GLOBALS]:
 Displays the values of the CATPAC global parameters.

\end{description}

\end{description}

\end{description}

\section{Quotas to run CATPAC}
\label{ap:quotas}

The quotas shown below are the advised quotas for running {\small KAPPA}. In
all but exceptional cases they should be sufficient for running {\small
CATPAC}.

\begin{tabular}[c]{lrlr}
\\
\hspace{4ex} CPU limit: & Infinite & \hspace{4ex} Direct I/O limit: & 18\\
\hspace{4ex} Buffered I/O byte count quota: & 20480 & \hspace{4ex} Buffered I/O limit: & 18\\
\hspace{4ex} Timer queue entry quota: & 9 & \hspace{4ex} Open file quota: & 50\\
\hspace{4ex} Paging file quota: & 40000 & \hspace{4ex} Subprocess quota: & 10\\
\hspace{4ex} Default page fault cluster: & 32 & \hspace{4ex} AST limit: & 23\\
\hspace{4ex} Enqueue quota: & 30 & \hspace{4ex} Shared file limit: & 0\\
\hspace{4ex} Max detached processes: & 0 & \hspace{4ex} Max active jobs: & 0\\
\hspace{4ex} JTQUOTA: & 3072 & & \\
\end{tabular}

\newpage
\section{Specifications of CATPAC applications}
\label{ap:full}
\subsection{Explanatory Notes}

\begin{description}

\item [Layout]
In this layout the specification of parameters has the following
format. 

\begin{verbatim}
     name  =  type (access)
        description
\end{verbatim}

This format also includes a {\em Usage} entry this shows how the
application is invoked from the command line.   It lists the positional
parameters in order followed by any prompted keyword parameters using 
a {\mbox ``KEYWORD=?''} syntax.  Defaulted
keyword parameters do not appear.  Positional parameters
that are normally defaulted are indicated by being enclosed in square
brackets.

Some parameters will only be used when another parameter has a certain
value or mode. These are indicated by the name of the mode in
parentheses at the end of the parameter description, but before any
default, {\it e.g.}\ parameter OUTREJECT in SAMPLE is only
relevant when parameter REJECT is {\tt "TRUE"}.

\end {description}

\newcommand {\mantt}{\tt}

{\mantt \%name} means the value of parameter {\it name}.

The description entry has a notation scheme to indicate 
normally defaulted parameters, {\it i.e.}\ those for which there will
be no prompt.
For such parameters a matching pair of square brackets ({\mantt []})
terminates the description.  The content between the brackets mean

\begin{description}

\item[{\mantt []}]
Empty brackets means that the default is created dynamically
by the application, and may depend on the values of other parameters.
Therefore, the default cannot be given explicitly.

\item[{\mantt [,]}]
As above, but there are two default values that are created dynamically.

\item[{\mantt [}{\rm default}{\mantt ]}]
Occasionally, a description of the default is given in normal type,
{\it e.g.}\ the size of the plotting region in a graphics application,
where the exact default values depend on the device chosen. 

\item[{\mantt [default]}]
If the brackets contain a value in teletype-fount, this is the explicit
default value.

\end{description}

\subsection{Abbreviations}

Abbreviations  only been used where it was considered necessary to reduce
the length of the commands. For example  {\small DELETECATALOGUE} becomes
{\small DELCAT}. The full list of abbreviations used is:

\begin{description}
\begin{description}
\item [CAT]: CATALOGUE
\item [DEL]: DELETE
\item [FLD]: FIELD
\item [PARAM]: PARAMETER
\end{description}
\end{description}

%+
%  Name:
%     LAYOUT.TEX

%  Purpose:
%     Define Latex commands for laying out documentation produced by PROLAT.

%  Language:
%     Latex

%  Type of Module:
%     Data file for use by the PROLAT application.

%  Description:
%     This file defines Latex commands which allow routine documentation
%     produced by the SST application PROLAT to be processed by Latex. The
%     contents of this file should be included in the source presented to
%     Latex in front of any output from PROLAT. By default, this is done
%     automatically by PROLAT.

%  Notes:
%     The definitions in this file should be included explicitly in any file
%     which requires them. The \include directive should not be used, as it
%     may not then be possible to process the resulting document with Latex
%     at a later date if changes to this definitions file become necessary.

%  Authors:
%     RFWS: R.F. Warren-Smith (STARLINK)

%  History:
%     10-SEP-1990 (RFWS):
%        Original version.
%     10-SEP-1990 (RFWS):
%        Added the implementation status section.
%     12-SEP-1990 (RFWS):
%        Added support for the usage section and adjusted various spacings.
%     {enter_further_changes_here}

%  Bugs:
%     {note_any_bugs_here}

%-

%  Define length variables.
\newlength{\sstbannerlength}
\newlength{\sstcaptionlength}

%  Define a \tt font of the required size.
\font\ssttt=cmtt10 scaled 1095

%  Define a command to produce a routine header, including its name,
%  a purpose description and the rest of the routine's documentation.
\newcommand{\sstroutine}[3]{
   \goodbreak
   \rule{\textwidth}{0.5mm}
   \vspace{-7ex}
   \newline
   \settowidth{\sstbannerlength}{{\Large {\bf #1}}}
   \setlength{\sstcaptionlength}{\textwidth}
   \addtolength{\sstbannerlength}{0.5em}
   \addtolength{\sstcaptionlength}{-2.0\sstbannerlength}
   \addtolength{\sstcaptionlength}{-4.45pt}
   \parbox[t]{\sstbannerlength}{\flushleft{\Large {\bf #1}}}
   \parbox[t]{\sstcaptionlength}{\center{\Large #2}}
   \parbox[t]{\sstbannerlength}{\flushright{\Large {\bf #1}}}
   \begin{description}
      #3
   \end{description}
}

%  Format the description section.
\newcommand{\sstdescription}[1]{\item[Description:] #1}

%  Format the usage section.
\newcommand{\sstusage}[1]{\item[Usage:] \mbox{} \\[1.3ex] {\ssttt #1}}

%  Format the invocation section.
\newcommand{\sstinvocation}[1]{\item[Invocation:]\hspace{0.4em}{\tt #1}}

%  Format the arguments section.
\newcommand{\sstarguments}[1]{
   \item[Arguments:] \mbox{} \\
   \vspace{-3.5ex}
   \begin{description}
      #1
   \end{description}
}

%  Format the returned value section (for a function).
\newcommand{\sstreturnedvalue}[1]{
   \item[Returned Value:] \mbox{} \\
   \vspace{-3.5ex}
   \begin{description}
      #1
   \end{description}
}

%  Format the parameters section (for an application).
\newcommand{\sstparameters}[1]{
   \item[Parameters:] \mbox{} \\
   \vspace{-3.5ex}
   \begin{description}
      #1
   \end{description}
}

%  Format the examples section.
\newcommand{\sstexamples}[1]{
   \item[Examples:] \mbox{} \\
   \vspace{-3.5ex}
   \begin{description}
      #1
   \end{description}
}

%  Define the format of a subsection in a normal section.
\newcommand{\sstsubsection}[1]{\item[{#1}] \mbox{} \\}

%  Define the format of a subsection in the examples section.
\newcommand{\sstexamplesubsection}[1]{\item[{\ssttt #1}] \mbox{} \\}

%  Format the notes section.
\newcommand{\sstnotes}[1]{\item[Notes:] \mbox{} \\[1.3ex] #1}

%  Provide a general-purpose format for additional (DIY) sections.
\newcommand{\sstdiytopic}[2]{\item[{\hspace{-0.35em}#1\hspace{-0.35em}:}] \mbox{} \\[1.3ex] #2}

%  Format the implementation status section.
\newcommand{\sstimplementationstatus}[1]{
   \item[{Implementation Status:}] \mbox{} \\[1.3ex] #1}

%  Format the bugs section.
\newcommand{\sstbugs}[1]{\item[Bugs:] #1}

%  Format a list of items while in paragraph mode.
\newcommand{\sstitemlist}[1]{
  \mbox{} \\
  \vspace{-3.5ex}
  \begin{itemize}
     #1
  \end{itemize}
}
% ------
% End of LAYOUT.TEX macros
% ------


\newpage

\begin{small}
\sstroutine{
   ADDPARAM
}{
   Add a Parameter to a Catalogue
}{
   \sstdescription{
      Add a new parameter to a catalogue. The new parameter requires a
      name, format, value and comment.
   }
   \sstusage{
      ADDPARAM INPUT NAME FORMAT VALUE COMMENT
   }
   \sstparameters{
      \sstsubsection{
         INPUT = \_CHAR (Read)
      }{
         Name of the catalogue.
      }
      \sstsubsection{
         NAME = \_CHAR (Read)
      }{
         Name of the new parameter.
      }
      \sstsubsection{
         FORMAT = \_CHAR (Read)
      }{
         Format of the new parameter. A FORTRAN format detailing how the
         parameter value should be interpreted.
      }
      \sstsubsection{
         VALUE = \_CHAR (Read)
      }{
         Value to be associated with the new parameter.
      }
      \sstsubsection{
         COMMENT = \_CHAR (Read)
      }{
         Comment to be associated with the new parameter.
      }
   }
   \sstnotes{
      The value of a parameter is always a character string. The format
      determines how the value should be interpreted.
   }
   \sstdiytopic{
      Example
   }{
      ADDPARAM TEST CLASS I2 25 {\tt '}Catalogue Class{\tt '}
   }
   \sstbugs{
      None known.
   }
}
\sstroutine{
   ASCIITOCAT
}{
   Create a catalogue from an ASCII file
}{
   \sstdescription{
      Create a new catalogue that contains the data from an ASCII file.
      ASCIITOCAT allows you to put your own data into a catalogue. The
      data must be tabular. Before running this application you should
      examine your data and decide which columns/fields in the data
      are to included in the catalogue. For each of these fields note
      the start position and the FORTRAN format that should be used to
      read the data in this field. A wide range of sexagesimal formats are
      also available. A good idea for finding the correct
      start positions is to copy the first few entries of the data into
      a separate file and include a first line containing
      123456789012.....
      Have a copy of this handy when you run the application. The
      application first prompts for the name of the catalogue being
      created and the name of the file from which the data is to be
      read. During the next stage the application repeatedly prompts for
      information about fields. For each field you will be prompted for
      a field name, format, units, null value, comment and start
      position. When you have defined all the fields use the ADAM null
      value ! at the field name prompt to move onto the next stage.
      During the final stage the application repeatedly prompts for
      information about catalogue parameters. For each parameter you
      will be prompted for a paramter name, format, value and comment
      When you have defined all the parameters use the ADAM null
      value ! at the parameter name prompt to finish.
   }
   \sstusage{
      ASCIITOCAT INPUT DATAFILE
   }
   \sstparameters{
      \sstsubsection{
         INPUT = \_CHAR (Read)
      }{
         Name of the catalogue to be created.
      }
      \sstsubsection{
         DATAFILE = \_CHAR (Read)
      }{
         Name of the file containing the tabular ascii data.
      }
      \sstsubsection{
         FNAME = \_CHAR (Read)
      }{
         Name of the next field.
         [Repeatedly prompted for, terminated with a !]
      }
      \sstsubsection{
         FFORMAT = \_CHAR (Read)
      }{
         Format of the field.
         [Repeatedly prompted for.]
      }
      \sstsubsection{
         FUNIT = \_CHAR (Read)
      }{
         Units to be associated with this field.
         [Repeatedly prompted for.]
      }
      \sstsubsection{
         FNULL = \_CHAR (Read)
      }{
         Null value  to be associated with this field.
         [Repeatedly prompted for.]
      }
      \sstsubsection{
         FCOMMENT = \_CHAR (Read)
      }{
         Comment to be associated with this field.
         [Repeatedly prompted for.]
      }
      \sstsubsection{
         STARTPOS = \_INTEGER (Read)
      }{
         Start position, column number, of the data for this field.
         [Repeatedly prompted for.]
      }
      \sstsubsection{
         PNAME = \_CHAR (Read)
      }{
         Name of the next parameter.
         [Repeatedly prompted for, terminated with a !]
      }
      \sstsubsection{
         PFORMAT = \_CHAR (Read)
      }{
         Format of the parameter.
         [Repeatedly prompted for.]
      }
      \sstsubsection{
         PVALUE = \_CHAR (Read)
      }{
         Value to be associated with this parameter.
         [Repeatedly prompted for.]
      }
      \sstsubsection{
         PCOMMENT = \_CHAR (Read)
      }{
         Comment to be associated with this parameter.
         [Repeatedly prompted for.]
      }
   }
   \sstnotes{
      This application performs no checking on your field definition.
      This allows a degree of flexibility when interpreting your data.
      You may, for example, have a field STARID with format A10
      starting at position 15 and field STARNUM with format I4 starting
      at position 21.
   }
   \sstdiytopic{
      Example
   }{
      ASCIITOCAT TEST TESTDATA.DAT
   }
   \sstbugs{
      None known.
   }
}
\sstroutine{
   CALCFLD
}{
   Calculate a new field
}{
   \sstdescription{
      Creates a new catalogue which includes an extra field. The value of the
      new field is calculated for each entry in a catalogue using
      a user defined CATPAC Parser expression. The new field requires a
      format, units, nullvalue and comment.
   }
   \sstusage{
      CALCFLD INPUT NAME EXPRESSION FORMAT UNITS NULLVALUE COMMENT
   }
   \sstparameters{
      \sstsubsection{
         INPUT = \_CHAR (Read)
      }{
         Name of the catalogue.
      }
      \sstsubsection{
         NAME = \_CHAR (Read)
      }{
         Name of the field to be updated.
      }
      \sstsubsection{
         EXPRESSION = \_CHAR (Read)
      }{
         Expression to be evaluated.
      }
      \sstsubsection{
         FORMAT = \_CHAR (Read)
      }{
         Format of the updated field.
      }
      \sstsubsection{
         UNITS = \_CHAR (Read)
      }{
         Units of the updated field.
      }
      \sstsubsection{
         NULLVALUE = \_CHAR (Read)
      }{
         Null value of the updated field.
      }
      \sstsubsection{
         COMMENT = \_CHAR (Read)
      }{
         Comment of the updated field.
      }
   }
   \sstdiytopic{
      Example
   }{
      CALCFLD TEST VALUE1 VALUE2$*$2 F6.2 MAG -99.99 {\tt '}Recalibrated Value{\tt '}
   }
   \sstbugs{
      None known.
   }
}
\sstroutine{
   CATPAC
}{
   Top-level ADAM monolith routine for the CATPAC package
}{
   \sstdescription{
      This routine interprets the action name passed to it and calls
      the appropriate routine to perform the specified action. An error
      will be reported and STATUS will be set if the action name is not
      recognised.
   }
   \sstbugs{
      None known.
   }
}
\sstroutine{
   CATRENAME
}{
   Rename a catalogue
}{
   \sstdescription{
      Rename a catalogue to a new name.
   }
   \sstusage{
      CATRENAME INPUT NEWNAME
   }
   \sstparameters{
      \sstsubsection{
         INPUT = \_CHAR (Read)
      }{
         Name of the catalogue.
      }
      \sstsubsection{
         NEWNAME = \_CHAR (Read)
      }{
         New name of the catalogue.
      }
   }
   \sstdiytopic{
      Example
   }{
      CATRENAME TEST NEWTEST
   }
   \sstbugs{
      None known.
   }
}
\sstroutine{
   CATREPORT
}{
   Produce a catalogue report
}{
   \sstdescription{
      Produces a catalogue report. A simple report is produced with or
      without a header, selecting fields or reporting all fields. The report
      is made either to the screen or to a file ($<$catalogue name$>$.REP),
   }
   \sstusage{
      CATREPORT TEST [HEADER] [SCREEN] [ALLFLDS] [FLDNAMES]
   }
   \sstparameters{
      \sstsubsection{
         INPUT = \_CHAR (Read)
      }{
         Name of the catalogue.
      }
      \sstsubsection{
         HEADER = \_LOGICAL (Read)
      }{
         Add header information to the report T/F?
         [TRUE]
      }
      \sstsubsection{
         SCREEN = \_LOGICAL (Read)
      }{
         Output to the screen (or a file) T/F?
         [TRUE]
      }
      \sstsubsection{
         ALLFLDS = \_LOGICAL (Read)
      }{
         All fields to be reported T/F?
         [TRUE]
      }
      \sstsubsection{
         FLDNAMES = \_CHAR (Read)
      }{
         List of field names to be reported
      }
   }
   \sstexamples{
      \sstexamplesubsection{
         REPORT TEST HEADER=F SCREEN=F ALLFLDS=N FLDNAMES=[RA,DEC,VALUE1]
      }{
         Produce a report with no header, output to a file and selecting
         fields.
      }
   }
   \sstnotes{
      A report to the screen is limited to 80 characters and to file 132
      characters. Excess fields are ignored and a warning issued.
      Requested fields that do not appear in the catalogue are ignored.
   }
   \sstbugs{
      None known.
   }
}
\sstroutine{
   CATSEARCH
}{
   Select entries from a catalogue
}{
   \sstdescription{
      Create a new catalogue that contains entries from a catalogue
      that pass some selection criteria. The criteria should be a legal
      CATPAC Parser expression.
   }
   \sstusage{
      CATSEARCH INPUT OUTPUT CRITERIA
   }
   \sstparameters{
      \sstsubsection{
         INPUT = \_CHAR (Read)
      }{
         Name of the catalogue.
      }
      \sstsubsection{
         OUTPUT = \_CHAR (Read)
      }{
         Name of the catalogue to contain the selected entries.
      }
      \sstsubsection{
         CRITERIA = \_CHAR (Read)
      }{
         Logical expression of the selection criteria.
      }
   }
   \sstnotes{
      REJECT and CATSEARCH complement each other.
   }
   \sstdiytopic{
      Example
   }{
      CATSEARCH TEST SEARCHTEST VALUE2.GT.300.AND.VALUE2.LT.500
   }
   \sstbugs{
      None known.
   }
}
\sstroutine{
   CATSORT
}{
   Create a new catalogue that is sorted on given fields
}{
   \sstdescription{
      Create a new catalogue that is sorted on given fields. The functionality
      of sort is twofold. The first function is to create indexes associated
      with the catalogue that allow efficient searching and joining. The second
      function is a by product of the sort and that is to order a catalogue
      for presentation.

      Consider sorting the data in a telephone directory. For presentation
      purposes sort the data by field SURNAME (Primary field) if several
      entries are found with the same surname order these by ordering on the
      field FIRSTINITIAL (Secondary field) and if entries are found with the
      same surname and first initial order these by ordering on the field
      SECONDINITIAL (Tertiary field). This catalogue would now be presented in
      a useful way. More importantly the system has created an
      index that allows it to perform an efficient search and join in certain
      cases. For example, a request for entries where the SURNAME is BROWN and
      the FIRSTINITIAL is J.

      The order of field names in the SORTFLDS parameter is significant.
      SORTFLDS(1) must contain the primary field, SORTFLDS(2) and
      SORTFLDS(3) contain the secondary and tertiary fields.
      Omitting either the secondary or tertiary position simply indicates
      that there should be no secondary or tertiary ordering.

      The direction of the sort for each field in given in the corresponding
      position of the SORTDIR parameter. TRUE for ascending.
   }
   \sstusage{
      CATSORT INPUT OUTPUT SORTFLDS SORTDIR
   }
   \sstparameters{
      \sstsubsection{
         INPUT = \_CHAR (Read)
      }{
         Name of the catalogue.
      }
      \sstsubsection{
         OUTPUT = \_CHAR (Read)
      }{
         Name of the new sorted catalogue.
      }
      \sstsubsection{
         SORTFLDS = \_CHAR (Read)
      }{
         Names of the Primary, Secondary and Tertiary fields for an index.
      }
      \sstsubsection{
         SORTDIR = \_LOGICAL (Read)
      }{
         Direction of sort TRUE for descending FALSE ascending.
      }
   }
   \sstnotes{
      SORTFLDS and SORTDIR must correspond.
   }
   \sstbugs{
      None known.
   }
}
\sstroutine{
   COPYCAT
}{
   Copy a catalogue
}{
   \sstdescription{
      Copy a catalogue.
   }
   \sstusage{
      COPYCAT INPUT OUTPUT
   }
   \sstparameters{
      \sstsubsection{
         INPUT = \_CHAR (Read)
      }{
         Name of the catalogue.
      }
      \sstsubsection{
         OUTPUT = \_CHAR (Read)
      }{
         New catalogue.
      }
   }
   \sstdiytopic{
      Example
   }{
      COPYCAT TEST NEWTEST
   }
   \sstbugs{
      None known.
   }
}
\sstroutine{
   CORRELATE
}{
   Non-parametric correlation  between fields
}{
   \sstdescription{
      Computes the non-parametric correlation coefficients between
      given numeric fields in a catalogue using Kendall{\tt '}s Tau and
      Spearman{\tt '}s Rho and writes the results to file
      $<$catalogue name$>$.LIS. These coefficients measure the strength of
      any monotonic relation between the two fields. They both take
      values of $+$1.0 for perfect correlation, -1.0 for perfect
      anti-correlation and 0.0 for no correlation, though in general
      the values of the two coefficients are not equal. These
      coefficients are further discussed in the NAG manual (Chapter
      G02).

      If a field contains the Null Value for that field the values
      for that entry are excluded form the correlation. This is a
      casewise treatment of missing values in the terminology of the
      NAG manual. This approach has the disadvantage that some good
      data is discarded, but it ensures that all coefficients are based
      on the same number of data points.
   }
   \sstusage{
      CORRELATE TEST [ALLFLDS] [FLDNAMES]
   }
   \sstparameters{
      \sstsubsection{
         INPUT = \_CHAR (Read)
      }{
         Name of the catalogue.
      }
      \sstsubsection{
         ALLFLDS = \_LOGICAL (Read)
      }{
         All numeric fields to be seleceted T/F?
         [TRUE]
      }
      \sstsubsection{
         FLDNAMES = \_CHAR (Read)
      }{
         List of field names to be correlated.
      }
   }
   \sstexamples{
      \sstexamplesubsection{
         CORRELATE TEST ALLFLDS=N FLDNAMES=[VALUE1,VALUE2]
      }{
         Selecting the fields to be included in the correlation.
      }
   }
   \sstnotes{
      If there are more than a few hundred entries in the catalogue
      the time required to compute the coefficients is not negligible.
      The time required increases in proportion to the number of entries
      and the number of fields being correlated.
   }
   \sstbugs{
      None known.
   }
}
\sstroutine{
   DELCAT
}{
   Deletes a catalogue from the system
}{
   \sstdescription{
      Delete a catalogue from the system.
   }
   \sstusage{
      DELCAT INPUT
   }
   \sstparameters{
      \sstsubsection{
         INPUT = \_CHAR (Read)
      }{
         Name of the catalogue.
      }
   }
   \sstnotes{
      Catalogues not created by the user may be protected and in these
      cases the Delete Catalogue application will have no effect.
   }
   \sstdiytopic{
      Example
   }{
      DELCAT TEST
   }
   \sstbugs{
      None known.
   }
}
\sstroutine{
   DELPARAM
}{
   Delete a Parameter from a Catalogue
}{
   \sstdescription{
      Delete a parameter from a catalogue.
   }
   \sstusage{
      DELPARAM INPUT NAME
   }
   \sstparameters{
      \sstsubsection{
         INPUT = \_CHAR (Read)
      }{
         Name of the catalogue.
      }
      \sstsubsection{
         NAME = \_CHAR (Read)
      }{
         Name of the parameter to be deleted.
      }
   }
   \sstdiytopic{
      Example
   }{
      DELPARAM TEST CLASS
   }
}
\sstroutine{
   DELSORT
}{
   Delete the sort information from a catalogue
}{
   \sstdescription{
      Delete the sort information from a catalogue.
   }
   \sstusage{
      DELSORT INPUT
   }
   \sstparameters{
      \sstsubsection{
         INPUT = \_CHAR (Read)
      }{
         Name of the catalogue.
      }
   }
   \sstdiytopic{
      Example
   }{
      DELSORT TEST
   }
   \sstbugs{
      None known.
   }
}
\sstroutine{
   ENTRIES
}{
   Finds the number of entries in a catalogue
}{
   \sstdescription{
      Finds the number of entries in a catalogue.
   }
   \sstusage{
      ENTRIES INPUT
   }
   \sstparameters{
      \sstsubsection{
         INPUT = \_CHAR (Read)
      }{
         Name of the catalogue.
      }
   }
   \sstdiytopic{
      Example
   }{
      ENTRIES TEST
   }
   \sstbugs{
      None known.
   }
}
\sstroutine{
   FIELDINFO
}{
   Find specific information about a field in a catalogue
}{
   \sstdescription{
      Finds the FORMAT, UNITS, NULLVALUE or COMMENT associated with a
      field in a catalogue. For example find the comment associated
      with the field VALUE1 in the TEST catalogue.
   }
   \sstusage{
      FIELDINFO INPUT NAME INFREQ
   }
   \sstparameters{
      \sstsubsection{
         INPUT = \_CHAR (Read)
      }{
         Name of the catalogue.
      }
      \sstsubsection{
         NAME = \_CHAR (Read)
      }{
         Name of the field whose information is required.
      }
      \sstsubsection{
         INFREQ = \_CHAR (Read)
      }{
         Information required. (FORMAT, UNITS, NULLVALUE or COMMENT)
      }
   }
   \sstdiytopic{
      Example
   }{
      FIELDINFO TEST VALUE1 COMMENT
   }
   \sstbugs{
      None known.
   }
}
\sstroutine{
   FIELDS
}{
   Find the number and names of fields in a catalogue
}{
   \sstdescription{
      Find the number of fields and the names of thoses fields
      in a catalogue.
   }
   \sstusage{
      FIELDS INPUT
   }
   \sstparameters{
      \sstsubsection{
         INPUT = \_CHAR (Read)
      }{
         Name of the catalogue.
      }
   }
   \sstdiytopic{
      Example
   }{
      FIELDS TEST
   }
   \sstbugs{
      None known.
   }
}
\sstroutine{
   FK425
}{
   Convert FK4 coordinates to FK5
}{
   \sstdescription{
      Create a catalogue containing new fields for the Right Ascension,
      Declination, Parallax, Radial velocity and proper motions after a
      conversion has been made from the FK4 system coordinates. The
      new fields are calculated using SLA\_FK425. See SUN 67

      Conversion from Besselian epoch 1950.0 to Julian epoch 2000.0 only
      is provided. Proper motion corrections can be made using PROPERM.

      Proper motions should be given in sec/yr and arcsecs/yr
      Parallax should be given in arcseconds.
      Radial velocity should be given in km/sec ($+$ve if receeding)

      If necessary use UPDATE to convert fields into the appropriate units.
   }
   \sstusage{
      FK425 INPUT OUTPUT RAFK4 DECFK4 RAPMFK4 DECPMFK4 PARLXFK4 RADVELFK4
      RAFK5
   \newline
      DECFK5 RAPMFK5 DECPMFK5 PARLXFK5 RADELFK5
   }
   \sstparameters{
      \sstsubsection{
         INPUT = \_CHAR (Read)
      }{
         Name of the catalogue.
      }
      \sstsubsection{
         OUTPUT = \_CHAR (Read)
      }{
         Name of the output catalogue.
      }
      \sstsubsection{
         RAFK4 = \_CHAR (Read)
      }{
         Name of the RA field in FK4 system.
      }
      \sstsubsection{
         DECFK4 = \_CHAR (Read)
      }{
         Name of the DEC field in FK4 system.
      }
      \sstsubsection{
         RAPMFK4 = \_CHAR (Read)
      }{
         Name of the RA proper motion field in FK4 system.
      }
      \sstsubsection{
         DECPMFK4 = \_CHAR (Read)
      }{
         Name of the DEC proper motion field in FK4 system.
      }
      \sstsubsection{
         PARLXFK4 = \_CHAR (Read)
      }{
         Name of the parallax field in FK4 system.
      }
      \sstsubsection{
         RADVELFK4 = \_CHAR (Read)
      }{
         Name of the radial velocity field in FK4 system.
      }
      \sstsubsection{
         RAFK5 = \_CHAR (Read)
      }{
         Name of the RA field in FK5 system.
      }
      \sstsubsection{
         DECFK5 = \_CHAR (Read)
      }{
         Name of the DEC field in FK5 system.
      }
      \sstsubsection{
         RAPMFK5 = \_CHAR (Read)
      }{
         Name of the RA proper motion field in FK5 system.
      }
      \sstsubsection{
         DECPMFK5 = \_CHAR (Read)
      }{
         Name of the DEC proper motion field in FK5 system.
      }
      \sstsubsection{
         PARLXFK5 = \_CHAR (Read)
      }{
         Name of the parallax field in FK5 system.
      }
      \sstsubsection{
         RADVELFK5 = \_CHAR (Read)
      }{
         Name of the radial velocity field in FK5 system.
      }
   }
   \sstnotes{
      This application creates a new catalogue that contains extra fields for
      Right Ascension, Declination, Parallax, Radial velocity and proper
      motions after a conversion to FK5 system.
      The naming of these new fields can lead to confusion. Traditionally
      the field names RA and DEC are used for the Right Ascension and
      Declination fields in the catalogue. It is stongly suggested that the
      field names of Right Ascension and Declination in the new system (FK5)
      take the form RAJ, DECJ for RA Julian and DEC Julian. In the same way
      PARLX would become PARLXJ and RADVEL RADVELJ etc. You may go on to
      rename the field RA to RAB and DEC to DECB and RAJ to RA and DECJ to DEC
      using the UPFIELD application but you must also then update the Parallax,
      Radial velocity and proper motions and the catalogues Equinox parameter.

      Care should also be taken when renaming fields. Renaming RAJ to RA
      before renaming RA to RAB, in the above example, would result in two
      fields named RA in the same catalogue.
   }
   \sstdiytopic{
      Example
   }{
      FK425 TEST TESTFK5 RA DEC RA\_PM DEC\_PM PARALLAX RAD\_VEL RAJ DECJ
      PARALLAXJ RADVELJ
   }
   \sstbugs{
      None known.
   }
}
\sstroutine{
   FK45Z
}{
   Convert FK4 coordinates to FK5
}{
   \sstdescription{
      Create a catalogue containing new fields for the Right Ascension
      and Declination. The new fields are calculated using SLA\_FK45Z.
      See SUN 67

      Conversion from Besselian epoch 1950.0 to Julian epoch 2000.0 only
      is provided.

      If necessary use UPDATE to convert fields into the appropriate units.
   }
   \sstusage{
      FK45Z INPUT OUTPUT RAFK4 DECFK4 BEPOCH RAFK5 DECFK5
   }
   \sstparameters{
      \sstsubsection{
         INPUT = \_CHAR (Read)
      }{
         Name of the catalogue.
      }
      \sstsubsection{
         OUTPUT = \_CHAR (Read)
      }{
         Name of the output catalogue.
      }
      \sstsubsection{
         RAFK4 = \_CHAR (Read)
      }{
         Name of the RA field in FK4 system.
      }
      \sstsubsection{
         DECFK4 = \_CHAR (Read)
      }{
         Name of the DEC field in FK4 system.
      }
      \sstsubsection{
         BEBOCH = \_REAL (Read)
      }{
         Epoch
      }
      \sstsubsection{
         RAFK5 = \_CHAR (Read)
      }{
         Name of the RA field in FK5 system.
      }
      \sstsubsection{
         DECFK5 = \_CHAR (Read)
      }{
         Name of the DEC field in FK5 system.
      }
   }
   \sstnotes{
      This application creates a new catalogue that contains extra fields for
      Right Ascension, Declination
      The naming of these new fields can lead to confusion. Traditionally
      the field names RA and DEC are used for the Right Ascension and
      Declination fields in the catalogue. It is stongly suggested that the
      field names of Right Ascension and Declination in the new system (FK5)
      take the form RAJ, DECJ for RA Julian and DEC Julian. You may go on to
      rename the field RA to RAB and DEC to DECB and RAJ to RA and DECJ to DEC
      using the UPFIELD application but you must also then update the Parallax,
      Radial velocity and proper motions and the catalogues Equinox parameter.

      Care should also be taken when renaming fields. Renaming RAJ to RA
      before renaming RA to RAB, in the above example, would result in two
      fields named RA in the same catalogue.
   }
   \sstdiytopic{
      Example
   }{
      FK45Z TEST TESTFK5 RA DEC BEPOCH RAJ DECJ
   }
   \sstbugs{
      None known.
   }
}
\sstroutine{
   FK524
}{
   Convert FK5 coordinates to FK4
}{
   \sstdescription{
      Create a catalogue containing new fields for the Right Ascension,
      Declination, Parallax, Radial velocity and proper motions after a
      conversion has been made from the FK5 system coordinates. The
      new fields are calculated using SLA\_FK524. See SUN 67

      Conversion from Julian epoch 2000.0 to Besselian epoch 1950.0 only
      is provided. Proper motion corrections can be made using PROPERM.

      Proper motions should be given in sec/yr and arcsecs/yr
      Parallax should be given in arcseconds.
      Radial velocity should be given in km/sec ($+$ve if receeding)

      If necessary use UPDATE to convert fields into the appropriate units.
   }
   \sstusage{
      FK524 INPUT OUTPUT RAFK5 DECFK5 RAPMFK5 DECPMFK5 PARLXFK5 RADVELFK5
      \newline
      RAFK4 DECFK4 RAPMFK4 DECPMFK4 PARLXFK4 RADVELFK4
   }
   \sstparameters{
      \sstsubsection{
         INPUT = \_CHAR (Read)
      }{
         Name of the catalogue.
      }
      \sstsubsection{
         OUTPUT = \_CHAR (Read)
      }{
         Name of the output catalogue.
      }
      \sstsubsection{
         RAFK5 = \_CHAR (Read)
      }{
         Name of the RA field in FK5 system.
      }
      \sstsubsection{
         DECFK5 = \_CHAR (Read)
      }{
         Name of the DEC field in FK5 system.
      }
      \sstsubsection{
         RAPMFK5 = \_CHAR (Read)
      }{
         Name of the RA proper motion field in FK5 system.
      }
      \sstsubsection{
         DECPMFK5 = \_CHAR (Read)
      }{
         Name of the DEC proper motion field in FK5 system.
      }
      \sstsubsection{
         PARLXFK5 = \_CHAR (Read)
      }{
         Name of the parallax field in FK5 system.
      }
      \sstsubsection{
         RADVALFK5 = \_CHAR (Read)
      }{
         Name of the radial velocity field in FK5 system.
      }
      \sstsubsection{
         RAFK4 = \_CHAR (Read)
      }{
         Name of the RA field in FK4 system.
      }
      \sstsubsection{
         DECFK4 = \_CHAR (Read)
      }{
         Name of the DEC field in FK4 system.
      }
      \sstsubsection{
         RAPMFK4 = \_CHAR (Read)
      }{
         Name of the RA proper motion field in FK4 system.
      }
      \sstsubsection{
         DECPMFK4 = \_CHAR (Read)
      }{
         Name of the DEC proper motion field in FK4 system.
      }
      \sstsubsection{
         PARLXFK4 = \_CHAR (Read)
      }{
         Name of the parallax field in FK4 system.
      }
      \sstsubsection{
         RADVALFK4 = \_CHAR (Read)
      }{
         Name of the radial velocity field in FK4 system.
      }
   }
   \sstnotes{
      This application creates a new catalogue that contains extra fields for
      Right Ascension, Declination, Parallax, Radial velocity and proper
      motions after a conversion to FK4 system.
      The naming of these new fields can lead to confusion. Traditionally
      the field names RA and DEC are used for the Right Ascension and
      Declination fields in the catalogue. It is stongly suggested that the
      field names of Right Ascension and Declination in the new system (FK4)
      take the form RAB, DECB for RA Besselian and DEC Besselian. In the same
      way PARLX would become PARLXB and RADVEL RADVELB etc. You may go on to
      rename the field RA to RAJ and DEC to DECJ and RAB to RA and DECB to DEC
      using the UPFIELD application but you must also then update the Parallax,
      Radial velocity and proper motions and the catalogues Equinox parameter.

      Care should also be taken when renaming fields. Renaming RAB to RA
      before renaming RA to RAJ, in the above example, would result in two
      fields named RA in the same catalogue.
   }
   \sstdiytopic{
      Example
   }{
      FK524 TEST TESTFK4 RA DEC RAPM DECPM PARALLAX RAD\_VEL RAB DECB
      RAPMB
      \newline
      DECPMB PARALLAXB RADVELB
   }
   \sstbugs{
      None known.
   }
}
\sstroutine{
   FK54Z
}{
   Convert FK5 coordinates to FK4
}{
   \sstdescription{
      Create a catalogue containing new fields for the Right Ascension
      and Declination. The new fields are calculated using SLA\_FK54Z.
      See SUN 67

      Conversion from Julian epoch 2000.0 to Besselian epoch 1950.0 only
      is provided.

      If necessary use UPDATE to convert fields into the appropriate units.
   }
   \sstusage{
      FK54Z INPUT OUTPUT RAFK5 DECFK5 BEPOCH RAFK4 DECFK4
   }
   \sstparameters{
      \sstsubsection{
         INPUT = \_CHAR (Read)
      }{
         Name of the catalogue.
      }
      \sstsubsection{
         OUTPUT = \_CHAR (Read)
      }{
         Name of the output catalogue.
      }
      \sstsubsection{
         RAFK5 = \_CHAR (Read)
      }{
         Name of the RA field in FK5 system.
      }
      \sstsubsection{
         DECFK5 = \_CHAR (Read)
      }{
         Name of the DEC field in FK5 system.
      }
      \sstsubsection{
         BEBOCH = \_REAL (Read)
      }{
         Epoch
      }
      \sstsubsection{
         RAFK4 = \_CHAR (Read)
      }{
         Name of the RA field in FK4 system.
      }
      \sstsubsection{
         DECFK4 = \_CHAR (Read)
      }{
         Name of the DEC field in FK4 system.
      }
   }
   \sstnotes{
      This application creates a new catalogue that contains extra fields for
      Right Ascension, Declination
      The naming of these new fields can lead to confusion. Traditionally
      the field names RA and DEC are used for the Right Ascension and
      Declination fields in the catalogue. It is stongly suggested that the
      field names of Right Ascension and Declination in the new system (FK5)
      take the form RAB, DECB for RA Besslian and DEC Besslian. You may go on to
      rename the field RA to RAJ and DEC to DECJ and RAB to RA and DECB to DEC
      using the UPFIELD application but you must also then update the Parallax,
      Radial velocity and proper motions and the catalogues Equinox parameter.

      Care should also be taken when renaming fields. Renaming RAB to RA
      before renaming RA to RAJ, in the above example, would result in two
      fields named RA in the same catalogue.
   }
   \sstdiytopic{
      Example
   }{
      FK54Z TEST TESTFK4 RA DEC BEPOCH RAB DECB
   }
   \sstbugs{
      None known.
   }
}
\sstroutine{
   GLOBALS
}{
   Displays the values of the CATPAC global parameters
}{
   \sstdescription{
      This procedure lists the meanings and values of the CATPAC global
      parameters.  If a global parameter does not have a value, the
      string {\tt "}$<$undefined$>${\tt "} is substituted where the value would have been
      written.
   }
   \sstusage{
      GLOBALS
   }
   \sstbugs{
      None known.
   }
}
\sstroutine{
   JOIN
}{
   JOIN two catalogues
}{
   \sstdescription{
      Create a new catalogue by joining two catalogues. The effect of the join
      is as follows. Consider a large catalogue that contains all the fields
      from the INPUT1 catalogue and all the fields from the INPUT2 catalogue.
      Into this catalogue put an entry for each combination of entries in
      catalogues INPUT1 and INPUT2. The resulting catalogue will have N$*$M
      entries where N is the number of entries in the INPUT1 catalogue and
      M the number in the INPUT2 catalogue. Now search this catalogue for
      those entries that satisfy the given expression.

      Another way of looking at join is to say. Take every entry in turn
      from catalogue INPUT1. Match this entry against every entry in
      catalogue INPUT2 and if the EXPRESSion is satisfied combine both entries
      to write to a new catalogue.

      The expression should be a legal CATPAC Parser expression.

      Field names in the expression must be unique so append an {\tt '}\_{\tt '} followed
      by the first four characters of the catalogue name. 
      \newline
      Eg. RA\_IRPS, DEC\_YALE
      VALUE2\_CAT1.GT.300.AND.VALUE2\_CAT2.LT.500
   }
   \sstusage{
      JOIN INPUT1 INPUT2 OUTPUT EXPRESS
   }
   \sstparameters{
      \sstsubsection{
         INPUT1 = \_CHAR (Read)
      }{
         Name of the first input catalogue.
      }
      \sstsubsection{
         INPUT2 = \_CHAR (Read)
      }{
         Name of the second input catalogue.
      }
      \sstsubsection{
         OUTPUT = \_CHAR (Read)
      }{
         Name of the catalogue to contain the merged entries.
      }
      \sstsubsection{
         EXPRESS = \_CHAR (Read)
      }{
         The join expression.
      }
   }
   \sstdiytopic{
      Example
   }{
      JOIN CAT1 CAT2 JOINCAT VALUE2\_CAT1.GT.300.AND.VALUE2\_CAT2.LT.500
   }\\{
      JOIN CAT1 CAT2 JOINCAT GREAT\_CIRCLE(RA\_CAT1,DEC\_CAT1,RA\_CAT2,DEC\_CAT2)
   }{
      .LT.CONVERT({\tt "}ARCSEC{\tt "},{\tt "}56{\tt "})
   }
   \sstbugs{
      None known.
   }
}
\sstroutine{
   LINCOR
}{
   Linear correlation between fields
}{
   \sstdescription{
      Computes the Pearson product-moment linear correlation
      coefficients between the given numeric fields in a catalogue and
      writes the results to file $<$catalogue name$>$.LIS. The coefficient
      takes the value of $+$1.0 for perfect correlation, -1.0 for
      perfect anti-correlation and 0.0 for no correlation. These
      coefficients are further discussed in the NAG manual (Chapter
      G02).
      If a field contains the Null Value for that field the values for
      that entry are excluded form the correlation. This is a casewise
      treatment of missing values in the terminology of the NAG manual.
      This approach has the disadvantage that some good data is
      discarded, but it ensures that all coefficients are based on the
      same number of data points.
   }
   \sstusage{
      LINCOR TEST [ALLFLDS] [FLDNAMES]
   }
   \sstparameters{
      \sstsubsection{
         INPUT = \_CHAR (Read)
      }{
         Name of the catalogue.
      }
      \sstsubsection{
         ALLFLDS = \_LOGICAL (Read)
      }{
         All numeric fields to be selected T/F?
         [TRUE]
      }
      \sstsubsection{
         FLDNAMES = \_CHAR (Read)
      }{
         List of field names to be correlated.
         Prompted for only if required.
      }
   }
   \sstexamples{
      \sstexamplesubsection{
         LINCOR TEST ALLFLDS=N FLDNAMES=[VALUE1,VALUE2]
      }{
         Selecting the fields to be included in the correlation.
      }
   }
   \sstnotes{
      If there are more than a few hundred entries in the catalogue
      the time required to compute the coefficients is not negligible.
      The time required increases in proportion to the number of entries
      and the number of fields being correlated.
   }
   \sstbugs{
      None known.
   }
}
   \goodbreak
   \rule{\textwidth}{0.5mm}
   \vspace{-7ex}
   \newline
   \settowidth{\sstbannerlength}{{\Large {\bf LISTIN}}}
   \setlength{\sstcaptionlength}{\textwidth}
   \addtolength{\sstbannerlength}{0.5em}
   \addtolength{\sstcaptionlength}{-2.0\sstbannerlength}
   \addtolength{\sstcaptionlength}{-4.45pt}
   \parbox[t]{\sstbannerlength}{\flushleft{\Large {\bf LISTIN}}}
   \parbox[t]{\sstcaptionlength}{\center
      {\Large    Create a catalogue from a free format ASCII file}}
   \parbox[t]{\sstbannerlength}{\flushright{\Large {\bf LISTIN}}}
   \begin{description}
   \sstdescription{
      Create a new catalogue that contains the data from a free format ASCII
      file. LISTIN allows you to put your own data into a catalogue. The
      data must be preceded by two lines of information. The first line
      should contain the destination field names separated by spaces. The
      second line should contain the formats to be associated with the fields
      again separated by spaces. When the field is a character string or in
      sexagesimal format the length of the field is taken from the format.
      The full range of CATPAC sexagesimal formats are available including
      HH MM SS, SDD MM SS, DEGREES, HHMMSS etc.
      Subsequent lines contain the free format data.
      Remember that the catalogue will be created with no parameters and
      dummy values for the UNITs, NULLVALUES, and COMMENTS of the fields.
      These can be added using applications such as ADDPARAM and UPFIELD
      A typical file may look like
   }
   \end{description}

\begin{verbatim} 

      NAME RA DEC VAL1 VAL2
       A5 HH MM SS    SDD MM SS      F10.4 F10.3
       STAR1      12 30 00   +44 30 00  1223.78000  1348.76000
       STAR2  12 35 00     +43 45 00  1472.16000  1302.45000
       STAR3  12 00 00   +45 00 00  1624.63000  1234.34000
         STAR4  12 22 00  +44 40 00  1558.32000  1419.83000
        STAR5  12 14 00   +43 52 00   0000.00000  1285.77000
       STAR6  12 06 00   +43 21 00        1498.23000  1379.48000
       STAR7 12 18 00   +44 08 00   1604.32000  0000.00000
         STAR8   12 26 00   +43 58 00  1273.72000  1230.54000
       STAR9       12 02 00   +44 13 00  1508.36000  1267.44000

\end{verbatim}
\begin{description}
   \sstusage{
      LISTIN INPUT DATAFILE
   }
   \sstparameters{
      \sstsubsection{
         INPUT = \_CHAR (Read)
      }{
         Name of the catalogue to be created.
      }
      \sstsubsection{
         DATAFILE = \_CHAR (Read)
      }{
         Name of the file containing the ascii data.
      }
   }
   \sstdiytopic{
      Example
   }{
      LISTIN TEST TESTDATA.DAT
   }
   \sstbugs{
      None known.
   }
   \end{description}
\sstroutine{
   LITTLEBIG
}{
   Extract entries with largest or smallest values of a given field
}{
   \sstdescription{
      Create a new catalogue that contains a user selected number of
      entries. The entries are those with the littlest, or biggest, values
      in a user selected field. For example create a new catalogue that
      contains the 3 brightest objects in a catalogue TEST.
      By using the REJECT option a second catalogue can be
      created that contains those entries that were not selected.
   }
   \sstusage{
      LITTLEBIG INPUT OUTPUT FIELD NUMSEL BIGGEST [REJECT] [OUTREJECT]
   }
   \sstparameters{
      \sstsubsection{
         INPUT = \_CHAR (Read)
      }{
         Name of the catalogue.
      }
      \sstsubsection{
         OUTPUT = \_CHAR (Read)
      }{
         Name of the catalogue to be created.
      }
      \sstsubsection{
         FIELD = \_CHAR (Read)
      }{
         Name of the field on which the selection is to be made.
      }
      \sstsubsection{
         NUMSEL = \_INTEGER (Read)
      }{
         Number of entries to be selected.
      }
      \sstsubsection{
         BIGGEST = \_LOGICAL (Read)
      }{
         {\tt '}T{\tt '} To select the entries with the biggest values in the
         selected field. {\tt '}F{\tt '} for the entries with the smallest values.
      }
      \sstsubsection{
         REJECT = \_LOGICAL (Read)
      }{
         Do you require a rejects catalogue (T/F)?
         [FALSE]
      }
      \sstsubsection{
         OUTREJECT = \_CHAR (Read)
      }{
         Name of the catalogue to contain the rejected entries.
      }
   }
   \sstbugs{
      None known.
   }
}
\sstroutine{
   MERGE
}{
   Merge two catalogues
}{
   \sstdescription{
      Create a new catalogue by merging two catalogues. If the catalogues were
      ASCII files merging two catalogue would be the same as appending the
      two ASCII files to create a new file, providing the data is aligned
      correctly. To merge two catalogues use the MERGELFDS parameter to select
      the fields in the first catalogue and the fields in the second catalogue
      that are to be merged to create fields in the new third catalogue.
      MERGEFLDS is repeatedly prompted and is terminated with a !.
      So you may have two
      catalogues one containing fields DATE, TIME and RESULT and the other DATE,
      TIME and VALUE. To merge these catalogues associate (DATE, DATE, DATE
      using MEGERFLDS=[DATE,DATE,DATE]) and (TIME, TIME, TIME using
      MERGEFLDS=[TIME, TIME, TIME]) and (RESULT, VALUE, READING using
      MEGERFLDS=[RESULT, VALUE, READING]. Finish with MERGEFLDS=!

      An alternative to entering the merge fields in interactive mode is to
      use parameter FROMFILE=T to read the merge fields from a file given
      by the MERGEFILE parameter. The file should contain three fieled names
      on each line separated by spaces. The file for the above example would
      be.

      DATE DATE DATE
      TIME TIME TIME
      RESULT VALUE READING

      The output catalogue will contain
      data from every entry in the first catalogue and every entry in the
      second. The merged fields must be of the same type.
      The resulting catalogue has fields TIME, DATE and READING.
      MERGEFLDFS will accept [RESULT, NULL, READING] in which case the field
      RESULT in the first catalogue has no field to be merged with it in the
      second catalogue. The null value of the field RESULT will be inserted
      when data is being read from the second catalogue.

      Each field in the new catalogue requires a format, units, null value and
      comment. These are taken from the fields in the first catalogue that were
      used in the merge.

      The new catalogue contains no parameters. Use ADDPARAM
      to add parameters to the catalogue.
   }
   \sstusage{
      MERGE INPUT1 INPUT2 OUTPUT MERGEFLDS [FROMFILE] [MERGEFILE]
   }
   \sstparameters{
      \sstsubsection{
         INPUT1 = \_CHAR (Read)
      }{
         Name of the first input catalogue.
      }
      \sstsubsection{
         INPUT2 = \_CHAR (Read)
      }{
         Name of the second input catalogue.
      }
      \sstsubsection{
         OUTPUT = \_CHAR (Read)
      }{
         Name of the catalogue to contain the merged entries.
      }
      \sstsubsection{
         FROMFILE = \_LOGICAL (Read)
      }{
         Names of the two fields to be merged and the field to be created.
         [FALSE]
      }
      \sstsubsection{
         MERGFILE = \_CHAR (Read)
      }{
         Name of the file containing the merge fields.
      }
      \sstsubsection{
         MERGFLDS = \_CHAR (Read)
      }{
         Names of the two fields to be merged and the field to be created.
         [Repeatedly prompted for terminate with !]
      }
   }
   \sstnotes{
      MERGE is an interactive application.
   }
   \sstdiytopic{
      Example
   }{
      MERGE TEST1 TEST2 MERTEST FROMFILE=T MERGEFILE=MERGDATA.DAT
   }
   \sstbugs{
      None known.
   }
}
\sstroutine{
   PARAMINFO
}{
   Find specific information about a parameter in a catalogue
}{
   \sstdescription{
      Finds the FORMAT, VALUE or COMMENT associated with a field in a
      catalogue. For example find the VALUE associated with the
      parameter AUTHOR in the TEST catalogue.
   }
   \sstusage{
      PARAMINFO TEST AUTHOR VALUE
   }
   \sstparameters{
      \sstsubsection{
         INPUT = \_CHAR (Read)
      }{
         Name of the catalogue.
      }
      \sstsubsection{
         NAME = \_CHAR (Read)
      }{
         Name of parameter.
      }
      \sstsubsection{
         INFREQ = \_CHAR (Read)
      }{
         Information required.(FORMAT, VALUE or COMMENT)
      }
   }
   \sstbugs{
      None known.
   }
}
\sstroutine{
   PARAMS
}{
   Find the number and names of parameters in a catalogue
}{
   \sstdescription{
      Find the number of parameters and the names of those parameters in
      a catalogue.
   }
   \sstusage{
      PARAMS INPUT
   }
   \sstparameters{
      \sstsubsection{
         INPUT = \_CHAR (Read)
      }{
         Name of the catalogue.
      }
   }
   \sstdiytopic{
      Example
   }{
      PARAMS TEST
   }
   \sstbugs{
      None known.
   }
}
\sstroutine{
   POLYGON
}{
   Create a polygon definition for use with WITHIN
}{
   \sstdescription{
       Creates a polygon definition for use by the WITHIN
       application. WITHIN is an application that searches a
       catalogue for objects  within a polygon in two dimensional space.
       The space is defined by two fields in the catalogue. See
       WITHIN for further details.
       POLYGON defaults to being an interactive application. The co-ordinates
       for the corners of the polygon are repeatedly prompted for
       allowing a many cornered polygon shape to be built up. The
       polygon need not be closed as the application automatically
       inserts the last side of the polygon from the current position
       back to the start position.
       Using the FROMFILE parameter allows the coordinate values to be read
       from a file instead of interactively. This can be usefull if a large
       number of coordinate values are being used. The X-axis value should be
       given before the Y-axis, The X and Y coordinates should be given on the
       same line and subsequent coordinate values should be given on subsequent
       lines.

       For Example

       20   20
      20    50
         75   55
      80   15
   }
   \sstusage{
      POLYGON TESTPOLY
   }
   \sstparameters{
      \sstsubsection{
         OUTPUT = \_CHAR (Read)
      }{
         Name of the catalogue to contain the polygon definition.
      }
      \sstsubsection{
         FROMFILE = \_LOGICAL (Read)
      }{
         TRUE if the coordinate values are to be taken from a file.
         [FALSE]
      }
      \sstsubsection{
         POLYFILE = \_CHAR (Read)
      }{
         Name of the file containing the coordinate values.
      }
      \sstsubsection{
         XCOORD = \_REAL (Read)
      }{
         X Co-ordinate of polygon corner.
         [Repeatedly prompted for terminate polygon with !]
      }
      \sstsubsection{
         YCOORD = \_REAL (Read)
      }{
         Y Co-ordinate of polygon corner.
         [Repeatedly prompted for]
      }
   }
   \sstnotes{
      POLYGON is, by default, an interactive application.
   }
   \sstbugs{
      None known.
   }
}
\sstroutine{
   PROPERM
}{
   Apply proper motion correction to a catalogue
}{
   \sstdescription{
      Create a catalogue containing new fields for the new Right Ascension and
      Declination after the correction has been made for proper motion.
      Calculated using SLA\_PM. See SUN 67.

      Proper motions should be given in secs/yr and arcsec/yr.
      Parallax should be given in arcseconds.
      Radial velocity should be given in km/sec ($+$ve if receeding)

      If necessary use UPDATE to convert fields into the appropriate units.
   }
   \sstusage{
      PROPERM INPUT OUTPUT RAEP0 DECEP0 RAPM DECPM PARLLAX RADVEL RAEP1
      DECEP1
      \newline
      EP0 EP1
   }
   \sstparameters{
      \sstsubsection{
         INPUT = \_CHAR (Read)
      }{
         Name of the catalogue.
      }
      \sstsubsection{
         OUTPUT = \_CHAR (Read)
      }{
         Name of the output catalogue.
      }
      \sstsubsection{
         RAEP0 = \_CHAR (Read)
      }{
         Name of the RA field at epoch 0.
      }
      \sstsubsection{
         DECEP0 = \_CHAR (Read)
      }{
         Name of the DEC field at epoch 0.
      }
      \sstsubsection{
         RAPM = \_CHAR (Read)
      }{
         Name of the RA proper motion field.
      }
      \sstsubsection{
         DECPM = \_CHAR (Read)
      }{
         Name of the DEC proper motion field.
      }
      \sstsubsection{
         PARALLAX = \_CHAR (Read)
      }{
         Name of the parallax field.
      }
      \sstsubsection{
         RADVEL = \_CHAR (Read)
      }{
         Name of the radial velocity field.
      }
      \sstsubsection{
         RAEP1 = \_CHAR (Read)
      }{
         Name of the RA field at epoch 1.
      }
      \sstsubsection{
         DECEP1 = \_CHAR (Read)
      }{
         Name of the DEC field at epoch 1.
      }
      \sstsubsection{
         EP0 = \_REAL (Read)
      }{
         Start Epoch.
      }
      \sstsubsection{
         EP1 = \_REAL (Read)
      }{
         End epoch.
      }
   }
   \sstnotes{
      This application creates a new catalogue that contains extra fields for
      Right Ascension and Declination after a proper motion correction.
      The naming of these new fields can lead to confusion. Traditionally
      the field names RA and DEC are used for the Right Ascension and
      Declination fields in the catalogue. It is stongly suggested that the
      field names of Right Ascension and Declination at a new epoch take the
      form RA1970, DEC1970. You may go on to rename the field RA to RA1950 and
      DEC to DEC1950 and RA1970 to RA and DEC1970 to DEC using the UPFIELD
      application but you must also then update the catalogues Epoch parameter
      to 1970.

      Care should also be taken when renaming fields. Renaming RA1970 to RA
      before renaming RA to RA1950, in the above example, would result in two
      fields named RA in the same catalogue.
   }
   \sstdiytopic{
      Example
   }{
      PROPERM TEST TEST1970 RA DEC RA\_PM DEC\_PM PARALLAX RAD\_VEL 1950 1970
      RA1970 DEC1970
   }
   \sstbugs{
      None known.
   }
}
\sstroutine{
   REJECT
}{
   Select rejected entries from a catalogue
}{
   \sstdescription{
      In most cases it is more convenient to SEARCH a catalogue
      for entries that obey some critera but occasionally it is more
      convenient to select entries that are REJECTed by the criteria.
      REJECT does this and creates a new catalogue to contain the
      rejected entries. The criteria should be a legal CATPAC Parser
      expression.
   }
   \sstusage{
      REJECT INPUT OUTPUT CRITERIA
   }
   \sstparameters{
      \sstsubsection{
         INPUT = \_CHAR (Read)
      }{
         Name of the catalogue.
      }
      \sstsubsection{
         OUTPUT = \_CHAR (Read)
      }{
         Name of the catalogue to contain the rejected entries.
      }
      \sstsubsection{
         CRITERIA = \_CHAR (Read)
      }{
         Logical expression of the reject criteria.
      }
   }
   \sstnotes{
      REJECT and SEARCH complement each other.
   }
   \sstdiytopic{
      Example
   }{
      REJECT TEST REJTEST VALUE1.GT.300.AND.VALUE1.LT.500
   }
   \sstbugs{
      None known.
   }
}
\sstroutine{
   SAMPLE
}{
   Select every Nth entry from a catalogue
}{
   \sstdescription{
      Sample a catalogue at frequency N creating a new catalogue to
      contain the selected entries. By using the REJECT option a second
      catalogue can be created that contains those entries that were not
      selected.
   }
   \sstusage{
      SAMPLE INPUT OUTPUT FREQUENCY [REJECT] [OUTREJECT]
   }
   \sstparameters{
      \sstsubsection{
         INPUT = \_CHAR (Read)
      }{
         Name of the catalogue.
      }
      \sstsubsection{
         OUTPUT = \_CHAR (Read)
      }{
         Name of the catalogue to contain the sampled entries.
      }
      \sstsubsection{
         FREQUENCY = \_INTEGER (Read)
      }{
         Sample frequency N
      }
      \sstsubsection{
         REJECT = \_LOGICAL (Read)
      }{
         Do you require a rejects catalogue (T/F)?
         [FALSE]
      }
      \sstsubsection{
         OUTREJECT = \_CHAR (Read)
      }{
         Name of the catalogue to contain the rejected entries.
      }
   }
   \sstdiytopic{
      Example
   }{
      SAMPLE TEST SAMPTEST 20
   }
   \sstbugs{
      None known.
   }
}
\sstroutine{
   SELECTFLDS
}{
   Create a new catalogue containing only selected fields
}{
   \sstdescription{
      Create a new catalogue that contains only selected fields of
      the input catalogue.
   }
   \sstusage{
      SELECTFLDS INPUT OUTPUT FLDNAMES
   }
   \sstparameters{
      \sstsubsection{
         INPUT = \_CHAR (Read)
      }{
         Name of the catalogue.
      }
      \sstsubsection{
         OUTPUT = \_CHAR (Read)
      }{
         Name of the new catalogue.
      }
      \sstsubsection{
         FLDNAMES = \_CHAR (Read)
      }{
         List of field names to be included in the new catalogue.
      }
   }
   \sstnotes{
      Requested fields that do not appear in the catalogue are ignored.
   }
   \sstdiytopic{
      Example
   }{
      SELECTFLDS TEST SELTEST [NAME,VALUE1,VALUE2]
   }
   \sstbugs{
      None known.
   }
}
\sstroutine{
   SORTFLDS
}{
   Get the sort information from a catalogue
}{
   \sstdescription{
      Get the sort information from a catalogue. Get the number of indices and
      the names of the fields and the direction of each sort field associated
      with each index.
   }
   \sstusage{
      SORTFLDS INPUT
   }
   \sstparameters{
      \sstsubsection{
         INPUT = \_CHAR (Read)
      }{
         Name of the catalogue.
      }
   }
   \sstdiytopic{
      Example
   }{
      SORTFLDS TEST
   }
   \sstbugs{
      None known.
   }
}
\sstroutine{
   UPDATE
}{
   Update a field in a catalogue
}{
   \sstdescription{
      Updates the value of a field for each entry in a catalogue using
      a user defined CATPAC Parser expression. The format, units, nullvalue
      and comment for the field can be changed if required.
   }
   \sstusage{
      UPDATE INPUT NAME EXPRESSION [FORMAT] [UNITS] [NULLVALUE] [COMMENT]
   }
   \sstparameters{
      \sstsubsection{
         INPUT = \_CHAR (Read)
      }{
         Name of the catalogue.
      }
      \sstsubsection{
         NAME = \_CHAR (Read)
      }{
         Name of the field to be updated.
      }
      \sstsubsection{
         EXPRESSION = \_CHAR (Read)
      }{
         Expression to be evaluated.
      }
      \sstsubsection{
         FORMAT = \_CHAR (Read)
      }{
         Format of the updated field.
         [UNCHANGED]
      }
      \sstsubsection{
         UNITS = \_CHAR (Read)
      }{
         Units of the updated field.
         [UNCHANGED]
      }
      \sstsubsection{
         NULLVALUE = \_CHAR (Read)
      }{
         Null value of the updated field.
         [UNCHANGED]
      }
      \sstsubsection{
         COMMENT = \_CHAR (Read)
      }{
         Comment of the updated field.
         [UNCHANGED]
      }
   }
   \sstdiytopic{
      Example
   }{
      UPDATE TEST VALUE1 VALUE2$*$2
   }
   \sstbugs{
      None known.
   }
}
\sstroutine{
   UPFIELD
}{
   Update Field information in a Catalogue
}{
   \sstdescription{
      Update the information associated with a field in a catalogue. The
      field has a name, format, unit, null value and comment. Any of these
      can be updated by setting the relevant flag.
   }
   \sstusage{
      UPFIELD INUPT NAME [NAMEFLG] [NEWNAME] [FORMFLG] [FORMAT]
      [UNITFLG]
      \newline
      [UNITS] [NULLFLG] [NULLVAL] [COMFLG] [COMMENT]
   }
   \sstparameters{
      \sstsubsection{
         INPUT = \_CHAR (Read)
      }{
         Name of the catalogue.
      }
      \sstsubsection{
         NAME = \_CHAR (Read)
      }{
         Name of the field.
      }
      \sstsubsection{
         NAMEFLG = \_LOGICAL (Read)
      }{
         Name flag. Do you want to update the field name T/F?
         [FALSE]
      }
      \sstsubsection{
         NEWNAME = \_CHAR (Read)
      }{
         New name for the field.
      }
      \sstsubsection{
         FORMFLG = \_LOGICAL (Read)
      }{
         Format flag. Do you want to update the field format T/F?
         [FALSE]
      }
      \sstsubsection{
         FORMAT = \_CHAR (Read)
      }{
         New format for the format. A FORTRAN format detailing how the
         field value should be displayed.
      }
      \sstsubsection{
         UNITFLG = \_LOGICAL (Read)
      }{
         Unit flag. Do you want to update the field units T/F?
         [FALSE]
      }
      \sstsubsection{
         UNITS = \_CHAR (Read)
      }{
         New units to be associated with the field.
      }
      \sstsubsection{
         NULLFLG = \_LOGICAL (Read)
      }{
         Null value flag. Do you want to update the field null value T/F?
         [FALSE]
      }
      \sstsubsection{
         NULLVAL = \_CHAR (Read)
      }{
         New null value to be associated with the field.
      }
      \sstsubsection{
         COMFLG = \_LOGICAL (Read)
      }{
         Comment flag. Do you want to update the field comment T/F?
         [FALSE]
      }
      \sstsubsection{
         COMMENT = \_CHAR (Read)
      }{
         New comment to be associated with the field.
      }
   }
   \sstnotes{
      UPFIELD only updates the field information not the data. Use UPDATE
      to modify the data in a catalogue.
   }
   \sstdiytopic{
      Example
   }{
      UPFIELD TEST VALUE1 COMFLG=T COMMENT={\tt "}Possible error 
      $+$ or $-$ 2{\tt "}
   }
   \sstbugs{
      None known.
   }
}
\sstroutine{
   UPPARAM
}{
   Update a Parameter in a Catalogue
}{
   \sstdescription{
      Update a parameter in a catalogue. The parameter has a name
      format, value and comment. Any of these can be updated
      by setting the relevant flag
   }
   \sstusage{
      UPPARAM INPUT NAME [NAMEFLG] [NEWNAME] [FORMFLG] [FORMAT]
      [VALFLG] [VALUE]
      \newline
      [COMFLG] [COMMENT]
   }
   \sstparameters{
      \sstsubsection{
         INPUT = \_CHAR (Read)
      }{
         Name of the catalogue.
      }
      \sstsubsection{
         NAME = \_CHAR (Read)
      }{
         Name of the parameter.
      }
      \sstsubsection{
         NAMEFLG = \_LOGICAL (Read)
      }{
         Name flag. Do you want to update the parameter name T/F?
         [FALSE]
      }
      \sstsubsection{
         NEWNAME = \_CHAR (Read)
      }{
         New name for the parameter.
      }
      \sstsubsection{
         FORMFLG = \_LOGICAL (Read)
      }{
         Format flag. Do you want to update the parameter format T/F?
         [FALSE]
      }
      \sstsubsection{
         FORMAT = \_CHAR (Read)
      }{
         New format for the parameter. A FORTRAN format detailing how the
         parameter value should be interpreted.
      }
      \sstsubsection{
         VALFLG = \_LOGICAL (Read)
      }{
         Value flag. Do you want to update the parameter value T/F?
         [FALSE]
      }
      \sstsubsection{
         VALUE = \_CHAR (Read)
      }{
         New value to be associated with the parameter.
      }
      \sstsubsection{
         COMFLG = \_LOGICAL (Read)
      }{
         Comment flag. Do you want to update the parameter comment T/F?
         [FALSE]
      }
      \sstsubsection{
         COMMENT = \_CHAR (Read)
      }{
         New comment to be associated with the parameter.
      }
   }
   \sstnotes{
      The value of a parameter is always a character string. The format
      determines how the value should be interpreted. An error will be
      reported if the value and format become inconsistent.
   }
   \sstdiytopic{
      Example
   }{
      UPPARAM TEST CLASS FORMFLG=T FORMAT=I2
   }
   \sstbugs{
      None known.
   }
}
\sstroutine{
   WITHIN
}{
   Select entries within a polygon
}{
   \sstdescription{
      Create a new catalogue that contains entries from a catalogue that
      lie within a predefined polygon in 2-dimensional space. (The
      2-dimensional space is defined by 2 user selected numeric fields
      in the input catalogue.) Use the POLYGON application to
      create the polygon definition catalogue. The INSIDE parameter
      defaults to TRUE but can be set to FALSE to select entries that
      lie outside the polygon.
   }
   \sstusage{
      WITHIN INPUT OUTPUT XFIELD YFIELD POLYCAT [INSIDE] [REJECT] [OUTREJECT]
   }
   \sstparameters{
      \sstsubsection{
         INPUT = \_CHAR (Read)
      }{
         Name of the catalogue.
      }
      \sstsubsection{
         OUTPUT = \_CHAR (Read)
      }{
         Name of the catalogue to contain the selected entries.
      }
      \sstsubsection{
         XFIELD = \_CHAR (Read)
      }{
         Name of the field containing the X co-ordinates. XFIELD and
         YFIELD define the 2-dimensional space.
      }
      \sstsubsection{
         YFIELD = \_CHAR (Read)
      }{
         Name of the field containing the Y co-ordinates. XFIELD and
         YFIELD define the 2-dimensional space.
      }
      \sstsubsection{
         POLYCAT = \_CHAR (Read)
      }{
         Name of the catalogue defining the polygon. Created using
         POLYGON
      }
      \sstsubsection{
         INSIDE = \_LOGICAL (Read)
      }{
         Select entries inside the polygon (T/F)?
         [TRUE]
      }
      \sstsubsection{
         REJECT = \_LOGICAL (Read)
      }{
         Do you require a rejects catalogue (T/F)?
         [FALSE]
      }
      \sstsubsection{
         OUTREJECT = \_CHAR (Read)
      }{
         Name of the catalogue to contain the rejected entries.
      }
   }
   \sstnotes{
      The polygon definition is created using the POLYGON application.
   }
   \sstdiytopic{
      Example
   }{
      WITHIN TEST INTEST VALUE1 VALUE2 POLYCAT
   }
   \sstbugs{
      None known.
   }
}
\end{small}

\newpage
\section{The CATPAC Parser}
\label{ap:parser}

The CATPAC parser:

\begin{small}
\begin{verbatim}
     ? has -3 as its precedence
     (     -2
     ,     -1
     )      0
     .OR.   1
     .XOR.  1
     .AND.  2
     .NOT.  3
     <      4
     {      4
     =      4
     }      4
     >      4
     #      4
     .LT.   4
     .LE.   4
     .EQ.   4
     .GE.   4
     .GT.   4
     .NE.   4
     //     6
     +      7 
     -      7
     *      8
     /      8
     **     9
     +      10 when acting as a unary operator
     -      10 when acting as a unary operator
\end{verbatim}
\end{small}

In addition to these operators the function names given below are known to the
system.  Functions have the same precedence as '(' that is -2.

\begin{small}
\begin{verbatim}
     SQRT
     LOG10
     LOG
     EXP
     SIN
     COS
     TAN
     ASIN
     ACOS
     ATAN
     SINH
     COSH
     TANH
     ABS
     STRING
     DIFF
     GREAT_CIRCLE
     IRAS_TEMP23
     IRAS_TEMP12
     IRAS_TEMP34
     INT
     NINT
     MIN
     MAX
     UCASE
     LCASE
     CONVERT
\end{verbatim}
\end{small}

Abbreviations used:

\begin{small}
\begin{verbatim}
     <arop> arithmetic operators  

     <unop> unary operators 

     <chrop> character operator  

     <reop> relational operators  

     <loop> logical operators 

     <sloop> single logical operator

     <loexp> logical expression
     
     <arexp> arithmetic expression
     
     <chrexp> character expression
     
     <loconst> logical constant
     
     <numconst> numeric constant
     
     <chrconst> character constant
     
     <logfname> field name of a logic field 
     
     <numfname> field name of a numeric field
     
     <chrfname> field name of a character field
     
     <logfunction> logical function
     
     <numfunction> numeric function
     
     <chrfunction> character function
     
     <fldname> field name

     <numargs> numeric arguments

     <numarg> numeric argument

     <chrarg> character argument

     <logarg> logical argument

     <unitarg> argument of the CONVERT function specifying how the character
               argument is to be interpreted.

     <realconst> real constant

     <intconst> integer constant

     <expon> exponent part of an integer or real

     <dpconstant> decimal point constant
\end{verbatim}
\end{small}

The definition:

\begin{small}
\begin{verbatim}
     <loexp> :=   <arexp><reop><arexp> 
                | <loexp><loop><loexp>
                | <chrexp><reop><chrexp>
                | <sloop><loexp>
                | <logconst>
                | <logfname>
                | <logfunction>

     <arexp> :=   <arexp><arop><arexp>
                | <numconst>
                | <numfunction>
                | <numfname>

     <chrexp> :=  <chrexp><chrop><chrexp>
                | <chrconst>
                | <chrfname>

     
     <loop> :=   .OR.
               | .XOR.
               | .AND.

     <relop> :=   .LT.
                | .LE.
                | .EQ.
                | .GE.
                | .GE.
                | .NE.
                | <
                | {
                | =
                | }
                | >
                | #

     Note: In character relational expressions "less then" means precedes 
           in the ASCII collating sequence and "greater than" means follows 
           in the ASCII collating sequence.

     <unop> :=    +
                | -

     <arop> :=   +
               | -
               | *
               | /
               | **

     <chrop> :=  //

     <sloop> := .NOT.

     <chrfname> :=  <fldname> 
  
     Note: The field must be of type character

     <logfname> :=  <fldname>

     Note: The field must be of type logical

     <numfname> :=  <fldname>

     Note: The field must be of type real, integer or double

     <numfunction> :=  SQRT(<numarg>)
                     | LOG10(<numarg>)
                     | LOG(<numarg>)
                     | EXP(<numarg>)
                     | SIN(<numarg>)
                     | COS(<numarg>)
                     | TAN(<numarg>)
                     | ASIN(<numarg>)
                     | ACOS(<numarg>)
                     | ATAN(<numarg>)
                     | SINH(<numarg>)
                     | COSH(<numarg>)
                     | TANH(<numarg>)
                     | ABS(<numarg>)
                     | DIFF(<numarg>,<numarg>)
                     | GREAT_CIRCLE(<numarg>,<numarg>,<numarg>,<numarg>)
                     | IRAS_TEMP23(<numarg>,<numarg>)
                     | IRAS_TEMP12(<numarg>,<numarg>)
                     | IRAS_TEMP34(<numarg>,<numarg>)
                     | INT(<numarg>)
                     | NINT(<numarg>)
                     | MIN(<numargs>)
                     | MAX(<numargs>)
                     | CONVERT(<unitarg>,<chrconst>)

     <chrfunction> :=  STRING(<chrarg>)
                     | UCASE(<chrarg>)
                     | LCASE(<chrarg>)

     <numargs> :=  <numargs>,<numarg>
                 | <numarg>

     <numarg> :=  <numfname>
                | <numconst>

     <chrarg> :=  <chrfname>
                | <chrconst>
 
     <unitarg> :=  "DEGREE"
                 | "ARCMIN"
                 | "ARCSEC"     
                 | "SDD MM SS.S"
                 | "SDD:MM:SS.S"
                 | "SDD MM SS"
                 | "SDD:MM:SS"
                 | "SDD MM"
                 | "SDD:MM"
                 | "MINUTE"
                 | "SECOND"     
                 | "HH MM SS.S"
                 | "HH:MM:SS.S"
                 | "HH MM SS"
                 | "HH:MM:SS"
                 | "HH MM"
                 | "HH:MM"

     <chrconst> :=  "<chrstring>"

     <chrstring> :=  <chrstring><chrstring>
                    | <space>
                    | <alphanumeric>

     <logconst> :=  .TRUE.
                  | .FALSE.

     <numconst> :=  <realconst>
                  | <intconst>
              
     <realconst> :=  <intconst>
                   | <dpconst>
                   | <dpconst>E<expon>
                   | <dpconst>D<expon>

     <expon> :=  <unop><digits>
               | <digits>

     <dpconst> :=  <digits>.<digits>
                 | <digits>.
                 | .<digits>

     <intconst> :=  <digits>
                  | <digits>E<digits>

     <fldname> :=  <character><alphanumerics>

     <alphanumerics> :=  <alphanumeric><alphanumerics>
                       | <alphanumeric>

     <alphanumeric> :=  <character>
                      | <digit>

     <digits> :=  <digit><digits>
                | <digit>

     <character> :=    a | b | c | d | e | f | g | h | i | j | k | l 
                     | m | n | o | p | q | r | s | t | u | v | w | x 
                     | y | z | A | B | C | D | E | F | G | H | I | J
                     | K | L | M | N | O | P | Q | R | S | T | U | V 
                     | W | X | Y | Z |
   
     <digit> :=   0 | 1 | 2 | 3 | 4 | 5 | 6 | 7 | 8 | 9
\end{verbatim}
\end{small}

Functions that return more than one result, for example coordinate system
conversions, do not belong in the parser. A suite of applications exist
to perform these types of functions.

\section{CATPAC and SCAR}

CATPAC is a replacement for SCAR. Not all applications available in SCAR are
available in CATPAC yet but in future releases all SCAR application will be 
included in CATPAC. There are a number of reasons for replacing SCAR with 
CATPAC
They include:

\begin{itemize}

\item Improved documentation.

\item Improved online help facilities.

\item STARLINK standard use of ADAM parameters

\item New applications.

\item Improved existing applications.

\end{itemize}

But the most important reason for replacing SCAR is not apparent to the user.
SCAR is dependent on a database system called ADC and ADC is no longer  being
supported. CATPAC applications are based on the well documented,  flexible and
portable CHI routines. See SUN/119 for more details on the benefits of using
the CHI routines.

\end{document}
