\documentstyle[11pt]{article} 
\pagestyle{myheadings}

%------------------------------------------------------------------------------
\newcommand{\stardoccategory}  {Starlink User Note}
\newcommand{\stardocinitials}  {SUN}
\newcommand{\stardocnumber}    {18.4}
\newcommand{\stardocauthors}   {Martin Ricketts}
\newcommand{\stardocdate}      {21 February 1994}
\newcommand{\stardoctitle}     {RPS --- ROSAT Proposal Submission}
%------------------------------------------------------------------------------

\newcommand{\stardocname}{\stardocinitials /\stardocnumber}
\markright{\stardocname}
\setlength{\textwidth}{160mm}
\setlength{\textheight}{230mm}
\setlength{\topmargin}{-2mm}
\setlength{\oddsidemargin}{0mm}
\setlength{\evensidemargin}{0mm}
\setlength{\parindent}{0mm}
\setlength{\parskip}{\medskipamount}
\setlength{\unitlength}{1mm}

\begin{document}
\thispagestyle{empty}
SCIENCE \& ENGINEERING RESEARCH COUNCIL \hfill \stardocname\\
RUTHERFORD APPLETON LABORATORY\\
{\large\bf Starlink Project\\}
{\large\bf \stardoccategory\ \stardocnumber}
\begin{flushright}
\stardocauthors\\
\stardocdate
\end{flushright}
\vspace{-4mm}
\rule{\textwidth}{0.5mm}
\vspace{5mm}
\begin{center}
{\Large\bf \stardoctitle}
\end{center}
\vspace{5mm}

\section{Introduction}

\subsection{Overview}

The RPS package provides all the tools necessary for the generation and
electronic submission of observing proposals for the UK ROSAT Pointed 
Observation Programme. Briefly RPS allows the user to do the following:

\begin{enumerate}

\item create a new ROSAT Proposal Form (RPF) by `filling it in'
interactively

\item edit an existing RPF, for example to make minor changes to
observation details

\item summarise the details of an RPF

\item produce a laser printer version of the RPF for submission to the
UK Programme

\item produce a version of the RPF that can be transmitted over the
network and submit this electronically to the ROSAT UK Data Centre
(UKDC) at RAL

\item check when a target will be observable by Rosat and get an
estimate of the survey exposure

\end{enumerate}

Before starting to use RPS, the user ought to be familar with the
requirements of the UK ROSAT Pointed Observation Programme as laid out
in the Announcement of Opportunity (AO) cover letter and the Technical
Description. Copies of these documents are available from the UKDC at
RAL. {\bf Previous users of RPS should note that the paperwork has been
reduced by eliminating separate pages for each target - see Section
\ref{sse:forms}}

The RPS package is designed to enable guest observers to fulfil the
most important requirements of the UK Programme, namely to submit a
computer-readable version of their RPF to the UKDC and also to submit 2
paper copies of the complete proposal --- consisting of the RPF plus
the scientific case for the proposal.

The use of the RPS package offers a number of advantages to users, for
example it will check the data you enter into each form, ensuring that
the form you submit will not be rejected because of incorrectly entered
or incomplete data. There is now a {\bf Unix} version of the package;
the main difference is that it only has the line-entry mode and not the
screen mode whereas the {\bf VMS} version has both.

If RPS is not yet available at your site, it may be obtained on a VMS
Backup saveset file or Unix tar file from the UKDC, mail address {\tt
SAVAX::ROSATMAIL} or {\tt rosatmail@uk.ac.rl.sa750} (Janet) or {\tt
rosatmail@sa750.rl.ac.uk} (Internet).  {\bf NB.} this is different from
the address given in the Announcement, due to problems with that address.
The start-up file (VMS version) is commented
to indicate what may need changing at installation.

\subsection{Proposal Forms}
\label{sse:forms}

The ROSAT Proposal Form (RPF) for one Proposal has been changed for
{\bf AO5} in that, although the data entry is still made in four
sections, namely Cover, General, Target and Constraints `forms', the
printed RPF no longer has separate sheets for each target you may be
glad to hear. Instead, there is one or more sheets summarising the
Targets, and then if there are any constraints or remarks these are
each listed on separate sheets.

So the paperwork now consists of the following pages:

\begin{enumerate}

\item The {\bf Cover Page} which contains details of the Principal
Investigator (PI), the proposal title, subject category and the number
of targets proposed.  \item The {\bf General Form} which contains the
details of any Co-Investigators (Co-Is), the official endorsement of
the PI's institution and the PI declaration.

\item One or more {\bf Target Summary} sheets which contain the
technical details of the proposed observation(s) of each target.

\item If any targets have time constraints, these are summarised on a
{\bf Constraints Summary} sheet. The types of time-constraint that can
be applied to an observation are explained in Chapter 9 of the
Technical Description.

\item If there are any remarks, these are printed on one or more separate 
{\bf Remarks Summary} sheets.

\end{enumerate}

In addition, the abstract of the scientific case has to be entered.  In
{\bf VMS} use an editor to create a separate file; the {\bf Unix}
version contains an abstract editor.

\subsection{File names}

RPS uses various files to store and output the RPF, as follows, where
\verb+<propname>+ is the name you enter when starting up the program:

\begin{list}%
{}{\setlength{\leftmargin}{55mm} \setlength{\labelwidth}{40mm} \setlength{\labelsep}{6mm}
\setlength{\listparindent}{0mm} }

\item[\verb+<propname>.DAT+ \hfill]  holds the cover and general form
data, where \verb+<propname>+ is the name entered by the user when
first creating the RPF. {\bf NB.} These are binary files in FACTS
format, as used within the SCAR package, and are {\bf not} transferable
between different platforms.

\item[\verb+<propname>\_TARGET.DAT+ \hfill] holds the target data. 

\item[\verb+<propname>.ABSTRACT+ \hfill] is created separately with an
editor in the {\bf VMS} version, or within RPS in {\bf Unix}.
This must be a simple ASCII text file (ie without Latex
commands).  Without an abstract the RPF can be printed, but not
submitted by electronic mail. Note that the Unix version leaves a file,
fort.99 which can be deleted (if I knew why it wouldn't be there!)

\item[\verb+<propname>.TEX+ \hfill] \LaTeX\ input file This file can be
processed outside RPS in the usual way if desired.

\item[\verb+<propname>.POST+ \hfill] computer-readable version of the
RPS for submission to the UKDC.  If possible users should submit their
RPFs to the UKDC using the option within RPS. The {\bf VMS} and {\bf
Unix} versions use different formats for this file.

\item[\verb+<propname>.LIS+ \hfill] will be created if you want the
{\em Summarise} output on a file rather than the terminal.  Can be
printed.

\end{list} Note that in the {\bf Unix} version the file names are in
uppercase, regardless of what you enter. Otherwise they are as
described above, ie with appropriate suffix.

\section{Getting Started}
 
With {\bf VMS}, assuming RPS is installed on your machine, it must
first be set-up by executing the start-up file:

\begin{verbatim}
    $ rps_start
\end{verbatim}

On {\bf Unix}, there is just one environment variable, RPS\_AUX, which
prefixes the file names. If RPS is installed this is set by
{\tt /star/etc/login}.

Both versions of the program are initiated by typing:

\begin{verbatim}
    $ rps
\end{verbatim}

You are asked initially to enter a filename (without suffix) to be used
by RPS for all files created. This is the \verb+<propname>+ in the
filenames above.  Although there is a default ({\tt RPS\_FORM}), for all
purposes other than {\em Check Target} you should enter a suitable
name.

For {\bf VMS} only, you are then asked if you want to work in {\em
Screen} or {\em Line} mode. If you choose {\em Screen} mode you are
asked to confirm that the terminal has the normal tab settings (every 8
columns). Otherwise the {\em Screen} routines will not make use of
them, and the terminal output will take a lot longer.  Selection from
the menus in {\em Screen} mode can be done with the cursor keys or by
entering the appropriate number, in either case followed by {\tt
<ret>}.

Following this you will be presented with the main menu options, which
include accessing the Help library.

When form-filling, in {\em Screen} mode Help is available by the {\em
Help {\rm or} F2} function keys and in {\em Line} mode it is accessed
by entering `?' in any field.  This gives a description of the field
and the options allowed where appropriate.  The Help item {\tt
screen\_mode} gives a full list of the functions available.

\section{Main Options}

\subsection {Create New File}

This lets you create a new Proposal Form, using the Cover Page data
from a previous file if you supply a filename.  The fields from the
Cover page are presented, with defaults where appropriate.  See the
section on below Form Filling for a further description of the
facilities available.

After completing the cover page ( in screen mode entering
\verb+ctrl/z+) the validation is carried out.  If you have entered the
necessary fields such as your name, Institute, the Proposal name and
category, etc.  the page will be validated.  If the validation fails,
you will be given the first field where an error was detected and you
can go back to it by taking the default.  If you wish to exit from the
form-filling, enter `E' instead.

When you have finished with the Cover page, the General form will be
displayed.  Some of these details are only required on the paper
copies, but you can get them printed by entering them here.

Following this the first Target form is entered. Fill in the details of
your first target.  On completion (line mode) or entering \verb+Ctrl/z+
(screen mode) the validation will be done.  If this indicates the form
is unsatisfactory, then correct the appropriate field and try again. As
before, you can exit from an incomplete or incorrect form by entering
`E' in reply to the prompt.

If the Constraints flag is set on the Target record then after you have
completed it the Constraints page is entered for you to fill the
relevant fields.  If the validation fails twice on the constraints
form, the program returns to the Target form, in case you wish to
change anything there. There is also a warning if the moon constraint
affects a coordinated observation; you have the options of ignoring it,
or returning to the form, e.g. to change the time range.  On completion
of a target / constraints form you are asked if you want another
target. When you reply in the negative you will get the main menu
again.

\subsection{Edit Old File}

Once a form has been created, the form-filling procedures, outlined
above, can be used to change the contents. Also, on each Target form
there is a Target number; this is just incremented when a Form is
created; since this number may decide the precedence when it comes to
scheduling, you may wish to alter it.  The `Review' option allows
deletion of records and changing of Target numbers.

The options are:

\begin{description}
\begin{description}

\item [Edit Cover Page] --- Although the Cover page contains the number
of Targets, this is not displayed with the other fields as it is
automatically adjusted whenever records are added or deleted.

\item [Edit General Page] 

\item [Edit Target Data] --- The Constraints form is accessed after the
main Target form if selected.

\item [Add Target] --- to enter data for another target.

\item [Review Targets] --- This displays the Target numbers and names
for up to 20 Targets and then allows two main options. To change the
Target number of record n so it becomes Target M enter after the
prompt:

\begin{verbatim}
     > Cn,m
\end{verbatim}

and to delete {\em record} n (not {\em target number} n):

\begin{verbatim}
     > Dn
\end{verbatim}

Deleting a Target record does not automatically change any of the other
target numbers, but the records are concatenated.

\end{description}
\end{description}

When exiting from RPS after editing, you will be given the option of
deleting the old (default) version or the new one.

\subsection{Summarise}

This writes a few lines for each target in the form (as many as you
have filters selected).  The output can be to the screen (default) or a
listing file.  The details of the format are as follows:

\begin{center}
\begin{tabular}{rl}
\setlength{\leftmargin}{40mm}
\hspace{20mm}Heading	& Contents\\ \hline
Rec No		& Record number\\
Target Name	& Target name (16 chars)\\
Targ No		& Target number\\
Qual		& Quality indicator - `OK', or `F' if record failed check\\
R.A.		& RA - HH MM SS.S\\
Dec		& dec - DD MM SS.S\\
Start		& start of visible period, due to Sun constraint\\
End		& end visible period\\
		& (there may be two periods listed)\\
No. obs		& Number of observations\\
TC		& type of time constraint, or `none'\\
Time Ksec	& Time requested, ksec (to nearest integer)\\
2 $\times$ srv exp XRT	& Twice estimated XRT survey exposure, Ksec (cf AO fig. 6.1)\\
2 $\times$ srv exp WFC	& Twice estimated WFC survey exposure, Ksec \\
		& Data for each filter:\\
Instr		& Instrument. Z after WFC indicates Zoom on\\
Flt		& Filter\\
\%		& Time percentage requested\\
sec		& Time, total or for particular WFC filter \\
Det cps		& estimated cps for detection:\\
		& WFC:  see AO section 12.2, uses Bcell = 0.01\\
		& HRI:  see AO section 11.2, uses Rb = 0.000013\\
\end{tabular}
\end{center}

\subsection{Check Target}

This can be used to check the dates when a target satisfies the sun
constraint, and also to find what the expected survey exposure,
etc.\ is likely to be; these outputs are as for the {\em Summarise}
option.  Enter the RA and dec separately on request.  These can be in
any format that is valid for the form-filling procedure --- `?' will
give you the help information on the field. After one display typing
`Y' enables a further target to be tested. the default is to exit to
the main menu.

\subsection{Print File}

All the \LaTeX\ options create a file. In the {\bf vms} version this
can be processed by a batch job. Otherwise you process it after leaving
RPS.

The options are:

\begin{description}
\begin{description}

\item [All forms] --- \LaTeX\ file of whole proposal produced.

\item [Selected Pages] --- \LaTeX\ particular pages,

\item [Blank Forms] --- \LaTeX\ gives you a blank copy of each page
as given in the AO document.

\item [Resubmit] --- ({\bf vms} only) will submit a batch job to print
the \LaTeX\ file created in an earlier session.

\end{description}
\end{description}

\subsection{Proposal Submission}

This will try to create the .POST file from the proposal form. It might
fail because:

\begin{enumerate}

\item Errors are detected in the one or more of the pages. Use {\bf
Review Targets} or {\bf Summarise} for more information or enter the
Target or Cover Edit and type {\tt ctrl/z} to initiate the
verification.

\item The Target numbers do not form a sequence from 1 to the number of
target records.

\item There is no abstract file yet, or it is longer than 800 characters.

\end{enumerate}

You can choose to let the program mail it to the UKDC
({\tt rosatmail@uk.ac.rl.sa750} on Janet, {\tt rosatmail@sa750.rl.ac.uk} on
Internet) provided the facility is available; the display will tell you
what method will be used.  Alternatively, you can exit from RPS and
send it yourself.  A checksum is used to ensure that the paper copies
match the file sent over the network.

{\bf Two paper copies must also be sent to RAL, which have attached the
scientific case (see the AO document). } If you do not have the means
to produce the \LaTeX\ hard-copy version of the RPF forms at your
Institute, please produce your RPF by filling in your entries on a
photocopy of the RPF form as they appear in the AO Technical
Description. You should nevertheless submit the RPF electronically to
the UKDC.

The {\bf Data Protection Act 1984} applies to the database of
proposals; by submitting a proposal it is assumed that you accept its
provisions.

\section{Form Filling Procedure}

For each section, you will be presented with each field name, either as
a list on the screen, or one at a time if in line mode.

{\bf VMS:} In screen mode you will also see below the screen title the
field type and number of characters available.  Fields can be filled
from the left, or edited using various `standard' features --- see the
help for the details.  After entering the field data, {\tt <ret>} or
the cursor up / down keys will move to the next field unless an error
is detected, in which case the field should be edited or re-entered. On
typing {\tt ctrl/z} to exit from the page it is checked for required
fields and consistency. The first error to be detected is displayed and
you can either enter `R' or just \verb+<return>+ to go back to that
field on the form, so it can be edited, or `E' to exit, saving the
record.  Otherwise the next page is displayed. Upper and lower case are
allowed on the form, and the command entry is case-independent.


{\bf Unix:} The line mode options, apart from moving to the next field
with \verb+<return>+ are as follows:

\begin{center}
\begin{tabular}{|ll|ll|} \hline
Input	&Action&Input	&Action\\ \hline
\verb+%+&to exit out of the form at any time.		&\verb+&+&to go to the previous field.\\
\verb+^+&to erase information in character fields	&\verb+?+&for field help.\\
        &entering spaces will NOT erase!		&\verb+??+&for key help.\\
\verb+#+&to skip groups of similar fields.		&\verb+???+&for screen help.\\ \hline
\end{tabular}
\end{center}

For array entries a null entry causes a skip to the next array. Also
for each of the four types of Constraint, the data fields are skipped
if that constraint is not applicable.

\sloppy

\section{Form contents}

The table shows the fields for each section. The form files, {\tt
RPS\_EXAMPLE.DAT}, {\tt RPS\_EXAMPLE\_TARGET.DAT} are available in {\tt
RPS\_AUX}; if you wish to examine them with RPS, first copy them to
your own directory.

\footnotesize
\begin{tabular}{llrlr}
{\bf Form}	&{\bf Field name} &{\bf Format}&{\bf Field name}	&{\bf Format}\\[2mm]
{\bf Cover}	&LAST.NAME            &C*27	&FIRST.NAME           &C*17	\\
	&MIDDLE.NAME          &C*12	&PI.TITLE             &C*12  \\
	&DEPARTMENT          &C*60   	&INSTITUTE           &C*60   \\
	&ADDRESS             &C*60   	&CITY.TOWN           &C*32   \\
	&COUNTY              &C*32   	&POSTAL.CODE         &C*12   \\
	&COUNTRY             &C*24   	&TELEPHONE.NUMBER    &C*24   \\
	&TELEX.NUMBER        &C*20   	&FAX.NUMBER          &C*24   \\
	&NETWORK.NAME        &C*10   	&NETWORK.ADDRESS     &C*25   \\
	&PROPOSAL.TITLE(2)   &C*60   	&SUBJECT.CATEGORY    &I*1    \\
	&NUMBER.OF.TARGETS   &I*1    	&			& \\ [1mm]
{\bf General}	&COI.NAME(6)         &C*32  	&COI.INSTITUTE(6)    &C*32   \\
	&COI.CNTRY(6)        &C*24   	&ADMIN.NAME         &C*32    \\
	&ADMIN.POST         &C*60    	&AGENCY             &C*4     \\ [1mm]
{\bf Target}	&TARGET.NUMBER      &I*2 &TARGET.NAME        &C*20     \\
	&ALTERNATIVE.NAME   &C*20     	&TARGET.RA          &C*11     \\
	&TARGET.DEC         &C*11     	&TOTAL.OBS.TIME     &R*4      \\
	&NUMBER.OBS         &I*1      	&TIME.CRITICAL      &L*1      \\
	&HRI.CODE           &I*1      	&WFC.CODE           &I*1      \\
	&WFC.ZOOM.ON        &L*1      	&WFC.FILT.CODE(8)   &C*3      \\	
	&WFC.FILT.PCNT(8)   &I*1      	&WFC.FILT.MINT(8)   &R*4      \\
	&REMARKS(4)        &C*50     	&			&	\\[1mm]
{\bf Constraints}&COORD.OBSERVATION  &L*1 	&START.YEAR        &I*2      \\
	&START.MONTH        &I*1      	&START.DAY          &I*2      \\
	&START.HOUR         &I*1      	&START.MINUTE       &R*4      \\
	&END.YEAR          &I*2      	&END.MONTH          &I*1      \\
	&END.DAY            &I*2      	&END.HOUR           &I*1      \\
	&END.MINUTE         &R*4      	&MONITOR           &L*1      \\
	&TIME.INTERVAL      &R*4      	&PHASE.DEPENDENT    &L*1      \\
	&EPOCH              &R*4      	&PERIOD             &R*4      \\
	&CONTIGUOUS.OBS     &L*1      	&NUMBER.INTERVALS   &I*2      \\
\end{tabular}
\normalsize
\section{Acknowledgements}

Assistance has been provided by M Watson, of Leicester University
and at RAL Brian Stewart, Mark Jefferies, Martin Bush, and Dave Ewart.
For the Unix version, a lot of the work was done by Margo Duesterhaus and
Phillip Brisco at NASA Goddard.

\end{document}
