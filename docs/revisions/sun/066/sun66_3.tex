\documentstyle[11pt]{article} 
\pagestyle{myheadings}

%------------------------------------------------------------------------------
\newcommand{\stardoccategory}  {Starlink User Note}
\newcommand{\stardocinitials}  {SUN}
\newcommand{\stardocnumber}    {66.3}
\newcommand{\stardocauthors}   {Jeremy Bailey\\Anglo-Australian Observatory}
\newcommand{\stardocdate}      {9 May 1994}
\newcommand{\stardoctitle}     {TSP --- A Time Series/Polarimetry Package \\
                                [2ex] Version 2.1}
%------------------------------------------------------------------------------

%------------------------------------------------------------------------------
\newcommand{\stardocname}{\stardocinitials /\stardocnumber}
\renewcommand{\_}{{\tt\char'137}}     % re-centres the underscore
\markright{\stardocname}
\setlength{\textwidth}{160mm}
\setlength{\textheight}{230mm}
\setlength{\topmargin}{-2mm}
\setlength{\oddsidemargin}{0mm}
\setlength{\evensidemargin}{0mm}
\setlength{\parindent}{0mm}
\setlength{\parskip}{\medskipamount}
\setlength{\unitlength}{1mm}

%------------------------------------------------------------------------------
% Add any \newcommand or \newenvironment commands here
%------------------------------------------------------------------------------

\begin{document}
\thispagestyle{empty}
SCIENCE \& ENGINEERING RESEARCH COUNCIL \hfill \stardocname\\
RUTHERFORD APPLETON LABORATORY\\
{\large\bf Starlink Project\\}
{\large\bf \stardoccategory\ \stardocnumber}
\begin{flushright}
\stardocauthors\\
\stardocdate
\end{flushright}
\vspace{-4mm}
\rule{\textwidth}{0.5mm}
\vspace{5mm}
\begin{center}
{\LARGE\bf \stardoctitle}
\end{center}
\vspace{5mm}

%------------------------------------------------------------------------------
%  Add this part if you want a table of contents
\setlength{\parskip}{0mm}
\tableofcontents
\setlength{\parskip}{\medskipamount}
\markright{\stardocname}
%------------------------------------------------------------------------------

%------------------------------------------------------------------------------
%+                              M A N . S T Y
%
%  Module name:
%    MAN.STY
%
%  Function:
%    Default definitions for \LaTeX\ macros used in MAN output
%
%  Description:
%    As much as possible of the output from the MAN automatic manual generator
%    uses calls to user-alterable macros rather than direct calls to built-in
%    \LaTeX\ macros. This file contains the default definitions for these
%    macros.
%
%  Language:
%    \LaTeX
%
%  Support:
%    William Lupton, {AAO}
%-
%  History:
%    16-Nov-88 - WFL - Add definitions to permit hyphenation to work on
%		 words containing special characters and in teletype fonts.
%    21-Mar-90 - WFL - Use \makeatletter and \makeatother; tidy
%    14-Nov-91 - WFL - Add \manroutinebreakitem that inserts a line break
%	       after any text that appears on the initial line of the paragraph
%    15-Nov-91 - WFL - Always put \mbox{} after \item to protect against missing
%	       item text; remove spaces before { and [

\typeout{Default MAN macros. Released 14th November 1991}

% permit use of @ characters in names

\makeatletter

% permit hyphenation when in teletype font (support 9,10,11,12 point only -
% could extend), define lccodes for special characters so that the hyphen-
% ation algorithm is not switched off. Define underscore character to be
% explicit underscore rather than lots of kerns etc.

\hyphenchar\nintt=`-\hyphenchar\tentt=`-\hyphenchar\elvtt=`-\hyphenchar\twltt=`-

\lccode`_=`_\lccode`$=`$

\renewcommand{\_}{{\tt\char'137}}

%+                      M A N _ I N T R O
%
%  Section name:
%    MAN_INTRO
%                                 
%  Function:
%    Macros used in the .TEX_INTRO file
%
%  Description:
%    There are no such special macros.
%-

%+                      M A N _ S U M M A R Y
%
%  Section name:
%    MAN_SUMMARY
%
%  Function:
%    Macros used in the .TEX_SUMMARY file
%
%  Description:
%    There is a command to introduce a new section (mansection) and a list-like
%    environment (mansectionroutines) that handles the list of routines in the
%    current section. In addition a mansectionitem command can be used instead
%    of the item command to introduce a new routine in the current section.
%-

\newcommand{\mansection}[2]{\subsection{#1 --- #2}}

\newenvironment{mansectionroutines}{\begin{description}\begin{description}}%
{\end{description}\end{description}}

\newcommand{\mansectionitem}[1]{\item[#1:]\mbox{}}

%+                      M A N _ D E S C R
%
%  Section name:
%    MAN_DESCR
%
%  Function:
%    Macros used in the .TEX_DESCR file
%
%  Description:
%    There is a command to introduce a new routine (manroutine) and a list-like
%    environment (manroutinedescription) that handles the list of paragraphs
%    describing the current routine. In addition a manroutineitem or
%    manroutinebreakitem command can be used instead of the item command to
%    introduce a new paragraph for the current routine.
%
%    Two-column tables (the ones that can occur anywhere and which are
%    triggered by "=>" as the second token on a line) are bracketed by a
%    new environment (mantwocolumntable). Other sorts of table are introduced
%    by relevant  environments (manparametertable, manfunctiontable and
%    manvaluetable). The definitions of these environments call various other
%    user-alterable commands, thus allowing considerable user control over such
%    tables... (to be filled in when the commands have been written)
%-

\newcommand{\manrule}{\rule{\textwidth}{0.5mm}}

\newcommand{\manroutine}[2]{\subsection{#1 --- #2}}

\newenvironment{manroutinedescription}{\begin{description}}{\end{description}%
\manrule}

\newenvironment{mansubparameterdescription}{\begin{description}}%
{\end{description}}

\newcommand{\manroutineitem}[2]{\item[#1:] #2\mbox{}}

\newcommand{\manroutinebreakitem}[2]{\item[#1:] #2\hfill\\}

% two column tables

\newcommand{\mantwocolumncols}{||l|p{80mm}||}

\newcommand{\mantwocolumntop}{\hline}

\newcommand{\mantwocolumnblank}{\mantwocolumn@ss\mantwocolumn@hl%
\gdef\mantwocolumn@hl{}\gdef\mantwocolumn@ss{}}

\newcommand{\mantwocolumnbottom}{\mantwocolumn@ss\mantwocolumn@hl}

\newenvironment{mantwocolumntable}{\gdef\mantwocolumn@ss{}%
\gdef\mantwocolumn@hl{}\hspace*{\fill}\vspace*{-\partopsep}\begin{center}%
\begin{tabular}{\mantwocolumncols}\mantwocolumntop}{\mantwocolumnbottom%
\end{tabular}\end{center}}

\newcommand{\mantwocolumnentry}[1]{\mantwocolumn@ss\gdef\mantwocolumn@ss{\\}%
\gdef\mantwocolumn@hl{\hline}#1 & }

% parameter tables

\newcommand{\manparametercols}{lllp{80mm}}

\newcommand{\manparameterorder}[3]{#1 & #2 & #3 & }

\newcommand{\manparametertop}{}

\newcommand{\manparameterblank}{\gdef\manparameter@hl{}\gdef\manparameter@ss{}}

\newcommand{\manparameterbottom}{}

\newenvironment{manparametertable}{\gdef\manparameter@ss{}%
\gdef\manparameter@hl{}\hspace*{\fill}\vspace*{-\partopsep}\begin{trivlist}%
\item[]\begin{tabular}{\manparametercols}\manparametertop}{\manparameterbottom%
\end{tabular}\end{trivlist}}

\newcommand{\manparameterentry}[3]{\manparameter@ss\gdef\manparameter@ss{\\}%
\gdef\manparameter@hl{\hline}\manparameterorder{#1}{#2}{#3}}

% return tables

\newcommand{\manreturncols}{lllp{80mm}}

\newcommand{\manreturnorder}[3]{#1 & #2 & #3 & }

\newcommand{\manreturntop}{}

\newcommand{\manreturnblank}{\gdef\manreturn@hl{}\gdef\manreturn@ss{}}

\newcommand{\manreturnbottom}{}

\newenvironment{manreturntable}{\gdef\manreturn@ss{}%
\gdef\manreturn@hl{}\hspace*{\fill}\vspace*{-\partopsep}\begin{trivlist}%
\item[]\begin{tabular}{\manreturncols}\manreturntop}{\manreturnbottom%
\end{tabular}\end{trivlist}}

\newcommand{\manreturnentry}[3]{\manreturn@ss\gdef\manreturn@ss{\\}%
\gdef\manreturn@hl{\hline}\manreturnorder{#1}{#2}{#3}}

% function tables

\newcommand{\manfunctioncols}{||l|l|p{80mm}||}

\newcommand{\manfunctionorder}[2]{#1 & #2 & }

\newcommand{\manfunctiontop}{\hline}

\newcommand{\manfunctionblank}{\manfunction@ss\manfunction@hl%
\gdef\manfunction@ss{}\gdef\manfunction@hl{}}

\newcommand{\manfunctionbottom}{\manfunction@ss\manfunction@hl}

\newenvironment{manfunctiontable}{\gdef\manfunction@ss{}\gdef\manfunction@hl{}%
\hspace*{\fill}\vspace*{-\partopsep}\begin{center}\begin{tabular}%
{\manfunctioncols}\manfunctiontop}{\manfunctionbottom\end{tabular}\end{center}}

\newcommand{\manfunctionentry}[2]{\manfunction@ss\gdef\manfunction@ss{\\}%
\gdef\manfunction@hl{\hline}\manfunctionorder{#1}{#2}}

% value tables

\newcommand{\manvaluecols}{||l|l|l|p{80mm}||}

\newcommand{\manvalueorder}[3]{#1 & #2 & #3 & }

\newcommand{\manvaluetop}{\hline}

\newcommand{\manvalueblank}{\manvalue@ss\manvalue@hl\gdef\manvalue@ss{}%
\gdef\manvalue@hl{}}

\newcommand{\manvaluebottom}{\manvalue@ss\manvalue@hl}

\newenvironment{manvaluetable}{\gdef\manvalue@ss{}\gdef\manvalue@hl{}%
\hspace*{\fill}\vspace*{-\partopsep}\begin{center}\begin{tabular}%
{\manvaluecols}\manvaluetop}{\manvaluebottom\end{tabular}\end{center}}

\newcommand{\manvalueentry}[3]{\manvalue@ss\gdef\manvalue@ss{\\}%
\gdef\manvalue@hl{\hline}\manvalueorder{#1}{#2}{#3}}

% list environments

\newenvironment{manenumerate}{\begin{enumerate}}{\end{enumerate}}

\newcommand{\manenumerateitem}[1]{\item[#1]\mbox{}}

\newenvironment{manitemize}{\begin{itemize}}{\end{itemize}}

\newcommand{\manitemizeitem}{\item\mbox{}}

\newenvironment{mandescription}{\begin{description}\begin{description}}%
{\end{description}\end{description}}

\newcommand{\mandescriptionitem}[1]{\item[#1]\mbox{}}

% teletype conversion

\newcommand{\mantt}{\tt}

% "semi-verbatim" environment (modified from LaTeX source)

\def\@mansemiverbatim{\trivlist\item[]\if@minipage\else\vskip\parskip\fi%
\leftskip\@totalleftmargin\rightskip\z@%
\parindent\z@\parfillskip\@flushglue\parskip\z@%
\@tempswafalse\def\par{\if@tempswa\hbox{}\fi\@tempswatrue\@@par}%
\obeylines}

\def\mansemiverbatim{\@mansemiverbatim\frenchspacing\@vobeyspaces}

\let\endmansemiverbatim=\endtrivlist

% forbid use of @ characters in names

\makeatother
%------------------------------------------------------------------------------

%*******************************************************************************
% Document text starts here, at last!
%*******************************************************************************

\newpage

\section{Introduction}

TSP is an astronomical data reduction package which handles time series
data and polarimetric data. These facilities are missing from most
existing data reduction packages which are usually oriented towards
either spectroscopy or image processing or both. Where facilities for
polarimetry or time series data have been provided they have usually
not been sufficiently general to handle data from a variety of
different instruments.

TSP is currently used to process data from the following instruments:

\begin{itemize}

\item Spectropolarimetry data obtained with the AAO spectropolarimeters
using wave-plate or Pockels cell modulators in conjunction with either
IPCS or CCD detectors.

\item Infrared spectropolarimetry obtained with the IRPOL polarimeter
module in conjunction with the CGS2 grating spectrometer and the UKT6
and UKT9 CVF systems at UKIRT.

\item Infrared imaging polarimetry obtained with the IRIS instrument at
the AAT and with similar instruments

\item Time series imaging and polarimetry obtained with the AAO Faint
Object Polarimeter.

\item Time series polarimetry obtained with the Hatfield Polarimeter at
either UKIRT or AAT.

\item Time series polarimetry obtained with the University of Turku
UBVRI polarimeter.

\item Five channel time series photometry obtained with the Hatfield
polarimeter at the AAT in its high speed photometry mode.
                              
\item Time series infrared photometry obtained with the AAO Infrared
Photometer Spectrometer (IRPS).

\item Time series optical photometry obtained using the HSP3 high speed
photometry package at the AAT.

\end{itemize}
                                                                   
TSP is an ADAM application package and can be run from the UNIX shell
or from the ICL command language.  It uses the HDS data system for data
storage and the NCAR/SGS/GKS packages for graphics.

TSP does not attempt to duplicate features that are available in other
Starlink packages, and it is assumed that it will typically be used in
conjunction with packages such as KAPPA, FIGARO and CCDPACK. These
packages might be used to perform any standard image processing
operations required for CCD or IR array reduction with TSP then being
used to perform the polarimetric reduction.

\section{Running TSP}

\subsection{On UNIX systems}

Make sure you have sourced the /star/etc/login and /star/etc/cshrc
files (probably by including them in your own .login and .cshrc files)
and that you have a subdirectory of name adam in your home directory.

Type the following command at your shell prompt.

\begin{verbatim}

  csh> tsp

    TSP commands are now available -- (Version 2.1-0)

  csh>
\end{verbatim}

Any TSP command may now be entered. Note that on UNIX commands are case
sensitive and must be entered in lower case.

The following example shows the use of the PPLOT command to plot a
polarization spectrum. The SN1987A data file is included with the
software, so you can use this command to check that TSP is working.

\begin{verbatim}

    csh> pplot
    INPUT - Stokes Data to Plot> /star/bin/tsp/sn1987a
    BINERR - Error per bin (per cent) /0.5000000E-01/> 0.1
    AUTO - Autoscale Plot /TRUE/>
    LABEL - Label for plot> SN1987A  1987 Sep 2
    DEVICE - Plot Device> xwindows

\end{verbatim}


\subsection{On VAX/VMS systems}

To run TSP on VAX/VMS use the following commands:

\begin{verbatim}
    $ ADAMSTART
\end{verbatim}

This makes ADAM available. You can include this command in your LOGIN.COM
file if you use it frequently.

\begin{verbatim}
    $ ICL
\end{verbatim}

This starts the ICL command language which is used to run TSP.

\begin{verbatim}
    ICL> LOAD TSP_DIR:TSP
\end{verbatim}

The TSP commands are then defined. When the first TSP command is entered
there will be a delay as the TSP monolith is loaded. Subsequent commands
will give a faster response. The following example shows the use of the
PPLOT command to plot a polarization spectrum. The SN1987A data file is
included with the software, so you can use this command to check that TSP
is working.

\begin{verbatim}

    ICL> PPLOT
    INPUT - Stokes Data to Plot> TSP_DIR:SN1987A
    BINERR - Error per bin (per cent) /0.5000000E-01/> 0.1
    AUTO - Autoscale Plot /TRUE/>
    LABEL - Label for plot> SN1987A  1987 Sep 2
    DEVICE - Plot Device> XWINDOWS

\end{verbatim}

\subsection{TSP Commands}

TSP programs will prompt for all the parameters they need. Parameters
may also be specified on the command line.

The individual TSP commands are listed in appendix~\ref{app:commands}
and described in appendix~\ref{app:details}. Some information on using
TSP to reduce various types of data is given in subsequent sections.

\section{TSP Data Format}

The essence of the system is a common data format in which time series
polarimetry data can be represented. This format is based on the
Starlink standard data format described in SGP/38. In particular the
polarimetry example given in that document is the basis for the TSP
data format. The hierarchical structure of a typical TSP dataset is
shown below.

\begin{tabular}{lll}
\\
Component & Type & Description \\
\\
DATA\_ARRAY(N,M) & \_REAL & Stokes I array\\
VARIANCE(N,M) & \_REAL & Variance on I array\\
LABEL & \_CHAR & Label for data\\
UNITS & \_CHAR & Units for data\\
AXIS(2) & Structure Array & Axis information\\ 
\hspace*{1cm}AXIS(1) & AXIS & Wavelength Axis\\
\hspace*{2cm}.DATA\_ARRAY(N) & \_REAL & Wavelength Axis Data\\
\hspace*{2cm}.LABEL & \_CHAR & Wavelength Axis Label\\
\hspace*{2cm}.UNITS & \_CHAR & Wavelength Axis Units\\ 
\hspace*{1cm}AXIS(2) & AXIS & Time Axis\\
\hspace*{2cm}.DATA\_ARRAY(M) & \_DOUBLE & Time Axis Data\\
\hspace*{2cm}.LABEL & \_CHAR & Time Axis Label\\
\hspace*{2cm}.UNITS & \_CHAR & Time Axis Units\\
MORE & EXT & Extension structure \\
\hspace*{1cm}.POLARIMETRY & EXT & Polarimetry Extension \\
\hspace*{2cm}.STOKES\_Q & NDF & Q Stokes parameter structure\\
\hspace*{3cm}.DATA\_ARRAY(N,M) & \_REAL & Stokes Q array \\
\hspace*{3cm}.VARIANCE(N,M) & \_REAL & Variance on Q array\\
\hspace*{2cm}.STOKES\_U & NDF & U Stokes parameter structure\\
\hspace*{3cm}.DATA\_ARRAY(N,M) & \_REAL & Stokes U array \\
\hspace*{3cm}.VARIANCE(N,M) & \_REAL & Variance on U array \\
\end{tabular}                                        

The DATA\_ARRAY component shown here may actually be replaced by a
structure which holds the data array and other components.

The TSP structure is a special case of the NDF (Extensible
N-Dimensional Data Format) described in SGP/38. The main data array of
the NDF structure contains the I or Intensity Stokes parameter data. A
polarimetry extension structure contains up to 3 additional NDF
structures corresponding to the Stokes parameters containing the
polarization information. Any number from zero to three of these
additional Stokes structures may be present. The above example has the
two linear polarization Stokes parameters in structures STOKES\_Q and
STOKES\_U. For circular polarization data a structure called STOKES\_V
would be used.

The STOKES NDF structures contain only DATA\_ARRAY and optionally
VARIANCE components. The axis, label and units information pertaining
to the Stokes parameters is that in the main structure.

All the data arrays used in TSP are simple or primitive arrays. Other
types of data arrays such as SCALED arrays, SPACED arrays etc.
described in SGP/38 are not supported at present.

The TITLE, QUALITY and HISTORY components described in SGP/38 are not
currently used by TSP. If present they will be propagated from input to
output files.

TSP datasets may be one, two or three dimensional. A 1D dataset
represents a polarization spectrum. The axis is a wavelength axis. As
well as representing spectra the wavelength axis is also used to
contain the wavelengths of a small number of broad band channels for
data resulting from instruments such as the Hatfield Polarimeter.

2D datasets may be either time series polarization spectra, or
polarization images. For time series polarization spectra the first
axis is the wavelength axis, and the second axis is the time axis. The
time is represented in the form of the Modified Julian Date (MJD = JD -
2400000.5) in a double precision array. Note that special cases of the
time series dataset are those with a wavelength axis of size 1, and
with no additional Stokes parameters. Thus simple time series
photometry can be represented in this way.

3D datasets represent time series imaging (or imaging polarimetry) data.

\section{Relationship to FIGARO}

Although TSP normally uses the NDF format as described above, a number
of commands access raw data from FIGARO format files as this is the raw
data format produced by most AAO instruments. It used to be the case
that Figaro used a different format from the Starlink NDF format but
the latest version of Figaro supports both the old Figaro format (DST
files) and the Starlink NDF format. TSP is intended to be used in
conjunction with FIGARO for reducing spectropolarimetry data, with
FIGARO being used for standard spectroscopy operations such as arc
identification and wavelength calibration.

The environment variable or logical name FIGARO\_FORMATS controls which
of the two formats (DST or NDF) are used by TSP (and by Figaro)  when
accessing Figaro files. It is suggested that this name be set up as
follows:

\begin{verbatim}
   setenv FIGARO_FORMATS "NDF,DST"
\end{verbatim}
on UNIX systems, or
\begin{verbatim}
   $ DEFINE/JOB FIGARO_FORMATS "NDF,DST"
\end{verbatim}
on VAX/VMS systems.

Which will make the NDF format the default, but will also allow DST
files to be read. With this setting it is possible to use Figaro
commands on TSP files, but remember that Figaro operations will only
apply to the main data array.  Figaro will not see the additional data
arrays containing the Stokes parameters.

There are a number of commands in TSP with the same name as commands in
FIGARO ({\em e.g.}, XCOPY, SCRUNCH, SPFLUX). This is deliberate as these
commands do exactly the same as their FIGARO counterparts, but do it to
TSP files rather than to FIGARO files.

\section{Spectropolarimetry Reduction Algorithms}

Spectropolarimetry reduction involves determining the polarization of
the data from observations in two polarization states, the E and O rays
produced by a polarizing prism, and two pockels cell states or
waveplate positions (for each Stokes parameter) which are referred to
as A and B. Thus the polarization is derived from four measurements AE,
AO, BE and BO. The reduction should be such that it is insensitive to
polarization within the spectrograph, and flat field effects which will
give rise to systematic difference between the E and O states, and to
time variations in transmission and seeing which will give rise to
changes between A and B.

The IPCS2STOKES program uses the simple difference algorithm where the
polarization is given by:

\begin{equation}
P = \frac{AE-BE-(AO-BO)}{AE+BE+AO+BO}
\end{equation}

This algorithm does not fully correct for transmission changes between
the polarization states, but since the Pockels cell polarimeter
modulates rapidly such effects normally average out and do not cause
problems.

The CCD2STOKES and CCD2POL programs give a choice of algorithms. The
first algorithm, referred to as the OLD algorithm, since it is the
original one used by these programs is a modification of the difference
method where the O data is scaled by a factor F as follows:

\begin{equation}
F = \frac{AE+BE}{AO+BO}
\end{equation}
\begin{equation}
P = \frac{AE-BE-F(AO-BO)}{AE+BE+F(AO+BO)}
\end{equation}

This scaling makes the algorithm much less sensitive to transmission
variations.

The alternative algorithm is the RATIO algorithm which is as follows:

\begin{equation}
R^2 = \frac{AE/AO}{BE/BO}
\end{equation}

\begin{equation}
P = \frac{R-1}{R+1}
\end{equation}

The RATIO algorithm works very well on bright stars, but can fail on
faint objects (or on 100\% polarized calibration sources) through
attempting to take the sqaure root of a negative number. Under these
circumstances the OLD algorithm should be used.

\section{AAT Pockels Cell Spectropolarimeter Data}

\subsection{Introduction}

The AAT Pockels cell spectropolarimeter can be used with either CCD or
IPCS detectors. However, for use with CCD detectors it is now
superseded by the wave-plate polarimeter which is much more efficent.
The normal mode of operation is to use a two hole decker above the slit
defining star and sky apertures, together with a calcite beam splitter.
This gives images containing four spectra, star and sky for each of the
O and E modes of the calcite. Images of this type are recorded for the
two Pockels cell states (referred to as A and B). In the case of IPCS
data a single image contains both Pockels cell states. For CCD data two
separate images are taken, one in each state. For a more detailed
description of the instrument and its operation see McLean et al.
(MNRAS {\bf 209}, 655, 1984), and the AAO spectropolarimetry manual
(AAO UM 24).

A typical observing sequence would consist of observations of the
object in two Stokes parameters (Q and U, obtained by inserting
different quarter wave plates in the beam), and at two orientations of
the instrument usually (90 and 135 degrees). In addition there will be
calibration observations of a 100\% polarized source (C waveplate
position) to calibrate the efficiency of the system, and calibration
lamp observations for wavelength calibration. Flux standards may also
be observed if flux calibration is required.

\subsection{Data reduction sequence}

To reduce such a data set requires a combination of Figaro and TSP
commands.  Figaro is used for the standard spectroscopy parts of the
reduction such as arc fitting, and reducing flux standards. TSP is used
for the polarimetric parts of the reduction. The basic sequence of
reductions is shown in figure~\ref{fig:pokels}.

\begin{figure}
   \setlength{\unitlength}{1pt}
   \Large                
   \caption{Procedure for Reduction of Pockels Cell Spectropolarimetry}
   \label{fig:pokels}
   \large
   \begin{picture}(500,490)(0,0)
      \thicklines
      \put(120,455){\makebox(80,10){Raw Data}}
      \put(120,440){\makebox(80,10){Files}}
      \put(185,450){\vector(1,-1){30}}
      \put(175,380){\dashbox{4}(80,40){}}
      \put(175,388){\makebox(80,40){IMAGE}}
      \put(175,372){\makebox(80,40){ICUR}}
      \put(160,440){\vector(0,-1){80}}
      \put(110,320){\framebox(100,40){}}
      \put(120,328){\makebox(80,40){IPCS2STOKES}}
      \put(120,312){\makebox(80,40){CCD2STOKES}}
      \put(160,320){\vector(0,-1){20}}
      \put(120,260){\framebox(80,40){}}
      \put(120,268){\makebox(80,40){COMBINE}}
      \put(120,252){\makebox(80,40){FLIP}}
      \put(160,260){\vector(0,-1){20}}
      \put(120,220){\framebox(80,20){QUMERGE}}
      \put(160,220){\vector(0,-1){60}}
      \put(230,455){\makebox(80,10){Calibration}}
      \put(230,440){\makebox(80,10){Files}}
      \put(270,440){\vector(0,-1){80}}
      \put(220,320){\framebox(100,40){}}
      \put(230,328){\makebox(80,40){IPCS2STOKES}}
      \put(230,312){\makebox(80,40){CCD2STOKES}}
      \put(270,320){\vector(0,-1){20}}
      \put(230,260){\framebox(80,40){COMBINE}}
      \put(270,260){\vector(0,-1){60}}
      \put(230,180){\framebox(80,20){CALFIT}}
      \put(270,180){\line(0,-1){30}}
      \put(270,150){\vector(-1,0){70}}
      \put(120,140){\framebox(80,20){CALIB}} 
      \put(160,140){\vector(0,-1){20}}

      \put(340,395){\makebox(80,10){Arc}}
      \put(340,380){\makebox(80,10){Files}}
      \put(380,380){\vector(0,-1){20}}
      \put(330,320){\dashbox{4}(100,40){}}
      \put(340,328){\makebox(80,40){EXTRACT}}
      \put(340,312){\makebox(80,40){YSTRACT}}
      \put(380,320){\vector(0,-1){20}}
      \put(340,260){\dashbox{4}(80,40){IADD}}
      \put(380,260){\vector(0,-1){60}}
      \put(340,180){\dashbox{4}(80,20){ARC}}
      \put(380,180){\line(0,-1){70}}
      \put(380,110){\vector(-1,0){180}}
      \put(120,100){\framebox(80,20){XCOPY}}
      \put(160,100){\vector(0,-1){20}}
      \put(120,20){\framebox(80,60){}}
      \put(120,36){\makebox(80,60){PPLOT}}
      \put(120,20){\makebox(80,60){FPLOT}}
      \put(120,4){\makebox(80,60){QUPLOT}}
                  
      \put(1,405){Select Regions}
      \put(1,395){Containing data}

      \put(1,340){Reduce to}
      \put(1,330){Stokes Spectrum}

      \put(1,290){Combine runs for}
      \put(1,280){Each Stokes}
      \put(1,270){Parameter}

      \put(1,230){Merge Q and U} 

      \put(1,193){Fit Curve to}
      \put(1,183){Calibration data}

      \put(1,153){Efficiency}
      \put(1,143){Calibrate}

      \put(1,113){Wavelength}
      \put(1,103){Calibrate}

      \put(1,47){Plot Results}

   \end{picture}

   \begin{picture}(400,160)(0,0)
      \thicklines
      \put(100,40){\framebox(80,20){}}
      \put(180,40){\makebox(160,20){TSP Commands}}
      \put(100,100){\dashbox{4}(80,20){}}
      \put(180,100){\makebox(180,20){FIGARO Commands}}
   \end{picture}
\end{figure}

The first step in the reduction is to identify the regions of the image
containing the four spectra. Having done this the commands IPCS2STOKES
or CCD2STOKES can be used to reduce the raw images to TSP format Stokes
spectra.  Different observations can then be combined. Note that
rotating the instrument through 45 degrees effectively interchanges Q
and U. Thus to obtain true Q and U the observations must be combined as
follows:

\begin{equation}
   Q = Q_{90} + U_{135}
\end{equation}

\begin{equation}
   U = Q_{135} - U_{90}
\end{equation}

Thus when running IPCS2STOKES or CCD2STOKES the Stokes parameter should
be specified as Q for a U$_{135}$ observation and U for a Q$_{135}$
observation.  The sign of the U$_{90}$ observation can be inverted with
the FLIP command.  The COMBINE command can be used to combine different
observations in the same stokes parameter.

The Stokes parameter data can be plotted using the SPLOT command to
check the progress of the reduction, and the consistency of different
observations, before combining them.

The Q and U observations can be combined into a single file with the
QUMERGE command.

It may be found necessary to FLIP the signs of both Stokes parameters
to obtain correct position angles ({\em e.g.}, If the E and O spectra
were not correctly identified). It is useful to have observations of
polarized standard stars to check the position angle calibration.

\subsection{Efficiency Calibration}

The efficiency of the polarimetry system is not 100\%, and varies with
wavelength, particularly if a large wavelength range is being observed.
Calibration observations with a 100\% polarizer inserted can be used to
calibrate this effect. Such observations are reduced using the CCD2STOKES
or IPCS2STOKES commands, and then a Chebyshev polynomial is fitted to the
data using the CALFIT command. This calibration curve can then be used to 
correct other data using the CALIB command.

\subsection{Wavelength Calibration}

Calibration lamp observations can be fitted using the Figaro ARC program,
as described in the Figaro manual. The resulting wavelength calibration
can then be copied to a TSP file using the TSP command XCOPY, which is
similar to the Figaro command of the same name. If desired the TSP data
can then be scrunched (rebinned to a linear wavelength scale) using the
TSP SCRUNCH command.

\subsection{Flux Calibration}

Observations of standard stars can be used to derive a flux calibration
curve using the methods described in the Figaro documentation. This calibration
curve can then be applied to a TSP polarization spectrum to give a flux
calibrated polarization spectrum. Note that TSP does not keep track of the
total exposure time as spectra are combined, so the total value must be 
supplied as a parameter to SPFLUX.

\subsection{Plotting Data}

There are a number of commands to plot polarization spectra. SPLOT plots
a single Stokes parameter, as percentage polarization, together with the
intensity spectrum. PPLOT plots percentage polarization, position angle,
and intensity. Both these programs use a variable binning technique, to 
give a constant polarization error per bin.

FPLOT plots data in the form of polarized intensity (or flux). QUPLOT
plots a QU diagram.

\section{AAT Wave-Plate Spectropolarimeter Data}

\subsection{Introduction}

The AAT Wave-plate spectropolarimeter is normally used with CCD
detectors. The normal mode of operation is to use a two hole decker above
the slit defining star and sky apertures, together with a calcite beam
splitter. This gives images containing four spectra, star and sky for each
of the O and E modes of the calcite. Images of this type are recorded for
four positions of the half-wave plate at angles 0.0, 45.0, 22.5, 67.5 degrees. 

Calibration observations can be made by inserting calibration polarizers into
the beam. Rotation of the instrument to different position angles is not
necessary.

\subsection{Data reduction sequence}

To reduce such a data set requires a combination of Figaro and TSP
commands.  Figaro is used for the standard spectroscopy parts of the
reduction such as arc fitting, and reducing flux standards. TSP is used
for the polarimetric parts of the reduction. The basic sequence of
reductions is shown in figure~\ref{fig:wave}.

\begin{figure}
   \setlength{\unitlength}{1pt}
   \Large                
   \caption{Procedure for Reduction of Wave-Plate Spectropolarimetry}
   \label{fig:wave}
   \large
   \begin{picture}(500,490)(0,0)
      \thicklines
      \put(120,455){\makebox(80,10){Raw Data}}
      \put(120,440){\makebox(80,10){Files}}
      \put(185,450){\vector(1,-1){30}}
      \put(175,380){\dashbox{4}(80,40){}}
      \put(175,388){\makebox(80,40){IMAGE}}
      \put(175,372){\makebox(80,40){ICUR}}
      \put(160,440){\vector(0,-1){80}}
      \put(110,320){\framebox(100,40){}}
      \put(120,320){\makebox(80,40){CCD2POL}}
      \put(160,320){\vector(0,-1){20}}
      \put(120,260){\framebox(80,40){}}
      \put(120,260){\makebox(80,40){COMBINE}}
      \put(160,260){\vector(0,-1){100}}
      \put(230,455){\makebox(80,10){Calibration}}
      \put(230,440){\makebox(80,10){Files}}
      \put(270,440){\vector(0,-1){80}}
      \put(220,320){\framebox(100,40){}}
      \put(230,320){\makebox(80,40){CCD2POL}}
      \put(270,320){\vector(0,-1){20}}
      \put(230,260){\framebox(80,40){COMBINE}}
      \put(270,260){\vector(0,-1){60}}
      \put(230,180){\framebox(80,20){CALFITPA}}
      \put(270,180){\line(0,-1){30}}
      \put(270,150){\vector(-1,0){70}}
      \put(120,140){\framebox(80,20){CALPA}} 
      \put(160,140){\vector(0,-1){20}}

      \put(340,395){\makebox(80,10){Arc}}
      \put(340,380){\makebox(80,10){Files}}
      \put(380,380){\vector(0,-1){20}}
      \put(330,320){\dashbox{4}(100,40){}}
      \put(340,328){\makebox(80,40){EXTRACT}}
      \put(340,312){\makebox(80,40){YSTRACT}}
      \put(380,320){\vector(0,-1){20}}
      \put(340,260){\dashbox{4}(80,40){IADD}}
      \put(380,260){\vector(0,-1){60}}
      \put(340,180){\dashbox{4}(80,20){ARC}}
      \put(380,180){\line(0,-1){70}}
      \put(380,110){\vector(-1,0){180}}
      \put(120,100){\framebox(80,20){XCOPY}}
      \put(160,100){\vector(0,-1){20}}
      \put(120,20){\framebox(80,60){}}
      \put(120,36){\makebox(80,60){PPLOT}}
      \put(120,20){\makebox(80,60){FPLOT}}
      \put(120,4){\makebox(80,60){QUPLOT}}
                  
      \put(1,405){Select Regions}
      \put(1,395){Containing data}

      \put(1,340){Reduce to}
      \put(1,330){Stokes Spectrum}

      \put(1,280){Combine runs}

      \put(1,193){Fit Curve to}
      \put(1,183){Calibration data}

      \put(1,153){Position angle}
      \put(1,143){Calibrate}

      \put(1,113){Wavelength}
      \put(1,103){Calibrate}

      \put(1,47){Plot Results}

   \end{picture}

   \begin{picture}(400,160)(0,0)
      \thicklines
      \put(100,40){\framebox(80,20){}}
      \put(180,40){\makebox(160,20){TSP Commands}}
      \put(100,100){\dashbox{4}(80,20){}}
      \put(180,100){\makebox(180,20){FIGARO Commands}}
   \end{picture}
\end{figure}

The first step in the reduction is to identify the regions of the image
containing the four spectra. Having done this the command
CCD2POL can be used to reduce the raw images for the four plate positions
to TSP format polarization spectra.
Different observations can then be combined using the COMBINE command. 

The polarization data can be plotted using the PPLOT command to check the
progress of the reduction, and the consistency of different observations,
before combining them.

\subsection{Efficiency Calibration}

The efficiency of the wave-plate polarimeter system is very close to 100\%, 
and thus efficiency correction is hardly necessary. If required it can be done
with the CALFIT and CALIB commands as described for the Pockels cell
polarimeter.

\subsection{Position Angle Calibration}

An additional complication with the wave-plate polarimeter is that the apparent
position angle varies with wavelength as a consequence of the wavelength
dependence of the position of the fast axis of the superachromatic plate.
This effect can be calibrated by using observations with known position angle,
either of a star through the calibration polarizer, or of a polarized standard
star. The CALFITPA command can be used to fit a Chebyshev polynomial to the
wavelength dependence of the position angle, and this calibration curve can
then be applied to observations using the CALPA command.

\subsection{Wavelength Calibration}

Calibration lamp observations can be fitted using the Figaro ARC program,
as described in the Figaro manual. The resulting wavelength calibration
can then be copied to a TSP file using the TSP command XCOPY, which is
similar to the Figaro command of the same name. If desired the TSP data
can then be scrunched (rebinned to a linear wavelength scale) using the
TSP SCRUNCH command.

\subsection{Flux Calibration}

Observations of standard stars can be used to derive a flux calibration
curve using the methods described in the Figaro documentation. This calibration
curve can then be applied to a TSP polarization spectrum to give a flux
calibrated polarization spectrum. Note that TSP does not keep track of the
total exposure time as spectra are combined, so the total value must be 
supplied as a parameter to SPFLUX.

\subsection{Plotting Data}

There are a number of commands to plot polarization spectra. SPLOT plots
a single Stokes parameter, as percentage polarization, together with the
intensity spectrum. PPLOT plots percentage polarization, position angle,
and intensity. Both these programs use a variable binning technique, to 
give a constant polarization error per bin.

FPLOT plots data in the form of polarized intensity (or flux). QUPLOT
plots a QU diagram.

\section{CGS2 and CGS4 Spectropolarimetry Data}

The UKIRT cooled grating spectrometers CGS4 and CGS2 can be used for
spectropolarimetry in conjnuction with the IRPOL polarimetry module
which is used to rotate a half-wave plate in front of the instrument.
In conjunction with a wire grid polarizer in the dewar this results in
data from which linear polarization can be derived. A typical procedure
for reduction of such data is illustrated in figure~\ref{fig:cgs4}.
Note that the order of the various calibration steps is generally not
critical ({\em e.g.}, flux calibration may be done before PA and
efficiency calibration).

The same procedure should be applicable to CVF spectropolarimetry data
obtained with the UKT6 and UKT9 instruments.


\begin{figure}
   \setlength{\unitlength}{1pt}
   \Large                
   \caption{Procedure for Reduction of CGS4 Spectropolarimetry}
   \label{fig:cgs4}
   \large
   \begin{picture}(500,490)(0,0)
      \thicklines
      \put(120,455){\makebox(80,10){Raw Data}}
      \put(120,440){\makebox(80,10){Files}}
      \put(160,440){\vector(0,-1){80}}
      \put(120,320){\framebox(80,40){}}
      \put(120,320){\makebox(80,40){CGS4POL}}
      \put(160,320){\vector(0,-1){70}}
      \put(120,210){\framebox(80,40){ROTPA}}
      \put(160,210){\vector(0,-1){50}}
      \put(230,455){\makebox(80,10){Calibration}}
      \put(230,440){\makebox(80,10){Files}}
      \put(270,440){\vector(0,-1){80}}
      \put(230,320){\framebox(80,40){}}
      \put(230,320){\makebox(80,40){CGS4POL}}
      \put(270,320){\vector(0,-1){20}}
      \put(230,260){\framebox(80,40){ROTPA}}
      \put(270,260){\vector(0,-1){60}}
      \put(230,180){\framebox(80,20){CALFIT}}
      \put(270,180){\line(0,-1){30}}
      \put(270,150){\vector(-1,0){70}}
      \put(120,140){\framebox(80,20){CALIB}} 
      \put(160,140){\vector(0,-1){20}}

      \put(340,395){\makebox(80,10){Standard}}
      \put(340,380){\makebox(80,10){Files}}
      \put(380,380){\vector(0,-1){20}}
      \put(340,320){\framebox(80,40){}}
      \put(340,320){\makebox(80,40){CGS4POL}}
      \put(380,320){\vector(0,-1){210}}
      \put(380,110){\vector(-1,0){180}}
      \put(120,100){\framebox(80,20){IRFLUX}}
      \put(160,100){\vector(0,-1){20}}
      \put(120,20){\framebox(80,60){}}
      \put(120,36){\makebox(80,60){EPLOT}}
      \put(120,20){\makebox(80,60){PPLOT}}
      \put(120,4){\makebox(80,60){QUPLOT}}
                  
      \put(1,340){Reduce to}
      \put(1,330){Stokes Spectrum}

      \put(1,280){Rotate to PA zero}

      \put(1,233){Rotate to}
      \put(1,223){Correct PA}

      \put(1,193){Fit Curve to}
      \put(1,183){Calibration data}

      \put(1,153){Efficiency}
      \put(1,143){Calibrate}

      \put(1,113){Flux}
      \put(1,103){Calibrate}

      \put(1,47){Plot Results}

   \end{picture}

\end{figure}

\subsection{Reducing CGS4 data}

The program CGS4POL takes four CGS4 reduced group files containing
observations at waveplate angles of 0, 45, 22.5 and 67.5 degrees and
extracts the spectra to obtain a polarization spectrum. The data files
should contain positive and negative spectra, obtained by sliding
between two slit positions for the star.  Parameters of CGS4POL are the
positions of the two apretures to extract the star data from, and the
name (A or B) of the aperture containing the star.

\subsection{Reading CGS2 data files}

The RCGS2 command reads raw CGS2 data files and produces TSP
polarization spectra. RCGS2 can also perform despiking of data by
specifying a cutoff level.  Any points which deviate from the mean by
more than this cutoff times the sigma for the wavelength are removed.

At this stage the data can be plotted using the EPLOT command to judge the
quality of the data. 

\subsection{Position angle calibration}

The only correction needed is a correction for position angle zero
point. This can be determined by looking at an observation of a
standard with known position angle, and then applied to the data using
the ROTPA command, which rotates the position angle of a dataset
through a specified angle. The PTHETA command may be useful in
accurately determining the position angle for a section of the spectrum
of a standard.

\subsection{Efficiency calibration}

The half-wave plates and wire grid polarizers used will not give 100\%
efficiency, and the efficiency will normally be wavelength dependent.
To calibrate for this effect an observation of a 100\% polarized source
should be made using the calibration wire grid polarizer. This
observation can be reduced using CGS4POL or RCGS2. It should then be
rotated using ROTPA so that the position angle is zero which puts all
the polarization into the Q Stokes parameter. CALFIT can then be used
to fit a calibration curve to this data. The CALIB command is then used
to apply this calibration curve to other datasets.

\subsection{Flux Calibration}

The IRFLUX command can be used to flux calibrate data, using an
observation of a standard star. IRFLUX models the standard star as a
black body of a given temperature. The flux of the standard may be
specified as a magnitude in one of the standard bands (JHKLM) or as a
flux in mJy at a specified wavelength. The flux calibrated data is in
mJy. It may be converted to F$_{\lambda}$ using the FLCONV command if
desired.

\subsection{Plotting Data}

IR spectropolarimetry may be plotted using any of the commands
described for optical spectropolarimetry (PPLOT, FPLOT, QUPLOT).
However with the smaller number of spectral points the EPLOT command,
which plots flux, P and Theta with error bars, will often be found more
appropriate than the constant error binning approach used by PPLOT.
Data may be plotted at any stage in the reduction after RCGS2 or
CGS4POL.

\section{Time Series Data}

\subsection{Input of Data}

Time series data handled by TSP can come from a variety of sources, and
can range from simple single channel photometry, to multichannel
polarimetric data. There are a number of routines which allow time
series data to be read into the system:

\begin{description}

\item[RHATHSP] --- Reads high speed photometry data taken with the Hatfield
polarimeter, working in its 5 channel photometer mode.

\item[RHATPOL] --- Reads 6 channel polarimetry data taken with the Hatfield
polarimeter.

\item[RHSP3] --- Reads tapes created with the HSP3 high speed photometry
system at the AAT.

\item[RIRPS] --- Reads time series infrared photometry obtained with the
AAO Infrared Photometer Spectrometer (IRPS).

\item[RTURKU] --- Reads files containing data from the University of TURKU
UBVRI polarimeter. These data sets have 5 channels and can include either
linear or linear plus circular polarization.

\end{description}

Time series photometry and polarimetry can also be extracted from time series
imaging data as described later.

\subsection{Processing Data}

Processing of time series data includes the following options.

\begin{description}

\item[LTCORR] --- This corrects a time series dataset for light travel time,
yielding a heliocentric or barycentric time axis.

\item[TEXTIN] --- This corrects a time series dataset for atmospheric
extinction.

\item[TMERGE] --- This merges two time series datasets. The two datasets
must have the same number of channels. Because TSP does not require the time
axis of datasets to be evenly spaced it is possible to combine observations
taken even years apart into a single file.

\item[TBIN] --- Bin a time series into time bins of a specified size.

\item[TDERIV] --- Calculate a new time series which is the time derivative
of the intensity data in a time series.

\end{description}

\subsection{Plotting Time Series Data}

The following commands are available for plotting time series data:

\begin{description}

\item[TSPLOT] --- Plots time series data against time. There are lots of
options to plot up to six items which can be different channels, polarization
components etc. The plotted data can be binned if necessary.

\item[PHASEPLOT] --- Plots time series data against phase on some period.
There are similar options to those in TSPLOT.

\item[QPLOT] --- A `quick' version of TSPLOT with less options.

\end{description}

\section{Imaging Polarimetry Data}

TSP can be used to reduce imaging polarimetry data obtained with IRIS
at the AAT. The command IRISPOL is used to generate a polarization
image from a set of four observations at the four waveplate positions.
IRISAP can be used to derive aperture polarimetry of a star from such a
set of four images.  See the AAO IRIS manual for more details.

IMPOL can be used to generate a polarization image from a single beam
polarimeter such as IRPOL/IRCAM at UKIRT.

TSP polarization images can be plotted with the AAOPLOT program (not
part of TSP but a separate ADAM package). ROTPA can be used to
calibrate the position angle of an image and images can be combined
with COMBINE.

These commands may also be found useful for other imaging polarimeters
working at optical as well as IR wavelengths. Use IRISPOL for dual beam
instruments which give simultaneous images in E and O states, and use
IMPOL for single beam instruments.

\section{Three dimensional data}

TSP uses three dimensional datasets to represent time series images.
There are two commands which enable such data to be read into TSP.

\begin{description}

\item[BUILD3D] --- This command  is used to build a three dimensional dataset
from a number of individual Figaro frames.

\item[RCCDTS] --- Reads times series data obtained with the time series mode
of the AAO CCD systems. 

\end{description}

\subsection{Displaying time series images}

The DISPLAY command enables time series imaging data to be displayed on
an image display device. Once displayed a COMMAND mode enables a number
of options to be selected. Any indivdual frame may be displayed. A
series of frames may be displayed as a movie. A cursor may be put up to
read positions or data values.

\subsection{Extracting Light Curves from Time Series Images }

The commands described here are used to obtain the light curve of a
star from a time series image. First it is necessary to subtract sky
using the command SKYSUB which is based on the use of two sky areas on
either side of a star.

CCDPHOT can then be used to perform aperture photometry of the star to
obtain a light curve. CCDPOL is a similar command that obtains
polarimetry from observed through an instrument such as the AAO faint
object polarimeter which produces E and O images from a wollaston
prism.

As an alternative to CCDPHOT, the commands TSPROFILE and TSEXTRACT can
be used to determine a profile which is a smoothly varying function of
time, and extract the photometry using an optimal weighted combination
of pixels. The procedure is the 3 dimensional analogue of the optimal
extraction technique for extracting spectra of stars from long slit
data. For the technique to be succesful it is important that the image
does not show rapid motion ({\em e.g.}, due to seeing or tracking
problems) that cannot be adequately represented by a low order
polynomial fitted through the dataset.

\subsection{Image motion and software tip/tilt correction}

This pair of commands enable the image motion in a time series image to
be studied and allow the software analogue of `tip-tilt' correction to
be applied to the data. When applied to a time series which consists of
very short exposure images, the translational component of seeing can
be removed enabling significant reduction in image sizes. This has
enabled a FWHM for star images of 0.37 arc seconds to be achieved on
data taken with infrared camera IRIS on the AAT.



\section{Writing TSP programs}

Programs to access TSP data files should be written to make use of the
TSP subroutine library (TSPSUBS) rather than making direct use of the
HDS DAT\_ routines. The TSPSUBS routines are described in Appendices C
and D.  They allow the building of new TSP structures, the reading and
writing of items from the structures, and the mapping of data arrays
(data arrays are always accessed by mapping in TSP programs. The only
DAT\_ routines that are used in TSP programs are those associated with
the ADAM parameter system such as DAT\_CREAT and DAT\_ASSOC, and also
DAT\_ANNUL. Below is an example of an ADAM A-task that makes a copy of
a TSP data structure using the routine TSP\_COPY

\begin{small}
\begin{verbatim}
      SUBROUTINE COPY(STATUS)
*
*  Copy a TSP structure
*
      IMPLICIT NONE
      INTEGER STATUS
      INCLUDE 'SAE_PAR'
      CHARACTER*(DAT__SZLOC) LOC,LOC2


      CALL DAT_ASSOC('INPUT','READ',LOC,STATUS)
      CALL DAT_CREAT('OUTPUT','NDF',0,0,STATUS)
      CALL DAT_ASSOC('OUTPUT','WRITE',LOC2,STATUS)
      CALL TSP_COPY(LOC,LOC2,STATUS)
      CALL DAT_ANNUL(LOC,STATUS)
      CALL DAT_ANNUL(LOC2,STATUS)

      END       
\end{verbatim}
\end{small}

Such a program could be linked with the command:

\begin{verbatim}
      alink copy `tsp_link_adam`
\end{verbatim}
on UNIX systems, and:
\begin{verbatim}
      $ ALINK COPY,TSP_DIR:TSPSUBS
\end{verbatim}
on VAX/VMS systems.

\section{New Features in TSP version 2.1}

\subsection{UNIX version}

TSP 2.1 is the first UNIX release of TSP. All features of the VAX/VMS
version are now available on UNIX systems. On UNIX commands are entered
directly from the shell rather than from ICL, and must be entered in
lower case.

\subsection{Imaging Polarimetry Commands}

The commands IRISPOL and IRISAP have been added to reduce imaging
polarimetry data obtained with IRIS at the AAT. The IMPOL command can
be used to reduce data from single beam polarimeters such as
IRCAM/IRPOL at UKIRT. Other commands such as ROTPA and COMBINE have
been enhanced to handle imaging as well as spectropolarimetry data.

\subsection{CGS4 Spectropolarimetry}

The command CGS4POL has been added for reduction of spectropolarimetry
data obtained with the CGS4 instrument at UKIRT.

\section{New Features in TSP version 2.0}

\subsection{New Commands}

\subsubsection{Commands for spectropolarimetry reduction}

\begin{description}

\item{CALFITPA} --- Fit a calibration curve to polarization position angle

\item{CALPA} --- Position angle calibrate a polarization spectrum

\item{CCD2POL} --- Reduce CCD Spectropolarimetry Data

\item{DIVIDE} --- Divide a polarization spectrum by an intensity spectrum

\item{EXTIN} --- Correct a polarization spectrum for extinction

\item{FLCONV} --- Convert a flux calibrated spectrum to F-lambda

\item{LMERGE} --- Merge two polarization spectra

\end{description}

\subsubsection{IR Spectropolarimetry}

\begin{description}

\item{EPLOT} --- Plot a polarization spectrum as P, Theta with error bars

\item{IRFLUX} --- Apply Flux calibration to an infrared polarization spectrum

\item{RCGS2} --- Read CGS2 Polarimetry data

\item{ROTPA} --- Rotate position angle of a polarization spectrum

\end{description}

\subsubsection{Time Series Data}

\begin{description}

\item{LHATPOL} --- List Hatfield Polarimeter Infrared Data

\item{RHATPOL} --- Read Hatfield Polarimeter Data

\item{TEXTIN} --- Correct a time series dataset for extinction

\item{TLIST} --- List time series data to a file

\item{TSETBAD} --- Inteactively mark bad points in a time series

\end{description}

\subsubsection{Time Series Imaging Data}

\begin{description}

\item{BUILD3D} --- Insert a Figaro frame into a time series image

\item{CCDPHOT} --- Photometry of a star on a time series image

\item{CCDPOL} --- Polarimetry of a star on a time series image

\item{DISPLAY} --- Display a time series image on an image display device

\item{IMOTION} --- Analyze the image motion in a time series image

\item{RCCDTS} --- Read AAO CCD Time Series data

\item{SKYSUB} --- Subtract sky from a time series image

\item{SHIFTADD} --- Add frames correcting for image motion

\item{TSEXTRACT} --- Optimal extraction of a light curve from a time series
image

\item{TSPROFILE} --- Determine a spatial profile from a time series image

\end{description}

\subsubsection{General}

\begin{description}

\item{DSTOKES} --- Delete a Stokes Parameter from a dataset

\end{description}

\subsection{TSPSUBS library}

The TSPSUBS library has been extended to support 3 dimensional datasets
(time series images) and rewritten to make use of the NDF package
rather than direct calls to HDS. This means that TSP now supports
SIMPLE as well as PRIMITIVE NDFs and will benefit from subsequent
additions to the NDF package.

\subsection{Figaro File access}

All access to Figaro files is now through the DSA package, allowing TSP
to benefit from the new Figaro 3.0 feature of accessing NDF files as
well as DST files. Now that Figaro supports NDF it is not strictly
necessary for TSP to use DSA at all, since it could access Figaro files
via NDF. However, for the moment the use of DSA has been retained to
allow DST files to be accessed as well, which is particularly
convenient when working on old data.

\subsection{Bad Pixel Handling}

Most TSP commands now support the handling and propagation of bad pixels. 

\newpage
\appendix
\begin{small}

\section{TSP commands}
\label{app:commands}

\begin{mansectionroutines}
\mansectionitem{{\mantt{BUILD3D}}}
        Insert a figaro frame into a time series image

\mansectionitem{{\mantt{CALFIT}}}
        Fit a calibration curve to a polarization spectrum

\mansectionitem{{\mantt{CALFITPA}}}
        Fit a calibration curve to the polarization position angle

\mansectionitem{{\mantt{CALIB}}}
        Efficiency Calibrate a Polarization Spectrum

\mansectionitem{{\mantt{CALPA}}}
        Position Angle Calibrate a Polarization Spectrum

\mansectionitem{{\mantt{CCD2POL}}}
        Reduce {\mantt{CCD}} spectropolarimetry data.

\mansectionitem{{\mantt{CCD2STOKES}}}
        Reduce {\mantt{CCD}} spectropolarimetry data.

\mansectionitem{{\mantt{CCDPHOT}}}
        Photometry of a star on a time series image

\mansectionitem{{\mantt{CCDPOL}}}
        Polarimetry of a star on a time series image

\mansectionitem{{\mantt{CGS4POL}}}
        Reduce {\mantt{CGS4}} spectropolarimetry data.

\mansectionitem{{\mantt{CMULT}}}
        Multiply a polarization spectrum by a constant

\mansectionitem{{\mantt{COMBINE}}}
        Combine two Polarization Datasets

\mansectionitem{{\mantt{DIVIDE}}}
        Divide a polarization spectrum by an intensity spectrum

\mansectionitem{{\mantt{DSTOKES}}}
        Delete a Stokes parameter from a dataset.

\mansectionitem{{\mantt{EPLOT}}}
        Plot a polarization spectrum as P, Theta with error bars

\mansectionitem{{\mantt{EXTIN}}}
        Correct a polarization spectrum for extinction

\mansectionitem{{\mantt{FLCONV}}}
        Convert a flux calibrated spectrum to f-lambda

\mansectionitem{{\mantt{FLIP}}}
        Invert the sign of the Stokes parameter in a spectrum.

\mansectionitem{{\mantt{FPLOT}}}
        Plot a polarization spectrum as Polarized Intensity

\mansectionitem{{\mantt{IMOTION}}}
        Analyze the image motion in a time series image

\mansectionitem{{\mantt{IPCS2STOKES}}}
        Reduce {\mantt{IPCS}} spectropolarimetry data.

\mansectionitem{{\mantt{IRFLUX}}}
        Apply flux calibration to an infrared polarization spectrum

\mansectionitem{{\mantt{IRISAP}}}
        Measure polarization within an aperture for {\mantt{IRIS}} data

\mansectionitem{{\mantt{IRISPOL}}}
        Reduce {\mantt{IRIS}} imaging polarimetry data.

\mansectionitem{{\mantt{LHATPOL}}}
        List Hatfield Polarimeter Infrared Data

\mansectionitem{{\mantt{LMERGE}}}
        Merge two polarization spectra.

\mansectionitem{{\mantt{LTCORR}}}
        Apply Light Time corrections to the time axis of a data set.

\mansectionitem{{\mantt{PHASEPLOT}}}
        Plot time series data against phase.

\mansectionitem{{\mantt{PPLOT}}}
        Plot a polarization spectrum as P, Theta

\mansectionitem{{\mantt{PTHETA}}}
        Output the P and Theta values for a polarization spectrum

\mansectionitem{{\mantt{QPLOT}}}
        Quick plot of time series data.

\mansectionitem{{\mantt{QUMERGE}}}
        Merge Q and U spectra into single dataset.

\mansectionitem{{\mantt{QUPLOT}}}
        Plot a polarization spectrum in the Q,U plane.

\mansectionitem{{\mantt{QUSUB}}}
        Subtract a Q,U vector from a polarization spectrum.

\mansectionitem{{\mantt{RCCDTS}}}
        Read {\mantt{AAO}} {\mantt{CCD}} Time Series data

\mansectionitem{{\mantt{RCGS2}}}
        Read {\mantt{CGS2}} Polarimetry Data

\mansectionitem{{\mantt{REVERSE}}}
        Reverse a spectrum in the wavelength axis.

\mansectionitem{{\mantt{RFIGARO}}}
        Read a Stokes Parameter Spectrum from a Figaro image

\mansectionitem{{\mantt{RHATHSP}}}
        Read Hatfield Polarimeter High Speed Photometry Data

\mansectionitem{{\mantt{RHATPOL}}}
        Read Hatfield Polarimeter Data

\mansectionitem{{\mantt{RHDSPLOT}}}
        Read {\mantt{ASCII}} files of Hatfield Polarimeter Data.

\mansectionitem{{\mantt{RHSP3}}}
        Read an {\mantt{HSP3}} tape

\mansectionitem{{\mantt{RIRPS}}}
        Read {\mantt{IRPS}} Photometry Data

\mansectionitem{{\mantt{ROTPA}}}
        Rotate the Position Angle of a Polarization Dataset

\mansectionitem{{\mantt{RTURKU}}}
        Read {\mantt{ASCII}} files of Data from the Turku {\mantt{UBVRI}} %
Polarimeter.

\mansectionitem{{\mantt{SCRUNCH}}}
        Rebin a Polarization Spectrum.

\mansectionitem{{\mantt{SKYSUB}}}
        Subtract Sky from a time series image dataset

\mansectionitem{{\mantt{SPLOT}}}
        Plot a polarization spectrum with a single Stokes parameter

\mansectionitem{{\mantt{SUBSET}}}
        Take a subset of a dataset in wavelength or time axes.

\mansectionitem{{\mantt{SUBTRACT}}}
        Subtract two Polarization spectra.

\mansectionitem{{\mantt{TBIN}}}
        Bin a time series

\mansectionitem{{\mantt{TCADD}}}
        Add Channels of a time series dataset

\mansectionitem{{\mantt{TDERIV}}}
        Calculate Time Derivative of a Dataset.

\mansectionitem{{\mantt{TEXTIN}}}
        Correct a time series dataset for extinction.

\mansectionitem{{\mantt{TLIST}}}
        List time series data to a file.

\mansectionitem{{\mantt{TMERGE}}}
        Merge two time series datasets.

\mansectionitem{{\mantt{TSETBAD}}}
        Interactively mark bad points in time series

\mansectionitem{{\mantt{TSEXTRACT}}}
        Optimal extraction of a light curve from a time series image

\mansectionitem{{\mantt{TSHIFT}}}
        Apply a time shift to a dataset.

\mansectionitem{{\mantt{TSPLOT}}}
        Plot time series data.

\mansectionitem{{\mantt{TSPROFILE}}}
        Determine a spatial profile from a time series image

\mansectionitem{{\mantt{XCOPY}}}
        Copy Wavelength Data from a Figaro Spectrum

\mansectionitem{{\mantt{SLIST}}}
        Output a polarization spectrum in the form of an {\mantt{ASCII}} file

\mansectionitem{{\mantt{IMPOL}}}
        Reduce {\mantt{IR}} Polarization images

\end{mansectionroutines}

\newpage

\section{Detailed Command Descriptions}
\label{app:details}

These command descriptions (and the TSPSUBS descriptions) were generated from
comments in the source code using William Lupton's MAN utility. In the
parameter lists numbers indicate positions of parameters on the command
line. H indicates a hidden parameter, one that is not prompted for but must
be explicitly specified on the command line. C indicates a parameter which
is conditional on the value of some other parameter (i.e. MAX is only prompted
for if AUTO is False).

\manroutine{{\mantt{BUILD3D}}}{Insert a figaro frame into a time series image}{%
BUILD3D}
\begin{manroutinedescription}
\manroutineitem{Function}{}
        Insert a figaro frame into a time series image

\manroutineitem{Description}{}
        {\mantt{BUILD3D}} is used to create a time series image from a number
        of figaro images. Each invocation of {\mantt{BUILD3D}} inserts one frame
        into the time series. A new time series dataset can be created
        by specifying the {\mantt{NEW}} parameter and the required number of %
frames
        The date and time of each frame is obtained from the {\mantt{FITS}} %
header
        if possible - otherwise it is prompted for.

\manroutineitem{Parameters}{}
\begin{manparametertable}
\manparameterentry{1}{{\mantt{FIGARO}}}{Char}     The Figaro files to insert.
\manparameterentry{2}{{\mantt{FRAME}}}{Integer}  The frame number at which to %
insert it.
\manparameterentry{3}{{\mantt{NEW}}}{Logical}  {\mantt{TRUE}} to create a new %
time series.
\manparameterentry{4}{{\mantt{OUTPUT}}}{{\mantt{TSP}}, {\mantt{3D}}}  The %
output time series dataset.
\manparameterentry{}{{\mantt{FRAMES}}}{Integer}  The number of frames in the %
time series.
\manparameterentry{}{{\mantt{UTDATE}}}{Char}     The {\mantt{UT}} date of the %
frame
\manparameterentry{}{{\mantt{UT}}}{Char}     The {\mantt{UT}} time of the frame

\end{manparametertable}
\manroutineitem{Support}{Jeremy Bailey, {\mantt{AAO}}}
\manroutineitem{Version date}{07/03/1992}
\end{manroutinedescription}
\manroutine{{\mantt{CALFIT}}}{Fit a calibration curve to a polarization %
spectrum}{CALFIT}
\begin{manroutinedescription}
\manroutineitem{Function}{}
        Fit a calibration curve to a polarization spectrum

\manroutineitem{Description}{}
        The {\mantt{AAO}} Spectropolarimeter allows the insertion of a polarizer
        which gives a {\mantt{100\%{}}} circular polarization for calibrating %
the
        efficiency of the instrument. {\mantt{CALFIT}} is used to fit a %
Chebyshev
        polynomial to an observed stokes parameter spectrum obtained
        with this calibrator. The fitted curve is output as another
        Stokes spectrum which may be used to calibrate other datasets
        using the {\mantt{CALIB}} command.

\manroutineitem{Parameters}{}
\begin{manparametertable}
\manparameterentry{1}{{\mantt{INPUT}}}{{\mantt{TSP}}, {\mantt{1D}}}  The input %
dataset, a spectrum with one
                               Stokes parameter which will be fitted.
\manparameterentry{2}{{\mantt{DEGREE}}}{Integer}  The degree of the polynomial %
to be fitted.
\manparameterentry{3}{{\mantt{OUTPUT}}}{{\mantt{TSP}}, {\mantt{1D}}}  The %
output dataset, equivalent in structure
                               to {\mantt{INPUT}}, but with Intensity array %
set to
                               unity, and the Stokes array containing
                               the fitted curve. The variance is set to
                               zero.

\end{manparametertable}
\manroutineitem{Support}{Jeremy Bailey, {\mantt{AAO}}}
\manroutineitem{Version date}{28/4/1988}
\end{manroutinedescription}
\manroutine{{\mantt{CALFITPA}}}{Fit a calibration curve to the polarization %
position angle}{CALFITPA}
\begin{manroutinedescription}
\manroutineitem{Function}{}
        Fit a calibration curve to the polarization position angle

\manroutineitem{Description}{}
        A polarimeter using a rotating superachromatic half-wave plate
        made on the Pancharatnam design will result in position angles
        which have a slight wavelength dependence due to the variation
        of the angle of the plates fast axis with wavelength. {\mantt{CALFITPA}}
        can be used to fit a calibration curve to an observation of
        an object with known position angle (e.g. an observation with
        a calibration polarizer) which can then be used to
        calibrate other data. A Chebyshev polynomial is fitted to the
        position angle data to obtain an output spectrum with
        can then be used as input to the {\mantt{CALPA}} command.

\manroutineitem{Parameters}{}
\begin{manparametertable}
\manparameterentry{1}{{\mantt{INPUT}}}{{\mantt{TSP}}, {\mantt{1D}}}  The input %
dataset, a polarization spectrum
                               which will be fitted to.
\manparameterentry{2}{{\mantt{DEGREE}}}{Integer}  The degree of the polynomial %
to be fitted.
\manparameterentry{3}{{\mantt{PA}}}{Real}     The position angle of the %
calibration source.
\manparameterentry{4}{{\mantt{OUTPUT}}}{{\mantt{TSP}}, {\mantt{1D}}}  The %
output dataset, equivalent in structure
                               to {\mantt{INPUT}}, but with Intensity array %
set to
                               unity, and the Stokes arrays containing
                               the fitted curve. The variance is set to
                               zero.

\end{manparametertable}
\manroutineitem{Support}{Jeremy Bailey, {\mantt{AAO}}}
\manroutineitem{Version date}{19/11/1991}
\end{manroutinedescription}
\manroutine{{\mantt{CALIB}}}{Efficiency Calibrate a Polarization Spectrum}{%
CALIB}
\begin{manroutinedescription}
\manroutineitem{Function}{}
        Efficiency Calibrate a Polarization Spectrum

\manroutineitem{Description}{}
        A polarization spectrum is corrected for instrument efficiency
        by applying a calibration curve obtained using the {\mantt{CALFIT}}
        command. The spectrum to be corrected may have any number
        of Stokes Parameters.

        {\mantt{CALIB}} leaves the intensity data unafected, but scales the
        Stokes parameters according to the calibration curve, and the
        variances of the Stokes parameters by the square of the calibration
        value.

\manroutineitem{Parameters}{}
\begin{manparametertable}
\manparameterentry{1}{{\mantt{INPUT}}}{{\mantt{TSP}}, {\mantt{1D}}}  The %
Polarization spectrum to be corrected.
\manparameterentry{2}{{\mantt{CALIB}}}{{\mantt{TSP}}, {\mantt{1D}}}  The %
calibration spectrum.
\manparameterentry{3}{{\mantt{OUTPUT}}}{{\mantt{TSP}}, {\mantt{1D}}}  The %
output corrected dataset.

\end{manparametertable}
\manroutineitem{Support}{Jeremy Bailey, {\mantt{AAO}}}
\manroutineitem{Version date}{19/11/1991}
\end{manroutinedescription}
\manroutine{{\mantt{CALPA}}}{Position Angle Calibrate a Polarization Spectrum}{%
CALPA}
\begin{manroutinedescription}
\manroutineitem{Function}{}
        Position Angle Calibrate a Polarization Spectrum

\manroutineitem{Description}{}
        A polarization spectrum is corrected for wavelength dependent
        position angle zero point by applying a calibration curve obtained
        using the {\mantt{CALFITPA}} command.

        This command is needed for polarimeters which are based on
        the use of a superachromatic half-wave plate, since such a
        plate shows significant cyclic  wavelength variations of the
        position of its fast axis.

\manroutineitem{Parameters}{}
\begin{manparametertable}
\manparameterentry{1}{{\mantt{INPUT}}}{{\mantt{TSP}}, {\mantt{1D}}}  The %
Polarization spectrum to be corrected.
\manparameterentry{2}{{\mantt{CALIB}}}{{\mantt{TSP}}, {\mantt{1D}}}  The %
calibration spectrum.
\manparameterentry{3}{{\mantt{OUTPUT}}}{{\mantt{TSP}}, {\mantt{1D}}}  The %
output corrected dataset.

\end{manparametertable}
\manroutineitem{Support}{}
        Jeremy Bailey, {\mantt{AAO}}

\manroutineitem{Version date}{}
        20/11/1991

\end{manroutinedescription}
\manroutine{{\mantt{CCD2POL}}}{Reduce {\mantt{CCD}} spectropolarimetry data.}{%
CCD2POL}
\begin{manroutinedescription}
\manroutineitem{Function}{}
        Reduce {\mantt{CCD}} spectropolarimetry data.

\manroutineitem{Description}{}
        {\mantt{CCD2POL}} reduces data obtained with the {\mantt{AAO}} Half-%
wave plate
        spectropolarimeter with the {\mantt{CCD}} as detector. The data for a
        single observation consists of four Figaro files containing the
        frames for plate position 0, 45, 22.5 and 67.5 degrees. Within each
        frame there are four spectra corresponding to the O and E rays for
        each of two apertures (A and B). These spectra are combined
        to derive a polarization spectrum in {\mantt{TSP}} format.
        The {\mantt{CCD}} data are expected to be in raw {\mantt{CCD}} format %
which is
        the wrong way round for Figaro. i.e. the Y axis is the dispersion
        direction. Thus if the data is preprocessed using Figaro it will
        have to be rotated back.

        Two different algorithms may be selected for the polarimetry
        reduction. The two algorithms differ in the method used to
        compensate for transparency variations between the observations
        at the two plate positions.

        The variances on the polarization data are calculated from photon
        statistics plus readout noise.

\manroutineitem{Parameters}{}
\begin{manparametertable}
\manparameterentry{1}{{\mantt{POS1}}}{Char}     The Figaro data file for %
position 0.0.
\manparameterentry{2}{{\mantt{POS2}}}{Char}     The Figaro data file for %
position 45.0.
\manparameterentry{3}{{\mantt{POS3}}}{Char}     The Figaro data file for %
position 22.5.
\manparameterentry{4}{{\mantt{POS4}}}{Char}     The Figaro data file for %
position 67.5.
\manparameterentry{}{{\mantt{ASTART}}}{Integer}  The Start channel for the A %
aperture data.
\manparameterentry{}{{\mantt{BSTART}}}{Integer}  The Start channel for the B %
aperture data.
\manparameterentry{}{{\mantt{OESEP}}}{Integer}  The number of channels %
separating O and
                               E spectra.
\manparameterentry{}{{\mantt{WIDTH}}}{Integer}  The number of channels to %
include in each
                               spectrum.
\manparameterentry{}{{\mantt{APERTURE}}}{Char}     The aperture containing the %
star (A or B).
\manparameterentry{}{{\mantt{BIAS}}}{Real}     Bias level to be subtracted %
from data.
\manparameterentry{}{{\mantt{READNOISE}}}{Real}     {\mantt{CCD}} readout %
noise (electrons/pixel).
\manparameterentry{}{{\mantt{PHOTADU}}}{Real}     Photons per {\mantt{ADU}} %
for the {\mantt{CCD}} data.
\manparameterentry{}{{\mantt{ALGORITHM}}}{Char}     The Algorithm to use for %
stokes
                               parameter calculation ({\mantt{OLD}}, {\mantt{%
RATIO}})
\manparameterentry{}{{\mantt{OUTPUT}}}{{\mantt{TSP}}, {\mantt{1D}}}  The %
Output dataset.

\end{manparametertable}
\manroutineitem{Support}{Jeremy Bailey, {\mantt{AAO}}}
\manroutineitem{Version date}{2/10/1991}
\end{manroutinedescription}
\manroutine{{\mantt{CCD2STOKES}}}{Reduce {\mantt{CCD}} spectropolarimetry data.%
}{CCD2STOKES}
\begin{manroutinedescription}
\manroutineitem{Function}{}
        Reduce {\mantt{CCD}} spectropolarimetry data.

\manroutineitem{Description}{}
        {\mantt{CCD2STOKES}} reduces data obtained with the {\mantt{AAO}} %
Pockels cell
        spectropolarimeter with the {\mantt{CCD}} as detector. The data for a
        single observation consists of two Figaro files containing the
        A and B state frames. Within each A and B frame there are four
        spectra corresponding to the O and E rays for each of two
        apertures (A and B). These spectra are combined
        to derive a Stokes parameter spectrum in {\mantt{TSP}} format.
        The {\mantt{CCD}} data are expected to be in raw {\mantt{CCD}} format %
which is
        the wrong way round for Figaro. i.e. the Y axis is the dispersion
        direction. Thus if the data is preprocessed using Figaro it will
        have to be rotated back.

        Two different algorithms may be selected for the polarimetry
        reduction. The two algorithms differ in the method used to
        compensate for transparency variations between the observations
        at the two plate positions.

        {\mantt{CCD2STOKES}} can also be used to reduce circular polarization
        data obtained with the wave-plate polarimeter. The equivalent of
        the A and B state data are the frames taken at two positions of
        a quarter-wave plate spaced by 90 degrees.

\manroutineitem{Parameters}{}
\begin{manparametertable}
\manparameterentry{1}{{\mantt{AFIGARO}}}{Char}     The Figaro A-state data file.
\manparameterentry{2}{{\mantt{BFIGARO}}}{Char}     The Figaro B-state data file.
\manparameterentry{}{{\mantt{ASTART}}}{Integer}  The Start channel for the A %
aperture data.
\manparameterentry{}{{\mantt{BSTART}}}{Integer}  The Start channel for the B %
aperture data.
\manparameterentry{}{{\mantt{OESEP}}}{Integer}  The number of channels %
separating O and
                               E spectra.
\manparameterentry{}{{\mantt{WIDTH}}}{Integer}  The number of channels to %
include in each
                               spectrum.
\manparameterentry{}{{\mantt{APERTURE}}}{Char}     The aperture containing the %
star (A or B).
\manparameterentry{}{{\mantt{BIAS}}}{Real}     Bias level to be subtracted %
from data.
\manparameterentry{}{{\mantt{READNOISE}}}{Real}     {\mantt{CCD}} readout %
noise (electrons/pixel).
\manparameterentry{}{{\mantt{PHOTADU}}}{Real}     Photons per {\mantt{ADU}} %
for the {\mantt{CCD}} data.
\manparameterentry{}{{\mantt{STOKESPAR}}}{Char}     The Stokes parameter (Q,U,%
V).
\manparameterentry{}{{\mantt{ALGORITHM}}}{Char}     The Algorithm to use for %
stokes
                               parameter calculation ({\mantt{OLD}}, {\mantt{%
RATIO}})
\manparameterentry{}{{\mantt{OUTPUT}}}{{\mantt{TSP}}, {\mantt{1D}}}  The %
Output dataset.

\end{manparametertable}
\manroutineitem{Support}{Jeremy Bailey, {\mantt{AAO}}}
\manroutineitem{Version date}{20/11/1991}
\end{manroutinedescription}
\manroutine{{\mantt{CCDPHOT}}}{Photometry of a star on a time series image}{%
CCDPHOT}
\begin{manroutinedescription}
\manroutineitem{Function}{}
        Photometry of a star on a time series image

\manroutineitem{Description}{}
        Measure the brightness of a star on each frame of a time
        series image, and generate a {\mantt{2D}} {\mantt{TSP}} dataset %
containing the
        resulting time series photometry. The data should previously
        have been sky subtracted.

        The photometry is done by means of summing the signal
        within a specified aperture. An alternative method is to
        use the commands {\mantt{TSPROFILE}} and {\mantt{TSEXTRACT}} which %
perform
        photometry weighting according to a smoothed spatial
        profile.

\manroutineitem{Parameters}{}
\begin{manparametertable}
\manparameterentry{1}{{\mantt{INPUT}}}{{\mantt{TSP}}, {\mantt{3D}}}   The time %
series image dataset.
\manparameterentry{2}{{\mantt{OUTPUT}}}{{\mantt{TSP}}, {\mantt{2D}}}   The %
output photometry dataset
\manparameterentry{3}{X}{Real}      X position of centre of star
\manparameterentry{4}{Y}{Real}      Y position of centre of star
\manparameterentry{5}{{\mantt{RADIUS}}}{Real}      Radius of aperture (pixels)
\manparameterentry{6}{{\mantt{LAMBDA}}}{Real}      Wavelength of observation (%
microns)
\manparameterentry{7}{{\mantt{FLUXCAL}}}{Real}      Counts per Jansky

\end{manparametertable}
\manroutineitem{Support}{Jeremy Bailey, {\mantt{AAO}}}
\manroutineitem{Version date}{20/11/1991}
\end{manroutinedescription}
\manroutine{{\mantt{CCDPOL}}}{Polarimetry of a star on a time series image}{%
CCDPOL}
\begin{manroutinedescription}
\manroutineitem{Function}{}
        Polarimetry of a star on a time series image

\manroutineitem{Description}{}
        Measure the brightness of the O and E images of a star on each
        frame of a time series image, and generate a {\mantt{2D}} {\mantt{TSP}} %
dataset
        containing the resulting time series polarimetry. The data should
        previously have been sky subtracted.

        Aperture photometry is performed on each of the two star images
        and used to derive the polarization. A polarization offset can
        be applied to correct for instrumental effects.

\manroutineitem{Parameters}{}
\begin{manparametertable}
\manparameterentry{1}{{\mantt{INPUT}}}{{\mantt{TSP}}, {\mantt{3D}}}   The time %
series image dataset.
\manparameterentry{2}{{\mantt{OUTPUT}}}{{\mantt{TSP}}, {\mantt{2D}}}   The %
output photometry dataset
\manparameterentry{3}{{\mantt{XE}}}{Real}      X position of centre of E image
\manparameterentry{4}{{\mantt{YE}}}{Real}      Y position of centre of E image
\manparameterentry{5}{{\mantt{RADIUS}}}{Real}      Radius of aperture (pixels)
\manparameterentry{6}{{\mantt{XO}}}{Real}      X position of centre of O image
\manparameterentry{7}{{\mantt{YO}}}{Real}      Y position of centre of O image
\manparameterentry{8}{{\mantt{LAMBDA}}}{Real}      Wavelength of observation (%
microns)
\manparameterentry{9}{{\mantt{STOKESPAR}}}{Real}      Stokes Parameter (Q,U,V)
\manparameterentry{10}{{\mantt{OFFSET}}}{Real}      Polarization offset (per %
cent)
\manparameterentry{11}{{\mantt{FLUXCAL}}}{Real}      Counts per Jansky

\end{manparametertable}
\manroutineitem{Support}{Jeremy Bailey, {\mantt{JAC}}}
\manroutineitem{Version date}{1/11/1989}
\end{manroutinedescription}
\manroutine{{\mantt{CGS4POL}}}{Reduce {\mantt{CGS4}} spectropolarimetry data.}{%
CGS4POL}
\begin{manroutinedescription}
\manroutineitem{Function}{}
        Reduce {\mantt{CGS4}} spectropolarimetry data.

\manroutineitem{Description}{}
        {\mantt{CGS4POL}} reduces data obtained with the {\mantt{CGS4}} %
instrument at {\mantt{UKIRT}}
        used in its spectropolarimetry mode. The data for a
        single observation consists of four Figaro files containing the
        frames for plate position 0, 45, 22.5 and 67.5 degrees. Within each
        frame there should be two spectra corresponding to two slit
        positions. These spectra are combined
        to derive a polarization spectrum in {\mantt{TSP}} format.


\manroutineitem{Parameters}{}
\begin{manparametertable}
\manparameterentry{1}{{\mantt{POS1}}}{Char}     The Figaro data file for %
position 0.0.
\manparameterentry{2}{{\mantt{POS2}}}{Char}     The Figaro data file for %
position 45.0.
\manparameterentry{3}{{\mantt{POS3}}}{Char}     The Figaro data file for %
position 22.5.
\manparameterentry{4}{{\mantt{POS4}}}{Char}     The Figaro data file for %
position 67.5.
\manparameterentry{}{{\mantt{ASTART}}}{Integer}  The Start channel for the A %
aperture data.
\manparameterentry{}{{\mantt{BSTART}}}{Integer}  The Start channel for the B %
aperture data.
\manparameterentry{}{{\mantt{WIDTH}}}{Integer}  The number of channels to %
include in each
                               spectrum.
\manparameterentry{}{{\mantt{APERTURE}}}{Char}     The aperture containing the %
star (A or B).
\manparameterentry{}{{\mantt{OUTPUT}}}{{\mantt{TSP}}, {\mantt{1D}}}  The %
Output dataset.

\end{manparametertable}
\manroutineitem{Support}{Jeremy Bailey, {\mantt{AAO}}}
\manroutineitem{Version date}{10/11/1992}
\end{manroutinedescription}
\manroutine{{\mantt{CMULT}}}{Multiply a polarization spectrum by a constant}{%
CMULT}
\begin{manroutinedescription}
\manroutineitem{Function}{}
        Multiply a polarization spectrum by a constant

\manroutineitem{Description}{}
        The intensity and stokes parameters are multiplied
        by the specified factor and the variances are multiplied
        by the square of the specified factor

\manroutineitem{Parameters}{}
\begin{manparametertable}
\manparameterentry{1}{{\mantt{INPUT}}}{{\mantt{TSP}}, {\mantt{1D}}}  The input %
spectrum to be multiplied.
\manparameterentry{2}{{\mantt{FACTOR}}}{Real}     Factor to multiply by
\manparameterentry{3}{{\mantt{OUTPUT}}}{{\mantt{TSP}}, {\mantt{1D}}}  The %
Output dataset.

\end{manparametertable}
\manroutineitem{Support}{}
         Jeremy Bailey, {\mantt{AAO}}

\manroutineitem{Version date}{}
          20/11/1990

\end{manroutinedescription}
\manroutine{{\mantt{COMBINE}}}{Combine two Polarization Datasets}{COMBINE}
\begin{manroutinedescription}
\manroutineitem{Function}{}
        Combine two Polarization Datasets

\manroutineitem{Description}{}
        Two Polarization datasets are added to form a new one of
        higher signal to noise ratio. Any number of Stokes parameters
        may be present in the data, but only Stokes parameters present
        in both spectra will appear in the output.

        {\mantt{COMBINE}} adds the intensity, Stokes parameters and variances
        and is therefore appropriate for combining data in the form
        of {\mantt{IPCS}} or {\mantt{CCD}} counts, but not for combining flux %
calibrated
        data.

\manroutineitem{Parameters}{}
\begin{manparametertable}
\manparameterentry{1}{{\mantt{INPUT1}}}{{\mantt{TSP}}, nD}\begin{manitemize}
\manitemizeitem The first input dataset.
\end{manitemize}
\manparameterentry{2}{{\mantt{INPUT2}}}{{\mantt{TSP}}, nD}\begin{manitemize}
\manitemizeitem The second input dataset.
\end{manitemize}
\manparameterentry{3}{{\mantt{OUTPUT}}}{{\mantt{TSP}}, nD}\begin{manitemize}
\manitemizeitem The output dataset.
\end{manitemize}

\end{manparametertable}
\manroutineitem{Support}{Jeremy Bailey, {\mantt{AAO}}}
\manroutineitem{Version date}{8/5/1993}
\end{manroutinedescription}
\manroutine{{\mantt{DIVIDE}}}{Divide a polarization spectrum by an intensity %
spectrum}{DIVIDE}
\begin{manroutinedescription}
\manroutineitem{Function}{}
        Divide a polarization spectrum by an intensity spectrum

\manroutineitem{Description}{}
        Divide a polarization spectrum by the intensity spectrum from another
        dataset. This can be used to divide data by a smooth spectrum star
        to remove atmospheric features.

        The intensity and Stokes parameters of the first spectrum are
        divided by the intensity of the second spectrum. The variances are
        also scaled accordingly. The spectrum being divided by is assumed
        to have no errors.

\manroutineitem{Parameters}{}
\begin{manparametertable}
\manparameterentry{1}{{\mantt{INPUT1}}}{{\mantt{TSP}}, {\mantt{1D}}}  The %
input spectrum to be divided.
\manparameterentry{2}{{\mantt{INPUT2}}}{{\mantt{TSP}}, {\mantt{1D}}}  The %
spectrum to divide by.
\manparameterentry{3}{{\mantt{OUTPUT}}}{{\mantt{TSP}}, {\mantt{1D}}}  The %
Output dataset.

\end{manparametertable}
\manroutineitem{Support}{}
         Jeremy Bailey, {\mantt{AAO}}

\manroutineitem{Version date}{}
         5/12/1991

\end{manroutinedescription}
\manroutine{{\mantt{DSTOKES}}}{Delete a Stokes parameter from a dataset.}{%
DSTOKES}
\begin{manroutinedescription}
\manroutineitem{Function}{}
        Delete a Stokes parameter from a dataset.

\manroutineitem{Description}{}
        Delete a Stokes component from the polarimetry
        extension of a data structure.

\manroutineitem{Parameters}{}
\begin{manparametertable}
\manparameterentry{1}{{\mantt{INPUT}}}{{\mantt{TSP}}, nD}\begin{manitemize}
\manitemizeitem The input Stokes dataset.
\end{manitemize}
\manparameterentry{2}{{\mantt{STOKESPAR}}}{Char}\begin{manitemize}
\manitemizeitem The Stokes parameter (Q, U or V)
\end{manitemize}
\manparameterentry{3}{{\mantt{OUTPUT}}}{{\mantt{TSP}}, nD}\begin{manitemize}
\manitemizeitem The output dataset.
\end{manitemize}

\end{manparametertable}
\manroutineitem{Support}{Jeremy Bailey, {\mantt{AAO}}}
\manroutineitem{Version date}{8/3/1992}
\end{manroutinedescription}
\manroutine{{\mantt{EPLOT}}}{Plot a polarization spectrum as P, Theta with %
error bars}{EPLOT}
\begin{manroutinedescription}
\manroutineitem{Function}{}
        Plot a polarization spectrum as P, Theta with error bars

\manroutineitem{Description}{}
        {\mantt{EPLOT}} produces a plot of a polarization spectrum. The plot is
        divided into three panels. The lower panel is the total intensity,
        the center panel is the percentage polarization, the top panel
        is the position angle in degrees.
        Plotting is done with the {\mantt{NCAR}}/{\mantt{SGS}}/{\mantt{GKS}} %
graphics system.

\manroutineitem{Parameters}{}
\begin{manparametertable}
\manparameterentry{1}{{\mantt{INPUT}}}{{\mantt{TSP}}, {\mantt{1D}}}  The input %
dataset, a spectrum which must
                               have Q and U Stokes parameters present.
\manparameterentry{2}{{\mantt{DEVICE}}}{Device}   The Graphics device (any %
valid {\mantt{GKS}} device).
\manparameterentry{3}{{\mantt{LABEL}}}{Char}     A label for the plot.
\manparameterentry{}{{\mantt{AUTO}}}{Logical}  True if plot is to be autoscaled.
\manparameterentry{C}{{\mantt{IMIN}}}{Real}     Minimum Intensity level to plot.
\manparameterentry{C}{{\mantt{IMAX}}}{Real}     Maximum Intensity level to plot.
\manparameterentry{C}{{\mantt{PMIN}}}{Real}     Minimum Polarization level to %
plot.
\manparameterentry{C}{{\mantt{PMAX}}}{Real}     Maximum Polarization level to %
plot.
\manparameterentry{H}{{\mantt{THETA}}}{Real}     Shift in angle to apply to %
theta plot.
                               Plot range is {\mantt{THETA}} to 180+{\mantt{%
THETA}}.

\end{manparametertable}
\manroutineitem{Support}{}
         Jeremy Bailey, {\mantt{AAO}}

\manroutineitem{Version date}{}
         6/12/1991

\end{manroutinedescription}
\manroutine{{\mantt{EXTIN}}}{Correct a polarization spectrum for extinction}{%
EXTIN}
\begin{manroutinedescription}
\manroutineitem{Function}{}
        Correct a polarization spectrum for extinction

\manroutineitem{Description}{}
        Correct a polarization spectrum for extinction using a coefficient
        spectrum containing the interpolated extinxtion coefficients over
        the wavelength range of the spectrum. This can be generated using
        Figaro as described in the section on extinction in the Figaro
        manual.

\manroutineitem{Parameters}{}
\begin{manparametertable}
\manparameterentry{1}{{\mantt{INPUT}}}{{\mantt{TSP}}, {\mantt{1D}}}  The input %
spectrum to be corrected.
\manparameterentry{2}{{\mantt{COEFF}}}{Char}     The name of the Figaro file %
containing
                                the coefficient spectrum.
\manparameterentry{3}{{\mantt{AIRMASS}}}{Real}     The air-mass (approximately %
sec z) of
                                the observation.
\manparameterentry{4}{{\mantt{OUTPUT}}}{{\mantt{TSP}}, {\mantt{1D}}}  The %
Output dataset.

\end{manparametertable}
\manroutineitem{Support}{}
           Jeremy Bailey, {\mantt{AAO}}

\manroutineitem{Version date}{}
           6/12/1991

\end{manroutinedescription}
\manroutine{{\mantt{FLCONV}}}{Convert a flux calibrated spectrum to f-lambda}{%
FLCONV}
\begin{manroutinedescription}
\manroutineitem{Function}{}
        Convert a flux calibrated spectrum to f-lambda

\manroutineitem{Description}{}
        A polarization spectrum flux calibrated in f-nu (Jy, mJy or
        micro-Jy) is converted to f-lambda (ergs/sec/cm**2/A). The
        units of the original data are sensed from the {\mantt{UNITS}} field
        of the axis structure. The spectrum must have a wavelength
        axis in Angstroms.

        This program is similar to the Figaro command of the same name,
        but applies the calibration to the Stokes parameters as well
        as to the intensity data.

\manroutineitem{Parameters}{}
\begin{manparametertable}
\manparameterentry{1}{{\mantt{INPUT}}}{{\mantt{TSP}}, {\mantt{1D}}}  The input %
spectrum to be converted.
\manparameterentry{2}{{\mantt{OUTPUT}}}{{\mantt{TSP}}, {\mantt{1D}}}  The %
Output dataset.

\end{manparametertable}
\manroutineitem{Support}{}
        Jeremy Bailey, {\mantt{AAO}}

\manroutineitem{Version date}{}
        6/12/1991

\end{manroutinedescription}
\manroutine{{\mantt{FLIP}}}{Invert the sign of the Stokes parameter in a %
spectrum.}{FLIP}
\begin{manroutinedescription}
\manroutineitem{Function}{}
        Invert the sign of the Stokes parameter in a spectrum.

\manroutineitem{Description}{}
        The Stokes array in the input dataset is sign changed to
        produce the Stokes array of the output dataset.

\manroutineitem{Parameters}{}
\begin{manparametertable}
\manparameterentry{1}{{\mantt{INPUT}}}{{\mantt{TSP}}, {\mantt{1D}}}\begin{%
manitemize}
\manitemizeitem The input Stokes dataset.
\end{manitemize}
\manparameterentry{2}{{\mantt{OUTPUT}}}{{\mantt{TSP}}, {\mantt{1D}}}\begin{%
manitemize}
\manitemizeitem The output dataset.
\end{manitemize}

\end{manparametertable}
\manroutineitem{Support}{Jeremy Bailey, {\mantt{AAO}}}
\manroutineitem{Version date}{27/4/1988}
\end{manroutinedescription}
\manroutine{{\mantt{FPLOT}}}{Plot a polarization spectrum as Polarized %
Intensity}{FPLOT}
\begin{manroutinedescription}
\manroutineitem{Function}{}
        Plot a polarization spectrum as Polarized Intensity

\manroutineitem{Description}{}
        {\mantt{FPLOT}} produces a plot of a polarization spectrum. The plot is
        divided into two panels. The lower panel is the total intensity,
        the top panel is the polarized intensity (or polarized flux).

        The polarized intensity data is binned into fixed size bins of
        size specified by the {\mantt{BINSIZE}} parameter. Plotting is done %
with the
        {\mantt{NCAR}}/{\mantt{SGS}}/{\mantt{GKS}} graphics system.

\manroutineitem{Parameters}{}
\begin{manparametertable}
\manparameterentry{1}{{\mantt{INPUT}}}{{\mantt{TSP}}, {\mantt{1D}}}  The input %
dataset, a spectrum which must
                               have both Q and U Stokes parameters or
                               the V Stokes parameter present.
\manparameterentry{2}{{\mantt{BINSIZE}}}{Integer}  The number of spectral %
channels per bin
\manparameterentry{3}{{\mantt{DEVICE}}}{Device}   The Graphics device (any %
valid {\mantt{GKS}} device).
\manparameterentry{4}{{\mantt{LABEL}}}{Char}     A label for the plot.
\manparameterentry{}{{\mantt{AUTO}}}{Logical}  True if plot is to be autoscaled.
\manparameterentry{C}{{\mantt{IMIN}}}{Real}     Minimum Intensity level to plot.
\manparameterentry{C}{{\mantt{IMAX}}}{Real}     Maximum Intensity level to plot.
\manparameterentry{C}{{\mantt{PMIN}}}{Real}     Minimum Polarization level to %
plot.
\manparameterentry{C}{{\mantt{PMAX}}}{Real}     Maximum Polarization level to %
plot.

\end{manparametertable}
\manroutineitem{Support}{}
         Jeremy Bailey, {\mantt{AAO}}

\manroutineitem{Version date}{}
         9/12/1991

\end{manroutinedescription}
\manroutine{{\mantt{IMOTION}}}{Analyze the image motion in a time series image}%
{IMOTION}
\begin{manroutinedescription}
\manroutineitem{Function}{}
        Analyze the image motion in a time series image

\manroutineitem{Description}{}
        Given a time series image produce an output time series
        which is a measure of the image motion in the 2 axes.
        The first channel of the output time series is the image
        motion in X and the second channel is the image motion in Y

\manroutineitem{Parameters}{}
\begin{manparametertable}
\manparameterentry{1}{{\mantt{INPUT}}}{{\mantt{TSP}}, {\mantt{3D}}}   The time %
series image dataset.
\manparameterentry{2}{{\mantt{TEMPLATE}}}{{\mantt{TSP}}, {\mantt{2D}}}   An %
image to be used as a template against
                                which motion will be measured.
\manparameterentry{3}{{\mantt{OUTPUT}}}{{\mantt{TSP}}, {\mantt{2D}}}   The %
output photometry dataset.
\manparameterentry{4}{X}{Real}      X position of centre of star
\manparameterentry{5}{Y}{Real}      Y position of centre of star
\manparameterentry{6}{{\mantt{RADIUS}}}{Real}      Radius of aperture (pixels)

\end{manparametertable}
\manroutineitem{Support}{}
         Jeremy Bailey, {\mantt{AAO}}

\manroutineitem{Version date}{}
         16/11/1991

\end{manroutinedescription}
\manroutine{{\mantt{IMPOL}}}{Reduce {\mantt{IR}} Polarization images}{IMPOL}
\begin{manroutinedescription}
\manroutineitem{Function}{}
        Reduce {\mantt{IR}} Polarization images

\manroutineitem{Description}{}
        {\mantt{IMPOL}} derives a polarization image from a set of four %
observations
        made with a rotating half-wave plate polarimeter at angles of
        0, 22.5, 45 and 67.5 degrees. It is used to reduce polarization
        imaging data obtained with the {\mantt{IRIS}} {\mantt{IR}} camera and %
half-wave plate
        polarimeter at the {\mantt{AAT}} or the {\mantt{IRPOL}}/{\mantt{IRCAM}} %
polarimeter at {\mantt{UKIRT}}.
        It should also be useable with other similar instruments (not
        necessarily in the {\mantt{IR}}).

        The input images should be {\mantt{NDF}} files (not Figaro .{\mantt{%
DST}} files).

\manroutineitem{Parameters}{}
\begin{manparametertable}
\manparameterentry{1}{{\mantt{POS1}}}{Char}     The input image for position 0.%
0.
\manparameterentry{2}{{\mantt{POS2}}}{Char}     The input image for position 45%
.0.
\manparameterentry{3}{{\mantt{POS3}}}{Char}     The input image for position 22%
.5.
\manparameterentry{4}{{\mantt{POS4}}}{Char}     The input image for position 67%
.5.
\manparameterentry{}{{\mantt{OUTPUT}}}{{\mantt{TSP}}, {\mantt{2D}}}  The %
Output dataset.

\end{manparametertable}
\manroutineitem{Support}{Jeremy Bailey, {\mantt{AAO}}}
\manroutineitem{Version date}{2/6/1992}
\end{manroutinedescription}
\manroutine{{\mantt{IPCS2STOKES}}}{Reduce {\mantt{IPCS}} spectropolarimetry %
data.}{IPCS2STOKES}
\begin{manroutinedescription}
\manroutineitem{Function}{}
        Reduce {\mantt{IPCS}} spectropolarimetry data.

\manroutineitem{Description}{}
        {\mantt{IPCS2STOKES}} reduces data obtained with the {\mantt{AAO}} %
Pockels cell
        spectropolarimeter with the {\mantt{IPCS}} as detector. The data is read
        in the form of Figaro files each containing a pair of A and B
        state frames forming a single observation. Within each A and
        B frame there are four spectra corresponding to the O and E rays
        for each of two apertures (A and B). These spectra are combined
        to derive a Stokes parameter spectrum in {\mantt{TSP}} format.

\manroutineitem{Parameters}{}
\begin{manparametertable}
\manparameterentry{1}{{\mantt{FIGARO}}}{Char}     The input Figaro data file.
\manparameterentry{}{{\mantt{ASTART}}}{Integer}  The Start channel for the A %
aperture data.
\manparameterentry{}{{\mantt{BSTART}}}{Integer}  The Start channel for the B %
aperture data.
\manparameterentry{}{{\mantt{OESEP}}}{Integer}  The number of channels %
separating O and
                               E spectra.
\manparameterentry{}{{\mantt{WIDTH}}}{Integer}  The number of channels to %
include in each
                               spectrum.
\manparameterentry{}{{\mantt{APERTURE}}}{Char}     The aperture containing the %
star (A or B).
\manparameterentry{}{{\mantt{STOKESPAR}}}{Char}     The Stokes parameter (Q,U,%
V).
\manparameterentry{}{{\mantt{OUTPUT}}}{{\mantt{TSP}}, {\mantt{1D}}}  The %
Output dataset.

\end{manparametertable}
\manroutineitem{Support}{}
         Jeremy Bailey, {\mantt{AAO}}

\manroutineitem{Version date}{}
         27/4/1988

\end{manroutinedescription}
\manroutine{{\mantt{IRFLUX}}}{Apply flux calibration to an infrared %
polarization spectrum}{IRFLUX}
\begin{manroutinedescription}
\manroutineitem{Function}{}
        Apply flux calibration to an infrared polarization spectrum

\manroutineitem{Description}{}
        {\mantt{IRFLUX}} flux calibrates a polarization spectrum using a
        calibration spectrum (normally a standard star observation)
        which is assumed to be a black body.
        The parameters of the black body are specified as a temperature,
        and magnitude in one of the standard bands. As an alternative to
        the magnitude the flux at a specified wavelength may be given.

\manroutineitem{Parameters}{}
\begin{manparametertable}
\manparameterentry{1}{{\mantt{INPUT}}}{{\mantt{TSP}}, {\mantt{1D}}}  The input %
spectrum to be calibrated.
\manparameterentry{2}{{\mantt{CALSPECT}}}{{\mantt{TSP}}, {\mantt{1D}}}  The %
calibration spectrum.
\manparameterentry{3}{{\mantt{TEMP}}}{Real}     Temperature of black body.
\manparameterentry{4}{{\mantt{CALTYPE}}}{Char}     The type of calibration %
data. A single
                                character as follows:
\end{manparametertable}
\begin{mantwocolumntable}
\mantwocolumnentry{J}Magnitude in J band
\mantwocolumnentry{H}Magnitude in H band
\mantwocolumnentry{K}Magnitude in H band
\mantwocolumnentry{L}Magnitude in L' band
\mantwocolumnentry{M}Magnitude in M band
\mantwocolumnentry{F}Flux at specified wavelength
\end{mantwocolumntable}
\begin{manparametertable}
\manparameterentry{5}{{\mantt{MAG}}}{Real}     The magnitude of the standard.
\manparameterentry{}{{\mantt{FLUX}}}{Real}     Flux of standard at calibration %
wavelength.
\manparameterentry{}{{\mantt{WAVE}}}{Real}     Calibration wavelength.
\manparameterentry{6}{{\mantt{OUTPUT}}}{{\mantt{TSP}}, {\mantt{1D}}}  The %
Output dataset.

\end{manparametertable}
\manroutineitem{Support}{Jeremy Bailey, {\mantt{AAO}}}
\manroutineitem{Version date}{20/9/1990}
\end{manroutinedescription}
\manroutine{{\mantt{IRISAP}}}{Measure polarization within an aperture for {%
\mantt{IRIS}} data}{IRISAP}
\begin{manroutinedescription}
\manroutineitem{Function}{}
        Measure polarization within an aperture for {\mantt{IRIS}} data

\manroutineitem{Description}{}
        {\mantt{IRISAP}} reduces data obtained with the {\mantt{AAT}} {\mantt{%
IRIS}} polarimeter
        using the wollaston  prism polarizer. The data for a
        single observation consists of four Figaro files containing the
        frames for plate position 0, 45, 22.5 and 67.5 degrees. Within each
        frame are selected two star images corresponding to the O and E rays
        for the same star. The polarization is derived for these

        Two different algorithms may be selected for the polarimetry
        reduction. The two algorithms differ in the method used to
        compensate for transparency variations between the observations
        at the two plate positions.


\manroutineitem{Parameters}{}
\begin{manparametertable}
\manparameterentry{1}{{\mantt{POS1}}}{Char}     The Figaro data file for %
position 0.0.
\manparameterentry{2}{{\mantt{POS2}}}{Char}     The Figaro data file for %
position 45.0.
\manparameterentry{3}{{\mantt{POS3}}}{Char}     The Figaro data file for %
position 22.5.
\manparameterentry{4}{{\mantt{POS4}}}{Char}     The Figaro data file for %
position 67.5.
\manparameterentry{}{X}{Integer}  X coordinate of centre of aperture
\manparameterentry{}{Y}{Integer}  Y coordinate of centre of aperture
\manparameterentry{}{R}{Real}     Radius of aperture
\manparameterentry{}{{\mantt{XSEP}}}{Integer}  {\mantt{OE}} separation vector %
in X
\manparameterentry{}{{\mantt{YSEP}}}{Integer}  {\mantt{OE}} separation vector %
in Y

\end{manparametertable}
\manroutineitem{Support}{Jeremy Bailey, {\mantt{AAO}}}
\manroutineitem{Version date}{4/5/1993}
\end{manroutinedescription}
\manroutine{{\mantt{IRISPOL}}}{Reduce {\mantt{IRIS}} imaging polarimetry data.}%
{IRISPOL}
\begin{manroutinedescription}
\manroutineitem{Function}{}
        Reduce {\mantt{IRIS}} imaging polarimetry data.

\manroutineitem{Description}{}
        {\mantt{IRISPOL}} reduces data obtained with the {\mantt{AAT}} {\mantt{%
IRIS}} polarimeter
        using the wolaston  prism poarizer. The data for a
        single observation consists of four Figaro files containing the
        frames for plate position 0, 45, 22.5 and 67.5 degrees. Within each
        frame are selected two images corresponding to the O and E rays for
        a single mask slot. These spectra are combined
        to derive a polarization image in {\mantt{TSP}} format.

        Two different algorithms may be selected for the polarimetry
        reduction. The two algorithms differ in the method used to
        compensate for transparency variations between the observations
        at the two plate positions.

        The variances on the polarization data are calculated from photon
        statistics plus readout noise.

\manroutineitem{Parameters}{}
\begin{manparametertable}
\manparameterentry{1}{{\mantt{POS1}}}{Char}     The Figaro data file for %
position 0.0.
\manparameterentry{2}{{\mantt{POS2}}}{Char}     The Figaro data file for %
position 45.0.
\manparameterentry{3}{{\mantt{POS3}}}{Char}     The Figaro data file for %
position 22.5.
\manparameterentry{4}{{\mantt{POS4}}}{Char}     The Figaro data file for %
position 67.5.
\manparameterentry{}{{\mantt{X1}}}{Integer}  X coordinate of the bottom left %
corner of block
\manparameterentry{}{{\mantt{Y1}}}{Integer}  Y coordinate of the bottom left %
corner of block
\manparameterentry{}{{\mantt{WIDTH}}}{Integer}  Width of the block
\manparameterentry{}{{\mantt{HEIGHT}}}{Integer}  Height of the block
\manparameterentry{}{{\mantt{XSEP}}}{Integer}  {\mantt{OE}} separation vector %
in X
\manparameterentry{}{{\mantt{YSEP}}}{Integer}  {\mantt{OE}} separation vector %
in Y
\manparameterentry{}{{\mantt{ALGORITHM}}}{Char}     The Algorithm to use for %
stokes
                               parameter calculation ({\mantt{OLD}}, {\mantt{%
RATIO}})
\manparameterentry{}{{\mantt{OUTPUT}}}{{\mantt{TSP}}, {\mantt{1D}}}  The %
Output dataset.

\end{manparametertable}
\manroutineitem{Support}{Jeremy Bailey, {\mantt{AAO}}}
\manroutineitem{Version date}{3/5/1993}
\end{manroutinedescription}
\manroutine{{\mantt{LHATPOL}}}{List Hatfield Polarimeter Infrared Data}{LHATPOL}
\begin{manroutinedescription}
\manroutineitem{Function}{}
        List Hatfield Polarimeter Infrared Data

\manroutineitem{Description}{}
        {\mantt{LHATPOL}} lists the {\mantt{IR}} data files in Figaro format %
as produced
        by the Hatfield Polarimeter systems on {\mantt{UKIRT}}. Its principal
        use is to detect spikes for subsequent removal using {\mantt{TSETBAD}}.

        {\mantt{LHATPOL}} works on the original Figaro format file produced by %
the
        data acquisition system. However, the spikes must be removed from
        the data by using {\mantt{TSETBAD}} on the {\mantt{TSP}} file obtained %
by inporting the
        data using {\mantt{RHATPOL}}.

\manroutineitem{Parameters}{}
\begin{manparametertable}
\manparameterentry{1}{{\mantt{FIGARO}}}{Char}     The {\mantt{IRPS}} Figaro %
file to read.
\manparameterentry{2}{{\mantt{FILE}}}{File}      Name of listing file.

\end{manparametertable}
\manroutineitem{Support}{}
         Jeremy Bailey, {\mantt{AAO}}

\manroutineitem{Version date}{}
         1/4/1990

\end{manroutinedescription}
\manroutine{{\mantt{LMERGE}}}{Merge two polarization spectra.}{LMERGE}
\begin{manroutinedescription}
\manroutineitem{Function}{}
        Merge two polarization spectra.

\manroutineitem{Description}{}
        {\mantt{LMERGE}} merges two polarization spectra covering different %
wavelength
        ranges, to form a single dataset.

        {\mantt{LMERGE}} simply appends the data from the second dataset to %
the first.
        There is no guarantee that the output data will thus be in order
        of increasing wavelength and this may cause problems for some other
        programs. To ensure this does not occur the two datasets can be
        SUBSETed or SCRUNCHed before merging so that they do not overlap,
        and should be merged with the higher wavelength dataset as the
        second input file.

\manroutineitem{Parameters}{}
\begin{manparametertable}
\manparameterentry{1}{{\mantt{INPUT1}}}{{\mantt{TSP}}, {\mantt{1D}}}  The %
first input dataset.
\manparameterentry{2}{{\mantt{INPUT2}}}{{\mantt{TSP}}, {\mantt{1D}}}  The %
second input dataset.
\manparameterentry{3}{{\mantt{OUTPUT}}}{{\mantt{TSP}}, {\mantt{1D}}}  The %
output merged dataset.

\end{manparametertable}
\manroutineitem{Support}{Jeremy Bailey, {\mantt{JAC}}}
\manroutineitem{Version date}{15/8/1990}
\end{manroutinedescription}
\manroutine{{\mantt{LTCORR}}}{Apply Light Time corrections to the time axis of %
a data set.}{LTCORR}
\begin{manroutinedescription}
\manroutineitem{Function}{}
        Apply Light Time corrections to the time axis of a data set.

\manroutineitem{Description}{}
        {\mantt{LTCORR}} applies light time corrections to the time axis of a %
data
        set, converting observed times to heliocentric or barycentric
        times. If the parameter {\mantt{SINGLE}} is true a single correction is
        calculated for the mid point time of the dataset, and applied to
        all points in the dataset. If {\mantt{SINGLE}} is false the correction %
is
        recalculated for each data point.

\manroutineitem{Parameters}{}
\begin{manparametertable}
\manparameterentry{1}{{\mantt{INPUT}}}{{\mantt{TSP}}, {\mantt{2D}}}  The input %
time series dataset with
                               observed times.
\manparameterentry{2}{{\mantt{OUTPUT}}}{{\mantt{TSP}}, {\mantt{2D}}}  The %
output corrected dataset with heliocentric
                               or barycentric times.
\manparameterentry{3}{{\mantt{RA}}}{Char}     The {\mantt{B1950}} mean Right %
Ascension of the
                               observed source.
\manparameterentry{4}{{\mantt{DEC}}}{Char}     The {\mantt{B1950}} mean %
declination of the observed
                               source.
\manparameterentry{}{{\mantt{BARY}}}{Logical}  If True, correction is to the %
solar system
                               Barycentre. If False, to the heliocentre.
\manparameterentry{}{{\mantt{SINGLE}}}{Logical}  If True, a single correction %
is calculated
                               for the mid point time of the dataset. If
                               False, the correction is recalculated for
                               each point.
\manparameterentry{}{{\mantt{REVERSE}}}{Logical}  If True, a reverse %
correction is performed.
                               e.g. heliocentric times are converted to
                               observed times.

\end{manparametertable}
\manroutineitem{Support}{Jeremy Bailey, {\mantt{AAO}}}
\manroutineitem{Version date}{27/2/1988}
\end{manroutinedescription}
\manroutine{{\mantt{PHASEPLOT}}}{Plot time series data against phase.}{%
PHASEPLOT}
\begin{manroutinedescription}
\manroutineitem{Function}{}
        Plot time series data against phase.

\manroutineitem{Description}{}
        {\mantt{PHASEPLOT}} plots time series data against the phase of a %
periodic
        variation. Up to six items may be plotted against the same phase
        axis. Each item may be a different channel or Stokes parameter etc.
        The data may be binned (all points in a given phase bin averaged)
        or simply folded (each indivdual time point plotted). The
        plotted phase may range from -1.0 to +2.0 allowing more than
        one cycle. Plotting is done with the {\mantt{NCAR}}/{\mantt{SGS}}/{%
\mantt{GKS}} graphics system.

        Specifying the {\mantt{FILE}} parameter as {\mantt{TRUE}} causes the %
data points
        to be output as a text file, which can then be used in other
        plotting packages such as {\mantt{MONGO}} to provide greater control %
over
        the final plot.

\manroutineitem{Parameters}{}
\begin{manparametertable}
\manparameterentry{1}{{\mantt{INPUT}}}{{\mantt{TSP}}, {\mantt{2D}}}  The input %
time series dataset.
\manparameterentry{2}{{\mantt{NPLOTS}}}{Integer}  The number of items to plot (%
max 6).
\manparameterentry{3}{{\mantt{DEVICE}}}{Device}   The Graphics device (any %
valid {\mantt{GKS}} device).
\manparameterentry{}{{\mantt{WHOLE}}}{Logical}  If {\mantt{TRUE}}, All time %
points are used.
\manparameterentry{C}{{\mantt{XSTART}}}{Double}   First time value ({\mantt{%
MJD}}) to use.
\manparameterentry{C}{{\mantt{XEND}}}{Double}   Last time value ({\mantt{MJD}}) %
to use.
\manparameterentry{}{{\mantt{EPOCH}}}{Double}   The Epoch of phase zero ({%
\mantt{MJD}}).
\manparameterentry{}{{\mantt{PERIOD}}}{Double}   The Period (days).
\manparameterentry{}{{\mantt{PHSTART}}}{Double}   Starting Phase to plot.
\manparameterentry{}{{\mantt{PHEND}}}{Double}   End Phase to plot.
\manparameterentry{}{CHANn}{Integer}  Channel for nth plot. This and the
                               following parameters repeat for
                               n {\mantt{=}} 1 to {\mantt{NPLOTS}}.
\manparameterentry{}{ITEMn}{Char}     Item for nth plot (I,{\mantt{FLUX}},{%
\mantt{MAG}},Q,U,V,P,{\mantt{THETA}})
\manparameterentry{}{AUTOn}{Logical}  If True nth plot is autoscaled.
\manparameterentry{}{BINn}{Double}   Bin size (negative for no binning).
\manparameterentry{}{PLABELn}{Char}     Label for plot n.
\manparameterentry{C}{MINn}{Real}     Minimum scaling level for plot n.
\manparameterentry{C}{MAXn}{Real}     Maximum scaling level for plot n.
\manparameterentry{}{{\mantt{LABEL}}}{Char}     Label for Diagram.
\manparameterentry{H}{{\mantt{ERRORS}}}{Logical}  If True (default), Error %
bars are plotted.
\manparameterentry{H}{{\mantt{LINE}}}{Logical}  If True, the points are joined %
by a
                               continuous line. (Default False).
\manparameterentry{H}{{\mantt{PEN}}}{Integer}  {\mantt{SGS}} Pen number to %
plot in. (Default 1).
\manparameterentry{H}{{\mantt{SIZE}}}{Real}     Scaling Factor for character %
sizes (Default 1).
\manparameterentry{H}{{\mantt{PTOP}}}{Real}     Position of top of diagram. (%
Default 0.9).
\manparameterentry{H}{{\mantt{PBOTTOM}}}{Real}     Position of bottom of %
diagram. (Default 0.1).
\manparameterentry{H}{{\mantt{FILE}}}{Logical}  If true generate a text file %
of data.

\end{manparametertable}
\manroutineitem{Support}{Jeremy Bailey, {\mantt{AAO}}}
\manroutineitem{Version date}{1/3/1990}
\end{manroutinedescription}
\manroutine{{\mantt{PPLOT}}}{Plot a polarization spectrum as P, Theta}{PPLOT}
\begin{manroutinedescription}
\manroutineitem{Function}{}
        Plot a polarization spectrum as P, Theta

\manroutineitem{Description}{}
        {\mantt{PPLOT}} produces a plot of a polarization spectrum. The plot is
        divided into three panels. The lower panel is the total intensity,
        the center panel is the percentage polarization, the top panel
        is the position angle in degrees. The polarization data is binned
        into variable size bins to give a constant polarization error per
        bin. Plotting is done with the {\mantt{NCAR}}/{\mantt{SGS}}/{\mantt{%
GKS}} graphics system.

\manroutineitem{Parameters}{}
\begin{manparametertable}
\manparameterentry{1}{{\mantt{INPUT}}}{{\mantt{TSP}}, {\mantt{1D}}}  The input %
dataset, a spectrum which must
                               have Q and U Stokes parameters present.
\manparameterentry{2}{{\mantt{BINERR}}}{Real}     The percentage error for %
each polarization
                               bin.
\manparameterentry{3}{{\mantt{DEVICE}}}{Device}   The Graphics device (any %
valid {\mantt{GKS}} device).
\manparameterentry{4}{{\mantt{LABEL}}}{Char}     A label for the plot.
\manparameterentry{}{{\mantt{AUTO}}}{Logical}  True if plot is to be autoscaled.
\manparameterentry{C}{{\mantt{IMIN}}}{Real}     Minimum Intensity level to plot.
\manparameterentry{C}{{\mantt{IMAX}}}{Real}     Maximum Intensity level to plot.
\manparameterentry{C}{{\mantt{PMIN}}}{Real}     Minimum Polarization level to %
plot.
\manparameterentry{C}{{\mantt{PMAX}}}{Real}     Maximum Polarization level to %
plot.
\manparameterentry{H}{{\mantt{THETA}}}{Real}     Shift in angle to apply to %
theta plot.
                               Plot range is {\mantt{THETA}} to 180+{\mantt{%
THETA}}.
\manparameterentry{H}{{\mantt{TMIN}}}{Real}     Minimum position angle to plot
\manparameterentry{H}{{\mantt{TMAX}}}{Real}     Maximum position angle to plot

\end{manparametertable}
\manroutineitem{Support}{}
         Jeremy Bailey, {\mantt{AAO}}

\manroutineitem{Version date}{}
         15/8/1990

\end{manroutinedescription}
\manroutine{{\mantt{PTHETA}}}{Output the P and Theta values for a polarization %
spectrum}{PTHETA}
\begin{manroutinedescription}
\manroutineitem{Function}{}
        Output the P and Theta values for a polarization spectrum

\manroutineitem{Description}{}
        The polarization and position angle corresponding to a
        specified wavelength range of a polarization spectrum are
        calculated and output.

\manroutineitem{Parameters}{}
\begin{manparametertable}
\manparameterentry{1}{{\mantt{INPUT}}}{{\mantt{TSP}}, {\mantt{1D}}}\begin{%
manitemize}
\manitemizeitem The input dataset, a spectrum which must
                               have Q and U Stokes parameters present.
\end{manitemize}
\manparameterentry{2}{{\mantt{LSTART}}}{Real}\begin{manitemize}
\manitemizeitem The starting wavelength for the section
                               to be used.
\end{manitemize}
\manparameterentry{3}{{\mantt{LEND}}}{Real}\begin{manitemize}
\manitemizeitem The end wavelength for the section.
\end{manitemize}

\end{manparametertable}
\manroutineitem{Support}{Jeremy Bailey, {\mantt{AAO}}}
\manroutineitem{Version date}{15/6/1988}
\end{manroutinedescription}
\manroutine{{\mantt{QPLOT}}}{Quick plot of time series data.}{QPLOT}
\begin{manroutinedescription}
\manroutineitem{Function}{}
        Quick plot of time series data.

\manroutineitem{Description}{}
        {\mantt{QPLOT}} provides a quick means of plotting one item from a time
        series data set, but without the many options provided by {\mantt{%
TSPLOT}}.
        Plotting is done with the {\mantt{NCAR}}/{\mantt{SGS}}/{\mantt{GKS}} %
graphics system.

\manroutineitem{Parameters}{}
\begin{manparametertable}
\manparameterentry{1}{{\mantt{INPUT}}}{{\mantt{TSP}}, {\mantt{2D}}}  The input %
time series dataset.
\manparameterentry{2}{{\mantt{DEVICE}}}{Device}   The Graphics device (any %
valid {\mantt{GKS}} device).
\manparameterentry{3}{{\mantt{CHAN}}}{Integer}  Channel to plot.
\manparameterentry{4}{{\mantt{ITEM}}}{Char}     Item to plot (I,Q,U,V)
\manparameterentry{}{{\mantt{LABEL}}}{Char}     Label for Diagram.
\manparameterentry{}{AUTOn}{Logical}  If True plot is autoscaled.
\manparameterentry{C}{{\mantt{MIN}}}{Real}     Minimum level for scaling.
\manparameterentry{C}{{\mantt{MAX}}}{Real}     Maximum level for scaling.
\manparameterentry{}{{\mantt{WHOLE}}}{Logical}  If {\mantt{TRUE}}, All time %
points are used.
\manparameterentry{C}{{\mantt{XSTART}}}{Double}   First time value ({\mantt{%
MJD}}) to use.
\manparameterentry{C}{{\mantt{XEND}}}{Double}   Last time value ({\mantt{MJD}}) %
to use.
\manparameterentry{H}{{\mantt{ERRORS}}}{Logical}  If True (default), Error %
bars are plotted.
\manparameterentry{H}{{\mantt{LINE}}}{Logical}  If True, the points are joined %
by a
                               continuous line. (Default False).
\manparameterentry{H}{{\mantt{PEN}}}{Integer}  {\mantt{SGS}} Pen number to %
plot in. (Default 1).

\end{manparametertable}
\manroutineitem{Support}{Jeremy Bailey, {\mantt{AAO}}}
\manroutineitem{Version date}{28/2/1988}
\end{manroutinedescription}
\manroutine{{\mantt{QUMERGE}}}{Merge Q and U spectra into single dataset.}{%
QUMERGE}
\begin{manroutinedescription}
\manroutineitem{Function}{}
        Merge Q and U spectra into single dataset.

\manroutineitem{Description}{}
        {\mantt{QUMERGE}} merges two separate datasets containing Q and U Stokes
        parameters to form a single dataset containing Q and U. It can
        be used to combine data obtained with {\mantt{IPCS2STOKES}} or {\mantt{%
CCD2STOKES}}
        where the two stokes parameters have been obtained independently.

        The program {\mantt{CCD2POL}} is equivalent to using {\mantt{CC2STOKES}%
} to derive
        the two Stokes parameters, and then combining them with {\mantt{%
QUMERGE}}.

\manroutineitem{Parameters}{}
\begin{manparametertable}
\manparameterentry{1}{{\mantt{QQ}}}{{\mantt{TSP}}, {\mantt{1D}}}  The input Q %
dataset.
\manparameterentry{2}{U}{{\mantt{TSP}}, {\mantt{1D}}}  The input U dataset.
\manparameterentry{3}{{\mantt{OUTPUT}}}{{\mantt{TSP}}, {\mantt{1D}}}  The %
output merged dataset.

\end{manparametertable}
\manroutineitem{Support}{}
         Jeremy Bailey, {\mantt{AAO}}

\manroutineitem{Version date}{}
         19/8/1988

\end{manroutinedescription}
\manroutine{{\mantt{QUPLOT}}}{Plot a polarization spectrum in the Q,U plane.}{%
QUPLOT}
\begin{manroutinedescription}
\manroutineitem{Function}{}
        Plot a polarization spectrum in the Q,U plane.

\manroutineitem{Description}{}
        {\mantt{QUPLOT}} produces a plot of the polarization spectrum in the
        Q,U plane. The polarization data is first binned to a constant
        percentage polarization error per bin, and the resulting points
        are plotted. Plotting is done with the {\mantt{NCAR}}/{\mantt{SGS}}/{%
\mantt{GKS}} graphics system.

\manroutineitem{Parameters}{}
\begin{manparametertable}
\manparameterentry{1}{{\mantt{INPUT}}}{{\mantt{TSP}}, {\mantt{1D}}}  The input %
dataset, a spectrum which must
                               have Q and U Stokes parameters present.
\manparameterentry{2}{{\mantt{BINERR}}}{Real}     The percentage error for %
each polarization
                               bin.
\manparameterentry{3}{{\mantt{DEVICE}}}{Device}   The Graphics device (any %
valid {\mantt{GKS}} device).
\manparameterentry{4}{{\mantt{LABEL}}}{Char}     A label for the plot.
\manparameterentry{}{{\mantt{AUTO}}}{Logical}  True if plot is to be autoscaled.
\manparameterentry{C}{{\mantt{QMIN}}}{Real}     Minimum Q level to plot.
\manparameterentry{C}{{\mantt{QMAX}}}{Real}     Maximum Q level to plot.
\manparameterentry{C}{{\mantt{UMIN}}}{Real}     Minimum U level to plot.
\manparameterentry{C}{{\mantt{UMAX}}}{Real}     Maximum U level to plot.

\end{manparametertable}
\manroutineitem{Support}{}
         Jeremy Bailey, {\mantt{AAO}}

\manroutineitem{Version date}{}
         26/2/1988

\end{manroutinedescription}
\manroutine{{\mantt{QUSUB}}}{Subtract a Q,U vector from a polarization %
spectrum.}{QUSUB}
\begin{manroutinedescription}
\manroutineitem{Function}{}
        Subtract a Q,U vector from a polarization spectrum.

\manroutineitem{Description}{}
        {\mantt{QUSUB}} subtracts a percentage polarization expressed as a Q,U
        vector from a polarization spectrum. This can be used as a
        crude correction for interstellar polarization.

\manroutineitem{Parameters}{}
\begin{manparametertable}
\manparameterentry{}{{\mantt{INPUT}}}{{\mantt{TSP}}, {\mantt{1D}}}  The input %
dataset, a spectrum which must
                               have Q and U Stokes parameters present.
\manparameterentry{}{{\mantt{QVAL}}}{Real}     The Q value (per cent) to %
subtract.
\manparameterentry{}{{\mantt{UVAL}}}{Real}     The U value (per cent) to %
subtract.
\manparameterentry{}{{\mantt{OUTPUT}}}{{\mantt{TSP}}, {\mantt{1D}}}  The %
corrected dataset.

\end{manparametertable}
\manroutineitem{Support}{Jeremy Bailey, {\mantt{AAO}}}
\manroutineitem{Version date}{28/2/1988}
\end{manroutinedescription}
\manroutine{{\mantt{RCCDTS}}}{Read {\mantt{AAO}} {\mantt{CCD}} Time Series %
data}{RCCDTS}
\begin{manroutinedescription}
\manroutineitem{Function}{}
        Read {\mantt{AAO}} {\mantt{CCD}} Time Series data

\manroutineitem{Description}{}
        Read an {\mantt{AAO}} {\mantt{CCD}} time series data set from the raw %
figaro file
        and build a {\mantt{3D}} {\mantt{TSP}} dataset.

        The {\mantt{AAO}} time series mode takes a time series of data by %
shifting
        data out of the {\mantt{CCD}} on some regular period. The whole time %
series
        is treated as a single readout and therefore appears as a two
        dimensional array in which slices of the array are each indivdual
        frames of the time series. {\mantt{RCCDTS}} takes these frames out of %
the {\mantt{2D}}
        array and builds a {\mantt{3D}} {\mantt{TSP}} dataset to represent the %
resulting time
        series image. It also creates a time axis from the timing information
        contained in the {\mantt{FITS}} header.

\manroutineitem{Parameters}{}
\begin{manparametertable}
\manparameterentry{1}{{\mantt{FIGARO}}}{Char}     The Figaro file containing %
the time series data
\manparameterentry{2}{{\mantt{OUTPUT}}}{{\mantt{TSP}}, {\mantt{3D}}}  The %
output time series dataset.

\end{manparametertable}
\manroutineitem{Support}{Jeremy Bailey, {\mantt{JAC}}}
\manroutineitem{Version date}{26/10/1989}
\end{manroutinedescription}
\manroutine{{\mantt{RCGS2}}}{Read {\mantt{CGS2}} Polarimetry Data}{RCGS2}
\begin{manroutinedescription}
\manroutineitem{Function}{}
        Read {\mantt{CGS2}} Polarimetry Data

\manroutineitem{Description}{}
        {\mantt{RCGS2}} reads a polarimetry data file in Figaro format as %
produced
        by the {\mantt{CGS2}} Polarimetry system at {\mantt{UKIRT}} and %
reduces it to
        a {\mantt{TSP}} polarization spectrum.

        The {\mantt{CGS2}} Figaro files are 4 dimensional arrays produced by the
        {\mantt{DRT}}, in which the dimensions are: {\mantt{WAVEPLATE}} {%
\mantt{POSITIONS}} by {\mantt{SPECTRAL}}
        {\mantt{CHANNELS}} by {\mantt{BEAMS}} (i.e. {\mantt{OFFSET}} or {%
\mantt{MAIN}}) by {\mantt{CYCLES}}.

\manroutineitem{Parameters}{}
\begin{manparametertable}
\manparameterentry{1}{{\mantt{FIGARO}}}{Char}     The {\mantt{IRPS}} Figaro %
file to read.
\manparameterentry{2}{{\mantt{OUTPUT}}}{{\mantt{TSP}}, {\mantt{1D}}}  The %
output time series dataset.
\manparameterentry{3}{{\mantt{NSIGMA}}}{Real}     Sigma level for despiking.

\end{manparametertable}
\manroutineitem{Support}{}
         Jeremy Bailey, {\mantt{AAO}}

\manroutineitem{Version date}{}
         28/8/1990

\end{manroutinedescription}
\manroutine{{\mantt{REVERSE}}}{Reverse a spectrum in the wavelength axis.}{%
REVERSE}
\begin{manroutinedescription}
\manroutineitem{Function}{}
        Reverse a spectrum in the wavelength axis.

\manroutineitem{Description}{}
        All the data arrays contained in a polarization spectrum are
        reversed in order along the wavelength axis.

\manroutineitem{Parameters}{}
\begin{manparametertable}
\manparameterentry{1}{{\mantt{INPUT}}}{{\mantt{TSP}}, {\mantt{1D}}}  The input %
dataset.
\manparameterentry{2}{{\mantt{OUTPUT}}}{{\mantt{TSP}}, {\mantt{1D}}}  The %
output dataset.

\end{manparametertable}
\manroutineitem{Support}{}
         Jeremy Bailey, {\mantt{AAO}}

\manroutineitem{Version date}{}
         27/4/1988

\end{manroutinedescription}
\manroutine{{\mantt{RFIGARO}}}{Read a Stokes Parameter Spectrum from a Figaro %
image}{RFIGARO}
\begin{manroutinedescription}
\manroutineitem{Function}{}
        Read a Stokes Parameter Spectrum from a Figaro image

\manroutineitem{Description}{}
        {\mantt{RFIGARO}} reads an n by 3 Figaro image and creates a {\mantt{%
TSP}} Stokes
        parameter spectrum. The First row of the image becomes the
        intensity spectrum. The second row becomes the Stokes spectrum
        and the third row becomes the Stokes variance. It allows data
        files created by the old {\mantt{VISTA}} {\mantt{BASIC}} %
spectropolarimetry package
        to be read into {\mantt{TSP}}.

\manroutineitem{Parameters}{}
\begin{manparametertable}
\manparameterentry{1}{{\mantt{FIGARO}}}{Char}     The input Figaro data file.
\manparameterentry{2}{{\mantt{STOKESPAR}}}{Char}     The Stokes parameter (Q,U,%
V) for the
                               output data.
\manparameterentry{3}{{\mantt{OUTPUT}}}{{\mantt{TSP}}, {\mantt{1D}}}  The %
Output dataset.

\end{manparametertable}
\manroutineitem{Support}{}
          Jeremy Bailey, {\mantt{AAO}}

\manroutineitem{Version date}{}
          26/2/1988

\end{manroutinedescription}
\manroutine{{\mantt{RHATHSP}}}{Read Hatfield Polarimeter High Speed Photometry %
Data}{RHATHSP}
\begin{manroutinedescription}
\manroutineitem{Function}{}
        Read Hatfield Polarimeter High Speed Photometry Data

\manroutineitem{Description}{}
        {\mantt{RHATHSP}} reads data files in Figaro format as produced
        by the Hatfield Polarimeter at the {\mantt{AAT}} running in its
        5 channel high speed photometry mode.

        It outputs a 5 channel {\mantt{TSP}} time series dataset containing
        the light curves in each of the five channels. An accurate
        time axis array is created using the approximate start time
        and the additional timing information written to the sixth
        channel of the data array.

\manroutineitem{Parameters}{}
\begin{manparametertable}
\manparameterentry{1}{{\mantt{FIGARO}}}{Char}     The Hatfield Figaro file to %
read.
\manparameterentry{2}{{\mantt{OUTPUT}}}{{\mantt{TSP}}, {\mantt{2D}}}  The %
output time series dataset.

\end{manparametertable}
\manroutineitem{Support}{}
         Jeremy Bailey, {\mantt{AAO}}

\manroutineitem{Version date}{}
         2/12/1988

\end{manroutinedescription}
\manroutine{{\mantt{RHATPOL}}}{Read Hatfield Polarimeter Data}{RHATPOL}
\begin{manroutinedescription}
\manroutineitem{Function}{}
        Read Hatfield Polarimeter Data

\manroutineitem{Description}{}
        {\mantt{RHATPOL}} reads polarimetry data files in Figaro format as %
produced
        by the Hatfield Polarimeter systems on {\mantt{UKIRT}} or the {\mantt{%
AAT}}.
        A time series dataset is created containing the reduced linear
        or linear+circular polarimetry data.

        A calibration file is used to specify the values of calibration
        parameters (efficiency, position angle zero point, photometric
        zero point, anc circular calibration).

\manroutineitem{Parameters}{}
\begin{manparametertable}
\manparameterentry{1}{{\mantt{FIGARO}}}{Char}     The {\mantt{IRPS}} Figaro %
file to read.
\manparameterentry{2}{{\mantt{NPTS}}}{Integer}  Number of points per cycle (1 %
or 2).
\manparameterentry{3}{{\mantt{LINEAR}}}{Logical}  True for linear data, false %
for circular.
\manparameterentry{4}{{\mantt{OUTPUT}}}{{\mantt{TSP}}, {\mantt{2D}}}  The %
output time series dataset.
\manparameterentry{5}{{\mantt{CFILE}}}{File}     Name of calibration file.

\end{manparametertable}
\manroutineitem{Support}{}
         Jeremy Bailey, {\mantt{AAO}}

\manroutineitem{Version date}{}
         3/3/1990

\end{manroutinedescription}
\manroutine{{\mantt{RHDSPLOT}}}{Read {\mantt{ASCII}} files of Hatfield %
Polarimeter Data.}{RHDSPLOT}
\begin{manroutinedescription}
\manroutineitem{Function}{}
        Read {\mantt{ASCII}} files of Hatfield Polarimeter Data.

\manroutineitem{Description}{}
        {\mantt{RHDSPLOT}} reads {\mantt{ASCII}} files created by Tim Peacock'%
s {\mantt{HDSPLOT}} program
        from raw Hatfield Polarimeter Data. It outputs time series datasets
        with either 3 or 6 wavelengths channels, depending on which version
        of the polarimeter the data came from.

        This command is superseded by {\mantt{RHATPOL}} which can reduce %
Hatfield
        polarimetry data directly from the raw data files.

\manroutineitem{Parameters}{}
\begin{manparametertable}
\manparameterentry{}{{\mantt{FILENAME}}}{Char}\begin{manitemize}
\manitemizeitem The name of the {\mantt{HDSPLOT}} file to be read.
\end{manitemize}
\manparameterentry{}{{\mantt{NCHANS}}}{Integer}\begin{manitemize}
\manitemizeitem The number of wavelength channels.
\end{manitemize}
\manparameterentry{}{{\mantt{OUTPUT}}}{{\mantt{TSP}}, {\mantt{2D}}}\begin{%
manitemize}
\manitemizeitem The output dataset to be created.
\end{manitemize}
\manparameterentry{}{{\mantt{CHANNEL}}}{Char}\begin{manitemize}
\manitemizeitem Name of channel.
\end{manitemize}
\manparameterentry{}{{\mantt{WAVELENGTH}}}{Real}\begin{manitemize}
\manitemizeitem Wavelength of channel.
\end{manitemize}

\end{manparametertable}
\manroutineitem{Support}{}
         Jeremy Bailey, {\mantt{AAO}}

\manroutineitem{Version date}{}
         27/2/1988

\end{manroutinedescription}
\manroutine{{\mantt{RHSP3}}}{Read an {\mantt{HSP3}} tape}{RHSP3}
\begin{manroutinedescription}
\manroutineitem{Function}{}
        Read an {\mantt{HSP3}} tape

\manroutineitem{Description}{}
        {\mantt{RHSP3}} reads tapes produced by the {\mantt{HSP3}} high speed %
photometry
        software at the {\mantt{AAT}}. The current version is limited to 16 bit
        single channel data, and handles a maximum of 200000 time bins.

\manroutineitem{Parameters}{}
\begin{manparametertable}
\manparameterentry{}{{\mantt{DRIVE}}}{Device}   The tape drive to read from.
\manparameterentry{}{{\mantt{MJDZERO}}}{Double}   The {\mantt{MJD}} at 0h U.T. %
on the night of observation.
\manparameterentry{}{{\mantt{OUTPUT}}}{{\mantt{TSP}}, {\mantt{2D}}}  The %
output time series dataset.

\end{manparametertable}
\manroutineitem{Support}{}
        Jeremy Bailey, {\mantt{AAO}}

\manroutineitem{Version date}{}
        27/2/1988

\end{manroutinedescription}
\manroutine{{\mantt{RIRPS}}}{Read {\mantt{IRPS}} Photometry Data}{RIRPS}
\begin{manroutinedescription}
\manroutineitem{Function}{}
        Read {\mantt{IRPS}} Photometry Data

\manroutineitem{Description}{}
        {\mantt{RIRPS}} reads photometry data files in Figaro format as produced
        by the {\mantt{IRPS}} ({\mantt{AAO}} Infrared Photometer Spectrometer) %
{\mantt{ADAM}} system.
        A time series dataset is created. Either {\mantt{P1}} or {\mantt{P4}} %
data may be
        read.

\manroutineitem{Parameters}{}
\begin{manparametertable}
\manparameterentry{1}{{\mantt{FIGARO}}}{Char}     The {\mantt{IRPS}} Figaro %
file to read.
\manparameterentry{2}{{\mantt{NDWELLS}}}{Integer}  The Number of {\mantt{IRPS}} %
dwells to use for each
                               data point (1,2 or 4).
\manparameterentry{3}{{\mantt{OUTPUT}}}{{\mantt{TSP}}, {\mantt{2D}}}  The %
output time series dataset.
\manparameterentry{}{{\mantt{ZEROPT}}}{Real}     Magnitude zero point.

\end{manparametertable}
\manroutineitem{Support}{}
          Jeremy Bailey, {\mantt{AAO}}

\manroutineitem{Version date}{}
          27/2/1988

\end{manroutinedescription}
\manroutine{{\mantt{ROTPA}}}{Rotate the Position Angle of a Polarization %
Dataset}{ROTPA}
\begin{manroutinedescription}
\manroutineitem{Function}{}
        Rotate the Position Angle of a Polarization Dataset

\manroutineitem{Description}{}
        Rotate the position angle of a polarization Dataset
        through a specified amount

        This propgram can be used to correct the position angle
        for a constant (wavelength independent) calibration errror

\manroutineitem{Parameters}{}
\begin{manparametertable}
\manparameterentry{1}{{\mantt{INPUT}}}{{\mantt{TSP}}, nD}  The Polarization %
dataset to be corrected.
\manparameterentry{2}{{\mantt{THETA}}}{Real}     Angle to rotate through (%
degrees).
\manparameterentry{3}{{\mantt{OUTPUT}}}{{\mantt{TSP}}, nD}  The output %
corrected dataset.

\end{manparametertable}
\manroutineitem{Support}{}
         Jeremy Bailey, {\mantt{AAO}}

\manroutineitem{Version date}{}
         8/5/1993

\end{manroutinedescription}
\manroutine{{\mantt{RTURKU}}}{Read {\mantt{ASCII}} files of Data from the %
Turku {\mantt{UBVRI}} Polarimeter.}{RTURKU}
\begin{manroutinedescription}
\manroutineitem{Function}{}
        Read {\mantt{ASCII}} files of Data from the Turku {\mantt{UBVRI}} %
Polarimeter.

\manroutineitem{Description}{}
        {\mantt{RTURKU}} reads {\mantt{ASCII}} files of data from the Turku %
University {\mantt{UBVRI}}
        polarimeter. The raw data must first be reduced using the {\mantt{%
POLRED}}
        (for linear polarimetery) or {\mantt{CIRLIN}} (for simultaneous %
circular/linear
        polarimetry) programs. The resulting files are read by {\mantt{RTURKU}}
        (one file for the linear case, two for the circular/linear case).
        The data on a given star is selected from the file by specifying
        its number, and output as a {\mantt{TSP}} time series dataset.

\manroutineitem{Parameters}{}
\begin{manparametertable}
\manparameterentry{}{{\mantt{CIRLIN}}}{Logical}  {\mantt{TRUE}} for circular+%
linear data
                              {\mantt{FALSE}} for linear only data.
\manparameterentry{}{{\mantt{LINFILE}}}{Char}     The name of the linear data %
input file.
\manparameterentry{}{{\mantt{CIRFILE}}}{Char}     The name of the circular %
data input file.
\manparameterentry{}{{\mantt{STAR}}}{Integer}  The Star Number
\manparameterentry{}{{\mantt{OUTPUT}}}{{\mantt{TSP}}, {\mantt{2D}}}  The %
output dataset to be created.


\end{manparametertable}
\manroutineitem{Support}{}
        Jeremy Bailey, {\mantt{AAO}}

\manroutineitem{Version date}{}
        4/11/1988

\end{manroutinedescription}
\manroutine{{\mantt{SCRUNCH}}}{Rebin a Polarization Spectrum.}{SCRUNCH}
\begin{manroutinedescription}
\manroutineitem{Function}{}
        Rebin a Polarization Spectrum.

\manroutineitem{Description}{}
        {\mantt{SCRUNCH}} rebins a polarization spectrum onto a linear or
        logarithmic wavelength scale. {\mantt{SCRUNCH}} is closely based on
        the {\mantt{FIGARO}} program of the same name, and uses the same
        subroutine to perform the rebinning.

\manroutineitem{Parameters}{}
\begin{manparametertable}
\manparameterentry{1}{{\mantt{INPUT}}}{{\mantt{TSP}}, nD}  The input spectrum %
to be scrunched.
\manparameterentry{2}{{\mantt{WSTART}}}{Real}     The wavelength of the center %
of the first
                               bin of the resulting scrunched spectrum.
\manparameterentry{3}{{\mantt{WEND}}}{Real}     The wavelength of the center %
of the final
                               bin of the resulting scrunched spectrum.
                               If {\mantt{WEND}} is less than {\mantt{WSTART}}, %
then {\mantt{SCRUNCH}}
                               assumes that it is the increment that is
                               being specified rather than the final value.
                               If the scrunch is logarithmic and {\mantt{%
WSTART}}
                               is greater than {\mantt{WEND}}, {\mantt{SCRUNCH}%
} assumes that
                               the {\mantt{WEND}} value represents a velocity in
                               km/s.
\manparameterentry{4}{{\mantt{BINS}}}{Integer}  The number of bins for the %
resulting spectrum.
\manparameterentry{5}{{\mantt{OUTPUT}}}{{\mantt{TSP}}, nD}  The Output %
rebinned spectrum.
\manparameterentry{}{{\mantt{LOG}}}{Logical}  Bin into logarithmic wavelength %
bins.
\manparameterentry{}{{\mantt{MEAN}}}{Logical}  Conserve mean data level rather %
than flux.
\manparameterentry{}{{\mantt{QUAD}}}{Logical}  Use quadratic (rather than %
linear)
                               interpolation.


\end{manparametertable}
\manroutineitem{Support}{Jeremy Bailey, {\mantt{AAO}}}
\manroutineitem{Version date}{10/8/1988}
\end{manroutinedescription}
\manroutine{{\mantt{SKYSUB}}}{Subtract Sky from a time series image dataset}{%
SKYSUB}
\begin{manroutinedescription}
\manroutineitem{Function}{}
        Subtract Sky from a time series image dataset

\manroutineitem{Description}{}
        Sky subtraction from each frame of a time series image is
        performed by selecting two areas on each side of a star to
        be observed and linearly interpolating between the mean or
        median of the values in these.

\manroutineitem{Parameters}{}
\begin{manparametertable}
\manparameterentry{1}{{\mantt{INPUT}}}{{\mantt{TSP}}, {\mantt{3D}}}   The time %
series dataset to be sky subtracted.
\manparameterentry{2}{{\mantt{OUTPUT}}}{{\mantt{TSP}}, {\mantt{3D}}}   The %
dataset after sky subtraction.
\manparameterentry{3}{{\mantt{Y1}}}{Integer}   Lowest Y value to use
\manparameterentry{4}{{\mantt{Y2}}}{Integer}   Highest Y value to use
\manparameterentry{5}{{\mantt{XL1}}}{Integer}   Lowest X value for left sky %
region
\manparameterentry{6}{{\mantt{XL2}}}{Integer}   Highest X value for left sky %
region
\manparameterentry{7}{{\mantt{XR1}}}{Integer}   Lowest X value for right sky %
region
\manparameterentry{8}{{\mantt{XR2}}}{Integer}   Highest X value for right sky %
region
\manparameterentry{9}{{\mantt{MEDIAN}}}{Logical}   Use median rather than mean

\end{manparametertable}
\manroutineitem{Support}{}
         Jeremy Bailey, {\mantt{AAO}}

\manroutineitem{Version date}{}
         26/10/1989

\end{manroutinedescription}
\manroutine{{\mantt{SLIST}}}{Output a polarization spectrum in the form of an {%
\mantt{ASCII}} file}{SLIST}
\begin{manroutinedescription}
\manroutineitem{Function}{}
        Output a polarization spectrum in the form of an {\mantt{ASCII}} file

\manroutineitem{Description}{}
        A polarization spectrum is output in the form of an {\mantt{ASCII}} %
file.
        The file has 6 columns containing the wavelength, I,
        Q, error on Q, U error on U.


\manroutineitem{Parameters}{}
\begin{manparametertable}
\manparameterentry{1}{{\mantt{INPUT}}}{{\mantt{TSP}}, {\mantt{1D}}}  The input %
dataset, a spectrum which must
                               have Q and U Stokes parameters present.
\manparameterentry{2}{{\mantt{FILE}}}{File}     Output Ascii File.

\end{manparametertable}
\manroutineitem{Support}{Jeremy Bailey, {\mantt{AAO}}}
\manroutineitem{Version date}{17/2/1993}
\end{manroutinedescription}
\manroutine{{\mantt{SPLOT}}}{Plot a polarization spectrum with a single Stokes %
parameter}{SPLOT}
\begin{manroutinedescription}
\manroutineitem{Function}{}
        Plot a polarization spectrum with a single Stokes parameter

\manroutineitem{Description}{}
        {\mantt{SPLOT}} produces a plot of a polarization spectrum. The plot is
        divided into two panels. The lower panel is the total intensity,
        the top panel is the percentage polarization for a single Stokes
        parameter. If the dataset contains only one Stokes parameter that
        Stokes parameter is plotted. If the spectrum contains more than
        one Stokes parameter any one of them may be chosen for plotting.
        The polarization data is binned into variable size bins to give
        a constant polarization error per bin. Plotting is done with the
        {\mantt{NCAR}}/{\mantt{SGS}}/{\mantt{GKS}} graphics system.

\manroutineitem{Parameters}{}
\begin{manparametertable}
\manparameterentry{1}{{\mantt{INPUT}}}{{\mantt{TSP}}, {\mantt{1D}}}  The input %
dataset, a spectrum which must
                               have at least one Stokes parameter.
\manparameterentry{2}{{\mantt{BINERR}}}{Real}     The percentage error for %
each polarization
                               bin.
\manparameterentry{3}{{\mantt{DEVICE}}}{Device}   The Graphics device (any %
valid {\mantt{GKS}} device).
\manparameterentry{4}{{\mantt{LABEL}}}{Char}     A label for the plot.
\manparameterentry{}{{\mantt{STOKESPAR}}}{Char}     The Stokes parameter to be %
plotted (Q,U,V).
\manparameterentry{}{{\mantt{AUTO}}}{Logical}  True if plot is to be autoscaled.
\manparameterentry{C}{{\mantt{IMIN}}}{Real}     Minimum Intensity level to plot.
\manparameterentry{C}{{\mantt{IMAX}}}{Real}     Maximum Intensity level to plot.
\manparameterentry{C}{{\mantt{PMIN}}}{Real}     Minimum Polarization level to %
plot.
\manparameterentry{C}{{\mantt{PMAX}}}{Real}     Maximum Polarization level to %
plot.

\end{manparametertable}
\manroutineitem{Support}{}
         Jeremy Bailey, {\mantt{AAO}}

\manroutineitem{Version date}{}
         19/8/1988

\end{manroutinedescription}
\manroutine{{\mantt{SUBSET}}}{Take a subset of a dataset in wavelength or time %
axes.}{SUBSET}
\begin{manroutinedescription}
\manroutineitem{Function}{}
        Take a subset of a dataset in wavelength or time axes.

\manroutineitem{Description}{}
        A subset of the input file is taken covering a specified
        range in wavelength and time. The command works on
        either {\mantt{1D}} or {\mantt{2D}} data.

\manroutineitem{Parameters}{}
\begin{manparametertable}
\manparameterentry{1}{{\mantt{INPUT}}}{{\mantt{TSP}}, nD}  The input dataset.
\manparameterentry{2}{{\mantt{LSTART}}}{Real}     Starting wavelength for %
subset.
\manparameterentry{3}{{\mantt{LEND}}}{Real}     End wavelength for subset.
\manparameterentry{C}{{\mantt{TSTART}}}{Double}   Starting Time for subset.
\manparameterentry{C}{{\mantt{TEND}}}{Double}   End Time for subset.
\manparameterentry{4}{{\mantt{OUTPUT}}}{{\mantt{TSP}}, nD}  The output dataset.

\end{manparametertable}
\manroutineitem{Support}{}
         Jeremy Bailey, {\mantt{AAO}}

\manroutineitem{Version date}{}
         30/4/1988

\end{manroutinedescription}
\manroutine{{\mantt{SUBTRACT}}}{Subtract two Polarization spectra.}{SUBTRACT}
\begin{manroutinedescription}
\manroutineitem{Function}{}
        Subtract two Polarization spectra.

\manroutineitem{Description}{}
        Two Polarization spectra covering the same wavelength range
        are subtracted to form a new spectrum giving the difference of
        the intensity and Stokes parameters.

        Any number of Stokes parameters may be present in the
        spectra, but only Stokes parameters present in both spectra
        will appear in the output spectrum.



\manroutineitem{Parameters}{}
\begin{manparametertable}
\manparameterentry{1}{{\mantt{INPUT1}}}{{\mantt{TSP}}, {\mantt{1D}}}  The %
first input Stokes spectrum.
\manparameterentry{2}{{\mantt{INPUT2}}}{{\mantt{TSP}}, {\mantt{1D}}}  The %
second input Stokes spectrum.
\manparameterentry{3}{{\mantt{OUTPUT}}}{{\mantt{TSP}}, {\mantt{1D}}}  The %
output dataset.

\end{manparametertable}
\manroutineitem{Support}{}
         Jeremy Bailey, {\mantt{AAO}}

\manroutineitem{Version date}{}
         4/12/1988

\end{manroutinedescription}
\manroutine{{\mantt{TBIN}}}{Bin a time series}{TBIN}
\begin{manroutinedescription}
\manroutineitem{Function}{}
        Bin a time series

\manroutineitem{Description}{}
        {\mantt{TBIN}} creates a new time series by binning an input time series
        into bins of a specified size. All the points falling within
        one bin are averaged, and their time value is averaged to create
        the new bin time. Note that this means that the output is not
        necessarily equally spaced in time. It depends where the points
        fall within the bin.

\manroutineitem{Parameters}{}
\begin{manparametertable}
\manparameterentry{1}{{\mantt{INPUT}}}{{\mantt{TSP}}, {\mantt{2D}}}  The input %
time series dataset.
\manparameterentry{}{{\mantt{BIN}}}{Double}   The bin size (days).
\manparameterentry{}{{\mantt{WHOLE}}}{Logical}  If {\mantt{TRUE}}, All time %
points are used.
\manparameterentry{C}{{\mantt{XSTART}}}{Double}   First time value ({\mantt{%
MJD}}) to use.
\manparameterentry{C}{{\mantt{XEND}}}{Double}   Last time value ({\mantt{MJD}}) %
to use.
\manparameterentry{}{{\mantt{OUTPUT}}}{{\mantt{TSP}}, {\mantt{2D}}}  The %
output binned dataset.

\end{manparametertable}
\manroutineitem{Support}{}
         Jeremy Bailey, {\mantt{AAO}}

\manroutineitem{Version date}{}
         26/2/1988

\end{manroutinedescription}
\manroutine{{\mantt{TCADD}}}{Add Channels of a time series dataset}{TCADD}
\begin{manroutinedescription}
\manroutineitem{Function}{}
        Add Channels of a time series dataset

\manroutineitem{Description}{}
        {\mantt{TCADD}} adds a range of channels of a time series dataset
        to produce a 1-dimensional output array. The number of channels
        can be one so it can be used to extract a single channel.

\manroutineitem{Parameters}{}
\begin{manparametertable}
\manparameterentry{1}{{\mantt{INPUT}}}{{\mantt{TSP}}, {\mantt{2D}}}  The input %
time series dataset.
\manparameterentry{}{{\mantt{FIRST}}}{Integer}  First channel to extract.
\manparameterentry{}{{\mantt{LAST}}}{Integer}  Last channel to extract.
\manparameterentry{}{{\mantt{OUTPUT}}}{{\mantt{TSP}}, {\mantt{1D}}}  The %
output binned dataset.

\end{manparametertable}
\manroutineitem{Support}{}
         Jeremy Bailey, {\mantt{AAO}}

\manroutineitem{Version date}{}
         26/2/1988

\end{manroutinedescription}
\manroutine{{\mantt{TDERIV}}}{Calculate Time Derivative of a Dataset.}{TDERIV}
\begin{manroutinedescription}
\manroutineitem{Function}{}
        Calculate Time Derivative of a Dataset.

\manroutineitem{Description}{}
        {\mantt{TDERIV}} calculates the time derivative of the intensity
        data in a dataset and outputs it as a new dataset.
        For each point in the time series the slope of a straight
        line fitted through n points is used to obtain the derivative.

\manroutineitem{Parameters}{}
\begin{manparametertable}
\manparameterentry{1}{{\mantt{INPUT}}}{{\mantt{TSP}}, {\mantt{2D}}}  The input %
time series dataset.
\manparameterentry{2}{{\mantt{NPOINTS}}}{Integer}  Number of points for line %
fit.
\manparameterentry{3}{{\mantt{OUTPUT}}}{{\mantt{TSP}}, {\mantt{2D}}}  The %
output corrected dataset.

\end{manparametertable}
\manroutineitem{Support}{}
        Jeremy Bailey, {\mantt{AAO}}

\manroutineitem{Version date}{}
        1/3/1988

\end{manroutinedescription}
\manroutine{{\mantt{TEXTIN}}}{Correct a time series dataset for extinction.}{%
TEXTIN}
\begin{manroutinedescription}
\manroutineitem{Function}{}
        Correct a time series dataset for extinction.

\manroutineitem{Description}{}
        Correct a dataset for extinction by calculating the airmass
        of each point and correcting the intensity to a value for
        airmass 1.

        The extinction coefficients (magnitude per airmass) for each
        channel must be provided.

\manroutineitem{Parameters}{}
\begin{manparametertable}
\manparameterentry{1}{{\mantt{INPUT}}}{{\mantt{TSP}}, {\mantt{2D}}}  The input %
time series dataset.
\manparameterentry{2}{{\mantt{OUTPUT}}}{{\mantt{TSP}}, {\mantt{2D}}}  The %
output extinction corrected dataset.
\manparameterentry{3}{{\mantt{RA}}}{Char}     The {\mantt{B1950}} mean Right %
Ascension of the
                               observed source.
\manparameterentry{4}{{\mantt{DEC}}}{Char}     The {\mantt{B1950}} mean %
declination of the observed
                               source.
\manparameterentry{5}{{\mantt{OBS}}}{Char}     Observing station (? for list), %
{\mantt{NO}} to give
                               longitude and latitude explicitly.
\manparameterentry{}{{\mantt{LONG}}}{Double}   Longitude of site (degrees, %
west +ve)
\manparameterentry{}{{\mantt{LAT}}}{Double}   Geodetic latitude of site (%
degrees, north +ve)
\manparameterentry{}{{\mantt{EXTCOEF}}}{Real}     Extinction coefficient (%
prompt is repeated
                              for each channel).

\end{manparametertable}
\manroutineitem{Support}{Jeremy Bailey, {\mantt{AAO}}}
\manroutineitem{Version date}{24/2/1992}
\end{manroutinedescription}
\manroutine{{\mantt{TLIST}}}{List time series data to a file.}{TLIST}
\begin{manroutinedescription}
\manroutineitem{Function}{}
        List time series data to a file.

\manroutineitem{Description}{}
        List the time, intensity and percentage stokes parameters
        for a specified channel of a time series data set to an
        {\mantt{ASCII}} file

\manroutineitem{Parameters}{}
\begin{manparametertable}
\manparameterentry{1}{{\mantt{INPUT}}}{{\mantt{TSP}}, {\mantt{2D}}}  The input %
time series dataset.
\manparameterentry{2}{{\mantt{CHAN}}}{Integer}  Channel to plot.
\manparameterentry{}{{\mantt{WHOLE}}}{Logical}  If {\mantt{TRUE}}, All time %
points are used.
\manparameterentry{C}{{\mantt{XSTART}}}{Double}   First time value ({\mantt{%
MJD}}) to use.
\manparameterentry{C}{{\mantt{XEND}}}{Double}   Last time value ({\mantt{MJD}}) %
to use.
\manparameterentry{}{{\mantt{FILE}}}{File}     Name of file for output

\end{manparametertable}
\manroutineitem{Support}{}
         Jeremy Bailey, {\mantt{AAO}}

\manroutineitem{Version date}{}
         27/2/1990

\end{manroutinedescription}
\manroutine{{\mantt{TMERGE}}}{Merge two time series datasets.}{TMERGE}
\begin{manroutinedescription}
\manroutineitem{Function}{}
        Merge two time series datasets.

\manroutineitem{Description}{}
        {\mantt{TMERGE}} merges two time series datasets covering different %
times
        but with the same number of wavelength channels, to form a single
        dataset. The files should be merged in their time order, i.e. {\mantt{%
INPUT1}}
        should be earlier than {\mantt{INPUT2}} to retain increasing time in the
        merged dataset. The operation can be applied to both {\mantt{2D}} or {%
\mantt{3D}}
        datasets

\manroutineitem{Parameters}{}
\begin{manparametertable}
\manparameterentry{1}{{\mantt{INPUT1}}}{{\mantt{TSP}}, {\mantt{2D}} or {\mantt{%
3D}}}  The first input dataset.
\manparameterentry{2}{{\mantt{INPUT2}}}{{\mantt{TSP}}, {\mantt{2D}} or {\mantt{%
3D}}}  The second input dataset.
\manparameterentry{3}{{\mantt{OUTPUT}}}{{\mantt{TSP}}, {\mantt{2D}} or {\mantt{%
3D}}}  The output merged dataset.

\end{manparametertable}
\manroutineitem{Support}{}
          Jeremy Bailey, {\mantt{AAO}}

\manroutineitem{Version date}{}
          31/10/1989

\end{manroutinedescription}
\manroutine{{\mantt{TSETBAD}}}{Interactively mark bad points in time series}{%
TSETBAD}
\begin{manroutinedescription}
\manroutineitem{Function}{}
        Interactively mark bad points in time series

\manroutineitem{Description}{}
        Mark points in a time series as bad by specifying the channel
        number and time point number. The intensity and Stokes parameter
        values for the selected data points are flagged with a bad value
        which will be ignored by subsequent applications.

\manroutineitem{Parameters}{}
\begin{manparametertable}
\manparameterentry{1}{{\mantt{INPUT}}}{{\mantt{TSP}}, {\mantt{2D}}}  The input %
time series dataset
\manparameterentry{2}{{\mantt{OUTPUT}}}{{\mantt{TSP}}, {\mantt{2D}}}  The %
output dataset with
                               bad points flagged.
\manparameterentry{}{{\mantt{CHAN}}}{Integer}  Channel number of point.
\manparameterentry{}{X}{Integer}  Time bin number of point.

\end{manparametertable}
\manroutineitem{Support}{}
         Jeremy Bailey, {\mantt{AAO}}

\manroutineitem{Version date}{}
         1/3/1990

\end{manroutinedescription}
\manroutine{{\mantt{TSEXTRACT}}}{Optimal extraction of a light curve from a %
time series image}{TSEXTRACT}
\begin{manroutinedescription}
\manroutineitem{Function}{}
        Optimal extraction of a light curve from a time series image

\manroutineitem{Description}{}
        This command performs optimal extraction of a light curve of
        a star from a time series image, using an algorithm which is
        a 3 dimensional analogue of Horne's algorithm for optimal
        extraction of spectra from long slit data.
        The extraction is performed using a profile time series which is
        obtained using the {\mantt{TSPROFILE}} command.

        This command is an alternative to {\mantt{CCDPHOT}} which performs the
        same procedure using simple aperture photometry.

\manroutineitem{Parameters}{}
\begin{manparametertable}
\manparameterentry{1}{{\mantt{INPUT}}}{{\mantt{TSP}}, {\mantt{3D}}}   The time %
series image dataset.
\manparameterentry{2}{{\mantt{PROFILE}}}{{\mantt{TSP}}, {\mantt{3D}}}   The %
spatial profile dataset.
\manparameterentry{3}{{\mantt{OUTPUT}}}{{\mantt{TSP}}, {\mantt{2D}}}   The %
output photometry dataset.
\manparameterentry{4}{{\mantt{LAMBDA}}}{Real}      Wavelength of observation (%
microns)
\manparameterentry{5}{{\mantt{FLUXCAL}}}{Real}      Counts per Jansky

\end{manparametertable}
\manroutineitem{Support}{}
         Jeremy Bailey, {\mantt{AAO}}

\manroutineitem{Version date}{}
         16/11/1991

\end{manroutinedescription}
\manroutine{{\mantt{TSHIFT}}}{Apply a time shift to a dataset.}{TSHIFT}
\begin{manroutinedescription}
\manroutineitem{Function}{}
        Apply a time shift to a dataset.

\manroutineitem{Description}{}
        {\mantt{TSHIFT}} adds a constant to the time axis values of a time
        series dataset. It can be used to correct for time errors
        in the original data.

\manroutineitem{Parameters}{}
\begin{manparametertable}
\manparameterentry{1}{{\mantt{INPUT}}}{{\mantt{TSP}}, {\mantt{2D}}}  The input %
time series dataset.
\manparameterentry{2}{{\mantt{SHIFT}}}{Double}   Time shift to apply (days).
\manparameterentry{3}{{\mantt{OUTPUT}}}{{\mantt{TSP}}, {\mantt{2D}}}  The %
output corrected dataset.

\end{manparametertable}
\manroutineitem{Support}{}
         Jeremy Bailey, {\mantt{AAO}}

\manroutineitem{Version date}{}
         28/2/1988

\end{manroutinedescription}
\manroutine{{\mantt{TSPLOT}}}{Plot time series data.}{TSPLOT}
\begin{manroutinedescription}
\manroutineitem{Function}{}
        Plot time series data.

\manroutineitem{Description}{}
        {\mantt{TSPLOT}} plots time series data against time. Up to six items %
may be
        plotted. Each item may be a different channel or Stokes parameter etc.
        The data may be binned (all points in a given time bin averaged).
        or points plotted individually. Plotting is done with the
        {\mantt{NCAR}}/{\mantt{SGS}}/{\mantt{GKS}} graphics system.

\manroutineitem{Parameters}{}
\begin{manparametertable}
\manparameterentry{1}{{\mantt{INPUT}}}{{\mantt{TSP}}, {\mantt{2D}}}  The input %
time series dataset.
\manparameterentry{2}{{\mantt{NPLOTS}}}{Integer}  The number of items to plot (%
max 6).
\manparameterentry{3}{{\mantt{DEVICE}}}{Device}   The Graphics device (any %
valid {\mantt{GKS}} device).
\manparameterentry{}{{\mantt{WHOLE}}}{Logical}  If {\mantt{TRUE}}, All time %
points are used.
\manparameterentry{C}{{\mantt{XSTART}}}{Double}   First time value ({\mantt{%
MJD}}) to use.
\manparameterentry{C}{{\mantt{XEND}}}{Double}   Last time value ({\mantt{MJD}}) %
to use.
\manparameterentry{}{CHANn}{Integer}  Channel for nth plot. This and the %
following
                               parameters repeat for
                               n {\mantt{=}} 1 to {\mantt{NPLOTS}}.
\manparameterentry{}{ITEMn}{Char}     Item for nth plot (I,{\mantt{FLUX}},{%
\mantt{MAG}},Q,U,V,P,{\mantt{THETA}})
\manparameterentry{}{AUTOn}{Logical}  If True nth plot is autoscaled.
\manparameterentry{}{BINn}{Double}   Bin size (negative for no binning).
\manparameterentry{}{PLABELn}{Char}     Label for plot n.
\manparameterentry{C}{MINn}{Real}     Minimum scaling level for plot n.
\manparameterentry{C}{MAXn}{Real}     Maximum scaling level for plot n.
\manparameterentry{}{{\mantt{LABEL}}}{Char}     Label for Diagram.
\manparameterentry{H}{{\mantt{ERRORS}}}{Logical}  If True (default), Error %
bars are plotted.
\manparameterentry{H}{{\mantt{LINE}}}{Logical}  If True, the points are joined %
by a
                               continuous line. (Default False).
\manparameterentry{H}{{\mantt{PEN}}}{Integer}  {\mantt{SGS}} Pen number to %
plot in. (Default 1).

\end{manparametertable}
\manroutineitem{Support}{}
         Jeremy Bailey, {\mantt{AAO}}

\manroutineitem{Version date}{}
         28/2/1988

\end{manroutinedescription}
\manroutine{{\mantt{TSPROFILE}}}{Determine a spatial profile from a time %
series image}{TSPROFILE}
\begin{manroutinedescription}
\manroutineitem{Function}{}
        Determine a spatial profile from a time series image

\manroutineitem{Description}{}
        This command is used to generate a spatial profile time series
        image which is a smoothed representation of the actual stellar
        profile. The profile can be used for subsequent optimal extraction
        of a light curve of the star using the {\mantt{TSEXTRACT}} command.

\manroutineitem{Parameters}{}
\begin{manparametertable}
\manparameterentry{1}{{\mantt{INPUT}}}{{\mantt{TSP}}, {\mantt{3D}}}   The time %
series image dataset
\manparameterentry{2}{{\mantt{PROFILE}}}{{\mantt{TSP}}, {\mantt{3D}}}   The %
output profile
\manparameterentry{3}{{\mantt{RESIDUALS}}}{{\mantt{TSP}}, {\mantt{3D}}}   The %
residuals file
\manparameterentry{4}{X}{Integer}   X position of centre of star
\manparameterentry{5}{Y}{Integer}   Y position of centre of star
\manparameterentry{6}{{\mantt{SIZE}}}{Integer}   Size of box to determine %
profile over
\manparameterentry{7}{{\mantt{DEGREE}}}{Integer}   Degree of polynomial

\end{manparametertable}
\manroutineitem{Support}{Jeremy Bailey, {\mantt{AAO}}}
\manroutineitem{Version date}{14/11/1991}
\end{manroutinedescription}
\manroutine{{\mantt{XCOPY}}}{Copy Wavelength Data from a Figaro Spectrum}{XCOPY}
\begin{manroutinedescription}
\manroutineitem{Function}{}
        Copy Wavelength Data from a Figaro Spectrum

\manroutineitem{Description}{}
        {\mantt{XCOPY}} is equivalent to the Figaro program of the same name
        and copies the wavelength axis information from a Figaro
        spectrum to a {\mantt{TSP}} Polarization spectrum. This provides the
        basic method of wavelength calibrating {\mantt{TSP}} data. First do the
        Arc analysis using the Figaro arc fitting programs and then
        {\mantt{XCOPY}} the resulting calibration to the {\mantt{TSP}} %
polarization spectra.

\manroutineitem{Parameters}{}
\begin{manparametertable}
\manparameterentry{1}{{\mantt{INPUT}}}{{\mantt{TSP}}, {\mantt{1D}}}  The input %
spectrum to be calibrated.
\manparameterentry{2}{{\mantt{ARC}}}{Char}     The Figaro file containing the %
wavelengths.
\manparameterentry{3}{{\mantt{OUTPUT}}}{{\mantt{TSP}}, {\mantt{1D}}}  The %
Output dataset.

\end{manparametertable}
\manroutineitem{Support}{}
         Jeremy Bailey, {\mantt{AAO}}

\manroutineitem{Version date}{}
         16/8/1990

\end{manroutinedescription}



\newpage
                                       
\section{TSPSUBS routines}
\label{app:routines}

\mansection{{\mantt{TSPSUBS}}}{Routines to Create New Structures}
\begin{mansectionroutines}
\mansectionitem{{\mantt{TSP\_{}COPY}}}
     Copy a {\mantt{TSP}} structure from one locator to another

\mansectionitem{{\mantt{TSP\_{}CREATE\_{}1D}}}
     Create a {\mantt{1D}} {\mantt{TSP}} structure

\mansectionitem{{\mantt{TSP\_{}CREATE\_{}2D}}}
     Create a {\mantt{2D}} {\mantt{TSP}} structure

\mansectionitem{{\mantt{TSP\_{}CREATE\_{}3D}}}
     Create a {\mantt{3D}} {\mantt{TSP}} structure

\mansectionitem{{\mantt{TSP\_{}TEMP}}}
     Create a temporary array.

\end{mansectionroutines}
\mansection{{\mantt{TSPSUBS}}}{Routines to Find or Create Stokes Components}
\begin{mansectionroutines}
\mansectionitem{{\mantt{TSP\_{}GET\_{}STOKES}}}
     Get locator to a Stokes component of a polarimetry structure.

\mansectionitem{{\mantt{TSP\_{}STOKES}}}
     Find out which Stokes parameters are present in a dataset

\mansectionitem{{\mantt{TSP\_{}ADD\_{}STOKES}}}
     Add a new Stokes component to a polarimetry structure.

\mansectionitem{{\mantt{TSP\_{}DELETE\_{}STOKES}}}
     Delete a Stokes component from a polarimetry structure.

\end{mansectionroutines}
\mansection{{\mantt{TSPSUBS}}}{Routines to Map or Unmap Data Arrays}
\begin{mansectionroutines}
\mansectionitem{{\mantt{TSP\_{}MAP\_{}DATA}}}
     Map the data array of an {\mantt{NDF}} structure

\mansectionitem{{\mantt{TSP\_{}MAP\_{}SLICE}}}
     Map a slice of the data array of an {\mantt{NDF}} structure

\mansectionitem{{\mantt{TSP\_{}MAP\_{}VSLICE}}}
     Map a slice of the variance array of an {\mantt{NDF}} structure

\mansectionitem{{\mantt{TSP\_{}MAP\_{}VAR}}}
     Map the variance array of an {\mantt{NDF}} structure

\mansectionitem{{\mantt{TSP\_{}MAP\_{}LAMBDA}}}
     Map the wavelength axis of a {\mantt{TSP}} structure

\mansectionitem{{\mantt{TSP\_{}MAP\_{}X}}}
     Map the X axis of a {\mantt{TSP}} structure

\mansectionitem{{\mantt{TSP\_{}MAP\_{}Y}}}
     Map the Y axis of a {\mantt{TSP}} structure

\mansectionitem{{\mantt{TSP\_{}MAP\_{}TIME}}}
     Map the time axis of a {\mantt{TSP}} structure

\mansectionitem{{\mantt{TSP\_{}UNMAP}}}
     Unmap a mapped data array

\end{mansectionroutines}
\mansection{{\mantt{TSPSUBS}}}{Routines to Inquire or Alter the Data Array Size}
\begin{mansectionroutines}
\mansectionitem{{\mantt{TSP\_{}RESIZE}}}
     Change the size of all the data arrays in a structure

\mansectionitem{{\mantt{TSP\_{}SIZE}}}
     Return the dimensions of a {\mantt{TSP}} structure

\end{mansectionroutines}
\mansection{{\mantt{TSPSUBS}}}{Routines to Read or Write {\mantt{LABEL}} and {%
\mantt{UNITS}} strings}
\begin{mansectionroutines}
\mansectionitem{{\mantt{TSP\_{}RLU}}}
     Read the {\mantt{LABEL}} and {\mantt{UNITS}} of a structure

\mansectionitem{{\mantt{TSP\_{}RLU\_{}LAMBDA}}}
     Read the {\mantt{LABEL}} and {\mantt{UNITS}} of a the wavelength axis

\mansectionitem{{\mantt{TSP\_{}RLU\_{}X}}}
     Read the {\mantt{LABEL}} and {\mantt{UNITS}} of the X axis

\mansectionitem{{\mantt{TSP\_{}RLU\_{}Y}}}
     Read the {\mantt{LABEL}} and {\mantt{UNITS}} of the Y axis

\mansectionitem{{\mantt{TSP\_{}RLU\_{}TIME}}}
     Read the {\mantt{LABEL}} and {\mantt{UNITS}} of the time axis

\mansectionitem{{\mantt{TSP\_{}WLU}}}
     Write the {\mantt{LABEL}} and {\mantt{UNITS}} of a structure

\mansectionitem{{\mantt{TSP\_{}WLU\_{}LAMBDA}}}
     Write the {\mantt{LABEL}} and {\mantt{UNITS}} of the wavelength axis

\mansectionitem{{\mantt{TSP\_{}WLU\_{}X}}}
     Write the {\mantt{LABEL}} and {\mantt{UNITS}} of the X axis

\mansectionitem{{\mantt{TSP\_{}WLU\_{}Y}}}
     Write the {\mantt{LABEL}} and {\mantt{UNITS}} of the Y axis

\mansectionitem{{\mantt{TSP\_{}WLU\_{}TIME}}}
     Write the {\mantt{LABEL}} and {\mantt{UNITS}} of the time axis

\end{mansectionroutines}


\newpage
                          
\section{Detailed description of TSPSUBS routines}
\label{app:subdetails}

\manroutine{{\mantt{TSP\_{}ADD\_{}STOKES}}}{Add a new Stokes component to a %
polarimetry structure.}
\begin{manroutinedescription}
\manroutineitem{Function}{}
     Add a new Stokes component to a polarimetry structure.

\manroutineitem{Description}{}
     Given a locator to a polarimetry structure, add a new Stokes
     component to it.

\manroutineitem{Language}{}
     {\mantt{FORTRAN}}

\manroutinebreakitem{Call}{}
     {\mantt{CALL}} {\mantt{TSP\_{}ADD\_{}STOKES}} ({\mantt{LOC}},{\mantt{%
STOKES}},V,{\mantt{STATUS}})

\manroutineitem{Parameters}{(``{\mantt{>}}'' input, ``{\mantt{!}}'' modified, `%
`W'' workspace, ``{\mantt{<}}'' output)}
\begin{manparametertable}
\manparameterentry{{\mantt{>}}}{{\mantt{LOC}}}{Fixed string,descr} A locator %
to the polarimetry
                       structure.
\manparameterentry{{\mantt{>}}}{{\mantt{STOKES}}}{Fixed string,descr} The name %
of the
                       Stokes parameter ('Q', 'U' or 'V')
\manparameterentry{{\mantt{>}}}{V}{Logical,ref} True if a variance array is to %
be
                       included.
\manparameterentry{{\mantt{!}}}{{\mantt{STATUS}}}{Integer,ref} The Status

\end{manparametertable}
\manroutineitem{External subroutines / functions used}{}
     Various {\mantt{NDF}} routines,
     {\mantt{TSP\_{}SIZE}}
\manroutineitem{Support}{Jeremy Bailey, {\mantt{AAO}}}
\manroutineitem{Version date}{30/4/1988}
\end{manroutinedescription}
\manroutine{{\mantt{TSP\_{}COPY}}}{Copy a {\mantt{TSP}} structure from one %
locator to another}
\begin{manroutinedescription}
\manroutineitem{Function}{}
     Copy a {\mantt{TSP}} structure from one locator to another

\manroutineitem{Description}{}
     Given a locator to a {\mantt{TSP}} structure, a complete copy of the %
structure
     is created at a second locator.

\manroutineitem{Language}{}
     {\mantt{FORTRAN}}

\manroutinebreakitem{Call}{}
     {\mantt{CALL}} {\mantt{TSP\_{}COPY}} ({\mantt{LOC}},{\mantt{LOC2}},{%
\mantt{STATUS}})

\manroutineitem{Parameters}{(``{\mantt{>}}'' input, ``{\mantt{!}}'' modified, `%
`W'' workspace, ``{\mantt{<}}'' output)}
\begin{manparametertable}
\manparameterentry{{\mantt{>}}}{{\mantt{LOC}}}{Fixed string,descr} A locator %
to the
                       top level of the object to
                       be copied (e.g. supplied by {\mantt{DAT\_{}ASSOC}})
\manparameterentry{{\mantt{>}}}{{\mantt{LOC2}}}{Fixed string,descr} A locator %
to the top level
                       object of an empty structure in which the
                       copy will be created (e.g. supplied by {\mantt{DAT\_{}%
CREAT}})
\manparameterentry{{\mantt{!}}}{{\mantt{STATUS}}}{Integer,ref} The Status

\end{manparametertable}
\manroutineitem{External subroutines / functions used}{}
     Various {\mantt{NDF}} routines
\manroutineitem{Support}{Jeremy Bailey, {\mantt{AAO}}}
\manroutineitem{Version date}{26/2/1988}
\end{manroutinedescription}
\manroutine{{\mantt{TSP\_{}CREATE\_{}1D}}}{Create a {\mantt{1D}} {\mantt{TSP}} %
structure}
\begin{manroutinedescription}
\manroutineitem{Function}{}
     Create a {\mantt{1D}} {\mantt{TSP}} structure

\manroutineitem{Description}{}
     A {\mantt{1D}} {\mantt{TSP}} structure (representing a polarization %
spectrum) is
     created of the specified size.

\manroutineitem{Language}{}
     {\mantt{FORTRAN}}

\manroutinebreakitem{Call}{}
     {\mantt{CALL}} {\mantt{TSP\_{}CREATE\_{}1D}} ({\mantt{LOC}},{\mantt{SIZE}}%
,{\mantt{STOKES}},{\mantt{VI}},{\mantt{VS}},{\mantt{STATUS}})

\manroutineitem{Parameters}{(``{\mantt{>}}'' input, ``{\mantt{!}}'' modified, `%
`W'' workspace, ``{\mantt{<}}'' output)}
\begin{manparametertable}
\manparameterentry{{\mantt{>}}}{{\mantt{LOC}}}{Fixed string,descr} A locator %
to the
                       top level of the object to
                       be created (e.g. supplied by {\mantt{DAT\_{}CREAT}})
\manparameterentry{{\mantt{>}}}{{\mantt{SIZE}}}{Integer,ref} The size of the %
array to be created.
\manparameterentry{{\mantt{>}}}{{\mantt{STOKES}}}{Fixed string,descr} A string %
specifying which
                       Stokes parameters are to be included in the
                       structure. This must contain some combination
                       of the letters 'Q', 'U' and 'V'
\manparameterentry{{\mantt{>}}}{{\mantt{VI}}}{Logical,ref} True if the %
variance of the intensity
                       is to be included in the structure.
\manparameterentry{{\mantt{>}}}{{\mantt{VS}}}{Logical,ref} True if the %
variance of the Stokes
                       parameters is to be included in the structure.
\manparameterentry{{\mantt{!}}}{{\mantt{STATUS}}}{Integer,ref} The Status

\end{manparametertable}
\manroutineitem{External subroutines / functions used}{}
     Various {\mantt{NDF}} routines
\manroutineitem{Support}{Jeremy Bailey, {\mantt{AAO}}}
\manroutineitem{Version date}{26/2/1988}
\end{manroutinedescription}
\manroutine{{\mantt{TSP\_{}CREATE\_{}2D}}}{Create a {\mantt{2D}} {\mantt{TSP}} %
structure}
\begin{manroutinedescription}
\manroutineitem{Function}{}
     Create a {\mantt{2D}} {\mantt{TSP}} structure

\manroutineitem{Description}{}
     A {\mantt{2D}} {\mantt{TSP}} structure (representing a time series %
polarization spectrum) is
     created of the specified size.

\manroutineitem{Language}{}
     {\mantt{FORTRAN}}

\manroutinebreakitem{Call}{}
     {\mantt{CALL}} {\mantt{TSP\_{}CREATE\_{}2D}} ({\mantt{LOC}},{\mantt{LSIZE}%
},{\mantt{TSIZE}},{\mantt{STOKES}},{\mantt{VI}},{\mantt{VS}},{\mantt{STATUS}})

\manroutineitem{Parameters}{(``{\mantt{>}}'' input, ``{\mantt{!}}'' modified, `%
`W'' workspace, ``{\mantt{<}}'' output)}
\begin{manparametertable}
\manparameterentry{{\mantt{>}}}{{\mantt{LOC}}}{Fixed string,descr} A locator %
to the
                       top level of the object to
                       be created (e.g. supplied by {\mantt{DAT\_{}CREAT}})
\manparameterentry{{\mantt{>}}}{{\mantt{LSIZE}}}{Integer,ref} The size of the %
array to be created
                       in the wavelength axis.
\manparameterentry{{\mantt{>}}}{{\mantt{TSIZE}}}{Integer,ref} The size of the %
array to be created
                       in the wavelength axis.
\manparameterentry{{\mantt{>}}}{{\mantt{STOKES}}}{Fixed string,descr} A string %
specifying which
                       Stokes parameters are to be included in the
                       structure. This must contain some combination
                       of the letters 'Q', 'U' and 'V'
\manparameterentry{{\mantt{>}}}{{\mantt{VI}}}{Logical,ref} True if the %
variance of the intensity
                       is to be included in the structure.
\manparameterentry{{\mantt{>}}}{{\mantt{VS}}}{Logical,ref} True if the %
variance of the Stokes
                       parameters is to be included in the structure.
\manparameterentry{{\mantt{!}}}{{\mantt{STATUS}}}{Integer,ref} The Status

\end{manparametertable}
\manroutineitem{External subroutines / functions used}{}
     Various {\mantt{NDF}} routines
\manroutineitem{Support}{Jeremy Bailey, {\mantt{AAO}}}
\manroutineitem{Version date}{26/2/1988}
\end{manroutinedescription}
\manroutine{{\mantt{TSP\_{}CREATE\_{}3D}}}{Create a {\mantt{3D}} {\mantt{TSP}} %
structure}
\begin{manroutinedescription}
\manroutineitem{Function}{}
     Create a {\mantt{3D}} {\mantt{TSP}} structure

\manroutineitem{Description}{}
     A {\mantt{3D}} {\mantt{TSP}} structure (representing a time series image) %
is
     created of the specified size.

\manroutineitem{Language}{}
     {\mantt{FORTRAN}}

\manroutinebreakitem{Call}{}
     {\mantt{CALL}} {\mantt{TSP\_{}CREATE\_{}3D}} ({\mantt{LOC}},{\mantt{XSIZE}%
},{\mantt{YSIZE}},{\mantt{TSIZE}},{\mantt{STOKES}},{\mantt{VI}},{\mantt{VS}},{%
\mantt{STATUS}})

\manroutineitem{Parameters}{(``{\mantt{>}}'' input, ``{\mantt{!}}'' modified, `%
`W'' workspace, ``{\mantt{<}}'' output)}
\begin{manparametertable}
\manparameterentry{{\mantt{>}}}{{\mantt{LOC}}}{Fixed string,descr} A locator %
to the
                       top level of the object to
                       be created (e.g. supplied by {\mantt{DAT\_{}CREAT}})
\manparameterentry{{\mantt{>}}}{{\mantt{XSIZE}}}{Integer,ref} The size of the %
array to be created
                       in the X axis.
\manparameterentry{{\mantt{>}}}{{\mantt{YSIZE}}}{Integer,ref} The size of the %
array to be created
                       in the Y axis.
\manparameterentry{{\mantt{>}}}{{\mantt{TSIZE}}}{Integer,ref} The size of the %
array to be created
                       in the time axis.
\manparameterentry{{\mantt{>}}}{{\mantt{STOKES}}}{Fixed string,descr} A string %
specifying which
                       Stokes parameters are to be included in the
                       structure. This must contain some combination
                       of the letters 'Q', 'U' and 'V'
\manparameterentry{{\mantt{>}}}{{\mantt{VI}}}{Logical,ref} True if the %
variance of the intensity
                       is to be included in the structure.
\manparameterentry{{\mantt{>}}}{{\mantt{VS}}}{Logical,ref} True if the %
variance of the Stokes
                       parameters is to be included in the structure.
\manparameterentry{{\mantt{!}}}{{\mantt{STATUS}}}{Integer,ref} The Status

\end{manparametertable}
\manroutineitem{External subroutines / functions used}{}
     Various {\mantt{NDF}} routines
\manroutineitem{Support}{Jeremy Bailey, {\mantt{AAO}}}
\manroutineitem{Version date}{19/10/1989}
\end{manroutinedescription}
\manroutine{{\mantt{TSP\_{}DELETE\_{}STOKES}}}{Delete a Stokes component from %
a polarimetry structure.}
\begin{manroutinedescription}
\manroutineitem{Function}{}
     Delete a Stokes component from a polarimetry structure.

\manroutineitem{Description}{}
     Given a locator to a polarimetry structure, Delete a specified Stokes
     component from it.

\manroutineitem{Language}{}
     {\mantt{FORTRAN}}

\manroutinebreakitem{Call}{}
     {\mantt{CALL}} {\mantt{TSP\_{}DELETE\_{}STOKES}} ({\mantt{LOC}},{\mantt{%
STOKES}},{\mantt{STATUS}})

\manroutineitem{Parameters}{(``{\mantt{>}}'' input, ``{\mantt{!}}'' modified, `%
`W'' workspace, ``{\mantt{<}}'' output)}
\begin{manparametertable}
\manparameterentry{{\mantt{>}}}{{\mantt{LOC}}}{Fixed string,descr} A locator %
to the polarimetry
                       structure.
\manparameterentry{{\mantt{>}}}{{\mantt{STOKES}}}{Fixed string,descr} The name %
of the
                       Stokes parameter ('Q', 'U' or 'V')
\manparameterentry{{\mantt{!}}}{{\mantt{STATUS}}}{Integer,ref} The Status

\end{manparametertable}
\manroutineitem{External subroutines / functions used}{}
     Various {\mantt{NDF}} routines
\manroutineitem{Support}{Jeremy Bailey, {\mantt{AAO}}}
\manroutineitem{Version date}{11/8/1988}
\end{manroutinedescription}
\manroutine{{\mantt{TSP\_{}GET\_{}STOKES}}}{Get locator to a Stokes component %
of a polarimetry structure.}
\begin{manroutinedescription}
\manroutineitem{Function}{}
     Get locator to a Stokes component of a polarimetry structure.

\manroutineitem{Description}{}
     Given a locator to a polarimetry structure, return a locator
     to one of its Stokes parameters.

\manroutineitem{Language}{}
     {\mantt{FORTRAN}}

\manroutinebreakitem{Call}{}
     {\mantt{CALL}} {\mantt{TSP\_{}GET\_{}STOKES}} ({\mantt{LOC}},{\mantt{%
STOKES}},{\mantt{LOC2}},{\mantt{STATUS}})

\manroutineitem{Parameters}{(``{\mantt{>}}'' input, ``{\mantt{!}}'' modified, `%
`W'' workspace, ``{\mantt{<}}'' output)}
\begin{manparametertable}
\manparameterentry{{\mantt{>}}}{{\mantt{LOC}}}{Fixed string,descr} A locator %
to the polarimetry
                       structure.
\manparameterentry{{\mantt{>}}}{{\mantt{STOKES}}}{Fixed string,descr} The name %
of the
                       Stokes parameter ('Q', 'U' or 'V')
\manparameterentry{{\mantt{<}}}{{\mantt{LOC2}}}{Fixed string,descr} The %
locator to the
                       Stokes {\mantt{NDF}} object.
\manparameterentry{{\mantt{!}}}{{\mantt{STATUS}}}{Integer,ref} The Status

\end{manparametertable}
\manroutineitem{External subroutines / functions used}{}
     Various {\mantt{NDF}} routines
\manroutineitem{Support}{Jeremy Bailey, {\mantt{AAO}}}
\manroutineitem{Version date}{26/2/1988}
\end{manroutinedescription}
\manroutine{{\mantt{TSP\_{}MAP\_{}DATA}}}{Map the data array of an {\mantt{NDF}%
} structure}
\begin{manroutinedescription}
\manroutineitem{Function}{}
     Map the data array of an {\mantt{NDF}} structure

\manroutineitem{Description}{}
     Given a locator to an {\mantt{NDF}} structure map its main {\mantt{DATA\_{%
}ARRAY}}.

\manroutineitem{Language}{}
     {\mantt{FORTRAN}}

\manroutinebreakitem{Call}{}
     {\mantt{CALL}} {\mantt{TSP\_{}MAP\_{}DATA}} ({\mantt{LOC}},{\mantt{MODE}},%
{\mantt{PTR}},{\mantt{LOC2}},{\mantt{STATUS}})

\manroutineitem{Parameters}{(``{\mantt{>}}'' input, ``{\mantt{!}}'' modified, `%
`W'' workspace, ``{\mantt{<}}'' output)}
\begin{manparametertable}
\manparameterentry{{\mantt{>}}}{{\mantt{LOC}}}{Fixed string,descr} A locator %
to the {\mantt{NDF}}
                       structure.
\manparameterentry{{\mantt{>}}}{{\mantt{MODE}}}{Fixed string,descr} The access %
mode,
                       '{\mantt{READ}}','{\mantt{WRITE}}' or '{\mantt{UPDATE}}'.
\manparameterentry{{\mantt{<}}}{{\mantt{PTR}}}{Integer,ref} Pointer to the %
mapped data.
\manparameterentry{{\mantt{<}}}{{\mantt{LOC2}}}{Fixed string,descr} The %
locator to the
                       mapped data object - needed so that it can
                       be unmapped.
\manparameterentry{{\mantt{!}}}{{\mantt{STATUS}}}{Integer,ref} The Status

\end{manparametertable}
\manroutineitem{External subroutines / functions used}{}
     Various {\mantt{NDF}} routines
\manroutineitem{Support}{Jeremy Bailey, {\mantt{AAO}}}
\manroutineitem{Version date}{26/2/1988}
\end{manroutinedescription}
\manroutine{{\mantt{TSP\_{}MAP\_{}LAMBDA}}}{Map the wavelength axis of a {%
\mantt{TSP}} structure}
\begin{manroutinedescription}
\manroutineitem{Function}{}
     Map the wavelength axis of a {\mantt{TSP}} structure

\manroutineitem{Description}{}
     Given a locator to an {\mantt{TSP}} structure map its wavelength axis data.

\manroutineitem{Language}{}
     {\mantt{FORTRAN}}

\manroutinebreakitem{Call}{}
     {\mantt{CALL}} {\mantt{TSP\_{}MAP\_{}LAMBDA}} ({\mantt{LOC}},{\mantt{MODE}%
},{\mantt{PTR}},{\mantt{LOC2}},{\mantt{STATUS}})

\manroutineitem{Parameters}{(``{\mantt{>}}'' input, ``{\mantt{!}}'' modified, `%
`W'' workspace, ``{\mantt{<}}'' output)}
\begin{manparametertable}
\manparameterentry{{\mantt{>}}}{{\mantt{LOC}}}{Fixed string,descr} A locator %
to the {\mantt{NDF}}
                       structure.
\manparameterentry{{\mantt{>}}}{{\mantt{MODE}}}{Fixed string,descr} The access %
mode,
                       '{\mantt{READ}}','{\mantt{WRITE}}' or '{\mantt{UPDATE}}'.
\manparameterentry{{\mantt{<}}}{{\mantt{PTR}}}{Integer,ref} Pointer to the %
mapped data.
\manparameterentry{{\mantt{<}}}{{\mantt{LOC2}}}{Fixed string,descr} The %
locator to the
                       mapped data object - needed so that it can
                       be unmapped.
\manparameterentry{{\mantt{!}}}{{\mantt{STATUS}}}{Integer,ref} The Status

\end{manparametertable}
\manroutineitem{External subroutines / functions used}{}
     {\mantt{TSP\_{}MAP\_{}DATA}},
     Various {\mantt{NDF}} routines
\manroutineitem{Support}{Jeremy Bailey, {\mantt{AAO}}}
\manroutineitem{Version date}{26/2/1988}
\end{manroutinedescription}
\manroutine{{\mantt{TSP\_{}MAP\_{}SLICE}}}{Map a slice of the data array of an %
{\mantt{NDF}} structure}
\begin{manroutinedescription}
\manroutineitem{Function}{}
     Map a slice of the data array of an {\mantt{NDF}} structure

\manroutineitem{Description}{}
     Given a locator to an {\mantt{NDF}} structure map a slice of its main {%
\mantt{DATA\_{}ARRAY}}.

\manroutineitem{Language}{}
     {\mantt{FORTRAN}}

\manroutinebreakitem{Call}{}
     {\mantt{CALL}} {\mantt{TSP\_{}MAP\_{}SLICE}} ({\mantt{LOC}},{\mantt{NDIMS}%
},{\mantt{LOWER}},{\mantt{UPPER}},{\mantt{MODE}},{\mantt{PTR}},{\mantt{LOC2}},{%
\mantt{STATUS}})

\manroutineitem{Parameters}{(``{\mantt{>}}'' input, ``{\mantt{!}}'' modified, `%
`W'' workspace, ``{\mantt{<}}'' output)}
\begin{manparametertable}
\manparameterentry{{\mantt{>}}}{{\mantt{LOC}}}{Fixed string,descr} A locator %
to the {\mantt{NDF}}
                       structure.
\manparameterentry{{\mantt{>}}}{{\mantt{NDIMS}}}{Integer,ref} The number of %
dimensions.
\manparameterentry{{\mantt{>}}}{{\mantt{LOWER}}}{Integer array,ref} Array of %
lower bounds
                       for the slice.
\manparameterentry{{\mantt{>}}}{{\mantt{UPPER}}}{Integer array,ref} Array of %
upper bounds
                       for the slice.
\manparameterentry{{\mantt{>}}}{{\mantt{MODE}}}{Fixed string,descr} The access %
mode,
                       '{\mantt{READ}}','{\mantt{WRITE}}' or '{\mantt{UPDATE}}'.
\manparameterentry{{\mantt{<}}}{{\mantt{PTR}}}{Integer,ref} Pointer to the %
mapped data.
\manparameterentry{{\mantt{<}}}{{\mantt{LOC2}}}{Fixed string,descr} The %
locator to the
                       mapped data object - needed so that it can
                       be unmapped.
\manparameterentry{{\mantt{!}}}{{\mantt{STATUS}}}{Integer,ref} The Status

\end{manparametertable}
\manroutineitem{External subroutines / functions used}{}
     Various {\mantt{NDF}} routines
\manroutineitem{Support}{Jeremy Bailey, {\mantt{AAO}}}
\manroutineitem{Version date}{27/2/1988}
\end{manroutinedescription}
\manroutine{{\mantt{TSP\_{}MAP\_{}TIME}}}{Map the time axis of a {\mantt{TSP}} %
structure}
\begin{manroutinedescription}
\manroutineitem{Function}{}
     Map the time axis of a {\mantt{TSP}} structure

\manroutineitem{Description}{}
     Given a locator to an {\mantt{TSP}} structure, map its time axis data.

\manroutineitem{Language}{}
     {\mantt{FORTRAN}}

\manroutinebreakitem{Call}{}
     {\mantt{CALL}} {\mantt{TSP\_{}MAP\_{}TIME}} ({\mantt{LOC}},{\mantt{MODE}},%
{\mantt{PTR}},{\mantt{LOC2}},{\mantt{STATUS}})

\manroutineitem{Parameters}{(``{\mantt{>}}'' input, ``{\mantt{!}}'' modified, `%
`W'' workspace, ``{\mantt{<}}'' output)}
\begin{manparametertable}
\manparameterentry{{\mantt{>}}}{{\mantt{LOC}}}{Fixed string,descr} A locator %
to the {\mantt{NDF}}
                       structure.
\manparameterentry{{\mantt{>}}}{{\mantt{MODE}}}{Fixed string,descr} The access %
mode,
                       '{\mantt{READ}}','{\mantt{WRITE}}' or '{\mantt{UPDATE}}'.
\manparameterentry{{\mantt{<}}}{{\mantt{PTR}}}{Integer,ref} Pointer to the %
mapped data.
                       A double precision array of {\mantt{MJD}}.
\manparameterentry{{\mantt{<}}}{{\mantt{LOC2}}}{Fixed string,descr} The %
locator to the
                       mapped data object - needed so that it can
                       be unmapped.
\manparameterentry{{\mantt{!}}}{{\mantt{STATUS}}}{Integer,ref} The Status

\end{manparametertable}
\manroutineitem{External subroutines / functions used}{}
     Various {\mantt{NDF}} routines
\manroutineitem{Support}{Jeremy Bailey, {\mantt{AAO}}}
\manroutineitem{Version date}{26/2/1988}
\end{manroutinedescription}
\manroutine{{\mantt{TSP\_{}MAP\_{}VAR}}}{Map the variance array of an {\mantt{%
NDF}} structure}
\begin{manroutinedescription}
\manroutineitem{Function}{}
     Map the variance array of an {\mantt{NDF}} structure

\manroutineitem{Description}{}
     Given a locator to an {\mantt{NDF}} structure map its {\mantt{VARIANCE}} %
array.

\manroutineitem{Language}{}
     {\mantt{FORTRAN}}

\manroutinebreakitem{Call}{}
     {\mantt{CALL}} {\mantt{TSP\_{}MAP\_{}VAR}} ({\mantt{LOC}},{\mantt{MODE}},{%
\mantt{PTR}},{\mantt{LOC2}},{\mantt{STATUS}})

\manroutineitem{Parameters}{(``{\mantt{>}}'' input, ``{\mantt{!}}'' modified, `%
`W'' workspace, ``{\mantt{<}}'' output)}
\begin{manparametertable}
\manparameterentry{{\mantt{>}}}{{\mantt{LOC}}}{Fixed string,descr} A locator %
to the {\mantt{NDF}}
                       structure.
\manparameterentry{{\mantt{>}}}{{\mantt{MODE}}}{Fixed string,descr} The access %
mode,
                       '{\mantt{READ}}','{\mantt{WRITE}}' or '{\mantt{UPDATE}}'.
\manparameterentry{{\mantt{<}}}{{\mantt{PTR}}}{Integer,ref} Pointer to the %
mapped data.
\manparameterentry{{\mantt{<}}}{{\mantt{LOC2}}}{Fixed string,descr} The %
locator to the
                       mapped data object - needed so that it can
                       be unmapped.
\manparameterentry{{\mantt{!}}}{{\mantt{STATUS}}}{Integer,ref} The Status

\end{manparametertable}
\manroutineitem{External subroutines / functions used}{}
     Various {\mantt{NDF}} routines
\manroutineitem{Support}{Jeremy Bailey, {\mantt{AAO}}}
\manroutineitem{Version date}{26/2/1988}
\end{manroutinedescription}
\manroutine{{\mantt{TSP\_{}MAP\_{}VSLICE}}}{Map a slice of the variance array %
of an {\mantt{NDF}} structure}
\begin{manroutinedescription}
\manroutineitem{Function}{}
     Map a slice of the variance array of an {\mantt{NDF}} structure

\manroutineitem{Description}{}
     Given a locator to an {\mantt{NDF}} structure map a slice of its {\mantt{%
VARIANCE}} array.

\manroutineitem{Language}{}
     {\mantt{FORTRAN}}

\manroutinebreakitem{Call}{}
     {\mantt{CALL}} {\mantt{TSP\_{}MAP\_{}VSLICE}} ({\mantt{LOC}},{\mantt{%
NDIMS}},{\mantt{LOWER}},{\mantt{UPPER}},{\mantt{MODE}},{\mantt{PTR}},{\mantt{%
LOC2}},{\mantt{STATUS}})

\manroutineitem{Parameters}{(``{\mantt{>}}'' input, ``{\mantt{!}}'' modified, `%
`W'' workspace, ``{\mantt{<}}'' output)}
\begin{manparametertable}
\manparameterentry{{\mantt{>}}}{{\mantt{LOC}}}{Fixed string,descr} A locator %
to the {\mantt{NDF}}
                       structure.
\manparameterentry{{\mantt{>}}}{{\mantt{NDIMS}}}{Integer,ref} The number of %
dimensions.
\manparameterentry{{\mantt{>}}}{{\mantt{LOWER}}}{Integer array,ref} Array of %
lower bounds
                       for the slice.
\manparameterentry{{\mantt{>}}}{{\mantt{UPPER}}}{Integer array,ref} Array of %
upper bounds
                       for the slice.
\manparameterentry{{\mantt{>}}}{{\mantt{MODE}}}{Fixed string,descr} The access %
mode,
                       '{\mantt{READ}}','{\mantt{WRITE}}' or '{\mantt{UPDATE}}'.
\manparameterentry{{\mantt{<}}}{{\mantt{PTR}}}{Integer,ref} Pointer to the %
mapped data.
\manparameterentry{{\mantt{<}}}{{\mantt{LOC2}}}{Fixed string,descr} The %
locator to the
                       mapped data object - needed so that it can
                       be unmapped.
\manparameterentry{{\mantt{!}}}{{\mantt{STATUS}}}{Integer,ref} The Status

\end{manparametertable}
\manroutineitem{External subroutines / functions used}{}
     Various {\mantt{NDF}} routines
\manroutineitem{Support}{Jeremy Bailey, {\mantt{AAO}}}
\manroutineitem{Version date}{28/2/1988}
\end{manroutinedescription}
\manroutine{{\mantt{TSP\_{}MAP\_{}X}}}{Map the X axis of a {\mantt{TSP}} %
structure}
\begin{manroutinedescription}
\manroutineitem{Function}{}
     Map the X axis of a {\mantt{TSP}} structure

\manroutineitem{Description}{}
     Given a locator to a {\mantt{3D}} {\mantt{TSP}} structure map its X axis %
data.

\manroutineitem{Language}{}
     {\mantt{FORTRAN}}

\manroutinebreakitem{Call}{}
     {\mantt{CALL}} {\mantt{TSP\_{}MAP\_{}X}} ({\mantt{LOC}},{\mantt{MODE}},{%
\mantt{PTR}},{\mantt{LOC2}},{\mantt{STATUS}})

\manroutineitem{Parameters}{(``{\mantt{>}}'' input, ``{\mantt{!}}'' modified, `%
`W'' workspace, ``{\mantt{<}}'' output)}
\begin{manparametertable}
\manparameterentry{{\mantt{>}}}{{\mantt{LOC}}}{Fixed string,descr} A locator %
to the {\mantt{NDF}}
                       structure.
\manparameterentry{{\mantt{>}}}{{\mantt{MODE}}}{Fixed string,descr} The access %
mode,
                       '{\mantt{READ}}','{\mantt{WRITE}}' or '{\mantt{UPDATE}}'.
\manparameterentry{{\mantt{<}}}{{\mantt{PTR}}}{Integer,ref} Pointer to the %
mapped data.
\manparameterentry{{\mantt{<}}}{{\mantt{LOC2}}}{Fixed string,descr} The %
locator to the
                       mapped data object - needed so that it can
                       be unmapped.
\manparameterentry{{\mantt{!}}}{{\mantt{STATUS}}}{Integer,ref} The Status

\end{manparametertable}
\manroutineitem{External subroutines / functions used}{}
     {\mantt{TSP\_{}MAP\_{}DATA}},
     Various {\mantt{NDF}} routines
\manroutineitem{Support}{Jeremy Bailey, {\mantt{AAO}}}
\manroutineitem{Version date}{19/10/1989}
\end{manroutinedescription}
\manroutine{{\mantt{TSP\_{}MAP\_{}Y}}}{Map the Y axis of a {\mantt{TSP}} %
structure}
\begin{manroutinedescription}
\manroutineitem{Function}{}
     Map the Y axis of a {\mantt{TSP}} structure

\manroutineitem{Description}{}
     Given a locator to a {\mantt{3D}} {\mantt{TSP}} structure map its Y axis %
data.

\manroutineitem{Language}{}
     {\mantt{FORTRAN}}

\manroutinebreakitem{Call}{}
     {\mantt{CALL}} {\mantt{TSP\_{}MAP\_{}Y}} ({\mantt{LOC}},{\mantt{MODE}},{%
\mantt{PTR}},{\mantt{LOC2}},{\mantt{STATUS}})

\manroutineitem{Parameters}{(``{\mantt{>}}'' input, ``{\mantt{!}}'' modified, `%
`W'' workspace, ``{\mantt{<}}'' output)}
\begin{manparametertable}
\manparameterentry{{\mantt{>}}}{{\mantt{LOC}}}{Fixed string,descr} A locator %
to the {\mantt{NDF}}
                       structure.
\manparameterentry{{\mantt{>}}}{{\mantt{MODE}}}{Fixed string,descr} The access %
mode,
                       '{\mantt{READ}}','{\mantt{WRITE}}' or '{\mantt{UPDATE}}'.
\manparameterentry{{\mantt{<}}}{{\mantt{PTR}}}{Integer,ref} Pointer to the %
mapped data.
\manparameterentry{{\mantt{<}}}{{\mantt{LOC2}}}{Fixed string,descr} The %
locator to the
                       mapped data object - needed so that it can
                       be unmapped.
\manparameterentry{{\mantt{!}}}{{\mantt{STATUS}}}{Integer,ref} The Status

\end{manparametertable}
\manroutineitem{External subroutines / functions used}{}
     {\mantt{TSP\_{}MAP\_{}DATA}},
     Various {\mantt{NDF}} routines
\manroutineitem{Support}{Jeremy Bailey, {\mantt{AAO}}}
\manroutineitem{Version date}{19/10/1989}
\end{manroutinedescription}
\manroutine{{\mantt{TSP\_{}RESIZE}}}{Change the size of all the data arrays in %
a structure}
\begin{manroutinedescription}
\manroutineitem{Function}{}
     Change the size of all the data arrays in a structure

\manroutineitem{Description}{}
     Given a locator to a polarimetry structure, change the
     size of the arrays in the structure. If the change
     consists solely of a change to the last dimension then
     data in the arrays will retain values from corresponding
     components in the original array. With more complex
     changes the data values will be lost. Axis arrays will
     retain their values where still present in the output array.

\manroutineitem{Language}{}
     {\mantt{FORTRAN}}

\manroutinebreakitem{Call}{}
     {\mantt{CALL}} {\mantt{TSP\_{}RESIZE}} ({\mantt{LOC}},{\mantt{NDIM}},{%
\mantt{DIMS}},{\mantt{STATUS}})

\manroutineitem{Parameters}{(``{\mantt{>}}'' input, ``{\mantt{!}}'' modified, `%
`W'' workspace, ``{\mantt{<}}'' output)}
\begin{manparametertable}
\manparameterentry{{\mantt{>}}}{{\mantt{LOC}}}{Fixed string,descr} A locator %
to the polarimetry
                       structure.
\manparameterentry{{\mantt{>}}}{{\mantt{NDIM}}}{Integer,ref} The new number of %
Dimensions for
                       the structure
\manparameterentry{{\mantt{>}}}{{\mantt{DIMS}}}{Integer Array,ref} The new %
dimensions for the
                       structure.
\manparameterentry{{\mantt{!}}}{{\mantt{STATUS}}}{Integer,ref} The Status

\end{manparametertable}
\manroutineitem{External subroutines / functions used}{}
     Various {\mantt{NDF}} routines
\manroutineitem{Support}{Jeremy Bailey, {\mantt{AAO}}}
\manroutineitem{Version date}{1/3/1988}
\end{manroutinedescription}
\manroutine{{\mantt{TSP\_{}RLU}}}{Read the {\mantt{LABEL}} and {\mantt{UNITS}} %
of a structure}
\begin{manroutinedescription}
\manroutineitem{Function}{}
     Read the {\mantt{LABEL}} and {\mantt{UNITS}} of a structure

\manroutineitem{Description}{}
     Given a locator to an {\mantt{TSP}} structure, return the {\mantt{LABEL}} %
and {\mantt{UNITS}}
     of the data array.

\manroutineitem{Language}{}
     {\mantt{FORTRAN}}

\manroutinebreakitem{Call}{}
     {\mantt{CALL}} {\mantt{TSP\_{}RLU}} ({\mantt{LOC}},{\mantt{LABEL}},{%
\mantt{UNITS}},{\mantt{STATUS}})

\manroutineitem{Parameters}{(``{\mantt{>}}'' input, ``{\mantt{!}}'' modified, `%
`W'' workspace, ``{\mantt{<}}'' output)}
\begin{manparametertable}
\manparameterentry{{\mantt{>}}}{{\mantt{LOC}}}{Fixed string,descr} A locator %
to the {\mantt{NDF}}
                       structure.
\manparameterentry{{\mantt{<}}}{{\mantt{LABEL}}}{Fixed string,descr} Label %
string.
\manparameterentry{{\mantt{<}}}{{\mantt{UNITS}}}{Fixed string,descr} Units %
string.
\manparameterentry{{\mantt{!}}}{{\mantt{STATUS}}}{Integer,ref} The Status

\end{manparametertable}
\manroutineitem{External subroutines / functions used}{}
     Various {\mantt{NDF}} routines
\manroutineitem{Support}{Jeremy Bailey, {\mantt{AAO}}}
\manroutineitem{Version date}{11/3/1988}
\end{manroutinedescription}
\manroutine{{\mantt{TSP\_{}RLU\_{}LAMBDA}}}{Read the {\mantt{LABEL}} and {%
\mantt{UNITS}} of a the wavelength axis}
\begin{manroutinedescription}
\manroutineitem{Function}{}
     Read the {\mantt{LABEL}} and {\mantt{UNITS}} of a the wavelength axis

\manroutineitem{Description}{}
     Given a locator to an {\mantt{TSP}} structure, return the {\mantt{LABEL}} %
and {\mantt{UNITS}}
     of the wavelength axis.

\manroutineitem{Language}{}
     {\mantt{FORTRAN}}

\manroutinebreakitem{Call}{}
     {\mantt{CALL}} {\mantt{TSP\_{}RLU\_{}LAMBDA}} ({\mantt{LOC}},{\mantt{%
LABEL}},{\mantt{UNITS}},{\mantt{STATUS}})

\manroutineitem{Parameters}{(``{\mantt{>}}'' input, ``{\mantt{!}}'' modified, `%
`W'' workspace, ``{\mantt{<}}'' output)}
\begin{manparametertable}
\manparameterentry{{\mantt{>}}}{{\mantt{LOC}}}{Fixed string,descr} A locator %
to the {\mantt{NDF}}
                       structure.
\manparameterentry{{\mantt{<}}}{{\mantt{LABEL}}}{Fixed string,descr} Label %
string.
\manparameterentry{{\mantt{<}}}{{\mantt{UNITS}}}{Fixed string,descr} Units %
string.
\manparameterentry{{\mantt{!}}}{{\mantt{STATUS}}}{Integer,ref} The Status

\end{manparametertable}
\manroutineitem{External subroutines / functions used}{}
     {\mantt{TSP\_{}RLU}},
     Various {\mantt{NDF}} routines
\manroutineitem{Support}{Jeremy Bailey, {\mantt{AAO}}}
\manroutineitem{Version date}{27/2/1988}
\end{manroutinedescription}
\manroutine{{\mantt{TSP\_{}RLU\_{}TIME}}}{Read the {\mantt{LABEL}} and {\mantt{%
UNITS}} of the time axis}
\begin{manroutinedescription}
\manroutineitem{Function}{}
     Read the {\mantt{LABEL}} and {\mantt{UNITS}} of the time axis

\manroutineitem{Description}{}
     Given a locator to an {\mantt{TSP}} structure, return the {\mantt{LABEL}} %
and {\mantt{UNITS}}
     of the time axis.

\manroutineitem{Language}{}
     {\mantt{FORTRAN}}

\manroutinebreakitem{Call}{}
     {\mantt{CALL}} {\mantt{TSP\_{}RLU\_{}TIME}} ({\mantt{LOC}},{\mantt{LABEL}}%
,{\mantt{UNITS}},{\mantt{STATUS}})

\manroutineitem{Parameters}{(``{\mantt{>}}'' input, ``{\mantt{!}}'' modified, `%
`W'' workspace, ``{\mantt{<}}'' output)}
\begin{manparametertable}
\manparameterentry{{\mantt{>}}}{{\mantt{LOC}}}{Fixed string,descr} A locator %
to the {\mantt{NDF}}
                       structure.
\manparameterentry{{\mantt{<}}}{{\mantt{LABEL}}}{Fixed string,descr} Label %
string.
\manparameterentry{{\mantt{<}}}{{\mantt{UNITS}}}{Fixed string,descr} Units %
string.
\manparameterentry{{\mantt{!}}}{{\mantt{STATUS}}}{Integer,ref} The Status

\end{manparametertable}
\manroutineitem{External subroutines / functions used}{}
     {\mantt{TSP\_{}RLU}},
     Various {\mantt{NDF}} routines
\manroutineitem{Support}{Jeremy Bailey, {\mantt{AAO}}}
\manroutineitem{Version date}{27/2/1988}
\end{manroutinedescription}
\manroutine{{\mantt{TSP\_{}RLU\_{}X}}}{Read the {\mantt{LABEL}} and {\mantt{%
UNITS}} of the X axis}
\begin{manroutinedescription}
\manroutineitem{Function}{}
     Read the {\mantt{LABEL}} and {\mantt{UNITS}} of the X axis

\manroutineitem{Description}{}
     Given a locator to an {\mantt{TSP}} structure, return the {\mantt{LABEL}} %
and {\mantt{UNITS}}
     of the X axis.

\manroutineitem{Language}{}
     {\mantt{FORTRAN}}

\manroutinebreakitem{Call}{}
     {\mantt{CALL}} {\mantt{TSP\_{}RLU\_{}X}} ({\mantt{LOC}},{\mantt{LABEL}},{%
\mantt{UNITS}},{\mantt{STATUS}})

\manroutineitem{Parameters}{(``{\mantt{>}}'' input, ``{\mantt{!}}'' modified, `%
`W'' workspace, ``{\mantt{<}}'' output)}
\begin{manparametertable}
\manparameterentry{{\mantt{>}}}{{\mantt{LOC}}}{Fixed string,descr} A locator %
to the {\mantt{NDF}}
                       structure.
\manparameterentry{{\mantt{<}}}{{\mantt{LABEL}}}{Fixed string,descr} Label %
string.
\manparameterentry{{\mantt{<}}}{{\mantt{UNITS}}}{Fixed string,descr} Units %
string.
\manparameterentry{{\mantt{!}}}{{\mantt{STATUS}}}{Integer,ref} The Status

\end{manparametertable}
\manroutineitem{External subroutines / functions used}{}
     {\mantt{TSP\_{}RLU}},
     Various {\mantt{NDF}} routines
\manroutineitem{Support}{Jeremy Bailey, {\mantt{AAO}}}
\manroutineitem{Version date}{19/10/1989}
\end{manroutinedescription}
\manroutine{{\mantt{TSP\_{}RLU\_{}Y}}}{Read the {\mantt{LABEL}} and {\mantt{%
UNITS}} of the Y axis}
\begin{manroutinedescription}
\manroutineitem{Function}{}
     Read the {\mantt{LABEL}} and {\mantt{UNITS}} of the Y axis

\manroutineitem{Description}{}
     Given a locator to an {\mantt{TSP}} structure, return the {\mantt{LABEL}} %
and {\mantt{UNITS}}
     of the Y axis.

\manroutineitem{Language}{}
     {\mantt{FORTRAN}}

\manroutinebreakitem{Call}{}
     {\mantt{CALL}} {\mantt{TSP\_{}RLU\_{}Y}} ({\mantt{LOC}},{\mantt{LABEL}},{%
\mantt{UNITS}},{\mantt{STATUS}})

\manroutineitem{Parameters}{(``{\mantt{>}}'' input, ``{\mantt{!}}'' modified, `%
`W'' workspace, ``{\mantt{<}}'' output)}
\begin{manparametertable}
\manparameterentry{{\mantt{>}}}{{\mantt{LOC}}}{Fixed string,descr} A locator %
to the {\mantt{NDF}}
                       structure.
\manparameterentry{{\mantt{<}}}{{\mantt{LABEL}}}{Fixed string,descr} Label %
string.
\manparameterentry{{\mantt{<}}}{{\mantt{UNITS}}}{Fixed string,descr} Units %
string.
\manparameterentry{{\mantt{!}}}{{\mantt{STATUS}}}{Integer,ref} The Status

\end{manparametertable}
\manroutineitem{External subroutines / functions used}{}
     {\mantt{TSP\_{}RLU}},
     Various {\mantt{NDF}} routines
\manroutineitem{Support}{Jeremy Bailey, {\mantt{AAO}}}
\manroutineitem{Version date}{19/10/1989}
\end{manroutinedescription}
\manroutine{{\mantt{TSP\_{}SIZE}}}{Return the dimensions of a {\mantt{TSP}} %
structure}
\begin{manroutinedescription}
\manroutineitem{Function}{}
     Return the dimensions of a {\mantt{TSP}} structure

\manroutineitem{Description}{}
     Given a locator to an {\mantt{TSP}} structure, return the dimensions and
     number of dimensions.

\manroutineitem{Language}{}
     {\mantt{FORTRAN}}

\manroutinebreakitem{Call}{}
     {\mantt{CALL}} {\mantt{TSP\_{}SIZE}} ({\mantt{LOC}},{\mantt{MAXDIM}},{%
\mantt{DIMS}},{\mantt{ACTDIM}},{\mantt{STATUS}})

\manroutineitem{Parameters}{(``{\mantt{>}}'' input, ``{\mantt{!}}'' modified, `%
`W'' workspace, ``{\mantt{<}}'' output)}
\begin{manparametertable}
\manparameterentry{{\mantt{>}}}{{\mantt{LOC}}}{Fixed string,descr} A locator %
to the {\mantt{NDF}}
                       structure.
\manparameterentry{{\mantt{>}}}{{\mantt{MAXDIM}}}{Integer,ref} Size of {\mantt{%
DIMS}}
\manparameterentry{{\mantt{<}}}{{\mantt{DIMS}}}{Integer array,ref} Array to %
receive the
                       size of each dimension.
\manparameterentry{{\mantt{<}}}{{\mantt{ACTDIM}}}{Integer,ref} Actual number %
of dimensions.
\manparameterentry{{\mantt{!}}}{{\mantt{STATUS}}}{Integer,ref} The Status

\end{manparametertable}
\manroutineitem{External subroutines / functions used}{}
     Various {\mantt{NDF}} routines
\manroutineitem{Support}{Jeremy Bailey, {\mantt{AAO}}}
\manroutineitem{Version date}{27/2/1988}
\end{manroutinedescription}
\manroutine{{\mantt{TSP\_{}STOKES}}}{Find out which Stokes parameters are %
present in a dataset}
\begin{manroutinedescription}
\manroutineitem{Function}{}
     Find out which Stokes parameters are present in a dataset

\manroutineitem{Description}{}
     Given a locator to a polarimetry structure, return the number
     of Stokes parameters, and their identities.

\manroutineitem{Language}{}
     {\mantt{FORTRAN}}

\manroutinebreakitem{Call}{}
     {\mantt{CALL}} {\mantt{TSP\_{}STOKES}} ({\mantt{LOC}},{\mantt{NUM}},Q,U,V,%
{\mantt{STATUS}})

\manroutineitem{Parameters}{(``{\mantt{>}}'' input, ``{\mantt{!}}'' modified, `%
`W'' workspace, ``{\mantt{<}}'' output)}
\begin{manparametertable}
\manparameterentry{{\mantt{>}}}{{\mantt{LOC}}}{Fixed string,descr} A locator %
to the polarimetry
                       structure.
\manparameterentry{{\mantt{<}}}{{\mantt{NUM}}}{Integer,ref} The number of Stokes
                       parameters in the dataset
\manparameterentry{{\mantt{<}}}{Q}{Logical,ref} True if the Q Stokes parameter
                       is present
\manparameterentry{{\mantt{<}}}{U}{Logical,ref} True if the U Stokes parameter
                       is present
\manparameterentry{{\mantt{<}}}{V}{Logical,ref} True if the V Stokes parameter
                       is present
\manparameterentry{{\mantt{!}}}{{\mantt{STATUS}}}{Integer,ref} The Status

\end{manparametertable}
\manroutineitem{External subroutines / functions used}{}
     Various {\mantt{NDF}} routines
\manroutineitem{Support}{Jeremy Bailey, {\mantt{AAO}}}
\manroutineitem{Version date}{29/2/1988}
\end{manroutinedescription}
\manroutine{{\mantt{TSP\_{}TEMP}}}{Create a temporary array.}
\begin{manroutinedescription}
\manroutineitem{Function}{}
     Create a temporary array.

\manroutineitem{Description}{}
     Create a temporary array, map it, and return a pointer to it.

\manroutineitem{Language}{}
     {\mantt{FORTRAN}}

\manroutinebreakitem{Call}{}
     {\mantt{CALL}} {\mantt{TSP\_{}TEMP}} ({\mantt{SIZE}},{\mantt{TYPE}},{%
\mantt{PTR}},{\mantt{LOC}},{\mantt{STATUS}})

\manroutineitem{Parameters}{(``{\mantt{>}}'' input, ``{\mantt{!}}'' modified, `%
`W'' workspace, ``{\mantt{<}}'' output)}
\begin{manparametertable}
\manparameterentry{{\mantt{>}}}{{\mantt{SIZE}}}{Integer,ref} The size of the %
array to be created.
\manparameterentry{{\mantt{>}}}{{\mantt{TYPE}}}{Fixed string,descr} The type %
of the array -
                       one of the {\mantt{HDS}} primitive type names.
\manparameterentry{{\mantt{<}}}{{\mantt{PTR}}}{Integer,ref} Pointer to the %
array created.
\manparameterentry{{\mantt{<}}}{{\mantt{LOC}}}{Fixed string,descr} A locator %
to the array
                       structure (so that it can be unmapped).
\manparameterentry{{\mantt{!}}}{{\mantt{STATUS}}}{Integer,ref} The Status

\end{manparametertable}
\manroutineitem{External subroutines / functions used}{}
     Various {\mantt{NDF}} routines
\manroutineitem{Support}{Jeremy Bailey, {\mantt{AAO}}}
\manroutineitem{Version date}{27/2/1988}
\end{manroutinedescription}
\manroutine{{\mantt{TSP\_{}UNMAP}}}{Unmap a mapped data array}
\begin{manroutinedescription}
\manroutineitem{Function}{}
     Unmap a mapped data array

\manroutineitem{Description}{}
     Given a locator to a mapped object, unmap it, and annul the locator.

\manroutineitem{Language}{}
     {\mantt{FORTRAN}}

\manroutinebreakitem{Call}{}
     {\mantt{CALL}} {\mantt{TSP\_{}UNMAP}}({\mantt{LOC}},{\mantt{STATUS}})

\manroutineitem{Parameters}{(``{\mantt{>}}'' input, ``{\mantt{!}}'' modified, `%
`W'' workspace, ``{\mantt{<}}'' output)}
\begin{manparametertable}
\manparameterentry{{\mantt{>}}}{{\mantt{LOC}}}{Fixed string,descr} A locator %
to a mapped
                       object - returned by one of the {\mantt{TSP\_{}MAP}}... %
routines
\manparameterentry{{\mantt{!}}}{{\mantt{STATUS}}}{Integer,ref} The Status

\end{manparametertable}
\manroutineitem{External subroutines / functions used}{}
     Various {\mantt{NDF}} routines
\manroutineitem{Support}{Jeremy Bailey, {\mantt{AAO}}}
\manroutineitem{Version date}{27/2/1988}
\end{manroutinedescription}
\manroutine{{\mantt{TSP\_{}WLU}}}{Write the {\mantt{LABEL}} and {\mantt{UNITS}} %
of a structure}
\begin{manroutinedescription}
\manroutineitem{Function}{}
     Write the {\mantt{LABEL}} and {\mantt{UNITS}} of a structure

\manroutineitem{Description}{}
     Given a locator to an {\mantt{TSP}} structure, write values for the {%
\mantt{LABEL}} and {\mantt{UNITS}}
     of the data array.

\manroutineitem{Language}{}
     {\mantt{FORTRAN}}

\manroutinebreakitem{Call}{}
     {\mantt{CALL}} {\mantt{TSP\_{}WLU}} ({\mantt{LOC}},{\mantt{LABEL}},{%
\mantt{UNITS}},{\mantt{STATUS}})

\manroutineitem{Parameters}{(``{\mantt{>}}'' input, ``{\mantt{!}}'' modified, `%
`W'' workspace, ``{\mantt{<}}'' output)}
\begin{manparametertable}
\manparameterentry{{\mantt{>}}}{{\mantt{LOC}}}{Fixed string,descr} A locator %
to the {\mantt{NDF}}
                       structure.
\manparameterentry{{\mantt{<}}}{{\mantt{LABEL}}}{Fixed string,descr} Label %
string.
\manparameterentry{{\mantt{<}}}{{\mantt{UNITS}}}{Fixed string,descr} Units %
string.
\manparameterentry{{\mantt{!}}}{{\mantt{STATUS}}}{Integer,ref} The Status

\end{manparametertable}
\manroutineitem{External subroutines / functions used}{}
     Various {\mantt{NDF}} routines
\manroutineitem{Support}{Jeremy Bailey, {\mantt{AAO}}}
\manroutineitem{Version date}{27/2/1988}
\end{manroutinedescription}
\manroutine{{\mantt{TSP\_{}WLU\_{}LAMBDA}}}{Write the {\mantt{LABEL}} and {%
\mantt{UNITS}} of the wavelength axis}
\begin{manroutinedescription}
\manroutineitem{Function}{}
     Write the {\mantt{LABEL}} and {\mantt{UNITS}} of the wavelength axis

\manroutineitem{Description}{}
     Given a locator to an {\mantt{TSP}} structure, write values for the {%
\mantt{LABEL}} and {\mantt{UNITS}}
     of the wavelength axis.

\manroutineitem{Language}{}
     {\mantt{FORTRAN}}

\manroutinebreakitem{Call}{}
     {\mantt{CALL}} {\mantt{TSP\_{}WLU\_{}LAMBDA}} ({\mantt{LOC}},{\mantt{%
LABEL}},{\mantt{UNITS}},{\mantt{STATUS}})

\manroutineitem{Parameters}{(``{\mantt{>}}'' input, ``{\mantt{!}}'' modified, `%
`W'' workspace, ``{\mantt{<}}'' output)}
\begin{manparametertable}
\manparameterentry{{\mantt{>}}}{{\mantt{LOC}}}{Fixed string,descr} A locator %
to the {\mantt{NDF}}
                       structure.
\manparameterentry{{\mantt{<}}}{{\mantt{LABEL}}}{Fixed string,descr} Label %
string.
\manparameterentry{{\mantt{<}}}{{\mantt{UNITS}}}{Fixed string,descr} Units %
string.
\manparameterentry{{\mantt{!}}}{{\mantt{STATUS}}}{Integer,ref} The Status

\end{manparametertable}
\manroutineitem{External subroutines / functions used}{}
     {\mantt{TSP\_{}WLU}},
     Various {\mantt{NDF}} routines
\manroutineitem{Support}{Jeremy Bailey, {\mantt{AAO}}}
\manroutineitem{Version date}{27/2/1988}
\end{manroutinedescription}
\manroutine{{\mantt{TSP\_{}WLU\_{}TIME}}}{Write the {\mantt{LABEL}} and {%
\mantt{UNITS}} of the time axis}
\begin{manroutinedescription}
\manroutineitem{Function}{}
     Write the {\mantt{LABEL}} and {\mantt{UNITS}} of the time axis

\manroutineitem{Description}{}
     Given a locator to an {\mantt{TSP}} structure, write values for the {%
\mantt{LABEL}} and {\mantt{UNITS}}
     of the time axis.

\manroutineitem{Language}{}
     {\mantt{FORTRAN}}

\manroutinebreakitem{Call}{}
     {\mantt{CALL}} {\mantt{TSP\_{}WLU\_{}TIME}} ({\mantt{LOC}},{\mantt{LABEL}}%
,{\mantt{UNITS}},{\mantt{STATUS}})

\manroutineitem{Parameters}{(``{\mantt{>}}'' input, ``{\mantt{!}}'' modified, `%
`W'' workspace, ``{\mantt{<}}'' output)}
\begin{manparametertable}
\manparameterentry{{\mantt{>}}}{{\mantt{LOC}}}{Fixed string,descr} A locator %
to the {\mantt{NDF}}
                       structure.
\manparameterentry{{\mantt{<}}}{{\mantt{LABEL}}}{Fixed string,descr} Label %
string.
\manparameterentry{{\mantt{<}}}{{\mantt{UNITS}}}{Fixed string,descr} Units %
string.
\manparameterentry{{\mantt{!}}}{{\mantt{STATUS}}}{Integer,ref} The Status

\end{manparametertable}
\manroutineitem{External subroutines / functions used}{}
     {\mantt{TSP\_{}WLU}},
     Various {\mantt{NDF}} routines
\manroutineitem{Support}{Jeremy Bailey, {\mantt{AAO}}}
\manroutineitem{Version date}{27/2/1988}
\end{manroutinedescription}
\manroutine{{\mantt{TSP\_{}WLU\_{}X}}}{Write the {\mantt{LABEL}} and {\mantt{%
UNITS}} of the X axis}
\begin{manroutinedescription}
\manroutineitem{Function}{}
     Write the {\mantt{LABEL}} and {\mantt{UNITS}} of the X axis

\manroutineitem{Description}{}
     Given a locator to an {\mantt{TSP}} structure, write values for the {%
\mantt{LABEL}} and {\mantt{UNITS}}
     of the X axis.

\manroutineitem{Language}{}
     {\mantt{FORTRAN}}

\manroutinebreakitem{Call}{}
     {\mantt{CALL}} {\mantt{TSP\_{}WLU\_{}Y}} ({\mantt{LOC}},{\mantt{LABEL}},{%
\mantt{UNITS}},{\mantt{STATUS}})

\manroutineitem{Parameters}{(``{\mantt{>}}'' input, ``{\mantt{!}}'' modified, `%
`W'' workspace, ``{\mantt{<}}'' output)}
\begin{manparametertable}
\manparameterentry{{\mantt{>}}}{{\mantt{LOC}}}{Fixed string,descr} A locator %
to the {\mantt{NDF}}
                       structure.
\manparameterentry{{\mantt{<}}}{{\mantt{LABEL}}}{Fixed string,descr} Label %
string.
\manparameterentry{{\mantt{<}}}{{\mantt{UNITS}}}{Fixed string,descr} Units %
string.
\manparameterentry{{\mantt{!}}}{{\mantt{STATUS}}}{Integer,ref} The Status

\end{manparametertable}
\manroutineitem{External subroutines / functions used}{}
     {\mantt{TSP\_{}WLU}},
     Various {\mantt{NDF}} routines
\manroutineitem{Support}{Jeremy Bailey, {\mantt{AAO}}}
\manroutineitem{Version date}{19/10/1989}
\end{manroutinedescription}
\manroutine{{\mantt{TSP\_{}WLU\_{}Y}}}{Write the {\mantt{LABEL}} and {\mantt{%
UNITS}} of the Y axis}
\begin{manroutinedescription}
\manroutineitem{Function}{}
     Write the {\mantt{LABEL}} and {\mantt{UNITS}} of the Y axis

\manroutineitem{Description}{}
     Given a locator to an {\mantt{TSP}} structure, write values for the {%
\mantt{LABEL}} and {\mantt{UNITS}}
     of the Y axis.

\manroutineitem{Language}{}
     {\mantt{FORTRAN}}

\manroutinebreakitem{Call}{}
     {\mantt{CALL}} {\mantt{TSP\_{}WLU\_{}Y}} ({\mantt{LOC}},{\mantt{LABEL}},{%
\mantt{UNITS}},{\mantt{STATUS}})

\manroutineitem{Parameters}{(``{\mantt{>}}'' input, ``{\mantt{!}}'' modified, `%
`W'' workspace, ``{\mantt{<}}'' output)}
\begin{manparametertable}
\manparameterentry{{\mantt{>}}}{{\mantt{LOC}}}{Fixed string,descr} A locator %
to the {\mantt{NDF}}
                       structure.
\manparameterentry{{\mantt{<}}}{{\mantt{LABEL}}}{Fixed string,descr} Label %
string.
\manparameterentry{{\mantt{<}}}{{\mantt{UNITS}}}{Fixed string,descr} Units %
string.
\manparameterentry{{\mantt{!}}}{{\mantt{STATUS}}}{Integer,ref} The Status

\end{manparametertable}
\manroutineitem{External subroutines / functions used}{}
     {\mantt{TSP\_{}WLU}},
     Various {\mantt{NDF}} routines
\manroutineitem{Support}{Jeremy Bailey, {\mantt{AAO}}}
\manroutineitem{Version date}{19/10/1989}
\end{manroutinedescription}

\end{small}

\end{document}


