\documentstyle{article} 
\pagestyle{myheadings}

%------------------------------------------------------------------------------
\newcommand{\stardoccategory}  {Starlink User Note}
\newcommand{\stardocinitials}  {SUN}
\newcommand{\stardocnumber}    {53.2}
\newcommand{\stardocauthors}   {William Lupton}
\newcommand{\stardocdate}      {22nd December 1989}
\newcommand{\stardoctitle}     {ECHWIND --- UCL Echelle Spectrograph \\
				Observation Planning Program}
%------------------------------------------------------------------------------

\newcommand{\stardocname}{\stardocinitials /\stardocnumber}
\markright{\stardocname}
\setlength{\textwidth}{160mm}
\setlength{\textheight}{240mm}
\setlength{\topmargin}{-5mm}
\setlength{\oddsidemargin}{0mm}
\setlength{\evensidemargin}{0mm}
\setlength{\parindent}{0mm}
\setlength{\parskip}{\medskipamount}
\setlength{\unitlength}{1mm}

%------------------------------------------------------------------------------
% Add any \newcommand or \newenvironment commands here
%------------------------------------------------------------------------------

\newcommand{\coude}{coud\'{e}}

\begin{document}
\thispagestyle{empty}
SCIENCE \& ENGINEERING RESEARCH COUNCIL \hfill \stardocname\\
RUTHERFORD APPLETON LABORATORY\\
{\large\bf Starlink Project\\}
{\large\bf \stardoccategory\ \stardocnumber}
\begin{flushright}
\stardocauthors\\
\stardocdate
\end{flushright}
\vspace{-4mm}
\rule{\textwidth}{0.5mm}
\vspace{5mm}
\begin{center}
{\Large\bf \stardoctitle}
\end{center}
\vspace{5mm}

%------------------------------------------------------------------------------
\setlength{\parskip}{0mm}
\tableofcontents
\setlength{\parskip}{\medskipamount}
\markright{\stardocname}
%------------------------------------------------------------------------------

\begin{quote}
{\small\em The ECHWIND program which is described in this document was formerly
known as MANUAL. MANUAL is completely superceded by ECHWIND, which in addition
offers many new facilities.}
\end{quote}

\section{Introduction}

ECHWIND is an ADAM A-task which was originally written to assist in the
planning of observations with the UCL \coude\ echelle spectrograph (UCLES). It
can be run at your home institution, when preparing an observing proposal, or
before or during an observing run. Much of the dependence on UCLES has now been
removed and it should be a useful tool for use with other similar echelle
spectrographs of similar design. For further details on using it with
spectrographs other than UCLES, see the description of the spectrograph model
package in section~\ref{The Spectrograph Model}.   

The program allows the user to view interactively that part of the spectrum
which will fall on the detector for a given central wavelength. This is
achieved by drawing a complete echellogram on the ARGS or IKON along with a box
representing the exact size and shape of the detector. The positions of
individual spectral lines can also be marked on the display. The user can then
move the detector window around using the trackerball / mouse (henceforth
referred to as the mouse) or else by explicitly specifying the desired central
wavelength. At each position the screen displays the minimum, central and
maximum wavelengths and order numbers falling on the detector as well as an
estimate of the length of the detector in Angstroms.  The position of the box
can be marked on the screen so as to ease preparation of observations where
multiple overlapping wavelength ranges are required, and facilities are
available for saving the current detector position in a text file or in a form
suitable for later use by either the AAT or WHT echelle spectrograph data
acquisition systems. For further details, see the detailed description of how
to run the program in section~\ref{Detailed Description}.

\section{Getting Going}

Before running the program, you must type @SSC:LOGIN to define Starlink
commands (many people do this in their LOGIN.COM files) and ECHWINDSTART to
define the ECHWIND DCL command. To run the program, type ECHWIND at the DCL
prompt.

If you don't explicitly specify an image display device, ECHWIND will use
the same one that was used the last time that it was run. It will only prompt
for it if this is the first time that the program has been run from this
username. If you wish to change the image display device you must use the DEV
keyword, as in
\begin{quote}
\$ ECHWIND DEV=IKON2
\end{quote}

IKON will always use the default IKON and ARGS will always use the default
ARGS. If your site has more than one of a particular type of image display, use
a name of the form IKONn or ARGSn.

You can specify the echelle and detector to use as the first and second
positional parameters respectively, or you can let the program prompt you for
them. The echellogram is now drawn on the image display and you are free to
move the detector using the mouse. As you move it, the minimum, central and
maximum wavelengths and order numbers are re-calculated and displayed.

Various menu options are available for zooming, saving configurations, marking
detector boxes, positioning to specified wavelengths and so on. To obtain the
menu, press any mouse button other than the left-most one. Now use the mouse to
highlight your choice and then press any mouse button. The menu will disappear
and whatever is necessary will happen --- this may involve having to respond to
a prompt.

Several keywords can be specified on the command line, but in nearly all cases
the same effect can be achieved via one of the menu options. Nevertheless, it
can be useful to set initial central wavelengths, zoom factors, line lists etc.
The following table lists the commonly used keywords.

\begin{tabular}{lp{115mm}}
KEYWORD=value	& Description \\ \hline
ANGWAVE=angwave	& Specify slit rotation angle or wavelength at which orders are
		to be aligned with the detector \\
DEV=device	& Specify image display device \\
FORM=format	& Specify format of saved configurations \\
LINES=file	& Specify file containing spectral lines to be displayed on the 
		echellogram \\
WAVE=wavelength	& Specify central wavelength \\
ZOOM=factor	& Specify zoom factor
\end{tabular}

To exit from the program, select the EXIT menu option.

For full details of all parameters, keywords and menu options, see the detailed
description of how to run the program in section~\ref{Detailed Description}.

\section{Detailed Description}
\label{Detailed Description}

This section contains a detailed description of all of the facilities offered
by ECHWIND. You can use command-line keywords to control things such as which
image display device to use and to specify the initial echelle, detector and
central wavelength. Subsequently you can use the mouse and keyboard to alter
most of these parameters.

\subsection{Selecting the Image Display}

ECHWIND uses an early version of IDI (Image Display Interface) which supports
both the ARGS and the IKON. Except for the first time that you ever run the
program from a given username, when you will be prompted, you can only alter
the image display device by using the DEV keyword to specify it on the command
line. The same device will continue to be used until again overridden in the
same manner.
\begin{quote}
\$ ECHWIND DEV=ARGS1
\end{quote}

Specifying ARGS or IKON will use logical names ARGS\_DEFAULT or IKON\_DEFAULT.
If your site has more than one of a given type of image display, ARGSn or IKONn
will use logical names IDI\_ARGS\_n or IDI\_IKON\_n. If you specify something
illegal then logical name IDI\_DEFAULT will be used. Your system manager will
have ensured that all of these logical names have sensible values.

\subsection{Selecting the Echelle Grating}

You can specify the name of the echelle grating as the first positional
parameter or by using the ECH keyword. If you don't do so, you will be prompted
for it.
\begin{quote}
\$ ECHWIND 31 \\
\$ ECHWIND ECH=79
\end{quote}

Although ``31'' and ``79'' are supported as always, there is actually support
for names of the form ``instrument/echelle/camera'' and ``31'' really means
``UCLES/31/LONG''. A spectrograph parameter file is read on startup and this
defines the available and default instruments, echelles and cameras (so in fact
``31'' means ``UCLES/31/LONG'' only if the default instrument is UCLES and
the default UCLES camera is LONG). It is easy to support spectrographs such as
the Utrecht Echelle Spectrograph (UES) which have the same basic design as
UCLES but which have, for example, different collimator or camera focal
lengths, or different echelle gratings. For further details, see the
description of the spectrograph model package in section~\ref{The Spectrograph
Model}.

You can change the echelle grating from within the program by selecting the
``New ECHELLE'' menu option. This simply switches between the 31 lines/mm and
the 79 lines/mm gratings and will always revert to the default instrument and
camera.

\subsection{Selecting the Detector}

You can specify the detector name as the second positional parameter or by
using the DET keyword. If you don't do so, you will be prompted for it.
\begin{quote}
\$ ECHWIND 31 THOMSON \\
\$ ECHWIND DET=THOMSON
\end{quote}

The only information pertaining to the detector that is actually needed is its
size and for unsupported detectors you can specify the size in mm in the form
``x,y'' (the quotes are needed if specifying it on the command line) rather
than giving the name.

The prompt for the detector name includes the list of supported detectors
(provided that it is not too long). Your system manager can set up a
system-wide list and in addition you can use the DETLIST keyword to specify
your own private list, as in
\begin{quote}
\$ ECHWIND DETLIST=MYLIST DET=MYDET
\end{quote}
The list is read from a text file with a default file extension of .DAT ---
MYLIST.DAT in the above example. The file format is very simple: each line
consists of the detector name in single quotes (like a Fortran CHARACTER
constant) followed by the detector X and Y sizes in mm. The items are separated
with spaces, tabs or commas. For example, this is the default system-wide
detector list.
\begin{verbatim}
    'IPCS'          33.9,18.6
    'THOMSON'       19.0,19.0
    'RCA'           15.4,9.6
    'GEC'           12.7,8.5
\end{verbatim}
If a detector name appears both in the system-wide list and in your
user-supplied list then your definition of it is used rather than the
system-wide one.

You can change the detector from within the program by selecting the ``New
DETECTOR'' menu option. You will then be prompted for a new detector name.

\subsection{Selecting a Line List}

Often you will be interested in specific spectral lines or wavelength ranges.
You can prepare a text file containing a list of wavelengths and text strings
with which to label those wavelengths and can use the LINES keyword on the
command line to give it to the program. The lines will be plotted in the
correct positions on the echellogram, in white where they appear within the
free spectral range and in grey where they do not. Each line has the specified
text string plotted near to it. By default this file has an extension of .DAT
so
\begin{quote}
\$ ECHWIND LINES=MYLINES
\end{quote}
will read the file MYLINES.DAT. The file format is very simple: each line
consists of the wavelength in Angstroms followed by the associated text in
single quotes (like a Fortran CHARACTER constant), as in
\begin{verbatim}
    6562.80 'A'
    4861.32 'B'
    4340.46 'C'
    4101.73 'D'
\end{verbatim}

You can select a new line list from within the program by selecting the ``New
LINES'' menu option. This prompts you for a new line list.

\subsection{Selecting a Central Wavelength}

By default the detector will initially be placed at the origin of the
dispersion and cross dispersion axes (about 4120 Angstroms for UCLES). If you
wish to override this you can do so using the WAVE keyword, as in
\begin{quote}
\$ ECHWIND WAVE=8000
\end{quote}
which will initially place the centre of the detector at 8000 Angstroms.

You can select a new central wavelength from within the program by selecting
the ``New WAVELENGTH'' menu option. This prompts you for a new central
wavelength.

\subsection{Selecting a Zoom Factor}

By default the echellogram will initially be displayed with a zoom factor of 1
(which displays an area of $100 \times 100$ mm). If you wish to override this
you can do so using the ZOOM keyword, as in
\begin{quote}
\$ ECHWIND ZOOM=3
\end{quote}

You can zoom, unzoom and pan from within the program. Select one of the ``x
1.3'', ``x 2.0'' and ``x 3.0'' menu options to zoom in; select one of the ``x
0.77'', ``x 0.5'' and ``x 0.3'' menu options to zoom out. The detector is
always placed at the centre of the display after zooming. Thus the ``CENTRE''
menu option is simply a pan operation (it does what ``x 1.0'' would do).

\subsection{Selecting a Camera Rotation Angle}

It is common observing practice with UCLES (and presumably will also be with
UES) to align the detector with the orders. This has the dual advantages that
it makes it easy to extract orders without having to track them, and that you
may get an extra complete order to fit on to the detector. You can use the
ANGWAVE keyword to specify the wavelength in Angstroms at which you plan to
align the orders with the detector, and the display of the orders will be
rotated accordingly.
\begin{quote}
\$ ECHWIND ANGWAVE=6000
\end{quote}

Alternatively, if you specify a value that is less than or equal to 1000, it is
assumed that you are providing a value in degrees for the slit angle offset.
This offset is automatically added to the slit angle by the UCLES (and
presumably the UES) instrument control software and compensates for the
detector rotation so as to give vertical slit images at the centre of the
detector. It is calculated by the UCLES setup routines and, once calculated, is
normally left unaltered for the entire observing run. It only makes sense to
supply negative values for this offset since positive values will always rotate
the slit image in the wrong direction. Provided that a sensible value is given,
the corresponding wavelength at which the orders will be aligned with the
detector is reported. Conversely, when a wavelength is given, the corresponding
slit angle offset as reported. {\em Note that there is currently a problem here
and the reported slit angle offsets do not correspond to actual behaviour.
Consequently, until the problem is resolved you should specify only
wavelengths for the ANGWAVE parameter.}

You can select a new camera rotation angle from within the program by selecting
the ``New ANGLE'' menu option. This prompts you for a new rotation
angle or corresponding wavelength.

\subsection{Saving Configurations}

ECHWIND supports the saving of spectrograph configurations in files for later
reference or use by the spectrograph control software. It supports three
different file formats but a given run of the program can only write a file of
a single format. The three formats are

\begin{tabular}{ll}
AAT	& AAT UCLES control file \\
WHT	& WHT UES default list \\
TEXT	& ASCII text file
\end{tabular}

and you control the file format by using the FORM keyword, as in
\begin{quote}
\$ ECHWIND FORM=AAT
\end{quote}

The FORM keyword behaves like the DEV keyword in that you are only ever
prompted for it the first time that you use the program from a given username.
On subsequent runs, unless explicitly overridden, it will retain its previous
value. Of course, AAT users will usually use FORM=AAT and WHT users will
usually use FORM=WHT.

To save the current spectrograph configuration (or detector position, however
you like to think of it), select the ``SAVE'' menu option. This will mark the
current position of the detector in green (like ``Mark BOX'') and will issue an
appropriate prompt for the file name. AAT control files (which are HDS files
with a default file extension of .SDF) and WHT default lists (which are ASCII
files with a default file extension of .DFL) are created for each SAVE
operation. Text files (which are ASCII files with a default file extension of
.TXT) are created only for the first SAVE operation and are appended to for
subsequent operations.

Enough information is written to allow the spectrograph control software to
set the spectrograph hardware into the saved configuration. The AAT control
file is an HDS file and using TRACE on it gives output like
\begin{verbatim}
    CONFIG          INSTRUMENT              <structure>
       _CHAR*80        ECHELLE                 '79'
       _CHAR*80        ECH_ENABLE              '79'
       _REAL           ECH_THETA               1.391196
       _REAL           ECH_GAMMA               -1.163817
       _REAL           SLIT_ANGLE              -7.185203
       _REAL           PRISM_POS               61.30878
       _CHAR*80        COLLIMATOR              'WIDE'
    CONFIG          MISC                    <structure>
       _REAL           WAVE_CEN                5999.962
       _INTEGER        ORDER_CEN               38
\end{verbatim}

The WHT default lists are of the form
\begin{verbatim}
    default list : test
    hardware list : ues

    ues.echelle       79
    ues.central_wave  5999.962
    ues.central_order 38
    ues.disp_offset   1.5605927E-02
    ues.xdisp_offset  3.0708313E-04
\end{verbatim}
where {\tt disp\_offset} and {\tt xdisp\_offset} are the offsets in mm
necessary to get from the centre of the order nearest to the detector centre to
the detector centre.

The text file is of the form
\begin{verbatim}
    echelle  wavelength  order  dispoff  xdispoff
    -------  ----------  -----  -------  --------

        79     5999.96     38     0.02      0.00
\end{verbatim}

\subsection{Marking Detector Positions}

You can select the ``Mark BOX'' menu option to mark the current position of the
detector in green (``SAVE'' does an implicit ``Mark BOX'' as well). Selecting
``Delete BOX'' will delete the most recently marked box. Selecting ``Clear
BOXES'' will delete all boxes.

\subsection{Doing Nothing}

If you have selected the menu but then decide that you made a mistake in so
doing, you can move the mouse so that no option is highlit and then press a
mouse button to return without taking any action.

\subsection{Suppressing Image Display Reset}

By default the image display is reset each time that the program starts up.
Occasionally this causes problems with the IKON. If this happens, try using the
NOERASE keyword, as in
\begin{quote}
\$ ECHWIND NOERASE
\end{quote}

\subsection{Exiting}

To exit from the program, select the ``EXIT'' menu option.

\section{The Spectrograph Model}
\label{The Spectrograph Model}

ECHWIND uses the ECH package of routines to predict the behaviour of the
spectrograph. Use of this package, most of which dates back to the period at
which UCLES was being designed, is documented in SSN/3. It is easy to use
different values for some of the spectrograph parameters, thus allowing
spectrographs other than UCLES to be modelled. Such spectrographs must be
similar in design to UCLES but can differ in
\begin{itemize}
\item collimator focal length (which is in fact irrelevant to the model package)
\item cross dispersing prism apex angle
\item echelle ruling frequency
\item echelle blaze angle
\item camera focal length
\end{itemize}

It is not capable of dealing with spectrographs which differ in any more major
ways from UCLES, although it could be extended to do so should the need arise.

At startup, ECHWIND reads a spectrograph parameter file which defines the list
of available spectrographs. By default this file is accessed using the logical
name ECHWIND\_SPECTROGRAPHS, but this can be overridden using the SPECLIST
keyword, as in
\begin{quote}
\$ ECHWIND SPECLIST=MYSPECS
\end{quote}

The default version of this file is
\begin{verbatim}
    !+ SPECTROGRAPHS.DAT
    !
    !  Function:
    !    Define parameters for spectrographs whose behaviour can be
    !    modelled using the UCLES model library.
    !
    !  History:
    !    22-Dec-89 - WFL - Original version

    INSTRUMENT UCLES
       FCOL 6000.0
       ANGLE 54.1
       ECHELLE 31
          D 31.6046
          M0 138
          WAVE0 4119.68
          THETAB 64.6
       ECHELLE 79
          D 79.0115
          M0 55
          WAVE0 4097.99
          THETAB 63.55
       CAMERA LONG
          FCAM 700.0
       CAMERA SHORT
          FCAM 450.0

    INSTRUMENT UES
       FCOL 1800.0
       ANGLE 54.1
       ECHELLE 31
          D 31.6046
          M0 138
          WAVE0 4119.68
          THETAB 64.6
       CAMERA LONG
          FCAM 700.0
\end{verbatim}
\begin{itemize}
\item Everything from a ``!'' onwards in a line is ignored (so lines beginning
with ``!'' are comment lines).
\item The file is regarded as being a set of ``keyword value'' pairs.
\item INSTRUMENT declares and names a new current instrument. The following
items are assumed to pertain to the current instrument. \\
\begin{tabular}{ll}
FCOL	& Collimator focal length (mm) \\
ANGLE	& Prism apex angle (degrees) \\
ECHELLE	& Echelle grating \\
CAMERA	& Camera
\end{tabular}
\item ECHELLE declares and names a new current echelle. This echelle is
associated with the current instrument. The following items are assumed to
pertain to the current echelle. \\
\begin{tabular}{ll}
D	& Number of lines/mm \\
M0	& Order number closest to field centre \\
WAVE0	& Central wavelength of this order (Angstroms) \\
THETAB	& Blaze angle (degrees)
\end{tabular}
\item CAMERA declares and names a new current camera. This camera is associated
with the current instrument. The following item is assumed to pertain to the
current camera. \\
\begin{tabular}{ll}
FCAM	& Camera focal length (mm)
\end{tabular}
\end{itemize}

Any omitted values will default to the corresponding values for the previously
defined instrument, echelle or camera, or, if there is no such value available,
to the corresponding values for UCLES.

ECHWIND allows use of the syntax ``instrument/echelle/camera'' in response to
the prompt for the echelle name. The user response is split up into tokens and
then the first token is checked against the list of defined instruments. If it
matches one of them, attention switches to the next token. If it doesn't, the
first instrument defined in the spectrograph parameter file is assumed (UCLES
in the above example) and the same token is regarded as a candidate echelle
name. The list of defined echelles for the current instrument is searched in
exactly the same way, as is the list of defined cameras. Attention only moves
to the next token if a match is found to the current token. If, after all this,
any unmatched tokens remain, an error is reported and the first instrument / 
echelle / camera combination defined in the spectrograph parameter file is
assumed (UCLES/31/LONG in the above example). To use the UES, which has only a
single defined echelle and camera, it would be sufficient to respond ``UES''.

\section{Installation}

This section is only relevant to system managers and includes a checklist for
installation of ECHWIND. Note that no abnormal quotas should be required in
order to run ECHWIND.

Two classes of release are available: the ``source only'' release occupies
about 1100 disc blocks and the ``source plus executable'' release occupies
about 1600 disc blocks. If you have the source only release, you must build the
object libraries and executable image by doing
\begin{verbatim}
    $ @SSC:LOGIN
    $ ADAMSTART
    $ ADAMDEV
    $ @ECHWIND_COMPILE
    $ @ECHWIND_LINK
\end{verbatim}
from within the directory into which you have put the release. There may also
be other circumstances where it is convenient to be able to re-build the
executable image from its source files.

The following is a checklist of things that must be done in order to install
ECHWIND.
\begin{itemize}
\item Place all supplied ECHWIND files in a directory somewhere.
\item If you have the source only release, build the executable image as
described above.
\item Edit LSTARDISK:[STARLOCAL]LOGIN.COM so that the symbol ECHWINDSTART will
execute ECHWIND.COM from that directory.
\item Edit ECHWIND.COM to define the appropriate image display device names for
your site.
\item Edit DETECTORS.DAT to define the appropriate detector names for your
site.
\item Edit SPECTROGRAPHS.DAT to define the appropriate spectrograph parameters
for your site.
\end{itemize}

\end{document}
