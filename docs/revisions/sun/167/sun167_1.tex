\documentstyle[11pt]{article} 
\pagestyle{myheadings}

%------------------------------------------------------------------------------
\newcommand{\stardoccategory}  {Starlink User Note}
\newcommand{\stardocinitials}  {SUN}
\newcommand{\stardocnumber}    {167.1}
\newcommand{\stardocauthors}   {V S Dhillon\\
                                Royal Greenwich Observatory\\
                                La Palma}
\newcommand{\stardocdate}      {18 May 1993}
\newcommand{\stardoctitle}     {PERIOD --- A Time-Series Analysis Package}
%------------------------------------------------------------------------------

\newcommand{\stardocname}{\stardocinitials /\stardocnumber}
\renewcommand{\_}{{\tt\char'137}}     % re-centres the underscore
\markright{\stardocname}
\setlength{\textwidth}{160mm}
\setlength{\textheight}{230mm}
\setlength{\topmargin}{-2mm}
\setlength{\oddsidemargin}{0mm}
\setlength{\evensidemargin}{0mm}
\setlength{\parindent}{0mm}
\setlength{\parskip}{\medskipamount}
\setlength{\unitlength}{1mm}

%------------------------------------------------------------------------------
% Add any \newcommand or \newenvironment commands here

\newcommand{\rf}{\par\noindent\hangindent 15pt{}}

\newenvironment{cozy}[1]%
{\begin{list}{}{%
\settowidth{\labelwidth}{\large\bf #1}%
\setlength{\labelsep}{5mm}%
\setlength{\leftmargin}{\labelwidth}\addtolength{\leftmargin}{\labelsep}%
\setlength{\parsep}{\medskipamount}%
}}{\end{list}}
%------------------------------------------------------------------------------

\begin{document}
\thispagestyle{empty}
SCIENCE \& ENGINEERING RESEARCH COUNCIL \hfill \stardocname\\
RUTHERFORD APPLETON LABORATORY\\
{\large\bf Starlink Project\\}
{\large\bf \stardoccategory\ \stardocnumber}
\begin{flushright}
\stardocauthors\\
\stardocdate
\end{flushright}
\vspace{-4mm}
\rule{\textwidth}{0.5mm}
\vspace{5mm}
\begin{center}
{\Large\bf \stardoctitle}
\end{center}
\vspace{5mm}

%------------------------------------------------------------------------------
%  Add this part if you want a table of contents
%  \setlength{\parskip}{0mm}
%  \tableofcontents
%  \setlength{\parskip}{\medskipamount}
%  \markright{\stardocname}
%------------------------------------------------------------------------------

\section{Introduction}

This document describes how to use {\tt PERIOD} (version 3.0), a software
package designed to search for periodicities in data. It is important to note
that {\tt PERIOD} will {\em not} run if {\tt QDP/PLT}, the plotting/fitting
package written by Allyn Tennant and described in SUN 128, is not properly
installed on the system. 


\section{Initializing and Running}

To allow for a substantial amount of data to be processed at one time, {\tt
PERIOD} uses a number of large data arrays. Therefore, a page file quota of at
least 27000 (approx.) is required to run  {\tt PERIOD}.

To run {\tt PERIOD}, the package must first be initialized using the command
\verb@PERIODSTART@. This may be placed in the users
\verb@SYS$LOGIN:LOGIN.COM@ file if desired. The package can then be run 
by typing {\tt PERIOD}.

\section{Data Format and Storage}
\label{slots}

{\tt PERIOD} only accepts ASCII data files as input. The files may contain
up to a maximum of 10000 rows and 20 columns. It is possible to specify which
column refers to the $x$-axis, which column refers to the $y$-axis and which 
column refers to the $y$-axis errors (optional). 

Data is stored and processed within {\tt PERIOD} using a {\em slot} system.
A data slot is simply an array holding one dataset. The maximum number
of data slots which can be handled at any one time in {\tt PERIOD} is 40;
hence {\tt PERIOD} has the capability of analysing a large amount of data
simultaneously. 

The first command that is usually run in {\tt PERIOD} is {\tt INPUT}
(see section~\ref{menu}), which loads datasets into data slots:

\begin{verbatim}
Enter first and last slots for input (0,0 to quit) : 
\end{verbatim}

In order to load the first slot with a single dataset, one should reply 
{\tt 1,1} to the above prompt. Similarly, if one wishes to load slots
4 through to 9 with 6 datasets, one should reply {\tt 4,9} to the above
prompt. It is important to note that slots {\em can} be overwritten.
Typing {\tt 0,0} will return the user to the menu. 

Most {\tt PERIOD} commands prompt not only for an input slot, but also for
an output slot:

\begin{verbatim}
Enter first and last slots for input  (0,0 to quit) : 
Enter first and last slots for output (0,0 to quit) : 
\end{verbatim}

The input should contain the dataset (or datasets) and the output should
contain the result of the operation on the dataset (or datasets). For example,
given a set of 5 time-series (which have previously been loaded into slots 1 to
5 using {\tt INPUT}) which need to be fitted with a sine curve (using the
command {\tt FIT}, see section~\ref{menu}), one should type {\tt 1,5} in reply
to the first prompt and {\tt 6,10} in reply to the second. {\tt PERIOD} will
then fit sine curves to the data files loaded in slots 1 to 5 and put the fits
in slots 6 to 10. Clearly, the number of output slots must be equal to the
number of input slots. In addition, if any selected input slot is empty, {\tt
PERIOD} will abort the operation and return the user to the main menu. It should
be noted that in order to save on storage space one could have typed {\tt 1,5} 
in reply to the second prompt and the original data-files would have been 
overwritten by the resulting sine curves.

\section{Using this Document}

There are two main sections in this document. Section~\ref{menu} should be
consulted by experienced users who require detailed information on individual
{\tt PERIOD} commands. Section~\ref{recipe} is intended for inexperienced 
users who require an introductory recipe detailing the steps one should 
follow when searching for periodicities in data. 

Please note that {\tt PERIOD} also has an extensive on-line help facility.
This provides information at a level of detail similar to that found in 
Section~\ref{menu}. However, it also provides a detailed description of 
the individual prompts, something which cannot be found in this document.

\section{Menu Options}
\label{menu}

{\tt PERIOD} is a menu-driven package. On entering {\tt PERIOD}, one is
confronted with the following menu options, which are described in greater
detail below.

\begin{verbatim}
          
|**************************************************|
|**| PERIOD  :>  A time-series analysis package |**|
|**| Version :>  3.0                            |**|
|**| Date    :>  17-March-1993                  |**|
|**************************************************|
|**|
|**| Acknowledgments would be greatly appreciated.
|**| Please report bugs, problems or suggestions to:
|**|
|**| Author  :>  Vikram Singh Dhillon
|**| Address :>  Royal Greenwich Observatory, La Palma
|**| E-mail  :>  (SPAN) 29146::VSD
        
 
Options.
--------

INPUT    --  Input data.
FAKE     --  Create fake data.
NOISE    --  Add noise to data.
DETREND  --  Detrend the data.
WINDOW   --  Set data points to unity.
OPEN     --  Open a log file.
CLOSE    --  Close the log file.
PERIOD   --  Find periodicities.
FIT      --  Fit sine curve to folded data.
FOLD     --  Fold data on given period.
SINE     --  +, -, / or * sine curves.
PLT      --  Call PLT.
STATUS   --  Information on stored data.
OUTPUT   --  Output data.
HELP     --  On-line help.
QUIT     --  Quit PERIOD.

PERIOD> 

\end{verbatim}

Any one of these commands can be entered by typing anything from the shortest
unambiguous string up to the full command name. Therefore, {\tt P} would be
ambiguous, but {\tt PE} would not.

\subsection*{\tt INPUT}

As described in section~\ref{slots}, this option allows the user to input data
into {\tt PERIOD}. The routine determines the number of columns in the input
files and then prompts the user for which columns refer to the $x$-axis,
$y$-axis and $y$-axis errors (if desired). For example, if the user is
inputting radial velocity data, the $x$-axis would most probably be HJD's, the
$y$-axis the heliocentric radial velocities and there would most likely be
errors associated with each radial velocity value. 

\subsection*{\tt FAKE}

Allows the user to create fake data with which to test or experiment with {\tt
PERIOD}. Two options are catered for: periodic data or chaotic data. The
periodic data are calculated using sine functions with parameters specified by
the user. The chaotic data are created using a simple logistic equation (see,
for example, Scargle 1990a, 1990b) with parameters specified by the user. 

\subsection*{\tt NOISE}

Using this option, it is possible to add noise to data or randomize data. The
latter operation is carried out by specifying the {\tt [N]}ew dataset option,
which will construct an artificial dataset of the same mean value and the same
standard deviation as the original. Selecting the {\tt [O]}ld dataset allows
the user to apply noise to data, create errors on the data points, and add
noise to the data sampling (so that, for instance, an evenly sampled dataset
becomes unevenly sampled). With most of the noise options, there is the
possibility of applying either Gaussian noise or Poissonian noise. This routine
is useful, not only in creating realistic artificial datasets (in conjunction
with {\tt FAKE}), but also in investigating the effects of noise on a period
detection. 

\subsection*{\tt DETREND}

This option removes the D.C. bias from data, which if not removed gives rise to
significant power at 0 Hz. There are two options: If the data show no long term
trends, it is best to simply subtract the mean and divide by the standard
deviation (the {\tt [M]} option). This gives a dataset with a mean of zero and
a standard deviation of one. Otherwise, it is best to subtract a low-order
polynomial fit to the data (the {\tt [P]} option), since if these are not
removed, the Fourier transform will put in a lot of power at the frequency of
the long term variations. 

\subsection*{\tt WINDOW}

One of the main problems with the {\em classical}
periodogram\footnote{Throughout the {\tt PERIOD} package and this document, the
terms {\em power spectrum} and {\em periodogram} are used interchangeably,
although strictly speaking the power spectrum is a theoretical quantity defined
as an integral over continuous time, of which the periodogram is merely an
estimate based on a finite amount of discrete data (Scargle 1982).}  (see
Scargle 1982 for a definition), is {\em spectral leakage}, of which there are
several forms. Leakage to nearby frequencies (sidelobes) is due to the finite
total interval over which the data is sampled. Leakage to distant frequencies
is due to the finite size of the interval between samples. The {\tt WINDOW}
option sets all the $y$-axis data points to unity. A discrete Fourier transform
of the resulting data would yield the window function (or spectrum), which
would show the effects of spectral leakage. 

\subsection*{\tt OPEN}

It is possible to store the fits calculated by {\tt SINE} and {\tt PEAKS} in a
log file. This option opens a new log file (if it does not already exist), or
else re-opens an old log file and skips over the existing entries. 

\subsection*{\tt CLOSE}

This option closes the currently open log file. 

\subsection*{\tt PERIOD}

This is where all the work is done. One is confronted with the following 
sub-menu: 

\begin{verbatim}


Options.
--------

SELECT   --  Select data slots.
FREQ     --  Set/show frequency search limits.
CHAOS    --  Chaos theory return map (Yn+i vs Yn).
CHISQ    --  Chi-squared of sine fit vs frequency.
CLEAN    --  CLEANed power spectrum.
FT       --  Discrete Fourier power spectrum.
MEM      --  Maximum entropy method power spectrum.
PDM      --  Phase dispersion minimization.
SCARGLE  --  Lomb-Scargle normalised periodogram.
STRING   --  String-length vs frequency.
PEAKS    --  Calculate period from periodogram.
HELP     --  On-line help.
QUIT     --  Quit PERIOD_PERIOD.

PERIOD_PERIOD> 

\end{verbatim}

\begin{itemize}

\item {\tt SELECT} -- Selects input and output slots for processing, as
described in section~\ref{slots}. The input slots should contain the
time-series, the output slots will contain, for example, the power spectra.
{\tt SELECT} must be run {\em every} time a periodicity-finding option is run. 

\item {\tt FREQ} -- Sets the frequency search parameters. The user can specify
the minimum frequency, maximum frequency and frequency interval. Note that the
maximum number of frequencies is 10000, so the search parameters should be
selected accordingly. Alternatively, by entering 0's, default values can be
accepted. Note that the default values are set on entering the {\tt PERIOD}
package and thus the {\tt FREQ} option need not be run if default frequencies
are required. The default values are calculated as follows: minimum frequency =
0 (ie. infinite period), maximum frequency = 1 / (2 $\times$ Smallest Data
Interval) (ie. Nyquist), frequency interval = 1 / (4 $\times$ Total Time
Interval). Note that the default values could demand greater than 10000
frequencies, in which case they will have to be modified by the user. 

\item {\tt CHAOS} -- Creates a return map by plotting Y(n+i) against Y(n),
where i is the {\em correlation integer} prompted for in the option.
See Scargle (1990a, 1990b) for a description of the use of return maps. 

\item {\tt CHISQ} -- This is a straight-forward technique where the input data
is folded on a series of trial periods. At each trial period, the data is
fitted with a sine curve. The resulting reduced-$\chi^2$ values are plotted as
a function of trial frequency and the minima in the plot suggest the most
likely periods. See Horne, Wade and Szkody (1986) for an example of the use of
this method, which is ideally suited to the study of radial velocity data or
any other sinusoidal variations. 

\item {\tt CLEAN} -- The {\tt CLEAN} algorithm was originally developed for use
in aperture synthesis and was later applied to one-dimensional data by Roberts,
Leh\'{a}r and Dreher (1987). An adapted version of Lehar's code is used here. 
The algorithm basically deconvolves the spectral window from the dirty spectrum.
This produces a {\tt CLEAN} spectrum, which is largely free of the many effects
of spectral leakage. The user is prompted for two parameters -- the loop gain
and the number of iterations. The loop gain must lie between 0 and 2, typical
values being between 0.1 and 1. The number of iterations should be large if the
loop gain is small, typical values lying between 1 and 100. See Roberts,
Leh\'{a}r and Dreher (1987) for further details. 

\item {\tt FT} -- This option performs a classical discrete Fourier transform
on the data and sums the mean-square-amplitudes of the result to form a power
spectrum (see, for example, Deeming 1975). 

\item {\tt MEM} -- Calculates the Maximum Entropy Method (MEM) power spectrum,
as described by Press {\it et al.} (1986) (see also Fahlman and Ulrych 1982).
The user is prompted for the order (or number of poles) of the MEM
approximation, which should be set to a few times the number of sharp spectral
features that one desires it to fit. With this restricted number of poles, the
method will smooth the spectrum somewhat, but this is often a desirable
property. While exact values depend on the application, one might take the
number of poles as 10 or 20 or 50 for 1000 or 5000 data points. {\tt MEM} can
produce strange results, especially if the number of poles or data points is
too large. Data with extremely sharp spectral features may produce split peaks
even at modest orders, and the peaks may shift with the phase of the sine wave.
Also, with noisy input data, if too high an order is chosen there will be many
spurious peaks. The use of this algorithm is often recommended in conjunction
with more conservative methods in order to help choose the correct model order
and to avoid being misled by spurious spectral features. The biggest drawback
of {\tt MEM} is that the data need to be evenly spaced. This is because the
frequencies are calculated as a function of the sampling interval, which is
determined from the {\em median} of the $x$-axis data. 

\item {\tt PDM} -- The phase dispersion minimization (PDM) technique is 
simply an automated version of the classical method of distinguishing
between possible periods, in which the period producing the least
observational scatter about the mean light curve (or, for example,
radial velocity curve) is chosen. This technique (which is described in
detail by Stellingwerf 1978) is well suited to cases in 
which only a few observations are available over a limited period of time,
especially if the light curve is highly non-sinusoidal.
The data is first folded on a series of trial frequencies. For each trial 
frequency, the full phase interval (0,1) is divided into a user-specified 
number of bins. The width of each bin is specified by the user, such that 
a point need not be picked (if a bin width more narrow than the bin spacing
is selected) or a point can belong to more than one bin (if a bin width
wider than the bin spacing is selected). The variance of each of these
bins (or samples) is then calculated. This gives a measure of the scatter
around the mean light curve defined by the means of the data in each sample.
The PDM statistic can then be calculated by dividing the overall variance
of all the samples by the variance of the original (unbinned) dataset. This 
process is then repeated for the next trial frequency. Note that windowed data 
cannot be passed to this option since its variance is zero. If the trial 
period is 
not a true period, then the overall sample variance will be approximately equal 
to the variance of the original dataset (ie. the PDM statistic will be
approximately equal to 1). If the trial period is a correct period, the PDM
statistic will reach a local minimum compared with neighbouring periods, 
hopefully near zero. 

The PDM method also gives an estimate of the {\em significance} 
of a minimum. This
is the probability that the value of the PDM statistic is due to random
fluctuations (it is calculated by determining the area of the F distribution
above 1/PDM in a two-sided test). The probability approaches unity as the
PDM statistic approaches unity and should be near zero for the minimum to be
regarded as reliable. Note that the significance is actually output in 
{\tt PEAKS} (see below) and not in this option. 

\item {\tt SCARGLE} -- Lomb (1976) and Scargle (1982) developed a novel type of
periodogram analysis, quite powerful for finding, and testing the significance
of, weak periodic signals in otherwise random, {\em unevenly sampled} data.
Horne and Baliunas (1986) have elaborated on the method and Press and Rybicki
(1989) present a fast implementation of the algorithm, a modified version of
which is used here. Note that windowed data cannot be passed to this option
(since it needs to divide by the variance (which is zero) to calculate the
false alarm probability). Otherwise, this option is simple to use and not only
produces a periodogram but also gives an estimate of the significance of peaks
in the periodogram -- the {\em false-alarm probability}. This represents the
probability that a peak of a given height will occur, {\em assuming that the
data are pure noise}. Therefore, a peak with probability 0 is definitely not
noise, while a peak with probability 1 is definitely noise. Note that the
false-alarm probability is actually output in {\tt PEAKS} (see below) and not
in this option. 

\item {\tt STRING} -- The string-length method is an intuitively simple method,
described in detail by Dworetsky (1983) and Friend {\it et al.} (1990). The
data is folded on a series of trial periods and at each period the sum of the
lengths of line segments joining successive points (the string-length) is
calculated. Minima in a plot of string-length versus trial frequency indicate
possible periods. Note that for sinusoidal data, the string-length should lie
between 1.4637 (the theoretical minimum) and 3 (the value below which it is
significant). 

\item {\tt PEAKS} -- This option should be run once a periodogram has been
obtained. It prompts for a periodogram and the time-series data that was used
to create it. It then finds the highest peak in the power spectrum (or lowest
trough if it is a string-length, PDM or reduced-$\chi^2$ plot) between
user-specified frequencies. The resulting period is calculated, along with a
crude error determined from the half-width of the peak. This option then goes
on to refine the period using a parabolic interpolation of the peak and the two
adjacent points. Finally, a more accurate error is derived using the analytical
expression given by Horne and Baliunas (1986). In addition, a number of useful
quantities, such as the false alarm probability, the string length or the PDM
significance (depending on the input periodogram) are output. The results may 
be written to a log file. For a useful discussion on the accuracy of period
determination, see Schwarzenberg-Czerny (1991). 

\item {\tt HELP} -- This command provides on-line help for {\tt PERIOD}. 
Detailed information about individual commands can be obtained by typing 
{\tt HELP 'COMMAND'} (eg. {\tt HELP SCARGLE}). 

\item {\tt QUIT} (or {\tt EXIT}) -- This quits the \verb@PERIOD_PERIOD@ 
sub-menu and returns the user to the main {\tt PERIOD} menu. 

\end{itemize}

Returning to the main {\tt PERIOD} menu:

\subsection*{\tt FIT}

Folds the data on a given period and zero point. This routine then fits the
data with a sine curve and outputs the fit parameters (which can be written to
a log file) and the resulting sine curve. 

\subsection*{\tt FOLD}

This routine folds the data on a given period and zero point. Hence, this
option transforms the data onto a phase scale, where one phase unit is equal to
one period and phase zero is defined by the zero point. Note that if a value of
0 is given as the zero point, the data is folded by taking the first data point
as the zero point. 

This option is useful for checking whether derived periods actually give
sensible results when applied to the data. In addition to normal folding
(which one can obtain by responding with a 0 to the number of phase bins
prompt), the user can phase bin the data, which folds the data and then
averages all the data points falling into each bin. 

\subsection*{\tt SINE}

Allows the user to add, subtract, multiply or divide a user-specified sine
curve. This option is useful for removing or adding known periods from data, 
thus enabling or testing the detection of other periods. 

\subsection*{\tt PLT}

This routine calls {\tt PLT}, the subroutine mode of {\tt QDP} -- the
plotting/fitting package written by Allyn Tennant; for further details see SUN
128 or the {\tt QDP/PLT} manual. This option is used to plot any desired data
slots. One can then manipulate the plot (just as if one were using {\tt QDP})
and create publication-ready diagrams. In addition, {\tt QDP} contains an
excellent set of fitting routines which can be used to fit a variety of
different functions to the data. In order to receive on-line help, simply type
{\tt HELP} at the {\tt PERIOD-PLT} prompt. To exit {\tt PERIOD-PLT} and return
to the {\tt PERIOD} menu, type {\tt EXIT}. 

\subsection*{\tt STATUS}

Returns information on the data slots or on the stored fits in the log file.
This command is useful in order to check which slots contain which datasets
and also as a means of obtaining some elementary statistics on the stored data. 
One can also use this option to check the fits stored in the log file without
having to exit the package and read the log file. 

\subsection*{\tt OUTPUT}

Writes any selected slot to an ASCII file on disk. This is the only way of
saving data created by {\tt PERIOD}, and should therefore be run before {\tt
QUIT}ing in order to store, say, a power spectrum. 

\subsection*{\tt HELP}

This command provides on-line help for {\tt PERIOD}. Detailed information about
individual commands can be obtained by typing {\tt HELP 'COMMAND'} (eg. {\tt
HELP PERIOD}). 

\subsection*{{\tt QUIT} (or {\tt EXIT})}

This option quits a {\tt PERIOD} session. However, it does provide a last
chance to stay in the package. This is essential, since any data files created
using {\tt PERIOD} will be lost on exit from the package {\em unless} one {\tt
OUTPUT}s the data first. 

\section{A Simple Recipe}
\label{recipe}

A simple guide designed to introduce inexperienced users to the steps involved 
in detecting periodicities is outlined below. Detailed descriptions of the
individual {\tt PERIOD} commands can be found in section~\ref{menu} and in
the on-line help facility.

\begin{enumerate}
\item Create an ASCII data file containing the time-series.
\item Read the data into {\tt PERIOD} using {\tt INPUT}. 
\item Detrend the data using the {\tt [M]} option in {\tt DETREND} (if the
data show long term variations, use the {\tt [P]} option instead).
\item Open a log file for the fits using {\tt OPEN}. 
\item Enter the \verb@PERIOD_PERIOD@ menu by typing {\tt PERIOD}.
\item Select the slots which contain the time-series data and specify the output
slots for the periodograms using {\tt SELECT}.
\item Set the frequency search limits using {\tt FREQ}. If you have no idea
what the period is, accept the default values by typing 0 (or alternatively, 
by not typing {\tt FREQ} in the first place). 
\item Calculate the Lomb-Scargle periodogram by typing {\tt SCARGLE}.
\item Now run {\tt PEAKS} on the resulting periodogram, specifying the detrended
time-series and the frequency range which contains the peak you wish to 
measure (you may enter 0,0 if you wish to process the entire range). Write 
the fits to the log file.
\item Now reselect the time-series slots and different output slots for a new
periodogram using {\tt SELECT}.
\item Type {\tt CLEAN} with 3 iterations and a loop gain of 0.5, for example.
\item Run {\tt PEAKS} on the resulting periodogram and store the fits.
\item Now quit the \verb@PERIOD_PERIOD@ sub-menu by typing {\tt QUIT}.
\item Plot the periodograms using {\tt PLT}. Check to see the validity of the
highest peak selected.
\item Check the fits in the log file using {\tt STATUS}. In particular, look
closely at the false alarm probability of the Lomb-Scargle periodogram.
\item Fold the {\em original} data on the most likely period using {\tt FOLD}.
\item Plot the folded data using {\tt PLT}. If this looks sensible, the period
may well be correct. Add labels using the QDP commands {\tt LA X HJD}, {\tt LA
Y Velocity (km/s)} (for example) and {\tt PLOT}. Make a hardcopy by typing {\tt
HARDCOPY 2600}. 
\item Output the periodograms and folded data to an ASCII file on disk using
{\tt OUTPUT}. 
\item Exit {\tt PERIOD} by typing {\tt QUIT}. 
\end{enumerate}

The above description
is intended only to be a very brief guide. Clearly, a great deal more
experimentation is required before one can definitely say that a period has
been detected. For example, one should investigate other large peaks in the
periodogram, try smaller or larger frequency ranges, or try one of the other
periodicity-finding options (a useful comparison of a number of different
options is given by Carbonell, Oliver and Ballester 1992). Other analysis
techniques should also be attempted, such as subtracting a sine curve from the
data in order to investigate its effects on the harmonics and enable the
detection of less-dominant periods. 

\section{Future Improvements}

\begin{itemize}
\item Port to UNIX and MS-DOS operating systems.
\item Write a more flexible slot system whereby the selected slots need not
be contiguous.
\item Add more sophisticated CHAOS analysis techniques, such as those described
by Canizzo and Goodings (1988) and Scargle (1990b).
\item Include more periodicity-finding options, such as AoV 
(Schwarzenberg-Czerny 1989).
\end{itemize}

\section{Acknowledgments}

The author gratefully acknowledges the use of subroutines written by Joseph
Lehar (Cambridge), Keith Horne (STScI) and Tom Marsh (Oxford), as well as those
written by Press {\it et al.} (1986), Press and Rybicki (1989) and Allyn
Tennant ({\tt QDP/PLT}, see SUN 128). I thank the bug-finding guinea-pigs
Martin Still (Sussex), Mark O'Dell (Sussex), Erik Kuulkers (Amsterdam) and 
Thomas Augusteijn (Amsterdam) for their many useful comments and suggestions. 

\section{References}

\noindent \rf{Canizzo, J. \&\ Goodings, D., 1986. {\em Astrophys. J. Lett.}, {\bf 334}, L31.}
\rf{Carbonell, M., Oliver, R. \&\ Ballester, J.~L., 1992. {\em Astron. Astrophys.}, {\bf 264}, 350.}
\rf{Deeming, T.~J., 1975. {\em Astrophys. Space Sci.}, {\bf 36}, 137.}
\rf{Dworetsky, M.~M., 1983. {\em Mon. Not. R. astr. Soc.}, {\bf 203}, 917.}
\rf{Fahlman, G.~G. \&\ Ulrych, T.~J., 1982. {\em Mon. Not. R. astr. Soc.}, {\bf 199}, 53.}
\rf{Friend, M.~T., Martin, J.~S., Smith, R.~C. \&\ Jones, D.~H.~P., 1990. {\em Mon. Not. R. astr. Soc.}, {\bf 246}, 637.}
\rf{Horne, J.~H. \&\ Baliunas, S.~L., 1986. {\em Astrophys. J.}, {\bf 302}, 757.}
\rf{Horne, K., Wade, R.~A. \&\ Szkody, P., 1986. {\em Mon. Not. R. astr. Soc.}, {\bf 219}, 791.}
\rf{Lomb, N.~R., 1976. {\em Astrophys. Space Sci.}, {\bf 39}, 447.}
\rf{Press, W.~H., Flannery, B.~P., Teukolsky, S.~A. \&\ Vetterling, W.~T., 1986. {\em Numerical Recipes: The Art of Scientific Computing}, Cambridge University Press, New York.}
\rf{Press, W.~H. \&\ Rybicki, G.~B., 1989. {\em Astrophys. J.}, {\bf 338}, 277.}
\rf{Roberts, D.~H., L\'{e}har, J. \&\ Dreher, J.~W., 1987. {\em Astron. J.}, {\bf 93}, 968.}
\rf{Scargle, J.~D., 1982. {\em Astrophys. J.}, {\bf 263}, 835.}
\rf{Scargle, J.~D., 1990a. {\em Astrophys. J.}, {\bf 359}, 469.}
\rf{Scargle, J.~D., 1990b. In {\em Errors, Bias \&\ Uncertainties in Astronomy}, Eds. Jaschek, C. \&\ Murtagh, F., Cambridge University Press, Cambridge.}
\rf{Schwarzenberg-Czerny, A., 1989. {\em Mon. Not. R. astr. Soc.}, {\bf 241}, 153.}
\rf{Schwarzenberg-Czerny, A., 1991. {\em Mon. Not. R. astr. Soc.}, {\bf 253}, 198.}
\rf{Stellingwerf, R.~F., 1978. {\em Astrophys. J.}, {\bf 224}, 953.}
\end{document}
