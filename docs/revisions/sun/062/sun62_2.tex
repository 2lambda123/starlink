\documentstyle[11pt]{article} 
\pagestyle{myheadings}

%------------------------------------------------------------------------------
\newcommand{\stardoccategory}  {Starlink User Note}
\newcommand{\stardocinitials}  {SUN}
\newcommand{\stardocnumber}    {62.2}
\newcommand{\stardocauthors}   {M Denby}
\newcommand{\stardocdate}      {3 September 1992}
\newcommand{\stardoctitle}     {WFCSORT --- Sort ROSAT Wide Field Camera data}
%------------------------------------------------------------------------------

\renewcommand{\_}{{\tt\char'137}}     % re-centres the underscore
\newcommand{\stardocname}{\stardocinitials /\stardocnumber}
\markright{\stardocname}
\setlength{\textwidth}{160mm}
\setlength{\textheight}{230mm}
\setlength{\topmargin}{-2mm}
\setlength{\oddsidemargin}{0mm}
\setlength{\evensidemargin}{0mm}
\setlength{\parindent}{0mm}
\setlength{\parskip}{\medskipamount}
\setlength{\unitlength}{1mm}

%------------------------------------------------------------------------------
% Add any \newcommand or \newenvironment commands here
%------------------------------------------------------------------------------

\begin{document}
\thispagestyle{empty}
SCIENCE \& ENGINEERING RESEARCH COUNCIL \hfill \stardocname\\
RUTHERFORD APPLETON LABORATORY\\
{\large\bf Starlink Project\\}
{\large\bf \stardoccategory\ \stardocnumber}
\begin{flushright}
\stardocauthors\\
\stardocdate
\end{flushright}
\vspace{-4mm}
\rule{\textwidth}{0.5mm}
\vspace{5mm}
\begin{center}
{\Large\bf \stardoctitle}
\end{center}
\vspace{5mm}

\section{Introduction}

The programs described in this document provide the mechanism for
producing ASTERIX (HDS) datasets from Wide Field Camera data
collected during the pointed phase of the ROSAT mission. A sort program
generates ASTERIX datasets ({\em e.g.}~time series, images, etc.) 
from pre-processed event files supplied as part of
the Rosat WFC Observation Datasets (RWODs). An exposure program
corrects these datasets allowing for
instrument characteristics. In addition a simple database manager
allows an index of RWODs to be maintained and searched.
The programs make use of the ADAM and ICL environment and a number of ICL
procedures have been provided to perform some of the more commonly
required operations such as extracting background subtracted lightcurves.


\section{Getting Started}
Assuming that WFCSORT is installed at your site, it may be started 
by executing the command

\begin{verbatim}
      $ WFCSTART
\end{verbatim}

The tasks are callable from DCL by simply typing the 
task name in response to the VMS prompt {\em e.g.}

\begin{verbatim}
      $ WFCSORT
\end{verbatim}

Alternatively, in order to make use of the facilities provided by ICL,
the tasks together with a series of ICL procedures may be
loaded into the users ICL session. This is achieved by means of the following
ICL commands

\begin{verbatim}
      ICL> LOAD WFCDIR:WFCLOAD
      ICL> SET NOCHECKPARS
\end{verbatim}

For convenience the user may include the above lines in a default ICL
login procedure located in SYS\$LOGIN:LOGIN.ICL. In order to be effective
this procedure should first be defined in the users DCL LOGIN.COM file {\em
e.g.}

\begin{verbatim}
      $ DEFINE ICL_LOGIN SYS$LOGIN:LOGIN.ICL
\end{verbatim}

\section{The ROSAT WFC Observation Dataset (RWOD)}
 
The RWOD is the set of data files distributed to the ROSAT WFC observer. Each
RWOD contains data from a single pointed phase observation in the form of a
Backup saveset. The software  described in this document does not require the
data to be stored in a  predefined directory. After starting ICL the data in an
RWOD may be recovered to disk as follows:

\begin{verbatim}
      ICL> DEFAULT location        ! The target directory
      ICL> WFCDISK MUB0:123456.BCK ! Replace with the tape drive and 
                                   ! the saveset required
\end{verbatim}

There are 6 files in the saveset which are essential for the sorting software
to work. These are:

\begin{verbatim}
      CAL_WFC_MASTER.SDF  The WFC Master Calibration File
      <RWOD>.ATT          Aspect file
      <RWOD>.HKR          Housekeeping rates file
      <RWOD>.HKC          Housekeeping configuration file
      <RWOD>.EVE          Pre-processed event file 
      <RWOD>.POS          Satellite position file
\end{verbatim}

CAL\_WFC\_MASTER.SDF defines the time history of the 
WFC calibration for the sorting software. WFCDISK copies the calibration
file from tape to the location pointed to by the name CAL\_WFC\_MASTER 
only if it has a more recent creation date than the file already in existence.

$<${\tt RWOD}$>$.ATT is a VMS ISAM file whose records define the pointing
and rate of change of the pointing over consecutive time intervals.

$<${\tt RWOD}$>$.HKR is a VMS ISAM file whose records contain the instrument
counters (see Section~\ref{se:wfchk} at 8 secs. resolution.

$<${\tt RWOD}$>$.HKC is a VMS ISAM file whose records define the instrument
configuration (with a new record generated for each change of configuration).

$<${\tt RWOD}$>$.POS is a VMS ISAM file whose records define the 
satellite location and instrument pointing at 60 second intervals.

$<${\tt RWOD}$>$.EVE is a VMS direct access file containing a list of event
parameters ordered in both time and spatial senses. For each event the
following values are stored:

\begin{itemize}
\item
The telemetered coordinates of the event on the surface of the
detector (in the range 0-511)
\item
The linearised coordinates of the event relative to the
nominal optical axis of the instrument (in the range -2.5 to 2.5 deg and
to 9 arc sec resolution)
\item
The local coordinates of the event relative to the nominal
pointing direction shown in the file header (in the range -2.5 to 2.5 deg and
to 6 arc secs resolution)
\item
The time tag of the event in seconds relative to the base time shown
in the file header (BASE\_MJD) to 1/32 sec resolution.
\end{itemize}

 A summary of the contents of an observation may be obtained by showing 
the EVE file header records using the WFCDBM program (described in
Section \ref{se:progs}). Below is an example of the header records from a typical EVE
file:

\begin{small}
\begin{verbatim}
      ROR Sequence #          : B50026    
      Axis Ra                 :  359.38
      Axis Dec                :   20.92
      Roll                    :   -8.50
      POP_FLAGS               : 00000B03
      Target                  : MKN335 (Day 46, ROR 150026 offset)      
      Observer                : WCALIB                                  
      Instrument              : WFC                                     
      Base date               : 16-JUL-90  
      Base MJD                : 48088.8086
      End MJD                 : 48088.8369
      Total Events            :    20879
      File Revision Level     : 5.2 


      Slot        Stime            Etime        Durn.  Flt  Fmt  Zm  Win   HT
             d:hh:mm:ss       d:hh:mm:ss

         1   0:00:02:59       0:00:07:46         288.  UV         z       161
         2   0:00:08:03       0:00:40:51        1968.  P1         z       161
\end{verbatim}
\end{small}

A fixed element of the header defines a number of general observation
parameters.

The Base MJD is the (nominal) start time of the observation and serves as the
origin for the event time tags. The Axis centre is the (nominal) observation
pointing and serves as the origin for the event local coordinate frame. The
Roll is the angle, in degrees, between Celestial North and the Wide Field
Camera elevation (defined in a positive clockwise sense when viewing the sky).
Note that all ROSAT coordinates are for  equinox 2000.0.

The fixed header is followed by information for a variable number of `slots'.
Each time interval during which events were collected and the instrument
configuration ({\it e.g.} the filter wheel position, HTs)  remained constant
defines a slot. For each slot the following information is displayed:

\begin{itemize}
\item
The start and end time in units of days, hours, mins, secs after Base MJD.
\item
The slot duration in secs.
\item
The filter configuration ({\it e.g.} S1A, P1, etc).
\item
The telemetry format. Screening at the pre-processing stage rejects data except
when the telemetry format was in normal or high-speed
modes. When the telemetry is in normal mode this field is blank,
otherwise the {\tt Hi} flag is displayed. The dead-time characteristics of 
the data are different in the two modes, there being a 200 count per second 
limit in normal mode and a 408 count per second limit in high-speed mode.
\item
The Zoom flag. The WFC has an enhanced spatial resolution mode in
which events from the central quarter of the detector only are
processed and the linear resolution is doubled. The field is blank
for non-zoomed data, otherwise the {\tt z} flag is displayed.
\item
The Window flag. Events from pre-set regions of the detector may have been
discarded before insertion in the telemetry
stream. When the window mode is in operation (denoted by a {\tt w} flag in
the slot summary) events from a rectangular area of the detector
have been excluded, leaving events from outside the window unaffected. For
inverted window mode (denoted by an {\tt i} flag in the summary) the event
selection is the reverse of the above ({\it i.e.} events have been accepted
from within the window and rejected from outside).
\item
The detector HT voltage setting. Whilst it is not anticipated that this
will change, it should be noted that the instrument
characteristics may change with different HT settings.
\end{itemize}

\section{The WFCSORT Programs}
\label{se:progs}

The programs involved are:
\begin{verbatim}
      WFCDBM          Data Base Manager
      WFCSORT         Sorts events into images, time series, etc.
      WFCEXP          Exposure corrects a binned data set
      WFCPOSX         Gives exposure at a point in the FOV
      WFCHK           Sorts HK rates to an HDS container
      WFCSLOTS        Selects a set of filter defined time slots
      WFCSLMRG        Merges a pair of time slot files
\end{verbatim}

These programs are described below in terms of the parameters which they require
from the user. 
Each program may be started by typing the name in response to the ICL prompt
together with on-line parameters, if required, in the KEYWORD= format:

\begin{verbatim}
      ICL>WFCSORT RA=270. DEC=66.6 \
\end{verbatim}

All programs have context sensitive help (using the VMS help system) which 
may be obtained by replying with the `?' character to any program prompt.


\subsection{WFCDBM}
Provides update and search facilities for all RWODs held at a particular
institute.

\subsubsection{Parameters}
\begin{description}
\item [CMND]
A number of commands, specified by the CMND parameter, are available 
and are specified below:
\begin{verbatim}
      (C)reate      Create an empty Index file
      (U)pdate      Update the index with a new RWOD
      (S)earch      Search for datasets matching various criteria.
      (H)ead        Display the header records from an event (EVE) file.
      (G)roup       Associate a group of slots within an EVE file.
\end{verbatim}
\item[EVEF   ]
The name of the input event file. The default extension for the file is .EVE.
\item[LISF   ]
The name of a file to hold listing o/p. The default is SYS\$OUTPUT.
\item[INDF   ]
The name the Index file which holds information on all the RWODs
in the library. If this file does not already exist it should be created 
using the CREATE option in WFCDBM. Thereafter the index file may be 
updated and searched.
\item[SMJD   ]
The start MJD for a search.
\item[EMJD   ]
The end MJD for a search.
\item[RAL    ]
The Right ascension lower limit for a search (decimal degs)
\item[RAH    ]
The Right ascension upper limit for a search (decimal degs)
\item[DECL   ]
The Declination lower limit for a search (decimal degs)
\item[DECH   ]
The Declination upper limit for a search (decimal degs)
\item[PI   ]
Any unique string which identifies the Principal Investigator
\item[TARGET   ]
Any unique string which identifies the target
\item[SELECT1]
The search function is performed on a series of selection criteria
which are entered in response to the repeated SELECTn prompts. The valid 
requests are

\begin{verbatim}
      (M)jd      select on an MJD window
      (R)a       select on an RA window
      (D)ec      select on a  DEC window
      (P)i       select on a  Principal Investigator string
      (T)arget   select on a  Target string
      (E)xit     combine selection criteria and perform the search
\end{verbatim}

\item[GROUP  ]
The sequence number of a group of slots to be associated together. (There
will be ab many sequence numbers for an observation as there are different
filter configurations). The slots
are present in the event file header records and the command groups together
slots with the same filter at the focus. The slot list is returned in the
SLOTS parameter. If the number of groups present in the data is exceeded
by the GROUP command SLOTS is returned as `-999'
\item[SLOTS  ]
A text string which specifies which observation slots have been 
associated. The text string is constructed
by the GROUP command and may be used to associate slots with the same filter
without a priori knowledge of the contents of the header records of an
event file.
\end{description}

\subsection{WFCSORT}

This task is used to sort pre-processed event files from the RWOD into 
ASTERIX (HDS) datasets.
\subsubsection{Parameters}
\begin{description}
\item[EVEF]
The name of the input event file.
The default extension for the file is .EVE.
\item[SHOW]
Requests that a sub-set of the information from the header records of the
event file is displayed at the user terminal. In a situation where many
operations are performed on the same event file the user may wish to 
reply `Y' to the first occurence of the prompt but `N' to all subsequent
occurences.
\item[OUTPUT]
The name of the output dataset. The default extension for the file is 
.SDF and should NOT be provided by the user.
\item[DTYPE]
The type of dataset to be produced by the sort. Valid types are

\begin{verbatim}
      (I)mage   to create an ASTERIX sky coord image dataset
      (R)aw     to create a detector coord image
      (L)in     to create a linearised detector coord image
      (T)ime    to create a time series dataset
      (E)vent   to create an event dataset.
\end{verbatim}

\item[RA]
The Right Ascension of the sort field centre (decimal degs). By default this
is set to the axis RA value  given in the header.
\item[DEC]
The Declination of the sort field centre (decimal degs). By default this
is set to the axis DEC value in the header.
\item[DAZ]
The half extent of the square field to be sorted (degs), 
the default value is 3 degrees.
\item[INR]
In time series mode INR specifies the inner radius of an annulus
from which to select events. The radius should be specified in arcminutes
and may be set to zero in order to select from a circular region.
\item[OUTR]
In time series mode OUTR specifies the outer radius of an annulus
from which to select events. The radius should be specified in arcminutes.
\item[NXPIX]
The number of pixels along a side of the image. For the default field half
extent of 3 degrees NXPIX = 256 
would give a pixel width of approx 1.4 arcminutes.
\item[SLOTS]
A text string which specifies which observation slots to include in
the sort. (These are the slots which are displayed when SHOW = Y is selected).
Specify the slots in the format i-j,k,l-m. The slots selected
should all have the same filter at the focus. If a slot value of `0' is
selected the selection mode changes and the user is prompted for the 
filter type via the FILTER parameter.
\item[START\_T]
The start time of a selection window which will be combined with
the slot and discrimination windows. The unit may be either seconds
from the BASE\_MJD given in the EVE file header or an absolute MJD
(in the latter case the MJD value should be preceded by an `M').
The default value is the seconds offset of the start of the first selected slot. 
It is also possible to input the start and end times from a text file,
with one time pair per record. In this case the time pairs should be
specified in MJDs, each preceded by the `M' character and separated by
a space. When performing background subtraction of two concentric time
series it is important that the two time series be co-binned and so 
START\_T should be chosen to be the same for both.
\item[END\_T ]
The end time of a selection window which will be combined with
the slot and discrimination windows. The unit may be either seconds 
from the BASE\_MJD given in the EVE file header or an absolute MJD (in 
the latter case the MJD value should be preceded by an `M'). The default 
value is the seconds offset of the end of the last selected slot. It is also possible
to input the start and end times from a text file, with one time pair per
record. In this case the time pairs should be specified in MJDs, each 
preceded by the `M' character and separated by a space.
\item[USEF]
The user may optionally specify the time windows from which events
will be selected by means of a simple text file.
\item[SFILE]
The name of a text file from which the time windows will be read.
Each record in the file should contain a pair of start and stop MJDs
separated by a space and with each MJD preceded by the letter `M'.
\item[NBINS]
The number of bins into which the time extent of the sort 
is to be divided. If a value `0' is given the program prompts for the
time duration for each bin (see TBIN).
\item[TBIN]
If NBIN is specified as 0 then TBIN should be set to the required
time duration of a bin (in seconds).
\item[FILTER]
A text string definition of the filter to use. This is selected from
the second column of the following table

\begin{verbatim}
      MCF #        Filter        Material

      1             UV           UV CAL
      2             S2B          Le/Be
      3             P1           Le/Al
      4             S1B          Le/C
      5             OPQ          Le/Al
      6             S2A          Le/Be
      7             P2           Sn/Al
      8             S1A          Le/C/B4C
 
      (MCF = Master Calibration File)
\end{verbatim}

\item[REJECT]
Wide Field camera data is sometimes contaminated by background features
whose exact nature and cause are not well understood. It is
however possible to improve the quality of badly affected data by 
choosing to reject data on a series of constraints. Data may
be excluded from a sort operation by selecting one of the LEVS or VIEW
options described below. Alternatively the NONE option should be selected 
to retain all the events.
\item[LEVS]
Rejects data when the limited event rate was above a user defined
threshold user is then prompted for the LEVS threshold to use, a typical
rate which gives noticeable improvement in a large number of cases is
50 c/s.
\item[VIEW]
Rejects data when the Velocity-View (or ram) angle or the solar
zenith angle were below certain limits. The limits are set as hidden
defaults of the program but may be overridden by giving values for
the VELVIEW and/or SOLZEN parameters on the command line (or using the 
PROMPT qualifier on the command line).
\item[ROOT]
The File root which defines the location of the .HKR file used for
LEVS discrimination. This defaults to the root of the EVE file. Note that
the root is also used to derive the name of the .POS file which is
used by WFCSORT to perform earth occultation screening of the data. 
Slots in the EVE file are clipped whenever the angle between the 
instrument viewing direction and the zenith exceeds a preset threshold 
of 130 degrees.
\end{description}

\subsection{WFCEXP}

This program takes as input an ASTERIX dataset (time series or image)
whose data array contains detected events. It creates
a dataset whose data array contains event flux (c/s) normalised
to the on-axis response of the instrument. The flux calculation takes
into account the following instrument dependencies:
\begin{itemize}
\item
Instrument dead time effects (the event rates needed for the 
calculation are stored in the $<${\tt RWOD}$>$.HKR file which is part
of the standard observation dataset).
\item
Instrument vignetting effects. The Instrument pointing is derived
from the $<${\tt RWOD}$>$.ATT file. The vignetting at points within the
field is derived from the filter and detector information 
stored in the WFC Calibration File (with the logical name 
CAL\_WFC\_MASTER).
\item
Optionally the program may correct the fluxes in a dataset for the 
degradation of the detector efficiency. The time history of the efficiency
is stored in the calibration file and is used to determine an interpolated
value at the epoch appropriate to the dataset. The user is warned if the
dataset epoch is beyond the end of the current time history in the
calibration file. In this event the latest value for detector efficiency
is used.
\item
Optionally the program may correct the fluxes in a one dimensional dataset 
for the fraction of the point response function which fell outside the source 
centred circle from which events were originally accepted by the WFCSORT
program.
\end{itemize}
\subsubsection{Parameters}
\begin{description}
\item[INP]
The name of an input ASTERIX dataset. The dataset may be either a time series
or an image.
\item[OUT]
The name of an output ASTERIX dataset. The dataset is cloned from the input and 
then the raw count array is replaced by an exposure corrected 
count/sec array.
\item[EFFCOR]
Selects correction of fluxes for the degradation of detector efficiency.
\item[PSFCOR]
Selects correction of fluxes in a one-dimensional dataset for the point
spread fraction.
\end{description}

\subsection{WFCPOSX}

This program takes as input an ASTERIX image dataset as produced by WFCSORT
and reports exposure information for any location within the image.
The quoted effective exposure is the value by which a raw count derived
at the location should be divided in order to produce a count rate 
corrected to the axis of the instrument at the epoch of the data set.
In addition the program displays the efficiency of the detector for the
epoch of the data set relative to the launch efficiency. This relative
efficiency may be used to increase a measured flux to that appropriate 
to a non-degraded detector. The exposure calculation works in a similar manner
to that in the WFCEXP program described above.
\subsubsection{Parameters}
\begin{description}
\item[INP]
The name of an input ASTERIX image dataset.
\item[RA]
The right ascension of the point in the FOV at which exposure information
is required. (decimal degrees).
\item[DEC]
The declination of the point in the FOV at which exposure information
is required. (decimal degrees).
\end{description}

\subsection{WFCHK}
\label{se:wfchk}

This program extracts the counters present in a reduced
housekeeping rates file ($<${\tt RWOD}$>$.HKR) together with the
pointing information present in an aspect file ($<${\tt RWOD}$>$.ATT) 
into a series of arrays in an HDS 
file. (Note that for flexibility this file does not obey
the ASTERIX standard format. This means that an 
extension must be given when giving filenames to ASTERIX, 
e.g. DRAW $<${\tt file}$>$.TEVS $<${\tt device}$>$.)

 The following parameters are stored in the HDS file (in arrays
of the same names).
\begin{verbatim}
      TEVS            Total events output by the detector preamplifiers
      VEVS            Valid events, i.e. TEVS within pulse height limits
      LEVS            Limited events, i.e. VEVS within the electronic f.o.v.
      AEVS            Accepted events, i.e LEVS entered into telemetry queue.
      FEVS            Final events, i.e. AEVS telemetered to ground
      RA              Pointing direction RA (degrees)
      DEC             Pointing direction DEC (degrees)
\end{verbatim}

The program gives the option of binning directly from an HKR file ({\it i.e.}
using the file start and end MJDs and intrinsic bin size of 8 secs)  or using
the bin structure from an existing time series dataset. In this latter mode the
user may create arrays of background counts aligned with an existing bin
structure. (BUT note that the binning mechanism is spot sample at the bin
centre {\it i.e.} summing over the bin duration is not performed.

\subsubsection{Parameters}
\begin{description}
\item[USE]
Requests that the binning information should be extracted from an
existing time series.
\item[INPUT]
If USE is selected the user is prompted for the name of a time series
dataset.
\item[HKR]
The name of an HKR reduced rates file. The .HKR extension is
optional.
\item[OUTPUT]
The name of the created HDS file.
\end{description}

\subsection{WFCSLOTS}

This program is used to select time slots with the WFC in a particular filter
configuration. The time slots are written to a text file with each record
containing one pair of start and end times (MJDs). The file may be merged
(using the WFCSLMRG program) with a similiar file created from XRT data and in
this way common time slots, when the two instruments were simultaneously
collecting data, may be defined and used to control event selection in the
WFCSORT program.

\subsubsection{Parameters}
\begin{description}
\item[EVEF]
The name of the input event file.
The default extension for the file is .EVE.
\item[SHOW]
Requests that a sub-set of the information from the header records of the
event file is displayed at the user terminal. In a situation where many
operations are performed on the same event file the user may wish to 
reply `Y' to the first occurence of the prompt but `N' to all subsequent
occurences.
\item[SLOTS]
A text string which specifies which observation slots to include in
the sort. (These are the slots which are displayed when SHOW = Y is selected).
Specify the slots in the format i-j,k,l-m. The slots selected
should all have the same filter at the focus. If a slot value of `0' is
selected the selection mode changes and the user is prompted for the 
filter type via the FILTER parameter.
\item[SFILE]
The name of a text file into which the time windows will be written.
Each record in the file contains a pair of start and stop MJDs
separated by a space and with each MJD preceded by the letter `M'.
\item[FILTER]
A text string definition of the filter to use. This is selected from
the second column of the following table
\begin{verbatim}
       MCF #        Filter        Material

       1             UV           UV CAL
       2             S2B          Le/Be
       3             P1           Le/Al
       4             S1B          Le/C
       5             OPQ          Le/Al
       6             S2A          Le/Be
       7             P2           Sn/Al
       8             S1A          Le/C/B4C
 
       (MCF = Master Calibration File)
\end{verbatim}

\end{description}
\subsection{WFCSLMRG}
The program is used to merge time slot files into a single file of common
time slots. The merged file may then be used to control event selection in 
the WFCSORT program.
\subsubsection{Parameters}
\begin{description}
\item[SFILE1]
The name of the first time slot file to merge.
\item[SFILE2]
The name of the first time slot file to merge.
\item[SOFILE]
The name of the output merged time slot file.
\end{description}

\subsection{An example}
 The following example shows how WFCSORT and WFCEXP may be used to
create and correct a time series from an event file. Parameter prompts and 
typical user responses are shown.

\begin{small}
\begin{verbatim}
      $ WFCSTART                           ! Define necessary VMS symbols etc.

      ICL> LOAD WFCDIR:WFCLOAD
      ICL> SET NOCHECKPARS                 ! Or do these 2 lines via LOGIN.ICL

      ICL> WFCSORT
          EVEF - 015079                   ! Select an event file
          SHOW - Y                        ! Display the header records
          DTYPE - TIME                    ! Select a time series o/p
          RA - 123.45                     ! Set the Ra (decimal degs)
          DEC - 67.89                     ! Set the Dec
          INRAD - 0                       ! Select a circular region on source
          OUTRAD - 6                      ! Circle radius = 6 arcmin
          SLOTS - 0                       ! Select by filter rather than by slots
          FILTER - P1                     ! Set the filter
          NBINS - 0                       ! Select by bin size rather than number
          TBIN - 5.                       ! Set the bin width to 5 secs
          REJECT - NONE                   ! Don't do event discrimination
          OUTPUT - TIM_P1                 ! O/p Asterix dataset
 
      ICL> WFCEXP
          INP - TIM_P1                    ! I/p raw time series
          OUT - TIM_P1C                   ! O/p corrected time series
\end{verbatim}
\end{small}

\section{The ICL procedures}

The WFCLOAD procedure defines a set of ICL procedures which perform
commonly required series of operations with minimum user interaction.
The procedures described here are:

\begin{verbatim}
      WFCDISK         Load data from an RWOD onto disk
      WFCAUTO         Standard analysis of a series of RWODs
      WFCIMAGE        Auto imaging of the events from an EVE file
      WFCLIGHT        Produce a light curve at a point in the current image
      WFCSPC          Produce a WFC spectrum at a point in the current image
      WFCSUB          Subtracts two contemporary time series
\end{verbatim}

The procedures have been written to exploit the NOCHECKPARS switch in ICL
so that if command line arguments are omitted then they are prompted
for. 
The procedures are described below in terms of the command line arguments
which they require (shown in angle parentheses).

\subsection{WFCDISK $<$SAVESET$>$ $<$TARGET$>$}

The procedure copies the files from a user specified RWOD save set 
to the users target directory. The SAVESET specifier must include both device
and file names (eg MUB0:123456.BCK). Each RWOD contains a copy of the
WFC calibration file and, after recovery to disk, a check is made of the
creation date of this file. If the file is more recent than the default
version, referenced by the CAL\_WFC\_MASTER logical, then this default
version is replaced.

\subsection{WFCAUTO $<$RDIR$>$}

The procedure works through all the RWODs located on a requested directory
producing hard copy images for each separate filter configuration present
in each RWOD. RDIR gives the full specification for the directory
containing the RWOD files.

\subsection{WFCHKR $<$RWOD$>$ $<$OUTF$>$}

The procedure creates an HDS file containing the various WFC instrument
count rates together with the pointing direction of the instrument.
The parameters are all binned into an HDS container at the fundamental
resolution of the housekeeping data (8 secs). The input RWOD, including
full directory spec., is specified by the RWOD argument. The output 
container file is specified by the OUTF argument.

\subsection{WFCIMAGE $<$RWOD$>$ $<$FILT$>$ $<$OUTF$>$}

The procedure creates a full field, 256 pixel, image of the
data specified by the RWOD argument for the filter specified by the
FILT argument. An ASTERIX image is formed in the file specified
by the OUTF argument. In addition an exposure corrected image and
and exposure field are created in files called $<$OUTF$>$\_C 
and $<$OUTF$>$\_X respectively.

\subsection{WFCLIGHT $<$TBIN$>$ $<$OUTF$>$}

The procedure is intended for interactive use only. It requests the
location of source and background boxes from the currently displayed
image and produces a background subtracted, exposure corrected, light
curve for the source in the file specified by the argument OUTF. The
bin size in seconds is specified via the TBIN argument.

\subsection{WFCSPC $<$OUTF$>$ $<$MERGE$>$ $<$XTIM$>$}
The procedure is intended for interactive use only. It requests the
location of source and background boxes from the currently displayed
image and produces a background subtracted, exposure corrected, spectral
dataset for the source in the file specified by the argument OUTF.
The procedure may optionally arrange to collect data from periods only
when the WFC and XRT systems were collecting data simultaneously. If
this option is required the MERGE option should be set to `YES' and
the XTIM argument should contain the name of a simple text file
in which each record contains a pair of MJDs between which the XRT
was collecting data. The MJDs should be separated by a space 
and each preceded by the letter `M'.

\subsection{WFCSUB $<$SRCF$>$ $<$BKDF$>$ $<$OUTF$>$}
The procedure subtracts a background light curve, specified by the
BKGF argument, from a source light curve, specified by the SRCF
argument. The subtracted light curve is formed in a file specified
by the OUTF argument. The two input light curves must have been
exposure corrected. The fluxes are normalised by the areas of the
two original boxes from which events were sorted, then subtraced.

\end{document}
