\documentstyle[11pt]{article} 
\pagestyle{myheadings}

%------------------------------------------------------------------------------
\newcommand{\stardoccategory}  {Starlink User Note}
\newcommand{\stardocinitials}  {SUN}
\newcommand{\stardocnumber}    {182.1}
\newcommand{\stardocauthors}   {M J Bly}
\newcommand{\stardocdate}      {29 March 1994}
\newcommand{\stardoctitle}     {EMAIL --- E-mail address searching}
%------------------------------------------------------------------------------

\newcommand{\stardocname}{\stardocinitials /\stardocnumber}
\renewcommand{\_}{{\tt\char'137}}     % re-centres the underscore
\markright{\stardocname}
\setlength{\textwidth}{160mm}
\setlength{\textheight}{230mm}
\setlength{\topmargin}{-2mm}
\setlength{\oddsidemargin}{0mm}
\setlength{\evensidemargin}{0mm}
\setlength{\parindent}{0mm}
\setlength{\parskip}{\medskipamount}
\setlength{\unitlength}{1mm}

%------------------------------------------------------------------------------
% Add any \newcommand or \newenvironment commands here
%------------------------------------------------------------------------------

\begin{document}
\thispagestyle{empty}
SCIENCE \& ENGINEERING RESEARCH COUNCIL \hfill \stardocname\\
RUTHERFORD APPLETON LABORATORY\\
{\large\bf Starlink Project\\}
{\large\bf \stardoccategory\ \stardocnumber}
\begin{flushright}
\stardocauthors\\
\stardocdate
\end{flushright}
\vspace{-4mm}
\rule{\textwidth}{0.5mm}
\vspace{5mm}
\begin{center}
{\Large\bf \stardoctitle}
\end{center}
\vspace{5mm}

%------------------------------------------------------------------------------
%  Add this part if you want a table of contents
%  \setlength{\parskip}{0mm}
%  \tableofcontents
%  \setlength{\parskip}{\medskipamount}
%  \markright{\stardocname}
%------------------------------------------------------------------------------

\section{Introduction}

There exists an extensive list of the electronic mail addresses of the
world's Astronomers, maintained by Chris Benn of the RGO. Starlink also
has its lists of users and their email addresses. The EMAIL utility is
an easy way conducting to search these lists on-line by name to find
the email address of elusive collaborators. 

\section{Searching for E-mail addresses}

The facilities offered be email vary slightly on Unix and VMS. Both
versions offer the searching facility to find usernames and email
addresses.  The VMS version has an on-line help facility giving details
about email addressing methods for VMS users.

Both versions search the following files:

\begin{itemize}

\item The list of Astronomical Institutions:
{\tt /star/etc/email/astroplaces.lis}

\item The list of local users: {\tt /star/local/admin/usernames.lis}

\item The World list of Astronomers: {\tt /star/etc/email/astropersons.lis}.

\end{itemize}

Thus EMAIL can be used to search for Institutional email
addresses as well as individuals.

\subsection{Unix}

The Unix version is used thus:

\begin{verbatim}
      % email keyword keyword ...
\end{verbatim}

where \verb+keyword keyword ...+ is a list of one or more
keywords for the search. It is thus possible to search for many
addresses at once, or conduct a search based on similar spellings of a
name in one pass.

\subsection{VMS}

To use the VMS version, type:

\begin{verbatim}
      $ EMAIL keyword,keyword,...
\end{verbatim}

where \verb+keyword,keyword,...+ is a list of one or more
keywords for the search. It is thus possible to search for many
addresses at once, or conduct a search based on similar spellings of a
name in one pass.

To access the EMAIL help system, type:

\begin{verbatim}
      $ EMAIL
\end{verbatim}

which will start the VMS style help facility.

\end{document}
