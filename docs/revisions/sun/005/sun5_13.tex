\documentstyle{article}
\pagestyle{myheadings}

%------------------------------------------------------------------------------
\newcommand{\stardoccategory}  {Starlink User Note}
\newcommand{\stardocinitials}  {SUN}
\newcommand{\stardocnumber}    {5.13}
\newcommand{\stardocauthors}   {P T Wallace}
\newcommand{\stardocdate}      {6th December 1993}
\newcommand{\stardoctitle}     {ASTROM --- Basic astrometry program}
%------------------------------------------------------------------------------

\newcommand{\stardocname}{\stardocinitials /\stardocnumber}
\markright{\stardocname}
\setlength{\textwidth}{160mm}
\setlength{\textheight}{240mm}
\setlength{\topmargin}{-5mm}
\setlength{\oddsidemargin}{0mm}
\setlength{\evensidemargin}{0mm}
\setlength{\parindent}{0mm}
\setlength{\parskip}{\medskipamount}
\setlength{\unitlength}{1mm}

\begin{document}
\thispagestyle{empty}
SCIENCE \& ENGINEERING RESEARCH COUNCIL \hfill \stardocname\\
RUTHERFORD APPLETON LABORATORY\\
{\large\bf Starlink Project\\}
{\large\bf \stardoccategory\ \stardocnumber}
\begin{flushright}
\stardocauthors\\
\stardocdate
\end{flushright}
\vspace{-4mm}
\rule{\textwidth}{0.5mm}
\vspace{5mm}
\begin{center}
{\Large\bf \stardoctitle}
\end{center}
\vspace{5mm}
\setlength{\parskip}{0mm}
%\tableofcontents
\setlength{\parskip}{\medskipamount}
\markright{\stardocname}

%------------------------------------------------------------------------------

\newcommand{\radec}     {$[\alpha,\delta\,]$}
\newcommand{\xy}        {$[x,y\,]$}
\newcommand{\xieta}     {$[\xi,\eta\,]$}

%------------------------------------------------------------------------------

\section{Introduction}
ASTROM is a simple plate reduction utility, designed to allow the
non-specialist user to get good results with a minimum of trouble and
esoteric knowledge.  The user supplies a text file containing
information about the
exposure and the positions of reference and unknown stars;  ASTROM
performs the various coordinate transformation and fitting operations
required, displays a synopsis report on the command terminal, and
prepares a detailed report for later printing.

\section{Operating Instructions}
ASTROM is run by means of the following command (DCL on VAX/VMS,
C-shell on Unix, MS-DOS on PC):
\begin{quote}
 \verb|astrom| [{\it input}] [{\it report}]
\end{quote}
where the input file {\it input}\, defaults to \verb|astrom.dat| and
the report file {\it report}\, defaults to \verb|astrom.lis|.  A
parameter can be defaulted either by leaving it out or
by using a pair of double-quotes.

The synopsis report, which normally appears on the command terminal,
should be monitored and any reference stars with large
residuals noted.  The input file can then be edited as
necessary and the job rerun.  Finally, the report file (which
contains Fortran printer format codes) can
be printed in the normal way.  An example report file is
reproduced in Appendix~B.

\section{The Input File}
The input file is an ordinary text file and is
terminated either by end-of-file or by a record
beginning $'${\tt E}$\,'$.  
Uppercase and lowercase are both acceptable throughout
the file and may be mixed
freely;  leading spaces are ignored.  A comment, beginning
$'${\tt *}$\,'$, can be appended to any record.
Completely blank records (and any
beginning with $'${\tt *}$\,'$) are ignored and can be used to improve
layout and provide commentary.

Most of the records consist of (or contain) various numbers of
numeric fields, separated by spaces (or commas).
In many
instances it is simply the number of fields present which enables
ASTROM to determine which sort of record has been read.
Free-format number decoding is used throughout;  spaces can
be freely inserted between fields, and many other freedoms
are permitted (see the documentation for the SLALIB
routines DFLTIN and DBJIN, in SUN/67).

Each file typically specifies
a single plate reduction;  however, multiple
{\it sequences}\, of records,
each specifying a complete
and separate plate reduction, can be used,
each sequence being separated from the next
by a record beginning
$'${\tt /}$\,'$.  This feature is used by the CHART utility
(see SUN/32), which is a convenient way of generating
ASTROM input without having to copy out star catalogue
entries by hand.

\goodbreak
The overall layout of each sequence is as follows:

\begin{center}
\begin{tabular}{|l|c|l|}
\hline
group & records & mandatory? \\
\hline
results equinox  &   1           & no \\
telescope type   &   1           & no \\
plate data       &   1           & yes \\
observation data &   1-3         & no \\
reference stars  &  2-3 per star & at least 2 stars \\
unknown stars    &  1-2 per star & no \\
\hline
\end{tabular}
\end{center}

Several sorts of record involve celestial positions.  Although
ASTROM can be made to work internally in {\it observed}\,
coordinates ({\it i.e.}\ as affected by refraction {\it etc.}),
all input data and reports are in terms of various sorts of
{\it mean}\, \radec.  In any particular instance, the mean coordinate
system is specified by quoting an {\it equinox}.  An
{\it epoch}\, is also required, for the calculation of
proper motion;  in the case of catalogue stars
this is frequently the same as the equinox.

Both the old pre~IAU~1976 (loosely FK4) system and the
new post~IAU~1976 (loosely FK5) system are supported, and
data in the two systems can be mixed freely.  ASTROM
follows the established convention of using the equinox to
distinguish between the two systems.  If the equinox is
prefixed by $'${\tt B}$\,'$ (which stands for
{\it Besselian}) then the position
is an old FK4 one;  if a prefix of $'${\tt J}$\,'$ is used (standing for
{\it Julian}), the position is a new FK5 one.  If no prefix is used,
pre 1984.0 equinoxes indicate the old FK4 system, and equinoxes
of 1984.0 or later indicate the new FK5 system.  The $'${\tt B}$\,'$
or $'${\tt J}$\,'$
prefix may also be used with epochs although the distinction is
unlikely to be significant.  The two most common equinoxes are
B1950 and J2000.  All FK4 positions include E-terms of aberration,
consistent with published catalogues.

When using the old system ({\it e.g.}\ B1950)
to specify the position of an object whose
proper motion is presumed to be
negligible, it is necessary to specify an epoch (as well as
the equinox).  This is because the old system is not
an inertial frame; failure to recognize galactic rotation
at the time the system was first established means that the
FK4 frame is rotating, and that even extragalactic
objects have fictitious proper motions which need to be taken
into account in precise work.  ASTROM accepts an optional
format of reference star data for such cases, where the
proper motions are omitted and an epoch is mandatory.

Appendix~A contains a detailed specification of the syntax of
the ASTROM input file, together with a comprehensive example.
As an introduction, we will look at a simple but typical example
of such a file:

\begin{quote}
\begin{small}
\begin{tabular}{|l|}
\hline
\\
\verb|B1950                                     * Results in FK4| \\
\verb|SCHM                                      * Schmidt geometry| \\
\verb|19 04 00.0  -65 00 00  B1950.0  1974.5    * Plate centre, and epoch| \\
\verb|18 56 39.426  -63 25 13.23  -0.0002  -0.036  B1950.0  * Ref 1| \\
\verb|44.791   85.643| \\
\verb|19 11 53.909  -63 17 57.57   0.0058  -0.044  1950.0   * Ref 2| \\
\verb|-46.266   92.337| \\
\verb|19 01 13.606  -63 49 14.84   0.0020  -0.026  1950.0   * Ref 3| \\
\verb|17.246   64.945| \\
\verb|19 08 29.088  -63 57 42.79   0.0016   0.018  1950.0   * Ref 4| \\
\verb|-25.314   57.456| \\
\verb|19 02 10.088  -63 29 16.73   0.0012  -0.019  1950.0   * Ref 5| \\
\verb|11.890   82.766| \\
\verb|-5.103    58.868                      *  Candidate| \\
\verb|19 09 46.2  -63 51 27  J2000.0        *  Radio pos| \\
\verb|END| \\
\\
\hline
\end{tabular}
\end{small}
\end{quote}

\goodbreak
Taking each record (or group of records) in turn:

\begin{quote}
\begin{tabular}{|l|}
\hline
\verb|B1950                                     * Results in FK4| \\
\hline
\end{tabular}
\end{quote}
This is the {\bf results equinox} record.
It specifies the mean equator and equinox for the coordinate
system of the report.
If this record is omitted, the results will be in J2000.

\goodbreak
\begin{quote}
\begin{tabular}{|l|}
\hline
\verb|SCHM                                      * Schmidt geometry| \\
\hline
\end{tabular}
\end{quote}
This is the {\bf telescope type} record, which
describes the projection geometry.
The telescope type is given by the first four characters of the
record;  there are currently six options.
$'${\tt SCHM}$\,'$, as used here, is for Schmidt telescopes.
$'${\tt ASTR}$\,'$ selects the tangent plane or {\it gnomonic}\,
projection, produced by conventional astrographic telescopes
(and by pinhole cameras).
Then there are three special AAT options: $'${\tt AAT2}$\,'$
and $'${\tt AAT3}$\,'$ for
the Prime Focus doublet and triplet correctors, and $'${\tt AAT8}$\,'$ for
the {\it f}/8 Ritchey-Chretien focus (using the vacuum plateholder).
The option $'${\tt JKT8}$\,'$ models the field distortion
of the JKT ({\it f}/8 Harmer-Wynne focus).
Finally, the option $'${\tt GENE}$\,'$ specifies generalized
pincushion/barrel distortion, the magnitude of which is given
by a single numeric parameter $q$ following the telescope
type string; further details are given in Section~4.

\goodbreak
\begin{quote}
\begin{tabular}{|l|}
\hline
\verb|19 04 00.0  -65 00 00  B1950.0  1974.5    * Plate centre, and epoch| 
\\
\hline
\end{tabular}
\end{quote}
This is the mandatory {\bf plate data} record, specifying the point on
the sky corresponding to the pole of the projection
geometry (which is usually, but not
necessarily\footnote{JKT {\it f}/8 plates, for example, are
mounted eccentrically.  The plate data record must specify the
celestial coordinates of the optical axis, preferably to
within a millimetre or so.  Section~7 gives details of how the
plate centre (and the radial distortion) can be determined
automatically.}, at the geometrical centre of the plate)
and the date on which the exposure occurred.
The \radec\ is expressed
as h~m~s~$^\circ$~$'$~$''$.  The hours, minutes, degrees and
arcminutes fields must all be integers.  The sign of the $\delta$
precedes the degrees.  The \radec\ must be followed
by an equinox.
The epoch specifies when the picture
was taken, needed for the proper motion calculation.
The epoch can be omitted if more precise information is to be supplied
later via the optional observation data records.

\goodbreak
\begin{quote}
\begin{tabular}{|l|}
\hline
\verb|18 56 39.426  -63 25 13.23  -0.0002  -0.036  B1950.0  * Ref 1| \\
\verb|44.791   85.643| \\
\hline
\end{tabular}
\end{quote}
This is the first of several record pairs describing the
{\bf reference stars}.  At least two such pairs are needed in
order to run ASTROM, and three if both sorts of linear fit are to
be done.  At least 10 stars are required if fitting of the
radial distortion and/or plate centre is to be attempted.  A typical
number for an ordinary linear fit would be about a dozen stars;
a thorough job covering a large area of a plate and with
automatic determination of the radial distortion and plate centre
selected would require perhaps 50.
The current limit is 1000.
The first record contains \radec, proper
motions in seconds per year and arcseconds per year
respectively ({\it n.b.}\ not centuries as in some
catalogues), and equinox, followed optionally by
epoch and/or parallax.  If the epoch is omitted (as in the above example),
it is assumed to be the same as the equinox.  Parallax, which is
in arcseconds, defaults to zero.  For reference
stars whose positions are given in the old
FK4 system, and whose proper motions are presumed to be zero in an
inertial sense, an alternative format is available,
with the proper motions omitted and the epoch mandatory:
\begin{quote}
\begin{tabular}{|l|}
\hline
\verb|18 56 39.422  -63 25 14.00  B1950.0  1971.3           * Ref 1| \\
\hline
\end{tabular}
\end{quote}
(Because the FK4 system is not inertial, using the
format described earlier and simply putting zero for the
proper motions would {\bf not} give the same effect.)  The
first 10 characters of any comment
are picked up and appear on the reports as `name'.
The second record of the reference star pair is the
measured \xy.  For the 4-coefficient model to
work properly, $x$ and $y$ must be in the same units.
The reports will look best if the units are millimetres or
thereabouts and the offsets from zero are reasonably small.
Orientation and handedness are immaterial; x=east and y=north
is the recommended convention as it matches the run of $\alpha$ and
$\delta$.

\goodbreak
\begin{quote}
\begin{tabular}{|l|}
\hline
\verb|-5.103    58.868                      *  Candidate| \\
\verb|19 09 46.2  -63 51 27  J2000.0        *  Radio pos| \\
\hline
\end{tabular}
\end{quote}
These two records both specify {\it unknown stars}.  The
first is \xy, from which \radec\ will be determined.
The second is \radec\ and equinox, from which
\xy\ will be determined.
It is not necessary to include any
unknown star records at all, if the intention is
simply to determine a plate scale or to check
a set of plate measurements.

The above example gives a position for the first unknown
star of `\verb|19 05 01.794 -63 56 16.70|'.  The equinox was
specified in the results equinox record, and the epoch in
the plate data record.  One way to express this
information in a publication might be as follows:

\begin{center}
19~05~01.79~~-63~56~16.7~~B1950~~epoch~1974.5
\end{center}

The example does not include the
optional {\bf observation data} and {\bf colour}
records, which enable ASTROM to
reconstruct the precise appearance of the field rather
than allowing various predictable rotations and
distortions to be absorbed into the fit.  Also omitted from
the example are requests to include the radial distortion
and plate centre in the fit.  Details of
these refinements are given in Sections~6 and~7.

\section{Method}
For each input sequence, up to three astrometric solutions are
reported. The first is a four coefficient linear model (zero
points, scale and orientation),
requiring at least two reference stars.  The second, computed
in addition to the 4-coefficient model if there are at least
three reference stars, is a six coefficient linear model (zero
points, scales in $x$ and $y$, orientation and 
nonperpendicularity).  The third solution, which is performed on
request and providing at least 10 reference stars have been
supplied, has 7-9 coefficients and includes in the model the radial
distortion coefficient and/or the plate centre,
along with the six linear terms.

The 4-coefficient model
is useful (1)~for rough and ready astrometry, {\it e.g.}\ from a
print using a ruler or graph paper, and (2)~for identifying
an erroneous reference star, the higher order fits
tending to disguise the error.  On most occasions, the 6-coefficient
solution will be the most useful.

Internally, the modelling is done in idealized ``plate
coordinates'', and the various \radec\ and
\xy\ data input to or output from ASTROM are converted
to and from this internal standard as required.
The conversion
from \radec\ to plate coordinates consists of the following steps:
\begin{enumerate}
 \item Appropriate operations to transform the supplied
       \radec\ into either observed coordinates (if the
       optional observation data have been provided) or
       mean coordinates at the plate epoch (if not).
 \item Conventional gnomonic projection, using the given
       plate centre \radec, to obtain tangential coordinates
       \xieta.
 \item A small adjustment to allow for departures from
       tangent-plane geometry.
\end{enumerate}
The distortion model in step 3 is the usual ``cubic'' one, where
the vector from the plate centre to the star image is
lengthened by an amount
proportional to the cube of the length of this vector.
The adjustment is carried out by multiplying each of
$\xi$ and $\eta$ by the factor $(1 + q (\xi^{2}+\eta^{2}))$,
the coefficient $q$ depending on the telescope type specified.  The
values for each telescope type are given in the following table:

\goodbreak
\begin{center}
\begin{tabular}{|c|l|c|}
\hline
{\it telescope type} & {\it description} & $q$ \\
\hline
$'${\tt ASTR}$\,'$ & astrograph & zero \\
$'${\tt SCHM}$\,'$ & Schmidt & $-1/3$ \\
$'${\tt AAT2}$\,'$ & AAT PF doublet & +147.1 \\
$'${\tt AAT3}$\,'$ & AAT PF triplet & +178.6 \\
$'${\tt AAT8}$\,'$ & AAT $f/8$ & +21.2 \\
$'${\tt JKT8}$\,'$ & JKT $f/8$ & +14.7 \\
$'${\tt GENE}$\,'$ & general & specified \\
\hline
\end{tabular}
\end{center}

\goodbreak
Notes:
\begin{itemize}
 \item Positive $q$ values correspond to pincushion distortion,
       negative to barrel distortion.
 \item In the case of telescope type $'${\tt GENE}$\,'$
       (generalized pincushion/barrel distortion), $q$ is specified
       directly as a numeric parameter, and therefore can be used for
       any telescope or camera which is adequately described by
       the distortion model.
 \item The difference between the Schmidt and tangent-plane
       projections is conventionally assumed to be that between $r$
       and $\tan r$; ASTROM's $q = -1/3$ is equivalent to
       making the approximation $\tan r \simeq r + r^3/3$.
 \item The coefficient $q$ can, optionally, be determined automatically.
\end{itemize}

For the 4- and 6-coefficient linear models, the fitting process
consists of finding a set of coefficients which transform
the measured reference star \xy\ data into plate coordinates
which approximate those calculated from the \radec\ data.
For the 7-9 coefficient solutions, revised estimates of
the plate centre \radec\ and/or radial distortion
coefficient $q$ are made as well.

The models relate the following three types of coordinate:
\begin{itemize}
 \item The {\it estimated}\, coordinates $[x_{e},y_{e}]$, derived
       from the reference star \radec\
       by gnomonic projection and the application of radial
       distortion, using the current estimates of the plate centre
       and radial distortion coefficient.
 \item The {\it measured}\, coordinates $[x_{m},y_{m}]$, as supplied.
 \item The {\it predicted}\, coordinates $[x_{p},y_{p}]$, derived by the
       application of the current linear model to the measured coordinates.
\end{itemize}

Two varieties of {\bf 4-coefficient linear model} are tried, one
the mirror-image of the other.  The {\it standard}\, model is:
\begin{eqnarray*}
x_{e} & \simeq & a_{1} + a_{2} x_{m} + a_{3} y_{m} \\
y_{e} & \simeq & b_{1} - a_{3} x_{m} + a_{2} y_{m}
\end{eqnarray*}
The {\it laterally inverted}\, model is:
\begin{eqnarray*}
x_{e} & \simeq & a_{1} + a_{2} x_{m} + a_{3} y_{m} \\
y_{e} & \simeq & b_{1} + a_{3} x_{m} - a_{2} y_{m}
\end{eqnarray*}
The one delivering the smallest RMS error is selected.  If only
two reference stars have been supplied, the standard model is
used.

The {\bf 6-coefficient linear model} is as follows:
\begin{eqnarray*}
x_{e} & \simeq & a_{1} + a_{2} x_{m} + a_{3} y_{m} \\
y_{e} & \simeq & b_{1} + b_{2} x_{m} + b_{3} y_{m}
\end{eqnarray*}

Instead of the coefficients $a_{n},b_{n}$ being found
directly, the fits are, in fact, implemented in terms of
corrections
$\Delta a_{n},\Delta b_{n}$ to assumed approximate values of
$a_{n},b_{n}$.  For example, the 6-coefficient model is fitted as:
\begin{eqnarray*}
x_{e} - x_{p} & \simeq & \Delta a_{1}
          + \Delta a_{2} x_{m} + \Delta a_{3} y_{m} \\
y_{e} - y_{p} &\simeq & \Delta b_{1}
          + \Delta b_{2} x_{m} + \Delta b_{3} y_{m}
\end{eqnarray*}

When determining the {\bf plate centre}, the following extra non-linear
terms are added to the basic 6-coefficient linear model:
\begin{eqnarray*}
x_{e} - x_{p} & \simeq & \cdots + p_{1} (x_{p}^{2} + q (3 x_{p}^{2} + y_{p}^{2}))
                    + p_{2} (x_{p}y_{p} + q (2 x_{p}y_{p})) \\
y_{e} - y_{p} & \simeq & \cdots + p_{1} (x_{p}y_{p} + q (2 x_{p}y_{p}))
                    + p_{2} (y_{p}^{2} + q (x_{p}^{2} + 3 y_{p}^{2}))
\end{eqnarray*}

The coefficients $p_{1}$ and $p_{2}$ estimate the offset between
the pole of projection and the current $[x_{p},y_{p}]$ origin.  This
offset is used to improve the plate centre \radec\ (and
to correct the zero point $[a_{1},b_{1}]$) prior to recomputing
$[x_{p},y_{p}]$ for each reference star.

When determining the {\bf radial distortion coefficient}, the
following extra
terms are added:
\begin{eqnarray*}
x_{e} - x_{p} & \simeq & \cdots - \Delta q (x_{p}^{2} + y_{p}^{2}) x_{p} \\
y_{e} - y_{p} & \simeq & \cdots - \Delta q (x_{p}^{2} + y_{p}^{2}) y_{p}
\end{eqnarray*}

The $\Delta q$ obtained from the fit
is added to the current $q$ to provide a better estimate.

The above expressions are similar to those derived by Murray in sections
8.3.1{\it ff}\, of {\it Vectorial Astrometry}\, (Adam~Hilger, 1983).  The
main difference is that in ASTROM the centres of the gnomonic projection
and cubic distortion are assumed to be coincident.

All three types of solution are found by the iterative application of a
least-squares
algorithm based on {\it singular value decomposition}\, of the {\it design
matrix}.  (See sections 2.9 and 14.3 of {\it Numerical Recipes},
Press {\it et al.}, Cambridge University Press, 1986.)
This algorithm gives identical results to the traditional
{\it normal equations}\,
approach, but copes better with the ill-conditioned character of the
7-9 coefficient model.
The fit minimizes $\Sigma ((x_{e}-x_{p})^{2}+(y_{e}-y_{p})^{2})$.
Each reference star thus produces two rows of design matrix -- one for
$x$ and one for $y$.

In the case of the 4- and 6-coefficient linear models, a single iteration
is, in principle, all that is needed, whatever the starting values for
the coefficients.  However, a second iteration is performed in order to
minimize rounding errors.

The 7-9 coefficient models are highly nonlinear, with adjustments
of plate centre and -- especially -- radial distortion producing
large changes in the scales and zero points which depend on the
distribution of reference stars.  To ensure convergence, given
reasonable starting values for the plate centre and radial
distortion coefficient, the following strategy is used:
\begin{itemize}
 \item ASTROM insists that at least 10 reference stars be available
       if a non-linear fit is to be attempted.
 \item A fixed and ample number of iterations is used -- currently 20.
 \item Iterations which fit the plate centre and/or the distortion
       alternate with ones containing the six linear terms alone.
       The final iteration is the linear model.
 \item Where fitting of both the plate centre and the distortion
       has been requested, for the first few iterations only one
       or other of these is included in the fit, in alternation.
       Once reliable estimates of each have been obtained the full
       model is fitted at once.
\end{itemize}

\section{Limitations}
ASTROM aims to deliver results better than 1~arcsec from typical
Schmidt plate measurements, and better than 0.1~arcsec from
carefully measured
JKT and AAT plates {\it etc}.  Astrometric specialists will, nonetheless,
be aware of a number of shortcomings, including the following:
\begin{itemize}
 \item The fit is limited to a 6-coefficient linear model plus
       cubic distortion and plate tilt.  Colour effects -- arising for example
       from chromatic aberrations in the camera optics -- are not allowed
       for, no magnitude or image shape terms are included in the model,
       and the refraction cannot be adjusted automatically.
 \item The zonal distortions of the reference catalogues are neglected.
 \item There is no provision for the simultaneous fitting of
       more than one plate.  This prevents an extended area being
       modelled via overlapping plates, and the determination
       of proper motion and parallax from plates
       taken at different epochs.
 \item Only rudimentary error information is produced.
\end{itemize}
Despite these limitations, which stem mainly from the need for
simplicity of use, the accuracy of the result tends in practice
to be dominated by the quality of the input data rather
than by ASTROM itself.

\section{Reduction in Observed Place}
Normally, the ASTROM reduction is carried out
(internally) using {\it mean}\, places for the
epoch of the plate --
positions corrected for precession, but not for
nutation, aberration, deflection, and refraction, the effects of
which are simply absorbed into the fit.  This approach
keeps the input file simple, and delivers perfectly
adequate results for most practical purposes.
However, there are some occasions on which a more precise reduction
may be worthwhile.

Although the nutation, aberration
and deflection are always relatively innocuous -- the nutation produces a
small and harmless rotation, the aberration varies very slowly
across the sky, and the deflection is tiny except close to the
Sun -- the effects of atmospheric refraction can be quite
important.  As far as ASTROM is concerned, the refraction has
two aspects:
\begin{itemize}
 \item {\it Differential refraction}\, causes the picture to be
       distorted.  The distortion is in the form of a non-linear scale
       reduction in the vertical direction, the reduction being
       larger near the bottom;  this cannot be fully corrected by
       ASTROM's linear model.
 \item {\it Atmospheric dispersion}, important for detectors of
       wide spectral coverage, causes the images
       of stars of different colour to appear shifted vertically
       from their nominal positions.
\end{itemize}
Both these effects can be eliminated if the
optional time, observatory, meteorological and colour
records are included in the input file.  The advantages
of bothering to do this are as follows:
\begin{itemize}
 \item A more accurate result.  The improvement is likely to be
       modest in most instances, but may be significant for low
       elevations and wide fields.
 \item More nearly equal scales will be reported in x and y,
       and will be constant all over the sky.  Apart from
       providing additional reassurance that the fit is good,
       accurate knowledge of the scales is clearly vital
       if the optical parameters of the
       telescope are being measured for later use in
       predictions for guide stars or fibre feeds {\it etc}.
 \item The effects of colour may be important and can
       at least be quantified.
\end{itemize}
The optional records all begin with an explicit identifier, of
which only the first character (T, O, M or C) is significant.
They must immediately follow the plate data record, but
can be in any order.   Here is an example:
\begin{quote}
\begin{tabular}{|l|}
\hline
\verb|Time 1984 01 20  16 00| \\
\verb|Obs 149 04.0  -31 16.6  1164| \\
\verb|Met 288 899| \\
\verb|Col 600| \\
\hline
\end{tabular}
\end{quote}
If the time, observatory and meteorological records are all
omitted, any colour records
subsequently encountered will be ignored.  In the absence
of full information, ASTROM makes plausible guesses
to make good the deficiencies.
If insufficient information for the
observed place predictions is available, warnings are issued and
the astrometry is done using mean place.
If any of the three observation data records appears twice,
the new information supplants the old, and no error is
reported.

The TIME record can either specify the UT date and time
or, optionally, the local sidereal time, for mid-exposure.
If the time record specifies
a UT, and an epoch is specified on the plate data record,
the latter is ignored.
If the ST option is used, the epoch on the plate data
record must be specified (and should be accurate to a day
or two if the annual aberration and solar deflection are
to be correctly computed).  In the absence of a
UT, it is reasonable to guess that the exposure occurred
near upper culmination, which simply requires the ST to
be set equal to the plate centre $\alpha$.  For example,
to perform an observed place reduction on a plate
of a field at $\alpha=19^{h}13^{m}$, the following
TIME record might be used:
\begin{quote}
\begin{tabular}{|l|}
\hline
\verb|Time 19 13   * Estimated LST| \\
\hline
\end{tabular}
\end{quote}
The OBSERVATORY record can either specify one of the
observatory identifiers recognized by the SLALIB routine
OBS (see SUN/67):
\begin{quote}
\begin{tabular}{|l|}
\hline
\verb|Obs AAT| \\
\hline
\end{tabular}
\end{quote}
or the observatory position can be given
explicitly as in the example given earlier.
If the TIME record specifies sidereal time, the observatory
longitude may optionally be omitted.
The height (metres
above sea level) is of limited importance unless the
meteorological record is absent, in which case the height
is used to estimate the pressure.

The METEOROLOGICAL record specifies the temperature and pressure
at the telescope, in $^\circ$K and mB respectively.  The
temperature defaults to $278^\circ$K; the default pressure
is computed from the observatory height.

COLOUR records can appear anywhere after the time,
observatory and meteorological records, except between a
pair of reference star records.  Here is an example:
\begin{quote}
\begin{tabular}{|l|}
\hline
\verb|Colour 550| \\
\hline
\end{tabular}
\end{quote}
The effective
wavelength specified by such a record applies to all
stars from that point onwards.
Should two colour records
follow consecutively, the second supplants the first, and
no error is reported.
Prior to the first colour record,
a default of 500nm is assumed.
Appendix~C contains rough
estimates
of the effective wavelength for sources of different
colour temperature and detectors of different passband.
For the photographic case, the following
table (compiled with the help of D.\,Malin) suggests effective
wavelengths for some common combinations of emulsion, filter and
star colour;  the {\it blue}\, and {\it red}\, columns refer
to very blue and very
red (thermal) sources respectively (the effects may, of
course, be more extreme for emission-line objects and other
non-blackbody sources).  For a star of spectral type G0, the
effective wavelength will lie about halfway between the
{\it blue}\, and {\it red}\, figures.

\begin{center}
\begin{tabular}{|c|c|c||c|c|}
\hline
{\it band} & {\it emulsion} & {\it filter} & {\it blue} & {\it red} \\
\hline \hline
U & O & UG\,1 & 365 & 365 \\
  & J &       &     &     \\
\hline
B & IIa\,O & GG\,385 & 410 & 420 \\
  &        & GG\,395 &     &     \\
\cline{2-5}
  & IIa\,J & GG\,385 & 410 & 480 \\
  &        & GG\,395 &     &     \\
\hline
V & IIa\,D & GG\,495 & 550 & 600 \\
\hline
R & IIIa\,F & RG\,610 & 675 & 675 \\
  & 103a\,E & RG\,630 &     &     \\
\cline{3-5}
  & 098-04 & GG\,495 & 600 & 675 \\
\hline
I & IV-N & GG\,695 & 800 & 800 \\
\hline
\end{tabular}
\end{center}

It must again be pointed out that there may be other important
colour effects, apart from atmospheric dispersion, notably where
refracting optics have been used.  There is no attempt in ASTROM
to model such phenomena.

\section{Fitting Plate Centre and Radial Distortion}
Given a sufficient number of reference stars, measured to high
accuracy and evenly distributed over the whole plate, it is
possible to supplement the normal 4- and 6-coefficient linear
solutions with one in which the plate centre and/or the radial
distortion are determined automatically.

The option of fitting the plate centre ({\it i.e.}\ the \radec\ of
the centre of projection, which {\it a priori}\, may well not be known
to adequate accuracy) is selected simply by beginning the
plate centre
record with the tilde character $'$\verb|~|$\,'$ (meaning ``approximately''):

\goodbreak
\begin{quote}
\begin{tabular}{|l|}
\hline
\verb|~ 12 53 00.0  -42 00 00  B1950.0  1974.5  * Approx plate centre| \\
\hline
\end{tabular}
\end{quote}
Even though the plate centre is to be adjusted, it is advisable to start
off with the best available estimate.  The difference between this
and the actual centre of the projection pattern is what textbooks
refer to as {\it tilt}.  Determination of the tilt is most secure where
the radial distortion is pronounced.  Schmidt astrometry is relatively
insensitive to tilt, and attempting to fit the plate centre may be
unwise unless the reference stars are numerous and well-distributed.

A further option (intended for investigating the properties of
previously unmodelled telescopes rather than for routine use), is to
fit the radial distortion
coefficient.  This is selected by prefixing the telescope type
record with the tilde character $'$\verb|~|$\,'$:
\goodbreak
\begin{quote}
\begin{tabular}{|l|}
\hline
\verb|~ AAT3                * Guess| \\
\hline
\end{tabular}
\end{quote}
Tilt and distortion may be fitted simultaneously.
Neither adjustment will be attempted unless at least 10
reference stars are supplied.  If the fit proves to be
unacceptably ill-conditioned, or if
the adjustments are unrealistically large, the fit is rejected.

Though no check is made, it is clearly unwise to request that the
tilt and distortion be included in the model if the reduction is
not taking place in {\it observed}\, coordinates (see the previous
section).

\section{Parallax}
As described in Section~3, reference star celestial positions
may be expressed in either of two formats.  The most usual format includes
proper motions and has an optional epoch which defaults to that
of the equinox.  The other common format, used for reference stars whose
proper motions are assumed to be zero in an inertial frame,
has no proper motions and must have an epoch as well as an equinox.

The first format ({\it i.e.}\ with proper motions) has the
supplementary option of allowing the annual parallax to be specified,
following or instead of the epoch.  Here are two fictitious
reference star \radec\ records each of which includes parallax:

\goodbreak
\begin{quote}
\begin{tabular}{|l|}
\hline
\verb|14 39 36.087  -60 50 07.14  -0.49486 +0.6960  J2000.0  0.752 * Ref 1| \\
\verb|09 16 19.03 -10 52 23.2 -0.0401 -0.006 B1950.0 1978.9 0.032 * Ref 2| \\
\hline
\end{tabular}
\end{quote}

In the case where the parallax is
supplied without an epoch, which of the two is meant is deduced from
the size of the number given.

In the case where an epoch is supplied as well as well as a parallax,
it is assumed that the parallax has yet to be applied.  In other words,
the option to have the parallax removed from a reference
star at the given catalogue epoch and then put back in for
the epoch of the plate is {\bf not} provided.  The parallax is
only taken into account (except for second-order effects on
the proper motion) when a reduction in observed place
has been requested (by supplying observatory, time and
refraction information -- see Section~6).  Note that no
provision is made to specify the radial velocity of a
reference star.  This would only matter in cases where the
plate epoch was very distant from the reference star epoch
and where both radial velocity and parallax were large.

No provision exists in ASTROM for specifying the parallax (or proper motion)
of the unknown stars.

\pagebreak
{\bf APPENDIX A -- The Input File} \\
\vspace{3mm}

This appendix gives a more formal and complete specification of
the input file than is given in the main text, and concludes
with a more comprehensive example.
\vspace{3mm}

{\bf INPUT FILE}:

\begin{quote}
 SEQUENCE \\
 $'$\verb|/|$\,'$ \\
 SEQUENCE \\
 $'$\verb|/|$\,'$ \\
 $\vdots$ \\
 SEQUENCE \\
 $'$\verb|END|$\,'$ \\
 Blank records are ignored.
 Any record may end in a comment, which begins with an asterisk.
 Where records contain numbers these are free-format,
 decoded by the DBJIN and DFLTIN routines
 in SLALIB (see SUN/67), separated
 by spaces or a single comma.
 Lowercase and uppercase can be freely mixed.
 Any number of sequences is permitted.
\end{quote}

\goodbreak
{\bf SEQUENCE}:

\begin{quote}
 RESULTS EQUINOX RECORD, optional \\
 TELESCOPE TYPE RECORD, optional \\
 PLATE DATA RECORD \\
 OBSERVATION DATA, 0-3 records \\
 REFERENCE STARS, 2-3 per star for 2-1000 stars \\
 UNKNOWN STARS, 1-2 per star for any number of stars
\end{quote}

\goodbreak
{\bf OBSERVATION DATA}:

\begin{quote}
 TIME RECORD \\
 OBSERVATORY RECORD \\
 METEOROLOGICAL RECORD \\
 (Any order; any selection; repeats harmless.)
\end{quote}

\goodbreak
{\bf REFERENCE STAR}:

\begin{quote}
 COLOUR RECORD (optional; repeats harmless) \\
 REFERENCE STAR RA,DEC RECORD \\
 REFERENCE STAR X,Y RECORD \\
 A colour record applies to all subsequent stars,
 both reference and unknown.
\end{quote}

\goodbreak
{\bf UNKNOWN STAR}

\begin{quote}
 COLOUR RECORD (optional; repeats harmless) \\
 {\it and/or:} \\
 UNKNOWN STAR X,Y RECORD \\
 {\it or:} \\
 UNKNOWN STAR RA,DEC RECORD
\end{quote}

\goodbreak
{\bf RESULTS EQUINOX RECORD}:

\begin{quote}
 EQUINOX \\
 In the absence of this record, J2000.0 is used.
\end{quote}

\goodbreak
{\bf TELESCOPE TYPE RECORD}:

\begin{quote}
 [APPROX] PROJECTION \\
 In the absence of this record, $'$\verb|SCHM|$\,'$ is assumed.
\end{quote}

\goodbreak
{\bf PLATE DATA RECORD}:

\begin{quote}
 [APPROX] RA DEC EQUINOX [EPOCH] \\
 The position specified is that of the plate centre.
 The epoch is optional only if the information is supplied
 later in a time record (see next item).
\end{quote}

\goodbreak
{\bf TIME RECORD}:

\begin{quote}
 {\it Either:} \\
 $'$\verb|T...|$\,'$ UT \\
 {\it or:} \\
 $'$\verb|T...|$\,'$ ST \\
 The first of these forms allows the epoch to be omitted from
 the plate data record.
\end{quote}

\goodbreak
{\bf OBSERVATORY RECORD}:

\begin{quote}
 {\it Either:} \\
 $'$\verb|O...|$\,'$ STATION \\
 {\it or:} \\
 $'$\verb|O...|$\,'$ [LONGITUDE] LATITUDE [HEIGHT] \\
 The longitude may be omitted only if the sidereal
 time has been or will be specified via a time record.
 The height defaults to an estimate based on the air pressure.
\end{quote}

\goodbreak
{\bf METEOROLOGICAL RECORD}:

\begin{quote}
 $'$\verb|M...|$\,'$ TEMPERATURE [PRESSURE] \\
 The temperature defaults to $278^\circ$K.
 The pressure defaults to an estimate based on the height.
\end{quote}

\goodbreak
{\bf COLOUR RECORD}:

\begin{quote}
 $'$\verb|C...|$\,'$ WAVELENGTH \\
 A wavelength, once specified, applies to all stars from then on.
 The starting default is 500~nm.
\end{quote}

\goodbreak
{\bf REFERENCE STAR RA,DEC RECORD}:

\begin{quote}
 {\it Either:} \\
 RA DEC PMR PMD EQUINOX [EPOCH] [PARALLAX] [NAME] \\
 {\it or:} \\
 RA DEC EQUINOX EPOCH [NAME] \\
 The second format implies inertially zero proper motion.
\end{quote}

\goodbreak
{\bf REFERENCE STAR X,Y RECORD}:

\begin{quote}
 X Y
\end{quote}

\goodbreak
{\bf UNKNOWN STAR X,Y RECORD}:

\begin{quote}
 X Y [NAME]
\end{quote}

\goodbreak
{\bf UNKNOWN STAR RA,DEC RECORD}:

\begin{quote}
 RA DEC EQUINOX [NAME]
\end{quote}

\goodbreak
{\bf EQUINOX}:

\begin{quote}
The epoch of the equator and equinox of a mean \radec\
coordinate system.  A Besselian epoch implies the pre~IAU~1976
system (as used in the FK4 catalogue) and a Julian epoch implies
the post~IAU~1976 system (as used in the FK5 catalogue).
\end{quote}

\goodbreak
{\bf APPROX}:

\begin{quote}
 $'$\verb|~|$\,'$ (tilde) \\
At the start of the telescope type and plate data
records, this specifies that the radial
distortion and plate centre, respectively, are to be fitted.
\end{quote}

\goodbreak
{\bf PROJECTION}:

\begin{quote}
 $'$\verb|ASTR...|$\,'$ = astrograph \\
 {\it or} \\
 $'$\verb|SCHM...|$\,'$ = Schmidt (default) \\
 {\it or} \\
 $'$\verb|AAT2...|$\,'$ = AAT prime focus doublet \\
 {\it or} \\
 $'$\verb|AAT3...|$\,'$ = AAT prime focus triplet \\
 {\it or} \\
 $'$\verb|AAT8...|$\,'$ = AAT $f/8$ with vacuum plateholder \\
 {\it or} \\
 $'$\verb|JKT8...|$\,'$ = JKT $f/8$ Harmer-Wynne \\
 {\it or} \\
 $'$\verb|GENE...|$\,'$ DISTORTION = generalized pincushion/barrel distortion
\end{quote}

\goodbreak
{\bf DISTORTION}:

\begin{quote}
A number, the parameter $q$ in the pincushion/barrel
distortion expression
$\mbox{\boldmath $r'$}=(1+q\mid\mbox{\boldmath $r$}\mid^2)\mbox{\boldmath $r$}$,
where \mbox{\boldmath $r'$} is the radial vector to the star image
from the intersection of the optical axis and the plate,
and \mbox{\boldmath $r$} is the same vector but assuming tangent-plane
geometry.
The vectors are in units of one focal length.
\end{quote}

\goodbreak
{\bf EPOCH}:

\begin{quote}
A Besselian or Julian epoch:  a single number resembling
years AD,
optionally preceded by $'${\tt B}$\,'$ (for
{\it Besselian}) or $'${\tt J}$\,'$ (for {\it Julian}).  In the
absence of a prefix, epochs before 1984.0 are assumed to
be in the Besselian timescale, and epochs from 1984.0 onwards
are assumed to be in the Julian timescale.
\end{quote}

\goodbreak
{\bf RA DEC}:

\begin{quote}
A mean \radec\, expressed as six numbers: hours,
minutes, seconds, degrees, arcminutes, arcseconds.
The seconds and arcseconds can be given to any reasonable
precision;  the others must be integers.
All the numbers except the degrees
must be positive; southern $\delta$ is indicated by
minus degrees (even if zero).
\end{quote}

\goodbreak
{\bf UT}:

\begin{quote}
The UT epoch of observation expressed as
six numbers: years AD, month, day, hours, minutes.
All but the minutes must be integers.
The year, month and day must form a valid date
in the Gregorian calendar.
\end{quote}

\goodbreak
{\bf ST}:

\begin{quote}
The local (apparent) sidereal time of observation expressed as
two numbers: hours, minutes.
The hours must be an integer.
\end{quote}

\goodbreak
{\bf STATION}

\begin{quote}
 A character string specifying one of the observatories
 supported by the OBS routine in SLALIB (see SUN/67).
\end{quote}

\goodbreak
{\bf LONGITUDE}

\begin{quote}
 The east longitude, expressed as two numbers: degrees (which
 must be an integer) and arcminutes.  West longitudes may
 be indicated
 either by minus degrees (even if zero) or by east
 longitude $> 180^\circ$.
\end{quote}

\goodbreak
{\bf LATITUDE}

\begin{quote}
 The (geodetic) latitude, expressed as two numbers: degrees (which
 must be an integer) and arcminutes.  South latitude is indicated
 by minus degrees (even if zero).
\end{quote}

\goodbreak
{\bf HEIGHT}

\begin{quote}
 A single number, the height above sea level in metres.
\end{quote}

\goodbreak
{\bf TEMPERATURE}

\begin{quote}
 A single number, the ambient temperature in $^\circ$K.
\end{quote}

\goodbreak
{\bf PRESSURE}

\begin{quote}
 A single number, the pressure in mB.
\end{quote}

\goodbreak
{\bf WAVELENGTH}

\begin{quote}
 A single number, the effective wavelength in nm.
\end{quote}

\goodbreak
{\bf PMR PMD}

\begin{quote}
 Two numbers, the proper motions in seconds and arcseconds
 per year respectively.
\end{quote}

\goodbreak
{\bf PARALLAX}

\begin{quote}
 A single number, the annual parallax in arcseconds.
\end{quote}

\goodbreak
{\bf NAME}

\begin{quote}
 The name field, which is always optional,
 is simply the first 10 characters of
 the comment, excluding the $'$\verb|*|$\,'$ and any leading spaces.
\end{quote}

\goodbreak
{\bf X Y}

\begin{quote}
 Two numbers, the Cartesian coordinates on the plate;
 the units of $x$ and $y$ should preferably be the same and in the
 $\mu$m to m range ({\it e.g.}\ mm), and the zero points
 should not be too far from the region of measurement.
\end{quote}


The example input file which appears on the next page
includes two sequences.
The first is for a typical
run using measurements from a Schmidt plate, while the second
is for precise reduction, in observed place, of AAT~$f/8$ measurements,
including automatic determination of the radial distortion and
plate centre.
For brevity, neither sequence includes as many reference stars as
would normally be advisable.

\pagebreak
\begin{center}
\begin{small}
\begin{tabular}{|l|}
\hline
\verb|B1950                                     * Results in FK4| \\
\verb|SCHM                                      * Schmidt geometry| \\
\verb|19 04 00.0  -65 00 00  B1950.0  1974.5    * Plate centre, and epoch| \\
\verb|18 56 39.426  -63 25 13.23  -0.0002  -0.036  B1950.0  * Ref 1| \\
\verb|44.791   85.643| \\
\\
\verb|19 11 53.909  -63 17 57.57   0.0058  -0.044  1950.0   * Ref 2| \\
\verb|-46.266   92.337| \\
\\
\verb|19 01 13.606  -63 49 14.84   0.0020  -0.026  1950.0   * Ref 3| \\
\verb|17.246   64.945| \\
\\
\verb|19 08 29.088  -63 57 42.79   0.0016   0.018  1950.0   * Ref 4| \\
\verb|-25.314   57.456| \\
\\
\verb|19 02 10.088  -63 29 16.73   0.0012  -0.019  1950.0   * Ref 5| \\
\verb|11.890   82.766| \\
\\
\verb|-5.103    58.868                      *  Candidate| \\
\verb|19 09 46.2  -63 51 27  J2000.0        *  Radio pos| \\
\\
\verb|/                                     *  End of first sequence| \\
\\
\verb|* AAT plate 2266 (f/8 RC)  NGC 3114| \\
\verb|~ AAT8    * To be fitted| \\
\verb|~ 10 01 00.0 -59 53 01 B1950   * To be fitted| \\
\verb|Time 1984 01 20  16 00| \\
\verb|Obs AAT| \\
\verb|Met 288 899| \\
\verb|Colour 450    *  Default colour for reference stars| \\
\verb|10 01 21.203 -59 52 14.05 B1950 J1984.1| \\
\verb|9.0353 18.4211 *130| \\
\verb|10 00 16.401 -59 52 52.16 B1950 J1984.1| \\
\verb|1.7304 17.9282 *70| \\
\verb|10 00 18.516 -59 53 10.20 B1950 J1984.1| \\
\verb|1.9669 17.6566 *73| \\
\verb|10 00 19.620 -59 49 01.62 B1950 J1984.1| \\
\verb|2.1223 21.3760 *74| \\
\verb|10 00 20.525 -59 52 01.09 B1950 J1984.1| \\
\verb|2.2025 18.6888 *75| \\
\verb|10 00 21.416 -59 51 30.27 B1950 J1984.1| \\
\verb|2.3067 19.1501 *76| \\
\verb|10 00 22.896 -59 53 49.60 B1950 J1984.1| \\
\verb|2.4544 17.0626 *80| \\
\verb|10 01 26.159 -59 50 38.50 B1950 J1984.1| \\
\verb|9.6143 19.8435 *134| \\
\verb|10 01 28.328 -59 51 16.86 B1950 J1984.1| \\
\verb|9.8509 19.2653 *138| \\
\verb|10 01 54.446 -59 54 39.28 B1950 J1984.1| \\
\verb|12.7495 16.1963 *156| \\
\verb|10 01 54.523 -59 50 01.72 B1950 J1984.1| \\
\verb|12.8193 20.3493 *157| \\
\verb|10 01 57.438 -59 51 26.29 B1950 J1984.1| \\
\verb|13.1292 19.0793 *161| \\
\verb|10 00 12.385 -60 08 08.51 B1950 J1984.1| \\
\verb|1.1637 4.2188 *65| \\
\verb|C 500| \\
\verb|5.8265 12.7252 *104 red| \\
\verb|C 400| \\
\verb|5.8265 12.7252 *104 blue| \\
\\
\verb|END| \\
\hline
\end{tabular}
\end{small}
\end{center}

\pagebreak
{\bf APPENDIX B -- Example Report}

\vspace{5mm}
Here is an example of the report produced by ASTROM.  It is the
result of a run using the specimen input file presented in Section~3.
The first part of the report lists the raw data.  The second part gives
the results of the 4-coefficient solution.  The third part (next
page) gives the results of the 6-coefficient solution.  The
fourth part gives the predictions for the unknown stars.

\vspace{20mm}

%%%%%%%%%%%%%%%%%%%%%%%%%%%%%%%%%%%%%%%%%%%
% LaTeX source generated by TEXLP program %
%        P.T.Wallace     Starlink         %
%%%%%%%%%%%%%%%%%%%%%%%%%%%%%%%%%%%%%%%%%%%
\begin{tiny}

\noindent
\begin{picture}(159.90, 52.00)( -2.60, -4.00)
\put( -0.65, -2.50){\framebox(158.60, 48.00){}}
\put(0.16,42){$\ast$}
\put(2.76,42){$\ast$}
\put(5.36,42){$\ast$}
\put(7.96,42){$\ast$}
\put(10.56,42){$\ast$}
\put(13.16,42){$\ast$}
\put(15.76,42){$\ast$}
\put(18.36,42){$\ast$}
\put(20.96,42){$\ast$}
\put(23.56,42){$\ast$}
\put(26.16,42){$\ast$}
\put(28.76,42){$\ast$}
\put(31.36,42){$\ast$}
\put(0.16,40){$\ast$}
\put(3.70,40){A}
\put(6.55,40){S}
\put(8.94,40){T}
\put(11.50,40){R}
\put(14.08,40){O}
\put(16.51,40){M}
\put(19.40,40){E}
\put(21.94,40){T}
\put(24.50,40){R}
\put(27.07,40){Y}
\put(31.36,40){$\ast$}
\put(0.16,38){$\ast$}
\put(2.76,38){$\ast$}
\put(5.36,38){$\ast$}
\put(7.96,38){$\ast$}
\put(10.56,38){$\ast$}
\put(13.16,38){$\ast$}
\put(15.76,38){$\ast$}
\put(18.36,38){$\ast$}
\put(20.96,38){$\ast$}
\put(23.56,38){$\ast$}
\put(26.16,38){$\ast$}
\put(28.76,38){$\ast$}
\put(31.36,38){$\ast$}
\put(-0.10,30){E}
\put(1.33,30){q}
\put(2.61,30){u}
\put(4.19,30){i}
\put(5.20,30){n}
\put(6.60,30){o}
\put(7.83,30){x}
\put(10.57,30){f}
\put(11.80,30){o}
\put(13.22,30){r}
\put(15.31,30){m}
\put(17.03,30){e}
\put(18.28,30){a}
\put(19.50,30){n}
\put(22.23,30){c}
\put(23.50,30){o}
\put(24.80,30){o}
\put(26.22,30){r}
\put(27.31,30){d}
\put(28.89,30){i}
\put(29.90,30){n}
\put(31.28,30){a}
\put(32.75,30){t}
\put(33.93,30){e}
\put(35.31,30){s}
\put(37.80,30){o}
\put(39.17,30){f}
\put(41.82,30){r}
\put(43.03,30){e}
\put(44.41,30){s}
\put(45.51,30){u}
\put(47.12,30){l}
\put(48.35,30){t}
\put(49.61,30){s}
\put(51.06,30){:}
\put(54.47,30){B}
\put(56.02,30){1}
\put(57.32,30){9}
\put(58.62,30){5}
\put(59.92,30){0}
\put(61.48,30){.}
\put(62.52,30){0}
\put(-0.06,26){P}
\put(1.52,26){r}
\put(2.70,26){o}
\put(4.25,26){j}
\put(5.33,26){e}
\put(6.63,26){c}
\put(8.05,26){t}
\put(9.39,26){i}
\put(10.50,26){o}
\put(11.70,26){n}
\put(14.36,26){g}
\put(15.73,26){e}
\put(17,26){o}
\put(17.91,26){m}
\put(19.63,26){e}
\put(21.05,26){t}
\put(22.32,26){r}
\put(23.47,26){y}
\put(25.06,26){:}
\put(28.65,26){S}
\put(30.03,26){c}
\put(31.20,26){h}
\put(32.21,26){m}
\put(34.09,26){i}
\put(35.11,26){d}
\put(36.65,26){t}
\put(-0.06,22){P}
\put(1.63,22){l}
\put(2.68,22){a}
\put(4.15,22){t}
\put(5.33,22){e}
\put(7.93,22){c}
\put(9.23,22){e}
\put(10.40,22){n}
\put(11.95,22){t}
\put(13.22,22){r}
\put(14.43,22){e}
\put(15.96,22){:}
\put(19.62,22){1}
\put(20.92,22){9}
\put(23.52,22){0}
\put(24.82,22){4}
\put(27.42,22){0}
\put(28.72,22){0}
\put(30.28,22){.}
\put(31.32,22){0}
\put(36.37,22){$-$}
\put(37.82,22){6}
\put(39.12,22){5}
\put(41.72,22){0}
\put(43.02,22){0}
\put(45.62,22){0}
\put(46.92,22){0}
\put(54.50,22){E}
\put(55.93,22){q}
\put(57.21,22){u}
\put(58.79,22){i}
\put(59.80,22){n}
\put(61.20,22){o}
\put(62.43,22){x}
\put(64.87,22){B}
\put(66.42,22){1}
\put(67.72,22){9}
\put(69.02,22){5}
\put(70.32,22){0}
\put(71.88,22){.}
\put(72.92,22){0}
\put(80.50,22){E}
\put(81.93,22){p}
\put(83.30,22){o}
\put(84.63,22){c}
\put(85.80,22){h}
\put(88.27,22){B}
\put(89.82,22){1}
\put(91.12,22){9}
\put(92.42,22){7}
\put(93.72,22){4}
\put(95.28,22){.}
\put(96.32,22){5}
\put(97.62,22){0}
\put(98.92,22){0}
\put(-0.19,16){R}
\put(1.43,16){e}
\put(2.77,16){f}
\put(4.03,16){e}
\put(5.42,16){r}
\put(6.63,16){e}
\put(7.80,16){n}
\put(9.23,16){c}
\put(10.53,16){e}
\put(13.21,16){s}
\put(14.55,16){t}
\put(15.68,16){a}
\put(17.12,16){r}
\put(18.41,16){s}
\put(19.86,16){:}
\put(19.50,12){n}
\put(38.80,12){R}
\put(40.10,12){A}
\put(57.04,12){D}
\put(58.63,12){e}
\put(59.93,12){c}
\put(72.83,12){p}
\put(73.81,12){m}
\put(75.20,12){R}
\put(81.93,12){p}
\put(82.91,12){m}
\put(84.34,12){D}
\put(89.60,12){E}
\put(91.03,12){q}
\put(92.31,12){u}
\put(93.89,12){i}
\put(94.90,12){n}
\put(96.30,12){o}
\put(97.53,12){x}
\put(102.60,12){E}
\put(104.03,12){p}
\put(105.40,12){o}
\put(106.73,12){c}
\put(107.90,12){h}
\put(117.03,12){p}
\put(118.33,12){x}
\put(131.08,12){X}
\put(132.31,12){m}
\put(134.03,12){e}
\put(135.28,12){a}
\put(136.71,12){s}
\put(147.97,12){Y}
\put(149.21,12){m}
\put(150.93,12){e}
\put(152.18,12){a}
\put(153.61,12){s}
\put(-0.19,8){R}
\put(1.43,8){e}
\put(2.77,8){f}
\put(5.32,8){1}
\put(19.62,8){1}
\put(32.62,8){1}
\put(33.92,8){8}
\put(36.52,8){5}
\put(37.82,8){6}
\put(40.42,8){3}
\put(41.72,8){9}
\put(43.28,8){.}
\put(44.32,8){4}
\put(45.62,8){2}
\put(46.92,8){6}
\put(50.67,8){$-$}
\put(52.12,8){6}
\put(53.42,8){3}
\put(56.02,8){2}
\put(57.32,8){5}
\put(59.92,8){1}
\put(61.22,8){3}
\put(62.78,8){.}
\put(63.82,8){2}
\put(65.12,8){3}
\put(68.87,8){$-$}
\put(70.32,8){0}
\put(71.88,8){.}
\put(72.92,8){0}
\put(74.22,8){0}
\put(75.52,8){0}
\put(76.82,8){2}
\put(79.27,8){$-$}
\put(80.72,8){0}
\put(82.28,8){.}
\put(83.32,8){0}
\put(84.62,8){3}
\put(85.92,8){6}
\put(89.57,8){B}
\put(91.12,8){1}
\put(92.42,8){9}
\put(93.72,8){5}
\put(95.02,8){0}
\put(96.58,8){.}
\put(97.62,8){0}
\put(101.27,8){B}
\put(102.82,8){1}
\put(104.12,8){9}
\put(105.42,8){5}
\put(106.72,8){0}
\put(108.28,8){.}
\put(109.32,8){0}
\put(110.62,8){0}
\put(111.92,8){0}
\put(115.82,8){0}
\put(117.38,8){.}
\put(118.42,8){0}
\put(119.72,8){0}
\put(121.02,8){0}
\put(129.80,8){$+$}
\put(131.42,8){4}
\put(132.72,8){4}
\put(134.28,8){.}
\put(135.32,8){7}
\put(136.62,8){9}
\put(137.92,8){1}
\put(146.70,8){$+$}
\put(148.32,8){8}
\put(149.62,8){5}
\put(151.18,8){.}
\put(152.22,8){6}
\put(153.52,8){4}
\put(154.82,8){3}
\put(-0.19,6){R}
\put(1.43,6){e}
\put(2.77,6){f}
\put(5.32,6){2}
\put(19.62,6){2}
\put(32.62,6){1}
\put(33.92,6){9}
\put(36.52,6){1}
\put(37.82,6){1}
\put(40.42,6){5}
\put(41.72,6){3}
\put(43.28,6){.}
\put(44.32,6){9}
\put(45.62,6){0}
\put(46.92,6){9}
\put(50.67,6){$-$}
\put(52.12,6){6}
\put(53.42,6){3}
\put(56.02,6){1}
\put(57.32,6){7}
\put(59.92,6){5}
\put(61.22,6){7}
\put(62.78,6){.}
\put(63.82,6){5}
\put(65.12,6){7}
\put(68.70,6){$+$}
\put(70.32,6){0}
\put(71.88,6){.}
\put(72.92,6){0}
\put(74.22,6){0}
\put(75.52,6){5}
\put(76.82,6){8}
\put(79.27,6){$-$}
\put(80.72,6){0}
\put(82.28,6){.}
\put(83.32,6){0}
\put(84.62,6){4}
\put(85.92,6){4}
\put(89.57,6){B}
\put(91.12,6){1}
\put(92.42,6){9}
\put(93.72,6){5}
\put(95.02,6){0}
\put(96.58,6){.}
\put(97.62,6){0}
\put(101.27,6){B}
\put(102.82,6){1}
\put(104.12,6){9}
\put(105.42,6){5}
\put(106.72,6){0}
\put(108.28,6){.}
\put(109.32,6){0}
\put(110.62,6){0}
\put(111.92,6){0}
\put(115.82,6){0}
\put(117.38,6){.}
\put(118.42,6){0}
\put(119.72,6){0}
\put(121.02,6){0}
\put(129.97,6){$-$}
\put(131.42,6){4}
\put(132.72,6){6}
\put(134.28,6){.}
\put(135.32,6){2}
\put(136.62,6){6}
\put(137.92,6){6}
\put(146.70,6){$+$}
\put(148.32,6){9}
\put(149.62,6){2}
\put(151.18,6){.}
\put(152.22,6){3}
\put(153.52,6){3}
\put(154.82,6){7}
\put(-0.19,4){R}
\put(1.43,4){e}
\put(2.77,4){f}
\put(5.32,4){3}
\put(19.62,4){3}
\put(32.62,4){1}
\put(33.92,4){9}
\put(36.52,4){0}
\put(37.82,4){1}
\put(40.42,4){1}
\put(41.72,4){3}
\put(43.28,4){.}
\put(44.32,4){6}
\put(45.62,4){0}
\put(46.92,4){6}
\put(50.67,4){$-$}
\put(52.12,4){6}
\put(53.42,4){3}
\put(56.02,4){4}
\put(57.32,4){9}
\put(59.92,4){1}
\put(61.22,4){4}
\put(62.78,4){.}
\put(63.82,4){8}
\put(65.12,4){4}
\put(68.70,4){$+$}
\put(70.32,4){0}
\put(71.88,4){.}
\put(72.92,4){0}
\put(74.22,4){0}
\put(75.52,4){2}
\put(76.82,4){0}
\put(79.27,4){$-$}
\put(80.72,4){0}
\put(82.28,4){.}
\put(83.32,4){0}
\put(84.62,4){2}
\put(85.92,4){6}
\put(89.57,4){B}
\put(91.12,4){1}
\put(92.42,4){9}
\put(93.72,4){5}
\put(95.02,4){0}
\put(96.58,4){.}
\put(97.62,4){0}
\put(101.27,4){B}
\put(102.82,4){1}
\put(104.12,4){9}
\put(105.42,4){5}
\put(106.72,4){0}
\put(108.28,4){.}
\put(109.32,4){0}
\put(110.62,4){0}
\put(111.92,4){0}
\put(115.82,4){0}
\put(117.38,4){.}
\put(118.42,4){0}
\put(119.72,4){0}
\put(121.02,4){0}
\put(129.80,4){$+$}
\put(131.42,4){1}
\put(132.72,4){7}
\put(134.28,4){.}
\put(135.32,4){2}
\put(136.62,4){4}
\put(137.92,4){6}
\put(146.70,4){$+$}
\put(148.32,4){6}
\put(149.62,4){4}
\put(151.18,4){.}
\put(152.22,4){9}
\put(153.52,4){4}
\put(154.82,4){5}
\put(-0.19,2){R}
\put(1.43,2){e}
\put(2.77,2){f}
\put(5.32,2){4}
\put(19.62,2){4}
\put(32.62,2){1}
\put(33.92,2){9}
\put(36.52,2){0}
\put(37.82,2){8}
\put(40.42,2){2}
\put(41.72,2){9}
\put(43.28,2){.}
\put(44.32,2){0}
\put(45.62,2){8}
\put(46.92,2){8}
\put(50.67,2){$-$}
\put(52.12,2){6}
\put(53.42,2){3}
\put(56.02,2){5}
\put(57.32,2){7}
\put(59.92,2){4}
\put(61.22,2){2}
\put(62.78,2){.}
\put(63.82,2){7}
\put(65.12,2){9}
\put(68.70,2){$+$}
\put(70.32,2){0}
\put(71.88,2){.}
\put(72.92,2){0}
\put(74.22,2){0}
\put(75.52,2){1}
\put(76.82,2){6}
\put(79.10,2){$+$}
\put(80.72,2){0}
\put(82.28,2){.}
\put(83.32,2){0}
\put(84.62,2){1}
\put(85.92,2){8}
\put(89.57,2){B}
\put(91.12,2){1}
\put(92.42,2){9}
\put(93.72,2){5}
\put(95.02,2){0}
\put(96.58,2){.}
\put(97.62,2){0}
\put(101.27,2){B}
\put(102.82,2){1}
\put(104.12,2){9}
\put(105.42,2){5}
\put(106.72,2){0}
\put(108.28,2){.}
\put(109.32,2){0}
\put(110.62,2){0}
\put(111.92,2){0}
\put(115.82,2){0}
\put(117.38,2){.}
\put(118.42,2){0}
\put(119.72,2){0}
\put(121.02,2){0}
\put(129.97,2){$-$}
\put(131.42,2){2}
\put(132.72,2){5}
\put(134.28,2){.}
\put(135.32,2){3}
\put(136.62,2){1}
\put(137.92,2){4}
\put(146.70,2){$+$}
\put(148.32,2){5}
\put(149.62,2){7}
\put(151.18,2){.}
\put(152.22,2){4}
\put(153.52,2){5}
\put(154.82,2){6}
\put(-0.19,0){R}
\put(1.43,0){e}
\put(2.77,0){f}
\put(5.32,0){5}
\put(19.62,0){5}
\put(32.62,0){1}
\put(33.92,0){9}
\put(36.52,0){0}
\put(37.82,0){2}
\put(40.42,0){1}
\put(41.72,0){0}
\put(43.28,0){.}
\put(44.32,0){0}
\put(45.62,0){8}
\put(46.92,0){8}
\put(50.67,0){$-$}
\put(52.12,0){6}
\put(53.42,0){3}
\put(56.02,0){2}
\put(57.32,0){9}
\put(59.92,0){1}
\put(61.22,0){6}
\put(62.78,0){.}
\put(63.82,0){7}
\put(65.12,0){3}
\put(68.70,0){$+$}
\put(70.32,0){0}
\put(71.88,0){.}
\put(72.92,0){0}
\put(74.22,0){0}
\put(75.52,0){1}
\put(76.82,0){2}
\put(79.27,0){$-$}
\put(80.72,0){0}
\put(82.28,0){.}
\put(83.32,0){0}
\put(84.62,0){1}
\put(85.92,0){9}
\put(89.57,0){B}
\put(91.12,0){1}
\put(92.42,0){9}
\put(93.72,0){5}
\put(95.02,0){0}
\put(96.58,0){.}
\put(97.62,0){0}
\put(101.27,0){B}
\put(102.82,0){1}
\put(104.12,0){9}
\put(105.42,0){5}
\put(106.72,0){0}
\put(108.28,0){.}
\put(109.32,0){0}
\put(110.62,0){0}
\put(111.92,0){0}
\put(115.82,0){0}
\put(117.38,0){.}
\put(118.42,0){0}
\put(119.72,0){0}
\put(121.02,0){0}
\put(129.80,0){$+$}
\put(131.42,0){1}
\put(132.72,0){1}
\put(134.28,0){.}
\put(135.32,0){8}
\put(136.62,0){9}
\put(137.92,0){0}
\put(146.70,0){$+$}
\put(148.32,0){8}
\put(149.62,0){2}
\put(151.18,0){.}
\put(152.22,0){7}
\put(153.52,0){6}
\put(154.82,0){6}
\end{picture}

\noindent
\begin{picture}(159.90, 68.00)( -2.60, -4.00)
\put( -0.65, -2.50){\framebox(158.60, 64.00){}}
\put(-0.06,58){P}
\put(1.63,58){l}
\put(2.68,58){a}
\put(4.15,58){t}
\put(5.33,58){e}
\put(8.01,58){s}
\put(9.20,58){o}
\put(10.72,58){l}
\put(11.71,58){u}
\put(13.25,58){t}
\put(14.59,58){i}
\put(15.70,58){o}
\put(16.90,58){n}
\put(18.56,58){:}
\put(20.92,58){4}
\put(22.07,58){$-$}
\put(23.53,58){c}
\put(24.80,58){o}
\put(26.13,58){e}
\put(27.47,58){f}
\put(28.77,58){f}
\put(30.19,58){i}
\put(31.33,58){c}
\put(32.79,58){i}
\put(33.93,58){e}
\put(35.10,58){n}
\put(36.65,58){t}
\put(-0.03,56){$-$}
\put(1.27,56){$-$}
\put(2.57,56){$-$}
\put(3.87,56){$-$}
\put(5.17,56){$-$}
\put(6.47,56){$-$}
\put(7.77,56){$-$}
\put(9.07,56){$-$}
\put(10.37,56){$-$}
\put(11.67,56){$-$}
\put(12.97,56){$-$}
\put(14.27,56){$-$}
\put(15.57,56){$-$}
\put(16.87,56){$-$}
\put(18.17,56){$-$}
\put(19.47,56){$-$}
\put(20.77,56){$-$}
\put(22.07,56){$-$}
\put(23.37,56){$-$}
\put(24.67,56){$-$}
\put(25.97,56){$-$}
\put(27.27,56){$-$}
\put(28.57,56){$-$}
\put(29.87,56){$-$}
\put(31.17,56){$-$}
\put(32.47,56){$-$}
\put(33.77,56){$-$}
\put(35.07,56){$-$}
\put(36.37,56){$-$}
\put(4.98,52){X}
\put(6.79,52){,}
\put(7.57,52){Y}
\put(10.28,52){=}
\put(13.13,52){e}
\put(14.33,52){x}
\put(15.63,52){p}
\put(17.03,52){e}
\put(18.33,52){c}
\put(19.75,52){t}
\put(20.93,52){e}
\put(22.11,52){d}
\put(24.73,52){p}
\put(26.32,52){l}
\put(27.38,52){a}
\put(28.85,52){t}
\put(30.03,52){e}
\put(32.63,52){c}
\put(33.90,52){o}
\put(35.20,52){o}
\put(36.62,52){r}
\put(37.71,52){d}
\put(39.29,52){i}
\put(40.30,52){n}
\put(41.68,52){a}
\put(43.15,52){t}
\put(44.33,52){e}
\put(45.71,52){s}
\put(48.26,52){(}
\put(49.62,52){r}
\put(50.78,52){a}
\put(52.01,52){d}
\put(53.59,52){i}
\put(54.68,52){a}
\put(55.90,52){n}
\put(57.41,52){s}
\put(58.76,52){)}
\put(4.98,48){X}
\put(7.68,48){=}
\put(11.50,48){$+$}
\put(13.12,48){0}
\put(14.68,48){.}
\put(15.72,48){2}
\put(17.02,48){3}
\put(18.32,48){2}
\put(19.62,48){8}
\put(20.92,48){0}
\put(22.22,48){7}
\put(23.52,48){2}
\put(24.60,48){E}
\put(25.97,48){$-$}
\put(27.42,48){0}
\put(28.72,48){3}
\put(66.07,48){Y}
\put(68.78,48){=}
\put(72.77,48){$-$}
\put(74.22,48){0}
\put(75.78,48){.}
\put(76.82,48){7}
\put(78.12,48){3}
\put(79.42,48){9}
\put(80.72,48){2}
\put(82.02,48){2}
\put(83.32,48){8}
\put(84.62,48){5}
\put(85.70,48){E}
\put(87.07,48){$-$}
\put(88.52,48){0}
\put(89.82,48){3}
\put(11.67,46){$-$}
\put(13.12,46){0}
\put(14.68,46){.}
\put(15.72,46){3}
\put(17.02,46){2}
\put(18.32,46){7}
\put(19.62,46){4}
\put(20.92,46){5}
\put(22.22,46){4}
\put(23.52,46){7}
\put(24.60,46){E}
\put(25.97,46){$-$}
\put(27.42,46){0}
\put(28.72,46){3}
\put(31.36,46){$\ast$}
\put(33.58,46){X}
\put(34.81,46){m}
\put(36.53,46){e}
\put(37.78,46){a}
\put(39.21,46){s}
\put(72.77,46){$-$}
\put(74.22,46){0}
\put(75.78,46){.}
\put(76.82,46){5}
\put(78.12,46){5}
\put(79.42,46){4}
\put(80.72,46){3}
\put(82.02,46){9}
\put(83.32,46){3}
\put(84.62,46){9}
\put(85.70,46){E}
\put(87.07,46){$-$}
\put(88.52,46){0}
\put(89.82,46){6}
\put(92.46,46){$\ast$}
\put(94.68,46){X}
\put(95.91,46){m}
\put(97.63,46){e}
\put(98.88,46){a}
\put(100.31,46){s}
\put(11.67,44){$-$}
\put(13.12,44){0}
\put(14.68,44){.}
\put(15.72,44){5}
\put(17.02,44){5}
\put(18.32,44){4}
\put(19.62,44){3}
\put(20.92,44){9}
\put(22.22,44){3}
\put(23.52,44){9}
\put(24.60,44){E}
\put(25.97,44){$-$}
\put(27.42,44){0}
\put(28.72,44){6}
\put(31.36,44){$\ast$}
\put(33.57,44){Y}
\put(34.81,44){m}
\put(36.53,44){e}
\put(37.78,44){a}
\put(39.21,44){s}
\put(72.60,44){$+$}
\put(74.22,44){0}
\put(75.78,44){.}
\put(76.82,44){3}
\put(78.12,44){2}
\put(79.42,44){7}
\put(80.72,44){4}
\put(82.02,44){5}
\put(83.32,44){4}
\put(84.62,44){7}
\put(85.70,44){E}
\put(87.07,44){$-$}
\put(88.52,44){0}
\put(89.82,44){3}
\put(92.46,44){$\ast$}
\put(94.67,44){Y}
\put(95.91,44){m}
\put(97.63,44){e}
\put(98.88,44){a}
\put(100.31,44){s}
\put(-0.22,38){X}
\put(1.01,38){m}
\put(2.73,38){e}
\put(3.98,38){a}
\put(5.41,38){s}
\put(7.68,38){=}
\put(11.50,38){$+$}
\put(13.12,38){0}
\put(14.68,38){.}
\put(15.72,38){7}
\put(17.02,38){0}
\put(18.32,38){7}
\put(19.62,38){1}
\put(20.92,38){3}
\put(22.22,38){6}
\put(23.52,38){0}
\put(60.87,38){Y}
\put(62.11,38){m}
\put(63.83,38){e}
\put(65.08,38){a}
\put(66.51,38){s}
\put(68.78,38){=}
\put(73.90,38){$+$}
\put(75.52,38){2}
\put(77.08,38){.}
\put(78.12,38){2}
\put(79.42,38){5}
\put(80.72,38){8}
\put(82.02,38){6}
\put(83.32,38){9}
\put(84.62,38){6}
\put(12.97,36){$-$}
\put(14.42,36){3}
\put(15.72,36){0}
\put(17.02,36){5}
\put(18.32,36){3}
\put(19.88,36){.}
\put(20.92,36){8}
\put(22.22,36){4}
\put(23.52,36){9}
\put(31.36,36){$\ast$}
\put(33.58,36){X}
\put(74.07,36){$-$}
\put(75.52,36){5}
\put(77.08,36){.}
\put(78.12,36){1}
\put(79.42,36){7}
\put(80.72,36){0}
\put(82.02,36){2}
\put(83.32,36){8}
\put(84.62,36){9}
\put(92.46,36){$\ast$}
\put(94.68,36){X}
\put(12.97,34){$-$}
\put(14.42,34){5}
\put(15.98,34){.}
\put(17.02,34){1}
\put(18.32,34){7}
\put(19.62,34){0}
\put(20.92,34){2}
\put(22.22,34){8}
\put(23.52,34){9}
\put(31.36,34){$\ast$}
\put(33.57,34){Y}
\put(73.90,34){$+$}
\put(75.52,34){3}
\put(76.82,34){0}
\put(78.12,34){5}
\put(79.42,34){3}
\put(80.98,34){.}
\put(82.02,34){8}
\put(83.32,34){4}
\put(84.62,34){9}
\put(92.46,34){$\ast$}
\put(94.67,34){Y}
\put(6.43,28){P}
\put(8.13,28){l}
\put(9.18,28){a}
\put(10.65,28){t}
\put(11.83,28){e}
\put(14.51,28){s}
\put(15.73,28){c}
\put(16.98,28){a}
\put(18.52,28){l}
\put(19.63,28){e}
\put(22.26,28){(}
\put(23.69,28){i}
\put(24.70,28){n}
\put(27.01,28){m}
\put(28.73,28){e}
\put(29.98,28){a}
\put(31.41,28){s}
\put(32.51,28){u}
\put(34.02,28){r}
\put(35.39,28){i}
\put(36.40,28){n}
\put(37.76,28){g}
\put(40.31,28){u}
\put(41.60,28){n}
\put(43.19,28){i}
\put(44.45,28){t}
\put(45.71,28){s}
\put(47.06,28){)}
\put(48.46,28){:}
\put(63.82,28){6}
\put(65.12,28){7}
\put(66.68,28){.}
\put(67.72,28){5}
\put(69.02,28){4}
\put(70.32,28){2}
\put(72.88,28){a}
\put(74.32,28){r}
\put(75.53,28){c}
\put(76.91,28){s}
\put(78.13,28){e}
\put(79.43,28){c}
\put(33.58,26){O}
\put(35.32,26){r}
\put(36.69,26){i}
\put(37.83,26){e}
\put(39,26){n}
\put(40.55,26){t}
\put(41.68,26){a}
\put(43.15,26){t}
\put(44.49,26){i}
\put(45.60,26){o}
\put(46.80,26){n}
\put(48.46,26){:}
\put(63.50,26){$+$}
\put(65.12,26){0}
\put(66.68,26){.}
\put(67.72,26){0}
\put(69.02,26){9}
\put(70.32,26){7}
\put(72.81,26){d}
\put(74.23,26){e}
\put(75.46,26){g}
\put(78.08,26){a}
\put(79.30,26){n}
\put(80.61,26){d}
\put(83.53,26){l}
\put(84.58,26){a}
\put(86.05,26){t}
\put(87.23,26){e}
\put(88.62,26){r}
\put(89.78,26){a}
\put(91.32,26){l}
\put(92.63,26){l}
\put(93.67,26){y}
\put(96.49,26){i}
\put(97.50,26){n}
\put(98.83,26){v}
\put(100.23,26){e}
\put(101.62,26){r}
\put(102.95,26){t}
\put(104.13,26){e}
\put(105.31,26){d}
\put(-0.19,20){R}
\put(1.43,20){e}
\put(2.77,20){f}
\put(4.03,20){e}
\put(5.42,20){r}
\put(6.63,20){e}
\put(7.80,20){n}
\put(9.23,20){c}
\put(10.53,20){e}
\put(13.21,20){s}
\put(14.55,20){t}
\put(15.68,20){a}
\put(17.12,20){r}
\put(18.41,20){s}
\put(19.86,20){:}
\put(36.01,18){M}
\put(37.83,18){e}
\put(39.08,18){a}
\put(40.30,18){n}
\put(42.70,18){R}
\put(44,18){A}
\put(45.79,18){,}
\put(46.64,18){D}
\put(48.23,18){e}
\put(49.53,18){c}
\put(62.30,18){E}
\put(63.73,18){q}
\put(65.01,18){u}
\put(66.59,18){i}
\put(67.60,18){n}
\put(69,18){o}
\put(70.23,18){x}
\put(72.67,18){B}
\put(74.22,18){1}
\put(75.52,18){9}
\put(76.82,18){5}
\put(78.12,18){0}
\put(79.68,18){.}
\put(80.72,18){0}
\put(89.60,18){E}
\put(91.03,18){p}
\put(92.40,18){o}
\put(93.73,18){c}
\put(94.90,18){h}
\put(97.37,18){B}
\put(98.92,18){1}
\put(100.22,18){9}
\put(101.52,18){7}
\put(102.82,18){4}
\put(104.38,18){.}
\put(105.42,18){5}
\put(106.72,18){0}
\put(108.02,18){0}
\put(123.30,18){R}
\put(124.93,18){e}
\put(126.31,18){s}
\put(127.69,18){i}
\put(128.71,18){d}
\put(130.01,18){u}
\put(131.38,18){a}
\put(132.93,18){l}
\put(134.11,18){s}
\put(136.66,18){(}
\put(137.88,18){a}
\put(139.32,18){r}
\put(140.53,18){c}
\put(141.91,18){s}
\put(143.13,18){e}
\put(144.43,18){c}
\put(145.86,18){)}
\put(19.50,16){n}
\put(46.93,16){c}
\put(48.18,16){a}
\put(49.65,16){t}
\put(50.78,16){a}
\put(52.32,16){l}
\put(53.40,16){o}
\put(54.66,16){g}
\put(55.91,16){u}
\put(57.33,16){e}
\put(85.93,16){c}
\put(87.18,16){a}
\put(88.72,16){l}
\put(89.83,16){c}
\put(91.01,16){u}
\put(92.63,16){l}
\put(93.68,16){a}
\put(95.15,16){t}
\put(96.33,16){e}
\put(97.51,16){d}
\put(120.91,16){d}
\put(121.98,16){X}
\put(133.91,16){d}
\put(134.97,16){Y}
\put(146.91,16){d}
\put(148,16){R}
\put(-0.19,12){R}
\put(1.43,12){e}
\put(2.77,12){f}
\put(5.32,12){1}
\put(19.62,12){1}
\put(36.52,12){1}
\put(37.82,12){8}
\put(40.42,12){5}
\put(41.72,12){6}
\put(44.32,12){3}
\put(45.62,12){9}
\put(47.18,12){.}
\put(48.22,12){4}
\put(49.52,12){2}
\put(50.82,12){1}
\put(54.57,12){$-$}
\put(56.02,12){6}
\put(57.32,12){3}
\put(59.92,12){2}
\put(61.22,12){5}
\put(63.82,12){1}
\put(65.12,12){4}
\put(66.68,12){.}
\put(67.72,12){1}
\put(69.02,12){1}
\put(75.52,12){1}
\put(76.82,12){8}
\put(79.42,12){5}
\put(80.72,12){6}
\put(83.32,12){3}
\put(84.62,12){9}
\put(86.18,12){.}
\put(87.22,12){4}
\put(88.52,12){6}
\put(89.82,12){8}
\put(93.57,12){$-$}
\put(95.02,12){6}
\put(96.32,12){3}
\put(98.92,12){2}
\put(100.22,12){5}
\put(102.82,12){1}
\put(104.12,12){4}
\put(105.68,12){.}
\put(106.72,12){2}
\put(108.02,12){7}
\put(116.80,12){$+$}
\put(118.42,12){0}
\put(119.98,12){.}
\put(121.02,12){3}
\put(122.32,12){1}
\put(123.62,12){9}
\put(129.97,12){$-$}
\put(131.42,12){0}
\put(132.98,12){.}
\put(134.02,12){1}
\put(135.32,12){4}
\put(136.62,12){5}
\put(144.42,12){0}
\put(145.98,12){.}
\put(147.02,12){3}
\put(148.32,12){5}
\put(149.62,12){0}
\put(-0.19,10){R}
\put(1.43,10){e}
\put(2.77,10){f}
\put(5.32,10){2}
\put(19.62,10){2}
\put(36.52,10){1}
\put(37.82,10){9}
\put(40.42,10){1}
\put(41.72,10){1}
\put(44.32,10){5}
\put(45.62,10){4}
\put(47.18,10){.}
\put(48.22,10){0}
\put(49.52,10){5}
\put(50.82,10){1}
\put(54.57,10){$-$}
\put(56.02,10){6}
\put(57.32,10){3}
\put(59.92,10){1}
\put(61.22,10){7}
\put(63.82,10){5}
\put(65.12,10){8}
\put(66.68,10){.}
\put(67.72,10){6}
\put(69.02,10){5}
\put(75.52,10){1}
\put(76.82,10){9}
\put(79.42,10){1}
\put(80.72,10){1}
\put(83.32,10){5}
\put(84.62,10){4}
\put(86.18,10){.}
\put(87.22,10){0}
\put(88.52,10){8}
\put(89.82,10){2}
\put(93.57,10){$-$}
\put(95.02,10){6}
\put(96.32,10){3}
\put(98.92,10){1}
\put(100.22,10){7}
\put(102.82,10){5}
\put(104.12,10){8}
\put(105.68,10){.}
\put(106.72,10){0}
\put(108.02,10){0}
\put(116.80,10){$+$}
\put(118.42,10){0}
\put(119.98,10){.}
\put(121.02,10){2}
\put(122.32,10){2}
\put(123.62,10){6}
\put(129.80,10){$+$}
\put(131.42,10){0}
\put(132.98,10){.}
\put(134.02,10){6}
\put(135.32,10){3}
\put(136.62,10){8}
\put(144.42,10){0}
\put(145.98,10){.}
\put(147.02,10){6}
\put(148.32,10){7}
\put(149.62,10){7}
\put(-0.19,8){R}
\put(1.43,8){e}
\put(2.77,8){f}
\put(5.32,8){3}
\put(19.62,8){3}
\put(36.52,8){1}
\put(37.82,8){9}
\put(40.42,8){0}
\put(41.72,8){1}
\put(44.32,8){1}
\put(45.62,8){3}
\put(47.18,8){.}
\put(48.22,8){6}
\put(49.52,8){5}
\put(50.82,8){5}
\put(54.57,8){$-$}
\put(56.02,8){6}
\put(57.32,8){3}
\put(59.92,8){4}
\put(61.22,8){9}
\put(63.82,8){1}
\put(65.12,8){5}
\put(66.68,8){.}
\put(67.72,8){4}
\put(69.02,8){8}
\put(75.52,8){1}
\put(76.82,8){9}
\put(79.42,8){0}
\put(80.72,8){1}
\put(83.32,8){1}
\put(84.62,8){3}
\put(86.18,8){.}
\put(87.22,8){5}
\put(88.52,8){8}
\put(89.82,8){0}
\put(93.57,8){$-$}
\put(95.02,8){6}
\put(96.32,8){3}
\put(98.92,8){4}
\put(100.22,8){9}
\put(102.82,8){1}
\put(104.12,8){5}
\put(105.68,8){.}
\put(106.72,8){8}
\put(108.02,8){8}
\put(116.97,8){$-$}
\put(118.42,8){0}
\put(119.98,8){.}
\put(121.02,8){4}
\put(122.32,8){8}
\put(123.62,8){9}
\put(129.97,8){$-$}
\put(131.42,8){0}
\put(132.98,8){.}
\put(134.02,8){4}
\put(135.32,8){0}
\put(136.62,8){7}
\put(144.42,8){0}
\put(145.98,8){.}
\put(147.02,8){6}
\put(148.32,8){3}
\put(149.62,8){6}
\put(-0.19,6){R}
\put(1.43,6){e}
\put(2.77,6){f}
\put(5.32,6){4}
\put(19.62,6){4}
\put(36.52,6){1}
\put(37.82,6){9}
\put(40.42,6){0}
\put(41.72,6){8}
\put(44.32,6){2}
\put(45.62,6){9}
\put(47.18,6){.}
\put(48.22,6){1}
\put(49.52,6){2}
\put(50.82,6){7}
\put(54.57,6){$-$}
\put(56.02,6){6}
\put(57.32,6){3}
\put(59.92,6){5}
\put(61.22,6){7}
\put(63.82,6){4}
\put(65.12,6){2}
\put(66.68,6){.}
\put(67.72,6){3}
\put(69.02,6){5}
\put(75.52,6){1}
\put(76.82,6){9}
\put(79.42,6){0}
\put(80.72,6){8}
\put(83.32,6){2}
\put(84.62,6){9}
\put(86.18,6){.}
\put(87.22,6){0}
\put(88.52,6){1}
\put(89.82,6){6}
\put(93.57,6){$-$}
\put(95.02,6){6}
\put(96.32,6){3}
\put(98.92,6){5}
\put(100.22,6){7}
\put(102.82,6){4}
\put(104.12,6){2}
\put(105.68,6){.}
\put(106.72,6){7}
\put(108.02,6){2}
\put(116.97,6){$-$}
\put(118.42,6){0}
\put(119.98,6){.}
\put(121.02,6){7}
\put(122.32,6){3}
\put(123.62,6){8}
\put(129.97,6){$-$}
\put(131.42,6){0}
\put(132.98,6){.}
\put(134.02,6){3}
\put(135.32,6){6}
\put(136.62,6){5}
\put(144.42,6){0}
\put(145.98,6){.}
\put(147.02,6){8}
\put(148.32,6){2}
\put(149.62,6){4}
\put(-0.19,4){R}
\put(1.43,4){e}
\put(2.77,4){f}
\put(5.32,4){5}
\put(19.62,4){5}
\put(36.52,4){1}
\put(37.82,4){9}
\put(40.42,4){0}
\put(41.72,4){2}
\put(44.32,4){1}
\put(45.62,4){0}
\put(47.18,4){.}
\put(48.22,4){1}
\put(49.52,4){1}
\put(50.82,4){7}
\put(54.57,4){$-$}
\put(56.02,4){6}
\put(57.32,4){3}
\put(59.92,4){2}
\put(61.22,4){9}
\put(63.82,4){1}
\put(65.12,4){7}
\put(66.68,4){.}
\put(67.72,4){2}
\put(69.02,4){0}
\put(75.52,4){1}
\put(76.82,4){9}
\put(79.42,4){0}
\put(80.72,4){2}
\put(83.32,4){1}
\put(84.62,4){0}
\put(86.18,4){.}
\put(87.22,4){2}
\put(88.52,4){2}
\put(89.82,4){0}
\put(93.57,4){$-$}
\put(95.02,4){6}
\put(96.32,4){3}
\put(98.92,4){2}
\put(100.22,4){9}
\put(102.82,4){1}
\put(104.12,4){6}
\put(105.68,4){.}
\put(106.72,4){9}
\put(108.02,4){2}
\put(116.80,4){$+$}
\put(118.42,4){0}
\put(119.98,4){.}
\put(121.02,4){6}
\put(122.32,4){8}
\put(123.62,4){2}
\put(129.80,4){$+$}
\put(131.42,4){0}
\put(132.98,4){.}
\put(134.02,4){2}
\put(135.32,4){8}
\put(136.62,4){0}
\put(144.42,4){0}
\put(145.98,4){.}
\put(147.02,4){7}
\put(148.32,4){3}
\put(149.62,4){8}
\put(105.10,0){R}
\put(106.21,0){M}
\put(107.95,0){S}
\put(110.86,0){:}
\put(118.42,0){0}
\put(119.98,0){.}
\put(121.02,0){5}
\put(122.32,0){3}
\put(123.62,0){0}
\put(131.42,0){0}
\put(132.98,0){.}
\put(134.02,0){4}
\put(135.32,0){0}
\put(136.62,0){1}
\put(144.42,0){0}
\put(145.98,0){.}
\put(147.02,0){6}
\put(148.32,0){6}
\put(149.62,0){5}
\end{picture}

\noindent
\begin{picture}(159.90, 74.00)( -2.60, -4.00)
\put( -0.65, -2.50){\framebox(158.60, 70.00){}}
\put(-0.06,64){P}
\put(1.63,64){l}
\put(2.68,64){a}
\put(4.15,64){t}
\put(5.33,64){e}
\put(8.01,64){s}
\put(9.20,64){o}
\put(10.72,64){l}
\put(11.71,64){u}
\put(13.25,64){t}
\put(14.59,64){i}
\put(15.70,64){o}
\put(16.90,64){n}
\put(18.56,64){:}
\put(20.92,64){6}
\put(22.07,64){$-$}
\put(23.53,64){c}
\put(24.80,64){o}
\put(26.13,64){e}
\put(27.47,64){f}
\put(28.77,64){f}
\put(30.19,64){i}
\put(31.33,64){c}
\put(32.79,64){i}
\put(33.93,64){e}
\put(35.10,64){n}
\put(36.65,64){t}
\put(-0.03,62){$-$}
\put(1.27,62){$-$}
\put(2.57,62){$-$}
\put(3.87,62){$-$}
\put(5.17,62){$-$}
\put(6.47,62){$-$}
\put(7.77,62){$-$}
\put(9.07,62){$-$}
\put(10.37,62){$-$}
\put(11.67,62){$-$}
\put(12.97,62){$-$}
\put(14.27,62){$-$}
\put(15.57,62){$-$}
\put(16.87,62){$-$}
\put(18.17,62){$-$}
\put(19.47,62){$-$}
\put(20.77,62){$-$}
\put(22.07,62){$-$}
\put(23.37,62){$-$}
\put(24.67,62){$-$}
\put(25.97,62){$-$}
\put(27.27,62){$-$}
\put(28.57,62){$-$}
\put(29.87,62){$-$}
\put(31.17,62){$-$}
\put(32.47,62){$-$}
\put(33.77,62){$-$}
\put(35.07,62){$-$}
\put(36.37,62){$-$}
\put(4.98,58){X}
\put(6.79,58){,}
\put(7.57,58){Y}
\put(10.28,58){=}
\put(13.13,58){e}
\put(14.33,58){x}
\put(15.63,58){p}
\put(17.03,58){e}
\put(18.33,58){c}
\put(19.75,58){t}
\put(20.93,58){e}
\put(22.11,58){d}
\put(24.73,58){p}
\put(26.32,58){l}
\put(27.38,58){a}
\put(28.85,58){t}
\put(30.03,58){e}
\put(32.63,58){c}
\put(33.90,58){o}
\put(35.20,58){o}
\put(36.62,58){r}
\put(37.71,58){d}
\put(39.29,58){i}
\put(40.30,58){n}
\put(41.68,58){a}
\put(43.15,58){t}
\put(44.33,58){e}
\put(45.71,58){s}
\put(48.26,58){(}
\put(49.62,58){r}
\put(50.78,58){a}
\put(52.01,58){d}
\put(53.59,58){i}
\put(54.68,58){a}
\put(55.90,58){n}
\put(57.41,58){s}
\put(58.76,58){)}
\put(4.98,54){X}
\put(7.68,54){=}
\put(11.50,54){$+$}
\put(13.12,54){0}
\put(14.68,54){.}
\put(15.72,54){2}
\put(17.02,54){4}
\put(18.32,54){5}
\put(19.62,54){7}
\put(20.92,54){7}
\put(22.22,54){6}
\put(23.52,54){1}
\put(24.60,54){E}
\put(25.97,54){$-$}
\put(27.42,54){0}
\put(28.72,54){3}
\put(66.07,54){Y}
\put(68.78,54){=}
\put(72.77,54){$-$}
\put(74.22,54){0}
\put(75.78,54){.}
\put(76.82,54){7}
\put(78.12,54){2}
\put(79.42,54){9}
\put(80.72,54){6}
\put(82.02,54){7}
\put(83.32,54){7}
\put(84.62,54){8}
\put(85.70,54){E}
\put(87.07,54){$-$}
\put(88.52,54){0}
\put(89.82,54){3}
\put(11.67,52){$-$}
\put(13.12,52){0}
\put(14.68,52){.}
\put(15.72,52){3}
\put(17.02,52){2}
\put(18.32,52){7}
\put(19.62,52){4}
\put(20.92,52){7}
\put(22.22,52){4}
\put(23.52,52){2}
\put(24.60,52){E}
\put(25.97,52){$-$}
\put(27.42,52){0}
\put(28.72,52){3}
\put(31.36,52){$\ast$}
\put(33.58,52){X}
\put(34.81,52){m}
\put(36.53,52){e}
\put(37.78,52){a}
\put(39.21,52){s}
\put(72.77,52){$-$}
\put(74.22,52){0}
\put(75.78,52){.}
\put(76.82,52){5}
\put(78.12,52){2}
\put(79.42,52){5}
\put(80.72,52){3}
\put(82.02,52){0}
\put(83.32,52){8}
\put(84.62,52){8}
\put(85.70,52){E}
\put(87.07,52){$-$}
\put(88.52,52){0}
\put(89.82,52){6}
\put(92.46,52){$\ast$}
\put(94.68,52){X}
\put(95.91,52){m}
\put(97.63,52){e}
\put(98.88,52){a}
\put(100.31,52){s}
\put(11.67,50){$-$}
\put(13.12,50){0}
\put(14.68,50){.}
\put(15.72,50){7}
\put(17.02,50){2}
\put(18.32,50){3}
\put(19.62,50){5}
\put(20.92,50){1}
\put(22.22,50){6}
\put(23.52,50){2}
\put(24.60,50){E}
\put(25.97,50){$-$}
\put(27.42,50){0}
\put(28.72,50){6}
\put(31.36,50){$\ast$}
\put(33.57,50){Y}
\put(34.81,50){m}
\put(36.53,50){e}
\put(37.78,50){a}
\put(39.21,50){s}
\put(72.60,50){$+$}
\put(74.22,50){0}
\put(75.78,50){.}
\put(76.82,50){3}
\put(78.12,50){2}
\put(79.42,50){7}
\put(80.72,50){3}
\put(82.02,50){2}
\put(83.32,50){9}
\put(84.62,50){9}
\put(85.70,50){E}
\put(87.07,50){$-$}
\put(88.52,50){0}
\put(89.82,50){3}
\put(92.46,50){$\ast$}
\put(94.67,50){Y}
\put(95.91,50){m}
\put(97.63,50){e}
\put(98.88,50){a}
\put(100.31,50){s}
\put(-0.22,44){X}
\put(1.01,44){m}
\put(2.73,44){e}
\put(3.98,44){a}
\put(5.41,44){s}
\put(7.68,44){=}
\put(11.50,44){$+$}
\put(13.12,44){0}
\put(14.68,44){.}
\put(15.72,44){7}
\put(17.02,44){4}
\put(18.32,44){5}
\put(19.62,44){5}
\put(20.92,44){9}
\put(22.22,44){2}
\put(23.52,44){8}
\put(60.87,44){Y}
\put(62.11,44){m}
\put(63.83,44){e}
\put(65.08,44){a}
\put(66.51,44){s}
\put(68.78,44){=}
\put(73.90,44){$+$}
\put(75.52,44){2}
\put(77.08,44){.}
\put(78.12,44){2}
\put(79.42,44){3}
\put(80.72,44){0}
\put(82.02,44){3}
\put(83.32,44){7}
\put(84.62,44){9}
\put(12.97,42){$-$}
\put(14.42,42){3}
\put(15.72,42){0}
\put(17.02,42){5}
\put(18.32,42){3}
\put(19.88,42){.}
\put(20.92,42){6}
\put(22.22,42){6}
\put(23.52,42){5}
\put(31.36,42){$\ast$}
\put(33.58,42){X}
\put(74.07,42){$-$}
\put(75.52,42){4}
\put(77.08,42){.}
\put(78.12,42){9}
\put(79.42,42){0}
\put(80.72,42){0}
\put(82.02,42){6}
\put(83.32,42){1}
\put(84.62,42){4}
\put(92.46,42){$\ast$}
\put(94.68,42){X}
\put(12.97,40){$-$}
\put(14.42,40){6}
\put(15.98,40){.}
\put(17.02,40){7}
\put(18.32,40){4}
\put(19.62,40){9}
\put(20.92,40){6}
\put(22.22,40){9}
\put(23.52,40){4}
\put(31.36,40){$\ast$}
\put(33.57,40){Y}
\put(73.90,40){$+$}
\put(75.52,40){3}
\put(76.82,40){0}
\put(78.12,40){5}
\put(79.42,40){5}
\put(80.98,40){.}
\put(82.02,40){0}
\put(83.32,40){1}
\put(84.62,40){1}
\put(92.46,40){$\ast$}
\put(94.67,40){Y}
\put(5.13,34){P}
\put(6.82,34){l}
\put(7.88,34){a}
\put(9.35,34){t}
\put(10.53,34){e}
\put(13.21,34){s}
\put(14.43,34){c}
\put(15.68,34){a}
\put(17.22,34){l}
\put(18.33,34){e}
\put(19.71,34){s}
\put(22.26,34){(}
\put(23.69,34){i}
\put(24.70,34){n}
\put(27.01,34){m}
\put(28.73,34){e}
\put(29.98,34){a}
\put(31.41,34){s}
\put(32.51,34){u}
\put(34.02,34){r}
\put(35.39,34){i}
\put(36.40,34){n}
\put(37.76,34){g}
\put(40.31,34){u}
\put(41.60,34){n}
\put(43.19,34){i}
\put(44.45,34){t}
\put(45.71,34){s}
\put(47.06,34){)}
\put(48.46,34){:}
\put(56.98,34){X}
\put(63.82,34){6}
\put(65.12,34){7}
\put(66.68,34){.}
\put(67.72,34){5}
\put(69.02,34){4}
\put(70.32,34){6}
\put(72.88,34){a}
\put(74.32,34){r}
\put(75.53,34){c}
\put(76.91,34){s}
\put(78.13,34){e}
\put(79.43,34){c}
\put(56.97,32){Y}
\put(63.82,32){6}
\put(65.12,32){7}
\put(66.68,32){.}
\put(67.72,32){5}
\put(69.02,32){1}
\put(70.32,32){7}
\put(72.88,32){a}
\put(74.32,32){r}
\put(75.53,32){c}
\put(76.91,32){s}
\put(78.13,32){e}
\put(79.43,32){c}
\put(53.01,30){m}
\put(54.73,30){e}
\put(55.98,30){a}
\put(57.20,30){n}
\put(63.82,30){6}
\put(65.12,30){7}
\put(66.68,30){.}
\put(67.72,30){5}
\put(69.02,30){3}
\put(70.32,30){2}
\put(72.88,30){a}
\put(74.32,30){r}
\put(75.53,30){c}
\put(76.91,30){s}
\put(78.13,30){e}
\put(79.43,30){c}
\put(23.24,28){N}
\put(24.80,28){o}
\put(26,28){n}
\put(27.33,28){p}
\put(28.73,28){e}
\put(30.12,28){r}
\put(31.23,28){p}
\put(32.63,28){e}
\put(33.80,28){n}
\put(35.11,28){d}
\put(36.69,28){i}
\put(37.83,28){c}
\put(39.01,28){u}
\put(40.62,28){l}
\put(41.68,28){a}
\put(43.12,28){r}
\put(44.49,28){i}
\put(45.75,28){t}
\put(46.86,28){y}
\put(48.46,28){:}
\put(63.50,28){$+$}
\put(65.12,28){0}
\put(66.68,28){.}
\put(67.72,28){0}
\put(69.02,28){3}
\put(70.32,28){5}
\put(72.81,28){d}
\put(74.23,28){e}
\put(75.46,28){g}
\put(33.58,26){O}
\put(35.32,26){r}
\put(36.69,26){i}
\put(37.83,26){e}
\put(39,26){n}
\put(40.55,26){t}
\put(41.68,26){a}
\put(43.15,26){t}
\put(44.49,26){i}
\put(45.60,26){o}
\put(46.80,26){n}
\put(48.46,26){:}
\put(63.50,26){$+$}
\put(65.12,26){0}
\put(66.68,26){.}
\put(67.72,26){1}
\put(69.02,26){0}
\put(70.32,26){9}
\put(72.81,26){d}
\put(74.23,26){e}
\put(75.46,26){g}
\put(78.08,26){a}
\put(79.30,26){n}
\put(80.61,26){d}
\put(83.53,26){l}
\put(84.58,26){a}
\put(86.05,26){t}
\put(87.23,26){e}
\put(88.62,26){r}
\put(89.78,26){a}
\put(91.32,26){l}
\put(92.63,26){l}
\put(93.67,26){y}
\put(96.49,26){i}
\put(97.50,26){n}
\put(98.83,26){v}
\put(100.23,26){e}
\put(101.62,26){r}
\put(102.95,26){t}
\put(104.13,26){e}
\put(105.31,26){d}
\put(-0.19,20){R}
\put(1.43,20){e}
\put(2.77,20){f}
\put(4.03,20){e}
\put(5.42,20){r}
\put(6.63,20){e}
\put(7.80,20){n}
\put(9.23,20){c}
\put(10.53,20){e}
\put(13.21,20){s}
\put(14.55,20){t}
\put(15.68,20){a}
\put(17.12,20){r}
\put(18.41,20){s}
\put(19.86,20){:}
\put(36.01,18){M}
\put(37.83,18){e}
\put(39.08,18){a}
\put(40.30,18){n}
\put(42.70,18){R}
\put(44,18){A}
\put(45.79,18){,}
\put(46.64,18){D}
\put(48.23,18){e}
\put(49.53,18){c}
\put(62.30,18){E}
\put(63.73,18){q}
\put(65.01,18){u}
\put(66.59,18){i}
\put(67.60,18){n}
\put(69,18){o}
\put(70.23,18){x}
\put(72.67,18){B}
\put(74.22,18){1}
\put(75.52,18){9}
\put(76.82,18){5}
\put(78.12,18){0}
\put(79.68,18){.}
\put(80.72,18){0}
\put(89.60,18){E}
\put(91.03,18){p}
\put(92.40,18){o}
\put(93.73,18){c}
\put(94.90,18){h}
\put(97.37,18){B}
\put(98.92,18){1}
\put(100.22,18){9}
\put(101.52,18){7}
\put(102.82,18){4}
\put(104.38,18){.}
\put(105.42,18){5}
\put(106.72,18){0}
\put(108.02,18){0}
\put(123.30,18){R}
\put(124.93,18){e}
\put(126.31,18){s}
\put(127.69,18){i}
\put(128.71,18){d}
\put(130.01,18){u}
\put(131.38,18){a}
\put(132.93,18){l}
\put(134.11,18){s}
\put(136.66,18){(}
\put(137.88,18){a}
\put(139.32,18){r}
\put(140.53,18){c}
\put(141.91,18){s}
\put(143.13,18){e}
\put(144.43,18){c}
\put(145.86,18){)}
\put(19.50,16){n}
\put(46.93,16){c}
\put(48.18,16){a}
\put(49.65,16){t}
\put(50.78,16){a}
\put(52.32,16){l}
\put(53.40,16){o}
\put(54.66,16){g}
\put(55.91,16){u}
\put(57.33,16){e}
\put(85.93,16){c}
\put(87.18,16){a}
\put(88.72,16){l}
\put(89.83,16){c}
\put(91.01,16){u}
\put(92.63,16){l}
\put(93.68,16){a}
\put(95.15,16){t}
\put(96.33,16){e}
\put(97.51,16){d}
\put(120.91,16){d}
\put(121.98,16){X}
\put(133.91,16){d}
\put(134.97,16){Y}
\put(146.91,16){d}
\put(148,16){R}
\put(-0.19,12){R}
\put(1.43,12){e}
\put(2.77,12){f}
\put(5.32,12){1}
\put(19.62,12){1}
\put(36.52,12){1}
\put(37.82,12){8}
\put(40.42,12){5}
\put(41.72,12){6}
\put(44.32,12){3}
\put(45.62,12){9}
\put(47.18,12){.}
\put(48.22,12){4}
\put(49.52,12){2}
\put(50.82,12){1}
\put(54.57,12){$-$}
\put(56.02,12){6}
\put(57.32,12){3}
\put(59.92,12){2}
\put(61.22,12){5}
\put(63.82,12){1}
\put(65.12,12){4}
\put(66.68,12){.}
\put(67.72,12){1}
\put(69.02,12){1}
\put(75.52,12){1}
\put(76.82,12){8}
\put(79.42,12){5}
\put(80.72,12){6}
\put(83.32,12){3}
\put(84.62,12){9}
\put(86.18,12){.}
\put(87.22,12){3}
\put(88.52,12){9}
\put(89.82,12){5}
\put(93.57,12){$-$}
\put(95.02,12){6}
\put(96.32,12){3}
\put(98.92,12){2}
\put(100.22,12){5}
\put(102.82,12){1}
\put(104.12,12){4}
\put(105.68,12){.}
\put(106.72,12){2}
\put(108.02,12){2}
\put(116.97,12){$-$}
\put(118.42,12){0}
\put(119.98,12){.}
\put(121.02,12){1}
\put(122.32,12){7}
\put(123.62,12){4}
\put(129.97,12){$-$}
\put(131.42,12){0}
\put(132.98,12){.}
\put(134.02,12){1}
\put(135.32,12){1}
\put(136.62,12){2}
\put(144.42,12){0}
\put(145.98,12){.}
\put(147.02,12){2}
\put(148.32,12){0}
\put(149.62,12){7}
\put(-0.19,10){R}
\put(1.43,10){e}
\put(2.77,10){f}
\put(5.32,10){2}
\put(19.62,10){2}
\put(36.52,10){1}
\put(37.82,10){9}
\put(40.42,10){1}
\put(41.72,10){1}
\put(44.32,10){5}
\put(45.62,10){4}
\put(47.18,10){.}
\put(48.22,10){0}
\put(49.52,10){5}
\put(50.82,10){1}
\put(54.57,10){$-$}
\put(56.02,10){6}
\put(57.32,10){3}
\put(59.92,10){1}
\put(61.22,10){7}
\put(63.82,10){5}
\put(65.12,10){8}
\put(66.68,10){.}
\put(67.72,10){6}
\put(69.02,10){5}
\put(75.52,10){1}
\put(76.82,10){9}
\put(79.42,10){1}
\put(80.72,10){1}
\put(83.32,10){5}
\put(84.62,10){4}
\put(86.18,10){.}
\put(87.22,10){0}
\put(88.52,10){3}
\put(89.82,10){1}
\put(93.57,10){$-$}
\put(95.02,10){6}
\put(96.32,10){3}
\put(98.92,10){1}
\put(100.22,10){7}
\put(102.82,10){5}
\put(104.12,10){8}
\put(105.68,10){.}
\put(106.72,10){7}
\put(108.02,10){0}
\put(116.97,10){$-$}
\put(118.42,10){0}
\put(119.98,10){.}
\put(121.02,10){1}
\put(122.32,10){3}
\put(123.62,10){4}
\put(129.97,10){$-$}
\put(131.42,10){0}
\put(132.98,10){.}
\put(134.02,10){0}
\put(135.32,10){4}
\put(136.62,10){7}
\put(144.42,10){0}
\put(145.98,10){.}
\put(147.02,10){1}
\put(148.32,10){4}
\put(149.62,10){2}
\put(-0.19,8){R}
\put(1.43,8){e}
\put(2.77,8){f}
\put(5.32,8){3}
\put(19.62,8){3}
\put(36.52,8){1}
\put(37.82,8){9}
\put(40.42,8){0}
\put(41.72,8){1}
\put(44.32,8){1}
\put(45.62,8){3}
\put(47.18,8){.}
\put(48.22,8){6}
\put(49.52,8){5}
\put(50.82,8){5}
\put(54.57,8){$-$}
\put(56.02,8){6}
\put(57.32,8){3}
\put(59.92,8){4}
\put(61.22,8){9}
\put(63.82,8){1}
\put(65.12,8){5}
\put(66.68,8){.}
\put(67.72,8){4}
\put(69.02,8){8}
\put(75.52,8){1}
\put(76.82,8){9}
\put(79.42,8){0}
\put(80.72,8){1}
\put(83.32,8){1}
\put(84.62,8){3}
\put(86.18,8){.}
\put(87.22,8){6}
\put(88.52,8){3}
\put(89.82,8){3}
\put(93.57,8){$-$}
\put(95.02,8){6}
\put(96.32,8){3}
\put(98.92,8){4}
\put(100.22,8){9}
\put(102.82,8){1}
\put(104.12,8){5}
\put(105.68,8){.}
\put(106.72,8){4}
\put(108.02,8){8}
\put(116.97,8){$-$}
\put(118.42,8){0}
\put(119.98,8){.}
\put(121.02,8){1}
\put(122.32,8){4}
\put(123.62,8){9}
\put(129.97,8){$-$}
\put(131.42,8){0}
\put(132.98,8){.}
\put(134.02,8){0}
\put(135.32,8){0}
\put(136.62,8){6}
\put(144.42,8){0}
\put(145.98,8){.}
\put(147.02,8){1}
\put(148.32,8){4}
\put(149.62,8){9}
\put(-0.19,6){R}
\put(1.43,6){e}
\put(2.77,6){f}
\put(5.32,6){4}
\put(19.62,6){4}
\put(36.52,6){1}
\put(37.82,6){9}
\put(40.42,6){0}
\put(41.72,6){8}
\put(44.32,6){2}
\put(45.62,6){9}
\put(47.18,6){.}
\put(48.22,6){1}
\put(49.52,6){2}
\put(50.82,6){7}
\put(54.57,6){$-$}
\put(56.02,6){6}
\put(57.32,6){3}
\put(59.92,6){5}
\put(61.22,6){7}
\put(63.82,6){4}
\put(65.12,6){2}
\put(66.68,6){.}
\put(67.72,6){3}
\put(69.02,6){5}
\put(75.52,6){1}
\put(76.82,6){9}
\put(79.42,6){0}
\put(80.72,6){8}
\put(83.32,6){2}
\put(84.62,6){9}
\put(86.18,6){.}
\put(87.22,6){1}
\put(88.52,6){3}
\put(89.82,6){2}
\put(93.57,6){$-$}
\put(95.02,6){6}
\put(96.32,6){3}
\put(98.92,6){5}
\put(100.22,6){7}
\put(102.82,6){4}
\put(104.12,6){2}
\put(105.68,6){.}
\put(106.72,6){3}
\put(108.02,6){7}
\put(116.80,6){$+$}
\put(118.42,6){0}
\put(119.98,6){.}
\put(121.02,6){0}
\put(122.32,6){3}
\put(123.62,6){4}
\put(129.97,6){$-$}
\put(131.42,6){0}
\put(132.98,6){.}
\put(134.02,6){0}
\put(135.32,6){2}
\put(136.62,6){6}
\put(144.42,6){0}
\put(145.98,6){.}
\put(147.02,6){0}
\put(148.32,6){4}
\put(149.62,6){3}
\put(-0.19,4){R}
\put(1.43,4){e}
\put(2.77,4){f}
\put(5.32,4){5}
\put(19.62,4){5}
\put(36.52,4){1}
\put(37.82,4){9}
\put(40.42,4){0}
\put(41.72,4){2}
\put(44.32,4){1}
\put(45.62,4){0}
\put(47.18,4){.}
\put(48.22,4){1}
\put(49.52,4){1}
\put(50.82,4){7}
\put(54.57,4){$-$}
\put(56.02,4){6}
\put(57.32,4){3}
\put(59.92,4){2}
\put(61.22,4){9}
\put(63.82,4){1}
\put(65.12,4){7}
\put(66.68,4){.}
\put(67.72,4){2}
\put(69.02,4){0}
\put(75.52,4){1}
\put(76.82,4){9}
\put(79.42,4){0}
\put(80.72,4){2}
\put(83.32,4){1}
\put(84.62,4){0}
\put(86.18,4){.}
\put(87.22,4){1}
\put(88.52,4){8}
\put(89.82,4){1}
\put(93.57,4){$-$}
\put(95.02,4){6}
\put(96.32,4){3}
\put(98.92,4){2}
\put(100.22,4){9}
\put(102.82,4){1}
\put(104.12,4){7}
\put(105.68,4){.}
\put(106.72,4){0}
\put(108.02,4){1}
\put(116.80,4){$+$}
\put(118.42,4){0}
\put(119.98,4){.}
\put(121.02,4){4}
\put(122.32,4){2}
\put(123.62,4){2}
\put(129.80,4){$+$}
\put(131.42,4){0}
\put(132.98,4){.}
\put(134.02,4){1}
\put(135.32,4){9}
\put(136.62,4){1}
\put(144.42,4){0}
\put(145.98,4){.}
\put(147.02,4){4}
\put(148.32,4){6}
\put(149.62,4){3}
\put(105.10,0){R}
\put(106.21,0){M}
\put(107.95,0){S}
\put(110.86,0){:}
\put(118.42,0){0}
\put(119.98,0){.}
\put(121.02,0){2}
\put(122.32,0){2}
\put(123.62,0){4}
\put(131.42,0){0}
\put(132.98,0){.}
\put(134.02,0){1}
\put(135.32,0){0}
\put(136.62,0){2}
\put(144.42,0){0}
\put(145.98,0){.}
\put(147.02,0){2}
\put(148.32,0){4}
\put(149.62,0){6}
\end{picture}

\noindent
\begin{picture}(159.90, 22.00)( -2.60, -4.00)
\put( -0.65, -2.50){\framebox(158.60, 18.00){}}
\put(-0.21,12){U}
\put(1.30,12){n}
\put(2.63,12){k}
\put(3.90,12){n}
\put(5.30,12){o}
\put(6.33,12){w}
\put(7.80,12){n}
\put(10.61,12){s}
\put(11.95,12){t}
\put(13.08,12){a}
\put(14.52,12){r}
\put(15.81,12){s}
\put(17.26,12){:}
\put(23.27,12){B}
\put(24.82,12){1}
\put(26.12,12){9}
\put(27.42,12){5}
\put(28.72,12){0}
\put(30.28,12){.}
\put(31.32,12){0}
\put(33.51,12){m}
\put(35.23,12){e}
\put(36.48,12){a}
\put(37.70,12){n}
\put(40.33,12){p}
\put(41.92,12){l}
\put(42.98,12){a}
\put(44.33,12){c}
\put(45.63,12){e}
\put(47.01,12){s}
\put(49.57,12){f}
\put(50.80,12){o}
\put(52.22,12){r}
\put(54.73,12){e}
\put(55.93,12){p}
\put(57.30,12){o}
\put(58.63,12){c}
\put(59.80,12){h}
\put(62.27,12){B}
\put(63.82,12){1}
\put(65.12,12){9}
\put(66.42,12){7}
\put(67.72,12){4}
\put(69.28,12){.}
\put(70.32,12){5}
\put(71.62,12){0}
\put(72.92,12){0}
\put(15.96,8){:}
\put(36.52,8){4}
\put(37.67,8){$-$}
\put(39.13,8){c}
\put(40.40,8){o}
\put(41.73,8){e}
\put(43.07,8){f}
\put(44.37,8){f}
\put(45.79,8){i}
\put(46.93,8){c}
\put(48.39,8){i}
\put(49.53,8){e}
\put(50.70,8){n}
\put(52.25,8){t}
\put(54.31,8){m}
\put(56,8){o}
\put(57.21,8){d}
\put(58.63,8){e}
\put(60.12,8){l}
\put(80.96,8){:}
\put(102.82,8){6}
\put(103.97,8){$-$}
\put(105.43,8){c}
\put(106.70,8){o}
\put(108.03,8){e}
\put(109.37,8){f}
\put(110.67,8){f}
\put(112.09,8){i}
\put(113.23,8){c}
\put(114.69,8){i}
\put(115.83,8){e}
\put(117,8){n}
\put(118.55,8){t}
\put(120.61,8){m}
\put(122.30,8){o}
\put(123.51,8){d}
\put(124.93,8){e}
\put(126.42,8){l}
\put(148.56,8){:}
\put(15.96,6){:}
\put(20.58,6){X}
\put(21.81,6){m}
\put(23.53,6){e}
\put(24.78,6){a}
\put(26.21,6){s}
\put(32.27,6){Y}
\put(33.51,6){m}
\put(35.23,6){e}
\put(36.48,6){a}
\put(37.91,6){s}
\put(51.80,6){R}
\put(53.10,6){A}
\put(70.04,6){D}
\put(71.63,6){e}
\put(72.93,6){c}
\put(80.96,6){:}
\put(85.58,6){X}
\put(86.81,6){m}
\put(88.53,6){e}
\put(89.78,6){a}
\put(91.21,6){s}
\put(97.27,6){Y}
\put(98.51,6){m}
\put(100.23,6){e}
\put(101.48,6){a}
\put(102.91,6){s}
\put(116.80,6){R}
\put(118.10,6){A}
\put(136.34,6){D}
\put(137.93,6){e}
\put(139.23,6){c}
\put(148.56,6){:}
\put(15.96,4){:}
\put(80.96,4){:}
\put(148.56,4){:}
\put(-0.13,2){C}
\put(1.38,2){a}
\put(2.60,2){n}
\put(3.91,2){d}
\put(5.49,2){i}
\put(6.51,2){d}
\put(7.88,2){a}
\put(9.35,2){t}
\put(10.53,2){e}
\put(15.96,2){:}
\put(20.77,2){$-$}
\put(22.22,2){5}
\put(23.78,2){.}
\put(24.82,2){1}
\put(26.12,2){0}
\put(27.42,2){3}
\put(31,2){$+$}
\put(32.62,2){5}
\put(33.92,2){8}
\put(35.48,2){.}
\put(36.52,2){8}
\put(37.82,2){6}
\put(39.12,2){8}
\put(41.57,2){$-$}
\put(42.78,2){$>$}
\put(45.62,2){1}
\put(46.92,2){9}
\put(49.52,2){0}
\put(50.82,2){5}
\put(53.42,2){0}
\put(54.72,2){1}
\put(56.28,2){.}
\put(57.32,2){6}
\put(58.62,2){9}
\put(59.92,2){7}
\put(63.67,2){$-$}
\put(65.12,2){6}
\put(66.42,2){3}
\put(69.02,2){5}
\put(70.32,2){6}
\put(72.92,2){1}
\put(74.22,2){7}
\put(75.78,2){.}
\put(76.82,2){1}
\put(78.12,2){3}
\put(80.96,2){:}
\put(85.77,2){$-$}
\put(87.22,2){5}
\put(88.78,2){.}
\put(89.82,2){1}
\put(91.12,2){0}
\put(92.42,2){3}
\put(96,2){$+$}
\put(97.62,2){5}
\put(98.92,2){8}
\put(100.48,2){.}
\put(101.52,2){8}
\put(102.82,2){6}
\put(104.12,2){8}
\put(106.57,2){$-$}
\put(107.78,2){$>$}
\put(110.62,2){1}
\put(111.92,2){9}
\put(114.52,2){0}
\put(115.82,2){5}
\put(118.42,2){0}
\put(119.72,2){1}
\put(121.28,2){.}
\put(122.32,2){7}
\put(123.62,2){9}
\put(124.92,2){4}
\put(126.22,2){2}
\put(129.97,2){$-$}
\put(131.42,2){6}
\put(132.72,2){3}
\put(135.32,2){5}
\put(136.62,2){6}
\put(139.22,2){1}
\put(140.52,2){6}
\put(142.08,2){.}
\put(143.12,2){7}
\put(144.42,2){0}
\put(145.72,2){2}
\put(148.56,2){:}
\put(-0.19,0){R}
\put(1.38,0){a}
\put(2.61,0){d}
\put(4.19,0){i}
\put(5.30,0){o}
\put(7.83,0){p}
\put(9.20,0){o}
\put(10.61,0){s}
\put(15.96,0){:}
\put(20.77,0){$-$}
\put(22.22,0){5}
\put(23.78,0){.}
\put(24.82,0){1}
\put(26.12,0){2}
\put(27.42,0){0}
\put(31,0){$+$}
\put(32.62,0){5}
\put(33.92,0){8}
\put(35.48,0){.}
\put(36.52,0){8}
\put(37.82,0){6}
\put(39.12,0){2}
\put(41.48,0){$<$}
\put(42.87,0){$-$}
\put(45.62,0){1}
\put(46.92,0){9}
\put(49.52,0){0}
\put(50.82,0){5}
\put(53.42,0){0}
\put(54.72,0){1}
\put(56.28,0){.}
\put(57.32,0){8}
\put(58.62,0){7}
\put(59.92,0){4}
\put(63.67,0){$-$}
\put(65.12,0){6}
\put(66.42,0){3}
\put(69.02,0){5}
\put(70.32,0){6}
\put(72.92,0){1}
\put(74.22,0){7}
\put(75.78,0){.}
\put(76.82,0){5}
\put(78.12,0){4}
\put(80.96,0){:}
\put(85.77,0){$-$}
\put(87.22,0){5}
\put(88.78,0){.}
\put(89.82,0){1}
\put(91.12,0){1}
\put(92.42,0){1}
\put(96,0){$+$}
\put(97.62,0){5}
\put(98.92,0){8}
\put(100.48,0){.}
\put(101.52,0){8}
\put(102.82,0){5}
\put(104.12,0){6}
\put(106.48,0){$<$}
\put(107.87,0){$-$}
\put(110.62,0){1}
\put(111.92,0){9}
\put(114.52,0){0}
\put(115.82,0){5}
\put(118.42,0){0}
\put(119.72,0){1}
\put(121.28,0){.}
\put(122.32,0){8}
\put(123.62,0){7}
\put(124.92,0){4}
\put(126.22,0){1}
\put(129.97,0){$-$}
\put(131.42,0){6}
\put(132.72,0){3}
\put(135.32,0){5}
\put(136.62,0){6}
\put(139.22,0){1}
\put(140.52,0){7}
\put(142.08,0){.}
\put(143.12,0){5}
\put(144.42,0){3}
\put(145.72,0){9}
\put(148.56,0){:}
\end{picture}
 
\end{tiny}
%%%%%%%%%%%%%%%%%%%%%%%%%%%%%%%%%%%%%%%%%%%

\vspace{20mm}

The residuals \verb|dX| and \verb|dY| are, loosely, in
{\it standard coordinates}; except at the poles,
positive \verb|dX| means
that the measured position of the image lies to the east of the
expected position, and positive \verb|dY| means that the measured position
lies to the north of the expected position.

If a 7-9 coefficient fit is requested, and providing there are at
least ten reference stars, an extra section of report precedes the
unknowns.  If the fit is successful, the two solutions used to process
the unknown stars are the 6 and 7-9 coefficient ones, and the
4-coefficient model is not used.

\pagebreak
{\bf APPENDIX C -- Effective Wavelengths}
\vspace{5mm}

Most ASTROM reductions work in mean \radec\ of date, absorbing
various small rotations and distortions of the field into the
fit.  This normally delivers results of adequate accuracy.
However, for slightly improved accuracy, especially at low
elevations and with wide fields, ASTROM can optionally be supplied with
additional information (an accurate date and time,
observatory location, {\it etc.}) to enable reduction in
{\it observed}\, place.  The further possibility is then
available of correcting for atmospheric dispersion, by
supplying effective colours for the reference and unknown stars;
this can sometimes be important.

The following tables estimate the effective wavelength for different
spectral types and detector/filter passbands.  The figures given are the
median wavelength of the appropriate blackbody spectrum within the
specified band, and are thus only a very rough guide.  Some detectors
peak strongly; some filters leak outside their nominal
passband; both may roll off gradually at the edges of the passband;
some stellar spectra have pronounced absorption or emission
features.

The tables cover blue cutoffs from 320~nm and red cutoffs up to 800~nm,
in steps of 20~nm and with a minimum passband of 100~nm, for 13
colour temperatures.  B$-$V and U$-$V colours and
the nearest main-sequence spectral types are also given, as listed
in Table~99 of {\it Astrophysical Quantities}.

For any given exposure, only one line in the tables will be
needed, corresponding to the spectral response of the detector plus
filter used.  The appropriate effective wavelength for any star in
the exposure can then be found by looking in the appropriate
column.

\clearpage

%%%%%%%%%%%%%%%%%%%%%%%%%%%%%%%%%%%%%%%%%%%%
% LaTeX source generated by program EWLLTX %
%  P.T.Wallace   Starlink   17 March 1989  %
%%%%%%%%%%%%%%%%%%%%%%%%%%%%%%%%%%%%%%%%%%%%

\begin{tiny}

\noindent
\begin{center}
\begin{tabular}{|c|c
@{\hspace{2ex}}c
@{\hspace{2ex}}c
@{\hspace{2ex}}c
@{\hspace{2ex}}c
@{\hspace{2ex}}c
@{\hspace{2ex}}c
@{\hspace{2ex}}c
@{\hspace{2ex}}c
@{\hspace{2ex}}c
@{\hspace{2ex}}c
@{\hspace{2ex}}c
@{\hspace{2ex}}c|l}
\cline{1-14}
Band
&  3000
&  3800
&  4500
&  5400
&  6000
&  6700
&  7600
&  9000
& 11100
& 15400
& 23000
& 38000
& 70000
& T$_c$ ($^\circ$K) \\
\cline{1-14}
320-420 & 392 & 386 & 382 & 377 & 375 & 373 & 371 & 368 & 365 & 362 & 360 & 359 & 358 & \\
320-440 & 408 & 400 & 395 & 389 & 386 & 383 & 380 & 377 & 373 & 369 & 366 & 364 & 363 & \\
320-460 & 424 & 415 & 408 & 401 & 397 & 394 & 390 & 385 & 381 & 375 & 372 & 369 & 367 & \\
320-480 & 440 & 430 & 422 & 413 & 408 & 404 & 399 & 393 & 387 & 381 & 377 & 373 & 371 & \\
320-500 & 456 & 444 & 435 & 424 & 419 & 413 & 407 & 401 & 394 & 387 & 381 & 377 & 375 & \\
320-520 & 472 & 458 & 448 & 436 & 429 & 423 & 416 & 408 & 400 & 391 & 385 & 381 & 378 & \\
320-540 & 488 & 472 & 460 & 447 & 439 & 432 & 424 & 415 & 406 & 396 & 389 & 384 & 381 & \\
320-560 & 503 & 486 & 473 & 457 & 449 & 441 & 432 & 421 & 411 & 400 & 392 & 387 & 384 & \\
320-580 & 519 & 500 & 485 & 468 & 458 & 449 & 439 & 428 & 416 & 404 & 395 & 389 & 386 & \\
320-600 & 534 & 513 & 497 & 478 & 468 & 457 & 446 & 434 & 421 & 408 & 398 & 392 & 388 & \\
320-620 & 549 & 527 & 508 & 488 & 476 & 465 & 453 & 439 & 426 & 411 & 401 & 394 & 390 & \\
320-640 & 563 & 540 & 520 & 498 & 485 & 473 & 459 & 444 & 430 & 414 & 403 & 396 & 391 & \\
320-660 & 578 & 552 & 531 & 507 & 493 & 480 & 466 & 449 & 434 & 417 & 405 & 397 & 393 & \\
320-680 & 592 & 565 & 542 & 516 & 501 & 487 & 472 & 454 & 437 & 419 & 407 & 399 & 394 & \\
320-700 & 607 & 577 & 552 & 525 & 509 & 493 & 477 & 459 & 441 & 422 & 409 & 400 & 395 & \\
320-720 & 620 & 589 & 563 & 533 & 516 & 500 & 482 & 463 & 444 & 424 & 410 & 402 & 396 & \\
320-740 & 634 & 601 & 573 & 541 & 523 & 506 & 488 & 467 & 447 & 426 & 412 & 403 & 397 & \\
320-760 & 648 & 612 & 582 & 549 & 530 & 512 & 492 & 470 & 450 & 428 & 413 & 404 & 398 & \\
320-780 & 661 & 623 & 592 & 557 & 537 & 517 & 497 & 474 & 452 & 430 & 414 & 405 & 399 & \\
320-800 & 674 & 634 & 601 & 564 & 543 & 522 & 501 & 799 & 799 & 799 & 799 & 799 & 799 & \\
340-440 & 410 & 404 & 400 & 396 & 394 & 392 & 390 & 387 & 385 & 382 & 380 & 379 & 378 & \\
340-460 & 426 & 418 & 413 & 407 & 405 & 402 & 399 & 396 & 393 & 389 & 387 & 385 & 383 & \\
340-480 & 441 & 432 & 426 & 419 & 415 & 412 & 408 & 404 & 400 & 396 & 392 & 390 & 388 & \\
340-500 & 457 & 446 & 438 & 430 & 426 & 421 & 417 & 412 & 407 & 401 & 397 & 394 & 393 & \\
340-520 & 473 & 460 & 451 & 441 & 436 & 431 & 426 & 419 & 413 & 407 & 402 & 398 & 396 & \\
340-540 & 488 & 474 & 464 & 452 & 446 & 440 & 434 & 427 & 420 & 412 & 406 & 402 & 400 & \\
340-560 & 504 & 488 & 476 & 463 & 456 & 449 & 442 & 433 & 425 & 416 & 410 & 405 & 403 & \\
340-580 & 519 & 502 & 488 & 473 & 465 & 457 & 449 & 440 & 431 & 421 & 413 & 408 & 406 & \\
340-600 & 534 & 515 & 500 & 483 & 474 & 465 & 456 & 446 & 436 & 425 & 417 & 411 & 408 & \\
340-620 & 549 & 528 & 511 & 493 & 483 & 473 & 463 & 452 & 440 & 428 & 419 & 414 & 410 & \\
340-640 & 564 & 541 & 523 & 503 & 492 & 481 & 470 & 457 & 445 & 432 & 422 & 416 & 412 & \\
340-660 & 578 & 554 & 534 & 512 & 500 & 488 & 476 & 462 & 449 & 435 & 425 & 418 & 414 & \\
340-680 & 593 & 566 & 545 & 521 & 508 & 495 & 482 & 467 & 453 & 438 & 427 & 420 & 416 & \\
340-700 & 607 & 578 & 555 & 530 & 516 & 502 & 488 & 472 & 456 & 440 & 429 & 421 & 417 & \\
340-720 & 621 & 590 & 565 & 538 & 523 & 508 & 493 & 476 & 460 & 443 & 431 & 423 & 418 & \\
340-740 & 634 & 602 & 575 & 546 & 530 & 515 & 498 & 480 & 463 & 445 & 433 & 424 & 420 & \\
340-760 & 648 & 613 & 585 & 554 & 537 & 520 & 503 & 484 & 466 & 447 & 434 & 426 & 421 & \\
340-780 & 661 & 624 & 595 & 562 & 544 & 526 & 508 & 488 & 469 & 449 & 436 & 427 & 422 & \\
340-800 & 674 & 635 & 603 & 569 & 550 & 531 & 512 & 799 & 799 & 799 & 799 & 799 & 799 & \\
360-460 & 428 & 422 & 418 & 415 & 413 & 411 & 409 & 407 & 405 & 402 & 401 & 399 & 399 & \\
360-480 & 443 & 436 & 431 & 426 & 423 & 421 & 418 & 415 & 412 & 409 & 407 & 405 & 404 & \\
360-500 & 459 & 450 & 443 & 437 & 434 & 430 & 427 & 423 & 420 & 416 & 413 & 410 & 409 & \\
360-520 & 474 & 463 & 456 & 448 & 444 & 440 & 436 & 431 & 427 & 422 & 418 & 415 & 414 & \\
360-540 & 490 & 477 & 468 & 459 & 454 & 449 & 444 & 438 & 433 & 427 & 423 & 419 & 418 & \\
360-560 & 505 & 491 & 480 & 469 & 463 & 458 & 452 & 445 & 439 & 432 & 427 & 423 & 421 & \\
360-580 & 520 & 504 & 492 & 479 & 473 & 466 & 460 & 452 & 445 & 437 & 431 & 427 & 424 & \\
360-600 & 535 & 517 & 504 & 489 & 482 & 475 & 467 & 458 & 450 & 441 & 435 & 430 & 427 & \\
360-620 & 550 & 530 & 515 & 499 & 491 & 482 & 474 & 464 & 455 & 445 & 438 & 433 & 430 & \\
360-640 & 564 & 543 & 526 & 509 & 499 & 490 & 481 & 470 & 460 & 449 & 441 & 436 & 432 & \\
360-660 & 579 & 555 & 537 & 518 & 508 & 497 & 487 & 476 & 464 & 452 & 444 & 438 & 434 & \\
360-680 & 593 & 568 & 548 & 527 & 516 & 505 & 493 & 481 & 469 & 456 & 446 & 440 & 436 & \\
360-700 & 607 & 580 & 558 & 536 & 523 & 511 & 499 & 486 & 472 & 459 & 449 & 442 & 438 & \\
360-720 & 621 & 592 & 569 & 544 & 531 & 518 & 505 & 490 & 476 & 461 & 451 & 444 & 440 & \\
360-740 & 635 & 603 & 579 & 552 & 538 & 524 & 510 & 494 & 480 & 464 & 453 & 446 & 441 & \\
360-760 & 648 & 615 & 588 & 560 & 545 & 530 & 515 & 499 & 483 & 466 & 455 & 447 & 443 & \\
360-780 & 662 & 626 & 598 & 568 & 551 & 536 & 520 & 502 & 486 & 468 & 456 & 448 & 444 & \\
360-800 & 674 & 636 & 607 & 575 & 558 & 541 & 524 & 799 & 799 & 799 & 799 & 799 & 799 & \\
380-480 & 446 & 441 & 437 & 434 & 432 & 430 & 428 & 427 & 425 & 423 & 421 & 420 & 419 & \\
380-500 & 461 & 454 & 449 & 444 & 442 & 440 & 437 & 435 & 432 & 429 & 427 & 426 & 425 & \\
380-520 & 476 & 467 & 461 & 455 & 452 & 449 & 446 & 443 & 440 & 436 & 433 & 431 & 430 & \\
380-540 & 491 & 481 & 473 & 466 & 462 & 458 & 455 & 450 & 446 & 442 & 438 & 436 & 435 & \\
380-560 & 506 & 494 & 485 & 476 & 472 & 467 & 463 & 458 & 453 & 447 & 443 & 440 & 439 & \\
380-580 & 521 & 507 & 497 & 486 & 481 & 476 & 471 & 465 & 459 & 452 & 448 & 444 & 443 & \\
380-600 & 536 & 520 & 508 & 496 & 490 & 484 & 478 & 471 & 465 & 457 & 452 & 448 & 446 & \\
380-620 & 551 & 533 & 520 & 506 & 499 & 492 & 485 & 477 & 470 & 462 & 456 & 452 & 449 & \\
380-640 & 565 & 545 & 531 & 516 & 508 & 500 & 492 & 483 & 475 & 466 & 459 & 455 & 452 & \\
380-660 & 580 & 558 & 542 & 525 & 516 & 507 & 499 & 489 & 480 & 470 & 462 & 457 & 454 & \\
380-680 & 594 & 570 & 552 & 534 & 524 & 515 & 505 & 494 & 484 & 473 & 465 & 460 & 457 & \\
380-700 & 608 & 582 & 563 & 542 & 532 & 521 & 511 & 499 & 488 & 476 & 468 & 462 & 459 & \\
380-720 & 622 & 594 & 573 & 551 & 539 & 528 & 517 & 504 & 492 & 480 & 470 & 464 & 461 & \\
380-740 & 636 & 606 & 583 & 559 & 546 & 534 & 522 & 509 & 496 & 482 & 473 & 466 & 462 & \\
380-760 & 649 & 617 & 592 & 567 & 553 & 540 & 528 & 513 & 499 & 485 & 475 & 468 & 464 & \\
380-780 & 662 & 628 & 602 & 575 & 560 & 546 & 533 & 517 & 503 & 488 & 477 & 470 & 465 & \\
380-800 & 675 & 638 & 611 & 582 & 566 & 552 & 537 & 799 & 799 & 799 & 799 & 799 & 799 & \\
400-500 & 464 & 459 & 456 & 453 & 451 & 450 & 448 & 446 & 445 & 443 & 441 & 440 & 440 & \\
400-520 & 479 & 472 & 468 & 463 & 461 & 459 & 457 & 455 & 452 & 450 & 448 & 446 & 445 & \\
400-540 & 494 & 485 & 479 & 474 & 471 & 468 & 465 & 462 & 459 & 456 & 454 & 452 & 451 & \\
400-560 & 508 & 498 & 491 & 484 & 480 & 477 & 474 & 470 & 466 & 462 & 459 & 457 & 455 & \\
400-580 & 523 & 511 & 502 & 494 & 490 & 486 & 482 & 477 & 473 & 468 & 464 & 461 & 460 & \\
400-600 & 538 & 524 & 514 & 504 & 499 & 494 & 489 & 484 & 479 & 473 & 469 & 466 & 464 & \\
400-620 & 552 & 536 & 525 & 514 & 508 & 502 & 497 & 490 & 484 & 478 & 473 & 469 & 467 & \\
400-640 & 567 & 549 & 536 & 523 & 517 & 510 & 504 & 497 & 490 & 482 & 477 & 473 & 471 & \\
400-660 & 581 & 561 & 547 & 532 & 525 & 518 & 511 & 503 & 495 & 486 & 480 & 476 & 474 & \\
400-680 & 595 & 573 & 557 & 541 & 533 & 525 & 517 & 508 & 500 & 490 & 484 & 479 & 476 & \\
400-700 & 609 & 585 & 568 & 550 & 541 & 532 & 523 & 513 & 504 & 494 & 487 & 482 & 479 & \\
\cline{1-14}
nm & M5
& M0
& K5
& K0
& G5
& G0
& F5
& F0
& A5
& A0
& B5
& B0
& O5 & \\
& $+1.61$
& $+1.39$
& $+1.11$
& $+0.84$
& $+0.70$
& $+0.57$
& $+0.45$
& $+0.30$
& $+0.16$
& $0.00$ 
& $-0.17$
& $-0.31$
& $-0.45$ & B$-$V \\
& $+1.19$
& $+1.24$
& $+1.06$
& $+0.46$
& $+0.20$
& $+0.04$
& $-0.01$
& $+0.02$
& $+0.09$
& $0.00$ 
& $-0.56$
& $-1.07$
& $-1.20$ & U$-$B \\
\cline{1-14}
\end{tabular}
\end{center}
\clearpage

\noindent
\begin{center}
\begin{tabular}{|c|c
@{\hspace{2ex}}c
@{\hspace{2ex}}c
@{\hspace{2ex}}c
@{\hspace{2ex}}c
@{\hspace{2ex}}c
@{\hspace{2ex}}c
@{\hspace{2ex}}c
@{\hspace{2ex}}c
@{\hspace{2ex}}c
@{\hspace{2ex}}c
@{\hspace{2ex}}c
@{\hspace{2ex}}c|l}
\cline{1-14}
Band
&  3000
&  3800
&  4500
&  5400
&  6000
&  6700
&  7600
&  9000
& 11100
& 15400
& 23000
& 38000
& 70000
& T$_c$ ($^\circ$K) \\
\cline{1-14}
400-720 & 623 & 597 & 578 & 558 & 548 & 539 & 529 & 519 & 508 & 497 & 490 & 484 & 481 & \\
400-740 & 637 & 608 & 588 & 567 & 556 & 545 & 535 & 523 & 512 & 501 & 492 & 486 & 483 & \\
400-760 & 650 & 620 & 597 & 575 & 563 & 551 & 540 & 528 & 516 & 504 & 495 & 489 & 485 & \\
400-780 & 663 & 631 & 607 & 582 & 569 & 557 & 545 & 532 & 520 & 506 & 497 & 490 & 487 & \\
400-800 & 676 & 641 & 615 & 589 & 576 & 563 & 550 & 799 & 799 & 799 & 799 & 799 & 799 & \\
420-520 & 483 & 478 & 475 & 472 & 470 & 469 & 468 & 466 & 464 & 463 & 461 & 461 & 460 & \\
420-540 & 497 & 490 & 486 & 482 & 480 & 478 & 476 & 474 & 472 & 470 & 468 & 467 & 466 & \\
420-560 & 511 & 503 & 498 & 492 & 490 & 487 & 485 & 482 & 479 & 476 & 474 & 472 & 471 & \\
420-580 & 526 & 516 & 509 & 502 & 499 & 496 & 493 & 490 & 486 & 482 & 480 & 478 & 476 & \\
420-600 & 540 & 528 & 520 & 512 & 508 & 505 & 501 & 497 & 493 & 488 & 485 & 482 & 481 & \\
420-620 & 555 & 541 & 531 & 522 & 517 & 513 & 508 & 503 & 499 & 493 & 489 & 487 & 485 & \\
420-640 & 569 & 553 & 542 & 531 & 526 & 521 & 516 & 510 & 504 & 498 & 494 & 491 & 489 & \\
420-660 & 583 & 565 & 553 & 541 & 534 & 529 & 523 & 516 & 510 & 503 & 498 & 494 & 492 & \\
420-680 & 597 & 577 & 563 & 550 & 543 & 536 & 529 & 522 & 515 & 507 & 502 & 498 & 495 & \\
420-700 & 611 & 589 & 573 & 558 & 550 & 543 & 536 & 528 & 520 & 511 & 505 & 501 & 498 & \\
420-720 & 624 & 601 & 584 & 567 & 558 & 550 & 542 & 533 & 524 & 515 & 508 & 504 & 501 & \\
420-740 & 638 & 612 & 593 & 575 & 565 & 557 & 548 & 538 & 528 & 518 & 511 & 506 & 503 & \\
420-760 & 651 & 623 & 603 & 583 & 573 & 563 & 553 & 543 & 532 & 522 & 514 & 509 & 505 & \\
420-780 & 664 & 634 & 612 & 591 & 579 & 569 & 559 & 547 & 536 & 525 & 516 & 511 & 507 & \\
420-800 & 677 & 644 & 621 & 598 & 586 & 575 & 563 & 799 & 799 & 799 & 799 & 799 & 799 & \\
440-540 & 502 & 497 & 494 & 491 & 490 & 489 & 487 & 486 & 484 & 483 & 482 & 481 & 480 & \\
440-560 & 515 & 509 & 505 & 501 & 499 & 498 & 496 & 494 & 492 & 490 & 488 & 487 & 486 & \\
440-580 & 529 & 521 & 516 & 511 & 509 & 507 & 504 & 502 & 499 & 497 & 494 & 493 & 492 & \\
440-600 & 543 & 534 & 527 & 521 & 518 & 515 & 512 & 509 & 506 & 503 & 500 & 498 & 497 & \\
440-620 & 557 & 546 & 538 & 531 & 527 & 524 & 520 & 516 & 513 & 508 & 505 & 503 & 502 & \\
440-640 & 571 & 558 & 549 & 540 & 536 & 532 & 528 & 523 & 519 & 514 & 510 & 508 & 506 & \\
440-660 & 585 & 570 & 559 & 549 & 544 & 540 & 535 & 530 & 524 & 519 & 515 & 512 & 510 & \\
440-680 & 599 & 582 & 570 & 558 & 553 & 547 & 542 & 536 & 530 & 524 & 519 & 516 & 514 & \\
440-700 & 613 & 593 & 580 & 567 & 561 & 554 & 548 & 542 & 535 & 528 & 523 & 519 & 517 & \\
440-720 & 626 & 605 & 590 & 576 & 568 & 561 & 555 & 547 & 540 & 532 & 526 & 522 & 520 & \\
440-740 & 640 & 616 & 600 & 584 & 576 & 568 & 561 & 552 & 544 & 536 & 530 & 525 & 523 & \\
440-760 & 653 & 627 & 609 & 592 & 583 & 575 & 566 & 557 & 549 & 539 & 533 & 528 & 525 & \\
440-780 & 666 & 638 & 619 & 600 & 590 & 581 & 572 & 562 & 553 & 543 & 535 & 530 & 527 & \\
440-800 & 678 & 648 & 627 & 607 & 596 & 587 & 577 & 799 & 799 & 799 & 799 & 799 & 799 & \\
460-560 & 520 & 516 & 513 & 511 & 509 & 508 & 507 & 506 & 504 & 503 & 502 & 501 & 501 & \\
460-580 & 534 & 528 & 524 & 521 & 519 & 517 & 516 & 514 & 512 & 510 & 509 & 508 & 507 & \\
460-600 & 547 & 540 & 535 & 530 & 528 & 526 & 524 & 522 & 519 & 517 & 515 & 513 & 513 & \\
460-620 & 561 & 552 & 546 & 540 & 537 & 535 & 532 & 529 & 526 & 523 & 521 & 519 & 518 & \\
460-640 & 575 & 564 & 556 & 549 & 546 & 543 & 540 & 536 & 533 & 529 & 526 & 524 & 523 & \\
460-660 & 588 & 575 & 567 & 559 & 555 & 551 & 547 & 543 & 539 & 534 & 531 & 529 & 527 & \\
460-680 & 602 & 587 & 577 & 568 & 563 & 559 & 554 & 549 & 545 & 539 & 536 & 533 & 531 & \\
460-700 & 616 & 599 & 587 & 576 & 571 & 566 & 561 & 555 & 550 & 544 & 540 & 537 & 535 & \\
460-720 & 629 & 610 & 597 & 585 & 579 & 573 & 567 & 561 & 555 & 549 & 544 & 540 & 538 & \\
460-740 & 642 & 621 & 607 & 593 & 586 & 580 & 574 & 567 & 560 & 553 & 547 & 544 & 541 & \\
460-760 & 655 & 632 & 616 & 601 & 594 & 587 & 580 & 572 & 565 & 557 & 551 & 547 & 544 & \\
460-780 & 668 & 643 & 626 & 609 & 601 & 593 & 585 & 577 & 569 & 560 & 554 & 550 & 547 & \\
460-800 & 681 & 653 & 634 & 616 & 607 & 599 & 591 & 799 & 799 & 799 & 799 & 799 & 799 & \\
480-580 & 539 & 535 & 533 & 530 & 529 & 528 & 527 & 526 & 524 & 523 & 522 & 521 & 521 & \\
480-600 & 552 & 547 & 543 & 540 & 538 & 537 & 535 & 534 & 532 & 530 & 529 & 528 & 527 & \\
480-620 & 566 & 558 & 554 & 550 & 547 & 546 & 544 & 541 & 539 & 537 & 535 & 534 & 533 & \\
480-640 & 579 & 570 & 564 & 559 & 556 & 554 & 552 & 549 & 546 & 543 & 541 & 540 & 539 & \\
480-660 & 592 & 582 & 575 & 568 & 565 & 562 & 559 & 556 & 553 & 549 & 547 & 545 & 544 & \\
480-680 & 606 & 593 & 585 & 577 & 574 & 570 & 567 & 563 & 559 & 555 & 552 & 549 & 548 & \\
480-700 & 619 & 604 & 595 & 586 & 582 & 578 & 574 & 569 & 565 & 560 & 556 & 554 & 552 & \\
480-720 & 632 & 616 & 605 & 595 & 590 & 585 & 580 & 575 & 570 & 565 & 561 & 558 & 556 & \\
480-740 & 645 & 627 & 615 & 603 & 597 & 592 & 587 & 581 & 575 & 569 & 565 & 562 & 560 & \\
480-760 & 658 & 638 & 624 & 611 & 605 & 599 & 593 & 587 & 580 & 574 & 569 & 565 & 563 & \\
480-780 & 671 & 648 & 633 & 619 & 612 & 605 & 599 & 592 & 585 & 578 & 572 & 568 & 566 & \\
480-800 & 683 & 658 & 642 & 626 & 619 & 612 & 604 & 799 & 799 & 799 & 799 & 799 & 799 & \\
500-600 & 558 & 554 & 552 & 550 & 549 & 548 & 547 & 546 & 544 & 543 & 542 & 542 & 541 & \\
500-620 & 571 & 566 & 562 & 559 & 558 & 557 & 555 & 554 & 552 & 550 & 549 & 548 & 548 & \\
500-640 & 584 & 577 & 573 & 569 & 567 & 565 & 563 & 561 & 559 & 557 & 556 & 554 & 554 & \\
500-660 & 597 & 588 & 583 & 578 & 576 & 573 & 571 & 569 & 566 & 564 & 562 & 560 & 559 & \\
500-680 & 610 & 600 & 593 & 587 & 584 & 582 & 579 & 576 & 573 & 570 & 567 & 565 & 564 & \\
500-700 & 623 & 611 & 603 & 596 & 593 & 589 & 586 & 583 & 579 & 575 & 572 & 570 & 569 & \\
500-720 & 636 & 622 & 613 & 605 & 601 & 597 & 593 & 589 & 585 & 581 & 577 & 575 & 573 & \\
500-740 & 649 & 633 & 623 & 613 & 609 & 604 & 600 & 595 & 590 & 585 & 582 & 579 & 577 & \\
500-760 & 662 & 644 & 632 & 621 & 616 & 611 & 606 & 601 & 596 & 590 & 586 & 583 & 581 & \\
500-780 & 675 & 655 & 642 & 629 & 623 & 618 & 613 & 606 & 601 & 595 & 590 & 587 & 585 & \\
500-800 & 687 & 665 & 650 & 637 & 630 & 624 & 618 & 799 & 799 & 799 & 799 & 799 & 799 & \\
520-620 & 577 & 574 & 571 & 569 & 568 & 567 & 567 & 566 & 565 & 563 & 563 & 562 & 562 & \\
520-640 & 590 & 585 & 582 & 579 & 578 & 576 & 575 & 574 & 572 & 571 & 570 & 569 & 568 & \\
520-660 & 603 & 596 & 592 & 588 & 586 & 585 & 583 & 581 & 580 & 578 & 576 & 575 & 574 & \\
520-680 & 615 & 607 & 602 & 597 & 595 & 593 & 591 & 589 & 586 & 584 & 582 & 581 & 580 & \\
520-700 & 628 & 618 & 612 & 606 & 604 & 601 & 599 & 596 & 593 & 590 & 588 & 586 & 585 & \\
520-720 & 641 & 629 & 622 & 615 & 612 & 609 & 606 & 602 & 599 & 596 & 593 & 591 & 590 & \\
520-740 & 653 & 640 & 631 & 624 & 620 & 616 & 613 & 609 & 605 & 601 & 598 & 596 & 594 & \\
520-760 & 666 & 651 & 641 & 632 & 628 & 624 & 619 & 615 & 611 & 606 & 603 & 600 & 599 & \\
520-780 & 679 & 661 & 650 & 640 & 635 & 630 & 626 & 621 & 616 & 611 & 607 & 604 & 602 & \\
520-800 & 691 & 671 & 659 & 648 & 642 & 637 & 632 & 799 & 799 & 799 & 799 & 799 & 799 & \\
540-640 & 597 & 593 & 591 & 589 & 588 & 587 & 586 & 586 & 585 & 584 & 583 & 582 & 582 & \\
540-660 & 609 & 604 & 601 & 599 & 597 & 596 & 595 & 594 & 592 & 591 & 590 & 589 & 589 & \\
540-680 & 621 & 615 & 611 & 608 & 606 & 605 & 603 & 601 & 600 & 598 & 596 & 595 & 595 & \\
540-700 & 634 & 626 & 621 & 617 & 615 & 613 & 611 & 609 & 607 & 604 & 603 & 601 & 600 & \\
540-720 & 646 & 637 & 631 & 626 & 623 & 621 & 618 & 616 & 613 & 610 & 608 & 607 & 606 & \\
\cline{1-14}
nm & M5
& M0
& K5
& K0
& G5
& G0
& F5
& F0
& A5
& A0
& B5
& B0
& O5 & \\
& $+1.61$
& $+1.39$
& $+1.11$
& $+0.84$
& $+0.70$
& $+0.57$
& $+0.45$
& $+0.30$
& $+0.16$
& $0.00$ 
& $-0.17$
& $-0.31$
& $-0.45$ & B$-$V \\
& $+1.19$
& $+1.24$
& $+1.06$
& $+0.46$
& $+0.20$
& $+0.04$
& $-0.01$
& $+0.02$
& $+0.09$
& $0.00$ 
& $-0.56$
& $-1.07$
& $-1.20$ & U$-$B \\
\cline{1-14}
\end{tabular}
\end{center}
\clearpage

\noindent
\begin{center}
\begin{tabular}{|c|c
@{\hspace{2ex}}c
@{\hspace{2ex}}c
@{\hspace{2ex}}c
@{\hspace{2ex}}c
@{\hspace{2ex}}c
@{\hspace{2ex}}c
@{\hspace{2ex}}c
@{\hspace{2ex}}c
@{\hspace{2ex}}c
@{\hspace{2ex}}c
@{\hspace{2ex}}c
@{\hspace{2ex}}c|l}
\cline{1-14}
Band
&  3000
&  3800
&  4500
&  5400
&  6000
&  6700
&  7600
&  9000
& 11100
& 15400
& 23000
& 38000
& 70000
& T$_c$ ($^\circ$K) \\
\cline{1-14}
540-740 & 659 & 647 & 641 & 634 & 631 & 628 & 626 & 622 & 619 & 616 & 614 & 612 & 611 & \\
540-760 & 671 & 658 & 650 & 643 & 639 & 636 & 633 & 629 & 625 & 622 & 619 & 617 & 615 & \\
540-780 & 683 & 669 & 659 & 651 & 647 & 643 & 639 & 635 & 631 & 627 & 623 & 621 & 620 & \\
540-800 & 695 & 678 & 668 & 658 & 654 & 650 & 645 & 799 & 799 & 799 & 799 & 799 & 799 & \\
560-660 & 616 & 613 & 611 & 609 & 608 & 607 & 606 & 605 & 605 & 604 & 603 & 602 & 602 & \\
560-680 & 628 & 623 & 621 & 618 & 617 & 616 & 615 & 614 & 612 & 611 & 610 & 609 & 609 & \\
560-700 & 640 & 634 & 630 & 627 & 626 & 624 & 623 & 621 & 620 & 618 & 617 & 616 & 615 & \\
560-720 & 652 & 645 & 640 & 636 & 634 & 632 & 631 & 629 & 627 & 625 & 623 & 622 & 621 & \\
560-740 & 664 & 655 & 650 & 645 & 643 & 640 & 638 & 636 & 633 & 631 & 629 & 627 & 626 & \\
560-760 & 677 & 666 & 659 & 654 & 651 & 648 & 645 & 642 & 640 & 637 & 634 & 633 & 632 & \\
560-780 & 689 & 676 & 669 & 662 & 658 & 655 & 652 & 649 & 646 & 642 & 639 & 638 & 636 & \\
560-800 & 700 & 686 & 677 & 670 & 666 & 662 & 659 & 799 & 799 & 799 & 799 & 799 & 799 & \\
580-680 & 635 & 632 & 630 & 629 & 628 & 627 & 626 & 625 & 625 & 624 & 623 & 623 & 622 & \\
580-700 & 647 & 643 & 640 & 638 & 637 & 636 & 635 & 634 & 632 & 631 & 630 & 630 & 629 & \\
580-720 & 659 & 653 & 650 & 647 & 645 & 644 & 643 & 641 & 640 & 638 & 637 & 636 & 636 & \\
580-740 & 671 & 664 & 660 & 656 & 654 & 652 & 651 & 649 & 647 & 645 & 643 & 642 & 642 & \\
580-760 & 683 & 674 & 669 & 664 & 662 & 660 & 658 & 656 & 654 & 651 & 649 & 648 & 647 & \\
580-780 & 695 & 685 & 678 & 673 & 670 & 668 & 665 & 663 & 660 & 657 & 655 & 653 & 652 & \\
580-800 & 706 & 694 & 687 & 681 & 678 & 675 & 672 & 799 & 799 & 799 & 799 & 799 & 799 & \\
600-700 & 655 & 652 & 650 & 648 & 648 & 647 & 646 & 645 & 645 & 644 & 643 & 643 & 643 & \\
600-720 & 666 & 662 & 660 & 658 & 657 & 656 & 655 & 654 & 653 & 651 & 651 & 650 & 649 & \\
600-740 & 678 & 673 & 669 & 667 & 665 & 664 & 663 & 661 & 660 & 658 & 657 & 657 & 656 & \\
600-760 & 690 & 683 & 679 & 675 & 674 & 672 & 670 & 669 & 667 & 665 & 664 & 663 & 662 & \\
600-780 & 701 & 693 & 688 & 684 & 682 & 680 & 678 & 676 & 674 & 672 & 670 & 669 & 668 & \\
600-800 & 712 & 703 & 697 & 692 & 689 & 687 & 685 & 799 & 799 & 799 & 799 & 799 & 799 & \\
620-720 & 674 & 671 & 670 & 668 & 668 & 667 & 666 & 666 & 665 & 664 & 663 & 663 & 663 & \\
620-740 & 685 & 682 & 679 & 677 & 676 & 675 & 675 & 674 & 673 & 672 & 671 & 670 & 670 & \\
620-760 & 697 & 692 & 689 & 686 & 685 & 684 & 683 & 681 & 680 & 679 & 678 & 677 & 676 & \\
620-780 & 708 & 702 & 698 & 695 & 693 & 692 & 690 & 689 & 687 & 685 & 684 & 683 & 683 & \\
620-800 & 719 & 712 & 707 & 703 & 701 & 699 & 698 & 799 & 799 & 799 & 799 & 799 & 799 & \\
640-740 & 694 & 691 & 689 & 688 & 687 & 687 & 686 & 686 & 685 & 684 & 684 & 683 & 683 & \\
640-760 & 705 & 701 & 699 & 697 & 696 & 695 & 695 & 694 & 693 & 692 & 691 & 690 & 690 & \\
640-780 & 716 & 711 & 708 & 706 & 705 & 704 & 703 & 701 & 700 & 699 & 698 & 697 & 697 & \\
640-800 & 727 & 721 & 717 & 714 & 713 & 711 & 710 & 799 & 799 & 799 & 799 & 799 & 799 & \\
660-760 & 713 & 711 & 709 & 708 & 707 & 707 & 706 & 706 & 705 & 704 & 704 & 703 & 703 & \\
660-780 & 724 & 721 & 719 & 717 & 716 & 715 & 715 & 714 & 713 & 712 & 711 & 711 & 710 & \\
660-800 & 735 & 730 & 728 & 725 & 724 & 723 & 722 & 799 & 799 & 799 & 799 & 799 & 799 & \\
680-780 & 733 & 730 & 729 & 728 & 727 & 727 & 726 & 726 & 725 & 724 & 724 & 724 & 723 & \\
680-800 & 743 & 740 & 738 & 736 & 736 & 735 & 734 & 799 & 799 & 799 & 799 & 799 & 799 & \\
700-800 & 752 & 750 & 748 & 747 & 747 & 746 & 746 & 799 & 799 & 799 & 799 & 799 & 799 & \\
\cline{1-14}
nm & M5
& M0
& K5
& K0
& G5
& G0
& F5
& F0
& A5
& A0
& B5
& B0
& O5 & \\
& $+1.61$
& $+1.39$
& $+1.11$
& $+0.84$
& $+0.70$
& $+0.57$
& $+0.45$
& $+0.30$
& $+0.16$
& $0.00$ 
& $-0.17$
& $-0.31$
& $-0.45$ & B$-$V \\
& $+1.19$
& $+1.24$
& $+1.06$
& $+0.46$
& $+0.20$
& $+0.04$
& $-0.01$
& $+0.02$
& $+0.09$
& $0.00$ 
& $-0.56$
& $-1.07$
& $-1.20$ & U$-$B \\
\cline{1-14}
\end{tabular}
\end{center}
\clearpage
\end{tiny}
%%%%%%%%%%%%%%%%%%%%%%%%%%%%%%%%%%%%%%%%%%%%

\pagebreak
{\bf APPENDIX D -- Error and Warning Messages}
\vspace{5mm}

Except where otherwise stated, the following messages appear
on both the command device (typically the terminal) and the
report device (typically the file to be printed).

\begin{quote}
\verb|Can't open file| {\it nn}
\end{quote}
The above message may appear when ASTROM is started and
usually means either that the specified data file doesn't exist,
or that file protection problems prevent the report file from
being written.

\begin{quote}
\verb|Please run ASTROM from the correct script!| \\
\verb|ASTROM improperly invoked!| \\
\verb|GETARG error!|
\end{quote}
These are examples of other messages which appear when ASTROM is
started and which depend on the particular type of computer
being used.  They indicate either software errors or some
other gross error.  Attempting to run the ASTROM
executable image directly rather than through the correct
command procedure is one possible cause.

\begin{quote}
\verb|^^^^^^  POSSIBLE DATA ERROR?  ^^^^^^|
\end{quote}
This message is output to the command device only.  It means that
at least one
of the fields in the plate centre \radec\ was suspect --
unexpectedly negative, fractional when it should have been
integer, or outside the conventional range.

\begin{quote}
\verb|Observation data were incomplete and will be ignored|
\end{quote}
Many of the observation data may be omitted and have sensible
defaults, but certain combinations of omission make it impossible
for ASTROM to carry out a reduction in observed place.  When this
happens, the above message is output and a mean place reduction
is carried out.

\begin{quote}
\verb|For the given observation data, the plate centre ZD is|
{\it xxx.x}\ \verb|degrees!|\\
\verb|Reduction will be in MEAN place.|
\end{quote}
This indicates that the
zenith distance of the field centre, if the observation data are to
be believed, is greater than $90^{\circ}$.  The most likely causes
are an incorrectly specified time or observatory site.

\begin{quote}
\verb|sla_DS2TP status |{\it n}
\end{quote}
The above message
indicates something badly wrong with the reference star or plate
centre positions.

\begin{quote}
\verb|Impossible sla_SVD error!|
\end{quote}
If this message appears please ask your Starlink Site Manager
to contact the Starlink User Support group.

\begin{quote}
\verb|sla_SVD warning |{\it n}
\end{quote}
Certain very rare combinations of (perfectly valid)
data can provoke the above message.
If it appears,
please check the rest of the output to see if it looks
reasonable.  If you
prefer, you can almost certainly make the condition disappear
by changing one of the data values {\it very}\, slightly.
If the results look suspect, or if changing the
data fails to eliminate the problem, please
ask your Starlink Site Manager
to contact the Starlink User Support group.

\begin{quote}
\verb|Fit was ill-conditioned!|
\end{quote}
This warns that the input data failed to
define all the fitted parameters adequately.  The cure is to
use more reference stars or not to fit the plate centre
and/or the distortion.

\begin{quote}
\verb|Radial distortion coefficient cannot reliably be determined!|
\end{quote}
This message, which appears only on the report device, warns that
the data were of insufficient quality to allow the distortion
to be determined.  This can happen, for example, if the
distortion is very small or if there aren't enough reference stars.

\begin{quote}
\verb|Plate centre cannot reliably be determined!|
\end{quote}
The above message, which appears only on the report device, warns that
the data were of insufficient quality to allow the plate
centre to be determined.  This can happen, for example, if the
radial distortion is very small or if there aren't enough reference stars.

\begin{quote}
\verb|*************** INVALID DATA ****************|
\end{quote}
This indicates that there is something fatally wrong with the
most recent input record.

\begin{quote}
\verb|------------ PREMATURE END OF DATA ------------|
\end{quote}
This indicates that the input file has ended, or that an
end-of-sequence record or an end-of-file record has been read at a time
when more input records are needed before a plate solution can be
attempted.

\begin{quote}
\verb|****** PLATE EPOCH WAS NOT SPECIFIED ******|
\end{quote}
This happens if no plate epoch is available.  This
information must be supplied, either as part of the plate data record or
in a time record.

\begin{quote}
\verb|****** TOO MANY REFERENCE STARS; MAX |{\it n}\ \verb|******|
\end{quote}
ASTROM internal workspace limits have been reached;  reduce the
number of reference stars.  If you have a legitimate requirement
for a larger number of reference stars than is currently allowed,
please ask your Starlink Site Manager to contact the Starlink
User Support Group.

\begin{quote}
\verb|------------ FIT |{\it n}\ \verb|ABORTED ------------|
\end{quote}
This message always follows one of those described earlier,
which will indicate the source of the problem.

\begin{quote}
\verb|---------- UNKNOWNS BUT NO SOLUTION ------------|
\end{quote}
An unknown star record has been encountered before the records defining
a plate solution have been input.  Check the order of records in the
input file.
\end{document}
