\documentstyle[11pt]{article} 
\pagestyle{myheadings}

%------------------------------------------------------------------------------
\newcommand{\stardoccategory}  {Starlink User Note}
\newcommand{\stardocinitials}  {SUN}
\newcommand{\stardocnumber}    {42.4}
\newcommand{\stardocauthors}   {Nicholas Eaton}
\newcommand{\stardocdate}      {21 January 1992}
\newcommand{\stardoctitle}     {DAOPHOT --- Stellar Photometry Package}
%------------------------------------------------------------------------------

\newcommand{\stardocname}{\stardocinitials /\stardocnumber}
\renewcommand{\_}{{\tt\char'137}}     % re-centres the underscore
\markright{\stardocname}
\setlength{\textwidth}{160mm}
\setlength{\textheight}{230mm}
\setlength{\topmargin}{-2mm}
\setlength{\oddsidemargin}{0mm}
\setlength{\evensidemargin}{0mm}
\setlength{\parindent}{0mm}
\setlength{\parskip}{\medskipamount}
\setlength{\unitlength}{1mm}

%------------------------------------------------------------------------------
% Add any \newcommand or \newenvironment commands here
%------------------------------------------------------------------------------

\begin{document}
\thispagestyle{empty}
SCIENCE \& ENGINEERING RESEARCH COUNCIL \hfill \stardocname\\
RUTHERFORD APPLETON LABORATORY\\
{\large\bf Starlink Project\\}
{\large\bf \stardoccategory\ \stardocnumber}
\begin{flushright}
\stardocauthors\\
\stardocdate
\end{flushright}
\vspace{-4mm}
\rule{\textwidth}{0.5mm}
\vspace{5mm}
\begin{center}
{\Large\bf \stardoctitle}
\end{center}
\vspace{5mm}

%------------------------------------------------------------------------------
%  Add this part if you want a table of contents
%  \setlength{\parskip}{0mm}
%  \tableofcontents
%  \setlength{\parskip}{\medskipamount}
%  \markright{\stardocname}
%------------------------------------------------------------------------------

\section{Introduction}

DAOPHOT is a stellar photometry package designed to deal with crowded fields.
and written by Peter Stetson at DAO,
The package performs various tasks including finding objects, aperture
photometry, obtaining the point-spread function, and profile-fitting photometry.
Profile fitting in crowded regions is performed iteratively which improves the
accuracy of the photometry.

This document is an introduction to DAOPHOT II: The Next Generation. It
replaces the previous version which is known as DAOPHOT Classic. The main
changes concern the choice of the point-spread fitting function and the
handling of undersampled data. This version uses image data written in the
Starlink N-Dimensional Data Format (NDF).

Reference [1] gives the background to the algorithms used by DAOPHOT. A user
guide is also available through Starlink.

DAOPHOT does not directly use an image display, which is one of the
reasons why the package has been successfully ported round the world.
Three additional routines have therefore been supplied which allow results
obtained with DAOPHOT to be displayed on an image device.

The routine DAOGREY will display a grey scale image of the data on a suitable
device. DAOPLOT will indicate the positions of objects found with DAOPHOT on
top of the grey image. DAOCURS will put up a cursor on the display to allow
positions to be measured from the screen. Although KAPPA routines could be used
for the image display and cursor interaction, DAOPHOT does not comply with the
Starlink coordinate system convention (SSN/22) and thus any coordinates would
be half a pixel out.

DAOPHOT is copyright of Peter Stetson at the Dominion Astrophysical Observatory.
The algorithms should not be changed without permission.

\section{Running DAOPHOT}

DAOPHOT is invoked by the simple command
\begin{verbatim}
        $ DAOPHOT
\end{verbatim}
DAOPHOT is a single executable image, and so once invoked the user remains
within DAOPHOT until the command
\begin{verbatim}
        Command: EXIT
\end{verbatim}
is given. Control will then return to the environment. The data to be analysed
is made known to DAOPHOT using the command
\begin{verbatim}
        Command: ATTACH <filename>
\end{verbatim}
The file has to be in the NDF format of the Starlink Standard Data Structures,
reference [2]. The default file extension is .SDF.

At present DAOPHOT can only handle NDF files that have the DATA\_ARRAY
structure at the top level. If the program fails to find the data structure in
the container file, then the message
\begin{verbatim}
        Failed to attach <filename>
\end{verbatim}
will be displayed. If you do not expect this message, then this is the moment
to seek help. If you do not get this message, then you can carry on with
DAOPHOT.

The complete set of DAOPHOT commands can be found in the user manual.

There are three options files which are used by DAOPHOT at various stages of
the reduction. There are examples of these in the DAO2\_DIR directory. The file
DAOPHOT.OPT is used when the program starts up. Parameter definitions can be
added or removed from the options file as required. An option is specified
using a two letter mnemonic followed by its value. If a copy of this file is
not present in the default directory, or the parameters have invalid values,
then the program will prompt for those parameters that it cannot continue
without, namely the  read-out noise and the gain. This is signified by the
message
\begin{verbatim}
        Value unacceptable --- please re-enter
\end{verbatim}
followed by the parameter prompt. The user has to give values for the read-out
noise and gain before the program will continue. Both parameters are used at
various stages of processing and the user guide keeps stressing that it is
highly advisable to have the most accurate values as possible for these
parameters. Although the program will execute with guesstimates for these, the
results should be treated with caution. The sections on the {\bf PEAK} and {\bf
NSTAR} commands in the user guide suggest a method of checking the validity of
the values given.

The file PHOTO.OPT contains the list of aperture radii used by DAOPHOT when
performing aperture photometry.

An alternative to the iterative profile fitting routine NSTAR in DAOPHOT has
been written by Peter Stetson and is called ALLSTAR. It is described in the
user manual and can be executed by typing
\begin{verbatim}
        $ ALLSTAR
\end{verbatim}
The file ALLSTAR.OPT contains parameters for this program.
\section{DAOGREY}

DAOPHOT does not directly use an image display, instead it produces image files
of intermediate results which can be displayed on any available display system.
It is often desirable to indicate the positions of stars on top of the displayed
image. To achieve this some special routines have been created.

DAOGREY is the first of these routines and this will display the image data on
a suitable device.
DAOGREY, DAOPLOT and DAOCURS use the ADAM parameter system and therefore the
command
\begin{verbatim}
        $ ADAMSTART
\end{verbatim}
has to be given first.
DAOGREY is executed by typing
\begin{verbatim}
        $ DAOGREY
\end{verbatim}
The following parameters are requested:
\begin{description}
\begin{description}
\item[IMAGE] :
The name of the file containing the image data.
\item[XSTART, XEND, YSTART, YEND] :
These specify the limits in pixel coordinates of the area of the data array to
be plotted.
The choice of values allows subsections of the data to be viewed.
\item[LOW, HIGH] :
These specify the data values corresponding to black and white on the display.
The grey levels in between are set automatically.
\item[DEVICE] :
The name of the display device to be used.
The image is plotted in the centre of the display, best filling the display
area.
\end{description}
\end{description}

\section{DAOPLOT}

DAOPLOT will take a file produced by DAOPHOT and overlay the positions
of objects in the file on top of the grey image produced by DAOGREY.
The positions are indicated by a green cross.
The file can be any of the types .COO, .AP, .GRP or .NST produced by
DAOPHOT.
If the file is not suitable then error messages will be given and the routine
will end.
If the filename was entered wrongly, then just run the task again.

To run DAOPLOT type
\begin{verbatim}
        $ DAOPLOT
\end{verbatim}
The following parameter is requested:
\begin{description}
\begin{description}
\item[PFILE] :
The name of the file containing the positions of the objects to be indicated.
The filename and extension should be given.
\end{description}
\end{description}

\section{DAOCURS}

DAOCURS will display a cursor over the grey image produced by DAOGREY so that
positions can be read off the screen.
The cursor is positioned with the mouse and the position recorded
by pressing any of the keys, except the break key on the mouse.
The break key ends the cursor session.
Positions are written to the terminal as well as a file.

To run DAOCURS type
\begin{verbatim}
        $ DAOCURS
\end{verbatim}
The following parameter is requested:
\begin{description}
\begin{description}
\item[RFILE] :
The name of the file to receive the positions selected by the cursor operation.
\end{description}
\end{description}

\section{Database}

DAOGREY, DAOPLOT and DAOCURS have been specially tailored round each other.
DAOPLOT and DAOCURS will not at present work on an image displayed with any
other routine than DAOGREY\. DAOGREY uses the AGI database system to record the
position of the image on the screen. DAOPLOT and DAOCURS use the database file
to set up the screen coordinates. When an entire session has been completed the
database file can be deleted.

\section{References}

1 --- Stetson, P.B., 1987, DAOPHOT: A computer program for crowded-field
stellar photometry, \\
\hspace*{2.0em} Publ.astr.Soc.Pacific, {\bf 99}, 191-222.

2 --- SUN/33: NDF - A Subroutine Library for Accessing NDF Data Structures.

3 --- SUN/48: AGI - Applications Graphics Interface Programmers Guide.

4 --- SSN/22: Coordinate Systems for Pixel Arrays.

\end{document}
