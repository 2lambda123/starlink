\documentstyle{article} 
\pagestyle{myheadings}
\renewcommand{\_}{{\tt\char'137}}     % re-centres the underscore
\markright{SUG}
\setlength{\textwidth}{100mm}
\setlength{\textheight}{154mm}
\setlength{\topmargin}{-0mm}
\setlength{\oddsidemargin}{0mm}
\setlength{\evensidemargin}{0mm}
\setlength{\parindent}{0mm}
\setlength{\parskip}{\medskipamount}
\setlength{\unitlength}{1mm}

\begin{document}
\thispagestyle{empty}
SCIENCE \& ENGINEERING RESEARCH COUNCIL \hfill SUG \\
RUTHERFORD APPLETON LABORATORY\\
{\Large\bf Starlink Project\\}
\begin{flushright}
M D Lawden\\
10 September 1993
\end{flushright}
\vspace{-8mm}
\rule{\textwidth}{0.5mm}
\vspace{5mm}
\begin{center}
{\LARGE\bf STARLINK USER'S GUIDE}
\end{center}
\vspace{5mm}

\setlength{\parskip}{0mm}
\tableofcontents
\setlength{\parskip}{\medskipamount}
\markright{SUG}

\newpage

blank page

\newpage
\setcounter{page}{1}

\section{The Starlink Project}

Starlink is a computing facility provided for UK astronomers by the Science and
Engineering Research Council (SERC) under its Particles, Space and Astronomy
Board (PSAB).
In April 1994 the SERC will cease to exist, and its astronomical work will be
taken over by the new Particle Physics and Astronomy Research Council (PPARC).

Starlink started in 1980.
Its main objectives are to:
\begin{itemize}
\item Provide and coordinate interactive data reduction and analysis facilities
for use as a research tool by UK astronomers.
\item Encourage software sharing and standardisation to prevent unnecessary
duplication of effort.
\item Provide systems software support for astronomers.
\end{itemize}
To achieve these objectives, Starlink provides:
\begin{itemize}
\item A compatible set of computer hardware installed at astronomical centres
throughout the country.
These Starlink sites are connected by a communications network.
\item A collection of astronomical software which is distributed to every site.
\item Management, programming and user support staff.
\end{itemize}
For its first ten years, Starlink was based on DEC VAX hardware, which made
software sharing easy.
However, in 1991 it was decided to change over to Unix-based hardware as this
was becoming the {\em de facto}\, standard in Astronomy, and would enable the
Project to exploit the most cost-effective hardware.
The speed of change is governed by financial constraints and the need to
port Starlink software to the new environment.
It is anticipated that the changeover will effectively be completed by the
end of 1995.
The transition period is complicated by the existence of both VMS and Unix
systems running in parallel.
This guide deals with both.

In this guide, commands and file references for Unix systems are shown in
this typeface: {\tt /star/\-admin/\-whoswho}.
The VMS equivalent is shown in parentheses following the Unix specification
in this typeface: ({\tt ADMINDIR:\-WHOSWHO}).

\section{Organisation and Management}

The central management is located at Rutherford Appleton Laboratory (RAL) and
is headed by the Project Manager.
There is also a Project Scientist, who acts as an independent interface
between the users and the Project.
There are various committees which look after things:
\begin{description}
\item [SP] --- Starlink Panel.
This determines overall policy, monitors progress, and reports to the
Astronomy and Astrophysics Committee (AAC) of the PSAB.
\item [SLUG] --- Starlink Local User Group.
Individual sites may hold meetings at which any user may comment on its
operation and policies.
At least one of these meetings each year is attended by Starlink Project staff.
If you feel strongly about something, go along to your SLUG and make your
views known.
Meeting dates are advertised locally.
\item [HAG] --- Hardware Advisory Group.
This meets from time to time in order to assess current computer hardware trends
and to advise Starlink management on appropriate purchases.
\end{description}
Starlink's computer hardware can be regarded as a set of nodes linked by a
communications network.
Current details of Starlink sites, managers, network addresses, staff, and
committees are stored in file {\tt /star/\-admin/\-whoswho} 
({\tt ADMINDIR:\-WHOSWHO})\footnote{All VMS files mentioned in this paper have
the file type `.LIS' unless otherwise specified.}---you should get a copy of
this when you register as a user.
A brief list of Starlink sites is also given in Appendix A.
Most of the nodes serve their local population of astronomers.
However, there are two special nodes which are located at RAL:
\begin{description}
\item [PRO] --- This serves the requirements of the central Project management
team and is used for administration, software development and distribution, and
document production.
\item [CEN] --- This provides all users with a central data and software
facility and a VMS service (SUN/30).
It stores astronomical catalogues which are accessed via a database management
system called SCAR (SUN/70).
It also stores some general purpose software which is only licensed for use
on this node.
Its network name is STADAT.
\end{description}
Your main point of contact with the Project is your Site Manager, from whom you
should seek help if you run into difficulties or have special
requirements (like using STADAT).
Some sites have ``Assistant Site Managers" or ``User Support Assistants", but in
general there are no computer operators; you are expected to load your own
tapes and take your own output off printers and plotters.

\section{User Registration}

Every Starlink user must be formally accredited before he or she can use
Starlink computers.
If you want to use Starlink, get a User Application Form from the Manager of
the site you want to use, complete and return it.
Authorisation to use the computers for interactive data reduction will
normally be granted immediately by the Site Manager.
All applications are vetted by the Project Scientist.

When your application has been approved, you will be given a username, password
and other resources on your local Starlink computer system.
Each site has its own distribution policy for documentation, but you should get
this guide, a local guide, and an introduction to Unix (SUN/145); add whatever
else you need.
{\em Please read the documents!}\,
Your local guide tells you what resources are available and how to login to
the computers, and the Unix introduction tells you how to create, edit,
compile, and link programs, manage files and directories, and use the e-mail
system.

You may register at more than one site, but this is rarely necessary and
is discouraged.
Consult your Site Manager if your requirements are not satisfied by a single
username.

The Data Protection Act (UK) controls the storage of personal information on
computers.
Starlink policy is to make all such information completely open to inspection.
It is held on computers solely to enable users to be supported and resources
to be managed efficiently.
A full list of the files involved is in file {\tt /star/\-local/\-admin/\-dpa}
({\tt RLVAD::LADMINDIR:DPA}) at the PRO site, but the most important data files
are the local {\tt /star/\-local/\-admin/\-usernames} ({\tt LADMINDIR:USERNAMES}) and
{\tt /star/\-local/\-admin/\-users} ({\tt LADMINDIR:USERS}) files stored locally
at each site.
Please inform your Site Manager if any information stored about you is
incorrect, and please tell him or her when you leave!

\section{Using Starlink Computers}

Do {\em not}\, share your username and password with anyone else, and make
passwords non-obvious---computer networks are vulnerable to criminals called
hackers.
If you tell someone your password for what you believe at the time is a
legitimate reason, allow them access to the computer for as short a time as
possible and change your password immediately they have finished their login
session.
Tell your Site Manager what you have done.
Anyone whose username is used by a hacker, and who has not taken reasonable
precautions against its misuse, will bear the responsibility for any wasted
resources or damage.

In order to use a Starlink Unix computer effectively, you must be familiar
with the contents of SUN/145.
In particular:
\begin{itemize}
\item know how to use the {\tt man} command to find information on Unix
commands.
\item know how to display and print text files such as those referred to in this
guide (you must consult your local guide for information as printing
techniques are locally determined).
\item look at {\tt /star/\-docs/\-news} ({\tt DOCSDIR:NEWS}) frequently to keep abreast
of changes to Starlink software and documentation.
\item know how to use an e-mail system ({\tt pine} is recommended), and login
frequently to check for messages.
The usernames and addresses needed to send e-mail to other people are
specified in {\tt /star/\-admin/\-unixnames} and {\tt /star/\-admin/\-usernames}
({\tt ADMINDIR:\-UNIXNAMES} and {\tt ADMINDIR:\-USERNAMES}).
\item learn how to use the network facilities described in SUN/36.
\end{itemize}

Please report suspected errors in Starlink documents and software to your Site
Manager and to the Starlink Software Librarian ({\tt ussc@star.rl.ac.uk}).
Please give sufficient detail in software bug reports to enable another person
to understand and reproduce the bug.
Known software bugs are reported in a VAXnotes conference called BUG\_REPORTS
(SGP/1, SUN/44).

\section{Finding Information}

Starlink is a big project (25 sites, 55 staff, 1600 users, 600Mbyte software,
400 documents); a great deal of information is available about its facilities,
and it is important that you know how to find it.
There are three main sources:
\begin{description}
\item [Personal contact] ---
This can be the most efficient source when you start to use Starlink.
Being shown how to do something cuts through the documentation barrier which is
such an intimidating feature of big systems.
Find out how other users work and what they find useful.
Don't stop there though.
There is a lot of useful information available and the people you talk to may
not use it or be aware of it.
If you are thinking of using a software package or analysing a standard type of
data ({\em e.g.}\, IUE images), seek out and make friends with other users of
them.
They can tell you what the documentation leaves out, and they may have solved
the problems that are worrying you.
Your Site Manager should be able to put you in touch with them.
\item [Paper documents] ---
These are mainly:
\begin{itemize}
\item Starlink documents.
\item Books on Unix and related software, such as X-windows.
\end{itemize}
These sources are described in more detail below.
\item [On-line files] ---
These are convenient for interactive users and may be printed out for reference.
Compared with paper documents, they are more likely to be up to date and the
computer can help search for information.
However, they are not as easy to use, may be incomplete, and diagrams are not
normally available.
The text of most Starlink documents is available on-line (but may not be easily
readable as most of them are meant for processing by the \LaTeX\ document
preparation system).
Unix has an on-line manual which can be examined with the {\tt man} command.
\end{description}

\subsection{Starlink documents}

Starlink classifies its documents as follows:
{\small
\begin{tabbing}
xx\=MUDxx\=- Starlink General Papersxxx\=LUNxx\=x\kill
\>{\bf SG}\>-- Starlink Guides\\
\>{\bf SUN}\>-- Starlink User Notes     \>{\bf LUN}\>-- Local User Notes\\
\>{\bf SGP}\>-- Starlink General Papers \>{\bf LGP}\>-- Local General Papers\\
\>{\bf SSN}\>-- Starlink System Notes   \>{\bf LSN}\>-- Local System Notes\\
\>{\bf MUD}\>-- Miscellaneous User Documents
\end{tabbing}
}
{\em Starlink}\, documents are project-wide and are available at every site.
{\em Local}\, documents are issued by individual sites and contain information
of local relevance; they are not normally available away from the site of issue.
\begin{itemize}
\item {\em Guides}\, provide tutorial information on large or complex software
 items.
\item {\em User Notes}\, describe how to use a particular program or package.
\item {\em General Papers}\, contain a wide variety of material concerning the
development and management of Starlink.
\item {\em System Notes}\, describe system software and give implementation and
installation details for applications software.
\end{itemize}
Starlink documents are indexed by type and serial number in
{\tt /star/\-docs/\-docs\_lis} ({\tt DOCSDIR:\-DOCS}) and by subject and keyword in
{\tt /star/\-docs/\-subject\_lis} ({\tt DOCSDIR:\-SUBJECT}).
Local documents are indexed in {\tt /star/\-local/\-docs/\-docs\_lis}
({\tt LDOCSDIR:\-DOCS}) at the local site.
However, the best way to find which documents describe a specific software item
is to look in {\tt /star/\-docs/\-analysis\_lis} ({\tt DOCSDIR:\-ANALYSIS}).

Starlink recommends \LaTeX\ for document production (SGP/28).
Anyone may contribute documents to the Starlink series, but please use the
standard templates held in files such as {\tt /star/\-docs/\-sun.tex}
({\tt DOCSDIR:\-SUN.TEX}).
This makes life much easier for Starlink staff as many potential formatting
traps have been thought out and {\em solved}\, in these templates, and their use
gives Starlink documentation a uniform appearance.
The Project reserves the right to make minor corrections and stylistic changes
to submitted documents in order to achieve a uniform Starlink style.
When revising documents you {\em must,} therefore, start from the released
version stored in directory {\tt /star/\-docs} ({\tt DOCSDIR:}) rather than from
your own copy.

\subsection{On-line files}

On-line files exist in two main forms:
\begin{description}
\item [HELP libraries] --- for access via the VMS HELP utility.
The main HELP library is that provided by DEC to support their DCL command
language and other VMS features.
Some Starlink packages have their own HELP libraries.
There is also a general Starlink HELP library which may be accessed by typing:
\begin{verbatim}
    $ HELP STARLINK
\end{verbatim}
\item [Text files] --- for printing or for display on a terminal.
They can be classified as follows:
\begin{itemize}
\item Starlink documents.
These are stored in the following directories:
\begin{itemize}
\item {\tt /star/docs} ({\tt DOCSDIR:}) contains {\em Starlink} documents
 (SG, SUG, SUN, SGP, SSN).
\item {\tt /star/local/docs} ({\tt LDOCSDIR:}) contains {\em Local} documents
 (LUN, LGP, LSN).
\item {\tt /star/local/mud} ({\tt LMUDSDIR:}) contains Miscellaneous documents
 where these are available in on-line form.
\end{itemize}
\item Information summaries.
These are maintained at the PRO site (usually in {\tt /star/\-local/\-admin} or
{\tt /star/\-local/\-docs}) as a reference for the whole project.
File {\tt /star/\-local/\-docs/\-online} ({\tt RLVAD::\-LDOCSDIR:\-ONLINE}) contains
an annotated list of such files.
Each summary includes the date on which it was last revised.
\item Internal documents.
\end{itemize}
\end{description}

\subsection{Computer searching}

The {\tt grep} ({\tt SEARCH}) command examines files for a given text string
and displays all the lines found to contain it.
As an example, suppose you want to find documents describing programs for
handling magnetic tapes.
You can search the document and subject indexes for the string `TAPE' with the
command:
{\small
\begin{verbatim}
    % grep tape /star/docs/docs_lis /star/docs/subject_lis
   ($ SEARCH  DOCSDIR:DOCS,SUBJECT TAPE)
\end{verbatim}
}
{\tt grep} ({\tt SEARCH}) can also be used to find a person's username:
{\small
\begin{verbatim}
    % grep surname /star/admin/usernames /star/admin/unixnames
   ($ SEARCH  ADMINDIR:USERNAMES,UNIXNAMES)
\end{verbatim}
}
You can then e-mail messages to this username.

On VMS systems you can find a Starlink document on a particular topic by typing
\begin{verbatim}
    $ DOCFIND topic
\end{verbatim}

\subsection{Summary}

Here is a table showing sources of information on the main topics you are
likely to want to know about.
{\footnotesize
\begin{center}
\begin{tabular}{|l|l|l|l|}
\hline
{\em Unix source} & {\em VMS source} & {\em Contents} \\
\hline
\hline
{\tt /star/docs/docs\_lis} & {\tt DOCSDIR:DOCS} & Starlink series \\
{\tt /star/docs/mud\_lis}  & {\tt DOCSDIR:MUD} & Miscellaneous \\
{\tt /star/docs/analysis\_lis} & {\tt DOCSDIR:ANALYSIS} & Acronym \\
{\tt /star/docs/subject\_lis} & {\tt DOCSDIR:SUBJECT} & Keyword \\
\hline
{\tt /star/admin/ssi} & {\tt ADMINDIR:SSI}  & Software index \\
SUN/1                 & SUN/1               & General intro \\
SG/4                  & SG/4                & ADAM intron \\
                      & LOOKUP command      & Commands \\
                      & HELP STARLINK       & On-line help \\
\hline
{\tt /star/admin/whoswho} & {\tt ADMINDIR:WHOSWHO} & Project staff \\
{\tt /star/admin/usernames} & {\tt ADMINDIR:USERNAMES} & Users \\
{\tt /star/admin/unixnames} & {\tt ADMINDIR:UNIXNAMES} & e-mail addresses \\
                            & {\tt EMAIL name}     & e-mail addresses \\
\hline
SGP/31                & SGP/31              & Purpose/history \\
Starlink Bulletin     & Starlink Bulletin   & Newsletter \\
\hline
Local guide           & Local guide         & Local guide \\
\hline
\end{tabular}
\end{center}
}

\section{Hardware}

The computer hardware operated by Starlink consists of Unix workstations and
servers from various manufacturers, together with VAX computers running VMS
which are being phased out.
Your local guide should describe the hardware available at your site.
The most common computers that run Unix are Sun workstations and DEC
DECstations and Alpha workstations.
The computers at each site are linked together by Local Area Networks.

Each site is connected to the UK-wide JANET communications network.
The links between different sites differ in speed, but even on the slowest
sections it is possible to transfer several thousand Kbyte an hour over the
network.

\section{Software}

Starlink software is managed by the Starlink Software Librarian at RAL.
It exists in two forms: the {\em Starlink Software Collection} (SSC), which
runs on VAXes, and the {\em Unix Starlink Software Collection} (USSC), which
runs under Unix.
These are similar, except that some of the older software has not been ported
to Unix, and some of the most recent software is not available under VMS.
Most of it was written and supported by people outside RAL, but it is installed
and distributed from there.
Its total size is currently about 300 Mbyte (600 Mbyte for VMS), and there are
about 3 million lines of source code.
It is called a {\em Collection}\, rather than a {\em Package}\, or
{\em System}\, because it consists of about a hundred {\em software items}\,
which are largely autonomous.

Wherever possible, standard software interfaces are used.
These include the NAG library of mathematical routines and the international
standard GKS graphics package.
The Flexible Image Transport System (FITS) is recommended for data interchange
via magnetic tape.

SUN/1 describes the items in the Collection.
SGP/20 describes the definition, organisation, and management of the Collection.
SGP/13 discusses the nature, philosophy, and development of Starlink
applications software.

Updates to Starlink software are routinely distributed by network to all
Starlink sites.
The Collection has also been distributed to over 100 non-Starlink sites, most of
which are outside the UK.
These include the AAO, LPO, ESO/ST-ECF, JACH, STScI and VILSPA, so astronomers
visiting these places can use their favourite Starlink software just like at
home.

Starlink supports this software to the best of its abilities, within the
limited resources available.
Expert help is available for most things, but Starlink may have difficulties
with some items, especially as many of the original writers were not software
professionals and/or have since left.
Software support is, in fact, one of the most intractable problems facing
Starlink.
The support level of each item is specified in {\tt /star/\-local/\-admin/\-support}
at the PRO site.
Some reported bugs take a long time to fix, although some others can be cured
and the updated software released within a day or two.

Starlink recognises the need for a {\em Software Environment}\, into which its
applications software can be integrated.
The principle is to provide a standard user interface (for example a command
language) and programmer interface (subroutine libraries), and to allow
part-processed data to be transferred easily from application to application
during a processing run.
The command language enables you to manipulate data in a concise but natural
way.
The subroutine libraries give programmers efficient and easy access to
command parameters, bulk data, graphics devices, and other resources.
The standard Starlink environment is called ADAM (SG/4) and most Starlink
applications run within it.

\appendix

\newpage

\section {Starlink Sites}

Up-to-date site information is kept in {\tt /star/\-admin/\-whoswho}
(or {\tt ADMINDIR:WHOSWHO}).
This includes the names of Site Managers and their e-mail addresses.

Each site has a three letter code which can be used to identify it.
These are:
\begin{description}
\begin{description}
\item [ARM] : Armagh Observatory
\item [BEL] : Queen's University of Belfast --- Pure \& Applied Physics Dept
\item [BIR] : Birmingham University --- School of Physics \& Space Research
\item [CAM] : Cambridge: Institute of Astronomy, Royal Greenwich Observatory, 
 Mullard Radio Astronomy Observatory.
\item [CAR] : University of Wales College of Cardiff --- Physics \& Astronomy
 Dept
\item [DUR] : Durham University --- Physics Dept
\item [GLA] : Glasgow University --- Physics \& Astronomy Dept
\item [HAT] : Hertfordshire University, Hatfield --- Physics \& Astronomy Dept
\item [IMP] : Imperial College, London --- Physics Dept
\item [JOD] : Nuffield Radio Astronomy Lab, Jodrell Bank
\item [KEE] : Keele University --- Physics Dept
\item [KEN] : Kent University --- Electronic Engineering Lab
\item [LEI] : Leicester University --- Physics \& Astronomy Dept
\item [LJM] : Liverpool John Moores University --- School of Chemical \&
 Physical Sciences
\item [MAN] : Manchester University --- Astronomy Dept
\item [OXF] : Oxford University --- Astrophysics Dept
\item [PRE] : Central Lancashire University, Preston --- Physics \& Astronomy
 School
\item [PRO] : Rutherford Appleton Laboratory --- Project Management
\item [QMW] : Queen Mary \& Westfield College, London --- Physics Dept
\item [RAL] : Rutherford Appleton Laboratory --- Astrophysics Division
\item [ROE] : Royal Observatory Edinburgh, Edinburgh University --- Astronomy
 Dept
\item [STA] : St Andrews University --- Physics \& Astronomy Dept
\item [SOU] : Southampton University --- Physics Dept
\item [SUS] : Sussex University --- Astronomy Centre
\item [UCL] : University College London --- Physics \& Astronomy Dept
\end{description}
\end{description}

\newpage

\section {Checklist}

Do you know how to:
\begin{itemize}
\item Get help?\\
\hspace*{10mm} {\em (Ask your Starlink Site Manager)}
\item Login and logout?\\
\hspace*{10mm} {\em (SUN/145 \& Local Guide)}
\item Interpret directory and file names?\\
\hspace*{10mm} {\em (SUN/145)}
\item Find information on a Unix Command?\\
\hspace*{10mm} {\em (\% man command)}
\item Change your password?\\
\hspace*{10mm} {\em (\% yppasswd)}
\item Send and receive MAIL and purge your mail file?\\
\hspace*{10mm} {\em (\% pine)}
\item Find a person's username or a username's owner?\\
\hspace*{10mm} {\em (\% grep surname or username /star/\-admin/\-usernames)}
\item Examine a file?\\
\hspace*{10mm} {\em (\% more filename)}
\item Print a file?\\
\hspace*{10mm} {\em (Ask your Site Manager or look at your Local Guide)}
\item Get a copy of a document?\\
\hspace*{10mm} {\em (Get a copy from your local document store)}
\item Find a document about a Starlink software item?\\
\hspace*{10mm} {\em (Look at /star/\-docs/\-analysis\_lis)}
\item Find out about the latest software and documentation releases?\\
\hspace*{10mm} {\em (Look at /star/\-docs/\-news)}
\item Find out about Starlink personnel, sites, and committees?\\
\hspace*{10mm} {\em (Look at /star/\-admin/\-whoswho)}
\item Find out about a specific topic?\\
\hspace*{10mm} {\em (Look at /star/\-docs/\-subject\_lis)}
\item Find out about on-line information?\\
\hspace*{10mm} {\em (Look at /star/\-local/\-docs/\-online)}
\item Guard your files against loss?\\
\hspace*{10mm} {\em (Use the tar utility to write to tape)}
\item Report an error in Starlink software or documentation?\\
\hspace*{10mm} {\em (Send a mail message to ussc@star.rl.ac.uk)}
\end{itemize}
\end{document}
