\documentclass[twoside]{article}
\pagestyle{myheadings}

% -----------------------------------------------------------------------------
% ? Document identification
\newcommand{\stardoccategory}  {Starlink User's Guide}
\newcommand{\stardocinitials}  {SUG}
\newcommand{\stardocsource}    {sug}
\newcommand{\stardocauthors}   {M D Lawden}
\newcommand{\stardocdate}      {17 October 1997}
\newcommand{\stardoctitle}     {STARLINK USER'S GUIDE}
\newcommand{\stardocabstract}{
This is a general introduction to the Starlink Project for new users.
You should also read the local guide for your site before attempting
to use Starlink computers.  This will tell you how to access your local
machine and what facilities areavailable.  As these documents cannot be
updated very frequently, you should refer to the on-line files
mentioned for current details of Starlink staff, sites, documents, and
software.
}

% ? End of document identification
% -----------------------------------------------------------------------------

\markright{SUG}
\setlength{\textwidth}{100mm}
\setlength{\textheight}{154mm}
\setlength{\topmargin}{-0mm}
\setlength{\oddsidemargin}{0mm}
\setlength{\evensidemargin}{0mm}
\setlength{\parindent}{0mm}
\setlength{\parskip}{\medskipamount}
\setlength{\unitlength}{1mm}

% -----------------------------------------------------------------------------
%  Hypertext definitions.
%  ======================
%  These are used by the LaTeX2HTML translator in conjunction with star2html.

%  Comment.sty: version 2.0, 19 June 1992
%  Selectively in/exclude pieces of text.
%
%  Author
%    Victor Eijkhout                                      <eijkhout@cs.utk.edu>
%    Department of Computer Science
%    University Tennessee at Knoxville
%    104 Ayres Hall
%    Knoxville, TN 37996
%    USA

%  Do not remove the %begin{latexonly} and %end{latexonly} lines (used by
%  star2html to signify raw TeX that latex2html cannot process).
%begin{latexonly}
\makeatletter
\def\makeinnocent#1{\catcode`#1=12 }
\def\csarg#1#2{\expandafter#1\csname#2\endcsname}

\def\ThrowAwayComment#1{\begingroup
    \def\CurrentComment{#1}%
    \let\do\makeinnocent \dospecials
    \makeinnocent\^^L% and whatever other special cases
    \endlinechar`\^^M \catcode`\^^M=12 \xComment}
{\catcode`\^^M=12 \endlinechar=-1 %
 \gdef\xComment#1^^M{\def\test{#1}
      \csarg\ifx{PlainEnd\CurrentComment Test}\test
          \let\html@next\endgroup
      \else \csarg\ifx{LaLaEnd\CurrentComment Test}\test
            \edef\html@next{\endgroup\noexpand\end{\CurrentComment}}
      \else \let\html@next\xComment
      \fi \fi \html@next}
}
\makeatother

\def\includecomment
 #1{\expandafter\def\csname#1\endcsname{}%
    \expandafter\def\csname end#1\endcsname{}}
\def\excludecomment
 #1{\expandafter\def\csname#1\endcsname{\ThrowAwayComment{#1}}%
    {\escapechar=-1\relax
     \csarg\xdef{PlainEnd#1Test}{\string\\end#1}%
     \csarg\xdef{LaLaEnd#1Test}{\string\\end\string\{#1\string\}}%
    }}

%  Define environments that ignore their contents.
\excludecomment{comment}
\excludecomment{rawhtml}
\excludecomment{htmlonly}

%  Hypertext commands etc. This is a condensed version of the html.sty
%  file supplied with LaTeX2HTML by: Nikos Drakos <nikos@cbl.leeds.ac.uk> &
%  Jelle van Zeijl <jvzeijl@isou17.estec.esa.nl>. The LaTeX2HTML documentation
%  should be consulted about all commands (and the environments defined above)
%  except \xref and \xlabel which are Starlink specific.

\newcommand{\htmladdnormallinkfoot}[2]{#1\footnote{#2}}
\newcommand{\htmladdnormallink}[2]{#1}
\newcommand{\htmladdimg}[1]{}
\newenvironment{latexonly}{}{}
\newcommand{\hyperref}[4]{#2\ref{#4}#3}
\newcommand{\htmlref}[2]{#1}
\newcommand{\htmlimage}[1]{}
\newcommand{\htmladdtonavigation}[1]{}

% Define commands for HTML-only or LaTeX-only text.
\newcommand{\html}[1]{}
\newcommand{\latex}[1]{#1}

% Use latex2html 98.2.
\newcommand{\latexhtml}[2]{#1}

%  Starlink cross-references and labels.
\newcommand{\xref}[3]{#1}
\newcommand{\xlabel}[1]{}

%  LaTeX2HTML symbol.
\newcommand{\latextohtml}{{\bf LaTeX}{2}{\tt{HTML}}}

%  Define command to re-centre underscore for Latex and leave as normal
%  for HTML (severe problems with \_ in tabbing environments and \_\_
%  generally otherwise).
\newcommand{\setunderscore}{\renewcommand{\_}{{\tt\symbol{95}}}}
\latex{\setunderscore}

%  Redefine the \tableofcontents command. This procrastination is necessary
%  to stop the automatic creation of a second table of contents page
%  by latex2html.
\newcommand{\latexonlytoc}[0]{\tableofcontents}

% -----------------------------------------------------------------------------
%  Debugging.
%  =========
%  Remove % on the following to debug links in the HTML version using Latex.

% \newcommand{\hotlink}[2]{\fbox{\begin{tabular}[t]{@{}c@{}}#1\\\hline{\footnotesize #2}\end{tabular}}}
% \renewcommand{\htmladdnormallinkfoot}[2]{\hotlink{#1}{#2}}
% \renewcommand{\htmladdnormallink}[2]{\hotlink{#1}{#2}}
% \renewcommand{\hyperref}[4]{\hotlink{#1}{\S\ref{#4}}}
% \renewcommand{\htmlref}[2]{\hotlink{#1}{\S\ref{#2}}}
% \renewcommand{\xref}[3]{\hotlink{#1}{#2 -- #3}}
%end{latexonly}
% -----------------------------------------------------------------------------
% ? Document specific \newcommand or \newenvironment commands.
% ? End of document specific commands
% -----------------------------------------------------------------------------
%  Title Page.
%  ===========
\renewcommand{\thepage}{\roman{page}}
\begin{document}
\thispagestyle{empty}

%  Latex document header.
%  ======================
\begin{latexonly}
   {\small CCLRC / {\sc Rutherford Appleton Laboratory} \hfill {\bf SUG}}\\
   {Particle Physics \& Astronomy Research Council}\\
   {Starlink Project\\}
   \begin{flushright}
   \stardocauthors\\
   \stardocdate
   \end{flushright}
   \vspace{-4mm}
   \rule{\textwidth}{0.5mm}
   \vspace{5mm}
   \begin{center}
   {\LARGE\bf  STARLINK USER'S GUIDE}
   \end{center}
   \vspace{5mm}

% ? Heading for abstract if used.
   \vspace{10mm}
   \begin{center}
      {\Large\bf Abstract}
   \end{center}
% ? End of heading for abstract.
\end{latexonly}

%  HTML documentation header.
%  ==========================
\begin{htmlonly}
   \xlabel{}
   \begin{rawhtml} <H1> \end{rawhtml}
      \stardoctitle\\
      \stardocversion\\
      \stardocmanual
   \begin{rawhtml} </H1> \end{rawhtml}

% ? Add picture here if required.
% ? End of picture

   \begin{rawhtml} <P> <I> \end{rawhtml}
   \stardoccategory\ \stardocnumber \\
   \stardocauthors \\
   \stardocdate
   \begin{rawhtml} </I> </P> <H3> \end{rawhtml}
      \htmladdnormallink{CCLRC}{http://www.cclrc.ac.uk} /
      \htmladdnormallink{Rutherford Appleton Laboratory}
                        {http://www.cclrc.ac.uk/ral} \\
      \htmladdnormallink{Particle Physics \& Astronomy Research Council}
                        {http://www.pparc.ac.uk} \\
   \begin{rawhtml} </H3> <H2> \end{rawhtml}
      \htmladdnormallink{Starlink Project}{http://www.starlink.ac.uk/}
   \begin{rawhtml} </H2> \end{rawhtml}
   \htmladdnormallink{\htmladdimg{source.gif} Retrieve hardcopy}
      {http://www.starlink.ac.uk/cgi-bin/hcserver?\stardocsource}\\

%  HTML document table of contents.
%  ================================
%  Add table of contents header and a navigation button to return to this
%  point in the document (this should always go before the abstract \section).
  \label{stardoccontents}
  \begin{rawhtml}
    <HR>
    <H2>Contents</H2>
  \end{rawhtml}
  \renewcommand{\latexonlytoc}[0]{}
  \htmladdtonavigation{\htmlref{\htmladdimg{contents_motif.gif}}
        {stardoccontents}}

% ? New section for abstract if used.
  \section{\xlabel{abstract}Abstract}
% ? End of new section for abstract
\end{htmlonly}

% -----------------------------------------------------------------------------
% ? Document Abstract. (if used)
%   ==================
\stardocabstract
% ? End of document abstract
% -----------------------------------------------------------------------------
% ? Latex document Table of Contents (if used).
%  ===========================================
\newpage
\begin{latexonly}
   \setlength{\parskip}{0mm}
   \latexonlytoc
   \setlength{\parskip}{\medskipamount}
   \markright{SUG}
\end{latexonly}
% ? End of Latex document table of contents
% -----------------------------------------------------------------------------
\newpage
\renewcommand{\thepage}{\arabic{page}}
\setcounter{page}{1}

\section{Quickstart}

Want to get going fast?

One way is to use the Starlink information on the World Wide Web -- your
Site Manager can show you how to use it.
Every Starlink site has its own web pages, but there is a central page that
you can use as a starting point.
It's address (URL) is:
\begin{quote}
\htmladdnormallink{\bf http://www.starlink.ac.uk/}{http://www.starlink.ac.uk/}
\end{quote}
You can get to {\bf your own site's pages} by following the links:
\begin{quote}
\htmladdnormallink{\bf Sites}{http://www.starlink.ac.uk/sites.html},
{\bf Site-Code}
\end{quote}
where ``Site-Code" is the 3-letter code next to your site name
(see \htmlref{Appendix A}{AppendixA}).

If you want to find out about {\bf Starlink software}, follow the
\begin{quote}
\htmladdnormallink{\bf What it does}
{http://www.starlink.ac.uk/star/docs/sun1.htx/sun1.html}
\end{quote}
link (under {\em Software}).
If you know the name of the software you want, try the
\begin{quote}
\htmladdnormallink{\bf List of items}
{http://www.starlink.ac.uk/softalpha.html}
\end{quote}
link in the same column.
Scroll down to find the name -- appropriate links follow.

If you want to search {\bf Starlink documents}, follow the
\begin{quote}
\htmladdnormallink{\bf Documentation}
{http://www.starlink.ac.uk/docs.html}
\end{quote}
link (top line of links).
This leads you to a selection of links, each designed for a particular type of
query.
Look for documents with codes starting with SC, SG, and SUN -- this is the
user's stuff.
If you are interested in a particular topic, try the
\begin{quote}
\htmladdnormallink{\bf Software names}
{http://www.starlink.ac.uk/cgi-bin/ssilist/ssilist} {\rm or}
\htmladdnormallink{\bf Topics}
{http://www.starlink.ac.uk/cgi-bin/htxfinder}
\end{quote}
links (under {\em Search}).
For example, under {\em Topics}\/ you could enter the keyword
{\em Fourier}\/ and find out where the stuff on Fourier transforms is kept.
If you want a paper copy of a Starlink document, ask your Site Manager.

If you want to find out about other {\bf Starlink users}, try the
\begin{quote}
\htmladdnormallink{\bf People}{http://www.starlink.ac.uk/people.html}
\end{quote}
link.
This takes you to the {\em Starlink People}\/ page which has links to lots of
information about Starlink staff and users.
If you just want to find someone's e-mail address, use the Unix command:
\begin{quote}
{\bf \% email} {\it surname}
\end{quote}
If you want to know {\bf what is going on}, try the links in the {\em News}\/
column.
The
\begin{quote}
\htmladdnormallink{\bf New products}
{http://www.starlink.ac.uk/\~{}cac/publicity/new_products.html}
\end{quote}
link describes the latest software and document releases.
The
\begin{quote}
\htmladdnormallink{\bf Bulletin}{http://www.starlink.ac.uk/bulletin.html}
\end{quote}
link takes you to our Newsletter, which may not be as up-to-date as the other
links but is a better ``read."
Finally, the
\begin{quote}
\htmladdnormallink{\bf FAQ}
{http://www.starlink.ac.uk/\~{}cac/publicity/starlink_faq.html}
\end{quote}
link (in the {\em Project}\/ column) takes you to a list of frequently asked
questions about Starlink.

The above links will get you up-to-speed fast.
You will then have the ``Big Picture" and be more able to find the information
you want.

\newpage

\section{The Starlink Project}

Starlink is a computing facility provided for UK astronomers by the
Particle Physics and Astronomy Research Council (PPARC).

Starlink started in 1980.
Its main objectives are to:

\begin{itemize}
\item Provide and coordinate interactive data reduction and analysis facilities
for use as a research tool by UK astronomers.
\item Encourage software sharing and standardisation to prevent unnecessary
duplication of effort.
\item Provide systems software support for astronomers.
\end{itemize}

To achieve these objectives, Starlink provides:

\begin{itemize}
\item A compatible set of computer hardware installed at astronomical centres
throughout the country.
\item A collection of astronomical software which is distributed to every site.
\item Management, programming, and user support staff.
\end{itemize}

\newpage

\section{Organisation and Management}

The central management is located at Rutherford Appleton Laboratory (RAL) and
is headed by a Project Manager (Mr P T Wallace).
There is also a Project Scientist (Dr A J Penny), who acts as an independent
interface between the users and the Project.
There are various committees which look after things:

\begin{description}
\item [ACP] -- {\em Astronomical Computing Panel.}\\
This oversees all astronomical computing (except supercomputing),
including Starlink.
\item [SLUG] -- {\em Starlink Local User Groups.}\\
Individual sites may hold meetings at which any user may comment on its
operation and policies.
At least one of these meetings each year is attended by Starlink Project staff.
If you feel strongly about something, go along to your SLUG and make your
views known.
Meeting dates are advertised locally.
\item [SSG] -- {\em Software Strategy Groups.}\\
These provide specialist information to the Project on software matters.
The seven existing groups (1997) cover:
\begin{quote}
\begin{description}
\item [Spectroscopy]
\item [Image processing]
\item [Public domain/commercial software]
\item [Information services \& databases]
\item [Graphics \& infrastructure]
\item [Radio, mm, \& sub-mm astronomy]
\item [X-ray astronomy]
\end{description}
\end{quote}
\end{description}

Local management at individual sites is flexible
(\xref{SGP/41}{sgp41}{}),
and is overseen by a Site Chairman.
Many sites operate a ``Local Management Committee."
Day-to-day management and operation of the sites is carried out by the
Site Managers, who are your main point of contact.

Starlink's computer hardware can be regarded as a set of nodes (groups of
computers) linked by a communications network.
Current details of Starlink sites, managers, chairmen, postal and network
addresses, staff, committees, and so on, are stored in file
{\tt /star/\-admin/\-whoswho}.
A brief list of sites is also given in
\htmlref{Appendix A}{AppendixA}.
\newpage

\section{User Registration}

If you want to use Starlink, get a {\em User Application Form}\/ from the
Manager of the site you want to use, complete it, and return it to the Manager.
Provisional authorisation will normally be granted immediately by the Site
Manager, but this will be formally vetted by the Project Scientist.

You will be given a username, password, and other resources on your local
Starlink computer system, together with some initial documentation.
Each site has its own document distribution policy, but as a start
you should get this guide, a local guide, and an introduction to Unix
(\xref{SUN/145}{sun145}{});
add whatever else you need from your site's local document library --
{\em please read them!}

Your local guide tells you what resources are available and how to login to
the computers, and the Unix introduction tells you how to create, edit,
compile, link, and run programs, manage files and directories, and use e-mail.
You should also find out how to access information on the World Wide Web.
The ``web" contains masses of information about Starlink which is easy to
search.

You may register at more than one site, but this is rarely necessary and
is discouraged.
Consult your Site Manager if your requirements are not satisfied by a single
username.

The Data Protection Act (UK) controls the storage of personal information on
computers.
The Starlink Application Form mentions this; by signing it
you give Starlink the right to store basic details about you on its computers.
Starlink policy is to make all such information completely open to inspection.
It is held on computers solely to enable users to be supported and resources
to be managed efficiently.
The most important files containing personal data are the local
{\tt usernames.lis} and {\tt users.lis} files stored at each site,
usually in directory {\tt /star/local/admin}.
Please inform your Site Manager if any information stored about you is
incorrect or needs to be modified, and please tell him or her when you leave
or move site!

\newpage

\section{Using Starlink Computers}

Do {\em not}\/ share your username and password with anyone else, and make
passwords non-obvious -- computer networks are vulnerable to criminals called
(in the UK at least) hackers.
If you tell someone your password for what you believe at the time is a
legitimate reason, allow them access to the computer for as short a time as
possible and change your password immediately they have finished their login
session.
Tell your Site Manager what you have done.
Anyone whose username is used by a hacker, and who has not taken reasonable
precautions against its misuse, will bear the responsibility for any wasted
resources or damage incurred.

In order to use a Starlink computer effectively, you should be familiar
with the Unix guide
(\xref{SUN/145}{sun145}{}).
In particular:

\begin{itemize}
\item know how to use the {\tt man} command to find information on Unix
commands.
\item know how to display and print text files, such as those referred to in
this guide (you must consult your local guide for information about this
as printing techniques are locally determined).
\item know how to use an e-mail system ({\tt pine} is recommended), and login
frequently to check for messages.
People's e-mail addresses are specified in file {\tt /star/admin/unixnames}.
\end{itemize}

Please report suspected errors in Starlink documents and software to your Site
Manager and to the Starlink Software Librarian {\tt starlink@jiscmail.ac.uk}).
Please give sufficient detail in software bug reports to enable another person
to understand and reproduce the bug.

\newpage

\section{Finding Information}

Starlink is a big project (30 sites, 50 staff, 2000 users, 1Gbyte software,
400 documents), and a lot of information is available about it.
You need to know how to find this information.
There are three main sources:

\begin{description}

\item [On-line information] --
This may be more up-to-date than paper documents, and the computer can help
you search it (by using the {\tt findme} command, as described later).
Most Starlink documents is available in this form.
Unix has an on-line manual which can be examined with the {\tt man} command.
The World Wide Web is the easiest way to search for information about
Starlink; see below.

\item [Paper documents] --
These are mainly:
\begin{itemize}
\item Starlink documents -- see below for more details.
\item Books on Unix and related software.
\end{itemize}

\item [Personal contact] --
This can be the best source of information when you start using Starlink.
Your primary contact is your Site Manager, but find out how other users work
and what they find useful.
If you want to use a software package or analyse a standard data type,
consult people with relevant experience.
Don't stop there though.
There's a lot of information available and the people you talk to may
not be aware of it.
\end{description}

\newpage

\subsection{Starlink documents}

Starlink publishes its
\htmladdnormallink{documents}{http://www.starlink.ac.uk/docscode.html}
in the following series:

\begin{latexonly}
{\small
\begin{tabbing}
xx\=MUDxx\=- Starlink General Papersxxx\=LUNxx\=x\kill
\>{\bf SC}\>-- Starlink Cookbooks\\
\>{\bf SG}\>-- Starlink Guides\\
\>{\bf SUN}\>-- Starlink User Notes     \>{\bf LUN}\>-- Local User Notes\\
\>{\bf SGP}\>-- Starlink General Papers \>{\bf LGP}\>-- Local General Papers\\
\>{\bf SSN}\>-- Starlink System Notes   \>{\bf LSN}\>-- Local System Notes\\
\\
\>{\bf MUD}\>-- Miscellaneous User Documents
\end{tabbing}
}
\end{latexonly}

\begin{htmlonly}
\begin{tabular}{llll}
{\bf SC}   & -- Starlink Cookbooks      &           & \\
{\bf SG}   & -- Starlink Guides         &           & \\
{\bf SUN}  & -- Starlink User Notes     & {\bf LUN} & -- Local User Notes\\
{\bf SGP}  & -- Starlink General Papers & {\bf LGP} & -- Local General Papers\\
{\bf SSN}  & -- Starlink System Notes   & {\bf LSN} & -- Local System Notes\\
\\
{\bf MUD}  & -- Miscellaneous User Documents &     &
\end{tabular}
\smallskip
\end{htmlonly}


{\em Starlink}\/ documents are Project-wide and are available at every site.
{\em Local}\/ documents are issued by individual sites and contain information
of local relevance; they are not normally available away from their site of
issue.

\begin{itemize}
\item {\em Cookbooks}\/ provide step-by-step instructions on how to use a
 particular program or package to perform specific tasks.
\item {\em Guides}\/ provide tutorial information on large or complex software
 items.
\item {\em User Notes}\/ describe in detail how to use a particular program,
 package, or subroutine library.
\item {\em General Papers}\/ contain a wide variety of material concerning the
 development and management of Starlink.
\item {\em System Notes}\/ describe system software, and give implementation and
 installation details for applications software.
\end{itemize}

The following documents are general introductions to Starlink which
supplement the information contained in the one you are currently reading:

\newpage

\begin{quote}
\begin{description}
\item [\xref{SUN/1}{sun1}{}]
-- A survey of Starlink software, organised by function and category.
\item [\xref{SUN/145}{sun145}{}]
-- An introduction to Unix.
\item [\xref{SGP/31}{sgp31}{}]
-- An introduction to the Starlink Project.
\item [\htmladdnormallink{Starlink Bulletin}{http://www.starlink.ac.uk/bulletin.html}]
-- A newsletter, published twice a year, to keep you in touch with Starlink's
progress.
You should get one delivered to you by your Site Manager.
If not, ask for a copy.
\end{description}
\end{quote}

There are also:

\begin{quote}
\begin{description}
\item [Quick Reference Cards]
-- for Starlink software and documents (SUN/1),
Unix commands (SUN/145),
and the KAPPA package (SUN/95).
\item [Glossies]
-- introduce you to various popular Starlink packages (currently
CCDPACK, CGS4DR, CURSA, FIGARO, GAIA, KAPPA, PISA).
\item [Calling Cards]
-- these are credit-card-sized notes that provide useful summaries of
Starlink information.
\end{description}
\end{quote}

Starlink recommends \LaTeX\ for document production
(\xref{SGP/28}{sgp28}{}).
Anyone may contribute documents to the Starlink series, but if you produce one,
please use the standard templates held in files such as
{\tt /star/docs/sun.tex}.
This makes life much easier for Starlink staff as it will give your document
a standard style.
The Project reserves the right to make minor corrections and stylistic changes
to submitted documents in order to achieve uniform standards.
When revising documents you {\em must}, therefore, start from the released
version stored in directory {\tt /star/docs} rather than from your own copy.

\subsection{World Wide Web}

All
\htmladdnormallink{Starlink sites}{http://www.starlink.ac.uk/sites.html}
provide information about themselves on the World Wide Web.
There is a lot available and it is easy to access, as long as you are
connected to the Internet and have a browser, such as Netscape, available.

The
\htmladdnormallink{central Starlink home page}{http://www.starlink.ac.uk/}
is installed at RAL and has the following address (URL):
\begin{quote}
{\tt http://www.starlink.ac.uk/}
\end{quote}
You should be able to find information quite easily by following obvious
links.
Look at the links under {\em News}\/ to find out about new
Project developments.
You can go to pages provided by other Starlink sites by following the
{\em \htmladdnormallink{Sites}{http://www.starlink.ac.uk/sites.html}}\/ link.
You should also explore the
{\em \htmladdnormallink{Documentation}
{http://www.starlink.ac.uk/docs.html}}\/
link as this provides access to the Starlink documentation set through
several different types of index, and also the links under {\em Software}\/
which give you access to lots of information about Starlink software.

Get to know your way around your local site's web pages as these are likely to
be the most relevant to your needs.
However, check Starlink's central web pages (as described above) for the
latest releases of software and documents (the
{\em \htmladdnormallink{New products}
{http://www.starlink.ac.uk/\~{}cac/publicity/new_products.html}}\/
link)
because these may be more up-to-date than your local versions.

\subsection{On-line files}

Starlink documents are stored in directory {\tt /star/docs}.
The files have obvious names ({\tt sun1.tex} stores the text for SUN/1, and so
on).
However, they are in \LaTeX\ format, so they are not suitable for reading
directly.
You can copy the files and use \LaTeX\ to produce a readable version
(\xref{SUN/9}{sun9}{}),
or you can look at the version on our World Wide Web (follow the
{\em \htmladdnormallink{Documentation}
{http://www.starlink.ac.uk/docs.html}}\/
link).

Many Starlink software packages have their own on-line help system, and also
may have pages under the
{\em \htmladdnormallink{List of items}
{http://www.starlink.ac.uk/softalpha.html}}\/
link on the Starlink home page.

Several on-line files contain summaries or indexes of Starlink information:

\newpage

\begin{description}

\item [Starlink Newsgroups] \hspace*{\fill}

\begin{description}
\item [uk.org.starlink.announce] -- news.
\end{description}

\item [Software] \hspace*{\fill}

\begin{description}
\item [/star/docs/news] -- latest software releases.
\item [/star/admin/status] -- current release level.
\item [/star/admin/ssi] -- full index of Starlink software.
\item [/star/admin/support] -- support levels.
\end{description}

\item [Documents] \hspace*{\fill}

\begin{description}
\item [/star/docs/docs\_lis] --  by type and serial number.
\item [/star/docs/mud\_lis] -- by serial number (MUDs only).
\item [/star/docs/analysis\_lis] -- by software name.
\item [/star/docs/subject\_lis] --  by subject and keyword.
\item [/star/docs/news] -- latest document releases.
\end{description}

\item [People and Sites] \hspace*{\fill}

\begin{description}
\item [/star/admin/usernames] -- site numbers.
\item [/star/admin/location] -- location codes.
\item [/star/admin/whoswho] -- Project staff and site details.
\end{description}

\end{description}

\subsection{Automated search and display}

Utilities are provided to help you find things and keep in touch:

\begin{description}
\item [showme] -- simply displays the document you name, for example:
 \begin{quote}
 {\tt \% showme sun188}
 \end{quote}
 will display the top page of the document
 (\xref{SUN/188}{sun188}{})
 which describes {\bf showme} and {\bf findme}.
\item [findme] -- performs keyword searching of Starlink documentation,
 for example, if you were interested in analysing CCD data:
 \begin{quote}
 {\tt \% findme CCD}
 \end{quote}
 will search for the word ``CCD" in document titles, and produce links in a
 web browser to the documents found.
\item [docfind] -- searches Starlink document indexes for keywords you supply,
 for example:
 \begin{quote}
 {\tt \% docfind tape}
 \end{quote}
 It is fairly crude and limited as it only looks at titles and keywords.
\item [email] -- searches for a person's e-mail address, for example:
 \begin{quote}
 {\tt \% email Smith}
 \end{quote}
 It searches Starlink's user lists first, then a world-wide list (usually
 somewhat out-of-date).
 A search for a more general class of astronomer is available at\\
 \htmladdnormallink{http://www.ast.cam.ac.uk/astrosearch.html}
 {http://www.ast.cam.ac.uk/astrosearch.html}.
\item [news] -- activates the on-line News system.
 It tells you about the latest software releases, jobs, and other happenings.
\end{description}

You can also use the {\tt grep} command to examine files for a given text
string and display all the lines found to contain it.
As an example, suppose you want to find documents describing programs for
handling magnetic tapes.
You can search the document and subject indexes for the string `tape' with the
command:
{\small
\begin{verbatim}
    % grep -i tape /star/docs/docs_lis /star/docs/subject_lis
\end{verbatim}
}
This command can also be used to find a person's username:
{\small
\begin{verbatim}
    % grep -i Smith /star/admin/unixnames
\end{verbatim}
}
You can then e-mail messages to this username.
({\bf email} is a better way of finding people's e-mail addresses.)

\newpage

\section{Hardware}

The computer hardware operated by Starlink consists of workstations and
servers from various manufacturers.
Your local guide should describe the hardware available at your site.
The most common computers are Sun and Alpha, and there are many X terminals
and PCs around.
The computers at each site are linked together by Local Area Networks.

Each site is connected to the UK-wide JANET communications network, which is
part of the Internet.
The links between different sites differ in speed, but even on the slowest
sections it is possible to transfer several thousand Kbyte per hour.

\newpage

\section{Software}

The {\em Starlink Software Collection}\/ (SSC) is a set of packages, utilities,
subroutine libraries, and infrastructure which is useful to astronomers.
It is managed by the Starlink Software Librarian (Martin Bly) at RAL.
It is called a {\em Collection}\/ rather than a {\em Package}\/ or
{\em System}\/ because it consists of over a hundred {\em software items}\/
which are largely autonomous.

The Collection is installed at every Starlink site.
It is also available on CD from the Starlink Software Librarian, or via
the World Wide Web on the
{\em \htmladdnormallink{Starlink Software Store}
{http://www.starlink.ac.uk/cgi-store/storetop}}.
Starlink software runs on three different platforms: SUN/Solaris,
Alpha/Digital Unix, PC/Linux.

Starlink software is intended, in general, for non-profit astronomical research.
A full statement of the terms and conditions of use of Starlink software
will be included in a forthcoming revision of SGP/21.

\xref{SUN/1}{sun1}{} describes the items in the Collection.
\xref{SGP/20}{sgp20}{} describes the definition, organisation, and management
of the Collection.

Updates to Starlink software are distributed on CD to all Starlink sites
twice a year -- Spring and Autumn (see \xref{SUN/212}{sun212}{}).
The Collection has also been distributed to over 100 non-Starlink sites, most of
which are outside the UK.
These include the AAO, LPO, ESO/ST-ECF, JACH, STScI and VILSPA, so astronomers
visiting these places can use their favourite Starlink software, just like at
home.

Starlink supports this software to the best of its abilities, within the
resources available.
Expert help is available for most things, but Starlink may have difficulties
with some items, especially as many of the original writers were not software
professionals and/or have since left.
Software support is, in fact, one of the most intractable problems facing
Starlink.
Some reported bugs take a long time to fix, although others can be cured
and the updated software released within a day or two.
Where the original author is available, response can be very rapid.

Starlink recognises the need for a {\em Software Environment}\/ into which its
applications software can be integrated.
The principle is to provide a standard user interface (for example a command
language or GUI) and programmer interface (subroutine libraries), and to allow
part-processed data to be transferred easily from application to application
during processing.
The command language enables you to manipulate data in a concise but natural
way.
The subroutine libraries give programmers efficient and easy access to
command parameters, bulk data, graphics devices, and other resources.
The standard Starlink environment is described in
\xref{SG/4}{sg4}{},
and most Starlink applications run within it.

A new software service for users was introduced in 1997.
This is the {\em Quick}\/ programming service (described on the web at URL
\htmladdnormallink{http://www.starlink.ac.uk/quick/}
{http://www.starlink.ac.uk/quick/}).
It is meant for software problems which a Starlink programmer can solve in
about a day (or less).
If you have such a problem, send it by e-mail to:
\begin{quote}
{\tt quick@star.rl.ac.uk}
\end{quote}

\newpage

\section{Publications}

Whenever you use Starlink resources -- Starlink hardware, software,
user-support, or manager/programmer effort -- please remember to acknowledge
Starlink in your publications.
This will help Starlink get the resources it needs to help you.
In particular:
\begin{itemize}
\item Cite Starlink software by name ({\em e.g.}\/ FIGARO, KAPPA) at
appropriate points in your text, with references to the authors or a citation
of the appropriate Starlink User Notes.
\item In your acknowledgements, add the sentence: ``We acknowledge the support
[software, data analysis facilities, {\em etc.}\/ as appropriate] provided
by the Starlink Project which is run by CCLRC on behalf of PPARC."
\item Include Starlink User Notes, {\em etc.}\/ in your reference list, using
the format:\\
{\em Author, A.N., 1997, Starlink User Note 123, Rutherford Appleton
Laboratory.}
\end{itemize}

\newpage

\section {Complaints}

If you are unhappy with some aspect of Starlink's service:

\begin{itemize}

\item  discuss it with your local Site Manager and, if it can't be resolved
this way, with your Site Chairman.

\item  if your complaint is still unresolved, take it up with the Starlink
Project Manager at RAL, with correspondence copied to the Chairman of the
Astronomical Computing Panel (to find names and addresses, look in file
{\tt /star/admin/whoswho}).

\end{itemize}

This procedure should not be seen as the only way of raising complaints
-- it is intended to be in addition to, not instead of, the informal channels
that have been used since the Project began.

\appendix

\newpage

\section {\label{AppendixA}Starlink Sites}

Up-to-date site information is kept in {\tt /star/\-admin/\-whoswho}.
This includes the names of Site Managers and Site Chairmen, with their e-mail
addresses.

Each site has a three letter code which can be used to identify it, as shown
below:
\begin{center}
\begin{tabular}{|l|l|}
\hline
{\bf Site Code} & {\bf Site Name} \\
\hline
\hline
 ARM & Armagh Observatory\\
 BEL & Queen's University of Belfast\\
     & {\em Pure \& Applied Physics Dept}\\
\hline
 BIR & Birmingham University\\
     & {\em School of Physics \& Space Research}\\
\hline
 BRI & Bristol University\\
     & {\em Physics Dept}\\
\hline
 CAM & Cambridge: Institute of Astronomy\\
     & Cambridge: Royal Greenwich Observatory\\
 MRA & Cambridge: {\small Mullard Radio Astronomy Observatory}\\
\hline
 CAR & University of Wales College of Cardiff\\
     & {\em Physics \& Astronomy Dept}\\
\hline
 DUR & Durham University \\
     & {\em Physics Dept}\\
\hline
 EDI & Royal Observatory Edinburgh\\
     & Edinburgh University \\
     & {\em Institute for Astronomy}\\
\hline
 GLA & Glasgow University \\
     & {\em Physics \& Astronomy Dept}\\
\hline
 HAT & Hertfordshire University, Hatfield \\
     & {\em Physical Sciences Dept}\\
\hline
 IMP & Imperial College, London \\
     & {\em Physics Dept}\\
\hline
 JOD & Nuffield Radio Astronomy Lab, Jodrell Bank\\
\hline
 KEE & Keele University \\
     & {\em Physics Dept}\\
\hline
\end{tabular}
\end{center}

\newpage

\begin{center}
\begin{tabular}{|l|l|}
\hline
{\bf Site Code} & {\bf Site Name} \\
\hline
\hline
 KEN & Kent University \\
     & {\em Electronic Engineering Lab}\\
\hline
 LEE & Leeds University \\
     & {\em Physics \& Astronomy Dept}\\
\hline
 LEI & Leicester University \\
     & {\em Physics \& Astronomy Dept}\\
\hline
 LJM & Liverpool John Moores University \\
     & {\em School of Electrical Engineering, Electronics} \\
     & {\em \& Physics}\\
\hline
 MAN & Manchester University \\
     & {\em Astronomy Dept}\\
\hline
 OXF & Oxford University \\
     & {\em Astrophysics Dept}\\
\hline
 PRE & Central Lancashire University, Preston \\
     & {\em Centre for Astrophysics}\\
\hline
 QMW & Queen Mary \& Westfield College, London \\
     & {\em Physics Dept and School of Mathematical Sciences}\\
\hline
 PRO & Rutherford Appleton Laboratory \\
     & {\em Project Management}\\
 RAL & Rutherford Appleton Laboratory \\
     & {\em SSD, Astrophysics Division}\\
\hline
 SHE & Sheffield University \\
     & {\em Physics Dept}\\
\hline
 STA & St Andrews University \\
     & {\em School of Physics \& Astronomy}\\
\hline
 SOU & Southampton University \\
     & {\em Physics Dept}\\
\hline
 SUS & Sussex University \\
     & {\em Astronomy Centre}\\
\hline
 UCL & University College London \\
     & {\em Physics \& Astronomy Dept}\\
\hline
\end{tabular}
\end{center}

\newpage

\section {\label{AppendixB}Checklist}
Do you know how to:
\begin{itemize}
\item Browse Starlink information easily?\\
\hspace*{10mm} {\em (Access
               \htmladdnormallink{Starlink World Wide Web}
                                 {http://www.starlink.ac.uk/} at\\
\hspace*{10mm} http://www.starlink.ac.uk/)}
\item Get help?\\
\hspace*{10mm} {\em (Ask your Site Manager)}
\item Login and logout?\\
\hspace*{10mm} {\em (\xref{SUN/145}{sun145}{} \& Local Guide)}
\item Interpret directory and file names?\\
\hspace*{10mm} {\em (\xref{SUN/145}{sun145}{})}
\item Find information on a Unix Command?\\
\hspace*{10mm} {\em (\% man command)}
\item Change your password?\\
\hspace*{10mm} {\em (\% yppasswd or nispasswd -- ask your Site Manager)}
\item Send and receive mail and purge your mail file?\\
\hspace*{10mm} {\em (\% pine)}
\item Find a person's username?\\
\hspace*{10mm} {\em (\% email surname)}
\item Find a username's owner?\\
\hspace*{10mm} {\em (\% grep username /star/\-admin/\-unixnames)}
\item Examine a file?\\
\hspace*{10mm} {\em (\% more filename)}
\item Print a file?\\
\hspace*{10mm} {\em (Ask your Site Manager, or look in your Local Guide)}
\item Get a copy of a document?\\
\hspace*{10mm} {\em (Get a copy from your local document store)}
\item Find a document about a Starlink software item?\\
\hspace*{10mm} {\em (\htmladdnormallink{http://www.starlink.ac.uk/docs.html}
{http://www.starlink.ac.uk/docs.html})}
\item Find out about the latest software and documentation releases?\\
\hspace*{10mm}
{\em (http://www.starlink.ac.uk/\~{}cac/publicity/new\_products.html)}
\item Find out about Starlink personnel, sites, and committees?\\
\hspace*{10mm} {\em (Look at /star/\-admin/\-whoswho)}
\item Find out about a specific topic?\\
\hspace*{10mm} {\em (\htmladdnormallink{http://www.starlink.ac.uk/cgi-bin/htxfinder}
{http://www.starlink.ac.uk/cgi-bin/htxfinder})}
\item Guard your files against loss?\\
\hspace*{10mm} {\em (Use the tar utility to write to tape)}
\item Report an error in Starlink software or documentation?\\
\hspace*{10mm} {\em (Send a mail message to starlink@jiscmail.ac.uk)}
\end{itemize}
\end{document}
