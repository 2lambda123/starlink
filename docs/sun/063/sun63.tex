\documentclass[11pt,twoside]{article}
\pagestyle{myheadings}

% -----------------------------------------------------------------------------
% ? Document identification
\newcommand{\stardoccategory}  {Starlink User Note}
\newcommand{\stardocinitials}  {SUN}
\newcommand{\stardocsource}    {sun\stardocnumber}
\newcommand{\stardocnumber}    {63.4}
\newcommand{\stardocauthors}   {Paul Rees}
\newcommand{\stardocdate}      {28 July 1992}
\newcommand{\stardoctitle}     {SPAG\\[2ex]
                               Spaghetti Unscrambler}
\newcommand{\stardocabstract}  {
SPAG is a program for unscrambling ``spaghetti'' Fortran code by
re-ordering blocks of statements in such a way that the structure of the
Fortran is improved, whilst remaining logically equivalent to the original
program.
The result improves the readability and maintainability of badly-written
Fortran.
}
% ? End of document identification

% -----------------------------------------------------------------------------

\newcommand{\stardocname}{\stardocinitials /\stardocnumber}
\markright{\stardocname}
\setlength{\textwidth}{160mm}
\setlength{\textheight}{230mm}
\setlength{\topmargin}{-2mm}
\setlength{\oddsidemargin}{0mm}
\setlength{\evensidemargin}{0mm}
\setlength{\parindent}{0mm}
\setlength{\parskip}{\medskipamount}
\setlength{\unitlength}{1mm}

% -----------------------------------------------------------------------------
%  Hypertext definitions.
%  ======================
%  These are used by the LaTeX2HTML translator in conjunction with star2html.

%  Comment.sty: version 2.0, 19 June 1992
%  Selectively in/exclude pieces of text.
%
%  Author
%    Victor Eijkhout                                      <eijkhout@cs.utk.edu>
%    Department of Computer Science
%    University Tennessee at Knoxville
%    104 Ayres Hall
%    Knoxville, TN 37996
%    USA

%  Do not remove the %begin{latexonly} and %end{latexonly} lines (used by
%  star2html to signify raw TeX that latex2html cannot process).
%begin{latexonly}
\makeatletter
\def\makeinnocent#1{\catcode`#1=12 }
\def\csarg#1#2{\expandafter#1\csname#2\endcsname}

\def\ThrowAwayComment#1{\begingroup
    \def\CurrentComment{#1}%
    \let\do\makeinnocent \dospecials
    \makeinnocent\^^L% and whatever other special cases
    \endlinechar`\^^M \catcode`\^^M=12 \xComment}
{\catcode`\^^M=12 \endlinechar=-1 %
 \gdef\xComment#1^^M{\def\test{#1}
      \csarg\ifx{PlainEnd\CurrentComment Test}\test
          \let\html@next\endgroup
      \else \csarg\ifx{LaLaEnd\CurrentComment Test}\test
            \edef\html@next{\endgroup\noexpand\end{\CurrentComment}}
      \else \let\html@next\xComment
      \fi \fi \html@next}
}
\makeatother

\def\includecomment
 #1{\expandafter\def\csname#1\endcsname{}%
    \expandafter\def\csname end#1\endcsname{}}
\def\excludecomment
 #1{\expandafter\def\csname#1\endcsname{\ThrowAwayComment{#1}}%
    {\escapechar=-1\relax
     \csarg\xdef{PlainEnd#1Test}{\string\\end#1}%
     \csarg\xdef{LaLaEnd#1Test}{\string\\end\string\{#1\string\}}%
    }}

%  Define environments that ignore their contents.
\excludecomment{comment}
\excludecomment{rawhtml}
\excludecomment{htmlonly}

%  Hypertext commands etc. This is a condensed version of the html.sty
%  file supplied with LaTeX2HTML by: Nikos Drakos <nikos@cbl.leeds.ac.uk> &
%  Jelle van Zeijl <jvzeijl@isou17.estec.esa.nl>. The LaTeX2HTML documentation
%  should be consulted about all commands (and the environments defined above)
%  except \xref and \xlabel which are Starlink specific.

\newcommand{\htmladdnormallinkfoot}[2]{#1\footnote{#2}}
\newcommand{\htmladdnormallink}[2]{#1}
\newcommand{\htmladdimg}[1]{}
\newenvironment{latexonly}{}{}
\newcommand{\hyperref}[4]{#2\ref{#4}#3}
\newcommand{\htmlref}[2]{#1}
\newcommand{\htmlimage}[1]{}
\newcommand{\htmladdtonavigation}[1]{}

% Define commands for HTML-only or LaTeX-only text.
\newcommand{\html}[1]{}
\newcommand{\latex}[1]{#1}

% Use latex2html 98.2.
\newcommand{\latexhtml}[2]{#1}

%  Starlink cross-references and labels.
\newcommand{\xref}[3]{#1}
\newcommand{\xlabel}[1]{}

%  LaTeX2HTML symbol.
\newcommand{\latextohtml}{{\bf LaTeX}{2}{\tt{HTML}}}

%  Define command to re-centre underscore for Latex and leave as normal
%  for HTML (severe problems with \_ in tabbing environments and \_\_
%  generally otherwise).
\newcommand{\setunderscore}{\renewcommand{\_}{{\tt\symbol{95}}}}
\latex{\setunderscore}

% -----------------------------------------------------------------------------
%  Debugging.
%  =========
%  Remove % from the following to debug links in the HTML version using Latex.

% \newcommand{\hotlink}[2]{\fbox{\begin{tabular}[t]{@{}c@{}}#1\\\hline{\footnotesize #2}\end{tabular}}}
% \renewcommand{\htmladdnormallinkfoot}[2]{\hotlink{#1}{#2}}
% \renewcommand{\htmladdnormallink}[2]{\hotlink{#1}{#2}}
% \renewcommand{\hyperref}[4]{\hotlink{#1}{\S\ref{#4}}}
% \renewcommand{\htmlref}[2]{\hotlink{#1}{\S\ref{#2}}}
% \renewcommand{\xref}[3]{\hotlink{#1}{#2 -- #3}}
%end{latexonly}
% -----------------------------------------------------------------------------
% ? Document-specific \newcommand or \newenvironment commands.
% ? End of document-specific commands
% -----------------------------------------------------------------------------
%  Title Page.
%  ===========
\renewcommand{\thepage}{\roman{page}}
\begin{document}
\thispagestyle{empty}

%  Latex document header.
%  ======================
\begin{latexonly}
   CCLRC / {\sc Rutherford Appleton Laboratory} \hfill {\bf \stardocname}\\
   {\large Particle Physics \& Astronomy Research Council}\\
   {\large Starlink Project\\}
   {\large \stardoccategory\ \stardocnumber}
   \begin{flushright}
   \stardocauthors\\
   \stardocdate
   \end{flushright}
   \vspace{-4mm}
   \rule{\textwidth}{0.5mm}
   \vspace{5mm}
   \begin{center}
   {\Huge\bf  \stardoctitle}
   \end{center}
   \vspace{5mm}

% ? Heading for abstract if used.
   \vspace{10mm}
   \begin{center}
      {\Large\bf Abstract}
   \end{center}
% ? End of heading for abstract.
\end{latexonly}

%  HTML documentation header.
%  ==========================
\begin{htmlonly}
   \xlabel{}
   \begin{rawhtml} <H1> \end{rawhtml}
      \stardoctitle
   \begin{rawhtml} </H1> \end{rawhtml}

% ? Add picture here if required.
% ? End of picture

   \begin{rawhtml} <P> <I> \end{rawhtml}
   \stardoccategory\ \stardocnumber \\
   \stardocauthors \\
   \stardocdate
   \begin{rawhtml} </I> </P> <H3> \end{rawhtml}
      \htmladdnormallink{CCLRC}{http://www.cclrc.ac.uk} /
      \htmladdnormallink{Rutherford Appleton Laboratory}
                        {http://www.cclrc.ac.uk/ral} \\
      \htmladdnormallink{Particle Physics \& Astronomy Research Council}
                        {http://www.pparc.ac.uk} \\
   \begin{rawhtml} </H3> <H2> \end{rawhtml}
      \htmladdnormallink{Starlink Project}{http://www.starlink.ac.uk/}
   \begin{rawhtml} </H2> \end{rawhtml}
   \htmladdnormallink{\htmladdimg{source.gif} Retrieve hardcopy}
      {http://www.starlink.ac.uk/cgi-bin/hcserver?\stardocsource}\\

%  HTML document table of contents.
%  ================================
%  Add table of contents header and a navigation button to return to this
%  point in the document (this should always go before the abstract \section).
  \label{stardoccontents}
  \begin{rawhtml}
    <HR>
    <H2>Contents</H2>
  \end{rawhtml}
  \htmladdtonavigation{\htmlref{\htmladdimg{contents_motif.gif}}
        {stardoccontents}}

% ? New section for abstract if used.
  \section{\xlabel{abstract}Abstract}
% ? End of new section for abstract
\end{htmlonly}

% -----------------------------------------------------------------------------
% ? Document Abstract. (if used)
%  ==================
\stardocabstract
% ? End of document abstract
% -----------------------------------------------------------------------------
% ? Latex document Table of Contents (if used).
%  ===========================================
\newpage
\begin{latexonly}
   \setlength{\parskip}{0mm}
   \tableofcontents
   \setlength{\parskip}{\medskipamount}
   \markright{\stardocname}
\end{latexonly}
% ? End of Latex document table of contents
% -----------------------------------------------------------------------------
\newpage
\renewcommand{\thepage}{\arabic{page}}
\setcounter{page}{1}

\section {Introduction\xlabel{introduction}}

SPAG is a program for unscrambling ``spaghetti'' Fortran code by
re-ordering blocks of statements in such a way that the structure of the
Fortran is improved, whilst remaining logically equivalent to the original
program.
The result improves the readability and maintainability of badly-written
Fortran.
On STARLINK, SPAG may be used to convert unstructured Fortran
77 into the structured and indented Fortran 77 required by the
STARLINK Programming Standard
(\xref{SGP/16}{sgp16}{}).
SPAG can also be used to update Fortran 66 code.
The SPAG software is part of a commercial suite of tools for software
maintenance called plusFORT (formerly PRISM), marketed by {\it Polyhedron
Software}, and is available on STARLINK DECnet node STADAT (the STARLINK
multi-r\^{o}le MicroVAX).
The other plusFORT tools are not available on STARLINK.


\section{Getting Started\xlabel{getting_started}}

SPAG is only available on STARLINK DECnet node STADAT, which is accessible
to all STARLINK registered users (contact your Site Manager for details).
Once logged onto STADAT, in order to use SPAG you must first type

\begin {quote}
\begin{verbatim}
$ SPAGINIT
\end{verbatim}
\end {quote}

This will set up several logical names and symbols used by the SPAG software.
Having done this, SPAG may be run on your Fortran file, say SOURCE.FOR, by
typing

\begin {quote}
\begin{verbatim}
$ SPAGRUN SOURCE.FOR
\end{verbatim}
\end {quote}

The asterix may be used as a wild card in the file name, {\em e.g.}

\begin {quote}
\begin{verbatim}
$ SPAGRUN S*.FOR
\end{verbatim}
\end {quote}

This will process all files beginning with ``S'' and with extension
``.FOR'' and put all the output into one file.
The output from SPAG will go into a file called SPAG.OUT by default.
An output file may be specified using the ``TO='' qualifier, {\em e.g.}

\begin {quote}
\begin{verbatim}
$ SPAGRUN SOURCE.FOR TO=SOURCE.OUT
\end{verbatim}
\end {quote}

Note that the qualifiers used with the SPAGRUN command do not need to
be preceded by a ``/''.
If no file name is given when SPAGRUN is used, all files with the
extension ``.FOR'' in the directory will be used and the output sent to
SPAG.OUT.
By default, SPAG will also create a file SPAG.LOG which contains a log of the
information written to the screen during execution.

The qualifier ``RUNTYPE='' may be used to control the effect of SPAG on the
source code.
There are three modes available using RUNTYPE:

\begin {itemize}
\item PP ---
for pretty print, or indent, only.
\item LAB ---
for re-labelling and pretty print.
\item FUL ---
for a full re-structure, re-label and pretty print (the default).
\end {itemize}

Also, the ``ERR='' qualifier may be used to route error messages generated by
SPAG to a file.
Here is an example:

\begin {quote}
\begin{verbatim}
$ SPAGRUN SOURCE.FOR TO=SOURCE.OUT ERR=SOURCE.ERR RUNTYPE=PP
\end{verbatim}
\end {quote}

This will only indent the Fortran code in SOURCE.FOR,
putting the output in SOURCE.OUT and any error messages in SOURCE.ERR.
The file name and qualifiers may occur in any order after the SPAGRUN command.

There are further qualifiers which can be used to control the effect of
SPAG on Fortran code.
The action of each one is fully documented in the plusFORT Reference Manual
(which is available from your Site Manager).
This manual gives a detailed description of how to run SPAG successfully.

The SPAG output file may be split up into individual files, each named after
and containing the code for one subroutine and with the ``.FOR'' extension,
using the QSPLIT command.
The syntax for QSPLIT is simply

\begin {quote}
\begin{verbatim}
$ QSPLIT SPAG.OUT
\end{verbatim}
\end {quote}

where SPAG.OUT is the name of the output file from SPAG.


\section{Some Warnings\xlabel{some_warnings}}

Here are some points to watch for when using SPAG:

\begin {itemize}
\item SPAG is only intended as a starting point when tidying up existing poor
quality Fortran code.
It certainly should not be seen as an inducement to writing poor quality code
and then relying on SPAG to unscramble it!

\item SPAG cannot recognise the Fortran standard to which the source code was
written.
Care should therefore be taken when using SPAG to update code written to the
Fortran 66 standard to ensure that the program's functioning remains unaffected
by the restructuring.

\item Should it be necessary, the output from SPAG can be customised by
modifying the SPAG configuration file used.
However, you should note that the precise format of this file changes with each
update of SPAG released by {\it Polyhedron Software}.
If your private configuration file stops working, check that SPAG has not been
updated.
If it has, then the remedy is to use the configuration file provided in
PRISMDIR as a template.
\end {itemize}

\section{Example\xlabel{example}}

There follows (overleaf) an example of the effect SPAG has when run on a
piece of ``poor'' code, taken from a matrix inversion
subroutine written in Fortran 66.

\twocolumn
\begin {center}
{\bf\large Before SPAG:}
\end {center}
\begin {verbatim}

C
C  Final row and column interchange
C
      k=n
   20 k=k-1
      if(k)21,21,22
22    i=l(k)
      if(i-k)23,23,24
24    jq=n*(k-1)
         jr=n*(i-1)
      do 25 j=1,n
      jk=jq+j
      hold=a(jk)
        ji=jr+j
      a(jk)=-a(ji)
   25 a(ji)=hold
23    j=m(k)
         if(j-k)20,20,26
 26   ki=k-n
       do 27 i=1,n
       ki=ki+n
       hold=a(ki)
      ji=ki-k+j
      a(ki)=-a(ji)
27     a(ji)=hold
      goto 20
 21   return
       end
\end{verbatim}
\pagebreak

\begin {center}
{\bf\large After SPAG:}
\end {center}
\begin {verbatim}

C
C  Final row and column interchange
C
      K = N
 100  CONTINUE
      K = K - 1
      IF ( K.LE.0 ) THEN
         RETURN
      ELSE
         I = L(K)
         IF ( I.GT.K ) THEN
            JQ = N*(K-1)
            JR = N*(I-1)
            DO J = 1,N
               JK = JQ + J
               HOLD = A(JK)
               JI = JR + J
               A(JK) = -A(JI)
               A(JI) = HOLD
            END DO
         END IF
         J = M(K)
         IF ( J.GT.K ) THEN
            KI = K - N
            DO I = 1,N
               KI = KI + N
               HOLD = A(KI)
               JI = KI - K + J
               A(KI) = -A(JI)
               A(JI) = HOLD
            END DO
         END IF
         GO TO 100
      END IF
      END
\end{verbatim}
\end {document}
