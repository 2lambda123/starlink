\documentstyle[11pt]{article}
\pagestyle{myheadings}

% -----------------------------------------------------------------------------
% ? Document identification
\newcommand{\stardoccategory}  {Starlink User Note}
\newcommand{\stardocinitials}  {SUN}
\newcommand{\stardocsource}    {sun\stardocnumber}
\newcommand{\stardocnumber}    {58.7}
\newcommand{\stardocauthors}   {C. Lloyd}
\newcommand{\stardocdate}      {16 March 1995}
\newcommand{\stardoctitle}     {IUEDA\\[2ex]
                               Access to the IUE data archive}
% ? End of document identification

% -----------------------------------------------------------------------------

\newcommand{\stardocname}{\stardocinitials /\stardocnumber}
\markright{\stardocname}
\setlength{\textwidth}{160mm}
\setlength{\textheight}{230mm}
\setlength{\topmargin}{-2mm}
\setlength{\oddsidemargin}{0mm}
\setlength{\evensidemargin}{0mm}
\setlength{\parindent}{0mm}
\setlength{\parskip}{\medskipamount}
\setlength{\unitlength}{1mm}

% -----------------------------------------------------------------------------
%  Hypertext definitions.
%  ======================
%  These are used by the LaTeX2HTML translator in conjunction with star2html.

%  Comment.sty: version 2.0, 19 June 1992
%  Selectively in/exclude pieces of text.
%
%  Author
%    Victor Eijkhout                                      <eijkhout@cs.utk.edu>
%    Department of Computer Science
%    University Tennessee at Knoxville
%    104 Ayres Hall
%    Knoxville, TN 37996
%    USA

%  Do not remove the %\begin{rawtex} and %\end{rawtex} lines (used by 
%  star2html to signify raw TeX that latex2html cannot process).
%\begin{rawtex}
\makeatletter
\def\makeinnocent#1{\catcode`#1=12 }
\def\csarg#1#2{\expandafter#1\csname#2\endcsname}

\def\ThrowAwayComment#1{\begingroup
    \def\CurrentComment{#1}%
    \let\do\makeinnocent \dospecials
    \makeinnocent\^^L% and whatever other special cases
    \endlinechar`\^^M \catcode`\^^M=12 \xComment}
{\catcode`\^^M=12 \endlinechar=-1 %
 \gdef\xComment#1^^M{\def\test{#1}
      \csarg\ifx{PlainEnd\CurrentComment Test}\test
          \let\html@next\endgroup
      \else \csarg\ifx{LaLaEnd\CurrentComment Test}\test
            \edef\html@next{\endgroup\noexpand\end{\CurrentComment}}
      \else \let\html@next\xComment
      \fi \fi \html@next}
}
\makeatother

\def\includecomment
 #1{\expandafter\def\csname#1\endcsname{}%
    \expandafter\def\csname end#1\endcsname{}}
\def\excludecomment
 #1{\expandafter\def\csname#1\endcsname{\ThrowAwayComment{#1}}%
    {\escapechar=-1\relax
     \csarg\xdef{PlainEnd#1Test}{\string\\end#1}%
     \csarg\xdef{LaLaEnd#1Test}{\string\\end\string\{#1\string\}}%
    }}

%  Define environments that ignore their contents.
\excludecomment{comment}
\excludecomment{rawhtml}
\excludecomment{htmlonly}
%\end{rawtex}

%  Hypertext commands etc. This is a condensed version of the html.sty
%  file supplied with LaTeX2HTML by: Nikos Drakos <nikos@cbl.leeds.ac.uk> &
%  Jelle van Zeijl <jvzeijl@isou17.estec.esa.nl>. The LaTeX2HTML documentation
%  should be consulted about all commands (and the environments defined above)
%  except \xref and \xlabel which are Starlink specific.

\newcommand{\htmladdnormallinkfoot}[2]{#1\footnote{#2}}
\newcommand{\htmladdnormallink}[2]{#1}
\newcommand{\htmladdimg}[1]{}
\newenvironment{latexonly}{}{}
\newcommand{\hyperref}[4]{#2\ref{#4}#3}
\newcommand{\htmlref}[2]{#1}
\newcommand{\htmlimage}[1]{}
\newcommand{\htmladdtonavigation}[1]{}

%  Starlink cross-references and labels.
\newcommand{\xref}[3]{#1}
\newcommand{\xlabel}[1]{}

%  LaTeX2HTML symbol.
\newcommand{\latextohtml}{{\bf LaTeX}{2}{\tt{HTML}}}

%  Define command to re-centre underscore for Latex and leave as normal
%  for HTML (severe problems with \_ in tabbing environments and \_\_
%  generally otherwise).
\newcommand{\latex}[1]{#1}
\newcommand{\setunderscore}{\renewcommand{\_}{{\tt\symbol{95}}}}
\latex{\setunderscore}

%  Redefine the \tableofcontents command. This procrastination is necessary 
%  to stop the automatic creation of a second table of contents page
%  by latex2html.
\newcommand{\latexonlytoc}[0]{\tableofcontents}

% -----------------------------------------------------------------------------
%  Debugging.
%  =========
%  Remove % on the following to debug links in the HTML version using Latex.

% \newcommand{\hotlink}[2]{\fbox{\begin{tabular}[t]{@{}c@{}}#1\\\hline{\footnotesize #2}\end{tabular}}}
% \renewcommand{\htmladdnormallinkfoot}[2]{\hotlink{#1}{#2}}
% \renewcommand{\htmladdnormallink}[2]{\hotlink{#1}{#2}}
% \renewcommand{\hyperref}[4]{\hotlink{#1}{\S\ref{#4}}}
% \renewcommand{\htmlref}[2]{\hotlink{#1}{\S\ref{#2}}}
% \renewcommand{\xref}[3]{\hotlink{#1}{#2 -- #3}}
% -----------------------------------------------------------------------------
% ? Document specific \newcommand or \newenvironment commands.
% ? End of document specific commands
% -----------------------------------------------------------------------------
%  Title Page.
%  ===========
\renewcommand{\thepage}{\roman{page}}
\begin{document}
\thispagestyle{empty}

%  Latex document header.
%  ======================
\begin{latexonly}
   CCLRC / {\sc Rutherford Appleton Laboratory} \hfill {\bf \stardocname}\\
   {\large Particle Physics \& Astronomy Research Council}\\
   {\large Starlink Project\\}
   {\large \stardoccategory\ \stardocnumber}
   \begin{flushright}
   \stardocauthors\\
   \stardocdate
   \end{flushright}
   \vspace{-4mm}
   \rule{\textwidth}{0.5mm}
   \vspace{5mm}
   \begin{center}
   {\Huge\bf  \stardoctitle}
   \end{center}
   \vspace{5mm}

\end{latexonly}

%  HTML documentation header.
%  ==========================
\begin{htmlonly}
   \xlabel{}
   \begin{rawhtml} <H1> \end{rawhtml}
      \stardoctitle
   \begin{rawhtml} </H1> \end{rawhtml}

% ? Add picture here if required.
% ? End of picture

   \begin{rawhtml} <P> <I> \end{rawhtml}
   \stardoccategory \stardocnumber \\
   \stardocauthors \\
   \stardocdate
   \begin{rawhtml} </I> </P> <H3> \end{rawhtml}
      \htmladdnormallink{CCLRC}{http://www.cclrc.ac.uk} /
      \htmladdnormallink{Rutherford Appleton Laboratory}
                        {http://www.cclrc.ac.uk/ral} \\
      \htmladdnormallink{Particle Physics \& Astronomy Research Council}
                        {http://www.pparc.ac.uk} \\
   \begin{rawhtml} </H3> <H2> \end{rawhtml}
      \htmladdnormallink{Starlink Project}{http://star-www.rl.ac.uk/}
   \begin{rawhtml} </H2> \end{rawhtml}
   \htmladdnormallink{\htmladdimg{source.gif} Retrieve hardcopy}
      {http://star-www.rl.ac.uk/cgi-bin/hcserver?\stardocsource}\\

%  HTML document table of contents. 
%  ================================
%  Add table of contents header and a navigation button to return to this 
%  point in the document (this should always go before the abstract \section). 
  \label{stardoccontents}
  \begin{rawhtml} 
    <HR>
    <H2>Contents</H2>
  \end{rawhtml}
  \renewcommand{\latexonlytoc}[0]{}
  \htmladdtonavigation{\htmlref{\htmladdimg{contents_motif.gif}}
        {stardoccontents}}

\end{htmlonly}

% ? Latex document Table of Contents (if used).
%  ===========================================
 \newpage
 \begin{latexonly}
   \setlength{\parskip}{0mm}
   \latexonlytoc
   \setlength{\parskip}{\medskipamount}
   \markright{\stardocname}
 \end{latexonly}
% ? End of Latex document table of contents
% -----------------------------------------------------------------------------
\newpage
\renewcommand{\thepage}{\arabic{page}}
\setcounter{page}{1}

\section {Introduction\xlabel{introduction}}

The {\it International Ultraviolet Explorer\/} satellite (IUE) is one of the
most successful, versatile and productive astronomical satellites, providing
ultraviolet spectra of a wide range of sources for the whole astronomical
community.
The UK hosts one of the three full archives of IUE data
which is supported and maintained by members of the UK IUE 
Project in the Space Science Department at RAL. 
The archive comprises some 100\,000 images,
and a catalogue which gives details of the exposures. 
The aim is to provide UK astronomers with quick and painless 
access to IUE data.
This document describes how to interrogate the catalogue and extract data from
the archive. 
There is also a subsidiary archive of extracted low-resolution spectra, the 
Uniform Low Dispersion Archive,
which is described in 
\xref{SUN/20}{sun20}{}: ULDA/USSP - Accessing the IUE ULDA.

\subsection {IUE satellite}
The IUE satellite was the product
of a collaboration between NASA, ESA and (originally) SRC for the UK.
The satellite is in a geosynchronous orbit where it is permanently visible from
the NASA tracking station at the Goddard Space Flight Center (GFSC) and visible
for over 8 hours from the ESA tracking station at Villafranca del Castillo,
outside Madrid (Vilspa).
Observing time on IUE is divided into 8-hour shifts, two NASA, US1
and US2, and one combined ESA-UK shift.
The US2 shift usually suffers from the highest particle background.
IUE is run as an observatory where guest observers make decisions 
about their observing programme in real time; a style of operation 
that has been often praised but seldom emulated.

IUE was launched in 1978 and despite the failure of most of the gyros continues to
operate as originally intended.
Over that time the instruments have suffered only a
small loss of sensitivity.
IUE carries two echelle spectrographs which cover the wavelength ranges
1100 - 2100\AA, (the short wavelength, SW)
and 1900 - 3300\AA, (the long wavelength, LW).
The spectrographs may be used in two resolution modes.
At high resolution (HIRES) $\Delta\lambda\sim$ 0.1\AA\ the spectrograph is 
used in full echelle mode and $\sim$60 spectral orders fill the camera 
faceplate.
In low resolution (LORES) mode $\Delta\lambda\sim$ 6\AA\ the echelle cross 
disperser is used alone to produce a single spectral order covering the full
wavelength range.
The spectrographs have two apertures. 
The large aperture (LAP) is 10$\times$20 arcsec and at LORES
may be used like a long slit
for single, multiple, trailed or extended images with spatial information
perpendicular to the dispersion.
At HIRES the length of the large aperture is effectively along the dispersion
direction.
At LORES both apertures may be used to record two spectra on a single image.
LORES images through the LAP are absolutely calibrated but HIRES images are
not.
The small aperture (SAP) is $\sim$ 3 arcsec in diameter and vignettes the beam.
Imprecision in centring the star make the transmission of the SAP 
unpredictable so these spectra lose their absolute photometry.
Both spectrographs have a prime (P) and redundant (R) camera. The SWP camera
has been used throughout the mission while the LWR was used until 1984 and
the LWP since then.
The SWR camera has a fault in its read section and was used for only a small
number of images during the commissioning phase, early in the mission.

\subsection {IUE data}

IUE images are referred to by the camera name and a running image number,
eg. SWP22861.
Each image is held as a number of files which contain the {\it raw\/} image,
some intermediate partly processed images and the final extracted 1-d spectra.
The data in the archive is in what is known as Guest Observer (GO) format. 
The raw IUE image is a 2-d scan of the camera target. 
The intensity in each pixel is given as a Data Number (DN) in the
range 0 -- 255. 
The image is superimposed on a `pedestal' of $\sim$ 20 DN and
overexposed ({\it saturated}) parts of the spectrum are set to 255 DN. 

In the photometric image the
DN values for each pixel are converted to fluxes via an Intensity Transfer
Function (ITF) and are given in terms of a Flux Number (FN).
Two types of photometric image have been used.
Until about 1981 the raw image was geometrically as well as photometrically
corrected to produce a {\it gphot\/} image. 
The geometric correction removes the distortion in the camera and 
the read process. 
More recently the raw image has been photometrically
corrected in raw space, to yield a {\it phot\/} image, and the geometric 
correction left to a later part of the processing. 
The timing of the change over of from gphot to phot was different for
the two ground stations and different dispersions (see Table
\ref{change}).
It should also be remembered that some early images may have been
reprocessed at a later date and have a different set of output files.
The type of the photometric file is usually given at the end of the image 
header, see Section \ref{readiue}.

\begin{table}[htb]
\caption{Change over from gphot to phot}
\label{change}
\begin{center}
\begin{tabular}{rcc} 
	& Goddard & Vilspa \\ \hline
LORES	& ~3 Nov 1980 & 10 Mar 1981 \\
HIRES	& 10 Nov 1981 & 11 Mar 1981 \\

\end{tabular}
\end{center}
\end{table}

At high resolution the third and final file contains the extracted spectrum,
produced with the version of the standard image processing software (IUESIPS)
in use at the time.
The extracted spectrum file contains independent extractions of
each spectral order, wavelength calibrated and corrected for the echelle
ripple, in time-integrated FN.
The low-resolution extracted spectrum file contains the absolutely calibrated,
time-integrated flux in erg/cm$^{2}$/\AA, one file for each aperture.

Low resolution images
usually have in addition an (extended) line-by-line file ((E)LBL)
which is a re-sampled photometric image with a number of scan lines parallel 
to the dispersion direction.
Later versions of this file are extended, they 
contain more lines than the earlier files.
The LBL file may be used for extracting spectra of extended sources or
multiple exposures of a single source made in different parts of the LAP.
Early LORES images also have a rectified image segment file which is of little
practical use.
LORES images may contain spectra taken through both apertures and a very small
number of images intentionally contain spectra taken at both resolutions.
Over the lifetime of the satellite the standard image processing (IUESIPS)
has evolved and although the extracted spectra at LORES 
are absolutely calibrated they are not uniformly processed. 
Some early images have been reprocessed with more recent, but not necessarily
the most recent, processing scheme.

The dynamic range of IUE's detectors limits the signal to noise ratio of the
spectra to $\sim$ 20, in the optimally exposed parts of the spectrum.
In general it will be rather less than this but may be improved by co-adding
spectra, with a small loss of resolution. 
Limitations in the image processing (fixed pattern noise)
restricts the improvement, which falls off
rapidly with increasing number of spectra

There are a variety of marks and blemishes on the detectors that may interfere
with the spectra.
Embedded in the camera faceplate is a grid of fiducial marks which is used in
the geometric correction.
Each reseau creates an essentially black spot about 2$\times$2 pixels on the 
raw image and some of these marks inevitably appear in the spectrum.
The processing usually recognises them and they are flagged accordingly.
There are also some other permanent blemishes, hot spots and other features and
these are described in detail in Section \ref{blemish}.

Cosmic-ray hits produce bright features on the image but these range from the 
obvious spots to diffuse comet-like structures, the effects of which are not 
easily seen in the spectrum. These problems may be identified by viewing the
whole raw or photometric image.

Some images, particularly those from the LWR, may be affected by microphonic
noise, which introduces a spurious varying signal of over 20 DN on a few scan 
lines, almost anywhere on the raw image.
Depending on the resolution and where it occurs the microphonic noise may or
may not interfere with the spectrum.
Steps were introduced to move any affected region away from the spectrum (LWR
heater warmup) and reduce its impact.
Images may also suffer from brief periods of data dropouts. These are flagged
by the processing and are comparatively rare.

\subsection {IUE archive}

The archive contains all the files for each IUE image and a catalogue.
New images are continually
added to the archive and the catalogue is updated every month.
IUE observations are made as part of a Guest Observer programme and 
are formally 
available to the whole astronomical community about six months after
they are made.
Details of new observations appear in the catalogue shortly before release.

Access to the IUE archive is simply through e-mailed de-archiving requests.
The procedure is described in detail in Section \ref{dearch}.
The IUE catalogue is accessed through the captive IUE account on the VMS
database machine at RAL.
The captive account may also be used to perform other 
IUE related tasks as given in Section \ref{captive}.

\subsection {IUE analysis}

In the UK it is common practice to extract the spectra from the photometric
image using the IUEDR package (SUN/37)
which offers significant advantages over IUESIPS.
IUEDR may also be used to create LBL spectra and merge IUESIPS extracted 
spectra.
The \$IUEDR directory contains tables to correct for the camera sensitivity
degredation.
A helpful introduction to using IUEDR may be found in SG/7: IUE Analysis - A
Tutorial and in the more comprehensive but rather old MUD/45: IUEDR User
Guide.
The DIPSO package (SUN/50) has many features specifically designed for IUE
spectra and is widely used for the analysis.

For some work the Uniform Low Dispersion Archive (SUN/20) may be appropriate.
The ULDA is an on-line archive of LORES extracted spectra which have had the 
details of the targets and exposures verified. 
However it should be remembered that 
these spectra are {\it not} uniformly processed, and are identical to the 
extracted spectra in the full archive.
The ULDA may still be accessed from the various STARLINK site accounts on
stadat.

\subsection {The future}

All IUE data are being reprocessed with completely new software to produce what
will hopefully be a lasting final archive.
The Final Archive should offer uniformly processed data with improved signal to
noise.
Some LORES images have been reprocessed and these should become widely
available during the next year.
HIRES images will be reprocessed after the completion of LORES and will
probably take a further two years to complete.

\section {The IUE captive account
\xlabel{the_iue_captive_account}\label{captive}}

The IUE captive account is on the (VMS) database machine at RAL 
(stadat.rl.ac.uk).
It is principally used to access the IUE catalogue but also enables IUE
observing details, beta angles, to be calculated.
To use the captive account 

{\tt \% telnet stadat.rl.ac.uk}

and login to username IUE (no password is required).
You will be asked for a name, which is used for a subdirectory to contain your 
files, and then the system displays a list of options. These give access to the 
iue catalogue and other 
facilities, and enable a few DCL commands to be issued.
These options are:

HELP - An online help facility which gives brief descriptions of how to use
the various options in IUE.

MAIL - Gives access to the VMS mail facility.

IUELOG - Searches the IUE catalogue.

CLASSKEY - Displays a list of the various classification codes used in the
IUE catalogue.

TYPE - Same as the DCL command 'TYPE'.
It can be used to view any of the files in the IUE directory. 
Wild card characters can not be used in the TYPE command.

DIR - Displays the contents of the IUE directory.
It may be used with the same options as the DCL equivalent.

FTP - Files which have been created in the IUE directory can be
transferred to one of your own directories with this option.

BETA - Calculates beta and other angles for targets observed with IUE. To keep
a listing of the output use the unix script command before connecting to
stadat.

LOG - End the session and log-out.

It is possible to run an interactive version of the registration process and 
de-archiving program, IUEDA, from the captive account, but there is no real 
need to do so. 

\subsection {Consulting the catalogue}

To consult the catalogue select the IUELOG option. 
There is usually a page of information about the latest log update and then
you are prompted for a {\it label} which will be used to identify any files you
want to keep.
You should only need to search the MAIN catalogue (the default).
You will be asked,

{\tt Enter search key}

which may be any of the following.

{\tt HIRES} selects high resolution images

{\tt LORES} selects low resolution images, these two are mutually exclusive. 
If you do not select either then by default both resolutions are selected.

\smallskip
{\tt LWP n, m}

{\tt LWR n, m}

{\tt SWP n, m}

{\tt SWR n, m} where both n and m are optional image numbers specifying a range
of required images. If no range is given then all images are selected.
Only one camera may be specifically selected at a time but by default all 
cameras are selected. 

\smallskip
{\tt OC n, m} where n and m (optional) specify a range of object classes
otherwise 
all object classes are selected. A single object class may be specified.

\smallskip
{\tt RA hh mm ss.s, hh mm ss.s} where the ra's specify the limits of a range in
right ascension.
The RA's may be truncated to any level. Single digit hours, minutes or seconds
may be given as a single digit or with a preceding zero. Hours, minutes and
seconds {\it must} be separated by spaces and the two limits {\it must} be
separated by a comma. The limits are in the sense, from and increasing to, 
and it is possible to search across the boundary from 24 to 0 hours, although
a warning will be issued.
A single RA may be given which defines the centre of
a search area the size of which is determined by the BOX keyword (see below).

\smallskip
{\tt DEC $\pm$dd mm ss, $\pm$dd mm ss} where the Dec's specify the limits of 
a range in
declination. The general comments for RA also apply. Positive Dec's do not
require a sign.
The RA and DEC keywords may be used completely independently.

\smallskip
{\tt BOX n} where n 
specifies the size of the search box in arc minutes around a central
RA and Dec. If RA and Dec {\it limits} are given then the value of BOX is
irrelevant. There is no Dec dependence of the dimension of the box in RA.
The default value of n is 30.

\smallskip
{\tt NAME string} where string specifies a target name. The string may include
spaces and wildcard (*) characters. The time taken for a name search, including
wildcards, may be up to 90 seconds.

\smallskip
{\tt PROG string} where string specifies a programme identification code.
The string may include wildcard (*) characters.

\smallskip
{\tt DATE year day, year day} where the dates define the limits of the search.
The year may be given with or without the preceding 19.
The year and day are separated by spaces and the two limits by a comma.
The year only may be given and a single year may also be selected. 

\smallskip
{\tt CLEAR} clears all the search criteria to their default values

\smallskip
{\tt GO} starts the search

\smallskip
{\tt Q} quits the search program.

\smallskip
{\tt ?} brief outline of the keywords

\smallskip
The search keys are entered one per line in response to the prompt.
Any combination, order and number of search keys may be used. 
If an error is made simply re-enter the search key at any time.

The results of the search are stored in a workfile which you may name as
appropriate.
The workfiles may be searched (using the old system) but with multiple search
keys this should not be necessary and the new system alone should provide the 
list you want. 
Any of the workfiles may be kept with the KEEP option
and will be saved as {\it label}.IUE.
Initially the workfiles are
in RA order but it is possible to sort them using the ORDER option 
by either date or camera and image number. Enter DATE or CIM as appropriate, or
Q to quit the order option.
All the KEEPed and ORDERed files are saved as different
generations of [IUE]{\it label}.IUE. 
Note that the label and not the workfile name is used.
The {\it label}.IUE files may be transferred or MAILed to your own account.


Any files you create in the IUE directory will be deleted automatically after
two weeks.
If you wish to keep a files they must be transferred to 
one of your own directories using FTP or MAIL.


\section {De-archiving IUE data
\xlabel{dearchiving_iue_data}\label{dearch}}

IUE data may be de-archived using the procedure IUEDA. It enables near-line
access to IUE images which are transferred as disk files. The files can
only be read by IUEDRv3 and not by earlier versions.
It is also assumed that IUEDR will be running under Unix and the files will by
default be in this format.
If for some reason the files should be VMS format then contact
iues@star.rl.ac.uk.

To use IUEDA a file containing some control information and a list of
images to be de-archived is e-mailed to iues (NOT IUE) @stadat.rl.ac.uk.
This request is automatically processed and a summary of the request
and outcome of
the de-archiving job will be e-mailed back to the requestor. 
The files are
copied directly from the archive tapes to the scratch disk on stadat and
will be held for about two weeks. They may be retrieved by ftp. 
At present the automatic procedure is run every hour so there will not be 
an immediate response.

\subsection{Registration}

Before using IUEDA it is necessary to register.
The registration procedure is run once and for all, and avoids the unnecessary
repetition of information in archive requests.
To register send a mail message to iues@stadat.rl.ac.uk
with the subject

{\tt register}

The first line should contain

{\tt start}

Then give a short string to identify yourself (eg. your initials).
This will be used as a unique prefix to {it all\/} your de-archiving requests. 

{\tt cl}

Then give your first name...

{\tt Chris}

...surname...

{\tt Lloyd}

...and the name of your institute, one word, ed UCL, Keele. These inputs are
case sensitive.

{\tt RAL}

Then give your IP e-mail address.
This will be used for all communication with you. 

{\tt cl@ast.star.rl.ac.uk}

...and the last line

{\tt end}

So the text of the mail message will be

{\tt start}
\newline
{\tt cl}
\newline
{\tt Chris}
\newline
{\tt Lloyd}
\newline
{\tt RAL}
\newline
{\tt cl@ast.star.rl.ac.uk}
\newline
{\tt end}

After the mail message has been processed you will be sent confirmation 
of the registration or some diagnostics if there were any problems, eg. if your 
identifier is already in use.

\subsection{To run IUEDA}

To submit a de-archiving request
send a mail message to iues@stadat.rl.ac.uk with the subject

{\tt iueda}

The first line should contain

{\tt start}

Then give the identifier you have registered under

{\tt cl}

Then give the request name. This can be anything relatively brief but
descriptive to identify this request. Don't add anything to identify 
yourself as all your requests will be prefixed by your identifier.
The request name may be the same as one used previously, for example to keep
all the images of one target or programme together.

{\tt tausco}

Then give the image codes for the images you want, which may be any combination
of the following. 
\label{codes}
\begin{center}
\begin{tabular}{cl} 
R & Raw image file (.raw file extension) \\
G & Photometric image file gphot or phot (.phot file extension) \\
E & Extracted spectrum file (.ext file extension) \\
L & (Extended) line-by-line file (.lbl file extension) \\
S & Image segment file (.is file extension) \\
A & All files \\

\end{tabular}
\end{center}

{\tt ge}

Then give the list of image numbers; an edited list from the IUELOG output may
also be used.

{\tt swp9788}
\newline
{\tt SWP 16574}
\newline
{\tt LWP21674}
\newline
.~~~~~~~~~\}
\newline
.~~~~~~~~~\} more images
\newline
.~~~~~~~~~\}

and finally

{\tt end}

So the text of the mail message will be

{\tt start}
\newline
{\tt cl}
\newline
{\tt tausco}
\newline
{\tt ge}
\newline
{\tt swp9788}
\newline
{\tt SWP 48580}
\newline
{\tt LWP21674}
\newline
.
\newline
.
\newline
.
\newline
{\tt end}

When the de-archiving request is processed you will be e-mailed a file giving
details of the images (target name, resolution etc.) and any problems that may
have been found, for example if the image is not yet releasable.
Also when the request has finished you will
receive an e-mail containing a summary of the copying process. 
You should go through these carefully to make sure that all your images have
been copied. If any images are not copied then send an e-mail message to
iues@star.rl.ac.uk.

\subsection{To access the files}

To ftp the files to your host machine...

{\tt ftp star-gw.rl.ac.uk}

{\tt login: anonymous}

{\tt password:} your id

{\tt ftp> cd pub/iue/}identifier\_requestname~~~in this case {\tt /cl\_tausco}

{\tt ftp> binary}

{\tt ftp> prompt}

{\tt ftp> mget *}

{\tt ftp> bye}

If there are multiple versions, for example when both apertures have been 
used, then they will appear with the suffix .1, .2 etc.
In some cases the information in the index is ambiguous and this also 
results in two images being dearchived, but one will be incorrect.

\subsection{Index errors}

If there is an error in the index then the file copied may not contain the 
image you requested, so {\it always\/} check that the header contains what you
expect. 
If you discover an index error then please contact
iues@star.rl.ac.uk.

\subsection{Output request file}
When a de-archiving request is received the procedure checks the image number
against the catalogue and outputs a summary of the target and processing
information. 
On receiving the output request file you should check that the targets
correspond to what you expect.
\tt
\begin{verbatim}
 --------
 SWP15350  RA 00 20 18.0  Dec -12 29 15  Station G  Object 1835
 SWP15350  LOW  DISPERSION  NEW  EXTRACTION TWO  APERTURES
 --------
\end{verbatim}
\rm
The procedure also checks that the image is outside the six month release
period and if not you will be told when the image will be released.
\tt
\begin{verbatim}
 --------
 LWP27383  RA 11 07 56.9  Dec -60 42 27  Station G  Object    97152            
 LWP27383 * This image will be released in  87 days
 --------
\end{verbatim}
\rm
On some occasions there is an ambiguity in the index and procedure will try to
retrieve two images, assuming that both tracking stations acquired the image.
There will be a pair of messages in the output file and two sets of files will
usually be copied, however only one of these will be correct. It is also
possible to have one set of files copied and an error message generated in the
de-archiving summary, described below, which would not be a real
error.
\tt
\begin{verbatim}
--------
SWP 6572 * Ambiguous station Goddard assumed
SWP 6572  RA 21 37 24.0  Dec +57 16 00  Station ?  Object 206267
SWP 6572  HIGH DISPERSION  ???? EXTRACTION ONE  APERTURES
--------
SWP 6572 * Ambiguous station Vilspa assumed
SWP 6572  RA 21 37 24.0  Dec +57 16 00  Station ?  Object 206267
SWP 6572  HIGH DISPERSION  ???? EXTRACTION ONE  APERTURES
\end{verbatim}
\rm
If an image is in the catalogue but for whatever reason does not appear in the
archive then the following message is generated. Such images are usually close
to the release date.
\tt
\begin{verbatim}
--------
SWP51660  RA 16 32 45.9  Dec -28 06 51  Station G  Object 00149438
SWP51660 * Image not available
\end{verbatim}
\rm
Finally if there is an error in the camera name or the image number is outside 
a permitted range then an error message will be generated.
\tt
\begin{verbatim}
--------
LWL23434 * Error in camera name
--------
SWR30989 * Error in image number
--------
SWP  876 * Error reading index
--------
LWR    0 * Image unavailable in index
--------
\end{verbatim}
\rm
\subsection{De-archiving summary}
When the images have been copied a summary file is e-mailed to the requestor.
The file simply contains a list of the files that were or were not copied.
The listing should be compared with the output request file taking note of any
duplicate files that may have been generated.
\tt
\begin{verbatim}
 error: SWP15350.PHOT was not copied
 error: SWP15350.EXT  was not copied
 error: SWP15350.EXT  was not copied
 image: SWP15341.PHOT copied
 image: SWP15341.EXT  copied
 image: SWP15342.PHOT copied
 image: SWP15342.EXT  copied
\end{verbatim}
\rm
If any images are not copied then you should report them to iues@star.rl.ac.uk.


\section{Running IUEDR
\xlabel{running_iuedr}\label{readiue}}

When reading an IUE image file check that it contains
the image you expect.
As well as the obvious the header contains much more information essential to
the correct running of IUEDR.
To inspect the header use the {\tt listiue}, {\tt readiue} or {\tt readsips} 
command with the
parameter {\tt nline=-1}, which will list the whole header of perhaps 100 lines.
With X-terminals it is a simple matter to scroll back through the listing to
find the relevant information.
At the {\tt DRIVE} prompt give the name of the file, a photometric image for 
{\tt readiue}, an extracted file for {\tt readsips} or any file type 
for {\tt listiue}.
The image number and target name, resolution, aperture and exposure time 
appear at the start of the header in line 3. In some early images this
information may be more scattered or even missing.
The THDA, which is required for temperature corrections, appears as the fourth
item on line 5, 10.2 in this case, but this helpful format has only been used
at Vilspa in recent years. 
The THDA may also be found amongst the processing information
in the header of the extracted file.
\tt
\begin{verbatim}
> listiue nlines=-1
DRIVE - Tape Drive or File Name. /'swp7471.phot'/ > swp45203.phot
-------------------------------------------------------------------------
                         0001000104212048   1 2 023045203   #101     1  C
     99*   3*21JUL-1 *   *   *  1200*      *   *  * * * * * *     *  2  C
 HD152270,SWP45203,HRES,LAP,20M0S,20:12:39                           3  C
 920721,SPREP,MAXG,LOREAD,PA145,LLOYD                                4  C
 10,5845,FO,10.2,,102D38M45.4S,289,435,2021,FO                       5  C
 FESBCK:NOT MEASURABLE                                               6  C
                                                                     7  C
\end{verbatim}
\rm

The next section of the header contains the event round robin, only part of
which is shown here. 

\tt
\begin{verbatim}
                                                                     8  C
                                                                     9  C
 92203204652* 10  * 218 *        *201424 TLM,LWPROM               * 10  C
 185730 TARGET FROM SWLA         *201459 READPREP 1 IMAGE 23560   * 11  C
 190153 S/C MANEUVERING          *201533 SCAN READLO SS 1 G3 47   * 12  C
 190450 CAMINIT                  *201550 X 53 Y 71 G1 97 HT 106   * 13  C
 190453  23559 18676 45202 1181  *203925 TLM,FES2ROM              * 14  C
 190516 TLM,LWPROM               *204010 FIN 3 T 1199 S 97 U 109  * 15  C
 191036 SPREP 1                  *204112 TARGET FROM SWLA         * 16  C
.
.
.
\end{verbatim}
\rm
Towards the end of the header is the line which gives the type of photometric
image, whether it was
photometrically or geometrically and photometrically corrected; phot or gphot
is required by IUEDR.
The gphot images are usually clearly identified but phot
images are not always, and
recourse may have to be made to the Table (\ref{change}).
\tt                                                                             
\begin{verbatim}

 92072120345203L HO 0920722930201000000000000201239002000540V        1APC
 PA145VHD152270            101650486414420- 10.205845FO3    0697     2APC
          LLOYD      000000000000                                    3APC
 ***** PHOTOMETRICALLY CORRECTED IMAGE *****                            C
 ***** SCHEME NAME:  F3HLAC, VILSPA RELEASE: R14 *****                  C
\end{verbatim}
\rm
Immediately following is the ITF table which identifies the ITF used in the
processing.
Use the table in the IUEDR Reference Manual under
ITFMAX to match the first line of the ITF table
\tt                                                                             
\begin{verbatim}
           0       1684       3374       6873       9091      10586   1PC
\end{verbatim}
\rm
to the value of ITFMAX.
\tt                                                                             
\begin{verbatim}
 PCF C/** DATA REC. 11 1   1   1 768 8448 5 3  6.1  5.0 2536   .00000 1PC
           0       1684       3374       6873       9091      10586   1PC
       14371      17745      21524      25105      28500              1PC
      11.000     11.000     11.000     11.000     11.000     11.000   1PC
      11.000     11.000     11.000     11.000     11.000              1PC
 TUBE   3 SEC EHT  6.1 ITT EHT  5.0 WAVELENGTH 2536 DIFFUSER 0        1PC
      C     MODE : FACTOR   .178E 00                                  1PC
 *PHOTOM   08:38Z JUL 22,'92                                           HC
 *ARCHIVE   08:38Z JUL 22,'92                                          HL
 Data part contains 768 lines of size 1536 bytes.
 This is a photometric image (IUE_GPHOT or IUE_PHOT).
>
\end{verbatim}
\rm
The header for the extracted file contains additional information about the 
processing and the spacecraft velocity data.
Use either of the commands {\tt listiue} or {\tt readsips} to see this header.
The THDA which is used by IUEDR for
temperature dependent corrections may be found here.
If the values are different the one for RESEAU MOTION is used.
\tt
\begin{verbatim}

 THDA FOR RESEAU MOTION = 10.18                                         C
 THDA FOR SPECTRUM MOTION = 10.18                                       C
 THERMAL SHIFTS:   LINE =   .193   SAMPLE =  -.338                      C
 REGISTRATION SHIFTS:   LINE = -1.597  SAMPLE =  2.036   AUTO           C
 *SORTHI   08:38Z JUL 22,'92                                            C
 RIPPLE CONSTANTS: K= 137730.0 -3.064*M +.0335*M**2      A= .856        C
 OBSERVATION DATE(GMT): YR=92 DAY=203 HR=20 MIN=30, (JD): 2448825.3542  C
 TARGET COORD, (1950):  RT. ASC.=16 50 48.6  DECL.=-41 44 20            C
 IUE VELOCITY (KM/S):     VX=   .5  VY= -2.7  VZ=  1.5                  C
 EARTH VELOCITY (KM/S):   VX= 25.5  VY= 13.3  VZ=  5.8                  C
 NET VELOCITY CORRECTION TO HELIOCENTRIC COORD.=-18.1                   C
 *ARCHIVE   08:38Z JUL 22,'92                                          HL
 Data part contains 421 lines of size 2048 bytes.
 This is a High Resolution Spectrum (IUE_HIRES) from IUESIPS#2.
>
\end{verbatim}
\rm

\section{Permanent features
\xlabel{permanent_features}\label{blemish}}

Tables \ref{low} and \ref{high} 
give a list of permanent features in IUE images. Some of
these fall on the spectra or background. 
Most are weak and appear only on long exposures.

\begin{table}[ht]
\caption{LORES hot spots}
\label{low}
\begin{center}
\begin{tabular}{cccccl} 
Camera	& \AA\	& LAP 	& SAP 	& Background \\
LWP	& None  &  \\ \hline
LWR  	& 1775	&	&$\bullet$&$\bullet$\\
	& 1780	&$\bullet$ &	&	\\
        & 2130	&	&$\bullet$&	\\
        & 2190	&$\bullet$&	&	\\
        & 2256	&$\bullet$&	&	\\
        & 3087	&$\bullet$&	&	\\ \hline
SWP     & 1279	&$\bullet$&	&	\\
        & 1288	&$\bullet$&	&	\\
        & 1491	&$\bullet$&	&	\\
        & 1535	&$\bullet$&	&	\\
        & 1663	&$\bullet$&	&	\\
        & 1750	&$\bullet$&	&	\\
        & 1795	&$\bullet$&	&$\bullet$\\
\end{tabular}
\end{center}
\end{table}

In the SWP LORES images there is also a weak spurious absorption feature
1515\AA.


\begin{table}[t]
\caption{HIRES hot spots and features}
\label{high}
\begin{center}
\begin{tabular}{ccc|ccc|ccc}
Camera	& Line	& Sample & Camera  & Line  & Sample &Camera  & Line  & Sample \\ \hline
LWP	& 101 & 525 &	   LWR & 126 & 291 &SWP     & 292 & 413  \\
	& 205 & 319 &          & 169 & 499 &        & 353 & 501  \\
$^{1}$	& 396 & 384 &		& 170 & 200 &        & 392 & 127  \\
$^{2}$	& 409 & 208 &   	& 175 & 369 &        & 398 & 521  \\
	& 426 & 435 &          & 178 & 610 &        & 410 & 535  \\
	& 455 & ~35 &          & 208 & 391 &        & 482 & 342  \\
	&     &     &          & 215 & 326 &        & 568 & 127  \\
        &     &     &          & 257 & 323 &        & 611 & 387  \\
        &     &     &          & 333 & 317  \\
        &     &     &          & 364 & ~60    \\
        &     &     &          & 412 & 385      \\
        &     &     &          & 434 & 479        \\
        &     &     &          & 518 & 545          \\
        &     &     &          & 532 & 307            \\
        &     &     &          & 680 & 332              \\

\end{tabular}
\end{center}
1. Fuzzy patch at $\sim$ 2482\AA\ in order 93

2. Hole at $\sim$ 2880\AA\ in order 80
\end{table}

\end{document}
