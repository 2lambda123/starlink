\documentstyle{article} 
\pagestyle{myheadings}

%------------------------------------------------------------------------------
\newcommand{\stardoccategory}  {Starlink User Note}
\newcommand{\stardocinitials}  {SUN}
\newcommand{\stardocnumber}    {127.1}
\newcommand{\stardocauthors}   {Julian Osborne}
\newcommand{\stardocdate}      {2 July 1991}
\newcommand{\stardoctitle}     {The EXOSAT Database System}
%------------------------------------------------------------------------------

\newcommand{\stardocname}{\stardocinitials /\stardocnumber}
\renewcommand{\_}{{\tt\char'137}}     % re-centres the underscore
\markright{\stardocname}
\setlength{\textwidth}{160mm}
\setlength{\textheight}{240mm}
\setlength{\topmargin}{-5mm}
\setlength{\oddsidemargin}{0mm}
\setlength{\evensidemargin}{0mm}
\setlength{\parindent}{0mm}
\setlength{\parskip}{\medskipamount}
\setlength{\unitlength}{1mm}

\begin{document}
\thispagestyle{empty}
SCIENCE \& ENGINEERING RESEARCH COUNCIL \hfill \stardocname\\
RUTHERFORD APPLETON LABORATORY\\
{\large\bf Starlink Project\\}
{\large\bf \stardoccategory\ \stardocnumber}
\begin{flushright}
\stardocauthors\\
University of Leicester\\
\stardocdate
\end{flushright}
\vspace{-4mm}
\rule{\textwidth}{0.5mm}
\vspace{5mm}
\begin{center}
{\Large\bf \stardoctitle}
\end{center}
\vspace{5mm}

The purpose of this note is to alert users to the existence of the EXOSAT 
database system, to point them in the direction of the manuals and to 
provide a very quick guide to accessing the system. It is not intended to be 
a complete guide to all the facilities available, the on-line 
and paper documentation fill these needs.

The EXOSAT database system is a relational database management system 
originally written by the EXOSAT observatory team and now being developed 
at the HEASARC in the Goddard Space Flight Centre. The database system has 
many associated 
astronomical databases. Some of these databases have data 
products associated with them, these may be plotted and analysed using 
software within the EXOSAT database system. 

The EXOSAT database system is installed at Leicester University to provide 
UK astronomers with easy access to the databases. It is also installed for 
public use at 
ESTEC and at the HEASARC in GSFC. The core databases are those resulting 
from an automatic analysis of the data from the European X-ray astronomy 
satellite EXOSAT. This data, together with the associated data products, is
installed at each site. Other databases may be installed worldwide or only 
locally. 

At Leicester there are currently 24 databases available. Apart from the 
EXOSAT databases, there are a number of EINSTEIN databases and many 
astronomical catalogues, {\it e.g.}~SAO, 
IRAS point source catalogue, HD, Variable Star Catalogue, etc. 
An increasing number of ROSAT-related 
databases will be made available as the mission progresses. An important 
ROSAT database currently available is {\tt ROSSTL}, 
the ROSAT short term timeline. 
This database allows users to see the exact planned times of ROSAT 
observations and is updated every week or so. Individual SCAR catalogues can 
be incorporated into the EXOSAT database system on demand, subject to disc 
space limitations.

Extensive help is available to guide users of the system. Three ESA 
technical memos (TM-11, TM-12, TM-13) have been distributed by ESA and 
Starlink. These are:
\begin{itemize}
\item The EXOSAT Database System: Browse User's Guide
\item The EXOSAT Database System: On-line User's Guide
\item The EXOSAT Database System: Available Databases
\end{itemize}
Inevitably, given the significant development effort at the HEASARC, these 
documents will become more or less dated. For this reason, substantial 
on-line help is also available within the EXOSAT database system. Users 
experiencing difficulties can e-mail me (LTVAD::JULO) for guidance.

To use the EXOSAT database system at Leicester, users must login to the  
captive account XRAY on the database machine LTXDB. No login password is 
required, but users will asked for a password if the system does not recognise 
them. 

The main database programme is called {\tt browse}. 
Enter {\tt browse} {\it dbname}, 
where {\it dbname} is the database name, to access a database, or {\tt browse ?}
to list the available databases. Within {\tt browse}, 
enter {\tt ?} to get a list 
of commands, and {\tt help} {\it command} to get help on that command. 

A trivial sample login session is illustrated below. On login the user is 
alerted to new bulletins. The command {\tt bulletin} is used to read these.
In this sesion the following 
{\tt browse} commands are used: {\tt sn} -- search by name; 
{\tt lpa} -- list parameters; {\tt cdb} -- change database; {\tt ex} -- exit.

\begin{verbatim}
$ set host ltxdb

Leicester Starlink VAXcluster  -  Node LTXDB

Username: XRAY


                    The EXOSAT Database System at Leicester



*********************************NEW BULLETINS**********************************
DESCRIPTION                                           FROM         DATE
-----------                                           ----         ----
ROSAT timeline databases                              JULO         13-JUN-1991
New Raymond & Smith code in XSPEC                     JULO         30-MAY-1991
Acknowledgements                                      JULO         18-APR-1991
********************************************************************************
Welcome julian from LTMVB::JULO

type help for help
Leicester> browse me
BROWSE 3.11:  2-JUL-91 15:35:10 BST. Address=LTMVB::JULO (XRAY)
Session initialisation, please wait
For further information type HELP, KEYWORDS or DATABASES (to end use EXIT)
Current equinox year: 1950
Plot device not defined, use cpd command
Loading ME database sample TOTAL indexed on DEC
ME_TOTAL_DEC > sn aeaqr

         name         time  seq QF   exp  count rate error   ra        dec
                    (yy.day) (#)    (sec)                  (1950)     (1950)

>  1 AEAQR           84.191  898 2   5070   2.37 +/- 0.10  16 52 11  +39 50.0
   2 AEAQR           84.190  897 2  14976  -0.63 +/- 0.13  20 37 34  -01  3.0
ME_TOTAL_NAM 2> lpa
 #1   - NAME                               NAME (16 ASCII CHARS)
 #2   - ID NUMBER                          SOURCE ORDER NO.
 #3   - CLASS                              CLASSIFICATION FLAG
 #4   - PROPOSAL TYPE                      OBSERVATION TYPE
 #5   - POINTRA                            POINTING RA     (DEGREES)
 #6   - POINTDEC                           POINTING DEC    (DEGREES)
 #7   - ROLL ANGLE                         ROLL ANGLE      (DEG *10)
 #8   - BETA ANGLE                         BETA ANGLE      (DEGREES)
 #9   - LII                                LII OF SOURCE   (DEGREES)
 #10  - BII                                BII OF SOURCE   (DEGREES)
 #11  - RA                                 SOURCE RA       (DEGREES)
 #12  - DEC                                SOURCE DEC      (DEGREES)
 #13  - TIME                               SHF KEY OF START
 #14  - EXPOSURE TIME                      DURATION    (SECONDS)
 #15  - LAST UPDATE                        SHF KEY OF CREATION TIME
 #16  - PROCESS DATE                       SHF KEY OF MACRO RUN
 #17  - COUNT RATE                         CTS/SEC/4D     (CH:6:34)
 #18  - COUNT RATE ERROR                   ERR/SEC/4D
 <CR> continues, to exit type any character <CR> or Ctrlz 
ME_TOTAL_NAM 2> cdb ?
The following SYSTEM databases are available:

    MSS         Einstein MSS catalog
    HRI         Einstein HRI database
    IPC         Einstein IPC catalog
    CMA         Exosat CMA Database
    GS          Exosat GS database
    LE          Exosat LE Database
    ME          Exosat ME database
    PR          Exosat PR database
    TGS         Exosat TGS database
    EXO_PUBS    Exosat bibliography
    EXO_LOG     Exosat observation log
    HD          HD catalog
    IR          IRAS catalog
    PULSAR      Lyne Pulsar Catalogue
    ROS1        ROSAT AO-1 & PV sched obs
    ROSSTL      ROSAT short term timeline
    RITTER      Ritter X-ray binary catalogue
    SAO         SAO catalog
 <CR> continues, to exit type any character <CR> or Ctrlz 

Database name: sao
Loading SAO database sample TOTAL indexed on DEC
SAO_TOTAL_DEC > ex
Leicester> lo
  XRAY         logged out at  2-JUL-1991 15:40:16.80

\end{verbatim}

\end{document}

