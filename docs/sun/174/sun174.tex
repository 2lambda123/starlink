\documentstyle[11pt]{article}
\pagestyle{myheadings}

% -----------------------------------------------------------------------------
% ? Document identification

% An Astronomer's Guide to On-line Bibliographic Databases and 
% Information Services

% Copyright 1993  A.C. Davenhall

% I reserve my legal rights as the author of this document.  However, I
% grant permission for it to be freely copied, modified and distributed
% provided only that it is used for non-commercial, non-profit making,
% bona fide academic and scholarly purposes.

% I believe the information in this document to be correct at the time 
% of writing (though much of it is necessarily ephemeral) and it is 
% offered in good faith.  However, no guarantee or warranty 
% whatsoever is offered or implied.  Neither myself, the University of 
% Leicester nor the Science and Engineering Research Council accept any 
% responsibility for loss, damage or injury resulting the use of this 
% document.

% A.C. Davenhall,
% 25th November 1993.


%-----------------------------------------------------------------------
\newcommand{\stardoccategory}  {Starlink User Note}
\newcommand{\stardocinitials}  {SUN}
\newcommand{\stardocsource}    {sun\stardocnumber}
\newcommand{\stardocnumber}    {174.1}
\newcommand{\stardocauthors}   {A C Davenhall \\
{\it Department of Physics and Astronomy, University of Leicester} }
\newcommand{\stardocdate}      {29 November 1993}
\newcommand{\stardoctitle}     {An Astronomer's Guide to\\[1ex]
On-line Bibliographic Databases and\\[1ex]
Information Services}
\newcommand{\stardocabstract}  {
This document is a guide to on-line bibliographic databases and other 
information services which might be useful to astronomers. It 
concentrates on facilities which are either provided within the 
astronomical community, or which are available free of charge. For
each facility listed a brief description is given, together with details
of how to access it and how to obtain further information about it.
The document aims to provide sufficient information for you to decide 
which facilities will be useful to you, and then obtain further 
information on them, rather than providing comprehensive documentation 
for each facility. It should be possible to access most of the 
facilities described here from Starlink computers. This document is 
complementary to SUN/162 which describes some of the observation and 
data archives available on-line.
}
% ? End of document identification

% -----------------------------------------------------------------------------

\newcommand{\stardocname}{\stardocinitials /\stardocnumber}
\markright{\stardocname}
\setlength{\textwidth}{160mm}
\setlength{\textheight}{230mm}
\setlength{\topmargin}{-2mm}
\setlength{\oddsidemargin}{0mm}
\setlength{\evensidemargin}{0mm}
\setlength{\parindent}{0mm}
\setlength{\parskip}{\medskipamount}
\setlength{\unitlength}{1mm}

% -----------------------------------------------------------------------------
%  Hypertext definitions.
%  ======================
%  These are used by the LaTeX2HTML translator in conjunction with star2html.

%  Comment.sty: version 2.0, 19 June 1992
%  Selectively in/exclude pieces of text.
%
%  Author
%    Victor Eijkhout                                      <eijkhout@cs.utk.edu>
%    Department of Computer Science
%    University Tennessee at Knoxville
%    104 Ayres Hall
%    Knoxville, TN 37996
%    USA

%  Do not remove the %\begin{rawtex} and %\end{rawtex} lines (used by 
%  star2html to signify raw TeX that latex2html cannot process).
%\begin{rawtex}
\makeatletter
\def\makeinnocent#1{\catcode`#1=12 }
\def\csarg#1#2{\expandafter#1\csname#2\endcsname}

\def\ThrowAwayComment#1{\begingroup
    \def\CurrentComment{#1}%
    \let\do\makeinnocent \dospecials
    \makeinnocent\^^L% and whatever other special cases
    \endlinechar`\^^M \catcode`\^^M=12 \xComment}
{\catcode`\^^M=12 \endlinechar=-1 %
 \gdef\xComment#1^^M{\def\test{#1}
      \csarg\ifx{PlainEnd\CurrentComment Test}\test
          \let\html@next\endgroup
      \else \csarg\ifx{LaLaEnd\CurrentComment Test}\test
            \edef\html@next{\endgroup\noexpand\end{\CurrentComment}}
      \else \let\html@next\xComment
      \fi \fi \html@next}
}
\makeatother

\def\includecomment
 #1{\expandafter\def\csname#1\endcsname{}%
    \expandafter\def\csname end#1\endcsname{}}
\def\excludecomment
 #1{\expandafter\def\csname#1\endcsname{\ThrowAwayComment{#1}}%
    {\escapechar=-1\relax
     \csarg\xdef{PlainEnd#1Test}{\string\\end#1}%
     \csarg\xdef{LaLaEnd#1Test}{\string\\end\string\{#1\string\}}%
    }}

%  Define environments that ignore their contents.
\excludecomment{comment}
\excludecomment{rawhtml}
\excludecomment{htmlonly}
%\end{rawtex}

%  Hypertext commands etc. This is a condensed version of the html.sty
%  file supplied with LaTeX2HTML by: Nikos Drakos <nikos@cbl.leeds.ac.uk> &
%  Jelle van Zeijl <jvzeijl@isou17.estec.esa.nl>. The LaTeX2HTML documentation
%  should be consulted about all commands (and the environments defined above)
%  except \xref and \xlabel which are Starlink specific.

\newcommand{\htmladdnormallinkfoot}[2]{#1\footnote{#2}}
\newcommand{\htmladdnormallink}[2]{#1}
\newcommand{\htmladdimg}[1]{}
\newenvironment{latexonly}{}{}
\newcommand{\hyperref}[4]{#2\ref{#4}#3}
\newcommand{\htmlref}[2]{#1}
\newcommand{\htmlimage}[1]{}
\newcommand{\htmladdtonavigation}[1]{}

%  Starlink cross-references and labels.
\newcommand{\xref}[3]{#1}
\newcommand{\xlabel}[1]{}

%  LaTeX2HTML symbol.
\newcommand{\latextohtml}{{\bf LaTeX}{2}{\tt{HTML}}}

%  Define command to re-centre underscore for Latex and leave as normal
%  for HTML (severe problems with \_ in tabbing environments and \_\_
%  generally otherwise).
\newcommand{\latex}[1]{#1}
\newcommand{\setunderscore}{\renewcommand{\_}{{\tt\symbol{95}}}}
\latex{\setunderscore}

%  Redefine the \tableofcontents command. This procrastination is necessary 
%  to stop the automatic creation of a second table of contents page
%  by latex2html.
\newcommand{\latexonlytoc}[0]{\tableofcontents}

% -----------------------------------------------------------------------------
%  Debugging.
%  =========
%  Remove % on the following to debug links in the HTML version using Latex.

% \newcommand{\hotlink}[2]{\fbox{\begin{tabular}[t]{@{}c@{}}#1\\\hline{\footnotesize #2}\end{tabular}}}
% \renewcommand{\htmladdnormallinkfoot}[2]{\hotlink{#1}{#2}}
% \renewcommand{\htmladdnormallink}[2]{\hotlink{#1}{#2}}
% \renewcommand{\hyperref}[4]{\hotlink{#1}{\S\ref{#4}}}
% \renewcommand{\htmlref}[2]{\hotlink{#1}{\S\ref{#2}}}
% \renewcommand{\xref}[3]{\hotlink{#1}{#2 -- #3}}
% -----------------------------------------------------------------------------
% ? Document specific \newcommand or \newenvironment commands.
% ? End of document specific commands
% -----------------------------------------------------------------------------
%  Title Page.
%  ===========
\renewcommand{\thepage}{\roman{page}}
\begin{document}
\thispagestyle{empty}

%  Latex document header.
%  ======================
\begin{latexonly}
   CCLRC / {\sc Rutherford Appleton Laboratory} \hfill {\bf \stardocname}\\
   {\large Particle Physics \& Astronomy Research Council}\\
   {\large Starlink Project\\}
   {\large \stardoccategory\ \stardocnumber}
   \begin{flushright}
   \stardocauthors\\
   \stardocdate
   \end{flushright}
   \vspace{-4mm}
   \rule{\textwidth}{0.5mm}
   \vspace{5mm}
   \begin{center}
   {\Huge\bf  \stardoctitle}
   \end{center}
   \vspace{5mm}

% ? Heading for abstract if used.
   \vspace{10mm}
   \begin{center}
      {\Large\bf Abstract}
   \end{center}
% ? End of heading for abstract.
\end{latexonly}

%  HTML documentation header.
%  ==========================
\begin{htmlonly}
   \xlabel{}
   \begin{rawhtml} <H1> \end{rawhtml}
      \stardoctitle
   \begin{rawhtml} </H1> \end{rawhtml}

% ? Add picture here if required.
% ? End of picture

   \begin{rawhtml} <P> <I> \end{rawhtml}
   \stardoccategory \stardocnumber \\
   \stardocauthors \\
   \stardocdate
   \begin{rawhtml} </I> </P> <H3> \end{rawhtml}
      \htmladdnormallink{CCLRC}{http://www.cclrc.ac.uk} /
      \htmladdnormallink{Rutherford Appleton Laboratory}
                        {http://www.cclrc.ac.uk/ral} \\
      \htmladdnormallink{Particle Physics \& Astronomy Research Council}
                        {http://www.pparc.ac.uk} \\
   \begin{rawhtml} </H3> <H2> \end{rawhtml}
      \htmladdnormallink{Starlink Project}{http://star-www.rl.ac.uk/}
   \begin{rawhtml} </H2> \end{rawhtml}
   \htmladdnormallink{\htmladdimg{source.gif} Retrieve hardcopy}
      {http://star-www.rl.ac.uk/cgi-bin/hcserver?\stardocsource}\\

%  HTML document table of contents. 
%  ================================
%  Add table of contents header and a navigation button to return to this 
%  point in the document (this should always go before the abstract \section). 
  \label{stardoccontents}
  \begin{rawhtml} 
    <HR>
    <H2>Contents</H2>
  \end{rawhtml}
  \renewcommand{\latexonlytoc}[0]{}
  \htmladdtonavigation{\htmlref{\htmladdimg{contents_motif.gif}}
        {stardoccontents}}

% ? New section for abstract if used.
  \section{\xlabel{abstract}Abstract}
% ? End of new section for abstract
\end{htmlonly}

% -----------------------------------------------------------------------------
% ? Document Abstract. (if used)
%  ==================
\stardocabstract
% ? End of document abstract
% -----------------------------------------------------------------------------
% ? Latex document Table of Contents (if used).
%  ===========================================
 \newpage
 \begin{latexonly}
   \setlength{\parskip}{0mm}
   \latexonlytoc
   \setlength{\parskip}{\medskipamount}
   \markright{\stardocname}
 \end{latexonly}
% ? End of Latex document table of contents
% -----------------------------------------------------------------------------
\newpage
~
\newpage
\renewcommand{\thepage}{\arabic{page}}
\setcounter{page}{1}

\section{Introduction\xlabel{introduction}}

This document is a guide to on-line bibliographic databases and other 
information services which might be useful to astronomers. It 
concentrates on facilities which are either provided within the 
astronomical community, or which are available free of charge. For
each facility listed a brief description is given, together with details
of how to access it and how to obtain further information about it.
The document aims to provide sufficient information for you to decide 
which facilities will be useful to you, and then obtain further 
information on them, rather than providing comprehensive documentation 
for each facility. The document is aimed at Starlink users in the 
United Kingdom. However, where practicable, I have tried to provide 
sufficient information to make it useful to astronomers overseas. 

The structure of the document is as follows:

\begin{itemize}

  \item Part I -- bibliographic databases,

  \item Part II -- other information services,

  \item Part III -- INTERNET tools which can be used to access the
   facilities listed in Parts I and II.

\end{itemize}

It should be possible to access most of the facilities described here 
from Starlink computers. This document is complementary to
\xref{SUN/162}{sun162}{} which
describes some of the observation and data archives available on-line.
There is some overlap between the two documents because some of the
systems described in
\xref{SUN/162}{sun162}{} include bibliographic information.

The recent compilation {\it Intelligent Information Retrieval: The
Case of Astronomy and Related Space Sciences}\, edited by A. Heck and
F. Murtagh (see Section~\ref{FURTHER}) covers very similar topics to 
this document. However it is more concerned with the rationale behind
the developments, rather than being a cookbook of what is available
and how to access it.  You are referred to it for a more comprehensive
discussion of many of the developments reported here.

Many of the facilities described in this document, particularly those 
overseas, but also increasingly ones in the United Kingdom, are accessed
through the INTERNET communications network (which is, more accurately, 
an interlinked set of communications networks, all running a common 
communications protocol). A wide variety of facilities are available 
through INTERNET, and some powerful and sophisticated software tools 
are becoming available to access them. An excellent and thoroughly 
recommended introduction to the INTERNET, and the facilities and 
services available in it, is {\it The Whole Internet User's Guide and 
Catalog}\, by Ed Krol (see Section~\ref{FURTHER}). In particular, this 
book gives further details of many of the INTERNET tools described
Part~\ref{INTERNET_TOOLS}.

Because existing facilities evolve and new ones become available I 
anticipate that the document will be revised periodically. In order that
it should remain useful and accurate I welcome corrections, comments and
suggestions for additional facilities to include. If you have any such 
comments please send them to: Clive Davenhall, Department of Physics and
Astronomy, University of Leicester, University Road, Leicester, LE1 7RH.
E-mail: LTVAD::ACD (Starlink DECNET), 19838::ACD (SPAN) or 
acd@star.le.ac.uk (JANET and INTERNET).


\pagebreak
\part{Bibliographic Databases}

\section{Overview\xlabel{overview}} 

Traditionally locating articles, papers or books on some specific 
astronomical topic has involved searching through the annual volumes
of {\it Astronomy and Astrophysics Abstracts}\, published by the 
Astronomisches Rechen-Institut of Heidelberg, or for older material, its
predecessor the {\it Astronomischer Jahresbericht}. Bibliographic 
databases containing similar sorts of information have existed as 
commercial products for well over a decade. Typically such databases 
contain information covering a wide range of subjects, rather than 
being specifically astronomical. The INSPEC database described 
briefly below is typical of such large, commercial bibliographic
databases. The use of these systems in astronomy has been reviewed
by J.M. Rey-Watson (1988, in {\it Astronomy from Large Databases}\, eds.
F. Murtagh and A. Heck, Garching: ESO conference and workshop 
proceedings no. 28, pp453-458) and J.M. Watson (1991, in {\it 
Databases and On-Line Data in Astronomy}, eds. D. Egret and M. 
Albrecht, Dordrecht: Kluwer, pp199-210).

The INSPEC database produced by the United Kingdom Institution of
Electrical Engineers (IEE) is  a typical large, commercial,
multi-subject database. The subject areas which it covers are: physics
(including astrophysics), electrical engineering, electronics, 
computing, control and information technology. It is the principal
English-language bibliographic database in these fields.

INSPEC was established in 1969 and coverage is continuous from that
date (though the IEE has published abstracts journals on paper for
somewhat longer, starting in 1898). INSPEC contains entries for articles
appearing in journals and conference proceedings and also for books,
reports and dissertations (though books, reports and dissertations 
amount for only a little over 2\% of the entries in the database). For
every article entered full bibliographic details and an abstract are
given. This abstract is prepared specifically for the INSPEC database
and is not necessarily the same as any abstract in the original article.
For articles published in languages other than English both the original
title and an English translation are provided; the abstract is prepared
in English. Entries are selected for timeliness and importance prior to 
being added to the database, so complete coverage of any journal
cannot be guaranteed (though in practice coverage is complete for 
eminent journals in the core subjects of the database). 

In October 1993 INSPEC contained about 4,500,000 entries, with new
entries being added at a rate of about 250,000 entries {\it per annum}.
These entries are selected from articles in about 4300 journals and
nearly 2000 conference proceedings {\it per annum}. All the eminent
astronomical journals are covered.

INSPEC may be accessed in a number of ways. It can be supplied on 
magnetic tape for users to process with their own software. A CD-ROM
version is also available, which comes with search software for IBM PC
compatible personal computers. Finally, it can be accessed on-line from
a number of third-party vendors. Typically such vendors will make a
number of bibliographic databases available, having homogenized their
appearance and customized their contents to a greater or lesser extent.
This method is the usual way of accessing the databases and there are
many such database vendors.

In principle, automatic searches of electronic bibliographic databases 
are less tedious, time consuming and error prone than manual searches of
a printed book. However, usually these database systems have been 
difficult and expensive to use, and in practice only trained librarians 
have accessed them. Improvements in computer hardware and software are
making access to these databases easier. BIDS (see Section~\ref{BIDS},
below) is the first national on-line bibliographic information system
designed for routine use by non-expert users.

In addition to the large multi-discipline bibliographic databases, a
number of specifically astronomical bibliographic databases have been
developed within the astronomical community. These databases are 
typically limited in scope, perhaps only covering some aspect of 
astronomy, but are usually available free of charge. Some of the systems
available are described below.

There is some overlap between the bibliographic databases described here
and the observation and data archives described in SUN/162. In 
particular, SIMBAD, NED, ADS, and ESIS, which all contain bibliographic
information, are described in SUN/162. The bibliographic aspects of the
ADS and ESIS were not covered in SUN/162, so they are summarized in
Section~\ref{ADS}, below. Conversely, the bibliographic aspects of NED
and SIMBAD were covered adequately in SUN/162, so the description is not
repeated here. However, to recap briefly, SIMBAD contains bibliographic 
data, without abstracts, for individual objects and NED contains 
bibliographic data, with abstracts, for individual extra-galactic 
objects. A charge is made for the use of SIMBAD.

None of these systems are direct replacements for {\it Astronomy and 
Astrophysics Abstracts}. Rather, they complement the paper {\it
Abstracts}\, by permitting large and complex searches which were not
practical hitherto.


\section{BIDS
\xlabel{bids}\label{BIDS}}

BIDS (Bath Information and Data Services) is run by the Computing 
Service of the University of Bath under the \ae gis of 
CHEST\footnote{Combined Higher Education Software Team; an organization
which negotiates licence arrangements for software on behalf of United
Kingdom Higher Educational Institutions.} and provides on-line access 
to a number of bibliographic databases. BIDS is the first national 
on-line bibliographic information system designed for routine use by 
non-expert users. Currently, BIDS provides access to three databases:

\begin{itemize}

  \item ISI,

  \item Inside Information,

  \item Embase.

\end{itemize}

Embase is a bio-medical database that is unlikely to be of interest to
astronomers. The remaining two databases are discussed further below.
In order to use BIDS services you must be associated with a subscribing 
institution. Institutions subscribe to the various BIDS services
individually; a given institution will not necessarily subscribe to them
all. BIDS services are not available outside the United Kingdom. The 
BIDS data are copyright and their use is subject to agreements between 
CHEST and the vendors of the databases. These agreements usually 
permit use for `teaching, research and personal educational 
development', but exclude consultancy and services where the data are 
used commercially. In case of doubt you should consult the text of the 
agreement for the database which you wish to use, which is accessible 
on-line through the BIDS service in question.

In order to gain access to BIDS services you should, in the first 
instance, contact your local University (or Institutional) library, who 
should be able to advise. If your institution subscribes to the 
appropriate service, you will be required to sign a form agreeing to 
the conditions for using it, and you will be given a username, password 
and a user guide. BIDS services are accessed via JANET or INTERNET. 
Local arrangements for logging on to them will vary, but will probably 
be done from a computer belonging to your university's computer centre, 
rather than a Starlink machine.

\subsection{BIDS ISI
\xlabel{bids_isi}\label{BIDS_ISI}}

BIDS ISI provides access to three multi-disciplinary citation indices:
the Science Citation Index (SCI), the Social Science Citation Index 
(SSCI) and the Arts and Humanities Citation Index (A\&HCI). These 
citation indices are supplied and owned by the Institute for Scientific 
Information Inc.~(ISI) in the United States. They contain details of 
articles drawn from over 7000 journals and the Index to Scientific and 
Technical Proceedings (ISTP) which contains details of the papers 
presented at over 400 conferences {\it per annum}. For each article,
bibliographic details, but not abstracts, are available. Keywords for
each article have been included since 1990. The period covered is from 
1981 to the present. The database is updated once per week. Articles 
on astronomical topics are usually within the scope of the SCI, though 
the A\&HCI contains some articles about the history of astronomy. 
Table~\ref{BIDS_JNLS} shows the astronomical journals which are covered 
by the BIDS ISI SCI. It excludes any physics or general journals which 
carry astronomical papers except {\it Nature}.

\begin{table}[htbp]

\begin{center}
\begin{tabular}{l}
Acta Astronomica  \\
Advances in Geophysics  \\
Advances in Space Research  \\
Annual Review of Astronomy and Astrophysics  \\
Annual Review of Earth and Planetary Sciences  \\
Astronomy and Astrophysics Review  \\
Astrophysics and Space Science  \\
Astronomy and Astrophysics  \\
Astronomy and Astrophysics Supplement Series  \\
Astronomische Nachrichten  \\
Astronomical Journal  \\
Astronomicheskii Zhurnal  \\
Astrophysical Journal  \\
Astrophysical Journal Supplement Series  \\
Astrophysical Letters and Communications  \\
Bulletin of the Astronomical Institutes of Czechoslovakia  \\
Classical and Quantum Gravity  \\
Earth and Planetary Science Letters  \\
Earth-science Reviews  \\
ESA Bulletin-European Space Agency  \\
ESA Journal-European Space Agency  \\
General Relativity and Gravitation  \\
Geophysical and Astrophysical Fluid Dynamics  \\
Indian Journal of Radio and Space Physics  \\
IAU Symposia  \\
Icarus  \\
Journal of Astrophysics and Astronomy  \\
Journal of Geophysical Research-Planets  \\
Journal of Geophysical Research-Solid Earth and Planets  \\
Journal of Geophysical Research-Space Physics  \\
Journal of Geophysical Research-Solid Earth  \\
Journal of Physics of the Earth  \\
Journal of the Royal Astronomical Society of Canada \\
Monthly Notices of the Royal Astronomical Society  \\
Meteoritics  \\
Nature  \\
Observatory  \\
Proceedings of the Astronomical Society of Australia  \\
Proceedings of the Indian Academy of Sciences-Earth and Planetary Sciences  \\
Proceedings of the Lunar and Planetary Science Conference  \\
Physics of the Earth and Planetary Interiors  \\
Publications of the Astronomical Society of Japan  \\
Publications of the Astronomical Society of the Pacific  \\
Quarterly Journal of the Royal Astronomical Society  \\
Revista Mexicana de Astronomia y Astrofisica  \\
Soviet Astronomy Letters \\
Space Science Reviews  \\
Transactions of the Japan Society for Aeronautical and Space Sciences  \\
\end{tabular}
\end{center}

\caption{Astronomical journals covered by the BIDS ISI database}
\label{BIDS_JNLS}

\end{table}

Virtually all the `old' (pre-1990) universities have subscribed to BIDS
ISI, though subscription amongst the former polytechnics is less 
complete. It can be accessed from a dumb terminal. Once you have logged 
on to it, a simple menu system allows you to
perform searches to identify articles of interest, examine their 
details and obtain copies of these details, via e-mail, on your host 
computer. Such details can optionally be directed straight to a 
Starlink computer, and do not necessarily have to go to the university 
computer centre machine that you are using to access BIDS. An important
feature of BIDS ISI is the ability to find all the citations of a given 
article. A table of contents can be displayed for recent issues of a
selected journal. Also you can request a paper copy of selected 
articles (a charge is made for this service). All these facilities are 
described in the short, but entirely adequate, user guide.

The advantages of using BIDS ISI include: the simple, menu-driven user 
interface, which is easy to learn and use, the large number of journals 
and conference proceedings covered and the ability to find all the 
citations of a given paper. The principal disadvantage is probably the 
lack of abstracts. Additionally, the user interface, though easy to use,
is of limited functionality. In particular, the details of selected 
articles must be individually inspected and retrieved. Searches can only
operate on a single year, so they must be repeated if a range of years 
is to be covered. The search cannot be limited to astronomical 
journals, which is both an advantage and a disadvantage; astronomical 
papers in non-astronomical journals will be found automatically, but 
spurious references may also be found. Finally, searches must be for 
words, not numbers. This restriction precludes searches for many 
astronomical object designations.

\subsection{BIDS Inside Information
\xlabel{bids_inside_information}\label{BIDS_II}}

BIDS Inside Information provides details of articles appearing in the
10,000 most requested periodicals at the British Library Document Supply
Centre, Boston Spa. It is a new database and the period covered is only 
since October 1992. It is however, very up to date, with the details of
new journals being input within a few days of their receipt. It is 
anticipated that details for about one million articles will be added
{\it per annum}. For each article full bibliographic details, but not
abstracts, are included. Most eminent astronomical journals are
covered.

BIDS Inside Information is only available at some universities; fewer
institutions have subscribed to it than have subscribed to BIDS ISI.
It can be accessed from a dumb terminal. Once you have logged on to it, 
a simple menu system allows you to perform searches to 
identify articles of interest and to examine their details. This system
is very similar to the BIDS ISI menu system (see Section~\ref{BIDS_ISI},
above). It is also possible to display a contents list for a selected 
issue of a given journal. The details of selected articles can be 
retrieved, via e-mail, to your host computer and it is also possible to 
request a paper copy of selected articles (a charge is made for this
service).


\section{ADS and ESIS
\xlabel{ads_and_esis}\label{ADS}}

ADS (Astrophysics Data System) is a distributed information 
system run by NASA. It includes a bibliographic database. This database 
is a subset extracted from the NASA RECON system, which is supplied by 
the NASA Scientific and Technical Information (STI) programme. It is 
compiled from articles appearing in over 200 journals, conference 
proceedings and internal NASA reports. The subject areas of RECON
which are included in the ADS database are listed in 
Table~\ref{ADS_SUBJ}. All the eminent astronomical journals are 
covered. For each article the service provides full bibliographic 
details, including an abstract. This abstract was prepared specifically 
for the RECON system and is not necessarily the same as any abstract 
in the original publication. The period covered is from 1975 to the 
present. Complete coverage of any of the sources is not guaranteed.

\begin{table}[htbp]

\begin{center}
\begin{tabular}{lc}
Subject                               & NASA classification  \\ \hline
{\bf Astronomy}                       & {\bf 89}  \\
Solar Astronomy                       & 89-01     \\
Stellar Astronomy and Cosmology       & 89-02     \\
Meteors and Meteorites                & 89-03     \\
{\bf Astrophysics}                    & {\bf 90}  \\ 
Gravitation                           & 90-01     \\
Astrophysical Plasmas                 & 90-02     \\
{\bf Lunar and Planetary Exploration} & {\bf 91}  \\ 
The Moon                              & 91-01     \\
Planetary Sciences and Exploration    & 91-02     \\
{\bf Solar Physics}                   & {\bf 92}  \\ 
{\bf Space Radiation}                 & {\bf 93}  \\
Cosmic Radiation                      & 93-01     \\
Solar Radiation and Activity          & 93-02     \\
Radiation Belts                       & 93-03     \\
\end{tabular}
\end{center}

\caption{RECON subject areas included in the ADS database}
\label{ADS_SUBJ}

\end{table}

Currently (November 1993) the ADS bibliographic database is only 
available to users outside the United States on an experimental basis.
It is possible that in the future the service may be restricted to users
within the United States.

ESIS (European Space Information System) is a distributed information 
system run by ESA. It also includes access to a bibliographic database. 
This database is also extracted from the NASA RECON system and is very 
similar to the one available through the ADS.

See SUN/162 for details of accessing the ADS and ESIS. You should 
beware, however, that the current speed of data communication lines in 
Europe makes ESIS un-usable at most institutions. The imminent release 
of a client-server version of ESIS may improve this situation.

{\it Reference:}

The ADS abstract service is described by M.J. Kurtz, T. Karakashian, 
C.S. Grant, G. Eichhorn, S.S. Murray, J.M. Watson, P.G. Ossorio and 
J.L. Stoner, 1993, in {\it Astronomical Data Analysis Software and
Systems II}, eds. R.J. Hanisch, R.J.V. Brissenden and J. Barnes,
Astronomical Society of the Pacific Conference Series, {\bf 52}, 
pp132-136.


\section{STELAR
\xlabel{stelar}\label{STELAR}}

STELAR (STudy of Electronic Literature for Astronomical Research) is a 
project to investigate computerized access to astronomical literature. 
It is managed by the Astrophysics Data Facility (ADF) at the NASA 
Goddard Space Flight Center. Its formal goal is to `explore the use of
electronic means for improving access to scientific literature; using
astronomical publications to evaluate distribution, search and retrieval
techniques for full text and graphics display.' There are several 
aspects to the STELAR project, and much of the work is still 
experimental, rather than running an operational system. However, as a 
pilot project, a database of bibliographic details, including abstracts,
for articles appearing in eight eminent astronomical journals (see 
Table~\ref{STELAR_JNL}) is publicly available. These abstracts have been
supplied by NASA/STI from a database prepared for the NASA RECON system
by an independent abstraction service. They were prepared specifically 
for the RECON system and are not necessarily the same as any abstract 
in the original publication. The database contains some articles dating 
back to the 1960s, though the coverage is pretty complete since the 
mid-1970s. The database is updated approximately fortnightly.

\begin{table}[htbp]

\begin{center}
\begin{tabular}{l}
Astrophysical Journal  \\
Astrophysical Journal Supplement  \\
Astronomical Journal  \\
Publications of the Astronomical Society of the Pacific  \\
Astronomy and Astrophysics  \\
Astronomy and Astrophysics Supplement \\
Monthly Notices of the Royal Astronomical Society  \\
Journal of Geophysical Research \\
\end{tabular}
\end{center}

\caption{Journals covered by the STELAR database}
\label{STELAR_JNL}

\end{table}

The STELAR database can be queried with WAIS (see Section~\ref{WAIS}), 
though first you should obtain a copy of the source file {\tt 
abstracts.src}. A copy of this file can be obtained from the European 
Space Telescope Coordinating Facility (again see Section~\ref{WAIS},
especially Table~\ref{WAIS_SOURCE}).

Further information on the STELAR project can be obtained by sending
an e-mail message to: stelar-info@hypatia.gsfc.nasa.gov (INTERNET).

{\it References:}

M.E. Van Steenberg, 1992, in {\it DeskTop Publishing in Astronomy and 
Space Sciences}, ed. A. Heck (World Scientific, Singapore), pp149-154.

A. Warnock, J.E. Gass, L.E. Brotzman, M.E. Van Steenberg, D. Kovalsky 
and F. Giovane, October 1992, {\it Amer. Astron. Soc. Newsletter} {\bf 
62}, insert p10.

M.E. Van Steenberg, J.E. Gass, L.E. Brotzman, A. Warnock, D. Kovalsky 
and F. Giovane, October 1992, {\it Amer. Astron. Soc. Newsletter} {\bf 
62}, insert p11.

M.E. Van Steenberg, Fall 1992, {\it NSSDC News}, {\bf 8}, number 3, p14.

A. Warnock, M.E. Van Steenberg, L.E. Brotzman, J.E. Gass, D. Kovalsky 
and F. Giovane, 1993, in {\it Astronomical Data Analysis Software and
Systems II}, eds. R.J. Hanisch, R.J.V. Brissenden and J. Barnes,
Astronomical Society of the Pacific Conference Series, {\bf 52}, 
pp137-141.


\section{Lunar and Planetary Bibliography
\xlabel{lunar_and_planetary_bibliography}\label{LPI_BIB}}

The NASA Lunar and Planetary Institute (LPI), Houston provides a 
bibliographic database for lunar and planetary studies. It is accessed
as part of the information services offered by the LPI (see 
Section~\ref{LPI_SERVE}). After logging on, simply select the 
appropriate option from the menu. The database is searched using simple
menu-driven software; just follow the instructions. Public access to the
database is allowed, so anyone can browse through it. However, you must 
register if you want to save the results of your searches as files on 
the LPI computers (for example, so that you can retrieve copies to your 
local machine). Instructions for registration are included in the menus.


\section{SISSA Preprint Archive\xlabel{sissa_preprint_archive}}

The International School of Advanced Studies (SISSA/ISAS), 
Trieste\footnote{The full postal address is: Scuola Internazionale 
Superiore di Studi Avanzati (International School for Advanced Studies),
Via Beirut, 2-4 -- 34014, Trieste, Italy.} operates an electronic 
archive of preprints of astronomical papers. In many ways this archive
is intended as a replacement for the traditional system whereby 
preprints of papers are exchanged between astronomical institutions, but
without the associated costs and delays. The system operates as follows:
authors submit an electronic copy of their preprint to the archive. The
preprint then becomes publicly available, so that anyone may examine 
it, and if they so desire, retrieve a copy. The system operates 
completely automatically.

The archive is intended to cover the entire field of astronomy, 
astrophysics and cosmology. It was established in April 1992 and by
August 1993 contained some 245 preprints, with a further 135
cross-references to preprints in other archives maintained by SISSA 
(see below). Preprints are added at a rate of about thirty five per 
month. The preprints cover a wide variety of astronomical topics, but 
with cosmology predominating. By April 1993 about 1200 users had 
accessed the archive.

Preprints are stored as either simple text files, or Latex or Tex
source files. In some cases the diagrams associated with a preprint
may not be included.

The archive can be accessed in two ways: by standardized commands 
submitted using e-mail and by anonymous ftp. The former is the
preferred method of using the archive. In this mode you issue mail
messages with standard commands in the `subject' field (the text of
the messages should be blank) and send them to the archive. It replies,
also by e-mail, with the requested abstracts, preprints or
information. Some of the facilities available include:

\begin{itemize}

  \item getting information on the preprints available,

  \item retrieving selected abstracts and preprints,

  \item `subscribing' to the archive, so that you will receive 
   automatically details (including title, authors and abstract) of new
   preprints as they are added to the archive (reluctantly I must admit 
   that I have not tried this facility because I do not fancy being 
   bombarded with e-mail messages containing abstracts of papers on 
   cosmology),

  \item contributing your own preprint.

\end{itemize}

Full details of using the system can be obtained by sending an e-mail
message with the single word {\tt help} in the subject field and no 
text in the main body to address astro-ph@babbage.sissa.it (INTERNET). 
Preprints in the archive are arranged by the month in which they were 
entered, and a list of the months for which preprints are available may
be obtained by sending a message with the single word {\tt listing} in 
the subject field, and again no text in the main body to the same 
address. There do not appear to be any facilities to search the entire
archive for preprints by a given author or on a given topic.

The details of accessing the system via anonymous ftp (see 
Section~\ref{ANON_FTP}) are as follows:

{\it INTERNET address: } babbage.sissa.it or 147.122.1.21
\newline {\it Username:} \verb-anonymous-
\newline {\it Password:} Type your username and e-mail address

Once you are connected, the relevant directories are {\tt listings} 
which contains abstracts and {\tt papers} which contains the full text 
of the preprints. Each preprint is kept in its own subdirectory, the 
name of the subdirectory being the internal number assigned to it by 
the archive.

In addition to the astronomy, astrophysics and cosmology archive, SISSA
maintains several other archives which are relevant to various branches
of physics. These archives operate in exactly the same way as the 
astronomy one, but have a different e-mail address. They are
listed, together with their addresses, in Table~\ref{SISSA}.

\begin{table}[htbp]

\begin{center}
\begin{tabular}{ll}
Subject   &  Electronic mail address  \\ \hline
{\bf astronomy, astrophysics and cosmology}  & {\bf astro-ph@babbage.sissa.it} \\
condensed matter                      & cond-mat@babbage.sissa.it \\
functional analysis                   & funct-an@babbage.sissa.it \\
general relativity and quantum cosmology  & gr-qc@babbage.sissa.it \\
phenomenological high energy physics   & hep-ph@babbage.sissa.it \\
formal high energy physics            & hep-th@babbage.sissa.it \\
nuclear physics theory                & nucl-th@babbage.sissa.it \\
database of e-mail addresses & e-mail@babbage.sissa.it \\
\end{tabular}
\end{center}

\caption{Preprint archives available at SISSA}
\label{SISSA}

\end{table}


\section{Other Preprint Archives and Lists
\xlabel{other_preprint_archives_and_lists}}

In addition to the SISSA archive, preprint collections are maintained
at the Los Alamos National Laboratory and the European Centre for 
Nuclear Research. Both of these collections cover many branches of 
physics, rather than being specifically astronomical. They may, however,
contain some preprints of interest to astronomers. Also the Space 
Telescope Science Institute and the National Radio Astronomy 
Observatory maintain on-line lists of their preprint collections.

The article by G. Taubes (see Section~\ref{LANL}, below) serves as a
general introduction to preprint archives as well as a description of 
the Los Alamos archive.

\subsection{LANL
\xlabel{lanl}\label{LANL}}

The preprint archive at the Los Alamos National Laboratory (LANL), New 
Mexico may be queried either remotely using standardized mail messages 
or via anonymous ftp. To obtain further information on the archive send 
a blank e-mail message with {\tt HELP} in the subject field to address 
hepth@xxx.lanl.gov (INTERNET).

The details of accessing the system via anonymous ftp (see 
Section~\ref{ANON_FTP}) are as follows:

{\it INTERNET address: } xxx.lanl.gov or 128.165.23.9
\newline {\it Username:} \verb-anonymous-
\newline {\it Password:} Type your username and e-mail address

{\it Reference:}

G. Taubes, 26 February 1993, {\it Science}, {\bf 259}, pp1246-1248.

\subsection{CERN\xlabel{cern}}

The preprint archive at the European Centre for Nuclear Research (CERN),
Geneva includes all preprints produced at CERN, a copy of the LANL 
archive and various other documents. It is accessed via anonymous ftp 
(see Section~\ref{ANON_FTP}), the details being as follows.

{\it INTERNET address: } asis01.cern.ch or 128.141.201.136
\newline {\it Username:} \verb-anonymous-
\newline {\it Password:} Type your username and e-mail address

Once you are connected, type {\tt cd preprints} to select the directory 
containing the preprints. This directory is divided into a large number 
of subdirectories, one per institution (or group of institutions) which 
have submitted preprints.

\subsection{STScI
\xlabel{stsci}\label{STSCI}}

The Space Telescope Science Institute (STScI), Baltimore maintains an 
on-line list of its preprint collection. The collection includes: all
refereed papers published using Hubble Space Telescope (HST)
observations, papers written at the STScI and preprints received from 
external institutions. It covers a wide range of astronomical topics. 
The list contains only bibliographic details, not the full text of the 
preprint, nor an abstract. When a preprint in the list is published its 
full bibliographic details are added. The list contains about 5000 
entries and is updated fortnightly.

The list can be queried with WAIS (see Section~\ref{WAIS}), though first
you should obtain a copy of the source file {\tt stsci-preprint-db.src}.
A copy of this file can be obtained from the European Space Telescope
Coordinating Facility (again see Section~\ref{WAIS}, especially 
Table~\ref{WAIS_SOURCE}).

\subsection{NRAO
\xlabel{nrao}\label{NRAO}}

The National Radio Astronomy Observatory (NRAO), Charlottesville, 
Virginia maintains an on-line list of its preprint collection. The 
collection includes papers written at the NRAO and preprints received 
from external institutions. It covers a wide range of astronomical 
topics. The list contains only bibliographic details, not the full text 
of the preprint, nor an abstract. When a preprint in the list is 
published its full bibliographic details are added. The list covers the 
period from October 1986 to the present and on the 1st May 1993 
contained about 13,000 entries. It is updated once per week.

The list can be queried with WAIS (see Section~\ref{WAIS}), though first
you should obtain a copy of the source file {\tt nrao-raps.src}.
A copy of this file can be obtained from the European Space Telescope
Coordinating Facility (again see Section~\ref{WAIS}, especially 
Table~\ref{WAIS_SOURCE}).


\section{Journal Tables of Contents\xlabel{journal_tables_of_contents}}

Tables of contents for recent issues of journals can be displayed using
either BIDS ISI or BIDS Inside Information (see Sections~\ref{BIDS_ISI}
and \ref{BIDS_II}). Two alternative systems are available, which, 
unlike BIDS, can be accessed from outside the United Kingdom.

\subsection{CfA\xlabel{cfa}}

The Harvard-Smithsonian Center for Astrophysics (CfA), Cambridge, 
Massachusetts maintains tables of contents for the following journals:
{\it Astrophysical Journal, Astrophysical Journal Supplement, 
Astronomical Journal} and {\it Publications of the Astronomical Society 
of the Pacific}. The period covered is from January 1988 to the present.

The details of accessing the system are as follows:

{\it INTERNET address:} cfa$n$.harvard.edu (where $n = 3, 4, 5, 7$ or
$8$)
\newline {\it Username:} \verb-APJAJ-
\newline {\it Password:} No password required; captive account

In order to access this system, you should log on to a local Starlink
Unix system and type, for example:

\vspace{2.0 mm}
\verb:telnet cfa3.harvard.edu:
\vspace{2.0 mm}

Reply {\tt APJAJ} in response to the `Username:' prompt; no password is
required. You will be logged into a captive account and some 
introductory information will be displayed. Thereafter simply follow 
the instructions.

\subsection{STScI\xlabel{stsci}}

The Space Telescope Science Institute (STScI), Baltimore maintains a
table of contents for the {\it Publications of the Astronomical
Society of the Pacific} (PASP). Abstracts for the individual papers are
also available. This system is new and coverage only extends from 
January 1993 to the present.

The system is accessed via anonymous ftp (see Section~\ref{ANON_FTP}).
The details are as follows:

{\it INTERNET address:} stsci.edu or 130.167.1.2
\newline {\it Username:} \verb-anonymous-
\newline {\it Password:} Type your username and e-mail address

Once you are connected to the system type \verb-cd pasp- to select the
directory containing the abstracts. The files in directory \verb-pasp- 
are all text files suitable for printing. The contents and abstract for 
each monthly issue of the journal are kept in a separate file. These 
files have names of the form:

\begin{verbatim}
contents_yy_mm
\end{verbatim}

For example, file \verb-contents_93_02- contains details for the
February 1993 issue. The remaining files contain a miscellany of useful 
information pertaining to PASP, for example, instructions on preparing 
papers for submission.


\section{Choosing a Bibliographic Database
\xlabel{choosing_a_bibliographic_database}}

All the systems listed here have their strengths. The one which you will
find most useful depends as much on your requirements and circumstances 
as the inherent strengths and weaknesses of the system.

The bibliography at the Lunar and Planetary Institute is the obvious
choice for lunar and planetary studies.

The SISSA preprint archive is certainly worth investigating if you are 
interested in cosmology or related fields.

Similarly, you should try the STScI preprint list if you are interested
in HST observations.

If you are looking for references pertaining to an individual, named
astronomical object then you should use either SIMBAD or NED: NED for
external galaxies and other extra-galactic objects and SIMBAD for 
\pagebreak
everything else. Remember that NED has abstracts and SIMBAD does not.
Also you must register to use SIMBAD, and a fee is charged. Both SIMBAD 
and NED include a dictionary of synonyms for astronomical object 
designations. Most astronomical objects have several different
designations\footnote{For example, some of the alternative designations
for Aldebaran are: 

\begin{center}
\begin{tabular}{ll}
$\alpha$ Tau, &  (Bayer Letter),  \\
87 Tau,       &  (Flamsteed number),  \\
BS 1457       &  (Yale Bright Star catalogue),  \\
HR 1457       &  (Revised Harvard Photometry),  \\
BD +16$^{\circ}$ 629  &  (Bonner Durchmusterung),  \\
HD 29139      &  (Henry Draper catalogue),  \\
SAO 094027    &  (Smithsonian Astrophysical Observatory catalogue).  \\
\end{tabular}
\end{center}
}, any of which can appear in
article titles and abstracts. A dictionary of synonyms allows data and
references to an object to be identified, irrespective of which of its
designations the user supplied. This facility is very useful when 
searching for named objects. BIDS ISI is particularly unsuitable for 
searching for named astronomical objects because a feature in the BIDS 
ISI search software prevents numbers being included in the search 
strings. This restriction precludes searches for most astronomical 
object designations.

If you are searching for some specific topic rather than an object
designation then the choice is probably between BIDS ISI (if you have 
access to it) and one of the systems derived from the NASA RECON system.
I suspect that BIDS ISI offers a wider range of journals and more 
complete coverage of those journals, but it does not have abstracts, 
which the RECON based systems do. Thus, use one of the RECON based 
systems if abstracts are important to you and BIDS ISI if they are not. 
Which of the three RECON based systems (ADS, ESIS, STELAR) you would 
choose depends on which you can get access to most conveniently. There 
seems little to choose, in terms of journal coverage, between the ADS 
and ESIS systems. You would probably only use STELAR if you could not 
get adequate access to either ADS or ESIS and were prepared to accept 
the limited number of journals covered by STELAR.

Both the NRAO and STScI preprint lists are sufficiently extensive to
offer a reasonable alternative to BIDS ISI, and they have the advantage
that, unlike BIDS ISI, they are accessible outside the United Kingdom.

Finally, it is worth remembering that the printed {\it Astronomy and 
Astrophysics Abstracts}\, offer extremely comprehensive coverage; none
of the on-line systems are as exhaustive.

\pagebreak
\part{Information Services}

\section{Overview\xlabel{overview}}

This section lists a miscellany of astronomical information services
other than bibliographic databases and observation archives. The 
distinction between the various sorts of facility is not always clear.
For example, STEIS (see Section~\ref{STEIS}) can be used to access both
the HST data archive (see SUN/162) and the STScI preprint collection
(see Section~\ref{STSCI}). Most of the systems listed are specifically
astronomical. However, the final few items are more general, but 
nonetheless potentially useful.

The list of services discussed here is by no means complete. A number
of additional facilities can be accessed via STEIS (see 
Section~\ref{STEIS}). Some of these services are listed in 
Tables~\ref{AIR_WEB} to \ref{AIR_FTP}. Section~\ref{STEIS} includes 
details of how to access them.


\section{IAU Circulars\xlabel{iau_circulars}}

Astronomy is an observational science. Often events occur which
require the rapid dissemination of information in order to facilitate
further observations of new or transient phenomena. Examples include
unexpected behaviour in variable stars, or newly discovered novae,
supernovae, comets or asteroids. There is a well-established system of
International Astronomical Union (IAU) Circulars and Minor Planet 
Circulars which distribute such information to subscribing 
observatories and institutions. This system, which predates electronic
computers, distributes the Circulars on paper. It is run by the Central
Bureau for Astronomical Telegrams and Minor Planet Center, Smithsonian 
Astrophysical Observatory, Cambridge, Massachusetts 02138 in the United 
States.

It is now possible to receive the IAU Circulars and material from the
Minor Planet Circulars as e-mail messages. In order to receive these
e-mail copies it is necessary to subscribe to the printed circulars
and a modest additional charge is made. Also there are restrictions on
re-distributing the electronic version of the circulars outside your
institution. Typically several circulars are issued per week.

For further information contact B.G. Marsden, the Director of the Bureau
for Astronomical Telegrams or D.W.E. Green, the Associate Director.
Their e-mail addresses are 
\newline marsden@cfa.harvard.edu and green@cfa.harvard.edu 
respectively (INTERNET).


\section{Solar and Auroral Activity\xlabel{solar_and_auroral_activity}}

Reports of solar and auroral activity can be obtained from the 
University of Lethbridge, Alberta. You should access this service from
a Starlink Unix system. Brief details of the information available and
how to access it are as follows.

\begin{itemize}

  \item A solar and geophysical activity report, updated at three-hourly
   intervals. To access it type:

  \nopagebreak
  \vspace{2.0 mm}
  \verb:finger solar@xi.uleth.ca: ~~ or ~~ 
  \verb:finger solar@142.66.3.29:
  \vspace{2.0 mm}

  \item A solar and geophysical activity report, updated daily. To 
   access it type:

  \vspace{2.0 mm}
  \verb:finger daily@xi.uleth.ca: ~~ or ~~ 
  \verb:finger daily@142.66.3.29:
  \vspace{2.0 mm}

  \item A report including predictions of auroral activity, sightings
   of auroras,  aurora watches etc. This information is updated hourly.
   To access it type:

  \vspace{2.0 mm}
  \verb:finger aurora@xi.uleth.ca: ~~ or ~~ 
  \verb:finger aurora@142.66.3.29:
  \vspace{2.0 mm}

\end{itemize}

Note that these reports are accessed using the Unix command {\tt
finger}. No username or password is required. The reports can be
quite lengthy and are probably best retrieved into a text file.
However, unfortunately, the {\tt finger} command when used across 
INTERNET inserts an unprintable character at the end of each line, which
makes the resulting file unsuitable for display. A Unix magic
incantation to remove this character prior to writing to the file
is, for example:

\begin{verbatim}
finger aurora@xi.uleth.ca | sed '1,$s/.$/ /' > aurora.lis
\end{verbatim}

This example writes the auroral activity report to file {\tt 
aurora.lis}.

A database of solar and auroral activity is also maintained at the
University of Lethbridge, and can be accessed by anonymous ftp (see 
Section~\ref{ANON_FTP}). The details  are as follows:

{\it INTERNET address:} xi.uleth.ca or 142.66.3.29
\newline {\it Username:} \verb-anonymous-
\newline {\it Password:} Type your username and e-mail address

Once you are connected, the relevant directory is {\tt /pub/solar}.
File {\tt 00-index.txt} in this directory contains an index of the
database contents. A particularly useful file is 
\newline {\tt /pub/solar/Docs/glossary.doc} which gives a glossary of 
technical terms used in the reports.

Further information and assistance can be obtained from Cary Oler,
e-mail Oler@rho.uleth.ca (INTERNET).


\section{Finding Usernames and Electronic Mail Addresses
\xlabel{finding_usernames_and_electronic_mail_addresses}}

Several facilities are available to assist you to find the username and 
e-mail address of a colleague whose name and institution you already 
know. Some of these facilities are summarized below.

\subsection{Lists of Starlink Users\xlabel{lists_of_starlink_users}}

All Starlink computers contain files holding lists of Starlink users. 
The details differ between VAX/VMS and Unix systems.

\subsubsection{VAX/VMS\xlabel{vaxvms}}

On Starlink VAX/VMS systems lists of users and related information are
kept in the following files:

\begin{description}

  \item[{\tt LADMINDIR:USERNAMES.LIS}] lists local Starlink users at 
   your node,

  \item[{\tt ADMINDIR:USERNAMES.LIS}] lists all Starlink users at all
   nodes,

  \item[{\tt ADMINDIR:UNIXNAMES.LIS}] lists Starlink Unix users and 
   gives codes for their locations,

  \item[{\tt ADMINDIR:LOCATIONS.LIS}] explains the codes used to 
   describe the locations of Starlink users.

\end{description}

All these lists are updated regularly. The local {\tt USERNAMES.LIS} is
more likely to be up to date than the Starlink-wide version, though the
latter is probably more self-consistent. All the files are simple text
files, and can be searched using standard operating system commands,
for example:

\begin{verbatim}
SEARCH  ADMINDIR:USERNAMES.LIS  Davenhall
\end{verbatim}

\subsubsection{Unix\xlabel{unix}}

On Starlink Unix systems lists of users and their addresses are 
maintained in file 
\newline {\tt /star/admin/unixnames}. It is a simple text file, and
can be searched using standard operating system commands, such as {\tt 
grep}.

\subsection{International List of Astronomical Electronic Mail Addresses
\xlabel{international_list}\label{EMAIL}}

Chris Benn and Ralph Martin of the Royal Greenwich Observatory, 
Cambridge have compiled a comprehensive list of the e-mail addresses of
astronomers throughout the world. This list is distributed widely and
is available at most (but not all) Starlink sites. Your site manager 
should be able to advise whether it is available at your site.

The utility {\tt email} is provided on Starlink VAX/VMS systems to 
search these files for a given user. This facility is not available on
Starlink Unix systems. If the lists are available at your site, simply 
type, for example:

\begin{verbatim}
email  Davenhall
\end{verbatim}

Normally you will not need to know the names and locations of the files
holding these lists, but for completeness they are usually:

\begin{description}

  \item[{\tt LDOCSDIR:ASTROPERSONS.LIS }] which gives, for each person 
   listed, their username, e-mail address and a code for their 
   institution.

  \item[{\tt LDOCSDIR:ASTROPLACES.LIS }] which gives, for each 
   astronomical institution listed, its e-mail address, and telephone 
   and fax numbers.

  \item[{\tt LDOCSDIR:ASTROPOSTAL.LIS }] which gives, for each 
   astronomical institution listed, its full postal address.

\end{description}

In all cases the files are simple text files, so you can, if you so 
desire, search them with standard operating system commands, edit them
in read-only mode etc. Note, however, that the details of a given
institution occupy several lines of {\tt ASTROPLACES.LIS} and {\tt 
ASTROPOSTAL.LIS}.

In addition to the Starlink versions, these lists are distributed 
widely, so you may be able to access them, even if you are not a
Starlink user. Versions can be accessed through the ADS and ESIS.
A copy of the list is also available through STEIS (see 
Section~\ref{STEIS}). The procedure to access this version is as
follows.

\begin{enumerate}

  \item Connect to STEIS using gopher (see Section~\ref{STEIS_GOPHER};
   the following description assumes you are using an X-terminal gopher 
   client).

  \item Choose menu {\tt STEIS search tools} and then menu {\tt
   Search Astronomer EMail Address (STScI version)}.

  \item A window will appear containing a box labelled {\tt Search of:}.
   Enter the surname of the person you are looking for in this box,
   and hit return.

  \item After a few moments a window will appear with a box containing
   the words {\tt Matches for} {\it the-surname-you-entered}.
   Double-click on this text and information about the person will be
   displayed.

  \item When you have finished click on the boxes marked either `done'
   or `previous directory' to retrace your steps through the menus.

\end{enumerate}

The lists are updated annually and any comments or additional 
information are gratefully received. They should be sent by e-mail
to email@ast-star.cam.uk (JANET and INTERNET).

{\it Documentation:}

A description of the lists is available in {\it Electronic Mail}\, by
C.R. Benn and R. Martin, 1993. If the lists are installed on
your node this document will be held as file 
\newline {\tt LDOCSDIR:ASTROGUIDE.TEX}. It is a Tex source file.

A number of other directories of astronomers and astronomical 
institutions are also available. They are reviewed by A. Heck (1991, in 
{\it Databases and On-Line Data in Astronomy}, eds. D.~Egret and 
M.~Albrecht, Dordrecht: Kluwer, pp211-224).


\subsection{Non-Astronomical Users\xlabel{nonastronomical_users}}

A number of more general services for finding the e-mail addresses for 
users outside the astronomical community are available. These services, 
often called `white pages' (in an analogy to a telephone directory), 
are beyond the scope of this document, but are described in {\it The 
Whole Internet User's Guide and Catalog}, especially Chapter Ten (see 
Section~\ref{FURTHER}).

One simple but useful facility is the Unix command {\tt finger}. Usually
this command can be used in the common case where you want to find out 
the username of a colleague, but you already know (or can find out) the 
INTERNET address of a computer running Unix which he uses. Beware 
however, that some systems will refuse remote {\tt finger} requests. 
From a Starlink Unix system simply type:

\vspace{2.0 mm}
{\tt finger ~ }{\it surname}{\tt @}{\it host}
\vspace{2.0 mm}

for example {\tt Davenhall@star.le.ac.uk}. If an INTERNET connection
to the given address can be established, some information about the 
user, including their username, will be displayed.  The same technique 
can be used in the less common case where you know a username and want 
to find the corresponding actual name of its owner. Here you would 
type, for example:

\begin{verbatim}
finger  acd@star.le.ac.uk
\end{verbatim}

and the same information would be displayed.

A given name may also be substituted for either the surname or 
username, but there are a couple of reasons why this approach is usually
less satisfactory. Firstly, a given name is less likely to be unique
(in the case of ambiguity all the matches are displayed) and more
importantly there is no way to predict whether the system will know
someone by their proper given name or some contraction or nickname
based on it (Dave for David etc). Usually it is better to use the 
surname.

If you want to find out about a user on your local machine simply omit
the INTERNET address. For example:

\vspace{2.0 mm}
\verb:finger  Davenhall: ~~ or ~~ \verb:finger  acd:
\vspace{2.0 mm}


\section{STEIS
\xlabel{steis}\label{STEIS}}

The Space Telescope Electronic Information System (STEIS) is maintained
by the Space Telescope Science Institute (STScI), Baltimore. It provides
a variety of information about the Hubble Space Telescope (HST) and the
STScI. It also allows convenient access to a number of other 
astronomical information services. Some of the facilities available
include:

\begin{itemize}

  \item copies of HST documents, such as proposal instructions, 
   instrument handbooks, a manual for the data archive and cookbooks.
   Both PostScript and simple text file versions are available,

  \item some of the HST operational files, including: the satellite
   status, the weekly summary and the time line. If you have been 
   awarded HST observing time you can search these files to check on 
   the progress of your observations,

  \item access to the HST data archive (see \xref{SUN/162}{sun162}{}),

  \item a version of the Royal Greenwich Observatory's worldwide list 
   of astronomer's e-mail addresses (see Section~\ref{EMAIL}),

  \item the STScI preprint list (see Section~\ref{STSCI}),

  \item remote access to the STELAR bibliographic database (see 
   Section~\ref{STELAR}),

  \item access to a large number of other astronomical information
   services which are available on the INTERNET.

\end{itemize}

The preferred method to access STEIS is to use gopher. However, it can
also be accessed using the World Wide Web, anonymous ftp and remotely 
using a listserver. Details of all four of these methods are given 
below. Further information and assistance can be obtained by sending an 
e-mail message to usb@stsci.edu (INTERNET).

STEIS provides convenient access to a large number of other astronomical
information services. The services available (in November 1993) are 
listed in Tables~\ref{AIR_WEB} to \ref{AIR_FTP}. You should note: the
number of services available is increasing, some of the services are not
always available, and some of them do not seem to have been set up very 
well: {\it caveat emptor}. Most of these services can be accessed 
through STEIS using gopher and they can all be accessed through STEIS 
using the Web. Details are included in the appropriate section, below.

% STEIS Web -------------------------------

\begin{table}[htbp]

\begin{center}
\begin{tabular}{l}
Astronomical Institute, Utrecht University  \\
Astrophysics Data System (ADS)   \\
Cagliari Astronomical Observatory   \\
Canadian Astronomy Data Centre   \\
Center for EUV Astrophysics EUVE Guest Observer Center   \\
Centre de Donn\'{e}es astronomiques de Strasbourg (CDS)   \\
European Space Information System   \\
Institute of Astronomy and Royal Greenwich Observatory, Cambridge  \\
Mount Stromlo and Siding Spring Observatories  \\
NASA / Goddard Space Flight Center   \\
NASA / Langley Research Center   \\
NASA / Scientific and Technical Information Project   \\
NASA / Space Science Data Operations Office   \\
NASA / STELAR Project STELAR   \\
National Center for Atmospheric Research NCAR   \\
National Optical Astronomy Observatories NOAO   \\
National Radio Astronomy Observatory NRAO   \\
National Solar Observatory   \\
Planetary Data System, NASA JPL  \\
Princeton University Observatory   \\
Sloan Digital Sky Survey (Fermilab)   \\
Space Astrophysics Laboratory, Ontario \\
Space Index   \\
STECF Space Telescope European Coordinating Facility   \\
STScI Space Telescope Science Institute   \\
University of Massachusetts at Amherst Department of Astronomy   \\
University of Southampton, Astronomy Group   \\
UUNA Astronomy Listings   \\
\end{tabular}
\end{center}

\caption{World Wide Web resources available through STEIS}
\label{AIR_WEB}

\end{table}


% STEIS gopher -------------------------------

\begin{table}[htbp]

\begin{center}
\begin{tabular}{l}
Australian National Telescope Facility   \\
Canadian Astronomy Data Center   \\
Center for EUV Astrophysics EUVE Guest Observer Center   \\
Chilean Weather Forecast   \\
Goddard Space Flight Center (NSSDC)   \\
Harvard-Smithsonian Center for Astrophysics Theory Group   \\
High Energy Astrophysics Science Archive Research Center   \\
Institute of Astronomy and Royal Greenwich Observatory, Cambridge  \\
Johnson Space Center, NASA   \\
National Center for Atmospheric Research   \\
NASA Network Application and Information Center (NAIC)   \\
National Optical Astronomy Observatories   \\
LANL Physics Information Service   \\
Physics Resources (Experimental)   \\
UCO/Lick Gopher Information Service   \\
\end{tabular}
\end{center}

\caption{Gopher resources available through STEIS}
\label{AIR_GOPHER}

\end{table}


% STEIS WAIS -------------------------------

\begin{table}[htbp]

\begin{center}
\begin{tabular}{l}
AAS Job Register   \\
AAS Meeting Abstracts  \\ 
AAS Summer 92 Meeting   \\
AAS Summer 93 Meeting   \\
AAS Winter 93 Meeting   \\
Astronomy Site Addresses   \\
Astronomy Person Addresses   \\
Astronomy Postal Addresses   \\
EROS Data Center   \\
NASA Directory of WAIS Servers (STELAR)   \\
NASA Missions   \\
NSSDC CD-ROM   \\
SPACEWARN   \\
STELAR Voyager Images (experimental)   \\
STScI Searchtools   \\
\end{tabular}
\end{center}

\caption{WAIS resources available through STEIS}
\label{AIR_WAIS}

\end{table}


% STEIS telnet  -------------------------------

\begin{table}[htbp]

\begin{center}
\begin{tabular}{l}
AXAF Science Center   \\
Compton/GRO Guest Observer Facility - GOF   \\
Compton/GRO News - GRONEWS   \\
DIRA2, Bologna   \\
Einstein On-Line Service - Einline   \\
ESA Data Dissemination Network   \\
ESIS - Astronomy   \\
ESIS - Space Physics   \\
ESO Bulletin Board   \\
ESOSAT Database - ESTEC  \\ 
HST Archive   \\
IPAC IRSKY Remote Access Tool   \\
IPAC XCATSCAN Catalog Scanning Tool   \\
IRAS Low Resolution Spectra   \\
IUE VILSPA Archive  \\ 
JCMT Data Archive   \\
La Palma Data Archive   \\
Lunar and Planetary Institute   \\
Lyon-Meudon Extragalactic Database (LEDA)   \\
NASA/IPAC Extragalactic Database - NED   \\
National Oceanic and Atmospheric Admin. Database   \\
NSI On-Line Network Aide - NONA   \\
Planetary Data Systems (JPL)   \\
SIMBAD   \\
StarWays - European Space Information System - ESIS   \\
STECF STARCAT   \\
STECF STINFO   \\
UK and ESO Schmidt, AAT Plate Catalogs   \\
VLA Information System   \\
\end{tabular}
\end{center}

\caption{{\tt Telnet} resources available through STEIS}
\label{AIR_TEL}

\end{table}


% STEIS ftp -------------------------------

\begin{table}[htbp]

\begin{center}
\begin{tabular}{l}
American Astronomical Society   \\
Anglo-Australian Observatory   \\
Arecibo Observatory   \\
Big Bear Solar Observatory   \\
Canada France Hawaii Telescope   \\
Centre de Donn\'{e}es astronomiques de Strasbourg (CDS)   \\
Cerro Tololo Inter-American Observatory   \\
Compton/GRO FTP Directories   \\
Dominion Astrophysical Observatory   \\
ESIS ftp Tree   \\
ESIS User Guide   \\
European Southern Observatory   \\
EUVE Guest Observer Center   \\
High Altitude Observatory   \\
Infrared Processing and Analysis Center (IPAC)   \\
JPL Public Access   \\
Joint Astronomy Centre (UKIRT and JCMT)   \\
Kitt Peak National Observatory   \\
Lowell Observatory   \\
Lunar and Planetary Institute   \\
Meetings List (CFHT)   \\
NASA Research Announcements   \\
National Radio Astronomy Observatory   \\
Parkes Radio Observatory   \\
sci.astro.hubble Archive   \\
SIMBAD  \\
Smithsonian Astrophysical Observatory (SAO)   \\
Solar, Auroral, Ionospheric, Information   \\
Steward Observatory   \\
ST-ECF Archives   \\
VLA Sky Survey   \\
Wilcox Solar Observatory (Stanford Univ.)   \\
WIYN - KPNO   \\
\end{tabular}
\end{center}

\caption{ftp resources available through STEIS}
\label{AIR_FTP}

\end{table}

\subsection{STEIS via Gopher
\xlabel{steis_via_gopher}\label{STEIS_GOPHER}}

The preferred method of accessing STEIS is gopher (see 
Section~\ref{GOPHER}). If you access gopher from a local client the
details will be specific to your site; your site manager should be
able to advise. If you are using an X interface then first you must
redirect X output to your terminal. The details will vary, but your site
manager should be able to advise. Then type perhaps:

\begin{verbatim}
xgopher stsci.edu 70
\end{verbatim}

If you are accessing gopher on a remote client you should log on to it
and then find the STEIS server by working through something like the
following menu items (the details may vary between clients):

\begin{verse}
{\tt Other Gopher and Information Servers  \\
USA  \\
General  \\
Space Telescope Electronic Information System}
\end{verse}

To access other astronomical information services, choose STEIS menu
{\tt Astronomical Internet Resources} and then the service you require.

\subsection{STEIS via the Web
\xlabel{steis_via_the_web}\label{STEIS_WEB}}

Details of accessing STEIS through the Web (see Section~\ref{WEB}) are
as follows. The URL of the STEIS Astronomical Information Resources
home page is:

\begin{verbatim}
http://stsci.edu/net-resources.html
\end{verbatim}

(note that the last character is the letter `l', not the digit `1').
To navigate to STEIS from the public Web client at CERN, use the
following list of links:

\begin{verse}
{\tt The Virtual Library  \\
Astronomy and Astrophysics  \\
See also Space} ~~ (the last link in the list)  \\
{\tt Space Telescope Science Institute electronic information service}\\
\end{verse}

A similar route should be possible from most Web clients.

To access other astronomical information services, choose either link
{\tt Astronomical Internet Resources} or {\tt STEIS WWW Top Level}
and then the service you require.

\subsection{STEIS via Anonymous ftp
\xlabel{steis_via_anonymous_ftp}\label{STEIS_FTP}}

The details of accessing STEIS using anonymous ftp (see 
Section~\ref{ANON_FTP}) are as follows:

{\it INTERNET address:} stsci.edu or 130.167.1.2
\newline {\it Username:} \verb-anonymous-
\newline {\it Password:} Type your username and e-mail address

\subsection{STEIS Listserver\xlabel{steis_listserver}}

STEIS provides a `listserver' to which you may subscribe. Once you have
subscribed it will automatically mail you a copy of some frequently
used files whenever they are updated. The files available (in November
1993) are listed in Table~\ref{STEIS_FILES}.

\begin{table}[htbp]

\begin{center}
\begin{tabular}{ll}
Description                                &  Filename        \\ \hline
Lists observations for the coming year     & {\tt long-range-plane}  \\
Detailed weekly schedule of observations   & {\tt timeline}   \\
Daily activity/instrument status reports   & {\tt hst-status} \\
List of all observations completed to date & {\tt completed-observations} \\
Point spread functions for the WFPC \dag   & {\tt wfpc-psf-library}  \\
Delta flat field information for the WFPC \dag  & {\tt wfpc-delta-flats} \\
Information about the Faint Object Camera (FOC) & {\tt foc-news} \\
\end{tabular}

\vspace{4.0 mm}

\dag Wide Field and Planetary Camera
\end{center}

\caption{Files available from the STEIS listserver}
\label{STEIS_FILES}

\end{table}

You may choose which of the files you receive. To use the system you
send e-mail messages to username listserv@stsci.edu (INTERNET).
The subject field should be blank and the text should comprise 
standardized messages as follows.

\begin{itemize}

  \item To subscribe to a file the message text should be:

  \vspace{2.0 mm}
  {\tt subscribe ~} {\it filename} ~~ {\it your-name}
  \vspace{2.0 mm}

   where {\it filename}\, is one of the filenames listed in 
   Table~\ref{STEIS_FILES} and {\it your-name} is your full name.

  \item To `unsubscribe' to file {\it filename:} 

  \vspace{2.0 mm}
  {\tt unsubscribe ~} {\it filename}
  \vspace{2.0 mm}

  \item For a summary of the commands available: {\tt HELP}

\end{itemize}


\section{Lunar and Planetary Institute
\xlabel{lunar_and_planetary_institute}\label{LPI_SERVE}}

The NASA Lunar and Planetary Institute (LPI), Houston provides a number
of services relating to lunar and planetary studies. These services
include:

\begin{itemize}

  \item a bibliographic database (see Section~\ref{LPI_BIB}),

  \item the Image Retrieval and Processing System (IRPS),

  \item information and abstracts for meetings,

  \item the electronic journal {\it Lunar and Planetary Information
   Bulletin}.

\end{itemize}

The primary means of accessing the system is {\tt telnet} and you need
a VT100 compatible terminal. The details are as follows:

{\it INTERNET address: } lpi.jsc.nasa.gov or 146.154.14.11
\newline {\it Username:} \verb-LPI-
\newline {\it Password:} No password required; captive account

Log on to a local Starlink Unix system and type either:

\vspace{2.0 mm}
\verb:telnet lpi.jsc.nasa.gov : ~~ or ~~ \verb:telnet 146.154.14.11:
\vspace{2.0 mm}

In response to the `Username:' prompt reply {\tt LPI}; no password is
required. Once logged on you will enter a simple menu system. Use the 
arrow keys to navigate the options and hit return to select an option.

Some of the LPI services can also be accessed using anonymous ftp (see
Section~\ref{ANON_FTP}). The details are as follows:

{\it INTERNET address: } lpi.jsc.nasa.gov or 146.154.14.11
\newline {\it Username:} \verb-anonymous-
\newline {\it Password:} Type your username and e-mail address

Note that the LPI computers are VAXes running the VMS operating system,
and hence when specifying directories within ftp you should use the
VAX/VMS notation. Various data are available. For example, directory
{\tt [ANONYMOUS.AMLAMP]} contains a database of meteorite falls in 
Antarctica. Retrieve files {\tt README.1ST} and {\tt README.2ND} from 
this directory for details of this database.


\section{JCMT\xlabel{jcmt}}

Two information services are available for the millimetre-wave James 
Clerk Maxwell Telescope (JCMT) in Hawaii: JCMT\_FILESERV and 
JCMT\_INFORM.

\subsection{JCMT\_FILESERV\xlabel{jcmtfileserv}}

JCMT\_FILESERV is a file distribution service provided by the Joint
Astronomy Centre (JAC), Hawaii. You interrogate this service by e-mail 
and it responds with e-mail messages containing the requested 
information. Information in the service is divided into a number of 
`packages', and each package comprises one or more files. The name 
of an individual file comprises the package name and a `part name', in 
the form {\tt package.part}. Table~\ref{JCMT_FILE} lists the packages 
available (in November 1993).

\begin{table}[htbp]

\begin{center}
\begin{tabular}{ll}
Description                                      & Package  \\ \hline
Information pertinent to requests for JCMT time   &  APPLICATIONS \\
General information for visitors                  &  HAWAII       \\
Job opportunities                                 &  JOBS         \\
Monthly JAC news                                  &  NEWS         \\
Technical/administrative articles from Newsletter &  NEWSLETTER   \\
Basic information on planet fluxes etc.           &  PLANETS      \\
Information to allow planning observing runs      &  PLANNING     \\
Source position information for pointing etc.     &  POINTING     \\
Notes on UKT14 polarimeter                        &  POLARIMETER  \\
Information on JCMT `front ends'                  &  RECEIVERS    \\
JCMT/HP bookings form                             &  RESERVATIONS \\
SCUBA progress news items                         &  SCUBA        \\
JCMT schedule access                              &  SCHEDULE     \\
Information on the service observing programs     &  SERVICE      \\
Notes and information on JCMT spectrometers       &  SPECTROMETER \\
JCMT usage statistics                             &  STATISTICS   \\
Information on UKT14 (a photometer)               &  UKT14        \\
User Guide (incomplete)                           &  USERGUIDE    \\
\end{tabular}
\end{center}

\caption{JCMT\_FILESERV packages}
\label{JCMT_FILE}

\end{table}

You use the system by sending e-mail messages to username 
JCMT\_INFO@jach.hawaii.edu (INTERNET). The subject field should be blank
and the text should comprise standardized messages as follows.

\begin{itemize}

  \item To list all the packages available: {\tt LIST}

  \item To retrieve all the parts of package {\it package:}

  \vspace{2.0 mm}
   {\tt SENDME} ~~ {\it package}
  \vspace{2.0 mm}

  \item To retrieve part {\it part} of package {\it package:}

  \vspace{2.0 mm}
   {\tt SENDME} ~~ {\it package.part}
  \vspace{2.0 mm}

  \item For further information: {\tt HELP}

\end{itemize}

Several commands may be included in a single message, but each must
occupy its own line.

Further information and assistance may be obtained by sending an e-mail
message to 
\newline JCMT\_INFO-Mgr@jach.hawaii.edu (INTERNET).  Queries about the
JCMT receivers may be sent to Henry Matthews, e-mail: 
HEM@jach.hawaii.edu (INTERNET).

\subsection{JCMT\_INFORM\xlabel{jcmtinform}}

JCMT\_INFORM is a captive account on the Starlink VAX cluster at the
Royal Observatory Edinburgh (ROE). The details of accessing it are:

{\it INTERNET address:} star.roe.ac.uk or 192.108.120.10
\newline {\it Username:} \verb-JCMTINFORM-
\newline {\it Password:} No password required; captive account

Log on to a local Starlink Unix system and type either:

\vspace{2.0 mm}
\verb:telnet star.roe.ac.uk: ~~ or ~~ \verb:telnet 192.108.120.10:
\vspace{2.0 mm}

In response to the `Username:' prompt reply {\tt JCMTINFORM}; no
password is required. The facilities available include:

\begin{itemize}

  \item recent articles from the JCMT newsletter,

  \item several useful utility programs,

  \item a version of the current JCMT observing schedule,

  \item access to the JCMT data archive (see \xref{SUN/162}{sun162}{}).
   Note that access
   to the archive is under the menu for the utility programs.

\end{itemize}

Further information may be obtained from Graeme Watt, e-mail:
GDW@star.roe.ac.uk
\newline (INTERNET).


\section{NISS
\xlabel{niss}\label{NISS}}

National Information Services and Systems (NISS) is an extensive
information system available at the University of Bath. It provides
access to a large number of information services and databases,
including a number of bibliographic databases. None of this information
is specifically astronomical, but nonetheless much of it is potentially
of interest.

NISS is accessed via the United Kingdom JANET academic network and via
INTERNET. It can be accessed from sites outside the United Kingdom. The
details are as follows:

{\it INTERNET address:} niss.ac.uk or 193.63.76.2
\newline {\it Username:} No username required; captive account
\newline {\it Password:} No password required; captive account

From a Starlink Unix system you would type:

\vspace{2.0 mm}
\verb:telnet niss.ac.uk: ~~ or ~~ \verb:telnet 193.63.76.2:
\vspace{2.0 mm}

No username or password is required and you will be presented with a
simple menu. The main items on this menu correspond to the various
services offered by NISS. They are listed in Table~\ref{NISS_OPT}
and a brief explanation for each is given below.

\begin{table}[htbp]

\begin{center}
\begin{tabular}{l}
The NISS Bulletin Board (NISSBB)  \\
The NISS Public Access Collections (NISSPAC)  \\
The NISS Wide Area Information Server (NISSWAIS)  \\
NISS Newspapers and Journals Services  \\
  \\
Library Catalogues (OPACs)  \\
Campus Information Systems  \\
Bibliographic Services (for example, Melvyl, CARL, STN, BUBL)  \\
Directory Services (for example, Yellow Pages, Paradise, WAIS)  \\
Archive Services (for example, HENSA, Mailbase, BIRON)  \\
General Services (for example, Guest-Telnet, ASK, CONCISE)  \\
\end{tabular}
\end{center}

\caption{NISS services}
\label{NISS_OPT}

\end{table}

\paragraph{The NISS Bulletin Board (NISSBB)}
is intended to facilitate the exchange of 
information within the United Kingdom Higher Education Community.
Amongst the topics covered are: the use of computers in education,
training and seminars and jobs.

\paragraph{The NISS Public Access Collections (NISSPAC)}
provides on-line catalogues of software and datasets held at
United Kingdom Higher Education Institutions. Currently three such 
catalogues are available:

\begin{itemize}

  \item the NISS Software and Datasets Catalogue,

  \item the VAX/VMS Applications Software Directory,

  \item the CHEST (Combined Higher Education Software Team) Directory; 
   a list of commercial software packages for which academic discounts 
   are available.
 
\end{itemize}

\paragraph{The NISS Wide Area Information Server (NISSWAIS)}
is an alternative method of searching the NISS collections using WAIS
(see Section~\ref{WAIS}).

\paragraph{NISS Newspapers and Journals Services}
provides on-line access to back-issues of newspapers and journals. 
Currently only {\it The Times}\, newspaper is available. The text for 
the current edition is not available. Since mid-April 1993 editions
have been added at about 11:00pm on the day after publication. Editions
of the {\it The Sunday Times}\, are added on the Wednesday following
publication. This service is available under an agreement between CHEST
and the American Cybercasting Corporation (ACC).

\paragraph{Library Catalogues (OPACs)}
provides access to On-line Public Access Catalogues (OPACs) at 
educational establishments. Catalogues are available for a substantial 
number of United Kingdom Higher Education Institutions.

\paragraph{Campus Information Systems}
provides access to Campus Wide Information Systems 
\newline (CWISs) at 
educational establishments. Catalogues are available for a significant 
number of United Kingdom Higher Education Institutions.

\paragraph{Bibliographic Services}
provides access to a number of bibliographic databases, covering, for
example, medical and chemical subjects (Data-Star), library holdings
(OCLC, Melvyl), and mathematics communications and publishing
({\bf e-MATH}). You have to register to use some of these databases. 
Instructions for registration are included in the on-line help 
information.

\paragraph{Directory Services}
provides access to a number of directory services, 
including information about the European Economic Community (ECHO), 
worldwide information on computing organizations (PARADISE X.500) and
distance teaching and education information (ICDL). There is also access
to Archie (see Section~\ref{ARCHIE}) and WAIS (see Section~\ref{WAIS}).
However, WAIS is accessed using the {\tt swais} client which is 
difficult and unfriendly to use.

\paragraph{Archive Services}
provides access to a number of archives, including PC and Unix software 
archives (HENSA and UKUUG respectively) and the information exchange 
service MailBase.

\paragraph{General Services}
provides access to a miscellany of services, including the on-line 
information service for German Universities (ASK), the e-mail and 
computer conferencing service EuroKOM and news about the JANET network.

\vspace{7.0 mm}

Inquiries about NISS services should be sent to the appropriate
e-mail address in Table~\ref{NISS_INQ}.

\begin{table}[htbp]

\begin{center}
\begin{tabular}{ll}
General queries    &  niss@niss.ac.uk     \\
NISSBB             &  nissbb@niss.ac.uk   \\
NISSPAC            &  nisspac@niss.ac.uk  \\
NISS Gateway       &  gateway@niss.ac.uk  \\
NISSWAIS           &  nisswais@niss.ac.uk \\
\end{tabular}
\end{center}

\caption{E-mail addresses for inquiries about NISS services}
\label{NISS_INQ}

\end{table}


\section{General Summaries of INTERNET facilities
\xlabel{general_summaries_of_internet_facilities}}

A large number of services providing a great deal of diverse information
can be accessed via the INTERNET. Most of this information is 
non-astronomical, though much of it is potentially useful. Even a
summary of the services available is beyond the scope of this
document. However, various summaries are available. Some useful and
readily available ones are listed below. The December and Yanoff lists
are updated frequently and distributed widely. Copies can be obtained
from a number of sources. However, the sources given are maintained
by the authors, and thus by using them you will retrieve the most recent
version of the lists.

\paragraph{The Whole Internet User's Guide and Catalog} (see 
Section~\ref{FURTHER}). The final part of this book is an extensive
catalogue of INTERNET information services.

\paragraph{The December List} This list is compiled by John December
(e-mail: decemj@rpi.edu, 
\newline INTERNET). It provides an extensive set of
references and other pointers to sources of information about the
INTERNET, computer networks in general and computer-mediated 
communication. A copy may be retrieved by anonymous ftp (see
Section~\ref{ANON_FTP}). The details are as follows:

{\it INTERNET address: } ftp.rpi.edu or 128.113.1.5
\newline {\it Username:} \verb-anonymous-
\newline {\it Password:} Type your username and e-mail address

The list is held in file {\tt pub/communications/internet-cmc}.
\newline File {\tt pub/communications/internet-cmc.readme} may also be 
of interest.

\paragraph{The Yanoff List} This list is compiled by Scott Yanoff
(e-mail: yanoff@csd4.csd.uwm.edu, INTERNET). It comprises an extensive
list of on-line information services, arranged by subject. There are
several ways to retrieve this list, though perhaps the most convenient
is anonymous ftp (see Section~\ref{ANON_FTP}). The details are as 
follows:

{\it INTERNET address: } csd4.csd.uwm.edu or 129.89.7.4
\newline {\it Username:} \verb-anonymous-
\newline {\it Password:} Type your username and e-mail address

The list is held in file {\tt pub/inet.services.txt}. For details
of other ways of retrieving the list type:

\begin{verbatim}
finger  yanoff@csd4.csd.uwm.edu
\end{verbatim}

\paragraph{Internet Resources Guide}

The United States National Science Foundation (NSF) has compiled
an extensive {\it Internet Resources Guide}. This document is currently
available by anonymous ftp (see Section~\ref{ANON_FTP}). The
details are:

{\it INTERNET address:} ds.internic.net or 198.49.45.10
\newline {\it Username:} \verb-anonymous-
\newline {\it Password:} Type your username and e-mail address

The {\it Guide} is held as file {\tt 
/resource-guide/resource-guide.ps.tar.Z}. It is a PostScript file
which has been compressed and entered into a tar archive. See 
Chapter Six of {\it The Whole Internet User's Guide and Catalog}\, 
(see Section~\ref{FURTHER}) for a description of how to un-compress
and unpack such a file. However, before  you retrieve and print
out a copy you should realise that it is a weighty tome.
%  of some XXX pages.

Files containing the individual chapters are also available, and you
could retrieve these if you are only interested in a particular topic.
Table~\ref{IRG_CHAP} lists The topics covered in each chapter. These 
files are also kept in directory {\tt /resource-guide}.

\begin{table}[htbp]

\begin{center}
\begin{tabular}{lc}
Topic             & Chapter  \\ \hline
Computer Centres  & 1  \\
Libraries         & 2  \\
Archives          & 3  \\
Directories       & 4  \\
Service providers & 5  \\
NIC               & 6  \\
Miscellaneous     & M  \\
\end{tabular}
\end{center}

\caption{Chapters of the NSF {\it Internet Resources Guide}}
\label{IRG_CHAP}

\end{table}

The {\it Internet Resource Guide} will continue to be available for some
time, but there will be no further development of it. It will be 
replaced by an on-line system which can be searched for information on
a specific topic. For further information on these developments send an
e-mail message to admin@ds.internic.net (INTERNET).


\section{Public Domain Software
\xlabel{public_domain_software}}

A large amount of public domain software is available on-line across
the INTERNET. The software available includes both specifically 
astronomical software and more general software which may be of interest
to astronomers. Finding and obtaining public domain software is beyond 
the scope of this document, but an excellent introduction is given by 
E.D. Feigelson and F. Murtagh (1992, {\it PASP}, {\bf 104}, pp574-581).

Various public domain software archives may be accessed through the
{\tt Archive Services} menu of NISS (see Section~\ref{NISS}). Archie
(see Section~\ref{ARCHIE}) is useful for searching for software.

Users outside the United Kingdom might be interested to know that
most of the Starlink Software Collection is publicly available, free of
charge, for use by astronomers in non-commercial, non-profit making
research. The items which cannot be supplied are commercial products
such as the NAG numerical algorithms library. For further details 
contact the Starlink Software Librarian, Rutherford Appleton Laboratory,
R68, Chilton, DIDCOT, Oxfordshire, OX11 0QX, United Kingdom, e-mail:
star@star.rl.ac.uk (INTERNET).


\pagebreak
\part{INTERNET Tools}
\label{INTERNET_TOOLS}

\section{Overview\xlabel{overview}}

A number of tools are available for accessing information held on remote
computers linked to your local machine by the INTERNET 
network. These tools range from simple simulated remote logins which 
allow you to examine and retrieve remote files to complex hypertext 
browsers which allow you to read distributed, structured documents whose
components are scattered around the world. Some of the facilities 
described in the previous sections are accessed using these tools; for
example the STELAR database (see Section~\ref{STELAR}) is queried using 
the tool WAIS. The tools also allow access to a plethora of other 
information, most of it non-astronomical. Necessarily this section 
presents only a very brief summary of the tools, orientated towards
providing enough information to allow you to use them to access the
facilities described in previous sections. More thorough discussions
are given in {\it The Whole Internet User's Guide and Catalog}\, and
{\it Intelligent Information Retrieval: The Case of Astronomy and 
Related Space Sciences}\, (see Section~\ref{FURTHER}).

In order to use many of the tools you must run an appropriate `client'
program which interacts with the remote facility. You can run a client
in one of two ways: either install an appropriate client on your local
Starlink machine or login remotely to a machine which already has a
client installed. The former option is preferable. There are perfectly 
adequate, public domain versions of the clients available free of 
charge and there is no reason not to install them on your local 
Starlink machine. Usually these clients are for Unix systems. Similarly,
in practice, most of the remote facilities will run on Unix systems, 
though many different sorts of computers can be connected to the 
INTERNET, so occasionally you may encounter a non-Unix machine.

Clients for some or all of the tools may already be installed at your 
site; your site manager should be able to advise. Some of the following
sections give explicit instructions for installing clients. If no such
instructions are given you can use Archie (see Section~\ref{ARCHIE}) to
find a suitable client, though in practice you will probably find one in
the Unix public domain software archive maintained at Imperial College
London. However, it is neither possible nor desirable for you to install
clients yourself. You should consult your site manager and, if a client
is not already installed, he will locate and install one on your behalf
(provided, of course, that he judges that it will be compatible with the
proper operation of your Starlink node). As usual for Unix there are 
several different versions of each client: one for X-terminals, one for 
dumb terminals, another which provides extra features etc. Your site
manager will choose one which bests suits local conditions. 

Finally, many of these tools are relatively new and are still being 
enhanced. You will sometimes encounter problems when using them: {\it 
caveat emptor}.


\section{Anonymous ftp
\xlabel{anonymous_ftp}\label{ANON_FTP}}

Many of the facilities described in Parts I and II allow you to retrieve
files from them using the so-called `anonymous ftp' mechanism. ftp
(file transfer protocol) is an INTERNET facility for copying files
between computers. Normally it requires you to have a username on
both the host and destination computers. Anonymous ftp is a variant 
which allows you to retrieve files from a remote computer without having
a username on that computer. A full description of anonymous ftp is
beyond the scope of this document. However, a brief summary is given 
below.

ftp is best used from a Starlink Unix system. Type:

\vspace{2.0 mm}
{\tt ftp} {\it remote-address}
\vspace{2.0 mm}

where {\it remote-address} is the INTERNET address of the remote system
which you wish to access. Either the alphabetic or numeric version of 
the INTERNET address may be used. For example, to connect to STEIS (see
Section~\ref{STEIS}) you would type:

\vspace{2.0 mm}
\verb:ftp stsci.edu : ~~ or ~~ \verb:ftp 130.167.1.2:
\vspace{2.0 mm}

If a connection can be established successfully you will be prompted to 
supply a username (or login name) and password. You should respond as 
follows:

{\it Login:} \verb-anonymous-
\newline {\it Password:} Type your username and e-mail address

Most anonymous ftp sites will also accept {\tt ftp} as the login name.
By convention your username and e-mail address should be given in the
form `username@address', for example 
\newline {\tt acd@star.le.ac.uk}. You are
now ready to examine and retrieve files from the remote system. Various
versions of the ftp program are available, and the commands differ 
slightly between them. However, the following frequently used commands,
some of which are similar to the corresponding Unix commands, are 
usually available.

\begin{description}

  \item[{\tt cd} {\it directory}] -- select directory {\it directory}\, 
   as the current directory on the remote machine\footnote{Directories 
   are usually specified using the Unix notation. If the remote system 
   is not running Unix you may have to use the notation appropriate to 
   its operating system.}.

  \item[{\tt dir}] -- list the files in the current directory.

  \item[{\tt get} {\it filename}] -- retrieve a copy of file {\it 
   filename}\, to your local computer. ftp should be in ASCII mode when
   retrieving text files and in binary mode when retrieving binary
   files (see commands {\tt ascii} and {\tt binary}). A description of
   compressed and archived (`tar') files is beyond the scope of this
   document; see Chapter Six of {\it The Whole Internet User's Guide 
   and Catalog}\, (see Section~\ref{FURTHER}). However, it is worth
   noting that ftp should be in binary mode when retrieving such files.

  \item[{\tt pwd}] -- display the current directory.

  \item[{\tt ascii}] -- switch to ASCII mode; ASCII is the default mode.

  \item[{\tt binary}] -- switch to binary mode.

  \item[{\tt help}] -- will often display information about the commands
   available.

  \item[{\tt quit}] -- exits ftp.

\end{description}

\pagebreak
A useful trick for displaying the contents of a remote file on your
terminal is the command:

\begin{verse}
{\tt get} {\it filename}\, {\tt "$|$ more"}
\end{verse}

{\it Documentation:}

ftp is described in Chapter Six of {\it The Whole Internet 
User's Guide and Catalog}, Chapter Six (see Section~\ref{FURTHER}).

Some additional useful documents can be obtained by anonymous ftp
from Dartmouth College, Hanover, New Hampshire. The details are:

{\it INTERNET address:} pilot.njin.net or 128.6.7.38
\newline {\it Username:} \verb-anonymous-
\newline {\it Password:} Type your username and e-mail address

Once connected you should select directory {\tt pub/ftp-list}. The
following files may be useful.

\begin{description}

  \item[{\tt ftp.help}] An introduction to using ftp (unfortunately you
   have to be at least moderately familiar with ftp to retrieve a copy).

  \item[{\tt suggestions}] A set of useful hints about setting up your 
   own anonymous ftp site.

  \item[{\tt ftp.list}] A list of anonymous ftp sites, with a summary of
   the type of software available at each site.  There are entries for
   about 900 sites. Maintenance of the list was discontinued in
   December 1991 because the development of systems like Archie (see 
   Section~\ref{ARCHIE}) reduced the need for it.

\end{description}


\section{Archie
\xlabel{archie}\label{ARCHIE}}

Archie is a system for locating remote files. It allows you to search 
for files which are available by anonymous ftp throughout INTERNET,
without having to know the details of the machines and directories where
they are held. Using Archie it appears to be possible to search all the
anonymous ftp directories as though they were a collection of 
directories on your local machine. Archie works as follows. There are
a number of Archie servers around the world. Each server knows about the
various anonymous ftp directories. It regularly interrogates these
directories to determine which files they contain, and saves the result.
When you run Archie you interrogate these lists of files. All the 
Archie servers are identical.

Archie is easier to use if you have some idea of what the file you are
looking for is called. It is probably most useful (and used) for finding
software.

If you access Archie from a local client the details will be specific
to your site; your site manager should be able to advise. If you are 
using an X interface then first you must redirect X output to your 
terminal. The details will vary, but your site manager should be able 
to advise. Then type perhaps:

\vspace{2.0 mm}
\verb:xarchie: ~~ or ~~ \verb:archie:
\vspace{2.0 mm}

the former invokes a client with an X-terminal interface. If you do not
have a local client you must run Archie remotely from a server. The
most convenient server for Starlink users is at Imperial College London.
To access it type:

\vspace{2.0 mm}
\verb:telnet archie.doc.ic.ac.uk: ~~ or ~~ \verb:telnet 146.169.11.3: 
\vspace{2.0 mm}

Users outside the United Kingdom might find another server more 
convenient. Table~\ref{ARCHIE_SERVE} gives a list of them. Usually you
will choose the one geographically closest to you. Remember that they
are all equivalent.

\begin{table}[htbp]

\begin{center}
\begin{tabular}{lll}
Country              & INTERNET address      & INTERNET number \\ \hline
Australia            & archie.au                  & 139.130.4.6  \\
Austria              & archie.edvz.uni-linz.ac.at & 140.78.3.8  \\
Austria              & archie.univie.ac.at        & 131.130.1.23  \\
Canada               & archie.uqam.ca             & 132.208.250.10  \\
Finland              & archie.funet.fi            & 128.214.6.100  \\
Germany              & archie.th-darmstadt.de     & 130.83.22.60  \\
Israel               & archie.ac.il               & 132.65.6.15  \\
Italy                & archie.unipi.it            & 131.114.21.10  \\
Japan                & archie.wide.ad.jp          & 133.4.3.6  \\
Korea                & archie.kr                  & 128.134.1.1  \\
Korea                & archie.sogang.ac.kr        & 163.239.1.11  \\
Spain                & archie.rediris.es          & 130.206.1.2  \\
Sweden               & archie.luth.se             & 130.240.18.4  \\
Switzerland          & archie.switch.ch           & 130.59.1.40  \\
Taiwan               & archie.ncu.edu.tw          & 140.115.19.24  \\
{\bf United Kingdom} & {\bf archie.doc.ic.ac.uk}  & {\bf 146.169.11.3}\\
United States (Maryland)    & archie.sura.net     & 128.167.254.179  \\
United States (New England) & archie.unl.edu      & 129.93.1.14  \\
United States (New Jersey)  & archie.internic.net & 198.48.45.10  \\
United States (New Jersey)  & archie.rutgers.edu  & 128.6.18.15  \\
United States (New York)    & archie.ans.net      & 147.225.1.10  \\
\end{tabular}
\end{center}

\caption{Archie servers}
\label{ARCHIE_SERVE}

\end{table}

In response to the `login:' prompt reply {\tt archie}; no password is
required. Once you are logged on, type {\tt help} for information on the
commands available. A full description of Archie is beyond the scope of 
this document. However, a brief introduction is given below. This
discussion refers to the dumb-terminal Archie client, which is the 
version accessed remotely at Imperial College. The X-terminal client has
the same functionality, though the interface is different.

Suppose that you were searching for an archie client to install
on your local machine, and guessed that such a client would have the
string `archie' as part of its name. You would type:

\begin{verbatim}
set search sub
\end{verbatim}

to specify that searches were to operate on substrings, rather than
requiring an exact match to the entire file name. Then perform the
search:

\begin{verbatim}
prog archie
\end{verbatim}

and a list of files (and their locations) which match the criterion will
be displayed. You can mail a copy of this back to your local computer
so that you can examine it at your leisure by:

\vspace{2.0 mm}
{\tt mail} ~~ {\it your-INTERNET-e-mail-address}
\vspace{2.0 mm}

for example {\tt mail acd@star.le.ac.uk}. When you have finished, type
{\tt quit} to exit Archie. {\tt whatis} is another useful command. It
searches a database containing a description of each file known to 
Archie and can be used to find files relating to some topic. Once you 
have found the names of such files you can locate them in the usual way 
using {\tt prog}.

{\it Documentation:}

Archie is described in Chapter Nine of {\it The Whole Internet 
User's Guide and Catalog}\, (see Section~\ref{FURTHER}).

It is also described by E. Emtage in Chapter Seven of {\it Intelligent 
Information Retrieval: The Case of Astronomy and Related Space 
Sciences}\, (see Section~\ref{FURTHER}).


\section{Gopher
\xlabel{gopher}\label{GOPHER}}

Gopher is a tool which allows you to browse through the INTERNET 
identifying and displaying information of interest. Its name is a pun
on this purpose; its function is to `go fer' things. Gopher presents
remote directories as though they were a hierarchy of local directories 
or menus and allows you to navigate through them whilst hiding the 
details of the addresses of the remote machines and the directory paths.
Once you locate files of interest you can display them; text files will 
be listed and more capable, X-terminal based gopher clients can also 
display images. Files may also be retrieved to your local computer. 
Gopher does not allow you to access any information which could not be 
made available using anonymous ftp or {\tt telnet}, but it does make 
accessing this information much simpler.

There are various gopher servers around the world. They are all 
inter-linked, so it does not much matter which one you access first,
though if you know that you are going to access a given facility, such
as STEIS (see Section~\ref{STEIS}), you may as well go straight to the
appropriate server. If you access gopher from a local client the 
details will be specific to your site; your site manager should be able 
to advise. If you are using an X interface then first you must redirect 
X output to your terminal. The details will vary, but your site manager 
should be able to advise. Then type perhaps:

\vspace{2.0 mm}
\verb:xgopher: ~~ or ~~ \verb:xgopher stsci.edu 70:
\vspace{2.0 mm}

The former accesses the default server for your client, the latter a
specified server. If you do not have a local client you must run gopher 
from a public access client on a remote computer. A list of public 
access clients is given in Table~\ref{GOPHER_PUBLIC}. For example, to
access the European public client you would type:

\vspace{2.0 mm}
\verb:telnet gopher.sunet.se: ~~ or ~~ \verb:telnet 192.36.125.2:
\vspace{2.0 mm}

In response to the `login:' prompt reply {\tt gopher}; no password is
required.

\begin{table}[htbp]

\begin{center}
\begin{tabular}{llll}
Area          & INTERNET address     & INTERNET number & Login \\ \hline
Australia     & info.anu.edu.au          & 150.203.84.20  & info    \\
Ecuador       & ecnet.ec                 & 157.100.45.2   & gopher  \\
{\bf Europe} & {\bf gopher.sunet.se} & {\bf 192.36.125.2} & {\bf gopher} \\
North America & consultant.micro.umn.edu & 134.84.132.4   & gopher  \\
North America & gopher.uiuc.edu          & 128.174.33.160 & gopher  \\
North America & panda.uiowa.edu          & 128.255.40.201 & panda   \\
South America & tolten.puc.cl            & 146.155.1.16   & gopher  \\
Sweden        & gopher.chalmers.se       & 129.16.221.40  & gopher  \\

\end{tabular}
\end{center}

\caption{Public gopher clients}
\label{GOPHER_PUBLIC}

\end{table}

Gopher allows you to navigate through the directory tree on the
server you are using. When you encounter files rather than directories
you may examine them and retrieve copies. Somewhere in the directory
tree for the server (usually quite close to the top) there will be an
entry called something like {\tt Other Gopher and Information Servers}.
Selecting this entry displays details of other gopher servers and 
allows you to select one of them (in a transparent way which gives the
appearance that you are just navigating to another directory). You may
then examine the directories of this server, or proceed from it to yet
another server.

{\it Documentation:}

Gopher is described in Chapter Eleven of {\it The Whole Internet 
User's Guide and Catalog}\, (see Section~\ref{FURTHER}).

It is also described by F. Anklesaria and M. McCahill in Chapter Nine
of {\it Intelligent Information Retrieval: The Case of Astronomy and 
Related Space Sciences}\, (see Section~\ref{FURTHER}).

A list of frequently asked questions (FAQs in INTERNET jargon) about
gopher is available by anonymous ftp from STEIS (see 
Section~\ref{STEIS_FTP}). It is held as file {\tt .Gopher-FAQ} in the 
top level directory. Note that because the file name begins with a dot 
it will not appear in normal Unix directory listings.


\section{WAIS
\xlabel{wais}\label{WAIS}}

Wide Area Information Servers (WAIS) is another tool for locating,
searching and retrieving remote information. It is based on the ANSI
Z39.50 draft standard for remote searches. It is usually used for 
searching text, but can be used for any information for which an 
appropriate index has been built. WAIS is the primary interface to 
several of the bibliographic databases described in Part I. This note
concentrates on using WAIS to access these databases. A more 
comprehensive (and balanced) description is given in {\it The Whole 
Internet User's Guide and Catalog}, Chapter Twelve (see 
Section~\ref{FURTHER}).

A WAIS client may already be installed at your site; your site manager
should be able to advise. If your site does not have a client installed
you can obtain one by anonymous ftp (see Section~\ref{ANON_FTP}). The 
details are as follows:

{\it INTERNET address:} think.com or 131.239.2.1
\newline {\it Username:} \verb-anonymous-
\newline {\it Password:} Type your username and e-mail address

You should copy file {\tt /public/wais/wais-8-b5.tar.Z}. Installation
instructions are included in this tar file.

Alternatively, you could use Archie (see Section~\ref{ARCHIE}) to locate
a suitable client. Note however that the dumb-terminal client {\tt
swais} is particularly unfriendly and difficult to use; the X-terminal
version is much to be preferred. Also note that you cannot install the
client yourself; your site manager must do so on your behalf.

WAIS tries to hide the details of where the server which hosts the
documents that you are searching is located. It is possible to search
a `directory of servers' to find suitable servers, and then interrogate
them in greater detail to find the information that you are looking for.
However, if you know you want to search one of the bibliographic 
databases described in Part I you can bypass this step. The procedure 
is as follows.

\begin{enumerate}

  \item Create a directory called {\tt wais-sources} as a subdirectory 
   of your login directory.

  \item For each database that you are going to access you need to 
   insert in this directory a `source file' describing how to access
   it. By convention these source files have file type {\tt .src}.

   Fortunately copies of the relevant source files are kept at the ESA 
   Space Telescope European Coordinating Facility, Garching bei 
   M\"{u}nchen, from where copies may be retrieved by anonymous ftp
   (see Section~\ref{ANON_FTP}). The details are:

   {\it INTERNET address:} ecf.hq.eso.org  or 134.171.11.4
   \newline {\it Username:} \verb-anonymous-
   \newline {\it Password:} Type your username and e-mail address

   The source files are in directory {\tt 
   pub/swlib/various-wais-sources}

  \item Table~\ref{WAIS_SOURCE} gives the names of the source files
   for the various bibliographic databases. You may as well retrieve
   copies of all three because you will end up trying them all. Also
   it is worth quickly perusing the directory because you might spot 
   some other source files that look interesting.

\end{enumerate}

\begin{table}[htbp]

\begin{center}
\begin{tabular}{lcl}
Database            &  Section      & Source file \\   \hline
STELAR              & \ref{STELAR}  & {\tt abstracts.src} \\
STScI preprint list & \ref{STSCI}   & {\tt stsci-preprint-db.src} \\
NRAO preprint list  & \ref{NRAO}    & {\tt nrao-raps.src} \\
\end{tabular}
\end{center}

\caption{WAIS source files}
\label{WAIS_SOURCE}

\end{table}

You are now ready to use WAIS. You will probably run WAIS from an 
X-terminal and first you must redirect X output to your terminal. The 
details will vary, but your site manager should be able to advise. Then 
type perhaps:

\begin{verbatim}
xwais &
\end{verbatim}

An outline of the procedure to find articles is as follows (this
description assumes that you are using the X-terminal WAIS client).

\begin{enumerate}

  \item Click on the `new question' button. This button is labelled
   {\tt New} and is enclosed in the {\tt Question} box, in the upper
   half of the window.

  \item A new window should appear. Click on the button marked {\tt
   Add source}. Yet another window with a list of all your sources 
   should appear.

  \item Choose the source corresponding to the database which you wish
   to search by double-clicking on it. For example, double-click on
   {\tt nrao-raps.src} to choose the NRAO preprint lists. This source 
   should then be listed as an active source.

  \item In the box labelled {\tt Tell me about:} enter the words which
   you wish to search for. You can enter as many words as you like, 
   separated by spaces. The order of the words is not important. Hit 
   return to start the search.

  \item A list of articles which satisfy your criteria will be 
   displayed. You can select an article by clicking on it. There are
   facilities for displaying all the information about the article
   (which will include the abstract in the case of STELAR) and saving
   the details as a text file on your local machine. Such files are
   created in directory {\tt wais-documents}, which is a subdirectory
   of your login directory.

\end{enumerate}

There are a number of problems or restrictions with the WAIS software
which you should be aware of. Firstly, because WAIS is not designed
specifically for bibliographic searches it does not contain facilities
to limit the search to a given journal or range of years.

The search software looks for articles containing all the words you
have supplied. That is, the words are combined using a logical (or 
boolean) `and' operation; there are no facilities for making more
complex searches involving other logical operators, such as `or' or 
`not'\footnote{Standard WAIS clients such as {\tt xwais} do not support
boolean operators or `wild cards' in searches. However, recent versions
of the Z39.50 standard include these concepts. Some more recent clients 
and servers which understand the revised protocol provide these 
facilities.}. However, the search will find articles containing most of 
the words specified, as well as articles containing all of them. For 
each article it selects, WAIS makes an estimate of how well the article
matches the search criteria you specified. This estimate is expressed
as a number in the range 0 to 1000. This feature is known as `relevancy
feedback'. The searches match only exact words, not substrings which are
part of a word. This behaviour can cause problems because of plurals and
variant spelling. As an example of the sort of problem which can arise, 
suppose you were searching for a catalogue of multi-colour photometry of
planetary nebulae. As the search string you might enter: `catalogue 
multi-colour photometry planetary nebulae'. Rendered into American this 
search string becomes `catalog multi-color photometry planetary 
nebulas'\footnote{I am not being entirely fair here: `nebulae' is the 
preferred plural in both American and British English, but, sadly, 
`nebulas' often occurs in both.}. Because WAIS will still select an 
article even if not all the words in the search string occur in it, 
the recommended procedure is to enter all relevant variant spellings.

Finally, the WAIS client is rather poor at reporting errors, 
particularly when it cannot establish a connection to a remote server.
Some of the bibliographic databases are quite often unavailable (at
least, when accessed from the United Kingdom); if you appear not to
be able to get through you should simply try again later.

{\it Documentation:}

\nopagebreak
WAIS is described in Chapter Twelve of {\it The Whole Internet
User's Guide and Catalog}\, (see Section~\ref{FURTHER}).

It is also described by J. Fullton in Chapter Eight of {\it Intelligent 
Information Retrieval: The Case of Astronomy and Related Space 
Sciences}\, (see Section~\ref{FURTHER}).

Thinking Machines Corporation has compiled a bibliography of WAIS.
It is available via anonymous ftp (see Section~\ref{ANON_FTP}). The 
details are:

{\it INTERNET address:} wais.com or 192.216.46.98
\newline {\it Username:} \verb-anonymous-
\newline {\it Password:} Type your username and e-mail address

It is held in file {\tt /pub/wais-inc-doc/bibliography.txt}.

A list of frequently asked questions (FAQs in INTERNET jargon) about
WAIS is available by anonymous ftp from STEIS (see 
Section~\ref{STEIS_FTP}). It is held as file {\tt .WAIS-FAQ} in the 
top level directory. Note that because the file name begins with a dot 
it will not appear in normal Unix directory listings.


\section{World Wide Web
\xlabel{world_wide_web}\label{WEB}}

The World Wide Web (WWW, or just `the Web') is a hypertext system in
which components of documents can be distributed around the INTERNET.
A full description of either the Web or hypertext is beyond the scope
of this document. However, briefly, hypertext offers an alternative to
traditional, linear narrative text. In a hypertext document the author
selects certain words or phrases as `links'. Links are highlighted in
some way when a hypertext document is displayed and a reader may choose
to display further information on a link, rather than continuing to
read the document sequentially. If he does so, further information
pertaining to, or describing the link will be displayed. This text may
itself contain further links. Thus a reader may `navigate' through a
hypertext document, investigating the topics which interest him to the
depth that he chooses\footnote{In a large hypertext document with many
links it is possible for a reader to loose track of where he is. This
unfortunate condition is aptly referred to as being `lost in 
hyperspace'.}. The first page to appear when a hypertext client (or 
`browser' in hypertext terminology) is started is called the `home
page'. Usually the home page contains introductory material.

The novel concept of the Web is that components of a Web
document may be scattered around the INTERNET, rather than limited to
a single machine. Also, Web links may lead to WAIS indices, gopher
menus and {\tt telnet} connections to remote facilities.

You will probably access the Web from an X-terminal. First you must
redirect X output to your terminal. The details will vary, but your site
manager should be able to advise. Then you will probably type something
like:

\begin{verbatim}
xmosaic
\end{verbatim}

The home page will be displayed and you can either follow the linear
narrative of the text, or explore links, as you choose. In the {\tt
xmosaic} browser links are either shown in a different colour, or by 
underlining (depending on whether or not you are using a colour 
X-terminal). On the basis of limited experience, compared to the clients
for some of the other tools, {\tt xmosaic} seems to make more onerous
demands on the local computer, so the performance may be somewhat 
sluggish.

Your client will be set up to access the home page of a given document
by default. Every Web document has a unique `Uniform Resource Locator'
(URL\footnote{Previously the acronym stood for `Universal Resource 
Locator'.}) which identifies it. You can control the default document
by specifying the URL. For {\tt xmosaic} the default URL is specified
by setting an environment variable prior to starting the browser. For
example:

\begin{verbatim}
setenv  WWW_HOME  http://stsci.edu/net-resources.html
\end{verbatim}

(note that the last character is the letter `l', not the digit `1').
This URL specifies the index of Astronomical INTERNET resources 
available as part of STEIS (see Section~\ref{STEIS_WEB}).

Much of the development of the Web occurred at the European Centre for 
Nuclear Research (CERN), Geneva. This institution runs a public access
Web browser. If a client is not available at your site you can access it
to try out the Web. From a Starlink Unix machine type:

\vspace{2.0 mm}
\verb:telnet info.cern.ch: ~~ or ~~ \verb:telnet 128.141.201.74:
\vspace{2.0 mm}

You will be logged on to the client without having to give a username
or password. This client has a simple user interface suitable for
dumb terminals. Links are indicated by numbers in square brackets
(`[~]') immediately after the linked text. You simply type the number
of the link to proceed to its text, or hit return to carry on with the
main document. One route in the CERN client which may be of interest is 
to proceed first to the {\tt The Virtual Library} link and 
thence to the {\tt Astronomy and Astrophysics} link.

{\it Documentation:}

The Web is described in Chapter Thirteen of {\it The Whole Internet
User's Guide and Catalog}\, (see Section~\ref{FURTHER}).

It is also described by B. White in Chapter Ten of {\it Intelligent 
Information Retrieval: The Case of Astronomy and Related Space 
Sciences}\, (see Section~\ref{FURTHER}).

A list of frequently asked questions (FAQs in INTERNET jargon) about
the Web is available by anonymous ftp from STEIS (see 
Section~\ref{STEIS_FTP}). It is held as file {\tt .WWW-FAQ} in the 
top level directory. Note that because the file name begins with a dot 
it will not appear in normal Unix directory listings.

{\it Getting Started with NCSA Mosaic}\, by M. Andreessen, May 1993 is
an introductory article about the {\tt xmosaic} client. However, it is
intended more for site managers who have to install it than for users.
It includes details of how to obtain a copy of the client. It is 
available through Starlink as MUD/147.

{\it Hypertext and Hypermedia} by J. Nielsen, 1990 (Academic Press,
San Francisco). This book is about hypertext in general, rather than 
being specifically about the Web.

The issue of {\it Communications of the ACM}\, for July 1998 contained a
special section devoted to hypertext. However, it was about hypertext 
in general, rather than specifically about the Web (indeed, it predates 
the Web).

\section{The Relation Between Gopher, WAIS and WWW
\xlabel{the_relation_between_gopher_wais_and_www}}

Gopher, WAIS and the Web are all similar in that they are all 
sophisticated tools for interrogating and retrieving remote 
information, and all are implemented using a `client-server' model. In
fact, they are more similar than might be apparent from their 
description in the previous sections because if a facility is accessible
by one of the tools it is probably accessible by the other two. 
Specifically:

\begin{itemize}

  \item gopher can search WAIS databases (though the relevancy feedback 
   is lost),

  \item the Web can search gopher and WAIS databases (preserving any 
   feedback).

\end{itemize}

The difference between the tools is the paradigm they use to represent
information.

\begin{description}

  \item[Gopher] represents each item of information as either a menu,
   a document, an index or a {\tt telnet} connection.

  \item[WAIS] represents all information as an index, and everything
   selected by searching an index is a document.

  \item[The Web] represents all information as a hypertext document,
   which optionally may have links to other documents, and which may
   be searchable.

\end{description}

The different paradigms are best suited to different sorts of tasks and
the tools complement each other.

\pagebreak
\part{~}

\section{Further Reading
\xlabel{further_reading}\label{FURTHER}}

\begin{itemize}

  \item {\it The Whole Internet User's Guide and
   Catalog}\, by Ed Krol, 1992 (O'Reilly and Associates Inc, 
   Sebastopol, California). This book is an excellent introduction to
   the INTERNET and the tools available for it.

  \item {\it Intelligent Information Retrieval: The Case of Astronomy
   and Related Space Sciences}\, eds. A. Heck and F. Murtagh, 1993
   (Kluwer, Dordrecht). This book covers much the same field as the 
   present document, but is more concerned with the principles behind
   the facilities provided, rather than being a cookbook of what is
   available and how to access it.

  \item {\it A Guide to Astronomical Catalogues, Databases and Archives
   available through Starlink} by A.C. Davenhall, March 1993
   (\xref{SUN/162}{sun162}{}).
   This SUN includes descriptions of SIMBAD, NED, ADS and ESIS, all of 
   which contain bibliographic information.

  \item F. Murtagh and H. -M. Adorf in {\it The ESO Messenger (El 
   Mensajero)}, June 1993, Number {\bf 72}, pp45-47. This paper is
   similar in scope and purpose to the present document.

  \item {\it Network Resources for Astronomers}\, by R.J. Hanisch, 
   October 1992 and {\it On-line Data and Information in Astronomy} 
   by H. Andernach, February 1993. These documents are similar to the
   present document and \xref{SUN/162}{sun162}{} and describe many on-line
   databases and 
   archives, some of which contain bibliographic information. They are 
   available through Starlink as MUD/135 and MUD/136.

\end{itemize}


\section{Acknowledgements\xlabel{acknowledgements}}

I learnt of some of the bibliographic databases from F. Murtagh and 
H. -M. Adorf's paper in {\it The ESO Messenger}. Dr.~B.G.~Marsden 
kindly supplied useful information about the IAU Circulars. Ed Krol's   
{\it The Whole Internet User's Guide and Catalog}\, explained the 
various INTERNET tools, which previously had seemed capricious and 
baffling. The descriptions of the various facilities were mostly 
abstracted from the introductory sections of the corresponding 
documentation. Tables~\ref{AIR_WEB} to \ref{AIR_FTP} were extracted
from STEIS. In addition, many people have also contributed to this 
document, either by supplying information, by offering advice and 
assistance or by loaning copies of journals and books. It is a pleasure 
to thank (in alphabetical order): Heinz~Andernach, Christine~Done, 
Dave~Green, Mike~Lawden, Geoff~Mellor, Julian~Osborne, Clive~Page, 
John~Pye, Francois~Simien and Lindy~Wilson. Special thanks are due to 
Geoff~Martin who checked all the network addresses, account names and 
passwords.


\typeout{  }
\typeout{***********************************************************}
\typeout{  }
\typeout{Note: in order to process this document ab initio, starting}
\typeout{with only the Latex source file, it needs to be run through}
\typeout{Latex THREE times in order to resolve the references and}
\typeout{table of contents.}
\typeout{  }
\typeout{***********************************************************}
\typeout{  }

\end{document}
