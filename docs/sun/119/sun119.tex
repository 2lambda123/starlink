\documentclass[11pt,nolof]{starlink}

%------------------------------------------------------------------------------
\stardoccategory    {Starlink User Note}
\stardocinitials    {SUN}
\stardocauthors     {A R Wood}
\stardocdate        {18 August 1992}
\stardoctitle       {CHI --- Catalogue Handling Interface}
\stardocnumber      {119.2}
\stardocversion     {Version 1.1}
\stardocmanual      {Programmer's Manual}
\stardocabstract{%
The CHI (Catalogue Handling Interface) and CHP (Catalogue Handling Plus)
routines provide access to the STARLINK database systems.
Applications written using the CHI and CHP routines will
run on the current STARLINK database (ADC) and any future STARLINK database
systems.
}
%------------------------------------------------------------------------------

\begin{document}
\scfrontmatter

\section {Introduction}

This note is meant for those people who intend to write catalogue handling
applications using the
Catalogue Handling Interface (CHI) and the Catalogue Handling Interface Plus
(CHP). Users should already be familiar with

\begin {description}

\item [\xref{SUN/120}{sun120}{}]  CATPAC --- CATalogue handling applications PACkage.

\end {description}

which describes the catalogue handling applications already available.

\section {What are CHI and CHP?}

The CHI and CHP routines provide access to the STARLINK database systems.
Applications written using the CHI and CHP routines will
run on the current STARLINK database (ADC) and any future STARLINK database
systems.
\begin{itemize}
   \item The CHI routines provide low level access to the underlying databases.
   \item The CHP routines (CHI Plus) are an extension to the CHI routines.
They include
functionality not found in the CHI routines.
This new functionality falls into two areas.
\begin{enumerate}
  \item A CHP routine may add general functionality applicable
to all catalogues (\emph{e.g.}, CHP\_SAMPLE Select every Nth row from a catalogue)
  \item A CHP routine may add specialist functionality dealing with the
manipulation of catalogue data in a domain specific way
(\emph{e.g.}, Astronomical catalogues with sexagesimal
field formats ).
\end{enumerate}
\end{itemize}
The CHP routines access the underlying database system through the CHI routines
so CHI and CHP routines can be freely mixed.

\section{Routines and Constants}

\subsection{Routine Names}

Catalogues are created, accessed, modified and deleted in programs by means of
calls to CHI and CHP routines. The routine names have the following structure:

\begin{terminalv}
<pck>_<func><qual>
\end{terminalv}

where \verb+<pck>+ is the package name, \verb+<func>+ represents the
function performed by the routine and \verb+<qual>+ is a qualifier which is
used to identify different versions of the \textbf{GETSVAL}, \textbf{GETVAL},
and \textbf{NEWFLD} routines.

\subsection{Program Structure}

Catalogue handling applications must have the following structure:

\begin{terminalv}
      <declarations>
      STATUS = SAI__OK
      CALL CHI_OPEN( STATUS )
      <executable statements>
      CALL CHI_CLOSE( STATUS )
      END
\end{terminalv}

\subsection{Symbolic Names and Include Files}

Various symbolic names should be used for important constant values in CHI
programs to make the programs clearer and to insulate them from possible future
changes. These symbolic names are defined by several Fortran ``include'' files.
The following include files are available:

\begin{description}

\item [SAE\_PAR:]
This file is not actually part of CHI, but it defines the global symbolic
constant SAI\_\_OK (the value of the status return indicating success) and will
be required by nearly all routines which call CHI. It should normally be
included as a matter of course.

\item [CHI\_PAR:]
This defines symbolic constants for the CHI and CHP routines.

\begin{tabbing}
....................................\=................................................................................\=....\kill
\textbf{CHI\_\_SZNAME}    \>The maximum size of a catalogue name.     \>8\\
\textbf{CHI\_\_NUMCATS}   \>The maximum number of catalogues.         \>512\\
\textbf{CHI\_\_NUMPARS}   \>The maximum number of parameters.         \>512\\
\textbf{CHI\_\_SZPNAME}   \>The maximum size of a parameter name.     \>17\\
\textbf{CHI\_\_SZPFMT}    \>The maximum size of the parameter format. \>6\\
\textbf{CHI\_\_SZPVAL}    \>The maximum size of the parameter value.  \>17\\
\textbf{CHI\_\_SZPCMT}    \>The maximum size of the parameter comment.\>50\\
\textbf{CHI\_\_NUMFLDS}   \>The maximum number of fields.             \>512\\
\textbf{CHI\_\_SZFNAME}   \>The maximum size of a field name.         \>17\\
\textbf{CHI\_\_SZFFMT}    \>The maximum size of the field format.     \>6\\
\textbf{CHI\_\_SZFUNIT}   \>The maximum size of the field units.      \>10\\
\textbf{CHI\_\_SZFNVAL}   \>The maximum size of the field null value. \>17\\
\textbf{CHI\_\_SZFCMT}    \>The maximum size of the field comment.    \>50\\
\textbf{CHI\_\_SZCVAL}    \>The maximum size of a character value.    \>512\\
\textbf{CHI\_\_SZEXP}     \>The maximum size of an expression         \>80\\
                       \>(search, reject, join, merge criteria)\\
\textbf{CHP\_\_SZFFMT}    \>The maximum size of a CHP field format.     \>6\\
\textbf{CHP\_\_SZFCMT}    \>The maximum size of a CHP field comment.    \>50\\
\end{tabbing}

These values are subject to change in subsequent releases.
\end{description}

If it is required to test for specific error conditions, the appropriate
include file should be used and the symbolic names (listed in
Appendix~\ref{appendix:errors}) used in the test. Here is an example of how to
use these symbols:

\begin{quote}
\begin{terminalv}
      INCLUDE 'SAE_PAR'
      INCLUDE 'CHI_PAR'
      INCLUDE 'CHI_ERR'
      ...

      CHARACTER * ( CHI__SZNAME ) NAME
      INTEGER NUMENTS
      INTEGER STATUS
      ...

* Find the number of entries in a catalogue.
      CALL CHI_GETNUMENTS( INPUT, NUMENTS, STATUS )

* Check the status value returned.
      IF ( STATUS .EQ. SAI__OK ) THEN
        <normal action>
      ELSE IF ( STATUS .EQ. CHI__CATNOTFND ) THEN
        <take appropriate action for catalogue not found>
      ELSE
        <action on other errors>
      ENDIF
\end{terminalv}
\end{quote}


\section {Using CHI and CHP}

In general any new catalogue handling application will fall into two parts
\begin{enumerate}

 \item Interacting with the user via the ADAM environment. Applications make
a small number of calls to CHP routines to perform the required catalogue
manipulation.

 \item Performing the required catalogue manipulation. New CHP routines that
perform the required catalogue manipulation will be added to the CHP library.
These CHP routines are
independent of the ADAM environment
and so can be reused by other applications.

\end{enumerate}

By adopting this approach the library of CHP Routines will be extended
providing a wealth of catalogue manipulation functionality that future
applications can draw on. Any new CHP routines should be submitted, with
documentation, for inclusion in this library.

\section{Efficiency Consideration when JOINing or SEARCHing}

Joining and searching catalogues can be very inefficient. Consider searching
a dictionary for a word if the words were not sorted into alphabetical order.
Sorting the catalogue (dictionary) can drastically improve searching efficiency.
The problem can be even more severe when joining catalogues. Consider joining
two dictionaries so that each word in the new dictionary has the definition of
that word from both the original dictionaries. If both dictionaries are sorted
alphabetically the join can be done efficiently. If only one of the dictionaries
is sorted the join is still managable. If neither of the dictionaries are
sorted the join will be highly inefficient.

The CHI\_TURBOSEARCH and CHI\_TURBOJOIN routines try to perform an efficient
search or join. If this is not possible an error and diagnostic information
is reported. Searching an unsorted dictionary for WORD equals `CAT' will
fail with a diagnostic message `Catalogue not sorted on field WORD.'
The catalogue can either be sorted on field WORD, using CHI\_SORT or
CHI\_ADDIND, and CHI\_TURBOSEARCH recalled or an inefficient search can be
made using CHI\_SEARCH. In most cases it is better to sort the catalogue. A
dictionary is more useful when sorted on field WORD. The sort takes more
time than an inefficient search but subsequent searches will be efficient.
There are cases where an inefficient search is more appropriate. Consider
a telephone directory sorted in the normal way. Searching for numbers given a
name and initial will be efficient. Searching for a number given an address,
because the name has been lost, will be inefficient but resorting the
directory is not required.

CHI\_SEARCH and CHI\_JOIN always try to perform an efficient search or join
but if this fails the error is cancelled and the routines go no to perform
inefficient searches or joins.

\section{Compiling and Linking}

\subsection{VAX/VMS Systems}

Users of ADAM may compile and link applications which call CHI and CHP routines
without the need for a special ``log in'', since all necessary  activation of
CHI will automatically be performed by the normal \texttt{\$ ADAMSTART}, \texttt{\$
ADAMDEV} and \texttt{\$ CATPAC} procedures. Thus, to compile and link an ADAM
application called ADAMPROG, the \texttt{\$ ALINK} command is used as normal,
thus:

\begin{terminalv}
$ FORTRAN ADAMPROG
$ ALINK ADAMPROG,CHILINK/OPT
\end{terminalv}

\appendix

\newpage
\section {Classified List of the CHI and CHP Routines}

\subsection {The CHP Routines}

The CHP routines can conveniently classified according to their functionality.

\begin{itemize}

\item Routines that analyze data in the catalogue.

 \begin{enumerate}

  \item CHP\_CATCORR Non parametric correlation handling null values.

  \item CHP\_CATLINCOR Linear correlation handling null values.

 \end{enumerate}

\item Routines that create catalogues.

 \begin{enumerate}

  \item CHP\_FK425 Apply the SLALIB SLA\_FK425 function (See \xref{SUN/67}{sun67}{})

  \item CHP\_FK45Z Apply the SLALIB SLA\_FK45Z function (See \xref{SUN/67}{sun67}{})

  \item CHP\_FK524 Apply the SLALIB SLA\_FK524 function (See \xref{SUN/67}{sun67}{})

  \item CHP\_FK54Z Apply the SLALIB SLA\_FK54Z function (See \xref{SUN/67}{sun67}{})

  \item CHP\_INCAT Input a catalogue from data in an ASCII file.

  \item CHP\_LISTIN Input a catalogue from free format data.

  \item CHP\_LITTLEBIG Extract the rows with the largest or smallest values in
a given column.

  \item CHP\_NOENT Create a new table with no entries.

  \item CHP\_PM Apply the SLALIB SLA\_PM function (See \xref{SUN/67}{sun67}{})

  \item CHP\_SAMPLE Select every Nth row from a table.

  \item CHP\_WITHIN Select entries within a polygon.

 \end{enumerate}

\item Routines that manipulate data about the catalogue.

 \begin{enumerate}

  \item CHP\_CHKFLD Check if a field exists in a catalogue.

  \item CHP\_CHKNAMES Check a list of fieldnames against a catalogue.

\item CHP\_GETFINF Get a particular piece of information associated with this
field. (\emph{e.g.}, get the FORMAT associated with this field NAME). If this
routine is used to find the FORMAT the fields sexagesimal format will be found.
An equivalent call to CHI\_GETFINF would return the FORMAT of the underlying
radian value. The NULLVALUE is also converted to sexagesimal format when this
is appropriate.

 \end{enumerate}


\item Routines that put information into catalogues.

 \begin{enumerate}


  \item CHP\_PUTENT Put a set of values that constitute an entry into a
      catalogue allowing sexagesimal formats.

 \end{enumerate}

\item Routines that get information out of a catalogue.

 \begin{enumerate}

  \item  CHP\_GETVALALL Get a set of values for fields of different types
  from the next entry in the catalogue. Sexagesimal field formats
  are allowed in character fields.

 \end{enumerate}

\item Routines that display information from a catalogue.

 \begin{enumerate}

  \item  CHP\_CATREP Make a report on the contents of a catalogue.

 \end{enumerate}

\item Miscellaneous routines.

 \begin{enumerate}

  \item  CHP\_CONVFMAT   Convert general sexagesimal format value.

  \item  CHP\_RECFMAT Convert a general sexagesimal format.

 \end{enumerate}

\end{itemize}

\subsection {The CHI Routines}

The CHI routines can conveniently classified according to their functionality.

\begin{itemize}
\item Routines that list available catalogues.

 \begin{enumerate}

  \item CHI\_AVAILCATS Get a list of the currently available catalogues.

 \end {enumerate}

\item Routines to open and close the system.

 \begin{enumerate}

  \item CHI\_OPEN Activate the interface.

  \item CHI\_CLOSE Deactivate the interface.

\end{enumerate}

\item Routines that manipulate data about the catalogue.

 \begin{enumerate}

  \item CHI\_ADDIND Add a new sort index to a sorted catalogue.

  \item CHI\_ADDP Add a parameter to a catalogue. Add this FORMAT, VALUE and
  COMMENT to the list of parameters
  for this catalogue and associate it with this NAME.

  \item CHI\_DELNOTES Delete given lines of the notes associated with this
   catalogue.

  \item CHI\_DELP Delete the parameter associated with a parameter NAME.

  \item CHI\_DELSORT Delete the sort information from a catalogue.

  \item CHI\_GETALLF Get all the information associated with all the fields.

  \item CHI\_GETALLP Get all the information associated with all parameters.

  \item CHI\_GETF Get all the field names and the number of fields.

  \item CHI\_GETFINF Get a particular piece of information associated with this
  field. (\emph{e.g.}, get the FORMAT associated with this field NAME)

  \item CHI\_GETNOTES Get the notes or part of the notes associated with this
  catalogue.

  \item CHI\_GETNUMENTS Get the number of entries in a catalogue.

  \item CHI\_GETONEF Get all the information associated with this field.

  \item CHI\_GETONEP Get all the information associated with this parameter.

  \item CHI\_GETP Get all the parameter names and the number of parameters.

  \item CHI\_GETPINF Get a particular piece of information associated with this
  parameter. (\emph{e.g.}, get the VALUE associated with this parameter NAME)

  \item CHI\_PUTNOTES Adds a line to the notes associated with this catalogue.

  \item CHI\_RENAME Rename a catalogue.

  \item CHI\_SORTFLDS Get the sort information from a catalogue.
   catalogue.

  \item CHI\_UPFLD Update the information associated with a field.

  \item CHI\_UPPAR Update the information associated with a parameter.

 \end{enumerate}

\item Routines that create catalogues.
 \begin{enumerate}

  \item CHI\_CALCFLD Create a new catalogue by adding a new field, calculated
  by evaluating an expression, to an existing catalogue.

  \item CHI\_COPYCAT Create a copy of a catalogue

  \item CHI\_CREATDUP Create a duplicate catalogue with no entries.

  \item CHI\_JOIN Create a new catalogue by joining two existing catalogues
  according to some join criteria.

  \item CHI\_MERGE Create a new catalogue by merging two existing catalogues.

  \item CHI\_NEWFLDx Create a new catalogue by adding a new field to an
  existing catalogue. The new
  catalogue has all the fields in the original catalogue plus the extra field.

  \item CHI\_NOENT Create a new catalogue with no entries.

  \item CHI\_REJECT Similar to CHI\_SEARCH but the new catalogue contains only
  those entries that fail the selection criteria.

  \item CHI\_SEARCH Create a new catalogue from an existing catalogue. All
      entries in the new catalogue obey some selection criteria. (\emph{e.g.},
  `MAG .GT. 8')

  \item CHI\_SELENTS Create a new catalogue from an existing catalogue. Only
      those entries selected are carried into the new catalogue.

  \item CHI\_SELFLDS Create a new catalogue from an existing catalogue. Only
      those fields selected are carried into the new catalogue.

  \item CHI\_SORT Create a new catalogue that is sorted on given fields.

  \item CHI\_TURBOJOIN Efficiently create a new catalogue by joining two
  existing catalogues according to some join criteria.

  \item CHI\_TURBOSEARCH Efficiently create a new catalogue from an existing
   catalogue. All in the new catalogue obey some selection criteria. (\emph{e.g.},
  `MAG .GT. 8')

 \end{enumerate}

\item Routine that deletes catalogues.

 \begin{enumerate}

  \item CHI\_DELCAT Delete a catalogue.

 \end{enumerate}

\item Routines that put data into catalogues.

 \begin{enumerate}

  \item CHI\_PUTENT Put a set of values that constitute an entry into a
      catalogue.

  \item CHI\_UPDATE Update a field.

 \end{enumerate}

\item Routines that get data from one entry in a catalogue.

 \begin{enumerate}

  \item  CHI\_GETVALx Get the value or values for a field or fields
  of the same type from the next entry in the catalogue.

  \item  CHI\_GETVALALL Get values for fields of different types
  from the next entry in the catalogue.

  \item  CHI\_RESET Reset a catalogue to the start after getting data
  entry by entry from a catalogue.

 \end{enumerate}

\item Routines that get data from many entries in a catalogue.

 \begin{enumerate}

  \item  CHI\_GETSVALx Get an array or arrays of values for a field or fields,
  of the same type, from the catalogue taking values from the given start
  position to the given start position plus the number of entries required.

  \item  CHI\_GETSVALALL Get an array or arrays of values for a field or
  fields, of different types, from the catalogue taking values from the given
  start position to the given start position plus the number of entries
  required.

 \end{enumerate}

\item Routines that allow direct access to the parser.

 \begin{enumerate}

  \item  CHI\_1PAR Parse an expression that contains field names from one
  catalogue.

  \item  CHI\_2PAR Parse an expression that contains field names from two
  catalogues.

  \item  CHI\_APPLY Apply the expression to the given data.

  \item  CHI\_EVAL Evaluate the expression using the values from the next
  entry in the catalogue.

 \end{enumerate}

\item Routines that allow direct access to the database.

 \begin{enumerate}

  \item  CHI\_GETCD Get an ADC Catalogue Descriptor.

  \item  CHI\_SPLITNAME Split a name into a database part and a catalogue part.

 \end{enumerate}

\end{itemize}


\newpage
\section{Alphabetical List of Routines}
\label{appendix:alphalist}

\begin{description}
\item [CHP\_CATCORR] ---  Non parametric correlation handling null values.
\item [CHP\_CATLINCOR] ---  Linear correlation handling null values.
\item [CHP\_CATREP] ---  Make a report on the contents of a catalogue.
\item [CHP\_CHKFLD] ---   Check if a field exists in a catalogue.
\item [CHP\_CHKNAMES] ---   Check a list of fieldnames against a catalogue.
\item [CHP\_CONVFMAT] ---   Convert general sexagesimal format value.
\item [CHP\_FK425] ---  Apply the SLALIB SLA\_FK425 function (See \xref{SUN/67}{sun67}{}).
\item [CHP\_FK45Z] ---  Apply the SLALIB SLA\_FK45Z function (See \xref{SUN/67}{sun67}{}).
\item [CHP\_FK524] ---  Apply the SLALIB SLA\_FK524 function (See \xref{SUN/67}{sun67}{}).
\item [CHP\_FK54Z] ---  Apply the SLALIB SLA\_FK54Z function (See \xref{SUN/67}{sun67}{}).
\item [CHP\_GETFINF] ---  Get a piece of information associated with a field.
\item [CHP\_GETVALALL] ---  Get a set of values for fields of different types.
\item [CHP\_INCAT] ---  Input a catalogue from data in an ASCII file.
\item [CHP\_LISTIN] ---  Input a catalogue from free format data.
\item [CHP\_LITTLEBIG] ---  Extract the rows with the largest or smallest
values in a given column.
\item [CHP\_NOENT] ---  Create a new table with no entries.
\item [CHP\_PM] ---  Apply the SLALIB SLA\_PM function (See \xref{SUN/67}{sun67}{}).
\item [CHP\_PUTENT] ---  Put a set of values that constitute an entry into a
catalogue.
\item [CHP\_RECFMAT] ---  Convert general sexagesimal format.
\item [CHP\_SAMPLE] ---  Select every Nth row from a table.
\item [CHP\_WITHIN] ---  Select entries within a polygon.
\item [CHI\_1PAR] ---  Parse an expression that contains field names from one
catalogue.
\item [CHI\_2PAR] ---  Parse an expression that contains field names from two
catalogues.
\item [CHI\_ADDIND] ---  Add a new sort index to a sorted catalogue.
\item [CHI\_ADDP] ---  Add a parameter to a catalogue.
\item [CHI\_APPLY] ---  Apply the expression to the given data.
\item [CHI\_AVAILCATS] ---  Get a list of the currently available catalogues.
\item [CHI\_CALCFLD] ---  Create a new catalogue by adding a new field,
calculated by evaluating an expression, to an existing catalogue.
\item [CHI\_GETCD] ---  Get an ADC Catalogue Descriptor.
\item [CHI\_CLOSE] ---  Deactivate the interface.
\item [CHI\_COPYCAT] ---  Create a copy of a catalogue
\item [CHI\_CREATDUP] ---  Create a duplicate catalogue with no entries.
\item [CHI\_DELCAT] ---  Delete a catalogue.
\item [CHI\_DELNOTES] ---  Delete given lines of the notes associated with this
catalogue.
\item [CHI\_DELP] ---  Delete the parameter associated with a parameter NAME.
\item [CHI\_DELSORT] ---  Delete the sort information from a catalogue.
\item [CHI\_EVAL] ---  Evaluate an expression.
\item [CHI\_GETALLF] ---  Get all the information associated with all the
fields.
\item [CHI\_GETALLP] ---  Get all the information associated with all
parameters.
\item [CHI\_GETF] ---  Get all the field names and the number of fields.
\item [CHI\_GETFINF] ---  Get a particular information associated with a field.
\item [CHI\_GETNOTES] ---  Get the notes or part of the notes associated with
this catalogue.
\item [CHI\_GETNUMENTS] ---  Get the number of entries in a catalogue.
\item [CHI\_GETONEF] ---  Get all the information associated with this field.
\item [CHI\_GETONEP] ---  Get all the information associated with this
parameter.
\item [CHI\_GETP] ---  Get all the parameter names and the number of parameters.
\item [CHI\_GETPINF] ---  Get information associated with this parameter.
\item [CHI\_GETSVALx] ---  Get an array or arrays of values for a field or
fields.
\item [CHI\_GETSVALALL] ---  Get an array or arrays of values for a field or
fields of different types.
\item [CHI\_GETVALx] ---  Get the value or values for a field or fields.
\item [CHI\_GETVALALL] ---  Get values for fields of different types.
\item [CHI\_JOIN] ---  Create a new catalogue by joining two existing
catalogues.
\item [CHI\_MERGE] ---  Create a new catalogue by merging two existing
catalogues.
\item [CHI\_NEWFLDx] ---  Create a new catalogue by adding a new field to an
existing catalogue.
\item [CHI\_NOENT] ---  Create a new catalogue with no entries.
\item [CHI\_OPEN] ---  Activate the interface.
\item [CHI\_PUTENT] ---  Put an entry into a catalogue.
\item [CHI\_PUTNOTES] ---  Adds a line to the notes associated with this
catalogue.
\item [CHI\_REJECT] ---  Similar to CHI\_SEARCH but the new catalogue contains
only those entries that fail the selection criteria.
\item [CHI\_RENAME] ---  Rename a catalogue.
\item [CHI\_RESET] ---  Reset a catalogue to the start.
\item [CHI\_SEARCH] ---  Create a new catalogue by searching an existing
catalogue.
\item [CHI\_SELENTS] --- Select entries from a catalogue.
\item [CHI\_SELFLDS] --- Select fields from a catalogue.
\item [CHI\_SPLITNAME] --- Split a name into a database part and a catalogue
part.
\item [CHI\_SORT] ---  Create a new catalogue that is sorted on given fields.
\item [CHI\_SORTFLDS] ---  Get the sort information from a catalogue.
\item [CHI\_TURBOJOIN] ---  Efficiently join two catalogues.
\item [CHI\_TURBOSEARCH] ---  Efficiently search a catalogue.
\item [CHI\_UPDATE] ---  Update a field.
\item [CHI\_UPFLD] ---  Update the information associated with a field.
\item [CHI\_UPPAR] ---  Update the information associated with a parameter.
\end{description}

\newpage

\section {Routine Descriptions}

\label{appendix:routines}

This appendix gives specifications for all the CHP and CHI routines.
A detailed description of the CHP Interface routines, in
alphabetical order, is followed by a description of the CHI routines.
Variables chi\_\_szfcmnt etc. are defined
in the include file CHI\_PAR. Full details can be found in the Include Files
and Error Handling Section of this document.

\subsection {Detailed Description of the CHP Routines}
\begin{small}

\sstroutine{
   CHP\_CATCORR
}{
   Non-parametric correlation of fields in a catalogue
}{
   \sstdescription{
      Perform non-parametric correlation tests on selected numeric
      fields in a catalogue including nulls.
   }
   \sstinvocation{
      CALL CHP\_CATCORR( INPUT, FNAMES, NUMFLD, STATUS)
   }
   \sstarguments{
      \sstsubsection{
         INPUT = CHARACTER $*$ ( CHI\_\_SZNAME ) (Given)
      }{
         Name of the catalogue whose fields are to be correlated.
      }
      \sstsubsection{
         FNAMES = CHARACTER $*$ ( CHI\_\_SZFNAME ) (Given)
      }{
         Name of the fields to be correlated.
      }
      \sstsubsection{
         NUMFLD = INTEGER (Given)
      }{
         Number of fields.
      }
      \sstsubsection{
         STATUS = INTEGER (Given and Returned)
      }{
         Global status.
      }
   }
   \sstnotes{
      Only numeric fields of types INTEGER, REAL and DOUBLE
      PRECISION are considered.
   }
   \sstdiytopic{
      Anticipated Errors
   }{
      CHI\_\_CATNOTFND
   }
}
\sstroutine{
   CHP\_CATLINCOR
}{
   Linear correlation of fields in a catalogue
}{
   \sstdescription{
      Perform linear correlation tests on selected numeric
      fields in a catalogue including null values.
   }
   \sstinvocation{
      CALL CHP\_CATLINCOR( INPUT, FNAMES, NUMFLD, STATUS)
   }
   \sstarguments{
      \sstsubsection{
         INPUT = CHARACTER $*$ ( CHI\_\_SZNAME ) (Given)
      }{
         Name of the catalogue whose fields are to be correlated.
      }
      \sstsubsection{
         FNAMES = CHARACTER $*$ ( CHI\_\_SZFNAME ) (Given)
      }{
         Name of the fields to be correlated.
      }
      \sstsubsection{
         NUMFLD = INTEGER (Given)
      }{
         Number of fields.
      }
      \sstsubsection{
         STATUS = INTEGER (Given and Returned)
      }{
         Global status.
      }
   }
   \sstnotes{
      Only numeric fields of types INTEGER, REAL and DOUBLE
      PRECISION are considered.
   }
   \sstdiytopic{
      Anticipated Errors
   }{
      CHI\_\_CATNOTFND
   }
}
\sstroutine{
   CHP\_CATREP
}{
   Produce a catalogue report
}{
   \sstdescription{
      Produce a catalogue report with or without a header and report it to
      the screen or to a file.
   }
   \sstinvocation{
      CALL  CHP\_CATREP(INPUT, HEADER, SCREEN, FNAMES, NUMFLD, ALL, STATUS)
   }
   \sstarguments{
      \sstsubsection{
         INPUT = CHARACTER $*$ ( CHI\_\_SZNAME ) (Given)
      }{
         Name of the catalogue whose fields are to be correlated.
      }
      \sstsubsection{
         HEADER = LOGICAL (Given)
      }{
         Add a header to the output.
      }
      \sstsubsection{
         SCREEN = LOGICAL (Given)
      }{
         Output to the screen or, false, to a file.
      }
      \sstsubsection{
         FNAMES( CHI\_\_NUMFLDS ) = CHARACTER $*$ ( CHI\_\_SZFNAME ) (Given)
      }{
         Name of the fields to be reported.
      }
      \sstsubsection{
         NUMFLD = INTEGER (Given)
      }{
         Number of fields.
      }
      \sstsubsection{
         ALL = LOGICAL (Given)
      }{
         All fields to be output.
      }
      \sstsubsection{
         STATUS = INTEGER (Given and Returned)
      }{
         Global status.
      }
   }
   \sstdiytopic{
      Anticipated Errors
   }{
      CHI\_\_CATNOTFND
   }
}
\sstroutine{
   CHP\_CHKFLD
}{
   Check if a field exists in a catalogue
}{
   \sstdescription{
      Check for the existence of a field in a catalogue
   }
   \sstinvocation{
      CALL CHP\_CHKFLD (INPUT, FLDFNAME, RESULT, STATUS)
   }
   \sstarguments{
      \sstsubsection{
         INPUT = CHARACTER $*$ ( CHI\_\_SZNAME ) (Given)
      }{
         Name of the catalogue whose fields are to be correlated.
      }
      \sstsubsection{
         FLDNAME = CHARACTER $*$ ( CHI\_\_SZFNAME ) (Given)
      }{
         Name of the field to be checked for.
      }
      \sstsubsection{
         RESULT = LOGICAL (Given)
      }{
         Logical result.
      }
      \sstsubsection{
         STATUS = INTEGER (Given and Returned)
      }{
         Global status.
      }
   }
   \sstdiytopic{
      Anticipated Errors
   }{
      CHI\_\_CATNOTFND
   }
}
\sstroutine{
   CHP\_CHKNAMES
}{
   Check that the list of field names are in the catalogue
}{
   \sstdescription{
      Check that the list of fieldnames are in the catalogue returning a new
      list containing only those that are and a new number of fields.
   }
   \sstinvocation{
       CALL CHP\_CHKNAMES (INPUT, NUMFLDS, FNAMES, NEWFLDS,
\newline
      :     NEWNAMES, STATUS)
   }
   \sstarguments{
      \sstsubsection{
         INPUT = CHARACTER $*$ ( CHI\_\_SZNAME ) (Given)
      }{
         Name of the catalogue whose fields are to be correlated.
      }
      \sstsubsection{
         NUMFLDS = INTEGER (Given)
      }{
         Number of fields.
      }
      \sstsubsection{
         FNAMES( CHI\_\_NUMFLDS ) = CHARACTER $*$ ( CHI\_\_SZFNAME ) (Given)
      }{
         Name of the fields to be checked for.
      }
      \sstsubsection{
         NEWFLDS = INTEGER (Given)
      }{
         New number of fields.
      }
      \sstsubsection{
         NEWNAMES( CHI\_\_NUMFLDS ) =
      }{
         \textbf{CHARACTER $*$ ( CHI\_\_SZFNAME ) (Given)}
      }{ \\
         Fields from the list that appeared in the catalogue.
      }
      \sstsubsection{
         STATUS = INTEGER (Given and Returned)
      }{
         Global status.
      }
   }
   \sstdiytopic{
      Anticipated Errors
   }{
      CHI\_\_CATNOTFND
   }
}
\sstroutine{
   CHP\_CONVFMAT
}{
   Convert general sexagesimal format value into a recommended format
   value
}{
   \sstdescription{
      Convert a general sexagesimal format into the most suitable recommended
      format. The input value is read using the input format and the most
      appropriate of the recommended formats is selected and the conversion made.
   }
   \sstinvocation{
      CALL CHP\_CONVFMAT( INFORMAT, INVALUE, OUTFORMAT, OUTVALUE, STATUS )
   }
   \sstarguments{
      \sstsubsection{
         INFORMAT = CHARACTER $*$ ( CHP\_\_SZFFMT ) (Given)
      }{
         Input format.
      }
      \sstsubsection{
         INVALUE = CHARACTER $*$ ( CHI\_\_SZFFMT ) (Given)
      }{
         Input value.
      }
      \sstsubsection{
         OUTFORMAT = CHARACTER $*$ ( CHP\_\_SZFFMT ) (Returned)
      }{
         Output format selected from the list of recommended formats.
      }
      \sstsubsection{
         OUTVALUE = CHARACTER $*$ ( CHI\_\_SZFFMT ) (Returned)
      }{
         Output value.
      }
      \sstsubsection{
         STATUS = INTEGER (Given and Returned)
      }{
         Global status.
      }
   }
}
\sstroutine{
   CHP\_FK425
}{
   Convert FK4 coordinates to FK5
}{
   \sstdescription{
      Create a catalogue containing new fields for the Right Ascension,
      Declination, Parallax, Radial velocity and proper motions after a
      conversion has been made from the FK4 system coordinates. The
      new fields are calculated using SLA\_FK425. See \xref{SUN/67}{sun67}{}

      Conversion from Besselian epoch 1950.0 to Julian epoch 2000.0 only
      is provided.

      Proper motions should be given in sec/yr and arcsecs/yr
      Parallax should be given in arcseconds.
      Radial velocity should be given in km/sec ($+$ve if receding)
   }
   \sstinvocation{
      CALL CHP\_FK425( INPUT, OUTPUT, RAFK4, DECFK4, RAPMFK4, DECPMFK4,
\newline
      PARLXFK4, RADVELFK4, RAFK5, DECFK5, RAPMFK5, DECPMFK5, PARLXFK5,
\newline
      RADVELFK5, STATUS )
   }
   \sstarguments{
      \sstsubsection{
         INPUT = CHARACTER $*$ ( CHI\_\_SZNAME ) (Given)
      }{
         Name of the catalogue.
      }
      \sstsubsection{
         OUTPUT = CHARACTER $*$ ( CHI\_\_SZNAME ) (Given)
      }{
         Name of the output catalogue.
      }
      \sstsubsection{
         RAFK4 = CHARACTER $*$ ( CHI\_\_SZFNAME ) (Given)
      }{
         Name of the RA field in FK4 system.
      }
      \sstsubsection{
         DECFK4 = CHARACTER $*$ ( CHI\_\_SZFNAME ) (Given)
      }{
         Name of the DEC field in FK4 system.
      }
      \sstsubsection{
         RAPMFK4 = CHARACTER $*$ ( CHI\_\_SZFNAME ) (Given)
      }{
         Name of the RA proper motion field in FK4 system.
      }
      \sstsubsection{
         DECPMFK4 = CHARACTER $*$ ( CHI\_\_SZFNAME ) (Given)
      }{
         Name of the DEC proper motion field in FK4 system.
      }
      \sstsubsection{
         PARLXFK4 = CHARACTER $*$ ( CHI\_\_SZFNAME ) (Given)
      }{
         Name of the parallax field in FK4 system.
      }
      \sstsubsection{
         RADVELFK4 = CHARACTER $*$ ( CHI\_\_SZFNAME ) (Given)
      }{
         Name of the radial velocity field in FK4 system.
      }
      \sstsubsection{
         RAFK5 = CHARACTER $*$ ( CHI\_\_SZFNAME ) (Given)
      }{
         Name of the RA field in FK5 system.
      }
      \sstsubsection{
         DECFK5 = CHARACTER $*$ ( CHI\_\_SZFNAME ) (Given)
      }{
         Name of the DEC field in FK5 system.
      }
      \sstsubsection{
         RAPMFK5 = CHARACTER $*$ ( CHI\_\_SZFNAME ) (Given)
      }{
         Name of the RA proper motion field in FK5 system.
      }
      \sstsubsection{
         DECPMFK5 = CHARACTER $*$ ( CHI\_\_SZFNAME ) (Given)
      }{
         Name of the DEC proper motion field in FK5 system.
      }
      \sstsubsection{
         PARLXFK5 = CHARACTER $*$ ( CHI\_\_SZFNAME ) (Given)
      }{
         Name of the parallax field in FK5 system.
      }
      \sstsubsection{
         RADVELFK5 = CHARACTER $*$ ( CHI\_\_SZFNAME ) (Given)
      }{
         Name of the radial velocity field in FK5 system.
      }
   }
   \sstdiytopic{
      Anticipated Errors
   }{
      CHI\_\_CATNOTFND \\
      CHI\_\_FLDNOTFND
   }
   \sstbugs{
      None
   }
}
\sstroutine{
   CHP\_FK45Z
}{
   Convert FK4 coordinates to FK5 (SLALIB FK45Z)
}{
   \sstdescription{
      Create a catalogue containing new fields for the Right Ascension
      and Declination after a conversion has been made from the FK4 system
      coordinates. The new fields are calculated using SLA\_FK45Z. See \xref{SUN/67}{sun67}{}
   }
   \sstinvocation{
       CALL CHP\_FK45Z( INPUT, OUTPUT, R1950, D1950, BEPOCH, R2000, D2000,
\newline
       STATUS )
   }
   \sstarguments{
      \sstsubsection{
         INPUT = CHARACTER $*$ ( CHI\_\_SZNAME ) (Given)
      }{
         Name of the catalogue.
      }
      \sstsubsection{
         OUTPUT = CHARACTER $*$ ( CHI\_\_SZNAME ) (Given)
      }{
         Name of the output catalogue.
      }
      \sstsubsection{
         R1950 = CHARACTER $*$ ( CHI\_\_SZFNAME ) (Given)
      }{
         Name of the RA field in FK4 system.
      }
      \sstsubsection{
         D1950 = CHARACTER $*$ ( CHI\_\_SZFNAME ) (Given)
      }{
         Name of the DEC field in FK4 system.
      }
      \sstsubsection{
         BEPOCH = REAL (Given)
      }{
         Besselian epoch.
      }
      \sstsubsection{
         R2000 = CHARACTER $*$ ( CHI\_\_SZFNAME ) (Given)
      }{
         Name of the RA field in FK5 system.
      }
      \sstsubsection{
         D2000 = CHARACTER $*$ ( CHI\_\_SZFNAME ) (Given)
      }{
         Name of the DEC field in FK5 system.
      }
   }
   \sstdiytopic{
      Anticipated Errors
   }{
      CHI\_\_CATNOTFND \\
      CHI\_\_FLDNOTFND
   }
   \sstbugs{
      None
   }
}
\sstroutine{
   CHP\_FK524
}{
   Convert FK5 coordinates to FK4
}{
   \sstdescription{
      Create a catalogue containing new fields for the Right Ascension,
      Declination, Parallax, Radial velocity and proper motions after a
      conversion has been made from the FK4 system coordinates. The
      new fields are calculated using SLA\_FK524. See \xref{SUN/67}{sun67}{}

      Conversion from Julian epoch 2000.0 to Besselian epoch 1950.0 only
      is provided.

      Proper motions should be given in sec/yr and arcsecs/yr
      Parallax should be given in arcseconds.
      Radial velocity should be given in km/sec ($+$ve if receding)
   }
   \sstinvocation{
      CALL CHP\_FK524( INPUT, OUTPUT, RAFK5, DECFK5, RAPMFK5, DECPMFK5,
\newline
      PARLXFK5, RADVELFK5, RAFK4, DECFK4, RAPMFK4, DECPMFK4, PARLXFK4,
\newline
      RADVELFK4, STATUS )
   }
   \sstarguments{
      \sstsubsection{
         INPUT = CHARACTER $*$ ( CHI\_\_SZNAME ) (Given)
      }{
         Name of the catalogue.
      }
      \sstsubsection{
         OUTPUT = CHARACTER $*$ ( CHI\_\_SZNAME ) (Given)
      }{
         Name of the output catalogue.
      }
      \sstsubsection{
         RAFK5 = CHARACTER $*$ ( CHI\_\_SZFNAME ) (Given)
      }{
         Name of the RA field in FK5 system.
      }
      \sstsubsection{
         DECFK5 = CHARACTER $*$ ( CHI\_\_SZFNAME ) (Given)
      }{
         Name of the DEC field in FK5 system.
      }
      \sstsubsection{
         RAPMFK5 = CHARACTER $*$ ( CHI\_\_SZFNAME ) (Given)
      }{
         Name of the RA proper motion field in FK5 system.
      }
      \sstsubsection{
         DECPMFK5 = CHARACTER $*$ ( CHI\_\_SZFNAME ) (Given)
      }{
         Name of the DEC proper motion field in FK5 system.
      }
      \sstsubsection{
         PARLXFK5 = CHARACTER $*$ ( CHI\_\_SZFNAME ) (Given)
      }{
         Name of the parallax field in FK5 system.
      }
      \sstsubsection{
         RADVELFK5 = CHARACTER $*$ ( CHI\_\_SZFNAME ) (Given)
      }{
         Name of the radial velocity field in FK5 system.
      }
      \sstsubsection{
         RAFK4 = CHARACTER $*$ ( CHI\_\_SZFNAME ) (Given)
      }{
         Name of the RA field in FK4 system.
      }
      \sstsubsection{
         DECFK4 = CHARACTER $*$ ( CHI\_\_SZFNAME ) (Given)
      }{
         Name of the DEC field in FK4 system.
      }
      \sstsubsection{
         RAPMFK4 = CHARACTER $*$ ( CHI\_\_SZFNAME ) (Given)
      }{
         Name of the RA proper motion field in FK4 system.
      }
      \sstsubsection{
         DECPMFK4 = CHARACTER $*$ ( CHI\_\_SZFNAME ) (Given)
      }{
         Name of the DEC proper motion field in FK4 system.
      }
      \sstsubsection{
         PARLXFK4 = CHARACTER $*$ ( CHI\_\_SZFNAME ) (Given)
      }{
         Name of the parallax field in FK4 system.
      }
      \sstsubsection{
         RADVELFK4 = CHARACTER $*$ ( CHI\_\_SZFNAME ) (Given)
      }{
         Name of the radial velocity field in FK4 system.
      }
   }
   \sstdiytopic{
      Anticipated Errors
   }{
      CHI\_\_CATNOTFND \\
      CHI\_\_FLDNOTFND
   }
   \sstbugs{
      None
   }
}
\sstroutine{
   CHP\_FK54Z
}{
   Convert FK5 coordinates to FK4 (SLALIB FK54Z)
}{
   \sstdescription{
      Create a catalogue containing new fields for the Right Ascension
      and Declination after a conversion has been made from the FK5 system
      coordinates. The new fields are calculated using SLA\_FK54Z. See \xref{SUN/67}{sun67}{}
   }
   \sstinvocation{
       CALL CHP\_FK54Z( INPUT, OUTPUT, R2000, D2000, BEPOCH, R1950, D1950,
\newline
       STATUS )
   }
   \sstarguments{
      \sstsubsection{
         INPUT = CHARACTER $*$ ( CHI\_\_SZNAME ) (Given)
      }{
         Name of the catalogue.
      }
      \sstsubsection{
         OUTPUT = CHARACTER $*$ ( CHI\_\_SZNAME ) (Given)
      }{
         Name of the output catalogue.
      }
      \sstsubsection{
         R2000 = CHARACTER $*$ ( CHI\_\_SZFNAME ) (Given)
      }{
         Name of the RA field in FK5 system.
      }
      \sstsubsection{
         D2000 = CHARACTER $*$ ( CHI\_\_SZFNAME ) (Given)
      }{
         Name of the DEC field in FK5 system.
      }
      \sstsubsection{
         BEPOCH = REAL (Given)
      }{
         Besselian epoch.
      }
      \sstsubsection{
         R1950 = CHARACTER $*$ ( CHI\_\_SZFNAME ) (Given)
      }{
         Name of the RA field in FK4 system.
      }
      \sstsubsection{
         D1950 = CHARACTER $*$ ( CHI\_\_SZFNAME ) (Given)
      }{
         Name of the DEC field in FK4 system.
      }
   }
   \sstdiytopic{
      Anticipated Errors
   }{
      CHI\_\_CATNOTFND \\
      CHI\_\_FLDNOTFND
   }
   \sstbugs{
      None
   }
}
\sstroutine{
   CHP\_GETFINF
}{
   Get specific information about a field
}{
   \sstdescription{
      Gets a specific piece of information about a field. The
      information required is one from FORMAT, UNITS, NULLVALUE or
      COMMENT. For example get the format of the field FLUX. Sexagesimal
      formats are allowed.
   }
   \sstinvocation{
      CALL CHP\_GETFINF( INPUT, FNAME, FREQ, FVALUE, STATUS )
   }
   \sstarguments{
      \sstsubsection{
         INPUT = CHARACTER $*$ ( CHI\_\_SZNAME ) (Given)
      }{
         Name of the catalogue from which the field information is
         required.
      }
      \sstsubsection{
         FNAME = CHARACTER $*$ ( CHI\_\_SZFNAME ) (Given)
      }{
         Name of the field whose information is required.
      }
      \sstsubsection{
         FREQ = CHARACTER $*$ ( 9 ) (Given)
      }{
         Information required. One from FORMAT, UNITS, NULLVALE or
         COMMENT.
      }
      \sstsubsection{
         FVALUE = CHARACTER $*$ ( CHI\_\_SZFCMT ) (Returned)
      }{
         The required information.
      }
      \sstsubsection{
         STATUS = INTEGER (Given and Returned)
      }{
         Global status.
      }
   }
   \sstdiytopic{
      Anticipated Errors
   }{
      CHI\_\_CATNOTFND \\
      CHI\_\_FLDNOTFND \\
      CHI\_\_IVLDFREQ
   }
}
\sstroutine{
   CHP\_GETVALALL
}{
   Get all the data from the next entry in a catalogue
}{
   \sstdescription{
      Get all the data from the next entry in a catalogue allowing sexagesimal
      formats. Get data from all the fields in the catalogue. The data is
      taken from the next entry in the catalogue  but see Notes if you are
      mixing different CHI\_GET calls. Data is returned in the appropriate
      element of the appropriate array. FNAMES will contain the names of the
      fields and FLDTYPES the type of field. So if FLDTYPES(3) is an \texttt{'}I\texttt{'} the
      array INTVALS(3) will contain the data from the field whose name is
      given in FNAMES(3).
   }
   \sstinvocation{
      CALL CHP\_GETVALALL( INPUT, FNAMES, NUMFLDS, CHARVALS, DOUBVALS,
\newline
      INTVALS, LOGVALS, REALVALS, FLDTYPES, STATUS )
   }
   \sstarguments{
      \sstsubsection{
         INPUT = CHARACTER $*$ ( CHI\_\_SZNAME ) (Given)
      }{
         Name of the catalogue from which the data is required.
      }
      \sstsubsection{
         FNAMES( CHI\_\_NUMFLDS ) =
      }{
         \textbf{CHARACTER $*$ ( CHI\_\_SZFNAME ) (Returned)}
      }{ \\
         Names of the fields whose data is required.
      }
      \sstsubsection{
         NUMFLDS = INTEGER (Returned)
      }{
         Number of fields whose data is required.
      }
      \sstsubsection{
         INTVALS( CHI\_\_NUMFLDS ) = INTEGER (Returned)
      }{
         Array to receive the data from integer fields.
      }
      \sstsubsection{
         REALVALS( CHI\_\_NUMFLDS ) = REAL (Returned)
      }{
         Array to receive the data from real fields.
      }
      \sstsubsection{
         DOUBVALS( CHI\_\_NUMFLDS ) = DOUBLE PRECISION (Returned)
      }{
         Array to receive the data from double precision fields.
      }
      \sstsubsection{
         LOGVALS( CHI\_\_NUMFLDS ) = LOGICAL (Returned)
      }{
         Array to receive the data from logical fields.
      }
      \sstsubsection{
         CHARVALS( CHI\_\_NUMFLDS ) =
      }{
         \textbf{CHARACTER $*$ ( CHI\_\_SZCVAL ) (Returned)}
      }{ \\
         Array to receive the data from character fields.
      }
      \sstsubsection{
         FLDTYPES( CHI\_\_NUMFLDS ) = CHARACTER $*$ ( 1 ) (Returned)
      }{
         Array to receive the types of fields.
      }
      \sstsubsection{
         STATUS = INTEGER (Given and Returned)
      }{
         Global status.
      }
   }
   \sstnotes{
      All the CHI\_GET and CHI\_EVAL routines are interlinked so a CHI\_GETVALR
      followed by a CHI\_GETVALC will cause the character values to be taken
      from the 2nd entry in the catalogue. In the same way a CHI\_GETSVALR
      getting data from 10 entries starting at entry 5 followed by a CHI\_EVALA
      will cause the arithmetic expression to be evaluated for the 15th entry
      in the catalogue. Use CHI\_RESET to return to the start of the catalogue.
   }
   \sstdiytopic{
      Anticipated Errors
   }{
      CHI\_\_CATNOTFND
   }
}
\sstroutine{
   CHP\_INCAT
}{
   Create a new catalogue from data in an ASCII file
}{
   \sstdescription{
      Create a catalogue and load data from an ASCII file into it.
   }
   \sstinvocation{
      CALL CHP\_INCAT(INPUT, DATAFILE, NUMPARAM, PNAMES, PFORMATS,
\newline
        PVALUES, PCOMMENTS, NUMFLD, FNAMES, INFFORMATS, OUTFORMATS, FUNITS,
\newline
        FNULLS, FCOMMENTS, STARTPOS, STATUS)
   }
   \sstarguments{
      \sstsubsection{
         INPUT = CHARACTER $*$ ( CHI\_\_SZNAME ) (Given)
      }{
         Name of the catalogue to be created.
      }
      \sstsubsection{
         DATAFILE = CHARACTER $*$ ( CHI\_\_SZNAME ) (Given)
      }{
         Name of the file containing the ASCII data.
      }
      \sstsubsection{
         NUMPARAM = INTEGER (Given)
      }{
         Number of parameters.
      }
      \sstsubsection{
         PNAMES( CHI\_\_NUMPARS ) = CHARACTER $*$ ( CHI\_\_SZPNAME ) (Given)
      }{
         Names of the parameters.
      }
      \sstsubsection{
         PFORMATS( CHI\_\_NUMPARS ) =
      }{
         \textbf{CHARACTER $*$ ( CHI\_\_SZPFMT ) (Given)}
      }{ \\
         Formats of the parameters.
      }
      \sstsubsection{
         PVALUES( CHI\_\_NUMPARS ) = CHARACTER $*$ ( CHI\_\_SZPVAL ) (Given)
      }{
         Values of the parameters.
      }
      \sstsubsection{
         PCOMMENTS( CHI\_\_NUMPARS ) =
      }{
         \textbf{CHARACTER $*$ ( CHI\_\_SZPCMT ) (Given)}
      }{ \\
         Comments associated with the parameters.
      }
      \sstsubsection{
         NUMFLD = INTEGER (Returned)
      }{
         Number of fields.
      }
      \sstsubsection{
         FNAMES( CHI\_\_NUMFLDS ) = CHARACTER $*$ ( CHI\_\_SZFNAME ) (Given)
      }{
         Names of the fields.
      }
      \sstsubsection{
         INFFORMATS( CHI\_\_NUMFLDS ) =
      }{
         \textbf{CHARACTER $*$ ( CHI\_\_SZFFMT ) (Given)}
      }{ \\
         Formats of the fields being read.
      }
      \sstsubsection{
         OUTFFORMATS( CHI\_\_NUMFLDS ) =
      }{
         \textbf{CHARACTER $*$ ( CHI\_\_SZFFMT ) (Given)}
      }{ \\
         Formats of the fields when displayed.
      }
      \sstsubsection{
         FUNITS( CHI\_\_NUMFLDS ) = CHARACTER $*$ ( CHI\_\_SZFUNIT ) (Given)
      }{
         Units of the fields.
      }
      \sstsubsection{
         FNULLS( CHI\_\_NUMFLDS ) = CHARACTER $*$ ( CHI\_\_SZFNVAL ) (Given)
      }{
         Null values of the fields.
      }
      \sstsubsection{
         FCOMMENTS( CHI\_\_NUMFLDS ) =
      }{
         \textbf{CHARACTER $*$ ( CHI\_\_SZFCMT ) (Given)}
      }{ \\
         Comments associated with the fields.
      }
      \sstsubsection{
         STATUS = INTEGER (Given and Returned)
      }{
         Global status.
      }
   }
   \sstdiytopic{
      Anticipated Errors
   }{
      CHI\_\_IVLDFFMT \\
      CHI\_\_IVLDPFMT
   }
}
\sstroutine{
   CHP\_LISTIN
}{
   Create a new catalogue from data in a free format ASCII file
}{
   \sstdescription{
      Create a catalogue and load data from a free format ASCII file into it.
   }
   \sstinvocation{
      CALL CHP\_LISTIN(INPUT, DATAFILE, STATUS)
   }
   \sstarguments{
      \sstsubsection{
         INPUT = CHARACTER $*$ ( CHI\_\_SZNAME ) (Given)
      }{
         Name of the catalogue to be created.
      }
      \sstsubsection{
         DATAFILE = CHARACTER $*$ ( CHI\_\_SZNAME ) (Given)
      }{
         Name of the file containing the ASCII data.
      }
      \sstsubsection{
         STATUS = INTEGER (Given and Returned)
      }{
         Global status.
      }
   }
}
\sstroutine{
   CHP\_LITTLEBIG
}{
   Extracts largest or smallest nos. from a catalogue
}{
   \sstdescription{
      Extracts a given number of the objects in a catalogue
      with the smallest or largest values for a given field.
      These values are output to a second catalogue.
   }
   \sstinvocation{
       CALL CHP\_LITTLEBIG (INPUT, OUTPUT, NUMSEL, BIGEST, FIELD, REJFLG,
\newline
                       REJECTS, STATUS)
   }
   \sstarguments{
      \sstsubsection{
         INPUT = CHARACTER $*$ ( CHI\_\_SZNAME ) (Given)
      }{
         Name of the catalogue whose fields are to be correlated.
      }
      \sstsubsection{
         OUTPUT = CHARACTER $*$ ( CHI\_\_SZNAME ) (Given)
      }{
         Name of the catalogue to be created.
      }
      \sstsubsection{
         NUMSEL = INTEGER (Given)
      }{
         Number of entries required.
      }
      \sstsubsection{
         BIGEST = LOGICAL (Given)
      }{
         Biggest or smallest entries required.
      }
      \sstsubsection{
         FIELD = CHARACTER $*$ ( CHI\_\_SZFNAME ) (Given)
      }{
         The field on whose value the biggest or smallest depends.
      }
      \sstsubsection{
         REJFLG = LOGICAL(Given)
      }{
         Rejects catalogue required.
      }
      \sstsubsection{
         REJECTS = CHARACTER $*$ ( CHI\_\_SZNAME ) (Given)
      }{
         Name of the rejects catalogue to be created.
      }
      \sstsubsection{
         STATUS = INTEGER (Given and Returned)
      }{
         Global status.
      }
   }
   \sstdiytopic{
      Anticipated Errors
   }{
      CHI\_\_CATNOTFND \\
      CHI\_\_FLDNOTFND
   }
}
\sstroutine{
   CHP\_NOENT
}{
   Create a new catalogue that contains no entries
}{
   \sstdescription{
      Creates a new catalogue that contains no entries. Sexagesimal formats
      are allowed. The CHI routines
      that write data into this catalogue will be more efficient if you can
      provide an estimate for the size of the catalogue. (The number
      of entries). The field formats and the parameter formats are checked. If
      an invalid format is found an error is reported and the offending
      parameter or field are returned in PNAMES(1),
      PFORMATS(1) or FNAMES(1), FFORMATS(1).
      In addition the parameter formats are checked against the parameter
      values
      and the field formats are checked against the null values and an error
      is reported if an inconsistency is found. The offending
      parameter or field name, format and value or null value are returned in
      PNAMES(1), PFORMATS(1) and PVALUES(1) or FNAMES(1), FFORMATS(1) and
      FNULLS(1) respectively.
   }
   \sstinvocation{
      CALL CHP\_NOENT( INPUT, ESTNUMENTS, NUMPARS, PNAMES, PFORMATS,
\newline
      PVALUES, PCOMMENTS, NUMFLDS, FNAMES, FFORMATS, FUNITS, FNULLS,
\newline
      FCOMMENTS, STATUS )
   }
   \sstarguments{
      \sstsubsection{
         INPUT = CHARACTER $*$ ( CHI\_\_SZNAME ) (Given)
      }{
         Name of the catalogue being created.
      }
      \sstsubsection{
         ESTNUMENTS = INTEGER (Given)
      }{
         Estimate for the number of entries that will be put into the catalogue.
      }
      \sstsubsection{
         NUMPARS = INTEGER (Given)
      }{
         Number of parameters in the catalogue.
      }
      \sstsubsection{
         PNAMES( CHI\_\_NUMPARS ) =
      }{
         \textbf{CHARACTER $*$ ( CHI\_\_SZPNAME ) (Given and Returned)}
      }{ \\
         Names of the parameters in the catalogue.
      }
      \sstsubsection{
         PFORMATS( CHI\_\_NUMPARS ) =
      }{
         \textbf{CHARACTER $*$ ( CHI\_\_SZPFMT ) (Given and Returned)}
      }{ \\
         Formats of the parameters in the catalogue.
      }
      \sstsubsection{
         PVALUE( CHI\_\_NUMPARS ) =
      }{
         \textbf{CHARACTER $*$ ( CHI\_\_SZPVAL ) (Given and Returned)}
      }{ \\
         Values of the parameters in the catalogue.
      }
      \sstsubsection{
         PCOMMENTS( CHI\_\_NUMPARS ) =
      }{
         \textbf{CHARACTER $*$ ( CHI\_\_SZPCMT ) (Given)}
      }{ \\
         Comments associated with the parameters in the catalogue.
      }
      \sstsubsection{
         NUMFLDS = INTEGER (Returned)
      }{
         Number of fields in the catalogue.
      }
      \sstsubsection{
         FNAMES( CHI\_\_NUMFLDS ) =
      }{
         \textbf{CHARACTER $*$ ( CHI\_\_SZFNAME ) (Given and Returned)}
      }{ \\
         Names of the fields in the catalogue.
      }
      \sstsubsection{
         FFORMATS( CHI\_\_NUMFLDS ) =
      }{
         \textbf{CHARACTER $*$ ( CHI\_\_SZFFMT ) (Given and Returned)}
      }{ \\
         Formats of the fields in the catalogue.
      }
      \sstsubsection{
         FUNITS( CHI\_\_NUMFLDS ) = CHARACTER $*$ ( CHI\_\_SZFUNIT ) (Given)
      }{
         Units of the fields in the catalogue.
      }
      \sstsubsection{
         FNULLS( CHI\_\_NUMFLDS ) = CHARACTER $*$ ( CHI\_\_SZFNVAL ) (Given)
      }{
         Null values of the fields in the catalogue.
      }
      \sstsubsection{
         FCOMMENTS( CHI\_\_NUMFLDS ) =
      }{
         \textbf{CHARACTER $*$ ( CHI\_\_SZFCMT ) (Given)}
      }{ \\
         Comments associated with the fields in the catalogue.
      }
      \sstsubsection{
         STATUS = INTEGER (Given and Returned)
      }{
         Global status.
      }
   }
   \sstdiytopic{
      Anticipated Errors
   }{
      CHI\_\_IVLDFFMT \\
      CHI\_\_IVLDPFMT
   }
}
\sstroutine{
   CHP\_PM
}{
   Apply proper motion correction to a catalogue
}{
   \sstdescription{
      Create a catalogue containing new fields for the new Right Ascension and
      Declination after the correction has been made for proper motion.
      Calculated using SLA\_PM. See \xref{SUN/67}{sun67}{}.

      Proper motions should be given in radians per year of epoch.
      Parallax should be given in arcseconds.
      Radial velocity should be given in km/sec ($+$ve if receding)
   }
   \sstinvocation{
       CALL CHP\_PM( INPUT, OUTPUT, RAEP0, DECEP0, RAPM, DECPM, PARALLAX, RADVEL,
\newline
       EP0, EP1, RAEP1, DECEP1, STATUS )
   }
   \sstarguments{
      \sstsubsection{
         INPUT = CHARACTER $*$ ( CHI\_\_SZNAME ) (Given)
      }{
         Name of the catalogue.
      }
      \sstsubsection{
         OUTPUT = CHARACTER $*$ ( CHI\_\_SZNAME ) (Given)
      }{
         Name of the output catalogue.
      }
      \sstsubsection{
         RAEP0 = CHARACTER $*$ ( CHI\_\_SZFNAME ) (Given)
      }{
         Name of the RA field at epoch 0.
      }
      \sstsubsection{
         DECEP0 = CHARACTER $*$ ( CHI\_\_SZFNAME ) (Given)
      }{
         Name of the DEC field at epoch 0.
      }
      \sstsubsection{
         RAPM = CHARACTER $*$ ( CHI\_\_SZFNAME ) (Given)
      }{
         Name of the RA proper motion field.
      }
      \sstsubsection{
         DECPM = CHARACTER $*$ ( CHI\_\_SZFNAME ) (Given)
      }{
         Name of the DEC proper motion field.
      }
      \sstsubsection{
         PARALLAX = CHARACTER $*$ ( CHI\_\_SZFNAME ) (Given)
      }{
         Name of the parallax field.
      }
      \sstsubsection{
         RADVEL = CHARACTER $*$ ( CHI\_\_SZFNAME ) (Given)
      }{
         Name of the radial velocity field.
      }
      \sstsubsection{
         RAEP0 = CHARACTER $*$ ( CHI\_\_SZFNAME ) (Given)
      }{
         Name of the RA field at epoch 0.
      }
      \sstsubsection{
         DECEP0 = CHARACTER $*$ ( CHI\_\_SZFNAME ) (Given)
      }{
         Name of the DEC field at epoch 0.
      }
      \sstsubsection{
         EP0 = REAL (Given)
      }{
         Start Epoch.
      }
      \sstsubsection{
         EP1 = REAL (Given)
      }{
         End epoch.
      }
   }
   \sstdiytopic{
      Anticipated Errors
   }{
      CHI\_\_CATNOTFND \\
      CHI\_\_FLDNOTFND
   }
}
\sstroutine{
   CHP\_PUTENT
}{
   Put an entry into a catalogue
}{
   \sstdescription{
      Add an entry to a catalogue allowing sexagesimal formats. For each
      field name in FNAMES CHP\_PUTENT checks in respective position in
      FLDTYPES to find the type of data.
      The data for this field in the entry will be taken from the respective
      element of the appropriate array. So if FLDTYPES(3) is I then the data
      for the field whose name is given in FNAMES(3) will be taken from
      INTVALS(3).

      There are eight levels of checking which are selected using the argument
      CHECK. Modes 1 and 2 are efficient and should be used whenever possible.
      You are advised not to change the level of checking when putting data
      into a catalogue unless you are moving to a mode where checking is done.

      CHECK=1. The lowest level of checking. The routine processes the field
      names given in FNAMES. Only if the field appears in the catalogue and
      the type agrees with the that given in FLDTYPES will data be put into
      the field for this entry. All other fields take their null values. The
      routine remembers where it found the data for each field (\emph{e.g.}, FLUX1 data
      in REALVALS(5)). On subsequent calls the routine assumes that the FLUX1
      value will be in REALVALS(5). Any ordering information in the catalogue
      is ignored, but not destroyed. Use this mode when you know all the field
      names and types are correct and that the entries are being put in the
      correct order.

      CHECK=11. The same level of checking as in mode 1 but any ordering
      information is destroyed.

      CHECK=2. The routine processes the field names given in FNAMES and only
      if the all the fields in the catalogue are given in FNAMES and the
      types are correct will the entry be put into the catalogue. If, during
      checking, an error is found the error is reported and the name of the
      offending field is returned in FNAMES(1). Again after
      the first call the routine remembers where it found the data for each
      field and on subsequent calls the routine assumes that the data will be
      in the same place. This mode ensures that genuine data is put into the
      catalogue. Fields cannot be overlooked. Any ordering information in the
      catalogue is ignored but not destroyed. Use this mode if you are not
      sure that the field
      names and types are correct or if the entries being put into the
      catalogue are not appropriately ordered.

      CHECK=21. The same level of checking as in mode 2 but any ordering
      information is destroyed.

      Modes 3 and 4 reflect the an equivalent level of checking in the
      CHI\_GET routines and only in exceptional cases will they be used
      when putting data into a catalogue.

      CHECK=3. In this mode the routine processes the field names given in
      FNAMES and their fieldtypes every time the routine is called. This means
      that different fields can contribute to each entry. Only if the field
      appears in the catalogue and
      the type agrees with that given in FLDTYPES will data be put into
      the field for this entry. All other fields take their null values.
      Any ordering information in the catalogue
      is ignored, but not destroyed.

      CHECK=31. The same level of checking as in mode 3 but any ordering
      information is destroyed.

      CHECK=4. Again in this mode the routine processes the field names given
      in FNAMES and their fieldtypes every time the routine is called.  Only
      if the all the fields in the catalogue are given in FNAMES and the
      types are correct will the entry be put into the catalogue.  If, during
      checking, an error is found the error is reported and the name of the
      offending field is returned in FNAMES(1). Subsequent
      calls to the routine may have the fields in FNAMES in a different order
      but they must all be their and the respective field types must be correct.
      Any ordering information in the catalogue is ignored but not destroyed.

      CHECK=41. The same level of checking as in mode 4 but any ordering
      information is destroyed.
   }
   \sstinvocation{
      CALL CHP\_PUTENT( INPUT, FNAMES, NUMFLDS, CHECK, CHARVALS, DOUBVALS,
\newline
      INTVALS, LOGVALS, REALVALS, FLDTYPES, STATUS )
   }
   \sstarguments{
      \sstsubsection{
         INPUT = CHARACTER $*$ ( CHI\_\_SZNAME ) (Given)
      }{
         Name of the catalogue into which the data is to be put.
      }
      \sstsubsection{
         FNAMES( CHI\_\_NUMFLDS ) =
      }{
         \textbf{CHARACTER $*$ ( CHI\_\_SZFNAME ) (Given and Returned)}
      }{ \\
         Names of the fields whose data is being supplied.
      }
      \sstsubsection{
         NUMFLDS = INTEGER (Given)
      }{
         Number of fields whose data is being supplied.
      }
      \sstsubsection{
         CHECK = INTEGER (Given)
      }{
         Set to 1,2,3 or 4 according to the level of checking required.
      }
      \sstsubsection{
         INTVALS( CHI\_\_NUMFLDS ) = INTEGER (Returned)
      }{
         Array containing the data for integer fields.
      }
      \sstsubsection{
         REALVALS( CHI\_\_NUMFLDS ) = REAL (Returned)
      }{
         Array containing the data for real fields.
      }
      \sstsubsection{
         DOUBVALS( CHI\_\_NUMFLDS ) = DOUBLE PRECISION (Returned)
      }{
         Array containing the data for double precision fields.
      }
      \sstsubsection{
         LOGVALS( CHI\_\_NUMFLDS ) = LOGICAL (Returned)
      }{
         Array containing the data for logical fields.
      }
      \sstsubsection{
         CHARVALS( CHI\_\_NUMFLDS ) =
      }{
         \textbf{CHARACTER $*$ ( CHI\_\_SZCVAL ) (Returned)}
      }{ \\
         Array containing the data for character fields.
      }
      \sstsubsection{
         FLDTYPES( CHI\_\_NUMFLDS ) = CHARACTER $*$ ( 1 ) (Returned)
      }{
         Array containing the types of fields.
      }
      \sstsubsection{
         STATUS = INTEGER (Given and Returned)
      }{
         Global status.
      }
   }
   \sstdiytopic{
      Anticipated Errors
   }{
      CHI\_\_CATNOTFND \\
      CHI\_\_FLDNOTSUP \\
      CHI\_\_IVLDFLDTYP
   }
}
\sstroutine{
   CHP\_RECFMAT
}{
   Convert general sexagesimal format to a recommended format
}{
   \sstdescription{
      Convert a general sexagesimal format into the most suitable recommended
      format.
   }
   \sstinvocation{
      CALL CHP\_RECFMAT( FORMAT, STATUS )
   }
   \sstarguments{
      \sstsubsection{
         FORMAT = CHARACTER $*$ ( CHP\_\_SZFFMT ) (Given and Returned)
      }{
         Format.
      }
      \sstsubsection{
         STATUS = INTEGER (Given and Returned)
      }{
         Global status.
      }
   }
}
\sstroutine{
   CHP\_SAMPLE
}{
   Select every Nth object in a catalogue
}{
   \sstdescription{
      Create an output catalogue containing an entry for every Nth
      object in an input catalogue. Optionally a second output
      catalogue containing the rejected objects can be created.
      The user can select the value required for N.
   }
   \sstinvocation{
      CALL CHP\_SAMPLE(INPUT, OUTPUT, REJFLG, REJECTS, FREQ, STATUS)
   }
   \sstarguments{
      \sstsubsection{
         INPUT = CHARACTER $*$ ( CHI\_\_SZNAME ) (Given)
      }{
         Name of the catalogue whose fields are to be correlated.
      }
      \sstsubsection{
         OUTPUT = CHARACTER $*$ ( CHI\_\_SZNAME ) (Given)
      }{
         Name of the output catalogue.
      }
      \sstsubsection{
         REJFLG = LOGICAL(Given)
      }{
         Is a rejects catalogue required.
      }
      \sstsubsection{
         REJECTS = CHARACTER $*$ ( CHI\_\_SZNAME ) (Given)
      }{
         Name of the rejects catalogue.
      }
      \sstsubsection{
         FREQ = INTEGER(Given)
      }{
         Frequency at which to sample.
      }
      \sstsubsection{
         STATUS = INTEGER (Given and Returned)
      }{
         Global status.
      }
   }
   \sstdiytopic{
      Anticipated Errors
   }{
      CHI\_\_CATNOTFND
   }
}
\sstroutine{
   CHP\_WITHIN
}{
   Select points inside or outside a polygon
}{
   \sstdescription{
      Objects in a catalogue are selected according to whether they
      lie inside or outside a 2 dimensional polygon. The corners
      defining the polygon are input as a second catalogue. Points
      either inside or outside the polygon may be selected. The
      selected objects are written to an output catalogue and
      optionally the rejected objects may be written to a second
      catalogue.
   }
   \sstinvocation{
       CALL CHP\_WITHIN( INPUT, POLYCAT, OUTPUT, REJFLG, REJECTS, INSIDE,
\newline
       XFIELD, YFIELD, STATUS )
   }
   \sstarguments{
      \sstsubsection{
         INPUT = CHARACTER $*$ ( CHI\_\_SZNAME ) (Given)
      }{
         Name of the catalogue.
      }
      \sstsubsection{
         POLYCAT = CHARACTER $*$ ( CHI\_\_SZNAME ) (Given)
      }{
         Name of the polygon catalogue.
      }
      \sstsubsection{
         OUTPUT = CHARACTER $*$ ( CHI\_\_SZNAME ) (Given)
      }{
         Name of the catalogue to be created.
      }
      \sstsubsection{
         REJFLG = LOGICAL (Given )
      }{
         Is a rejects catalogue required?
      }
      \sstsubsection{
         REJECTS = CHARACTER $*$ ( CHI\_\_SZNAME ) (Given)
      }{
         Name of the rejects catalogue.
      }
      \sstsubsection{
         INSIDE = LOGICAL( Given )
      }{
         Flag to select points inside or outside the polygon.
      }
      \sstsubsection{
         XFIELD = CHARACTER $*$ ( CHI\_\_SZFNAME ) (Given)
      }{
         Name of the X field in the catalogue.
      }
      \sstsubsection{
         YFIELD = CHARACTER $*$ ( CHI\_\_SZFNAME ) (Given)
      }{
         Name of the Y field in the catalogue.
      }
   }
   \sstdiytopic{
      Anticipated Errors
   }{
      CHI\_\_CATNOTFND
   }
}

\subsection {Detailed Description of the CHI Routines}

\sstroutine{
   CHI\_1PAR
}{
   Parse an expression for one catalogue
}{
   \sstdescription{
      This routine will not normally be required because all common uses of
      the parser have been anticipated and included in other CHI routines.
      CHI\_SEARCH, CHI\_REJECT, CHI\_JOIN, CHI\_UPDATE, CHI\_EVAL and
      CHI\_CALCFLD call the parser on you behalf. CHI\_1PAR and CHI\_2PAR give
      you direct access to the parser when 1 or 2 catalogues are involved.
      This routine parses an expression which contains field names from a
      single catalogue. To apply this expression to a set of data see the
      CHI\_APPLY routine. CHI\_1PAR uses the arguments FNAMES and FTYPES to
      inform the user of how the data is to be supplied. See CHI\_APPLY for
      more detail.
   }
   \sstinvocation{
      CALL CHI\_1PAR( INPUT, EXPRESS, FNAMES, FTYPES, STATUS )
   }
   \sstarguments{
      \sstsubsection{
         INPUT = CHARACTER$*$(CHI\_\_SZNAME) (Given)
      }{
         Name of the catalogue.
      }
      \sstsubsection{
         EXPRESS = CHARACTER$*$(CHI\_\_SZEXP) (Given)
      }{
         Expression to be parsed.
      }
      \sstsubsection{
         FNAMES(CHI\_\_NUMFLDS) = CHARACTER$*$(CHI\_\_SZFNMAE) (Returned)
      }{
         Names of the fields.
      }
      \sstsubsection{
         FTYPES(CHI\_\_NUMFLDS) = CHARACTER$*$(1) (Returned)
      }{
         Types of the fields.
      }
      \sstsubsection{
         STATUS = INTEGER (Given and Returned)
      }{
         The global status.
      }
   }
   \sstdiytopic{
      Anticipated Errors
   }{
      CHI\_\_CATNOTFND
   }
}
\sstroutine{
   CHI\_2PAR
}{
   Parse an expression two catalogues
}{
   \sstdescription{
      This routine will not normally be required because all common uses of
      the parser have been anticipated and included in other CHI routines.
      CHI\_SEARCH, CHI\_REJECT, CHI\_JOIN, CHI\_UPDATE, CHI\_EVAL and
      CHI\_CALCFLD call the parser on you behalf. CHI\_1PAR and CHI\_2PAR give
      you direct access to the parser when 1 or 2 catalogues are involved.
      This routine parses an expression which contains field names from
      two catalogues. To apply this expression to a set of data see the
      CHI\_APPLY routine. CHI\_2PAR uses the arguments FNAMES and FTYPES to
      inform the user of how the data is to be supplied. See CHI\_APPLY for
      more detail.
   }
   \sstinvocation{
      CALL CHI\_2PAR( INPUT1, INPUT2, EXPRESS, FNAMES, FTYPES, STATUS )
   }
   \sstarguments{
      \sstsubsection{
         INPUT1 = CHARACTER$*$(CHI\_\_SZNAME) (Given)
      }{
         Name of the first catalogue.
      }
      \sstsubsection{
         INPUT2 = CHARACTER$*$(CHI\_\_SZNAME) (Given)
      }{
         Name of the second catalogue.
      }
      \sstsubsection{
         EXPRESS = CHARACTER$*$(CHI\_\_SZEXP) (Given)
      }{
         Expression to be parsed.
      }
      \sstsubsection{
         FNAMES(CHI\_\_NUMFLDS) = CHARACTER$*$(CHI\_\_SZFNMAE) (Returned)
      }{
         Names of the fields.
      }
      \sstsubsection{
         FTYPES(CHI\_\_NUMFLDS) = CHARACTER$*$(1) (Returned)
      }{
         Types of the fields.
      }
      \sstsubsection{
         STATUS = INTEGER (Given and Returned)
      }{
         The global status.
      }
   }
   \sstdiytopic{
      Anticipated Errors
   }{
      CHI\_\_CATNOTFND
   }
}
\sstroutine{
   CHI\_ADDIND
}{
   Add a new sort index to a sorted catalogue
}{
   \sstdescription{
      Add a new sort index to a sorted catalogue. Creating sort indexes
      associated with catalogues allows efficient searching and joining in
      certain circumstances.

      Consider the data in a telephone directory catalogue. The data is already
      sorted on the field names  SURNAME (Primary field) FIRSTINITIAL
      (Secondary field) and SECONDINITIAL (Tertiary field). See CHI\_SORT.

      It is possible to more indexes to the catalogue. In the above example as
      well as finding the telephone number for a given name we may also want
      to find the surname the person given the telephone number. To do this
      efficiently the catalogue would have to be ordered by field TELNUMBER.
      This is achieved by adding an additional index sort on field TELNUMBER.

      The order of field name in the SORTFLDS array is significant.
      SORTFLDS(1) must contain the primary field, SORTFLDS(2) and
      SORTFLDS(3) contain the secondary and tertiary fields.
      Spaces in either the secondary or tertiary position simply indicates
      that there should be no secondary or tertiary ordering.

      The direction of the sort for each field in given in the corresponding
      element of the SORTDIR array. TRUE for ascending.
   }
   \sstinvocation{
      CALL CHI\_ADDIND( INPUT, SORTFLDS, SORTDIR, STATUS )
   }
   \sstarguments{
      \sstsubsection{
         INPUT = CHARACTER $*$ ( CHI\_\_SZNAME ) (Given)
      }{
         Name of the catalogue to be sorted.
      }
      \sstsubsection{
         SORTFLDS( 3 ) = CHARACTER $*$ ( CHI\_\_SZFNAME ) (Given)
      }{
         Names of the sort fields.
      }
      \sstsubsection{
         SORTDIR( 3 ) = LOGICAL (Given)
      }{
         Direction of sort for each field. (TRUE for ascending).
      }
      \sstsubsection{
         STATUS = INTEGER (Given and Returned)
      }{
         Global status.
      }
   }
   \sstnotes{
      If a field name does not exist in the catalogue an error will be reported.
   }
   \sstdiytopic{
      Anticipated Errors
   }{
      CHI\_\_CATNOTFND \\
      CHI\_\_FLDNOTFND
   }
}
\sstroutine{
   CHI\_ADDP
}{
   Add a parameter to a catalogue
}{
   \sstdescription{
      Add a new parameter to a catalogue. A parameter consist of a NAME,
      FORMAT, VALUE and COMMENT.
      The format is checked against the value and an error is reported
      if an inconsistency is found.
   }
   \sstinvocation{
      CALL CHI\_ADDP( INPUT, PNAME, PFORMAT, PVALUE, PCOMMENT, STATUS )
   }
   \sstarguments{
      \sstsubsection{
         INPUT = CHARACTER $*$ ( CHI\_\_SZNAME ) (Given)
      }{
         Name of the catalogue to which the parameter is to be added.
      }
      \sstsubsection{
         PNAME = CHARACTER $*$ ( CHI\_\_SZPNAME ) (Given)
      }{
         Name of the new parameter.
      }
      \sstsubsection{
         PFORMAT = CHARACTER $*$ ( CHI\_\_SZPFMT ) (Given)
      }{
         Format of the new parameter.
      }
      \sstsubsection{
         PVALUE = CHARACTER $*$ ( CHI\_\_SVPVAL ) (Given)
      }{
         Value of the new parameter.
      }
      \sstsubsection{
         PCOMMENT = CHARACTER $*$ ( CHI\_\_SZPCMT ) (Given)
      }{
         Comment associated with the new parameter.
      }
   }
   \sstnotes{
      CHI\_ADDP will not overwrite an existing parameter, an error is reported.
      Use CHI\_UPPAR to update the information associated with an existing
      parameter or CHI\_DELP to delete the parameter.
   }
   \sstdiytopic{
      Anticipated Errors
   }{
      CHI\_\_CATNOTFND \\
      CHI\_\_PAREXISTS \\
      CHI\_\_IVLDPFMT
   }
}
\sstroutine{
   CHI\_APPLY
}{
   Apply an expression to a set of data
}{
   \sstdescription{
      This routine will not normally be required because all common uses of
      the parser have been anticipated and included in other CHI routines.
      CHI\_SEARCH, CHI\_REJECT, CHI\_JOIN, CHI\_UPDATE and CHI\_CALCFLD call
      the parser on you behalf. CHI\_1PAR and CHI\_2PAR give you direct access
      to the parser when 1 or 2 catalogues are involved and must be called
      before CHI\_APPLY can be called. This routine applies the expression to
      a set of data. Data must be of the correct type and supplied in the
      correct position as defined by CHI\_1PAR or CHI\_2PAR. CHI\_1PAR and
      CHI\_2PAR return the
      fieldnames and the fieldtypes which specify how the data is expected
      for CHI\_APPLY. If the FNAMES(1) is STARNAME and FTYPE(1) is character
      then CHARVALS(1) must contain the value of the STARNAME field as
      given in the next entry in the catalogue. If the FNAMES(3) is VALUE1
      and FTYPE(3) is real then REALVALS(3) must contain the value of the
      VALUE1 field as given in the next entry in the catalogue.
      For ease of use the CHI\_1PAR routine expects values in the same form
      CHI\_GETVALALL returns so a CHI\_1PAR followed by a sequence of
      CHI\_GETVALALL and CHI\_APPLY will produce the correct result.
   }
   \sstinvocation{
      CALL CHI\_APPLY( CVALS, DVALS, IVALS, LVALS, RVALS, NUMFLDS,
\newline
                  CVALUE, DVALUE, IVALUE, LVALUE, RVALUE, RESTYPE,
\newline
                 STATUS )
   }
   \sstarguments{
      \sstsubsection{
         CVALS( CHI\_\_NUMFLDS ) = CHARACTER$*$(CHI\_\_SZCVAL) (Given)
      }{
         Character value being supplied.
      }
      \sstsubsection{
         DVALS( CHI\_\_NUMFLDS ) = DOUBLE PRECISION (Given)
      }{
         Double precision value being supplied.
      }
      \sstsubsection{
         IVALS = INTEGER (Given)
      }{
         Integer value being supplied.
      }
      \sstsubsection{
         LVALS = LOGICAL (Given)
      }{
         Logical value being supplied.
      }
      \sstsubsection{
         RVALS = REAL (Given)
      }{
         Real value returned.
      }
      \sstsubsection{
         NUMFLDS = INTEGER (Given)
      }{
         Number of fields for which data is being supplied.
      }
      \sstsubsection{
         CVALUE = CHARACTER$*$(CHI\_\_SZCVAL) (Returned)
      }{
         Character value returned.
      }
      \sstsubsection{
         DVALUE = DOUBLE PRECISION (Returned)
      }{
         Double precision value returned.
      }
      \sstsubsection{
         IVALUE = INTEGER (Returned)
      }{
         Integer value returned.
      }
      \sstsubsection{
         LVALUE = LOGICAL (Returned)
      }{
         Logical value returned.
      }
      \sstsubsection{
         RVALUE = REAL (Returned)
      }{
         Real value returned.
      }
      \sstsubsection{
         RESTYPE = CHARACTER $*$ ( 1 ) (Returned)
      }{
         Result type (C,D,I,L,R)
      }
      \sstsubsection{
         STATUS = INTEGER (Given and Returned)
      }{
         The global status.
      }
   }
   \sstdiytopic{
      Anticipated Errors
   }{
      CHI\_\_CATNOTFND
   }
}
\sstroutine{
   CHI\_AVAILCATS
}{
   Find the names of the catalogues available to the current user
}{
   \sstdescription{
      Find the names of all the catalogues available to the current user or find
      just the names of the catalogues that have been created by the
      user.
   }
   \sstinvocation{
      CALL CHI\_AVAILCATS( ALL, CATNAMES, STATUS )
   }
   \sstarguments{
      \sstsubsection{
         ALL = LOGICAL (Given)
      }{
         Logical flag. Set to .TRUE. for all available catalogues to be
         returned and .FALSE. for only catalogues created by this user.
      }
      \sstsubsection{
         CATNAMES( CHI\_\_NUMCATS) =
      }{
         \textbf{CHARACTER $*$ ( CHI\_\_SZNAME ) (Returned)}
      }{ \\
         Names of the available catalogues.
      }
   }
   \sstnotes{
      Catalogues that were not created by the user, but which are still
      available to him, are likely to restricted access (Read Only).
   }
   \sstdiytopic{
      Anticipated Errors
   }{
      None
   }
}
\sstroutine{
   CHI\_CALCFLD
}{
   Create a new catalogue by adding a new field
}{
   \sstdescription{
      Create a new catalogue by adding a new field to an existing catalogue.
      The data values
      of the new field are calculated by applying the given EXPRESSion to each
      entry in the catalogue. The new field also requires a name,format, units,
      null value and comment. The format and the nullvalue of the field are
      checked to ensure that the format is valid and that the null
      value and format are consistent. If an error is reported the offending
      format is returned in FFORMAT.
   }
   \sstinvocation{
       CALL CHI\_CALCFLD( INPUT, OUTPUT, EXPRESS, FNAME, FFORMAT, FUNIT,
\newline
       FNULL, FCOMMENT,  STATUS )
   }
   \sstarguments{
      \sstsubsection{
         INPUT = CHARACTER $*$ ( CHI\_\_SZNAME ) (Given)
      }{
         Name of the catalogue.
      }
      \sstsubsection{
         OUTPUT = CHARACTER $*$ ( CHI\_\_SZNAME ) (Given)
      }{
         Name of the new catalogue that includes the new field.
      }
      \sstsubsection{
         EXPRESS = CHARACTER $*$ ( CHI\_\_SZEXPE ) (Given)
      }{
         Expression to be evaluated.
      }
      \sstsubsection{
         FNAME = CHARACTER $*$ ( CHI\_\_SZFNAME ) (Given)
      }{
         Name of the new field.
      }
      \sstsubsection{
         FFORMAT = CHARACTER $*$ ( CHI\_\_SZFFMT ) (Given)
      }{
         Format of the new field.
      }
      \sstsubsection{
         FUNIT = CHARACTER $*$ ( CHI\_\_SZFUNIT ) (Given)
      }{
         Units of the new field.
      }
      \sstsubsection{
         FNULL = CHARACTER $*$ ( CHI\_\_SZFNVAL ) (Given)
      }{
         Null value of the new field.
      }
      \sstsubsection{
         FCOMMENT = CHARACTER $*$ ( CHI\_\_SZFCMT ) (Given)
      }{
         Comment associated with the new field.
      }
      \sstsubsection{
         STATUS = INTEGER (Given and Returned)
      }{
         Global status.
      }
   }
   \sstdiytopic{
      Anticipated Errors
   }{
      CHI\_\_CATNOTFND \\
      CHI\_\_IVLDFFMT
   }
}
\sstroutine{
   CHI\_CLOSE
}{
   Close the CHI system
}{
   \sstdescription{
      Closes the CHI system and performs house keeping task to release
      resources. CHI\_OPEN should be the first CHI call in your
      application and CHI\_CLOSE the last.
   }
   \sstinvocation{
      CALL CHI\_CLOSE( STATUS )
   }
   \sstarguments{
      \sstsubsection{
         STATUS = INTEGER $*$ ( CHI\_\_SZNAME ) (Given and Returned)
      }{
         Global status.
      }
   }
   \sstdiytopic{
      Anticipated Errors
   }{
      None
   }
}
\sstroutine{
   CHI\_COPYCAT
}{
   Create a copy of a catalogue
}{
   \sstdescription{
      Create a copy of a catalogue.
   }
   \sstinvocation{
      CALL CHI\_COPYCAT( INPUT, OUTPUT, STATUS )
   }
   \sstarguments{
      \sstsubsection{
         INPUT = CHARACTER $*$ ( CHI\_\_SZNAME ) (Given)
      }{
         Name of the catalogue from which the entries are to be selected.
      }
      \sstsubsection{
         OUTPUT = CHARACTER $*$ ( CHI\_\_SZNAME ) (Given)
      }{
         Name of the new catalogue containing only the selected entries.
      }
      \sstsubsection{
         STATUS = INTEGER (Given and Returned)
      }{
         Global status.
      }
   }
   \sstdiytopic{
      Anticipated Errors
   }{
      CHI\_\_CATNOTFND
   }
}
\sstroutine{
   CHI\_CREATDUP
}{
   Create a duplicate catalogue with no entries
}{
   \sstdescription{
      Create a duplicate catalogue with the same parameters and the same
      fields but no entries. If the original catalogue contains ordering
      information this can be carried into the duplicate catalogue using the
      retain ordering argument (RETORDER). See the CHI\_PUT routines for
      details about adding entries to an ordered catalogue.
   }
   \sstinvocation{
      CALL CHI\_CREATDUP( INPUT, ESTNUMENTS, OUTPUT, RETORDER, STATUS )
   }
   \sstarguments{
      \sstsubsection{
         INPUT = CHARACTER $*$ ( CHI\_\_SZNAME ) (Given)
      }{
         Name of the catalogue to be duplicated.
      }
      \sstsubsection{
         OUTPUT = CHARACTER $*$ ( CHI\_\_SZNAME ) (Given)
      }{
         Name of the duplicate catalogue to be created.
      }
      \sstsubsection{
         ESTNUMENTS = INTEGER (Given)
      }{
         Estimate for the number of entries that will be put into the catalogue.
      }
      \sstsubsection{
         RETORDER = LOGICAL (Given)
      }{
         Set RETORDER to TRUE to retain the ordering information.
      }
      \sstsubsection{
         STATUS = INTEGER (Given and Returned)
      }{
         Global status.
      }
   }
   \sstdiytopic{
      Anticipated Errors
   }{
      CHI\_\_CATNOTFND
   }
}
\sstroutine{
   CHI\_DELCAT
}{
   Delete a catalogue
}{
   \sstdescription{
      Delete a catalogue from the system. Catalogues that do not belong
      to the user may be protected and will not be deleted.
   }
   \sstinvocation{
      CALL CHI\_DELCAT( INPUT, STATUS )
   }
   \sstarguments{
      \sstsubsection{
         INPUT = CHARACTER $*$ ( CHI\_\_SZNAME ) (Given)
      }{
         Name of the catalogue to be deleted.
      }
      \sstsubsection{
         STATUS = INTEGER (Given and Returned)
      }{
         Global status.
      }
   }
   \sstdiytopic{
      Anticipated Errors
   }{
      CHI\_\_CATNOTFND
   }
}
\sstroutine{
   CHI\_DELNOTES
}{
   Delete the notes, or part of the notes, associated with a catalogue
}{
   \sstdescription{
      Delete the notes, or part of the notes, associated with a catalogue. The
      notes are removed from the line starting at the position indicated
      by STARTPOS and finishing at the entry
      STARTPOS$+$NUMLINES-1.
      If you try to delete lines off the end of the note no error will be
      reported.
   }
   \sstinvocation{
      CALL CHI\_DELNOTES( INPUT, STARTPOS, NUMLINES, STATUS )
   }
   \sstarguments{
      \sstsubsection{
         INPUT = CHARACTER $*$ ( CHI\_\_SZNAME ) (Given)
      }{
         Name of the catalogue from which the note is required.
      }
      \sstsubsection{
         STARTPOS = INTEGER (Given)
      }{
         Line at which to start deleting the note.
      }
      \sstsubsection{
         NUMLINES = INTEGER (Given)
      }{
         Number of lines to be deleted from the note.
      }
      \sstsubsection{
         STATUS = INTEGER (Given and Returned)
      }{
         Global status.
      }
   }
   \sstdiytopic{
      Anticipated Errors
   }{
      CHI\_\_CATNOTFND
   }
}
\sstroutine{
   CHI\_DELP
}{
   Delete a parameter from a catalogue
}{
   \sstdescription{
      Delete a parameter from a catalogue.
   }
   \sstinvocation{
      CALL CHI\_DELP( INPUT, PNAME, STATUS )
   }
   \sstarguments{
      \sstsubsection{
         INPUT = CHARACTER $*$ ( CHI\_\_SZNAME ) (Given)
      }{
         Name of the catalogue from which the parameter is to be
         deleted.
      }
      \sstsubsection{
         PNAME = CHARACTER $*$ ( CHI\_\_SZPNAME ) (Given)
      }{
         Name of the parameter to be deleted.
      }
      \sstsubsection{
         STATUS = INTEGER (Given and Returned)
      }{
         Global status.
      }
   }
   \sstdiytopic{
      Anticipated Errors
   }{
      CHI\_\_CATNOTFND \\
      CHI\_\_PARNOTFND
   }
}
\sstroutine{
   CHI\_DELSORT
}{
   Delete the sort information from a catalogue
}{
   \sstdescription{
      Delete the sort information from a catalogue.
   }
   \sstinvocation{
      CALL CHI\_DELSORT( INPUT, STATUS )
   }
   \sstarguments{
      \sstsubsection{
         INPUT = CHARACTER $*$ ( CHI\_\_SZNAME ) (Given)
      }{
         Name of the catalogue.
      }
      \sstsubsection{
         STATUS = INTEGER (Given and Returned)
      }{
         Global status.
      }
   }
   \sstdiytopic{
      Anticipated Errors
   }{
      CHI\_\_CATNOTFND
   }
}
\sstroutine{
   CHI\_EVAL
}{
   Evaluate an expression using values from the next entry in the
   catalogue
}{
   \sstdescription{
      Evaluate an expression using values from the next entry in the
      catalogue. The data is taken from the next entry in the catalogue but
      see notes if you are mixing different CHI\_GET and CHI\_EVAL calls.
      If an invalid expression error is reported, EXPRESS is returned
      containing diagnostic information. Use CHI\_RESET to return to the start
      of the catalogue. The type of result returned is given in RESTYPE and
      the value will be found in one of CHARVAL, DOUBVAL, INTVAL, LOGVAL
      or REALVAL selected by RESTYPE being C,D,I,L,R respectively.

      CHECK=1. The lowest level of checking. The routine processes the
      expression only once. The routine remembers the expression and on
      subsequent calls the routine assumes that the same expression is to be
      used.

      CHECK=2. The routine processes the expression every time the routine
      is called. The expression can change between calls.
   }
   \sstinvocation{
       CALL CHI\_EVAL( INPUT, EXPRESS, CHECK, CHARVAL, DOUBVAL, INTVAL,
\newline
                     LOGVAL, REALVAL, RESTYPE, STATUS )
   }
   \sstarguments{
      \sstsubsection{
         INPUT = CHARACTER $*$ ( CHI\_\_SZNAME ) (Given)
      }{
         Name of the catalogue.
      }
      \sstsubsection{
         EXPRESS = CHARACTER $*$ ( CHI\_\_SZEXP ) (Given)
      }{
         Expression to be applied to the next entry in the catalogue.
      }
      \sstsubsection{
         CHECK = INTEGER (Given)
      }{
         Set to 1 or 2 according to the level of checking required.
      }
      \sstsubsection{
         CHARVAL = CHARACTER $*$ (CHI\_\_SZCVAL ) (Returned)
      }{
         Result of a character evaluation.
      }
      \sstsubsection{
         DOUBVAL = DOUBLE PRECISION (Returned)
      }{
         Result of a double precision evaluation.
      }
      \sstsubsection{
         INTVAL = INTEGER (Returned)
      }{
         Result of an integer evaluation.
      }
      \sstsubsection{
         LOGVAL = LOGICAL (Returned)
      }{
         Result of the evaluation.
      }
      \sstsubsection{
         REALVAL = REAL (Returned)
      }{
         Result of a real evaluation.
      }
      \sstsubsection{
         RESTYPE = CHARACTER $*$ ( 1 ) (Returned)
      }{
         Type of result being returned (C,D,I,L,R)
      }
      \sstsubsection{
         STATUS = INTEGER (Given and Returned)
      }{
         Global status.
      }
   }
   \sstnotes{
      All the CHI\_GET and CHI\_EVAL routines are interlinked so a CHI\_GETVALR
      followed by a CHI\_GETVALC will cause the character values to be taken
      from the 2nd entry in the catalogue. In the same way a CHI\_GETSVALR
      getting data from 10 entries starting at entry 5 followed by a CHI\_EVAL
      will cause the arithmetic expression to be evaluated for the 15th entry
      in the catalogue. Use CHI\_RESET to return to the start of the catalogue.
   }
   \sstdiytopic{
      Anticipated Errors
   }{
      CHI\_\_CATNOTFND \\
      CHI\_\_IVLDEXP
   }
}
\sstroutine{
   CHI\_GETALLF
}{
   Get all the information about all fields
}{
   \sstdescription{
      Gets all the information about all the fields in a catalogue. Each
      field has associated with it a name, format, units, null value,
      comment and a type.
   }
   \sstinvocation{
      CALL CHI\_GETALLF( INPUT, NUMFLDS, FNAMES, FFORMATS, FUNITS,
\newline
      FNULLS, FCOMMENTS, FTYPES, STATUS )
   }
   \sstarguments{
      \sstsubsection{
         INPUT = CHARACTER $*$ ( CHI\_\_SZNAME ) (Given)
      }{
         Name of the catalogue from which the field information is
         required.
      }
      \sstsubsection{
         NUMFLDS = INTEGER (Returned)
      }{
         Number of fields in the catalogue.
      }
      \sstsubsection{
         FNAMES( CHI\_\_NUMFLDS ) =
      }{
         \textbf{CHARACTER $*$ ( CHI\_\_SZFNAME ) (Returned)}
      }{ \\
         Names of the fields in the catalogue.
      }
      \sstsubsection{
         FFORMATS( CHI\_\_NUMFLDS ) =
      }{
         \textbf{CHARACTER $*$ ( CHI\_\_SZFFMT ) (Returned)}
      }{ \\
         Formats of the fields in the catalogue.
      }
      \sstsubsection{
         FUNITS( CHI\_\_NUMFLDS ) =
      }{
         \textbf{CHARACTER $*$ ( CHI\_\_SZFUNIT ) (Returned)}
      }{ \\
         Units of the fields in the catalogue.
      }
      \sstsubsection{
         FNULLS( CHI\_\_NUMFLDS ) =
      }{
         \textbf{CHARACTER $*$ ( CHI\_\_SZFNVAL ) (Returned)}
      }{ \\
         Null values of the fields in the catalogue.
      }
      \sstsubsection{
         FCOMMENTS( CHI\_\_NUMFLDS ) =
      }{
         \textbf{CHARACTER $*$ ( CHI\_\_SZFCMT ) (Returned)}
      }{ \\
         Comments associated with the fields in the catalogue.
      }
      \sstsubsection{
         FTYPES( CHI\_\_NUMFLDS ) = CHARACTER $*$ ( 1 ) (Returned)
      }{
         Types of the fields in the catalogue (C,D,I,L or R).
      }
      \sstsubsection{
         STATUS = INTEGER (Given and Returned)
      }{
         Global status.
      }
   }
   \sstdiytopic{
      Anticipated Errors
   }{
      CHI\_\_CATNOTFND
   }
}
\sstroutine{
   CHI\_GETALLP
}{
   Get all the information about all parameters
}{
   \sstdescription{
      Gets all the information about all the parameters in a catalogue.
      Each parameter has associated with it a name, format, value and
      comment.
   }
   \sstinvocation{
      CALL CHI\_GETALLP( INPUT, NUMPARS, PNAMES, PFORMATS, PVALUES,
\newline
      PCOMMENTS, STATUS )
   }
   \sstarguments{
      \sstsubsection{
         INPUT = CHARACTER $*$ ( CHI\_\_SZNAME ) (Given)
      }{
         Name of the catalogue from which the parameter information is
         required.
      }
      \sstsubsection{
         NUMPARS = INTEGER (Returned)
      }{
         Number of parameters in the catalogue.
      }
      \sstsubsection{
         PNAMES( CHI\_\_NUMPARS ) =
      }{
         \textbf{CHARACTER $*$ ( CHI\_\_SZPNAME ) (Returned)}
      }{ \\
         Names of the parameters in the catalogue.
      }
      \sstsubsection{
         PFORMATS( CHI\_\_NUMPARS ) =
      }{
         \textbf{CHARACTER $*$ ( CHI\_\_SZPFMT ) (Returned)}
      }{ \\
         Formats of the parameters in the catalogue.
      }
      \sstsubsection{
         PVALUE( CHI\_\_NUMPARS ) = CHARACTER $*$ ( CHI\_\_SZPVAL ) (Returned)
      }{
         Values of the parameters in the catalogue.
      }
      \sstsubsection{
         PCOMMENTS( CHI\_\_NUMPARS ) =
      }{
         \textbf{CHARACTER $*$ ( CHI\_\_SZPCMT ) (Returned)}
      }{ \\
         Comments associated with the parameters in the catalogue.
      }
      \sstsubsection{
         STATUS = INTEGER (Given and Returned)
      }{
         Global status.
      }
   }
   \sstdiytopic{
      Anticipated Errors
   }{
      CHI\_\_CATNOTFND
   }
}
\sstroutine{
   CHI\_GETCD
}{
   Get the catalogue descriptor for an ADC catalogue
}{
   \sstdescription{
      Not part of the CHI interface definition. CHI\_GETCD checks if the ADC
      catalogue is already open and returns the catalogue descriptor. If
      the catalogue is not yet open it is opened and the catalogue descriptor
      is returned.
   }
   \sstinvocation{
      CALL CHI\_GETCD( INPUT, ACCMODE, CD, STATUS )
   }
   \sstarguments{
      \sstsubsection{
         INPUT = CHARACTER $*$ ( CHI\_\_SZNAME ) (Given)
      }{
         Name of the catalogue as known in the ADC system.
      }
      \sstsubsection{
         ACCMODE = CHARACTER $*$ ( 7 ) (Given)
      }{
         Access mode to catalogue
      }
      \sstsubsection{
         CD = INTEGER (Returned)
      }{
         Catalogue descriptor
      }
      \sstsubsection{
         STATUS = INTEGER
      }{
         Global status.
      }
   }
   \sstdiytopic{
      Anticipated Errors
   }{
      CHI\_\_CATNOTFND
   }
}
\sstroutine{
   CHI\_GETF
}{
   Get names of the fields in a catalogue
}{
   \sstdescription{
      Get the names of all the fields in a catalogue and count the
      number of fields found.
   }
   \sstinvocation{
      CALL CHI\_GETF( INPUT, FNAMES, NUMFLDS, STATUS )
   }
   \sstarguments{
      \sstsubsection{
         INPUT = CHARACTER $*$ ( CHI\_\_SZNAME ) (Given)
      }{
         Name of the catalogue from which the field names are required.
      }
      \sstsubsection{
         FNAMES( CHI\_\_NUMFLDS ) =
      }{
         \textbf{CHARACTER $*$ ( CHI\_\_SZFNAME ) (Returned)}
      }{ \\
         Names of the fields in the catalogue.
      }
      \sstsubsection{
         NUMFLDS = INTEGER (Returned)
      }{
         Number of fields in the catalogue.
      }
      \sstsubsection{
         STATUS = INTEGER (Given and Returned)
      }{
         Global status.
      }
   }
   \sstdiytopic{
      Anticipated Errors
   }{
      CHI\_\_CATNOTFND
   }
}
\sstroutine{
   CHI\_GETFINF
}{
   Get specific information about a field
}{
   \sstdescription{
      Gets a specific piece of information about a field. The
      information required is one from FORMAT, UNITS, NULLVALUE or
      COMMENT. For example get the format of the field FLUX.
   }
   \sstinvocation{
      CALL CHI\_GETFINF( INPUT, FNAME, FREQ, FVALUE, STATUS )
   }
   \sstarguments{
      \sstsubsection{
         INPUT = CHARACTER $*$ ( CHI\_\_SZNAME ) (Given)
      }{
         Name of the catalogue from which the field information is
         required.
      }
      \sstsubsection{
         FNAME = CHARACTER $*$ ( CHI\_\_SZFNAME ) (Given)
      }{
         Name of the field whose information is required.
      }
      \sstsubsection{
         FREQ = CHARACTER $*$ ( 9 ) (Given)
      }{
         Information required. One from FORMAT, UNITS, NULLVALE or
         COMMENT.
      }
      \sstsubsection{
         FVALUE = CHARACTER $*$ ( CHI\_\_SZFCMT ) (Returned)
      }{
         The required information.
      }
      \sstsubsection{
         STATUS = INTEGER (Given and Returned)
      }{
         Global status.
      }
   }
   \sstdiytopic{
      Anticipated Errors
   }{
      CHI\_\_CATNOTFND \\
      CHI\_\_FLDNOTFND \\
      CHI\_\_IVLDFREQ
   }
}
\sstroutine{
   CHI\_GETNOTES
}{
   Get the notes associated with a catalogue
}{
   \sstdescription{
      Get the notes associated with a catalogue. The notes are taken from
      the line starting at the position indicated
      by STARTPOS and finishing at the entry STARTPOS$+$NUMLINES-1. The note
      is returned in the BUFFER. If you try to read off the end of the note
      no error will be reported and the note will be retrieved and NUMLINES
      is returned with the valid number of lines found.
   }
   \sstinvocation{
      CALL CHI\_GETNOTES( INPUT, STARTPOS, NUMLINES, BUFFER, STATUS )
   }
   \sstarguments{
      \sstsubsection{
         INPUT = CHARACTER $*$ ( CHI\_\_SZNAME ) (Given)
      }{
         Name of the catalogue from which the note is required.
      }
      \sstsubsection{
         STARTPOS = INTEGER (Given)
      }{
         Line at which to start getting the note.
      }
      \sstsubsection{
         NUMLINES = INTEGER (Given and Returned)
      }{
         Number of lines from the note that are required.
      }
      \sstsubsection{
         BUFFER(NUMLINES) = CHARACTER $*$ ( 80 ) (Returned)
      }{
         Array to receive the note.
      }
      \sstsubsection{
         STATUS = INTEGER (Given and Returned)
      }{
         Global status.
      }
   }
   \sstdiytopic{
      Anticipated Errors
   }{
      CHI\_\_CATNOTFND \\
      CHI\_\_EOF
   }
}
\sstroutine{
   CHI\_GETNUMENTS
}{
   Get the number of entries in a catalogue
}{
   \sstdescription{
      Get the number of entries in a catalogue.
   }
   \sstinvocation{
      CALL CHI\_GETNUMENTS( INPUT, NUMENTS, STATUS )
   }
   \sstarguments{
      \sstsubsection{
         INPUT = CHARACTER $*$ ( CHI\_\_SZNAME ) (Given)
      }{
         Name of the catalogue.
      }
      \sstsubsection{
         NUMENTS = INTEGER (Returned)
      }{
         Number of entries in the catalogue.
      }
      \sstsubsection{
         STATUS = INTEGER
      }{
         Global status.
      }
   }
   \sstdiytopic{
      Anticipated Errors
   }{
      CHI\_\_CATNOTFND
   }
}
\sstroutine{
   CHI\_GETONEF
}{
   Get all the information about one field
}{
   \sstdescription{
      Gets all the information about one field in a catalogue. Each
      field has associated with it a name, format, units, null value,
      comment and a type.
   }
   \sstinvocation{
      CALL CHI\_GETONEF( INPUT, FNAME, FFORMAT, FUNIT,
\newline
      FNULL, FCOMMENT, FTYPE, STATUS )
   }
   \sstarguments{
      \sstsubsection{
         INPUT = CHARACTER $*$ ( CHI\_\_SZNAME ) (Given)
      }{
         Name of the catalogue from which the field information is
         required.
      }
      \sstsubsection{
         FNAME = CHARACTER $*$ ( CHI\_\_SZFNAME ) (Given)
      }{
         Name of the field whose information is required.
      }
      \sstsubsection{
         FFORMAT = CHARACTER $*$ ( CHI\_\_SZFFMT ) (Returned)
      }{
         Format of the field.
      }
      \sstsubsection{
         FUNIT = CHARACTER $*$ ( CHI\_\_SZFUNIT ) (Returned)
      }{
         Units of the field.
      }
      \sstsubsection{
         FNULL = CHARACTER $*$ ( CHI\_\_SZFNVAL ) (Returned)
      }{
         Null value of the field.
      }
      \sstsubsection{
         FCOMMENT = CHARACTER $*$ ( CHI\_\_SZFCMT ) (Returned)
      }{
         Comment associated with the field.
      }
      \sstsubsection{
         FTYPE = CHARACTER $*$ ( 1 ) (Returned)
      }{
         Type of the field in the catalogue (C,D,I,L or R).
      }
      \sstsubsection{
         STATUS = INTEGER (Given and Returned)
      }{
         Global status.
      }
   }
   \sstdiytopic{
      Anticipated Errors
   }{
      CHI\_\_CATNOTFND \\
      CHI\_\_FLDNOTFND
   }
}
\sstroutine{
   CHI\_GETONEP
}{
   Get all the information about one parameter
}{
   \sstdescription{
      Gets all the information about one parameter in a catalogue.
      Each parameter has associated with it a name, format, value and
      comment.
   }
   \sstinvocation{
      CALL CHI\_GETONEP( INPUT, PNAME, PFORMAT, PVALUE,
\newline
      PCOMMENT, STATUS )
   }
   \sstarguments{
      \sstsubsection{
         INPUT = CHARACTER $*$ ( CHI\_\_SZNAME ) (Given)
      }{
         Name of the catalogue from which the parameter information is
         required.
      }
      \sstsubsection{
         PNAME = CHARACTER $*$ ( CHI\_\_SZPNAME ) (Given)
      }{
         Name of the parameter whose information is required.
      }
      \sstsubsection{
         PFORMAT = CHARACTER $*$ ( CHI\_\_SZPFMT ) (Returned)
      }{
         Format of the parameter.
      }
      \sstsubsection{
         PVALUE = CHARACTER $*$ ( CHI\_\_SZPVAL ) (Returned)
      }{
         Value of the parameter.
      }
      \sstsubsection{
         PCOMMENT = CHARACTER $*$ ( CHI\_\_SZPCMT ) (Returned)
      }{
         Comment associated with the parameter.
      }
      \sstsubsection{
         STATUS = INTEGER (Given and Returned)
      }{
         Global status.
      }
   }
   \sstdiytopic{
      Anticipated Errors
   }{
      CHI\_\_CATNOTFND \\
      CHI\_\_PARNOTFND
   }
}
\sstroutine{
   CHI\_GETP
}{
   Get names of the parameters in a catalogue
}{
   \sstdescription{
      Get the names of all the parameters in a catalogue. Also count the
      number of parameters found.
   }
   \sstinvocation{
      CALL CHI\_GETP( INPUT, PNAMES, NUMPARS, STATUS )
   }
   \sstarguments{
      \sstsubsection{
         INPUT = CHARACTER $*$ ( CHI\_\_SZNAME ) (Given)
      }{
         Name of the catalogue from which the parameter names are
         required.
      }
      \sstsubsection{
         PNAMES( CHI\_\_NUMPARS ) =
      }{
         \textbf{CHARACTER $*$ ( CHI\_\_SZPNAME ) (Returned)}
      }{ \\
         Names of the parameters in the catalogue.
      }
      \sstsubsection{
         NUMPARS = INTEGER (Returned)
      }{
         Number of parameters in the catalogue.
      }
      \sstsubsection{
         STATUS = INTEGER (Given and Returned)
      }{
         Global status.
      }
   }
   \sstdiytopic{
      Anticipated Errors
   }{
      CHI\_\_CATNOTFND
   }
}
\sstroutine{
   CHI\_GETPINF
}{
   Get specific information about a parameter
}{
   \sstdescription{
      Gets a specific piece of information about a parameter. The
      information required is one from FORMAT, VALUE or
      COMMENT. For example get the value of the parameter AUTHOR.
   }
   \sstinvocation{
      CALL CHI\_GETPINF( INPUT, PNAME, PREQ, PVALUE, STATUS )
   }
   \sstarguments{
      \sstsubsection{
         INPUT = CHARACTER $*$ ( CHI\_\_SZNAME ) (Given)
      }{
         Name of the catalogue from which the parameter information is
         required.
      }
      \sstsubsection{
         PNAME = CHARACTER $*$ ( CHI\_\_SZPNAME ) (Given)
      }{
         Name of the parameter whose information is required.
      }
      \sstsubsection{
         PREQ = CHARACTER $*$ ( 9 ) (Given)
      }{
         Information required. One from FORMAT, VALUE or
         COMMENT.
      }
      \sstsubsection{
         PVALUE = CHARACTER $*$ ( CHI\_\_SZPCMT ) (Returned)
      }{
         The required information.
      }
      \sstsubsection{
         STATUS = INTEGER (Given and Returned)
      }{
         Global status.
      }
   }
   \sstdiytopic{
      Anticipated Errors
   }{
      CHI\_\_CATNOTFND \\
      CHI\_\_PARNOTFND \\
      CHI\_\_IVLDPREQ
   }
}
\sstroutine{
   CHI\_GETSVALALL
}{
   Get columns of data from all fields in a catalogue
}{
   \sstdescription{
      Get columns of data from all fields in a catalogue. The data is taken from
      the entries in the catalogue starting at the position indicated
      by STARTPOS and finishing at the entry STARTPOS$+$NUMENTS-1. Data
      is returned in the respective column of the appropriate array (INTVALS,
      REALVALS, LOGVALS, DOUBVALS or CHARVALS). The type of the field is given
      in the respective column of FLDTYPES (C,D,I,L or R). So, if the field
      type of the field whose name is in FNAMES(1) is character then
      FLDTYPES(1) will be C and the data will be returned in
      CHARVALS(1,1) (2,1) (3,1) ....(NUMENTS,1).
      If you try to get data off the
      end of the catalogue an error will be reported but the data to the end
      of the catalogue will be retrieved and NUMENTS is returned with the number
      of valid entries found.

      See CHI\_GETVALALL for a simpler version of this routine that gets
      data from the next entry in the catalogue.
   }
   \sstinvocation{
      CALL CHI\_GETSVALALL( INPUT, FNAMES, NUMFLDS, STARTPOS, NUMENTS,
\newline
      CHARVALS, DOUBVALS, INTVALS, LOGVALS, REALVALS, FLDTYPES,
\newline
      STATUS )
   }
   \sstarguments{
      \sstsubsection{
         INPUT = CHARACTER $*$ ( CHI\_\_SZNAME ) (Given)
      }{
         Name of the catalogue from which the field names are required.
      }
      \sstsubsection{
         FNAMES( CHI\_\_NUMFLDS ) =
      }{
         \textbf{CHARACTER $*$ ( CHI\_\_SZFNAME ) (Returned)}
      }{ \\
         Names of the fields in the catalogue.
      }
      \sstsubsection{
         NUMFLDS = INTEGER (Returned)
      }{
         Number of fields in the catalogue.
      }
      \sstsubsection{
         STARTPOS = INTEGER (Given)
      }{
         Position at which to start getting the data.
      }
      \sstsubsection{
         NUMENTS = INTEGER (Given and Returned)
      }{
         Number of entries from which data is required.
      }
      \sstsubsection{
         CHARVALS(NUMENTS,CHI\_\_NUMFLDS) =
      }{
         \textbf{CHARACTER $*$ ( CHI\_\_SZCVAL ) (Returned)}
      }{ \\
         Array to receive the character data.
      }
      \sstsubsection{
         DOUBVALS(NUMENTS,CHI\_\_NUMFLDS) =
      }{
         \textbf{DOUBLE PRECISION (Returned)}
      }{ \\
         Array to receive the double precision data.
      }
      \sstsubsection{
         INTVALS(NUMENTS,CHI\_\_NUMFLDS) = INTEGER (Returned)
      }{
         Array to receive the integer data.
      }
      \sstsubsection{
         LOGVALS(NUMENTS,CHI\_\_NUMFLDS) = LOGICAL (Returned)
      }{
         Array to receive the logical data.
      }
      \sstsubsection{
         REALVALS(NUMENTS,CHI\_\_NUMFLDS) = REAL (Returned)
      }{
         Array to receive the real data.
      }
      \sstsubsection{
         STATUS = INTEGER (Given and Returned)
      }{
         Global status.
      }
   }
   \sstdiytopic{
      Anticipated Errors
   }{
      CHI\_\_CATNOTFND \\
      CHI\_\_ENDOFCAT
   }
}
\sstroutine{
   CHI\_GETSVALx
}{
   Get columns of data from single type fields in a catalogue
}{
   \sstdescription{
      Get columns of  data from the given single type fields in a catalogue.
      Replace 'x' with C for character, D for double precision, I for integer,
      L for logical or R for real. Eg CHI\_GETSVALC.
      Only get the data from the fields given in FNAMES. The data is taken from
      the entries in the catalogue starting at the position indicated
      by STARTPOS and finishing at the entry }\\{STARTPOS$+$NUMENTS-1. Data
      is returned in the respective column of CHARVALS. So the
      data for the field whose name is in FNAMES(1) will be returned in
      CHARVALS(1,1) (2,1) (3,1) ....(NUMENTS,1).

      The routine checks the field names given in FNAMES. Only if
      the field appears in the catalogue and is type character will data be
      retrieved from the field. If, during
      checking, an error is found the error is reported and the name of the
      offending field is returned in FNAMES(1). If you try to get data off the
      end of the catalogue an error will be reported but the data to the end
      of the catalogue will be retrieved and NUMENTS is returned with the number
      of valid entries found.

      See CHI\_GETVALx for a simpler version of this routine that gets
      character data from the next entry in the catalogue.
   }
   \sstinvocation{
      CALL CHI\_GETSVALC( INPUT, FNAMES, NUMFLDS, STARTPOS, NUMENTS,
\newline
      CHARVALS, STATUS )
   }
   \sstarguments{
      \sstsubsection{
         INPUT = CHARACTER $*$ ( CHI\_\_SZNAME ) (Given)
      }{
         Name of the catalogue from which the data is required.
      }
      \sstsubsection{
         FNAMES( CHI\_\_NUMFLDS ) = CHARACTER $*$ ( CHI\_\_SZFNAME )
      }{
        (Given and Returned)
      }{
         Names of the fields whose data is required.
      }
      \sstsubsection{
         NUMFLDS = INTEGER (Given and Returned)
      }{
         Number of fields.
      }
      \sstsubsection{
         STARTPOS = INTEGER (Given)
      }{
         Position at which to start getting the data.
      }
      \sstsubsection{
         NUMENTS = INTEGER (Given and Returned)
      }{
         Number of entries from which data is required.
      }
      \sstsubsection{
         CHARVALS(NUMENTS,CHI\_\_NUMFLDS) =
      }{
         \textbf{CHARACTER $*$ ( CHI\_\_SZCVAL ) (Returned)}
      }{ \\
         Array to receive the data.
      }
      \sstsubsection{
         STATUS = INTEGER (Given and Returned)
      }{
         Global status.
      }
   }
   \sstdiytopic{
      Anticipated Errors
   }{
      CHI\_\_CATNOTFND \\
      CHI\_\_FLDNOTFND \\
      CHI\_\_ENDOFCAT \\
      CHI\_\_IVLDFLDTYP
   }
}
\sstroutine{
   CHI\_GETSVALM
}{
   Get columns of data from mixed fields in a catalogue
}{
   \sstdescription{
      Get columns of data from mixed fields in a catalogue.
      Only get the data from the fields given in FNAMES. The data is taken from
      the entries in the catalogue starting at the position indicated
      by STARTPOS and finishing at the entry STARTPOS$+$NUMENTS-1. Data
      is returned in the respective column of the appropriate array (INTVALS,
      REALVALS, LOGVALS, DOUBVALS or CHARVALS). The type of the field is given
      in the respective column of FLDTYPES (C,D,I,L or R). So, if the field
      type of the field whose name is in FNAMES(1) is character then
      FLDTYPES(1) will be C and the data will be returned in
      CHARVALS(1,1) (2,1) (3,1) ....(NUMENTS,1).

      The routine checks the field names given in FNAMES. Only if
      the field appears in the catalogue will data be retrieved from the field.
      If, during checking, an error is found the error is reported and the name
      of the offending field is returned in FNAMES(1). If you try to get data
      off the end of the catalogue an error will be reported but the data to
      the end of the catalogue will be retrieved and NUMENTS is returned with
      the number of valid entries found.

      See CHI\_GETVALM for a simpler version of this routine that gets
      data from the next entry in the catalogue.
   }
   \sstinvocation{
      CALL CHI\_GETSVALM( INPUT, FNAMES, NUMFLDS, STARTPOS, NUMENTS,
\newline
      CHARVALS, DOUBVALS, INTVALS, LOGVALS, REALVALS, FLDTYPES,
\newline
      STATUS )
   }
   \sstarguments{
      \sstsubsection{
         INPUT = CHARACTER $*$ ( CHI\_\_SZNAME ) (Given)
      }{
         Name of the catalogue from which the field names are required.
      }
      \sstsubsection{
         FNAMES( CHI\_\_NUMFLDS ) =
      }{
         \textbf{CHARACTER $*$ ( CHI\_\_SZFNAME ) (Returned)}
      }{ \\
         Names of the fields in the catalogue.
      }
      \sstsubsection{
         NUMFLDS = INTEGER (Returned)
      }{
         Number of fields in the catalogue.
      }
      \sstsubsection{
         STARTPOS = INTEGER (Given)
      }{
         Position at which to start getting the data.
      }
      \sstsubsection{
         NUMENTS = INTEGER (Given and Returned)
      }{
         Number of entries from which data is required.
      }
      \sstsubsection{
         CHARVALS(NUMENTS,CHI\_\_NUMFLDS) =
      }{
         \textbf{CHARACTER $*$ ( CHI\_\_SZCVAL ) (Returned)}
      }{ \\
         Array to receive the character data.
      }
      \sstsubsection{
         DOUBVALS(NUMENTS,CHI\_\_NUMFLDS) =
      }{
         \textbf{DOUBLE PRECISION (Returned)}
      }{ \\
         Array to receive the double precision data.
      }
      \sstsubsection{
         INTVALS(NUMENTS,CHI\_\_NUMFLDS) = INTEGER (Returned)
      }{
         Array to receive the integer data.
      }
      \sstsubsection{
         LOGVALS(NUMENTS,CHI\_\_NUMFLDS) = LOGICAL (Returned)
      }{
         Array to receive the logical data.
      }
      \sstsubsection{
         REALVALS(NUMENTS,CHI\_\_NUMFLDS) = REAL (Returned)
      }{
         Array to receive the real data.
      }
      \sstsubsection{
         FLDTYPES( CHI\_\_NUMFLDS ) = CHARACTER $*$ ( 1 ) (Returned)
      }{
         Array to receive the field types one of C,D,I,L,R
      }
      \sstsubsection{
         STATUS = INTEGER (Given and Returned)
      }{
         Global status.
      }
   }
   \sstdiytopic{
      Anticipated Errors
   }{
      CHI\_\_CATNOTFND \\
      CHI\_\_FLDNOTFND \\
      CHI\_\_ENDOFCAT
   }
}
\sstroutine{
   CHI\_GETVALALL
}{
   Get all the data from the next entry in a catalogue
}{
   \sstdescription{
      Get all the data from the next entry in a catalogue. Get data from all
      the fields in the catalogue. The data is taken from the next entry in the
      catalogue  but see Notes if you are mixing different CHI\_GET calls.
      Data is returned in the appropriate element of the appropriate
      array. FNAMES will contain the names of the fields and FLDTYPES the
      type of field. So if FLDTYPES(3) is an \texttt{'}I\texttt{'} the array
      INTVALS(3) will contain the data from the field whose name is given in
      FNAMES(3).
   }
   \sstinvocation{
      CALL CHI\_GETVALALL( INPUT, FNAMES, NUMFLDS, CHARVALS, DOUBVALS,
\newline
      INTVALS, LOGVALS, REALVALS, FLDTYPES, STATUS )
   }
   \sstarguments{
      \sstsubsection{
         INPUT = CHARACTER $*$ ( CHI\_\_SZNAME ) (Given)
      }{
         Name of the catalogue from which the data is required.
      }
      \sstsubsection{
         FNAMES( CHI\_\_NUMFLDS ) =
      }{
         \textbf{CHARACTER $*$ ( CHI\_\_SZFNAME ) (Returned)}
      }{ \\
         Names of the fields whose data is required.
      }
      \sstsubsection{
         NUMFLDS = INTEGER (Returned)
      }{
         Number of fields whose data is required.
      }
      \sstsubsection{
         INTVALS( CHI\_\_NUMFLDS ) = INTEGER (Returned)
      }{
         Array to receive the data from integer fields.
      }
      \sstsubsection{
         REALVALS( CHI\_\_NUMFLDS ) = REAL (Returned)
      }{
         Array to receive the data from real fields.
      }
      \sstsubsection{
         DOUBVALS( CHI\_\_NUMFLDS ) = DOUBLE PRECISION (Returned)
      }{
         Array to receive the data from double precision fields.
      }
      \sstsubsection{
         LOGVALS( CHI\_\_NUMFLDS ) = LOGICAL (Returned)
      }{
         Array to receive the data from logical fields.
      }
      \sstsubsection{
         CHARVALS( CHI\_\_NUMFLDS ) =
      }{
         \textbf{CHARACTER $*$ ( CHI\_\_SZCVAL ) (Returned)}
      }{ \\
         Array to receive the data from character fields.
      }
      \sstsubsection{
         FLDTYPES( CHI\_\_NUMFLDS ) = CHARACTER $*$ ( 1 ) (Returned)
      }{
         Array to receive the types of fields.
      }
      \sstsubsection{
         STATUS = INTEGER (Given and Returned)
      }{
         Global status.
      }
   }
   \sstnotes{
      All the CHI\_GET and CHI\_EVAL routines are interlinked so a CHI\_GETVALR
      followed by a CHI\_GETVALC will cause the character values to be taken
      from the 2nd entry in the catalogue. In the same way a CHI\_GETSVALR
      getting data from 10 entries starting at entry 5 followed by a CHI\_EVALA
      will cause the arithmetic expression to be evaluated for the 15th entry
      in the catalogue. Use CHI\_RESET to return to the start of the catalogue.
   }
   \sstdiytopic{
      Anticipated Errors
   }{
      CHI\_\_CATNOTFND
   }
}
\sstroutine{
   CHI\_GETVALx
}{
   Get data from single type fields in the next entry in a catalogue
}{
   \sstdescription{
      Gets data from character fields in the next entry in a catalogue.
      Replace 'x' with C for character, D for double precision, I for integer,
      L for logical or R for real. Eg CHI\_GETVALC.
      Only
      get the data from the fields given in FNAMES. The data is taken from the
      next entry in the catalogue but see Notes if you are mixing different
      CHI\_GET and CHI\_EVAL calls. Data is returned in the CHARVALS array
      so if FNAMES(3) is the name of a field, the array CHARVALS(3) will
      contain the data from that field.
      Use CHI\_RESET to start at the begining of the catalogue again.

      There are two levels of checking which are selected using the argument
      CHECK. Mode 1 is more efficient and should be used whenever possible.

      CHECK=1. The routine checks the field names given in FNAMES. Only if
      the field appears in the catalogue and is type character will data be
      retrieved from the field. If, during
      checking, an error is found the error is reported and the name of the
      offending field is returned in FNAMES(1)
      The routine remembers where it has to put the retrieved data for each
      field (\emph{e.g.}, The value of the STARID field has to be put into CHARVALS(5)).
      On subsequent calls the routine assumes that it should place the data
      from the STARID into CHARVALS(5).

      CHECK=2. In this mode the routine processes the field names given in
      FNAMES every time the routine is called. Only if the field
      appears in the catalogue and
      is type character will data be retrieved from the field. If, during
      checking, an error is found the error is reported and the name of the
      offending field is returned in FNAMES(1).
      Because the fieldnames are checked every time the routine is called
      you can retrieve data from different character fields in each entry in
      the catalogue.
   }
   \sstinvocation{
      CALL CHI\_GETVALC( INPUT, FNAMES, NUMFLDS, CHECK, CHARVALS, STATUS )
   }
   \sstarguments{
      \sstsubsection{
         INPUT = CHARACTER $*$ ( CHI\_\_SZNAME ) (Given)
      }{
         Name of the catalogue from which the data is required.
      }
      \sstsubsection{
         FNAMES( CHI\_\_NUMFLDS ) = CHARACTER $*$ ( CHI\_\_SZFNAME ) (Given)
      }{
         Names of the fields whose data is required.
      }
      \sstsubsection{
         NUMFLDS = INTEGER (Given)
      }{
         Number of fields whose data is required.
      }
      \sstsubsection{
         CHECK = INTEGER (Given)
      }{
         Set to 1 or 2 according to the level of checking you require.
      }
      \sstsubsection{
         CHARVALS( CHI\_\_NUMFLDS ) =
      }{
         \textbf{CHARACTER $*$ ( CHI\_\_SZCVAL ) (Returned)}
      }{ \\
         Array to receive the data from character fields.
      }
      \sstsubsection{
         STATUS = INTEGER (Given and Returned)
      }{
         Global status.
      }
   }
   \sstnotes{
      All the CHI\_GET and CHI\_EVAL routines are interlinked so a CHI\_GETVALR
      followed by a CHI\_GETVALC will cause the character values to be taken
      from the 2nd entry in the catalogue. In the same way a CHI\_GETSVALR
      getting data from 10 entries starting at entry 5 followed by a CHI\_EVALA
      will cause the arithmetic expression to be evaluated for the 15th entry
      in the catalogue. Use CHI\_RESET to return to the start of the catalogue.
   }
   \sstdiytopic{
      Anticipated Errors
   }{
      CHI\_\_CATNOTFND \\
      CHI\_\_FLDNOTFND
   }
}
\sstroutine{
   CHI\_GETVALM
}{
   Get data from mixed fields in the next entry in a catalogue
}{
   \sstdescription{
      Gets data from the next entry in a catalogue. Only get data from the
      fields given in FNAMES. The data is taken from the next entry in the
      catalogue but see Notes if you are mixing different CHI\_GET and
      CHI\_EVAL calls. Data is returned in the appropriate element of the
      appropriate array. So if FNAMES(1) is the name of an integer field then
      FLDTYPES(1) will be \texttt{'}I\texttt{'} and the array INTVALS(1) will
      contain the data from the field.

      There are two levels of checking which are selected using the argument
      CHECK. Modes 1 is more efficient and should be used whenever possible.

      CHECK=1. The routine checks the field names given in FNAMES. Only if
      the field appears in the catalogue will data be
      retrieved from the field. If, during
      checking, an error is found the error is reported and the name of the
      offending field is returned in FNAMES(1)
      The routine remembers where it has to put the retrieved data for each
      field (\emph{e.g.}, The value of the STARID field has to be put into CHARVALS(5)).
      On subsequent calls the routine assumes that it should place the data
      from the STARID into CHARVALS(5). Use this mode when you know that the
      field names and types are correct.

      CHECK=2. In this mode the routine processes the field names given in
      FNAMES every time the routine is called. Only
      if the all the fields in appear in the catalogue
      will the data be retrieved from the catalogue. If, during
      checking, an error is found the error is reported and the name of the
      offending field is returned in FNAMES(1).
      Because the fieldnames are checked every time the routine is called
      you can retrieve data from different character fields in each entry in
      the catalogue.
   }
   \sstinvocation{
      CALL CHI\_GETVALM( INPUT, FNAMES, NUMFLDS, CHECK, CHARVALS, DOUBVALS,
\newline
      INTVALS, LOGVALS, REALVALS, FLDTYPES, STATUS )
   }
   \sstarguments{
      \sstsubsection{
         INPUT = CHARACTER $*$ ( CHI\_\_SZNAME ) (Given)
      }{
         Name of the catalogue from which the data is required.
      }
      \sstsubsection{
         FNAMES( CHI\_\_NUMFLDS ) = CHARACTER $*$ ( CHI\_\_SZFNAME ) (Given)
      }{
         Names of the fields whose data is required.
      }
      \sstsubsection{
         NUMFLDS = INTEGER (Given)
      }{
         Number of fields whose data is required.
      }
      \sstsubsection{
         CHECK = INTEGER (Given)
      }{
         Set to 1 or 2 according to the level of checking you require.
      }
      \sstsubsection{
         INTVALS( CHI\_\_NUMFLDS ) = INTEGER (Returned)
      }{
         Array to receive the data from integer fields.
      }
      \sstsubsection{
         REALVALS( CHI\_\_NUMFLDS ) = REAL (Returned)
      }{
         Array to receive the data from real fields.
      }
      \sstsubsection{
         DOUBVALS( CHI\_\_NUMFLDS ) = DOUBLE PRECISION (Returned)
      }{
         Array to receive the data from double precision fields.
      }
      \sstsubsection{
         LOGVALS( CHI\_\_NUMFLDS ) = LOGICAL (Returned)
      }{
         Array to receive the data from logical fields.
      }
      \sstsubsection{
         CHARVALS( CHI\_\_NUMFLDS ) =
      }{
         \textbf{CHARACTER $*$ ( CHI\_\_SZCVAL ) (Returned)}
      }{ \\
         Array to receive the data from character fields.
      }
      \sstsubsection{
         FLDTYPES( CHI\_\_NUMFLDS ) = CHARACTER $*$ ( 1 ) (Returned)
      }{
         Array to receive the types of fields.
      }
      \sstsubsection{
         STATUS = INTEGER (Given and Returned)
      }{
         Global status.
      }
   }
   \sstnotes{
      All the CHI\_GET and CHI\_EVAL routines are interlinked so a CHI\_GETVALR
      followed by a CHI\_GETVALC will cause the character values to be taken
      from the 2nd entry in the catalogue. In the same way a CHI\_GETSVALR
      getting data from 10 entries starting at entry 5 followed by a CHI\_EVALA
      will cause the arithmetic expression to be evaluated for the 15th entry
      in the catalogue. Use CHI\_RESET to return to the start of the catalogue.
   }
   \sstdiytopic{
      Anticipated Errors
   }{
      CHI\_\_CATNOTFND \\
      CHI\_\_FLDNOTFND
   }
}
\sstroutine{
   CHI\_JOIN
}{
   Create a new catalogue by joining two catalogues
}{
   \sstdescription{
      Create a new catalogue by joining two catalogues. The effect of the join
      is as follows. Consider a large catalogue that contains all the fields
      from the INPUT1 catalogue and all the fields from the INPUT2 catalogue.
      Into this catalogue put an entry for each combination of entries in
      catalogues INPUT1 and INPUT2. The resulting catalogue will have N$*$M
      entries where N is the number of entries in the INPUT1 catalogue and
      M the number in the INPUT2 catalogue. Now search this catalogue for
      those entries that satisfy the given expression.

      Another way of looking at join is to say. Take every entry in turn
      from catalogue INPUT1. Match this entry with every entry in
      catalogue INPUT2 and if the EXPRESSion in satisfied combine both entries
      to write to a new catalogue.

      CHI\_JOIN tries to perform an efficient join using CHI\_TURBOJOIN
      but if this fails the error is cancelled and the routine goes on to
      to check all the N$*$M combinations of entries against the expression as
      described above.
   }
   \sstinvocation{
      CALL CHI\_JOIN( INPUT1, INPUT2, OUTPUT, EXPRESS, STATUS )
   }
   \sstarguments{
      \sstsubsection{
         INPUT1 = CHARACTER $*$ ( CHI\_\_SZNAME ) (Given)
      }{
         Name of the first join input catalogue.
      }
      \sstsubsection{
         INPUT2 = CHARACTER $*$ ( CHI\_\_SZNAME ) (Given)
      }{
         Name of the second join input catalogue.
      }
      \sstsubsection{
         OUTPUT = CHARACTER $*$ ( CHI\_\_SZNAME ) (Given)
      }{
         Name of the new catalogue.
      }
      \sstsubsection{
         EXPESS = CHARACTER $*$ ( CHI\_\_SZEXP ) (Given)
      }{
         Expression to be applied during the join.
      }
      \sstsubsection{
         STATUS = INTEGER (Given and Returned)
      }{
         Global status.
      }
   }
   \sstnotes{
      Joining two catalogues by taking every entry in turn from catalogue
      INPUT1, matching this entry with every entry in catalogue INPUT2
      and then checking the result against the EXPRESSion is extremely
      time consumming. Applications that use this routine will run very
      slowly when CHI\_JOIN has to use this method of joining.
   }
   \sstdiytopic{
      Anticipated Errors
   }{
      CHI\_\_CATNOTFND \\
      CHI\_\_IVLDEXP
   }
}
\sstroutine{
   CHI\_MERGE
}{
   Create a new catalogue by merging two catalogues
}{
   \sstdescription{
      Create a new catalogue by merging two catalogues. If the catalogues were
      ASCII files merging two catalogue would be the same as appending the
      two ASCII files to create a new file, providing the data is aligned
      correctly. To merge two catalogues use the MERGELFDS argument to select
      the fields in the first catalogue and the fields in the second catalogue
      that are to be merged to create fields in the third catalogue.
      MERGEFLDS is 3 by NUMMFLDS array. So you may have two
      catalogues one containing fields DATE, TIME and RESULT and the other DATE,
      TIME and VALUE. To merge these catalogues associate (DATE, DATE, DATE in
      MEGERFLDS(1,1)(2,1)(3,1)) and (TIME, TIME, TIME in MERGEFLDS(1,2)(2,2)
      (3,2)) and (RESULT, VALUE, READING in  MEGERFLDS(1,3)(2,3)(3,3)).
      The output catalogue will contain
      data from every entry in the first catalogue and every entry in the
      second. The merged fields must be of the same type.
      The resulting catalogue has fields TIME, DATE and READING.
      MERGEFLDFS will accept (RESULT, NULL, READING) in which case the field
      RESULT in the first catalogue has no field to be merged with it in the
      second catalogue. The null value of the field RESULT will be inserted
      when data is being read from the second catalogue.

      Each field in the new catalogue requires a format, units, null value and
      comment. These are taken from the fields in the first catalogue that were
      used in the merge but there is an opportunity to update these values.
      Setting the flag  FMATFLG(5) to TRUE indicates that value of FFORMAT(5)
      is to  be used to update the format of the field in the output catalogue
      given in MERGEFLDS(3,5). In the same way the fields\texttt{'} units, nullvalue and
      comment can be updated with the values of FUNIT, FNULL and
      FCOMMENT by setting the flags UNITFLG, NULLFLG and COMFLG
      respectively. If the format or the nullvalue of any field are changed
      a check is made to ensure that the format is valid and that the null
      value and format are consistent. If an error is reported the offending
      format is returned in FFORMATS(1).

      The new catalogue contains no parameters. Use CHI\_ADDP
      to add parameters to the catalogue.

      There are two levels of checking which are selected using the argument
      CHECK. Modes 1 is more efficient and should be used whenever possible.

      CHECK=1. The lowest level of checking. The routine does not check that
      the fields are in the respective catalogues or if they are of the same
      type. If the field is missing or if the types do not match the field
      in the resulting catalogue will be undefined. Use this mode when you
      know that the field names and types are correct.

      CHECK=2. The routine processes the field names given in MERGEFLDS and only
      if the all the fields in appear in the catalogues and are of the correct
      type will the catalogues be merged.
      Use this mode if you are not sure that the field names
      and types are correct. Fieldnames input by the user may contain spelling
      mistakes.
   }
   \sstinvocation{
       CALL  CHI\_MERGE( INPUT1, INPUT2, OUTPUT, MERGEFLDS, NUMMFLS, FMATFLG,
\newline
       FFORMAT, UNITFLG, FUNIT, NULLFLG, FNULL, COMFLG, FCOMMENT, STATUS )
   }
   \sstarguments{
      \sstsubsection{
         INPUT1 = CHARACTER $*$ ( CHI\_\_SZNAME ) (Given)
      }{
         Name of the first merge input catalogue.
      }
      \sstsubsection{
         INPUT2 = CHARACTER $*$ ( CHI\_\_SZNAME ) (Given)
      }{
         Name of the second merge input catalogue.
      }
      \sstsubsection{
         OUTPUT = CHARACTER $*$ ( CHI\_\_SZNAME ) (Given)
      }{
         Name of the new catalogue.
      }
      \sstsubsection{
         MERGEFLDS( 3, NUMMFLDS ) =
      }{
         \textbf{CHARACTER $*$ ( CHI\_\_SZFNAME ) (Given)}
      }{ \\
         Fields to be merged.
      }
      \sstsubsection{
         NUMMFLDS = INTEGER (Given)
      }{
         Number of merge field associations.
      }
      \sstsubsection{
         FMATFLGS = LOGICAL( NUMMFLDS ) (Given)
      }{
         Set FMATFLGS to TRUE to update the field format with the
         value of FFORMAT.
      }
      \sstsubsection{
         FFORMATS( NUMMFLDS ) =
      }{
         \textbf{CHARACTER $*$ ( CHI\_\_SZFFMT ) (Given and Returned)}
      }{ \\
         New formats of the field.
      }
      \sstsubsection{
         UNITFLGS = LOGICAL( NUMMFLDS ) (Given)
      }{
         Set UNITFLGS to TRUE to update the fields units with the
         value of FUNITS.
      }
      \sstsubsection{
         FUNITS( NUMMFLDS ) = CHARACTER $*$ ( CHI\_\_SZFUNIT ) (Given)
      }{
         New units of the field.
      }
      \sstsubsection{
         NULLFLG = LOGICAL (Given)
      }{
         Set NULLFLGS to TRUE to update the fields null values with the
         value of FNULLS.
      }
      \sstsubsection{
         FNULLS( NUMMFLDS ) = CHARACTER $*$ ( CHI\_\_SZFNVAL ) (Given)
      }{
         New null values of the field.
      }
      \sstsubsection{
         COMFLG = LOGICAL (Given)
      }{
         Set COMFLG to TRUE to update the fields comments with the
         value of FCOMMENT.
      }
      \sstsubsection{
         FCOMMENTS( NUMMFLDS ) = CHARACTER $*$ ( CHI\_\_SZFCMT ) (Given)
      }{
         New comments associated with the fields.
      }
      \sstsubsection{
         STATUS = INTEGER (Given and Returned)
      }{
         Global status.
      }
   }
   \sstnotes{
      The new catalogue contains no ordering information.
   }
   \sstdiytopic{
      Anticipated Errors
   }{
      CHI\_\_CATNOTFND \\
      CHI\_\_FLDNOTFND \\
      CHI\_\_IVLDFLDTYP \\
      CHI\_\_IVLDFFMT
   }
}
\sstroutine{
   CHI\_NEWFLDx
}{
   Create a new catalogue by adding a new field
}{
   \sstdescription{
      Create a new catalogue by adding a new field.
      Replace 'x' with C for character, D for double precision, I for integer,
      L for logical or R for real. Eg CHI\_GETSVALC.
      The data values
      of the new field are character strings of length SIZECHAR and are given
      in VALUES. The size of this array is given by NUMENTS. If NUMENTS is
      less than the number of entries in the catalogue an error will be
      reported. If, however, NUMENTS is greater than the number of entries in
      the catalogue the routine will use only as many values as it requires
      and disregard the rest. The new field also requires a name,format, units,
      null value and comment. The format and the nullvalue of the field are
      checked to ensure that the format is valid and that the null
      value and format are consistent. If an error is reported the offending
      format is returned in FFORMAT. The format should reflect the length of
      the character strings as given by SIZECHAR.
   }
   \sstinvocation{
      CALL CHI\_NEWFLDC( INPUT, OUTPUT, FNAME, FFORMAT, FUNIT, FNULL,
\newline
      FCOMMENT, VALUES, SIZECHAR, NUMENTS, STATUS)
   }
   \sstarguments{
      \sstsubsection{
         INPUT = CHARACTER $*$ ( CHI\_\_SZNAME ) (Given)
      }{
         Name of the catalogue.
      }
      \sstsubsection{
         OUTPUT = CHARACTER $*$ ( CHI\_\_SZNAME ) (Given)
      }{
         Name of the new catalogue that includes the new field.
      }
      \sstsubsection{
         FNAME = CHARACTER $*$ ( CHI\_\_SZFNAME ) (Given)
      }{
         Name of the new field.
      }
      \sstsubsection{
         FFORMAT = CHARACTER $*$ ( CHI\_\_SZFFMT ) (Given)
      }{
         Format of the new field.
      }
      \sstsubsection{
         FUNIT = CHARACTER $*$ ( CHI\_\_SZFUNIT ) (Given)
      }{
         Units of the new field.
      }
      \sstsubsection{
         FNULL = CHARACTER $*$ ( CHI\_\_SZFNVAL ) (Given)
      }{
         Null value of the new field.
      }
      \sstsubsection{
         FCOMMENT = CHARACTER $*$ ( CHI\_\_SZFCMT ) (Given)
      }{
         Comment associated with the new field.
      }
      \sstsubsection{
         VALUES( NUMENTS ) = CHARACTER $*$ ( SIZECHAR ) (Given)
      }{
         Data values to be inserted into the new field.
      }
      \sstsubsection{
         SIZECHAR = INTEGER (Given)
      }{
         Size of character string data.
      }
      \sstsubsection{
         NUMENTS = INTEGER (Given)
      }{
         Size of array containing the data values to be inserted.
      }
      \sstsubsection{
         STATUS = INTEGER (Given and Returned)
      }{
         Global status.
      }
   }
   \sstdiytopic{
      Anticipated Errors
   }{
      CHI\_\_CATNOTFND \\
      CHI\_\_INSENTRIES \\
      CHI\_\_IVLDFFMT
   }
}
\sstroutine{
   CHI\_NOENT
}{
   Create a new catalogue that contains no entries
}{
   \sstdescription{
      Creates a new catalogue that contains no entries. The CHI routines
      that write data into this catalogue will be more efficient if you can
      provide an estimate for the size of the catalogue. (The number
      of entries). The field formats and the parameter formats are checked. If
      an invalid format is found an error is reported and the offending
      parameter or field are returned in PNAMES(1),
      PFORMATS(1) or FNAMES(1), FFORMATS(1).
      The parameter formats are checked against the parameter values
      and the field formats are checked against the null values and an error
      is reported if an inconsistency is found. The offending
      parameter or field name, format and value or null value are returned in
      PNAMES(1), PFORMATS(1) and PVALUES(1) or FNAMES(1), FFORMATS(1) and
      FNULLS(1) respectively.
   }
   \sstinvocation{
      CALL CHI\_NOENT( INPUT, ESTNUMENTS, NUMPARS, PNAMES, PFORMATS,
\newline
      PVALUES, PCOMMENTS, NUMFLDS, FNAMES, FFORMATS, FUNITS, FNULLS,
\newline
      FCOMMENTS, STATUS )
   }
   \sstarguments{
      \sstsubsection{
         INPUT = CHARACTER $*$ ( CHI\_\_SZNAME ) (Given)
      }{
         Name of the catalogue being created.
      }
      \sstsubsection{
         ESTNUMENTS = INTEGER (Given)
      }{
         Estimate for the number of entries that will be put into the catalogue.
      }
      \sstsubsection{
         NUMPARS = INTEGER (Given)
      }{
         Number of parameters in the catalogue.
      }
      \sstsubsection{
         PNAMES( CHI\_\_NUMPARS ) =
      }{
         \textbf{CHARACTER $*$ ( CHI\_\_SZPNAME ) (Given and Returned)}
      }{ \\
         Names of the parameters in the catalogue.
      }
      \sstsubsection{
         PFORMATS( CHI\_\_NUMPARS ) =
      }{
         \textbf{CHARACTER $*$ ( CHI\_\_SZPFMT ) (Given and Returned)}
      }{ \\
         Formats of the parameters in the catalogue.
      }
      \sstsubsection{
         PVALUE( CHI\_\_NUMPARS ) =
      }{
         \textbf{CHARACTER $*$ ( CHI\_\_SZPVAL ) (Given and Returned)}
      }{ \\
         Values of the parameters in the catalogue.
      }
      \sstsubsection{
         PCOMMENTS( CHI\_\_NUMPARS ) =
      }{
         \textbf{CHARACTER $*$ ( CHI\_\_SZPCMT ) (Given)}
      }{ \\
         Comments associated with the parameters in the catalogue.
      }
      \sstsubsection{
         NUMFLDS = INTEGER (Returned)
      }{
         Number of fields in the catalogue.
      }
      \sstsubsection{
         FNAMES( CHI\_\_NUMFLDS ) =
      }{
         \textbf{CHARACTER $*$ ( CHI\_\_SZFNAME ) (Given and Returned)}
      }{ \\
         Names of the fields in the catalogue.
      }
      \sstsubsection{
         FFORMATS( CHI\_\_NUMFLDS ) =
      }{
         \textbf{CHARACTER $*$ ( CHI\_\_SZFFMT ) (Given and Returned)}
      }{ \\
         Formats of the fields in the catalogue.
      }
      \sstsubsection{
         FUNITS( CHI\_\_NUMFLDS ) = CHARACTER $*$ ( CHI\_\_SZFUNIT ) (Given)
      }{
         Units of the fields in the catalogue.
      }
      \sstsubsection{
         FNULLS( CHI\_\_NUMFLDS ) = CHARACTER $*$ ( CHI\_\_SZFNVAL ) (Given)
      }{
         Null values of the fields in the catalogue.
      }
      \sstsubsection{
         FCOMMENTS( CHI\_\_NUMFLDS ) =
      }{
         \textbf{CHARACTER $*$ ( CHI\_\_SZFCMT ) (Given)}
      }{ \\
         Comments associated with the fields in the catalogue.
      }
      \sstsubsection{
         STATUS = INTEGER (Given and Returned)
      }{
         Global status.
      }
   }
   \sstdiytopic{
      Anticipated Errors
   }{
      CHI\_\_IVLDFFMT \\
      CHI\_\_IVLDPFMT
   }
}
\sstroutine{
   CHI\_OPEN
}{
   Open the CHI system
}{
   \sstdescription{
      Opens the CHI system and performs house keeping tasks.
      CHI\_OPEN should be the first CHI call in your application
      and CHI\_CLOSE the last.
   }
   \sstinvocation{
      CALL CHI\_OPEN( STATUS )
   }
   \sstarguments{
      \sstsubsection{
         STATUS = INTEGER $*$ ( CHI\_\_SZNAME ) (Given and Returned)
      }{
         Global status.
      }
   }
   \sstdiytopic{
      Anticipated Errors
   }{
      None
   }
}
\sstroutine{
   CHI\_PUTENT
}{
   Put an entry into a catalogue
}{
   \sstdescription{
      Add an entry to a catalogue. For each field name in FNAMES CHI\_PUTENT
      checks in respective position in FLDTYPES to find the type of data.
      The data for this field in the entry will be taken from the respective
      element of the appropriate array. So if FLDTYPES(3) is I then the data
      for the field whose name is given in FNAMES(3) will be taken from
      INTVALS(3).

      There are eight levels of checking which are selected using the argument
      CHECK. Modes 1 and 2 are efficient and should be used whenever possible.
      You are advised not to change the level of checking when putting data
      into a catalogue unless you are moving to a mode where checking is done reading from
      a catalogue $*$
      CHECK=1. The lowest level of checking. The routine processes the field
      names given in FNAMES. Only if the field appears in the catalogue and
      the type agrees with the that given in FLDTYPES will data be put into
      the field for this entry. All other fields take their null values. The
      routine remembers where it found the data for each field (\emph{e.g.}, FLUX1 data
      in REALVALS(5)). On subsequent calls the routine assumes that the FLUX1
      value will be in REALVALS(5). Any ordering information in the catalogue
      is ignored, but not destroyed. Use this mode when you know all the field
      names and types are correct and that the entries are being put in the
      correct order.

      CHECK=11. The same level of checking as in mode 1 but any ordering
      information is destroyed.

      CHECK=2. The routine processes the field names given in FNAMES and only
      if the all the fields in the catalogue are given in FNAMES and the
      types are correct will the entry be put into the catalogue. If, during
      checking, an error is found the error is reported and the name of the
      offending field is returned in FNAMES(1). Again after
      the first call the routine remembers where it found the data for each
      field and on subsequent calls the routine assumes that the data will be
      in the same place. This mode ensures that genuine data is put into the
      catalogue. Fields cannot be overlooked. Any ordering information in the
      catalogue is ignored but not destroyed. Use this mode if you are not
      sure that the field
      names and types are correct or if the entries being put into the
      catalogue are not appropriately ordered.

      CHECK=21. The same level of checking as in mode 2 but any ordering
      information is destroyed.

      Modes 3 and 4 reflect the an equivalent level of checking in the
      CHI\_GET routines and only in exceptional cases will they be used
      when putting data into a catalogue.

      CHECK=3. In this mode the routine processes the field names given in
      FNAMES and their fieldtypes every time the routine is called. This means
      that different fields can contribute to each entry. Only if the field
      appears in the catalogue and
      the type agrees with that given in FLDTYPES will data be put into
      the field for this entry. All other fields take their null values.
      Any ordering information in the catalogue
      is ignored, but not destroyed.

      CHECK=31. The same level of checking as in mode 3 but any ordering
      information is destroyed.

      CHECK=4. Again in this mode the routine processes the field names given
      in FNAMES and their fieldtypes every time the routine is called.  Only
      if the all the fields in the catalogue are given in FNAMES and the
      types are correct will the entry be put into the catalogue.  If, during
      checking, an error is found the error is reported and the name of the
      offending field is returned in FNAMES(1). Subsequent
      calls to the routine may have the fields in FNAMES in a different order
      but they must all be their and the respective field types must be correct.
      Any ordering information in the catalogue is ignored but not destroyed.

      CHECK=41. The same level of checking as in mode 4 but any ordering
      information is destroyed.
   }
   \sstinvocation{
      CALL CHI\_PUTENT( INPUT, FNAMES, NUMFLDS, CHECK, CHARVALS, DOUBVALS,
\newline
      INTVALS, LOGVALS, REALVALS, FLDTYPES, STATUS )
   }
   \sstarguments{
      \sstsubsection{
         INPUT = CHARACTER $*$ ( CHI\_\_SZNAME ) (Given)
      }{
         Name of the catalogue into which the data is to be put.
      }
      \sstsubsection{
         FNAMES( CHI\_\_NUMFLDS ) =
      }{
         CHARACTER $*$ ( CHI\_\_SZFNAME ) (Given and Returned)
      }{
         Names of the fields whose data is being supplied.
      }
      \sstsubsection{
         NUMFLDS = INTEGER (Given)
      }{
         Number of fields whose data is being supplied.
      }
      \sstsubsection{
         CHECK = INTEGER (Given)
      }{
         Set to 1,2,3 or 4 according to the level of checking required.
      }
      \sstsubsection{
         INTVALS( CHI\_\_NUMFLDS ) = INTEGER (Returned)
      }{
         Array containing the data for integer fields.
      }
      \sstsubsection{
         REALVALS( CHI\_\_NUMFLDS ) = REAL (Returned)
      }{
         Array containing the data for real fields.
      }
      \sstsubsection{
         DOUBVALS( CHI\_\_NUMFLDS ) = DOUBLE PRECISION (Returned)
      }{
         Array containing the data for double precision fields.
      }
      \sstsubsection{
         LOGVALS( CHI\_\_NUMFLDS ) = LOGICAL (Returned)
      }{
         Array containing the data for logical fields.
      }
      \sstsubsection{
         CHARVALS( CHI\_\_NUMFLDS ) =
      }{
         \textbf{CHARACTER $*$ ( CHI\_\_SZCVAL ) (Returned)}
      }{ \\
         Array containing the data for character fields.
      }
      \sstsubsection{
         FLDTYPES( CHI\_\_NUMFLDS ) = CHARACTER $*$ ( 1 ) (Returned)
      }{
         Array containing the types of fields.
      }
      \sstsubsection{
         STATUS = INTEGER (Given and Returned)
      }{
         Global status.
      }
   }
   \sstdiytopic{
      Anticipated Errors
   }{
      CHI\_\_CATNOTFND \\
      CHI\_\_FLDNOTSUP \\
      CHI\_\_IVLDFLDTYP
   }
}
\sstroutine{
   CHI\_PUTNOTES
}{
   Add a line to the notes associated with a catalogue
}{
   \sstdescription{
      Add a line to the notes associated with a catalogue.
   }
   \sstinvocation{
      CALL CHI\_PUTNOTES( INPUT, BUFFER, STATUS )
   }
   \sstarguments{
      \sstsubsection{
         INPUT = CHARACTER $*$ ( CHI\_\_SZNAME ) (Given)
      }{
         Name of the catalogue to which the line note is to be added.
      }
      \sstsubsection{
         BUFFER = CHARACTER $*$ ( 80 ) (Returned)
      }{
         Array containing the line of the note.
      }
      \sstsubsection{
         STATUS = INTEGER (Given and Returned)
      }{
         Global status.
      }
   }
   \sstdiytopic{
      Anticipated Errors
   }{
      CHI\_\_CATNOTFND
   }
}
\sstroutine{
   CHI\_REJECT
}{
   Create a new catalogue containing only entries that FAIL to meet the
   given criteria
}{
   \sstdescription{
      Create a new catalogue containing only entries that FAIL to meet the
      given criteria. If an invalid expression error is reported CRITERIA
      is returned containing diagnostic information. Any ordering information
      in the input catalogue will be preserved in the new catalogue.
   }
   \sstinvocation{
      CALL CHI\_REJECT( INPUT, OUTPUT, CRITERIA, STATUS )
   }
   \sstarguments{
      \sstsubsection{
         INPUT = CHARACTER $*$ ( CHI\_\_SZNAME ) (Given)
      }{
         Name of the catalogue from which the entries are to be selected.
      }
      \sstsubsection{
         OUTPUT = CHARACTER $*$ ( CHI\_\_SZNAME ) (Given)
      }{
         Name of the new catalogue containing only the selected entries.
      }
      \sstsubsection{
         CRITERIA = CHARACTER $*$ ( CHI\_\_SZEXP ) (Given)
      }{
         Criteria to be applied to each entry in the input catalogue to
         determine if this entry is to be copied into the output catalogue.
      }
      \sstsubsection{
         STATUS = INTEGER (Given and Returned)
      }{
         Global status.
      }
   }
   \sstnotes{
      Any ordering in the input catalogue will be preserved in the new
      catalogue.
   }
   \sstdiytopic{
      Anticipated Errors
   }{
      CHI\_\_CATNOTFND \\
      CHI\_\_IVLDEXP
   }
}
\sstroutine{
   CHI\_RENAME
}{
   Rename a catalogue
}{
   \sstdescription{
      Rename a catalogue.
   }
   \sstinvocation{
      CALL CHI\_RENAME( INPUT, NEWNAME, STATUS )
   }
   \sstarguments{
      \sstsubsection{
         INPUT = CHARACTER $*$ ( CHI\_\_SZNAME ) (Given)
      }{
         Name of the catalogue from which the entries are to be selected.
      }
      \sstsubsection{
         NEWNAME = CHARACTER $*$ ( CHI\_\_SZNAME ) (Given)
      }{
         New name of the catalogue.
      }
      \sstsubsection{
         STATUS = INTEGER (Given and Returned)
      }{
         Global status.
      }
   }
   \sstnotes{
      Any ordering information in the input catalogue will be preserved in
      the new catalogue.
   }
   \sstdiytopic{
      Anticipated Errors
   }{
      CHI\_\_CATNOTFND \\
      CHI\_\_IVLDEXP
   }
}
\sstroutine{
   CHI\_RESET
}{
   Reset a catalogue back to the start after getting data
}{
   \sstdescription{
      Reset a catalogue back to the start after getting data. A subsequent call
      to a CHI\_GET or CHI\_EVAL routine will get data from the first entry in
      the catalogue.
   }
   \sstinvocation{
      CALL CHI\_RESET( INPUT, STATUS )
   }
   \sstarguments{
      \sstsubsection{
         INPUT = CHARACTER $*$ ( CHI\_\_SZNAME ) (Given)
      }{
         Name of the catalogue to be reset.
      }
      \sstsubsection{
         STATUS = INTEGER (Given and Returned)
      }{
         Global status.
      }
   }
   \sstnotes{
      A CHI\_RESET is not needed the first time data is retrieved from the
      catalogue. The CHI\_GETVAL routines automatically start at the first
      record.
   }
   \sstdiytopic{
      Anticipated Errors
   }{
      CHI\_\_CATNOTFND
   }
}
\sstroutine{
   CHI\_SEARCH
}{
   Create a new catalogue containing only entries that meet the given
   criteria
}{
   \sstdescription{
      Create a new catalogue containing only entries that meet the given
      criteria. If an invalid expression error is reported CRITERIA
      is returned containing diagnostic information. Any ordering information
      in the input catalogue will be preserved in the new catalogue.
      CHI\_SEARCH tries to perfom an efficient search using CHI\_TURBOSEARCH
      but if this fails the error is cancelled and the routine goes on to
      check every entry in the catalogue against the criteria.
   }
   \sstinvocation{
      CALL CHI\_SEARCH( INPUT, OUTPUT, CRITERIA, STATUS )
   }
   \sstarguments{
      \sstsubsection{
         INPUT = CHARACTER $*$ ( CHI\_\_SZNAME ) (Given)
      }{
         Name of the catalogue from which the entries are to be selected.
      }
      \sstsubsection{
         OUTPUT = CHARACTER $*$ ( CHI\_\_SZNAME ) (Given)
      }{
         Name of the new catalogue containing only the selected entries.
      }
      \sstsubsection{
         CRITERIA = CHARACTER $*$ ( CHI\_\_SZEXP ) (Given)
      }{
         Criteria to be applied to each entry in the input catalogue to
         determine if this entry is to be copied into the output catalogue.
      }
      \sstsubsection{
         STATUS = INTEGER (Given and Returned)
      }{
         Global status.
      }
   }
   \sstnotes{
      Searching a catalogue entry by entry is very inefficient and applications
      that use this routine will run very slowly when CHI\_SEARCH has to use
      method of searching.
   }
   \sstdiytopic{
      Anticipated Errors
   }{
      CHI\_\_CATNOTFND \\
      CHI\_\_IVLDEXP
   }
}
\sstroutine{
   CHI\_SELENTS
}{
   Create a new catalogue containing only selected entries
}{
   \sstdescription{
      Create a new catalogue containing only selected entries. The integer
      values in the array SELENTS indicate which entries are to be copied
      and the order in which they are to be copied.
      to the new catalogue. The size of the SELENTS array is given by NUMENTS.
      Any ordering information in the input catalogue will be destroyed in the
      new catalogue.
   }
   \sstinvocation{
      CALL CHI\_SELENTS( INPUT, OUTPUT, SELENTS, NUMENTS, STATUS )
   }
   \sstarguments{
      \sstsubsection{
         INPUT = CHARACTER $*$ ( CHI\_\_SZNAME ) (Given)
      }{
         Name of the catalogue from which the entries are to be selected.
      }
      \sstsubsection{
         OUTPUT = CHARACTER $*$ ( CHI\_\_SZNAME ) (Given)
      }{
         Name of the new catalogue containing only the selected entries.
      }
      \sstsubsection{
         NUMENTS = INTEGER (Given)
      }{
         Number of entries in the SELENTS array.
      }
      \sstsubsection{
         SELENTS( NUMENTS ) = INTEGER (Given)
      }{
         1-D integer array indicating the entries and their order.
      }
      \sstsubsection{
         STATUS = INTEGER (Given and Returned)
      }{
         Global status.
      }
   }
   \sstnotes{
      If a value given in array SELENTS is greater than the number of entries
      in the catalogue an error is reported.
   }
   \sstdiytopic{
      Anticipated Errors
   }{
      CHI\_\_CATNOTFND \\
      CHI\_\_INSENTRIES
   }
}
\sstroutine{
   CHI\_SELFLDS
}{
   Create a new catalogue containing only selected fields
}{
   \sstdescription{
      Create a new catalogue containing only the selected fields. If the
      original catalogue contained ordering information this will only be
      preserved in the new catalogue if all the key fields have been selected.
      If a field is not found in the catalogue an error is reported and the
      offending field is returned in FNAMES(1).
   }
   \sstinvocation{
      CALL CHI\_SELFLDS( INPUT, OUTPUT, FNAMES, NUMFLDS, STATUS )
   }
   \sstarguments{
      \sstsubsection{
         INPUT = CHARACTER $*$ ( CHI\_\_SZNAME ) (Given)
      }{
         Name of the catalogue from which the fields are to be selected.
      }
      \sstsubsection{
         OUTPUT = CHARACTER $*$ ( CHI\_\_SZNAME ) (Given)
      }{
         Name of the new catalogue containing only the selected fields.
      }
      \sstsubsection{
         FNAMES( CHI\_\_NUMFLDS ) =
      }{
         \textbf{CHARACTER $*$ ( CHI\_\_SZFNAME ) (Given and Returned)}
      }{ \\
         Names of the fields to be selected.
      }
      \sstsubsection{
         NUMFLDS = INTEGER (Given and Returned)
      }{
         Number of fields selected.
      }
      \sstsubsection{
         STATUS = INTEGER (Given and Returned)
      }{
         Global status.
      }
   }
   \sstnotes{
      If a field name does not exist in the catalogue no error will be reported
      and the field will not be selected. On return FNAMES and NUMFLDS contain
      the names of the fields and the number of fields in the new catalogue.
   }
   \sstdiytopic{
      Anticipated Errors
   }{
      CHI\_\_CATNOTFND
   }
}
\sstroutine{
   CHI\_SORT
}{
   Create a new catalogue that is sorted on given fields
}{
   \sstdescription{
      Create a new catalogue that is sorted on given fields. The functionality
      of sort is twofold. The first function is to create indexes associated
      with the catalogue that allow efficient searching and joining. The second
      function is a by product of the sort and that is to order a catalogue
      for presentation.

      Consider sorting the data in a telephone directory. For presentation
      purposes sort the data by field SURNAME (Primary field) if several
      entries are found with the same surname order these by ordering on the
      field FIRSTINITIAL (Secondary field) and if entries are found with the
      same surname and first initial order these by ordering on the field
      SECONDINITIAL (Tertiary field). This catalogue would now be presented in
      a useful way. More importantly the system has created an
      index that allows it to perform an efficient search and join in certain
      cases. For example, a request for entries where the SURNAME is BROWN and
      the FIRSTINITIAL is J.

      It is possible to create more than one index to a catalogue but only
      the first index is considered for presentation. In the above example as
      well as finding the telephone number for a given name we may also want
      to find the surname the person given the telephone number. To do this
      efficiently the catalogue would have to be ordered by field TELNUMBER.
      This is achieved by creating an additional index sort on TELNUMBER.
      Additional indexes can be created when the catalogue is first sorted
      or by using CHI\_ADDIND.

      The order of field name in the SORTFLDS array is significant.
      SORTFLDS(1,1) must contain the primary field of the presentation index.
      SORTFLDS(1,2) and SORTFLDS(1,3) contain the secondary and tertiary fields.
      Spaces in either the secondary or tertiary position simply indicates
      that there should be no secondary or tertiary ordering.
      SORTFLDS(2,1), SORTFLDS(2,2) and SORTFLDS(2,3) contain the primary field,
      secondary field and tertiary field for an additional index. Spaces in the
      primary position indicates that there are no further additional indexes.
      SORTFLDS(3,1), SORTFLDS(3,2) and SORTFLDS(3,3) contain the primary field,
      secondary field and tertiary field for an additional index. Spaces in the
      primary position indicates that there are no further additional indexes
      and so on.

      The direction of the sort for each field in given in the corresponding
      element of the SORTDIR array. TRUE for ascending.
   }
   \sstinvocation{
      CALL CHI\_SORT( INPUT, OUTPUT, SORTFLDS, SORTDIR, STATUS )
   }
   \sstarguments{
      \sstsubsection{
         INPUT = CHARACTER $*$ ( CHI\_\_SZNAME ) (Given)
      }{
         Name of the catalogue to be sorted.
      }
      \sstsubsection{
         OUTPUT = CHARACTER $*$ ( CHI\_\_SZNAME ) (Given)
      }{
         Name of the new sorted catalogue.
      }
      \sstsubsection{
         SORTFLDS( CHI\_\_NUMFLDS, 3 ) =
      }{
         \textbf{CHARACTER $*$ ( CHI\_\_SZFNAME ) (Given)}
      }{ \\
         Names of the sort fields.
      }
      \sstsubsection{
         SORTDIR( CHI\_\_NUMFLDS, 3 ) = LOGICAL (Given)
      }{
         Direction of sort for each field. (TRUE for ascending).
      }
      \sstsubsection{
         STATUS = INTEGER (Given and Returned)
      }{
         Global status.
      }
   }
   \sstnotes{
      If a field name does not exist in the catalogue an error will be reported.
   }
   \sstdiytopic{
      Anticipated Errors
   }{
      CHI\_\_CATNOTFND \\
      CHI\_\_FLDNOTFND
   }
}
\sstroutine{
   CHI\_SORTFLDS
}{
   Get the sort information from a catalogue
}{
   \sstdescription{
      Get the sort information from a catalogue. Get the names of the sort
      fields and the direction of each sort field. TRUE direction is ascending.
      If the catalogue contains no ordering information NUMFLDS is returned 0.
   }
   \sstinvocation{
      CALL CHI\_SORTFLDS( INPUT, SORTFLDS, SORTDIR, NUMFLDS, STATUS )
   }
   \sstarguments{
      \sstsubsection{
         INPUT = CHARACTER $*$ ( CHI\_\_SZNAME ) (Given)
      }{
         Name of the catalogue to be sorted.
      }
      \sstsubsection{
         SORTFLDS( CHI\_\_NUMFLDS, 3 ) =
      }{
         \textbf{CHARACTER $*$ ( CHI\_\_SZFNAME ) (Returned)}
      }{ \\
         Names of the sort fields. Primary field first.
      }
      \sstsubsection{
         SORTDIR( CHI\_\_NUMFLDS, 3 ) = LOGICAL (Returned)
      }{
         Direction of sort for each field. (TRUE for ascending).
      }
      \sstsubsection{
         NUMFLDS = INTEGER (Returned)
      }{
         Number of sort fields.
      }
      \sstsubsection{
         STATUS = INTEGER (Given and Returned)
      }{
         Global status.
      }
   }
   \sstdiytopic{
      Anticipated Errors
   }{
      CHI\_\_CATNOTFND
   }
}
\sstroutine{
   CHI\_SPLITNAME
}{
   Split catalogue name into database name and internal name
}{
   \sstdescription{
      Splits the input catalogue name into a database identifier an a
      catalogue identifier. It is suggested that the first three
      characters of the catalogue name, as it is known to the user,
       will be the database name and the remaining characters the name
       of the catalogue in that database. ADCIRPS would be catalogue
       IRPS in the ADC database.
   }
   \sstinvocation{
      CALL CHI\_SPLITNAME( INPUT, DBNAME, CATNAME, STATUS )
   }
   \sstarguments{
      \sstsubsection{
         INPUT = CHARACTER $*$ ( CHI\_\_SZNAME ) (Given)
      }{
         Name of the catalogue as known be the user.
      }
      \sstsubsection{
         DBNAME = CHARACTER $*$ ( 3 ) (Returned)
      }{
         Database name.
      }
      \sstsubsection{
         CATNAME = CHARACTER $*$ ( CHI\_\_SZNAME ) (Returned)
      }{
         Name of the catalogue within the database.
      }
      \sstsubsection{
         STATUS = INTEGER
      }{
         Global status.
      }
   }
   \sstdiytopic{
      Anticipated Errors
   }{
      None
   }
}
\sstroutine{
   CHI\_TURBOJOIN
}{
   Efficiently create a new catalogue by joining two catalogues
}{
   \sstdescription{
      Create a new catalogue by joining two catalogues. The effect of the join
      is as follows. Consider a large catalogue that contains all the fields
      from the INPUT1 catalogue and all the fields from the INPUT2 catalogue.
      Into this catalogue put an entry for each combination of entries in
      catalogues INPUT1 and INPUT2. The resulting catalogue will have N$*$M
      entries where N is the number of entries in the INPUT1 catalogue and
      M the number in the INPUT2 catalogue. Now search this catalogue for
      those entries that satisfy the given expression.

      Another way of looking at join is to say. Take every entry in turn
      from catalogue INPUT1. Match this entry against every entry in
      catalogue INPUT2 and if the EXPRESSion in satisfied combine both entries
      to write to a new catalogue.

      CHI\_TURBOJOIN attempts to do an efficient join using any ordering
      information in the catalogues. If an efficient join is not possible
      diagnostic information is returned in DIAGNOST. The diagnostic
      information is in the form of a text message that is suitable for
      presentation to the user.
   }
   \sstinvocation{
      CALL CHI\_JOIN( INPUT1, INPUT2, OUTPUT, EXPRESS, STATUS )
   }
   \sstarguments{
      \sstsubsection{
         INPUT1 = CHARACTER $*$ ( CHI\_\_SZNAME ) (Given)
      }{
         Name of the first join input catalogue.
      }
      \sstsubsection{
         INPUT2 = CHARACTER $*$ ( CHI\_\_SZNAME ) (Given)
      }{
         Name of the second join input catalogue.
      }
      \sstsubsection{
         OUTPUT = CHARACTER $*$ ( CHI\_\_SZNAME ) (Given)
      }{
         Name of the new catalogue.
      }
      \sstsubsection{
         EXPESS = CHARACTER $*$ ( CHI\_\_SZEXP ) (Given)
      }{
         Expression to be applied during the join.
      }
      \sstsubsection{
         DIAGNOST( 10 ) = CHARACTER $*$ ( 80 ) (Given)
      }{
         Diagnostic information about why a turbo join is not possible.
      }
      \sstsubsection{
         STATUS = INTEGER (Given and Returned)
      }{
         Global status.
      }
   }
   \sstdiytopic{
      Anticipated Errors
   }{
      CHI\_\_CATNOTFND \\
      CHI\_\_IVLDEXP \\
      CHI\_\_IVLDJOIN
   }
}
\sstroutine{
   CHI\_TURBOSEARCH
}{
   Efficiently create a new catalogue containing only entries that meet
   the given criteria
}{
   \sstdescription{
      Create a new catalogue containing only entries that meet the given
      criteria. If an invalid expression error is reported CRITERIA
      is returned containing diagnostic information.
      Any ordering information in the input catalogue will be preserved in
      the new catalogue.
      CHI\_TURBOSEARCH attempts to do an efficient search using any ordering
      information in the catalogue. If an efficient search is not possible
      diagnostic information is returned in DIAGNOST. The diagnostic
      information is in the form of a text message that is suitable for
      presentation to the user.
   }
   \sstinvocation{
      CALL CHI\_TURBOSEARCH( INPUT, OUTPUT, CRITERIA, DIAGNOST, STATUS )
   }
   \sstarguments{
      \sstsubsection{
         INPUT = CHARACTER $*$ ( CHI\_\_SZNAME ) (Given)
      }{
         Name of the catalogue from which the entries are to be selected.
      }
      \sstsubsection{
         OUTPUT = CHARACTER $*$ ( CHI\_\_SZNAME ) (Given)
      }{
         Name of the new catalogue containing only the selected entries.
      }
      \sstsubsection{
         CRITERIA = CHARACTER $*$ ( CHI\_\_SZEXP ) (Given)
      }{
         Criteria to be applied to each entry in the input catalogue to
         determine if this entry is to be copied into the output catalogue.
      }
      \sstsubsection{
         DIAGNOST( 10 ) = CHARACTER $*$ ( 80 ) (Returned)
      }{
         Diagnostic information about why a turbo search is not possible.
      }
      \sstsubsection{
         STATUS = INTEGER (Given and Returned)
      }{
         Global status.
      }
   }
   \sstdiytopic{
      Anticipated Errors
   }{
      CHI\_\_CATNOTFND \\
      CHI\_\_IVLDEXP \\
      CHI\_\_IVLDSRCH
   }
}
\sstroutine{
   CHI\_UPDATE
}{
   Update the data in a field
}{
   \sstdescription{
      Update the data in a field using a given expression and optionally
      update the information associated with the field.  Setting the flag
      FMATFLG to TRUE indicates that value of FFORMAT is to be used to update
      the format associated with this field. If FMATFLG is FALSE the field
      format will be unchanged. In the same way the fields\texttt{'} units, nullvalue and
      comment can be updated with the values of FUNIT, FNULL and
      FCOMMENT by setting the flags FMATFLG, UNITFLG, NULLFLG and COMFLG
      respectively. If the format or the nullvalue of the field are changed
      a check is made to ensure that the format is valid and that the null
      value and format are consistent. If an error is reported the offending
      format is returned in FFORMATS

       For each entry in the catalogue the expression is
      evaluated and the result used to update the given field. A field
      (ELTIME elapsed time) in your catalogue may be
      in hours and you really want this data in seconds. EXPRESS will be
      ELTIME$*$3600 and the FMATFLG, UNITFLG and COMFLG will be TRUE so that
      the new values of FFORMAT (I5), FUNIT (seconds) and FCOMMENT (Time in
      seconds) are inserted as the new values for the field format, units and
      comment respectively. You may also want to update the null value.
   }
   \sstinvocation{
      CALL CHI\_UPDATE( INPUT, FNAME, EXPRESS, FMATFLG, FFORMAT,
\newline
      UNITFLG, FUNIT, NULLFLG, FNULL, COMFLG, FCOMMENT, STATUS )
   }
   \sstarguments{
      \sstsubsection{
         INPUT = CHARACTER $*$ ( CHI\_\_SZNAME ) (Given)
      }{
         Name of the catalogue containing the field to be updated.
      }
      \sstsubsection{
         FNAME = CHARACTER $*$ ( CHI\_\_SZFNAME ) (Given)
      }{
         Names of the field whose information is to be updated.
      }
      \sstsubsection{
         EXPRESS = CHARACTER $*$ (CHI\_\_SZEXP) (Given)
      }{
         Expression to be evaluated to update the field.
      }
      \sstsubsection{
         FMATFLG = LOGICAL (Given)
      }{
         Set FMATFLG to TRUE to update the field format with the
         value of FFORMAT.
      }
      \sstsubsection{
         FFORMAT = CHARACTER $*$ ( CHI\_\_SZFFMT ) (Given)
      }{
         New format of the field.
      }
      \sstsubsection{
         UNITFLG = LOGICAL (Given)
      }{
         Set UNITFLG to TRUE to update the field units with the
         value of FUNIT.
      }
      \sstsubsection{
         FUNIT = CHARACTER $*$ ( CHI\_\_SZFUNIT ) (Given)
      }{
         New units of the field.
      }
      \sstsubsection{
         NULLFLG = LOGICAL (Given)
      }{
         Set NULLFLG to TRUE to update the field null value with the
         value of FNULL.
      }
      \sstsubsection{
         FNULL = CHARACTER $*$ ( CHI\_\_SZFNVAL ) (Given)
      }{
         New null value of the field.
      }
      \sstsubsection{
         COMFLG = LOGICAL (Given)
      }{
         Set COMFLG to TRUE to update the field comment with the
         value of FCOMMENT.
      }
      \sstsubsection{
         FCOMMENT = CHARACTER $*$ ( CHI\_\_SZFCMT ) (Given)
      }{
         New comment associated with the field.
      }
      \sstsubsection{
         STATUS = INTEGER (Given and Returned)
      }{
         Global status.
      }
   }
   \sstexamples{
      \sstexamplesubsection{
         CALL CHI\_UPDATE( TESTCAT, TEMP, TEMP$*$1.1, .FALSE., FFORMAT,
      }{
         \texttt{.FALSE., FUNITS, .FALSE., FNULL, .TRUE., 'Recalibrated}
      }{
         \texttt{temperature', STATUS)}
      }{ \\
         In this example the field TEMP is being increased by 10\% to
         reflect an instrument recalibration. All the information
         associated with the field remains unchanged except the comment
         which is updated to note that this is the recalibrated
         temperature.
      }
   }
   \sstdiytopic{
      Anticipated Errors
   }{
      CHI\_\_CATNOTFND \\
      CHI\_\_FLDNOTFND \\
      CHI\_\_IVLDEXP \\
      CHI\_\_IVLDFFMT
   }
}
\sstroutine{
   CHI\_UPFLD
}{
   Update the information associated with a field
}{
   \sstdescription{
      Update the information associated with a field. Setting the flag
      NAMEFLG to TRUE indicates that value of NEWNAME is to be used to update
      the name of this field. If NAMEFLG is FALSE the field name will be
      unchanged. In the same way the fields\texttt{'} format, units, nullvalue and
      comment can be updated with the values of FFORMAT, FUNIT, FNULL and
      FCOMMENT by setting the flags FMATFLG, UNITFLG, NULLFLG and COMFLG
      respectively. If the format or the nullvalue of any field are changed
      a check is made to ensure that the format is valid and that the null
      value and format are consistent. An error is reported and the offending
      format is returned in FFORMATS.
   }
   \sstinvocation{
      CALL CHI\_UPFLD( INPUT, FNAME, NAMEFLG, NEWNAME, FMATFLG, FFORMAT,
\newline
      UNITFLG, FUNIT, NULLFLG, FNULL, COMFLG, FCOMMENT, FTYPE, STATUS )
   }
   \sstarguments{
      \sstsubsection{
         INPUT = CHARACTER $*$ ( CHI\_\_SZNAME ) (Given)
      }{
         Name of the catalogue containing the field to be updated.
      }
      \sstsubsection{
         FNAME = CHARACTER $*$ ( CHI\_\_SZFNAME ) (Given)
      }{
         Name of the field whose information is to be updated.
      }
      \sstsubsection{
         NAMEFLG = LOGICAL (Given)
      }{
         Set NAMEFLG to TRUE to update the field name with the value of
         NEWNAME.
      }
      \sstsubsection{
         NEWNAME = CHARACTER $*$ ( CHI\_\_SZFNAME ) (Given)
      }{
         New field name.
      }
      \sstsubsection{
         FMATFLG = LOGICAL (Given)
      }{
         Set FMATFLG to TRUE to update the field format with the
         value of FFORMAT.
      }
      \sstsubsection{
         FFORMAT = CHARACTER $*$ ( CHI\_\_SZFFMT ) (Given)
      }{
         New format of the field.
      }
      \sstsubsection{
         UNITFLG = LOGICAL (Given)
      }{
         Set UNITFLG to TRUE to update the field units with the
         value of FUNIT.
      }
      \sstsubsection{
         FUNIT = CHARACTER $*$ ( CHI\_\_SZFUNIT ) (Given)
      }{
         New units of the field.
      }
      \sstsubsection{
         NULLFLG = LOGICAL (Given)
      }{
         Set NULLFLG to TRUE to update the field null value with the
         value of FNULL.
      }
      \sstsubsection{
         FNULL = CHARACTER $*$ ( CHI\_\_SZFNVAL ) (Given)
      }{
         New null value of the field.
      }
      \sstsubsection{
         COMFLG = LOGICAL (Given)
      }{
         Set COMFLG to TRUE to update the field comment with the
         value of FCOMMENT.
      }
      \sstsubsection{
         FCOMMENT = CHARACTER $*$ ( CHI\_\_SZFCMT ) (Given)
      }{
         New comment associated with the field.
      }
      \sstsubsection{
         STATUS = INTEGER (Given and Returned)
      }{
         Global status.
      }
   }
   \sstdiytopic{
      Anticipated Errors
   }{
      CHI\_\_CATNOTFND \\
      CHI\_\_FLDNOTFND \\
      CHI\_\_IVLDFFMT
   }
}
\sstroutine{
   CHI\_UPPAR
}{
   Update the information associated with a parameter
}{
   \sstdescription{
      Update the information associated with a parameter. Setting the flag
      NAMEFLG to TRUE indicates that value of NEWNAME is to be used to update
      the name associated with this parameter. If NAMEFLG is FALSE the parameter
      name will be unchanged. In the same way the parameters\texttt{'} format, value and
      comment can be updated with the values of PFORMAT, PVALUE and PCOMMENT
      by setting the flags FMATFLG, VALFLG and COMFLG respectively.
      If the format or the value of the parameter are changed
      a check is made to ensure that the format is valid and that the
      value and format are consistent. An error is reported and the offending
      format is returned in PFORMAT.
   }
   \sstinvocation{
      CALL CHI\_UPPAR( INPUT, PNAME, NAMEFLG, NEWNAME, FMATFLG, PFORMAT, VALFLG,
\newline
      PVALUE, COMFLG, PCOMMENT, STATUS )
   }
   \sstarguments{
      \sstsubsection{
         INPUT = CHARACTER $*$ ( CHI\_\_SZNAME ) (Given)
      }{
         Name of the catalogue containing the parameter to be updated.
      }
      \sstsubsection{
         PNAME = CHARACTER $*$ ( CHI\_\_SZPNAME ) (Given)
      }{
         Name of the parameter whose information is to be updated.
      }
      \sstsubsection{
         NAMEFLG = LOGICAL (Given)
      }{
         Set NAMEFLG to TRUE to update the parameter name with the
         value of NEWNAME.
      }
      \sstsubsection{
         NEWNAME = CHARACTER $*$ ( CHI\_\_SZPNAME ) (Given)
      }{
         New parameter name.
      }
      \sstsubsection{
         FMATFLG = LOGICAL (Given)
      }{
         Set FMATFLG to TRUE to update the parameter format with the
         value of PFORMAT.
      }
      \sstsubsection{
         PFORMAT = CHARACTER $*$ ( CHI\_\_SZPFMT ) (Given)
      }{
         New format of the parameter.
      }
      \sstsubsection{
         VALFLG = LOGICAL (Given)
      }{
         Set VALFLG to TRUE to update the parameter value with the
         value of PVALUE.
      }
      \sstsubsection{
         PVALUE = CHARACTER $*$ ( CHI\_\_SZPVAL ) (Given)
      }{
         New value of the parameter.
      }
      \sstsubsection{
         COMFLG = LOGICAL (Given)
      }{
         Set COMFLG to TRUE to update the parameter comment with the
         value of PCOMMENT.
      }
      \sstsubsection{
         PCOMMENT = CHARACTER $*$ ( CHI\_\_SZPCMT ) (Given)
      }{
         New comments associated with the parameter.
      }
      \sstsubsection{
         STATUS = INTEGER (Given and Returned)
      }{
         Global status.
      }
   }
   \sstdiytopic{
      Anticipated Errors
   }{
      CHI\_\_CATNOTFND \\
      CHI\_\_PARNOTFND \\
      CHI\_\_IVLDPFMT
   }
}
\end{small}

\newpage
\section {Error Codes}
\label{appendix:errors}

An error status value (not equal to SAI\_\_OK) is returned by any CHI or CHP
routine
which detects an error condition. If a program is to perform specific tests on
these status values, then the CHI-supplied symbolic names described below
should be used rather than explicit numerical values. These symbolic status
names are prefixed with `CHI'. The appropriate symbol definitions are contained
in the include files CHI\_ERR (for general CHI errors) and CHIPAR\_ERR (for
CHI parser errors).
The symbols defined in these two files are shown in the
following table and described in more detail below.

\begin{center}
\begin{tabular}{|l|l|} \hline
\emph{Symbolic Name} & \emph{Meaning}\\ \hline \hline
\textbf{CHI\_\_CATNOTFND} & Catalogue not found\\
\textbf{CHI\_\_PAREXISTS} & Parameters already exists.\\
\textbf{CHI\_\_IVLDFREQ} & Invalid value required (field).\\
\textbf{CHI\_\_IVLDPREQ} & Invalid value required (parameter).\\
\textbf{CHI\_\_PARNOTFND} & Parameter not found.\\
\textbf{CHI\_\_FLDNOTFND} & Field not found.\\
\textbf{CHI\_\_INSENTRIES} & Insufficient entries.\\
\textbf{CHI\_\_IVLDEXP} & Invalid expression.\\
\textbf{CHI\_\_EOF} & End of file.\\
\textbf{CHI\_\_INDCAT} & Indexed catalogue.\\
\textbf{CHI\_\_INVLDFMT} & Invalid format.\\
\textbf{CHI\_\_CONVERR} &  Conversion error.\\
\textbf{CHI\_\_IVLDFFMT} & Invalid field format.\\
\textbf{CHI\_\_IVLDPFMT} & Invalid parameter format.\\
\textbf{CHI\_\_IVLDSRCH} & Invalid search expression.\\
\textbf{CHI\_\_IVLDJOIN} & Invalid join expression.\\
\textbf{CHI\_\_WSOFL} & Work stack overflow.\\
\textbf{CHI\_\_TOOIT} & Too many items in expression.\\
\textbf{CHI\_\_PRSER} & Parse system error.\\
\textbf{CHI\_\_IVSYN} & Operand expected.\\
\textbf{CHI\_\_TOOOP} & Too many operators.\\
\textbf{CHI\_\_UFLOP} & Underflow on operator stack.\\
\textbf{CHI\_\_ICMPT} & Operator and operand incompatible.\\
\textbf{CHI\_\_NTSUP} & Option not supported.\\
\textbf{CHI\_\_IVOPR} & Invalid operator.\\
\textbf{CHI\_\_TOORX} & Too many relations in expression.\\
\textbf{CHI\_\_NOREL} & Too many relational sub-expressions.\\
\textbf{CHI\_\_DECOD} & Operand decode error.\\
\textbf{CHI\_\_MISMA} & Mismatche quotes or parentheses.\\
\textbf{CHI\_\_IVFUN} & Invalid function.\\
\hline
\end{tabular}
\end{center}

\begin{description}

\item [CATNOTFND:]
Catalogue not found. The catalogue required was
not found. Check spelling of catalogue name.
\item [PAREXISTS:]
Parameters already exists. The parameter that you
are trying to add to this catalogue already exists. Delete the existing
parameter using CHI\_DELP, or think of a new name.
\item [IVLDFREQ:]
Invalid value required. The value required must
be one of UNITS, FORMAT, NULLVALUE or COMMENT.
\item [IVLDPREQ:]
Invalid value required. The value required must
be one of FORMAT, VALUE or COMMENT.
\item [PARNOTFND:]
Parameter not found. The parameter required was
not found in the catalogue. Check spelling of parameter name.
\item [FLDNOTFND:]
Field not found. The field required was
not found in the catalogue. Check spelling of the field name.
\item [INSENTRIES:]
Insufficient entries have been supplied. Use
CHI\_GETNUMENTS to find the number of entries that are required.
\item [IVLDEXP:]
Invalid expression. The expression was invalid.
\item [IVLDFLDTYP:]
Invalid field type. The field type must be one
from I (integer), R (real), L (logical), D (double precision), C (character)
\item [EOF:]
End of file. You have tried to read off the end
of the catalogue. The number of entries can be found using CHI\_GETNUMENTS.
\item [INDCAT:]
Indexed catalogue. This error is a special case
and is a hangover from ADC. The catalogue is indexed and so the entry can not
be added. SCAR users can use CAR\_CONVERT to create a master binary catalogue,
other users contact STADAT::ARW. This error should be eliminated in a
subsequent release.
\item [INVLDFMT:]
Invalid format. An invalid format has been
detected. If the format is of unusual sexagesimal type it may not be supported.
\item [CONVERR:]
Conversion error. An error occured when
converting to or from a sexagesimal format
\item [IVLDFFMT:]
Invalid field format. The null value of the
field is inconsistent with the format.
\item [IVLDPFMT:]
Invalid parameter format. The value of the
parameter is inconsistent with the format.
\item [IVLDSRCH:]
Invalid search expression for a turbosearch.
Further diagnostics information is available.
\item [IVLDJOIN:]
Invalid join expression for a turbojoin.
Further diagnostics information is available.
\end{description}

And the possible CHI parser errors are

\begin{description}
\item [WSOFL:]
Work stack overflow. Check expression.
\item [TOOIT:]
Too many items in expression. Check expression.
\item [PRSER:]
Parse system error. Check expression.
\item [IVSYN:]
Operand expected. Check expression.
\item [TOOOP:]
Too many operators. Check expression.
\item [UFLOP:]
Underflow on operator stack. Check expression.
\item [ICMPT:]
Operator and operand incompatible. Check expression.
\item [NTSUP:]
Option not supported. Check expression.
\item [IVOPR:]
Invalid operator. Check expression.
\item [TOORX:]
Too many relations in expression. Check expression.
\item [NOREL:]
Too many relational sub-expressions. Check
expression.
\item [DECOD:]
Operand decode error. Check expression.
\item [MISMA:]
Mismatched quotes or parentheses. Check expression.
\item [IVFUN:]
Invalid function. Check expression.
\end{description}

\newpage
\section {Examples}
\label{appendix:examples}
No explicit examples of catalogue handling applications are included in this
document. All the CATPAC applications can be considered as examples of using
the CHI and CHP routines. The CATPAC applications source files can be found in
the library CATPAC\_DIR:CATPAC.TLB.

Applications can be classified according to their function. There are five
common types of application.

\begin{description}
\item[\mbox{}]\mbox{}
\begin{enumerate}

\item
Applications that read entries from one catalogue and write selected,
unchanged, entries to a new catalogue. SAMPLE is an example of this type of
application.

\item
Applications that reads entries from one catalogue, perform
some analysis and write updated or extended entries to a new catalogue.
The new catalogue may contain fewer or more fields than the original catalogue.
FK425 for example.

\item
Applications that obtain information about a catalogue. ENTRIES for example.

\item
Applications that update information about a catalogue. ADDPARAM for example.

\item
Applications that read data from a catalogue and performs some analysis on
that data. CORRELATE for example.

\end{enumerate}

\end{description}

\newpage
\section {Background and Design Philosophy of CHI and CHP}

It is not necessary for the reader to appreciate why CHI was developed or
how it works but it may be of interest. STARLINK's existing database
system SCAR is based on an underlying system called ADC. Unfortunately ADC
is very poorly documented and in the past has been found to contain a great
number of bugs. It also includes a large number of common variables as well
as a large number of routines. These difficulties have lead to the situation
where very few people
were able or willing to develop new applications using ADC.
A replacement for ADC was clearly required. The requirements for
CHI are
\begin{itemize}

\item The new Catalogue Handling Interface should be well documented.

\item The number of routines should be kept to a minimum.

\item The programmer should be unaware of how the data is stored. Is it
ASCII data or BINARY data? Is the data itself stored or just a pointer to
data?

\item There should be no special cases. ADC includes cases such as `If the
function is GREAT\_\-CIRCLE then expect the constant to be ARCSECONDS.'

\item It should be independent of the underlying database system. Making the
interface independent of the underlying database means that the underlying
system or systems can be changed without affecting the applications. This
provides a migration path from the discredited ADC to other more powerful
systems and, importantly, to future systems that are not yet available.

\item It should be able to deal automatically with several underlying
database systems presenting them as if they were a single system. The user
is unaware of where the data is actually stored. The data may not even be
available locally but may be made available via network access to remote
database systems. (We expect that most catalogue handling will be done
locally on the prefered underlying database system where it can be done most
efficiently.)

\item It should be portable.

\item CHI (and CHP) routines should be efficient. Any powerful features
available on any of the underlying systems should be fully exploited.
A discussion of how the CHI and CHP routines should achieve maximum efficiency
is outside the scope of this document but the performance of any application
written using CHI and CHP should be very close to that of an application written
directly using the underlying database.

\end{itemize}

\newpage

\section{CHI and SQL}

SQL or the Structured Query Language is widely accepted ANSI/ISO standard
relational database language. Almost all the major commercial relational
database suppliers support SQL. At the present time the only available
underlying system supporting CHI is ADC but work is already progressing
to implement CHI on an SQL based system. When this is completed any database
that supports SQL will also support CHI and CHP and all the applications built
using CHI and CHP.

\end{document}
