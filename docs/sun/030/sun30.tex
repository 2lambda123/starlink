\documentstyle{article} 
\pagestyle{myheadings}

%------------------------------------------------------------------------------
\newcommand{\stardoccategory}  {Starlink User Note}
\newcommand{\stardocinitials}  {SUN}
\newcommand{\stardocnumber}    {30.2}
\newcommand{\stardocauthors}   {David Giaretta}
\newcommand{\stardocdate}      {22th Sept 1989}
\newcommand{\stardoctitle}     {Using the Starlink Database MicroVAX computer}
%------------------------------------------------------------------------------

\newcommand{\stardocname}{\stardocinitials /\stardocnumber}
\markright{\stardocname}
\setlength{\textwidth}{160mm}
\setlength{\textheight}{240mm}
\setlength{\topmargin}{-5mm}
\setlength{\oddsidemargin}{0mm}
\setlength{\evensidemargin}{0mm}
\setlength{\parindent}{0mm}
\setlength{\parskip}{\medskipamount}
\setlength{\unitlength}{1mm}

\begin{document}
\thispagestyle{empty}
SCIENCE \& ENGINEERING RESEARCH COUNCIL \hfill \stardocname\\
RUTHERFORD APPLETON LABORATORY\\
{\large\bf Starlink Project\\}
{\large\bf \stardoccategory\ \stardocnumber}
\begin{flushright}
\stardocauthors\\
\stardocdate
\end{flushright}
\vspace{-4mm}
\rule{\textwidth}{0.5mm}
\vspace{5mm}
\begin{center}
{\Large\bf \stardoctitle}
\end{center}
\vspace{5mm}

\tableofcontents

\newpage
\section{Introduction}

This paper tells you how to use the Starlink Database microVAX computer.
This is installed at the Rutherford Appleton Laboratory and may be used by any
registered Starlink user.
As its title implies, its 
principal purpose is to provide access to astronomical
databases, in particular: the star and galaxy catalogues and the IRAS catalogues
used by the SCAR database management system, the IUE Uniform Low Dispersion
Archive (ULDA), and a variety of spectral atlases.
It is not a general purpose computer, and users will not have permanent
file space or private usernames allocated on it.

Some facilities may conveniently be used directly from one's local node,
reading data files resident on STADAT.

The way the machine is run will evolve as requirements change and resources
allow.
Thus, this paper describes the current
hardware and software configurations
and operational procedures.
These may be changed in the future.


\subsection{Getting help}

Typing {\bf STADAT\_HELP} gives help on STADAT resources.
\begin{itemize}
\item Difficulties concerned with network
     access should be reported to STADAT::OPER
\item Problems with catalogue software and catalogue requests should be reported
      to STADAT::CATMAIL
\item Temporary increases in disk quota may be requested from STADAT::OPER
      (or STADAT::DATAMAN)
\end{itemize}

\section{Access routes}

The DECNET name of the computer is STADAT (for STArlink DATabase), and
the NRS name is UK.AC.RL.STADAT (full name
UK.AC.RUTHERFORD.STARLINKDATABASE).

STADAT is  in a Local Area Vax Cluster with the Starlink $\mu$VAX.
There are two access routes to STADAT from other computers.
The first is Starlink DECNET access (by SET HOST and COPY).
The second is Coloured Book software access (by TRANSFER only).

Users may only log into STADAT by SET HOST from a Starlink node
(major or minor):
\begin{verbatim}
    $ SET HOST STADAT
\end{verbatim}
This is a DECNET facility and can only be used from Starlink nodes on DECNET.
Access from non-Starlink DECNET nodes is barred and PAD cannot be used
to log in.
Users should copy files from STADAT using the TRANSFER command (see SUN/36,
Appendix A.1):
\begin{verbatim}
    $ TRANSFER/USERNAME=NETUSR,NETUSR/CODE=FAST -
         RL.STADAT::DISK$STARDATA:<file specification>  <file specification>
\end{verbatim}
This is more efficient than COPY.



\section{Usernames}

You can only log in to
the database machine after having logged into a
normal Starlink computer under your own username.
This is the basis of the security system.
There are two usernames set up on the machine for each 
Starlink node.
One of them is a private username for the Site Manager.
The other is a public username that you can use; this may be used
by {\em all} users at your site who have asked to use the machine.
Each site's public username is different and is managed separately.

If you wish to use the machine, ask your Site Manager for the appropriate
username and password.
The password will be changed periodically, and it must not appear in documents
or mail messages or be disclosed to other users.
Once logged into the machine, you will be able to use the installed software and
access the installed data.
However, your files will not be permanent and you should copy any files you
want to keep back to your local filespace when you have finished.
Remember that the username you will be using will be used by many other
people.
You should adapt your behaviour to this circumstance.
In particular, delete unwanted files before logging out, and don't expect
any files you do leave to remain in existence for very long.

{\bf Don't do anything antisocial like changing passwords or tinkering with the
LOGIN.COM file. The latter should call \@SSC:LOGIN.}

There are a few extra usernames on the machine for particular activities,
such as Operations, Software Maintenance, SCAR and USSP development, and
managing the IUE captive account.
These should not concern the normal user.

\section{Resources}

\subsection{Software available}

The following software is available:
\begin{itemize}
\item Standard Starlink Software.
This includes GKS, REXEC and the ADC library.
STADAT accesses the RLSTAR version of the Starlink Software
Collection (SSC) and hence always uses the
latest versions of software items.
\item USSP (see ULDA below for total size).
This is the IUE ULDA access software which was obtained from VILSPA.
See SUN/20 for more details.
\item SCAR, including source code.
See SUN/70 for more details.
\item SPAG, a code unscrambler. See SUN/63 for details.
\item Fortran compiler, Coloured Book, PSI-Access.
\end{itemize}


\subsection{Astronomical Data available}

The following data sets are available:
\begin{itemize}
\item CHART catalogues in directory [STARCATS].
\item The logical search path SCAR\_DAT\_PATH is set up by typing
SCARSTART, and points to directories containing SCAR catalogues.
\item IRAS LRS database --- see SUN/14.
\item ULDA (including USSP software).
This is the IUE Uniform Low Dispersion Archive.
Its size will increase as time goes on (rate unknown).
Its use is described in SUN/20.
\item IUE merged Observing Log, and use of IUEDEARCH (SUN/58).
\item Various Spectral Atlases, in directory SPECTRAL\_ATLASDIR:\\
      typing:
\begin{verbatim}
    $ @SPECTRAL_ATLASDIR:LOGIN
\end{verbatim}
sets up the appropriate logicals. The command ATLAS\_HELP provides help for
the atlases available. Most allow one to extract the spectra chosen directly
into DIPSO files,  after first listing the spectra available.
\begin{verbatim}
    $ ATLAS_HELP SUMMARY
\end{verbatim}
shows a summary of the spectral atlases available.
\item    JPL Ephemeris 1800-2050  : disk\$stardata:[ephemeris]jpl.dat (SUN/87)\\
         In the examples in SUN/87 you can assign the data file to the
         above instead of jpldir:de200.dat and pick up the full ephemeris.
\end{itemize}
Other items, such as an 1885-1985 Supernova database to be produced at RAL, and
the UKST and other telescope logs, will be installed as they become available.
Users will be informed by revisions of this note and by news items.

\subsection{Captive accounts}

\begin{itemize}

\item IUE \\
the IUE captive account allows use of services 
concerned with IUE data such as fast searching of the IUE log
and IUE image de-archiving (see SUN/58).
\item IRASLRS\\
the IRASLRS captive account allows access to the LRS database (see SUN/14).
\end{itemize}


\subsection{Things you can use without logging in}

An important point to note is that from one's local machine one can 
read data STADAT files over the network. Normally this would be
too time consuming to be worthwhile, for example searching SCAR catalogues.
However in certain instances, where one essentially points to a particular
record in a file and transfers a small number of bytes across the network,
it does make sense not to log in to STADAT.

The following is a list of facilities available and the logical names
which must be defined.

\begin{itemize}
\item Spectral Atlases\\
{\bf \$ define/job spectral\_atlasdir stadat::spectral\_atlasdir:\\
\$ @spectral\_atlasdir:login\\ }
\item IUEDEARCH (SUN/58)\\
{\bf \$ define/job iuedearchdisk stadat::iuedearchdisk:\\
\$ @iuedearchdisk:[iues.capaccnt]iuedearch }

\item UNSPL --- part of USSP\\
 unspl should be defined as ``run stadat::ussp\_dir:unspl''\\
{\bf \$ unspl}

\item    JPL Ephemeris 1800-2050  \\
         In the examples in SUN/87 you can assign the data file to
         stadat::disk\$stardata:[ephemeris]jpl.dat.

\end{itemize}



\section{Configuration and Management}

The database microVAX has the following hardware configuration at present:
\begin{itemize}
\item DEC $\mu$VAX II with 13 Mbyte memory.
\item One RD53 disk drive (71 Mbyte = 142000 blocks).
This will be used as the system disk and will store the operating system,
Fortran compiler, communications software, etc.
\item Two SI83 disk drives organised as a single logical unit
(1 Gbyte = 2048000 blocks).
This will be used for catalogues and data, and for user space and
applications software.
\item X-25 communications, via Ethernet and a non-Starlink $\mu$VAX.
\end{itemize}


\subsection{Limitations}

\begin{itemize}
\item Only one user is able to run the USSP software at any one time.
This is a limitation of the VILSPA software design, not an RAL operational
decision.
\item Only one user is able to use the IRASLRS account at any one time.
This is a limitation of the software design, not an RAL operational
decision.
\item STADAT only has a licence for 8 simultaneous users, so no more than 8
logins are allowed at any time.
\end {itemize}

\subsection{Management}

There is no formal user registration scheme on this machine, unlike other
Starlink machines.
Site Managers should keep a list of the names of their local users who have
asked to use the machine and are still active.
(This should be stored in file LADMINDIR:STADAT.LIS in the same format as
LADMINDIR:USERNAMES.LIS.)

David Giaretta (STADAT::DATAMAN) 
is responsible for the management policy, following Starlink
guidelines, and
Monica Kendall (STADAT::OPER) is responsible for Operations, 
Backup, and User installation.
Individual Site Managers are responsible for managing users of the microVAX
from their own sites.

Mail about catalogues and SCAR
should be sent to STADAT::CATMAIL, which is a mail-club account. A number 
of people  check mail in this account and so long delays in replies when
individuals are away should be avoided.

\section{References}

\begin{description}
\item [SUN/14] : Obtaining spectra from the IRAS LRS database
\item [SUN/20] : USSP --- The IUE ULDA Software Support Package (to be issued).
\item [SUN/36] : Starlink Networking.
\item [SUN/58] : IUEDEARCH --- access to IUE archive data
\item [SUN/63] : SPAG --- Spaghetti Unscrambler
\item [SUN/70] : SCAR --- Starlink catalogue access and reporting.
\item [SUN/87] : JPL --- Solar System ephemeris
\end{description}

\end{document}
