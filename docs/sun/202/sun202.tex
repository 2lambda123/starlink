\documentclass[twoside,11pt]{article}
\pagestyle{myheadings}

%------------------------------------------------------------------------------

% -----------------------------------------------------------------------------
% ? Document identification
\newcommand{\stardoccategory}  {Starlink User Note}
\newcommand{\stardocinitials}  {SUN}
\newcommand{\stardocsource}    {sun\stardocnumber}
\newcommand{\stardocnumber}    {202.2}
\newcommand{\stardocauthors}   {M.\,J.\,Bly}
\newcommand{\stardocdate}      {19 February 1998}
\newcommand{\stardoctitle}     {Starlink Subroutine Libraries \\[1.5ex]
                                A Guide for Program Development and Linking}
\newcommand{\stardocversion}   {software\_version}
\newcommand{\stardocmanual}    {manual\_type}
% ? End of document identification
% -----------------------------------------------------------------------------

\newcommand{\stardocname}{\stardocinitials /\stardocnumber}
\markboth{\stardocname}{\stardocname}
\setlength{\textwidth}{160mm}
\setlength{\textheight}{230mm}
\setlength{\topmargin}{-2mm}
\setlength{\oddsidemargin}{0mm}
\setlength{\evensidemargin}{0mm}
\setlength{\parindent}{0mm}
\setlength{\parskip}{\medskipamount}
\setlength{\unitlength}{1mm}

% -----------------------------------------------------------------------------
%  Hypertext definitions.
%  ======================
%  These are used by the LaTeX2HTML translator in conjunction with star2html.

%  Comment.sty: version 2.0, 19 June 1992
%  Selectively in/exclude pieces of text.
%
%  Author
%    Victor Eijkhout                                      <eijkhout@cs.utk.edu>
%    Department of Computer Science
%    University Tennessee at Knoxville
%    104 Ayres Hall
%    Knoxville, TN 37996
%    USA

%  Do not remove the %begin{latexonly} and %end{latexonly} lines (used by
%  LaTeX2HTML to signify text it shouldn't process).
%begin{latexonly}
\makeatletter
\def\makeinnocent#1{\catcode`#1=12 }
\def\csarg#1#2{\expandafter#1\csname#2\endcsname}

\def\ThrowAwayComment#1{\begingroup
    \def\CurrentComment{#1}%
    \let\do\makeinnocent \dospecials
    \makeinnocent\^^L% and whatever other special cases
    \endlinechar`\^^M \catcode`\^^M=12 \xComment}
{\catcode`\^^M=12 \endlinechar=-1 %
 \gdef\xComment#1^^M{\def\test{#1}
      \csarg\ifx{PlainEnd\CurrentComment Test}\test
          \let\html@next\endgroup
      \else \csarg\ifx{LaLaEnd\CurrentComment Test}\test
            \edef\html@next{\endgroup\noexpand\end{\CurrentComment}}
      \else \let\html@next\xComment
      \fi \fi \html@next}
}
\makeatother

\def\includecomment
 #1{\expandafter\def\csname#1\endcsname{}%
    \expandafter\def\csname end#1\endcsname{}}
\def\excludecomment
 #1{\expandafter\def\csname#1\endcsname{\ThrowAwayComment{#1}}%
    {\escapechar=-1\relax
     \csarg\xdef{PlainEnd#1Test}{\string\\end#1}%
     \csarg\xdef{LaLaEnd#1Test}{\string\\end\string\{#1\string\}}%
    }}

%  Define environments that ignore their contents.
\excludecomment{comment}
\excludecomment{rawhtml}
\excludecomment{htmlonly}

%  Hypertext commands etc. This is a condensed version of the html.sty
%  file supplied with LaTeX2HTML by: Nikos Drakos <nikos@cbl.leeds.ac.uk> &
%  Jelle van Zeijl <jvzeijl@isou17.estec.esa.nl>. The LaTeX2HTML documentation
%  should be consulted about all commands (and the environments defined above)
%  except \xref and \xlabel which are Starlink specific.

\newcommand{\htmladdnormallinkfoot}[2]{#1\footnote{#2}}
\newcommand{\htmladdnormallink}[2]{#1}
\newcommand{\htmladdimg}[1]{}
\newenvironment{latexonly}{}{}
\newcommand{\hyperref}[4]{#2\ref{#4}#3}
\newcommand{\htmlref}[2]{#1}
\newcommand{\htmlimage}[1]{}
\newcommand{\htmladdtonavigation}[1]{}

% Define commands for HTML-only or LaTeX-only text.
\newcommand{\html}[1]{}
\newcommand{\latex}[1]{#1}

% Use latex2html 98.2.
\newcommand{\latexhtml}[2]{#1}

%  Starlink cross-references and labels.
\newcommand{\xref}[3]{#1}
\newcommand{\xlabel}[1]{}

%  LaTeX2HTML symbol.
\newcommand{\latextohtml}{{\bf LaTeX}{2}{\tt{HTML}}}

%  Define command to re-centre underscore for Latex and leave as normal
%  for HTML (severe problems with \_ in tabbing environments and \_\_
%  generally otherwise).
\newcommand{\setunderscore}{\renewcommand{\_}{{\tt\symbol{95}}}}
\latex{\setunderscore}

% -----------------------------------------------------------------------------
%  Debugging.
%  =========
%  Remove % from the following to debug links in the HTML version using Latex.

% \newcommand{\hotlink}[2]{\fbox{\begin{tabular}[t]{@{}c@{}}#1\\\hline{\footnotesize #2}\end{tabular}}}
% \renewcommand{\htmladdnormallinkfoot}[2]{\hotlink{#1}{#2}}
% \renewcommand{\htmladdnormallink}[2]{\hotlink{#1}{#2}}
% \renewcommand{\hyperref}[4]{\hotlink{#1}{\S\ref{#4}}}
% \renewcommand{\htmlref}[2]{\hotlink{#1}{\S\ref{#2}}}
% \renewcommand{\xref}[3]{\hotlink{#1}{#2 -- #3}}
%end{latexonly}
% -----------------------------------------------------------------------------
% ? Document-specific \newcommand or \newenvironment commands.
% ? End of document-specific commands
% -----------------------------------------------------------------------------
%  Title Page.
%  ===========
\renewcommand{\thepage}{\roman{page}}
\begin{document}
\thispagestyle{empty}

%  Latex document header.
%  ======================
\begin{latexonly}
   CCLRC / {\sc Rutherford Appleton Laboratory} \hfill {\bf \stardocname}\\
   {\large Particle Physics \& Astronomy Research Council}\\
   {\large Starlink Project\\}
   {\large \stardoccategory\ \stardocnumber}
   \begin{flushright}
   \stardocauthors\\
   \stardocdate
   \end{flushright}
   \vspace{-4mm}
   \rule{\textwidth}{0.5mm}
   \vspace{5mm}
   \begin{center}
   {\Huge\bf  \stardoctitle \\ [2.5ex]}
%   {\LARGE\bf \stardocversion \\ [4ex]}
%   {\Huge\bf  \stardocmanual}
   \end{center}
   \vspace{5mm}

% ? Heading for abstract if used.
%   \vspace{10mm}
%   \begin{center}
%      {\Large\bf Abstract}
%   \end{center}
% ? End of heading for abstract.
\end{latexonly}

%  HTML documentation header.
%  ==========================
\begin{htmlonly}
   \xlabel{}
   \begin{rawhtml} <H1> \end{rawhtml}
      \stardoctitle
   \begin{rawhtml} </H1> \end{rawhtml}

% ? Add picture here if required.
% ? End of picture

   \begin{rawhtml} <P> <I> \end{rawhtml}
   \stardoccategory\ \stardocnumber \\
   \stardocauthors \\
   \stardocdate
   \begin{rawhtml} </I> </P> <H3> \end{rawhtml}
      \htmladdnormallink{CCLRC}{http://www.cclrc.ac.uk} /
      \htmladdnormallink{Rutherford Appleton Laboratory}
                        {http://www.cclrc.ac.uk/ral} \\
      \htmladdnormallink{Particle Physics \& Astronomy Research Council}
                        {http://www.pparc.ac.uk} \\
   \begin{rawhtml} </H3> <H2> \end{rawhtml}
      \htmladdnormallink{Starlink Project}{http://www.starlink.ac.uk/}
   \begin{rawhtml} </H2> \end{rawhtml}
   \htmladdnormallink{\htmladdimg{source.gif} Retrieve hardcopy}
      {http://www.starlink.ac.uk/cgi-bin/hcserver?\stardocsource}\\

%  HTML document table of contents.
%  ================================
%  Add table of contents header and a navigation button to return to this
%  point in the document (this should always go before the abstract \section).
  \label{stardoccontents}
  \begin{rawhtml}
    <HR>
    <H2>Contents</H2>
  \end{rawhtml}
  \htmladdtonavigation{\htmlref{\htmladdimg{contents_motif.gif}}
        {stardoccontents}}

% ? New section for abstract if used.
%  \section{\xlabel{abstract}Abstract}
% ? End of new section for abstract
\end{htmlonly}

% -----------------------------------------------------------------------------
% ? Document Abstract. (if used)
%   ==================
% ? End of document abstract
% -----------------------------------------------------------------------------
% ? Latex document Table of Contents (if used).
%  ===========================================
% \newpage
% \begin{latexonly}
%   \setlength{\parskip}{0mm}
%   \tableofcontents
%   \setlength{\parskip}{\medskipamount}
%   \markboth{\stardocname}{\stardocname}
% \end{latexonly}
% ? End of Latex document table of contents
% -----------------------------------------------------------------------------
%\newpage
\renewcommand{\thepage}{\arabic{page}}
\setcounter{page}{1}

\section{\xlabel{introduction}Introduction}
\label{introduction}

This note gives a general overview of the methods available for using
Starlink Infrastructure subroutine libraries with applications.
There is an outline of how to use the include files for a subroutine
library, and a guide to the methods available for linking with the
subroutine libraries.

All the Starlink Infrastructure libraries are organised in the same way,
so it is possible to give a general guide to the principles involved.
However, some libraries do differ, and for precise details of how to use a
particular subroutine library, you should consult the Starlink document
for that library.

\section{\xlabel{organisation}Organisation}
\label{organisation}

Suppose you wanted to use an Infrastructure library BLY (there isn't one
-- this is an example!).  In the Starlink installation, the BLY subroutine
library has several components:

\begin{enumerate}

\item a library file \texttt{libbly.a}

\item a shareable library \texttt{libbly.so}

\item a development script \texttt{bly\_dev}

\item Fortran \texttt{INCLUDE} files \texttt{bly\_err} and \texttt{bly\_par}

\item ADAM\footnote{See SG/4 `\emph{ADAM -- The Starlink Software
Environment}'} versions of the library and shareable library
\texttt{libbly\_adam.a} and \texttt{libbly\_adam.so}

\item link scripts \texttt{bly\_link} and \texttt{bly\_link\_adam}

\end{enumerate}

Most of the Infrastructure libraries have all these components, but some
have more \texttt{INCLUDE} files, and some do not have shareable libraries.
Those that do not have any \texttt{INCLUDE} files will lack a development
script.  A few libraries do not need separate ADAM versions, so will not
have libraries for use with ADAM.

It is best to consult the documentation for a particular library to see
what \texttt{INCLUDE} files and libraries are available.

The components may all be used as part of the development and linking of
programs that use the Starlink Infrastructure subroutine libraries.

\section{\xlabel{program_development}Program Development -- \texttt{INCLUDE} files}
\label{program_development}

If the Infrastructure library you want to use has \texttt{INCLUDE} files, you
need to be able to reference them from your source code.

You might wish to use the full \texttt{PATH} name for the \texttt{INCLUDE}
file, \emph{e.g.}, \texttt{/star/include/bly\_par}.  This is fine, and works,
but is not portable, and could lead to problems if you want to use a
development version of the library.

The recommended way to use the \texttt{INCLUDE} files for a particular
library is to create links in your development directory to the {\tt
INCLUDE} files, and reference the links in your source code.

A library `\texttt{dev}' script will create the links for you in your
working directory.  The links are \textbf{UPPER-CASE}, and it is these
upper-case links you reference in the source code, thus:

\begin{quote}
\begin{verbatim}
      PROGRAM MYPROG
*
*  demonstrate use of include file
*
      INCLUDE 'BLY_PAR'
*
      ...
*
      END
\end{verbatim}
\end{quote}

To generate the links, issue the development command:

\begin{quote}
\begin{verbatim}
% bly_dev
\end{verbatim}
\end{quote}

This will create links to ALL the \texttt{INCLUDE} files for the BLY library:

\begin{quote}
\begin{verbatim}
BLY_ERR ->  /star/include/bly_err
BLY_PAR ->  /star/include/bly_par
\end{verbatim}
\end{quote}

To remove the links, issue the command again with the \texttt{remove} option:

\begin{quote}
\begin{verbatim}
% bly_dev remove
\end{verbatim}
\end{quote}

You should keep the links in place while developing your program.  If you
want to move development directories, simply remove the links and create
new ones in the new directory.

The Starlink software building system uses the soft link strategy in its
makefiles, though the makefiles generate the links themselves.  The links
may easily be changed to pick up a development version of a library,
without having to edit source code.

\section{\xlabel{library_link_scripts}Library Link Scripts}
\label{library_link_scripts}

\subsection{\xlabel{link_scripts_background}Background}
\label{link_scripts_background}

The Starlink Infrastructure libraries depend upon one another in a
hierarchy of dependencies that is quite complicated -- dependencies that
mean one library may need several others at link time to get a full
resolution of all the subroutine calls.

So that users do not have to remember the dependencies of a particular
library, each library has a link script that contains references
to its own libraries, and all the other libraries that it depends upon.
Most of the other references will be to the link scripts for the other
libraries.

This in itself presents a problem -- the nested links scripts can
generate a long list of libraries, often with each library occurring more
than once.

To avoid this, each link script has an internal mechanism that trims
unnecessary occurrences of a library out of the list.

The link script writes a list of libraries to its standard output, so
to get the list into your compile or link command, you need to run the
script as part of the compile or link command.  To do this you just
back-quote the link script name thus: \texttt{`bly\_link`}.

The result is that when using the link scripts, you do not have to worry
about remembering which libraries that the one you need depends upon, and
the linker is provided with a simple list of dependant libraries in the
correct order to resolve all external references in your source code.

\subsection{\xlabel{compiling_and_linking_on_liunux}Compiling and Linking on Linux}
\label{compiling_and_linking_on_liunux}

This section applies to Linux systems only.

The Starlink libraries on Linux systems are compiled so that they are
compatible with the GNU \texttt{gcc/g77} system and the \texttt{f2c}
compiler.  To do this, the \texttt{`-fno-second-underscore'} compiler
flag is used.

This means that to compile and link code with the Starlink libraries
on Linux systems, you must use the \texttt{`-fno-second-underscore'}
flag for the \texttt{g77} compiler/linker, thus:

\begin{quote}
\begin{verbatim}
% g77 -O -fno-second-underscore myprog.f -o myprog .....
\end{verbatim}
\end{quote}

\subsection{\xlabel{static_linking}Static Linking}
\label{static_linking}

Your application depends upon the BLY library, so when you link (or
compile and link -- it does not matter), you need to tell it to link with
the \texttt{libbly.a} library.

In a one-stage compile and link, you would use the following:

\begin{quote}
\begin{verbatim}
% f77 -O myprog.f -o myprog -L/star/lib `bly_link`
\end{verbatim}
\end{quote}

In a two stage compile and link, you would use the following:

\begin{quote}
\begin{verbatim}
% f77 -O -c myprog.f
% f77 -O myprog.o -o myprog -L/star/lib `bly_link`
\end{verbatim}
\end{quote}

To use the ADAM versions of the library, if you are developing an ADAM
application, use the following:

\begin{quote}
\begin{verbatim}
% alink myprog.f -o myprog -L/star/lib `bly_link_adam`
\end{verbatim}
\end{quote}
or:
\begin{quote}
\begin{verbatim}
% f77 -O -c myprog.f
% alink myprog.o -o myprog -L/star/lib `bly_link_adam`
\end{verbatim}
\end{quote}

You should also include in the link phase any other library link scripts
for those libraries that your application calls directly, and any
libraries of your own \emph{e.g.}:

\begin{quote}
\begin{verbatim}
% f77 -O myprog.f -o myprog ./libmine.a -L/star/lib \
         `bly_link` `other_lib_link` -lmine2
\end{verbatim}
\end{quote}

where \texttt{`other\_lib\_link`} causes linking with another library
using the link script system, and \texttt{-lmine2} causes a link with a
library \texttt{libmine2.a}.

\subsection{\xlabel{dynamic_linking}Dynamic Linking}
\label{dynamic_linking}

If you use the link scripts and the \texttt{-L/star/lib} tag in your link
chain, your executable will be created using the ordinary libraries
available in \texttt{/star/lib} (unless the linker discovers a shareable
version of a library in \texttt{/star/lib} which it will use by default).

A statically linked binary includes all the necessary object code.  This
can create very large binaries, but at at load time, startup is quite
fast, because the runtime linker only has to resolve the system library
references.

In contrast, dynamically linked executables are much smaller -- the
linker just notes in the binary which shareable libraries were used to
resolve which references, and leaves it at that.

When the binary is loaded for execution, the runtime linker looks for the
shared libraries for which it finds references in the binary, and then does a
`fixup' to resolve all the external references using the libraries.  For
executables with a large list of shared libraries, this process can take a
considerable time.  What you save in link time during development and in
disk space, you may pay for when waiting for the binary to load.

Since your binary will need the shared libraries at install time, your
binary will only be portable to systems containing the Starlink shared
library set.

Starlink builds its applications statically linked against the
Infrastructure libraries, and dynamically linked with the system
libraries.  The \texttt{alink} command for ADAM applications triggers static
linking by default.

If you want to take advantage of the speed of dynamic linking and the
disk space savings, you can use the shared libraries in \texttt{/star/share}
to link with:

\begin{quote}
\begin{verbatim}
% f77 -O myprog.f -L/star/share `bly_link`
\end{verbatim}
\end{quote}

There are some caveats (apart from portability already mentioned):

\begin{enumerate}

\item This facility is only available on Intel Linux and SPARC Solaris
2 machines -- shareable Infrastructure libraries are not provided for
DEC Alpha Digital Unix machines.

\item At run time, the loader needs to find the shared libraries.  On a
Starlink system your \texttt{LD\_LIBRARY\_PATH} will have been set to enable
this to occur (if you use the Starlink login files).  If not, you should
add \texttt{/star/share} to your \texttt{LD\_LIBRARY\_PATH}.

\item Some Infrastructure libraries do not have shared versions because
it is not possible to generate them, even on Intel Linux and SPARC Solaris 2
systems.

\end{enumerate}

\end{document}
