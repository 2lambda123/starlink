\documentclass[11pt,nolof]{starlink}


%------------------------------------------------------------------------------
\stardoccategory    {Starlink User Note}
\stardocinitials    {SUN}
\stardocnumber      {169.3}
\stardocsource       {sun\stardocnumber}
\stardocauthors     {G.R.Mellor}
\stardocdate        {13 August 1994}
\stardoctitle       {PINE --- Electronic mail interface}
\stardoccopyright{Copyright 1994 Particle Physics \& Astronomy
  Research Council}
\stardocabstract{
The PINE mail interface is a user-friendly mail utility for Unix systems.
It has been adopted by Starlink as the recommended mail utility because
of its ease of use compared with the mail utilities supplied as standard with
the Unix operating system. PINE is intended to be intuitive and ``\emph{to be learned by exploration rather than reading manuals}''. Here however
are a few brief notes to get you started.
}

%------------------------------------------------------------------------------

%------------------------------------------------------------------------------
% Add any \providecommand or \newenvironment commands here
%------------------------------------------------------------------------------

\begin{document}
\scfrontmatter

\section{Using PINE}

PINE is started by typing:

\begin{terminalv}
% pine
\end{terminalv}

If you are invoking PINE for the first time, you will see some
administrative messages and then the PINE main menu. PINE functions
can all be accessed using single letter commands - some of the main
ones are shown on the main menu. Whilst using PINE, other relevant commands
(and their functions) will be listed at the bottom of the screen.
Note that often there are more commands than can be listed on one screen.
Option \texttt{O} will toggle to a list of ``other'' commands.

\subsection{Reading Mail}

Mail sent to your user account arrives in a system directory. PINE
see this directory as a ``mail folder'' called \texttt{INBOX}. From the
main menu, typing \texttt{I} will list the mail contained in a folder.
By default this will be \texttt{INBOX}. A mail message from the list
can be selected using arrow keys or entering its number. Then \texttt{$<$RETURN$>$} will display the message. Various other navigational commands
appear at the bottom of the screen.

To extract a mail message to a file, use the \texttt{E} command. You will be
prompted for a filename to save the message text in. By default, this
file will be created in your home directory. If you wish to save it
elsewhere, include either a relative (to your home directory)
or absolute directory specification with the filename.

\emph{e.g.}~~\texttt{savemail/message.1} will save file \texttt{message.1} to
\texttt{\$HOME/savemail/message.1}

To print mail, use the \texttt{Y} command. This will prompt for
a print command. A suitable default command can be configured as
shown in Section\ref{sec-cust}.

\subsection{Folders}

Once your incoming mail is read, it should be deleted (\texttt{D}) or
stored neatly in a folder. It is essential that mail you wish to save
is moved to your personal folders to avoid filling up the main system
mail area where your \texttt{INBOX} is situated.  To save a message to a
folder, type \texttt{S} and PINE will prompt for a folder name. If it does
not exist, PINE will create it (in your \texttt{Mail} subdirectory).
Option \texttt{L} will give a list of your folders which can be navigated
through using arrow keys and then acessed using option \texttt{I}.

\subsection{Sending Mail}

Typing \texttt{C} will produce the compose screen. The first few lines are
fields which are mostly self explanatory. The \texttt{Attchmnt:} field is
usually used for sending non-text files with your message (\emph{e.g.}
GIF image files).

If you wish to include a text file in your message, use the \texttt{$<$CTRL$>$-R} option when composing the message. You will be prompted
for a filename.  Note that by default, PINE assumes that the file is in
your home directory. If this is not the case, include either a relative
(to your home directory) or absolute directory specification with the
filename.

To write the mail message, a rudimentary editor called \texttt{Pico} is
invoked by default.  Its various functions are listed at the bottom of
the screen.  PINE can be configured to allow the use of your own
favourite editor (see section~\ref{sec-cust}). This cannot be invoked
by default however.  When commencing composition, it is necessary to
use command \texttt{$<$CTRL$>$-$<$underscore$>$} to invoke your personal
editor.  Afterwards, exit your editor in the usual way.

Once the message is finished (whichever editor was used for composition),
type  \texttt{$<$CTRL$>$-X} to send it.

\subsection{Replying/Forwarding Mail}

To reply to or forward a mail message, first select or read the
appropriate message. Then to reply, use option \texttt{R}.  This gives the
useful option of including the original mail message in your message
allowing you to respond to specific points. Option \texttt{F} will forward
mail. Note the flexibility of the system - it is possible to modify the
message, addressee, and/or subject before actually sending.

\subsection{Address Book and Distribution Lists}

In order to save memorising all those long email addresses, it is
possible to save them in your personal address book together with a
nickname.  When composing a message in future, the nickname can be used
and PINE will substitute the correct address. From the main menu,
option \texttt{A} will select the address book. Now option \texttt{A} will
add an address to the book. You are prompted for the person's full
name, mail address and nickname.

The address book also allows the creation of distribution lists.
Option \texttt{S} creates a list and option \texttt{Z} adds to an existing
list of users. To mail to this list, just type the list nickname in the
\texttt{To:} field and PINE will expand it.

\subsection{Customising PINE}
\label{sec-cust}

When you invoke PINE for the first time, a file called \texttt{.pinerc},
containing a set of commands to customise PINE, is created in the your
home directory. The initial settings are the system defaults but it is
possible to customise them for your own preferences.  Some of these
settings should not be changed if PINE is to work correctly, however
some of the configurable ones are listed below, together with
explanations (as comment lines beginning with \texttt{\#}).

\begin{small}
\begin{terminalv}
# default-fcc specifies a folder where a copy of outgoing mail is saved.
# To suppress saving of outgoing mail, set: default-fcc=""
default-fcc="saved-messages"

# signature-file specifies the name or path of a file containing text that
# will automatically be inserted in outgoing mail.
signature-file=/home/ltsun0/grm/signature.txt

# sort-key= order in which messages will be presented...
# Choose one: subject, from, arrival, date, size.
# Normal default is "arrival"
sort-key=date

# editor specifies the program invoked by ^_ in the Composer.
# This is normally an alternative to Pine's internal composer (Pico)
editor=/usr/local/bin/jed

# Program to view images if format such as GIF and TIFF
image-viewer=/soft2/bin/xv

# Your printer selection
printer=lpr -Pstar_laser

# Special print command if it isn't one of the standard printers
personal-print-command=lpr -Pstar_laser
\end{terminalv}
\end{small}

\section{Mail addresses}

For information on how to determine and construct mail addresses
see \xref{SUN/182}{sun182}{} "Email Addressing".

\section{Acknowledgements}

I would like to thank Dr A.C.Davenhall and Dr C.G.Page for their assistance
in preparing this document.

\end{document}
