\documentstyle[11pt]{article}
\pagestyle{myheadings}

%------------------------------------------------------------------------------
\newcommand{\stardoccategory}  {Starlink User Note}
\newcommand{\stardocinitials}  {SUN}
\newcommand{\stardocnumber}    {82.4}
\newcommand{\stardocauthors}   {D C Parsons, H J Walker}
\newcommand{\stardocdate}      {21 June 1993}
\newcommand{\stardoctitle}     {IRAS --- Data Products Primer}
%------------------------------------------------------------------------------

\newcommand{\stardocname}{\stardocinitials /\stardocnumber}
\markright{\stardocname}
\setlength{\textwidth}{160mm}
\setlength{\textheight}{240mm}
\setlength{\topmargin}{-5mm}

\setlength{\oddsidemargin}{0mm}
\setlength{\evensidemargin}{0mm}
\setlength{\parindent}{0mm}
\setlength{\parskip}{\medskipamount}
\setlength{\unitlength}{1mm}
\begin{document}
\thispagestyle{empty}
SCIENCE \& ENGINEERING RESEARCH COUNCIL \hfill \stardocname\\
RUTHERFORD APPLETON LABORATORY\\
{\large\bf Starlink Project\\}
{\large\bf \stardoccategory\ \stardocnumber}
\begin{flushright}
\stardocauthors\\
\stardocdate
\end{flushright}
\vspace{-4mm}
\rule{\textwidth}{0.5mm}
\vspace{5mm}
\begin{center}
{\Large\bf \stardoctitle}
\end{center}
\vspace{5mm}
%\setlength{\parskip}{0mm}
%\tableofcontents
\setlength{\parskip}{\medskipamount}
\markright{\stardocname}
\setlength{\parskip}{0mm}
\tableofcontents
\setlength{\parskip}{\medskipamount}
\markright{\stardocname}

\pagebreak
\section{Introduction}

The purpose of this note is to introduce the IRAS data products,
describe how to gain access or obtain copies of selected data,
and to guide potential users to Starlink software and documentation which
can help in the analysis of the data.

For ease of readability, I have referred to the user throughout this document
as he, rather than he or she, and I hope that those amongst its readership
who, like me, are female will excuse me.

\section{IRAS}
The InfraRed Astronomical Satellite flew in 1983. It carried three instruments.
The main detector array consisted of 62 detectors sensitive around 12$\mu$m,
25$\mu$m, 60$\mu$m, and 100$\mu$m. The slitless Low Resolution Spectrometer
(LRS) gave spectra between 8$\mu$m and 23$\mu$m with a five channel output. The
Chopped Photometric Channel (CPC) had detectors sensitive at 50$\mu$m and
100$\mu$m.

The main objective of the mission was to make a survey of the whole sky using
the main detector array, with multiple coverage. During this survey, the
telescope scanned almost
the same strip of sky during orbits separated in time anything from one orbit
of the satellite (103 minutes) up to a maximum of 36 hours. This enabled
two sets of data covering the same section of sky, and close in time, to be
compared so that the most obvious inconsistencies could be removed. This group
of observations is termed an HCON (for Hours Confirmed). 96\% of the sky was
covered by at least two HCONs. A three HCON coverage was achieved over 72\% of
the sky, and 15\% of the sky received more than three HCON coverages. A map of
the coverage is given in Appendix {\ref a:1} (from IRAS Catalogues and Atlases:
Explanatory Supplement, p. I--5).

In addition to the whole sky survey, interesting objects were scanned at higher
resolution with the main detector array, these are termed Pointed Observations.
(They may also be referred to as additional observations in earlier
documentation). The CPC maps were also of specific objects only.

The extraction of an LRS spectrum was triggered by the detectors passing over a
sufficiently bright point source.

The raw detector data from the IRAS whole sky survey was first processed to
remove glitches and calibrated to the internal source. The Calibrated
Reconstructed Detector Data (CRDD) comprises this minimally processed data. The
CRDD was subsequently processed to give highly reliable catalogues and images.

The IRAS Point Source Catalog contains 245,839 sources seen in at
least one waveband, however only 6343 were seen at all four wavelengths.
The IRAS Faint Source Catalogue contains a further 173,044 fainter sources
seen in the unconfused 80\% of the sky, but the reliability is slightly reduced.
In addition IRAS data were also processed to give more specialised catalogues.

IRAS images were prepared from both the original CRDD data, and, for the
unconfused higher ecliptic latitudes, from recalibrated data from which the
modelled zodiacal light has been removed. These are known respectively as
Skyflux Plates and IRAS Sky Survey Atlas (ISSA). Lower resolution maps of the
original image output are available.

Details of the IRAS satellite and mission, the calibration and data
reduction techniques used, reliability and other caveats on products are
described in the IRAS Catalogues and Atlases Explanatory Supplement.

The user must be aware that for most IRAS data products there are several
versions available, due to improved calibration. When you are extracting data,
check that you have the most recent version, and when comparing your data with
published results, check that they used the most recent version!

{\bf It cannot be over-emphasised that potential users of IRAS data products
MUST read the IRAS Catalogs and Atlases Explanatory Supplement before embarking
on a research programme utilising any of the products described here.
Particular attention should be given to section A.3 page I--2 of the Explanatory
Supplement which directs the user to certain caveats covering each data
product.}

\section {Catalogues}
\label{m:cats}
IRAS catalogues are of three types:---
\begin{itemize}
\item Detection catalogues containing IRAS fluxes or spectra, {\em e.g.}$\;$ the
IRAS Point Source Catalog,
\item Association catalogues containing one type of IRAS source,
{\em e.g.}$\;$ IRAS Point Source Catalog objects associated with galaxies
and quasars,
\item Position catalogues giving the positions at which observations were
made, but no other data {\em e.g.}$\;$ for Pointed Observations.
\end{itemize}
SUN/120 (CATPAC --- Catalogue Applications Package),
SUN/106 (SCAR --- Beginner's Guide) and SUN/70 (Starlink Catalogue Access and
Reporting) describe how to access the IRAS catalogues. The catalogues are
available on STADAT (the STArlink DATabase machine) and may also be available
at your node.

It is possible, using CATPAC and SCAR, to search through the catalogues using
a wide range of criteria, from names and positions to flux density ratios and
flux quality flags.
It is also possible to extract data from more than one catalogue for sources
present in several catalogues.
The package allows the user to print out information about sources extracted
from the catalogues, or to plot them.

When new IRAS catalogues are released, they are added to the STADAT
database.
There is an up to date on-line listing of the available catalogues on STADAT
which can be viewed using the SCAR CAT\_HELP facility.

\section{Images}
\label{m:images}
Until now astronomical infrared images have been very rare; IRAS has
revolutionised this situation by providing two types of data sets.
\begin{itemize}
\item The first, the Extended Emission data from the survey, consists of
the Infrared Sky Survey Atlas (ISSA), covering the higher ecliptic latitudes
(roughly $| \beta | >$ 20$^{\circ}$), and the infrared ``Skyflux~Plates''
(also called Skyflux images), covering the whole sky --- the infrared analogues
of the Palomar and SERC/ESO optical sky surveys.
There are also lower resolution maps from the survey data.
\item The second, from the IRAS Pointed Observation (PO) programme,
comprises thousands of deeper images of selected objects and regions of sky,
including many examples of virtually every known astronomical phenomenon.
These were made with small raster scans using either the survey array or the
Chopped Photometric Channel (CPC) instrument.
\end{itemize}
These two data bases are described below, followed by an outline of the
pertinent access and analysis software.

\subsection{The Extended Emission data set}

\subsubsection{Skyflux Plates}
The IRAS survey covered most parts of the sky three times during the 10 months
of data collection. Each of these coverages is distinct and is processed
separately, which results in three sets of the image-like products. The first
two coverages were of about 96\% of the sky, the third of about 72\%.
There are 212 high resolution `Skyflux Plates' covering the whole sky in the
first 2 HCONs, the 3rd HCON has 188 Skyflux Plates. All 3 HCONS are available,
the 3rd HCON having been re-released after recalibration.
Since the zodiacal light is different for each HCON, the HCONs are kept
separate.

Each plate has flux and statistical weight images stored for all 4 IRAS
wavebands of 12, 25, 60 and 100 $\mu$m.
Each image is about 16.7 degrees square and the plate centres are separated by
15 degrees in Right Ascension and Declination. The plates nominally have
499 by 499 2-arcmin-square pixels, but this varies from plate to plate.
The quoted effective resolution is 6 arcmin.
The data are stored as FITS images, in a Gnomonic projection which is
described in the header of each image.
The Skyflux data are stored at RAL on magnetic tape, users can obtain
images by contacting RLSTAR::IRASMAIL.

The Skyflux Plate images are available as photographic negatives as well as
digital data on tape.
The negatives have all four bands reproduced side by side in a rectangular
format approximately 5 inches square, which is intended for enlargement to
prints 16 inches by 20 inches.
In order to display the whole dynamic range of the data, the grey scale employed
corresponds to the fifth root of the surface brightness, which is similar in
effect to a logarithmic scale.
The grey-scale photographic product has been distributed to UK astronomical
institutes.

\subsubsection{IRAS Sky Survey Atlas}
The IRAS Sky Survey Atlas consists of images made from whole sky survey data
which has been recalibrated and from which a model of the zodiacal light has
been removed.
Images are formed from the data of each HCON separately, and a combined image
of all HCON data for each plate has also been made and is referred to as HCON0.

Each plate has flux images stored for all 4 IRAS wavebands of 12, 25, 60 and
100 $\mu$m.
Each image is 12.5 degrees square. The plate centres are separated by 10 degrees
in declination, with a variable spacing in Right Ascension due to convergence at
the poles. They have 500 by 500 1.5 arcmin-square pixels, and the resolution is
between 4 and 5 arcmin depending on wavelength.
The images are stored in FITS format.

The first release of ISSA images (Phase I) covers ecliptic latitudes more than
50$^{\circ}$ away from the ecliptic plane. These have been released on CD-ROM,
and a set of CDs has been sent to each Starlink node manager.
The CD set includes a program FFIELD, which tells the user which image
contains the source position required.
RAL has received the Phase II release (on Exabyte), and these images are stored
on the optical disk jukebox on the Space Data Centre VAX. These images cover
the area between approximately 20$^{\circ}$ and 50$^{\circ}$ away from the
ecliptic plane.
The user can obtain copies of Phase II images by contacting RLSTAR::IRASMAIL.
Brief details of
the processing used to obtain the ISSA images, and the caveats upon them are
included with copies of the CD-ROM. For a more detailed description the user
should refer to the IRAS Sky Survey Atlas Explanatory Supplement.

RAL also has a set of ISSA Reject images, available on the optical jukebox.
These are images less than 20$^{\circ}$ away from the ecliptic plane. These
data do not meet the standards imposed on the main product, and care is needed
in their use.
The user can obtain copies of Reject images by contacting RLSTAR::IRASMAIL.

\subsubsection{Galactic Plane Maps}
These cover the band of sky within \underline{+}10 degrees of the galactic
plane.
They are a set of 24 images, obtained by rebinning a subset of the Skyflux
Plates.

The maps are 16.7 by 20 degrees and have map centres separated by 15 degrees in
galactic longitude.
The maps nominally have 499 by 599 2-arcmin-square pixels.
They consist of 4 images for each map area, one for each of the IRAS
wavebands, 12, 25, 60 and 100 $\mu$m.
The data from each HCON are stored separately.
The rebinning gives some degradation of the 6 arcmin Skyflux plate resolution.
The maps are presented in the Lambert normal equivalent cylindrical projection
which is described in the FITS header of each image.
The associated statistical weight maps are not available directly, but can be
derived from those for the Skyflux plates.
The maps are available on tape at RAL.

\subsubsection{Low Resolution All Sky Maps}
These are derived from the first version of the Zodiacal Observation History
File (see section \ref{m:zodiac} below), split into the three HCONs.
The two maps each cover the whole sky, and are centred on the galactic centre
and anti-centre.
They are stored in galactic coordinates.
They each consist of 4 images, one for each of the IRAS wavebands, 12, 25,
60 and 100 $\mu$m.
The maps have 325 by 649 half degree pixels, the effective resolution being
about a degree.
The maps are in the Aitoff projection which is described in the FITS header of
each image.
Again the maps are available on tape at RAL.

\subsection{The Pointed Observation images}
For 40\% of its lifetime, IRAS was making pointed-mode observations (usually
small raster scans) of a large variety of interesting astronomical objects.
These observations were to make deep maps or photometry using the survey array.
The Pointed Observation data set consists of $\sim$10,000 fields.


\subsubsection{Observations with the survey array}
Details of the observations, and caveats on them, are given in `A Users Guide to
IRAS Pointed Observation Products'.
A list of the positions where Pointed Observations were made is available
in the IAOD catalogue accessed through SCAR, these observations were originally
known as additional observations hence the name of the catalogue (see the
section on Catalogues \ref{m:cats} above).

Each position is observed at least twice, and separate images are made for each
band in each coverage.

The AO images are available in FITS format, so the standard packages mentioned
below for image analysis can be used on these data, as if they were ordinary
extended emission maps. Further details of how to extract and process these
images are to be found in section \ref{m:imsoft} on Software for IRAS image
analysis below.

\subsubsection{The CPC images}
IRAS Chopped Photometric Channel made pointed observations (using a small
raster scan typically 9 arcmin in length) at 50$\mu$m and 100$\mu$m
simultaneously, and chopped to a reference source at a few degrees Kelvin. The
CPC has a higher spatial resolution than the main detector, around 1 arcmin.
However
the instrument had two major instrumental problems, the most likely cause of
which was too low a temperature in the focal plane. Maps which clearly suffered
from the too simple gain correction procedure were omitted from the set of
maps offered, 1500 observations are considered acceptable. But features running
parallel to the inscan direction should be treated with caution. Users are
referred to the `IRAS-DAX, Chopped Photometric Channel Explanatory Supplement'
for the original caveats, but users should be aware that reprocessed CPC data
was made available in 1990.

A list of the positions where CPC observations were made is available is given
in the CPCD catalogue accessed through SCAR.

The CPC maps are available in FITS format, each consists of a raw and a cleaned
image for each waveband, 50$\mu$m and 100$\mu$m. Details of how these images can
be processed is given in the section on Software for IRAS image analysis in
section \ref{m:imsoft} below.

\subsection{Software for IRAS image analysis}
\label{m:imsoft}
The information given in this section applies to Skyflux plates, ISSA Sky Survey
Atlas images, Galactic Plane and Low Resolution All Sky maps, and Pointed
Observation images described above.

The processing of IRAS images consists of two main stages, reading and
preprocessing the images, and examining them.

The reading and preprocessing stage consists of
\begin{itemize}
\item Determining whether data is available and its storage medium and identity.
\item Extracting the required data from the storage medium.
\item Carrying out processing on the data which makes the information into a
standard form which can be processed by image processing applications.
\end{itemize}
\begin{description}
\item [SUN/80.3] `IRAS90 --- Extraction and
Preparation of IRAS Images' provides details of all these
processes.
\item [SUN/163] `IRAS90 --- Survey and PO Data Analysis Package --- Reference
Guide' provides details of the full flexibility of the PREPARE preprocessor.
\end{description}

In the examination stage the user will probably be using image processing and
similar software to carry out the following types of analysis
\begin{itemize}
\item Displaying the image, or drawing a contour map of it. This includes
annotation with astronomical coordinate positions and user defined information,
and other aspects of presentation.
\item Finding flux at positions on the image including those specified by
astronomical coordinate positions.
\item More particular image processing applications such as colouring the images
to bring out certain features, or smoothing it.
\end{itemize}
\begin{description}
\item [SUN/161] `IRAS90 --- Introductory/User Guide to Calibrated Reconstructed
Detector Data Analysis' provides a walkthrough of how to carry out
frequently used examples of the first two types of processing, using KAPPA
for the image processing and IRAS90 software for applications dealing
with astronomical coordinates.
\item [SUN/163] `IRAS90 --- Survey and PO Data Analysis Package --- Reference
Guide' provides details of the full flexibility of each of the IRAS90
applications.
\item [SUN/95] `KAPPA --- Kernel Application Package' provides full details of
KAPPA and its applications. This is a very flexible image display and processing
suite of applications, however it does not deal with position in astronomical
coordinates, this facility is provided, for IRAS data, by the IRAS90
applications.
\end{description}

\section{Calibrated Raw Detector Data}
The Calibrated Raw Detector Data (CRDD) is the ultimate data set from which all
other IRAS survey array data products are derived.
It is available on 300 6250~bpi tapes held at RAL.

A recalibrated version of the CRDD data, known as PASS3, is soon to be made
available. These data will reside on the optical jukebox at RAL.

A updated suite of software to access, process and help in analysis of CRDD
data, called IRAS90 is now available. This runs under the Starlink ADAM
environment.
\begin{description}
\item [SUN/161] `IRAS90 --- Introductory/User Guide to Calibrated
Reconstructed Detector Data Analysis' provides details of the complete CRDD
processing cycle as carried out with the IRAS90 software and KAPPA.
\item [SUN/163] `IRAS90 --- Survey and PO Data Analysis Package --- Reference
Guide' provides details of the full flexibility of each of the IRAS90
applications.
\item [SUN/95] `KAPPA --- Kernel Application Package' provides full details of
KAPPA and its applications.
\end{description}

Staff at RAL will extract the data from the archive for you, this is a very
time consuming task. Details of the data you will receive are given in SUN/161.
Sources when processed are oval, since the resolution in-scan is better than the
resolution cross-scan and the maps usually have stripes in them, due to the
preliminary calibration made independently for each detector.

The CRDD is the least processed (therefore most inconvenient) form of IRAS data,
however using it is the best way to:
\begin{itemize}
\item Search for weak sources which fall below the IRAS point source thresholds
\item Check variability
\item Disentangle highly confused regions
\item Obtain the highest possible spatial resolution
\end{itemize}

\section{Zodiacal Observation History File}
\label{m:zodiac}
For convenience in the analysis and treatment of background emission from
interplanetary dust (zodiacal emission) and other extremely large scale emission
features of the sky, the IRAS survey data was averaged to half degree square
beam size and, along with pointing information, was preserved as a time ordered
data set.
Three versions of this file are available, of which the latest was prepared from
data from which the point sources from the IRAS point source catalogue have been
removed. Gradients in large scale emission can cause artifacts in this product.
These data are available as the Zodiacal Observation History File, on tape at
RAL. However, no software is currently planned to use them.

\section{LRS Spectra}
The IRAS survey instrumentation included a slitless, Low Resolution
Spectrometer which scanned simultaneously two overlapping wavelength ranges,
one extending from 7.7 to 13.4 $\mu$m and the other from 11.0 to 22.6 $\mu$m.
The resolution varies from $\sim$50 at 8 $\mu$m to $\sim$20 at 23 $\mu$m.
The spectrometer was in operation during the entire survey, providing spectra
for the brighter point sources.
The mean spectra can be found in the IRAS Low Resolution Spectrometer (LRS)
Atlas, which has been incorporated into the SCAR online database mentioned
in \ref{m:cats}.

Program I\_$\,$LRSEXT has been written to extract the 8 -- 13 $\mu$m and
11 -- 22 $\mu$m spectra for a source and write the spectra to files in a
format which can be handled by the spectral analysis program DIPSO (SUN/50).
DIPSO can then be used to display the spectra and perform some analysis on them.
The spectra are extracted from the catalogue on the basis of position
association, the positions being input by the user either interactively or from
a file.
For more details on the program I\_$\,$LRSEXT, reference should be made to
SUN/81.

The IRAS LRS Catalogue contains mean spectra, for sources in the Point
Source Catalog.
The Point Source Catalog flags the sources which have LRS spectra and gives
them a spectral classification, which is described in Section IX.D of the IRAS
Catalogs and Atlases Explanatory Supplement.
Therefore it is possible to select sub-sets of sources with LRS spectra on the
basis of this spectral classification or indeed of any other property of the
sources in the Point Source Catalog.
Alternatively, it is also possible to use SCAR software to associate a private
list of sources with either the Point Source Catalog or the LRS Atlas to
ascertain whether spectra exist for these sources.
However, as yet there is no interface to SCAR in the program I\_$\,$LRSEXT, so
after producing a sub-set using SCAR, you must input the positions to
I\_$\,$LRSEXT to extract the spectra.

The IRASLRS captive account on STADAT allows users to interrogate the LRS
database, see SUN/14.
This allows access to the full dataset of $\sim$170,000 spectra extracted
during the IRAS mission, and from which the LRS Atlas was created.
Some spectra are very obviously ``wrong'', due to incorrect data being stored,
or an extended object being observed.
The spectra are generally of lower quality, but examination of the individual
spectra making up an Atlas entry can be a useful check on a dubious spectral
feature.
Very noisy, but useful, spectra have been extracted for sources as faint as
4 Jy in the IRAS 12$\mu$m band.

\section{Access to Data}
Catalogue data is usually accessed using the network to reach the STADAT
(STArlink DATabase) computer at RAL, and then running SCAR.
Users should contact their local node manager for more information.
Problems and requests for new catalogues can be sent to STADAT::CATMAIL.

Image data (including CRDD) can be supplied on Exobyte for users to analyse at
their home institutes. Requests for this data should be made to RLSTAR::IRASMAIL
giving details of the type of data, and positions.

Requests for other types of data, information on IRAS, and problems with any
access, or analysis software (not catalogues or KAPPA related) should also
be sent to RLSTAR::IRASMAIL.

\section{Available IRAS Data}
The IRAS data come in several categories, each of which is summarised here.
\begin{enumerate}
\item CATALOGUES ---
some of these are accessible through the CATPAC and SCAR system, see NEWS for
current list\\
	\begin{enumerate}
	\item IRAS Point Source Catalog, Version 2 (IRPS)\\
	      A version (actually an index) of the IRPS catalogue sorted
	      by Galactic coordinates is also available (GIRL)
	\item Catalogue of associations to IRAS point sources (AIPS)
              --- the combination of IRPS and AIPS is called ASIR
        \item IRAS Faint Source Survey Catalog (V2FSSC Version 2 and FSSC the
              original version) --- the result of
              reprocessing and coadding the IRAS data for $|$b$|>$50$^{\circ}$
              having 2 to 3 times greater sensitivity than the IRPS.
        \item Catalogue of associations to the Faint Source Survey Catalog
              (FSSA)
	\item IRAS Low Resolution Spectrometer (LRS) Atlas (ILRS) --- 5425
              averaged spectra
	\item IRAS Small Scale Structure Catalog (ISSC) --- 16,740 sources
              with sizes $<$8$'$
        \item Catalogue of associations to the Small Scale Structure sources
              (ASSC)
        \item IRAS Serendipitous Survey Catalog (IRSS) --- 43,866 point
              sources derived from the additional observation fields
        \item Catalogue of associations to the Serendipitous Survey sources
              (AISS)
        \item IRAS Asteroid and Comet Catalog (IRAC) --- IRAS data for known
              asteroids and comets
        \item IRAS Asteroid Sightings Catalog (IASC)
	\item Catalogue listing the positions of Pointed Observations (POs)
              in IAOD
        \item Catalogue listing the positions of CPC observations (CPCD)
	\item Extragalactic sub-catalogue (CATX) --- IRAS Point Sources with
              extragalactic associations, created at RAL
        \item Catalogued galaxies and quasars, Version 2 (XCAT) --- official
              IRAS Point Sources with extragalactic associations
	\item Working Survey Database(WSDB) --- available on tape, but not
              easily accessible\\
              Contains all the HCON detections for all the sources in the
              IRAS Point Source Catalog.
              Hence it contains both the observation and data reduction
              history for each point source and can be used for more detailed
              studies such as variability.
	      The WSDB is more than three times the size of the IRAS Point
              Source Catalog.
	\item High Source Density Bins --- a list of those regions of the sky
              which received more stringent processing due to the large number
              of sources, each wavelength is handled independently.
	\end{enumerate}
\item IMAGES --- see Software for IRAS image analysis section \ref{m:imsoft}
above.
	Each image consists of 4 sets of data (images),
	corresponding to the 4 IRAS
	wavebands of 12, 25, 60 and 100 $\mu$m.
	\begin{enumerate}
	\item Extended Emission data set, consisting of (for each HCON)\\
		\begin{enumerate}
		\item 212 high resolution ``Skyflux Plates'' (188 for 3rd HCON)
                \item The complete set of ISSA high resolution, recalibrated,
                Zodiacal light removed images, including reject plates.
		\item 24 ``Galactic Plane Maps'' (a rebinning of Skyflux Plates)
		\item 2 ``Low Resolution All Sky Maps'', each covering the
		whole sky, one centred at the galactic centre the other
		at the anti-centre, with resolution of 0.5$^{\circ}$ (from
                ZOHF data)
		\end{enumerate}
	\item Pointed Observation (PO) deep images\\
        Images are available for approximately 10,000 positions.
	A list of the positions at which these images were taken, ordered by
        DEC is available in SCAR catalogue IAOD.
	\item FITS images of IRAS Observations of Large Optical Galaxies\\
         A FITS tape containing the ``Catalog of IRAS Observations of Large
         Optical Galaxies'' (Rice et al., 1988: Ap.J. Supp {\bf 68}, 91),
         with an accompanying report. This tape contains IRAS images, made
         from the co-added raw data, of 85 2nd Reference Catalogue bright
         galaxies with blue light isophotal diameters $>$8$'$.

	\end{enumerate}
\item CPC data \\
         The 1500 observations made with the CPC are available in FITS format,
         each observation comprising of 4 images, `clean' maps at 50 and 100
         microns, and raw maps at 50$\mu$m and 100$\mu$m.
         A list of the positions at which these images were taken are available
         in the SCAR catalogue CPCD.
\item ZODIACAL OBSERVATION HISTORY FILE (ZOHF) \\
         These give a time ordered history of emission as seen by IRAS at all
         4 wavelengths,
         with a resolution of 0.5$^{\circ}$, in ecliptic coordinates.
\item CALIBRATED RAW DETECTOR DATA (CRDD) --- see SUN/161\\
	This is the ultimate source of all other survey products.
	Users may obtain copies of data for small selected areas from RAL.
\item LOW RESOLUTION SPECTROMETER (LRS) SPECTRA --- see SUN/81 and SUN/14\\
There are two overlapping wavelength ranges: 7.7 -- 13.4 $\mu$m and
11.0 -- 22.6 $\mu$m.
The spectra of the 5425 brightest (and most reliable) sources in the IRAS Low
Resolution Spectrometer Atlas (ILRS) are available in SCAR, and through
LRSEXT.\\
The full dataset of $\sim$170,000 spectra for $\sim$40,000 sources is
available on STADAT using the captive account IRASLRS.
This includes the individual spectra contributing to the mean spectra in
the Atlas.
\end{enumerate}

\section{References}
{\em IRAS Catalogs and Atlases: Explanatory Supplement}, 1988. ed. C.A.
Beichman, G. Neugebauer, H.J. Habing, P.E. Clegg, and T.J. Chester (Washington,
DC:GPO) --- This will be found in the libraries of most institutes.

{\em IRAS Sky Survey Atlas Explanatory Supplement}, 1993. S Wheelock et al.

{\em A Users Guide to IRAS Pointed Observation Products}, 1985. E.T. Young et
al. IPAC preprint no. PRE-008N.

{\em IRAS-DAX Chopped Photometric Channel Explanatory Supplement} P.R. Wesselius
 et al. Laboratory for Space Research, Groningen, Netherlands.

\begin{tabular}{ccl}
SUN/14&---&IRASLRS --- Obtaining spectra from the IRAS LRS database\\
SUN/50&---&DIPSO --- A friendly spectrum analysis program\\
SUN/70&---&SCAR --- Starlink Catalogue Access and Reporting\\
SUN/80&---&IRAS90 --- Extraction and Preparation of IRAS Images\\
SUN/81&---&LRSEXT --- IRAS LRS Spectra Extraction\\
SUN/95&---&KAPPA --- Kernel Application Package\\
SUN/106&---&SCAR --- Beginner's Guide\\
SUN/120&---&CATPAC --- Catalogue Applications Package\\
SUN/161&---&IRAS90 --- Introductory/User Guide to Calibrated Reconstructed
Detector Data Analysis\\
SUN/163&---&IRAS90 --- Survey and PO Data Analysis Package --- Reference
Guide\\
\end{tabular}
\appendix
\section{A Map of the IRAS Whole Sky Survey Coverage}
\label{a}
\end{document}
