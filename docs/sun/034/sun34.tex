\documentstyle[11pt]{article} 
\pagestyle{myheadings}

%------------------------------------------------------------------------------
\newcommand{\stardoccategory}  {Starlink User Note}
\newcommand{\stardocinitials}  {SUN}
\newcommand{\stardocnumber}    {34.1}
\newcommand{\stardocauthors}   {G R Mellor}
\newcommand{\stardocdate}      {7 October 1991}
\newcommand{\stardoctitle}     {GNU EMACS --- Display Editor}
%------------------------------------------------------------------------------

\newcommand{\stardocname}{\stardocinitials /\stardocnumber}
\renewcommand{\_}{{\tt\char'137}}     % re-centres the underscore
\markright{\stardocname}
\setlength{\textwidth}{160mm}
\setlength{\textheight}{230mm}
\setlength{\topmargin}{-2mm}
\setlength{\oddsidemargin}{0mm}
\setlength{\evensidemargin}{0mm}
\setlength{\parindent}{0mm}
\setlength{\parskip}{\medskipamount}
\setlength{\unitlength}{1mm}

%------------------------------------------------------------------------------
% Add any \newcommand or \newenvironment commands here
%------------------------------------------------------------------------------

\begin{document}
\thispagestyle{empty}
SCIENCE \& ENGINEERING RESEARCH COUNCIL \hfill \stardocname\\
RUTHERFORD APPLETON LABORATORY\\
{\large\bf Starlink Project\\}
{\large\bf \stardoccategory\ \stardocnumber}
\begin{flushright}
\stardocauthors\\
\stardocdate
\end{flushright}
\vspace{-4mm}
\rule{\textwidth}{0.5mm}
\vspace{5mm}
\begin{center}
{\Large\bf \stardoctitle}
\end{center}
\vspace{5mm}

%------------------------------------------------------------------------------
%  Add this part if you want a table of contents
%  \setlength{\parskip}{0mm}
%  \tableofcontents
%  \setlength{\parskip}{\medskipamount}
%  \markright{\stardocname}
%------------------------------------------------------------------------------

\section{Introduction}

This document describes the GNU EMACS editor as distributed by Starlink
for UNIX systems. It is intended to provide an introduction for the
new user. For a comphrehensive description of the editor, refer to
the {\it GNU EMACS Manual}.


The GNU EMACS editor is a powerful editor with many advanced customizable
features. 
However, it is relatively simple to provide
an EDT style emulation 
for those users already familiar with VAX/VMS editors and the windowing
facilities similar to the EVE editor are also available.

\section{Getting Started}

An Emacs tutorial is provided with the editor and can be accessed
by first running Emacs by typing {\tt emacs} and then typing {\tt C-h t}
where {\tt C-h} means pressing the Control and h keys simultaneously.
This teaches the raw commands which are all key combinations commencing
with either the Control Key ({\tt C}) or the Meta/Escape Key ({\tt M}).
A more
user friendly system can be obtained by customizing the editor.
The customized definitions reside in an initialisation file {\tt .emacs}
in the home directory. A template {\tt .emacs} file providing
a basic EDT keypad emulation is available in {\tt /star/emacs}.
Some further basic commands that might also be required are listed 
in the GNU Emacs Reference Card which is appended to this document.

\section{Exiting Emacs}

There are two different ways of exiting Emacs: {\it suspending} and
{\it killing} Emacs. {\it killing} means destroying the Emacs job,
however {\it suspending} Emacs allows you to return to the session later.
The best way to finish a session is to save the file and then suspend
Emacs. Emacs can be resumed later with a {\tt \%emacs} command if you are 
using the C shell.

\end{document}
