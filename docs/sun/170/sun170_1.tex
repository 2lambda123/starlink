\documentstyle[11pt]{article} 
\pagestyle{myheadings}

%------------------------------------------------------------------------------
\newcommand{\stardoccategory}  {Starlink User Note}
\newcommand{\stardocinitials}  {SUN}
\newcommand{\stardocnumber}    {170.1}
\newcommand{\stardocauthors}   {C A Clayton}
\newcommand{\stardocdate}      {1 June 1993}
\newcommand{\stardoctitle}     {Editors and Mail on Unix}
%------------------------------------------------------------------------------

\newcommand{\stardocname}{\stardocinitials /\stardocnumber}
\renewcommand{\_}{{\tt\char'137}}     % re-centres the underscore
\markright{\stardocname}
\setlength{\textwidth}{160mm}
\setlength{\textheight}{230mm}
\setlength{\topmargin}{-2mm}
\setlength{\oddsidemargin}{0mm}
\setlength{\evensidemargin}{0mm}
\setlength{\parindent}{0mm}
\setlength{\parskip}{\medskipamount}
\setlength{\unitlength}{1mm}

%------------------------------------------------------------------------------
% Add any \newcommand or \newenvironment commands here
%------------------------------------------------------------------------------

\begin{document}
\thispagestyle{empty}
SCIENCE \& ENGINEERING RESEARCH COUNCIL \hfill \stardocname\\
RUTHERFORD APPLETON LABORATORY\\
{\large\bf Starlink Project\\}
{\large\bf \stardoccategory\ \stardocnumber}
\begin{flushright}
\stardocauthors\\
\stardocdate
\end{flushright}
\vspace{-4mm}
\rule{\textwidth}{0.5mm}
\vspace{5mm}
\begin{center}
{\Large\bf \stardoctitle}
\end{center}
\vspace{5mm}

\section{Introduction}

The purpose of this document is to give new Unix users some advice on how
to choose which editor and which mail utility to use on Unix machines.
Under VMS the choice is simple; virtually everyone uses TPU or EDT as
his or her editor and there is only one MAIL utility. Under Unix the default
editor and mail utilities are considered to be unfriendly and many users prefer
to use other more sophisticated alternatives. However, many such alternatives
exist; there is not one single editor or mail utility that everyone 
uses and hence each user must decide for himself or herself which to adopt. 


\section{Editors}

In this section I point out some pros and cons of each editor to help you
choose for yourself which you should adopt. When Starlink started the move to
Unix, we recommended that users use either \verb|vi| or \verb|emacs|, mainly
since these editors are very widely available (SGP/7).
We considered \verb|vi| adequate
for the average user with \verb|emacs| as an option for those who needed 
something friendlier and could cope with the greater
sophistication. However, this advice was not popular with all users. Some felt
that \verb|vi| was just too unfriendly 
and \verb|emacs| too complex. Our original recommendation still
stands, but I am also including here information on other editor options.

\subsection{vi}

\subsubsection*{Pros}

\begin{itemize}
\item used by lots of people
\item will be available on every Unix machine you encounter, anywhere
\item works with any keyboard since it only uses standard {\tt qwerty} keys
and not the keypad
\item no cost (in terms of software support and installation)
\item fast
\end{itemize}

\subsubsection*{Cons}

\begin{itemize}
\item unfriendly
\item not as powerful as some of the other editors
\end{itemize}


\subsection{emacs}

\subsubsection*{Pros}

\begin{itemize}
\item very powerful editor
\item available for most Unix platforms but sites must build and 
install it themselves
\item free and readily available
\item works with any keyboard since it only uses standard {\tt qwerty} keys
and not the keypad (unless you choose to configure it to do so)
\item available on non--Unix platforms (e.g. VMS and MS--DOS)
\end{itemize}

\subsubsection*{Cons}
\begin{itemize}
\item not necessarily available on every Unix machine you will meet
\item too complex for some users
\end{itemize}

\subsection{TPU (EDT) emulators}

\subsubsection*{Pros}

\begin{itemize}
\item easiest transition from VMS to Unix
\end{itemize}

\subsubsection*{Cons}

\begin{itemize}
\item costs money
\item emulation problems on some keyboards
\item you are {\bf very} unlikely to find a TPU emulator on non-Starlink machines
\item you {\bf cannot} take a copy of a TPU emulator with you to non-Starlink sites
\end{itemize}

\subsection{jed (see SUN/168) (and other public domain editors)}

\subsubsection*{Pros}

\begin{itemize}
\item can emulate EDT incompletely but quite usably
\item is free
\item you {\bf can} take a copy of \verb|jed| with you when you leave Starlink
\item available on non-Unix platforms (e.g. VMS and MS--DOS)
\end{itemize}

\subsubsection*{Cons}
\begin{itemize}
\item is less ubiquitous than \verb|vi| or \verb|emacs|
\item EDT emulation is not perfect (e.g. no highlight on cut/paste)
\end{itemize}


\subsection{X--windows based editors}

\subsubsection*{Pros}

\begin{itemize}
\item some users find this type of editor easy to use
\end{itemize}

\subsubsection*{Cons}
\begin{itemize}
\item use limited to an X--display
\item X--based editors are not part of Unix and thus can vary
enormously between different machine types. Hence you will probably 
have to learn several different X--editors if you wish to use all 
types of Starlink Unix machine.
\end{itemize}




\section{Mail utility}

The default mail utility in Unix is adequate but unfriendly. Hence, many users
prefer to use one of many freely available public-domain alternatives. As
with editors, different users prefer different mailers and there is no outright
leader in the field. After consultation with users and site managers, we have
decided to recommend a public-domain mailer called \verb|pine|.
This decision is not the result of a detailed evaluation of many different
mailers; whichever mailer we select for whatever reasons, it will be considered
the wrong choice by some people. We are picking \verb|pine|  partly because
users seem to like it and it is easy to use, but mostly because we  must
standardize on a single mailer to help new users.

\verb|pine| is a mailer designed specifically for ease-of-use with
the Unix novice in mind. For example, it has a high-tolerance of user mistakes
and always-present command menus. It is considered that \verb|pine| can be
learned by exploration rather than reading manuals, although \verb|pine| is in
fact fully documented (SUN/169).

Attractive features include:

\begin{itemize}

\item mail index showing a message summary which includes the status, sender,
size, date and subjects of messages

\item address book for saving long, complex addresses and personal 
distribution lists under a nickname

\item multiple folders and folder management screen for filing messages

\item message composer with easy-to-use editor and spelling checker

\item on--line help specific to each screen and context

\end{itemize}

\verb|pine| should be available on all Starlink Unix machines. If it is not, see your
Site Manager. 

\end{document}
