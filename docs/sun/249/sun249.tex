\documentclass[twoside,11pt]{article}

% ? Specify used packages
\usepackage{graphicx}        %  Use this one for final production.
% \usepackage[draft]{graphicx} %  Use this one for drafting.
% ? End of specify used packages

%% Need PDF to take preference over PNG
\DeclareGraphicsExtensions{.pdf,.png}

\pagestyle{myheadings}

% -----------------------------------------------------------------------------
% ? Document identification
% Fixed part
\newcommand{\stardoccategory}  {Starlink User Note}
\newcommand{\stardocinitials}  {SUN}
\newcommand{\stardocsource}    {sun\stardocnumber}
\newcommand{\stardoccopyright}
{Copyright \copyright\ 2000 Council for the Central Laboratory of the Research Councils}

% Variable part - replace [xxx] as appropriate.
\newcommand{\stardocnumber}    {249.2}
\newcommand{\stardocauthors}   {Mark Taylor}
\newcommand{\stardocdate}      {8 June 2005}
\newcommand{\stardoctitle}     {INT-WFC - Astrometric Calibration for the INT Wide Field Camera}
\newcommand{\stardocversion}   {}
\newcommand{\stardocmanual}    {User Note}
\newcommand{\stardocabstract}  {
Images observed using the Wide Field Camera instrument
on the Isaac Newton Telescope contain nonlinear optical distortions.
This note gives a quantitative description of these distortions.
It also explains how they can be corrected for easily using
the \xref{CCDPACK}{sun139}{} package.
}
% ? End of document identification
% -----------------------------------------------------------------------------

% +
%  Name:
%     sun.tex
%
%  Purpose:
%     Template for Starlink User Note (SUN) documents.
%     Refer to SUN/199
%
%  Authors:
%     AJC: A.J.Chipperfield (Starlink, RAL)
%     BLY: M.J.Bly (Starlink, RAL)
%     PWD: Peter W. Draper (Starlink, Durham University)
%
%  History:
%     17-JAN-1996 (AJC):
%        Original with hypertext macros, based on MDL plain originals.
%     16-JUN-1997 (BLY):
%        Adapted for LaTeX2e.
%        Added picture commands.
%     13-AUG-1998 (PWD):
%        Converted for use with LaTeX2HTML version 98.2 and
%        Star2HTML version 1.3.
%      1-FEB-2000 (AJC):
%        Add Copyright statement in LaTeX
%     {Add further history here}
%
% -

\newcommand{\stardocname}{\stardocinitials /\stardocnumber}
\markboth{\stardocname}{\stardocname}
\setlength{\textwidth}{160mm}
\setlength{\textheight}{230mm}
\setlength{\topmargin}{-2mm}
\setlength{\oddsidemargin}{0mm}
\setlength{\evensidemargin}{0mm}
\setlength{\parindent}{0mm}
\setlength{\parskip}{\medskipamount}
\setlength{\unitlength}{1mm}

% -----------------------------------------------------------------------------
%  Hypertext definitions.
%  ======================
%  These are used by the LaTeX2HTML translator in conjunction with star2html.

%  Comment.sty: version 2.0, 19 June 1992
%  Selectively in/exclude pieces of text.
%
%  Author
%    Victor Eijkhout                                      <eijkhout@cs.utk.edu>
%    Department of Computer Science
%    University Tennessee at Knoxville
%    104 Ayres Hall
%    Knoxville, TN 37996
%    USA

%  Do not remove the %begin{latexonly} and %end{latexonly} lines (used by
%  LaTeX2HTML to signify text it shouldn't process).
%begin{latexonly}
\makeatletter
\def\makeinnocent#1{\catcode`#1=12 }
\def\csarg#1#2{\expandafter#1\csname#2\endcsname}

\def\ThrowAwayComment#1{\begingroup
    \def\CurrentComment{#1}%
    \let\do\makeinnocent \dospecials
    \makeinnocent\^^L% and whatever other special cases
    \endlinechar`\^^M \catcode`\^^M=12 \xComment}
{\catcode`\^^M=12 \endlinechar=-1 %
 \gdef\xComment#1^^M{\def\test{#1}
      \csarg\ifx{PlainEnd\CurrentComment Test}\test
          \let\html@next\endgroup
      \else \csarg\ifx{LaLaEnd\CurrentComment Test}\test
            \edef\html@next{\endgroup\noexpand\end{\CurrentComment}}
      \else \let\html@next\xComment
      \fi \fi \html@next}
}
\makeatother

\def\includecomment
 #1{\expandafter\def\csname#1\endcsname{}%
    \expandafter\def\csname end#1\endcsname{}}
\def\excludecomment
 #1{\expandafter\def\csname#1\endcsname{\ThrowAwayComment{#1}}%
    {\escapechar=-1\relax
     \csarg\xdef{PlainEnd#1Test}{\string\\end#1}%
     \csarg\xdef{LaLaEnd#1Test}{\string\\end\string\{#1\string\}}%
    }}

%  Define environments that ignore their contents.
\excludecomment{comment}
\excludecomment{rawhtml}
\excludecomment{htmlonly}

%  Hypertext commands etc. This is a condensed version of the html.sty
%  file supplied with LaTeX2HTML by: Nikos Drakos <nikos@cbl.leeds.ac.uk> &
%  Jelle van Zeijl <jvzeijl@isou17.estec.esa.nl>. The LaTeX2HTML documentation
%  should be consulted about all commands (and the environments defined above)
%  except \xref and \xlabel which are Starlink specific.

\newcommand{\htmladdnormallinkfoot}[2]{#1\footnote{#2}}
\newcommand{\htmladdnormallink}[2]{#1}
\newcommand{\htmladdimg}[1]{}
\newcommand{\hyperref}[4]{#2\ref{#4}#3}
\newcommand{\htmlref}[2]{#1}
\newcommand{\htmlimage}[1]{}
\newcommand{\htmladdtonavigation}[1]{}

\newenvironment{latexonly}{}{}
\newcommand{\latex}[1]{#1}
\newcommand{\html}[1]{}
\newcommand{\latexhtml}[2]{#1}
\newcommand{\HTMLcode}[2][]{}

%  Starlink cross-references and labels.
\newcommand{\xref}[3]{#1}
\newcommand{\xlabel}[1]{}

%  LaTeX2HTML symbol.
\newcommand{\latextohtml}{\LaTeX2\texttt{HTML}}

%  Define command to re-centre underscore for Latex and leave as normal
%  for HTML (severe problems with \_ in tabbing environments and \_\_
%  generally otherwise).
\renewcommand{\_}{\texttt{\symbol{95}}}

% -----------------------------------------------------------------------------
%  Debugging.
%  =========
%  Remove % on the following to debug links in the HTML version using Latex.

% \newcommand{\hotlink}[2]{\fbox{\begin{tabular}[t]{@{}c@{}}#1\\\hline{\footnotesize #2}\end{tabular}}}
% \renewcommand{\htmladdnormallinkfoot}[2]{\hotlink{#1}{#2}}
% \renewcommand{\htmladdnormallink}[2]{\hotlink{#1}{#2}}
% \renewcommand{\hyperref}[4]{\hotlink{#1}{\S\ref{#4}}}
% \renewcommand{\htmlref}[2]{\hotlink{#1}{\S\ref{#2}}}
% \renewcommand{\xref}[3]{\hotlink{#1}{#2 -- #3}}
%end{latexonly}
% -----------------------------------------------------------------------------
% ? Document specific \newcommand or \newenvironment commands.

% Size of \tt text - looks better smaller in LaTeX but not HTML.
\newcommand{\ttsize}{\latexhtml{\small}{}}

% Environment for indenting and using a small font.
\newenvironment{myquote}{\begin{quote}\ttsize}{\end{quote}}

% Figure environment. Defined for latex2html. #1 is postscript file
% #2 the qualifiers, #3 the gif file, #4 the label, #5 the caption.
\newcommand{\myfig} [5] {
  \begin{figure}
    \centering\includegraphics[#2]{#1}
    \typeout{#1 inserted on page \arabic{page}}
    \caption{\label{#4}#5}
  \end{figure}
  }
\begin{htmlonly}
  \newcommand{\myfig}[5]{
    \htmladdimg{#3}\\
    Figure: \label{#4} #5
    }
\end{htmlonly}


% ? End of document specific commands
% -----------------------------------------------------------------------------
%  Title Page.
%  ===========
\renewcommand{\thepage}{\roman{page}}
\begin{document}
\thispagestyle{empty}

%  Latex document header.
%  ======================
\begin{latexonly}
   CCLRC / \textsc{Rutherford Appleton Laboratory} \hfill \textbf{\stardocname}\\
   {\large Particle Physics \& Astronomy Research Council}\\
   {\large Starlink Project\\}
   {\large \stardoccategory\ \stardocnumber}
   \begin{flushright}
   \stardocauthors\\
   \stardocdate
   \end{flushright}
   \vspace{-4mm}
   \rule{\textwidth}{0.5mm}
   \vspace{5mm}
   \begin{center}
   {\Huge\textbf{\stardoctitle \\ [2.5ex]}}
   {\LARGE\textbf{\stardocversion \\ [4ex]}}
   {\Huge\textbf{\stardocmanual}}
   \end{center}
   \vspace{5mm}

% ? Add picture here if required for the LaTeX version.
%   e.g. \includegraphics[scale=0.3]{filename.ps}
% ? End of picture

% ? Heading for abstract if used.
   \vspace{10mm}
   \begin{center}
      {\Large\textbf{Abstract}}
   \end{center}
% ? End of heading for abstract.
\end{latexonly}

%  HTML documentation header.
%  ==========================
\begin{htmlonly}
   \xlabel{}
   \begin{rawhtml} <H1> \end{rawhtml}
      \stardoctitle\\
      \stardocversion\\
      \stardocmanual
   \begin{rawhtml} </H1> <HR> \end{rawhtml}

% ? Add picture here if required for the hypertext version.
%   e.g. \includegraphics[scale=0.7]{filename.ps}
% ? End of picture

   \begin{rawhtml} <P> <I> \end{rawhtml}
   \stardoccategory\ \stardocnumber \\
   \stardocauthors \\
   \stardocdate
   \begin{rawhtml} </I> </P> <H3> \end{rawhtml}
      \htmladdnormallink{CCLRC / Rutherford Appleton Laboratory}
                        {http://www.cclrc.ac.uk} \\
      \htmladdnormallink{Particle Physics \& Astronomy Research Council}
                        {http://www.pparc.ac.uk} \\
   \begin{rawhtml} </H3> <H2> \end{rawhtml}
      \htmladdnormallink{Starlink Project}{http://www.starlink.ac.uk/}
   \begin{rawhtml} </H2> \end{rawhtml}
   \htmladdnormallink{\htmladdimg{source.gif} Retrieve hardcopy}
      {http://www.starlink.ac.uk/cgi-bin/hcserver?\stardocsource}\\

%  HTML document table of contents.
%  ================================
%  Add table of contents header and a navigation button to return to this
%  point in the document (this should always go before the abstract \section).
  \label{stardoccontents}
  \begin{rawhtml}
    <HR>
    <H2>Contents</H2>
  \end{rawhtml}
  \htmladdtonavigation{\htmlref{\htmladdimg{contents_motif.gif}}
        {stardoccontents}}

% ? New section for abstract if used.
  \section{\xlabel{abstract}Abstract}
% ? End of new section for abstract
\end{htmlonly}

% -----------------------------------------------------------------------------
% ? Document Abstract. (if used)
%  ==================
\stardocabstract
% ? End of document abstract

% -----------------------------------------------------------------------------
% ? Latex Copyright Statement
%  =========================
\begin{latexonly}
\newpage
\vspace*{\fill}
\stardoccopyright
\end{latexonly}
% ? End of Latex copyright statement

% -----------------------------------------------------------------------------
% ? Latex document Table of Contents (if used).
%  ===========================================
  \newpage
  \begin{latexonly}
    \setlength{\parskip}{0mm}
    \tableofcontents
    \setlength{\parskip}{\medskipamount}
    \markboth{\stardocname}{\stardocname}
  \end{latexonly}
% ? End of Latex document table of contents
% -----------------------------------------------------------------------------
\cleardoublepage
\renewcommand{\thepage}{\arabic{page}}
\setcounter{page}{1}

\section{Astrometric calibration for INT Wide Field Camera images}

The Wide Field Camera instrument on the Isaac Newton Telescope
contains four CCD chips of 2048 $\times$ 4096 pixels
positioned roughly, but not exactly, as shown in
figure~\ref{fig:ideal}.

\myfig{sun249_ideal}{}{sun249_ideal.png}{fig:ideal}{Idealised arrangement of WFC CCDs}

In order to do accurate astrometry with data obtained
from this instrument, it is necessary to correct for the
exact orientation and position of each CCD in relation to
the others, as well as for nonlinear distortions
away from the optical axis of the focal plane introduced by
the optics of the telescope.
The nonlinear distortion is very well modelled by a radial ``pincushion''
transformation, which has the form
\begin{displaymath}
  r' = r ( 1 + D r^2 )
\end{displaymath}

The following transformations take account of both these effects
to map the pixel coordinates $(x_i, y_i)$
of CCD\#$i$ into a new pixel-like coordinate system
$(x', y')$
of uniform scale, which is the same for all four CCDs.
The origin of this coordinate system is at pixel coordinates
$(1778, 3029)$ of CCD\#4, which is taken to be on the optical axis.
At this point, one unit of the new coordinate system is equivalent
to one pixel of the CCD\#4 system, so that the new coordinates
are almost equivalent to CCD\#4 coordinates, although they diverge
away from the origin thanks to the nonlinear distortion.
Each unit of the new coordinates has the same size,
which is approximately 0.333 arcsec.

The corrected coordinates $(x', y')$ are obtained from
each set of pixel coordinates $(x_i, y_i)$
by first translating so that the origin is on the optical
axis, then rotating to the correct angle,
then correcting for the radial distortion effect:
\begin{eqnarray*}
  x_i^{\rm shift} & = & x_i - X_i \\
  y_i^{\rm shift} & = & y_i - Y_i \\[1ex]
%
  x_i^{\rm rot}   & = & x_i^{\rm shift} \cos \theta_i
                      - y_i^{\rm shift} \sin \theta_i  \\
  y_i^{\rm rot}   & = & x_i^{\rm shift} \sin \theta_i
                      + y_i^{\rm shift} \cos \theta_i  \\[1ex]
%
  x'              & = & x_i^{\rm rot}
                        \left( 1 + D \left[ \left. x_i^{\rm rot} \right.^2
                                          + \left. y_i^{\rm rot} \right.^2
                                     \right] \right)  \\
  y'              & = & y_i^{\rm rot}
                        \left( 1 + D \left[ \left. x_i^{\rm rot} \right.^2
                                          + \left. y_i^{\rm rot} \right.^2
                                     \right] \right)
\end{eqnarray*}
Where $(X_i, Y_i)$ are the coordinates of the optical centre of
the instrument in the pixel coordinate system of CCD\#$i$,
$\theta_i$ is the angle at which CCD\#$i$ sits on the focal plane,
and $D$ is the pincushion distortion coefficient.

Values for these coefficients have been obtained by registering
two exposures of the same region of sky, in which the instrument
had been rotated by 180$^\circ$ between exposures.
The value of $(X_4, Y_4)$ was taken from the value used
for the
\htmladdnormallink{Wide Field Survey}{http://www.ast.cam.ac.uk/~wfcsur/astrometry.html}
data, and $\theta_4$ was chosen to be zero.
The values of the coefficients in these equations are approximately:
\begin{displaymath}
  \begin{array}{c@{\ =\ }r@{\hspace{3em}}c@{\ =\ }r@{\hspace{3em}}c@{\ =\ }r}
     X_1 & -336.74  &  Y_1 & 3039.14  &  \theta_1 &   0.01868^\circ  \\
     X_2 & 3180.68  &  Y_2 & 1729.67  &  \theta_2 & -90.62115^\circ  \\
     X_3 & 3876.73  &  Y_3 & 2996.30  &  \theta_3 &   0.11436^\circ  \\
     X_4 & 1778.00  &  Y_4 & 3029.00  &  \theta_4 &   0.00000^\circ
  \end{array}
\end{displaymath}
and
\begin{displaymath}
     D = -5.30 \times 10^{-10} {\rm pixel}^{-2}
\end{displaymath}
This value of $D$ corresponds in units of radians to
\latexhtml{$-203\,{\rm rad}^{-2}$}{-203 rad$^{-2}$},
compared to the value
\latexhtml{$-259.8\,{\rm rad}^{-2}$}{-259.8 rad$^{-2}$}
quoted in the ING observer handbook.

These values of the coefficients are thought to be correct to an
accuracy of 1 or 2 pixels, which is around half an arcsecond.

Figure~\ref{fig:exag} gives an exaggerated representation of the
shapes and positions of the CCDs as mapped on to the new coordinate system.

\myfig{sun249_exag4}{}{sun249_exag4.png}{fig:exag}{Exagerrated view of the
actual arrangement of the CCDs.  The dots in the corners mark the
pixel origin for each CCD and the ``$\times$'' marks the origin of
the unified coordinate system.}

\latexhtml{\newpage}{\htmlrule}
\section{Attaching modified coordinates using CCDPACK}

For users of Starlink software,
the calibrated coordinate information can be attached automatically
to image files using the
\htmladdnormallink{ASTIMP}{http://capc23.ast.cam.ac.uk/star/local/docs/sun139.htx/node70.html#xref_ASTIMP}
command, and the file {\tt INT-WFC.ast}.
This file is distributed in the CCDPACK package
(in directory {\tt \$CCDPACK\_DIR}, or {\tt bin/ccdpack}).

To use this facility,
it is best if the files are first converted to Starlink \xref{NDF}{sun33}{}
format using the \xref{CONVERT}{sun55}{} package as follows:
\begin{myquote}
\begin{verbatim}
% convert

   CONVERT commands are now available -- (Version 1.5-4)

   Defaults for automatic NDF conversion are set.

   Type conhelp for help on CONVERT commands.
   Type "showme sun55" to browse the hypertext documentation.

% fits2ndf in='r10628?.fits' out='*' container=true
\end{verbatim}
\end{myquote}
which will convert the FITS file {\tt r106280.fits}
into the NDF {\tt r106280.sdf} and so on.
The Starlink software will in fact work with FITS files,
but because it converts them on the fly between FITS and NDF formats
before and after each command,
this can be rather slow with files the size of WFC frames.

Once the file is in Starlink NDF format,
from the Unix C shell do the following:
\begin{myquote}
\begin{verbatim}
% ccdpack

   CCDPACK commands are now available -- (Version 4.0-16)

  For help use the commands ccdhelp or ccdwww

% astimp in='r10628?' astfile=$CCDPACK_DIR/INT-WFC.ast reset accept

    ASTIMP
    ======
  Framesets read from file INT-WFC.ast:
    FITS header "ROTSKYPA" used for rotation

     N    Base domain         Current domain      Frameset ID
     --   -----------         --------------      -----------
     1    PIXEL               INT-WFC             FITSID CHIPNAME 'A5506-4'
     2    PIXEL               INT-WFC             FITSID CHIPNAME 'A5383-17-7'
     3    PIXEL               INT-WFC             FITSID CHIPNAME 'A5530-3'
     4    PIXEL               INT-WFC             FITSID CHIPNAME 'A5382-1-7'
  4 NDFs accessed using parameter IN

  Processing NDF /data/cass58a/mbt/data/int4/r106280
    Matched with frameset ID "FITSID CHIPNAME 'A5506-4'"
    Rotating additional 180 degrees
    New frame in domain "INT-WFC" added

  Processing NDF /data/cass58a/mbt/data/int4/r106281
    Matched with frameset ID "FITSID CHIPNAME 'A5383-17-7'"
    Rotating additional 180 degrees
    New frame in domain "INT-WFC" added

  Processing NDF /data/cass58a/mbt/data/int4/r106282
    Matched with frameset ID "FITSID CHIPNAME 'A5530-3'"
    Rotating additional 180 degrees
    New frame in domain "INT-WFC" added

  Processing NDF /data/cass58a/mbt/data/int4/r106283
    Matched with frameset ID "FITSID CHIPNAME 'A5382-1-7'"
    Rotating additional 180 degrees
    New frame in domain "INT-WFC" added
\end{verbatim}
\end{myquote}
This command sets the Current coordinate frame of the specified
NDF files to the new coordinate system $(x',y')$ described above.
As well as applying the linear and nonlinear geometry terms whose
coefficients were given,
it also rotates the coordinates according to the value of the
ROTSKYPA FITS header card, which records the orientation of the
turntable when the observation was made;
in this case it was oriented at 180$^\circ$.
In this way, applying the {\tt INT-WFC.ast} file to any set of
image NDF files will result in them sharing coordinates which
are related by a simple shift in X and Y coordinates,
since the nonlinear optical distortions,
the exact positioning of the CCDs on the turntable,
and the orientation of the turntable itself will have been
accounted for.

Once the new coordinate system has been attached to the images,
then most Starlink applications will make use of this as appropriate.
If you display any of the images using
\xref{KAPPA}{sun95}{}'s
\xref{DISPLAY}{sun95}{DISPLAY}
application, the axes will show the coordinates in the new frame
as in figure~\ref{fig:display}:
\begin{myquote}
\begin{verbatim}
% kappa

   KAPPA commands are now available -- (Version 0.13-7)

   Type kaphelp for help on KAPPA commands
   Type "showme sun95" to browse the hypertext documentation

% display r106282 style='"grid=1,gap=500"'
\end{verbatim}
\end{myquote}

\myfig{sun249_r106282-s8-display}{}{sun249_r106282-display.png}{fig:display}{Display
of a single CCD image with coordinates grafted onto it using {\tt ASTIMP}.
Close inspection will show that the grid is not quite parallel to the
sides of the image, or straight,
because of the slight rotation from the vertical and nonlinear distortion.}

If some of the frames overlap, the corresponding X and Y
offsets can then easily be determined
using CCDPACK's object matching registration facilities.


\section{Acknowledgements}

This calibration was done on Starlink time with assistance from
Peter Draper and David Gilbank in Durham.

% ? End of main text
\end{document}
