\documentstyle [11pt] {article} 
\pagestyle{myheadings}

%------------------------------------------------------------------------------
\newcommand{\stardoccategory}  {Starlink User Note}
\newcommand{\stardocinitials}  {SUN}
\newcommand{\stardocnumber}    {107.3}
\newcommand{\stardocauthors}   {A C Charles, M D Lawden}
\newcommand{\stardocdate}      {15 July 1993}
\newcommand{\stardoctitle}     {MAPLE --- Mathematical Manipulation Language}
%------------------------------------------------------------------------------

\newcommand{\stardocname}{\stardocinitials /\stardocnumber}
\renewcommand{\_}{{\tt\char'137}}     % re-centres the underscore
\markright{\stardocname}
\setlength{\textwidth}{160mm}
\setlength{\textheight}{230mm}
\setlength{\topmargin}{-2mm}
\setlength{\oddsidemargin}{0mm}
\setlength{\evensidemargin}{0mm}
\setlength{\parindent}{0mm}
\setlength{\parskip}{\medskipamount}
\setlength{\unitlength}{1mm}

\begin{document}
\thispagestyle{empty}
SCIENCE \& ENGINEERING RESEARCH COUNCIL \hfill \stardocname\\
RUTHERFORD APPLETON LABORATORY\\
{\large\bf Starlink Project\\}
{\large\bf \stardoccategory\ \stardocnumber}
\begin{flushright}
\stardocauthors\\
\stardocdate
\end{flushright}
\vspace{-4mm}
\rule{\textwidth}{0.5mm}
\vspace{5mm}
\begin{center}
{\Large\bf \stardoctitle}
\end{center}
\vspace{5mm}

\begin{quote}{\em 

MAPLE is a commercial software package purchased from WATCOM, Waterloo,
Ontario, CANADA. It was developed by the Symbolic Computation Group at
the University of Waterloo. It is licensed to run on the Starlink Central
Data \& Software Facility (STADAT) and is only installed on this machine.
To use it, you must first log onto STADAT from your own account on your
own Starlink machine.

}
\end{quote}

\section{Introduction}

MAPLE is an interactive system for symbolic algebra computation. It can
perform hundreds of algebraic functions for use at all mathematical
levels, and can provide solutions for many types of problems:

\begin{itemize}
\item Arithmetic with integers, fractions, and polynomials.
\item Power series.
\item Differentiation and integration of functions.
\item Systems of equations.
\item Differential equations.
\item Linear optimization.
\item Tensor manipulation.
\item Symbolic and numeric approximation.
\item Automatic generation of Fortran code and \LaTeX\ source for mathematical
expressions.
\end{itemize}

In addition, plots can be generated to illustrate graphically any
function, including user-defined functions.

You can extend or redefine the numerous functions by writing MAPLE
programs in the built-in Pascal-like language to create specialized
functions.

\section{Documentation}

There are two on-line help systems. The first can be accessed from VMS by
typing:

\begin{verbatim}
    $ HELP MAPLE
\end{verbatim}

while logged into STADAT. The second (more extensive) system can be
accessed during a MAPLE session by typing:

\begin{verbatim}
    > help();
\end{verbatim}

There is also `An Introduction to Maple' written by David Harper (QMW).
Copies can be obtained from your Site Manager.

Serious users of MAPLE may also want to look at the two main MAPLE publications
which are provided with the software. These are:

\begin{itemize}
\item MAPLE --- First Leaves: A tutorial introduction to Maple; 3rd Edition;
Bruce W.\ Char, et al; WATCOM.
\item MAPLE --- Reference Manual; 5th Edition; Bruce W.\ Char, et al; WATCOM.
\end{itemize}

Copies of these documents are held at RAL but most sites should have a
copy of `First Leaves'.  Fortunately, the on-line help systems should
enable you to use MAPLE effectively without needing to look at the
manuals.

\section{Getting Started}

To use MAPLE, you must log in to STADAT:

\begin{verbatim}
    $ SET HOST STADAT
\end{verbatim}

using the appropriate username and password for the users of your site.
Get into an appropriate directory. Then start up MAPLE by typing:

\begin{verbatim}
    $ MAPLESTART
    $ MAPLE
\end{verbatim}

The program will announce that you are running MAPLEV and present you with
its prompt symbol:

\begin{verbatim}
    >
\end{verbatim}

You can then enter MAPLE commands, read and store files, and define and run
MAPLE procedures.

If you are using a colour workstation, you can use the DECWindows
interface to MAPLE.  To start, type: 

\begin{verbatim} 
      $ XDISPLAY 
      $ MAPLESTART 
      $ XMAPLE 
\end{verbatim}

The two most important things for a beginner to remember are:

\begin{itemize}

\item You {\em must} end every command with a semi-colon (;), otherwise,
MAPLE will ignore your input. If you forget, enter a semi-colon in
response to the next `$>$' prompt and MAPLE will then obey your command.

\item You end your session by typing:

\begin{verbatim}
    > quit
\end{verbatim}

(you don't need a semi-colon here).

\end{itemize}

The first thing you may wish to do is to browse around the on-line help
system:

\begin{verbatim}
    > ?
\end{verbatim}

or

\begin{verbatim}
    > help();
\end{verbatim}

(don't forget that semi-colon, or you'll be complaining that MAPLE
doesn't work). Of particular interest is the introduction:

\begin{verbatim}
    > help(intro);
\end{verbatim}

The MAPLE manuals refer to version 4.2 of the software. You can find out
the new features of the installed version (5.1) by:

\begin{verbatim}
    > help(updates,v5);
\end{verbatim}

You can execute VMS commands from within MAPLE by prefixing them with a
`!' character. For example,

\begin{verbatim}
    > !copy maple$demo:taylor1.; *
\end{verbatim}

will copy one of the demos supplied with MAPLE into your default
directory.

\section{Examples}

The following example of a MAPLE session should give you some insight
into the capabilities of the software, and how to use it. You type the
lines after the `$>$' prompt and the computer replies with the next line.
The notes after the example explain some of the features you may find
mysterious at first reading.

\begin{verbatim}
    > 3+4+5;
                                   12
    > e1:=expand(x*(x+1)*(x-1));
                                   3
                            e1 := x  - x
    > factor(");
                            x (x + 1) (x - 1)
    > evalf(sqrt(2));
                            1.414213562
    > solve(e1=0,x);
                            0, 1, -1
    > int(x*exp(x),x);
                            x exp(x) - exp(x)
    > help(plot);
         ... gives information on the plot command
    > help();
         ... describes the help command
    > help(index,library);
         ... lists the functions in the Maple library
    > quit
         ... escapes from Maple
\end{verbatim}

The second command illustrates the assignment operator `{\tt :=}'. It
assigns the expression on the right of the operator to the variable ({\tt
e1}) on the left.

The {\tt (")} symbol refers to the previous expression in the MAPLE
session. Thus the third command \verb+factor(")+ is equivalent to
\verb+factor(e1)+. There are also symbols {\tt ("")} and {\tt (""")}
which refer to the second previous and third previous expressions.

In the fifth command, the symbol `{\tt e1}' refers to the expression
defined in the second command. The `{\tt int}' function integrates the
expression `{\tt e1}' with respect to {\tt x}.

Many examples of how to use MAPLE are given in the on-line help text.
Other examples are stored in files in the {\tt MAPLE\$DEMO} directory.

\section{Functions}

MAPLE has a very large number of functions available. You can find an
explanation of these functions and examples of their use by typing the
command:

\begin{verbatim}
    > help(<function>);
\end{verbatim}

during a MAPLE session. For example,

\begin{verbatim}
    > help(int);
\end{verbatim}

will describe the `{\tt int}' function which performs integration.

Among the most important and commonly used functions are the following:

\begin{description}
\begin{description}

\item [expand, simplify, normal] --- simplify expressions.
\item [evalf] --- evaluate an expression using floating point arithmetic.
\item [solve] --- solve equations.
\item [int, diff] --- integrate/differentiate expressions.
\item [series] --- expand functions to Taylor or Laurant series.
\item [plot] --- plot functions.

\end{description}
\end{description}

Use \verb+help(<function>)+ to find out about these --- there are lots of
examples. Other commonly used functions are:

\begin{description}
\begin{description}

\item [array] --- create an array.
\item [coeff] --- extract a coefficient of a polynomial.
\item [collect] --- collect terms in a specified indeterminate.
\item [convert] --- convert an expression to a different form.
\item [degree] --- determine the highest degree of a polynomial.
\item [evalc] --- evaluate in the complex number field.
\item [ifactor] --- integer factorisation.
\item [limit] --- calculate function limits.
\item [map] --- apply a procedure to each operand of an expression.
\item [op] --- extract operands from an expression.
\item [product] --- product of a sequence.
\item [radsimp] --- simplification of an expression containing radicals.
\item [subs] --- substitute sub-expressions into an expression.
\item [sum] --- definite and indefinite summation.
\item [table] --- create a table with initial values.
\item [type] --- type checking function.

\end{description}
\end{description}

These are just a sample of the 182 functions in the standard and
miscellaneous MAPLE function library. In addition, there are a number of
packages which specialise in particular areas of mathematics. These are:

\begin{description}
\begin{description}

\item [combinat] --- combinatorial functions.
\item [difforms] --- differential forms.
\item [geometry] --- Euclidean geometry.
\item [grobner] --- Grobner bases.
\item [group] --- permutation and finitely-presented groups.
\item [linalg] --- linear algebra.
\item [np] --- Newman-Penrose formalism.
\item [numtheory] --- number theory.
\item [orthopoly] --- orthogonal polynomials.
\item [powseries] --- formal power series.
\item [projgeom] --- projective geometry.
\item [simplex] --- linear optimization.
\item [stats] --- statistics.
\item [student] --- student calculus.

\end{description}
\end{description}

Information on these packages can be obtained from within MAPLE by
entering:

\begin{verbatim}
    > help(<package>);
\end{verbatim}

and information on the functions within a package can be obtained by:

\begin{verbatim}
    > help(<package>,<topic>);
\end{verbatim}

Thus, if you were learning calculus you could find help by entering:

\begin{verbatim}
    > help(student);
\end{verbatim}

and if you were particularly interested in finding out how to integrate
by parts, enter:

\begin{verbatim}
    > help(student,intparts);
\end{verbatim}

Sounds a great way of doing your Maths homework.

\end{document}
