\documentclass[11pt,twoside]{article}
\usepackage{graphicx}
\pagestyle{myheadings}

% -----------------------------------------------------------------------------
% ? Document identification
\newcommand{\stardoccategory}  {Starlink User Note}
\newcommand{\stardocinitials}  {SUN}
\newcommand{\stardocsource}    {sun\stardocnumber}
\newcommand{\stardocnumber}    {9.7}
\newcommand{\stardocauthors}   {Anne Charles}
\newcommand{\stardocdate}      {26 June 1997}
\newcommand{\stardoctitle}     {\LaTeX \\ A Document Preparation System}
\newcommand{\stardocversion}   {Version 2e}
\newcommand{\stardocmanual}    {User's Guide}
\newcommand{\stardocabstract}  {This is a guide to the \LaTeX\ 
                                facilities available on Starlink. It gives a 
                                brief description of how \LaTeX\ works. 
                                It also tells how to display or print the 
                                document, 
                                and how to include a PostScript file.}
% ? End of document identification

% -----------------------------------------------------------------------------

\newcommand{\stardocname}{\stardocinitials /\stardocnumber}
\markboth{\stardocname}{\stardocname}
\setlength{\textwidth}{160mm}
\setlength{\textheight}{230mm}
\setlength{\topmargin}{-2mm}
\setlength{\oddsidemargin}{0mm}
\setlength{\evensidemargin}{0mm}
\setlength{\parindent}{0mm}
\setlength{\parskip}{\medskipamount}
\setlength{\unitlength}{1mm}

% -----------------------------------------------------------------------------
%  Hypertext definitions.
%  ======================
%  These are used by the LaTeX2HTML translator in conjunction with star2html.

%  Comment.sty: version 2.0, 19 June 1992
%  Selectively in/exclude pieces of text.
%
%  Author
%    Victor Eijkhout                                      <eijkhout@cs.utk.edu>
%    Department of Computer Science
%    University Tennessee at Knoxville
%    104 Ayres Hall
%    Knoxville, TN 37996
%    USA

%  Do not remove the %\begin{rawtex} and %\end{rawtex} lines (used by 
%  star2html to signify raw TeX that latex2html cannot process).
%\begin{rawtex}
\makeatletter
\def\makeinnocent#1{\catcode`#1=12 }
\def\csarg#1#2{\expandafter#1\csname#2\endcsname}

\def\ThrowAwayComment#1{\begingroup
    \def\CurrentComment{#1}%
    \let\do\makeinnocent \dospecials
    \makeinnocent\^^L% and whatever other special cases
    \endlinechar`\^^M \catcode`\^^M=12 \xComment}
{\catcode`\^^M=12 \endlinechar=-1 %
 \gdef\xComment#1^^M{\def\test{#1}
      \csarg\ifx{PlainEnd\CurrentComment Test}\test
          \let\html@next\endgroup
      \else \csarg\ifx{LaLaEnd\CurrentComment Test}\test
            \edef\html@next{\endgroup\noexpand\end{\CurrentComment}}
      \else \let\html@next\xComment
      \fi \fi \html@next}
}
\makeatother

\def\includecomment
 #1{\expandafter\def\csname#1\endcsname{}%
    \expandafter\def\csname end#1\endcsname{}}
\def\excludecomment
 #1{\expandafter\def\csname#1\endcsname{\ThrowAwayComment{#1}}%
    {\escapechar=-1\relax
     \csarg\xdef{PlainEnd#1Test}{\string\\end#1}%
     \csarg\xdef{LaLaEnd#1Test}{\string\\end\string\{#1\string\}}%
    }}

%  Define environments that ignore their contents.
\excludecomment{comment}
\excludecomment{rawhtml}
\excludecomment{htmlonly}
%\end{rawtex}

%  Hypertext commands etc. This is a condensed version of the html.sty
%  file supplied with LaTeX2HTML by: Nikos Drakos <nikos@cbl.leeds.ac.uk> &
%  Jelle van Zeijl <jvzeijl@isou17.estec.esa.nl>. The LaTeX2HTML documentation
%  should be consulted about all commands (and the environments defined above)
%  except \xref and \xlabel which are Starlink specific.

\newcommand{\htmladdnormallinkfoot}[2]{#1\footnote{#2}}
\newcommand{\htmladdnormallink}[2]{#1}
\newcommand{\htmladdimg}[1]{}
\newenvironment{latexonly}{}{}
\newcommand{\hyperref}[4]{#2\ref{#4}#3}
\newcommand{\htmlref}[2]{#1}
\newcommand{\htmlimage}[1]{}
\newcommand{\htmladdtonavigation}[1]{}

%  Starlink cross-references and labels.
\newcommand{\xref}[3]{#1}
\newcommand{\xlabel}[1]{}

%  LaTeX2HTML symbol.
\newcommand{\latextohtml}{{\bf LaTeX}{2}{\tt{HTML}}}

%  Define command to re-centre underscore for Latex and leave as normal
%  for HTML (severe problems with \_ in tabbing environments and \_\_
%  generally otherwise).
\newcommand{\latex}[1]{#1}
\newcommand{\setunderscore}{\renewcommand{\_}{{\tt\symbol{95}}}}
\latex{\setunderscore}

%  Redefine the \tableofcontents command. This procrastination is necessary 
%  to stop the automatic creation of a second table of contents page
%  by latex2html.
\newcommand{\latexonlytoc}[0]{\tableofcontents}

% -----------------------------------------------------------------------------
%  Debugging.
%  =========
%  Remove % on the following to debug links in the HTML version using Latex.

% \newcommand{\hotlink}[2]{\fbox{\begin{tabular}[t]{@{}c@{}}#1\\\hline{\footnotesize #2}\end{tabular}}}
% \renewcommand{\htmladdnormallinkfoot}[2]{\hotlink{#1}{#2}}
% \renewcommand{\htmladdnormallink}[2]{\hotlink{#1}{#2}}
% \renewcommand{\hyperref}[4]{\hotlink{#1}{\S\ref{#4}}}
% \renewcommand{\htmlref}[2]{\hotlink{#1}{\S\ref{#2}}}
% \renewcommand{\xref}[3]{\hotlink{#1}{#2 -- #3}}
% -----------------------------------------------------------------------------
% ? Document specific \newcommand or \newenvironment commands.
% ? End of document specific commands
% -----------------------------------------------------------------------------
%  Title Page.
%  ===========
\renewcommand{\thepage}{\roman{page}}
\begin{document}
\thispagestyle{empty}

%  Latex document header.
%  ======================
\begin{latexonly}
   CCLRC / {\sc Rutherford Appleton Laboratory} \hfill {\bf \stardocname}\\
   {\large Particle Physics \& Astronomy Research Council}\\
   {\large Starlink Project\\}
   {\large \stardoccategory\ \stardocnumber}
   \begin{flushright}
   \stardocauthors\\
   \stardocdate
   \end{flushright}
   \vspace{-4mm}
   \rule{\textwidth}{0.5mm}
   \vspace{5mm}
   \begin{center}
   {\Huge\bf  \stardoctitle \\ [2.5ex]}
   {\LARGE\bf \stardocversion \\ [4ex]}
   {\Huge\bf  \stardocmanual}
   \end{center}
   \vspace{5mm}

% ? Heading for abstract if used.
   \vspace{10mm}
   \begin{center}
      {\Large\bf Abstract}
   \end{center}
% ? End of heading for abstract.
\end{latexonly}

%  HTML documentation header.
%  ==========================
\begin{htmlonly}
   \xlabel{}
   \begin{rawhtml} <H1> \end{rawhtml}
      \stardoctitle\\
      \stardocversion\\
      \stardocmanual
   \begin{rawhtml} </H1> \end{rawhtml}

% ? Add picture here if required.
% ? End of picture

   \begin{rawhtml} <P> <I> \end{rawhtml}
   \stardoccategory \stardocnumber \\
   \stardocauthors \\
   \stardocdate
   \begin{rawhtml} </I> </P> <H3> \end{rawhtml}
      \htmladdnormallink{CCLRC}{http://www.cclrc.ac.uk} /
      \htmladdnormallink{Rutherford Appleton Laboratory}
                        {http://www.cclrc.ac.uk/ral} \\
      \htmladdnormallink{Particle Physics \& Astronomy Research Council}
                        {http://www.pparc.ac.uk} \\
   \begin{rawhtml} </H3> <H2> \end{rawhtml}
      \htmladdnormallink{Starlink Project}{http://star-www.rl.ac.uk/}
   \begin{rawhtml} </H2> \end{rawhtml}
   \htmladdnormallink{\htmladdimg{source.gif} Retrieve hardcopy}
      {http://star-www.rl.ac.uk/cgi-bin/hcserver?\stardocsource}\\

%  HTML document table of contents. 
%  ================================
%  Add table of contents header and a navigation button to return to this 
%  point in the document (this should always go before the abstract \section). 
  \label{stardoccontents}
  \begin{rawhtml} 
    <HR>
    <H2>Contents</H2>
  \end{rawhtml}
  \renewcommand{\latexonlytoc}[0]{}
  \htmladdtonavigation{\htmlref{\htmladdimg{contents_motif.gif}}
        {stardoccontents}}

% ? New section for abstract if used.
  \section{\xlabel{abstract}Abstract}
% ? End of new section for abstract
\end{htmlonly}

% -----------------------------------------------------------------------------
% ? Document Abstract. (if used)
%  ==================
\stardocabstract
% ? End of document abstract
% -----------------------------------------------------------------------------
% ? Latex document Table of Contents (if used).
%  ===========================================
 \newpage
 \begin{latexonly}
   \setlength{\parskip}{0mm}
   \latexonlytoc
   \setlength{\parskip}{\medskipamount}
   \markboth{\stardocname}{\stardocname}
 \end{latexonly}
% ? End of Latex document table of contents
% -----------------------------------------------------------------------------
\cleardoublepage
\renewcommand{\thepage}{\arabic{page}}
\setcounter{page}{1}
\input{texnames.sty}
\bibliographystyle{plain}

\begin{htmlonly}
  \renewcommand{\BIBTeX}{BIBTeX}
  \renewcommand{\SLITeX}{SLITeX}
  \renewcommand{\LaTeXe}{LaTeX 2e}
\end{htmlonly}

\section{\xlabel{introduction}\label{introduction}Introduction}

\LaTeX\ is a system for typesetting documents.
The style of the printed page is 
controlled by using commands (\TeX\ macros) within the text of the document.
The \texttt{latex} program interprets the commands and produces a 
device-independent (dvi) file.
A separate program then prints or displays the page on a variety of devices. 

For further details about \TeX\ and \LaTeX, see the following:

\begin{itemize}
  \item \emph{\LaTeX\ A Document Preparation System --- 
        User's Guide and Reference Manual}, 
        by Leslie Lamport\cite{lamport}.
  \item \emph{The \LaTeX Companion}, 
        by Michel Gossens, Frank Mittelbach, Alexander Samarin\cite{latexcomp}.
  \item \emph{The \TeX\ book}, by Donald Knuth\cite{texbook}.
  \item \xref{SC/9: \emph{\LaTeX\ Cook-Book}}{sc9}{}, 
        by Mike Lawden\cite{sc9}.
  \item \xref{SUN/93: \emph{\TeX\ --- A Superior document preparation system}}
        {sun93}{},
        by Jo Murray\cite{sun93}.
\end{itemize}

All of the above should be available at your site.
For \LaTeX, the most useful are SC/9
and the books by Lamport and by Gossens, \emph{et al}.
Many \LaTeX-related documents can be accessed via the WWW page 
\emph{\TeX\ and \LaTeX\ on Starlink}:

\begin{quote}
      \htmladdnormallink{\texttt{http://star-www.rl.ac.uk/\~{}acc/tex/tex.html}}
      {http://star-www.rl.ac.uk/\~{}acc/tex/tex.html}
\end{quote}

If you intend to use \LaTeX\ to write Starlink documentation (such as SUNs), 
please read the following paper:
\begin{itemize}
  \item \xref{SGP/28: 
         \emph{How to write good documents for Starlink}}{sgp28}{},
         by Mike Lawden\cite{sgp28}.
  \item \xref{SGP/50: 
         \emph{Starlink document styles}}{sgp50}{},
         by Mike Lawden\cite{sgp50}.
  \item \xref{SUN/199: 
         \emph{Star2HTML, Converting Starlink documents to hypertext}}{sun199}{},
         by Peter Draper\cite{sun199}.
\end{itemize}

This document, SUN/9, has been updated for the December 1996 release of \LaTeXe.
It mainly deals with how to process \LaTeX\
source files, not with the \LaTeX\ commands.

Throughout this document, the following conventions are used:

\begin{itemize}
  \item \texttt{\%} is the prompt for UNIX.
  \item \texttt{csh}, or equivalent, is assumed for the UNIX shell.
  \item \texttt{<} and \texttt{>} surround names which should be replaced
        by values.
\end{itemize}

\section{\xlabel{how_latex_works}\label{how_latex_works}How \LaTeX\ Works}

The processing of \LaTeX\ source is in two stages. First the
typesetting program is used to convert a source file to a device-independent
(dvi) form. Then, a translator is used to convert the dvi form to a format that
allows a particular device to display or print the result.

The process may be demonstrated by following an example.

\subsection{\xlabel{example}Example}
\label{se:example}
A good example of \LaTeX\ source comes with the release.
Copy it to your current directory:
\begin{verbatim}
      % cp /usr/local/teTeX/texmf/tex/latex/base/sample2e.tex .
\end{verbatim}
Invoke \LaTeX\ to convert the source 
\texttt{.tex} file to the device independent \texttt{.dvi} file:
\begin{verbatim}
      % latex sample2e
\end{verbatim}
The file extension is assumed to be \texttt{.tex}.
This creates the file \texttt{sample2e.dvi}.

To process the \texttt{.dvi} file for printing in PostScript mode, 
use the \texttt{dvips} program:\footnote{Use \texttt{dvican sample2e} 
for a Canon laser printer in native mode.} 

\begin{verbatim}
      % dvips sample2e
\end{verbatim}

The file extension is assumed to be \texttt{.dvi}.
This creates the file \texttt{sample2e.ps}\footnote{\texttt{sample.dvi-can} 
if you used \texttt{dvican}}.

To print the file:\footnote{\texttt{lpr -Pn sample2e.dvi-can} for a Canon laser
printer file.}
\begin{verbatim}
      % lpr -Pn sample2e.ps
\end{verbatim}

\texttt{n} specifies the printer.

After the document has been processed with \texttt{latex}, 
there will be several additional files
in your directory with the name \texttt{sample2e.nnn}:
\begin{quote}
  \texttt{sample2e.log} --- A log (transcript) of the \LaTeX\ run\\
  \texttt{sample2e.aux} --- Cross reference information
\end{quote}

Files with other extensions may also be generated, depending upon what
instructions are contained within the source file. 
All of these additional files may be deleted. However, it is best to keep them
because future invocations of \texttt{latex} may use them.

Always run \texttt{latex} from the directory in which the \texttt{.tex} file 
resides. This enables \LaTeX\ most easily to find its auxiliary files.

\subsection{\xlabel{latex_errors}\LaTeX\ Errors}
\label{se:latexerrs}

When \LaTeX\ encounters an error it reports the error, displays the problem 
line, and prompts with a `\texttt{?}'.  
The simplest response is to either:

\begin{itemize}
  \item type a \texttt{$<$cr$>$} to cause \LaTeX\ to continue, 
        improvising what it considers the optimum solution to the problem, or 
  \item type \texttt{x} to exit from \LaTeX.
\end{itemize}

See the \LaTeX\ manual\cite[sections 2.3 and 2.4 and Chapter 8]{lamport}
for a discussion about how to handle errors.

If you want to stop \LaTeX\ in the middle of its execution, 
type \texttt{<Ctrl-C>}. This will make \LaTeX\ stop as if it
had encountered an ordinary error. Return to command level by typing \texttt{x}.

\subsection{\xlabel{spelling}Spelling}
\label{se:unixspell}
To use the spell utility for finding spelling errors in a \LaTeX\ input file:
\begin{verbatim}
      % ispell <filename>
\end{verbatim}

For details on how to use the \texttt{ispell} utility, type \texttt{man ispell}. 

\section{\xlabel{how_to_display_the_document}\label{how_to_display_the_document}How to Display the Document}

\LaTeX\ produces files which are device-independent (dvi). To get a
viewable or printable version, you must use a dvi previewer or translator 
program which can interpret the dvi code for the device you want to use:\footnote{
\texttt{dvi2vdu} is a previewer for Pericom, Graphon and some VT terminals;
                  there is help in the program and a 
    \htmladdnormallink{manual}
    {http://star-www.rl.ac.uk/\~{}acc/tex/doc/programs/dviman.dvi}
    \cite{dvican}.
\texttt{dvican} is a translator for Canon LBP-8II and LBP-8III Laser 
                 printers in native Canon mode; see the 
    \htmladdnormallink{manual}
    {http://star-www.rl.ac.uk/\~{}acc/tex/doc/programs/dvitovdu.dvi}
    \cite{dvi2vdu} 
    for more information.}

\begin{itemize}
\item \texttt{xdvi} --- previewer for Xwindows terminals and Xworkstations; 
                     type \texttt{man xdvi} for more information.
\item \texttt{dvips} --- translator for PostScript devices; type \texttt{man dvips}
                  or see the 
    \htmladdnormallink{dvips manual}
    {http://star-www.rl.ac.uk/\~{}acc/tex/doc/programs/dvipsk.dvi}\cite{dvips}
    for more information.
\end{itemize}

To save paper, if you have access to
an Xwindows terminal or an Xworkstation\footnote{or a Pericom or Graphon
graphics terminal}, you should use the dvi previewer. 

\subsection{\xlabel{xdvi}XDVI}
\label{se:dvixdvi}

\subsubsection{Setting the Display Device}

The \verb+xdvi+ translator will work on Xterminals and Xworkstations.
Usually, \verb+xdvi+ knows where to send its output, but sometimes it must 
be told by using \xref{\texttt{xdisplay}}{sun129}{}\cite{sun129}.  
This will interrogate your terminal and access method: 
\begin{verbatim}
      % xdisplay 
\end{verbatim}
or
\begin{verbatim}
      % xdisplay <workstation_name>
\end{verbatim}
You only need to do this once per session.

For \texttt{xdisplay} to work properly you may need to authorize the remote
workstation to open a window on the local workstation: 
\begin{verbatim}
      % xhost +<workstation_name>
\end{verbatim}
where \verb+<workstation_name>+ is the full name or valid alias or TCP/IP
address of the remote workstation. The following commands are equivalent:
\begin{verbatim}
      % xhost +axp1.ast.man.ac.uk
      % xhost +axp1
      % xhost +130.88.9.41
\end{verbatim}

\subsubsection{Command Syntax}
The command:
\begin{verbatim}
      % xdvi <filename>
\end{verbatim}
creates a window on the default X-windows display and displays the first 
page of the file \texttt{<filename>.dvi}.

It can show the file shrunk by various
(integer) factors, and also has a ``magnifying glass'' which allows one
to see a small part of the unshrunk image momentarily.

\texttt{xdvi} has various buttons which can be used to select the page 
being viewed and the magnification. There are also various 
keystrokes which can be typed to control the display, \emph{e.g.}: 
\begin{quote}
  \texttt{u} to move up the page \\
  \texttt{d} to move down the page \\
  \texttt{n} to go to the next page \\
  \texttt{p} to go to the previous page \\ 
  \texttt{q} to quit
\end{quote}

Type \texttt{man xdvi} for more information.

\subsubsection{Resource Names}

Default values for the command line options may be set via the resource names 
given in the descriptions of the options. To do this
create a file \texttt{.Xdefaults} in your top level directory.
Include in this file the resource names
and arguments of each of the options you wish to specify.  For example:
\begin{verbatim}
   XDvi.copy: off
   XDvi.thorough: on
   XDvi.shrinkFactor: 2
   XDvi.Margin: 0.95
   XDvi*geometry: 1015x750+3+25
\end{verbatim}
   When \verb+xdvi+ is invoked, it behaves as if it had been invoked with the
   command:
\begin{verbatim}
   % xdvi +copy -thorough -s 2 -margins 0.95 -geometry 1015x750+3+25 dvifile
\end{verbatim}
   Specifying options on the command line will override any options listed
   in the \texttt{.Xdefaults} file.

\subsection{\xlabel{dvips}DVIPS}

Here are a few useful things you can do with \texttt{dvips}:

$\bullet$ \textbf{Print at 600 dpi}

Most PostScript printers print at 300dpi.  However, 
if your printer is capable of printing at 600dpi you can invoke it by:
\begin{verbatim}
      % dvips -D 600 -Z <filename>
\end{verbatim}

$\bullet$ \textbf{Print selected pages}

To print the pages which are numbered 5 through 10:

\begin{verbatim}
      % dvips -p5 -l10 <filename>
\end{verbatim}

To print the fifth through tenth pages:

\begin{verbatim}
      % dvips -p=5 -l=10 <filename>
\end{verbatim}

$\bullet$ \textbf{Print in landscape mode}

Use the \texttt{landscape} class option in the \texttt{.tex} file 
to set the default margins correctly:

\begin{verbatim}
      \documentclass[landscape]{article}
\end{verbatim}

Then invoke \texttt{dvips} as follows:

\begin{verbatim}
      % dvips -t landscape <filename>
\end{verbatim}

For more information about \texttt{dvips} options, type \texttt{man dvips} or
see the 
    \htmladdnormallink{dvips manual}
    {http://star-www.rl.ac.uk/\~{}acc/tex/doc/programs/dvipsk.dvi}\cite[section 7, p. 26]{dvips}.

\section{\xlabel{more_about_latex}\label{more_about_latex}More about \LaTeX}

\subsection{\xlabel{document_classes}Document Classes: standard classes, 
proc, slides}
\label{se:dstyle}

$\bullet$ \\textbf{Standard Classes}

\LaTeX\ provides the following standard document classes:
\texttt{article, book, letter, report} and \texttt{slides}. 
The \texttt{article} class is for short reports and is the standard class
for Starlink documents.  
You may specify class options for size of type, one or two-column formatting,
\emph{etc}.
Templates for producing the document in standard Starlink styles, 
{\em e.g.} SUNs, SGPs, \emph{etc}. are stored in 
\verb+/star/docs+ as \verb+sun.tex+, \verb+sgp.tex+, \verb+ssn.tex+ and 
\verb+sg.tex+.

If you use \LaTeX's default margin settings in your non-Starlink documents,
you should specify \texttt{a4paper} as a document-class option:
\begin{verbatim}
      \documentclass[a4paper]{article}
\end{verbatim}
otherwise it will default to \texttt{letterpaper}, American quarto.  
There has been much discussion about this on USENET News.  
The conclusion seems to be that it has to default to 
something and no matter what is chosen a lot of people will be unhappy.

$\bullet$ \\textbf{The \texttt{proc} Document Class}

The \texttt{proc} document class produces two-column output for conference
proceedings.  The command \hbox{\verb|\copyrightspace|} makes the blank
space at the bottom of the first column of the first page, where the
proceedings editor will insert a copyright notice.  This command works
by producing a blank footnote, so it is placed in the text of the first
column.  It must go after any \hbox{\verb|\footnote|} command that
generates a footnote in that column.

\LaTeX\ automatically numbers the output pages.  It is a good idea 
to identify the paper on each page of output.  Placing the following command
in the preamble (before the \hbox{\verb|\begin{document}|} command)
prints ``Jones---Foo'' at the bottom of each page:  

\begin{verbatim}
      \markright{Jones---Foo}
\end{verbatim}

$\bullet$ \\textbf{Slides}

\SLITeX\ has been replaced by the \texttt{slides} document class, 
which is accessed by specifying:

\begin{verbatim}
      \documentclass{slides}
\end{verbatim}
 
See the \LaTeX\ manual\cite{lamport} for more information.
There are other ways of making 
\htmladdnormallink{slides}
{http://star-www.rl.ac.uk/\~{}acc/tex/packages.html#slides}.
\begin{latexonly} See the 
WWW page mentioned in Section~\ref{introduction} more information.
\end{latexonly}

\subsection{\xlabel{packages}Packages and Programs}

$\bullet$ \\textbf{Packages}

An extensive range of 
\htmladdnormallink{\LaTeX\ packages}
{http://star-www.rl.ac.uk/\~{}acc/tex/packages.html}
is included in this release. 
\begin{latexonly} See the 
WWW page mentioned in Section~\ref{introduction} more information.
\end{latexonly}

Specify each package you use as follows:

\begin{verbatim}
      \documentclass{article}
      \usepackage{<package1>,<package2>,...}
\end{verbatim}


$\bullet$ \\textbf{BIBTeX}

\BIBTeX\ produces a bibliography from databases of sources. 
See the 
\htmladdnormallink{\BIBTeX\ manual}
{http://star-www.rl.ac.uk/\~{}acc/tex/doc/bibtex/base/btxdoc.dvi}\cite{bibtex} 
and the \LaTeX\ manual\cite[Section 4.3.1 and Appendix B]{lamport}  
for more information.


$\bullet$ \\textbf{MakeIndex}

MakeIndex is a tool for making an index for a document.
See the 
\htmladdnormallink{MakeIndex manual}
{http://star-www.rl.ac.uk/\~{}acc/tex/doc/makeindex/makeindex.dvi}\cite{makeindex}  
for more information.

\subsection{\xlabel{fonts}Fonts}
\label{se:fonts}
There are type sizes not
obtainable by \LaTeX's size-changing commands with the ordinary
document class options.  If you try to use a font 
that is not available, \LaTeX\ will issue a warning and substitute a font of 
the same size that is as close as possible in style to the one you wanted.

Table~\ref{tab:styles} tells you what size of type is used for each
\LaTeX\ type-size command in the various document-class options.  For
example, with the \texttt{12pt} option, the \hbox{\verb|\large|}
declaration causes \LaTeX\ to use 14-pt type.  

\begin{table}[htb]
\centering
\begin{tabular}{l|r|r|r|}
\multicolumn{1}{l}{size} & 
\multicolumn{1}{c}{default (10pt)} &
        \multicolumn{1}{c}{11pt option}  &
        \multicolumn{1}{c}{12pt option}\\
\cline{2-4}
\verb|\tiny|       & 5pt  & 6pt & 6pt\\
\cline{2-4}
\verb|\scriptsize| & 7pt  & 8pt & 8pt\\
\cline{2-4}
\verb|\footnotesize| & 8pt & 9pt & 10pt\\
\cline{2-4}
\verb|\small|      & 9pt  & 10pt & 11pt\\
\cline{2-4}
\verb|\normalsize| & 10pt & 11pt & 12pt \\
\cline{2-4}
\verb|\large|      & 12pt & 12pt & 14pt \\
\cline{2-4}
\verb|\Large|      & 14pt & 14pt & 17pt \\
\cline{2-4}
\verb|\LARGE|      & 17pt & 17pt & 20pt\\
\cline{2-4}
\verb|\huge|       & 20pt & 20pt & 25pt\\
\cline{2-4}
\verb|\Huge|       & 25pt & 25pt & 25pt\\
\cline{2-4}
\end{tabular}
\caption{Type sizes for \LaTeX\ size-changing commands.}\label{tab:styles}
\end{table}


\section{\xlabel{including_ps_in_latex}\label{including_ps_in_latex}
How to Include a PostScript File in a \LaTeX\ Document}

\subsection{\xlabel{creating_the_graphics_files}Creating the Graphics Files}

When running the application program which creates your graphics, 
specify the appropriate GKS workstation:\footnote{\texttt{canon\_ptex} and 
\texttt{canon\_ltex} for Canon portrait and landscape.}

\begin{quote}
  \texttt{epsf\_p} --- Encapsulated PostScript, portrait\\
  \texttt{epsf\_l} --- Encapsulated PostScript, landscape
\end{quote}

Note that the output from these workstations cannot be plotted directly on 
the printer. Each new frame generates a separate file.

Hint: The name of the output file will default to \texttt{gks74.ps}. 
However, you can enter it at the time of specifying the GKS workstation:

\begin{verbatim}
       epsf_p;graph.eps
\end{verbatim}

The output will then be in files \texttt{graph.eps}, 
\texttt{graph.eps.1}, \texttt{graph.eps.2}, \emph{etc}.

The \LaTeX\ graphics package contains options which allow you to 
adjust the size of the figure. 
However, you may change it within GKS by setting the workstation viewport to 
the required size, bearing in mind that:

\begin{itemize} 
  \item GKS sets the graphics origin to the top left corner of the workstation 
        viewport.
  \item The position of the viewport on the display surface is irrelevant.
  \item With some high level graphics packages, setting the workstation 
        viewport may interfere with the working of the package. 
\end{itemize}

You may change the size independently of GKS in any of the following ways:

\begin{quote}
  \begin{description}
    \item[SGS ---] Use \texttt{SGS\_ZSIZE} with a position specification of \texttt{TL}.
               See \xref{SUN/85}{sun85}{}\cite{sun85}.
    \item[PGPLOT ---] Use \texttt{PGPAPER}. See \xref{SUN/15}{sun15}{}\cite{sun15}.
  \end{description}
\end{quote}

\subsection{\xlabel{preparing_the_postscript_file}\label{preparing_the_postscript_file}Preparing the PostScript File}

If you have an Encapsulated PostScript (EPS) file (\texttt{.eps}), 
go on to section \ref{IncludePSFile}.
 
    Verify that your file has a \texttt{BoundingBox} line in 
    the header or the footer. 
    This line tells \LaTeX\ the size of the image to be included. 
    If it is in the footer, move it to the header. 

    If you do not have a \texttt{BoundingBox} but 
    do have a simple one-page PostScript file,  
    you can convert it to EPS by using 
    the \texttt{ghostscript} program \texttt{ps2epsi}. 
    For more information, see the 
    \htmladdnormallink{EPS manual}
    {http://star-www.rl.ac.uk/\~{}acc/tex/doc/latex/graphics/epslatex.ps}\cite[section 3.1, p 7]{eps}.
    If you are unable to use \texttt{ps2epsi}, you must determine what the 
    \texttt{BoundingBox}  should be and add it by hand. For information see the 
    \htmladdnormallink{EPS manual}
    {http://star-www.rl.ac.uk/\~{}acc/tex/doc/latex/graphics/epslatex.ps}\cite[section 3.1, \#2, p 8]{eps}.

\subsection{\xlabel{including_the_postscript_file}\label{includepsfile}Including the PostScript File\xlabel{IncludePSFile}
\label{IncludePSFile}}

The recommended method for including PostScript files in \LaTeX\ 
documents is the \LaTeXe\ \texttt{graphicx} package, which 
is enabled by the \verb+\usepackage+ command.
Use the \verb+\includegraphics+ command to include the PostScript file. 
For more information about the \texttt{graphicx} package, see the 
\htmladdnormallink{graphics manual}
{http://star-www.rl.ac.uk/\~{}acc/tex/doc/latex/graphics/grfguide.ps}\cite[section 4, pp 6-14]{graphics}, 
and the 
\htmladdnormallink{EPS manual}
{http://star-www.rl.ac.uk/\~{}acc/tex/doc/latex/graphics/epslatex.ps}\cite[part I, section 1,  p 5]{eps}.

Allow the image to float to the next page if it does not fit on the 
current page by placing the \verb+\includegraphics+ command inside a 
\texttt{figure} environment.
For more information about the \texttt{figure} environment, see the \LaTeX\ 
manual\cite[section 3.5.1]{lamport} and the 
\htmladdnormallink{EPS manual}
{http://star-www.rl.ac.uk/\~{}acc/tex/doc/latex/graphics/epslatex.ps}\cite[part IV, section 10, p 36]{eps}.

Centre the image by preceding the \verb+\includegraphics+ command with 
\verb+\centering+.

    \begin{verbatim}
      \documentclass{article}
      \usepackage{graphicx}
      ...
      \begin{figure}
        \centering
        \includegraphics{<filename>}
        \caption{<text>}
      \end{figure}
    \end{verbatim}

Specifying \texttt{draft} as an option to the \texttt{graphicx} package,
prints the filename in a box of the correct size. This can save time 
when writing the document:  

    \begin{verbatim}
      \documentclass{article}
      \usepackage[draft]{graphicx}
    \end{verbatim}

You can specify a scale factor, angle of rotation, image width, and/or 
image height within the \verb+\includegraphics+ command, for example: 
    \begin{verbatim}
      \includegraphics[scale=0.5]{<filename>}
      \includegraphics[angle=45]{<filename>}
      \includegraphics[width=2in]{<filename>}
      \includegraphics[totalheight=4in]{<filename>}
      \includegraphics[scale=0.5,totalheight=4in]{<filename>}
    \end{verbatim}
For a more detailed discussion, see the 
\htmladdnormallink{EPS manual}
{http://star-www.rl.ac.uk/\~{}acc/tex/doc/latex/graphics/epslatex.ps}\cite[part III, section 6, pp 15-19]{eps}.

\subsection{Manipulating the Image: \\
Placement, Landscape, Side-by-side, Text Around Image\xlabel{ManipulatingImage}}

$\bullet$ \textbf{Placement}

    Choose where \LaTeX\ should place the figure by specifying an 
    option to the \texttt{figure} environment:

    \begin{verbatim}
      \begin{figure}[<option>]
        \centering
        \includegraphics{<filename>}
        \caption{<text>}
      \end{figure}
    \end{verbatim}

    where the \texttt{figure} \emph{option} can be any combination of the 
    following:
    \begin{quote}
      \texttt{h} --- here (where the \texttt{figure} command is located)\\
      \texttt{t} --- at the top of the page\\
      \texttt{b} --- at the bottom of the page\\
      \texttt{p} --- on a separate page which contains only floats
    \end{quote}
   
    If you specify more than one of these options, \LaTeX\ will use the order 
    \texttt{h-t-b-p} regardless of the order you specify.
    Precede the specification with a \texttt{!}, \emph{e.g.}, \texttt{[!ht]},  
    to make \LaTeX\ try really hard to place the figure where you have 
    specified.  See the 
    \htmladdnormallink{EPS manual}
    {http://star-www.rl.ac.uk/\~{}acc/tex/doc/latex/graphics/epslatex.ps}\cite[part IV, section 10, p 36]{eps}.

$\bullet$ \textbf{Landscape}

    Use the \texttt{landscape} environment from the \texttt{lscape} package
    to print an image in the landscape orientation, while 
    keeping the page heading in the portrait orientation: 
    
    \begin{verbatim}
      \documentclass{article}
      \usepackage{lscape,graphicx}
      ...
      \begin{landscape}
        \begin{figure}
          \centering
          \includegrapics{<filename>}
          \caption{<text>}
        \end{figure}
      \end{landscape}
    \end{verbatim}

    For more information about the \texttt{landscape} environment and other 
    ways of producing landscape orientation, see the 
    \htmladdnormallink{EPS manual}
    {http://star-www.rl.ac.uk/\~{}acc/tex/doc/latex/graphics/epslatex.ps}\cite[part IV, section 11, p 38-41]{eps}.

    Note: I have found that the caption may not be rotated when viewed with 
    \texttt{xdvi}; however, it is rotated correctly by \texttt{dvips}. The 
    resulting \texttt{.ps} file can be viewed with \texttt{ghostview}.

$\bullet$ \textbf{Side-by-side}

    There are three common methods of organizing side-by-side graphics: 

    \begin{enumerate}
      \item Combined into a single figure
      \item Each in its own figure
      \item Each in a subfigure which form a single figure
    \end{enumerate}

    These are all discussed in detail in the 
    \htmladdnormallink{EPS manual}{http://star-www.rl.ac.uk/\~{}acc/tex/doc/latex/graphics/epslatex.ps}\cite[part IV, section 12, pp 41-48]{eps}. 
    The \texttt{subfigure} method is also documented in the 
    \htmladdnormallink{subfigure manual}{http://star-www.rl.ac.uk/\~{}acc/tex/doc/latex/styles/subfigure.dvi}\cite{subfigure}.

$\bullet$ \textbf{Text Around Image: floatflt}

  Use the \texttt{floatflt} package to get text to wrap around a small figure: 

  \begin{verbatim}
    \documentclass{article}
    \usepackage{floatflt,graphicx}
    ...
    \begin{floatingfigure}[<option>]{<width>}
      \centering
      \includegraphics{<filename>}
      \caption{<text>}
    \end{floatingfigure}
  \end{verbatim}

where \emph{option} indicates the placement of the figure on the page and 
can be any combination of the following: 
\begin{quote}
  \texttt{r} --- right\\
  \texttt{l} --- left\\
  \texttt{p} --- right for odd-numbered pages, left for even-numbered pages 
\end{quote}

\emph{width} is the width of the figure in the final document.

For example:

  \begin{verbatim}
    \documentclass{article}
    \usepackage{floatflt,graphicx}
    ...
    \begin{floatingfigure}[l]{5cm}
      \centering
      \includegraphics[scale=0.5]{graph.ps}
      \caption{An Interesting Graph}
    \end{floatingfigure}
  \end{verbatim}


  Detailed information, including a section about `Known problems and 
limitations', can be found in the 
  \htmladdnormallink{\texttt{floatflt} manual}{http://star-www.rl.ac.uk/\~{}acc/tex/doc/latex/floatflt/floatflt.dvi}\cite{floatflt}.

  Note: If the size of the image is larger than the size allowed by 
{\em width}, no figure will be created and no warning message will be 
given. To get the figure to appear, increase {\em width} and/or  
decrease \texttt{scale}. 


\subsection{\xlabel{converting_another_image_format_to_postscript}Converting Another Image Format to PostScript}

The following are tools for converting from another image format to 
PostScript:

\begin{enumerate}
  \item \texttt{xv} --- This is an interactive program for both displaying 
    and converting the format of images. \texttt{xv} is part of the base set 
    of software distributed by Starlink.\\
    To convert to PostScript format, do the following:
    \begin{quote}
      type \texttt{xv}\\
      click the right mouse button on the xv window to call-up the 
        `xv controls' window\\
      click `Load' on the `xv controls' window\\
      select the file and click on it twice (or on it once and then on `OK')\\
      click `Save'\\
      hold down the left mouse button after `Format' on the `xv save' window 
        and move the highlighted line to `PostScript' and release the mouse 
        button\\
      click `OK' on the `xv save' window\\
      click `OK' on the `xv PostScript' window
    \end{quote} 
    See the
    \htmladdnormallink{xv manual}
    {http://star-www.rl.ac.uk/\~{}acc/tex/xvdocs.ps}\cite[section 8 `The Load Window', section 9 `The Save Window', section 10 
    `The PostScript Window']{xv}.
  \item \htmladdnormallink{ImageMagick}
    {http://www.wizards.dupont.com/cristy/ImageMagick.html}
    --- This performs similar functions to \texttt{xv} but can be run from a 
    command line. ImageMagick is public domain software but is not currently 
    distributed by Starlink. 
    \begin{latexonly}
    You can find out more about it at:
    \begin{verbatim}
      http://www.wizards.dupont.com/cristy/ImageMagick.html
    \end{verbatim}
    \end{latexonly}
\end{enumerate}


\section{\xlabel{conversion}\label{conversion}Conversion between \LaTeX\ and Other Formats}

There are two sources of information about conversion between \TeX/\LaTeX\ and 
other formats:

\begin{itemize}
  \item \htmladdnormallink{\TeX/\LaTeX\ FAQ}
    {http://www.cogs.susx.ac.uk/cgi-bin/texfaq2html?keyword=font&question=48}
    \begin{latexonly}
      \begin{verbatim}
http://www.cogs.susx.ac.uk/cgi-bin/texfaq2html?keyword=font\&question=48
      \end{verbatim}
    \end{latexonly}
  \item \htmladdnormallink{Converters between \LaTeX\ and PC Textprocessors}
    {http://www.kfa-juelich.de/isr/1/texconv.html}
    \begin{latexonly}
      \begin{verbatim}
http://www.kfa-juelich.de/isr/1/texconv.html
      \end{verbatim}
    \end{latexonly}
\end{itemize}


\section{\xlabel{acknowledgements}\label{acknowledgements}Acknowledgements}

Thanks are due to Dave~Terrett, Alan~Lotts, Mike~Lawden, Martin~Bly,  
and Malcolm~Curry who wrote sections of this document. 

\section{\xlabel{trademarks}\label{trademarks}Trademarks}
\begin{itemize}
\item \TeX\ is a trademark of the American Mathematical Society,
\item PostScript is a trademark of Adobe Systems, Inc.,
\item UNIX is a trademark of Bell Laboratories,
\item SUN is a trademark of Sun Microsystems Ltd,
\item VAX/VMS is a trademark of Digital Equipment Corporation.
\end{itemize}

\newpage
\addcontentsline{toc}{section}{References}~\bibliography{sun9}

% recipe for generating master copy.
\typeout{}
\typeout{This document should be processed as folows:}
\typeout{}
\typeout{   > latex sun9}
\typeout{   > bibtex sun9}
\typeout{   > latex sun9}
\typeout{   > latex sun9}
\typeout{}
\typeout{in order to resolve all references.}
\typeout{}
\end{document}
