\documentclass[twoside,11pt]{article}
\pagestyle{myheadings}

%------------------------------------------------------------------------------
\newcommand{\stardoccategory}  {Starlink General Paper}
\newcommand{\stardocinitials}  {SGP}
\newcommand{\stardocnumber}    {1.1}
\newcommand{\stardocauthors}   {D J Rawlinson}
\newcommand{\stardocdate}      {8 September 1989}
\newcommand{\stardoctitle}     {Procedure for reporting bugs}
%------------------------------------------------------------------------------

\newcommand{\stardocname}{\stardocinitials /\stardocnumber}
\markright{\stardocname}
\setlength{\textwidth}{160mm}
\setlength{\textheight}{240mm}
\setlength{\topmargin}{-5mm}
\setlength{\oddsidemargin}{0mm}
\setlength{\evensidemargin}{0mm}
\setlength{\parindent}{0mm}
\setlength{\parskip}{\medskipamount}
\setlength{\unitlength}{1mm}

% -----------------------------------------------------------------------------
%  Hypertext definitions.
%  ======================
%  These are used by the LaTeX2HTML translator in conjunction with star2html.

%  Comment.sty: version 2.0, 19 June 1992
%  Selectively in/exclude pieces of text.
%
%  Author
%    Victor Eijkhout                                      <eijkhout@cs.utk.edu>
%    Department of Computer Science
%    University Tennessee at Knoxville
%    104 Ayres Hall
%    Knoxville, TN 37996
%    USA

%  Do not remove the %begin{latexonly} and %end{latexonly} lines (used by 
%  star2html to signify raw TeX that latex2html cannot process).
%begin{latexonly}
\makeatletter
\def\makeinnocent#1{\catcode`#1=12 }
\def\csarg#1#2{\expandafter#1\csname#2\endcsname}

\def\ThrowAwayComment#1{\begingroup
    \def\CurrentComment{#1}%
    \let\do\makeinnocent \dospecials
    \makeinnocent\^^L% and whatever other special cases
    \endlinechar`\^^M \catcode`\^^M=12 \xComment}
{\catcode`\^^M=12 \endlinechar=-1 %
 \gdef\xComment#1^^M{\def\test{#1}
      \csarg\ifx{PlainEnd\CurrentComment Test}\test
          \let\html@next\endgroup
      \else \csarg\ifx{LaLaEnd\CurrentComment Test}\test
            \edef\html@next{\endgroup\noexpand\end{\CurrentComment}}
      \else \let\html@next\xComment
      \fi \fi \html@next}
}
\makeatother

\def\includecomment
 #1{\expandafter\def\csname#1\endcsname{}%
    \expandafter\def\csname end#1\endcsname{}}
\def\excludecomment
 #1{\expandafter\def\csname#1\endcsname{\ThrowAwayComment{#1}}%
    {\escapechar=-1\relax
     \csarg\xdef{PlainEnd#1Test}{\string\\end#1}%
     \csarg\xdef{LaLaEnd#1Test}{\string\\end\string\{#1\string\}}%
    }}

%  Define environments that ignore their contents.
\excludecomment{comment}
\excludecomment{rawhtml}
\excludecomment{htmlonly}

%  Hypertext commands etc. This is a condensed version of the html.sty
%  file supplied with LaTeX2HTML by: Nikos Drakos <nikos@cbl.leeds.ac.uk> &
%  Jelle van Zeijl <jvzeijl@isou17.estec.esa.nl>. The LaTeX2HTML documentation
%  should be consulted about all commands (and the environments defined above)
%  except \xref and \xlabel which are Starlink specific.

\newcommand{\htmladdnormallinkfoot}[2]{#1\footnote{#2}}
\newcommand{\htmladdnormallink}[2]{#1}
\newcommand{\htmladdimg}[1]{}
\newenvironment{latexonly}{}{}
\newcommand{\hyperref}[4]{#2\ref{#4}#3}
\newcommand{\htmlref}[2]{#1}
\newcommand{\htmlimage}[1]{}
\newcommand{\htmladdtonavigation}[1]{}

% Define commands for HTML-only or LaTeX-only text.
\newcommand{\html}[1]{}
\newcommand{\latex}[1]{#1}

% Use latex2html 98.2.
\newcommand{\latexhtml}[2]{#1}

%  Starlink cross-references and labels.
\newcommand{\xref}[3]{#1}
\newcommand{\xlabel}[1]{}

%  LaTeX2HTML symbol.
\newcommand{\latextohtml}{{\bf LaTeX}{2}{\tt{HTML}}}

%  Define command to re-centre underscore for Latex and leave as normal
%  for HTML (severe problems with \_ in tabbing environments and \_\_
%  generally otherwise).
\newcommand{\setunderscore}{\renewcommand{\_}{{\tt\symbol{95}}}}
\latex{\setunderscore}

% -----------------------------------------------------------------------------
%  Debugging.
%  =========
%  Remove % from the following to debug links in the HTML version using Latex.

% \newcommand{\hotlink}[2]{\fbox{\begin{tabular}[t]{@{}c@{}}#1\\\hline{\footnotesize #2}\end{tabular}}}
% \renewcommand{\htmladdnormallinkfoot}[2]{\hotlink{#1}{#2}}
% \renewcommand{\htmladdnormallink}[2]{\hotlink{#1}{#2}}
% \renewcommand{\hyperref}[4]{\hotlink{#1}{\S\ref{#4}}}
% \renewcommand{\htmlref}[2]{\hotlink{#1}{\S\ref{#2}}}
% \renewcommand{\xref}[3]{\hotlink{#1}{#2 -- #3}}
%end{latexonly}

% -----------------------------------------------------------------------------
%------------------------------------------------------------------------------
% Add any \newcommand or \newenvironment commands here
%------------------------------------------------------------------------------

%  Title Page.
%  ===========
\renewcommand{\thepage}{\roman{page}}
\begin{document}
\thispagestyle{empty}

%  Latex document header.
%  ======================
\begin{latexonly}
   CCLRC / {\sc Rutherford Appleton Laboratory} \hfill {\bf \stardocname}\\
   {\large Science \& Engineering Research Council}\\
   {\large Starlink Project\\}
   {\large \stardoccategory\ \stardocnumber}
   \begin{flushright}
   \stardocauthors\\
   \stardocdate
   \end{flushright}
   \vspace{-4mm}
   \rule{\textwidth}{0.5mm}
   \vspace{5mm}
   \begin{center}
   {\Large\bf \stardoctitle}
   \end{center}
   \vspace{5mm}

% ? Heading for abstract if used.
%   \vspace{10mm}
%   \begin{center}
%      {\Large\bf Abstract}
%   \end{center}
% ? End of heading for abstract.
\end{latexonly}

%  HTML documentation header.
%  ==========================
\begin{htmlonly}
   \xlabel{}
   \begin{rawhtml} <H1> \end{rawhtml}
      \stardoctitle
   \begin{rawhtml} </H1> \end{rawhtml}

% ? Add picture here if required.
% ? End of picture

   \begin{rawhtml} <P> <I> \end{rawhtml}
   \stardoccategory\ \stardocnumber \\
   \stardocauthors \\
   \stardocdate
   \begin{rawhtml} </I> </P> <H3> \end{rawhtml}
      \htmladdnormallink{CCLRC}{http://www.cclrc.ac.uk} /
      \htmladdnormallink{Rutherford Appleton Laboratory}
                        {http://www.cclrc.ac.uk/ral} \\
      \htmladdnormallink{Science \& Engineering Research Council}
                        {http://www.pparc.ac.uk} \\
   \begin{rawhtml} </H3> <H2> \end{rawhtml}
      \htmladdnormallink{Starlink Project}{http://www.starlink.ac.uk/}
   \begin{rawhtml} </H2> \end{rawhtml}
   \htmladdnormallink{\htmladdimg{source.gif} Retrieve hardcopy}
      {http://www.starlink.ac.uk/cgi-bin/hcserver?\stardocsource}\\

%  HTML document table of contents. 
%  ================================
%  Add table of contents header and a navigation button to return to this 
%  point in the document (this should always go before the abstract \section). 
  \label{stardoccontents}
  \begin{rawhtml} 
    <HR>
    <H2>Contents</H2>
  \end{rawhtml}
  \htmladdtonavigation{\htmlref{\htmladdimg{contents_motif.gif}}
        {stardoccontents}}

% ? New section for abstract if used.
%  \section{\xlabel{abstract}Abstract}
% ? End of new section for abstract

\end{htmlonly}

% -----------------------------------------------------------------------------
% ? Document Abstract. (if used)
%  ==================
%\stardocabstract
% ? End of document abstract
% -----------------------------------------------------------------------------
% ? Latex document Table of Contents (if used).
%  ===========================================
% \newpage
%\begin{latexonly}
%   \setlength{\parskip}{0mm}
%   \tableofcontents
%   \setlength{\parskip}{\medskipamount}
%   \markright{\stardocname}
%\end{latexonly}
% ? End of Latex document table of contents
% -----------------------------------------------------------------------------

\section{Introduction}
The present procedure for reporting bugs in the Starlink Software Collection is
thought by many to be unsatisfactory. The purpose of this document is to outline
the procedures to be followed by the person discovering the bug, the Starlink 
Software Librarian and the person to whom the bug is reported. In the past,
bug reports have seemed to disappear down a black hole, with a fix (maybe)
appearing, without warning, sometime later. This was generally due to a lack of 
communication on all parts and it is hoped that the procedures outlined below
will keep all users informed of the progress of a bug through the system.

\section{Reporting a bug}
When a user discovers what s/he thinks is a bug, a mail message should be
sent to username {\bf STAR} at {\bf RLVAD}. The message should contain as much
relevant detail as possible. A message saying {\it ` FIGARO doesn't work.'}
isn't much use to anyone! STAR \footnote[1]{If STAR is unavailable, an
acknowledgement will be sent by OPER who will carry out the bug reporting
process.} will then acknowledge the user's note within $1/2$ working day.

\section{Tracing a bug's progress -- VAX Notes}
Having received the bug report, STAR will then forward all the details to the
person(s) responsible for supporting the software (as given in the Starlink
Software Index). A VAX NOTES conference has been set up to allow the
progress of a bug to be followed. Each note will be a particular 
bug report
with the name of the person dealing with the problem. The replies will consist
of information from the person supporting the software as to possible fixes.

\subsection{Classification of bugs}
{\em Question:} When is a bug not a bug? {\em Answer:} When it is a feature!
The VAX Notes conference will only allow contributions from the person
supporting the software and STAR. Each note will be assigned 4 keywords to
allow users to search the conference for a particular class of bugs.
\newpage
The four keywords will identify the bug by the following categories:
\begin{enumerate}
\item The name of the software item,
\item Whether the item is {\bf SUPPORTED} or {\bf UNSUPPORTED},
\item The  status of the bug, 
\begin{enumerate}
\item {\bf ACTIVE} -- under investigation.
\item {\bf FEATURE} -- may or may not be fixed in a forthcoming release.
\item {\bf FIXED} -- fixed, but awaiting release.
\item {\bf SUSPEND} -- non-reproduceable bug.
\item {\bf DEAD} -- released.
\end{enumerate}
\item The severity of the bug,
\begin{enumerate}
\item {\bf SEVERE} -- bug causes severe disruption to the software.
\item {\bf MINOR} -- bug is inconvenient to the user.
\item {\bf PATCH} -- minor bug with workround.
\end{enumerate}
\end{enumerate}
\subsubsection{Supported and unsupported software}
If the software item is supported, then a reply will be entered in the
conference, by that person, containing more information about the bug and the
actions being taken (if any) and the keyword {\bf SUPPORTED} assigned to the
entry. However, if the software is unsupported, it will
be assigned the keyword {\bf UNSUPPORTED}. Users are encouraged to send any
known workrounds or fixes to STAR who will put these suggestions in the
conference. The conference has restricted write access so that it does not
become a `free for all' and some order can be maintained.
\subsubsection{Bug status}
An {\bf ACTIVE} bug is currently being investigated by the support programmer
and its progress can be followed in the conference. A {\bf FEATURE}, is a `bug'
which has a known cause/effect and possibly has a workround. 
It may be that it will be
fixed in a forthcoming release or that it is a true feature of the software and
may not be fixed. A {\bf FIXED} bug is where a fix or patch has been supplied 
and is awaiting release in a Starlink Software Change. It may be that the bug 
is not reproduceable by the support programmer (over a period of time), the 
bug will go into a state of {\em suspended animation} and assigned the status 
of {\bf SUSPEND}. The keyword {\bf DEAD} will be assigned to a bug that has
been fixed and released. Periodically all DEAD bugs will be extracted from the
conference and archived.
\subsubsection{Severity of bugs}
This class of keywords is designed to inform the user as to whether the bug
will cause {\bf SEVERE} disruption or just a {\bf MINOR} inconvenience. The
keyword {\bf PATCH} indicates a minor bug with a possible workround 
until the bug is fixed properly.
\subsection{Using the VAX Notes KEYWORD facility}
As described above, each note will be assigned 4 keywords. To search for all
bugs on a specific software item, you would type:
\begin{verbatim}
        Notes> DIRECTORY/KEYWORD=<item>  
\end{verbatim}
Where {\em item} is ADAM, FIGARO  etc.
and similarly for the keywords in the other categories.
\newpage
\section{Procedure to be followed by support programmer}
Once the support programmer receives a bug report from STAR, s/he should put a
reply in the topic associated with that bug containing any information relating
to the bug. As progress is made in fixing the bug, replies should be entered
so that users can follow its progress. 
\end{document}
