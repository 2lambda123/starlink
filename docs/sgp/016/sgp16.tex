\documentclass[twoside,11pt]{article}
\pagestyle{myheadings}

%------------------------------------------------------------------------------
\newcommand{\stardoccategory}  {Starlink General Paper}
\newcommand{\stardocinitials}  {SGP}
\newcommand{\stardocnumber}    {16.10}
\newcommand{\stardocsource}    {sgp\stardocnumber}
\newcommand{\stardocauthors}   {P.~T.~Wallace}
\newcommand{\stardocdate}      {23rd~March~1992}
\newcommand{\stardoctitle}     {Starlink Application Programming Standard}
%------------------------------------------------------------------------------

\newcommand{\stardocname}{\stardocinitials /\stardocnumber}
\markright{\stardocname}
\setlength{\textwidth}{160mm}
\setlength{\textheight}{240mm}
\setlength{\topmargin}{-5mm}
\setlength{\oddsidemargin}{0mm}
\setlength{\evensidemargin}{0mm}
\setlength{\parindent}{0mm}
\setlength{\parskip}{\medskipamount}
\setlength{\unitlength}{1mm}

% -----------------------------------------------------------------------------
%  Hypertext definitions.
%  ======================
%  These are used by the LaTeX2HTML translator in conjunction with star2html.

%  Comment.sty: version 2.0, 19 June 1992
%  Selectively in/exclude pieces of text.
%
%  Author
%    Victor Eijkhout                                      <eijkhout@cs.utk.edu>
%    Department of Computer Science
%    University Tennessee at Knoxville
%    104 Ayres Hall
%    Knoxville, TN 37996
%    USA

%  Do not remove the %begin{latexonly} and %end{latexonly} lines (used by 
%  star2html to signify raw TeX that latex2html cannot process).
%begin{latexonly}
\makeatletter
\def\makeinnocent#1{\catcode`#1=12 }
\def\csarg#1#2{\expandafter#1\csname#2\endcsname}

\def\ThrowAwayComment#1{\begingroup
    \def\CurrentComment{#1}%
    \let\do\makeinnocent \dospecials
    \makeinnocent\^^L% and whatever other special cases
    \endlinechar`\^^M \catcode`\^^M=12 \xComment}
{\catcode`\^^M=12 \endlinechar=-1 %
 \gdef\xComment#1^^M{\def\test{#1}
      \csarg\ifx{PlainEnd\CurrentComment Test}\test
          \let\html@next\endgroup
      \else \csarg\ifx{LaLaEnd\CurrentComment Test}\test
            \edef\html@next{\endgroup\noexpand\end{\CurrentComment}}
      \else \let\html@next\xComment
      \fi \fi \html@next}
}
\makeatother

\def\includecomment
 #1{\expandafter\def\csname#1\endcsname{}%
    \expandafter\def\csname end#1\endcsname{}}
\def\excludecomment
 #1{\expandafter\def\csname#1\endcsname{\ThrowAwayComment{#1}}%
    {\escapechar=-1\relax
     \csarg\xdef{PlainEnd#1Test}{\string\\end#1}%
     \csarg\xdef{LaLaEnd#1Test}{\string\\end\string\{#1\string\}}%
    }}

%  Define environments that ignore their contents.
\excludecomment{comment}
\excludecomment{rawhtml}
\excludecomment{htmlonly}

%  Hypertext commands etc. This is a condensed version of the html.sty
%  file supplied with LaTeX2HTML by: Nikos Drakos <nikos@cbl.leeds.ac.uk> &
%  Jelle van Zeijl <jvzeijl@isou17.estec.esa.nl>. The LaTeX2HTML documentation
%  should be consulted about all commands (and the environments defined above)
%  except \xref and \xlabel which are Starlink specific.

\newcommand{\htmladdnormallinkfoot}[2]{#1\footnote{#2}}
\newcommand{\htmladdnormallink}[2]{#1}
\newcommand{\htmladdimg}[1]{}
\newenvironment{latexonly}{}{}
\newcommand{\hyperref}[4]{#2\ref{#4}#3}
\newcommand{\htmlref}[2]{#1}
\newcommand{\htmlimage}[1]{}
\newcommand{\htmladdtonavigation}[1]{}

% Define commands for HTML-only or LaTeX-only text.
\newcommand{\html}[1]{}
\newcommand{\latex}[1]{#1}

% Use latex2html 98.2.
\newcommand{\latexhtml}[2]{#1}

%  Starlink cross-references and labels.
\newcommand{\xref}[3]{#1}
\newcommand{\xlabel}[1]{}

%  LaTeX2HTML symbol.
\newcommand{\latextohtml}{{\bf LaTeX}{2}{\tt{HTML}}}

%  Define command to re-centre underscore for Latex and leave as normal
%  for HTML (severe problems with \_ in tabbing environments and \_\_
%  generally otherwise).
\newcommand{\setunderscore}{\renewcommand{\_}{{\tt\symbol{95}}}}
\latex{\setunderscore}

% -----------------------------------------------------------------------------
%  Debugging.
%  =========
%  Remove % from the following to debug links in the HTML version using Latex.

% \newcommand{\hotlink}[2]{\fbox{\begin{tabular}[t]{@{}c@{}}#1\\\hline{\footnotesize #2}\end{tabular}}}
% \renewcommand{\htmladdnormallinkfoot}[2]{\hotlink{#1}{#2}}
% \renewcommand{\htmladdnormallink}[2]{\hotlink{#1}{#2}}
% \renewcommand{\hyperref}[4]{\hotlink{#1}{\S\ref{#4}}}
% \renewcommand{\htmlref}[2]{\hotlink{#1}{\S\ref{#2}}}
% \renewcommand{\xref}[3]{\hotlink{#1}{#2 -- #3}}
%end{latexonly}

% -----------------------------------------------------------------------------
% ? Document-specific \newcommand or \newenvironment commands.
\newcounter{sruleno}
\newcommand{\srule}[1]{
    \addtocounter{sruleno}{1}
    \goodbreak
    \rule[0.5ex]{\textwidth}{0.3mm}
    {\Large #1 \hfill {\thesruleno}}
    \rule[0.5ex]{\textwidth}{0.1mm}
}

\begin{htmlonly}
  \newcommand{\srule}[1]{
       \addtocounter{sruleno}{1}
       \begin{rawhtml} <HR> \end{rawhtml}
       {\Large \thesruleno}~~~~{\Large #1}
       \begin{rawhtml} <HR> \end{rawhtml}
       \end{tabular}
  }
\end{htmlonly}

\renewcommand{\_}{{\tt\char'137}}

%------------------------------------------------------------------------------
% ? End of document-specific commands
% -----------------------------------------------------------------------------

%  Title Page.
%  ===========
\renewcommand{\thepage}{\roman{page}}
\begin{document}
\thispagestyle{empty}

%  Latex document header.
%  ======================
\begin{latexonly}
   CCLRC / {\sc Rutherford Appleton Laboratory} \hfill {\bf \stardocname}\\
   {\large Science \& Engineering Research Council}\\
   {\large Starlink Project\\}
   {\large \stardoccategory\ \stardocnumber}
   \begin{flushright}
   \stardocauthors\\
   \stardocdate
   \end{flushright}
   \vspace{-4mm}
   \rule{\textwidth}{0.5mm}
   \vspace{5mm}
   \begin{center}
   {\Large\bf \stardoctitle}
   \end{center}
   \vspace{5mm}

% ? Heading for abstract if used.
%   \vspace{10mm}
%   \begin{center}
%      {\Large\bf Abstract}
%   \end{center}
% ? End of heading for abstract.
\end{latexonly}

%  HTML documentation header.
%  ==========================
\begin{htmlonly}
   \xlabel{}
   \begin{rawhtml} <H1> \end{rawhtml}
      \stardoctitle
   \begin{rawhtml} </H1> \end{rawhtml}

% ? Add picture here if required.
% ? End of picture

   \begin{rawhtml} <P> <I> \end{rawhtml}
   \stardoccategory\ \stardocnumber \\
   \stardocauthors \\
   \stardocdate
   \begin{rawhtml} </I> </P> <H3> \end{rawhtml}
      \htmladdnormallink{CCLRC}{http://www.cclrc.ac.uk} /
      \htmladdnormallink{Rutherford Appleton Laboratory}
                        {http://www.cclrc.ac.uk/ral} \\
      \htmladdnormallink{Science \& Engineering Research Council}
                        {http://www.pparc.ac.uk} \\
   \begin{rawhtml} </H3> <H2> \end{rawhtml}
      \htmladdnormallink{Starlink Project}{http://star-www.rl.ac.uk/}
   \begin{rawhtml} </H2> \end{rawhtml}
   \htmladdnormallink{\htmladdimg{source.gif} Retrieve hardcopy}
      {http://star-www.rl.ac.uk/cgi-bin/hcserver?\stardocsource}\\

%  HTML document table of contents. 
%  ================================
%  Add table of contents header and a navigation button to return to this 
%  point in the document (this should always go before the abstract \section). 
  \label{stardoccontents}
  \begin{rawhtml} 
    <HR>
    <H2>Contents</H2>
  \end{rawhtml}
  \htmladdtonavigation{\htmlref{\htmladdimg{contents_motif.gif}}
        {stardoccontents}}

% ? New section for abstract if used.
%  \section{\xlabel{abstract}Abstract}
% ? End of new section for abstract

\end{htmlonly}

% -----------------------------------------------------------------------------
% ? Document Abstract. (if used)
%  ==================
%\stardocabstract
% ? End of document abstract
% -----------------------------------------------------------------------------
% ? Latex document Table of Contents (if used).
%  ===========================================
% \newpage
%\begin{latexonly}
%   \setlength{\parskip}{0mm}
%   \tableofcontents
%   \setlength{\parskip}{\medskipamount}
%   \markright{\stardocname}
%\end{latexonly}
% ? End of Latex document table of contents
% -----------------------------------------------------------------------------
\newpage

\begin{latexonly}\begin{center}
\vspace{10mm}
PREFACE
\end{center}
\end{latexonly}
\html{\section{\xlabel{preface}PREFACE}}

This Programming Standard should be read by all those who implement
programs on Starlink computers.  The techniques described are
requirements for Starlink-supported applications software but are equally
appropriate for private programming.  The standard applies mainly
to the implementation of freestanding VAX/VMS and Unix
applications, but also includes some material specific to developing
programs for the Starlink ADAM Software Environment.

Most of the recommendations made here have been available since
the appearance of the first edition in June~1981.
Apart from a few additions in August~1984, Starlink's advice
on general FORTRAN coding standards has not changed.
Recent versions of this document
have contained additional material on use
of graphics and on writing ADAM applications.

\newpage
\section{INTRODUCTION}

This note offers advice on how to write applications programs
for Starlink, so as to:
\begin{itemize}
\item maximize readability and comprehensibility
\item encourage uniformity
\item simplify maintenance
\item eliminate ego
\item assist portability
\item promote reliability and fault tolerance
\end{itemize}
Programming standards like this one are apt to be
treated with contempt by battle-hardened user-programmers,
at least those who have only
ever used one sort of computer and have never been faced with
having to look after someone else's code.  Even experienced professional
programmers are sometimes reluctant to accept advice, in the confident belief
that their own style constitutes the ideal balance between
discipline and pragmatism.
However, the lamentable standards of programming generally and Starlink's
objective of software sharing make some restrictions necessary and desirable.
The requirement that a program should work is merely
the beginning;  the effort expended in distributing and supporting even
the best software is well known to be much greater than that required to
write it in the first place.

Several existing standards were consulted before the Starlink
standard was drawn up.
Compared with most standards,
the Starlink one is rather liberal, omitting many of the rules in
these others while adding comparatively few of its own.

The `rules' in this standard are of variable importance, and range from mere
stylistic suggestions to firm requirements.
The relative importance of each rule is indicated by the following coding:
\begin{quote}
\begin{tabbing}
XXXXXXXX \= \kill
.   \> Suggestion \\
!   \> Strong recommendation \\
!!  \> Rule which can sometimes be waived \\
!!! \> Firm rule \\
\end{tabbing}
\end{quote}
Any program submitted to Starlink (whether written by a Starlink programmer or
a generous user) may be vetted for conformity with this standard, and any
violations taken into account when deciding whether Starlink is to undertake
distribution and support.
Major violations will only be acceptable if there are good reasons.
For example
(a)~the program may be of an interim nature and not require
long-term support,
(b)~it may be so urgently required that this has to override all other factors,
(c)~while departing from the standard in detail, it is so disciplined and
consistent that it can be accepted for support on its own terms.

Of course, this document does not pretend to be an exhaustive specification for
writing Starlink application programs.
More information on, respectively, the Fortran~77 language and general program
design and coding, can be found in the following two books:
\begin{itemize}
\item PROGRAMMING IN STANDARD FORTRAN~77 by A.Balfour and D.H.Marwick,
published by Heinemann.
This is a good reference for the language, and also contains some sound
programming advice.
Note that certain extensions to the published US standard are permitted in the
Starlink programming standard; these are not included in Balfour and Marwick.
\item THE ELEMENTS OF PROGRAMMING STYLE by B.W.Kernighan and P.J.Plauger,
published by McGraw-Hill.
Every Starlink programmer should read this classic
work and follow its recommendations
except in those rare instances where the Starlink standard differs.
\end{itemize}

\newpage
\section{GENERAL CODING STANDARD}

The rules and guidelines in this section apply to all applications programming
irrespective of what `software environment' they run under.
They fall into four groups -- language, design, quality and
presentation.

\goodbreak
\subsection{Language}

\srule{Fortran only !!!}
For the time being, writers of Starlink application programs are
urged to use Fortran exclusively.  Increasing amounts of Starlink
systems code are being written in ANSI~C (SGP/4 is the present document's
counterpart for the C language), but use of C is discouraged for
applications code.  A few pieces of Starlink system software use
various dialects of assembler and Pascal, but these are being
eliminated.  There may conceivably be future interest in ADA.

The only permitted dialect of Fortran
is ANSI X3.9-1978 (known as Fortran~77) plus certain
US Department of Defense extensions and -- in a few cases only -- some
extensions provided in DEC and several other proprietary Fortrans.

Discriminating Fortran users were disappointed with the 1978 standard, which
failed to contain many expected improvements.
Experience with preprocessors (RATFOR for example) had shown the feasibility of
turning Fortran into a much better language with a minimum of disruption;
unfortunately, many features which could easily have been added to the language
weren't.  DEC Fortran, however, does have a few of these features, and it
was decided that certain of these should be permitted within the Starlink
standard.  The rationale was
(i)~it would not be hard to eliminate these extensions manually if necessary,
(ii)~they have become almost universal, and
(iii)~similar facilities will be part of later Fortran standards.
Note that anyone wishing to adhere more rigidly to the Fortran~77 standard is
permitted to do so.
The latest version of Fortran, called Fortran~90, contains
many improvements over Fortran~77 and almost eliminates the need to
use {\bf any} platform-specific features.  Extra facilities not
available in existing Fortran~77 implementations (for example the
processing of whole arrays
as primitive data items) are also available.
The Fortran~90 standard identifies certain features as `deprecated'
or `obsolescent', candidates for deletion in future standards;
almost all of these are already prohibited by the Starlink standard.
Fortran~90 compilers are not presently provided on Starlink but
will be in due course.

The following extensions to the Fortran~77 standard are {\bf permitted}
within Starlink:
\begin{quote}
\begin{tabbing}
IMPLICITXNONEXXXXX\=\kill
IMPLICIT NONE\>encouraged\\
END DO\>permitted\\
DO WHILE\>permitted but worth avoiding\\
INCLUDE\>encouraged
\end{tabbing}
\end{quote}
IMPLICIT~NONE, END~DO, DO~WHILE and INCLUDE are all
in Fortran~90.  See Section~5, {\it Writing Portable
Programs}, for detailed advice on INCLUDE statements.

Here are some examples of features which are part of neither the Fortran~77
standard nor the Starlink standard, and are to be {\bf avoided}:
\begin{quote}
INTEGER*4, REAL*4, REAL*8, LOGICAL*1 etc.\ (see note 1, below)\\
Names $>$ 6 characters (note 2)\\
More than 19 continuation lines\\
Data initialization in a type declaration\\
Departures from Fortran~77 permitted statement order\\
Overlapping character substrings on both sides of an assignment\\
Data structures\\
Bit handling functions (note 3)\\
VMS specific OPEN keywords (note 4)\\
\$, O, Q, or Z edit descriptors in FORMAT statements (note 5)\\
\%VAL, \%REF, \%DESCR (note 6)
\end{quote}
Certain other non-standard features are dealt with later sections.
The DEC Fortran manual has all extensions to the ANSI standard
printed in blue, and these features should be avoided unless
expressly sanctioned in the Starlink standard.  If an
extension has to be used for any reason (because the required
result simply cannot be obtained any other way) this should
be highlighted with comments.  If you are unsure about whether
your program conforms to the Fortran~77 standard, try compiling
it with the /STANDARD qualifier (on VMS systems).

Notes:

\begin{enumerate}

\item INTEGER*2 and BYTE data types may be used where they are essential to
deal with input data sets containing 8 or 16 bit values.

\item Many programmers refuse point-blank to stick to 6 character
names, and some even like using
whole sentences with underscores between the words.  Opinions vary
on whether long names help readability all that much, but what is
certain is that several computer systems of considerable potential
importance to UK astronomers do {\bf not} support long
names (for example some mainframes,
certain attached processors, some workstations).  Some
computers accept names longer than 6 characters
but only a little longer; others
accept long names but ignore all but the first {\it n}\, characters.

Though names internal to a Starlink program must be no more than 6 characters
long, the use of names of more than 6 characters is permissible (and indeed
strongly encouraged) in the case of program unit and
labelled COMMON block names, in
order to reduce the chances of name clashes between different facilities
(especially the C run-time library).
Such names should be constructed by prefixing the name proper, which must be
no more than 6 characters long, with a facility name of the form `FAC\_'.
An example is SGS\_OPEN: the facility name is SGS and the name of the routine
OPEN.
For the utmost portability,
it is wise to limit the name proper to 5 characters
rather than the full 6; thus SGS\_ZSIZE can readily be preprocessed to SZSIZE
(for example) if merely dropping the SGS\_ gives name clashes.  Within a
given package, the name proper should be unique even if different
facility names are being used.
A further Starlink convention is that routines called only internally within a
package are given names prefixed with `FAC1\_' rather than simply `FAC\_'.
Application programmers should never use existing
facility names (see Appendix~A) unless contributing solicited
and debugged software for inclusion in the facility concerned.

Fortran~90 allows names of up to 32 characters.

\item Use of bit manipulation routines will be unavoidable in instrument
specific processing but is prohibited for general applications.  (Note
that Fortran~90 contains bit-manipulation facilities.)

\item Certain VAX extensions to the OPEN and INQUIRE statements
overcome serious
omissions from the Fortran~77 standard and may be used if necessary:
\begin{itemize}
\item READONLY
\item CARRIAGECONTROL=`LIST'
\item ACCESS=`APPEND' (worth avoiding)
\end{itemize}
Highlight these
VAX-specific items with comments, and use them only in short routines
which can be re-coded for other platforms.

\item In FORMAT statements, O (octal) and Z (hexadecimal) edit descriptors may
be required for the early stages of instrument specific processing, but must not
be used otherwise.  Q (number of characters input) and \$\ (suppress
carriage return) should be avoided.  Sometimes they have to be
used, typically at just one place in an application, and when this
happens they should be prominently commented as VAX-specific and
preferably isolated inside a short routine which can be re-coded
for other platforms.

\item The DEC Fortran \%VAL etc.\ may be required to interface to `software
environment' routines.
(ADAM is one such software environment -- see section 4.)
They should not be used for any other purpose.

\end{enumerate}

\srule{Stick to the Fortran~77 character set !!}
The Fortran~77 standard permits use of only the following characters
(except in comments and character strings):
\begin{verbatim}
    0-9, A-Z, space, currency symbol, +-*/(),'.:
\end{verbatim}
Though a few other characters are sufficiently common for trouble to be unlikely
it is best to avoid them even in comments as they may appear different on
different terminals and printers -- hash and pounds sign are particular
examples of this.  Backslash is best avoided because it has a special
meaning in almost all Unix Fortrans.
Applications should not depend on the presence of nonstandard characters --
stick to the Fortran~77 set.

An exception is made for lowercase letters
a-z, which are (a)~permitted in program
source code, and (b)~encouraged in output messages.
The recommended style is to use upper- and lowercase
freely in comments but to use uppercase in the Fortran proper.
If you must use lowercase characters in Fortran statements, it is
important to be consistent, not just to make the code easier to read,
but also because the Fortran compilers on some Unix systems regard
upper- and lowercase characters as different (so \verb|N| and \verb|n|
would be different entities for example).

The character set allowed by Fortran~90 includes \verb|_ ! " % & ; < > ? [ ]|.

Take care with collating sequence.  The letters A-Z appear in the
expected {\it order}, but you cannot assume
that they are {\it contiguous}\, in
the collating sequence.  Similar remarks apply to 0-9.  Do not
assume that the numbers appear before the letters.  Assume
nothing about the collating sequence of punctuation characters,
except that blank comes before both the letters and the numbers.

\srule{Input must not be case sensitive !!!}
Applications must not distinguish between upper- and lowercase in data they
receive.

\srule{Avoid end of line comments !}
End of line comments (preceded by `!') are not in the
Fortran~77 standard, and are best avoided, though they
are in Fortran~90 and are supported by many current compilers.
There is some justification for using them with data declarations.
If used at all they must be neatly aligned by means of spaces, {\bf not}
TAB characters (see the next rule).

\srule{Don't use TABs !!!}
The use of TAB characters is prohibited, both in source code and as required
data input to a program.
Though their use in source code may give an {\it illusion}\,
of good layout with a
minimum of effort, tab settings on different terminals and printers will in
general be different and will not reproduce what appeared on the screen when the
program was being written.
With most editors it is easy to program rarely
used keys or special function keys to produce multiple spaces -- for instance
3 and 6 -- which will reduce the number of keystrokes when entering Fortran.

\srule{Arithmetic IF banned !!!}
The use of the arithmetic IF:
\begin{verbatim}
    IF (X) 10,10,20
\end{verbatim}
is prohibited.
It offers even more opportunity for unstructured programming than the
discredited GO~TO and, moreover, doesn't read naturally in
English.
(The arithmetic IF is a fossil of one of the
machine instructions on the computer for which Fortran was
first devised, the IBM~704.)

\srule{Use the computed GO~TO sparingly !}
Don't use the computed GO~TO unless you have to, and then only to
implement a properly laid out `case' construct.  In such instances, a good
plan is to use comments to indicate the case for each destination.
For example:
\begin{verbatim}
    *  Switch according to command
          GO TO ( 1000, 2000, 3000, 9000 )
    *               Go  Hold  Stop  Abort
\end{verbatim}

\srule{ASSIGN banned !!!}
The ASSIGN statement is prohibited.  It offers endless opportunities
for incomprehensible code, and is a peculiar historical feature
not found in other languages.  Though it can be used in conjunction
with the assigned GO~TO (also banned) to implement a sort of
internal subroutine call, the temptation to do clever things
with saved or multiple return pointers etc.\ will prove too great
for some practitioners -- so the feature is outlawed by Starlink.  (See
also rule~29.)

\srule{Don't use PAUSE !!!}
PAUSE is prohibited.  It produces different results on different
sorts of computers and is sure to interfere with whatever
software environment you are using.

\srule{STOP, RETURN, ENTRY banned !!!}
STOP, RETURN and ENTRY are all prohibited.

STOP produces a message that is usually uninformative and is incompatible with
software environments.
It is also redundant -- the main program's END statement
causes program termination.

RETURN is redundant -- a subprogram's END statement returns to the caller.

The key reason for prohibiting STOP and RETURN is to force the programmer to
have one exit point only per routine, at the end, to aid debugging.
This complements the ban on multiple entry points (and hence on alternate
RETURN) in satisfying structured programming requirements.

Note that the ban on ENTRY and RETURN means that multiple ENTRYs
and RETURNs are similarly prohibited.

\srule{Don't use DIMENSION !!!}
The DIMENSION statement should not be used; arrays must be given explicit type
declarations.
For example:
\begin{verbatim}
    REAL A(512,512)
\end{verbatim}
must be used, rather than:
\begin{verbatim}
    DIMENSION A(512,512)
\end{verbatim}

\srule{Don't try to overprint !!!}
In FORMAT statements,
the use of the printer control character `+' is prohibited.
Assume that overprinting is not possible;
this is almost always the case.

\srule{Don't devise your own character handling mechanisms !!!}
Characters must be handled only by the character string mechanisms of
Fortran~77.  Private mechanisms (involving for example integer arrays) are
banned.

\srule{Remember to use SAVE !!!}
The SAVE statement must be used in all cases where
subprograms assume that their local variables retain
their values between successive invocations.  If you
don't do this, {\bf your programs will not run
correctly on some computers} even though they happen to work on VAX/VMS
and other compilers that allocate static storage by default.
Note that SAVE is
also required when using labelled COMMON, unless the COMMON block is
declared in at least one other program unit in the calling chain
(the main program for instance); in this case the SAVE specifies the
COMMON block itself rather than the
individual variables.  Avoid using the form of the SAVE statement
where the variable and COMMON block names are omitted.

The rules in the ANSI standard concerning SAVE are rather complex and
no further attempt will be made to summarize them here.

\srule{Statement labels on FORMAT and CONTINUE only !!!}
Statement labels are permitted on FORMAT and CONTINUE statements only.
They should be less than 10000, increase monotonically through the routine,
occupy columns 2-5 only and be consistently justified, left or right.

\srule{Declare everything explicitly !!!}
All variables, parametric constants and functions (except the Fortran~77
generic functions) must be given explicit type declarations.
To force this, it is strongly recommended that the IMPLICIT~NONE statement be
used, with the additional advantage
that undeclared arrays, functions, etc., and
many typing errors are exposed during debugging.

The use of a consistent naming scheme to indicate type, though
unfashionable, is worth considering.
The standard Fortran convention of
an initial I-N can be used to highlight integers, and where REAL and
DOUBLE~PRECISION variables are being mixed it can be helpful to
reserve initial D for the latter.  CHARACTER, LOGICAL and
COMPLEX entities are more rarely confused and there is
usually less need for naming conventions.

Arrays appearing in COMMON must have their type and dimensions declared in
separate statements which precede the COMMON declaration.
For example:
\begin{verbatim}
    REAL ARRAY(512)
    COMMON /fac_BLOCK/ ARRAY
\end{verbatim}

The types of subprogram arguments must be specified explicitly.

\srule{Use CHARACTER$\ast$($\ast$) !!}
Dummy arguments of type character, and character functions, should be declared
CHARACTER*(*) in order to pass the length implicitly.

\srule{Keep I/O out of mainline code !!}
Portability considerations make it desirable that all Fortran I/O (except to
internal files) be encapsulated within suitable primitive subroutines.

Avoid list-directed I/O, which may give different results on different machines.
There are library routines for input and output conversions
(see sla\_DFLTIN in SLALIB for example); use these.

In many `software environments' (ADAM for example -- see section 4), Fortran
I/O to the terminal is not permitted, and any Fortran I/O to files should be
done in accordance with any special rules of the software environment concerned.

\srule{Don't use literal I/O unit numbers !!!}
The external unit identifiers (logical unit numbers) used in OPEN, READ, WRITE,
etc.\ must not be hardwired -- the number itself should appear at a
maximum of just one place in the program, in a PARAMETER statement.
Keep to the range 0-255.

The various `software environments' usually have ways of supplying such logical
unit numbers from a pool, and this is recommended.

\srule{Don't use mixed mode arithmetic !!}
Avoid mixed mode arithmetic or change of type across an assignment statement.
Use explicit type conversions; for example:
\begin{verbatim}
    X = REAL(N-1)
\end{verbatim}
rather than:
\begin{verbatim}
    X = N-1
\end{verbatim}

Changing from one number format to another is a very significant
event in an algorithm and should be clearly expressed.  Great
circumspection should be employed when deciding where such
conversions should occur.

\srule{Data type must not change across a CALL !!!}
Data types must not be mixed across a function reference or subroutine call,
even in cases where it appears to work on a VAX (between REAL and
DOUBLE~PRECISION for example).

\srule{LOGICAL and INTEGER are different !!!}
LOGICAL and INTEGER usage must not be mixed.  For example:
\begin{verbatim}
    INTEGER JFLAG
       :
       :
    IF (.NOT.JFLAG) ...
\end{verbatim}
is not permitted.

\srule{Use only INTEGER DO-variables !!}
Don't use REAL or DOUBLE~PRECISION DO-variables, just integers.
The way the iteration count is determined and the
measures taken to avoid cumulative rounding errors are
important parts of the algorithm and should be spelt out.

\srule{Don't use EQUIVALENCE !!}
EQUIVALENCE should only be used if there is a very good reason, and even then
only in a straightforward way.  In particular:
\begin{itemize}
\item Do not use EQUIVALENCE to `extend' an array.
\item Do not EQUIVALENCE anything in COMMON.
\item Do not EQUIVALENCE entities of different data types.
\end{itemize}

\goodbreak
\subsection{Design}

\srule{Use Structured Programming !!}
The principles of {\em structured programming} should
be followed except in cases where this obscures the program logic.

Programs written using structured programming techniques
are built out of three basic elements:
(a)~processing sequences, (b)~decisions, and
(c)~loops.  Each of these three elements has
only one entry point (at the top) and only one exit (at the
bottom), and its relationship with the data it uses is
clearly defined.  The whole program consists of a hierarchy of
these elements.  In Fortran~77, (a) is just a series of
statements, (b) is an IF~THEN~ELSE construct and (c) is one
of the DO constructs.  The most
conspicuous feature of structured programs is the
absence of (or, depending on the suitability of the programming
language being used, the relative absence of)
GO~TO statements.  Such programs are incomparably easier to follow
than ones containing tangled GO~TO logic, but may be harder to
write, especially if you have not completely
grasped what you are trying to do before you start coding.

To conform to the Starlink standard, a program must {\it not}\, show signs of
having been written `bottom-up', or having `grown like Topsy'.
It can be a good plan to begin by writing the program in a `structured English'
pseudo-language; this can then become, as comments, part of the program.

The SPAG utility (SUN/63) can re-arrange GOTO- and arithmetic-IF-infested
code into block-structured form, and this is a good first step when
working up old code into a maintainable form.

\srule{Keep program units small !}
Modules should be small wherever possible, ideally a page or less (not
including prologue comments).
Programs which,
despite being longer than this,
are nevertheless not difficult to follow
-- because they consist of a simple top-to-bottom flow or contain a single main
loop with a simple flow inside it --- may be acceptable.

\srule{Avoid the GO~TO !!}
The GO~TO must not be used unnecessarily.
Use the DO and IF structures unless they make the program difficult to follow.
There are cases where use of GO~TO is justified by the need to jump downwards
in a program as the result of some exceptional condition, and use of
IF \ldots\ ELSE~IF together with indenting of code would give a less
satisfactory appearance.  If you have to form structures using
the GO~TO, clarify what is going on by means of commenting and indenting.

\goodbreak
If you are prepared to use the non-ANSI-standard DO~WHILE, you can write a
loop which includes tests for exit and repeat conditions,
without using the GO~TO:
\begin{quote}
\begin{verbatim}
LOGICAL LOOP
     :
LOOP = .TRUE.
DO WHILE (LOOP)
     :
  IF (`exit' condition) THEN
    LOOP = .FALSE.
     :
  ELSE IF (`next' condition) THEN
     :
  END IF
END DO
\end{verbatim}
\end{quote}

\srule{Don't loop using GO~TO !!}
Do not implement loops by jumping backwards using GO~TO.

\srule{Don't use GO~TO to drive internal subroutines !!!}
Internal subroutines driven by GO~TO statements are prohibited,
whether using the assigned GO~TO or not.  (See also rules 8 and 28.)

\srule{FUNCTIONs must not have side effects !!!}
FUNCTION subprograms must only be used when a single argument is returned and
there are no side-effects.
All other procedures should be SUBROUTINE subprograms.

\srule{Minimize use of COMMON !!!}
No unnecessary use of COMMON must be made.
The use of COMMON can lead to code which is difficult to maintain.
It should not be used simply as a lazy way of passing arguments to subprograms;
when it is used for passing arguments there must be a good reason and every item
referenced must be mentioned in the prologue comments just as for formal
arguments.

The names of entities in COMMON must not change from one program unit to
another; the use of dummies is discouraged.

All COMMON block source should be stored in separate files from the rest of the
source, and be inserted using the INCLUDE statement.

Blank common must not be used -- only labelled common (with a fac\_ prefix to
the name, as described in rule~1 note~2).

Note that the Fortran~77 standard does not permit initialization of common
blocks via DATA statements in main programs or normal subprograms;
BLOCK~DATA must be used.  (Avoid potential linking difficulties
when using BLOCK~DATA by referring to the name of the BLOCK~DATA module in an
EXTERNAL statement in at least one of the subprograms that makes use of
the relevant COMMON block.)

\srule{No garbage !!!}
The following are prohibited:
\begin{quote}
\begin{tabbing}
unused statement labels\\
unused declarations\\
unused FORMAT statements\\
unreachable code\\
\end{tabbing}
\end{quote}
FORCHECK (see SUN/73) will detect all of these conditions.

\srule{Be device-independent !!!}
Do not make unwarranted assumptions about the
hardware being used, especially the properties of
terminals or printers.  If your application requires
use of special features, either include mechanisms
external to the program which can be configured to
match the hardware available, or have the user make
explicit assertions.  A common crime, for example, is
to assume that the terminal is ANSI standard (VT100 etc.)
and to output escape sequences in order to control
scrolling or to output large characters etc.; these
will have unpredictable effects on other types of
terminal.  A solution is to include
in your program's repertoire of commands one which allows
the user to announce that he is on an ANSI terminal and
only to output ANSI escape sequences if that
command has been invoked.

\srule{Be environment independent !!!}
Be careful when specifying filenames or device names in application
programs.  There are, unfortunately, no platform-independent ways
of handling such names, but difficulties will in practice
be minimized if
filenames (i)~contain no uppercase letters, (ii)~are no longer than
eight characters or eight plus a period and a further three,
and (iii)~contain only letters, numbers and a maximum of
one period.

When soliciting a file or device name from the
user, make no assumptions about its format and pass on the
string received (to an OPEN statement typically) without altering
it or trying to deduce things from it.

\srule{Be aware of floating-point limitations !!!}
Programs must not rely on more than the following ranges and accuracies:
\begin{quote}
\begin{tabular}{ccc}
type & range & accuracy \\ \\
\verb|REAL| & $\pm10^{\pm38}$ \&\ zero & 6 dp \\ \\
\verb|DOUBLE PRECISION| & " & 14 dp
\end{tabular}
\end{quote}

\srule{Don't access uninitialized variables !!!}
{\bf Never} access an uninitialized variable.
Relying on variables being initially zero (usually
the case for VAX Fortran) is {\bf not} permitted.

\srule{Don't use VAX system service or RTL calls !!}
Any programs which call system services (QIO etc.) or Run-Time Library
routines will clearly be non-portable, and these techniques should
not in general be used.  This may not always be possible, and where
a VAX-specific call is required it should be kept separate from
the application proper by encapsulating it in a subprogram of
its own.  Occasional use in mainline code is excusable if
removing the call happens still to result in a runnable program (albeit
of reduced capability), and if prominent commenting is used to highlight
the VAX-specific code.

\goodbreak
\subsection{Quality}

\srule{Use meaningful names !!}
Use sensible names which (within the 6-character limit) offer some
indication of the meaning of the entity concerned.
The use of I, N, W, X, etc. for purely local and
temporary use is permissible; daft or misleading names are banned.

\srule{Don't re-use variables !!}
Use a variable for a specific purpose; use a different one if the meaning has
changed.

\srule{Define sizes parametrically !}
In general, the sizes of tables, queues, buffers and work arrays should be
defined parametrically.
For example:
\begin{verbatim}
    *  Reference star positions
          PARAMETER (NREFS=1000)
          REAL REFS(2,NREFS)
\end{verbatim}

If a size is required in more than one program unit, it should be
declared in an INCLUDE file.

\srule{Minimize rounding errors !!}
In cases where control over execution order within a
statement is important in order to minimize rounding, this
can be achieved by means of otherwise redundant parentheses.
For example, if DELTA1 and DELTA2 are small compared with B, their sum could be
computed without avoidable loss of precision as follows:
\begin{verbatim}
    A = B+(DELTA1+DELTA2)
\end{verbatim}

\srule{Validate inputs !!!}
Everything coming into the program from outside must be validated.

There are only two exemptions permitted:
\begin{itemize}
\item Subprograms called exclusively by programs which can guarantee to present
valid arguments.
\item Subprograms which explicitly put the onus of validation onto the caller to
achieve some real efficiency advantage.
\end{itemize}

\srule{Don't output error messages at too low a level !}
Subprograms will be more flexible if they
do not output error messages themselves but
instead leave this to the caller by returning a status.
(Using PAUSE and STOP is even worse -- see rules 9 \& 10.)

Applications running under a `software environment' (for example ADAM -- see
Section~4) should adopt the error reporting
strategy provided by that environment.  The
ERR\_ and MSG\_ packages (SUN/104) provide error reporting that works both
in ADAM-based and freestanding programs.  These packages enable subprograms
to report errors in detail while still allowing higher levels to
decide whether or not messages will actually appear on the user's
screen.

\srule{Don't terminate by count !!!}
Terminate input by end-of-file or by a special end record, {\bf not} by
count.

\srule{Don't test REALs for equality !!!}
REAL or DOUBLE~PRECISION variables must never be tested for equality
against non-zero numbers.  Testing for zero is also frowned on by most
experts but can be difficult to avoid.

(But see remarks about bad-pixel handling in ADAM applications -- section 4.)

\srule{Use the standard order for arguments !!}
The order of arguments in subroutines should be as follows:
\begin{quote}
Given\\
Given and altered\\
Returned\\
Status return
\end{quote}
Note that the Fortran~77 standard prohibits use of the same actual argument more
than once when calling a subprogram which gives one of the arguments concerned a
new value.  Thus a subroutine \mbox{P(A,B,R)},
which computes some function of A and B and finally returns it in R, must not
be called with arguments (X,Y,X) even
though this technique happens usually to work in VAX Fortran (for example).

\srule{Use generic names !}
Use the generic names of intrinsic functions, for example
MAX(A,B) rather than AMAX1(A,B).

\srule{Don't re-invent existing routines !!}
Starlink library routines should be used whenever possible, rather than
writing routines which duplicate (or almost duplicate) the functions of
existing Starlink routines.
It is permissible to {\it adapt}\, Starlink code where the required changes are
substantial and the code uses only published interfaces; the source must be
seamlessly blended with the new application and be given a new name.

Do not include in a package copies or slight variants of
Starlink routines in an attempt to make the package self-contained.
Assume the availability of the required Starlink library, and if the package has
to be run on an installation which does not have the Starlink software
collection make proper arrangements with the Starlink Software Librarian
to have the up-to-date libraries sent there.

Avoid using the Run-Time-Library routines available on VAX/VMS and other
platforms.  Many of the facilities included in these libraries are also
provided by POSIX, an industry-standard Portable Operating System Interface.
Fortran-callable versions of many of these routines are provided in the
Starlink PSX library.

Many of the libraries which form part of the ADAM Software
Environment (see Section~4) are available in two forms:  an
ADAM version and a free-standing version.  Programmers are
strongly recommended to use these libraries even where there
is no immediate intention of running under ADAM.

\srule{Don't write clever code !!}
Programs must not be obscure in the name of efficiency.
The first version of the program should be coded for clarity rather than
efficiency (within reason).
If there are found to be worthwhile (i.e.\ obvious to the user) improvements in
efficiency possible, at the expense of clarity, then changes can be made.
The reduction in clarity must then be made good by extra comments -- perhaps
including the original code.
(If the changes do not reduce the clarity of the program, then it was badly
written in the first place.)

\goodbreak
\subsection{Presentation}

\srule{Begin modules properly !!}
The first statement in a program unit must be one of the
following: PROGRAM,
FUNCTION,
SUBROUTINE or
BLOCK~DATA.  Preceding comments are discouraged,
to avoid confusion (either to the reader or
to software) over where each new module begins in a concatenated
sequence of such modules.  (Likewise, comments following an
END statement are not allowed.)

All program units must have sensible and self explanatory
names, as far as is possible given the limitations imposed by the
6 character or fac\_~+~5 character rule (see rule~1 note~2).

\srule{Include prologue comments !!!}
There must be one or more blocks of comments at the beginning of every program
unit, which together form a `prologue'.
The prologue must include the name of the program unit, a brief description of
what it does, and full details of its interactions with the calling environment.
The author, organization, and date should be given, expressed compactly.

To enable automatic recognition,
each block of prologue comments
must begin with a comment which starts
$\ast+$ and ends with a comment which starts
$\ast-$.  Elsewhere in the program, do not have any statements with
$+$ or $-$ in the second column.

Main programs must have a prologue which says what files will
be read or written, and gives the I/O unit identifiers
used (numbers or symbols).

Subprogram prologues must list all the arguments, clearly describing their
function, type (unless obvious), units (where applicable) and any special
properties (e.g.\ whether an array).
The words `given' and `returned' are recommended for direction, rather than
`input', `output', `source', `destination' and other possibly ambiguous terms.
Access to COMMON blocks should be treated similarly, with every item referenced
fully described.

It is recommended that the names of all subprograms
called (except the Fortran~77 intrinsic functions) be given in the prologue.
It is extremely important that all the information given in the prologue is
accurate and up to date.
Prologues can be automatically extracted from source held in text
libraries by using the LIBPRE facility (see SUN/8).

\goodbreak
Example:
\begin{quote}
\begin{footnotesize}
\begin{verbatim}
      SUBROUTINE sla_NUT (DATE, RMATN)
*+
*     - - - -
*      N U T
*     - - - -
*
*  Form the matrix of nutation for a given date (IAU 1980 theory).
*
*  (double precision)
*
*  References:
*     Final report of the IAU Working Group on Nutation,
*                                 chairman P.K.Seidelmann, 1980.
*     Kaplan,G.H., 1981, USNO circular no. 163, pA3-6.
*
*  Given:
*     DATE   dp         TDB (loosely ET) as Modified Julian Date
*                                           (=JD-2400000.5)
*  Returned:
*     RMATN  dp(3,3)    nutation matrix
*
*  The matrix is in the sense   V(true)  =  RMATN * V(mean) .
*
*  Called:   sla_NUTC, sla_DEULER
*
*  P.T.Wallace   Starlink   10 May 1990
*-
\end{verbatim}
\end{footnotesize}
\end{quote}
Note, however, that much of the freedom implied by the above recommendations 
may not be available to programs which conform to the documentation standards of
a particular `software environment', especially if automatic documentation
facilities are involved.
For details of the prologue requirements of the ADAM software environment, see
section 4.

\srule{Begin comments with $\ast$.}
The comment symbol `$\ast$' should be used in
preference to the old-fashioned `C'.

\srule{Use blank lines to improve layout.}
Blank lines should be used freely to break up code.

\srule{Make comments stand out from code !!}
Comments must be clearly distinguished from code.  The recommended
style is begin the text some fixed amount (1-3 characters) to the
left of the code, and to follow the same indenting scheme in
comments and code alike.  An alternative style, less likely
to be spoiled by editors and reformatters which change
the code indentation, is to begin the comments always in column
3 or 4.  Comments should use lowercase freely, but the code should
be in uppercase.

Comments beginning in column 7 are strongly discouraged, even if preceded by
\mbox{\tt{C*****}} and the like.
\srule{Be considerate !!!}
Every effort must be made to present the program in the clearest and most
agreeable way {\bf from the point
of view of the support programmer} -- who has a much
more difficult job to do than the original author.
The layout must be neat, clean and consistent, and there must
be liberal commenting, both at the start of each module and
within the code.

The comments, which should in general precede the code they
describe:
\begin{itemize}
\item must accurately reflect the behaviour of the program;
\item must be detailed without merely stating the obvious;
\item must not be cryptic -- `clues' are not enough;
\item must be in English, without spelling or grammatical errors;
\item must not contain attempts at humour, meretricious
phraseology, catchwords, superfluous pleasantries, private jargon
and clever abbreviations.
\end{itemize}

\srule{Modifications must blend in !!!}
When a program is changed, the modifications must be `invisibly mended' into the
coding;
scars are not permitted.
The original style and conventions of the program must be
preserved -- which will
be much easier and more palatable if the program conformed to the Starlink
standard in the first place.

There should be no need for elaborate and unsightly `revision-flag' schemes
except, perhaps, during debugging or where software that runs on more than one
machine has machine specific inclusions.

\srule{Fix bad code !!!}
Bad code must be rewritten, not merely commented.

\srule{Use `:' as the continuation character !}
Only one continuation character should be used, preferably `:'.  (If
you do choose something else, remember
it should be in the Fortran~77 character set. The dollar sign is
a popular choice because it has no syntactical function in ANSI Fortran
outside a character string.

Using 1,2,3 \ldots\ for successive lines is specifically discouraged as it leads
to annoying editing problems (and, after all, protection against shuffling is no
more necessary than for any other region of the program).

\srule{Lines longer than 72 characters are not allowed !!!}
Program lines (including comments)
must not be longer than the legal 72 character maximum (even though
compiler options such as /EXTEND\_SOURCE on VAX/VMS can override this
maximum).
A lengthy statement (that for some reason cannot be broken up into several
shorter statements) should be split at natural breakpoints and the pieces
neatly aligned.
Statements must not be split between lines simply by exploiting the break at
column 72.  Take care with character constants, where this can easily
happen inadvertently -- especially inside format specifications.

\srule{Use spaces to improve readability.}
Spaces should be used freely within Fortran statements
to make them easier to read.  A space before
and after the equals sign in an assignment statement is particularly
recommended.

\srule{Use indenting to show structure !!!}
It is essential that the structure of a program be reflected in a consistent and
pleasing indentation scheme.

A suggested scheme is as follows.
The normal starting columns for comments and code should be 5 and 7
respectively.
For each block between a DO and END~DO, or between IF, ELSE~IF,
ELSE and END~IF, both comments and
code should be indented a further 3 columns.
Blank lines should be used freely to improve presentation further.

\goodbreak
This scheme is used in the following example.  This is, of course,
only one acceptable layout, and tastes vary.
\begin{quote}
\begin{footnotesize}
\begin{verbatim}
*   Reset bad pixel count
      NBLEM = 0

*   Look at all pixels except edge
      DO IY = 2, NY-1
         DO IX = 2, NX-1

*         Reset blemish flag
            IBLEM = 0

*         Pick up pixel
            PXV = A(IX,IY)

*         Expected value = mean of surrounding eight
            S = 0.0
            DO J = IY-1, IY+1
               DO I = IX-1, IX+1
                  S = S+A(I,J)
               END DO
            END DO
            EV = (S-PXV)/8.0

*         'Blemish' criterion
            BLEM = SIGMAS*SQRT(ABS(EV))

*         Decide whether in star or not
            IF (EV.GT.STAR) THEN

*            In star
               IF (PXV.LT.EV/4.0-BLEM) IBLEM=1
               IF (PXV.GT.4.0*EV+BLEM) IBLEM=2
            ELSE

*            Not in star
               D = PXV-EV
               IF (D.LT.-BLEM) IBLEM=3
               IF (D.GT.BLEM) IBLEM=4
            END IF

*         Override if negative or small
            IF (PXV.LT.DIM) IBLEM=0
            IF (PXV.LT.0.0) IBLEM=5

*         If blemish, fix and report
            IF (IBLEM.NE.0) THEN
               B(IX,IY) = EV
               NBLEM = NBLEM+1
               WRITE (LINE,'(''Blemish at    ('','     //
     :                     'I4,'','',I4,'')'','        //
     :                     'G16.4,''  changed to'','   //
     :                     'G16.4)') IX-1,IY-1,PXV,EV
               CALL WRUSER(LINE,JSTAT)
            END IF
         END DO
      END DO

*   Final report
      WRITE (LINE,'(I6,''  blemishes removed'')') NBLEM
\end{verbatim}
\end{footnotesize}
\end{quote}

\srule{Put FORMAT statements inline !}
FORMAT statements should be inline (following the appropriate READs or WRITEs)
rather than, for example, massed at the end of the program.
When a given format
specification is required once only it can be incorporated into the READ
or WRITE itself as a character
constant, unless the result is less clear (for example because of
multiple apostrophes).  When using this technique, be sure
the constant doesn't continue through multiple lines (see rule~59,
and the example in the previous rule).

\newpage
\section{GRAPHICS}

\srule{Plot using an approved graphics package !!!}
All graphics operations must be carried out, ultimately,
by the GKS (SUN/83) and IDI (SUN/65) libraries.

{\bf GKS:~~} Direct use of raw GKS in applications is
discouraged, in favour of using one
of the Starlink-supported higher-level packages as an intermediary.
These include SGS, the \mbox{NCAR} utilities (and the SNX routines),
the NAG Graphical Supplement, and \mbox{PGPLOT}.  (Though the Starlink
version of the proprietary package MONGO uses GKS, use of
callable MONGO in application
programs is not encouraged in this standard as its functions
are provided elsewhere.  MONGO is supplied by Starlink principally
for interactive use, and is in any case being phased out in favour
of the Starlink \mbox{PGPLOT}-based PONGO package.)

{\bf SGS:~~} Where only low-level facilities are required (lines,
character strings, formatted numbers, simple shapes, but
not complete axes or whole graphs) the SGS package (SUN/85)
is recommended.  SGS consists mainly of convenient packaging
of GKS functions, including easy control of plotting zone
and a flexible workstation naming scheme, and can be used
in conjunction with direct GKS calls.

{\bf PGPLOT:~~} In standalone applications where complete graphs are to be
drawn and convenience is more important than flexibility,
the Caltech \mbox{PGPLOT} package (SUN/15) may be used.
\mbox{PGPLOT} was expressly designed to meet the requirements of
astronomers, and is especially
good at publication-quality laserprinter output.
\mbox{PGPLOT} exists in two forms, one using GKS and the other
with native low level and device driver layers, which gives
it certain portability advantages for astronomers
wishing to send their applications overseas.
The two forms are of comparable performance except in greyscale output,
where GKS is substantially faster.
\mbox{PGPLOT} is moderately flexible,
and gives good access to colour and multiple text fonts.
However, direct access to GKS during use of \mbox{PGPLOT}
is prohibited, limiting the package's capabilities
to those explicitly provided by \mbox{PGPLOT} itself.

{\bf NAG:~~} The NAG Graphical Supplement (currently available
only at certain sites --  see RAL~LUN/45) also draws complete graphs,
in a number of useful though spartan formats.  It has
portability advantages, however, especially for users of
UK university computing facilities.  Some direct access to GKS is possible.

{\bf NCAR \& SNX:~~} The \mbox{NCAR} package, (SUN/88) supplied by the
National Center for Atmospheric Research, is widely available and runs
on many different types of computer (n.b./ recent versions are proprietary).
The Starlink SNX routines (SUN/90) allow the \mbox{NCAR} routines
to be used in conjunction with SGS, and direct use of GKS
facilities is also possible.  The most important
of the \mbox{NCAR} utilities is \mbox{AUTOGRAPH}, which draws complete
plots of one variable against another.  Great flexibility
is available, and effects can be achieved which are
impossible with other packages of comparable level.  For
example numeric labels can be intercepted and replaced;
opposite axes can have different scales (including
non-linear ones), lines can have embedded captions, and so on.

{\bf IDI:~~} The IDI package (SUN/65) differs in its objectives from all
those mentioned so far, which are mainly oriented towards line-drawing.
IDI is a low-level interface for image display devices, and supports
concepts absent in GKS,
for instance re-configurable pixel memory.
Compared with the GKS-based packages,
IDI's drawing facilities are rather primitive, but IDI comes
into its own for highly interactive applications, involving
special cursors, pan/zoom, blink etc.

{\bf AGI \& GNS:~~} Two supporting packages are the Graphics Database library
AGI (SUN/48) and the Graphics Name Service library GNS (SUN/57).
AGI remembers what has been plotted where and allows one
applications to access the plotting of another.  GNS is a
unified naming scheme which is supported across all the
Starlink graphics packages.
Application programmers should be fully aware of what
these packages do, should make full use of these facilities,
and should not duplicate their functions in application code.

{\bf Mixtures:~~} With care, several of the plotting packages can
be used in conjunction with one another.  The various packages
interrelate with each other and with applications
in ways which are unfortunately too complicated and
varied to permit a simple diagram to be drawn or a simple
set of rules to be laid down.  The techniques available depend on the
particular combination of packages
involved, and guidance should be sought from
the documents describing those packages.   The SGS, SNX,
NCAR and GKS packages are more miscible than some of the
others.  It is normally safe to use the packages
sequentially;  the dangers lie in calls to one package
deranging the context of one of the others.  The difficulties
and complications of this area are a direct result of
offering the utmost flexibility.  Those who prefer
safety and simplicity would be well advised to stick to a single
package (for example \mbox{PGPLOT}) used in a straightforward way.

\srule{Don't interpret graphics device names or numbers !!!}
Do {\bf not} examine in applications the GKS/SGS/GNS workstation
type or name in order to control the behaviour of the
program.  For example, you are not allowed to say ``the
GKS workstation type is 3200, therefore this is an
Ikon image display, therefore colour is available".  This
information must be obtained by calling the appropriate
GKS, SGS, \mbox{PGPLOT} or GNS enquiry routines;  if the property
you are interested in is not available through such
enquiries then devise some mechanism external to the
program, or solicit information from the user.

\newpage
\section{THE ADAM SOFTWARE ENVIRONMENT}

\subsection{Introduction}
Applications running under Starlink's recommended software
environment ADAM should in most respects be programmed
according to the rules given so far.  However, ADAM has a number
of special requirements which may mean that one of the
general rules has to be reinterpreted -- in some cases
strengthened, in others relaxed.  There are, in addition,
several new rules which do not have to be obeyed when
writing non-ADAM applications.

Programming standards for ADAM applications written outside
Starlink may, of course, differ from those laid down by Starlink.

\goodbreak
\subsection{Rules for programming ADAM applications}

\srule{Initialize variables !!!}
It is {\bf essential} that variables are initialized.
Even the VAX's initialization to 0 cannot be relied upon as the task may or
may not be reloaded between invocations.
DATA statements must only be used to initialize data which will not change. 
(Emphasizes rule~36.)

\srule{Use the ADAM standard prologue !!}
ADAM standard prologues differ in some respects from the Starlink
standard, allowing less freedom but giving more opportunity for
the automatic production of documentation and help files.
For details, see SUN/105 and SUN/110.
(Modification of rule~51.)

\srule{Output messages via the MSG\_ routines !!!}
Message output must be done using the ADAM message system MSG\_ subroutines.
The ADAM message system is described in SUN/104.  A stand-alone
version of the MSG\_ package exists.
(Supplements rule~18.)

\srule{Don't use \$, \%, $\wedge$ in messages !}
The non-Fortran~77 characters \$, \% and $\wedge$ are
used as escape characters in the ADAM message system, and
special methods have to be employed if they are to be included
in messages (see SUN/104).  In general, it is better to
avoid using them.
(Reinforces rule~2.)

\srule{Report errors via the ERR\_ routines !!!}
Error reporting must be done using the ADAM error system ERR\_ subroutines
and the ADAM error strategy should be employed. 
The ADAM error system is described in SUN/104.   A stand-alone
version of the ERR\_ package exists.
(Reinforces rule~43.)

\srule{Set STATUS on failure !!!}
All applications which fail must return to the
environment with an error status value set. 
This is to enable the environment to detect the failure so that users can
write procedures which take appropriate action.
(Supplements rule~43.)

\srule{When setting STATUS, generate a message !!}
If a subroutine is entered with STATUS=SAI\_\_OK
but, during execution,
sets the STATUS (other than by calling another ADAM routine),
an appropriate error message must be generated using ERR\_REP
(see SUN/104).

\srule{Some routines have $>6$ character names.}
Some of the ADAM environment package subroutines have names and prefixes
greater than 6 characters.
Where it is necessary to call these, rule~1 note~2 must be relaxed.

\srule{Get parameters with the PAR\_ routines etc.\ !!!}
All program parameters must be obtained using the parameter system PAR\_
or pkg\_ASSOC subroutines. 
(Reinforces and supplements rule~18.)

\srule{Use symbols when testing for bad pixels !!!}
A REAL or DOUBLE~PRECISION variable may be equated to its corresponding
bad-pixel value, though explicit bad-pixel values, e.g.\ $-32768$, are banned.
The parameters VAL\_\_BADx
(see SUN/39), where x corresponds to the data type, must be
used.
(Relaxation of rule~45 for this special case.)

\srule{Avoid Fortran input/output !!!}
Use the environment facility packages MAG, FIO etc.\ wherever possible.
(Reinforces rule~18.)

If it is necessary to use Fortran I/O, obtain and release logical unit numbers
using FIO\_GUNIT and FIO\_PUNIT. 
(Supplements rule~19.)

\srule{Use symbolic names !!!}
Status values and package constants are given symbolic names such as
PAR\_\_NULL and DAT\_\_SZLOC by INCLUDE files for each package. 
These symbolic names should be used on every occasion that the constant is
required.  Follow these conventions when developing your own
INCLUDE files, and use file names in the INCLUDE statements
which conform to the convention {\it fac}\_{\it err} for
error codes and {\it fac}\_{\it par} for other symbolic
constants, where {\it fac} is the facility name.  Further
advice on INCLUDE statements can be found in Section~5, {\it Writing
Portable Programs}.

\srule{RETURN is permissible when testing status.}
The RETURN statement is allowed in the form:
\begin{verbatim}
    IF (STATUS.NE.SAI__OK) RETURN
\end{verbatim}
as the first executable statement in a subroutine. 
This avoids an extra, unhelpful IF clause and indentation.
Alternatively, use a GO~TO n, where line n is a CONTINUE statement immediately
preceeding the END statement.
(Relaxation of rule~10.)

\srule{In generic routines use only the standard tokens !!}
The preprocessor for
generic routines supports special tokens used by the \mbox{ASTERIX}
package (SUN/98), as well
as ones for general use.  Use only the standard tokens.

\srule{PAR\_\_ABORT status (!!) must abort the application !!!}
An application must terminate if the {\it abort} response (!!) is
made when a parameter has been requested.  Note that this rule
does not mean that the application has to test for the abort
status after every parameter is obtained;  the inherited
status will look after that.  What matters is the appearance to the
user of the application, who should:
\begin{itemize}
\item not be re-prompted for the parameter,
\item not be prompted for further parameters, and
\item not receive additional error messages merely because
the status was not OK.
\end{itemize}
An abort does not absolve the programmer from ensuring that the
application closes down in an orderly fashion.

\newpage
\section{WRITING PORTABLE PROGRAMS}

One of the key reasons for having a Starlink programming standard
is to promote {\it software portability}.  What is meant by this
term, and why is it important?

\subsection{Meaning of Portability}

Other things being equal, it is clearly desirable for applications
to be usable on different computers rather than be limited to
just one type.  Equally clearly,
there may be a tradeoff between the extra trouble of ensuring
that application code is highly machine-independent and the work of modifying
or rewriting programs from time to time.  The Starlink Application Programming
Standard recognizes this tradeoff and allows the programmer to choose what
degree of portability is appropriate, taking into account:
\begin{itemize}
\item the type of software;
\item its life expectancy;
\item who will support it long-term; and
\item the extent to which the programmer is prepared to rely on
infrastructure software provided by others.  
\end{itemize}
To put the recommendations of the Starlink Standard in context,
consider four degrees of portability, called here
{\it Absolute Portability}, {\it Portable Fortran},
{\it Adaptable Fortran}\, and {\it Laissez-Faire}.

{\bf ABSOLUTE PORTABILITY} is where application source code
compiles and runs on all types of computer without
any alterations whatsoever.  Once the programmer has completed work on
an application, the code need never again be touched.
To achieve this result,
the programmer must be fully insulated from the facilities offered
by the platform.  Because the Fortran~77 standard does not include
all those things which applications need to do, and in any case
compilers vary in their compliance with and interpretation of
the standard or have bugs, it is not possible to rely on pure standard Fortran.
(Similar arguments apply to other languages.)  The classic solution is to
write applications in a private programming language, and to
accommodate differences between computers, compilers and operating
systems by providing different versions of the language interpreter
software.  This approach has the benefit that for
any new platform, once a new version of the system software has
been written, unlimited quantities of application code will run.
However, having to use the systems's own programming language
provokes scepticism among users, introduces extra
training needs, produces code which can only run within the
system, and reduces the convenience and
effectiveness of online source code debuggers.
These drawbacks led Starlink to reject this approach.

{\bf PORTABLE FORTRAN}, Starlink's recommendation, is to write
applications in an industry-standard language --
Fortran~77 -- with controlled use of certain platform-dependent
features as sanctioned by Sections 2 and 4 of the present document.
Significant departures from
standard Fortran (for example the use of \verb|%VAL|) should be present
in only a small minority of modules, with most routines in {\it de facto}\,
standard Fortran.  These departures can, if and when
necessary, easily be edited using simple preprocessors like
\verb|forconv| or even by hand.  Furthermore, the programs can be
understood and modified by non-specialists.

{\bf ADAPTABLE FORTRAN}, also embraced by the Starlink Standard, differs
from {\it Portable Fortran}\, in the degree to which
departures from ANSI Fortran are tolerated.  While gratuitous use of
platform-specific features is frowned upon, it is accepted that
some use of such features will be convenient and relatively
harmless.  Programs of this general level of portability are easy to
write and to adapt manually for new platforms as required.

{\bf LAISSEZ-FAIRE} programming is where programmers can use whatever the
current machine's Fortran compiler accepts -- the objective is simply
to have a program that works.  If a new computer is introduced,
authors can decide whether to adapt, rewrite or scrap their applications.
This style of programming lies outside the Starlink Standard and
is deprecated for anything more than a casual one-off.

The {\it Absolute Portability}\, and {\it Portable Fortran}\,
categories presuppose substantial quantities of {\it infrastructure
software}, libraries and utilities which leave the
programmer free to concentrate on the application itself rather than
worrying about user interfaces, error handling, input/output and
so on.  At the lowest levels within the infrastructure there is a
small platform-specific kernel, which has to be
rewritten for each new machine.  The {\it Adaptable Fortran}\,
and {\it Laissez-Faire}\, categories allow
programmers to provide their own infrastructure if they wish.

{\bf Starlink's recommendation is to base programs on
the various standard tools and libraries, and to aim for the
PORTABLE FORTRAN level in applications code.}

\subsection{Why Portability Matters}

Despite the fact that for its first decade Starlink supported just
one platform -- VAX/VMS -- the importance of avoiding platform-specific
software has been stressed from the beginning.  There are two main
reasons for this.  Firstly, there were and are
collaborating astronomical institutions using non-Starlink types
of computer -- Data General, Fujitsu, Perkin-Elmer, CDC, Cray and various
Unix platforms -- and it is useful if they can run Starlink applications,
and programs written by Starlink users.
The second reason is to enable existing
software to run on different sorts of Starlink equipment -- currently
Sun and DECstation as well as VAX/VMS.  Quite apart from further
Unix-based platforms,
it is also possible that fast specialized processors will be added to
the existing Starlink systems, and there is interest in using
various types of Personal Computer.  Attention
to software portability -- which means resisting the temptation
to use Sun and DECstation features now
just as much as avoiding VAX dependency in the past -- means great
benefits in the long term.

\subsection{Achieving Portability}

Programmers who have followed the recommendations given earlier in
Section~2 are likely to encounter fewer difficulties in adapting their
code to run on new and multiple platforms than programmers who have
not.  Of those recommendations, the key one is to work only with
ANSI Standard Fortran~77 and only to use a VAX extension
or an extension available on some other platform when it is essential,
or safe, to do so.  This advice
should be borne in mind when reading the following
notes, many of which refer to problems that
afflict code where non-standard Fortran extensions have
been used.  The notes concentrate on the specific problem of adapting
VAX/VMS applications to run on Unix platforms but also serve to
illustrate more general portability issues.
\begin{itemize}
 \item ANSI FORTRAN 77 STANDARD:~~
  There are a number of violations of the Fortran standard that are
  allowed by the VMS compiler that will cause problems on a UNIX system;
  some are rejected by the compilers but others cannot be detected at
  compile time and will cause programs to fail.
  \begin{itemize}
   \item {\it Overlapping character substrings:~~}
    The character strings on either size of an assignment
    statement must not overlap.  If not detected at compile
    time, such an overlap will produce incorrect results on
    Unix platforms.
   \item {\it Illegal string concatenation:~~}
    The concatenation operator \verb|//| cannot be used in
    circumstances that would require the allocation of arbitrary
    amounts of dynamic memory at run-time.
    For example, a CHARACTER*(*)
    dummy argument of a subroutine cannot be concatenated
    with another character string in the argument list of a subroutine or
    function call.  (The ANSI standard puts it thus:  a passed-length
    character dummy argument may only be the operand of a concatenation
    operator within an assignment statement.)
   \item {\it Mixing character and numeric data in a COMMON block:~~}
    Separate COMMON blocks are required for character data on the one
    hand, and numeric and logical data on the other.  (Similarly,
    it is illegal to EQUIVALENCE character data with anything else.)
  \end{itemize}
 \item INPUT/OUTPUT:~~
  The Fortran I/O system is not tightly enough specified to avoid
  problems with different implementations:
  \begin{itemize}
   \item Most compilers have their own set of non-standard I/O keywords,
    especially in OPEN statements.  If use of such keywords is
    unavoidable they must only appear in explicitly platform-dependent
    routines, not in the middle of large programs.
   \item Compilers vary in their tolerance of illegal combinations of
    keywords, which must be avoided.
   \item {\it I/O unit numbers:~~}
    On Unix platforms the I/O unit numbers 0, 5 and 6 refer to the standard
    error, input and output channels respectively and cannot be used for
    anything else.  Furthermore, only those unit numbers can usefully be used
    for reading from and writing to the terminal; other logical units are
    buffered in a way that is inappropriate for terminal I/O.
   \item {\it Version numbers:~~}
    The Unix file system does not have file versions; opening a
    file with \verb|STATUS='NEW'| when the
    file already exists will either destroy
    the contents of the file or fail depending on the system.
   \item {\it READONLY:~~}
    On the DECstation, the non-standard keyword \verb|READONLY| is required in
    order to open a file that you do not have write access to. The Sun
    compiler issues a warning message if \verb|READONLY| is used.
   \item {\it RECORDTYPE:~~}
    On Unix platforms, opening an existing file with the non-standard
    keyword \verb|RECORDTYPE='FIXED'| requires
    that the RECL keyword is used as well because unlike on VMS the file
    system does not store the record length in the file header.
   \item {\it Unformatted direct-access files:~~}
    In the OPEN statement for direct-access files the Fortran standard
    requires the record length to be specified, by means of the
    \verb|RECL| keyword.
    In the case of a formatted file, the length is in characters;  however,
    the Fortran standard does not specify the units of length for
    an unformatted file.  For unformatted files the Sun uses bytes, whereas
    the VAX and DECstation use numeric storage units (the space required
    to store a REAL or INTEGER value).
   \item {\it Printer control codes:~~}
    In the OPEN statement, \verb|CARRIAGECONTROL='LIST'| is non-standard and
    is not supported by some platforms.  There is no machine-independent
    way of specifying whether a text file contains Fortran printer control
    codes or not, and the effect of typing out text files produced by
    Fortran programs or of reading such files into a text editor cannot be
    predicted.  This problem must be handled through per-platform code
    variations or by using per-platform utilities for processing the
    files (for example the \verb|fpr| command on the DECstation and Sun).
   \item {\it Prompt strings:~~}
    There is no portable way of suppressing CR/LF after
    a message has been output, though the VAX, DECstation and Sun all
    use the non-standard `\verb|$|' edit descriptor.  Provision must
    be made for per-platform variations at this point in an application.
   \item {\it Status:~~}
    The I/O status values returned by OPEN, CLOSE, READ and WRITE
    are non-portable and application code should avoid using them in
    anything more than a general way (or should use the ERR\_FIOERR
    routine -- see SUN/104).  Unfortunately, it is not even
    possible to map the
    numbers from the different platforms onto a single adopted set of
    values since the conditions that each platform reports
    as errors are different.
   \item {\it End-of-File:~~}
    A READ or WRITE statement which includes the ERR= specifier
    behaves differently on different platforms when end-of-file
    is encountered.  The condition is treated as an error on the
    VAX and DECstation but not on the Sun.  To comply with the ANSI
    standard, all platforms return an IOSTAT value of $-1$.
  \end{itemize}
 \item DATA STORAGE ALIGNMENT:~~
  The VAX is unusual in imposing no restriction on the addresses of data;
  many architectures generate a hardware error if, for example, a
  floating point operand has an odd rather than an even address. Both the
  Sun and the DECstation do this and although both operating systems
  handle the error successfully and allow the program to continue, it is
  at the expense of both a huge execution time overhead and a mysterious
  message being output. COMMON blocks should therefore be arranged such
  that the longer data types always appear before shorter data types.
 \item THE BACKSLASH CHARACTER:~~
  Unix compilers treat the backslash character as an escape character (so
  that for example \verb|\t| is translated into a tab character) and to
  insert a
  true backslash character the source must have two backslash characters
  (i.e.\ \verb|\| on VMS must be converted to \verb|\\| on Unix).
  The \verb|forconv| tool (SUN/111) can be used to insert the extra
  backslash character when converting source code.

  Some compilers have a switch that turns off the special meaning of the
  backslash character but using this is unwise -- see the remarks
  on compiler switches later.
 \item THINGS THAT WORK BY ACCIDENT ON VAX:~~
  There are a number of bugs that can go unnoticed on VMS but will cause
  programs to fail on other systems:
  \begin{itemize}
   \item {\it Argument mismatches across subroutine and function calls:~~}
    For example a DOUBLE~PRECISION argument passed to a subroutine
    expecting a REAL happens to work on VMS but is a bug.
   \item {\it Uninitialized variables:~~}
    On a VAX uninitialized variables will be set to zero;  on Suns and
    DECstations they will not (see section~2, rule~36).
   \item {\it Missing SAVE statements:~~}
    On VMS the values of variables local to a subroutine are retained
    between calls to the subroutine.  On other systems they may not be;
    on the DECstation, for example, it depends on compiler switches.
    (See section~2, rule~14.)
  \end{itemize}
 \item THINGS THAT WORK BY ACCIDENT ON UNIX:~~
  Similarly, there are bugs that go undetected on many Unix systems
  which will cause problems on VAX/VMS.  For example, if a CHARACTER*n
  argument is not declared as such in a subprogram, it is often
  possible to get away with this on Unix systems but not on VMS.
 \item UNUSED VARIABLES:~~
  The Unix compilers always complain about declared but unused variables.
  (See section~2, rule~32.)
 \item INCLUDES:~~
  INCLUDE statements, by their very nature, cannot avoid involvement with
  the syntax of file names which makes writing source code that will run
  on many machines with absolutely no change of source code difficult if
  not impossible. However, the following scheme keeps the changes to a
  minimum and allows what changes that may be necessary to be automated.
  \begin{itemize}
   \item On the VAX call your INCLUDE files \verb|xxxx.for| where
    \verb|xxxx| is some name of your own choosing.
   \item Include them with statement like:
    \begin{quote}
    \begin{verbatim}
      INCLUDE 'xxxx'
    \end{verbatim}
    \end{quote}
   \item Either compile your code in the same directory as the included files
    are stored or define \verb|xxxx| as a logical name.
   \item On Unix call the INCLUDE file \verb|xxxx| (with no file extension) and
    compile your code in the same directory.
  \end{itemize}
  The INCLUDE files that are used when calling Starlink subroutine
  libraries have logical names defined on the VAXs so that, for example,
  SAE\_PAR.FOR is included with the statement:
    \begin{quote}
    \begin{verbatim}
      INCLUDE 'SAE_PAR'
    \end{verbatim}
    \end{quote}
  On Unix the corresponding file is called \verb|sae_par| and is stored in
  \verb|/star/include| along with all the other Starlink INCLUDE files so that
  the INCLUDE statement must be changed to:
    \begin{quote}
    \begin{verbatim}
      INCLUDE '/star/include/sae_par'
    \end{verbatim}
    \end{quote}
  The \verb|forconv| program
  described in SUN/111 will accomplish the conversion from VMS
  to Unix. The reverse operation can be done with a simple edit script.

  It is also possible, though not at present the recommended
  technique, to set up a {\it soft link}\, file pointing to the required
  INCLUDE file and to specify the name of the soft link file in
  the INCLUDE statement.
 \item ONE MODULE PER SOURCE FILE:~~
  Code that is going to be inserted into a subroutine library (a Unix
  {\it archive}) must have just one routine per source file before it is
  compiled.  This is because the Unix equivalent (\verb|ar|) of the VMS
  librarian does not split object files into separate modules when it inserts
  them into a library.  The Unix command \verb|fsplit| will split a Fortran
  source file into separate files.
 \item COMPILER SWITCHES:~~
  It is unwise to do anything requiring use of special compiler
  switches.  There are sure to be problems in the future when someone --
  not the original author -- compiles the program, in good faith,
  without the switch.  Examples are the switches that allow code
  to extend beyond column~72 (see section~2, rule~59) and
  to disable the special meaning of the backslash character.
 \item WHEN ALL ELSE FAILS:~~
   Unavoidable per-machine variations can be handled either by using the
   \verb|forconv| preprocessor (SUN/111) or by using separate files.
   Where the latter technique is used, a code identifying the platform
   should be appended to the name:

   \begin{tabbing}
   xxxxx \= xxxxxxxxxxx \= \kill
     \> {\it suffix} \> {\it platform} \\ \\
     \> \verb|_vax| \> VAX/VMS \\
     \> \verb|_sun4| \> Sun SPARCstation etc. \\
     \> \verb|_mips| \> DECstation etc. \\
     \> \verb|_pcm| \> PC/Microsoft \\
     \> \verb|_ind| \> platform-independent substitute
   \end{tabbing}

   and the file extension should identify the language in the normal way:
   \begin{tabbing}
   xxxxx \= xxxxxxxxxxx \= \kill
     \> {\it extension} \> {\it language} \\ \\
     \> \verb|.f| \> FORTRAN \\
     \> \verb|.c| \> C
   \end{tabbing}

   Thus, different versions of a Fortran routine \verb|fsub|
   for, respectively, VAX and Sun, would be \verb|fsub_vax.f|
   and \verb|fsub_sun4.f|.  Care must be taken not to exceed
   15 characters or \verb|ar| will truncate the file name.
\end{itemize}

\newpage
\appendix
\section{Reserved Facility Names}

The following is a list of packages used by existing Starlink software.
Software written by others than those actually responsible for developing or
maintaining these packages must {\bf not} use existing package names, even if
the new routines resemble or complement the official set.
Programmers may wish to submit their own package names for inclusion in the
following lists.
\begin{quote}
\begin{tabbing}
XXXXXXXXXXXX\=\kill
ADAM\>ADAM interface file handling\\
ADAMCL\>ADAM Command Language\\
ADC\>Astronomical Data Catalogues\\
AG*\>AGI + interfaces to graphics packages\\
AIF\>Auxiliary ADAM interface routines\\
ANT\>ADAM networking\\
ARGS\>ARGSLIB\\
ARY\>ARRAY-structure access\\
AST\>Asterix\\
CHA\>CHART part of SCAR\\
CHI\>Catalogue Handling Interface\\
CHP\>Character Handling ``Plus''\\
CHR\>Character handling\\
CHT\>CHART\\
CLV\>ADAM command language variable system\\
CMP\>HDS component handling\\
CNF\>C 'n' Fortran\\
CNV\>Figaro data-conversion\\
CON\>CONVERT package routines\\
CTASK\>ADAM Ctasks\\
CTM\>Colour-Table Management\\
DAT\>HDS and extensions to it\\
DAU\>HDS internal routines\\
DCV\>Data conversion\\
DIA\>DIAGRAM\\
DIP\>Internal system for DIAGRAM\\
DSA\>Figaro data-structure access\\
DSK\>Disc output for PGPLOT\\
DTA\>Figaro data-structure access\\
DTASK\>ADAM Dtasks\\
DYN\>Figaro dynamic memory routines\\
EMS\>Error Message Service\\
ENG\>ADAM Engineering interface\\
EOS\>Extendable Object System\\
ERR\>Error reporting system\\
EXC\>HDS internal routines\\
FIG\>Figaro general purpose\\
FIO\>File I/O\\
FIT\>FITS processing (including Figaro FITS)\\
FTS\>KAPPA FITS library\\
GEN\>General utility routines\\
GKD\>Graphics dialogue routines\\
GKS\>Graphics Kernel System\\
GNS\>Graphics Name System\\
GWM\>Graphics Window Manager\\
HDS\>Hierarchical Data System\\
HELPSYS\>ADAM help system\\
HLP\>Starlink portable help system\\
ICH\>ICL real-to-character conversion\\
ICL\>Interactive Command Language\\
IDI\>Image display interface to ADAM\\
IMG\>Simple Image Interface\\
IOC\>Low-level C magtape routines\\
IRA--IRZ\>IRAS software\\
KPE\>KAPPA environment packaging\\
KPG\>KAPPA general routines\\
KPS\>KAPPA specific routines\\
LEX\>ADAM command line parsing\\
LOCK\>ADAM file sharing system\\
LOG\>ADAM logging system\\
MAG\>Magnetic tape handling high level\\
MCH\>Machine-dependent constants etc.\\
MESSYS\>ADAM message system implementation\\
MGO\>MONGO routines\\
MIO\>Magnetic tape handling low level\\
MON\>ADAM monitor parameter system\\
MSG\>Message reporting system\\
MSP\>Message System Primitives\\
NBS\>NoticeBoard System\\
NDF\>NDF access\\
NUM\>Primitive data arithmetic\\
PAR\>ADAM parameter system interface\\
PARSECON\>ADAM interface file parsing\\
PRM\>PRIMDAT\\
PSX\>POSIX interface\\
REC\>HDS internal routines\\
REF\>HDS reference handling\\
RIO\>Random-access I/O\\
REPORT\>ADAM reporting system\\
SLA\>Subprograms mainly concerned with positional astronomy\\
SGS\>Simple Graphics System\\
SMS\>ADAM Screen Management System\\
SNX\>NCAR/SGS integration\\
SST\>Simple Software Tools\\
STRING\>String handling\\
SUBPAR\>ADAM parameter system implementation level\\
TAP\>Theoretical astrophysics library\\
TASK\>ADAM task control\\
TCV\>ICL type conversion\\
TEL\>UKIRT clock handling\\
TPT\>TPOINT\\
TRA\>Trace\\
TRN\>TRANSFORM coordinate transformation facility\\
UFACE\>ADAM User Interface\\
UNI\>Unit conversion utilities\\
UTIL\>ADAM/VMS utility\\
VAL\>Primitive data arithmetic\\
VAR\>Figaro user variable routines\\
VEC\>Primitive vector arithmetic\\
VIO\>VDU I/O (Used by SCAR)
\end{tabbing}
\end{quote}
\goodbreak
In addition to the above, the following blocks of facility names
are reserved by Starlink:
\begin{quote}
\begin{tabbing}
XXXXXXXXXXXX\=\kill
DSx\>x = 0-9,A-Z\\
SLx\>x = 0-9,A-Z
\end{tabbing}
\end{quote}
\goodbreak
Though not strictly facility names, the following prefixes are used internally
within ADAM facilities and should be avoided:
\begin{quote}
\begin{tabbing}
XXXXXXXXXXXX\=\kill
ACT\>ADAM task activation errors\\
ADM\>ADAM general errors\\
PARSE\>PARSECON errors\\
SAI\>Starlink Applications Interface errors\\
USER\>Error values available for users\\
VAL\>Special values
\end{tabbing}
\end{quote}
\end{document}
