\documentstyle[11pt]{article}
\pagestyle{myheadings}

%------------------------------------------------------------------------------
\newcommand{\stardoccategory}  {Starlink General Paper}
\newcommand{\stardocinitials}  {SGP}
\newcommand{\stardocnumber}    {42.1}
\newcommand{\stardocsource}    {sgp42.1}
\newcommand{\stardocauthors}   {R.F.~Warren-Smith}
\newcommand{\stardocdate}      {18th May 1995}
\newcommand{\stardoctitle}     {Starlink Software Strategy}
%------------------------------------------------------------------------------

\newcommand{\stardocname}{\stardocinitials /\stardocnumber}
\renewcommand{\_}{{\tt\symbol{95}}}     % re-centres the underscore
\markright{\stardocname}
\setlength{\textwidth}{160mm}
\setlength{\textheight}{230mm}
\setlength{\topmargin}{-2mm}
\setlength{\oddsidemargin}{0mm}
\setlength{\evensidemargin}{0mm}
\setlength{\parindent}{0mm}
\setlength{\parskip}{\medskipamount}
\setlength{\unitlength}{1mm}

%------------------------------------------------------------------------------
% Hypertext definitions.
% These are used by the LaTeX2HTML translator in conjuction with star2html.

% Comment.sty: version 2.0, 19 June 1992
% Selectively in/exclude pieces of text.
%
% Author
%    Victor Eijkhout                                      <eijkhout@cs.utk.edu>
%    Department of Computer Science
%    University Tennessee at Knoxville
%    104 Ayres Hall
%    Knoxville, TN 37996
%    USA

%  Do not remove the %\begin{rawtex} and %\end{rawtex} lines (used by 
%  star2html to signify raw TeX that latex2html cannot process).
%\begin{rawtex}
\makeatletter
\def\makeinnocent#1{\catcode`#1=12 }
\def\csarg#1#2{\expandafter#1\csname#2\endcsname}

\def\ThrowAwayComment#1{\begingroup
    \def\CurrentComment{#1}%
    \let\do\makeinnocent \dospecials
    \makeinnocent\^^L% and whatever other special cases
    \endlinechar`\^^M \catcode`\^^M=12 \xComment}
{\catcode`\^^M=12 \endlinechar=-1 %
 \gdef\xComment#1^^M{\def\test{#1}
      \csarg\ifx{PlainEnd\CurrentComment Test}\test
          \let\html@next\endgroup
      \else \csarg\ifx{LaLaEnd\CurrentComment Test}\test
            \edef\html@next{\endgroup\noexpand\end{\CurrentComment}}
      \else \let\html@next\xComment
      \fi \fi \html@next}
}
\makeatother

\def\includecomment
 #1{\expandafter\def\csname#1\endcsname{}%
    \expandafter\def\csname end#1\endcsname{}}
\def\excludecomment
 #1{\expandafter\def\csname#1\endcsname{\ThrowAwayComment{#1}}%
    {\escapechar=-1\relax
     \csarg\xdef{PlainEnd#1Test}{\string\\end#1}%
     \csarg\xdef{LaLaEnd#1Test}{\string\\end\string\{#1\string\}}%
    }}

%  Define environments that ignore their contents.
\excludecomment{comment}
\excludecomment{rawhtml}
\excludecomment{htmlonly}
%\end{rawtex}

%  Hypertext commands etc. This is a condensed version of the html.sty
%  file supplied with LaTeX2HTML by: Nikos Drakos <nikos@cbl.leeds.ac.uk> &
%  Jelle van Zeijl <jvzeijl@isou17.estec.esa.nl>. The LaTeX2HTML documentation
%  should be consulted about all commands (and the environments defined above)
%  except \xref and \xlabel which are Starlink specific.

\newcommand{\htmladdnormallinkfoot}[2]{#1\footnote{#2}}
\newcommand{\htmladdnormallink}[2]{#1}
\newcommand{\htmladdimg}[1]{}
\newcommand{\externallabels}[2]{}
\newcommand{\externalref}[1]{}
\newcommand{\html}[1]{}
\newenvironment{latexonly}{}{}
\newcommand{\latex}[1]{#1}
\newcommand{\hyperref}[4]{#2\ref{#4}#3}
\newcommand{\htmlref}[2]{#1}
\newcommand{\htmlimage}[1]{}
\newcommand{\htmladdtonavigation}[1]{}

% Starlink cross-references and labels.
\newcommand{\xref}[3]{#1}
\newcommand{\xlabel}[1]{}

%------------------------------------------------------------------------------
%  Debugging.
%  =========
%  Un-comment the following to debug links in the HTML version using Latex.

%\newcommand{\hotlink}[2]{\fbox{\begin{tabular}[t]{@{}c@{}}#1\\\hline{\footnotesize #2}\end{tabular}}}
%\renewcommand{\htmladdnormallinkfoot}[2]{\hotlink{#1}{#2}}
%\renewcommand{\htmladdnormallink}[2]{\hotlink{#1}{#2}}
%\renewcommand{\hyperref}[4]{\hotlink{#1}{\S\ref{#4}}}
%\renewcommand{\htmlref}[2]{\hotlink{#1}{\S\ref{#2}}}
%\renewcommand{\xref}[3]{\hotlink{#1}{#2 -- #3}}
%------------------------------------------------------------------------------
% Add any document specific \newcommand or \newenvironment commands here

\newcommand{\hi}[1]{{\tt{#1}}}
\newcommand{\qt}[1]{``#1''}
\newcommand{\st}[1]{{\em{#1}}}
\begin{htmlonly}
   \renewcommand{\qt}[1]{{\tt{"}}#1{\tt{"}}}
%   \renewcommand{\_}{\_}
\end{htmlonly}
\newcommand{\dev}[1]{\htmlref{#1}{development}}
\newcommand{\dis}[1]{\htmlref{#1}{distribution}}
\newcommand{\mnt}[1]{\htmlref{#1}{maintenance}}
\newcommand{\spt}[1]{\htmlref{#1}{support}}

\newcommand{\ssgurl}[0]{http://star-www.rl.ac.uk/\~{}rfws/projects/ssginput.html}
\newcommand{\prgurl}[0]{http://star-www.rl.ac.uk/\~{}rfws/projects/index.html}
\newcommand{\linuxurl}[0]{http://bima.astro.umd.edu/nemo/linuxastro/linux.html}
\newcommand{\aipsurl}[0]{http://info.cv.nrao.edu/aips++/docs/html/aips++.html}
\newcommand{\midasurl}[0]{http://http.hq.eso.org/midas-info/midas.html}
\newcommand{\irafurl}[0]{http://iraf.noao.edu/iraf-homepage.html}
\newcommand{\ssgref}[1]{\htmladdnormallink{#1}{\ssgurl}}
\newcommand{\prgref}[1]{\htmladdnormallink{#1}{\prgurl}}
\newcommand{\mgrref}[1]{\xref{#1}{sgp25}{}}
\newcommand{\usscref}[1]{\xref{#1}{sun1}{}}
\newcommand{\figaroref}[1]{\xref{#1}{sun86}{}}
\newcommand{\ndffmtref}[1]{\xref{#1}{ssn20}{}}
\newcommand{\aipsref}[1]{\htmladdnormallink{#1}{\aipsurl}}
\newcommand{\midasref}[1]{\htmladdnormallink{#1}{\midasurl}}
\newcommand{\irafref}[1]{\htmladdnormallink{#1}{\irafurl}}
%------------------------------------------------------------------------------
%  Title page.
%  ==========
\begin{document}
\thispagestyle{empty}
\begin{htmlonly}
   \xlabel{}
   \label{stardoctoppage}
   \begin{rawhtml} <H1> \end{rawhtml}
      \stardoctitle
   \begin{rawhtml} </H1> \end{rawhtml}

%  Add picture here if appropriate.

   \begin{rawhtml} <P> <I> \end{rawhtml}
   \stardoccategory \stardocnumber \\
   \stardocauthors \\
   \stardocdate
   \begin{rawhtml} </I> </P> <H3> \end{rawhtml}
      \htmladdnormallink{CCL}{http://www.clrc.ac.uk/} /
      \htmladdnormallink{Rutherford Appleton Laboratory}
                        {http://www.clrc.ac.uk/ral/index.html} \\
      Particle Physics \& Astronomy Research Council \\
   \begin{rawhtml} </H3> <H2> \end{rawhtml}
      \htmladdnormallink{Starlink Project}{http://star-www.rl.ac.uk/}
   \begin{rawhtml} </H2> \end{rawhtml}
   \htmladdnormallink{\htmladdimg{source.gif} Retrieve hardcopy}
      {http://star-www.rl.ac.uk/cgi-bin/hcserver?\stardocsource}\\
\end{htmlonly}
\begin{latexonly}
   CCL / {\sc Rutherford Appleton Laboratory} \hfill {\bf \stardocname}\\
   {\large Particle Physics \& Astronomy Research Council}\\
   {\large Starlink Project\\}
   {\large \stardoccategory\ \stardocnumber}
   \begin{flushright}
   \stardocauthors\\
   \stardocdate
   \end{flushright}
   \vspace{-4mm}
   \rule{\textwidth}{0.5mm}
   \vspace{5mm}
   \begin{center}
   {\Large\bf \stardoctitle}
   \end{center}
   \vspace{5mm}
\end{latexonly}

%  Table of contents.
%  =================
\htmladdtonavigation{\htmlref{\htmladdimg{contents_motif.gif}}
                                         {stardoctoppage}}

%------------------------------------------------------------------------------
%  Introduction page.
%  =================

\section{INTRODUCTION}

This document contains the text of a paper presented to the Starlink
Panel in March 1995 and describes a strategy for Starlink software
activities over the next 3 years. It is based heavily on the input
provided by the 1993-94 Starlink Software Survey (SGP/43) and
subsequent discussions at Starlink Software Strategy Group
(SSG) meetings during late 1994.

\subsection{Definition of Terms}

The terms \qt{maintenance}, \qt{support}, \st{etc.}\ are used with a
variety of meanings in software discussions and this can potentially
lead to misunderstanding. In this paper, the generic term \qt{support}
is used in its loose sense, but italics are also used where necessary
to indicate the following specific meanings:

\begin{description}
\item[\label{distribution}Distribution:]
Assessing software, checking that it builds and runs on supported
platforms and meets rudimentary \qt{health checks}, packaging it to
facilitate distribution and installation (by \mgrref{Site Managers} and
others), and copying and distributing the documentation.

\item[\label{support}Support:]
Maintaining an awareness of what software items are available and how
they work, and providing advice on their use and diagnosis of problems
that may occur. Suggesting work-arounds where necessary.

\item[\label{maintenance}Maintenance:]
Repairing software when it is found not to behave as advertised or
expected. Includes bug fixing and updating of items that go out of
date or cease to function properly due to external changes.

\item[\label{development}Development:]
Adding new capabilities to software which go beyond its previously
advertised functionality. Includes writing new software from scratch
and writing substantially new documentation.

\end{description}

\section{CURRENT PROBLEMS AND TRENDS}

\subsection{\label{started}Documentation and Ease of Software Use}

One of the clearest messages to emerge from the 1993/94 Starlink
Software Survey was the need to improve the ease with which the
average user can learn about and utilise Starlink software. Issues
relating to this dominated the priority list for ways to improve both
Starlink software and documentation, and there appear to be a number
of closely related requirements:

\begin{enumerate}
\item To improve the \qt{friendliness} of software and the ease with
which people can understand and make use of it.

\item To improve the documentation so that it is both more accessible
in itself, and also allows users to identify the software they require
more easily.

\item To provide \qt{cookbooks} and examples of software use that allow
users to perform standard data reduction tasks by simply following a
pre-defined recipe without needing to become too expert in the
technical aspects of the processing.

\item To help users choose which tools best suit their purposes.
\end{enumerate}

In contrast to these issues, more traditional priorities such as
software performance, reliability, and the addition of new processing
capabilities must clearly be regarded as less important at present (we
expect them to re-emerge as priorities once the above problems have
been tackled).

To respond to this, we have begun \prgref{a programme} to address the
objective of improving documentation and software presentation, with
ease of use primarily in mind.  Given the amount of documentation and
software in use, it will require a sustained effort over a number
years before the problem is completely solved.  However, new
technologies such as hypertext (\st{e.g.}\ the World Wide Web) and
tools for developing more intuitive Graphical User Interfaces (GUIs)
make this a timely development and should allow us to make substantial
improvements within the next few years (see \S\ref{infra}).

\subsection{Software Diversity}

A theme which was much discussed at SSG meetings was the
increasing need to make use of many different software packages in
order to access the most powerful facilities at each stage of data
processing.\footnote{For instance, the Spectroscopy SSG identified no
less than 13 major software packages in active use for spectroscopic
data reduction (doubtless still overlooking a few) and most other
groups could have made similar lists. All of them felt that this
software diversity should be regarded as a valuable resource and be
exploited.}  The Software Survey also provides ample evidence of this
in the wide range of software packages in use.

The Project believes that this trend is extremely important and likely
to continue or even accelerate with the much easier access to software
from around the world that modern network tools provide. As a result,
good new software can find its way into widespread use remarkably
quickly, replacing previous methods.

It is easy to see the advantages of exploiting this (usually) free
software, much of it written by acknowledged experts who would not be
available to write software for Starlink.  Very often, the support
available is also good, with active user groups and regular postings by
knowledgeable people.

By depending on an international \qt{software market} such as this, one
also hopes to benefit from evolutionary forces which will, over time,
improve software quality and establish standards. If we wish to
benefit from this, our approach to software must be sufficiently
flexible to adapt to new approaches as they emerge.

\section{DISCUSSION}

\subsection{Implications for Starlink}

Traditionally, developers of astronomical software have made their
packages perform a wide range of tasks, making them \qt{complete} and
independent of other packages. However, such an approach is becoming
increasingly difficult to sustain because the resulting software is
large and inflexible, and can respond only slowly to new
expectations. The weaker areas of large packages become a burden and
can swiftly be overtaken by new, more specialised tools. The
increasing difficulty of maintaining large packages in the face of
such developments is causing users to diversify the range of software
they use -- despite the barriers they face when moving from one
software package to another.

Because of these weaknesses in the traditional approach, we believe
the trend towards freer exchange of software will be a very difficult
one to ignore, although the consequent diversity will be costly in
support effort.  In this situation, we do not believe it is good use of
Starlink's resources to be \st{\mnt{maintaining}} the weaker areas of any
large software package (\st{i.e.}\ simply bringing them up to par)
when there is already better software available. A better strategy is
to ensure that we can adapt to use the new software as conveniently as
possible. This is not to say that we should avoid \st{\dev{developing}} new
software in areas where it already exists, but that if we do, our
endeavours must be sufficiently well-focussed that our products are
{\bf significantly better} than the alternatives -- simply matching
them is not good enough.

This approach requires a clear recognition of the international
context of our work and a determination to concentrate resources where
they will be most effective in this context. It also means recognising
that increasing software diversity is something we must accept as a
price for the advantages of free software exchange. We must then find
ways to exploit this diversity and to manage the problems that it
brings.

\subsection{How Far should we Specialise?}

In many enterprises, the emergence of more open competition is a
signal to specialise. For Starlink, however, the degree to which we
should do this not so clear. Although we are a software supplier and
want to see our products succeed, we also provide a computing service
for UK astronomy which involves, almost uniquely,
\st{\dis{distributing}} and \st{\spt{supporting}} a great deal of
software which we have not ourselves developed. These two functions
are somewhat at odds.

If we were to concentrate all our work in a particular area (say
optical spectroscopy) we would doubtless be able to \st{\dev{develop}}
rather better software and offer far better \st{\mnt{maintenance}} and
\st{support}, but this would be at the price of withdrawing almost
completely from other areas of astronomy.  We believe this would
undermine much of the cross-discipline support that makes Starlink
unique and accounts for its success.

On the other hand, if we spread our resources evenly between every
available item of software, we would not have enough people even to
understand how it all works. Our software \st{development} would then
cease and our \st{maintenance} and \st{support} activities would be
reduced to minor tinkering -- contributing very little of value. With
this loss of value, our reasons for providing software
\st{\dis{distribution}} would also be eroded.

We would prefer to maintain a broad involvement in most areas of
astronomy, with the aim of providing UK astronomers with a choice of
the most up to date and powerful software for their data reduction
irrespective of where it comes from, together with advice on how to
use it. However, we need to concentrate our \st{development} effort
more if we are to produce world-class software capable of featuring
amongst the choices available.

\subsection{Finding our Place in the World}

What we propose, therefore, is the maintenance of a broad software
provision at the level of \st{\dis{distribution}} so that the scope of our
computing service does not change greatly, but with an increasing
degree of specialisation as we move through \st{\spt{support}},
\st{\mnt{maintenance}} and \st{\dev{development}} activities. This continues a
trend that has existed since the 1980s.

A consequence of this is that we see a future of increasing dependence
on software \st{developed} by other people and on the \st{support}
and \st{maintenance} they provide.  Our own \st{support} activities
will then need to shift towards making the resulting more diverse
software collection easier to use.  With these changes, there will be
less need to \st{develop} or \st{maintain} software for such a wide
range of purposes, and this will give us scope to concentrate our
\st{development} efforts on areas where we can make world-class
contributions.

We do not, however, anticipate complete withdrawal of our
\st{development} work from whole areas of astronomy.  Rather, we
envisage limiting the breadth of our \st{development} projects and
employing a finer level of granularity, so that they are not (as in
the past) all-embracing enterprises covering whole disciplines but
well-focussed projects aimed at specific problems. This might often
involve working at the level of individual applications rather than
entire packages.  In doing this, we must be prepared in many areas to
defer to non-UK workers, but we should also be prepared vigorously to
defend those areas where we have the best expertise and which are of
particular importance to UK astronomy.

The following sections discuss some of the practical implications of
these proposals.

\section{PROPOSED STRATEGY}

\subsection{Distribution}

Our software \st{\dis{distribution}} should, in future, be directed towards
a broad level of software provision for all currently-supported areas
of UK astronomy, but we should achieve this by depending to a greater
extent than in the early days on software from non-UK sources,
continuing the trend of the past few years.

Where software is already freely available from sources outside
Starlink, we should regard its re-distribution through Starlink as
an opportunity to add value. Typically this might consist of providing
UK-specific \st{\spt{support}} (over and above that available from its
suppliers), or the ability to make a particular trustworthy version
easily available so that it is standard at Starlink sites.  There are
many other pitfalls associated with using free software, and these
would also provide opportunities for Starlink to contribute.

We should adopt an increasingly international approach to distributing
software developed in the UK, involving at least:
\begin{itemize}
\item Configuring the software so that there is nothing \qt{special}
or restrictive about it that will prevent it being easily installed
and used at non-Starlink sites, and
\item Developing an automatic software distribution mechanism (based on the
WWW) that will make it readily available anywhere in the world.
\end{itemize}

These moves should be supported by long-overdue revised statements of
our \htmlref{international software distribution and support
policy}{international}\begin{latexonly} (see
below)\end{latexonly}. This would include any copyright and licensing
restrictions, \st{etc.}, which should be kept to a minimum.

We expect the \usscref{UNIX Starlink Software Collection} to continue
to grow in size, as there is \htmlref{little benefit}{USSC} in
attempting to reduce its size for its own sake\begin{latexonly} (see
\S\ref{USSC})\end{latexonly}. However, when considering new platforms,
we should examine each software item on its merits before porting it,
and expect the collection to be reduced in size as a result.

\subsection{Support and Maintenance Priorities}

We favour \htmlref{prioritising}{priority} the majority of our
software work in the form of specific projects\begin{latexonly} (see
\S\ref{priority})\end{latexonly}, but there will always be a spectrum
of \st{\mnt{maintenance}} and \st{\spt{support}} activities that
either cannot be foreseen or which are too small to warrant
consideration as individual projects.  The following describes current
practice in this area which is, we think, well adapted to future
needs:

\begin{enumerate}
\item Allocate an annual manpower budget to \qt{baseline} \st{support} and
\st{maintenance}. This governs the overall amount of work undertaken
in this category.

\item Allow this work to be event-driven by user requests. This
ensures that we concentrate only on problems that cause real
difficulty in software that people actually use.

\item Use discretion in deciding to what extent any particular request
requires action. This helps relate the response to the seriousness of
the problem.

\item If demand exceeds the allocated manpower, then decide priorities on
the basis of software usage statistics. This helps to ensure that the
minimum number of people are kept waiting.

\item If the work involved in a request is clearly too large to handle
in this way, then regard it as an identifiable project, to be
considered in competition with other projects (see \S\ref{priority}).

\end{enumerate}

The budget allocated to these activities in the current year is around
25\% of our total software manpower. This means that routine
\st{support} and \st{maintenance} is a significant but minor activity
-- the majority of our work being \st{\dev{development}}.

In line with the increasing dependence on non-UK software, we should in
future also depend increasingly on the associated external sources of
technical software \st{support}. Our own \st{support} activity must
instead concentrate on the added value of providing indexing,
cookbooks and advice, \st{etc.}\ to help users find the software they
need and learn how to use it (see \S\ref{infra}).  The wider diversity
of software which we anticipate will therefore probably increase our
\st{support} costs.

It will however, be very difficult to provide \st{maintenance} (such
as bug fixes) for other people's software and we should attempt to do
this only in exceptional circumstances, depending instead on the
original suppliers for this \st{maintenance} work. Consequently, we
would expect our \st{maintenance} costs to fall.

\label{international}To encourage wider use of UK-developed software,
we should extend our \st{support} and \st{maintenance} services to
non-UK users, but without guarantee and only when the
\htmlref{benefits}{foreign} are in balance with the small extra costs
involved\begin{latexonly} (see \S\ref{foreign})\end{latexonly}.

\subsection{\label{infra}Key Infrastructure Developments}

Our proposal to accept a more diverse range of software and to rely
increasingly on external suppliers means accepting that
rationalisation of astronomical functionality (so that a user need not
trudge from one package to another in search of the best applications)
is unlikely to happen in the near future. Some users may regret this,
but it seems an unavoidable conclusion given the many different
sources of software that now exist.  There will, as a result, be an
increased need for help in finding the best item of software, learning
how to use it, and facilitating the transfer of information between
packages. This trend is, of course, already apparent and brings us
back to \hyperref{where we started}{where we started (}{)}{started}.

We believe there are a number of promising approaches to help
alleviate these difficulties and to provide data reduction facilities
that are considerably more uniform and easier to use than can be
achieved simply by copying software from the Internet:

\begin{description}
\item[Improved Documentation:] We plan that a great deal of the guidance
about which software to use for a particular purpose should come from
documentation in a \qt{cookbook} style, providing recipes for common
types of data reduction. While we expect opinions to continue to
differ about which software is best, this will at least allow us to
provide recommendations that can identify \qt{primary} packages for
particular purposes, chosen freely from what is available.

We anticipate using hypertext extensively in this work because:
\begin{itemize}
\item It offers better opportunities to present pictorial and other
information in an easily accessible way.
\item It presents the user with an integrated documentation system in
which the boundaries between individual documents are hidden even
though they may actually come from disparate sources.
\item It opens up new indexing and searching possibilities.
\item It offers significant maintenance advantages because
cross-linking between documents results in far less duplication of
material.
\end{itemize}

\item[User Interfaces:] By designing new user interfaces
(often graphical) for specific purposes, it should be possible to
encapsulate software items from different sources (or items that are
currently hard to use) so that they are presented to a user as a
single coherent and intuitive system. In effect, this is an
\qt{interactive cookbook} that helps guide the inexperienced user
through the necessary processing and isolates him from the underlying
software, while not preventing the more knowledgeable from accessing
it directly.

\item[Interoperability:] We believe there is much we can do to improve
the ease with which the applications software we develop can be used
alongside software from other sources. This issue of
\qt{\htmlref{interoperability}{interop}} also has a number of other
benefits\begin{latexonly} (see \S\ref{interop})\end{latexonly} which
will help to make our software more relevant to the international
context and to users who are already familiar with different systems.

\end{description}

We regard these developments as very important because the assistance
we provide and the way that the \usscref{Starlink Software Collection}
is distributed, organised, presented and described will in future
constitute a larger part of the service we offer.

We expect to spend about one year developing and testing the
techniques described above and implementing a number of pilot projects
that make use of them. This will be followed by approximately two
years in which we put these into use to provide hypertext
documentation, cookbooks and improved user interfaces and
interoperability for a range of astronomical data reduction purposes.

\subsection{\label{priority}Identifying Development Priorities}

Going beyond the infrastructure developments above, which are designed
to make future software support tractable, we must also identify the
astronomical application areas that we will address.

For reasons outlined in \S\ref{concentrate}, we consider it unwise to
allocate software effort uncritically to \qt{packages}, as it is far too
hard to control their evolution, and this approach too easily becomes
a commitment to develop a wide range of software of arguable quality
across a whole discipline.  A better strategy for using resources
effectively is to \st{identify the functionality required} and then
\st{examine the best way of obtaining it}. In this way we can clearly
prioritise our objectives and need not expend effort on activities
that are irrelevant to achieving them. Developing existing packages
is, of course, not ruled out by this.

The system introduced last year, of identifying software plans
annually based on input from users and SSG meetings was
intended to identify specific high-priority software objectives and to
ensure that these objectives remain relevant with the passage of time.
This seems to have functioned quite well so far, although there is
still scope for increasing user involvement.

In assessing how to achieve the objectives that have been identified,
one must consider:
\begin{enumerate}
\item Identifying and using software that already exists,

\item Making enhancements to existing software and/or documentation,

\item Assembling a new facility by drawing components from other
existing systems,

\item Writing new material where there is no suitable basis from which
to start.
\end{enumerate}

Normally, one would expect the priority order to be that
above. However, the best approach will often hinge on complex issues
involving user requirements (and prejudice), technical considerations
and knowledge of the available products. As weeks or months of
investigation may be involved, this work has to be regarded as part
of the design phase of each project and handled by the software staff
who will implement it.

In delegating this work, the Project is well aware that critical
decisions should not rest with software implementors alone and that a
thorough and objective investigation of the alternatives must be
carried out.  A important part of this should be a survey of the
opinions of users who already use related software and will be the
likely users of the new product, and the Project's responsibility lies
in ensuring that this investigation is thorough and appropriate to the
project being undertaken. We have already developed resources such as
mailing lists to help us in this, and intend to monitor their
effectiveness and extend them as necessary.

\subsection{Concentrating Development Activity}

A further consideration is the timeliness and quality of our software
development, since very often a long development time can render a
product obsolete before it is finished. It may also be necessary to
devote more than a single person to a project if we are to improve on
what is already available (there is a limit to what a single person
can achieve).

For this reason, we favour identifying key projects, when necessary, on
which resources can be concentrated in order to achieve a high-quality
result over a suitably short period. We imagine that such projects
might be allowed to consume up to about 50\% of software manpower for
short periods (1 or 2 years, with annual reviews).

Clearly, such major projects can only be undertaken where the need is
clear and the benefits substantial.  The area must be sufficiently
narrow to be tractable, but with more than niche appeal.  Users and
SSGs should be informed that such possibilities exist, and that they
should consider proposing such projects where they are justified. It
would lie with the Starlink Panel to give final approval.

\section{FURTHER CONSIDERATIONS}

\subsection{Future Programming Languages}

It appears that our users' programming language requirements are
evolving only rather slowly at present, with little interest in new
languages for their own sake.  Most people seem to feel comfortable
with Fortran~77 and can do most of what they want with it (although
they would naturally like to be able to do it more easily, if they
could). There is some interest in C as an alternative, and a little
interest in Fortran~90 (mainly from the Theory community for
compatibility with other computing services).

Internationally, however, use of Fortran~77 seems to be declining
(certainly in the USA) and software written in it is becoming
noticeably less acceptable to projects evaluating products for their
own use. The most internationally-acceptable language at present is
undoubtedly C.

The \aipsref{AIPS++} group (and also, perhaps, \midasref{MIDAS}) are
looking towards C++ as the language of the future, and it is certainly
gaining popularity in the USA (outside astronomy) and in
commerce. However, there seems to be little evidence that these moves
are being driven by astronomical users -- indeed, there are serious
worries that the novelty and additional complexities of C++ may prove
unacceptable to many of these.  Instead, it seems to be mainly system
implementors who are increasingly having to cope with greater software
complexity that demands this radical change. With much existing
Fortran expertise in the UK, Fortran~90 may be an attractive
alternative to C++ (it has comparable capabilities although it lacks
standard interfaces to many widely-used libraries and operating
systems). However, neither is really yet in wide enough use in
scientific circles to be sure.

At present, Starlink provides facilities for writing applications only
in Fortran~77 but we do not expect this to remain adequate for long.
A significant fraction of our infrastructure software is already
implemented in C and it is our policy to use C exclusively for this
purpose in future.

Moving wholesale to either C++ or Fortran~90 (particularly the former)
would probably be very expensive for us: it could easily consume most
of our software effort for several years adapting to the new
programming paradigm, establishing new standards and re-educating
programmers and users. We do not feel that such a drastic move is yet
justified by user demand, and think it unlikely to be necessary in the
near future. Nevertheless, a measured involvement in investigating
these languages is warranted and may perhaps progress to a more
concrete proposal in time.

A more modest but practical alternative is a migration towards C
programming interfaces and, increasingly, writing applications in C
over the next 2 or 3 years. This is an achievable goal that would
provide worthwhile benefits for programmers as well as addressing the
need for more internationally accepted practice in our software work.

\subsection{New Platforms}

Sun and DEC seem likely to continue to provide competitively priced
workstation hardware in the near future, and this will probably be in
wide use in the international astronomical community. Hence, we do not
currently anticipate any major porting of software to new workstation
platforms.

There is, however, a growing recognition of the capabilities and
advantages of PC hardware. The availability of
\htmladdnormallinkfoot{Linux}{\linuxurl} (a free UNIX operating system
for PCs) and, very recently, a Fortran compiler for it means that many
more astronomers around the world will be able to use PCs in future
for their work. We therefore plan to introduce provision of Starlink
software for such machines over the next 2 years (although initially
at a lower level than on workstations). We have already embarked on
the first step, of porting the Starlink infrastructure software.

In the longer term, there may be much to gain from greater
exploitation of commercial software written for the PC market. In
particular, the high quality and relatively low price of this software
should allow us to make far better provision for (\st{e.g.}) text
processing and database/tables handling than we can at present. The
current obstacle is the lack of an accepted multi-tasking operating
system that will support both astronomical data reduction software and
commercial applications while running on widely-available hardware.
When this problem is solved (as seems likely in the next few years),
the potential benefits mean that we should then seriously consider
providing software to run on whatever turns out to be the next
successful PC operating system.

\section{SUMMARY -- KEY POINTS}

To summarise, the key points of the strategy we propose are:

\begin{enumerate}

\item To recognise the international context of our work and
the need to depend increasingly on software originating outside the
UK.

\item To maintain a broad level of software provision in the form of
\st{\dis{distribution}} in all current areas of UK astronomy, but to
introduce an increasing level of specialisation as we move through
software \st{\spt{support}}, \st{\mnt{maintenance}} and \st{\dev{development}}.

\item To depend increasingly on the \st{support} and \st{maintenance}
services provided by others, and to shift our own \st{support}
services towards managing the resulting increase in software diversity
and helping users to exploit the range of software available
(through cookbooks, \st{etc.}).

\item To concentrate our \st{maintenance} services (\st{e.g.}\ bug-fixing) on
software developed within the UK and to extend our \st{support} and
\st{maintenance} services for this software to non-UK users in a
limited form, so as to promote international use.

\item To focus our astronomical software \st{developments} in areas where
we can make world-class contributions or where no alternative software
exists and, where necessary, to devote larger proportions of our
development effort to key applications projects.

\item To develop key infrastructure technologies (\st{e.g.}\ automatic
software distribution, hypertext documentation, GUIs and
interoperability) to support the activities above.
\end{enumerate}

\newpage
\appendix
\section{FREQUENTLY ASKED QUESTIONS}

The following addresses a few frequently asked questions whose answers
are relevant to the arguments presented in this paper.

\subsection{\label{USSC}Does the Starlink Software Collection Need to
be so Big?}

{\Large \bf Q:} The \usscref{Starlink Software Collection} (USSC)
contains a large number of items. Surely these can't all be needed?
Wouldn't it be less expensive if we reduced its size?

{\Large \bf A:} Only a small subset of the USSC routinely demands
any \st{\mnt{maintenance}} or \st{\dev{development}} activity, so the costs associated
directly with its size are mainly those of \st{\dis{distribution}}. Since only
the most actively-used packages tend to be regularly updated and
re-installed at sites, this is where most of the effort is expended.

User surveys have always shown that usage of Starlink software is very
broadly distributed, with all items being used to some extent. There
is no clear cut-off between heavily-used and under-used items, so it
is not easy to remove items from the USSC without inconveniencing a
significant number of people. In some cases items have fallen out of
use through obsolescence and have been removed.  Since they had
remained untouched for years, however, the manpower saving was
negligible.  A more aggressive stance on deleting little-used items
would save very little manpower, as these items demand little
attention anyway. Indeed, supporting users who were inconvenienced by
software deletions would probably result in an overall increase in our
costs.

There is, in conclusion, little reason to prevent little-used software
items from remaining in the collection as they consume negligible
resources. There is, however, good reason to examine carefully the
need to make any new software item available, as it is the initial
release of any item which is usually the most time consuming.

\subsection{\label{duplication}Why is there so much Duplication?}

{\Large \bf Q:} The \usscref{Starlink Software Collection} contains
many alternative ways of doing the same thing. Surely this is
wasteful? Why can't it be rationalised by eliminating duplication to
save on support costs?

{\Large \bf A:} If we had planned and written the USSC as it currently
exists, it would, indeed, have been wasteful to have incorporated so
much duplication (to say nothing of the confusion and difficulty this
causes for users). However, it did not happen that way.

The vast majority of this software has not been written by Starlink
but was contributed by various authors over more than a decade.  When
it was submitted to Starlink its authors had already chosen how
compatible it should be with other software, and to what extent to
introduce duplication.  Unfortunately, the desire of many authors to
make their software independent of others has tended to drive them
towards duplicating features available elsewhere. Starlink has always
discouraged this practice and extolled the virtues of software
compatibility, but most of this work has taken place outside
Starlink's control.

\label{illfeeling}Having received such software, and appreciating that
it is valuable, Starlink is faced with two options:

\begin{enumerate}
\item Attempt to rationalise it by transferring its valuable functions
into other packages which we already distribute, or

\item Accept the duplication and the costs of distributing the new
software as it stands.
\end{enumerate}

Experience of attempting the former approach has shown that it nearly
always takes very much more effort than expected and succeeds in
alienating both the author (because the original character of the
software rarely survives) and potential users (who suffer considerable
delays in receiving the software). Overall, the latter approach has
proved by far the more acceptable and cheaper option.

It is not hard to see that attempts to rationalise software become
even more infeasible once the increasing amount of free software
available on the Internet is considered.  Starlink simply cannot hope
to assimilate the entire software output of the rest of the world's
astronomers in this way. It seems that duplication is an inevitable
consequence of the free exchange of software.

\subsection{\label{concentrate}Why not Concentrate on just a few Packages?}

{\Large \bf Q:} Starlink supports\footnote{Note we are really
discussing \st{\dev{development}} work here rather than \st{\spt{support}} --
people typically want new things added to the software they use.} so
many packages.  Wouldn't it be better to concentrate on just a few and
make them really good?

{\Large \bf A:} It is a common notion that most astronomers do much the
same operations on their data, so it should be possible to identify
whichever package is best at it, and concentrate on that one.

In reality, however, there is almost no agreement amongst Starlink
users about which software is best for any particular purpose.  People
want different things and choose the software that provides them.
There is plenty of evidence for this in the Starlink Software Survey
results, which show a very wide spread of usage between different
packages, and it becomes extremely obvious if you visit a range of
Starlink sites or gather a group of users in a room and ask them to
agree.

This means that whichever package you decided to concentrate on, you
would have to abandon a large number of users who needed features that
it did not provide. It then becomes difficult to resist the demand to
transfer these features into the preferred package. Performing this
rationalisation and supporting the affected users takes a great deal
of effort and generates ill feeling (see the \htmlref{previous
question}{illfeeling}). Although the resulting package may be a
little less work to support, the potential barrier to achieving this
is substantial and the net result in terms of increased functionality
is nil. By the time the work is finished, some better product is quite
likely to have come along.

Another difficulty is the \qt{umbrella effect}. Suppose we had just
designated \figaroref{Figaro} as the only package to be supported by
Starlink. An excellent package in many ways, Figaro is nevertheless
large with many old and weaker features that it would be of doubtful
value to support (given that better software exists elsewhere). By
sheltering under the Figaro umbrella, these weaker features start to
consume our resources, whereas we would otherwise expend this effort
more effectively on the better alternatives.

A related problem is that the designated package would grow rapidly in
size as every new software development advertised itself as being part
of it (which now guarantees it support from Starlink).  Large packages
are extremely difficult to \st{\mnt{maintain}} and respond slowly to
change, so we would soon be back to the original problem: a large and
uncontrollable accumulation of software with a commitment to support
it all -- but now with a reduced ability to do so.

In reality, it is often cheaper to accept that you have a spread of
requirements and to develop individual packages to satisfy each of
them. The economy comes through keeping the packages small, and
ensuring that you concentrate only on the strengths of each one and do
not allow it to expand into areas that are already well covered
elsewhere. The small level of duplication involved is more than made
up for by the saving in overheads of a single large package attempting
to cater for more diverse requirements.

\subsection{\label{foreign}Why Should we Support non-UK Users?}

{\Large \bf Q:} Why should we provide software support for users outside
the UK? Surely Starlink's priorities lie at home?

{\Large \bf A:} It is true that our activities should be
directed towards the support of UK astronomy. However, this astronomy,
like the software it depends on, cannot be isolated from its
international context. It is therefore important to consider both the
export of software from the UK as well as simply importing software
from others. There are good practical reasons for this:

\begin{itemize}
\item If software (and support) only flows into the UK, and
not out, then astronomers with international collaborations have
little choice but to use the software dictated by their foreign
collaborators. If this differs from that generally used in the UK,
then Starlink's scope for supporting them is greatly reduced.

\item There is an economic advantage in ensuring that any \st{ad hoc}
international standards that emerge are influenced by our own ideas
and software developments. The payoff for this is a reduced cost in
adapting to future developments and a greater likelihood that software
developed elsewhere will be useful to us. In practice, such standards
can only be established by supplying foreign users with successful
products and encouraging them to use them and base future developments
upon them. Lack of support, or an obviously \qt{UK only} attitude are
common obstacles that prevent this happening.

\item International contacts help to ensure that we are kept aware of
developments elsewhere and increase the chance that users will suggest
good new ideas (or even contribute software) which we can
exploit. Overall, this helps prevent our approach to software becoming
too UK-centred. An overseas user base is a cost-effective way of
maintaining this sort of contact.

\item There is, in reality, very little cost in providing a modest
international service once the initial investment has been made. This
is because the effort spent on many forms of \st{\mnt{maintenance}} and
\st{\spt{support}} activity benefits all users of the software, regardless
of who reports the problem.  The use of the WWW to distribute software
and documentation will also keep the cost to a minimum.

\end{itemize}

The question of software \st{\dev{development}} priorities is, of course, a
different matter and needs to reflect UK needs above all.

\subsection{\label{interop}What is Interoperability?}

{\Large \bf Q:} What is \qt{interoperability} and how can it benefit us?

{\Large \bf A:} At present, there are many different software packages
in use, with each trying to be complete in itself. This leads to
considerable duplication (\st{e.g.}\ every package has its own
\qt{add} and \qt{display} commands). Because of the effort of developing
this wide range of functions, most packages have weak areas that their
users have little option but to tolerate until the author gets around
to improving them.  The package as a whole therefore develops far more
slowly than if the author were free to concentrate on its strengths
and ignore its weaknesses.

This situation is maintained by the barriers that exist (intentionally
or otherwise) to prevent users moving freely between software
packages.  Users would, in fact, often be far better off using
selected applications from different packages, choosing whichever is
best.  Many do so already, but generally at a rather coarse level,
switching between packages for different phases of data reduction
because the barriers to inter-mixing applications more intimately are
too high.

There are considerable advantages in trying to reduce these barriers,
which is what interoperability in this context is all about:

\begin{itemize}
\item It makes it easier for users to select the best application for a given
purpose.

\item It exposes application developers to more open competition,
making it less attractive to develop new applications unless they have
clear advantages over what is already available.

\item As a result, it allows us to reduce support for applications which are
already better provided from elsewhere.

\item It allows small projects to compete at the level of individual
\qt{high value} applications against larger packages that enjoy much
higher levels of manpower.

\item It allows us to develop new applications that are useful to a wider
range of users, regardless of which package or environment they
already use.

\item It should help to promote standards as a result of developers
selecting the best features available from different sources.

\end{itemize}

In an ideal world, one might imagine writing applications that,
chameleon-like, could run in a number of different environments, so
that a user could not distinguish them from \qt{native} applications.
Whether one actually needs to go that far, however, is debatable (and
in any case there may be formidable technical obstacles to this
ideal).  More probably, a workable degree of inter-operability between
two or three of the major astronomical data reduction packages would
go a long way towards establishing common standards that others would
follow. This potential for catalysis is one of the attractions of this
approach.

There are currently three main barriers to the sort of
interoperability we envisage:

\begin{description}
\item[Data:] Applications must be able to read and write data in a
variety of formats, so that they can exchange information with other
applications with which they may be used. We have developed an
\ndffmtref{initial solution} to this problem and plan to release the
relevant software this year.

\item[User Interface:] Users would probably want to be presented with
a similar style of application command line, prompt, \st{etc.}, as a
\qt{native} application.  We are currently pursuing prospects for
compatibility with \irafref{IRAF} in this area, and have had a
favourable initial response from its developers.\footnote{The favoured
approach involves an additional process that lies between the user
interface and the application, effectively \qt{adapting} it to the
environment. The resulting deception is expected to be fairly good,
and is potentially extensible to other environments.} One possible
outcome of this might be the availability of Starlink applications
within the IRAF command language -- a capability that can already be
demonstrated up to a point.

\item[Graphics:] The need for graphical interoperability is somewhat
uncertain, as X-windows already provides a fair degree of
compatibility and very few applications in practice depend on more
intimate interaction (such as sharing the same plotting space). The
ability to share image servers might, however, be useful and this
possibility is being investigated.

\end{description}

\end{document}
