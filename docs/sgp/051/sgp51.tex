\documentclass[twoside,11pt]{article}
\pagestyle{myheadings}

\setcounter{tocdepth}{2}

% -----------------------------------------------------------------------------
% ? Document identification
\newcommand{\stardoccategory}  {Starlink General Paper}
\newcommand{\stardocinitials}  {SGP}
\newcommand{\stardocsource}    {sgp\stardocnumber}
\newcommand{\stardocnumber}    {51.1}
\newcommand{\stardocauthors}   {M.\,J.\,Bly\\R.\,F.\,Warren-Smith}
\newcommand{\stardocdate}      {10th March 1997}
\newcommand{\stardoctitle}     {Starlink Software Maintenance Procedures}
\newcommand{\stardocabstract}  {

This document sets out the steps involved in maintaining and developing
the Starlink Software Collection.  The intended audience is all those
involved in maintenance and development of Starlink software.   Some
sections will be of particular relevance to programmers while other
sections will be of interest to those involved in software
distribution.

It details the process from the passing of a bug report to the support
programmer through to the final release of a fix to a package.  It also
specifies the steps required for handling a new package, porting of
existing packages to new platforms, and developing updates to
packages.

There are also sections on beta-testing packages, and the production of
{\sc{CD-rom}}s.

It does not cover Base Set software, nor does it detail the use of
Starlink standard makefiles, and does not specify the mechanics of the
procedures, which evolve with operational demands.

}

% ? End of document identification
% -----------------------------------------------------------------------------

\newcommand{\stardocname}{\stardocinitials /\stardocnumber}
\markboth{\stardocname}{\stardocname}
\setlength{\textwidth}{160mm}
\setlength{\textheight}{230mm}
\setlength{\topmargin}{-2mm}
\setlength{\oddsidemargin}{0mm}
\setlength{\evensidemargin}{0mm}
\setlength{\parindent}{0mm}
\setlength{\parskip}{\medskipamount}
\setlength{\unitlength}{1mm}

% -----------------------------------------------------------------------------
%  Hypertext definitions.
%  ======================
%  These are used by the LaTeX2HTML translator in conjunction with star2html.

%  Comment.sty: version 2.0, 19 June 1992
%  Selectively in/exclude pieces of text.
%
%  Author
%    Victor Eijkhout                                      <eijkhout@cs.utk.edu>
%    Department of Computer Science
%    University Tennessee at Knoxville
%    104 Ayres Hall
%    Knoxville, TN 37996
%    USA

%  Do not remove the %begin{latexonly} and %end{latexonly} lines (used by
%  star2html to signify raw TeX that latex2html cannot process).
%begin{latexonly}
\makeatletter
\def\makeinnocent#1{\catcode`#1=12 }
\def\csarg#1#2{\expandafter#1\csname#2\endcsname}

\def\ThrowAwayComment#1{\begingroup
    \def\CurrentComment{#1}%
    \let\do\makeinnocent \dospecials
    \makeinnocent\^^L% and whatever other special cases
    \endlinechar`\^^M \catcode`\^^M=12 \xComment}
{\catcode`\^^M=12 \endlinechar=-1 %
 \gdef\xComment#1^^M{\def\test{#1}
      \csarg\ifx{PlainEnd\CurrentComment Test}\test
          \let\html@next\endgroup
      \else \csarg\ifx{LaLaEnd\CurrentComment Test}\test
            \edef\html@next{\endgroup\noexpand\end{\CurrentComment}}
      \else \let\html@next\xComment
      \fi \fi \html@next}
}
\makeatother

\def\includecomment
 #1{\expandafter\def\csname#1\endcsname{}%
    \expandafter\def\csname end#1\endcsname{}}
\def\excludecomment
 #1{\expandafter\def\csname#1\endcsname{\ThrowAwayComment{#1}}%
    {\escapechar=-1\relax
     \csarg\xdef{PlainEnd#1Test}{\string\\end#1}%
     \csarg\xdef{LaLaEnd#1Test}{\string\\end\string\{#1\string\}}%
    }}

%  Define environments that ignore their contents.
\excludecomment{comment}
\excludecomment{rawhtml}
\excludecomment{htmlonly}

%  Hypertext commands etc. This is a condensed version of the html.sty
%  file supplied with LaTeX2HTML by: Nikos Drakos <nikos@cbl.leeds.ac.uk> &
%  Jelle van Zeijl <jvzeijl@isou17.estec.esa.nl>. The LaTeX2HTML documentation
%  should be consulted about all commands (and the environments defined above)
%  except \xref and \xlabel which are Starlink specific.

\newcommand{\htmladdnormallinkfoot}[2]{#1\footnote{#2}}
\newcommand{\htmladdnormallink}[2]{#1}
\newcommand{\htmladdimg}[1]{}
\newenvironment{latexonly}{}{}
\newcommand{\hyperref}[4]{#2\ref{#4}#3}
\newcommand{\htmlref}[2]{#1}
\newcommand{\htmlimage}[1]{}
\newcommand{\htmladdtonavigation}[1]{}

%  Starlink cross-references and labels.
\newcommand{\xref}[3]{#1}
\newcommand{\xlabel}[1]{}

%  LaTeX2HTML symbol.
\newcommand{\latextohtml}{{\bf LaTeX}{2}{\tt{HTML}}}

%  Define command to re-centre underscore for Latex and leave as normal
%  for HTML (severe problems with \_ in tabbing environments and \_\_
%  generally otherwise).
\newcommand{\latex}[1]{#1}
\newcommand{\setunderscore}{\renewcommand{\_}{{\tt\symbol{95}}}}
\latex{\setunderscore}

% -----------------------------------------------------------------------------
%  Debugging.
%  =========
%  Remove % from the following to debug links in the HTML version using Latex.

% \newcommand{\hotlink}[2]{\fbox{\begin{tabular}[t]{@{}c@{}}#1\\\hline{\footnotesize #2}\end{tabular}}}
% \renewcommand{\htmladdnormallinkfoot}[2]{\hotlink{#1}{#2}}
% \renewcommand{\htmladdnormallink}[2]{\hotlink{#1}{#2}}
% \renewcommand{\hyperref}[4]{\hotlink{#1}{\S\ref{#4}}}
% \renewcommand{\htmlref}[2]{\hotlink{#1}{\S\ref{#2}}}
% \renewcommand{\xref}[3]{\hotlink{#1}{#2 -- #3}}
%end{latexonly}
% -----------------------------------------------------------------------------
% ? Document-specific \newcommand or \newenvironment commands.
% ? End of document-specific commands
% -----------------------------------------------------------------------------
%  Title Page.
%  ===========
\renewcommand{\thepage}{\roman{page}}
\begin{document}
\thispagestyle{empty}

%  Latex document header.
%  ======================
\begin{latexonly}
   CCLRC / {\sc Rutherford Appleton Laboratory} \hfill {\bf \stardocname}\\
   {\large Particle Physics \& Astronomy Research Council}\\
   {\large Starlink Project\\}
   {\large \stardoccategory\ \stardocnumber}
   \begin{flushright}
   \stardocauthors\\
   \stardocdate
   \end{flushright}
   \vspace{-4mm}
   \rule{\textwidth}{0.5mm}
   \vspace{5mm}
   \begin{center}
   {\huge\bf \stardoctitle}
   \end{center}
   \vspace{5mm}

% ? Heading for abstract if used.
   \vspace{10mm}
   \begin{center}
      {\Large\bf Abstract}
   \end{center}
% ? End of heading for abstract.
\end{latexonly}

%  HTML documentation header.
%  ==========================
\begin{htmlonly}
   \xlabel{}
   \begin{rawhtml} <H1> \end{rawhtml}
      \stardoctitle
   \begin{rawhtml} </H1> \end{rawhtml}

% ? Add picture here if required.
% ? End of picture

   \begin{rawhtml} <P> <I> \end{rawhtml}
   \stardoccategory\ \stardocnumber \\
   \stardocauthors \\
   \stardocdate
   \begin{rawhtml} </I> </P> <H3> \end{rawhtml}
      \htmladdnormallink{CCLRC}{http://www.cclrc.ac.uk} /
      \htmladdnormallink{Rutherford Appleton Laboratory}
                        {http://www.cclrc.ac.uk/ral} \\
      \htmladdnormallink{Particle Physics \& Astronomy Research Council}
                        {http://www.pparc.ac.uk} \\
   \begin{rawhtml} </H3> <H2> \end{rawhtml}
      \htmladdnormallink{Starlink Project}{http://www.starlink.ac.uk/}
   \begin{rawhtml} </H2> \end{rawhtml}
   \htmladdnormallink{\htmladdimg{source.gif} Retrieve hardcopy}
      {http://www.starlink.ac.uk/cgi-bin/hcserver?\stardocsource}\\

%  HTML document table of contents.
%  ================================
%  Add table of contents header and a navigation button to return to this
%  point in the document (this should always go before the abstract \section).
  \label{stardoccontents}
  \begin{rawhtml}
    <HR>
    <H2>Contents</H2>
  \end{rawhtml}
  \htmladdtonavigation{\htmlref{\htmladdimg{contents_motif.gif}}
        {stardoccontents}}

% ? New section for abstract if used.
  \section{\xlabel{abstract}Abstract}
% ? End of new section for abstract

\end{htmlonly}

% -----------------------------------------------------------------------------
% ? Document Abstract. (if used)
%  ==================
\stardocabstract
% ? End of document abstract
% -----------------------------------------------------------------------------
% ? Latex document Table of Contents (if used).
%  ===========================================
\newpage
\begin{latexonly}
   \setlength{\parskip}{0mm}
   \tableofcontents
   \setlength{\parskip}{\medskipamount}
   \markboth{\stardocname}{\stardocname}
\end{latexonly}
% ? End of Latex document table of contents
% -----------------------------------------------------------------------------

% MAINPART
\cleardoublepage
\markboth{\stardocname}{\stardocname}
\renewcommand{\thepage}{\arabic{page}}
\setcounter{page}{1}

\section{\label{introduction}\xlabel{introduction}Introduction}

The Starlink Software Collection is a large and complex set of packages.
It is maintained through the efforts of various support programmers and
the Starlink Software Librarian.  There are a large number of diverse
packages, each with its own requirements which have to be dealt with
when put through a change cycle.

To avoid having to remember several different ways of building and
manipulating software, Starlink has evolved a standard method of
manipulation which allows each element of the collection to be
treated separately, while remaining part of a cohesive whole.  This requires
software to use a standard {\tt{makefile}} style.

Any large software system needs a set of procedures that govern how
changes are made.  A number of steps have been identified through
which each software change must pass.  These are set out in this
document, and make up the Software Change Cycle.  Each step has a
series of requirements which must be satisfied before the process may
progress to the next step.

In some circumstances it may be appropriate to skip a step, but this
should only be done after consultation with the Starlink Software
Librarian.

\section{\label{the_software_change_cycle}\xlabel{the_software_change_cycle}The Software Change Cycle}

The following steps are {\bf{required}} for a complete software release
into the Starlink Software Collection (USSC).  The following sections
describe each step in more detail.

\begin{enumerate}

\item Fix the problem, produce an update, and package a submission

\item Submit the software to the Software Librarian

\item Check and test the submitted software

\item Build the checked software for release

\item Beta-test the software if required

\item Prepare a USSC update

\item Release the USSC update

\item Tidy up after the release
\end{enumerate}

There is an emphasis on checking and testing during the steps -- the
amount and scope of the testing depending on the step involved.  The
testing phases are particularly important, since late finding of
problems causes effort already expended to have been wasted.

\subsection{\label{fix_the_problem}\xlabel{fix_the_problem}Fix the problem and/or produce an update}

\begin{description}

\item[File the bug report]: Programmer and Software Librarian.

The bug report is frequently the first step in the change cycle -- the
support Programmer or Software Librarian receives a bug report for a
piece of software, and a fix is required.

All bug reports received by a programmer should be copied to the
Software Librarian to be included in the records for the package.  The
Software Librarian will print and keep paper copies of all bug reports
in the package file.

The Software Librarian will forward bug reports for a package to the
appropriate programmer, and print and file a copy of the report in the
package file.

\item[Find and fix the problem/bug]: Programmer.

The fix for a bug is up to the programmer, who must
find and solve the problem.

\item[Update the software to add new functionality]: Programmer.

In some cases a more detailed change is to be made and instead of a fix
to an individual problem, a series of fixes and possibly additional functions
are required.  Updates are usually assigned by the Head
of Software as part of the annual Software Plan (for details see
\xref{SGP/48}{sgp48}{}), though some requests for enhancements or
modifications may come from the Software Librarian.

\item[Provide integration to common source where appropriate]: Programmer.

Once a fix has been made, or an update developed, the programmer should
integrate the fix with the common source set for the package, and ensure
the bug is fixed or the update works correctly on all supported
hardware and software combinations.

\item[Update the news file]: Programmer.

Create or update the {\tt{package.news}} file to detail the fix, update
or new functionality.  The news file is very important because it is the
first notification users will see which describe the changes to a package.
It should detail clearly and concisely the changes and fixes made, and
any changes in behaviour.  It should also reference the documentation.

\item[Update the documentation]: Programmer and Document Librarian

If required, the documentation for the package should be updated.  It
is the programmer's responsibility to generate new documents, or
update existing documents, but the Document Librarian is
responsible for clearing a document for release.

\begin{description}

\item [New Documents]:  Programmer.

\begin{enumerate}

\item New documents should be provided by the programmer.

\item New document numbers (not revision numbers) are allocated by the
Starlink Document Librarian at RAL and should be obtained by the
programmer before the document is submitted for clearance.

\item A new document should have the revision number {\tt{1(Draft)}}
until it is has been cleared for release by the Documentation
Librarian.  Thereafter it should have the revision number {\tt{1}}.

\item New documents must conform to the Starlink House Style.  Details
may be found in \xref{SGP/50}{sgp50}{}.

\end{enumerate}

\item [Existing Documents]: Programmer.

\begin{enumerate}

\item Make all required changes to the document text and format.

\item Update document revision number and date.

The document revision number is updated when new sections are
added, or existing sections changed to enhance detail about the
package.  The revision number is the number after the period in the
{\verb+\stardocnumber+} command definition in the document header, and
may only be incremented, never decremented.

The document revision number is not changed if the only change since
the last release is to correct spelling and format errors.

\end{enumerate}

\item [All Documents]: Programmer and Document Librarian.

\begin{enumerate}

\item All Starlink User Notes (SUNs) describing software packages  must
have a title page, and the abstract and introduction sections.  All other
Starlink documentation should conform to the Starlink House Style (see
\xref{SGP/50}{sgp50}{}).

\item All Starlink User Notes (SUNs) for software `packages' are normally
produced with coloured covers in the Starlink House Style.  Programmers
should provide the Document Librarian with one or more
`screen-shot' pictures to be used on the coloured cover, otherwise a
default coloured cover will be used.  The Document Librarian is
responsible for deciding which documents (SUNs) should be issued with
coloured covers.

\item All {\verb+\section+} and {\verb+\subsection+} commands must be
labeled using {\verb+\label+} and {\verb+\xlabel+} commands, to allow
cross-referencing between documents.  The labels should be appropriate
to the section.

\item The document must be spell checked.

\item A new hypertext version of the document should be produced and
checked.  All details of any post-processing of the converted hypertext
should be recorded in comments inside the document source so that the
hypertext version may be produced without involving the programmer.

\begin{quote}{\em
The \LaTeX\ and hypertext versions of a document must match, except
where there are {\verb+\latexonly+} or {\verb+\htmlonly+} sections.}
\end{quote}
\item A new hypertext document archive should be created.

\item New documentation, and new sections to existing documentation,
must be submitted to the Document Librarian to be cleared for release.

Submission should be made by email, and contain the document source.
For large documents it may be appropriate for the programmer to copy
the document source to the RAL systems and notify the Document
Librarian of the location.

\item The Document Librarian should check that the submitted
document conforms where practicable to the Starlink House Style, and
that the document revision number and date have been updated
appropriately.

\item The Document Librarian must clear the document for release.

Where no changes to the submitted document are necessary, the Document
Librarian will clear the document for release by emailing the programmer
to tell him.

In some cases the Document Librarian may at his discretion make
modifications to the document before returning it to the programmer.  The
modified document source should be returned by email.

If modifications are made, a new hypertext version of the document should
be created (by the programmer).

If the Document Librarian does not clear the document for release, the
document source (modified or not) should be returned to the programmer by
email with notes detailing deficiencies in the document.  The programer
should address the deficiencies and resubmit the document for clearance.

\item After a document has been cleared for release, the programmer should
include the cleared document master, any attendant illustration files and
the hypertext archive in any submission of the software for release.

\end{enumerate}
\end{description}

\item[Update package version number]: Programmer.

The Package Version number in the {\tt{makefile}} should be updated to
show that a change has occurred.  The significance of the change is
indicated by how the Package Version number is changed.

Package Version numbers have the form: {\tt{major.minor-release}}

\begin{enumerate}

\item Bug fixes and {\it{minor}} changes should be indicated by
incrementing the {\tt{release}} number.  This typically occurs when
a change is made to a {\tt{makefile}} or {\tt{mk}} script, or the document
is modified.

\item More significant changes, updates, and new ports should be
indicated by incrementing the {\tt{minor}} version number, and
returning the {\tt{release}} number to zero.  Significant revisions to
documentation should also coincide with a release of this magnitude.

\item Major changes and updates to functionality, and additional
functions in a package should be indicated by incrementing the
{\tt{major}} version number, with both {\tt{minor}} and {\tt{release}}
numbers returned to zero.

\item If a beta-test is to be carried out, the letter {\tt{b}} should
be added to the {\tt{release}} number.  The {\tt{b}} will be removed
for final release after the beta-test period.

\end{enumerate}

For libraries, a Library Version number is defined, similar to the
Package Version number, but without the {\tt{release}} number.

\begin{enumerate}

\item The {\tt{minor}} version number should be incremented to indicate
bug fixes and {\it{minor}} updates.

\item The {\tt{major}} version number should only be incremented if a
dynamic linked program linked with the new library would not run with
the older versions of the library, {\em{e.g.}}, if new subroutines had
been added to the library.  The {\tt{minor}} version number should be
returned to zero.

\end{enumerate}

\item[Testing procedures]: Programmer

It is the responsibility of the programer to verify that a fix or update
for a piece of software works, and does not adversely affect the existing
functionality of the software.

It is also the responsibility of the programmer to provide information
that can be used by other testers or the Software Librarian to verify that
a newly submitted version of the software has the fix or update included.
This information should be provided in the submission notice (see next
section) sent to the Software Librarian, and/or in information provided to
other testers if appropriate (see next section).

Once the programmer is confident that the fix or update works, the
software must be subjected to and pass a testing regime, which must
include tests of the bug fix and/or update, and the standard package
tests.  The tests must be passed on all supported hardware and software
combinations:

\begin{enumerate}

\item The fix and/or update must verifiably have been included in the
the new source set for the package.

\item The {\tt{./mk test}} for the package should be passed on all platforms.

The {\tt{./mk test}} (the {\tt{test}} target in the {\tt{makefile}}) should,
after package installation, check that all appropriate binaries will run,
and that all required files are installed correctly.  It is not intended that
the {\tt{./mk test}} should do an exhaustive test of all functionality in a
package.

\item The package demonstration programs must run.  These are often used
by Starlink when testing software, and in providing demonstrations of
software capability to new users and visitors to sites.  It is important
that any demonstration programs provided as part of the package run smoothly.

\item The {\tt{export}} and {\tt{export\_run}} targets in the
makefile must be run to check that valid exportable sets are created.
Both {\tt{export}} and {\tt{export\_run}} sets must be capable of being
correctly installed.

\item Where the software is critical to the operation of other items of
software,  the interface between the two packages should be tested
to ensure that existing software continues to work with the fixed or updated
software.

It is the responsibility of the programmer to maintain in fixed or
updated software any existing compatibility with other software, where
reasonably practicable.  Any instance where this is not possible should
be discussed with the Software Librarian.

The requirement for maintaining existing capability does not preclude
the possibility of changes in functionality breaking existing
capability, however, this should be the exception and all such
possibilities must be discussed with the Software Librarian and the
Head of Software.

\end{enumerate}

The aim is to ensure that as much of the software as is possible and
appropriate is tested before the submission is made.

\end{description}

\subsection{\label{software_submission}\xlabel{software_submission}Software submission}

When testing is completed, the package should be prepared for
submission.  All the stages in the preceeding step must have been
completed satisfactorily.

\begin{description}

\item[Risk assessment]: Programmer and Software Librarian.

An assessment of the risk involved in the fix or update must be made, in
order to determine whether a beta-test of the package is required.

There is a risk to other packages in the Software Collection from any
release -- changes in functionality in the package may break existing
software, and major updates may introduce new bugs to the package that
may not show up in testing before release.

In most cases the risks are slight, but an assessment of the scope
of any such effects must always be made by the programmer.

If the risk of disruption is sufficient, a beta-test of the package will
be deemed appropriate by the Software Librarian.

The programmer should provide factual information to the Software
Librarian about the scope and impact of the fix or update.  This
information will be used by the Software Librarian in determining
whether a beta-test is needed, though if a programmer thinks fit, he
may recommend a beta-test be carried out.

The Software Librarian will normally consult the Head of Software when
determining whether a beta-test is to be made, but it is the
responsibility of the Software Librarian to make the risk assessment,
and the responsibility of the programmer to provide the information on
which a decision can be based.

\item[Release Priorities]: Programer and Software Librarian.

Each package submitted has to be queued for release along with all the
other packages in the system.  The scheduling is fluid, so that items
can be queued according to their individual priority.

The programmer should supply a statement of priority with the
submission.  This information will be taken into account by the
Software Librarian when constructing the release schedule.  He will use
his discretion to resolve conflicts of priority.

If a submission is dependent on another package submission, {\em{e.g.}},
an application requires a new version of a subroutine library, this
information must also be provided by the programmer in the submission note.

The up-to-date release schedule will be maintained on-line so that
programmers may track their submissions through the release procedure
after submission.

\item[Create a {\tt{package.tar.Z}} archive]: Programmer.

Once a package is ready for submission, the programmer should create an
archive (compressed tar file) using the {\tt{export\_source}} target in
the {\tt{makefile}}:

\begin{quote}
\begin{verbatim}
% ./mk export_source
\end{verbatim}
\end{quote}

This creates a {\tt{package.tar.Z}} archive.

\item[Upload to RAL or local {\tt{ftp}} system]: Programmer.

The {\tt{package.tar.Z}} archive should be uploaded to the local
anonymous {\tt{ftp}} system at the programmer's Site, or the
{\tt{/pub/incoming}} directory on the Starlink anonymous {\tt{ftp}}
system at RAL.  Programmers with access to the RAL systems should leave
the {\tt{package.tar.Z}} file in their own home directory tree on the RAL
systems.

\item[Notify Software Librarian of submission]: Programmer.

The programmer should notify the Software Librarian of a submission
by sending an email message to {\tt{starlink@jiscmail.ac.uk}}.

The message should contain:

\begin{enumerate}

\item the location of the {\tt{package.tar.Z}} archive,

\item any special instructions for installation, and details of
any new command aliases and environment variables required.

\item a statement of priority, and any specific scheduling requirements.

\item information on the impact and scope of the fix or update in
order that a risk assessment may be made.

\item a request for a beta-test (if the programmer thinks it necessary).

\item information on which tests are appropriate to verify the fixing
of a bug, and the location of any required test data.  Where data is
needed for testing, it should also be made available with (but not in)
the {\tt{package.tar.Z}} archive.

\item a copy of the {\tt{package.news}} file `attached' (using
{\tt{Pine}}) for reference.

\end{enumerate}

Once a submission message has been submitted, a programmer should NOT
update or modify the submitted {\tt{package.tar.Z}} archive without
first checking with the Software Librarian, to make sure that the
package has not already been entered into the submission checking
system.

\end{description}

\subsection{\xlabel{checking_submission}\label{checking_submission}Checking the submission}

The Software Librarian will initiate a check of the submitted software,
which will be carried out either by the Software Librarian or by a
designated checker.

The check will normally be carried out over a period of a few days,
and the programmer will be notified beforehand that the check is about
to occur, so that any last minute changes may be incorporated and
notified to the Software Librarian before the check starts.

Any major last minute changes will cause the submission to be handed
back to the programmer for proper resubmission.

The person assigned to check the submission is responsible for ensuring that
the submission is checked according to the procedure below.

\begin{description}

\item[Version numbers and dates]\mbox{}

The package and library version numbers, and document version number
and date should be checked to see if they have been updated correctly.

\item[{\tt{Makefile}} and {\tt{mk}} script]\mbox{}

The {\tt{makefile}} and {\tt{mk}} script should be checked where
appropriate to make sure they conform to the Starlink standards.

\item[Documentation standard]\mbox{}

The Document Librarian should be consulted to see that the document
has been cleared for release.  If the document has not been cleared,
the submission should be rejected.

\item[Build and installation tests]\mbox{}

A build and installation should be performed on all systems to ensure
a `clean' build and install, {\em{i.e.}} the build occurs in one pass
without failing, and the installation occurs without failing.
The test building should be done on the development systems at RAL, but
not in the main USSC directories, or by the owner of those directories.

\item[Testing]\mbox{}

The package should be tested to verify that a fix or update is
included, and the {\tt{./mk test}} runs successfully.  Where available,
package demonstration scripts should also be used to provide more
extended verification that the package runs.

\item[Deinstallation, cleaning, and export tests]\mbox{}

The package {\tt{export}} and {\tt{export\_run}} distribution archives
should be created and tested to ensure that they can be installed
correctly.

The package deinstallation procedure should be tested to
ensure that the deinstallation occurs without failing.

The package {\tt{clean}} and {\tt{unbuild}} procedures should be tested to
ensure that no extraneous files remain after their operation.

\item[Clearing for release]\mbox{}

When the check is complete the checker should confirm to the
Software Librarian that the submission has passed all the checking steps.

The Software Librarian will then clear the submission for release.

Any submission not passing all the checking steps will
be handed back to the programmer for correction.   This should be the
exception rather than the rule.

\end{description}

\subsection{\label{build_package_for_release}\xlabel{build_package_for_release}Building the package for release}

The Software Librarian will build the submission from scratch on the
RAL development systems, using the {\tt{`star'}} account which owns the
USSC file hierarchy.  This is required for all supported hardware and
software combinations.

\begin{enumerate}

\item {\bf{Deinstall existing version}}

Any existing version of the package should be deinstalled, and its
source directory renamed from {\tt{package}} to {\tt{old.package}}.

\item {\bf{Create a new directory}}

A new {\tt{package}} source directory should be created, and the
distribution {\tt{package.tar.Z}} archive should be unpacked into it.

\item {\bf{Build the package}}

The software should be built, with log files kept in the logs directory
for future reference.

\item {\bf{Clean, strip and install}}

After a `clean' build (to completion in one pass), the software should be
cleaned, any binaries stripped, and the package installed on each system.

\item {\bf{Final Testing}}

The built-for-release version should be tested by running the {\tt{./mk
test}} and all other appropriate tests using a different account than
used for building or checking.  This is normally done by the Software
Librarian using a personal userid, and verifies that the package runs
under a generic account as well as the special Maintenance accounts.

\end{enumerate}

If all the tests are passed, the package should be released.

\subsection{\label{beta-testing}\xlabel{beta-testing}Beta-testing}

If a beta-test has been decided upon prior to or during the risk
assessments, a beta-test release should be prepared and distributed.
Otherwise, this step may be omitted.

\begin{description}

\item[Importance]\mbox{}

Beta-testing of software is vital to the Starlink software strategy,
and it is important that beta-tests are carried out properly, even
if this means that software fails the test and has to be revised.

The Software Librarian, Site Managers at beta-test sites, and the
Programmers each have a responsibility to ensure that a beta-test is
completed successfully, and this is best achieved by cooperation.

The following sections detail the responsibilities of each.

\item[Identifying beta-test sites]: Programmer, Software Librarian.

Appropriate beta-test sites should be suggested by the Programmer.  The
Head of Software and the Software Librarian may also be able to suggest
beta-test sites.

The Software Librarian should request a site to participate in
the beta-test.

Site Managers requested to participate in a beta-test are strongly
encouraged to participate.

\item[Build beta-test version of package]: Software Librarian.

The Software Librarian will build and test a beta-test release in the
same manner as a standard release.  The beta-test submission must have
passed all the previous steps.

\item[Prepare beta-test release note]: Software Librarian.

The Software Librarian will prepare a beta-test release note.  This
will detail:

\begin{enumerate}

\item the beta-release installation instructions

\item the instructions on how to switch back to the standard release

\item the procedure for reporting problems

\item the period over which the beta-test will run.

\end{enumerate}

\item[Upload the beta-test distribution sets]: Software Librarian.

The beta-test distribution sets will be made available in the
Starlink anonymous {\tt{ftp}} system beta-test directory,
with the beta-test release note.

\item[Issue beta-test note]: Software Librarian.

The beta-test release note should be emailed to the beta-test sites.
It should not be posted on Usenet.

\item[Installation of beta-tests]: Site Managers.

Site Managers at sites participating in beta tests must inform the
Software Librarian that the beta-test has been installed, and should
ensure that users are aware of it and are able to use the beta-test
software.

If a Site Manager at a proposed beta-test site is unable to run a
beta-test, the Software Librarian must be informed immediately.

Site Managers at beta-test sites are responsible for ensuring that the
beta-test is run, and that problem reports are passed on to the
Software Librarian.

\item[Problem reporting]: Site Managers, Software Librarian, Programmers.

It is important that any problem with beta-test software is reported,
no matter how trivial.

The Site Manager at a beta-test site is responsible for reporting all
problems with beta-test software that he is made aware of in addition to
any report of the problem by a user.

The Software Librarian will pass on to the Programmer any bug reports
from the beta-test sites.  It is the Programmer's responsibility to
follow up all problem reports.

\item[Completion of the beta-test period]: Site Managers, Software Librarian.

Site Managers at the beta-test sites must confirm the completion of the
beta-test at their site, and provide evidence that the users have both used
the beta-test software, and reported any problems they encountered.

When the test period is complete (minimum 2 weeks), the test should be
withdrawn.  The Software Librarian should issue a proper release, or
instructions for reverting to the old release.  All bug and problem
reports should be fed back to the Programmer responsible.

\end{description}

\subsection{\label{preparing_a_release}\xlabel{preparing_a_release}Preparing a release}

The Software Librarian is responsible for the tasks in this section.

When one or more packages are cleared for release, a release bundle
should be made.  The precise method of release depends upon the inventory
of packages to be released, but all releases follow the same basic pattern.

\begin{enumerate}

\item{\bf{Create temporary distribution directories}}

As well as the permanent preparation directory, each release requires
a set of release- and system-specific distribution directories.  These
should be created in the distribution staging area.

\item{\bf{Generate distribution archives}}

Create {\tt{export\_run}} and {\tt{export}} sets for each supported
system.  Large optional items should be chopped into smaller chunks.
Copies of the {\tt{export}} sets should be placed in the system
specific distribution staging directories.

\item{\bf{Create document masters}}

Where documents have been updated, create new paper masters, and pass them
to the Documentation Librarian for bulk copying.  Where possible,
paper masters should be printed double-sided.

The Software Librarian should use discretion when deciding whether
new paper masters are needed.  Minor spelling corrections would normally
not require a complete new copy run, but it would be appropriate to
generate a new paper master so that future copies are made from the
latest master.

\item{\bf{Create package news items}}

Each package should have a news item to announce a bug fix or update.

The news item should contain a short description of the package, and
details of the update in simple terms.  More complicated details of the
release may also be included, but should be restricted to subtopics of
the news item.

The Software Librarian should generate a suitable news item for each
package to be issued, based on the submitted {\tt{package.news}} file.
Distribution news items should be called {\tt{news\_package.txt}}.  The
news item should be placed in the permanent staging directory.

Note that providing a {\tt{package.news}} file is the responsibility of the programmer as detailed in Section~\ref{fix_the_problem}.

\item{\bf{Generate release note, implementation script and summary news item}}

Create a release implementation script and
release summary news item in the permanent staging directory. Also
create a release note, detailing the contents of the release, and
giving the implementation instructions.  These are created from
standard templates.

\item{\bf{Starlink database and information files}}

Update the Starlink database to contain details of changes to software,
documentation, personnel and contact information, as appropriate.  New
information files should be generated from the database in the
permanent staging directory.

\item{\bf{Check implementation script, notes and news item}}

Print out and check the implementation script, release news item and
release note for errors and typographical mistakes.  Once they are OK,
{\tt{touch}} the contents of the preparation directory to give all the
files a common date and time stamp.

\item{\bf{Copy the generic distribution files}}

Copy all the generic distribution files (information files, news items,
the implementation script) from the permanent staging directory into
each of the system-specific staging directories to create a complete
distribution directory for each system.

\end{enumerate}

\subsection{\label{sofwtare_release}\xlabel{software_release}Software release}

The Software Librarian is responsible for these stages.

The release of an update has two stages: first, the RAL control
versions are updated, followed by the development systems; and second,
the release is issued to sites.

\begin{description}

\item[Test the updates at RAL]\mbox{}

This stage is performed for two reasons.  The first is to test that the
actual update procedure works as intended, and the second is to update
all the RAL systems to keep them in step.

\begin{enumerate}

\item{\bf{Implement update on control systems}}

Implement the update on the control systems at RAL as if implementing
the update at a site, by running the update procedure.  Monitor the
size of the various packages so that size data may be included in the
distribution note.

\item{\bf{Implement update on development systems}}

Implement the update on the development versions at RAL.  It is usually
only necessary to do the compulsory parts of an update, to update the
information file set.

In some cases, special steps must be taken to allow an update to
proceed on an already updated system, for instance when new software is
added.  For example, new packages may have been built in a directory
that has the same name as that to be created and used by the update
procedure.

\end{enumerate}

\item[Issue the update]\mbox{}

\begin{enumerate}

\item{\bf{Update Software Store}}

Copy the package {\tt{export}}, {\tt{export\_run}} and
{\tt{export\_source}} sets to the Software Store if licence conditions
permit.  Each package has its own subdirectory of the main store
directory.

The Software Store web pages should be updated to show new packages (where
appropriate), and update information for existing packages.

\item{\bf{Update source archive}}

Delete the old version of the package source from the source archive
directory, and unpack the {\tt{export\_source}} set into the source
archive directory to replace it.

\item{\bf{Create release update archives}}

Create the update archives for the release from the system-specific
(or in some cases generic) staging directories.

\item{\bf{Add sizes data and news summary to the release note}}

Add data for the sizes of the released packages and the update archives
to the appropriate places in the release note.  Add the summary news
item to its place at the top of the release note.

\item{\bf{Copy update archives to the {\tt{ftp}} system}}

Copy all the update archives, one for each system, or generic archives
to the Starlink anonymous {\tt{ftp}} system, together with a copy of
the release note.

\item{\bf{Distribute the Release}}

Distribute the release note to sites using the {\tt{ussc\_mailer}}
script and the {\tt{ussc\_sites}} distribution list.

\item{\bf{Issue the release note to Usenet}}

The {\tt{ussc\_sites}} distribution list has an entry that sends the
release note to the moderator of the {\tt{uk.org.starlink.announce}}
news group, {\tt{announce@star.rl.ac.uk}}.  Using this account, post the
release note to the news group.

In some circumstances, wider distribution of release notes is appropriate.
In this case, the release notes will be posted to a relevant selection of
news groups.

\item{\bf{Print and archive implementation script and release note}}

Save a copy of the emailed release note and print it, and print
a copy of the implementation script.  File both in the archive folder, in
the Software Librarian's Office.

Archive the on-line copies of the release note and implementation script
using the {\tt{ussc\_arc}} script.

\end{enumerate}

\end{description}

\subsection{\label{tidying_up}\xlabel{tidying_up}Tidying up after a release}

After a release, the Software Librarian should tidy up the development
system used to build the release by removing old copies of software,
purging test directories and staging areas, and filing all email and
notes in the software package file.

\begin{enumerate}

\item{\bf{Delete copies of the release notes}}

Copies of the release notes and implementation script used for emailing
and on-line archiving should be deleted once the emailing has been completed.

\item{\bf{Remove old versions of software}}

The procedure for building software prior to release requires that a
`deinstalled' copy of the software be kept on-line in an {\tt{old.package}}
directory at the same level in the directory hierarchy as the {\tt{package}}
directory.  The {\tt{old.package}} directory should be deleted.

\item{\bf{Clean preparation, test and staging directories}}

The inventory of files comprising a release should be deleted from
the preparation directories.

Source copies of the released packages ({\tt{package.tar.Z}} files)
left in the {\tt{import}} staging area, and export copies left in the
{\tt{export}} staging area should be deleted.

In addition, any test copies remaining in the test directories should
be removed.

\item{\bf{File paper copies of notes}}

All email traffic and submission notes concerning each released package
should be filed in the appropriate package file.

\end{enumerate}

This ends the release cycle.

\newpage
\section{\label{porting_procedures}\xlabel{porting_procedures}Porting Procedures}

Porting Starlink Software to a new hardware and software combination is
not something to be undertaken lightly.  That said, following the
simple guidelines and procedures detailed here should result in a
successful port.

A successful port is defined as the complete conversion of a package
onto a new hardware and software combination, and the reintegration of
any necessary modifications into the common source for the package,
without compromising the functioning of the package on existing
hardware and software combinations.  A package will not be adopted
until it has passed all its tests.

\subsection{\label{where_to_start_a_port}\xlabel{where_to_start_a_port}Where to start a port}

A port of a package is either easy, in that you only have to port the
package itself, or hard, in that you have to port all the necessary
infrastructure as well.

\subsection{\label{who_does_the_port}\xlabel{who_does_the_port}Who does the port}

The port should be done either by the programmer supporting the
package, or when the port is part of a whole system port exercise (see
below), the programmer assigned to the port.

In some cases, the porter is not the support programmer or a Starlink
programmer doing a whole system port, but this makes little difference
to the procedures involved.

\subsection{\label{single_package_ports}\xlabel{single_package_ports}Single package ports}

A single package port is usually straight forward, if the originating
programmer has adhered to the Starlink coding standards, and not used
esoteric system-specific Fortran extensions.

The infrastructure is already available and all that is necessary is to
make the package run on the new system.

\subsubsection{Port procedure}

The procedure for porting (in terms of the Software Cycle) is to treat
the port exercise as a development update to the package, and use Steps
1 and 2 of the Software Cycle.

If the porter is not the support programmer for the package, he should
note all the modifications necessary to the package to complete the
port, and feed these back to the support programmer as an update
request. This will trigger the complete Software Cycle for the package,
and will lead to a release of the package for all supported systems.

It is the responsibility of the package support programmer to ensure that
the package is sufficiently tested to pass Step 2 of the Software Cycle.

\subsection{\label{porting_the_whole_ussc}\xlabel{porting_the_whole_ussc}Porting the whole USSC}

A complete port of the whole system to a new hardware and software
combination is best left to the Starlink Project, and is usually
assigned to one programmer.

\subsubsection{Port procedure}

The method is to start at the lowest level of the infrastructure and
work up the dependencies till the whole infrastructure is ported.
This must be done using Steps 1 and 2 of the Software Cycle applied
to each separate package in the infrastructure.

Notes must be kept of modifications require to each package, and these
should be fed back to the package support programmers so that they may
carry out integration and testing for Steps 1 and 2 of the Software
Cycle.

Once the infrastructure is ported, applications and utilities can be
ported, again by using Steps 1 and 2 of the Software Cycle, and
treating each port as a development update.

\subsection{\label{adpopting_a_new_port}\xlabel{adpopting_a_new_port}Adopting a new port}

The decision to adopt a new port is take by the Head of Software, in
consultation with the Software Librarian.  A port will be considered
for adoption only after it has been fully tested and integrated.

\newpage
\section{\label{producing_cdroms}\xlabel{producing_cdroms}Producing {\sc{CD-rom}}s}

The production of Starlink {\sc{CD-rom}}s is a relatively straightforward
process, requiring the production of a copy of the appropriate
software directories, which is then processed into ISO-9660 format and
dumped onto a reserved disk partition.  This copy is then used to
`burn' new {\sc{CD-rom}}s.  Many {\sc{CD-rom}}s can be made by
{\sc{CD-rom}} `burner', or a bulk order made from one master by a
commercial manufacturer.

\subsection{\label{cdrom_inventory}\xlabel{cdrom_inventory}{\sc{CD-rom}} inventory}

The Starlink Software Librarian will decide when a new {\sc{CD-rom}} is
to be made, and the inventory of software to be provided, in
consultation with the Head of Software.  In general, only versions of
software that have been stable in production service for a period will
be put onto {\sc{CD-rom}}.

Additionally, extra systems software, such as compilers and general
utilities, will be provided on {\sc{CD-rom}}.  The inventory of these will be
agreed upon within Operations group, and such software will  be
provided by Operations group and others as appropriate.

\subsection{\label{cdrom_snapshot}\xlabel{cdrom_snapshot}{\sc{CD-rom}} snapshot}

The {\sc{CD-rom}} will contain a snapshot of the RAL `control' version
of one of the supported USSC versions.  This is the version that is
maintained to be the same as the versions running at Starlink sites,
rather than the `development' versions used at RAL for testing,
which sometimes differ considerably from the `control' versions.

\subsection{\label{snapshot_testing}\xlabel{snapshot_testing}Snapshot testing}

When a release date and inventory have been agreed, changes to the
`control' version to be used for the snapshot will be locked out for an
extended period of testing by programmers and support personnel.  Testing
will also be done by Operations group, to check that the snapshot of
the USSC has no major flaws.

Any major problems found in a piece of software during this period will
be fixed prior to release on {\sc{CD-rom}}, or the software to be
withdrawn from the snapshot copy, particularly if the software is newly
written or ported.

\subsection{\label{update_the_cdrom_document}\xlabel{update_the_cdrom_document}Update the {\sc{CD-rom}} document}

There is a document detailing the use of the {\sc{CD-rom}} which
contains some details of the software inventory, and the layout of the
{\sc{CD-rom}}.  For the Linux distribution, this document is
\xref{SUN/212}{sun212}{}. The document must be updated to reflect the
new {\sc{CD-rom}} to be produced.

\subsection{\label{create_test_cdroms}\xlabel{create_test_cdroms}Create test {\sc{CD-rom}}s}

Before the {\sc{CD-rom}} can be mass produced for distribution, a small
number of test copies are made to test the integrity of the snapshot
and additional software, using the {\sc{CD-rom}} copy.

A complete hierarchy of directories to be placed on the {\sc{CD-rom}},
including a copy of the `snapshot', should be assembled and processed
into ISO-9660 format.

This copy should then be used by the {\sc{CD-rom}} burner to cut up to 5
test {\sc{CD-rom}}s.

\subsection{\label{pre-production_testing}\xlabel{pre-production_testing}Pre-production testing}

The Starlink Software Librarian and the Head of Applications will
select some test sites or individuals to receive a copy of the new
{\sc{CD-rom}}, and arrange for them to test it.
Additional testing will be done by Operations group.

At this stage, there should be no problems with the software, but
testing may highlight technical problems with file arrangements on
disc, or use of the {\sc{CD-rom}}.

The testers should confirm that testing is complete, and report any
problems the Starlink Software Librarian.  Any problems should then be
fixed, and a final release date set.

\subsection{\label{cdrom_inserts}\xlabel{cdrom_inserts}{\sc{CD-rom}} inserts}

A new {\sc{CD-rom}} version should be provided with a new jewel case insert
detailing the version number and date of the release.  This should also
have brief details of how to use the {\sc{CD-rom}}.

\subsection{\label{cdrom_production}\xlabel{cdrom_production}{\sc{CD-rom}} production}

Once the Starlink Software Librarian is happy that the test
{\sc{CD-rom}} is OK, he will clear the new {\sc{CD-rom}} for bulk
production and distribution.  Bulk production is the responsibility of
Operations group.

\end{document}
