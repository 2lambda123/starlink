\documentstyle[11pt]{article}
\pagestyle{myheadings}

% -----------------------------------------------------------------------------
% ? Document identification
\newcommand{\stardoccategory}  {Starlink General Paper}
\newcommand{\stardocinitials}  {SGP}
\newcommand{\stardocsource}    {sgp\stardocnumber}
\newcommand{\stardocnumber}    {49.1}
\newcommand{\stardocauthors}   {J.\,C.\,Sherman\\
                                C.\,A.\,Clayton}
\newcommand{\stardocdate}      {28th May 1996}
\newcommand{\stardoctitle}     {Starlink's Support for PCs}
\newcommand{\stardocabstract}  {

This paper outlines Starlink's policy on support for PCs.
Essentially, we aim to use PCs as low-end multi-user
workstations running Linux, a variant of Unix, and to
integrate them with existing Unix systems.  We also aim to
make most Starlink software available for the PC/Linux
platform.
 
Starlink's policy on support of PCs is likely to develop
over the next few years.  Readers are therefore advised
to check that they have the most up-to-date version of this
document.  The policy outlined here was agreed by the
Starlink Panel at their meeting in March 1996.

}
% ? End of document identification
% -----------------------------------------------------------------------------

\newcommand{\stardocname}{\stardocinitials /\stardocnumber}
\markright{\stardocname}
\setlength{\textwidth}{160mm}
\setlength{\textheight}{230mm}
\setlength{\topmargin}{-2mm}
\setlength{\oddsidemargin}{0mm}
\setlength{\evensidemargin}{0mm}
\setlength{\parindent}{0mm}
\setlength{\parskip}{\medskipamount}
\setlength{\unitlength}{1mm}

% -----------------------------------------------------------------------------
%  Hypertext definitions.
%  ======================
%  These are used by the LaTeX2HTML translator in conjunction with star2html.

%  Comment.sty: version 2.0, 19 June 1992
%  Selectively in/exclude pieces of text.
%
%  Author
%    Victor Eijkhout                                      <eijkhout@cs.utk.edu>
%    Department of Computer Science
%    University Tennessee at Knoxville
%    104 Ayres Hall
%    Knoxville, TN 37996
%    USA

%  Do not remove the %\begin{rawtex} and %\end{rawtex} lines (used by 
%  star2html to signify raw TeX that latex2html cannot process).
%\begin{rawtex}
\makeatletter
\def\makeinnocent#1{\catcode`#1=12 }
\def\csarg#1#2{\expandafter#1\csname#2\endcsname}

\def\ThrowAwayComment#1{\begingroup
    \def\CurrentComment{#1}%
    \let\do\makeinnocent \dospecials
    \makeinnocent\^^L% and whatever other special cases
    \endlinechar`\^^M \catcode`\^^M=12 \xComment}
{\catcode`\^^M=12 \endlinechar=-1 %
 \gdef\xComment#1^^M{\def\test{#1}
      \csarg\ifx{PlainEnd\CurrentComment Test}\test
          \let\html@next\endgroup
      \else \csarg\ifx{LaLaEnd\CurrentComment Test}\test
            \edef\html@next{\endgroup\noexpand\end{\CurrentComment}}
      \else \let\html@next\xComment
      \fi \fi \html@next}
}
\makeatother

\def\includecomment
 #1{\expandafter\def\csname#1\endcsname{}%
    \expandafter\def\csname end#1\endcsname{}}
\def\excludecomment
 #1{\expandafter\def\csname#1\endcsname{\ThrowAwayComment{#1}}%
    {\escapechar=-1\relax
     \csarg\xdef{PlainEnd#1Test}{\string\\end#1}%
     \csarg\xdef{LaLaEnd#1Test}{\string\\end\string\{#1\string\}}%
    }}

%  Define environments that ignore their contents.
\excludecomment{comment}
\excludecomment{rawhtml}
\excludecomment{htmlonly}
%\end{rawtex}

%  Hypertext commands etc. This is a condensed version of the html.sty
%  file supplied with LaTeX2HTML by: Nikos Drakos <nikos@cbl.leeds.ac.uk> &
%  Jelle van Zeijl <jvzeijl@isou17.estec.esa.nl>. The LaTeX2HTML documentation
%  should be consulted about all commands (and the environments defined above)
%  except \xref and \xlabel which are Starlink specific.

\newcommand{\htmladdnormallinkfoot}[2]{#1\footnote{#2}}
\newcommand{\htmladdnormallink}[2]{#1}
\newcommand{\htmladdimg}[1]{}
\newenvironment{latexonly}{}{}
\newcommand{\hyperref}[4]{#2\ref{#4}#3}
\newcommand{\htmlref}[2]{#1}
\newcommand{\htmlimage}[1]{}
\newcommand{\htmladdtonavigation}[1]{}

%  Starlink cross-references and labels.
\newcommand{\xref}[3]{#1}
\newcommand{\xlabel}[1]{}

%  LaTeX2HTML symbol.
\newcommand{\latextohtml}{{\bf LaTeX}{2}{\tt{HTML}}}

%  Define command to re-centre underscore for Latex and leave as normal
%  for HTML (severe problems with \_ in tabbing environments and \_\_
%  generally otherwise).
\newcommand{\latex}[1]{#1}
\newcommand{\setunderscore}{\renewcommand{\_}{{\tt\symbol{95}}}}
\latex{\setunderscore}

%  Redefine the \tableofcontents command. This procrastination is necessary 
%  to stop the automatic creation of a second table of contents page
%  by latex2html.
\newcommand{\latexonlytoc}[0]{\tableofcontents}

% -----------------------------------------------------------------------------
%  Debugging.
%  =========
%  Remove % on the following to debug links in the HTML version using Latex.

% \newcommand{\hotlink}[2]{\fbox{\begin{tabular}[t]{@{}c@{}}#1\\\hline{\footnotesize #2}\end{tabular}}}
% \renewcommand{\htmladdnormallinkfoot}[2]{\hotlink{#1}{#2}}
% \renewcommand{\htmladdnormallink}[2]{\hotlink{#1}{#2}}
% \renewcommand{\hyperref}[4]{\hotlink{#1}{\S\ref{#4}}}
% \renewcommand{\htmlref}[2]{\hotlink{#1}{\S\ref{#2}}}
% \renewcommand{\xref}[3]{\hotlink{#1}{#2 -- #3}}
% -----------------------------------------------------------------------------
% ? Document specific \newcommand or \newenvironment commands.
% ? End of document specific commands
% -----------------------------------------------------------------------------
%  Title Page.
%  ===========
\renewcommand{\thepage}{\roman{page}}
\begin{document}
\thispagestyle{empty}

%  Latex document header.
%  ======================
\begin{latexonly}
   CCLRC / {\sc Rutherford Appleton Laboratory} \hfill {\bf \stardocname}\\
   {\large Particle Physics \& Astronomy Research Council}\\
   {\large Starlink Project\\}
   {\large \stardoccategory\ \stardocnumber}
   \begin{flushright}
   \stardocauthors\\
   \stardocdate
   \end{flushright}
   \vspace{-4mm}
   \rule{\textwidth}{0.5mm}
   \vspace{5mm}
   \begin{center}
   {\Large\bf \stardoctitle}
   \end{center}
   \vspace{5mm}

% ? Heading for abstract if used.
   \vspace{10mm}
   \begin{center}
      {\Large\bf Abstract}
   \end{center}
% ? End of heading for abstract.
\end{latexonly}

%  HTML documentation header.
%  ==========================
\begin{htmlonly}
   \xlabel{}
   \begin{rawhtml} <H1> \end{rawhtml}
      \stardoctitle
   \begin{rawhtml} </H1> \end{rawhtml}

% ? Add picture here if required.
% ? End of picture

   \begin{rawhtml} <P> <I> \end{rawhtml}
   \stardoccategory \stardocnumber \\
   \stardocauthors \\
   \stardocdate
   \begin{rawhtml} </I> </P> <H3> \end{rawhtml}
      \htmladdnormallink{CCLRC}{http://www.cclrc.ac.uk} /
      \htmladdnormallink{Rutherford Appleton Laboratory}
                        {http://www.cclrc.ac.uk/ral} \\
      \htmladdnormallink{Particle Physics \& Astronomy Research Council}
                        {http://www.pparc.ac.uk} \\
   \begin{rawhtml} </H3> <H2> \end{rawhtml}
      \htmladdnormallink{Starlink Project}{http://star-www.rl.ac.uk/}
   \begin{rawhtml} </H2> \end{rawhtml}
   \htmladdnormallink{\htmladdimg{source.gif} Retrieve hardcopy}
      {http://star-www.rl.ac.uk/cgi-bin/hcserver?\stardocsource}\\

%  HTML document table of contents. 
%  ================================
%  Add table of contents header and a navigation button to return to this 
%  point in the document (this should always go before the abstract \section). 
  \label{stardoccontents}
  \begin{rawhtml} 
    <HR>
    <H2>Contents</H2>
  \end{rawhtml}
  \renewcommand{\latexonlytoc}[0]{}
  \htmladdtonavigation{\htmlref{\htmladdimg{contents_motif.gif}}
        {stardoccontents}}

% ? New section for abstract if used.
  \section{\xlabel{abstract}Abstract}
% ? End of new section for abstract

\end{htmlonly}

% -----------------------------------------------------------------------------
% ? Document Abstract. (if used)
%  ==================
\stardocabstract
% ? End of document abstract
% -----------------------------------------------------------------------------
% ? Latex document Table of Contents (if used).
%  ===========================================
% \newpage
 \begin{latexonly}
   \setlength{\parskip}{0mm}
   \latexonlytoc
   \setlength{\parskip}{\medskipamount}
   \markright{\stardocname}
 \end{latexonly}
% ? End of Latex document table of contents
% -----------------------------------------------------------------------------
\newpage
\renewcommand{\thepage}{\arabic{page}}
\setcounter{page}{1}

\section{Introduction}

The policy on Starlink's support for PCs, as outlined here, is our first
such policy but, we expect, not our last.  In other words, we expect
this initial policy to be modified in the light of experience.   We
are therefore starting with a modest level of support, which can be
increased subsequently if practical and desirable. Much of what
follows is aimed at minimizing the PC support load on Site Managers and
RAL staff, PC support being in addition to present duties which are
already demanding.

It is intended that Starlink's PCs are treated as low-end, multi-user
workstations (only a few users simultaneously), similar to present
low-end SPARC and Alpha workstations, and {\em not} as ``personal''
systems.  To this extent the title of this paper is misleading; a
better title might be {\em Starlink's Support for Low-cost Unix
Workstations}.  Some sites are already using PCs as X-terminals but the
present policy goes well beyond such use.  The multi-user approach has
many implications and consequences, as noted in the following sections,
and makes a large contribution towards minimizing the support load on
Site Managers and RAL.

\section{Why bother with PCs?}

Starlink is incorporating PC hardware into present SPARC and Alpha
systems for several reasons, as outlined in the following sections, but
not least of all because PCs are good value for money.   Compared with
SPARC workstations, for example, a Pentium based PC can be purchased
for less than a SPARC 4 but with more CPU power than a SPARC 10, or
even some models of SPARC 20.  Of course, such comparisons are not
straightforward because there are other parameters besides price and
CPU power, for example I/O speed and expandability.  Comparisons are
also difficult because prices and product specs are continually
changing.  Nevertheless, we believe that PCs represent good value for
money and could be valuable enhancements to existing systems.   

Some detailed comparisons of price and performance for Pentium PCs,
low-end SPARC workstations and low-end Alpha workstations are given in
Appendix A.


\section{What Operating System?}

Starlink supports only PCs running Linux.  In other
words, no significant support is provided for PCs
running any other operating system.  Linux
has been chosen because it is a version of Unix (and hence multi-user),
is free and is in widespread use by astronomers.   The advantages of
choosing a version of Unix are that (a)
differences with existing systems are minimized, for both users and
support staff and (b) porting Starlink software is relatively easy. Sun's
Solaris is also available for PCs but is not free and, more importantly,
its use is far less widespread than Linux. Linux is becoming the {\em
de facto} standard for astronomical applications on PCs --- IRAF, AIPS,
MIDAS and IDL are already available.   Furthermore, Linux comes with
various important Unix tools built-in, such as \LaTeX, {\tt pine, Tcl/Tk}
and compilers, and experience shows it to be reliable (at least as reliable
as Solaris).   Starlink supports only Linux
(as opposed to Linux plus a variety of other PC operating systems)
simply to minimize the support load.  The adoption of Linux rules out
support of Macs at present, which do not run Linux at the time of writing.

Of course, Linux is not without its disadvantages.  Because it is free,
support cannot be guaranteed, but our experience so far is that help
with problems is, nevertheless, readily available.  The usual Linux Fortran
compilers (g77 and f2c) also have some problems but the lack of VMS
extensions in one of them (f2c)  has been corrected as part of our
porting work (see the next section).  Finally, Linux has less support
for updating a cluster of machines than Solaris but we expect
Starlink's Linux upgrades to be infrequent.

All in all, the choice of Linux over other operating
systems for PCs seems clear and work on porting
Starlink software to Linux has been going on at a low level since
early 1995 (see the next section).   It is intended that the version(s)
of Linux adopted by Starlink will be compatible with IRAF, AIPS and IDL.

\section{Starlink Software under Linux}

At the time of writing, all of Starlink's infrastructure software has
been ported to Linux and has been tested subsystem by subsystem.  We
have six applications running at present, FIGARO, DIPSO, SPECDRE, KAPPA,
IRAS90 and
CATAPP, and work is continuing on testing these applications, testing
the infrastructure through them and porting further applications.
These first applications have been chosen because they
are NAG-free (except KAPPA) and because they are good tests of the
Linux compiler and infrastructure port. We expect to have most
Starlink applications ported by the summer of 1996.

Experience with running the applications ported so far indicates that
PC/Linux performance is well up to expectations, in line with the
positive reports from those running IRAF on this platform.  We have run
the Linux versions of SPECDRE and KAPPA as part of Starlink's
benchmarking suite but as yet don't have enough applications running
under Linux to provide a STARmark rating for PC/Linux.  Early
indications are that the memory required to properly exploit a Pentium
PC running Starlink applications is 32 Mbyte.  The same software will
also run on older 486 based PCs with less memory, but obviously with
lower performance. Nevertheless, we expect 486 performance to be
acceptable to a single user for many applications.

PC hardware requirements for various astronomical software systems are
listed in Appendix B. 

\section{Other Advantages}

As well as the value for money benefits noted in \S 2, there are
several other major advantages to UK astronomers which follow from
Starlink support of the Pentium PC/Linux platform.  They stem from the
fact that there are a significant number of ``non-Starlink'' PCs
already in use at Starlink sites.   If these PCs remain non-Starlink
(no integration with Starlink systems, no Site Manager support and no
maintenance) then Starlink software (under Linux) will nevertheless be
available for them, alongside IRAF, AIPS, IDL etc.   If, on the other
hand, these PCs become ``part-Starlink'', then as well as software the
benefits for astronomers would include integration into existing
Starlink systems, Site Manager support and maintenance.

As with other equipment which becomes part-Starlink, such PCs would
become part of the {\em site's} equipment under the control of the
local Management committee and would be available to
all.  However, the hardware configuration of some non-Starlink PCs may
be very different from that of ``standard'' PCs (see below), making the
adoption of such machines as part-Starlink impractical.  The
adoption and integration of non-Starlink PCs is decided case by case
and site by site, with the change to part-Starlink status going ahead
only if all parties (the original purchaser of the equipment, the site,
the Site Manager and Starlink) are in agreement.   If management or
support problems do arise, the PC concerned will revert to
non-Starlink status.

\section{Hardware Restrictions}

It is intended that Starlink PCs are fully integrated into existing
systems and are loaded with software from the network and from CD-ROM
(see the \S 8).  The {\em minimum} hardware requirement for a
``standard'' Starlink PC is therefore a powerful processor (Pentium),
16 Mbytes of memory (32Mb recommended), an ethernet interface and a sizeable hard disc
(half Gbyte or more).  This minimum requirement assumes a CD-ROM drive
can be accessed via the network.

There are many different varieties of PC on the market.  To provide
full support for all possible combinations of PC, add-on boards
(ethernet) and peripherals (hard disc, CD-ROM) would be an impossibly
large load on Site Managers and RAL staff.   Instead, 
full support is restricted to  (a) the PCs, add-on boards, discs etc.
purchased by RAL, and (b) the very similar equipment purchased by
others which has been given part-Starlink status.  Of course, Linux and
Starlink software will probably run on other ``non-standard'' hardware
combinations but support for such non-standard hardware will be limited
to only a best-efforts basis.

Furthermore, the equipment purchased by RAL for sites, and
therefore fully supported, is restricted initially to the type of
hardware RAL is already familiar with --- Viglen PCs. 
This range of hardware may expand somewhat as our experience
increases.  Advice on whether other hardware combinations could run
Starlink software or could be integrated into Starlink systems will be
provided, but again only on a best-efforts basis.   All the
restrictions noted here are aimed at minimizing the extra support load on
Site Managers and RAL staff which follows from support for a third
platform (PC/Linux) in addition to the present two platforms
(SPARC/Solaris and Alpha/Digital-Unix).

\section{Some Difficulties}

One obvious difficulty is that support for PCs will be in addition to
present duties.   However,  constant themes of this policy are: (a) to
minimize the extra workload and (b) to make PC support optional for
each site (it will not be optional for the Project at RAL or for our
contract programmers).  In addition,  the reduction over the past two
years in the number of platforms supported, from a peak of five
(MicroVAX/VMS, DECstation/Ultrix, SPARC/SunOS, SPARC/Solaris,
Alpha/Digital-Unix) down to two (SPARC/Solaris, Alpha/Digital-Unix), 
will at many sites open up the possibility of taking on PCs.

There are two, more serious, difficulties, both related to Starlink
users' perception of PCs:  \begin{itemize} \item Many see them as
personal machines and may be disappointed that Starlink is treating
them as multi-user.  Starlink may eventually reach a situation where
every user has an X-screen on his or her desk, or even where every user
has a CPU on their desk.  However, we are at present far from this
ideal situation (we have, roughly, 1 Unix system per 4 active on--site
research users) so it is inevitable that users have to share CPUs.
Hence, although it may disappoint some users, we have decided to run
the systems as shared, multi-user, low-end Unix workstations.

\item Many users may expect PCs to run Windows and may be disappointed
to find Starlink PCs running Linux.  \end{itemize}

Both difficulties can be ameliorated by allowing Starlink PCs to
occasionally run Windows, say 10\% of the time or less.  When they do
so they they would no longer be multi-user systems so the decisions on
when to change from one system to the other would be for the site as a
whole (the Site Manager and Site Chairman), not for individual users.
Starlink would not provide any support for Windows, the Site Managers'
maximum involvement with Windows being to shut down Linux and boot Windows
and vice versa.  We hope that sites that wish to work this way will be
able to devise suitable arrangements to allow a small amount of Windows
use to  co-exist with the majority Linux use. Sites wishing to provide 
user support for Windows will have to find the necessary effort from their
own resources.

Starlink recognizes the benefits to astronomers of Windows applications
such as DTP and spreadsheets but has concluded that the limited
resource available for PCs and PC support at present must be
concentrated on the PC/Linux platform, with only negligible resource
diverted to PC/Windows.

\section{Software Distribution}

PC software, both Linux and Starlink, is
distributed on CD-ROM (Starlink now has the capability of making a
small number of CD-ROMs).  For Linux, we distribute the Linux
CD-ROM set, some optional pre-built kernels suitable for Starlink
hardware, configuration files tailored to Starlink hardware, a
recommended book on running Linux and Starlink documentation on Linux
building and installation.   All of this is backed up by advice
and support from RAL.   Linux, like other versions of Unix, can be
configured in many different ways.   We hope that by distributing
pre-built kernels and configuration files and through the documentation
we can reduce diversity and so can reduce the support load on Site
Managers and RAL.   Starlink's Linux distributions take place
from time to time, depending on the changes in Linux since the previous
distribution.  We expect the frequency to be less than once per year.

Starlink software for PCs, both infrastructure and
applications, is available via the
network from Starlink's ``Software Store'', in just the same way as
software for SPARC/Solaris and Alpha/Digital-Unix.   In addition, at
roughly six month intervals, we will take a snap-shot of the PC
part of the software store and distribute it on CD-ROM to relevant
sites.  It is a well established policy that Starlink software for
SPARC/Solaris and Alpha/Digital-Unix stems from a single version of the
source code.  This policy has been extended to include
PC/Linux so that all three platforms share a common source.   To
this extent, the level of software development and support for all
three platforms is the same --- bug fixes and improvements made
to software running on one platform will at the same time benefit the
other platforms.

A major difference compared with software distribution for our two main
platforms is that there is {\em no} requirement for Site Managers to
update their copies of PC/Linux software whenever there is a change.
For SPARC/Solaris and Alpha/Digital-Unix software there is such a
requirement (the USSCs), so that sites remain up to date and in step
with each other,  but updates would be optional for PC/Linux.  
At a minimum sites should update PC/Linux software
whenever they receive a CD-ROM distribution, roughly once every six
months. Some sites will of course choose to keep their various versions of
the Starlink software in step for the convenience of their users.
As well as distributing a Starlink CD-ROM, we could also, if
sites wished, distribute CD-ROMs for other software of interest to
astronomers, for example IRAF.


\section{Ancillary Software}

As well as the operating system, infrastructure software and
astronomical applications, the SPARC and Alpha systems run a variety of
other software, such as {\tt pine}, \LaTeX, {\tt perl, ispell, jed,
NuTPU, Mosaic \& Netscape, Vbackup, pidentd} and {\tt httpd}.   
Where this software can be easily provided for PC/Linux
we will do so but where it would take considerable effort we will
not.  The latter arrangement is reasonable in view of the fact that all
the software is already available on the main SPARC and Alpha systems.

The details of what will and will not be made available for PC/Linux
systems have not yet been fully worked out, but some examples (covering
around half the software involved) are:  \begin{itemize} \item {\tt
pine}, \LaTeX and {\tt perl} come with Linux.  They might
      be different versions from those running on our SPARC and Alpha
      systems but any differences apparent to users are unlikely to be
      significant.  \item {\tt ispell} and {\tt jed} also come with
Linux.   To make them
      the same as the versions running on SPARC and Alpha systems, 
      we will copy across Starlink's (English) dictionaries for
      {\tt ispell} and our tailored configuration files for {\tt jed}.
\item {\tt NuTPU} is proprietary software which was purchased to
      make an EDT-like editor available on Unix.  It will not be
      made available on PCs.  \item {\tt Mosaic \& Netscape} should
be made available on PCs but at
      present they are not on the Linux CD-ROMs.   We therefore intend
      that RAL will obtain them and make them available with our Linux
      distributions.  \item {\tt Vbackup} is proprietary software which
was purchased to
      allow VMS backup tapes to be read on Unix systems.   It 
      will not be made available on PCs.  \item {\tt pidentd} is the
identity checking software which is required
      to run Starlink Forum.  Like {\tt Mosaic \& Netscape}, it does
      not come with Linux but RAL will make it available with our Linux
      distributions.  \item {\tt httpd} is the web server.  Since other
machines will
      probably be more suitable as web servers, it is not proposed to
      make this available on PCs.  \end{itemize}

Taking the above together with the
previous section (Software Distribution), it can be seen that all
software running on SPARC and Alpha systems will also be available on
PCs, with the exceptions that (a) Starlink's astronomy applications {\em may}
be up to six months behind and (b) some ancillary software may be
missing or a different version.

\section{Maintenance and other Support}
 
As for other equipment purchased by Starlink, Starlink would maintain
the PCs it purchased, either per-call or, if the numbers are sufficient,
via an on-site maintenance contract.
 
To help ensure that Starlink and part-Starlink PCs, like other
Starlink hardware, are accessible to all users at a site, we require
that they are located on-site, not, for example, at an astronomer's home.
Consistent with this, if non-Starlink
PCs are located at home then the support provided by Site Managers
for such PCs, for example, to do with logging on from home, should be
at most
on a best efforts basis.
 
PC hard discs will be backed up as part of the usual site-wide disc
backups in a similar way to discs on other low-end workstations.
%However, Linux does not include a version of the {\tt dump} program and
%hence including PC hard discs into existing procedures will create
%additional work for Site Managers.

\section{What Next?}
 
Sites wishing to obtain  Starlink PC(s) should bid for them
in their responses to Starlink's Announcements of Opportunity
(AO).  As described previously, the bids should be for PC(s)
to run Linux and for use in astronomical data reduction and
analysis.  The PC could run Starlink applications or
Linux versions of IRAF, AIPS, etc.
 
Sites wishing to obtain the Linux version of Starlink software,
for use on Starlink or non-Starlink PCs,
can do so via the Starlink Software Store 
or via CD-ROM.  Requests for CD-ROMs should be addressed to
Mr K P Duffey, Starlink, RAL.
 
Sites wishing to have existing non-Starlink PCs adopted as
part-Starlink should first make themselves familiar with
the conditions associated with such adoptions and then
contact Mr D J Rawlinson, RAL.
 

\section{Summary}
 
In summary, this policy states that:
\begin{itemize}
\item   PCs running Linux, bid for by sites and allocated by the Panel 
        as part of the
        AO round, will be integrated into existing Starlink systems as an
        alternative type of low-end Unix workstation;
\item   The support provided by RAL and Site Managers for such equipment
        is the same as provided for present low-end Unix workstations, with
        the exception that software updates need not be as frequent and
        not all ancillary software will be available (see \S 8 and \S 9);
\item   Existing PCs can be adopted as part-Starlink, if a site
        agrees, provided the hardware is sufficiently similar to
       ``standard'' hardware and provided the equipment becomes
        accessible to all users;
\item   A small amount of Windows use can, if a site wishes, be
        combined with the majority Linux use but almost all the Starlink 
        resource going into PCs and PC support must be concentrated on the 
        PC/Linux platform.
\end{itemize}
 
\newpage

\appendix

\section{Price and Performance Comparisons for ``low-end'' Machine (May 1996)}

The following table gives a simple price/performance comparison for
low--end Digital, Sun and PC-based systems. Various configurations of
varying compute power are given, but with otherwise identical hardware
features (where possible). Issues such as relative bus speed,
graphics acceleration and expandability have {\em not} been taken into
account. The price is given ex-VAT and after discount.

Specification:
\begin{itemize}
\item 17" Colour monitor
\item 32Mb RAM (greater on Digital Unix machines which require more RAM)
\item 1Gb internal disk
\item Ethernet card
\end{itemize}
 
\begin{table}[h]
\begin{center}
\begin {tabular}{|l|r|r|}
\hline
{\bf System}                 & {\bf Performance}  &   {\bf Price (ex-VAT)} \\
\hline
AlphaStation 200 4/100       &    92 SPECfp92 &   No longer available  \\
AlphaStation 200 4/166 (40Mb)&   134 SPECfp92 &   \pounds 2799  \\
AlphaStation 255 5/233 (64Mb)&   230 SPECfp92 &   \pounds 5000  \\
\hline
SPARCstation 4 Model 110     &    65 SPECfp92 &   \pounds 2562  \\
SPARCstation 10 Model 51     &    83 SPECfp92 &   No longer available  \\
Ultra 1 M140                 &   303 SPECfp92 &   \pounds 6104  \\
\hline 
Pentium 100                  &    90 SPECfp92 &   \pounds 1785  \\
Pentium 133                  &   110 SPECfp92 &   \pounds 1885  \\
\hline 
\end{tabular}
\caption{Price and Performance comparisons (May 1996)}
\end{center}
\end {table}

%\newpage

\section{PC Hardware Requirements}

To give an indication of the sort of hardware that is 
suitable for running astronomical applications, 
the recommendations for PC hardware to run IRAF, AIPS and IDL
are noted below. It is expected that the requirements of Starlink
software will be similar to those of IRAF and AIPS.  An ethernet
interface will also be required. The following notes come from the
software providers themselves (IRAF and IDL) or from Jodrell Bank (AIPS).
 
\begin{tabbing}
xxxxxxxxxxxxxxxxxxxxxxxx\=xxxxxxxxxxxxxxxx\kill
{\bf IRAF minimum}  \> 486-33DX CPU with 16 Mb memory, 500 Mb disc\\
                    \> and 15" monitor;\\
{\bf IRAF recommended} \> Pentium 90 CPU with PCI bus, 32 Mb memory,\\
                       \> 2 Gb disc and 17" monitor;\\
{\bf AIPS minimum}  \> 486-66DX2 CPU with 20 Mb memory, 1Gb disc\\
                    \> and 15" monitor;\\
{\bf AIPS recommended} \> Pentium 90 CPU with PCI bus, 32 Mb memory,\\
                       \> 2 Gb disc and 17" monitor;\\
{\bf IDL minimum}   \> 486-66DX2 CPU with 8 Mb memory.
\end{tabbing}
 
\newpage
 
\end{document}
