\documentstyle[twoside,11pt]{article}
\pagestyle{myheadings}

% -----------------------------------------------------------------------------
% ? Document identification
\newcommand{\stardoccategory}  {Starlink General Paper}
\newcommand{\stardocinitials}  {SGP}
\newcommand{\stardocsource}    {sgp\stardocnumber}
\newcommand{\stardocnumber}    {47.1}
\newcommand{\stardocauthors}   {Martin Clayton}
\newcommand{\stardocdate}      {11 March 1997}
\newcommand{\stardoctitle}     {Computer Algebra Software}
% ? End of document identification
% -----------------------------------------------------------------------------

\newcommand{\stardocname}{\stardocinitials /\stardocnumber}
\markboth{\stardocname}{\stardocname}
\setlength{\textwidth}{160mm}
\setlength{\textheight}{230mm}
\setlength{\topmargin}{-2mm}
\setlength{\oddsidemargin}{0mm}
\setlength{\evensidemargin}{0mm}
\setlength{\parindent}{0mm}
\setlength{\parskip}{\medskipamount}
\setlength{\unitlength}{1mm}

% -----------------------------------------------------------------------------
%  Hypertext definitions.
%  ======================
%  These are used by the LaTeX2HTML translator in conjunction with star2html.

%  Comment.sty: version 2.0, 19 June 1992
%  Selectively in/exclude pieces of text.
%
%  Author
%    Victor Eijkhout                                      <eijkhout@cs.utk.edu>
%    Department of Computer Science
%    University Tennessee at Knoxville
%    104 Ayres Hall
%    Knoxville, TN 37996
%    USA

%  Do not remove the %\begin{rawtex} and %\end{rawtex} lines (used by
%  star2html to signify raw TeX that latex2html cannot process).
%\begin{rawtex}
\makeatletter
\def\makeinnocent#1{\catcode`#1=12 }
\def\csarg#1#2{\expandafter#1\csname#2\endcsname}

\def\ThrowAwayComment#1{\begingroup
    \def\CurrentComment{#1}%
    \let\do\makeinnocent \dospecials
    \makeinnocent\^^L% and whatever other special cases
    \endlinechar`\^^M \catcode`\^^M=12 \xComment}
{\catcode`\^^M=12 \endlinechar=-1 %
 \gdef\xComment#1^^M{\def\test{#1}
      \csarg\ifx{PlainEnd\CurrentComment Test}\test
          \let\html@next\endgroup
      \else \csarg\ifx{LaLaEnd\CurrentComment Test}\test
            \edef\html@next{\endgroup\noexpand\end{\CurrentComment}}
      \else \let\html@next\xComment
      \fi \fi \html@next}
}
\makeatother

\def\includecomment
 #1{\expandafter\def\csname#1\endcsname{}%
    \expandafter\def\csname end#1\endcsname{}}
\def\excludecomment
 #1{\expandafter\def\csname#1\endcsname{\ThrowAwayComment{#1}}%
    {\escapechar=-1\relax
     \csarg\xdef{PlainEnd#1Test}{\string\\end#1}%
     \csarg\xdef{LaLaEnd#1Test}{\string\\end\string\{#1\string\}}%
    }}

%  Define environments that ignore their contents.
\excludecomment{comment}
\excludecomment{rawhtml}
\excludecomment{htmlonly}
%\end{rawtex}

%  Hypertext commands etc. This is a condensed version of the html.sty
%  file supplied with LaTeX2HTML by: Nikos Drakos <nikos@cbl.leeds.ac.uk> &
%  Jelle van Zeijl <jvzeijl@isou17.estec.esa.nl>. The LaTeX2HTML documentation
%  should be consulted about all commands (and the environments defined above)
%  except \xref and \xlabel which are Starlink specific.

\newcommand{\htmladdnormallinkfoot}[2]{#1\footnote{#2}}
\newcommand{\htmladdnormallink}[2]{#1}
\newcommand{\htmladdimg}[1]{}
\newenvironment{latexonly}{}{}
\newcommand{\hyperref}[4]{#2\ref{#4}#3}
\newcommand{\htmlref}[2]{#1}
\newcommand{\htmlimage}[1]{}
\newcommand{\htmladdtonavigation}[1]{}

%  Starlink cross-references and labels.
\newcommand{\xref}[3]{#1}
\newcommand{\xlabel}[1]{}

%  LaTeX2HTML symbol.
\newcommand{\latextohtml}{{\bf LaTeX}{2}{\tt{HTML}}}

%  Define command to re-centre underscore for Latex and leave as normal
%  for HTML (severe problems with \_ in tabbing environments and \_\_
%  generally otherwise).
\newcommand{\latex}[1]{#1}
\newcommand{\setunderscore}{\renewcommand{\_}{{\tt\symbol{95}}}}
\latex{\setunderscore}

%  Redefine the \tableofcontents command. This procrastination is necessary
%  to stop the automatic creation of a second table of contents page
%  by latex2html.
\newcommand{\latexonlytoc}[0]{\tableofcontents}

% -----------------------------------------------------------------------------
%  Debugging.
%  =========
%  Remove % on the following to debug links in the HTML version using Latex.

% \newcommand{\hotlink}[2]{\fbox{\begin{tabular}[t]{@{}c@{}}#1\\\hline{\footnotesize #2}\end{tabular}}}
% \renewcommand{\htmladdnormallinkfoot}[2]{\hotlink{#1}{#2}}
% \renewcommand{\htmladdnormallink}[2]{\hotlink{#1}{#2}}
% \renewcommand{\hyperref}[4]{\hotlink{#1}{\S\ref{#4}}}
% \renewcommand{\htmlref}[2]{\hotlink{#1}{\S\ref{#2}}}
% \renewcommand{\xref}[3]{\hotlink{#1}{#2 -- #3}}
% -----------------------------------------------------------------------------
% ? Document specific \newcommand or \newenvironment commands.

\newcommand{\sgspec}[2]{#1}
\begin{htmlonly}
\renewcommand{\sgspec}[2]{#2}
\end{htmlonly}

% star2HTML and post-processing
% =============================
%
% This following can be used to invoke star2html and post-process this
% document to generate HTML as the author intended.
%
% To run the script, automatically extracting from this file:
%
%    % egrep -e "^%%S2HPOST" sgp47.tex | sed -e "s/^%%S2HPOST//" | csh

%%S2HPOST#!/bin/csh
%%S2HPOST
%%S2HPOST# Definitions.
%%S2HPOST    set AUTH_NAME = 'Martin Clayton';
%%S2HPOST    set AUTH_EMAIL = 'mjc@star.ucl.ac.uk';
%%S2HPOST    set DOC_CODE = sgp47;
%%S2HPOST    set DOC_TITLE = 'Computer Algebra Software';
%%S2HPOST
%%S2HPOST# Star2html process.
%%S2HPOST    star2html -a "${AUTH_NAME}" -m "${AUTH_EMAIL}" -t \
%%S2HPOST            "${DOC_TITLE}" ${DOC_CODE}
%%S2HPOST
%%S2HPOST# Generate Perl script to do the work.
%%S2HPOST    cat >! ${DOC_CODE}$$.pl <<FOO
%%S2HPOST#!/usr/local/bin/perl
%%S2HPOST
%%S2HPOST# To be used in a pipe.
%%S2HPOST# Post-processing for star2html.
%%S2HPOST
%%S2HPOST\$last_line = '';
%%S2HPOST\$last_invisanchor = 0;
%%S2HPOST\$AUTH_EMAIL = '$AUTH_EMAIL';
%%S2HPOST
%%S2HPOST\$nlines = 0;
%%S2HPOSTwhile ( <> ) {
%%S2HPOST
%%S2HPOST# Add mailto URL if e-mail address is found.
%%S2HPOST    s#\$AUTH_EMAIL#<A HREF="mailto:\${AUTH_EMAIL}">\${AUTH_EMAIL}</A>#;
%%S2HPOST
%%S2HPOST# Some small adjustments - gets rid of blank space at page top.
%%S2HPOST    s/<P><ADDRESS>/<ADDRESS>/;
%%S2HPOST    s/<BR> <HR>/<HR>/g;
%%S2HPOST    s/<HR> <P>/<HR>/g;
%%S2HPOST    s/<DD>  <BR>/<DD>/g;
%%S2HPOST
%%S2HPOST# Convert em dashes into single hyphens.
%%S2HPOST    s/([^-])---([^-])/\1-\2/g;
%%S2HPOST
%%S2HPOST# Try to get rid of "invisible" anchors by tying them to the next word.
%%S2HPOST# Also try to handle "doubles" caused by \xlabel{}\label{} type things.
%%S2HPOST    s:(<H[1-9]><A NAME=SECTION[^>]*><A NAME=[^>]*>)&#160;(</A>)(<A NAME=[^>]*>)&#160;(</A>)([^<]*):\1\3\5\2\4:;
%%S2HPOST    s:(<H[1-9]><A NAME=SECTION[^>]*><A NAME=[^>]*>)&#160;(</A>)([^<]*):\1\3\2:;
%%S2HPOST
%%S2HPOST# Same procedure for invisible anchors in list items...
%%S2HPOST    s:(<LI> <b><A NAME=[^>]*>)&#160(</A>)([^<]*)(</b><BR>):\1\3\2\4:;
%%S2HPOST
%%S2HPOST# Minor correct for some lists.
%%S2HPOST    s:<LI> :<LI>:;
%%S2HPOST
%%S2HPOST# Get rid of double horizontal rules (shouldn't be any...)
%%S2HPOST    if ( \$last_line eq "<HR>\n" ) {
%%S2HPOST        if ( \$_ eq "<HR>\n" ) {
%%S2HPOST            next;
%%S2HPOST        }
%%S2HPOST
%%S2HPOST# Get rid of double Paragraph breaks.
%%S2HPOST    } elsif ( \$last_line eq "<P>\n" ) {
%%S2HPOST        if ( \$_ eq "<P>\n" ) {
%%S2HPOST            next;
%%S2HPOST        }
%%S2HPOST
%%S2HPOST# This is an attempt to convert invisible anchors introduced by \indexentry
%%S2HPOST# to be attached to the next word.
%%S2HPOST    } elsif ( \$last_invisanchor != 0 ) {
%%S2HPOST        if ( \$_ eq "<P>\n" ) {
%%S2HPOST            next;
%%S2HPOST
%%S2HPOST        } else {
%%S2HPOST            \$last_invisanchor = 0;
%%S2HPOST            ( \$firstword, \$rest ) = split( ' ', \$_, 2 );
%%S2HPOST            print "\$1\$firstword\$2 \$rest";
%%S2HPOST            \$last_line = \$_;
%%S2HPOST            next;
%%S2HPOST        }
%%S2HPOST    }
%%S2HPOST
%%S2HPOST# Note that an \indexentry introduced anchor is here.
%%S2HPOST    if ( m-^[ ]?(<A NAME=[^>]*>)&#160;(</A>)- ) {
%%S2HPOST        \$last_invisanchor = 1;
%%S2HPOST        \$last_line = \$_;
%%S2HPOST        next;
%%S2HPOST
%%S2HPOST    }
%%S2HPOST
%%S2HPOST    print;
%%S2HPOST    \$last_line = \$_;
%%S2HPOST
%%S2HPOST# Encapsulate page in HTML tags.
%%S2HPOST    if ( \$nlines == 0 ) {
%%S2HPOST       print "<HTML>\n";
%%S2HPOST       \$nlines = 1;
%%S2HPOST    }
%%S2HPOST}
%%S2HPOST
%%S2HPOST# End encapsulate
%%S2HPOSTprint "</HTML>\n";
%%S2HPOST
%%S2HPOST#
%%S2HPOST# End-of-file.
%%S2HPOSTFOO
%%S2HPOST    chmod 755 ${DOC_CODE}$$.pl;
%%S2HPOST
%%S2HPOST# Make changes.
%%S2HPOST    cd ${DOC_CODE}.htx;
%%S2HPOST    echo '';
%%S2HPOST    echo "! star2html post-processing ${DOC_CODE}.";
%%S2HPOST    echo -n "! General HTML changes";
%%S2HPOST    foreach f ( *.html )
%%S2HPOST        cat $f | ../${DOC_CODE}$$.pl >! temp$$.html;
%%S2HPOST        mv -f temp$$.html $f;
%%S2HPOST        echo -n '.';
%%S2HPOST    end
%%S2HPOST    rm -f ../${DOC_CODE}$$.pl;
%%S2HPOST    rm -f temp$$.html;
%%S2HPOST    echo '';
%%S2HPOST    echo -n '! Correcting References, file: ';
%%S2HPOST    set REFS_FILE = `grep -l 'value="References' n*.html`;
%%S2HPOST    echo ${REFS_FILE}.;
%%S2HPOST    perl -e 's:<DD> :<DD>:' -p ${REFS_FILE} > t$$.html;
%%S2HPOST    mv -f t$$.html ${REFS_FILE};
%%S2HPOST    echo '! Hyperlinking document against Master Document Set at RAL.';
%%S2HPOST    cd ..;
%%S2HPOST    setenv HTX_PATH .;
%%S2HPOST    hlink .;
%%S2HPOST    echo '! Hyperlink done.';
%%S2HPOST    echo '! star2html post-processing complete.'
%%S2HPOST
%%S2HPOSTexit

% ? End of document specific commands
% -----------------------------------------------------------------------------
%  Title Page.
%  ===========
\renewcommand{\thepage}{\roman{page}}
\begin{document}
\thispagestyle{empty}

%  Latex document header.
%  ======================
\begin{latexonly}
   CCLRC / {\sc Rutherford Appleton Laboratory} \hfill {\bf \stardocname}\\
   {\large Particle Physics \& Astronomy Research Council}\\
   {\large Starlink Project\\}
   {\large \stardoccategory\ \stardocnumber}
   \begin{flushright}
   \stardocauthors\\
   \stardocdate
   \end{flushright}
   \vspace{-4mm}
   \rule{\textwidth}{0.5mm}
   \vspace{5mm}
   \begin{center}
   {\Large\bf \stardoctitle}
   \end{center}
   \vspace{5mm}

% ? Heading for abstract if used.
   \vspace{10mm}
   \begin{center}
      {\Large\bf Abstract}
   \end{center}
% ? End of heading for abstract.
\end{latexonly}

%  HTML documentation header.
%  ==========================
\begin{htmlonly}
   \xlabel{}
   \begin{rawhtml} <H1> \end{rawhtml}
      \stardoctitle
   \begin{rawhtml} </H1> \end{rawhtml}

% ? Add picture here if required.
% ? End of picture

   \begin{rawhtml} <P> <I> \end{rawhtml}
   \stardoccategory \stardocnumber \\
   \stardocauthors \\
   \stardocdate
   \begin{rawhtml} </I> </P> <H3> \end{rawhtml}
      \htmladdnormallink{CCLRC}{http://www.cclrc.ac.uk} /
      \htmladdnormallink{Rutherford Appleton Laboratory}
                        {http://www.cclrc.ac.uk/ral} \\
      \htmladdnormallink{Particle Physics \& Astronomy Research Council}
                        {http://www.pparc.ac.uk} \\
   \begin{rawhtml} </H3> <H2> \end{rawhtml}
      \htmladdnormallink{Starlink Project}{http://star-www.rl.ac.uk/}
   \begin{rawhtml} </H2> \end{rawhtml}
   \htmladdnormallink{\htmladdimg{source.gif} Retrieve hardcopy}
      {http://star-www.rl.ac.uk/cgi-bin/hcserver?\stardocsource}\\

%  HTML document table of contents.
%  ================================
%  Add table of contents header and a navigation button to return to this
%  point in the document (this should always go before the abstract \section).
  \label{stardoccontents}
  \begin{rawhtml}
    <HR>
    <H2>Contents</H2>
  \end{rawhtml}
  \renewcommand{\latexonlytoc}[0]{}
  \htmladdtonavigation{\htmlref{\htmladdimg{contents_motif.gif}}
        {stardoccontents}}

% ? New section for abstract if used.
  \section{\xlabel{abstract}Abstract}
% ? End of new section for abstract

\end{htmlonly}

% -----------------------------------------------------------------------------
% ? Document Abstract. (if used)
%  ==================
This document is designed as an aid to those who need to find, or use,
software for computer algebra.
Some of the text is based on a review undertaken to determine whether
the existing Starlink provision for computer algebra is sufficient and
appropriate.
At present there is a copy of Maple, available to anyone with a Starlink
account, on the central stadat machine.

The document compares the two main packages available for computer algebra,
gives short descriptions of other popular packages, includes a list of
`pointers' to other sources of information, and offers some general hints
on selecting software.
% ? End of document abstract
% -----------------------------------------------------------------------------
% ? Latex document Table of Contents (if used).
%  ===========================================
 \newpage
 \begin{latexonly}
   \setlength{\parskip}{0mm}
   \setcounter{tocdepth}{2}
   \latexonlytoc
   \setlength{\parskip}{\medskipamount}
%   \markright{\stardocname}{\stardocname}
 \end{latexonly}
% ? End of Latex document table of contents
% -----------------------------------------------------------------------------
\cleardoublepage
\renewcommand{\thepage}{\arabic{page}}
\setcounter{page}{1}

%%%%%%%%%%%%%%%%%%%%%%%%%%%%%%%%%%%%%%%%%%%%%%%%%%%%%%%%%%%%%%%%%%%%%%%%%%%
\section{\xlabel{introduction}\label{se_introduction}Introduction}
\markboth{\stardocname}{Introduction}

This document is designed as an aid to those who need to find, or use,
software for computer algebra.
Some of the text is based on a review undertaken to determine whether
the existing Starlink provision for computer algebra is sufficient and
appropriate.
At present there is a copy of Maple, available to anyone with a Starlink
account, on the central stadat machine.

The document compares the two main packages available for computer algebra,
gives short descriptions of other popular packages, includes a list of
`pointers' to other sources of information, and offers some general hints
on selecting software.


%%%%%%%%%%%%%%%%%%%%%%%%%%%%%%%%%%%%%%%%%%%%%%%%%%%%%%%%%%%%%%%%%%%%%%%%%%%
\section{\xlabel{maple_mathematica}\label{se_maple_mathematica}Maple and
         Mathematica}
\markboth{\stardocname}{Maple \& Mathematica}

The two major software packages available for computer algebra are Maple
and Mathematica.  Both Packages are commercial and have similar
capabilities.  Both packages are available for a wide range of operating
systems, including those which Starlink supports.


\subsection{Maple}

Maple is the on-going product of a research project at the University of
Waterloo in Canada.  Maple was first developed by the authors in the early
1980s in response to their perception that the (then) existing software in
the field was inadequate.
The current version is Maple V Release~4.

Maple is a general-purpose package for mathematical manipulation.
It is programmable, and the source code for its internal routines is
available as part of the package.
This allows easy inspection of the algorithms used in standard functions,
and adaption of these functions to similar tasks.
The program has extensive support for graphical display of functions and
equations etc., on-screen, in a manner resembling normal mathematical
notation.

Waterloo Maple have a set of world-wide web pages which start at:

\begin{itemize}

\item \htmladdnormallink{{\tt http://www.mathematics.com/}}
      {http://www.mathematics.com/}

\end{itemize}


\subsection{Mathematica}

Mathematica is a product of Wolfram Research Inc.\ founded by the
`architect' of the system, Stephen Wolfram.  Mathematica first appeared in
1988.  The most recent release is Version 3.0 which is styled as the
`World's only fully integrated technical computing system'.
Wolfram are well known for their no-holds-barred hype.

Mathematica, like Maple, offers capabilities for symbolic and numerical
computations.  Numeric computations can be carried out to `arbitrary'
precision, though obviously the higher the precision, the more time required
to complete the calculation.  There is a full suite of functions supporting
2- and 3-dimensional plotting of data and functions.  Mathematica
incorporates a graphics language capability which can be used to produce
visualisations of complex objects.

There is an {\sl Applications Library} available which includes a range of
application-tailored tools written in Mathematica's language.  These include:
a 3-D real-time graphics tool; a control-system tool; an experimental data
analyser; mechanical, electrical, and signal analysers.  There are also
packages for financial work.

There have been several papers and reviews critical of Mathematica; this
is at least partly due to the marketing strategy of the company.  At least
some of these critical reviews have turned on the debate about the difference
between accuracy and precision (sic).  There is an extensive
review available via FTP\cite{fateman}\@.

Wolfram have a set of world-wide web pages which start at:

\begin{itemize}

\item \htmladdnormallink{{\tt http://www.wri.com/}}{http://www.wri.com/}

\end{itemize}


\subsection{Comparison of Maple \& Mathematica}

Maple and Mathematica have largely the same capabilities in the area
of computer algebra, however, they do differ in some important ways,
reflecting their designs.

Both Mathematica and Maple provide on-line help facilities which document
their capabilities.  These help facilities are useful for locating
functions (and functionality) you might require.  The on-line help for
Maple is more extensive than that in Mathematica, providing examples and
cross-references not normally present in the Mathematica help.

Maple is written in its own programming language and this source is
available to the user who can inspect and modify the code, perhaps
adapting the code to specific requirements.  The source code for
Mathematica is written in C and is not generally available.

Programming in the two systems is different.  The Maple language is
similar in style to Pascal.
Mathematica supports several styles of programming:
object-oriented, procedural, rule-based and functional.
The naming convention for internal functions in Maple is less consistent
than that of Mathematica.
All built-in Mathematica functions start with a capital letter.
Built-in Maple functions can be fully upper-case, capitalised or fully
lower-case.

Mathematica has a large, monolithic kernel and can take a small amount of
time to load into memory.
Maple, on the other hand, is modular in design;
the Maple kernel loads functions into memory on demand and thus can be
expected to generally run in less memory than Mathematica.  This would
only be a problem if the system is expected to run on a PC with less than
32 Megabytes of RAM\@.

Both Maple and Mathematica support output of equations and the like
in \LaTeX\ format.

\subsection{Maple or Mathematica?}

It's difficult to make a firm recommendation of either Maple or
Mathematica\sgspec{---}{ - }it depends on what you want to do, what
hardware you have available, and so on.  It may be the case that the
various `frills' which Mathematica offers will be of use to you,
for example, the provision of the ability to manifest a function
in sound.

University-based individuals might investigate local expertise in one or
both of the packages; there is no substitute for a local guru when trying
to get something to work.


%%%%%%%%%%%%%%%%%%%%%%%%%%%%%%%%%%%%%%%%%%%%%%%%%%%%%%%%%%%%%%%%%%%%%%%%%%%
\section{\xlabel{starlink_maple}\label{se_starlink_maple}Central Starlink
         Maple Service}
\markboth{\stardocname}{The Central Starlink Maple service}

Starlink provides a central Maple service for those who do not have access
to computer algebra facilities at their Starlink site.
The system is installed on the stadat machine at RAL.
If you do not already have access to the machine\sgspec{---}{ - }some
Starlink sites may have a single account for all their
users\sgspec{---}{ - }you will have to apply for an account (not as painful
as it might sound).
The service, how to access it, and an introduction to usage, can be found in
the Starlink document \xref{SUN/107, {\sl Maple\sgspec{---}{ - }Mathematical
Manipulation Language}}{sun107}{}\@.


%%%%%%%%%%%%%%%%%%%%%%%%%%%%%%%%%%%%%%%%%%%%%%%%%%%%%%%%%%%%%%%%%%%%%%%%%%%
\section{\xlabel{other_software}\label{se_other_software}Other Software}
\markboth{\stardocname}{Other Software}

At the time of writing there are no packages of comparable capabilities and
widespread use as Maple and Mathematica.
Both products attempt to be market leaders and undergo continuous development.
There are many packages with subsets of the capabilities available and these
are comprehensively listed in a database maintained at Berkeley (see the
\htmlref{Appendix}{se_packages}).

There is a newsgroups which may be of interest to those investigating
computer algebra software:

\begin{itemize}

\item \htmladdnormallink{{\tt news:sci.math.symbolic}}{sci.math.symbolic}

\end{itemize}

The appendix data are drawn from the extensive on-line database about
mathematical software at Berkeley.  The source file is:

\begin{itemize}

\item \htmladdnormallink{\tt
        ftp://math.berkeley.edu/pub/Symbolic\_Soft/Available\_Systems}
       {ftp://math.berkeley.edu/pub/Symbolic_Soft/Available_Systems}

\end{itemize}

There is more extensive information available on many of these packages
at this site.

One package mentioned only very briefly in the Appendix is MathSoft's
{\sl MathCad.}  MathCad offers many of the facilities present in Maple
and Mathematica, its symbolic technology is actually based on that of
maple.  MathCad is available for MS-Windows and Macintosh systems.
MathCad's world-wide web homepage is:

\begin{itemize}

\item \htmladdnormallink{\tt http://www.mathsoft.com/all60.htm}
       {http://www.mathsoft.com/all60.htm}

\end{itemize}

Per Bergehed maintains a `Math Software' web page which is a useful
resource:

\begin{itemize}

\item \sgspec{{\tt
       http://www.dtek.chalmers.se/\symbol{126}d0pbm/math/math.html}}
       {\htmladdnormallink{
       \verb+http://www.dtek.chalmers.se/~d0pbm/math/math.html+}
       {http://www.dtek.chalmers.se/\~{}d0pbm/math/math.html}}

\end{itemize}

%%%%%%%%%%%%%%%%%%%%%%%%%%%%%%%%%%%%%%%%%%%%%%%%%%%%%%%%%%%%%%%%%%%%%%%%%%%
\begin{thebibliography}{99}\addcontentsline{toc}{section}{References}
\markboth{\stardocname}{References}

\bibitem{leaves} Bruce W.~Char, Keith O.~Geddes, Gaston H.~Gonnet,
     Michael B.~Monagan and Stephen M.~Watt,\\
     {\sl MAPLE First Leaves, A Tutorial Introduction to Maple,}\\
     (Third Edition), Publ.~Waterloo Maple Publishing, 1990.

\bibitem{maple} Bruce W.~Char, Keith O.~Geddes, Gaston H.~Gonnet,
     Michael B.~Monagan and Stephen M.~Watt,\\
     {\sl MAPLE Reference Manual,}\\
     (Fifth Edition), Publ.~Waterloo Maple Publishing, March 1988.

\bibitem{fateman} Richard J.~Fateman,\\
     {\sl A Review of Mathematica,}\\
     \htmladdnormallink{{\tt
      ftp://math.berkeley.edu/pub/Symbolic\_Soft/Mathematica/Review.tex}}
     {ftp://math.berkeley.edu/pub/Symbolic_Soft/Mathematica/Review.tex}.

\bibitem{wolfram2} Stephen Wolfram,\\
     {\sl Mathematica, A system for Doing Mathematics by Computer,}\\
     (Second Edition), Publ.~Addison Wesley, July 1993.

\bibitem{wolfram3} Stephen Wolfram,\\
     {\sl The Mathematica Book,}\\
     (Third Edition), Wolfram Media/Cambridge University Press, 1996.

\bibitem{deSouza} Paulo Ney de Souza,\\
     {\sl Computer Algebra Systems,}\\
     Department of Mathematics, University of California, Berkeley,
     California.\\
     \htmladdnormallink{{\tt
      ftp://math.berkeley.edu/pub/Symbolic\_Soft/Available\_Systems}}
     {ftp://math.berkeley.edu/pub/Symbolic_Soft/Available_Systems}

\end{thebibliography}


%%%%%%%%%%%%%%%%%%%%%%%%%%%%%%%%%%%%%%%%%%%%%%%%%%%%%%%%%%%%%%%%%%%%%%%%%%%
\newpage
\appendix
\section{\xlabel{packages}\label{se_packages}Algebra Software Packages}
\markboth{\stardocname}{Algebra Software Packages}

{\sl Only those packages from the software list with computer algebra
capabilities, and available for PCs or Starlink-supported operating
systems, are included.  Some lengthy lists of supported operating systems
have been removed.}

This is the  list of  currently  developed  and   distributed   software  for
symbolic mathematical applications. It was started a few years ago by  Booker
Bense of the UC San Diego Supercomputer Center. He is responsible for     the
lay-out and for the idea of collecting this database. Most of the information
here is obtained from the developers and we try to keep it up to date, a  not
so easy task.

Electronic mail concerning this database and additions to it could be
sent to:

\begin{verbatim}
   ca@math.berkeley.edu
\end{verbatim}

\subsection{General Purpose}

\subsubsection{Axiom}
\begin{verbatim}
  Type:      Commercial
  Machines:  SUN Sparc, IBM RS 6000's and other IBM mainframe platforms.
  Contact:   USA:                         Rest of the world:
             ryan@nag.com                 valerie@nag.co.uk
             NAG Inc                      NAG Ltd
             1400 Opus Place, Suite 200   Wilkinson House
             Downers Grove, Il 60515-5702 Jordan Hill Rd.
             Phone: 708-971-2337          Oxford OX2 8DR
             Fax:   708-971-2706          United Kingdom
                                          Phone: [+44] 865-51-1245
                                          Fax:   [+44] 865-31-0139
             or
             axiom@watson.ibm.com
  Version:   1.2
  Comments:  General purpose. Object oriented with multiple inheritance
             based on algebraic concepts. Powerful type-inferencing
             techniques to minimise the need for type declarations.
             Hypertext browser and on-line documentation with source code
             available for all library functions. High level interactive
             language and powerful graphics capability. Design goal:
             unlimited extensibility without degradation in performance
             or usability. Release 2 will provide ability to create
             stand-alone packages and links to Nag Fortran Libraries of
             numerical and statistical routines and a compiler for Axiom
             programs.
\end{verbatim}

\newpage
\subsubsection{Derive}
\begin{verbatim}
  Type:      Commercial
  Machines:  Runs on PC's and HP 95's.
  Contact:   Soft Warehouse Inc. 3660 Waialae Ave, Suite 304
             Honolulu Hi 96816-3236 Phone: 808-734-5801
  Version:   2.59
  Comments:  Very robust, gets problems that other larger programs fail
             on. Low cost. Runs on the tiny palmtops like HP 95-100 and
             on meager machines like the PC XT.
\end{verbatim}

\subsubsection{FORM}
\begin{verbatim}
  Type:      Version 1 is public domain, Version 2 is commercial
  Machines:  MS-DOS, AtariSt, Amiga, Mac, Sun3, SunSparc, Apollo, NeXT,
             VAX/VMS, VAX/Ultrix, DECStation, HP, IBM Risc, IBM 3090,
             MIPS, Alliant, Gould and SGI.
  Contact:   t68@nikhef.nl (Jos Vermaseren)
             Binary versions of version 1 are available
             by anonymous FTP from nikhef.nikhef.nl (192.16.199.1)
             Version 2 is commercially distributed by:
             CAN, Kruislaan 419, 1098 VA Amsterdam,
             The Netherlands
             form@can.nl  Phone: [+31] 20 560-8400 Fax: [+31] 20 560-8448
  Version:   1 and 2.
  Comments:  General purpose, designed for BIG problems, batch-like
             interface.
\end{verbatim}

\subsubsection{GNU-calc}
\begin{verbatim}
  Type:      GNU copyleft
  Machines:  Where Emacs runs.
  Contact:   Free Software Foundation
  Comments:  It runs inside GNU Emacs and is written entirely in Emacs
             Lisp. It does the usual things: arbitrary precision integer,
             real, and complex arithmetic (all written in Lisp),
             scientific functions, symbolic algebra and calculus,
             matrices, graphics, etc. and can display expressions with
             square root signs and integrals by drawing them on the
             screen with ASCII characters. It comes with a well written
             600 page on-line manual. You can FTP it from any GNU site.
\end{verbatim}

\subsubsection{JACAL}
\begin{verbatim}
  Type:      Gnu CopyLeft
  Machines:  Needs Scheme ver. R4RS.
  Contact:   Aubrey Jaffer, 84 Pleasant St. Wakefield MA 01880, USA.
             jaffer@martigny.ai.mit.edu
  Version:   1a4
  Comments:  An IBM PC version on floppy for $99 is available from
             Aubrey Jaffer, 84 Pleasant St. Wakefield MA 01880, USA.
             There is a Scheme implementation available from the same
             sites as JACAL which will run it on just about any machine
             with a C compiler (Amiga, Atari-ST, MacOS, MS-DOS, OS/2,
             NOS/VE, Unicos, VMS, Unix and similar systems). FTP sites:
             altdorf.ai.mit.edu:archive/scm/jacal1a4.tar.gz
             prep.ai.mit.edu:pub/gnu/jacal/jacal1a4.tar.gz
             nexus.yorku.ca:pub/scheme/new/jacal1a4.tar.gz
             ftp.maths.tcd.ie:pub/bosullvn/jacal/jacal1a4.tar.gz
             JACAL is a symbolic mathematics system for the simplification
             and manipulation of equations and single and multiple valued
             algebraic expressions constructed of numbers, variables,
             radicals, and algebraic functions, differential, and
             holonomic functions.  In addition, vectors and matrices of
             the above objects are included.
\end{verbatim}

\subsubsection{Macsyma}
\begin{verbatim}
  Macsyma was developed at MIT and has spun-off a series of different
  versions that run on specific machines. The full list consists of:
  Macsyma, DOE-Macsyma, Maxima, Aljabr, Paramacs and Vaxima.

  (i) Macsyma
  Type:      Commercial
  Machines:  PC386/387 and 486 (no SX's) PC's.
             SPARC Solaris, SPARC SunOS, HP9000 RISC, and IBM
             RS/6000, and the SGI IRIX release will ship soon
  Contact:   Macsyma Inc,  20 Academy St., Arlington MA 02174-6436
             info-macsyma@macsyma.com Phone: 800-MACSYMA Fax: 617-646-3161
  Version:   Version 2 for PC's and 419 for Unix Workstations.
  Comments:  General purpose, many diverse capabilities, one of the
             oldest around. Includes proprietary improvements from
             Symbolics and Macsyma Inc. Descendant of MIT's Macsyma.
             Recent major enhancements include new capabilities in ODE's,
             Laplace transforms, integrations, inequalities, linear
             algebra and partial differential equations.

  (ii) DOE-Macsyma:
  Type:      Distribution fee only
  Machines:  The public domain Franz Lisp version, runs on Unix machines,
             including Suns and Vaxes using Unix.
  Contact:   ESTSC - Energy Science & Technology Software Center
             P. O. Box 1020 Oak Ridge TN 37831-1020
             Phone: 615-576-2606
  Comments:  Help with DOE-Macsyma; general and specific help with issues
             such as support, new versions, etc: lph@paradigm.com
             Leo Harten from Paradigm Associates, Inc. 29 Putnam Avenue,
             Suite 6, Cambridge, MA 02139 617-492-6079.

  (iii) Maxima
  Type:      License for a fee. Get license from ESTSC before download.
  Machines:  Unix workstations (Sun, MIPS, HP, PC's) and PC-DOS (beta).
  Contact:   Bill Schelter (wfs@math.utexas.edu)
  Version:   4.155
  Comments:  General purpose -  MIT Macsyma family. Common Lisp
             implementation by William F. Schelter, based on Kyoto
             Common Lisp. Modified version of DOE-Macsyma available
             to ESTSC (DOE) sites. Get the license from ESTSC (Phone:
             615-576-2606) and then download the software from
             DOS: math.utexas.edu:pub/beta-max.zip   or
             UNIX: rascal.ics.utexas.edu:pub/maxima-4-155.tar.Z
             Currently their charge for 1 machine license is $165 to
             universities. Site licenses are also available.

  (iv) Aljabr
  Type:      Commercial
  Machines:  Mac's with 4Meg of RAM.
  Contact:   Fort Pond Research, 15 Fort Pond Road, Acton MA  01720 US
             aljabr@fpr.com, Phone: 508-263-9692
  Version:   1.0
  Comments:  MIT Macsyma family descendant, uses Franz LISP.

  (v)  Paramacs
  Type:      Commercial
  Machines:  VAX-VMS, Sun-3, Sun-4, (SGI and Mac's on the works)
  Contact:   Paradigm Associates, Inc.
             29 Putnam Avenue, Suite 6, Cambridge, MA 02139
             lph@paradigm.com, Phone: 617-492-6079
  Version:   1
  Comments:  Improved SHARE library and enhanced ODE solver (over
             DOE-Macsyma) and animated graphics. Maintenance, phone
             services, and site licensing available.

  (vi) Vaxima
  Type:      Distribution fee only
  Machines:  VAX-Unix
  Contact:   ESTSC (see DOE-Macsyma above)
  Comments:  General purpose -  MIT Macsyma family descendant.
             Includes source and binaries with assembler for Macsyma
             and Franz Lisp Opus 38
\end{verbatim}

\subsubsection{Maple}
\begin{verbatim}
  Type:      Commercial
  Machines:  Most common systems.
  Contact:   Waterloo Maple Software, 450 Phillip Street,
             Waterloo, Ontario, Canada     N2L 5J2
             info@maplesoft.on.ca, support@maplesoft.on.ca
             Phone: 519-747-2373 and 800-267-6583
             Fax: 519-747-5284
  Version:   5 release 3.
  Comments:  General purpose, source available for most routines. Graphic
             and animation. On-screen and printable real-math notation.
             Worksheet interface on Amiga, Macintosh, NeXT, Unix X
             Windows, PC Windows and DEC VMS. Demo programs available
             for DOS, Windows 3.1 and Macintosh. The demo of the program
             for PC-DOS can be obtained from anonymous FTP at
             wuarchive.wustl.edu:/edu/math/msdos/modern.algebra/maplev.zip
             Student versions available for MAC and PC's. A share library
             contains many additional Maple routines, packages, and
             application worksheets. It is distributed with the product,
             and is also available through anonymous ftp from

                     daisy.uwaterloo.ca      129.97.140.58
                     neptune.inf.ethz.ch     129.132.101.33
                     ftp.inria.fr            128.93.2.54
                     canb.can.nl             192.16.187.2
                     ftp.maplesoft.on.ca

             Maple's symbolic technology is incorporated into MathCAD
             (MathSoft, Inc), MATLAB (The MathWORKS, Inc) and PV-WAVE
             (Visual Numerics, Inc).
\end{verbatim}

\subsubsection{MAS}
\begin{verbatim}
  Type:      Anonymous FTP
  Machines:  Atari ST (TDI and SPC Modula-2 compilers), IBM PC/AT
             (M2SDS and Topspeed Modula-2 compilers) and Commodore
             Amiga (M2AMIGA compiler).
  Contact:   H. Kredel. Computer Algebra Group
             University of Passau, Germany
  Version:   0.60
  Comments:  MAS is an experimental computer algebra system combining
             imperative programming facilities with algebraic
             specification capabilities for design and study of algebraic
             algorithms. MAS is available via anonymous FTP from:
             alice.fmi.uni-passau.de = 123.231.10.1
\end{verbatim}

\subsubsection{Matlab Symbolic Math Toolbox}
\begin{verbatim}
  Type:      Commercial
  Machines:  Microsoft Windows, Sun SPARC, HP 9000/700, DEC, IBM RS/6000,
             Silicon Graphics.  Under preparation: Macintosh
  Contact:   The MathWorks, Inc.
             24 Prime Park Way
             Natick, MA  01760-1500
             info@mathworks.com  Phone: 508-653-1415  Fax: 508-653-2997
  Version:   1.0
  Comments:  General purpose. Based on the Maple V computational kernel,
             with a set of MATLAB M-files designed to make symbolic
             computation more easy. The MATLAB Symbolic Math Toolbox
             includes the computational functions in the Maple kernel
             and the linear algebra package.  An Extended Symbolic Math
             Toolbox is also available that supports Maple procedure
             execution and includes other Maple packages for statistics,
             Grobner basis, combinatorial functions, number theory,
             Euclidean geometry, Lie symmetries, etc.
\end{verbatim}


\subsubsection{Mathematica}
\begin{verbatim}
  Type:      Commercial
  Machines:  Most common systems.
  Contact:   Wolfram Research, Inc.
             100 Trade Center Drive, Champaign IL 61820-7237
             Phone: 1-800-441-MATH  & 217-398-0700
             Fax: 217-398-0747
             email: info@wri.com

             Wolfram Research Europe Ltd.
             10 Blenheim Office Park, Lower Road, Long Hanborough,
             Oxfordshire OX8 8LN, UNITED KINGDOM
             Phone: +44-(0)1993-883400, Fax: +44-(0)1993-883800
             email: info-euro@wri.com

  Version:   2.2
  Comments:  General purpose, Notebook interface on Windows, NEXTSTEP, Mac,
             and most Unix platforms (Sun SPARC and Solaris, Silicon
             Graphics, HP, DEC RISC, IBM Risc System/6000) nice graphics.
             Packages include: MathTensor for Tensors and NCAlgebra for
             Non-Commutative Algebra  & Combinatorica for Graph Theory.
             Specialty packages for Finance and Electrical Eng. are sold by
             WRI. Student version available for MAC and PC's. MathLink
             libraries included for communicating with external programs
             (not MS-DOS version).
\end{verbatim}

\subsubsection{Mock-Mma}
\begin{verbatim}
  Type:      Anonymous FTP
  Machines:  Anywhere running Common LISP.
  Contact:   Richard Fateman
             Department of Comp. Science
             University of California
             Berkeley CA 94720
             fateman@cs.berkeley.edu
  Version:   1.5
  Comments:  It is a framework of a parser, display, polynomial
             manipulator, and a few other pieces, that allow one to
             program other pieces of a common-lisp systems for computer
             algebra. The pieces supplied are roughly consistent with the
             conventions adopted by the Mathematica system.  Available
             through anonymous FTP from from peoplesparc.berkeley.edu:
             /pub directory.
\end{verbatim}

\subsubsection{MuPAD}
\begin{verbatim}
  Type:      Anonymous FTP (free, but not PD; registration required)
  Machines:  Unix workstations (IBM, Sun),
             Prereleased Versions exist for PC-386 (DOS & Unix), Mac,
             Sequent. Please call for others.
  Contact:   MuPAD-Distribution, Fachbereich 17 Mathematik-Informatik,
             Univ. of Paderborn, Warburger Str. 100,
             D 4790 Paderborn, Germany.
             E-mail: MuPAD-distribution@uni-paderborn.de
             Phone: +49-5251-602633
  Version:   1 release 1.
  Comments:  General purpose, source available for library routines,
             graphics support, source code debugger, on-line hypertext
             help system.  Obtained by anonymous FTP at
             athene.uni-paderborn.de:unix/MuPAD
\end{verbatim}

\subsubsection{Reduce}
\begin{verbatim}
  Type:      Commercial
  Machines:  Just about anything you can think of.
  Contact:   Anthony C. Hearn, RAND, 1700 Main Street
             P. O. Box 2138, Santa Monica CA 90407-2138 U.S.A.
             Phone: 310-393-0411 Ext. 7681  Fax: 310-393-4818
             reduce@rand.org
  Version:   3.5
  Comments:  Interactive system and programming language for general
             purpose symbolic and numeric calculations. Specific
             application packages for various fields. Complete source
             available. Demos for MS DOS, Windows and Linux available
             from ftp.zib-berlin.de:pub/reduce/demo. Demos for Macintosh
             and MS DOS available from ftp.bath.ac.uk:pub/jpff/REDUCE.
             More complete information is available from URL
             http://www.rrz.uni-koeln.de/REDUCE/ or by sending the email
             message "send info-package" to reduce-netlib@rand.org.
\end{verbatim}


\subsubsection{SACLIB}
\begin{verbatim}
  Type:      Anonymous FTP, registration required.
  Machines:  Sun Sparc, Ultrix, Sequent, Generic Unix and Amiga.
  Contact:   saclib@risc.uni-linz.ac.at
             SACLIB Maintenance
             Research Institute for Symbolic Computation
             Johannes Kepler University
             4020 Linz
             Austria
  Version:   1.1
  Comments:  SACLIB is a library of C programs for computer algebra
             derived from the SAC2 system, and incorporating many
             improvements. It contains programs for list processing,
             infinite precision arithmetic (integer, rational and
             modular), operations on multivariate polynomials, polynomial
             real root isolation and refinement, and operations with real
             algebraic numbers and polynomials having algebraic number
             coefficients. Improved and extended versions of SACLIB, and
             several SACLIB application packages, are already in advanced
             stages of development.  These include arbitrary precision
             floating point and interval arithmetic, improved greatest
             common divisor and factorization algorithms for polynomials,
             isolation and refinement of complex roots of polynomials,
             Groebner basis computation and quantifier elimination.
\end{verbatim}

\subsubsection{SENAC}
\begin{verbatim}
  Type:      Commercial
  Machines:  Microsoft Windows (PCs), Unix workstations (DEC, HP, IBM,
             Sun, SGI), Convex, and Cray. For other implementations
             please contact us.
  Contact:   senac@ulcc.ac.uk, senac@waikato.ac.uk
             For Europe:                        For rest of the world:
             Minaz Punjani                      Kevin. A. Broughan
             University of London Comp Centre   Mathematical Soft Project
             20 Guilford Street                 University of Waikato
             London WC1N 1DZ                    Private Bag 3105
             England                            Hamilton, New Zealand
             Tel: [+44] 71 405 8400             Tel: [+64] 7 856 2889
             Fax: [+44] 71 242 1845             Fax: [+64] 7 838 4155
  Version:   6.0
  Comments:  General purpose  environment for numeric and algebraic
             computing. An interactive computer algebra host language, an
             interactive library including Numerical Recipes, graphics
             library with postscript output, a fully automated symbolic
             numeric interface to the NAG library, a symbolic graphic
             interface to the NAG Graphics library.
             Specialities in large scale optimisation and finite element
             modelling.
\end{verbatim}


\subsection{Algebra \& Number Theory}

\subsubsection{Cocoa}
\begin{verbatim}
  Type:      Anonymous FTP
  Machines:  Mac's; PC's on the works.
  Contact:   Gianfranco Niesi
             Dipartimento di Matematica
             Via L. B. Alberti, 4
             I-16134   Genova  Italy
             cocoa@igecuniv.bitnet
  Version:   1.5
  Comments:  Computations in commutative algebra. Available on the
             anonymous FTP at  ftp.dm.unipi.it:/pub/alpi-cocoa/cocoa
             or diskettes (2) sent in a self-addressed to the author.
\end{verbatim}

\subsubsection{GANITH}
\begin{verbatim}
  Type:      Version 1 is public domain. Current Version 3 is commercial
  Machines:  Unix Workstations under X-11 R5 (SUN, SGI, HP, ..)
  Contact:   Chandrajit Bajaj
             Department of Computer Science, Purdue University
             West Lafayette, IN 47907
             bajaj@cs.purdue.edu
  Comments:  GANITH is an algebraic geometry toolkit, for the computation
             and visualization of solutions to systems of algebraic
             equations.
             Example applications of this for geometric modeling and
             computer graphics are algebraic curve and surface
             display, curve-curve intersections, surface-surface
             intersections, global and local parameterizations,
             implicitization.
             GANITH also incorporates techniques for interpolation and
             least-squares approximation (multivariate data fitting)
             with algebraic curves and surfaces.
             Version 1 of GANITH is available from anonymous FTP at
             ftp.cs.purdue.edu in the file /pub/shastra/ganith-sun.tar.Z
             The latest release is being sold by the Technology Transfer
             Division of Purdue University for $250 (binary) and
             $1000 (source code).
\end{verbatim}

\subsubsection{Magma}
\begin{verbatim}
  Type:      Cost recovery.
  Machines:  Sun 3, Sun4, DECStation, HP9000/7000, IBM RS6000 and Mac
             running A/UX 2.01
  Contact:   The Secretary
             Computational Algebra Group
             School of Mathematics
             University of Sydney
             NSW 2006
             Australia
             Email: magma@maths.su.oz.au
             Telephone: +61 2 692 3338        Fax: +61 2 692 4534
  Version:   1.01-2
  Comments:  The system is designed to support computation in algebra,
             number theory, geometry and algebraic combinatorics. It
             has an advanced functional programming language with many
             novel features designed for concise and efficient
             specification of algebraic algorithms. The kernel has
             (coded in) the fundamental algorithms for ring theory
             (polynomial rings, matrix rings, integer rings), field
             theory (general algebraic number fields -- KANT V.2, finite
             fields, real and complex fields), module theory, group
             theory (fp groups, permutation groups, soluble groups and
             matrix groups) and algebraic combinatorics (coding theory
             and graph theory).
\end{verbatim}

\subsubsection{PARI}
\begin{verbatim}
  Type:      Anonymous FTP
  Machines:  Most  Unix workstations, NeXT, PC386 and  Mac. Special 64-bit
             version for the Dec Alpha machines.
  Contact:   pari@ceremab.u-bordeaux.fr
  Version:   1.38
  Comments:  Number theoretical computations, source available, key
             routines are in assembler, ASCII and Xwindows graphics.
             Complete package for algebraic number theory computations
             including handling of ideals, prime ideals, prime ideal
             factorization, p-adic factorization, etc ...
             Unix and Amiga versions available from:
             megrez.ceremab.u-bordeaux.fr (147.210.16.17), math.ucla.edu
             and ftp.inria.fr. PC-DOS version available from anonymous FTP
             at wuarchive.wustl.edu:/edu/math/msdos/modern.algebra/pari386
             or  math.ucla.edu (128.97.64.16) in the directory /pub/pari.
\end{verbatim}

\subsubsection{SIMATH}
\begin{verbatim}
  Type:      Anonymous FTP
  Machines:  Suns.
  Contact:   SIMATH-Gruppe, Lehrstuhl Prof. Dr. H.G. Zimmer,
             FB 9 Mathematik, Universitaet des Saarlandes,
             D-W-6600 Saarbruecken, Germany.
             E-mail: simath@math.uni-sb.de,
             phone: [0681]302-2206.
  Version:   3.7
  Comments:  SIMATH is written in C, contains an interactive
             calculator (simcalc) and many C-functions over
             algebraic structures as arbitrary long integers,
             rational numbers, floating point numbers, poly-
             nomials, Galoisfields, matrices, elliptic curves,
             algebraic number fields, modular integers, finite fields,
             Groebner basis, etc...
             Version 3.7 contains a handbook written in English.
             Available from: ftp.math.uni-sb.de (134.96.32.23)
             or ftp.math.orst.edu (128.193.80.160). You have to
             get one extra file by an Email-Server, see the
             installation hints.
\end{verbatim}


\subsection{PC Shareware}

\subsubsection{CLA}
\begin{verbatim}
  Type:      Anonymous FTP
  Machines:  PC-DOS
  Contact:   Lenimar Nunes de Andrade (ccendm03@brufpb.bitnet)
             UFPB - CCEN - Dep.  de Matema'tica
             58.059 - Joa~o Pessoa, PB - BRAZIL
  Version:   2.0
  Comments:  A linear or matrix algebra package which computes
             rank, determinant, row-reduced echelon form, Jordan
             canonical form, characteristic equation, eigenvalues,
             etc. of a matrix. Available from anonymous FTP at
             wuarchive.wustl.edu:/edu/math/msdos/linear.algebra/cla20.zip
\end{verbatim}

\subsubsection{Mercury}
\begin{verbatim}
  Type:      Shareware
  Machines:  PC-DOS
  Contact:   Roger Schlafly, Real Software, PO Box 1680, Soquel, CA 95073
             Phone or Fax: 408-476-3550, 76646.323@compuserve.com
             TTDX08A@prodigy.com
  Version:   2.06
  Comments:  Limited in symbolic capabilities, but is extremely adept
             at numerically solving equations and produces publication
             quality graphical output. This used to be Borland's Eureka!,
             but when Borland abandoned it, its original author started
             selling it as shareware under the name Mercury. Available
             from anonymous FTP at
             wuarchive.wustl.edu:/edu/math/msdos/calculus/mrcry206.zip
\end{verbatim}

\subsubsection{PFSA}
\begin{verbatim}
  Type:      Public Domain
  Machines:  PC-DOS and a Unix non-interactive version.
  Contact:   Don Stevens, Courant Institute, 251 Mercer St., NY NY 10012
             stevens@cims.nyu.edu, Phone: 212-998-3275
  Version:   5.46
  Comments:  Written if Fortran, it is very fast but is limited to
             polynomial algebra and calculus  The DOS version is available
             from the anonymous FTP at wuarchive.wustl.edu:/edu/math/msdos
             /modern.algebra/vol546.zip and the Unix version at
             math.berkeley.edu:pub/PFSA
\end{verbatim}

\subsubsection{SymbMath}
\begin{verbatim}
  Type:      Shareware, student and advanced versions.
  Machines:  IBM PC
  Contact:   Dr Weiguang Huang, Dept. of Analytical Chemistry
             University of New South Wales - Kensington NSW 2033 Australia
             Phone: [61] 2-697-4643, Fax: [61] 2-662-2835
             w.huang@unsw.edu.au  or s9300078@cumulus.csd.unsw.oz.au
  Version:   3.1
  Comments:  Runs on plain (640k) DOS machines (8086). An expert system
             that can learn from the user. More capable versions are
             available by mail-order from the author.
             Also available by anonymous FTP from:
             math.bekerley.edu: /pub/Software/SymbMath/sm31a.zip
             oak.oakland.edu: /pub/msdos/calculator/sm31a.zip
             garbo.uwasa.fi: /pc/math/sm31a.zip
             wuarchive.wustl.edu: /mirrors/msdos/calculator/sm31a.zip
             or by sending the following e-mail:
             /pdget mail pd:<msdos.calculator>sm31a.zip  to
             listserv@vm1.nodak.edu, or listserv@ndsuvm1.bitnet
\end{verbatim}


\end{document}
