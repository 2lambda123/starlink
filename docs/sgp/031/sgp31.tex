\documentclass[twoside,11pt]{article}
\pagestyle{myheadings}

% -----------------------------------------------------------------------------
% ? Document identification
\newcommand{\stardoccategory}  {Starlink General Paper}
\newcommand{\stardocinitials}  {SGP}
\newcommand{\stardocsource}    {sgp\stardocnumber}
\newcommand{\stardocnumber}    {31.11}
\newcommand{\stardocauthors}   {\small M D Lawden}
\newcommand{\stardocdate}      {\small 17 June 1996}
\newcommand{\stardoctitle}     {\Large\bf STARLINK}
% ? End of document identification
% -----------------------------------------------------------------------------

\newcommand{\stardocname}{\stardocinitials /\stardocnumber}
\markright{\stardocname}
\setlength{\textwidth}{100mm}
\setlength{\textheight}{154mm}
\setlength{\topmargin}{0mm}
\setlength{\oddsidemargin}{0mm}
\setlength{\evensidemargin}{0mm}
\setlength{\parindent}{0mm}
\setlength{\parskip}{\medskipamount}
\setlength{\unitlength}{1mm}

% -----------------------------------------------------------------------------
%  Hypertext definitions.
%  ======================
%  These are used by the LaTeX2HTML translator in conjunction with star2html.

%  Comment.sty: version 2.0, 19 June 1992
%  Selectively in/exclude pieces of text.
%
%  Author
%    Victor Eijkhout                                      <eijkhout@cs.utk.edu>
%    Department of Computer Science
%    University Tennessee at Knoxville
%    104 Ayres Hall
%    Knoxville, TN 37996
%    USA

%  Do not remove the %begin{latexonly} and %end{latexonly} lines (used by 
%  star2html to signify raw TeX that latex2html cannot process).
%begin{latexonly}
\makeatletter
\def\makeinnocent#1{\catcode`#1=12 }
\def\csarg#1#2{\expandafter#1\csname#2\endcsname}

\def\ThrowAwayComment#1{\begingroup
    \def\CurrentComment{#1}%
    \let\do\makeinnocent \dospecials
    \makeinnocent\^^L% and whatever other special cases
    \endlinechar`\^^M \catcode`\^^M=12 \xComment}
{\catcode`\^^M=12 \endlinechar=-1 %
 \gdef\xComment#1^^M{\def\test{#1}
      \csarg\ifx{PlainEnd\CurrentComment Test}\test
          \let\html@next\endgroup
      \else \csarg\ifx{LaLaEnd\CurrentComment Test}\test
            \edef\html@next{\endgroup\noexpand\end{\CurrentComment}}
      \else \let\html@next\xComment
      \fi \fi \html@next}
}
\makeatother

\def\includecomment
 #1{\expandafter\def\csname#1\endcsname{}%
    \expandafter\def\csname end#1\endcsname{}}
\def\excludecomment
 #1{\expandafter\def\csname#1\endcsname{\ThrowAwayComment{#1}}%
    {\escapechar=-1\relax
     \csarg\xdef{PlainEnd#1Test}{\string\\end#1}%
     \csarg\xdef{LaLaEnd#1Test}{\string\\end\string\{#1\string\}}%
    }}

%  Define environments that ignore their contents.
\excludecomment{comment}
\excludecomment{rawhtml}
\excludecomment{htmlonly}

%  Hypertext commands etc. This is a condensed version of the html.sty
%  file supplied with LaTeX2HTML by: Nikos Drakos <nikos@cbl.leeds.ac.uk> &
%  Jelle van Zeijl <jvzeijl@isou17.estec.esa.nl>. The LaTeX2HTML documentation
%  should be consulted about all commands (and the environments defined above)
%  except \xref and \xlabel which are Starlink specific.

\newcommand{\htmladdnormallinkfoot}[2]{#1\footnote{#2}}
\newcommand{\htmladdnormallink}[2]{#1}
\newcommand{\htmladdimg}[1]{}
\newenvironment{latexonly}{}{}
\newcommand{\hyperref}[4]{#2\ref{#4}#3}
\newcommand{\htmlref}[2]{#1}
\newcommand{\htmlimage}[1]{}
\newcommand{\htmladdtonavigation}[1]{}

% Define commands for HTML-only or LaTeX-only text.
\newcommand{\html}[1]{}
\newcommand{\latex}[1]{#1}

% Use latex2html 98.2.
\newcommand{\latexhtml}[2]{#1}

%  Starlink cross-references and labels.
\newcommand{\xref}[3]{#1}
\newcommand{\xlabel}[1]{}

%  LaTeX2HTML symbol.
\newcommand{\latextohtml}{{\bf LaTeX}{2}{\tt{HTML}}}

%  Define command to re-centre underscore for Latex and leave as normal
%  for HTML (severe problems with \_ in tabbing environments and \_\_
%  generally otherwise).
\newcommand{\setunderscore}{\renewcommand{\_}{{\tt\symbol{95}}}}
\latex{\setunderscore}

%  Redefine the \tableofcontents command. This procrastination is necessary 
%  to stop the automatic creation of a second table of contents page
%  by latex2html.
\newcommand{\latexonlytoc}[0]{\tableofcontents}

% -----------------------------------------------------------------------------
%  Debugging.
%  =========
%  Remove % on the following to debug links in the HTML version using Latex.

% \newcommand{\hotlink}[2]{\fbox{\begin{tabular}[t]{@{}c@{}}#1\\\hline{\footnotesize #2}\end{tabular}}}
% \renewcommand{\htmladdnormallinkfoot}[2]{\hotlink{#1}{#2}}
% \renewcommand{\htmladdnormallink}[2]{\hotlink{#1}{#2}}
% \renewcommand{\hyperref}[4]{\hotlink{#1}{\S\ref{#4}}}
% \renewcommand{\htmlref}[2]{\hotlink{#1}{\S\ref{#2}}}
% \renewcommand{\xref}[3]{\hotlink{#1}{#2 -- #3}}
%end{latexonly}
% -----------------------------------------------------------------------------
% ? Document specific \newcommand or \newenvironment commands.
% ? End of document specific commands
% -----------------------------------------------------------------------------
\newcommand{\latexonlysection}[1]{\section{#1}}
\newcommand{\latexonlysubsection}[1]{\subsection{#1}}
\newcommand{\latexonlysubsubsection}[1]{\subsubsection{#1}}
\begin{htmlonly}
   \newcommand{\latexonlysection}[1]{#1}
   \newcommand{\latexonlysubsection}[1]{#1}
   \newcommand{\latexonlysubsubsection}[1]{#1}
\end{htmlonly}
%  Title Page.
%  ===========
\renewcommand{\thepage}{\roman{page}}
\begin{document}
\thispagestyle{empty}

%  Latex document header.
%  ======================
\begin{latexonly}
   {\small CCLRC / {\sc Rutherford Appleton Laboratory} \hfill {\bf \stardocname}}\\
   Particle Physics \& Astronomy Research Council\\
   Starlink Project\\
   \stardoccategory\ \stardocnumber
   \begin{flushright}
   \stardocauthors\\
   \stardocdate
   \end{flushright}
   \vspace{-4mm}
   \rule{\textwidth}{0.5mm}
   \vspace{3mm}
   \begin{center}
   {\large\bf \stardoctitle}
   \end{center}
\end{latexonly}

%  HTML documentation header.
%  ==========================
\begin{htmlonly}
   \xlabel{}
   \begin{rawhtml} <H1> \end{rawhtml}
      \stardoctitle
   \begin{rawhtml} </H1> \end{rawhtml}

% ? Add picture here if required.
% ? End of picture

   \begin{rawhtml} <P> <I> \end{rawhtml}
   \stardoccategory\ \stardocnumber \\
   \stardocauthors \\
   \stardocdate
   \begin{rawhtml} </I> </P> <H3> \end{rawhtml}
      \htmladdnormallink{CCLRC}{http://www.cclrc.ac.uk} /
      \htmladdnormallink{Rutherford Appleton Laboratory}
                        {http://www.cclrc.ac.uk/ral} \\
      \htmladdnormallink{Particle Physics \& Astronomy Research Council}
                        {http://www.pparc.ac.uk} \\
   \begin{rawhtml} </H3> <H2> \end{rawhtml}
      \htmladdnormallink{Starlink Project}{http://star-www.rl.ac.uk/}
   \begin{rawhtml} </H2> \end{rawhtml}
   \htmladdnormallink{\htmladdimg{source.gif} Retrieve hardcopy}
      {http://star-www.rl.ac.uk/cgi-bin/hcserver?\stardocsource}\\

%  HTML document table of contents. 
%  ================================
%  Add table of contents header and a navigation button to return to this 
%  point in the document (this should always go before the abstract \section). 
  \label{stardoccontents}
  \begin{rawhtml} 
    <HR>
    <H2>Contents</H2>
  \end{rawhtml}
  \newcommand{\latexonlytoc}[0]{}
  \htmladdtonavigation{\htmlref{\htmladdimg{contents_motif.gif}}
        {stardoccontents}}

% ? New section for abstract if used.
%  \section{\xlabel{abstract}Abstract}
% ? End of new section for abstract

\end{htmlonly}

% -----------------------------------------------------------------------------
% ? Document Abstract. (if used)
%  ==================
% ? End of document abstract
% -----------------------------------------------------------------------------
% ? Latex document Table of Contents (if used).
%  ===========================================
% \newpage
% ? End of Latex document table of contents
% -----------------------------------------------------------------------------
\renewcommand{\thepage}{\arabic{page}}
\setcounter{page}{1}

\html{\section{\xlabel{support}Support for astronomical computing in
the UK}}
\begin{latexonly}
\begin{center}
{\large\bf Support for astronomical computing in the UK}
\end{center}
\end{latexonly}

\vspace*{5mm}
% Added the following line to obtain normalsize unbolden text outside
% the lists in the html version.  Do know why it's needed.  --- MJC
\html{\normalsize\rm}
Starlink was set up in 1980 to help astronomers use computers to analyse their
observations.
It is:

\begin{itemize}
\item A network of {\bf computers} used by UK astronomers; 
\item A collection of {\bf software} to reduce and analyse astronomical data,
and work on some types of theory;
\item A team of {\bf people} giving hardware, software, and administrative
support. 
\end{itemize}

Starlink's main objectives are to:

\begin{itemize}
\item Provide and coordinate 
{\bf interactive data reduction and analysis facilities} for use as a 
research tool by UK astronomers.
\item Encourage {\bf software sharing and standardisation} to prevent
unnecessary duplication of effort.
\item Provide {\bf systems software support} for astronomers.
\end{itemize}

\newpage

\section*{Work supported}

The {\bf types of astronomy} supported by Starlink are:

\begin{itemize}
\item Normal data reduction and analysis.
\item Limited theoretical work.
\end{itemize}

Work on Starlink computers has the following {\bf priorities}:

\begin{enumerate}
\item Astronomical data reduction that can only be done interactively.
\item Other astronomical data reduction.
\item Other astronomical research work that can only be done interactively.
\item Other astronomical research work.
\item Other PPARC astronomy work.
\end{enumerate}

(In this list `astronomical data' includes data obtained from telescopes
and data obtained from theoretical calculations made on other computers.)

\newpage

The following types of astronomy are {\bf not supported} by Starlink:

\begin{itemize}
\item Large scale theoretical work.
\item Instrument research and construction.
\item Observations taken with telescopes and space missions that can only
 observe solar system objects.
\item Geophysics and solar-terrestrial physics.
\item Undergraduate and other purely educational work (which is normally
 supported by the DfE).
\end{itemize}

These types of work are normally supported by other sources of funding, such as
PPARC grants.


\newpage

\section*{Sites}

Starlink facilities are concentrated at a number of
\htmladdnormallink{sites}{http://star-www.rl.ac.uk/sites.html}
spread around the country.
\begin{latexonly}
Details of individual sites are given in Appendix B.
\end{latexonly}

Day-to-day management of a Starlink site is the responsibility of its
Site Manager, who may be helped at large sites by an Assistant.
The overall management of a site is monitored by a Site Chairman, accredited by
the Project.

How many sites are there? -- That depends on your viewpoint:
\begin{description}
\item[\mbox{}]\mbox{}
\begin{description}
\item [Geographical] --
 The obvious one.
 We call them {\bf Sites}.
 There are {\bf 26} of them.
 The Cambridge site is split between IoA/RGO and MRAO, which are a 10-minute
 walk apart but regarded as a single Site.
\item [Administrative] --
 The administrator's view.
 Its primary criterion is the allocation of long-term  on-site management.
 We call them {\bf Nodes}.
 There are {\bf 26} of them.
 But these are not quite the same as the 26 sites.
 The sites at Armagh and Belfast are regarded as the ``Northern Ireland" node,
 while the RAL site contains two nodes: the ``Project" node and the
 ``RAL -- Astrophysics node."
\item [Operational] --
 This is the way things look when you want to send an e-mail message or
 count user numbers.
 We call them {\bf User-Centres}.
 There are {\bf 28} of them.
 The Northern Ireland node has two user-centres (Armagh and Belfast), and
 the Cambridge node also has two user-centres (IoA/RGO and MRAO).
 (In practice, the word ``site" is often used to refer to a user-centre.)
\end{description}
\end{description}
The thing to remember is that Northern Ireland has one node and two sites,
RAL has one site and two nodes, and Cambridge has two user-centres.
All the other sites have one of everything.

\subsection*{Remote User Groups}

In addition to Starlink Nodes, there are Remote User Groups (RUGs).
A RUG does not have funding for long-term on-site management but receives,
or expects to receive, hardware and software support from Starlink.
It may also receive ``remote" site management effort from other sites.
There are 5 RUGs, at present, which have not been counted in the numbers
given above.

\subsection*{Central facilities}

A further Starlink facility is a central computer system at RAL which stores
astronomical catalogues and proprietary software which is licensed only for
this one machine.
Any Starlink user may use it from any site, via a network connection.

\newpage

\section*{Users}

In May 1996 there were 1948 Starlink 
\htmladdnormallink{Users}{http://star-www.rl.ac.uk/people.html}.
\begin{latexonly}
Details may be found in Appendix B, which also includes a chart showing the
history of user numbers since the start of the Project.
\end{latexonly}

Users are classified in several ways.
The most commonly used are:

\begin{description}
\item[\mbox{}]\mbox{}
\begin{description}
\item [Uniqueness] --
 This is used to avoid double counting.
 The problem is that some users are registered at more than one site.
 This means that if you add up the number of users registered at each site,
 the grand total will be higher than the actual number of users.
 To solve this problem, each site classifies its users in two ways:
 \begin{itemize}
 \item Primary
 \item Secondary
 \end{itemize}
\item [Type] --
 This is used to help us estimate the number of users who do different types
 of work.
 It helps us allocate resources rationally.
 The Types are:
 \begin{description}
\item[\mbox{}]\mbox{}
 \begin{description}
 \item [r] -- Research astronomer actively processing data.
 \item [t] -- Scientific \& technical support staff.
 \item [o] -- Others, like secretaries.
 \item [u] -- Undergraduate engaged in an astronomical research programme.
 \item [a] -- Associate user (does not have an official position in a
              university or similar institution).
 \item [f] -- People who are usually not resident in the UK, such as visitors.
 \end{description}
 \end{description}
\item [Community] --
 Users are allocated a:
 \begin{itemize}
 \item Location code
 \end{itemize}
 Not all users work at the same place or in the same department as that in
 which a Starlink site is located.
 This code helps us get a better idea of the size and location of the different
 communities of workers we are supporting.
\end{description}
\end{description}

Application forms for prospective Starlink users are available from the Site
Manager of the site they wish use.

Postgraduate students will need it countersigned by their supervisor.
Undergraduates, unfortunately, cannot be accepted, except for collaboration in
research aimed at publication, as Starlink is not funded for educational work.
Applications from people not associated with an HEI, for example a serious
amateur astronomer doing research, are considered on a case-by-case basis.
Such a person should normally be working in collaboration with astronomers at
an HEI.

All applications are formally authorised by the Starlink Project Scientist.

\newpage

\section*{User support}

Users can get help from many places:

\begin{itemize}
\item \htmladdnormallink{Site Manager}
 {http://star-www.rl.ac.uk/managers.html}.
\item \htmladdnormallink{Site Chairman}
 {http://star-www.rl.ac.uk/whoswho}.
\item Starlink Local User Group (SLUG).
\item \htmladdnormallink{Application Programmers}
 {http://star-www.rl.ac.uk/programmers.html}.
\item \htmladdnormallink{Starlink Project Group}
 {http://star-www.rl.ac.uk/pro.html} at RAL.
\item \htmladdnormallink{Project Scientist}
 {http://star-www.rl.ac.uk/pro.html}.
\end{itemize}

The primary focus of user support is the Site Manager.
Informal contacts with colleagues can also resolve many problems.

Overall policy on how a site is run is overseen by the Site Chairman, in 
consultation with its users, with the Site Manager, and with Starlink
management.
The aim is to operate each site in accordance with the needs and wishes of its
users.
The exact nature of local support organisations can vary, but they usually
involve a Starlink Local User Group (SLUG) which all users may attend and
participate in.
The Site Chairman has a direct input to Starlink to report difficulties and
to recommend policy changes.

Starlink welcomes input from users.
They can contact the sources of support mentioned above.
Other communication routes are Software Strategy Groups, Software
Questionnaires, and the Starlink meeting at the annual National Astronomy
Meetings.
There is also an established complaints procedure.

\newpage

\subsection*{Information sources}

Extensive 
\htmladdnormallink{documentation}{http://star-www.rl.ac.uk/documentation.html}
is available on the Project itself, and on individual software items.
Many of the larger packages have on-line help systems.
The information sources are described in detail in the {\em Starlink User's
Guide} (\xref{SUG}{sug}{}), but the main source is the set of
{\em Starlink User Notes}\/ (\htmladdnormallink{SUN}{http://star-www.rl.ac.uk/sun.html}),
{\em Starlink Guides}\/ (\htmladdnormallink{SG}{http://star-www.rl.ac.uk/sg.html}),
and {\em Starlink Cookbooks}\/ (\htmladdnormallink{SC}{http://star-www.rl.ac.uk/sc.html}).
There are over 180 of these Notes and Guides (about 6000 pages), and they
are extensively indexed.
Also, a newsletter called the 
{\em \htmladdnormallink{Starlink Bulletin}
{http://star-www.rl.ac.uk/bulletin.html}} 
is published twice a year and circulated to all users to keep them in touch
with what is happening in the Project.

A large amount of information about Starlink is also provided on the World
Wide Web, including most of the Notes and Guides.
Most sites have their own home pages, but the central one is maintained
by the 
\htmladdnormallink{Project node}{http://star-www.rl.ac.uk/}
 and its URL is:
\begin{quote}
{\tt http://star-www.rl.ac.uk/}
\end{quote}
From here you can branch to all the other sites which provide web pages.

\newpage

\section*{Software}

The major software product provided by Starlink for its users is the
{\em Starlink Software Collection}\/ 
(\htmladdnormallink{SSC}{http://star-www.rl.ac.uk/software.html}).
It is managed and distributed by the Starlink Software Librarian at RAL.
It is installed at every Starlink site, and has also been distributed to
many other sites all over the world.

The SSC comprises about 120 items (40 packages, 40 utilities, 40 subroutine
libraries and infrastructure).
It is being actively developed and there are about 40 software releases per
year.
This means that published information like this gets out-of-date rapidly,
so it is worth checking with the Project on the latest information --- the
World Wide Web and the Starlink Bulletin are good sources of current
information.
\begin{latexonly}
A summary of the current software available is given in Appendix B.
\end{latexonly}
The best survey of the SSC as a whole is \xref{SUN/1}{sun1}{}.

Starlink software originates from many different sources and has different
levels of support.
Some is a legacy from Starlink's past.
Some is being actively developed and supported by Starlink's own programmers.
Some (like AIPS, IDL, IRAF) has been obtained from outside the Project.
Starlink software development is controlled by the Starlink Panel,
Software Strategy Groups, and the Project Group at RAL.

\newpage

\subsection*{Software Environments}

An important software issue is the {\em software environment}\/.
This term covers:
programming languages;
job control;
command languages;
data systems;
graphics;
documentation aids;
utilities;
error handling.
It is important both for reasons of software development efficiency, and
because of its central role in the organisation of Starlink software.

Environment designs lie on a continuum between two extreme positions.

At one extreme, the facilities provided by the computer manufacturer are used,
together with a collection of {\em ad hoc}\/ routines.
The dangers of this are:
huge monolithic programs offering facilities of limited flexibility;
a multiplicity of systems which are idiosyncratic to user and programmer;
fixed, inflexible data formats;
incompatibility between different systems;
duplication.

At the other extreme, an ideal system is created which is machine, operating
system, and programming language independent.
The dangers of this are:
might not be what the users want or need;
takes a long time to develop;
takes application programming effort;
inefficient and slow.

The first approach tends to be favoured by users innocent of the real cost
of software, who just want to get on with the job of analysing data as
quickly as possible.
The second approach tends to be favoured by computer scientists, who are more
interested in a clever system than one that is useful for astronomers.

Starlink, with the approval of the relevant UK astronomical committees,
adopted a middle course based on the ADAM environment.
This provides application programmers with a uniform and powerful system of
data storage, user interface, programming tools, job control, command language,
and graphics.
It is described in \xref{SG/4}{sg4}{}.

\newpage

\section*{Hardware}

Each Starlink site has a computer system based on Unix workstations and servers.
At present these are based on DEC Alpha/Digital Unix and Sun SPARC/Solaris
architectures,
although other systems may be adopted in the future (for example, Linux on PCs).
A residual VAX/VMS service is maintained at the Project Node to run ``legacy"
software.

In addition to Unix CPUs, the Project also supplies X-terminals, SCSI disks,
Exabyte and DAT tape drives (for data exchange, backups, etc.), printers
(mostly PostScript, and including some colour devices), networking equipment
and other more specialized pieces of hardware.

A central computing facility is also provide at the Project Node to 
give all Starlink users access to some expensive commercial software and large
databases.

Hardware enhancements are decided after annual bids to the Starlink Panel.
Hardware funded from other sources, e.g.\ grants, may be brought under the
Starlink umbrella.
If this is agreed, Starlink takes over the running costs and management.

\newpage

\section*{Organisation}

Starlink is run on behalf of the
{\em Particle Physics and Astronomy Research Council}\/ 
(\htmladdnormallink{PPARC}{http://www.pparc.ac.uk/}) 
by the
\htmladdnormallink{{\em Starlink Project Group}}
{http://star-www.rl.ac.uk/pro.html}\/ 
within the
{\em Space Science Department}\/
(\htmladdnormallink{SSD}
{http://star-www.rl.ac.uk/ssd.html})
of 
{\em Rutherford Appleton Laboratory}\/
(\htmladdnormallink{RAL}
{http://www.cclrc.ac.uk/ral/index.html}).
It has an annual budget of about 2 million pounds.

RAL is part of the
{\em Council for the Central Laboratory of the Research Councils}\/ 
(\htmladdnormallink{CCLRC}{http://www.cclrc.ac.uk/}).

PPARC is one of the
{\em \htmladdnormallink{Research Councils}
{http://www.nerc.ac.uk/joint_res_councils.html}}
funded by the
{\em Office of Science and Technology}\/ 
(\htmladdnormallink{OST}
{http://www.open.gov.uk/ost/osthome.htm}) 
within the
{\em Department of Trade and Industry}\/ 
(\htmladdnormallink{DTI}{http://www.dti.gov.uk/}).

\subsection*{Starlink Panel}

This is a PPARC committee which:

\begin{itemize}
\item Advises PPARC on the operation and development of Starlink and its
financial requirements.
\item Establishes scientific priorities for Starlink within PPARC policy and its
budget.
\item Approves Starlink expenditure.
\end{itemize}
It consists of a Chairman and seven others.
They are university staff with interests covering a wide range of computing
and astronomy.
It normally meets three times a year, in March, July, and November.

\newpage

\subsection*{Starlink Project Group}

This is located at RAL and is the focus of day-to-day Project management.
It:
\begin{itemize}
\item Manages and coordinates the software collection (SSC).
\item Manages and coordinates the software and Project documentation.
\item Manages the purchase of computers and their operating systems and
arranges maintenance.
\item Manages the contracts for its site managers and programmers.
\item Writes infrastructure, systems, network, and core applications software.
\item Writes general Project documentation.
\item Maintains an information database for the Project on software,
 documentation, and users.
\item Supports the Starlink Panel, advises PPARC on computing matters, and
advises sites on the purchase and operation of Starlink compatible equipment.
\end{itemize}

\newpage

\subsection*{Software Strategy Groups}

These groups are one of the main sources of advice for Starlink on the
development of its software.
They cover the following areas:
\begin{itemize}
\item Spectroscopy.
\item Image processing.
\item Theory \& statistical analysis.
\item Information services \& databases.
\item Graphics \& infrastructure.
\item Radio, mm, \& sub-mm astronomy.
\item X-ray astronomy.
\end{itemize}

\newpage

\section*{History}

In the 1970's it became clear that the data processing facilities available at
that time to UK astronomers were inadequate to deal with the anticipated
flood of data in digital form which would be generated by new data
acquisition techniques.

In April 1978, the Science Research Council (which became the SERC, and later
PPARC) set
up a {\em Panel on Astronomical Image and Data Processing}\/ under the
chairmanship of Professor M J Disney to ascertain the computing needs of
UK astronomers for the next 5 to 7 years.
This Panel reported in April 1979 and recommended the installation of 6
super-minicomputers connected together in a star network by leased lines; 
hence the name {\em Starlink}.

The computer chosen was the DEC VAX 11/780.
Between December 1979 and July 1980, these were installed at Cambridge,
Manchester, RAL, RGO, ROE, and UCL.
Astronomers at other sites used Starlink facilities via network links.
The initial Project staff were appointed by mid 1980 with the administrative
centre located at RAL.
Starlink was inaugurated on 24th October 1980 by Mr D N MacFarlane,
Parliamentary Under Secretary at the Department of Education and Science.

Starlink was the first astronomical data processing system to use networking
extensively.
The early links ran at 4800 baud and used DEC protocols exclusively.

A Scientific Advisory Group (SAG) and later a Starlink Users' Committee (SUC)
advised RAL management on the running of the Project.
There were also Area Management Committees (AMCs) for groups of sites
geographically close to each other.
Special Interest Groups (SIGs) considered software development in specific
application areas.
A Hardware Advisory Group (HAG) advised the Project on hardware purchase.

\subsection*{Sites and Users}

Since Starlink began in 1980 the number of Starlink sites has grown from 6
to 28 and the number of users has grown from about 200 to about 2000.
\begin{latexonly}
Details are given in Appendix B.
\end{latexonly}
This reflects an increase in the use of computers by astronomers, and a
move by university groups to transfer from non-Starlink to Starlink systems.
In 1990 the RGO node merged with the Cambridge node when RGO moved from
Sussex to Cambridge, and the RAL node split into the Project node and the
Astrophysics node.

\subsection*{Hardware}

Right from the outset, Starlink has used equipment from a variety of suppliers
to maximize cost-effectiveness.
For example, the original six VAX 780s were equipped with disks from System
Industries and printers from Printronix and Versatec.
 
As new Starlink sites were established, they were set up with VAX 750s and,
subsequently, MicroVAXes.
From 1988 onwards, MicroVAXes were first added to VAX 780 and 750 sites to
create VAX clusters, and then the 780s and 750s themselves were replaced by
MicroVAXes.
The last Starlink 780 was removed from Herstmonceux in early 1990, when RGO
moved to Cambridge.
 
By 1990, Starlink's hardware comprised clusters of MicroVAXes and VAXstations,
and workstation screens had replaced the earlier specialized image display
devices, such as the Sigma ARGS.
By basing our software on VMS alone since Starlink was established we had
economised on software costs, but Unix offered two major benefits:
\begin{itemize}
\item Compatibility with overseas astronomers (Unix was becoming a
{\em de facto} standard for astronomical computing).
\item A choice of suppliers offering cost-effective hardware.
\end{itemize}
It was decided, therefore, to move from VMS to Unix, and in 1992 the Willmore
Review of Starlink recommended that the move should be completed by April 1995.
Starlink's first Unix hardware was a DECstation 2100 running Ultrix, purchased
by the Project in late 1989 to investigate software porting questions.

The move to Unix took several years, but was completed by the April 1995
target.
During the move, Starlink supported up to five platforms in parallel 
-- VAX/VMS, SPARC/SunOS, SPARC/Solaris, DECstation/Ultrix, Alpha/OSF 
(Digital Unix).
Since completing the move we have rationalized support, concentrating
on our two main platforms: SPARC/Solaris and Alpha/Digital Unix.
Nothing stands still, however, and Starlink has recently started to offer
support for a third platform: PCs running Linux.

\subsection*{Software}

Software development plans were first discussed at a workshop held at Appleton
Laboratory in November 1979 and the following recommendations were made:

\begin{itemize}
\item An overall supervisory system should be written.
\item There should be a hierarchical data system.
\item There should be a command system with facilities for parameter handling,
defaulting mechanisms, symbols, help, prompting, multi-stream, and batch.
\item There should be a standard graphics system.
\item Data interchange on tape should be in FITS format.
\end{itemize}

The first software release occurred in March 1980.

The first Starlink software environment (INTERIM), first released in
September 1980, was a data and parameter system towards the pragmatic end of
the spectrum.
It also had a fairly primitive command language (DSCL).
It was tolerably efficient, very easy to use, and was available within 9
months of the start of the Project.
It served as the basis for a lot of application software which was widely
used.

The second Starlink software environment (SSE) was an ambitious
concept residing near the utopian end of the spectrum, and very similar to the
successful IRAF system which appeared some years later.
It was developed for 5 years but was never considered satisfactory and was
inadequately documented.
It was 44 times bigger than INTERIM (even though incomplete) and command
processing was unacceptably slow on the equipment of the time.
It contained some powerful and promising packages, in particular a good
graphics system based on GKS and a Hierarchical Data System (HDS) of
considerable elegance, power, and efficiency.
Its failure was due to its development being badly affected by catastrophic
losses of key staff and, more generally, by lack of central programming
effort in a Project whose customer base was expanding rapidly.
This serious situation was considered at two meetings at RAL in November 1985
and February 1986.

The result was that a new software environment (ADAM) was adopted for
development.
This did not represent a fresh start in the way that SSE was a
radically different design from INTERIM.
In fact it derived from many sources taken from many different places.
It owed a lot to SSE, and in some respects was a re-implementation
of it with a greater emphasis on efficiency.
The initial focus for this work was the production of a real-time telescope
and instrument control system at ROE and RGO, but this was now broadened to
encompass Starlink's data-analysis requirements.
The initial release was in September 1986, and it has since been developed
extensively.

All Starlink software development in the eighties was based on VMS.
A major challenge and opportunity in the nineties was presented by the arrival
of powerful Unix-based workstations.
It was necessary to port selected Starlink software to these machines; in
particular, a big effort was needed to port the ADAM environment.
In the event, this was achieved comfortably before the April 1995 deadline for
the move to Unix.
Software development is now concentrated on Unix systems.

\subsection*{Questionnaires}

In the spring of 1986, a major exercise was carried out to determine what the
users thought should be the future direction of the Starlink Project.
A 14-page questionnaire was distributed to 669 users and potential users and
394 were returned, a very high proportion.
The returns were analysed and the results influenced the direction of the
Project.

Another extensive questionnaire, concerned specifically with Starlink software,
was distributed to over 1600 users in January 1994.
Over 500 were returned and the results heavily influenced Starlink's software
development plans.

A further software questionnaire was distributed to nearly 2000 users in March
1996.
Once again, over 500 were returned and the results of their analysis will have
a similar influence on the Project.

\newpage

\section*{What have we learnt?}

The Starlink Project has been going for sixteen years.
What have we learnt during this time?

The VAX computer was an excellent initial choice.
It provided a powerful, well documented, user-friendly operating system.
For some time it was the standard hardware for mainstream, world-wide
astronomical computing.
Developments, such as microVAXes and VAX clusters, provided a natural growth
path for the Project.
The choice of VAX/VMS and UK-wide networking made Starlink a success, in spite
of many problems.
The arrival of powerful Unix-based scientific workstations changed the
situation, and we completed a move from VMS to Unix by early 1995.

Compatible hardware at each site allows software to be distributed easily
and avoids wasteful development of different versions for different
hardware.
Central purchase and maintenance of this hardware has allowed Starlink to
negotiate extra discounts from suppliers and so reduce costs.

The network has been vital.
It enables a basic set of software to be controlled centrally and distributed
rapidly, so that a common user environment exists at every site.
The electronic mail facilities are heavily used and extremely valuable.
They bind the astronomical community together.

Our first software environment (INTERIM) was modest but successful.
The second (SSE) proved to be over-ambitious in relation to our
limited resources, and on the computers of the time was just too slow for users
to tolerate.
The third (ADAM) seems to have been a good move forward.
Perhaps the biggest lesson is the importance of having a strong, central
programming team under firm control when embarking on this kind of project.
After prolonged efforts by Starlink management, this situation was
achieved by the creation at RAL of an Infrastructure support group.

Once a large collection of software has been distributed, the maintenance
and support requirements remain considerable.
This is not understood by some astronomers, and is one of Starlink's most
intractable problems.

Central management of the preparation and distribution of software releases
by a Software Librarian is vital to the cohesiveness of the Project.
A properly integrated set of software, documentation and administrative
information installed at multiple sites is difficult to achieve and has to
be well supported.
A related problem is the dispersion of different versions of specific software
items within different packages.
This leads to waste of storage space and difficulties with updates.
The solution lies in central co-ordination of software development and the
use of programming support environments.

\section*{Acknowledgements}

I would like to thank the following people for their significant contributions
to this document:
Chris Clayton,
Alan Penny,
Andrea Roberts,
John Sherman,
Dave Terrett,
Patrick Wallace.

\newpage

\appendix

\section{Glossary}

\vspace{-6mm}

{\small
\begin{tabbing}
XX\=ABCDEFGHI\=XX\kill
\>  \>  \\
\>AMC   \>Area Management Committee  \\
\>CCLRC \>Council for the Central Laboratory of the\\
\>      \>Research Councils\\
\>CPU   \>Central Processor Unit \\
\>DEC   \>Digital Equipment Corporation\\
\>DfE   \>Department for Education \\
\>DTI   \>Department of Trade and Industry \\
\>HAG   \>Hardware Advisory Group\\
\>HEI   \>Higher Educational Institute  \\
\>HST   \>Hubble Space Telescope\\
\>IoA   \>Institute of Astronomy, Cambridge\\
\>MRAO  \>Mullard Radio Astronomy Observatory  \\
\>OST   \>Office of Science and Technology \\
\>PC    \>Personal Computer (IBM compatible)\\
\>PPARC \>Particle Physics \& Astronomy Research Council\\
\>RAL   \>Rutherford Appleton Laboratory\\
\>RGO   \>Royal Greenwich Observatory\\
\>ROE   \>Royal Observatory Edinburgh\\
\>RUG   \> Remote User Group \\
\>SAG   \>Scientific Advisory Group\\
\>SC    \>Starlink Cookbook\\
\>SCSI  \>Simple Computer System Interface\\
\>SERC  \>Science and Engineering Research Council\\
\>SG    \>Starlink Guide\\
\>SIG   \>Special Interest Group\\
\>SLUG  \>Starlink Local User Group\\
\>SSC   \>Starlink Software Collection\\
\>SSD   \>Space Science Department (RAL)\\
\>SSE   \>Starlink Software Environment \\
\>SUC   \>Starlink User' Committee \\
\>SUG   \>Starlink User's Guide\\
\>SUN   \>Starlink User Note\\
\>UCL   \>University College London\\
\>UK    \>United Kingdom \\
\>URL   \>Universal Resource Locator\\
\>VAX   \>Virtual Address Extension (DEC)\\
\>VMS   \>Virtual Memory System (DEC)\\
\end{tabbing}
}

\begin{latexonly}
\newpage

\latexonlysection{Detailed information}

The pages that follow contain charts and tables that give detailed information
about Starlink.
These are described below in order of appearance.
The information was the latest available at the time of publication.

\begin{enumerate}

\item Shows the number of users at each User-Centre.
The {\em Prim}\/ block shows the number of Primary users at a site.
The {\em Sec}\/ block shows the number of Secondary users at a site.

\item Shows the proportions of different types of user (specified on page 6).

\item Shows the growth in the total number of Starlink users since the Project
began in 1980.

\item Shows the growth in the number of User-Centres since the Project began.
See page 4 for a definition of User-Centre.

\item Lists the names of the items in the Starlink Software Collection
(SSC), classified into functional area.

\item Lists the postal address of every Starlink User-Centre.
The RAL/Astrophysics User-Centre and the Project User-Centre are both located
at RAL and have the same postal address.
Thus, only 27 addresses are given in the list, although there are 28 Starlink
User-Centres

\end{enumerate}
\end{latexonly}

\end{document}
