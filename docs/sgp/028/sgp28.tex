\documentclass[twoside,11pt]{article}
\pagestyle{myheadings}

% -----------------------------------------------------------------------------
% ? Document identification
\newcommand{\stardoccategory}  {Starlink General Paper}
\newcommand{\stardocinitials}  {SGP}
\newcommand{\stardocsource}    {sgp\stardocnumber}
\newcommand{\stardocnumber}    {28.8}
\newcommand{\stardocauthors}   {M D Lawden\\
                                A C Charles}
\newcommand{\stardocdate}      {25 September 1997}
\newcommand{\stardoctitle}     {How to write good documents for Starlink}
\newcommand{\stardocabstract}  {
  This document gives advice on how to write good Starlink documents.
  It lists Starlink documents which users said they liked in the 1994
  Starlink Software Survey, and identifies common characteristics of style
  and organisation which can be recommended to authors.

  Various features of documentation which the authors believe to be harmful
  are specified, including common problems with pages produced for
  the World Wide Web.
  Many of these will inevitably reflect the taste of the authors, and these
  may not correspond to your own.
  However, they derive from our belief that it is right for an author
  to spend time and effort on a document in order to save the reader time
  and effort in reading it.
}
% ? End of document identification

% -----------------------------------------------------------------------------

\newcommand{\stardocname}{\stardocinitials /\stardocnumber}
\markright{\stardocname}
\setlength{\textwidth}{160mm}
\setlength{\textheight}{230mm}
\setlength{\topmargin}{-2mm}
\setlength{\oddsidemargin}{0mm}
\setlength{\evensidemargin}{0mm}
\setlength{\parindent}{0mm}
\setlength{\parskip}{\medskipamount}
\setlength{\unitlength}{1mm}

% -----------------------------------------------------------------------------
%  Hypertext definitions.
%  ======================
%  These are used by the LaTeX2HTML translator in conjunction with star2html.

%  Comment.sty: version 2.0, 19 June 1992
%  Selectively in/exclude pieces of text.
%
%  Author
%    Victor Eijkhout                                      <eijkhout@cs.utk.edu>
%    Department of Computer Science
%    University Tennessee at Knoxville
%    104 Ayres Hall
%    Knoxville, TN 37996
%    USA

%  Do not remove the %begin{latexonly} and %end{latexonly} lines (used by
%  star2html to signify raw TeX that latex2html cannot process).
%begin{latexonly}
\makeatletter
\def\makeinnocent#1{\catcode`#1=12 }
\def\csarg#1#2{\expandafter#1\csname#2\endcsname}

\def\ThrowAwayComment#1{\begingroup
    \def\CurrentComment{#1}%
    \let\do\makeinnocent \dospecials
    \makeinnocent\^^L% and whatever other special cases
    \endlinechar`\^^M \catcode`\^^M=12 \xComment}
{\catcode`\^^M=12 \endlinechar=-1 %
 \gdef\xComment#1^^M{\def\test{#1}
      \csarg\ifx{PlainEnd\CurrentComment Test}\test
          \let\html@next\endgroup
      \else \csarg\ifx{LaLaEnd\CurrentComment Test}\test
            \edef\html@next{\endgroup\noexpand\end{\CurrentComment}}
      \else \let\html@next\xComment
      \fi \fi \html@next}
}
\makeatother

\def\includecomment
 #1{\expandafter\def\csname#1\endcsname{}%
    \expandafter\def\csname end#1\endcsname{}}
\def\excludecomment
 #1{\expandafter\def\csname#1\endcsname{\ThrowAwayComment{#1}}%
    {\escapechar=-1\relax
     \csarg\xdef{PlainEnd#1Test}{\string\\end#1}%
     \csarg\xdef{LaLaEnd#1Test}{\string\\end\string\{#1\string\}}%
    }}

%  Define environments that ignore their contents.
\excludecomment{comment}
\excludecomment{rawhtml}
\excludecomment{htmlonly}

%  Hypertext commands etc. This is a condensed version of the html.sty
%  file supplied with LaTeX2HTML by: Nikos Drakos <nikos@cbl.leeds.ac.uk> &
%  Jelle van Zeijl <jvzeijl@isou17.estec.esa.nl>. The LaTeX2HTML documentation
%  should be consulted about all commands (and the environments defined above)
%  except \xref and \xlabel which are Starlink specific.

\newcommand{\htmladdnormallinkfoot}[2]{#1\footnote{#2}}
\newcommand{\htmladdnormallink}[2]{#1}
\newcommand{\htmladdimg}[1]{}
\newenvironment{latexonly}{}{}
\newcommand{\hyperref}[4]{#2\ref{#4}#3}
\newcommand{\htmlref}[2]{#1}
\newcommand{\htmlimage}[1]{}
\newcommand{\htmladdtonavigation}[1]{}

% Define commands for HTML-only or LaTeX-only text.
\newcommand{\html}[1]{}
\newcommand{\latex}[1]{#1}

% Use latex2html 98.2.
\newcommand{\latexhtml}[2]{#1}

%  Starlink cross-references and labels.
\newcommand{\xref}[3]{#1}
\newcommand{\xlabel}[1]{}

%  LaTeX2HTML symbol.
\newcommand{\latextohtml}{{\bf LaTeX}{2}{\tt{HTML}}}

%  Define command to re-centre underscore for Latex and leave as normal
%  for HTML (severe problems with \_ in tabbing environments and \_\_
%  generally otherwise).
\newcommand{\setunderscore}{\renewcommand{\_}{{\tt\symbol{95}}}}
\latex{\setunderscore}

% -----------------------------------------------------------------------------
%  Debugging.
%  =========
%  Remove % on the following to debug links in the HTML version using Latex.

% \newcommand{\hotlink}[2]{\fbox{\begin{tabular}[t]{@{}c@{}}#1\\\hline{\footnotesize #2}\end{tabular}}}
% \renewcommand{\htmladdnormallinkfoot}[2]{\hotlink{#1}{#2}}
% \renewcommand{\htmladdnormallink}[2]{\hotlink{#1}{#2}}
% \renewcommand{\hyperref}[4]{\hotlink{#1}{\S\ref{#4}}}
% \renewcommand{\htmlref}[2]{\hotlink{#1}{\S\ref{#2}}}
% \renewcommand{\xref}[3]{\hotlink{#1}{#2 -- #3}}
%end{latexonly}
% -----------------------------------------------------------------------------
% ? Document specific \newcommand or \newenvironment commands.
% ? End of document specific commands
% -----------------------------------------------------------------------------
%  Title Page.
%  ===========
\renewcommand{\thepage}{\roman{page}}
\begin{document}
\thispagestyle{empty}

%  Latex document header.
%  ======================
\begin{latexonly}
   CCLRC / {\sc Rutherford Appleton Laboratory} \hfill {\bf \stardocname}\\
   {\large Particle Physics \& Astronomy Research Council}\\
   {\large Starlink Project\\}
   {\large \stardoccategory\ \stardocnumber}
   \begin{flushright}
   \stardocauthors\\
   \stardocdate
   \end{flushright}
   \vspace{-4mm}
   \rule{\textwidth}{0.5mm}
   \vspace{5mm}
   \begin{center}
   {\Huge\bf  \stardoctitle \\ [2.5ex]}
   \end{center}
   \vspace{5mm}

% ? Heading for abstract if used.
   \vspace{10mm}
   \begin{center}
      {\Large\bf Abstract}
   \end{center}
% ? End of heading for abstract.
\end{latexonly}

%  HTML documentation header.
%  ==========================
\begin{htmlonly}
   \xlabel{}
   \begin{rawhtml} <H1> \end{rawhtml}
      \stardoctitle
   \begin{rawhtml} </H1> \end{rawhtml}

% ? Add picture here if required.
% ? End of picture

   \begin{rawhtml} <P> <I> \end{rawhtml}
   \stardoccategory\ \stardocnumber \\
   \stardocauthors \\
   \stardocdate
   \begin{rawhtml} </I> </P> <H3> \end{rawhtml}
      \htmladdnormallink{CCLRC}{http://www.cclrc.ac.uk} /
      \htmladdnormallink{Rutherford Appleton Laboratory}
                        {http://www.cclrc.ac.uk/ral} \\
      \htmladdnormallink{Particle Physics \& Astronomy Research Council}
                        {http://www.pparc.ac.uk} \\
   \begin{rawhtml} </H3> <H2> \end{rawhtml}
      \htmladdnormallink{Starlink Project}{http://www.starlink.ac.uk/}
   \begin{rawhtml} </H2> \end{rawhtml}
   \htmladdnormallink{\htmladdimg{source.gif} Retrieve hardcopy}
      {http://www.starlink.ac.uk/cgi-bin/hcserver?\stardocsource}\\

%  HTML document table of contents.
%  ================================
%  Add table of contents header and a navigation button to return to this
%  point in the document (this should always go before the abstract \section).
  \label{stardoccontents}
  \begin{rawhtml}
    <HR>
    <H2>Contents</H2>
  \end{rawhtml}
  \htmladdtonavigation{\htmlref{\htmladdimg{contents_motif.gif}}
        {stardoccontents}}

% ? New section for abstract if used.
  \section{\xlabel{abstract}Abstract}
% ? End of new section for abstract
\end{htmlonly}

% -----------------------------------------------------------------------------
% ? Document Abstract. (if used)
%  ==================
\stardocabstract
% ? End of document abstract
% -----------------------------------------------------------------------------
% ? Latex document Table of Contents (if used).
%  ===========================================
\newpage
\begin{latexonly}
   \setlength{\parskip}{0mm}
   \tableofcontents
   \setlength{\parskip}{\medskipamount}
   \markright{\stardocname}
\end{latexonly}
% ? End of Latex document table of contents
% -----------------------------------------------------------------------------
\newpage
\renewcommand{\thepage}{\arabic{page}}
\setcounter{page}{1}

\section{Introduction}

This note gives advice on how to prepare good documents for Starlink.
It applies particularly to Starlink User Notes (SUN), Starlink Cookbooks (SC),
and Starlink Guides (SG), as these are Starlink's main ways of telling its users
how to use its software.
Some extra advice is given for World Wide Web pages.

Any judgement about what constitues a good document is bound to be partly a
question of taste, and our taste may differ from yours.
Naturally, this document reflects our taste.
However, a lot of our advice is based on our analysis of the common
characteristics of documents which have been judged good by Starlink users.

We believe the following general principles apply to good documents:

\begin{itemize}
\item Basic literacy, including a respect for conventional spelling,
punctuation, and grammar, is a fundamental requirement.
\item The layout and organisation of a document should help the reader to
understand its structure and find information.
\item The text of a document should be physically easy to read.
\item A user document should have more readers than writers.
It is, therefore, cost-effective for a writer to spend his own time in order
to save his readers' time.
It is wrong for a writer to excuse a shoddy or difficult-to-understand document
by claiming that it saved him time doing it that way.
\item Avoid irritating your readers.
(Not an easy task, as no doubt this document shows).
\end{itemize}

The Starlink Document Librarian (Mike Lawden) is responsible for ensuring that
Starlink documents reach an acceptable standard.
If Starlink published poor quality documents it could suffer political damage
which, in turn, could damage our service to users.
Because of this, the Librarian has the power to refuse to accept poor
documents for distribution within Starlink because of the bad impression they
would give of the Project.
Alternatively, he may modify them in order to make them acceptable.

Three recent developments have influenced our views on what constitutes a good
Starlink document:

\begin{quote}
\begin{description}

\item [Starlink Software Survey, 1994] -- this provided valuable feedback
from users about what sort of documents they wanted, and what they thought
of current documents. (see SGP/43)

\item [Starlink Software Strategy] -- this is a response by Starlink to the
wishes expressed by its users in the Starlink Software Survey.
Among other things, it contains plans for improving Starlink documents.
(see \xref{SGP/42}{sgp42}{})

\item [World Wide Web] -- this is an important technique for publishing
information which makes it easier and faster to use, update, and distribute.

\end{description}
\end{quote}

The impact of these developments on Starlink's document production is discussed
in the next three sections.
This is followed by a section containing advice on how to achieve good style and
organisation in your documents, and a section describing practices which
we think are bad and should be avoided.
Various technical matters concerning Starlink documents and how to produce
them are contained in two appendices.

\section{Starlink Software Survey}

A demand for better documentation was a marked feature of the Starlink
Software Survey of 1994.
This is shown by the answers given to questions 3 and 4.

\begin{quote}
Question 3: {\em How could Starlink software be improved?}
\end{quote}

Out of 18 possible answers, the 2nd and 3rd most popular (totalling 17\%
of the vote) were:
\begin{itemize}
\item Better on-line documentation.
\item Better printed documentation.
\end{itemize}

\begin{quote}
Question 4: {\em How could Starlink Documentation be improved?}
\end{quote}

Out of 23 possible answers, the top four (getting 45\% of the vote) were:
\begin{itemize}
\item Cookbooks for specific types of work.
\item A more tutorial style.
\item More practical examples.
\item Better examples for programmers.
\end{itemize}

The message is loud and clear: users want more help with how to {\em use}
the facilities provided by a piece of software, rather than just plain
descriptions of the facilities provided.

\subsection{\label{BestDocs}Best documents}

The survey also asked users to mark the quality of the documents provided with
each software item they used.
The ones which got the highest average marks (starting with the best) were:
\begin{itemize}
\item \xref{SUN/40}{sun40}{}: {\em CHR -- Character handling routines.}
\item \xref{SUN/39}{sun39}{}: {\em PRIMDAT -- Processing of primitive numerical data.}
\item SUN/100: {\em TPOINT -- Telescope pointing analysis system.}
\item \xref{SUN/33}{sun33}{}: {\em NDF -- Accessing extensible n-D data.}
\item SUN/105: {\em STARLSE -- Starlink extensions to language sensitive editor.}
\item SUN/110: {\em SST -- A simple software tools package.}
\item \xref{SUN/31}{sun31}{}: {\em REF -- Handling references to HDS objects.}
\item \xref{SUN/87}{sun87}{}: {\em JPL -- Solar system ephemeris.}
\item SUN/90: {\em SNX -- Starlink extensions to NCAR graphics utilities.}
\end{itemize}
These are the documents the users said they liked best, so take a look at them
and consider modelling your documents on their organisation and style.
In \htmlref{section 5}{GoodStyle} we list what we think are common
characteristics of these documents.
These characteristics can be used as guides for your own writing.

\section{Planned improvements to Starlink documents}

Major improvements to Starlink's documents are underway
(see \xref{SGP/42}{sgp42}{}).
Two key ones are:

\begin{itemize}
\item Cookbooks.
\item Hypertext.
\end{itemize}

\subsection{Cookbooks}

These provide recipes for common types of data reduction.
Users will, therefore, not need to understand a package in depth before being
able to do useful work with it.
They are issued as {\em Starlink Cookbooks}\,
(\htmladdnormallink{SC}{http://www.starlink.ac.uk/sc.html}).

\subsection{Hypertext}

This publishes Starlink information on the World Wide Web in the
form of HTML files, and enables Starlink documents to be linked together
as an integrated whole, using a hypertext linker called HTX
(see \xref{SUN/188}{sun188}{}).
Starlink documents are written mostly in \LaTeX.
New documents are produced in both hypertext and paper form.
Old documents which only exist in paper form can be converted to HTML form by
translators such as {\tt Star2HTML}
(see \xref{SUN/199}{sun199}{}).

The hypertext infrastructure software and organisation required is now
operational
(see \xref{SUN/188}{sun188}{}).
The work of transforming the current document set to hypertext form is
proceeding rapidly.

A \htmladdnormallink{beginner's guide}{http://www.ncsa.uiuc.edu/General/Internet/WWW/HTMLPrimer.html}
to HTML has been distributed as MUD/149.
An updated version of this is available on the web at:
\begin{quote}
{\tt http://www.ncsa.uiuc.edu/General/Internet/WWW/HTMLPrimer.html}
\end{quote}

\section{World Wide Web}

As with paper documents, an effective way to write good pages for the web
is to model them on existing good pages.
You can easily examine the source text of pages that interest you, using
standard facilities of browsers such as Netscape and Mosaic.
Have a look at the Starlink pages.
There are lots of them, and a good place to start is the
\htmladdnormallink{central Starlink page}{http://www.starlink.ac.uk/} at:
\begin{quote}
{\tt http://www.starlink.ac.uk/}
\end{quote}
From here you can go to most Starlink sites from the {\em Sites}\, link.
There are also lots of Style Guides available to help you write good web pages.
An \htmladdnormallink{example}{http://www.w3.org/hypertext/WWW/Provider/Style/Overview.html} is:
\begin{quote}
{\tt http://www.w3.org/hypertext/WWW/Provider/Style/Overview.html}
\end{quote}

\section{\label{GoodStyle}Good style and organisation}

In \htmlref{Section 2.1}{BestDocs} we listed the Starlink documents which
the users voted the best available.
Do these have any common features which can be used as guides for ones own
writing?
We took a close look at them, and all or most of them have the following
characteristics:

\begin{itemize}
\item Well organised and formatted, so the structure of the document
is clear and it is easy to find things.
\item An abstract or description on the front page, summarising what the
document is about.
\item A contents list (at the front).
\item An introduction and overview.
\item A summary of the types of functions provided, and a classified list of
commands or routines (by function; not just alphabetical).
\item Clear instructions on how to start up the package or use the routines.
\item Plenty of examples.
\item Detailed descriptions of commands or routines, placed in a separate
section or appendix.
\item Descriptions of new features or changes introduced by the latest version
of the software, and possibly a description of limitations or restrictions.
\end{itemize}

Any non-trivial document describing a software package or routine library
should have this sort of sectioning.
Some (such as \xref{SUN/1}{sun1}{} which surveys the whole of Starlink
software) can also benefit from having an index (much easier to produce with
the new version of \LaTeX).

We also recommend the following further characteristics for your Starlink user
documents:

\begin{itemize}

\item Organise your paper into sections using sectioning commands (see
\xref{SC/9, Section2}{sc9}{document_structure}).

\item Use the {\em list}\/ facilities to highlight options or features
(see
\xref{SC/9, Section 4}{sc9}{lists}).

\item Indent example commands and show them as they appear to the user,
including any prompt symbol.
Leave a space between the prompt symbol and the command -- `what you read is
what you see'.
Thus:
\begin{verbatim}
    ICL> CAR_HELP SEARCH
\end{verbatim}
is clearer than
\begin{verbatim}
ICL>CAR_HELP SEARCH
\end{verbatim}

\item Try to keep your formatting techniques as simple and obvious as possible.
We find the basic techniques shown in \xref{SC/9}{sc9}{} to be adequate for
all normal purposes.

\item Documents are often revised, merged, or restructured many times, so
you should make it as easy as possible for this to be done on a computer.
In your source document we recommend that you start each sentence on a new
line; this makes future identification, insertion, selection, and movement
of text very easy.

\end{itemize}

Starlink is trying to achieve a common style in all its documents, particularly
in their format, so as to present a professional and uniform image.
To achieve this, we request that you include in your documents the standard
elements specified in appendix A.

\section{Document diseases}

The prime purpose of technical documents is to convey correct information
efficiently and clearly.
However, documents written by non-specialist authors often display features
which, we believe, decrease this efficiency.
Also, most authors type their own text, so various filters that cleaned raw
text in the past (like fussy typists and obsessive editors) have been bypassed.
The result has been an epidemic of document diseases which makes life harder
for the reader.
Here are some that we come across:

\begin{description}

\item [Verbosity:]

The most common fault in writing.
It is an insidious disease which afflicts us all, but it can be controlled if
you look for it.
There is far too much to read without having to wade through redundant waffle.
Good documents should be {\em short}\/ and {\em concise}.
Get to the point!
A reader does not want to waste his time ploughing through:
\begin{quote}
{\em The purpose of this document is to introduce the reader to the CHART
software system which has been developed to facilitate the production of
finding charts which, hopefully, will be of some assistance to the user's
astronomical investigations.}
\end{quote}
when he would be quite satisfied with:
\begin{quote}
{\em This program produces star finding charts.}
\end{quote}

\item [Members only:]

Authors sometimes get so wrapped up in their work that they cannot conceive
of anyone not being {\em au fait}\/ with what they are writing about.
Don't just write for members of your clique, but give outsiders some background
and context to your work.
Also, a brief introduction is very useful for me when I am adding notes on new
software to SUN/1.

\item [Where's the ON switch?:]

Believe it or not, people sometimes omit to describe how to start up their
software, or they may bury this information in the middle of a big paragraph.
Put this information near the front and make it clear, e.g.:
\begin{quote}
{\em To start the program, type the command:
\begin{verbatim}
    $ CHART
\end{verbatim}
You can then get an introductory summary of the principal commands by typing:
\begin{verbatim}
    $ PROMPT
\end{verbatim}}
\end{quote}

\item [Smothered punctuation marks:]

Most people accept the convention of leaving a space between individual words
as they recognise that it helps the reader.
However, the convention of leaving a space between sentences and after
punctuation marks is under attack.
One now sees text written like this:
\begin{quote}
{\em I can save my precious time,and energy,by missing out unnecessary
spaces;anyway,I'm impatient.I don't care if this makes my text
irritating to read,and infuriating to edit,as my time is more important than
yours,and this extra speed gives me a competitive edge.}
\end{quote}
This style of punctuation might be tolerable in a Fortran FORMAT statement
(where, perhaps, it originated), but it is barbaric and illiterate in text
meant for the human eye.

\item [Lonely parentheses:]

This is the inverse of the {\em Smothered punctuation marks} disease\/ --
too much space rather than too little.
It has the following symptom:
\begin{quote}
{\em Some authors ( not many ) do this.}
\end{quote}
No doubt, such authors think this is clearer than:
\begin{quote}
{\em Most (thank goodness) do this.}
\end{quote}
The problem with this disease is that it makes the syntactic structure of a
sentence less obvious -- you are writing for the human eye, not a parser.
Moreover, unless you go to special trouble, odd parentheses will stray onto the
previous or next line.

\item [Tabs:]

The trouble with tabs is that they may not be interpreted as you expect, since
this depends on the device or software your reader is using.
What you see on your terminal may not be what he sees on his.
The safest thing is not to use them.

\item [Missing files:]

A common disease of \TeX\ virtuosi is to submit document files which refer to
other files which only exist in their author's local environment.
Make sure you submit all the files required for someone else to produce
hardcopy.

\item [Clever dates:]

Many text processors have a facility to generate today's date automatically.
This can be very convenient in things like letter headings which are used
to produce a single document at a specific moment, but it is dangerous in
long-life source documents that may generate hardcopy several times.
A date should identify such a document uniquely and should not be vulnerable to
arbitrary change.
The use of clever dates can cause a number of versions of a document to go
into circulation which are identical, apart from the date in the heading.
This can cause great confusion.

\end{description}

\subsection{Web diseases}

The World Wide Web is a popular tool for publishing information,
and Starlink uses it extensively.
The enormous amount of material on the web and the arrival of commercial pages
has led to a competitive striving for attention by its information providers.
This has not always been to the advantage of the reader, as there is sometimes
a conflict between attracting a reader to your text, and making that text
easy to read.
Three forms of attention-seeking have spread across the web like cane toads in
Queensland.
We believe they compromise legibility:

\begin{description}

\item [Background images:]

Text is often presented against a background image.
The intention is to make the page look attractive and encourage people to
read it.
This technique may work when used with discretion in high-quality publications
printed on glossy paper, but we feel it doesn't work successfully on computer
terminals.
The reason is that most computer displays have much lower resolution and
dynamic range than glossy paper, so a background image seriously degrades the
legibility of the foreground text.
Another problem is that some browsing software just can't handle background
images properly.
If you use background images your text may become illegible (and your writing
efforts futile).

\item [Fancy colours and fonts:]

You don't have to use background images to make your reader abandon an
attempt to read what you have written.
This can also be achieved by extrovert choices of colour for text and
background, and also by using fancy fonts.

\item [Flashing lights:]

You can draw attention to the latest or most important information on your web
page by repeatedly flashing it on and off.
The effect on at least some readers is similar to that produced by those car
alarms that go off in the car park next to your office window.
The problem is the same: you can't turn the message off once you've got the
point.
It then becomes distracting and merely adds to the ambient noise.

\end{description}

No doubt when we all have multi-media terminals, information will be
accompanied by exciting background music and stimulating drum beats.
We are not looking forward to it.

\section{References}

\begin{enumerate}
\item MUD/149: {\em HTML -- A beginner's guide.}
\item \xref{SC/9}{sc9}{}: {\em \LaTeX\ Cook-book.}
\item \xref{SGP/42}{sgp42}{}: {\em Starlink software strategy.}
\item SGP/43: {\em Starlink software survey results, 1994.}
\item \xref{SUN/9}{sun9}{}: {\em \LaTeX\ -- A document preparation system.}
\item \xref{SUN/188}{sun188}{}: {\em HTX -- Hypertext cross-reference utilities.}
\item \xref{SUN/199}{sun199}{}: {\em Star2HTML -- Converting Starlink documents
  to hypertext.}
\item \xref{SUN/201}{sun201}{}: {\em Latex2HTML -- \LaTeX\ to HTML conversion.}
\item Leslie Lamport: {\em A Document Preparation System. \LaTeX\ User's
Guide \& Reference Manual}, 2nd ed, Addison-Wesley, 1994,
ISBN 0-201-52983-1
\end{enumerate}

\appendix

\newpage

\section{Standard elements of a Starlink document}

\subsection{Code, date, filename}

Every Starlink document has an identifier and a date of issue.
Identifiers have a format like {\tt SGP/28.7} where {\tt SGP} is the class code,
{\tt 28} is the number within that class, and {\tt 7} is the version number
starting at 1.
Identifiers are allocated by the Document Librarian
({\tt mdl@star.rl.ac.uk}).
When a note is revised, its version number should be incremented and the
date of issue updated.

Files which store Starlink documents should have a name like {\tt sgp28.tex},
where {\tt sgp} shows the document class, {\tt 28} tells you the number, and
{\tt tex} tells you it is a \TeX\ or \LaTeX\ source file.
Files containing Starlink classified documents are stored in {\tt /star/docs}.

\subsection{Standard headers, type size, page size, layout}

Use the \LaTeX\ template files like {\tt sun.tex} which are stored in
{\tt /star/docs}.
These have properly formatted headers and correct type size, page size, and
layout.
They also have the definitions required for you to produce hypertext, as
described in \xref{SUN/199}{sun199}{}.
Avoid using a type size smaller than 11 point.

\subsection{Title}

A user looking for information usually selects a note on the basis of its title,
and this should, therefore, be concise and informative.
It should contain the acronym used to refer to the software, and indicate its
function.
Remember, your note may have a long life, so phrases such as ``A New\ldots",
which will quickly become either obsolete or positively misleading, should be
avoided.
An example title is:

\begin{center}
{\bf ASPIC -- A set of image processing programs}
\end{center}

Don't assume a reader already knows what your software does.
For example, don't have a title like:

\begin{center}
{\bf MYPROG -- An introduction}
\end{center}

Say what MYPROG does:

\begin{center}
{\bf MYPROG -- An HTML editor: Introduction}
\end{center}

\newpage

\section{Producing a document}

\subsection{Starting or revising a document}

When {\em starting a new document}, copy the appropriate template file from
{\tt /star/docs}
into your own directory and change the variable information; then add your
own text.
This procedure is explained in some detail in
\xref{SUN/199}{sun199}{}.
Produce your original file in the form of a {\tt .tex} file ready for
processing by \LaTeX\ and Star2HTML.

When {\em updating a document}, copy the existing file from {\tt /star/docs}
and edit that; do {\em not} start from a private copy as your submitted file
may have been edited by a Librarian before final issue, and the
changes (such as spelling corrections) will be lost and have to be done again.

\subsection{Using \LaTeX}

For all but the simplest of notes you will probably wish to use a text
processing program.
Starlink recommends {\LaTeX}.
Don't use \TeX\ unless you enjoy making life difficult for yourself and
others; \LaTeX\ is much easier.
In our experience, the output from \TeX\ source is inferior to that of
standard Starlink \LaTeX\ documents, and invariably has idiosyncratic style and
format.

\xref{SC/9}{sc9}{}
contains many examples of how to use \LaTeX\ and shows the style preferred by
Starlink.
\xref{SUN/9}{sun9}{} describes how to use \LaTeX.
There are lots of books about \LaTeX\ to help you use it properly.
The basic reference is the book by Leslie Lamport listed in the References.

\subsection{Using Star2HTML}

If possible, you should add hypertext links to your document and use
Star2HTML to produce the hypertext version.
This procedure is explained in \xref{SUN/199}{sun199}{}.
The basic procedure for checking the hypertext version of your document
from within your working directory is as follows:
\begin{quote}
{\tt
\% star2html sgp28\\
\% hlink\\
\% showme ./sgp28}
\end{quote}
This will display your hypertext document (in this case, sgp28.tex) in a window
running a web browser.

\subsection{Submitting a document for distribution}

When you have finished your document, tell the Software Librarian
({\tt starlink@jiscmail.ac.uk}) where your file is.
If necessary, send paper copies of any additional material, such as diagrams,
which is required in the final document, but which doesn't appear in hardcopy
derived from your file.
The Librarians may edit your file to make it conform to Starlink standards, and
will then produce a paper master and get copies made.
They will also distribute the file in a software release, and send several
copies to every Starlink Site Manager for users.

\end{document}
