\documentstyle{article} 
\pagestyle{myheadings}

%------------------------------------------------------------------------------
\newcommand{\stardoccategory}  {Starlink General Paper}
\newcommand{\stardocinitials}  {SGP}
\newcommand{\stardocnumber}    {28.4}
\newcommand{\stardocauthors}   {M D Lawden}
\newcommand{\stardocdate}      {20 January 1989}
\newcommand{\stardoctitle}     {Starlink Document Production}
%------------------------------------------------------------------------------

\newcommand{\stardocname}{\stardocinitials /\stardocnumber}
\markright{\stardocname}
\setlength{\textwidth}{160mm}
\setlength{\textheight}{240mm}
\setlength{\topmargin}{-5mm}
\setlength{\oddsidemargin}{0mm}
\setlength{\evensidemargin}{0mm}
\setlength{\parindent}{0mm}
\setlength{\parskip}{\medskipamount}
\setlength{\unitlength}{1mm}

\begin{document}
\thispagestyle{empty}
SCIENCE \& ENGINEERING RESEARCH COUNCIL \hfill \stardocname\\
RUTHERFORD APPLETON LABORATORY\\
{\large\bf Starlink Project\\}
{\large\bf \stardoccategory\ \stardocnumber}
\begin{flushright}
\stardocauthors\\
\stardocdate
\end{flushright}
\vspace{-4mm}
\rule{\textwidth}{0.5mm}
\vspace{5mm}
\begin{center}
{\Large\bf \stardoctitle}
\end{center}
\vspace{5mm}

\section{Introduction}

This note gives advice on how to prepare documents that are meant to be
distributed through Starlink.
The emphasis is on documents that are meant to read in printed form; on-line
documentation is a specialised art and is not dealt with here.

Starlink documents fall into two broad categories:
\begin{itemize}
\item Unclassified.
\item Classified.
\end{itemize}
The format of the former is left almost entirely to the discretion of the
author, with the exception of a few general points outlined in the next section.
The format of the latter is more tightly controlled as they are intended to form
a uniform set of documents issued by the Project.
The Starlink Documentation Librarian (currently Mike Lawden) has editorial
powers to make format changes and obvious corrections to submitted classified
documents in order to preserve a uniform `Starlink style'; the information
content will not be affected.

\section{Unclassified documents}

When unclassified documents are available as computer files, they will be
stored in association with the software they describe, and will be a primary
source of information for people using this software and for programmers
maintaining it.
If appropriate, they will be indexed in DOCSDIR:MUD.LIS as Miscellaneous
User Documents.
If they are an important source of information for a user, they should be
referred to within a note in the Starlink SUN series.

The layout is left to the discretion and good taste of the author, but the
following specific points should be born in mind:
\begin{itemize}
\item If the text is meant to be readable on a VDU screen, lines must not
exceed 79 characters (80 character lines can cause double spacing on some
terminals, indeed some terminals can't handle more than 74), and underlining
(if essential) must be done in such a way that the text does not get destroyed
by the underlines.
\item If the document is meant for a Printronix or Sanders printer, it should
be possible to print it without spurious blank or overflow pages coming out.
It should also be possible to photocopy the printed document onto A4 paper with
adequate margins.
This implies lines of no more than 70 characters and a maximum of 57 lines per
page.
\end{itemize}

If the software has an on-line help system, remember that users often wish to
have hardcopies of this information, so bear this in mind when planning its
layout, and indicate in the manual how such copies can be made.

\section{Classified documents}

Classified Starlink documents which are project-wide are stored in DOCSDIR and
are indexed in the Starlink documentation index DOCSDIR:DOCS.LIS.
Each site may also have local documents stored in LDOCSDIR and indexed in
LDOCSDIR:DOCS.LIS.
These documents are the basic source material for Starlink users.
If they are of high quality, standard and uniformity, they make the Project
look professional and well managed.
Alternatively, documents which contain trivial errors and are presented in a
variety of styles and formats make the project look amateurish and poorly
managed.
Part of the Starlink Documentation Librarian's job is to achieve the former
state, rather than the latter.

There are five classes of document:
\begin{description}
\begin{description}
\item [SUG] --- Starlink User Guide

This is a general introduction to Starlink for new users.

\item [SG] --- Starlink Guides

These are guides to the effective use of specific Starlink software items.

\item [SUN] --- Starlink User Notes

These describe how to use Starlink software.

\item [SSN] --- Starlink System Notes

These give implementation and installation information on Starlink software.

\item [SGP] --- Starlink General Papers

These cover a wide range of topics such as project management, programming
standards, and proposals for future developments.
\end{description}
\end{description}

\subsection{Content}

A SUN may fully document a software item, or may just be an introduction
which references more complete documentation stored elsewhere.
In either case, remember that the reader may know nothing of either the specific
software item, or any other Starlink software; your note may be the first that a
new user reads!
It is reasonable to assume that the reader is an astronomer with some
familiarity with computers in general and the VAX in particular, and has read
(and understood!) the VAX/VMS Introduction (even though this may well not be
the case).

Do not assume that the user is reading the note just to find out how to use the
software (although of course he may be).
Notes are often read by people who wish to know what the software does, what it
is for, or (most difficult of all) if it will solve a particular problem.
To satisfy all three types of reader, a User Note will typically contain the
following sections, probably in the order given.
\begin{itemize}
\item An introduction outlining the intended use of the software or the
problem it addresses.
\item A general description of the methods used.
\item A tutorial style introduction to using it.
\item A detailed description of commands, subroutine interfaces or whatever is
appropriate.
\item A detailed description of the algorithms used.
\end{itemize}
Each section can be anything from a short paragraph followed by references to
other documents, to a complete description many pages long.
The first two sections should aim to be relatively complete, while the remainder
may in some cases be little more than a list of references.
Other sections may be required in many cases.

\subsection{Title}

A user looking for information usually selects a note on the basis of its title
(as listed in the documentation index) and it should therefore be as informative
and concise as possible about the content of the note.
It should contain the acronym generally used to refer to the software and
indicate its function.
Remember, the note may have a long life, so phrases such as ``A New\ldots" which
will quickly become either obsolete or positively misleading should be avoided.
An example title is:

\begin{center}
{\bf ASPIC --- A set of image processing programs}
\end{center}

\subsection{Document identification}

Each document has an identifier and a date of issue.
Identifiers have the format `class/nn.v' where `class' is the class,
`nn' is the number within that class, and `v' is the version number starting
at 1 (e.g.\ the identifier of this document is SGP/28.4).
Identifiers are allocated by the Documentation Librarian
(MDL@UK.AC.RL.STAR on JANET, or RLVAD::MDL on DECNET), who can also be
consulted on any other aspect of document production.
If you like, you can specify your document number as `nn' and leave the
Librarian to alter this to the required number.
When a note is revised, its version number should be incremented and the issue
date revised.

\subsection{Header layout}

You must conform to the standard style and layout for the header to a Starlink
document.
Use the \LaTeX\ template files SUN.TEX, etc., or the RUNOFF template files
SUN.RNO, etc., which are stored in DOCSDIR.

\subsection{File names}

Files which store Starlink documents should have a name of the form SUNn, etc.,
where `n' is the number of the document.
For example, this note is in a file called SGP28.
The file name extension should be `.TEX' for a \LaTeX\ version, `.RNO' for a
RUNOFF version, and `.LIS' for a readable version (only relevant for RUNOFF).
Manuals and guides may use other conventions, but if so these should be
clearly described in the accompanying Starlink Note.
Files containing classified documents are stored in DOCSDIR.

\section{Text processors}

For all but the simplest of notes you will probably wish to use a text
processing program.
Starlink recommends that you use {\LaTeX}, although RUNOFF is an acceptable
alternative (but note that such documents may be converted to \LaTeX\ by
Starlink staff).
Don't use GEROFF as it is no longer supported by Starlink and may stop
working.
Don't use \TeX\ unless you enjoy making life unnecessarily difficult for
yourself and others; \LaTeX\ is much easier.
In my experience the output from \TeX\ source (assuming you have access to all
the private macros the author thought everyone had) always turns out to be
inferior to that of standard Starlink documents and invariably has idiosyncratic
style and format.

Various aspects of \LaTeX\ are documented in SUN/9, 12, 34, and 77; there is
also an excellent book on the subject by Leslie Lamport.
\TeX\ is documented in SUN/93 and in the book by Donald Knuth.

\section{Starting or revising a document}

When starting a new document, copy the appropriate template file from DOCSDIR
into your own directory and change the variable information; then add your
own text.
Produce your original file in the form of a .TEX or .RNO file ready for
processing by the text processor.
Try to avoid making your document dependent on other files; it should be
self-sufficient.
If you must have things like `include' files, make sure you submit them at
the same time as your main file.

When updating a document, copy the existing file from DOCSDIR and edit
that; do {\em not} start from a private copy as your submitted file may have
been edited by the Documentation Librarian before final issue, and the changes
(such as spelling corrections) will be lost and have to be done again.

\section{Submitting a note for distribution}

When you have finished your document, tell the Documentation Librarian the
whereabouts of the file you have produced.
If necessary, send paper copies of any additional material required, such as
diagrams, which will not appear in the hardcopy derived from your file.
The Librarian may edit your file to make it conform to Starlink standards and
will then produce a paper master and make copies.
He will also distribute the file in a software release and send several paper
copies to every Starlink Site Manager for his users.

\section{Recommended techniques}

SUN/12 contains many examples of how to use \LaTeX\ and shows the style
preferred by Starlink.
An earlier version of this paper (SGP/28.3) contained recommended techniques
for using RUNOFF.

\section{Style}

Starlink is trying to achieve a common style in all its documents, particularly
in their format.
A number of stylistic recommendations are listed below, and the next section
warns you about various `Documentation diseases' that are rampant.

\begin{itemize}

\item Organise your paper into sections using sectioning commands (see SUN/12,
chapter 2).

\item Use the {\em list}\/ facilities to highlight options or features
(see SUN/12 chapter 4).

\item Indent example commands and show them as they appear to the user,
including any prompt symbol.
Leave a space between the prompt symbol and the command --- `what you read is
what you see'.
Thus:
\begin{verbatim}
    $ CAR_HELP SEARCH
\end{verbatim}
is clearer than
\begin{verbatim}
$CAR_HELP SEARCH
\end{verbatim}

\item Try to keep your formatting techniques as simple and obvious as possible.
I find the basic techniques shown in SUN/12 to be adequate for all normal
purposes.

\item Documentation is often revised, merged or restructured many times and
you should make it as easy as possible for this to be done on a computer.
For \LaTeX\ source, etc, you are recommended to start each sentence on a new
line; this makes insertions and movement of chunks of text very easy.

\end{itemize}

\section{Documentation diseases}

There are a number of faults in documents produced by non-specialist authors
which can be recognised and classified fairly easily.
I call them {\em documentation diseases}.
Some affect the content and organisation of a document.
Others are trivial typing faults.
The following are diseases of content.

\begin{description}

\item [Verbosity] :

This is a common fault in documentation submitted to Starlink.
It is an insidious disease which afflicts us all, but its symptoms can be
controlled if you check your own text for them.
There is far too much to read without having to wade through redundant waffle.
Good documents should be as {\em short}\/ and {\em concise}\/ as possible.
A reader does not want to waste his time ploughing through:
\begin{quote}
The purpose of this document is to introduce the reader to the CHART software
system which has been developed in the Fortran language on the RGO Starlink
computer (funded by the SERC) to assist the user in the production of finding
charts which, hopefully, will be of some assistance to the user's astronomical
investigations.
\end{quote}
when he would be quite satisfied with:
\begin{quote}
The CHART program produces star finding charts.
\end{quote}

\item [Members only] :

Authors sometimes get so wrapped up in their work that they cannot conceive
of anyone not being {\em au fait}\/ with what they are writing about.
I once attended an introductory lecture on Cybernetics at which the lecturer
handed out a sheet containing a very complex diagram of his model of human
visual perception.
The class was then asked to suggest improvements.
It was our first exposure to a subject he had been researching for years.
This kind of pedagogical incompetance is sometimes evident, in milder form, in
submitted Starlink User Notes.
Don't just write for members of your clique, but give faint hearts and
browsers some background and context to your work.
Also,a brief introduction is very useful for me when I am adding notes on new
software to SUN/1.

\item [Where's the ON switch?] :

Believe it or not, people sometimes omit to describe how to start up their
software, or they may bury this information in the middle of a big paragraph.
Make this information very clear and put it in front of information on how
to use your software; e.g.:
\begin{quote}
To start the program, type the command:
\begin{verbatim}
    $ CHART
\end{verbatim}
You can then get an introductory summary of the principal commands by typing:
\begin{verbatim}
    $ PROMPT
\end{verbatim}
\end{quote}
Clear, isn't it?

\end{description}

These days, most documentation is produced by the author typing his text into
a computer and producing hardcopy through a text processor.
This means that various filters that cleaned the raw text in the past (like
fussy typists and obsessive editors) have been eliminated.
The result has been an epidemic of trivial documentation diseases which
disfigure some of the text submitted to Starlink for release.
Some of these seem to be caused by the author's desire to save as much
of his own time as possible.
However, one assumes a document worth publishing will have more readers
than authors, so it is worth while for the author to spend a little extra
time to save his readers time and effort.
Here are some of these diseases I come across frequently:

\begin{description}

\item [Smothered punctuation marks] :

Most people accept the convention of leaving a space between individual words
as they recognise that it aids comprehension.
However, the old convention of leaving a space between sentences and after
punctuation marks is under attack.
One frequently sees text written like this:
\begin{quote}
I can save my precious time,and energy,by missing out unnecessary
spaces;anyway,I'm impatient.I don't care if this makes my text
irritating to read,and infuriating to edit,as my time is more important than
other peoples,and this extra speed gives me a competitive edge.Why do people
bother about such trivial things?Don't be so fussy!
\end{quote}
This style of punctuation might be tolerable in a Fortran FORMAT statement
(where, perhaps, it originated), but it is barbaric and illiterate in text
meant for the human eye.
It makes the reader angry with you.
Is that what you want?

\item [Lonely parentheses] :

This is the inverse of the {\em Smothered punctuation marks} disease\/ ---
too much space rather than too little.
It has the following symptom:
\begin{quote}
Some authors ( not many ) do this.
\end{quote}
No doubt, such authors think this is clearer than:
\begin{quote}
Most (thank goodness) do this.
\end{quote}
The problem with this disease is that it makes the syntactic structure of a
sentence less obvious --- you are writing for the human eye, not a parser.

\item [Tabs] :

The trouble with tabs is that they may not be interpreted as you expect, since
this depends on the device or software your reader is using.
What you see on your terminal may not be what he sees on his.
The safest thing is not to use them.

\item [Missing files]:

A very common disease of \TeX\ virtuosi and RUNOFF aficionados is to submit
document files which refer to other files which only exist in their author's
local environment.
Make sure you submit all the files required to produce hardcopy outside
your personal software environment.

\item [Clever dates]:

Many text processors have a facility which enables you to generate automatically
today's date when preparing hardcopy.
This can be very convenient in things like letter headings which are used
to produce a single document at a specific moment, but it is a source of
danger when used in long-life documents that can be generated several
times.
A date should identify such a document uniquely and should not be vulnerable to
arbitrary change.
The use of clever dates can cause a number of versions of a document to go
into circulation which are identical, apart from the date in the heading.
This can cause great confusion.

\end{description}

\section{References}

\begin{enumerate}
\item SUN/9, `\LaTeX\ --- A document preparation system'.
\item SUN/12, `\LaTeX\ Cook-book'.
\item SUN/34, `\TeX\ DVI file translators'.
\item SUN/77, `DVIDIS --- DVI file previwer'.
\item SUN/93, `\TeX\ --- A superior document preparation system'.
\item Leslie Lamport, `A Document Preparation System. \LaTeX\ User's Guide \&
 Reference Manual', Addison-Wesley, 1986.
\item Donald E Knuth, `The \TeX book', Addison-Wesley, 1986.
\end{enumerate}

\end{document}
