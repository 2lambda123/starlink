\documentstyle[11pt]{article}
\pagestyle{myheadings}

%------------------------------------------------------------------------------
\newcommand{\stardoccategory}  {Starlink General Paper}
\newcommand{\stardocinitials}  {SGP}
\newcommand{\stardocnumber}    {12.1}
\newcommand{\stardocauthors}   {D L Terrett}
\newcommand{\stardocdate}      {16 September 1992}
\newcommand{\stardoctitle}     {AVS as a GUI for ADAM}
%------------------------------------------------------------------------------

\newcommand{\stardocname}{\stardocinitials /\stardocnumber}
\renewcommand{\_}{{\tt\char'137}}     % re-centres the underscore
\markright{\stardocname}
\setlength{\textwidth}{160mm}
\setlength{\textheight}{230mm}
\setlength{\topmargin}{-2mm}
\setlength{\oddsidemargin}{0mm}
\setlength{\evensidemargin}{0mm}
\setlength{\parindent}{0mm}
\setlength{\parskip}{\medskipamount}
\setlength{\unitlength}{1mm}

%------------------------------------------------------------------------------
% Add any \newcommand or \newenvironment commands here
%------------------------------------------------------------------------------

\begin{document}
\thispagestyle{empty}
SCIENCE \& ENGINEERING RESEARCH COUNCIL \hfill \stardocname\\
RUTHERFORD APPLETON LABORATORY\\
{\large\bf Starlink Project\\}
{\large\bf \stardoccategory\ \stardocnumber}
\begin{flushright}
\stardocauthors\\
\stardocdate
\end{flushright}
\vspace{-4mm}
\rule{\textwidth}{0.5mm}
\vspace{5mm}
\begin{center}
{\Large\bf \stardoctitle}
\end{center}
\vspace{5mm}

%------------------------------------------------------------------------------
%  Add this part if you want a table of contents
%  \setlength{\parskip}{0mm}
%  \tableofcontents
%  \setlength{\parskip}{\medskipamount}
%  \markright{\stardocname}
%------------------------------------------------------------------------------

\section{Introduction}

The basis of this report is 2 days spent with an AVS expert from DEC's CERN
project office attempting to convert an ADAM application into an AVS module.
The experiment was successful in that we succeeded in running a KAPPA
application (ADD) as a module in an AVS network without modifying the
applications program code in any way. We took many short cuts and it became
clear that doing the job properly would be a major exercise, but we
learned enough to know that the job is feasible and gained a clear idea of
what the final system would look like and what it would be capable of.

\section{What is AVS}

At the heart of AVS is the concept of a ``network''. An AVS network is similar
to  the more familiar pipeline where the output of one application is fed into
the input of another application, but in a network applications can have
several inputs and several outputs and can therefore be connected up in a more
complex way. It is even possible to have loops that implement iterative
techniques. AVS uses the terms ``module'' for the individual elements of a
network and ``ports''  for the inputs and outputs.

Networks are created with the graphical ``network editor''; modules are dragged
from lists of available modules (organized into module libraries) with the
mouse and placed in a diagram of the network (each module represented by a box
with colour coded blobs representing the module's inputs and outputs). The
output ports of one module can then be connected to the input ports of other
modules, also with the mouse. Networks can be saved on disk and restored later.

As well as input and output ports, a module also has parameters; parameters
have their values set by the user via a ``control panel''. There are a variety
of styles of control (AVS uses the X11 term ``widget''), such as sliders,
dials, text input fields, etc. available. The layout of the control panel is
defined by code in the module but can be altered by the user with the ``layout
editor'' (for example, the widget for an integer parameter could be changed
from a dial to a text entry field) and parameters can be converted to input
ports so that they can be connected to the output ports of other modules
instead of being set directly from the control panel.

The layout editor can also move controls from one control panel to another and
could be used to create a single control panel  for an entire network with the
parameters renamed to use terminology appropriate to the particular job the
network was designed to do.

An AVS module is a program, and the program is run (in a sub-process) as it is
dragged onto the network diagram. At this point the module's initialisation
routine is called; this routine calls AVS subroutines to define the contents
and layout of the control panel which also pops up at this point. A single
executable program can contain several AVS modules (cf.\ ADAM monoliths).

The execution routine of a module is called whenever any of the module's input
ports or parameters change their values (this can happen because the user
changed the value of a parameter with the control panel, the output of some
other module in the network changed, or the connection to the input port was
changed). It may not always be appropriate to run the applications code at this
point; for example, all the necessary input ports may not be connected (e.g.\
while the network is being edited) and the user would normally want to set
several parameter values before running the application. There are AVS
subroutines that can be called to determine whether a parameter or input port
has changed its value etc.

If the applications code is run, the execution routine then calls subroutines
to inform AVS whether or not the module's output ports have changed their
values. If they have, this  will cause other modules in the network to execute
and so on until the change  has propagated throughout the network and a new
final output has been  generated. It is worth noting that there is no parallel
processing involved;  AVS invokes the execution routine of each module in turn,
waiting for it to  finish before invoking the next.

AVS comes with a large number of modules which implement a wide range of 2 and
3D image processing algorithms and data visualization techniques. It is access
to these data visualization modules that makes the use of AVS as a GUI for ADAM
particularly attractive.

\section{AVS Data Types}

Every parameter and port has a data type associated with it, and output ports
can only be connected to input ports with compatible data types. AVS defines
the usual simple types, integer, float, string etc, and a variety of more
complex types for representing 2D and 3D images, colour maps etc. User defined
data types can also be created but are limited to things that can be defined by
a C structure definition, so are essentially fixed size.

None of these data types gets anywhere near what is needed for NDFs or HDS
structures, so we elected to pass the names of HDS files between applications
as character strings. (A full implementation would probably define a user data
type for this purpose rather than using the built in character string type.)
When an NDF is being used for transferring data from one module to another
within a network it is necessary to generate ``scratch'' NDF names and care
needs to be taken to ensure that they are deleted when no longer required in
order to prevent the file system being overwhelmed by temporary NDFs.

\section{Modifications to ADAM}

In order to integrate ADAM applications properly with AVS the following
modifications to the ADAM system code would be required:
\begin{description}

\item[The fixed part]
The ADAM fixed part needs to be restructured somewhat to match the AVS module
structure of separate initialisation and invocation routines.

\item[The parameter system]
AVS cannot support the concept of an application rejecting the current value of
a parameter and prompting for a new value so the parameter system needs to be
modified to return an error rather than prompting under these circumstances.

\item[The message/error system]
Although an AVS module can write to its standard output or error channels, the
output appears in the terminal window from which AVS was invoked which is not
ideal. The error and message systems therefore should be modified to call
AVS subroutines that display messages in pop-up windows.

\end{description}

\section{Creating a Module}

For each ADAM application an initialisation routine, an execution routine and
possibly a destruction routine has to be written. Both C and Fortran are
supported but C is the easier option because the routines have to deal with
pointers and structures and this is somewhat convoluted in Fortran and didn't
appear very portable. (We initially started work in Fortran but switched to C
after failing to get certain operations to work---however, we were working
without manuals and the DEC programmer was not familar with Fortran.)

This initialisation routine has to do the following:
\begin{itemize}

\item Initialise the ADAM parameter system.

\item Create an input port for each input NDF.

\item Create an AVS parameter for all other input ADAM parameters and set their
  initial values (obtained from the ADAM parameter system).

\item Optionally set the widget type and layout for each parameter (there is a
  default widget type for each parameter data type).

\item Create output ports for all output ADAM parameters.

\item Create a ``GO'' parameter attached to a button widget that can be used by the
  user to indicate that all parameters have been set and that the application
  should execute.

\item Create AVS parameters of type character string for all input and output NDFs
  to allow the names of the NDFs to be set by the user. This enables the user
  to explicitly set the name of an input or output NDF instead of having some
  scratch file name generated. This is necessary when a module has ports not
  connected to any other module.

\end{itemize}
This is straightforward and could conceivably be generated automatically from
the interface file (perhaps with the addition of some new interface file
keywords). Alternatively, AVS contains a module generation tool that uses menus
to define parameters and ports and their properties and will generate the
neccessary code (in either C or Fortran).

The execution routine has to:
\begin{itemize}

\item Examine the state of its input ports in order to decide whether the
  application should be executed.

\end{itemize}
If the decision is to execute the application code:
\begin{itemize}

\item Copy the values of the AVS parameters to the corresponding ADAM parameters.

\item Generate names for any output NDFs which don't have their names specified
  by AVS parameters and set the corresponding ADAM parameters.

\item Execute the applications code.

\item Set the values of any output ports from the values of the corresponding ADAM
  parameters.

\end{itemize}
The decision whether to execute the application or not is quite tricky but is
something along the lines of:
\begin{quote}
	If all inputs have either a valid value or the corresponding NDF
	name parameter has a value then execute if either ``GO'' was
	pressed or one of the inputs has changed its value.
\end{quote}

The destruction routine is called when a module is removed from the network or
the network as a whole is deleted, and would be responsible for removing and
scratch NDFs created by the module.

\section{Other Tasks}

To produce a finished product there are a number of other things that would need
to be done, none of them particularly challenging:

\begin{itemize}

\item Organise ADAM applications into AVS module libraries, probably along the
  existing package lines but something like KAPPA may be too big for comfort.
  When building a network, modules have to be selected from a list of all the
  modules in a library; finding the one you want among 200 could be tedious.

\item Write a module (or modules) for converting NDFs to AVS internal data types so
  that the results of computations can be displayed using the wealth of
  visualisation tools that AVS provides.

\item Convert ADAM help information into AVS format.

\end{itemize}

\section{Problems}

What clouds does this particular silver lining have ?
\begin{itemize}

\item Only ADAM programs that are ``black boxes'' could be run from AVS; no
  interaction with the user other than the display of messages is allowed once
  the applications code has started running.

\item No dynamic defaults. Default values for parameters have to be set before
  the applications code runs.

\item Cost. The list price for AVS on a DECstation 5000 is 3K pounds. Chest deals
  etc. may make it cheaper (even free) for some but probably not everyone.

\item Hardware required. A fairly powerful system is needed; a DECstation 5000
  with 48 Mbyte of memory is probably a minimum for real data processing. The
  visualisation stuff really requires fancy graphics accelerators and
  preferably 24 bit displays. To quote the AVS manual:
\begin{quote}
	``AVS was originally implemented to run on very high performance,
	true colour (24 plane) workstations with considerable hardware
	support for 3D graphics rendering, including gouraud shading,
	lighting, texture mapping, and hardware-assisted polygon and
	sphere rendering and transformations.''
\end{quote}
  Having all the applications in a network loaded at once places
  considerable demands on things like system swap space, and the intermediate
  files created as the network executed would stretch the available disk space.
  Inadequate system resources could easily make the whole thing effectively
  unusable.

\item A non-negligible amount of effort would be required to implement the system,
  although probably not more (and quite possibly less) than other approaches to
  creating a GUI interface for ADAM. I would estimate perhaps 3 man-months to
  get a system that could demonstrate what could be achieved and another 3 to
  get a releaseable system. Elapsed time would probably be double this.

\item The ``global association'' mechanism used to connect the output of one ADAM
  application to the input of another is probably superfluous and may well
  conflict with way a network works. It may therefore be necessary to modify
  all interface files (alternatively it might be better to disable this
  mechanism within the parameter system).

\end{itemize}

\section{Data Acquisition}

That AVS would be a useful tool for data reduction is clear (to me at least!)
but what about in the data acquisition environment? I don't have the knowledge
to do more than speculate on this but the control panel interface seems
applicable and the network editor could be used to set up and then save the set
of ADAM tasks required for a particular instrument setup. There is no
requirement for all the modules in a network to be actually connected so they
can equally well exist as separate applications that are just executed
individually at the request of the user and AVS modules are allowed to execute
continuously (AVS calls them coroutines). The layout editor could be used to
design a unified interface for the whole set of applications.

\end{document}
