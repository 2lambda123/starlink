\documentclass[twoside,11pt]{article}
\pagestyle{myheadings}

%------------------------------------------------------------------------------
\newcommand{\stardoccategory}  {Starlink General Paper}
\newcommand{\stardocinitials}  {SGP}
\newcommand{\stardocnumber}    {38.2}
\newcommand{\stardocsource}    {sgp\stardocnumber}
\newcommand{\stardocauthors}   {Malcolm J Currie, P T Wallace \&
                                R F Warren-Smith}
\newcommand{\stardocdate}      {1989 January 20}
\newcommand{\stardoctitle}     {Starlink Standard Data Structures}
%------------------------------------------------------------------------------

\newcommand{\stardocname}{\stardocinitials /\stardocnumber}
\markright{\stardocname}
\setlength{\textwidth}{160mm}
\setlength{\textheight}{240mm}
\setlength{\topmargin}{-5mm}
\setlength{\oddsidemargin}{0mm}
\setlength{\evensidemargin}{0mm}
\setlength{\parindent}{0mm}
\setlength{\parskip}{\medskipamount}
\setlength{\unitlength}{1mm}

% -----------------------------------------------------------------------------
%  Hypertext definitions.
%  ======================
%  These are used by the LaTeX2HTML translator in conjunction with star2html.

%  Comment.sty: version 2.0, 19 June 1992
%  Selectively in/exclude pieces of text.
%
%  Author
%    Victor Eijkhout                                      <eijkhout@cs.utk.edu>
%    Department of Computer Science
%    University Tennessee at Knoxville
%    104 Ayres Hall
%    Knoxville, TN 37996
%    USA

%  Do not remove the %begin{latexonly} and %end{latexonly} lines (used by 
%  star2html to signify raw TeX that latex2html cannot process).
%begin{latexonly}
\makeatletter
\def\makeinnocent#1{\catcode`#1=12 }
\def\csarg#1#2{\expandafter#1\csname#2\endcsname}

\def\ThrowAwayComment#1{\begingroup
    \def\CurrentComment{#1}%
    \let\do\makeinnocent \dospecials
    \makeinnocent\^^L% and whatever other special cases
    \endlinechar`\^^M \catcode`\^^M=12 \xComment}
{\catcode`\^^M=12 \endlinechar=-1 %
 \gdef\xComment#1^^M{\def\test{#1}
      \csarg\ifx{PlainEnd\CurrentComment Test}\test
          \let\html@next\endgroup
      \else \csarg\ifx{LaLaEnd\CurrentComment Test}\test
            \edef\html@next{\endgroup\noexpand\end{\CurrentComment}}
      \else \let\html@next\xComment
      \fi \fi \html@next}
}
\makeatother

\def\includecomment
 #1{\expandafter\def\csname#1\endcsname{}%
    \expandafter\def\csname end#1\endcsname{}}
\def\excludecomment
 #1{\expandafter\def\csname#1\endcsname{\ThrowAwayComment{#1}}%
    {\escapechar=-1\relax
     \csarg\xdef{PlainEnd#1Test}{\string\\end#1}%
     \csarg\xdef{LaLaEnd#1Test}{\string\\end\string\{#1\string\}}%
    }}

%  Define environments that ignore their contents.
\excludecomment{comment}
\excludecomment{rawhtml}
\excludecomment{htmlonly}

%  Hypertext commands etc. This is a condensed version of the html.sty
%  file supplied with LaTeX2HTML by: Nikos Drakos <nikos@cbl.leeds.ac.uk> &
%  Jelle van Zeijl <jvzeijl@isou17.estec.esa.nl>. The LaTeX2HTML documentation
%  should be consulted about all commands (and the environments defined above)
%  except \xref and \xlabel which are Starlink specific.

\newcommand{\htmladdnormallinkfoot}[2]{#1\footnote{#2}}
\newcommand{\htmladdnormallink}[2]{#1}
\newcommand{\htmladdimg}[1]{}
\newenvironment{latexonly}{}{}
\newcommand{\hyperref}[4]{#2\ref{#4}#3}
\newcommand{\htmlref}[2]{#1}
\newcommand{\htmlimage}[1]{}
\newcommand{\htmladdtonavigation}[1]{}

% Define commands for HTML-only or LaTeX-only text.
\newcommand{\html}[1]{}
\newcommand{\latex}[1]{#1}

% Use latex2html 98.2.
\newcommand{\latexhtml}[2]{#1}

%  Starlink cross-references and labels.
\newcommand{\xref}[3]{#1}
\newcommand{\xlabel}[1]{}

%  LaTeX2HTML symbol.
\newcommand{\latextohtml}{{\bf LaTeX}{2}{\tt{HTML}}}

%  Define command to re-centre underscore for Latex and leave as normal
%  for HTML (severe problems with \_ in tabbing environments and \_\_
%  generally otherwise).
\newcommand{\setunderscore}{\renewcommand{\_}{{\tt\symbol{95}}}}
\latex{\setunderscore}

% -----------------------------------------------------------------------------
%  Debugging.
%  =========
%  Remove % on the following to debug links in the HTML version using Latex.

% \newcommand{\hotlink}[2]{\fbox{\begin{tabular}[t]{@{}c@{}}#1\\\hline{\footnotesize #2}\end{tabular}}}
% \renewcommand{\htmladdnormallinkfoot}[2]{\hotlink{#1}{#2}}
% \renewcommand{\htmladdnormallink}[2]{\hotlink{#1}{#2}}
% \renewcommand{\hyperref}[4]{\hotlink{#1}{\S\ref{#4}}}
% \renewcommand{\htmlref}[2]{\hotlink{#1}{\S\ref{#2}}}
% \renewcommand{\xref}[3]{\hotlink{#1}{#2 -- #3}}
%end{latexonly}

% -----------------------------------------------------------------------------

\newcommand{\numcir}[1]{\mbox{\hspace{3ex}$\bigcirc$\hspace{-1.7ex}{\small #1}}}
\begin{htmlonly}
   \newcommand{\numcir}[1]{~~~~~~ \htmlref{{#1}}{se:expnote{#1}}} 
\end{htmlonly}

\newcommand{\lsk}{\raisebox{-0.4ex}{\rm *}}

% -----------------------------------------------------------------------------

%  Title Page.
%  ===========
\renewcommand{\thepage}{\roman{page}}
\begin{document}
\thispagestyle{empty}

%  Latex document header.
%  ======================
\begin{latexonly}
   SCIENCE \& ENGINEERING RESEARCH COUNCIL \hfill \stardocname\\
   RUTHERFORD APPLETON LABORATORY\\
   {\large Starlink Project\\}
   {\large \stardoccategory\ \stardocnumber}
   \begin{flushright}
   \stardocauthors\\
   \stardocdate
   \end{flushright}
   \vspace{-4mm}
   \rule{\textwidth}{0.5mm}
   \vspace{5mm}
   \begin{center}
   {\Large\bf \stardoctitle}
   \end{center}
   \vspace{5mm}

% ? Heading for abstract if used.
%   \vspace{10mm}
%   \begin{center}
%      {\Large\bf Abstract}
%   \end{center}
% ? End of heading for abstract.
\end{latexonly}

%  HTML documentation header.
%  ==========================
\begin{htmlonly}
   \xlabel{}
   \begin{rawhtml} <H1> \end{rawhtml}
      \stardoctitle
   \begin{rawhtml} </H1> \end{rawhtml}

% ? Add picture here if required.
% ? End of picture

   \begin{rawhtml} <P> <I> \end{rawhtml}
   \stardoccategory\ \stardocnumber \\
   \stardocauthors \\
   \stardocdate
   \begin{rawhtml} </I> </P> <H3> \end{rawhtml}
      SCIENCE \& ENGINEERING RESEARCH COUNCIL /
      \htmladdnormallink{Rutherford Appleton Laboratory}
                        {http://www.cclrc.ac.uk/ral} \\
   \begin{rawhtml} </H3> <H2> \end{rawhtml}
      \htmladdnormallink{Starlink Project}{http://star-www.rl.ac.uk/}
   \begin{rawhtml} </H2> \end{rawhtml}
   \htmladdnormallink{\htmladdimg{source.gif} Retrieve hardcopy}
      {http://star-www.rl.ac.uk/cgi-bin/hcserver?\stardocsource}\\

%  HTML document table of contents. 
%  ================================
%  Add table of contents header and a navigation button to return to this 
%  point in the document (this should always go before the abstract \section). 
  \label{stardoccontents}
  \begin{rawhtml} 
    <HR>
    <H2>Contents</H2>
  \end{rawhtml}
  \htmladdtonavigation{\htmlref{\htmladdimg{contents_motif.gif}}
        {stardoccontents}}

% ? New section for abstract if used.
%  \section{\xlabel{abstract}Abstract}
% ? End of new section for abstract

\end{htmlonly}

% -----------------------------------------------------------------------------
% ? Document Abstract. (if used)
%  ==================
%\stardocabstract
% ? End of document abstract
% -----------------------------------------------------------------------------
% ? Latex document Table of Contents (if used).
%  ===========================================
% \newpage
\begin{latexonly}
   \setlength{\parskip}{0mm}
   \tableofcontents
   \setlength{\parskip}{\medskipamount}
   \markright{\stardocname}
\end{latexonly}
% ? End of Latex document table of contents
% -----------------------------------------------------------------------------

\section{Introduction}
\label{se:intro}

This document should be read by programmers wishing to write
applications software for Starlink.  It describes various
standard ways of arranging data, using the Hierarchical
Data System (HDS).  The most important of these is the NDF
(Extensible $n$-Dimensional-Data Format), which is suitable for
expressing a wide variety of data which occur in $n$-dimensional
arrays---for example pictures, spectra and time series.

The early part of the document deals mainly with
concepts, while later sections give more detailed
definitions of the data structures.

Standard data structures comprise an important part of the
{\it software environment}.
Their use enables software packages to share data, thereby
reducing the number of functions which must be
duplicated and hence
the total quantity of software needed.

To assist in designing standard data formats of adequate
versatility, Starlink has implemented HDS, the
{\it Hierarchical Data System},
a flexible system for storing and retrieving data in a
structured fashion.
HDS structures consist of assemblies of
{\it data objects}
arranged in trees.  Each tree resides within a
{\it container file},
one top-level tree per container file.
Data objects have a
{\it NAME},
which identifies that particular object, and a
{\it TYPE},
which describes what sort of object it is.
A data object can either be a
{\it structure}
or a
{\it primitive}.
Primitives are such things as integers,
character strings, and floating-point numbers, and express the
data themselves.  Both structures and primitives can
either be single items or multi-dimensional arrays.  The overall
philosophy is analogous to a computer's file system, where
the files themselves (the equivalent of HDS primitives)
are embedded within a hierarchy of directories (the equivalent
of HDS structures).

Most people find the ``tree'' picture the most natural when
visualising HDS objects.  However, on some occasions it can
be useful to regard an HDS object as a box (or $n$-dimensional
array of boxes), marked with a NAME (unique at that level)
and a TYPE (giving a clue to what might lie within).  When
the box is opened, further boxes are found.  Eventually,
you get to the items themselves (the primitives);  all the
rest is packaging.

Readers of the present document
should be familiar with the HDS Programmer's Guide, \xref{SUN/92}{sun92}.

In conjunction with a release of
the Starlink Software Environment ({\small SSE}---a forerunner of the
{\small ADAM} system),
Wright \& Giddings (1983) proposed a number of standard data structures
using HDS.  The discussions which followed raised many
new difficulties, and the proposals were never
ratified.  However, in the absence of anything more definitive,
the proposed standard structures were used by several
groups of implementors.  During 1987 the discussion was re-opened
following the selection of {\small ADAM} as the Starlink environment.  The
electronic correspondence on this topic runs into several hundred
messages filling three bulky
folders.  It has not been easy to produce a design which
will please everyone, for reasons that will be dealt with
in the next section.

\section{\xlabel{se_overview}Overview\label{se:overview}}

It has taken a long time and a large amount of effort
to agree the design which is described here.
The solutions to many apparently
straightforward problems came only after
protracted discussion and
thought.  Many superficially attractive ideas proved on careful
examination to be half-baked and unfortunately had to be dropped.
Some of the arguments are subtle, and it must be acknowledged that
not all contributors are fully convinced that the rejection
of their ideas was justified.  This is inevitable given the
size and diversity of the Starlink community.

\paragraph{Design Fundamentals}
The most fundamental issue (certainly the
most fruitful source of controversy) is whether we are
trying to design formats which are so comprehensive that
any piece of information, however specific to an instrument or
processing phase, has a defined location ready to receive
it, or whether we are instead trying to design
the simplest possible system which can do the job.

The first approach, sometimes called the ``we've thought of
everything'' philosophy or {\it TOE}, is the one that has been
traditionally employed.  The designers of such systems
have tried to predict everything that might be needed (in their
experience as optical spectroscopists, aperture synthesists,
X-ray observers, {\it etc.}) for the general case of a picture,
spectrum, time series, or whatever.  These designs generally
work well for their inventor, but when others try to use them
they find omissions, inconsistencies, ambiguities and
limitations, and either have to add new items of their
own or use existing items in a non-standard way.  Even where
a new form of data is expressed in what appears to be the
standard way, experience has shown that precise interpretation
by different application programs
of all the various ancillary items ({\it e.g.}\ exposure time,
astrometric parameters, {\it etc.}) cannot be relied on, and so
these items become little more than comments.

The Starlink designs reject the TOE
approach in favour of one where:
\begin{itemize}
\item The structures are simple.
\item The processing rules for all components are defined.
\item There is orderly treatment of extensions.
\end{itemize}

The third point---the treatment of extensions---is
crucial.  Most astronomer/programmers feel drawn to the
familiar TOE approach, where
there is a place to put the $\alpha,\delta$, exposure time,
polarimeter setting,
relative humidity, feed-horn
collimation parameters, {\it etc.}, and are unhappy that
many of the items they wish to include have to be
``demoted'' by being
moved into an extension.  Alternatively, they are
willing to accept the need for extensions, but only for
the idiosyncratic data required
by other astronomers.  It is
important to understand that the extensions in the
Starlink standard formats are an essential part of
the scheme, safe havens where important but specialised
items can reside, accessible to programs which
understand them, and automatically copied from
generation to generation.  All extensions should be registered
with the Starlink Head of Applications to avoid
clashes between different groups of applications.  Certain
general-purpose extensions will be highly standardised, and
will be used by many application packages.

The combination of (i) trying to keep the formats simple and
(ii) defining precisely how the different items should be
interpreted by application programs has produced a result
which has remarkably little evidence of astronomy in it.  This
should not be regarded as a worry;  the astronomical
information, relating to astrometry, radiometry, timing,
and so on, will reside in standard extensions which will be defined
in due course.  A byproduct of this conservatism (which
came largely from the need to reduce the task to a
manageable size) is that the standard structures, and
the general-purpose applications which process them, may have uses
outside astronomy.

\paragraph{The Extensible n-Dimensional-Data Format}
The Starlink standard data structures can be divided into two 
categories:
{\it low-level,} and {\it composite}.  Low-level structures are
self-contained; coupled with
individual data objects they are used to build the composite
formats, and include axis information, title, history, and quality.
The composite formats are akin to the Interim
Environment's Bulk Data Frame (see \xref{SUN/4}{sun4}).

Since the idea of some completely general data format has
been rejected (for reasons already presented), even for
astronomical data, a number of composite data formats will be
defined for various classes or forms of data as required. 

The only current example of a composite data format, the NDF, is
presented in Table~\ref{ta:example1}.  (NDF is short for {\it
Extensible n-Dimensional-Data Format} and will be described fully
\latex{in Section~\ref{se:ndf}.)}\html{\htmlref{here.)}{se:ndf}}  The
NDF is based on the $n$-dimensional data array, which is the most
natural way to express many sorts of astronomical data set---notably
spectra, pictures and time series.

Within an HDS container file, the NDF structure resides usually,
though not necessarily, at the top level.  For example, the top-level
structure within a container file might be built from several NDFs,
each an observation of the same source but through a different filter.

\begin{table}[htb]
\centering
\caption{Components of the Extensible $n$-Dimensional-Data
structure}
\label{ta:example1}
\begin{tabular}{|l|l|l|}
\hline
Component Name & TYPE & Brief Description \\ \hline
{[}{\bf VARIANT}{]} & $<${\bf \_CHAR}$>$ & variant of the $<${\bf NDF}$>$ \\
{[}{\bf TITLE}{]} & $<${\bf \_CHAR}$>$ & title of {[}{\bf DATA\_ARRAY}{]} \\
{[}{\bf DATA\_ARRAY}{]} & $<${\bf various}$>$ & NAXIS-dimensional data array \\
{[}{\bf LABEL}{]} & $<${\bf \_CHAR}$>$ & label describing the data array \\
{[}{\bf UNITS}{]} & $<${\bf \_CHAR}$>$ & units of the data array \\
{[}{\bf VARIANCE}{]} & $<${\bf s\_array}$>$ & variance of the data array \\
{[}{\bf BAD\_PIXEL}{]} & $<${\bf \_LOGICAL}$>$ & bad-pixel flag \\
{[}{\bf QUALITY}{]} & $<${\bf various}$>$ & quality of the data array \\
{[}{\bf AXIS}(NAXIS){]} & $<${\bf AXIS}$>$ & axis values, labels, units and errors \\
{[}{\bf HISTORY}{]} & $<${\bf HISTORY}$>$ & history structure \\ 
{[}{\bf MORE}{]} & $<${\bf EXT}$>$ & extension structure \\ \hline
\end{tabular}
\end{table}

(The $<$ and $>$ signs
are not actually part of the TYPE, 
nor are the brackets around the NAME.  They are
just a notation convention
to distinguish TYPE from NAME.)

There are several low-level objects within the NDF, each with a {\it
NAME} for access and identification, and a {\it TYPE} to control the
general processing.  They comprise:
\begin{itemize}
\item primitives: \begin{minipage}[t]{3in}
                  \begin{tabular}{l}
                  {[}{\bf TITLE}{]}, \\
                  {[}{\bf LABEL}{]}, \\
                  {[}{\bf UNITS}{]}, \\
                  {[}{\bf BAD\_PIXEL}{]}, \\
                  \end{tabular}\end{minipage}
\item structures: \begin{minipage}[t]{2in}
                  \begin{tabular}{l}
                  {[}{\bf HISTORY}{]}, \\
                  {[}{\bf MORE}{]}; \\
                  \end{tabular}\end{minipage}
\item arrays of structures: {[}{\bf AXIS}{]};
\item objects which can be either primitive or a structure: \begin{minipage}[t]{2in}
                  \begin{tabular}{l}
                  {[}{\bf DATA\_ARRAY}{]}, \\
                  {[}{\bf VARIANCE}{]}, \\
                  {[}{\bf QUALITY}{]}. \\
                  \end{tabular}\end{minipage}
\end{itemize}

Only the {[}{\bf DATA\_ARRAY}{]} is mandatory---so a primitive HDS
object containing a 2-D array of numbers (for example) is valid as the
only component of an NDF. A full description of the components is
given \latex{in Section~\ref{se:ndf}}\html{\htmlref{here}{se:ndf}} and
the meaning of the TYPEs in \latex{Section~\ref{se:natyvar}.}\html{\htmlref{Naming, Types and
Variants.}{se:natyvar}}

The omission from the NDF of the maximum and minimum
values of the data array, and other quantities which can be
deduced from the data, is deliberate.  This is
because experience has shown that sooner or later
applications come along which fail to keep these
numbers up to date.

In designing the NDF, efforts were made
to retain some compatibility
with the original Wright-Giddings proposals.  A limited but
useful degree of compatibility was achieved---the former
location of the main data array ({\it i.e.}\ at the top level) is
still recognised, and the name has been
retained---but it proved impossible to accommodate more
of the original proposals without enormously
adding to the complexity of the processing rules.

As was mentioned earlier, the NDF is a simple structure devoid
of any obviously astronomical items but which can be used
to express many different varieties of astronomical
data.  IPCS spectra, CCD pictures and Taurus datacubes, for example,
are all essentially $n$-dimensional arrays together with
some ancillary items, and fit naturally into the
form of the NDF, simple though it is.
In the next section, we will look at the
general question of {\bf layering}---how the
elaborate requirements of any particular data type can
be broken down into different levels of generality.  We
will then go on to consider
the topics of {\bf substructures} (how
common building blocks may be identified and exploited),
{\bf extensibility} (how items peculiar to a data type
or an application package may be accommodated), and
various processing considerations including the
{\bf propagation} of extensions.

\paragraph{Layering}
There are several well-defined levels of generality in designing
data formats, with each level building on those below it (see
Table~\ref{ta:layering}).  Having designed and implemented
HDS, the Starlink Project could have left it at that and,
beyond stipulating that applications must use HDS for their
data storage, have allowed each software package
to be autonomous.
This is the {\bf HDS} level.
The second level uses a {\bf mathematical}
representation to provide generality;  the rules for processing data
objects are mathematical abstractions, though chosen to map
well onto the processing of pictures and spectra and other
types of astronomical work.  The {\bf astronomical}
representation is much more complex than the mathematical level,
and contains information relating to astrometry, radiometry,
{\it etc.}
{\bf Specialist} structures for instrument- or application-specific
processing are still more complex.

\begin{table}[htb]
\centering
\caption{Layering of Starlink data formats.}
\label{ta:layering}
\begin{center}
\begin{tabular}{|c|}
\hline
specialist \\ \hline
astronomical \\ \hline
mathematical \\ \hline
HDS \\ \hline
\end{tabular}
\end{center}
\end{table}
\medskip

The HDS-level approach was rejected because it fails to ensure the
required degree of compatibility between application packages.
The astronomy level, on the other hand, could not
have been defined properly in the time available, and
the highest level will, of course,
only be supplied by the authors of specialist
packages. Therefore, Starlink initially selected the mathematical
representation. As well as offering the
possibility of a useful standard in a reasonable time (it will
be much more capable than INTERIM's BDF, which has nonetheless
proved extremely successful and versatile),
concentrating on the mathematical level
makes it easier to identify a subset of common
low-level data objects.  Because the processing rules are
well defined, software to handle
these low-level objects is relatively easy to write,
and once written can be used by many different packages. 

{\small KAPPA} ({\bf K}ernel {\bf Ap}plication {\bf Pa}ckage ---
see \xref{SUN/95}{sun95}) is a set of
applications which processes these abstracted data
objects.  {\small KAPPA} is
intended to be a quick, exploratory data-analysis tool, and its
applications will act as paradigms
or templates for specialist software packages.

Another important package is {\small FIGARO} (\xref{SUN/86}{sun86}), which
has been influential in defining the Starlink
standard data formats and is at present undergoing a refit
to bring it fully into line.
{\small FIGARO} is a large and mature set of picture processing
and spectral
applications, and though not as formally
correct as {\small KAPPA} will be the
dominant general purpose {\small ADAM} application package for some
time to come and will have a profound influence on the
design of other packages.

Standard software interfaces will be written to access the low-level
structures.  They are currently being designed
and will be described
in a future version of this document.

\paragraph{Substructures}
The essence of HDS is the ability to define
multiple levels rather than having everything at the
top level in a ``flat'' design.
Substructures make it easy to copy or delete parts of a structure, and
provide flexibility and extensibility.  Some implementors
mistrust multiple levels of hierarchy, and have advocated
the use of flat arrangements, combined with elaborate
naming schemes (with wildcarding) to distinguish between different
classes of object.  However, this approach has been
discredited as inefficient and arcane, and has not been used
in the Starlink standard formats.  It is also deprecated
within NDF extensions.

Low-level structures should be kept as small and simple as possible.
They should contain related objects, whose meanings are defined and
whose processing rules are specified.  If a
structure is created
which contains unrelated objects all lumped together, it
will be unwieldy, the
processing rules will be restrictive or highly complex, and it will
be a difficult object for programmers to manipulate.  Furthermore,
modular substructures can be used in different data formats without
duplication of software. The conventions for the interpretation of a
structure, as well as its components,
must be defined if there is to be compatability between
different software packages. 

On the other hand, over-elaborate structuring (structure which would
demand more work of the applications programmer and reduce the
efficiency of applications) should be avoided.  For example, it
would probably be inefficient to represent a
celestial position with separate components for hours, minutes, seconds
and degrees, arcminutes, arcseconds, rather than expressing
the two angles as floating-point numbers or using
a single character string (notwithstanding the example in 
\xref{SSN/27}{ssn27}, which is a demonstration rather than a recommendation!). 

\begin{table}[htb]
\centering
\caption{Components of $<${\bf ARRAY}$>$ Structure,
{[}{\bf VARIANT}{]} = {\tt `SIMPLE'}}
\label{ta:example2}
\begin{tabular}{|l|l|l|}
\hline
Component Name  & TYPE & Brief Description \\ \hline
{[}{\bf VARIANT}{]} & $<${\bf \_CHAR}$>$ & {\tt `SIMPLE'} \\
{[}{\bf ORIGIN}(NAXIS){]} & $<${\bf integer}$>$ & origin of the data array \\
{[}{\bf DATA}{]} & $<${\bf narray}$>$ & actual value at every pixel \\ \hline
\end{tabular}
\end{table}

\begin{table}[htb]
\centering
\caption{Components of $<${\bf ARRAY}$>$ Structure,
{[}{\bf VARIANT}{]} = {\tt `SCALED'}}
\label{ta:example3}
\begin{tabular}{|l|l|l|}
\hline
Component Name & TYPE & Brief Description \\ \hline
{[}{\bf VARIANT}{]} & $<${\bf \_CHAR}$>$ & {\tt `SCALED'} \\
{[}{\bf ORIGIN}(NAXIS){]} & $<${\bf integer}$>$ & origin of the data array \\
{[}{\bf DATA}{]} & $<${\bf narray}$>$ & scaled numeric value at every pixel \\
{[}{\bf SCALE}{]} & $<${\bf numeric}$>$ & scale factor \\
{[}{\bf ZERO}{]} & $<${\bf numeric}$>$ & zero point \\ \hline
\end{tabular}
\end{table}

Example structures are shown in
Tables~\ref{ta:example2}--\ref{ta:example4}.
The first two are {\it variants} of the $<${\bf ARRAY}$>$
structure, and are different ways of storing an $n$-dimensional array of
numbers.  A variant qualifies the TYPE and the processing rules of the
structure, and may appear in any structure.  The most basic form is
specified by
{[}{\bf VARIANT}{]} = {\tt `SIMPLE'}, which is the default should
{[}{\bf VARIANT}{]} not be
present. In this case $<${\bf ARRAY}$>$ comprises an
array of numbers plus the origin in each
dimension. {[}{\bf ORIGIN}{]} defines the zero point of the
pixel coordinate system. For {[}{\bf VARIANT}{]} = {\tt `SCALED'},
the array of numbers has been
scaled, perhaps as 16-bit integers to save disk space for large-format
data.  {[}{\bf SCALE}{]} and {[}{\bf ZERO}{]} are used to generate the
actual array values. 
In almost all cases, standard subroutines will deal automatically
with the different variants and
simply present the application with a locator to an array of numbers. 

\begin{table}[htb]
\centering
\caption{Contents of the $<${\bf HISTORY}$>$ Structure}
\label{ta:example4}
\begin{tabular}{|l|l|l|}
\hline
Component Name & TYPE & Brief Description \\ \hline
{[}{\bf CREATED}{]} & $<${\bf \_CHAR}$>$ & creation date and time \\
{[}{\bf EXTEND\_SIZE}{]} & $<${\bf \_INTEGER}$>$ & increment number of records \\
{[}{\bf CURRENT\_RECORD}{]} & $<${\bf \_INTEGER}$>$ & record number being used \\
{[}{\bf RECORDS}($m$){]} & $<${\bf HIST\_REC}$>$ & array of $m$ history records \\ \hline
\end{tabular}
\end{table}

All the objects in the first two examples have primitive TYPEs.  This
need not be the case.  The $<${\bf HISTORY}$>$ structure contains a further
structure {[}{\bf RECORDS}{]} to store history text and time tag. History
records are brief and are only intended to assist the user.  Their
contents must {\bf not} be used by applications to
control processing.

The rules for designing new structures are presented 
in \latex{Section~\ref{se:newdesign}.}\html{\htmlref{Creating New
Structures.}{se:newdesign}}

\paragraph{Extensibility}
No data format design will ever satisfy all requirements, and
some provision has to be made for accommodating ancillary
information, specific to a group of applications or a
particular instrument for example.  What should an application
do if it encounters objects which are not part of a
standard structure?

In most of the {\small ASPIC} applications (see \xref{SG/1}{sg1}) such objects ---
which have to be expressed in simple ways akin to {\small FITS}
descriptors ---
are simply ignored, and consequently do not appear in any new data
frames created by the applications.

A hierarchical data system like HDS enables all these specialist data
to be expressed in natural ways and to be attached
to the main data structure at appropriate places within the structure.
During
processing these extension substructures can be copied to
output data structures.
Of course, as a result of the computations there may
then be inconsistencies
between the specialist and the general data objects;  this is
unavoidable.  Rules will have to be laid down about what applications
can be run on what types of data and in what order, and sometimes
it may be necessary to resort to
utilities which edit HDS structures to fix up inconsistencies.
Specialist packages could implement all this transparently by
providing command procedures.

In the NDF, extensibility is provided through the [{\bf MORE}{]}
structure.  Information required by application packages (and
under the complete control of those packages) is arranged
into structures usually of TYPE $<${\bf EXT}$>$ and stored within
[{\bf MORE}{]}.
In order to allow
applications to recognise their own extension objects
without risk of clashes, the
names and types of the extension structures
must be registered with the Head of Applications.  An
example of an extension might be one called ASTROMETRY, which
would contain information about the relationship between
the data and the celestial reference frame.

[{\bf MORE}{]} structures can occur once
at the top level of the NDF, and
once in each [{\bf AXIS}{]} element.

As well as processing the extensions they recognise, applications
are obliged to propagate all other extensions to any output
structures.

Details of the defined extensions are given in other documents.

\paragraph{Propagation}
Although the data objects in the NDF are general, not all applications
will know how to process all the objects.  For example, the {[}{\bf VARIANCE}{]}
in the NDF becomes meaningless after thresholding.
Therefore, the rule for propagation,
is as follows.
\begin{itemize}
\item
Applications always propagate extensions.
\item
An application may only process and
propagate a data object whose processing rules it understands
and can fully implement, in order to avoid
rendering the object meaningless or wrong. Unknown or poorly
understood objects must be ignored---never fudged.
\end{itemize}

\paragraph{Tree walking}
``Tree walking'', {\it i.e.}\  by one means or another moving to a higher
position within an structure and processing data objects there, is
forbidden. If such objects are required their names must be acquired
from outside the application, and a new locator obtained. 

\paragraph{Instrumental data}
It will often be most convenient and efficient to delay conversion of
instrumental
data into standard formats until calibration,
reformatting and other preprocessing operations have been
completed.  The software for performing this
conversion is best provided by the
instrument builders.
However, in some instances it will prove
convenient to use a composite structure ({\it e.g.}\
an NDF) for uncalibrated data so that
general-purpose applications can be used for
for display and other quick-look operations.

\paragraph{Bad data}
Two methods are available for dealing with bad data.  The
{\bf magic value} method uses special values
stored within the data themselves to
indicate that a datum is undefined or invalid.  The second method is
{\bf quality}.  Each data value may have associated with
it a data-quality value, an 8-bit
code which flags whether the data value is special
in some way or combination of ways.

The NDF has a {[}{\bf BAD\_PIXEL}{]} flag, which allows applications
to find out in advance if there are any
magic-value data within the {[}{\bf DATA\_ARRAY}{]}.  If not, a
version of the algorithm which does not do the checks can be
invoked, in order to save time.  Note that in this document the term
{\em pixel} is used in its generic sense, {\it i.e.}\ equivalent to
an {\em array element} or {\em bin}, and therefore, {\em pixel}
does not imply membership of a two-dimensional array.

Full details of bad-pixel methods are presented \latex{in 
Section~\ref{se:badpixel}.}\html{\htmlref{here.}{se:badpixel}}

\section{\xlabel{se_natyvar}Naming, Types and
Variants\label{se:natyvar}}

\subsection{\xlabel{se_charset}Character Set\label{se:charset}}

NAMEs, TYPEs and the contents of strings in HDS
{\bf must} consist only of
printable ASCII characters ({\it i.e.}\ hexadecimal
20 through 7E).  Beyond this, HDS itself
(see \xref{SUN/92}{sun92}) imposes restrictions on NAMEs and TYPES,
to which we add the
requirement that the first character of a
(non-primitive) TYPE must {\bf not} be
underscore.

It is {\bf strongly recommended} by Starlink that NAMEs and TYPEs
be limited to letters, numbers, and the underscore character, and
that the first character be a letter.  This is to
prevent inconvenience
to users, who will find themselves having to resort
to special syntax (extra quote marks, for example) in order to
resolve
command-line ambiguity where unconventional NAMEs or TYPEs are
present.  Worse, the possibility cannot be ruled out that
unexpected interactions
between command-language features and imaginative NAMEs or TYPEs
will, at some stage, cause insurmountable difficulties.

The names of HDS container files should be chosen to be the
same as the names of their top-level data structures, or at
least to be closely related.  Note that filenames must {\bf not}
appear in applications code (see \xref{SGP/16}{sgp16}).

\subsection{\xlabel{se_roleoftype}The R\^{o}le of TYPE\label{se:roleoftype}}

TYPE describes the form of a data structure and the general rules
for interpreting or processing all structures of that form.  The NAME
identifies the particular object.  The TYPE tells the program what to 
expect and how to interpret the object.

It is possible to take the view that the TYPE attribute
of HDS objects is important only for
primitives ({\it e.g.}\  $<${\bf \_REAL}$>$) and is not needed for
higher-level structures.
Such structures would thus have a NAME but no TYPE, and applications
would interpret them by the presence or absence of NAMEd components, as
well as by data values within the structure.
Note, however, that the concept of type will be needed even if the
TYPE facility is not itself used.
TYPE is already available for this purpose and its use eliminates
the need to define naming schemes or other conventions, which could
differ between packages and ultimately cause clashes.  It is supported
directly by the HDS subroutines and is likely to be more efficient than
other methods.  It assists both the user (who will see it in a standard
place when structures are listed) and the programmer (who can expect a
structure of registered TYPE to conform to a given set of rules,
and to have associated interface routines).  A
further objection to naming schemes comes from the limited length of HDS
names (15 characters, chosen to maintain adequate storage efficiency). 

In order to minimise the total number of TYPEs to be recognised, 
the TYPE may be supplemented by
a {\it variant}, an HDS object
of TYPE $<${\bf \_CHAR}$>$
called [{\bf VARIANT}{]}.  In such cases,
the TYPE will define the general processing rules for a given
structure, while
{[}{\bf VARIANT}{]}
will specify detailed interpretation.  The variant also enables a data
structure to be developed and modified.  Therefore, applications must
ascertain the value of {[}{\bf VARIANT}{]} in every structure to
determine whether or not it can process that version of the structure. 
Mostly, this will merely be a check of the existence of {[}{\bf VARIANT}{]}.

To avoid incompatibilities between programs, it will be necessary to
register both TYPEs and variants with the Starlink Head of Applications. 

How do you decide whether a structure should have
a new TYPE or be a variant of a TYPE?  The guideline is as
follows.  {[}{\bf VARIANT}{]} should be used
if an algorithm written to process an existing TYPE could also process
the new structure, merely by interposing an automatic conversion routine
which requires no additional external information.
If {[}{\bf VARIANT}{]} is not present within a structure,
or it has the value {\tt `SIMPLE'},
then the structure is in its most basic and readily
comprehensible form.

Any general utility program, ({\it e.g.}\  one from the
{\small KAPPA} package) can use TYPE
when available, but must also be prepared
to rummage within a data structure
and identify the components it needs by other methods, if the particular
TYPE is not recognised.

\subsection{\xlabel{se_notation}Notation and Pseudo-types\label{se:notation}}

A component of an HDS structure is characterised
by its dimension(s), NAME, TYPE and meaning.  The
notation adopted in this document to describe these
attributes is as follows.

The {\bf dimensions} are given in a FORTRAN-like way.
For example:
\begin{verse}
   {[}{\bf DATA\_ARRAY}(NAXIS1,NAXIS2){]}
\end{verse}
is an NAXIS1 $\times$ NAXIS2 array with name
{[}{\bf DATA\_ARRAY}{]}.
The square brackets are a notation convention
in this document to indicate the
NAME of a data object. They are not part of the NAME.
As in this example, use is made of symbolic constants ({\it e.g.}\  NAXIS1)
in specifying array dimensions.
The meaning of these constants are described for each structure.

A {\bf NAME} in capitals is the NAME the component must take; where the NAME
is in lowercase it is generic, {\it i.e.}\  it may be chosen by the
programmer provided it does not duplicate a Starlink-reserved NAME.
NAME is not globally significant---its meaning is only defined within
the context of a containing structure.

{\bf TYPE} may be one of the {\it primitives} or one
allotted to a {\it structure},
denoted $<${\bf \_TYPE}$>$ and
$<${\bf TYPE}$>$ respectively. As mentioned earlier, the $>$ and
$<$ signs form a notation convention used in the text of this
document to signify the TYPE of a data object, and they are not
themselves part of the TYPE. Details of
HDS primitive TYPEs can be found in \xref{SUN/92}{sun92}.

Within Starlink standard data structures TYPE is
recognised globally---a given TYPE always means the same thing no 
matter where it appears in a structure.

In order to allow for
flexibility in the way in which data are represented, certain structure
components are allowed to take one of a number of options. This is
indicated in this document by ``pseudo-types'',
which are denoted by $<${\bf type}$>$;
the standard pseudo-types are listed in Table~\ref{ta:pseudotypes}.

\begin{table}[htb]
\centering
\caption{Standard Pseudo-types}
\label{ta:pseudotypes}
\begin{tabular}[t]{|l|l|}
\hline
Pseudo-type & Allowed Types \\ \hline
$<${\bf various}$>$  & structured or primitive (to be specified) \\
$<${\bf numeric}$>$  & one of the primitive numeric TYPEs \\
$<${\bf integer}$>$  & one of the primitive integer TYPEs \\
$<${\bf float}$>$    & one of the primitive floating-point TYPEs \\
$<${\bf narray}$>$   & a $<${\bf numeric}$>$ array \\
$<${\bf farray}$>$   & a $<${\bf float}$>$ array \\
$<${\bf iarray}$>$   & a $<${\bf integer}$>$ array \\
$<${\bf c\_array}$>$ & an array of complex numbers or an $<${\bf ARRAY}$>$ TYPE \\
$<${\bf p\_array}$>$ & a $<${\bf narray}$>$ or an $<${\bf ARRAY}$>$ TYPE \\
$<${\bf s\_array}$>$ & a $<${\bf numeric}$>$ scalar or $<${\bf p\_array}$>$ \\ \hline
\end{tabular}
\end{table}

The $<${\bf various}$>$ notation is used where the options do not
neatly fit into one of the categories in Table~\ref{ta:pseudotypes}.
\latex{Sections~\ref{se:basic}}\html{\htmlref{Low-Level
Structures}{se:basic}} and \latex{\ref{se:ndf}}\html{\htmlref{The
Extensible n-dimensional Format}{se:ndf}} contain tables of the
contents of standard structures, followed by descriptions of the
individual items. The possible options for each $<${\bf various}$>$
can be found within the description of the associated data object.

There is problem for the notation.
Certain data objects are arrays. The number and sizes of each dimension
of these arrays are mostly not fixed, and so cannot be specified explicitly
in a structure's table of contents.
To overcome this problem, the array pseudo-types
$<${\bf narray}$>$, $<${\bf farray}$>$
and $<${\bf iarray}$>$ are used instead of the scalars
$<${\bf numeric}$>$, $<${\bf float}$>$ and $<${\bf integer}$>$ respectively
in cases where the number and size of an array's dimensions cannot be
specified {\it a priori}.
Thus, if a vector object has its dimensions specified in a table,
it will not have an array pseudo-type assigned to it, because the 
dimensions are constant.

The first letter of the last-three types in Table~\ref{ta:pseudotypes}
may be an {\it aide m\'{e}moire} of their meanings: {\bf c} stands for
complex, {\bf p} for primitive and {\bf s} for scalar.

The notation $<${\bf c\_array}$>$ can mean either an $<${\bf ARRAY}$>$ or a
$<${\bf COMPLEX\_ARRAY}$>$ (described below). The \mbox{$<${\bf ARRAY}$>$} TYPE
illustrates some of the terminology used in this document.
\mbox{$<${\bf ARRAY}$>$} has
POLYNOMIAL, SCALED, SPARSE and SPACED variants.
These forms are all methods for expressing an array of numbers, intended
for use where a primitive array
would not be suitable.  Whenever one of
these special forms is used, the primitive TYPE of the
equivalent array (for example \_REAL), which we will call the
{\it equivalent primitive type}, must also
be specified to allow application programs to select the
most efficient form of processing.

Some examples to clarify the meanings of the pseudo-types are presented
in Table~\ref{ta:pseudoex}.
Parameterised dimensions indicate that their number and sizes are
fixed. Numerical dimensions are arbitrary.

\begin{table}[htb]
\centering
\caption{Pseudo-type Examples}
\label{ta:pseudoex}
\begin{tabular}[h]{|l|l||l|l|}
\hline
\multicolumn{2}{|c||}{Entry in a Table of Contents} & \multicolumn{2}{c|}{Example} \\ \hline
Name & Pseudo-type & Name and dimensions & Type \\ \hline
{[}{\bf BASE}{]} & $<${\bf numeric}$>$  & {[}{\bf BASE}{]} & $<${\bf \_INTEGER}$>$ \\ \hline
{[}{\bf LIST}(NAXIS,NDATA){]} & $<${\bf integer}$>$  & {[}{\bf LIST}(2,450){]} & $<${\bf \_WORD}$>$ \\ \hline
{[}{\bf TMIN}(NAXIS){]} & $<${\bf float}$>$ & {[}{\bf TMIN}(3){]} & $<${\bf \_DOUBLE}$>$ \\ \hline
{[}{\bf DATA}{]} & $<${\bf narray}$>$ & {[}{\bf DATA}(512,100,3){]} & $<${\bf \_REAL}$>$ \\ \hline
{[}{\bf VARIANCE}{]} & $<${\bf farray}$>$ & {[}{\bf VARIANCE}(100){]} & $<${\bf \_DOUBLE}$>$ \\ \hline
{[}{\bf DATA\_ARRAY}{]} & $<${\bf iarray}$>$ & {[}{\bf DATA\_ARRAY}(512){]} & $<${\bf \_INTEGER}$>$ \\ \hline
{[}{\bf DATA\_ARRAY}{]} & $<${\bf c\_array}$>$ & {[}{\bf DATA\_ARRAY}{]} & $<${\bf ARRAY}$>$ \\
 & & {\em or} {[}{\bf DATA\_ARRAY}{]} & $<${\bf COMPLEX\_ARRAY}$>$ \\ \hline
{[}{\bf QUALITY}{]} & $<${\bf p\_array}$>$ & {[}{\bf QUALITY}(384,512){]} & $<${\bf \_UBYTE}$>$ \\
 & & {\em or} {[}{\bf QUALITY}{]} & $<${\bf ARRAY}$>$ \\ \hline
{[}{\bf WIDTH}{]} & $<${\bf s\_array}$>$ & {[}{\bf VARIANCE}{]} & $<${\bf \_REAL}$>$ \\
 & & {\em or} {[}{\bf WIDTH}(2000){]} & $<${\bf \_INTEGER}$>$ \\
 & & {\em or} {[}{\bf WIDTH}{]} & $<${\bf ARRAY}$>$ \\ \hline
\end{tabular}
\end{table}

\section{\xlabel{se_labunit}Labels and Units\label{se:labunit}}

Often, data values stored in structures will be accompanied
by textual labels describing what they are and what units they
are represented in.
Whilst this paper describes abstracted data structures, clearly the
main use of the data formats is for astronomical data.  Therefore,
it is important that the form and content of these label and
units strings conform to a definite standard,
so that they are readable and unambiguous.

One important reason for consistency is so
that those general-purpose
applications which have more than one input
data array can test for equality the units of the various associated
data objects. For utility operations, like addition and subtraction,
the application must warn the user if the units are different, and
where an output object is being generated must
drop the units altogether.
In other cases it may not be possible to proceed with
the processing at all.

A minor reason for a rather definite standard is that future
applications (but not the present ones), may
have the capability of interpreting and processing
labels and units.
For example, consider a hypothetical application which
would calibrate an array of data
by dividing into it another
array containing the calibration information.
The attributes of the two arrays, and the result, {[}{\bf FLUX}{]}
might be:

\begin{center}
\begin{tabular}{lll}
Name & Label & Units \\ \hline
{[}{\bf DATA}{]} & {\tt flux} & {\tt count/s} \\
{[}{\bf CALIB}{]} & {\tt flux-calibration} & {\tt J/(m**2*Ang*count)} \\
{[}{\bf FLUX}{]} & {\tt flux-density} & {\tt J/(m**2*Ang*s)} \\ \hline
\end{tabular}
\end{center}
\medskip

where {\tt Ang} is \AA ngstrom to avoid a clash with {\tt A} (ampere),
and parentheses
are to bracket units in the denominator; {\tt J/m**2/Ang/s},
for example, would be easy to misinterpret.
In this case (and probably all others), it would
be impossible to predict
the label to be associated with the result.
However, the units could (with some care) be determined.
Towards this goal, 
present-day
applications should conform to the following guidelines when
generating the output {[}{\bf UNITS}{]} object as follows.
\begin{itemize}
\item Fortran conventions and intrinsic functions are to be used.
\item Parentheses should be
placed around the input {[}{\bf UNITS}{]} if an intrinsic
function or arithmetic operation has been applied to the associated data.
\item If a meaningless output {[}{\bf UNITS}{]} arises, say due to addition or
subtraction of different input
{[}{\bf UNITS}{]}, no output {[}{\bf UNITS}{]} should be created.
\end{itemize}
To clarify these rules here are some examples.

\smallskip
\begin{center}
\begin{tabular}{llll}
$1^{\rm st}$ input {[}{\bf UNITS}{]} & $2^{\rm nd}$ input {[}{\bf UNITS}{]} &
Operation & Output {[}{\bf UNITS}{]} \\ \hline
{\tt 'count/s'} & & arithmetic with a constant & {\tt 'count/s'} \\
{\tt 'count/s'} & & logarithm to base 10 & {\tt 'log10(count/s)'} \\
{\tt 'count/s'} & {\tt 's'} & multiplication of data & {\tt '(count/s)*(s)'} \\
{\tt 'count/s'} & {\tt 'W'} & subtraction of data & --- \\
{\tt 'J/(m**2*Ang*s)'} & & exponentiation to power of 2 &
{\tt '2.**(J/(m**2*Ang*s))'} \\ \hline
\end{tabular}
\end{center}
\medskip
Present applications will not have the ability to interpret or parse
{[}{\bf UNITS}{]} objects.
Until standard routines appear
which do this (which will not happen soon and may never
happen), applications must not attempt to do this themselves.

The standards for saying which units are to be used for each kind
of value will probably have to be determined by a group of
interested users and implementors.
However, a start can be made by looking at the scheme proposed along
with the {\small FITS} specification (Wells \& Greisen, 1979).
The {\small FITS} scheme may be summarised as follows:
\begin{itemize}
\item Consistent with the International System of Units (SI).
\item Add ``degrees'' for angles.
\item Add Janskys (Jy) for flux density.
\end{itemize}

The first proposal---conformity with SI ---
includes the standard prefixes used to
scale values by factor multiples of 1000.  There is one problem here:
micro- is abbreviated to $\mu$, which is not a character available
within HDS strings (see 
\latex{Section~\ref{se:charset};}\html{\htmlref{Character Set;}{se:charset}}
the character {\tt u} should be
used instead.  Also, the SI units include an
$\Omega$ abbreviation; here, the full unit name,
{\tt ohm}, should be used.
Using SI units is, on the face of it, clean and simple, and the right thing 
to do.
However, at present, SI units are quite alien to many in the
astronomical community, and their adoption as a rigid Starlink standard
would probably not be acceptable.  (There is also the
complication that to distinguish the abbreviations for
seconds and siemens case-sensitive testing would be required.)

\begin{table}[htb]
\centering
\caption{International System (SI) Nomenclature}
\label{ta:siunit}
\begin{tabular}[t]{|l|l|c|}
\hline
Physical Quantity & Name of Unit & Symbol for Unit \\ \hline
length$^*$ & metre & m \\
mass$^*$ & kilogram & kg \\
time$^*$ & second & s \\
electric current$^*$ & ampere & A \\
thermodynamic$^*$ & kelvin & K \\
amount of substance$^*$ & mole & mol \\
luminous intensity$^*$ & candela & cd \\
plane angle$^+$ & radian & rad \\
solid angle$^+$ & steradian & sr \\
frequency & hertz & Hz \\
energy & joule & J \\
force & newton & N \\
pressure & pascal & Pa \\
power & watt & W \\
electric charge & coulomb & C \\
electric potential & volt & V \\
electric resistance & ohm & $\Omega$ \\
electric conductance & siemens & S \\
electric capacitance & farad & F \\
magnetic flux & weber & Wb \\
inductance & henry & H \\
magnetic flux density & tesla & T \\
luminous flux & lumen & lm \\
illuminance & lux & lx \\
activity (of radioactive & becquerel & Bq \\
source) & & \\
absorbed dose (of & gray & Gy \\
ionising radiation) & & \\ \hline
\multicolumn{3}{l}{$^*$ SI Base Unit} \\
\multicolumn{3}{l}{$^+$ Supplementary Unit} \\
\end{tabular}
\end{table}

Some of the SI quantities are rarely, if ever, used in astronomy,
but are included in Table~\ref{ta:siunit} for completeness.

\begin{table}[htb]
\centering
\caption{The more commonly appearing quantities, their SI
units, and alternatives in current use}
\label{ta:units}
\begin{tabular}[t]{|l|c|l|}
\hline
Quantity & SI & Others in use and their abbreviations\\ \hline
length & {\tt m} & {\tt centimetre(cm), parsec(pc), astronomical unit(AU)} \\
mass & {\tt kg} & {\tt gram(g), solar mass(Mo)} \\
time & {\tt s} & {\tt minute(m), hour(h), day(d), year(yr), Julian date(JD)} \\
velocity & {\tt m/s} & {\tt km/s} \\
plane angle & {\tt rad} & {\tt degree, arcminute, arcsecond} \\
solid angle & {\tt sr} & {\tt square degree, square arcsecond} \\
wavelength & {\tt m} & {\tt Angstrom, micron, centimetre} \\
energy & {\tt J} & {\tt erg} \\
force & {\tt N} & {\tt dyne} \\
pressure & {\tt Pa} & {\tt dyne/cm**2} \\
density & {\tt kg/m**3} & {\tt g/cm**3} \\
power & {\tt W} & {\tt erg/s} \\
flux & {\tt W/m**2} & {\tt erg/(cm**2*s)} \\
luminous flux & {\tt lm} & {\tt erg/(cm**2*s*sr), magnitude(mag)} \\
spectral-flux density & {\tt Jy} & {\tt erg/(cm**2*s*keV), erg/(cm**2*s*Ang)} \\
surface brightness & {\tt Jy/sr} & {\tt magnitudes/arcsecond**2} \\
magnetic-flux density & {\tt T} & {\tt gauss} \\ \hline
\end{tabular}
\end{table}

Within the literature there is liberal misuse of the terms
{\it intensity}, {\it flux}
and {\it flux density}.  For example, ``flux'' is
frequently used where ``flux
density'' is the correct term.  Table~\ref{ta:units} shows the diversity
of units currently employed, and the duplication of abbreviations,
{\it e.g.} {\tt m} both for minute and metre.  Starlink's current
recommendation is to use SI units;  if this is unacceptable,
the names or abbreviations used must be unambiguous.

\section{\xlabel{se_errors}Errors\label{se:errors}}

``Error'' is a woolly concept;  it means different things to different
people, and is generally intimately tied to specialist data or
applications.  Even where some mathematical description is adopted,
with rigid rules describing the effects of different operations
on the error values, there is still no way of protecting the
user against invalid processing sequences and,
consequently, the generation of
incorrect and misleading error estimates.
However, there have been repeated and emphatic demands for there
to be some provision for error handling,
so that (at the very least) error bars can appear on plots.

The compromise adopted
in Starlink data structures is to allow
normal statistics to be assumed and to provide for
variances to be stored along with the data.  User-defined
structures may employ different representations of error
information.

On input to an application assume that the elements of the
{[}{\bf DATA\_ARRAY}{]} data object are independent and subject to
normal statistics, and that the contents of the {[}{\bf VARIANCE}{]}
data object are the variances of the corresponding elements of
{[}{\bf DATA\_ARRAY}{]}.
For most data, however, this will not be true, and therefore
the variance should be taken merely as a guide.  Ultimately, it will be
entirely the responsibility of the user to ensure that the result is
sensible.  For example, if two copies of the same data are added
together this will not be detected by the application, and
the variances will be wrong (by a factor of two).

The {[}{\bf VARIANCE}{]} data object will
be propagated in cases where the application can 
readily compute its processed values.  For example, it is relatively
easy to define the effect on the variances of
simple scalar and vector arithmetic operations, and so
variances will be computed and included
in output structures;  however more complicated operations,
including convolution, are not so amenable, and variances
will not be computed.  The programmer
should state for each program whether {[}{\bf VARIANCE}{]} is
processed or not, and the limitations of the variance computation.

\section{History}

For the storage of information related to the genealogy and
processing history of a data frame, the
Birmingham/Asterix design and subroutine-interface library
has been adopted.  Therefore the structure is named
{[}{\bf HISTORY}{]} and has TYPE $<${\bf HISTORY}$>$.  (See
Asterix Programmer Note 86\_008.)
A {[}{\bf HISTORY}{]} structure describes
the object in which it appears at the top level;
it never describes anything at higher levels.
This usually means that one history applies to one
main data array.

History is optional and is under the control of applications---only
they know whether they have done anything important to a file, and which
parameters are important, though the user will always
have the opportunity to add commentary via various patching
and notepad utilities.  There
should be a logical parameter in applications to enable/disable history
recording, which is normally defaulted in the interface file to
disabled.  Applications which do not modify or create structures containing
{[}{\bf HISTORY}{]} ({\it e.g.}\ 
$<${\bf NDF}$>$) do not need to update it ---
a display
application, for example, would not have to write a history
record.

History records must not be regarded by applications as
machine readable---they must not be used to specify parameters
or to control processing, even to re-enact automatically
a processing sequence.  They are present to assist the user:
``What did I do to these data?'', ``Did I flat-field this image?''.

History records should be brief.

Details are given in \latex{Section~\ref{se:shistory}.}\html{\htmlref{$<${\bf HISTORY}$>$
Structure.}{se:history}}

\section{\xlabel{se_badpixel}Bad-Pixel Methods\label{se:badpixel}}

Starlink standard data formats support two
methods of handling bad data: {\it magic value} (which
flags specified pixels as undefined)
and {\it data quality} (a more general mechanism, which may
be used to indicate any attribute of selected pixels, including
``badness'').  Magic value is simple
and efficient.  Data quality
is flexible and preserves the original data.

\subsection{\xlabel{se_quality}Quality\label{se:quality}}

To flag a data value as ``bad'', an associated {\it data-quality} 
value can be used.  This is an array of
8-bit positive integers, one per element of the data
array with which it is associated (a single value,
applying to all elements of the data array, is also
possible, but this will rarely be useful),
whose bits describe, in various ways,
attributes of the data value concerned.  The recommended
way to use data quality is to regard the 8 bits as eight
independent logical masks, one mask per attribute.

As its name implies, data-quality is a qualitative description of
the data value.  It is frequently
used to flag bad pixels,
but is also useful for ``good'' attributes,
{\it e.g.}\ which regions of a picture constitute the
sky sample.
It is not in any sense an error estimate (though groups of
bits might be used to convey some numerical meaning);
it finds application in circumstances where an error estimate is not
meaningful.  Here are some examples of how data quality might
be used:
\begin{itemize}
\item In an image where the intensities of some pixels have
been digitally truncated, there is only a lower limit to the actual
incident intensity; the upper limit is unbounded.  Data-quality
could be used to flag this condition, and application programs
could then decide whether to use the pixel value or to treat
it as missing.
\item Data quality is useful where a pixel has an accurate
intensity, but has
to be interpreted in a different way from other pixels.
The case of pixels affected
by fiducial marks ({\it e.g.}\  reseaux) is a
common example of this.  For most parts of the
processing, such pixels must be excluded.
However, in an application which locates the fiducial marks
themselves, they would clearly be crucial.
\item Where parts of a picture are vignetted,
data-quality allows these regions to be ignored when
appropriate (at the discretion of the
user, for example) without losing what information they contain.
\end{itemize}
Sometimes a simple true/false mask is not enough.  In such
cases it is possible to use combinations of bits to indicate both
the presence of the condition and to what subclass
of that condition the pixel belongs.  For example, a group of
three data quality bits could be used not only to flag
saturation but also to grade the degree of saturation, on a scale
of 1--7.

Clearly, not all values stored in the data system will have associated
data-quality; that would be unnecessary and quite wasteful of resources.
Normally, data-quality values are associated with basic observational
or measured data.

\subsection{\xlabel{se_magic}Magic or Undefined Value\label{se:magic}}

The alternative method for handling bad pixels is the so-called {\it
magic value} method,  where a pixel is assigned a special flag when it
has an undefined value---it corresponds to a dead element in a CCD
chip, for example, or is the result of division by zero.  This
terminology should not be confused with the HDS ``undefined state'',
where a data object exists, but has no value(s) assigned to it.  In this
document ``undefined'' means ``having a magic value'', unless explicitly
stated. An undefined pixel will always be bad, unless repaired in some
fashion, and so the data-quality technique is not applicable. 

The method is efficient on space: it can always be applied without 
increasing the data-storage requirement because the flag or magic value 
replaces the unwanted data value.  (For applications where it is important 
to retain pixel values, or where
there is a {\it degree} of badness, {\bf data quality} should
be used.)  The method enables an application to
discover whether a given pixel is bad as soon as it is
accessed.

Alternative techniques, based on a list of bad
pixels, would be less efficient, because
the list would have to be searched repeatedly to see whether given
pixels are bad.  Such methods would be especially inefficient
if large areas of pixels
were undefined.

Once a bad pixel has been detected, the application can take appropriate
action -- flagging the corresponding output pixel as bad, or attempting
a repair, perhaps {\it via} a choice of interpolation methods. 

The HDS undefined state must not be used to indicate bad pixels.  If an
application finds a data-object in this state, it must report an error,
so that the malfunctioning application which created the object can be
identified and corrected. The error is fatal.

\subsection{\xlabel{se_badimplement}Implementation\label{se:badimplement}}

General-purpose applications, like those in the
{\small KAPPA} package, should support
both magic-value and quality
arrays.  It will usually be best to look
only for the magic-value case
in the scientific algorithm part of the code, having
dealt with any data-quality information
in a preliminary pass which converts flagged
pixels into magic-value ones.

The groups of true/false logical flags involved in the
data quality mechanism are stored as integer values.  We
picture these integers as having conventional
binary encoding, and adopt the convention that 1$\equiv$true.
Specific VAX representations and conventions are {\bf not}
followed and are not in any way involved with the discussion.

\subsubsection{\xlabel{se_impquality}Data Quality\label{se:impquality}}

Quality is an 8-bit value associated with each datum, and is stored as
an unsigned byte.  A value of zero
({\it i.e.}\ all quality flags set 0$\equiv$false) implies a
``ordinary'' value which can be accepted at face value by
application programs.

\paragraph{General-purpose applications}
In general-purpose applications, the data-quality values
are regarded as
a set of 8 independent
masks, each of which is 1 bit deep.
Whether a given pixel is to be included in the
processing or not ({\it i.e.}\ whether it ``bad'')
is determined by comparing its quality value with a bit pattern
stored in a $<${\bf \_UBYTE}$>$ data object {[}{\bf BADBITS}{]} within
the $<${\bf QUALITY}$>$ structure.  The
following logical expression is evaluated: 
  \[ BAD  =  QUALITY \wedge BADBITS \]
where $\wedge$ is the logical AND operation.

Note that if a {[}{\bf BADBITS}{]} mask is zero ({\it i.e.}\ all
false), the corresponding data-quality mask is ignored.  This can be
used to turn off all 8 data-quality
masks and allow inspection or processing of the
pixels whatever their status.
For a single bit, the above expression has the following truth table:\\

\begin{picture}(110,25)
  \put(86,5.5){\makebox(0,0){0}}
  \put(86,10.5){\makebox(0,0){0}}
  \put(98,10.5){\makebox(0,0){0}}
  \put(98,5.5){\makebox(0,0){1}}
  \multiput(104,3)(0,5){3}{\line(-1,0){27}}
  \multiput(80,3)(12,0){3}{\line(0,1){13}}
  \put(53,5){\shortstack{{[}{\bf QUALITY}{]}\\ bit}}
  \put(78,18){{[}{\bf BADBITS}{]} bit}
  \put(86,15){\makebox(0,0){0}}
  \put(78.5,10.5){\makebox(0,0){0}}
  \put(78.5,5.5){\makebox(0,0){1}}
  \put(98,15){\makebox(0,0){1}}
\end{picture}
\medskip

and the overall logical value of BAD is the OR of the results
for all eight bits---just one of which has to be TRUE to make
the resulting pixel bad.

An example may clarify this.  Assume {[}{\bf BADBITS}{]} is 01001010
(where the bits of the binary number are written with the most
significant at the left, and are numbered from the right beginning
with zero).
For this
{[}{\bf BADBITS}{]} value, a pixel with a
{[}{\bf QUALITY}{]} value of 10100100 is interpreted
as non-bad, because bits 2, 5 and 7, which are set in the
data-quality value, are not set in {[}{\bf BADBITS}{]}.
However, a
{[}{\bf QUALITY}{]} value of 10100110 generates a bad value because
bit 1, which is set in the data-quality
value, is also set in {[}{\bf BADBITS}{]}.
If data object {[}{\bf BADBITS}{]} is not present its value is
assumed to be
to be 00000000, and general-purpose applications will
accept as ``good'' any pixel, irrespective of the
corresponding data-quality value.

The rules and conventions for the processing of data-quality
values and their associated data, taking into account the possible
presence of undefined values, are as follows.
\begin{description}
\item [Rules] ---
\begin{itemize}
\item Undefined pixels stay bad after processing.
\item Undefined pixels generated during the processing (other than
through data quality), {\it e.g.} logarithm of a negative data value,
are propagated to the output data value.
\item If processing would or might have changed the value
of a pixel, had the pixel
not been marked as bad through data quality, then it must propagate
an undefined pixel.  The input quality is propagated. In applications
where data value will not have been changed as a result of the
processing, the application is permitted either
(1) to propagate the original data
value and quality or (2) to propagate an undefined pixel. 
\end{itemize}
\item [Conventions] ---
\begin{itemize}
\item When there is more than one input data array 
({\it cf.}\ \latex{Section~\ref{se:propag}}\html{\htmlref{Propagation}{se:propag}}),
the {\it input} or {\it original} data quality is deemed to be that
associated with the {\it principal} data array.

However, in some cases it is hard to identify a
principal data array, or the principal data array does not have
quality and one or more of the others does.
Therefore, what is best depends on the
nature of the application.
For example, in the computation of the statistics of corresponding
pixels from each of a series of pictures, to produce mean or
standard-deviation arrays, it is vital to exclude {\bf all}
bad values from the calculations.  A related problem is
what the quality of output data arrays should be, and
here again programmers must make case-by-case judgements.
\item If an undefined pixel is generated during the processing, the data
quality of the output data value is nonetheless
the same as the input data quality, just as
if a good pixel had been generated.  (Although a
pixel is undefined, you may still need to know, for example, that the
pixel was a part of fiducial mark.)
\item The original data and quality values remain unchanged.
\end{itemize}

If a {[}{\bf QUALITY}{]} array is present it is
assumed that it is to be used to define bad pixels unless:
\begin{itemize}
\item there is a parameter which overrides the default;
\item the bit pattern of {[}{\bf BADBITS}{]} is 00000000, or
\item {[}{\bf BADBITS}{]} is omitted from the $<${\bf QUALITY}$>$ structure.
\end{itemize}
If {[}{\bf QUALITY}{]} is not present the magic-value method is assumed.
\end{description}

There is no one ideal way of handling data quality in general-purpose
routines.  Methods will evolve as experience with real applications and
data is gained.  The main considerations are:
\begin{itemize}
\item Disc space and virtual memory---use of full-size work arrays
may be unacceptable for large data frames.
\item Speed---checks for magic value and data quality
in pixel-by-pixel processing loops should be kept to a
minimum.  For example, the program could
first determine whether bad pixels and data quality
are present or not, and then call different
processing routines for the two cases.
\end{itemize}

\paragraph{Specialist applications}
Applications can be as sophisticated and specialised as they
like in their use of data quality,
and are at liberty to assign specific meanings to
values of data quality, {\it e.g.}\  a fiducial mark, vignetting,
saturation.
The details of
how data-quality information is encoded within the 8 bits
are specific to
each kind of data source and specialist package.  A description of how
quality will be interpreted must be given in the documentation for each
package that uses the technique.  However, it is possible to
identify some general features of data-quality processing.

Each data-quality value can be regarded as a
set of bit groups, each containing one or more bits.
The recommended approach is to use single bits, each
with an independent meaning, to form eight 1-bit deep logical
masks.  However, it is also permissible to take
several bits (which ought to be contiguous) and interpret them
as a positive integer.
Single bit fields are used to contain
a flag (1 = .TRUE., 0 = .FALSE.) for some feature ({\it e.g.}\  ``pixel
in fiducial'').
Multiple-bit fields are used to contain code numbers or degree of
quality.

It is envisaged that most manipulation of data-quality values will be
done quite transparently by those applications which know how to use them
to advantage, without the user being aware of the mechanism.
However, it is expected that there will be some cases where
users will want to manipulate data quality explicitly, and there
will be various data-quality editing applications, often
using graphics or image displays.
For example, there will be instances where the user wishes to
view a picture on a display and select which pixels are to
be temporarily flagged as ``wrong'', rather than trust some automatic
algorithm.

Since the data quality codes are stored separately from the actual data,
data-quality editing will normally be a reversible process, leaving
the data values themselves untouched.

({\it n.b.}\ The implementation of data quality is
largely unchanged from the
Wright-Giddings proposal.)

\subsubsection{\xlabel{se_impmagic}Magic Values\label{se:impmagic}}

For each of HDS's primitive TYPEs, the
magic-value method uses the values given in Table~\ref{ta:magicvalues}.
Alert readers will note that these are the
same as the bad values used by HDS.

\begin{table}[htb]
\centering
\caption{Magic values for bad pixels}
\label{ta:magicvalues}
\begin{tabular}{|l|l|l|}
\hline
Data TYPE & Value & Hexadecimal pattern \\ \hline
$<${\bf \_BYTE}$>$ & $-$128 & 80 \\
$<${\bf \_UBYTE}$>$ & \hspace{0.8em}255 & FF \\
$<${\bf \_WORD}$>$ & $-$32768 & 8000 \\
$<${\bf \_UWORD}$>$ & \hspace{0.8em}65535 & FFFF \\
$<${\bf \_INTEGER}$>$ & $-$2147483648 & 80000000 \\
$<${\bf \_REAL}$>$ & $-$1.7014117E+38 & FFFFFFFF \\
$<${\bf \_DOUBLE}$>$ & $-$1.701411834604923D+38 & FFFFFFFFFFFFFFFF \\ \hline
\end{tabular}
\end{table}

Use of ``undefined data'' flags must be restricted to three operations:
(i) setting a datum to ``undefined'', (ii) testing whether a datum is
in the undefined state, and (iii) replacing an undefined datum with a
valid value (using an assignment statement).  Arithmetic operations on
undefined data values are {\bf banned}.  Magic values are applicable to both
scalar and vector data objects.  There are some exceptions and these
are individually noted.

For efficiency, pixel values are tested inline for equality with the
magic value of the appropriate type.  However, the numerical values
given above must not be written explicitly in the code; instead,
variables called {\tt VAL\_\_BAD<T>}, where {\tt <T>} is the one or two-letter
type code ({\it e.g.}\  see \xref{SUN/7}{sun7}), should
be used.  These variables are specified
via an INCLUDE file with logical name BAD\_PAR.  Here is a trivial
example, which computes the mean of a one-dimensional REAL array.
({\it n.b.}\ Actual applications would include comments and
defences against rounding errors, excluded for brevity here.)

\goodbreak
\begin{verbatim}
      INTEGER I, N, NPIX
      PARAMETER (NPIX = 100)

      REAL DATA(NPIX), SUM, MEAN

      INCLUDE 'BAD_PAR'

      SUM = 0.0
      N = 0
      DO I = 1, NPIX
         IF ( DATA( I ) .NE. VAL__BADR ) THEN
            SUM = SUM + DATA( I )
            N = N + 1
         END IF
      END DO

      IF ( N .EQ. 0 ) THEN
         MEAN = VAL__BADR
      ELSE
         MEAN = SUM/ REAL( N )
      END
\end{verbatim}

Note that only valid pixels are counted and summed.

For reasons of efficiency of processor time and work space, and to
permit easier portability and adaptation of general-purpose subroutine
libraries, a flag called {[}{\bf BAD\_PIXEL}{]} may be
provided within a structure
to denote whether undefined pixels are present.  Only if it is present
and set to {\tt .FALSE.} will it be
permissible to bypass magic-value testing.  Thus, many packages
will support two sets of algorithmic subroutines; one which
tests magic values, and one which does not. 

\section{\xlabel{se_exten}Extensions\label{se:exten}}

The flexible and general-purpose character of the structures
described in this document stems from their simplicity.  This
was a conscious design decision, the more traditional ``we've
thought of everything'' approach having been rejected for
reasons given earlier.  By keeping the structures simple, it
is possible to be reasonably precise about the processing
rules, and this will enable users to mix applications from
different packages in their processing schemes.

However, the standard structures do not immediately cater for
package- or instrument-specific objects
and their processing, and indeed show very little evidence
of their planned astronomical r\^{o}le, which is realised
through the use of {\it extensions}.

Extension information can be stored only in specially
reserved places within
certain standard data structures (including the
NDF).  These places comprise optional structures of NAME
{[}{\bf MORE}{]}
TYPE $<${\bf EXT}$>$, which contain, in turn,
the extensions proper, usually of TYPE $<${\bf EXT}$>$.
The NAMEs and TYPEs of the extensions, self-contained assemblies of
related information peculiar to an application package or
data source, must be registered with the Starlink Head of
Application to avoid clashes.  (Which {[}{\bf MORE}{]} they
go in has also to be specified.)
Programmers should select names
which are sensible, descriptive and related to the
extension and the project concerned.
Simple generic names should be
avoided in case they are needed later for
Starlink general-purpose packages, {\it e.g.}
\mbox{{[}{\bf HEADERS}{]}}, {[}{\bf PHOTOMETRY}{]}.

Starlink will, in due course, define standard representations 
of time and celestial positions (using the rules 
of \latex{Section~\ref{se:newdesign}).}\html{\htmlref{Creating New
Structures).}{se:newdesign}}
They will be implemented as extensions, not additions to the
standard structures.  It has not yet been decided
whether applications in the {\small KAPPA} package
will process these standard extensions, or whether a
new package, of basic astronomy applications, will be written.

\section{\xlabel{se_propag}Propagation of Data Objects\label{se:propag}}

Many applications take one or more {\it input} objects and
generate one or more {\it output} objects.  In such cases,
applications have to make
decisions about what parts of the input
objects are to be propagated to the outputs, guided by the
following rules:
\begin{itemize}
\item
General-purpose applications will copy extensions 
({\it cf.} \latex{Section~\ref{se:exten}}\html{\htmlref{Extensions}{se:exten}})
to the same location within an output structure provided their
parent structure has been propagated; specialist applications are,
of course, allowed to process individual items within the extensions
they understand.
\item
Items whose properties are well understood in the context of a given
application may be processed by that application.
\item
Items whose properties are not well defined in the context of a given
application must {\bf not} be propagated by that application.  Thus, items
or structures rendered meaningless, wrong or even dubious by the
processing must not be propagated.  The {\it ad hoc} approach to
doubtful cases is strongly discouraged.
\item
``Rogue'' objects---anything outside the Starlink standard---should 
be ignored.
\item
Applications are not allowed to pollute standard structures
with rogue objects;  all application-specific items must reside
in registered extensions, within one of the {[}{\bf MORE}{]}
structures.
\end{itemize}

By stipulating that rogue objects must be ignored, we
avoid the problem of what should happen
if a component of the same name already exists in the output structure.
Also, not having to worry about rogue objects will greatly simplify the
checking and revision of application code that would follow
any revision or extension of the standard structure definitions
in the future.

In cases where applications access more than one composite data structure
to generate a single output structure, {\it e.g.} arithmetic of two data
arrays, there are additional problems about the propagation of certain
data objects.  These are mainly descriptive items {\it e.g.}
{[}{\bf HISTORY}{]}, {[}{\bf LABEL}{]}.
Should one, all or none of these data objects be propagated?
If the third rule above is followed rigorously the answer is none, but
many useful and relevant data objects would then be lost.  To assist
in deciding whether they can be
retained, the concept of a dominant or {\it principal data array} is 
introduced, whose various associated items will be the ones propagated.
By convention, the application's interface file
will be arranged so that, of the parameters specifying data
arrays, the principal array will be the first in sequence.  In many
cases, it will be necessary to advise the user which items have
been propagated and which have been lost;  if the result is
unsuitable, the application can then be re-run with the
parameters specified in a different order, or other programs can
be run to fine-tune the result---by replacing a label, for example.

The specific rules for HISTORY are as follows.

If there is a principal array, its history should be propagated with
a record appended for the latest operation, which could include the
name and title of the secondary data arrays.  If the data arrays are
equally important, it is probably best
to create a new {[}{\bf HISTORY}{]} structure,
again with the names/titles of its parents.  However, for cases
where this option will not suffice, a special
HDS editor application will be provided for grafting
in history records taken from
existing $<${\bf HISTORY}$>$ structures.

\section{\xlabel{se_basic}Low-Level Structures\label{se:basic}}

In this section, we begin by describing a number of
{\it low-level} structures, the basic building blocks which
can be used to build NDFs and other {\it composite} structures.

\subsection{\xlabel{se_polynomial}$<${\bf POLYNOMIAL}$>$ Structure\label{se:spolynomial}}

The $<${\bf POLYNOMIAL}$>$ structure is used for storing the coefficients of an
$n$-dimensional ordinary or Chebyshev polynomial.  Evaluation of
the polynomial yields a $<${\bf float}$>$ scalar result.

\begin{table}[hbt]
\centering
\caption{Contents of the $<${\bf POLYNOMIAL}$>$ Structure}
\begin{tabular}{|l|l|l|}
\hline
Component Name & TYPE & Brief Description \\ \hline
{[}{\bf VARIANT}{]} & $<${\bf \_CHAR}$>$ & registered variant \\
{[}{\bf DATA\_ARRAY}{]} & $<${\bf farray}$>$ & coefficients \\
{[}{\bf VARIANCE}{]}  & $<${\bf farray}$>$ & variance of the coefficients \\
{[}{\bf TMIN}(NAXIS){]} & $<${\bf float}$>$ & lower bound of co-ordinate range\\
{[}{\bf TMAX}(NAXIS){]} & $<${\bf float}$>$ & upper bound of co-ordinate range\\ \hline
\end{tabular}
\end{table}

\begin{description}
\item [{[}VARIANT{]}] At present, there are
two variants. The base variant
({[}{\bf VARIANT}{]} = {\tt `SIMPLE'}) is the ordinary polynomial,
while
{[}{\bf VARIANT}{]} = {\tt `CHEBYSHEV'} is used
for storing Chebyshev polynomials.
\item [{[}DATA\_ARRAY{]}]  This is mandatory.  The
interpretation depends on the value
of {[}{\bf VARIANT}{]}.  The
scalar (VALUE) which results from the
evaluation of the polynomial inherits the equivalent primitive TYPE
of the polynomial coefficients.

\begin{description}
\item [{[}{\bf VARIANT}{]} =] {\tt `SIMPLE'}
The {[}{\bf DATA\_ARRAY}{]} is the $NAXIS$-dimensional array containing
the $NAXIS$-di\-men\-sion\-al poly\-no\-mial
co\-efficients.  The function is evaluated as:
\[{\rm VALUE} = \sum {\rm DATA\underline{~~}ARRAY}(I_1,I_2,\ldots,I_{\rm NAXIS}) \prod_{j=1}^{\rm NAXIS} {\rm AXIS}_j\!^{I_j-1} \]
summed over all elements of the DATA\_ARRAY.

Here, AXIS$_n$ is the co-ordinate value along axis $n$, and
NAXIS$_n$ is the dimension of axis $n$.

\item [{[}{\bf VARIANT}{]} =] {\tt `CHEBYSHEV'}
The {[}{\bf DATA\_ARRAY}{]} contains the NAXIS-dimensional
array of Chebyshev-polynomial coefficients.  The function is evaluated as: 

\[{\rm VALUE}=\sum {\rm DATA\underline{~~}ARRAY}(I_1,I_2,\ldots,I_{\rm NAXIS})\prod_{j=1}^{\rm NAXIS} T(I_{j-1},S_j) \]
summed over all elements of the DATA\_ARRAY, where
$T(m,x)$ is the Chebyshev polynomial of order $m$ evaluated at
$x$ (in the range $-$1 to $+$1),
and
\[ S_n = ({\rm AXIS}_n - {\rm TMIN}(n))/({\rm TMAX}(n) - {\rm TMIN}(n)) \]

If the polynomial is evaluated outside this interval, the result is bad, and set
to the magic value.
\end{description}
\item [{[}VARIANCE{]}]
This optional component is used to describe the variance of the
errors associated with the polynomial coefficients given
by the {[}{\bf DATA\_ARRAY}{]} component, and
it must be an array of the same
dimension as {[}{\bf DATA\_ARRAY}{]}.
\item [NAXIS]
The dimensionality of the
{[}{\bf DATA\_ARRAY}{]} and therefore the dimensionality
of the space over which the polynomial function is defined.
\item [{[}TMIN{\rm ($n$)}{]}, {[}TMAX{\rm ($n$)}{]}]
These are the minimum and maximum values along
AXIS number $n$.  They represent the bounds of the normalised
co-ordinate range.  Only processed if
{[}{\bf VARIANT}{]} = {\tt `CHEBYSHEV'}, for which they are mandatory.
\end{description}

\subsection{\xlabel{se_array}$<${\bf ARRAY}$>$Structure\label{se:sarray}}

We will now describe the various structures for arrays of numbers 
introduced in \latex{Section~\ref{se:notation}.}\html{\htmlref{Notation 
and Pseudo-types.}{se:notation}}
An \mbox{$<${\bf ARRAY}$>$} structure has a number of variants.

\subsubsection{{[}{\bf VARIANT}{]} = `SIMPLE'}

\begin{table}[htb]
\centering
\caption{Components of $<${\bf ARRAY}$>$ Structure,
               {[}{\bf VARIANT}{]} = `SIMPLE'}
\begin{tabular}{|l|l|l|}
\hline
Component Name  & TYPE & Brief Description \\ \hline
{[}{\bf VARIANT}{]} & $<${\bf \_CHAR}$>$ & {\tt `SIMPLE'} \\
{[}{\bf ORIGIN}(NAXIS){]} & $<${\bf integer}$>$ & origin of the data array \\
{[}{\bf DATA}{]} & $<${\bf narray}$>$ & actual value at every pixel \\ \hline
\end{tabular}
\end{table}

\begin{description}
\item [{[}ORIGIN{]}]
The pixel origin in each dimension.  If this object is not present,
each origin is assumed to be 1.  Note that this
ability, though present in FORTRAN77, is
absent from HDS {\it per se}.

This information may be communicated to 
access routines via the upper and lower bounds on each array dimension, 
so that the Fortran dimensions would be 
\begin{verbatim}
    ARRAY(ORIGIN(1):DIMS(1)+ORIGIN(1)-1, ORIGIN(2):DIMS(2)+ORIGIN(2)-1, ...).
\end{verbatim}

{[}{\bf ORIGIN}{]} overcomes a number of quite serious problems,
including the old chestnut ``does the first pixel start at (1,1) or
(0,0)?'', because the origin is specified explicitly. Also negative
pixel indices become available, and therefore an application is freed
from having every array co-located at the first pixel, and it may extend
arrays in any direction. This property is particularly useful for
storing data which require negative indices for a natural
representation, such as Fourier-transform data, where zero often needs to
be in the centre of the array; or image data after (say) rotation about
any point.  Storing a pixel origin also enable applications to cut out parts
of a data array yet maintain the array's coordinate system (it can be
re-combined with the original if necessary). Since the coordinate
system is preserved, all the axis scaling is also preserved. This means,
for instance, that applications do not have to re-compute the polynomial
coefficients in an $<${\bf AXIS}$>$ structure when the array size is
changed.
\item [NAXIS]
The number of axes in the array being represented.
\item [{[}DATA{]}]
This mandatory item is the NAXIS-dimensional array of numbers.
\end{description}

\subsubsection{{[}{\bf VARIANT}{]} = `SCALED'}
A scaled \mbox{$<${\bf ARRAY}$>$} is sometimes referred
to as {\it block floating point}. 
When dealing with large-format data, disk space is a major
consideration; significant space may be saved by storing data in a
16-bit form via the block floating-point format.  Scaling may also be
used as a means of implementing global scalar changes in the array's
values without the need to change the array, or to move to a more
expensive storage type ({\it e.g.}\  to $<${\bf float}$>$ from
$<${\bf \_WORD}$>$) in order to work with very small or very large values. 

\begin{table}[htb]
\centering
\caption{Components of $<${\bf ARRAY}$>$ Structure, {[}{\bf VARIANT}{]} = {\tt `SCALED'}}
\begin{tabular}{|l|l|l|}
\hline
Component Name & TYPE & Brief Description \\ \hline
{[}{\bf VARIANT}{]} & $<${\bf \_CHAR}$>$ & {\tt `SCALED'} \\
{[}{\bf ORIGIN}(NAXIS){]} & $<${\bf integer}$>$ & origin of the data array \\
{[}{\bf DATA}{]} & $<${\bf narray}$>$ & scaled numeric value at every pixel \\
{[}{\bf SCALE}{]} & $<${\bf numeric}$>$ & scale factor \\
{[}{\bf ZERO}{]} & $<${\bf numeric}$>$ & zero point \\ \hline
\end{tabular}
\end{table}

General-purpose applications will process the scaled variant.
However, users may notice extra delays
while the array is converted to
$<${\bf narray}$>$ for processing
and rescaled for storage at the end of each application.  In such cases,
they may prefer the
alternative strategy of first running a utility application
to convert the data to primitive floating point in a new
container file, then performing their processing, and
finally running another utility to convert
the data back to the scaled-array
form.

The numeric value of a pixel is:
\begin{verbatim}
         ZERO + DATA element * SCALE.
\end{verbatim}

\begin{description}
\item [NAXIS]
The number of axes in the array being represented.
\item [{[}ORIGIN{]}]
The pixel origin in each dimension.  If this object is not present,
each origin is assumed to be 1. 
\item [{[}DATA{]}] 
The array stored in scaled (block-floating-point) format. Mandatory.
\item [{[}SCALE{]}]
A positive scale factor used to convert the element of
{[}{\bf DATA}{]} to its
unscaled $<${\bf numeric}$>$ form.  If it is not present,
or is negative, or is bad, an error condition results.
It is evaluated as follows:
\begin{verbatim}

      SCALE = (MAX - MIN)/(TMAX<T> - TMIN<T>)

\end{verbatim}
where {\tt MAX}, {\tt MIN} are the maximum and minimum valid ({\it i.e.}\  not bad)
data values respectively, and {\tt TMAX<T>}, {\tt TMIN<T>}  are
the largest and smallest valid values for the primitive type of {[}{\bf DATA}{]}, represented
by token {\tt <T>}.
\item [{[}ZERO{]}]
The zero point used to convert the element of {[}{\bf
DATA}{]} to its unscaled $<${\bf numeric}$>$ form.  If it is not
present,
or is undefined, a value of zero is assumed.  It is evaluated as follows:
\begin{verbatim}

      ZERO = TMIN<T> - SCALE * MIN

\end{verbatim}

The equivalent primitive type is the TYPE of {[}{\bf ZERO}{]} or, if
{[}{\bf ZERO}{]} is absent, the TYPE of {[}{\bf SCALE}{]}.  If both
{[}{\bf ZERO}{]} and {[}{\bf SCALE}{]} are missing the equivalent
$<${\bf numeric}$>$ array defaults to TYPE $<${\bf \_REAL}$>$. 
\end{description}

\subsubsection{{[}{\bf VARIANT}{]} = `SPACED'}
A spaced \mbox{$<${\bf ARRAY}$>$} is one where the data values
vary linearly with array index.
An example where this variant could be used is axis data, which are
often equally spaced.

General-purpose applications will handle these structures.  However,
extra time will be expended (though probably much less than for a
scaled \mbox{$<${\bf ARRAY}$>$}) whilst conversion to
and from a $<${\bf narray}$>$ is performed.
They do not contain magic-value bad pixels.

\begin{table}[htb]
\centering
\caption{Components of $<${\bf ARRAY}$>$ Structure,
{[}{\bf VARIANT}{]} = {\tt `SPACED'}}
\begin{tabular}{|l|l|l|}
\hline
Component Name & TYPE & Brief Description \\  \hline
{[}{\bf VARIANT}{]} & $<${\bf \_CHAR}$>$ & {\tt `SPACED'} \\
{[}{\bf ORIGIN}(NAXIS){]} & $<${\bf integer}$>$ & origin of the data array \\
{[}{\bf DIMENSIONS}(NAXIS){]} & $<${\bf integer}$>$ & dimensions of equivalent numeric array\\
{[}{\bf SCALE}(NAXIS){]} & $<${\bf numeric}$>$ & scale factors \\
{[}{\bf BASE}(NAXIS){]} & $<${\bf numeric}$>$ & zero points \\ \hline
\end{tabular}
\end{table}

The value of an element of the equivalent
$<${\bf numeric}$>$ array would be:
\[{\rm VALUE} = \sum_{i=1}^{\rm NAXIS} ({\rm BASE}(i) + (k_i-1) * {\rm SCALE}(i) ) \]
where $k_i$ is the element number of the $i^{\rm th}$ dimension.
\begin{description}
\item [NAXIS]
The number of axes in the array being represented.
\item [{[}ORIGIN{]}]
The pixel origin in each dimension.  If this object is not present,
each origin is assumed to be 1. 
\item [{[}{\bf SCALE}{]}]
The scale factor for each dimension used to convert the corresponding
array index of {[}{\bf DATA}{]} to its
$<${\bf numeric}$>$ form.  If it is not present, or is undefined,
a value of 1 is assumed for each dimension.
\item [{[}{\bf BASE}{]}]
The zero point for each dimension used to convert the corresponding
array index of {[}{\bf DATA}{]} to its
$<${\bf numeric}$>$ form.  If it is not present, or is undefined, a value of 0
is assumed for each dimension.
\item [{[}DIMENSIONS{]}]
The axis dimensions of the full array.  Mandatory.
The equivalent primitive type
is the TYPE of {[}{\bf BASE}{]}, or if {[}{\bf BASE}{]} is
absent, the TYPE of {[}{\bf SCALE}{]}.  If both
{[}{\bf BASE}{]} and {[}{\bf SCALE}{]} are missing, the
equivalent $<${\bf numeric}$>$ array defaults to
TYPE $<${\bf \_REAL}$>$.
\end{description}

\subsubsection{{[}{\bf VARIANT}{]} = `SPARSE'}
A sparse \mbox{$<${\bf ARRAY}$>$} structure is used for storing
the values of an array where most of the pixels are equal in value
(zero perhaps).
The structure is composed of a list of array indices for the
specified ``non-grey''
elements, and their corresponding data values.
The ``grey'' value adopted for the majority of the array elements is
also supplied.

\begin{table}[htb]
\centering
\caption{Components of $<${\bf ARRAY}$>$ Structure, {[}{\bf VARIANT}{]} = {\tt `SPARSE'}}
\begin{tabular}{|l|l|l|}
\hline
Component Name & TYPE & Brief Description \\  \hline
{[}{\bf VARIANT}{]} & $<${\bf \_CHAR}$>$ & {\tt `SPARSE'} \\
{[}{\bf ORIGIN}(NAXIS){]} & $<${\bf integer}$>$ & origin of the data array \\
{[}{\bf DATA}(NDATA){]} & $<${\bf numeric}$>$ & data values for pixel subset \\
{[}{\bf LIST}(NAXIS,NDATA){]} & $<${\bf integer}$>$ & list of pixel array indices \\
{[}{\bf DIMENSIONS}(NAXIS){]} & $<${\bf integer}$>$ &
dimension of equivalent $<${\bf numeric}$>$ array \\
{[}{\bf GREY}{]} & $<${\bf numeric}$>$ & grey value \\ \hline
\end{tabular}
\end{table}

\begin{description}
\item [NAXIS]
The number of axes in the array being represented.
\item [{[}ORIGIN{]}]
The pixel origin in each dimension.  If this object is not present,
each origin is assumed to be 1. 
\item [NDATA]
This is the number of non-grey array elements.
\item [{[}DATA{]}]
These are the data values for significant array elements.
Mandatory.
\item [{[}LIST{]}]
These are the array indices for each significant array value.
Mandatory.
\item [{[}DIMENSIONS{]}]
This array contains the axis dimensions of the full data array.
It is used to generate the equivalent $<${\bf narray}$>$.
Mandatory.
\item [{[}GREY{]}]
This is the value used to fill the array elements not
referenced in the {[}{\bf LIST}{]} component.  It should
have the same primitive
TYPE as {[}{\bf DATA}{]}.  If absent,
it is assumed to be
the bad-pixel magic value
(for {[}{\bf DATA}{]}'s primitive type) since the pixel's
value is undefined.
\end{description}

The equivalent primitive type is the TYPE of {[}{\bf DATA}{]}.

\subsubsection{{[}{\bf VARIANT}{]} = `POLYNOMIAL'}
The array is represented by an $n$-dimensional
polynomial.
(This variant does not conform to the 
guideline about variants since conversion of an array of numbers to a
polynomial requires additional information.  However, for clarity and
because the reverse operation is possible---polynomial to an array of
numbers---a variant of $<${\bf ARRAY}$>$ has been used.)

\begin{table}[htb]
\centering
\caption{Components of $<${\bf ARRAY}$>$ Structure, {[}{\bf VARIANT}{]} = {\tt `POLYNOMIAL'}}
\begin{tabular}{|l|l|l|}
\hline
Component Name & TYPE & Brief Description \\ \hline
{[}{\bf VARIANT}{]} & $<${\bf \_CHAR}$>$ & {\tt `POLYNOMIAL'} \\
{[}{\bf ORIGIN}(NAXIS){]} & $<${\bf integer}$>$ & origin of the data array \\
{[}{\bf DATA}{]} & $<${\bf POLYNOMIAL}$>$ & polynomial representing array \\
{[}{\bf DIMENSIONS}(NAXIS){]} & $<${\bf integer}$>$ & dimension of equivalent \\
& & $<${\bf numeric}$>$ array \\ \hline
\end{tabular}
\end{table}

\begin{description}
\item [NAXIS]
The number of axes in the array being represented.
\item [{[}ORIGIN{]}]
The pixel origin in each dimension.  If this object is not present,
each origin is assumed to be 1.  This avoids recomputing the 
polynomial coefficients if a subsection of a larger array is created.
\item [{[}DATA{]}]
The NAXIS-dimensional polynomial representing the array of numbers.
Mandatory.  The equivalent primitive type has the same TYPE as
the polynomial coefficients. To generate the equivalent primitive
array, the polynomial is evaluated at pixel centres, which start
at 0.5 for the first pixel given a uniform bin.
\item [{[}DIMENSIONS{]}]
This array contains the axis dimensions of the full data array.
It is used to generate the equivalent $<${\bf narray}$>$.
Mandatory.
\end{description}

The `SPACED' variant may appear merely
to be a degenerate form of the `POLYNOMIAL' variant.  However, there are reasons for
having a `SPACED' variant:
\begin{itemize}
\item the case where the equivalent primitive type is $<${\bf
integer}$>$ is common, whereas the polynomial variant only produces
$<${\bf float}$>$ results. 
\item The spaced $<${\bf ARRAY}$>$ uses the pixel indices
to generate the values, whereas a
polynomial array uses the pixel co-ordinates.  Pixel
indices are integers, starting at 1; pixel
co-ordinates are real numbers, which start at 0.5
for uniformly-sized bins.
\item The spaced $<${\bf ARRAY}$>$ format will be easier to understand
in a structure listing. 
\end{itemize}

To reiterate, general-purpose applications will be able to process
\mbox{$<${\bf ARRAY}$>$} structures, though their performance will be
degraded when the {[}{\bf VARIANT}{]} is not {\tt `SIMPLE'} because of
the extra processing required. These options do not create a significant
overhead for the normal case, the base variant.
When the {[}{\bf VARIANT}{]} is not {\tt `SIMPLE'},
users will probably be warned that their data are in a special format,
be given the option to abort the application, and be directed to
documentation for applications better suited to the data. 

\subsection{\xlabel{se_complex}$<${\bf COMPLEX\_ARRAY}$>$ Structure\label{se:scomplex}}

HDS does not provide a $<${\bf \_COMPLEX}$>$ primitive type, and
therefore if complex data are to be stored a $<${\bf COMPLEX\_ARRAY}$>$
structure, defined in Table~\ref{ta:complex} must be used.

\begin{table}[htb]
\centering
\caption{Components of $<${\bf COMPLEX\_ARRAY}$>$ Structure}
\label{ta:complex}
\begin{tabular}{|l|l|l|}
\hline
Component Name & TYPE & Brief Description \\ \hline
{[}{\bf VARIANT}{]} & $<${\bf \_CHAR}$>$ & registered variant \\
{[}{\bf REAL}{]} & $<${\bf p\_array}$>$ & real-component data values  \\
{[}{\bf IMAGINARY}{]} & $<${\bf p\_array}$>$ & imaginary-component data values \\ \hline
\end{tabular}
\end{table}

Currently, the [{[}VARIANT{]}] can only be {\tt `SIMPLE'}.

The dimensions of {[}{\bf REAL}{]} and {[}{\bf IMAGINARY}{]} must be identical. 
Both components
are mandatory.  The pixel origin will be taken from {[}{\bf REAL}{]} component's
structure, {\it i.e.}\  {[}{\bf REAL.ORIGIN}{]}.

If a pixel is undefined, the magic value should be assigned to both
corresponding elements of the real and imaginary arrays.  However,
should the application encounter the case where only
one element is flagged, the whole pixel should be regarded as undefined.

The {[}{\bf ORIGIN]} components should match, but only {[}{\bf REAL.ORIGIN}{]}
is used.

\subsection{\xlabel{se_axis}$<${\bf AXIS}$>$ Structure\label{se:saxis}}

This structure has TYPE $<${\bf AXIS}$>$ and is used to
store information describing the size and spacing of the
pixels in a multi-dimensional data array.
In use, $<${\bf AXIS}$>$ structures are arranged as elements of
a 1-dimensional array (of structures).  The number of elements
(structures) in this array is equal to the dimensionality
(number of axes) of the data array being described, so that
each $<${\bf AXIS}$>$ structure relates to a single
data dimension.  The following description applies to one
element of that multi-dimensional structure.

\begin{table}[htb]
\centering
\caption{Components of $<${\bf AXIS}$>$ structure element}
\begin{tabular}{|l|l|l|}
\hline
Component Name & TYPE & Brief Description \\ \hline
{[}{\bf VARIANT}{]} & $<${\bf \_CHAR}$>$ & registered variant \\
{[}{\bf DATA\_ARRAY}{]} & $<${\bf p\_array}$>$ & axis value at each pixel \\
{[}{\bf LABEL}{]} & $<${\bf \_CHAR}$>$ & axis label \\
{[}{\bf UNITS}{]} & $<${\bf \_CHAR}$>$ & axis units \\
{[}{\bf VARIANCE}{]} & $<${\bf s\_array}$>$ & axis variance in {[}{\bf DATA\_ARRAY}{]} \\
{[}{\bf NORMALISED}{]} & $<${\bf \_LOGICAL}$>$ & true if the data have been normalised \\
{[}{\bf WIDTH}{]} & $<${\bf s\_array}$>$ & bin width of each pixel \\
{[}{\bf MORE}{]} & $<${\bf EXT}$>$ & extension structure \\ \hline
\end{tabular}
\end{table}

\begin{description}
\item [{[}VARIANT{]}]
Currently, this can only be {\tt `SIMPLE'}.
\item [{[}{\bf DATA\_ARRAY}{]}]
This mandatory component, which
can be the only item within the $<${\bf AXIS}$>$ structure,
is used to provide co-ordinates along the axis.
Each value corresponds to the {\it centre} of a pixel.  For an
$n$-dimensional data array, with no axis calibration, and equally spaced
data, the Starlink standard is for the $n^{\rm th}$ pixel centre to be at
$(n-1)$.5 in each axis, {\it e.g.}\  for display purposes
({\it cf.}\  \xref{SSN/22}{ssn22}). However, in some applications the pixel
index is adequate and clearer.  Programmers should take care to use
the appropriate scheme.

The co-ordinates are given as
a vector with length corresponding to that of the corresponding
dimension of the data array. In practice, it will often be
the polynomial or spaced variant of TYPE \mbox{$<${\bf ARRAY}$>$}.

\item [{[}LABEL{]}]
This is a textual description of the co-ordinate type.  It may
contain the ``control'' codes that are used by graphics packages
to produce special characters---backslashes {\it etc}.
\item [{[}UNITS{]}]
This is a textual description of the co-ordinate units.
\item [{[}VARIANCE{]}]
This component is used to describe the variance of the
errors associated with co-ordinate
values given by the {[}{\bf DATA\_ARRAY}{]} component.
It must either be an array of
the same dimension as {[}{\bf DATA\_ARRAY}{]} or,
should a single variance apply to all elements of {[}{\bf 
DATA\_ARRAY}{]},
a scalar.
\item [{[}NORMALISED{]}]
If true, the data have been normalised to the pixel size
({\it i.e.}\ divided by the pixel width along this axis).
In other words if the data are 
normalised the data values reduce as the pixels are stretched and {\it 
vice versa}. Optional. If absent, it defaults to false.
\item [{[}WIDTH{]}]
The width or extent of each bin for irregularly spaced
and/or overlapping data.
If it is an array, it must have
the same dimensions as the {[}{\bf DATA\_ARRAY}{]}.
The scalar form applies to all the pixels and is intended for overlapping
pixels.  Optional.  If absent, the extent of the $n^{\rm th}$
element is assumed to be
\[ 0.5~(D_{n-1}+D_n)~~~{\rm to}~~~0.5~(D_n+D_{n+1}) , \]
where $D_n$ is the $n^{\rm th}$ centre
value given by {[}{\bf DATA\_ARRAY}{]}. The
extreme values are twice the available half widths, {\it viz.}
\[ D_n - 0.5 * {\rm WIDTH} ~~~ {\rm to} ~~~ D_n + 0.5 * {\rm WIDTH} \]
for the $n^{\rm th}$ element.
\item [{[}MORE{]}] The extension in which application-specific
axis-related information can be stored,
{\it e.g.} whether the axis is cyclic, the type of axis: spectral,
spatial, temporal {\it etc.}
\end{description}

\subsection{\xlabel{se_squality}$<${\bf QUALITY}$>$ Structure\label{se:squality}}

This structure is used for storing data-quality information.

\begin{table}[htb]
\centering
\caption{Components of $<${\bf QUALITY}$>$ structure}
\begin{tabular}{|l|l|l|}
\hline
Component Name & TYPE & Brief Description \\ \hline
{[}{\bf VARIANT}{]} & $<${\bf \_CHAR}$>$ & registered variant \\
{[}{\bf QUALITY}{]} & $<${\bf p\_array}$>$ & quality value of each pixel \\
{[}{\bf BADBITS}{]} & $<${\bf \_UBYTE}$>$ &
      mask of allowed bad bits in quality \\ \hline
\end{tabular}
\end{table}

\begin{description}
\item [{[}VARIANT{]}]
Currently can only be {\tt `SIMPLE'}.
\item [{[}QUALITY{]}]
The data-quality value(s).  If it is scalar it applies
to all the pixels.  The actual or
equivalent primitive type must be $<${\bf \_UBYTE}$>$.  Mandatory.

The r\^{o}le of data-quality and how it can be used to advantage were
described earlier.
\item [{[}BADBITS{]}]
A set of 8 one-bit flags which flag that the corresponding
data quality bits indicate bad pixels;
if omitted a default value of zero ({\it i.e.}\ data quality
is ignored in deciding whether a pixel is bad or not)
is assumed.
\end{description}

\subsection{\xlabel{se_shistory}$<${\bf HISTORY}$>$ Structure\label{se:shistory}}

This structure has NAME
{[}{\bf HISTORY}{]} and
TYPE $<${\bf HISTORY}$>$.  Its components are shown in
Table~\ref{ta:history}.

\begin{table}[htb]
\centering
\caption{Contents of the $<${\bf HISTORY}$>$ Structure}
\label{ta:history}
\begin{tabular}{|l|l|l|}
\hline
Component Name & TYPE & Brief Description \\ \hline
{[}{\bf VARIANT}{]} & $<${\bf \_CHAR}$>$ & registered variant \\
{[}{\bf CREATED}{]} & $<${\bf \_CHAR}$>$ & creation date and time \\
{[}{\bf EXTEND\_SIZE}{]} & $<${\bf \_INTEGER}$>$ & increment number of records \\
{[}{\bf CURRENT\_RECORD}{]} & $<${\bf \_INTEGER}$>$ & record number being used \\
{[}{\bf RECORDS}($m$){]} & $<${\bf HIST\_REC}$>$ & array of $m$ history records \\ \hline
\end{tabular}
\end{table}

\begin{description}
\item [{[}VARIANT{]}]
Currently, this can only be {\tt `SIMPLE'}.
\item [{[}CREATED{]}]
The date and time when the HISTORY structure was created,
derived from the computer's clock.  Mandatory.  The format for
dates and times is as following examples:
\begin{verbatim}
    1988/AUG/07 23:59:59.999
    2001-JAN-01 01:02:03.456
\end{verbatim}
\item [{[}EXTEND\_SIZE{]}]
The number of history records to be created whenever the existing
records have been filled.  The default is 5.
Initially 10 records are created.  (The item
{[}{\bf INITIAL\_SIZE}{]}
in the original ASTERIX history system is redundant and has been
removed.)
\item [{[}CURRENT\_RECORD{]}]
The number of the current history record.  Mandatory.
\item [{[}RECORDS{]}]  An array of $<${\bf HIST\_REC}$>$
structures containing the
history records.  Mandatory.
\end{description}

\begin{table}[htb]
\centering
\caption{Contents of the $<${\bf HIST\_REC}$>$ Structure}
\begin{tabular}{|l|l|l|}
\hline
Component Name & TYPE & Brief Description \\ \hline
{[}{\bf VARIANT}{]} & $<${\bf \_CHAR}$>$ & registered variant \\
{[}{\bf DATE}{]} & $<${\bf \_CHAR}$>$ & creation date and time \\
{[}{\bf COMMAND}{]} & $<${\bf \_CHAR}$>$ & application name
      and version \\
{[}{\bf TEXT}($m$){]} & $<${\bf \_CHAR}$>$ &
      $m$ lines of text \\ \hline
\end{tabular}
\end{table}

\begin{description}
\item [{[}VARIANT{]}]
Currently, this can only be {\tt `SIMPLE'}.
\item [{[}DATE{]}]
The date and time when the record was created, derived from the
computer's clock.  Mandatory.
\item [{[}COMMAND{]}]
The name and version number of the application or command
which modified or created the structure at
or below the level in the hierarchy of this
$<${\bf HISTORY}$>$ object.  Mandatory.
\item [{[}TEXT{]}]  History text, which can span an
unlimited number of records.  It may
contain commentary or details of input parameters.  Parameters should be
converted to text using environment calls.
\end{description}

Note there are currently no variants
of the $<${\bf HISTORY}$>$ and $<${\bf HIST\_REC}$>$
structures.

\xlabel{se_ndf}\section{The Extensible n-Dimensional-Data Format\label{se:ndf}}

The most-common structure for data that are not instrument specific is
what has become known as the {\it bulk-data frame}.  To avoid
confusion with the Interim Environment's BDF,
the new Starlink standard for storing bulk-data frames
is called the Extensible $n$-Dimensional-Data Format (NDF for short).
It has no specific HDS NAME because a container file may
have several $<${\bf NDF}$>$ structures at a given level.  It
has an optional
TYPE of $<${\bf NDF}$>$ that will not be tested by general-purpose
applications but is recommended to assist
recognition by human readers of structure listings.
NDFs may be structured recursively---see the 
polarimetry example below, for example.

It was not, in fact, possible to keep strictly to
the rules in \latex{Section~\ref{se:newdesign}}\html{\htmlref{Creating
New Structures}{se:newdesign}} when designing the NDF structure;
compromises were necessary in order to
allow old Asterix and Wright-Giddings-formatted data,
of which there is a great deal,
to be processed by the new general-purpose applications.
The NDF structure comprises a title, a data array and its associated objects
(a {[}{\bf DATA\_ARRAY}{]} structure in the Wright-Giddings terminology),
axis information, history and one or more registered named objects containing
application-specific components.  
Note that everything at the top level is intended to be under Starlink
control, and although general-purpose applications will (for
an initial period) tolerate
non-standard components at this level, such
{\it rogue objects} will not be processed beyond being
copied to the same place within an output structure.  This
Starlink-components-only restriction, which does
not preclude extensibility (done through the MORE objects),
simplifies the job of applications,
relieving them of the responsibility
of keeping track of arbitrary
numbers of extra objects.
It is recommended that if an
application detects the presence of a
rogue object it should
display a warning message, to alert the user to
take some action (for example to run the
appropriate format conversion utility).

\begin{table}[htb]
\centering
\caption{Components of the Extensible $n$-Dimensional-Data structure}
\begin{tabular}{|l|l|l|}
\hline
Component Name & TYPE & Brief Description \\ \hline
{[}{\bf VARIANT}{]} & $<${\bf \_CHAR}$>$ & variant of the $<${\bf NDF}$>$ type \\
{[}{\bf TITLE}{]} & $<${\bf \_CHAR}$>$ & title of $<${\bf NDF}$>$ \\
{[}{\bf DATA\_ARRAY}{]} & $<${\bf various}$>$ & NAXIS-dimensional data array \\
{[}{\bf LABEL}{]} & $<${\bf \_CHAR}$>$ & label describing the data array \\
{[}{\bf UNITS}{]} & $<${\bf \_CHAR}$>$ & units of the data array \\
{[}{\bf VARIANCE}{]} & $<${\bf s\_array}$>$ & variance of the data array \\
{[}{\bf BAD\_PIXEL}{]} & $<${\bf \_LOGICAL}$>$ & bad pixel flag \\
{[}{\bf QUALITY}{]} & $<${\bf various}$>$ & quality of the data array \\
{[}{\bf AXIS}(NAXIS){]} & $<${\bf AXIS}$>$ & axis values, labels, units and errors \\
{[}{\bf HISTORY}{]} & $<${\bf HISTORY}$>$ & history structure \\ 
{[}{\bf MORE}{]} & $<${\bf EXT}$>$  & extension structure \\ \hline
\end{tabular}
\end{table}

\begin{description}
\item [{[}VARIANT{]}]  Specifies which
sort of $<${\bf NDF}$>$ structure.  The variant
must be one
of the registered strings, of which only {\tt `SIMPLE'} is currently 
available.
\item [{[}TITLE{]}]  A title for the data which
may be used to annotate plots and
listings, and which will help identify the NDF.
(A single line of text will obviously be too
brief to describe the contents
of a dataset in detail, but will
be useful for display purposes.)
\item [{[}{\bf DATA\_ARRAY}{]}]  This is the primary
$n$-dimensional array of data values.
It is the ONLY obligatory structure element.  The
{[}{\bf DATA\_ARRAY}{]} can be 
present in one of these forms:
\begin{enumerate}
\item A $<${\bf narray}$>$.
This primitive form, {\it i.e.}\ just a $<${\bf numeric}$>$ array of numbers,
is available to give compatibility with the old Wright-Giddings proposals.
Although its
use is discouraged for new applications, it
is recommended that general-purpose applications
propagate the $<${\bf narray}$>$ format, as input,
rather than convert to an
\mbox{$<${\bf ARRAY}$>$} structure.
\item A $<${\bf c\_array}$>$.
\end{enumerate}
\item [{[}LABEL{]}]  This is a textual description for the kind of quantity
stored in the {[}{\bf DATA\_ARRAY}{]} array.
\item [{[}UNITS{]}]  This is a textual description for the units in which the
data values are given. If more than one NDF is being processed, the various
{[}{\bf UNITS}{]} text may be tested for equality.  Should they prove
unequal, the
application must inform the user, who then may have an opportunity to 
permit processing to continue;  however, {[}{\bf UNITS}{]}
would {\bf not} under these circumstances be
propagated to the output NDF if any.
\item [{[}VARIANCE{]}]  This is used to store the variance of the
errors associated with {[}{\bf DATA\_ARRAY}{]}.
It is used for computing symmetric error bars.
The array dimensions must correspond to those of the
{[}{\bf DATA\_ARRAY}{]} component.
If all values in the data array have the same error, this can
be represented by the scalar option. 
Other, more complex, forms of error representation
({\it e.g.}\  asymmetric errors)
can be stored in specialised extensions, yet to be defined.
\item [{[}BAD\_PIXEL{]}]  If this
is false, applications may assume that
{[}{\bf DATA\_ARRAY}{]} and {[}{\bf VARIANCE}{]} contain no magic-value
pixels. If it is either true or absent, applications
must either test for magic-value pixels or---if incapable
of performing bad-pixel processing---give up.
\item [{[}QUALITY{]}]  The data-quality values for the
corresponding elements of {[}{\bf DATA\_ARRAY}{]}.
Its TYPE is either $<${\bf narray}$>$ or $<${\bf QUALITY}$>$.
Note that the array can be stored
in a sparse variant \mbox{$<${\bf ARRAY}$>$} structure;
however, the dimensions of the sparse array must correspond to those of the
{[}{\bf DATA\_ARRAY}{]} component, and the actual or equivalent
primitive type for the data-quality values must be $<${\bf \_UBYTE}$>$.
The $<${\bf narray}$>$ option is to allow compatability with existing data in
Wright-Giddings format; there was no
{[}{\bf BADBITS}{]} flag in the Wright-Giddings format,
and so when such existing data are processed by
general-purpose applications
non-zero data quality will not be interpreted as
bad pixels as would have occurred formerly.
\item [NAXIS]  The dimensionality of
the {[}{\bf DATA\_ARRAY}{]}, and therefore
the number of elements (structures) in the AXIS array
(of structures).
\item [{[}AXIS{]}] This is an array of $<${\bf AXIS}$>$ structures, where
{[}{\bf AXIS}($n$){]} corresponds to the $n^{\rm th}$ dimension of the
{[}{\bf DATA\_ARRAY}{]}.
If {[}{\bf AXIS}{]} is not present the pixel index is used, starting
from the associated value(s) of {[}{\bf DATA\_ARRAY.ORIGIN}{]}, or 1 if the
origin data object does not exist. If a simple pixel index is required,
then the {[}{\bf AXIS}{]} should be omitted from the $<${\bf NDF}$>$.
\item [{[}MORE{]}]  This is a wrapper containing extensions; it is {\em 
not} itself an extension. The extensions and their components
are outside the scope of the NDF definition, and they will
be defined separately and in many cases will belong
to specific applications packages.
Each extension must have a unique NAME, by which it is recognised.
Its TYPE may be any one of the Starlink-defined standard TYPEs, or
may a new one defined according to rules in 
\latex{Section~\ref{se:newdesign}.}\html{\htmlref{Creating New
Structures.}{se:newdesign}}
Each extension (with the NAME, TYPE and variants)
must be registered with the Starlink Head of Applications.
Further NDFs may be located within these structures, and these
may in turn contain extensions.  To reduce the task of registration,
and to minimise the risk of clashes, it is strongly recommended
that one structure per application package be used rather than multiple
minor items. It is also recommended that hierarchical structuring be used
within extensions (rather than just `flat' lists of components)
so as to group related data objects, {\it e.g.}\  by
processing or instrument.  
\end{description}

\goodbreak
Notes:
\begin{enumerate}
\item Locating the data array.

General-purpose applications expecting an $<${\bf NDF}$>$ structure should 
be prepared to process
the data array of Wright-Giddings formats as well.  Also, it should not 
matter in either case whether the name of the structure containing the
data array or the name of the data array itself is supplied by the user.
However, only when the name of the $<${\bf NDF}$>$ structure is given can
other data objects in the NDF be processed, because of the no-tree-walking
rule. An outline algorithm to achieve the required functionality is:

\begin{tabbing}
Giv\=en \=nam\=e o\=f o\=bje\=ct\\
Find its type\\
{\bf if} (type {\bf not} primitive) {\bf then}\+ \\
   {\bf if} (type {\bf not} $<${\bf c\_array}$>$) {\bf then}\+ \\
      {\bf if} (type {\bf not} $<${\bf NDF}$>$){\bf then}\+ \\
         issue warning but proceed\- \\
      {\bf endif} \\
      look for {[}{\bf DATA\_ARRAY}{]} \\
      {\bf if} ({[}{\bf DATA\_ARRAY}{]} {\bf not} found) {\bf then}\+ \\
         No data processed \\
         Exit\- \\
      {\bf else}\+ \\
         Search for other required items\- \\
      {\bf endif}\- \\
   {\bf endif}\- \\
{\bf endif} \\
Process {[}{\bf DATA\_ARRAY}{]}.
\end{tabbing}

\item Accessing part of a {[}{\bf DATA\_ARRAY}{]}

Some general-purpose applications will need to be able to access subsets
of a data array.  The problem is twofold: first, the method of
implementation needs to be specified, and second, the representation of
each axis must be identified.  An example is a general image-display
routine which expects to be supplied a two-dimensional image but which is
instead given a three-dimensional
data cube.  Such an application must have a
means to select the whole or part of a
slice from the cube.  One method is simply to use
two applications one after the other: first run MANIC 
(a {\small KAPPA} application) on the input data array to create a
new dataset containing the required
data; and second, run the required processing
application on those extracted data.  However,
this means extra work for the user, and extra scratch
space requirements, and in the case of frequently-used
applications it
will be more natural to provide the necessary
`slicing' capability directly.
In these cases, applications will be able to
exploit MANIC's component subroutines,
which will first obtain the parameter values
to specify the data subset required, and
then extract the subset efficiently and store it in
internal workspace ready for processing.
Through the applications interface file, it will be
possible to set up default parameter values
tailored to the application concerned.
When the selection of axes is being made (specifying in what
direction the 2-D cut through the 3-D data cube is to be made,
for example),
the application should display to the user the
axis labels (if present) to assist identification.

\item Higher-level structures

Various specialised data objects and structures may be packaged around
the NDF structure, using the NDF as a building block.  One common
requirement is for a series of related spectra or pictures;  this
could be implemented simply as a sequence of NDFs as follows:

\begin{latexonly}
\begin{tabbing}

na\=me\=~~~~~~sp\=eci\=al\_ty\=pe\+ \\
  {[}{\bf name1}{]} \> \> $<${\bf NDF}$>$\+ \\
    $\vdots$ \> \>$\vdots$\- \\
  {[}{\bf name2}{]} \> \> $<${\bf NDF}$>$\+ \\
    $\vdots$ \> \>$\vdots$\- \\
  {[}{\bf name3}{]} \> \> $<${\bf NDF}$>$\+ \\
    $\vdots$ \> \>$\vdots$\- \\

\end{tabbing}
\end{latexonly}

\begin{htmlonly}
\begin{quote}

name~~~~~~~~~~special\_type\\
~~~{[}{\bf name1}{]}~~~~~~$<${\bf NDF}$>$ \\
~~~~~~~$\vdots$~~~~~~~~~~~~~~~~~~$\vdots$\- \\
~~~{[}{\bf name2}{]}~~~~~~$<${\bf NDF}$>$ \\
~~~~~~~$\vdots$~~~~~~~~~~~~~~~~~~$\vdots$\- \\
~~~{[}{\bf name3}{]}~~~~~~$<${\bf NDF}$>$ \\
~~~~~~~$\vdots$~~~~~~~~~~~~~~~~~~$\vdots$\- \\

\end{quote}
\end{htmlonly}

Another approach would be to use an HDS array, each
element of which is an NDF.

\item Merging two or more $<${\bf NDF}$>$ structures

The merging of history records has been discussed 
in \latex{Section~\ref{se:shistory},}\html{\htmlref{$<${\bf
HISTORY}$>$ Structure,}{se:shistory}} and the same approach
is followed for other data objects within an $<${\bf NDF}$>$.
Thus, cases are divided into (i)
those with a principal data array, where only the
components of its $<${\bf NDF}$>$ structure are
processed/copied to an output array, and
(ii) those where the
data arrays have equal importance, and
the application, by convention, assumes the first $<${\bf NDF}$>$ supplied
contains the
principal data array.  There will be an HDS editor and $<${\bf NDF}$>$
``dressing/undressing'' utilities when this is not satisfactory.  It is
suggested that a common {\small ADAM} parameter name be assigned to this `principal'
$<${\bf NDF}$>$, {\it e.g.}\  MAIN\_ARRAY.
\end{enumerate}

\xlabel{se_expolarimetry}\subsection{Polarimetry Example\label{se:expolarimetry}}

{\it Stokes parameters} are the most common method for storing and analysis
of polarimetric data.  Here is an illustrative
example of how they might be stored
using the $<${\bf NDF}$>$ structure, taking the approach that
the $I$ data is the principal data array, and is therefore
stored at the top-level of the structure.  The $Q$, $U$ and $V$-parameter data
are $<${\bf NDF}$>$ structures called, respectively,
{[}{\bf STOKES\_Q}{]}, {[}{\bf STOKES\_U}{]}
and {[}{\bf STOKES\_V}{]}, and
located within a polarimetry extension.

The obvious alternative approach would be simply to
add to the {[}{\bf DATA\_ARRAY}{]} an extra dimension so that
the different Stokes parameters could all be stored in a
single data array.  Thus, for example, the four Stokes pictures from a
512$\times$512 imaging polarimeter would be stored as different planes of a
4$\times$512$\times$512 data cube.
Though superficially more elegant than using
separate arrays for each Stokes parameter, such
an approach would introduce the danger
of invalid processing, because the Stokes
parameters are intrinsically different from each other; they cannot
be combined (for example adding a $Q$ pixel
value to its $V$ value would be meaningless)
whereas analogous arithmetic between values in the spatial
time and wavelength/energy dimensions (for example rebinning)
would, of course, be valid.

\begin{table}[htb]
\centering
\caption{Example Polarimetry extension.}
\begin{tabular}{|l|l|l|}
\hline
Component Name & TYPE & Brief Description \\ \hline
{[}{\bf STOKES\_Q}{]} & $<${\bf NDF}$>$ & Stokes $Q$ data objects \\
{[}{\bf STOKES\_U}{]} & $<${\bf NDF}$>$ & Stokes $U$ data objects \\
{[}{\bf STOKES\_V}{]} & $<${\bf NDF}$>$ & Stokes $V$ data objects \\ \hline
\end{tabular}
\end{table}

\begin{table}[htb]
\centering
\caption{A polarimetry example using the $<${\bf NDF}$>$}
\begin{tabular}{|l|l|l|}
\hline
Component Name & TYPE & Brief Description \\ \hline
{[}{\bf TITLE}{]} & $<${\bf \_CHAR}$>$ & title of $<${\bf NDF}$>$ \\
{[}{\bf DATA\_ARRAY}{]} & $<${\bf various}$>$ & Stokes $I$ data array \\
{[}{\bf LABEL}{]} & $<${\bf \_CHAR}$>$ & label describing the data array \\
{[}{\bf UNITS}{]} & $<${\bf \_CHAR}$>$ & units of the data array \\
{[}{\bf VARIANCE}{]} & $<${\bf s\_array}$>$ & variance of the data array \\
{[}{\bf QUALITY}{]} & $<${\bf various}$>$ & quality of the data array \\
{[}{\bf AXIS}(NAXIS){]} & $<${\bf AXIS}$>$ & axis values, labels, units and errors\\
& & applicable to all Stokes parameters \\
{[}{\bf HISTORY}{]} & $<${\bf HISTORY}$>$ & history structure \\
{[}{\bf MORE}{]} & $<${\bf EXT}$>$ & extension structure \\
\hspace{14mm}{[}{\bf .POLARIMETRY}{]} & $<${\bf EXT}$>$ & polarimetry extension \\ \hline
\end{tabular}
\end{table}

Usually, the different Stokes parameters will
have the same axis information and,
using the structure above, specialist polarimetry applications will
be able to exploit this fact.  However, general-purpose applications will
not be able to do so,
because of the rule on ``tree-walking''.  To obtain
other than default
axis data using a general-purpose application, say displaying
{[}{\bf STOKES\_Q.DATA\_ARRAY}{]}, the axis information
must be duplicated in the
{[}{\bf STOKES\_Q}{]} structure; alternatively,
application should have
a parameter which
specifies where the axis information is to be found.
If the default is taken, the application should look
for the $<${\bf AXIS}$>$ structure in
the normal place, {\it i.e.}\  within {[}{\bf STOKES\_Q.}{]} 

\xlabel{se_sndf}\subsection{Simplified $<${\bf NDF}$>$ Structure\label{se:sndf}}

The support software associated with the
standard data structures described above
will take a long time to develop. In the meantime, some astronomers
and programmers will want to convert their applications to {\small ADAM} and
to start using HDS data structures.  Therefore, a simple and limited
form of the $<${\bf NDF}$>$ structure is available for their use.  It will be
comprehensible to the standard interfaces once they are ready, so
that existing
applications would then require minor modification, but the data files would
not.

\begin{table}[htb]
\centering
\caption{Components of the Simple Extensible $n$-Dimensional-Data structure}
\begin{tabular}{|l|l|l|}
\hline
Component Name & TYPE & Brief Description \\ \hline
{[}{\bf TITLE}{]} & $<${\bf \_CHAR}$>$ & title of {[}{\bf DATA\_ARRAY}{]} \\
{[}{\bf DATA\_ARRAY}{]} & $<${\bf narray}$>$ & NAXIS-dimensional data array \\
{[}{\bf AXIS}(NAXIS){]} & $<${\bf AXIS}$>$ & axis values, labels, units and errors \\ \hline
\end{tabular}
\end{table}

\begin{description}
\item [{[}TITLE{]}]  As described in the full $<${\bf NDF}$>$.
\item [{[}{\bf DATA\_ARRAY}{]}]  This is the primary
$n$-dimensional array of data values.
It is the ONLY obligatory structure element.  Note it can only be an array
of numbers in the simplified $<${\bf NDF}$>$.
\item [NAXIS]  No change from the standard $<${\bf NDF}$>$.
\item [{[}AXIS{]}] Note these are simplified $<${\bf AXIS}$>$ structures.
If they are not present, the pixel coordinates are used for the
axis arrays, which have the 
same dimensions as the {[}{\bf DATA\_ARRAY}{]}
\end{description}

\begin{table}[htb]
\centering
\caption{Components of the Simplified $<${\bf AXIS}$>$ structure element}
\begin{tabular}{|l|l|l|}
\hline
Component Name & TYPE & Brief Description \\ \hline
{[}{\bf DATA\_ARRAY}{]} & $<${\bf narray}$>$ & axis value at each pixel \\
{[}{\bf LABEL}{]} & $<${\bf \_CHAR}$>$ & axis label \\
{[}{\bf UNITS}{]} & $<${\bf \_CHAR}$>$ & axis units \\ \hline
\end{tabular}
\end{table}

\begin{description}
\item [{[}{\bf DATA\_ARRAY}{]}]  Note this can only be an array of numbers. 
Mandatory.
\item [{[}LABEL{]}]  No change from the normal $<${\bf AXIS}$>$ structure.
\item [{[}UNITS{]}]  No change.
\end{description}
Applications must test pixels for magic values---there is no
{[}{\bf BAD\_PIXEL}{]} flag in the simplified $<${\bf NDF}$>$.

\xlabel{se_comparison}\section{Comparison with Wright-Giddings proposals\label{se:comparison}}

The term ``structure'' in the Wright-Giddings (WG) proposals meant
a collection of related data objects, so that one level of the hierarchy
could contain more than one ``structure'', {\it e.g.}\  GLOBAL and
{[}{\bf DATA\_ARRAY}{]}. 

Of the GLOBAL structure only {[}{\bf TITLE}{]} has been copied into the
$<${\bf NDF}$>$ structure.

\begin{table}[htb]
\centering
\caption{Comparison of $<${\bf NDF}$>$ and the Wright-Giddings ``DATA\_ARRAY 
structure''}
\begin{tabular}{|l|l|l|}
\hline
Wright-Giddings & NDF name & Comments \\ \hline
& {[}{\bf TITLE}{]} & same name in WG GLOBAL \\
& {[}{\bf VARIANT}{]} & no counterpart in WG \\
{[}{\bf DATA\_ARRAY}{]} & {[}{\bf DATA\_ARRAY}{]} & unchanged in name or meaning \\
{[}{\bf DATA\_MIN}{]} & & no counterpart in $<${\bf NDF}$>$\\
{[}{\bf DATA\_MAX}{]} & & no counterpart in $<${\bf NDF}$>$ \\
{[}{\bf DATA\_BLANK}{]} & & no counterpart in $<${\bf NDF}$>$ \\
{[}{\bf DATA\_SCALE}{]} & & {[}{\bf SCALE}{]} in scaled $<${\bf narray}$>$ \\
{[}{\bf DATA\_ZERO}{]} & & {[}{\bf ZERO}{]} in scaled $<${\bf narray}$>$ \\
{[}{\bf DATA\_LABEL}{]} & {[}{\bf LABEL}{]} & unchanged in meaning \\
{[}{\bf DATA\_UNITS}{]} & {[}{\bf UNITS}{]} & unchanged in meaning \\
{[}{\bf DATA\_QUALITY}{]} & {[}{\bf QUALITY}{]} & unchanged in meaning \\
& {[}{\bf BAD\_PIXEL}{]} & no counterpart in WG \\
{[}{\bf DATA\_ERROR}{]} & {[}{\bf VARIANCE}{]} & variance rather than $\sigma$ \\
{[}{\bf AXIS\_CALIB}{]} & & no counterpart in $<${\bf NDF}$>$ \\
{[}{\bf AXIS$<$n$>$\_DATA}{]} & {[}{\bf AXIS($<$n$>$).DATA\_ARRAY}{]} &
{[}{\bf DATA\_ARRAY}{]} for consistency \\
{[}{\bf AXIS$<$n$>$\_UNITS}{]} & {[}{\bf AXIS($<$n$>$).UNITS}{]} & unchanged in meaning \\
{[}{\bf AXIS$<$n$>$\_LABEL}{]} & {[}{\bf AXIS($<$n$>$).LABEL}{]} & unchanged in meaning \\
{[}{\bf AXIS$<$n$>$\_ERROR}{]} & {[}{\bf AXIS($<$n$>$).VARIANCE}{]} & variance rather than $\sigma$ \\
{[}{\bf AXIS$<$n$>$\_CALIB}{]} & & no counterpart in $<${\bf NDF}$>$ \\
& {[}{\bf HISTORY}{]} & no counterpart in WG \\
& {[}{\bf MORE}{]} & no counterpart though it can\\
& & store WG GLOBAL objects \\ \hline
\end{tabular}
\end{table}

WG components no longer used in $<${\bf NDF}$>$ :
\begin{description}
\item [{[}DATA\_MIN{]}, {[}DATA\_MAX{]}]  Following the precept of making
life easier for
applications programmers forces their exclusion.  Principally they are
used for scaling for display, yet they often fail to produce the desired
result, especially for data with a wide dynamic range (as is often
found in astronomy), because the extrema only give information about
two pixels and are therefore unrepresentative.  What often happens is that
the display process is repeated several times (albeit more efficiently
if {[}{\bf DATA\_MIN}{]} and {[}{\bf DATA\_MAX}{]} are stored), each time
guessing the correct range more accurately.  Even in those cases where
the maximum and minimum provide a good scaling, their computation accounts
for only about 20 per cent of the total time taken by the display
process.
A more-sophisticated approach at the outset pays off because the data
need only be displayed once.  The most successful methods are related to
the distribution of pixel values.  Histogram equalisation, central $x$
per cent, or minus $a$ to plus $b$ standard deviations all work well.

Although there might be some hope that a min-max system could work if all
algorithmic access to the datasets were through Starlink-supplied
routines, it is inevitable that some programmers/users would forget
to call the subroutines or fail to realise that their application may
have affected the extrema.
\item [{[}DATA\_BLANK{]}]  This has been superseded by
system-defined magic values,
available to applications.
\item [{[}DATA\_LABEL{]}, {[}DATA\_UNITS{]} and {[}DATA\_QUALITY{]}]  These
have changed to allow
standard subroutines to work on different labels, units and quality,
{\it e.g.}\  in $<${\bf AXIS}$>$, and not be tied into a naming scheme,
{\it e.g.}\  {[}{\bf AXIS\_LABEL}{]}.
The interpretation of {[}{\bf DATA\_QUALITY}{]} has also changed for general-purpose
applications.
\item [{[}DATA\_ERROR{]}]  This has changed as for {[}{\bf DATA\_LABEL}{]}
{\it etc.,} but also because the interpretation has changed to variance.
\item [{[}AXIS\_CALIB{]} and {[}AXIS$<$n$>$\_CALIB{]}]  These are too specialised and the
processing rules are unknown for general-purpose applications.
Calibration data should be situated in {[}{\bf MORE}{]} within an extension, perhaps
which shadows the main {[}{\bf AXIS}{]} structures.
\end{description}

\xlabel{se_newdesign}\section{Creating new structures\label{se:newdesign}}

The methods presented in this section are intended to ensure that the rules 
given in \latex{Section~\ref{se:overview}}\html{\htmlref{Overview}{se:overview}}
are adhered to. Particularly, it
makes sure that standard structures are used wherever possible, and
therefore encourages the building of new structures (when needed) by
gathering together existing structures, so ensuring the maximum
commonality.

New structures must not be constructed simply by adding new
components into standard ones (which is illegal), but instead by
adding a new layer. The
HDS hierarchy then provides a natural barrier between separate
structures, and ensures that further components can later be added at
any level without risking naming conflicts.  

\xlabel{se_newdef}\subsection{Definitions\label{se:newdef}}

In the description
of the design process the total hierarchy of structures is called a
{\it dataset} to distinguish it from a single component structure. 
It is equivalent to the contents of an HDS container file.
The term {\it structure} refers to a set of related data items; it
corresponds to a single level of a dataset.
Often a dataset will consist simply of a single structure.

\xlabel{se_newalg}\subsection{Algorithm\label{se:newalg}}

A summary of the algorithm to be used when creating a new dataset is 
given below.  The \latex{circled} numbering refers to expanded notes
in the next subsection.

\begin{latexonly}
\begin{tabbing}
xxx\=xxx\=xxx\=xxx\=xxx\=xxx\=xx\kill
Define what the software is going to do and not going to do \numcir{1}\\
Identify, in concept, the datasets required \\
Determine their interrelations, perhaps via a tree diagram \numcir{2}\\
Start at the most deeply nested level of the hierarchy\\
{\bf for} each structure\+ \\
   Identify the data components\numcir{3}\\
   {\bf if} an existing standard structure can be used {\bf then}\+ \\
      Use it \\
      Place remaining associated items in a new structure or in an extension\- \\
   {\bf else}\+ \\
      Assign a unique HDS TYPE to the new structure \numcir{4}\\
      Assign a NAME to each component \numcir{5}\\
      Determine the rules and restrictions governing the way the data will be stored in the\\
      ~~various components\\
      Assign a TYPE to each component \numcir{6}\\
      Identify the sorts of operation to be performed on the structure and ensure they are\\
      ~~meaningfully defined \numcir{7}\\
      {\bf if} the processing of a component cannot be defined in some cases {\bf then} remove the\\
      ~~component from the structure\\
      Implement and document the software needed to process the new structure \numcir{8}\\
      {\bf if} the new structure is to become a standard type {\bf then}\+ \\
         Submit it and its software to Starlink for approval\- \numcir{9}\\
      {\bf endif}\- \\
   {\bf endif}\- \\
{\bf endfor} \\
\end{tabbing}
\end{latexonly}

\begin{htmlonly}
\begin{quote}
Define what the software is going to do and not going to do \numcir{1} \\
Identify, in concept, the datasets required \\
Determine their interrelations, perhaps via a tree diagram \numcir{2}\\
Start at the most deeply nested level of the hierarchy\\
{\bf for} each structure \\
~~~~Identify the data components\numcir{3}\\
~~~~{\bf if} an existing standard structure can be used {\bf then} \\
~~~~~~~~Use it \\
~~~~~~~~Place remaining associated items in a new structure or in an extension \\
~~~~{\bf else} \\
~~~~~~~~Assign a unique HDS TYPE to the new structure \numcir{4}\\
~~~~~~~~Assign a NAME to each component \numcir{5}\\
~~~~~~~~Determine the rules and restrictions governing the way the data will be stored in the various components\\
~~~~~~~~Assign a TYPE to each component \numcir{6}\\
~~~~~~~~Identify the sorts of operation to be performed on the structure and ensure they are meaningfully defined \numcir{7}\\
~~~~~~~~{\bf if} the processing of a component cannot be defined in some cases {\bf then} remove the component from the structure\\
~~~~~~~~Implement and document the software needed to process the new structure \numcir{8}\\
~~~~~~~~{\bf if} the new structure is to become a standard type {\bf then} \\
~~~~~~~~~~~~Submit it and its software to Starlink for approval \numcir{9}\\
~~~~~~~~{\bf endif} \\
~~~~{\bf endif} \\
{\bf endfor} \\
\end{quote}
\end{htmlonly}


\xlabel{se_newexn}\subsection{Explanatory Notes\label{se:newexn}}

\begin{enumerate}
\html{\label{se:expnote1}}
\item
\begin{itemize}
\item
It is important to define the scope of the software initially and not to
let it expand arbitrarily during implementation.  If the software design
does subsequently need to be revised, then the dataset may also need
re-designing. 
\item
During the following stages of the design process, the original outline of
the dataset may prove to be incorrect or inadequate, especially in
more-complex hierarchies.  In such cases you have to start again.
\end{itemize}
\html{\label{se:expnote2}}
\item
\begin{itemize}
\item
The interrelations between structures specify how they should be
organised hierarchically.  Drawing a dendrogram should help. 
\item
The design process is bottom-up.
Multiple-level datasets are built up from from the lowest (most deeply
nested) level of the hierarchy.  Design each and every structure at
the current HDS level before going to the next higher level.
\end{itemize}
\html{\label{se:expnote3}}
\item
\begin{itemize}
\item
Check to see whether any of the required data components
are standard structures or are components of standard
structures.  If suitable standard
structures already exist, then use them.
If not, and you have to design new structures, try to make
them general so that they might later become
standard structures themselves.
\item
The original dendrogram design of the dataset may grow some
extra branches if standard structures can be used, because 
there may be a net increase in the number of structures.
\item
Certain standard structures include provision for extension
structures, and may thus
be used even if there is no
appropriate place in the standard structure itself
for some of the items to be stored.
\item
Using existing standard structures gives the obvious advantages of being
able to use existing software.  (Starlink will maintain a list of
standard structures and their conventions.)  The standards and
conventions associated with standard structures must be observed by all
new software which uses them.
\end{itemize}
\html{\label{se:expnote4}}
\item
\begin{itemize}
\item
The TYPE should reflect the sort of data to be held in the structure, but
must not conflict with the TYPE of any other standard data structure. 
Starlink management should be consulted when defining TYPEs.
\end{itemize}
\html{\label{se:expnote5}}
\item
\begin{itemize}
\item
The names should preferably identify the r\^{o}le which each component
plays.  Although the name will have no global significance outside the
structure, it may still be sensible to have a naming convention for
certain common types of structure to avoid confusion. 
\item
There may be any number of rules and conventions governing use of the
structure.  For instance, some components may be optional, and the
presence of some components may depend on others (as with the VARIANT
concept). These rules must be explicitly stated and obeyed by all
software which uses the structure.  If this software is likely to be
written by many different people, then the rules should obviously be
kept simple.
\end{itemize}
\html{\label{se:expnote6}}
\item
\begin{itemize}
\item
Only primitives or structures of a TYPE already defined may be used.
\item
Since only defined TYPEs (which includes primitives) may be used, any
substructures must already have been defined along with the rules for
processing them.  It  might also occasionally be appropriate to
define structures ``recursively'' by including components of the same
TYPE as that being defined. 
\item
Often, standard subroutines will already exist for
processing the data components
from which the new structure is being built, and these can therefore be
used to process the components of the new structure. 
\end{itemize}
\html{\label{se:expnote7}}
\item
\begin{itemize}
\item
Ensure that all the operations are meaningfully defined in terms of what
will happen to each component when the structure is processed. Consider
all valid combinations of structure components. 
\item
Many packages ``grow'' indefinitely, so it may not be possible to
enumerate all possible operations.  However, if
the initial (global) stage of
the design was obeyed, it should be possible to identify them as broad
classes, such as [image display, arithmetic, spatial smoothing$\ldots$], or
[create history, append history, search for history record$\ldots$]. 
\item
It may be necessary to reject some components if you cannot meaningfully
define what will happen to them in all circumstances.
\end{itemize}
\html{\label{se:expnote8}}
\item
\begin{itemize}
\item
The software should obey all the conventions appropriate to
the new structure (and any other structures it uses).  When accessing a
structure, software should first check its TYPE---this specifies how the
structure contents are to be interpreted. Any component not covered by
the structure definition should be completely ignored. 
\item
From time to time, ignorance and independence of spirit will no doubt
lead implementors and users into inserting extra components into a
structure, but these are illegal and will be ignored.  This is {\bf not}
a valid way of defining a new structure.
\end{itemize}
\html{\label{se:expnote9}}
\item
\begin{itemize}
\item
If the new structure is to become a standard type, submit the design
(providing details of the NAME, TYPE, meaning and processing rules
for each data object) to the Starlink Head of Applications for approval
and registration.  If appropriate, a
subroutine interface should be written for handling the structure;
this would ensure that the conventions governing its use are enforced.
Any associated software should also be submitted to the Head of Applications.
\item
Once a new standard structure has been accepted,
anyone is free to use the structure and to
incorporate it in any new structures he or she may create. Once
this point is reached, it may be difficult to change the structure
definition without upsetting somebody;
changes in the form of additions to the structure are the least likely
to cause trouble.
\end{itemize}
\end{enumerate}

\subsection{Extensions}
Once a ``core'' of fairly simple standard structures exists, the
process of designing more specialised structures will be devolved to
the various SIGs, who can use the simpler structures as building
blocks.  This avoids the problems of the `all or nothing'
monolithic approach.  When a more complex (and therefore
highly specialised) structure is built out of simpler ones, software
will then automatically exist for processing all its substructures in
a more general way. This should give a high degree of flexibility.

There will be independent extensions, each
having a uniquely defined TYPE together with rules for its
interpretation.
Though many extensions will be independent and self-contained,
some will form hierarchies.
The design of each extension should be kept straightforward
and appropriate to the kind of software which will use it.
Simple and specialised,
simple and general, and
complex and specialised are all acceptable,
but implementors should beware of attempting to
design extensions which are both complex and general.  By
introducing a strict criterion to decide whether a given component is
acceptable (``do we know how to process it?''), it is ensured that the
problem is broken into manageable
pieces, the complexity of which does not exceed our software-writing
abilities.

\section{Acknowledgments}
\label{se:acknow}

Parts of this document have been adapted from earlier proposals. The
contributions of Keith Shortridge and Trevor Ponman
were particularly helpful.  The comments, criticisms and ideas of the
participants of the Great Debate greatly improved the
content of this document.

\section{References}
\label{se:refs}

\parindent=-3mm
\begin{verse}
\hspace{-9.2mm}
Wright, S.L. \& Giddings, J.R.  1983 {\it Standard Starlink Data 
Structures} (draft proposal).
\vspace{-2mm}
\end{verse}

\begin{verse}
\hspace{-9.2mm}
Wells, D.C. \& Greisen, E.W.  1979 in {\it Image Processing in Astronomy},
eds. Sedmak, K., Capaccioli, M. \& Allen, R.J., Osservatorio Astronomico
di Trieste, p.445.
\vspace{-2mm}
\end{verse}
\end{document}
