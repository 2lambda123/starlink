\documentclass[twoside,11pt]{article}
\pagestyle{myheadings}

% -----------------------------------------------------------------------------
% ? Document identification
\newcommand{\stardoccategory}  {Starlink General Paper}
\newcommand{\stardocinitials}  {SGP}
\newcommand{\stardocsource}    {sgp\stardocnumber}
\newcommand{\stardocnumber}    {25.13}
\newcommand{\stardocauthors}   {Starlink Project\footnote[1]{Contributors:
 M J Bly, C A Clayton, M J Currie, M D Lawden, A J Penny, D J Rawlinson,
 A V Roberts, J C Sherman, P T Wallace.
 Edited by M D Lawden}}
\newcommand{\stardocdate}      {9 December 1997}
\newcommand{\stardoctitle}     {Starlink Site Manager's Guide}
\newcommand{\stardocabstract}  {This guide helps Starlink Site
Managers carry out their r\^ole within the Starlink Project.
It details their responsibilities and describes the methods of communication
and administration used to manage Starlink.
It also describes the operating standards for Starlink computers and Starlink
software.
Finally, it describes the management of Starlink users.}
% ? End of document identification
% -----------------------------------------------------------------------------

\newcommand{\stardocname}{\stardocinitials /\stardocnumber}
\markright{\stardocname}
\setlength{\textwidth}{160mm}
\setlength{\textheight}{230mm}
\setlength{\topmargin}{-2mm}
\setlength{\oddsidemargin}{0mm}
\setlength{\evensidemargin}{0mm}
\setlength{\parindent}{0mm}
\setlength{\parskip}{\medskipamount}
\setlength{\unitlength}{1mm}

% -----------------------------------------------------------------------------
%  Hypertext definitions.
%  ======================
%  These are used by the LaTeX2HTML translator in conjunction with star2html.

%  Comment.sty: version 2.0, 19 June 1992
%  Selectively in/exclude pieces of text.
%
%  Author
%    Victor Eijkhout                                      <eijkhout@cs.utk.edu>
%    Department of Computer Science
%    University Tennessee at Knoxville
%    104 Ayres Hall
%    Knoxville, TN 37996
%    USA

%  Do not remove the %begin{latexonly} and %end{latexonly} lines (used by
%  star2html to signify raw TeX that latex2html cannot process).
%begin{latexonly}
\makeatletter
\def\makeinnocent#1{\catcode`#1=12 }
\def\csarg#1#2{\expandafter#1\csname#2\endcsname}

\def\ThrowAwayComment#1{\begingroup
    \def\CurrentComment{#1}%
    \let\do\makeinnocent \dospecials
    \makeinnocent\^^L% and whatever other special cases
    \endlinechar`\^^M \catcode`\^^M=12 \xComment}
{\catcode`\^^M=12 \endlinechar=-1 %
 \gdef\xComment#1^^M{\def\test{#1}
      \csarg\ifx{PlainEnd\CurrentComment Test}\test
          \let\html@next\endgroup
      \else \csarg\ifx{LaLaEnd\CurrentComment Test}\test
            \edef\html@next{\endgroup\noexpand\end{\CurrentComment}}
      \else \let\html@next\xComment
      \fi \fi \html@next}
}
\makeatother

\def\includecomment
 #1{\expandafter\def\csname#1\endcsname{}%
    \expandafter\def\csname end#1\endcsname{}}
\def\excludecomment
 #1{\expandafter\def\csname#1\endcsname{\ThrowAwayComment{#1}}%
    {\escapechar=-1\relax
     \csarg\xdef{PlainEnd#1Test}{\string\\end#1}%
     \csarg\xdef{LaLaEnd#1Test}{\string\\end\string\{#1\string\}}%
    }}

%  Define environments that ignore their contents.
\excludecomment{comment}
\excludecomment{rawhtml}
\excludecomment{htmlonly}

%  Hypertext commands etc. This is a condensed version of the html.sty
%  file supplied with LaTeX2HTML by: Nikos Drakos <nikos@cbl.leeds.ac.uk> &
%  Jelle van Zeijl <jvzeijl@isou17.estec.esa.nl>. The LaTeX2HTML documentation
%  should be consulted about all commands (and the environments defined above)
%  except \xref and \xlabel which are Starlink specific.

\newcommand{\htmladdnormallinkfoot}[2]{#1\footnote{#2}}
\newcommand{\htmladdnormallink}[2]{#1}
\newcommand{\htmladdimg}[1]{}
\newenvironment{latexonly}{}{}
\newcommand{\hyperref}[4]{#2\ref{#4}#3}
\newcommand{\htmlref}[2]{#1}
\newcommand{\htmlimage}[1]{}
\newcommand{\htmladdtonavigation}[1]{}

% Define commands for HTML-only or LaTeX-only text.
\newcommand{\html}[1]{}
\newcommand{\latex}[1]{#1}

% Use latex2html 98.2.
\newcommand{\latexhtml}[2]{#1}

%  Starlink cross-references and labels.
\newcommand{\xref}[3]{#1}
\newcommand{\xlabel}[1]{}

%  LaTeX2HTML symbol.
\newcommand{\latextohtml}{{\bf LaTeX}{2}{\tt{HTML}}}

%  Define command to re-centre underscore for Latex and leave as normal
%  for HTML (severe problems with \_ in tabbing environments and \_\_
%  generally otherwise).
\newcommand{\setunderscore}{\renewcommand{\_}{{\tt\symbol{95}}}}
\latex{\setunderscore}

% -----------------------------------------------------------------------------
%  Debugging.
%  =========
%  Remove % from the following to debug links in the HTML version using Latex.

% \newcommand{\hotlink}[2]{\fbox{\begin{tabular}[t]{@{}c@{}}#1\\\hline{\footnotesize #2}\end{tabular}}}
% \renewcommand{\htmladdnormallinkfoot}[2]{\hotlink{#1}{#2}}
% \renewcommand{\htmladdnormallink}[2]{\hotlink{#1}{#2}}
% \renewcommand{\hyperref}[4]{\hotlink{#1}{\S\ref{#4}}}
% \renewcommand{\htmlref}[2]{\hotlink{#1}{\S\ref{#2}}}
% \renewcommand{\xref}[3]{\hotlink{#1}{#2 -- #3}}
%end{latexonly}
% -----------------------------------------------------------------------------
% ? Document-specific \newcommand or \newenvironment commands.
% ? End of document-specific commands
% -----------------------------------------------------------------------------
%  Title Page.
%  ===========
\renewcommand{\thepage}{\roman{page}}
\begin{document}
\thispagestyle{empty}

%  Latex document header.
%  ======================
\begin{latexonly}
   CCLRC / {\sc Rutherford Appleton Laboratory} \hfill {\bf \stardocname}\\
   {\large Particle Physics \& Astronomy Research Council}\\
   {\large Starlink Project\\}
   {\large \stardoccategory\ \stardocnumber}
   \begin{flushright}
   \stardocauthors\\
   \stardocdate
   \end{flushright}
   \vspace{-4mm}
   \rule{\textwidth}{0.5mm}
   \vspace{5mm}
   \begin{center}
   {\Huge\bf \stardoctitle}
   \end{center}
   \vspace{5mm}

% ? Heading for abstract if used.
   \vspace{10mm}
   \begin{center}
      {\Large\bf Abstract}
   \end{center}
% ? End of heading for abstract.
\end{latexonly}

%  HTML documentation header.
%  ==========================
\begin{htmlonly}
   \xlabel{}
   \begin{rawhtml} <H1> \end{rawhtml}
      \stardoctitle
   \begin{rawhtml} </H1> \end{rawhtml}

% ? Add picture here if required.
% ? End of picture

   \begin{rawhtml} <P> <I> \end{rawhtml}
   \stardoccategory\ \stardocnumber \\
   \stardocauthors \\
   \stardocdate
   \begin{rawhtml} </I> </P> <H3> \end{rawhtml}
      \htmladdnormallink{CCLRC}{http://www.cclrc.ac.uk} /
      \htmladdnormallink{Rutherford Appleton Laboratory}
                        {http://www.cclrc.ac.uk/ral} \\
      \htmladdnormallink{Particle Physics \& Astronomy Research Council}
                        {http://www.pparc.ac.uk} \\
   \begin{rawhtml} </H3> <H2> \end{rawhtml}
      \htmladdnormallink{Starlink Project}{http://www.starlink.ac.uk/}
   \begin{rawhtml} </H2> \end{rawhtml}
   \htmladdnormallink{\htmladdimg{source.gif} Retrieve hardcopy}
      {http://www.starlink.ac.uk/cgi-bin/hcserver?\stardocsource}\\

%  HTML document table of contents.
%  ================================
%  Add table of contents header and a navigation button to return to this
%  point in the document (this should always go before the abstract \section).
  \label{stardoccontents}
  \begin{rawhtml}
    <HR>
    <H2>Contents</H2>
  \end{rawhtml}
  \htmladdtonavigation{\htmlref{\htmladdimg{contents_motif.gif}}
        {stardoccontents}}

% ? New section for abstract if used.
  \section{\xlabel{abstract}Abstract}
% ? End of new section for abstract

\end{htmlonly}

% -----------------------------------------------------------------------------
% ? Document Abstract. (if used)
%  ==================
\stardocabstract
% ? End of document abstract
% -----------------------------------------------------------------------------
% ? Latex document Table of Contents (if used).
%  ===========================================
\newpage
\begin{latexonly}
   \setlength{\parskip}{0mm}
   \tableofcontents
   \setlength{\parskip}{\medskipamount}
   \markright{\stardocname}
\end{latexonly}
% ? End of Latex document table of contents
% -----------------------------------------------------------------------------
\newpage
~
\newpage
\renewcommand{\thepage}{\arabic{page}}
\setcounter{page}{1}

\section {\label{introduction}Introduction}

This paper is intended primarily for the managers of Starlink sites,
the majority of whom are supported by Starlink via site contracts.
Managers should be familiar with the terms of these contracts, copies of which
are available from the University or from Starlink.

This paper will help you understand the Starlink Project and your
r\^{o}le in it.
It assumes you are aware of the documentation that vendors provide for managers
of their computers and associated hardware, and so it contains only the extra
information you need for Starlink purposes.
It makes many references to other Starlink documents (indicated by a code
number, such as \xref{SGP/41}{sgp41}{}); you need to become familiar with these
in order to do your job properly.

There are many people you can turn to for help.
These include other Site Managers and Project Staff with particular areas
of expertise.
Their names, addresses (electronic and postal), and functions are specified in
file {\tt /star/admin/whoswho}.
In general, use e-mail to communicate with people as your message can then be
dealt with at a convenient time and you have a record of it.
However, don't be afraid to telephone if you have a serious or urgent problem.

All Starlink documents, including this one and the ``whoswho", are available
on the World Wide Web (called the ``web" hereafter).
Starting from the Project's home page, simply follow the link
\htmladdnormallink{{\bf Documentation}}{http://www.starlink.ac.uk/docs.html}.

\subsection{What's New?}

The whole text has been revised to bring minor details up-to-date and
to clarify meaning.
In particular, the following changes have been made:

\begin{itemize}
\item {\bf Section \ref{meetings}}:
 the Astronomical Computing Panel has replaced the Starlink Panel.
\item {\bf Section \ref{distribution}}:
 new procedures for the release and distribution of Starlink Software.
\item {\bf Section \ref{qps}}:
 new quick programming service for users.
\item {\bf Section \ref{amateurs}}:
 support for amateur astronomers.
\item {\bf Section \ref{NUDP}}:
 include quick reference cards in the New User Document-Pack.
\item {\bf Section \ref{documentlibrary}}:
 revised numbers of documents distributed to sites by default.
\item {\bf Section \ref{references}}:
 new documents (\xref{SUN/212}{sun212}{} and \xref{SGP/51}{sgp51}{}).
\item {\bf Appendix \ref{starlinkcontacts}}:
 new r\^{o}les (PC/Linux Support and Communications).
\item {\bf Appendix \ref{sitecodes}}:
 new codes (ULS and UMI).
\item {\bf Appendix \ref{cronjob}}:
 what to do when a cron job fails.
\end{itemize}

\newpage

\section {\label{sitemanagersresponsibilities}Site Manager's Responsibilities}

A Site Manager's responsibilities are many and varied.
The following list specifies those duties which apply at every Starlink site;
individual Site Managers may be asked by the Project to perform additional
duties from time to time.

\begin{itemize}

\item {\bf Implement Starlink policy}.
This means both Project-wide policy (as communicated to you by Project staff)
which is relevant to your site, and local policy recommended by your Site
Chairman and/or Local Management Committee
(\xref{SGP/41}{sgp41}{}).
Communicate Starlink policy and developments to your users.
Feed back user views to the Project.

\item {\bf Liaise with Starlink central management}, especially with regard to
hardware and software changes, and inform central management at the first hint
of problems with maintenance companies.

\item {\bf Liaise with field service organisations} and other
maintainers of equipment and software, and deal with equipment faults.

\item {\bf Operate the computers and associated equipment} to provide a
satisfactory day-to-day service to your users, both on- and off-site users.
Check system performance by running the Starlink benchmark suite
(\xref{SSN/23}{ssn23}{}),
especially after installing new hardware or after a major software change.

\item Find out if there any local rules concerning the {\bf testing and safety
of electrical equipment}.
If there are, make sure you carry them out on Starlink equipment.

\item Install and maintain the {\bf Starlink Software Collection and licensed
software}.
Software changes issued by RAL should be implemented as soon as possible.

\item Provide {\bf advice and assistance to users of Starlink software} and
answer questions regarding operating systems and other commercial software.
You should be the first person that users turn to when they have problems,
although you may need to refer some of their questions to others for an
answer.

\item Encourage users who are developing software to do so according to
{\bf Starlink programming standards}
(\xref{SGP/16}{sgp16}{} and
\xref{SGP/4}{sgp4}{}) and to use the {\bf Starlink Software Infrastructure}
(\xref{SG/4}{sg4}{}), in order to help them produce
high-quality code which could, eventually, be supported by others.

\item {\bf Register new users} and ensure that only accredited people use the
system.
{\bf Remove details of users who have finished their work} at your site from
your list of registered users.
Maintain a Local Usernames File containing current details of your active
registered users.
Arrange for this file to be copied to RAL every month for merging with those of
other sites in order to produce a complete list of registered Starlink users.

\item Maintain copies of {\bf paper Starlink documents} for reference by, or
loan to, your users.
Keep these up-to-date, complete, and easily accessible.
Distribute new issues of the Starlink Bulletin to {\em all}\, your
{\em Primary}\, users.

\item Keep appropriate {\bf non-users} at your site informed about Starlink.
Give them things like the Starlink Bulletin, Starlink Glossies, and the
general introduction to Starlink
(\xref{SGP/31}{sgp31}{}).

\item Maintain on-line and up-to-date {\bf inventory lists}.

\item Keep your on-line {\bf maintenance schedules} at RAL up-to-date and
accurate.
You are encouraged to produce configuration diagrams for your own use,
but this is not mandatory.

\item {\bf Keep in touch} with relevant Starlink newsgroups, web pages, and
Starlink Forum conferences.

\item Help manage the {\bf finances} of the node where possible (this is not
appropriate at some sites).
An example of such financial concerns is consumables expenditure.

\end{itemize}

In carrying out these duties you will sometimes need advice or information
from Project Staff at RAL.
Details of Starlink contacts there can be found in file
{\tt /star/admin/whoswho}, or on the back of the Starlink Bulletin, or in
{\em Appendix~\ref{starlinkcontacts}},
or in Starlink's web pages.

Some of your duties are more important than others.
{\bf The primary aim of a Site Manager is User Support}, {\em i.e.\ to
help astronomers at your site to get the most out of Starlink's (PPARC's)
investment in hardware and software.}
Consequently, duties directly related to this are the most important, but this
does {\em not}\, mean that others can be neglected.

To fulfill your primary aim you should be aware of the work your users are
doing, or trying to do.
Thus, when the opportunity arises, ask them about their work methods --  you may
be able to suggest better ones.
A receptive audience for such advice is each year's intake of post-graduate
students.
If you would like assistance in advising users, for example a presentation on
some aspects of application software, then Project Staff or Starlink's Contract
Programmers will try to help.
You should also tell the Project about user concerns, such as local hardware
or software problems.

Many Site Manager's activities are ``event driven."
For example, if a hardware or software fault has halted your system, then
getting it running again (and letting users know what is happening) clearly has
top priority.
In these circumstances it is easy to lose sight of longer-term activities, but
a balance between event-driven and longer-term work is essential for a
Starlink node to be successful.

\newpage

\section {\label{communications}Communications}

\subsection {\label{meetings}Meetings}

There are four series of meetings which you must attend:

\begin{quote}
\begin{description}
\item [SMM -- Site Managers' Meeting:]
 This brings Site Managers and appropriate Project Staff together to discuss
 problems, report  developments, and agree on common standards and procedures.
 Meetings are held twice a year, once at RAL and once at one of the other
 Starlink sites.
 Attendance is optional (but recommended) for managers who are only partly
 funded by Starlink, but is mandatory for fully-funded managers.
\item [AGM -- Annual General Meeting:]
 This is held once a year at RAL (Cosener's House).
 It is similar to an SMM, but also involves Contract Application Programmers and
 all Site Managers.
 It is, therefore, considerably larger and lasts longer than an SMM.
 It is mandatory for {\em all}\, Site Managers.
\item [SLUG -- Starlink Local User Group:]
 These are informally constituted and comprise all the users of a site.
 These meetings are:
 \begin{itemize}
 \item the main forum for comments and suggestions about all aspects of the
  running of the node.
 \item an important source of suggestions for new hardware and software to
  include in your site's annual bid to the Starlink Panel.
 \end{itemize}
 They may be chaired by a user, with the manager acting as secretary, or
 {\it vice versa}.
 The Site Manager and Site Chairman should ensure that SLUG meetings are held
 regularly and are properly advertised.
 We recommend a minimum of two per year, plus others if there is a demand from
 users or if important issues arise that they would like to discuss.
 At least one meeting a year {\em must}\, be open to an address by a Project
 Representative.

 A great strength of the SLUG system is that no user need feel excluded from
 decisions about the running of his or her site.
 It is important, therefore, that users see their Site Chairman and the local
 committee as representing them and taking account of their recommendations.

\item [LMC -- Local Management Committee:]
 See \xref{SGP/41}{sgp41}{}.
\end{description}
\end{quote}

The dates of all Starlink meetings are usually known well in advance, so clashes
with other commitments can be avoided.

There are two other series of meetings which you need to know about:

\begin{quote}
\begin{description}

\item [Astronomical Computing Panel:]
This is a PPARC committee which oversees all astronomical computing,
including Starlink, but excluding supercomputing.
It advises the PPARC executive and PPARC's Astronomy Committee, and meets at
least times a year.
Its membership is listed in {\tt /star/admin/whoswho}.
The Project Management attends these meetings when Starlink is on the agenda.

\item [SSG -- Software Strategy Groups:]
These are user groups which consider software development in specific areas.
You may be asked to be a member of an SSG if you have special expertise,
so volunteer if you believe you can contribute.
When an SSG meeting is held at your site, by all means attend as an observer
even if you are not a member.

\end{description}
\end{quote}

\subsection {Starlink Forum}

{\em Starlink Forum}\, is a hypertext-based conferencing system on the web
(\xref{SUN/205}{sun205}{} and
\xref{SSN/33}{ssn33}{}).
It is used to share information and opinions.
It operates from the Starlink Project Node at RAL and is used primarily by
Starlink Site Managers and Programmers.

To access it, use the
\htmladdnormallink{{\bf Forum}}{http://rlsaxps.bnsc.rl.ac.uk/Forum}
link on the Starlink Project home page or set
up your own link to the URL:
\begin{quote}
{\tt http://rlsaxps.bnsc.rl.ac.uk/Forum}
\end{quote}
You should regularly read and contribute to the Conferences listed below
(they have restricted access).
If you can't access them, contact the Site Manager at the RAL Project Node.

\begin{quote}
\begin{description}
\item [Systems-Management] --
 This is the {\em most important} conference for Site Managers.
 It contains valuable advice and instructions from other managers and
 staff which are found nowhere else ({\em e.g.}\ the availability of new
 versions of system software).
 You should read this conference at least once a week, preferably daily.
\item [Faults] --
 Use this to log all hardware faults.
 We can use this information to identify design faults, unreliable batches of
 hardware, and as ammunition when trying to demonstrate a fault to a supplier.
\item [Maintenance] --
 Maintenance contract issues.
\item [Linux] --
 Linux-specific operations.
\end{description}
\end{quote}

\subsection{Starlink News Groups}

Starlink has two Usenet News Groups:

\begin{quote}
{\bf uk.org.starlink.announce\\
     uk.org.starlink.misc}
\end{quote}

The most important one for you is {\bf uk.org.starlink.announce}.
This is where software updates are announced.
It also carries news items which you should post at your site.
You should read this group several times a week.
Access to these groups can be provided by RAL if your local news feed
does not carry them.

\subsection{\label{web}World Wide Web}

The World Wide Web (the ``web") is now one of Starlink's most important
channels for providing information to the user community and Site Managers.
The Project maintains a home page at URL:

\begin{quote}
\htmladdnormallink{{\bf http://www.starlink.ac.uk/}}{http://www.starlink.ac.uk/}
\end{quote}

and this contains many useful links.
In particular, you should review the
\htmladdnormallink{{\bf Operations}}
{http://www.starlink.ac.uk/operations.html}
link regularly as this gives access to information about operational aspects of
particular types of Starlink equipment.
For example, there are pages to help managers of SPARC/Solaris,
Alpha/Digital-Unix, and Linux computers.
There are also links to publicity material, to help you demonstrate and
advertise Starlink facilities, and to your maintenance schedules.

\subsection {Liaison between Starlink and its Users}

One of your key duties is to pass on information about Starlink to your users,
and to help them find it for themselves.
Do this in several ways so as to reach as many of them as possible (some don't
read news items or login messages) --  find out from them which are the best
methods to use.
Small sites often rely on personal contact; larger sites must rely on several
methods, including:

\begin{itemize}
\item Regular and interesting {\bf SLUG} meetings.
\item Informal Starlink {\bf seminars}.
\item Starlink {\bf News} -- to submit a news item (such as a job advert) to
 this medium for distribution within Starlink, mail it to
 {\tt announce@star.rl.ac.uk}.
 Items should be in plain ASCII text (no hypertext), and should include an
 expiry date.
\item {\bf motd} (message of the day).
\item {\bf Web} pages -- your site's, other sites', and the Project's.
\item Global {\bf E-mail}.
\item A physical Starlink {\bf Notice Board}.
\item {\bf Starlink Documents}, particularly the Starlink User's Guide,
 Starlink Unix Guide, Starlink Bulletin, plus any local guides.
 Encourage your users to read these.
\end{itemize}

You should also talk to your users about Starlink and feed their views back to
the Project.
For example, we need to know what software is used at your site and what users
think of it.
We also need to know of new developments at your site which might require
advanced planning by the Project to provide new services.
You are also welcome to write an interesting article for the Starlink Bulletin,
preferably with lots of coloured pictures (send it to Anne Charles).

\newpage

\section {\label{administration}Administration}

\subsection {Maintenance Schedules}

{\em Site Managers are responsible for ensuring that their Starlink-maintained
equipment is on an appropriate maintenance contract administered by RAL.}

Maintenance schedules are stored on-line at RAL, and are available on the
web to all managers via the
\htmladdnormallink{{\bf Operations}}{http://www.starlink.ac.uk/operations.html}
link on the Project home page.
You will need a special username and password to access the maintenance
schedules -- these  are available from the {\em maintenance-contact}\, at RAL
on request.

Please check these pages regularly (at least once a month), especially if any
amendments have been requested for your site.

The stored schedule lists are exact copies of the details that each maintenance
company has for each site.
{\em If details of a piece of equipment are not on-line and it has not been
requested as an amendment by yourself, then we do not have any record of it
for maintenance purposes.}
For amendments, additions, or deletions required for your site, please send
full details of the changes to the {\em maintenance-contact}\, at RAL, stating
schedule numbers where known.
The amendments will not be put on-line until confirmation of change has been
received from the maintenance company.

It is vital that these maintenance schedules are kept up to date.
Failure to do so will cause substantial delays (weeks, not days) in
repairing faults and can cost many hundreds of pounds.
When you receive new equipment from Starlink, it is vital that you inform the
{\em maintenance-contact}\, at RAL of the arrival of this kit and all relevant
serial numbers.
{\em New equipment cannot be put onto maintenance without these serial
numbers}.
Do not assume that because Starlink bought the equipment, it will automatically
be put onto maintenance for you.
This needs input from you.

\subsection{Inventory Lists}

It is important that RAL has accurate and up-to-date records of the equipment
installed at your site.
A hardware inventory of such equipment (in the approved format) should be
maintained and sent once a month to the {\em maintenance-contact}\, at RAL.
Even if the inventory hasn't changed, change the ``update date" in the file to
the current date and send the listing to RAL as above.

\subsection{Expenses}

All Site Managers are entitled to reimbursement of their expenses when
they travel on Starlink business ({\em e.g.}\, to attend Site Managers'
meetings, Software Strategy Group meetings, or {\em ad hoc}\, meetings called
by the Project).

The managers of university sites should submit a claim to the University,
consistent with the University's normal regulations.
Subsequently, the University can claim reimbursement from Starlink by
including the expenses on one of the quarterly site contract invoices.
These arrangements are noted in the site contract document, which also notes
that prior approval must have been obtained from Starlink before any claim can
be paid.
Prior approval has already been given for all travel and subsistence expenses
in connection with several ``standard" purposes, this approval being in
the form of a letter from John Sherman dated 28 July 1989, and subsequent
updates dated 16 December 1992 and 8 December 1997.
If the letter, with its list of ``standard'' purposes, has been mislaid or
if your site joined Starlink after December 1997, please ask the
{\em Head of Operations}\, for a copy.
Prior approval must still be obtained, case by case, for any travel and
subsistence expenses in connection with visits which are not in the
``standard'' list, or where there is some doubt.

RAL imposes strict rules about who can be paid travelling expenses, so it is
worth contacting the Project beforehand in dubious or unusually expensive cases
({\em e.g.}\, where overnight accommodation will be required or where several trips
per day are claimed).

If your site is still grant-funded, prior to being adopted by Starlink, you
will need to fill in a Visitor's expense claim form (available from Andrea
Roberts) for expenses incurred while on Starlink business.
Please return it to Andrea for signature and forwarding to the RAL claims
office.

\subsection{Consumables Budget}

Most sites have a consumables budget based on the number of research
and technical users on site.
Management of this fund {\em may}\, be the responsibility of the Site Manager --
check with your Site Chairman.
The University will claim money spent under this budget (up to an agreed
amount) via the site contract.
Books bought for your sites' users come out of this budget.

\subsection {Additions to a Node via Grants}

It is recommended that additions to your node which are proposed by users and
which are not funded by Starlink ({\em e.g.}\, those funded by PPARC block
grant or by {\em ad hoc}\, PPARC or other grants) be discussed with yourself
and with the Starlink Project well before the proposal is finalised.
The advantages of early discussion with the Project are:

\begin{itemize}
\item Short-term and long-term maintenance can be discussed.
\item In the case of PPARC grants, subsequent review by the relevant
 committees and panels will be simplified.
\end{itemize}

\newpage

\section{\label{operatingstandards}Operating Standards}

There are some practical aspects to consider when setting up and operating a
Starlink computer, some of which are described below.
If necessary, you will be helped to set up your system by an experienced
manager, so this section does not describe the set-up procedure in detail.
Make the most of Site Managers' Meetings to find out what ingenious, but
doubtless undocumented, procedures are used at other sites.

The names of the staff at RAL responsible for technical support
of the various types of hardware can be found in
{\em Appendix~\ref{starlinkcontacts}}.
Advice on setting up a Starlink SPARC/Solaris System can be found in
\xref{SSN/18}{ssn18}{}.

\subsection {Backing-up File Systems}

Backing-up your file systems is one of your most important duties.
The purpose of backup is to protect your computer against accidental file
deletion, usually caused by user error or disk failure -- in that order.
Scratch space need not be backed up, but users should be made aware of what this
means to them if this is the policy at your site.

The following recommendations should not be treated as rigid rules -- there is
room for local variation.
Nevertheless, any site which deviates substantially from them should consider
why their procedures are different from those at other sites.

\begin{itemize}
\item Zero level (full) dumps should be run at least every few weeks.
\item Incremental dumps should be run on user partitions at least once a week
 (most sites run incrementals once a day).
 The frequency of incrementals depends on local factors, such as the number and
 type of tape drives available.
 Incremental dumps of system partitions ({\em e.g.}\ {\tt /usr}) need not be done as
 frequently as for user partitions.
\item Where possible, dumps should be done outside prime time, for example
 during early morning or evening or weekend, and high capacity cartridge drives
 should be used to enable backups to be run overnight.
\item Consider storing your backup media away from the systems to reduce
 the possibility of both the on-line data and the backups being destroyed by the
 same disaster.
\item Be very careful when you brief a stand-in to act as manager in your
 absence.
 Starlink's experience is that, despite the best of intentions, stand-ins have
 been a major cause of damage to system files.
 Please ask the stand-in to call RAL or another Site Manager if they get stuck.
 A stand-in who presses on with fingers crossed when things don't go quite as
 the manager had led them to expect can do a surprising amount of damage when
 armed with the root password.
\item If you have not yet retrieved files from your dumps, use a scratch disk
 to make an end-to-end test of your procedures.
\end{itemize}

Scripts for dumps are available from the RAL Project Site Manager.

\subsection {Special Usernames}

You can contact other Site Managers by sending e-mail messages to the {\tt star}
username at their nodes  -- they will use the same username to send e-mail to
you.
This is the primary way of communicating between sites and
should be monitored regularly; a deputy must be instructed to do this if you
are away, even for a day.
Username {\tt star} is also used for maintaining Starlink software.

Some sites can't use {\tt star} for e-mail due to local restrictions.
The back page of the Starlink Bulletin and the {\tt /star/admin/whoswho} file
give appropriate addresses for the Site Managers at these sites.

Every manager is permitted a username on the Project system at RAL.
This is required to allow you to download some proprietary software.
Its password should never be divulged to users.

\subsection {Benchmarks}

One of your duties is to ensure the efficient running of your hardware.
A suite of benchmarking software
(\xref{SSN/23}{ssn23}{}),
based on Starlink Software and on IRAF, is available to allow you to confirm
that your system is running as it should.
Running this software has, on many occasions, shown up problems with system
set-up which were previously overlooked.

\subsection {Games}

Game playing, along with any other flippant or improper uses of Starlink
computers, is not allowed.

\newpage

\section {\label{starlinksoftware}Starlink Software}

The Starlink Software Collection is comprehensively documented.
The following documents provide general background to Starlink software
development:

\begin{quote}
\begin{description}
\item [\xref{SUN/1:}{sun1}{}] Starlink software collection.
\item [\xref{SGP/4:}{sgp4}{}] Starlink C programming standard.
\item [\xref{SGP/16:}{sgp16}{}] Starlink application programming standard.
\item [\xref{SGP/42:}{sgp42}{}] Starlink software strategy.
\item [\xref{SGP/46:}{sgp46}{}] Starlink software work during last year.
\end{description}
\end{quote}

Other documents describe the management of the Collection:

\begin{quote}
\begin{description}
\item [\xref{SUN/212:}{sun212}{}] Starlink software CD-ROMs, user's guide.
\item [\xref{SGP/21:}{sgp21}{}] Starlink software distribution policy.
\item [\xref{SGP/51:}{sgp51}{}] Starlink software maintenance procedures.
\item [\xref{SSN/9:}{ssn9}{}] Installing the Unix Starlink software.
\end{description}
\end{quote}

You should read these to understand Starlink software and its management.

Every Starlink site should have an
up-to-date copy of the required parts of the Starlink Software Collection
installed on its system.
\xref{SSN/9}{ssn9}{}
tells you how to install the Collection from scratch.

Do not distribute any part of the Collection to anyone else without
permission from Starlink management
(\xref{SGP/21}{sgp21}{}) --
contact the {\em Software Librarian}\, in the first instance.
In particular, proprietary software, such as the NAG libraries, {\em must not}\,
be distributed.
If you do receive permission to distribute Starlink software to another site,
give the {\em Software Librarian}\, the following details about the site:

\begin{itemize}
\item Full postal address, network address, telephone, and fax number.
\item What you sent, when you sent it, and the type of media used.
\item Name of a site contact.
\item What support you intend to give.
 (Will you answer queries from users?
 Will you supply updates?)
\end{itemize}

The GKS graphics package may be distributed to non-profit-making astronomical
research centres for use in connection with other Starlink software.
This involves a registration procedure and must be done by the Project.

If you, or a user, have trouble with a software item, send details to the
{\em Software Librarian}.
Do not just make a local fix and forget about it; inform the {\em Librarian}\,
and get the correction properly distributed through the official release
mechanism.
When reporting bugs, please include sufficient detail to make the bug
repeatable -- a bug which cannot be repeated cannot, in general, be fixed.
In urgent cases, if the affected user wants to contact the author of the
software directly, this is permissible.
However, please also submit a proper bug report.

Don't let your version of Starlink software become chronically different from
RAL's.
This has caused major problems in the past.
Also, software developed at another site has, on occasion, been submitted
for release and did not work at RAL because it relied on some local
modification.
Conversely, there have been cases where software which was released by RAL
didn't work at another site because of local peculiarities.
Be warned: if you allow your site to get into this kind of mess you may find
that new or improved software doesn't run at your site.

Some items in the Collection belong to the {\em Base Set}.
These are non-astronomical, proprietary, and public domain software.
Base set items are specified in the Starlink Software Index
({\tt /star/admin/ssi}).
All base set items must be installed at all sites.

\subsection {\label{distribution}Distribution and Maintenance}

In October 1997, Starlink changed the way it distributes the Starlink
Software Collection from the old system, based on frequently issued
software updates via the network, to a new one using CD-ROMs and the Starlink
Software Store.
This enables a Starlink Software installation to be tailored to match the
precise requirements of a Site.

The primary distribution medium will be CD-ROM.
The Software Librarian will build all the libraries, packages, and utilities
from scratch every six months (March and September), for all supported
systems, and issue a complete system on a multi-CD set.
The set will contain a CD for each supported platform with a ready-to-run
Software Collection, and a CD of all the software packaged individually, again
for all supported systems.
In addition, rebuilt versions of the Base Set software will be provided.
Each release will be documented by a new version of SUN/212.

Sites will be able to update their local Starlink Software installation
from the CD-ROM in one go using the tools provided, or update selected
packages in batches, depending on local needs.
Sites may also replace their local installation completely with the one on
the CD-ROM should they wish, or run the software from the CD-ROM itself,
although this may only be practicable on a PC running Linux with a fast (8X)
CD-ROM drive.

At the same time that the CD-ROM sets are issued, the Starlink Software
Store will be updated with distribution sets of the software packages, so
that the software versions in the Store match those on the CD-ROM.
Thereafter, until the next CD-ROM, the Store will be updated with any
bug fixes required for the packages.
Sites may update their installations with patches from the Software Store, or
wait until the next CD-ROM is available.

Some software will not be included on the CD-ROM, because it uses the
proprietary NAG libraries.
Starlink has removed any dependency on the NAG libraries from the majority of
packages, replacing the calls to NAG routines with calls to appropriate
routines in the Starlink Public Domain Algorithms (PDA) library.
Such NAG-free packages will be included on the CDs.
The NAG libraries themselves will not be included on the CDs, nor will the
MEMSYS (Maximum Entropy) libraries.
Instead, they will be available to Starlink Sites from RAL using a modified
version of the existing distribution method.

Starlink regularly distributed information files, such as lists of users and
document indexes, as part of the old distribution system.
These files, and documents not associated with a specific software package,
will now be updated once a month by a simplified version of the previous
network-based distribution method.

\subsection {\label{qps}QUICK -- Quick Programming Service}

In 1997, Starlink launched {\em Quick} -- a custom Programming Service aimed at
solving individual user's small data reduction and analysis problems.
The service provides quick solutions to specific astronomical data
reduction/analysis problems by, for example, supplying a recipe or script
using existing software, or adding minor functionality to an existing
application.

The problems should be those which a programmer can solve in about
a day (or less), and should be in areas of Starlink expertise so that the
knowledge gained can be fed back to the Starlink user community.
Longer jobs will have to vie with other contenders to be included in future
software plans.

{\em Quick}\/ is experimental.
Its future depends on its popularity, and the availability of appropriate
Starlink staff.
Please encourage your users to use it.
A successful service will help Starlink too.
For instance, if common themes appear in the requests, this will pinpoint
gaps in our software, and areas where Starlink needs to provide new or better
documentation, such as cookbooks.

Requests for the service should be e-mailed to {\tt quick@star.rl.ac.uk}.
There is also a Web page ({\tt http://www.starlink.ac.uk/quick}) where a
selection of programming solutions will be made public to benefit
other users.

\newpage

\section {\label{usermanagement}User Management}

Previous sections have already mentioned several of your responsibilities
towards your site's users, such as providing ``advice and assistance."
This section is concerned specifically with how to manage your users.
The term {\em User Management}\, covers the following aspects of your work:
\begin{itemize}
\item Who is allowed to use Starlink facilities, and how are users
      classified?
\item What do I do when a new user arrives or an existing user departs?
\item What documents do I give to, and keep in stock for, my users?
\item What information do I keep on my users, and in what format?
\item Who do I tell about my users and how often do I tell it?
\item Is there any software to help me maintain my user information?
\item How does the Data Protection Act affect the way I keep information on
      users?
\end{itemize}

\subsection{\label{worktype}Types of Work supported by Starlink}

The types of work that are appropriate for Starlink facilities are described
in \xref{SGP/31}{sgp31}{}.
They are:

\begin{itemize}
\item Normal astronomical data reduction and analysis.
\item Limited theoretical astronomy.
\end{itemize}

Within these broad guidelines, different types of work have the following
priorities:

\begin{enumerate}
\item Astronomical data reduction that can only be done interactively.
\item Other astronomical data reduction.
\item Other astronomical research work that can only be done interactively.
\item Other astronomical research work.
\item Other PPARC astronomical work.
\end{enumerate}

If you are not sure whether or not some proposed work falls within Starlink's
terms of reference, refer to
\xref{SGP/31}{sgp31}{}
or consult the {\em Project Scientist}.
A difficult question is where precisely to draw the line between, for
example, routine telemetry processing associated with the operation of
a particular satellite or instrument, which is clearly not Starlink
work, and analysis of calibrated scientific data, which clearly is.
The analysis and visualization of theoretical data generated on other more
powerful machines is considered mainstream Starlink work, but the generation of
those data is not.
(Supercomputers are regarded as the theoreticians' telescopes or satellites.)

\subsubsection{\label{amateurs}Support for Amateur Astronomy}

Although Starlink is not funded to support amateur astronomers, it allows the
Starlink Software Collection to be used for non-profit, astronomical use by
amateurs.
Starlink is, however, unable to provide support for this class of use.
The Project is aware that fruitful pro-am collaborations do take place,
leading to peer-reviewed publications.
Amateurs collaborating with professionals should rely on their collaborators
to give guidance on the reduction of data and related support.
In addition, amateurs are expected to form a self-help group run over e-mail
and the web (many of the relevant people already have access to this
technology).
It is this group's responsibility to provide amateur-specific advice on how to
use the software to solve particular problems.
For example, the group could produce its own cookbooks for the standard amateur
CCD cameras.
Structures for organizing this type of group already exist in the amateur
community, as well as motivated individuals who are willing to put in effort to
make this software available.

When new Starlink Software CD-ROMs are pressed, a limited number will be
sent to a ``senior" member of the UK amateur community (nominated by Starlink)
who is then responsible for advertising the CD-ROMs and distributing them to the
relevant people.
This person is Starlink's sole interface to the amateur community in terms of
software distribution.
Amateurs can also get the software from the Starlink Software Store.
Direct requests to the Project for CD-ROMs from anyone other than the official
amateur contact will be acknowledged, but cannot otherwise be responded to.
The same applies to technical queries.

Starlink values the views of amateur astronomers regarding its software, and,
through the official amateur/Starlink contact, welcomes feedback.

\subsubsection{\label{teaching}Support for Teaching}

Starlink is intended to support astronomical research, not teaching.
Consequently, the Starlink resources, including Site Manager's time, that can
be spent on teaching are strictly limited.
This applies to teaching both in schools and at undergraduate level in
universities (but see section~\ref{usertypes} for undergraduates doing
research).
Nevertheless, Starlink software can be used for teaching, as long as no
significant support from Starlink is required and as long as the standard
restrictions on software distribution are observed.

\subsection{\label{usertypes}Types of Starlink User}

There are a large number of Starlink users and they use Starlink facilities
for many different reasons.
It is useful, indeed essential, to classify them in some way.
For example, user classification helps to determine appropriate resource
allocation for Starlink sites, and is essential for determining consumables
funding.

The current classification is based on the following coding system:

\begin{quote}
\begin{description}
\item [r] -- UK-resident research astronomer actively processing data
 (includes post-graduate students).
\item [t] -- UK-resident scientific or technical staff member supporting
 astronomy research (includes Site Managers).
\item [o] -- Other UK-resident user ({\em e.g.}\, secretaries and e-mail-only
 users).
\item [u] -- Undergraduate engaged in an astronomical research programme.
\item [a] -- Associate user (a person who does not have an official
 astronomical position in a university or recognised research institution).
\item [f] -- Non UK-resident user, abroad for at least half the current academic
 year.
\end{description}
\end{quote}

Starlink Staff are ``{\bf t}" users because their primary r\^{o}le is to support
astronomers rather than to do astronomical research themselves.
Of course, some staff may do astronomical research in addition to their other
duties.
As always, classification schemes throw up difficult borderline cases.
The answer is to do what seems reasonable.

The sub-sections below discuss in more detail the rationale behind the two most
recent user classes to have been introduced -- those indicated by
``{\bf u}" and ``{\bf a}" codes.
Users in these classes must be registered in the same way as other Starlink
users and should be given individual accounts.
The amount of site management support given to them should be determined by the
Site Chair and will normally be at a level substantially below that given to
other users.
They will not count towards the allocation of Starlink resources to a site,
neither do they qualify for consumables support.
The Site Chair should regularly review their research output and their impact
on other Starlink users, and should ensure that they understand Starlink rules
concerning the distribution of Starlink software.

\subsubsection{Undergraduate User -- (u)}

There is a growing trend for universities to incorporate a research project in
undergraduate education.
In part, this is due to the introduction of 4-year courses.

As noted in section~\ref{teaching},
and with the exception of software, significant Starlink resources cannot be
used for undergraduate teaching.
However, Starlink resources {\em can} be used for an {\em undergraduate
research project}\, (at the discretion of the local Site Chair) provided that:

\begin{itemize}
\item The normal post-graduate research work at the site is not seriously
 affected.
\item The output of the project is likely to form a significant part of a
 published work.
\end{itemize}

\subsubsection{Associate User -- (a)}

Associate users are people who do not have an official astronomical position
in a university or research institution, {\em e.g.}\ amateur
astronomers.
(An unpaid position, such as Honorary Research Fellow, is regarded here as an
official position, and the other user classifications apply.)

At the discretion of the local Site Chair such people can use Starlink, but
only in specific cases.
Each case must be examined individually.
If in doubt, consult the {\em Project Scientist}.
In general, a person qualifies if:

\begin{itemize}
\item The output of the proposed work is likely to form a significant part of
 an article in a research publication.
\item The associate user collaborates actively with an existing Starlink user.
 This is not absolutely essential, but an associate user whose work is
 completely separate from other work at the site would need special
 justification.
\item The work will not seriously affect the site.
\end{itemize}

\subsection{Registration and Termination of Starlink Users}

If someone wants to use the facilities of your site, they must complete a
{\em Starlink User Application Form} and return it to you -- this is true,
even if they are already registered at another site.

To save time you may register new users yourself, in advance of formal
accreditation, as long as you are reasonably sure that the work involved is
appropriate for Starlink
(see sections~\ref{worktype} and \ref{usertypes}).
Dubious cases should be referred to the {\em Project Scientist}\, before you
allow the user onto your system.

Here is a checklist of things you should do to register a new user:

\begin{itemize}
\item Check the form and ensure that it has been completed satisfactorily.
 (Is it legible?
 Have all the questions been answered satisfactorily?)
\item If the user is a research student, make sure his or her Supervisor has
 signed it.
\item When you are happy, sign the form yourself {\em and date it}.
 Fill in the section headed {\em``For completion by the Site Manager"}\, at
 the bottom of the back page.
\item Make a copy, and file the original with your other completed application
 forms.
\item Send the copy to the {\em Project Scientist}\, at RAL.
 He will check it and sign it if he is happy, or get back to you with any
 queries.
 If the proposed work does not meet Starlink criteria, he will ask you to
 withdraw the username.
\item Add details of your new user to your local {\em Starlink User Database}
 (see section \ref{SUD}).
 In particular, {\bf add a record for the user to your usernames.lis file and
 change its date.}
\end{itemize}

Please use the latest version of the application form (currently dated
``95/11" in the bottom left-hand corner of the front page).
Stocks are obtainable from the {\em Document Librarian}.

Each form should cover one person and one username only.
Shared usernames should be avoided unless absolutely essential as they are a
security risk.
If you have such usernames, make sure that their use is properly controlled.

As a user's work evolves, it may change in nature and become less legitimate.
If so, ask the user to make a new application to use your site and treat this
like any other new application.

A person may have a username at more than one Starlink site, although this
should rarely be necessary.
Such a user should specify one of these sites as his {\em Primary Node}\,
which is then responsible for his primary support.
His other sites are called {\em Secondary Nodes}.
A user is called a {\em Primary User}\, at their Primary Node, and a
{\em Secondary User}\, at their Secondary Nodes.
(People registered at only one site are Primary Users of that site by default.)
The distinction between Primary and Secondary nodes avoids us counting users
more than once when figuring out how many users we've got.
The {\em Document Librarian}\, keeps an eye on multiple registrations and makes
sure that such users have only one Primary Node.

When a user leaves permanently (say for more than 3 months), remove (or archive)
his details from your Starlink User Database and his files from your system.
In particular, {\bf delete his record from your usernames.lis file and change
the date}.
This should be done within one month of departure.

\subsection{Documentation for Users}

It is important that your users are aware of, and use, relevant documents
distributed by Starlink.

\subsubsection{\label{NUDP}New User Document-Pack}

You should give new users a {\em ``New User Document-Pack."}
(If they have previously used other sites, they may only need any local
documents you think appropriate.)
Its contents are at your discretion, but should include:

\begin{quote}
\begin{description}
\item [\xref{SUG}{sug}{}] -- Starlink user's guide.
\item [\xref{SUN/1}{sun1}{}] -- Starlink software collection (including
quick reference card)
\item [\xref{SUN/145}{sun145}{}] -- Unix: An introduction (including quick
reference card).
\item [Glossies] -- A folder containing glossies on Starlink, our software
 environment, and popular applications.
\end{description}
\end{quote}

You should add anything else they need in order to work effectively, such
as a local guide to your site or descriptions of software they intend to use.
You may wish to bind together into a single volume a set of local documents
for your users.
A generic coloured cover is available entitled ``Starlink Documents" together
with a matching back cover.
These can be used to make your locally-produced volumes match Starlink's
``package" documents in style, and can be supplied by the Document Librarian.

\subsubsection{World Wide Web}

As mentioned in section \ref{web},
most Starlink documentation is available on the web, together with a lot of
extra information that is useful to users.
Two particularly useful commands which use the web to find and display
information are {\bf showme} and {\bf findme}.
They are illustrated in the \xref{Starlink User's Guide}{sug}{}, and
you should make sure that your users are aware of them.

\subsubsection{Starlink Bulletin}

Twice a year, Starlink issues a newsletter called the
\htmladdnormallink{{\em Starlink Bulletin}}
{http://www.starlink.ac.uk/bulletin.html}.

Copies are sent to every site and you must send one to all your {\em Primary
Users}.
(Your {\em Secondary Users}\, should receive a copy from their Primary Node.)
Also, give one to any local non-users you think may be interested.

The Bulletin is one of the most important ways in which Starlink informs its
users about Project status, new software, developments, policies, and plans.
Please distribute it promptly.

\subsubsection{\label{documentlibrary}Paper Document Library}

A list of Starlink documents is kept in files
    {\tt /star/docs/docs\_lis}
and {\tt /star/docs/mud\_lis},
but an easier way to find out what is available is to follow the
\htmladdnormallink{{\bf Documentation}}{http://www.starlink.ac.uk/docs.html}
link on the Starlink Project web home page.

Your users should have access to paper copies of Starlink documents if they
want them, and they should know where to find them.
Keep this {\em Document Library}\, stocked, up-to-date, and visible (use the
Starlink sticky coloured labels to identify your binders and filing cabinets).

You must decide the best method of managing your Document Library -- different
sites have different needs.
At RAL we keep the Library in well-labelled filing cabinets.
Each drawer holds lots of individual pockets, each labelled with a specific
document code.
Each such pocket contains several copies of the document corresponding to its
label.
Above these there is an overflow shelf which holds extra copies of some
documents (particularly large ones) so that the pockets in the filing cabinets
do not get over-stuffed.
We do {\em not}\, keep a reference set in binders because when someone wants a
document they go straight to the filing cabinet.
This works fine at RAL but you may find that a reference set in binders with
some sort of backup stock is more suitable for your environment.
The important thing is that your users should know where to find Starlink
documents and that your Library should be reliable ({\em i.e.}\, complete
and up-to-date).

Most Starlink documents are issued as stapled photocopies of A4 sheets.
Some of the more important ones are issued in bound form to make them more
attractive and easier to use.
Some managers tear off these covers and file the tattered remains in binders;
not what Starlink intends: try putting them on shelves next to the binders.

RAL sends every Starlink site one paper copy of every new Project-wide document
(many sites issue local documents, but normally these are not distributed to
other sites).
If the document is issued with a special binding and cover (like `software
package' documents), then more than one copy will be sent.
If you want more of these special documents, ask the {\em Document Librarian}.
If you want more of the plain stapled documents, you are expected to
produce them yourself from the source file(s) in {\tt /star/docs}.

\subsection{\label{SUD}Starlink User Database}

You should set up and maintain a {\em Starlink User Database}\, containing
information about your sites's users.
You can keep any information you like on your users (within the constraints
of the Data Protection Act, see section~\ref{DPA}), but we recommend as a
minimum the following three files (normally stored in directory
{\tt /star/local/admin}):

\begin{quote}
\begin{description}
\item [usernames.lis] -- names, usernames, classification codes, location
 codes, and e-mail addresses.
\item [users.lis] -- more detailed information.
\item [people.adr] -- postal addresses.
\end{description}
\end{quote}

Of these, the first ({\bf usernames.lis}) is {\em mandatory}\, and is
described in section~\ref{LUF}.
The others are optional.

Many sites maintain a {\bf users.lis} file, but their format varies widely.
We recommend that the {\tt staradmin} utility (see section~\ref{staradmin})
is used to set it up and maintain it.
The file should contain things like room and phone numbers, research interests,
expected leaving date, and so on.
Its purpose is to help you find out things about your users fast (maybe in
response to a question from the Project).

The {\bf people.adr} file is used to hold user's postal addresses in a form
that can be used by a program called STICKY (LSN/28 (PRO)).
This is used at RAL to print sticky address labels to help us send things
to users and sites.
Each address can be tagged with one or more codes to indicate specific
distribution lists for various types of reader.
STICKY then extracts the addresses on a specified list and writes a file for
printing.
Other sites find it useful for the same job and you can get the program
from Kevin Duffey on request ({\tt kpd@star.rl.ac.uk}).

\subsubsection{\label{LUF}The local usernames File -- usernames.lis}

You {\em must} install and maintain a {\bf Local Usernames File}
called {\bf usernames.lis} containing information about your users.
Keeping this file accurate and up-to-date is {\em vital}.
The set of Local Usernames Files collected from every Starlink site is the
best way we have of determining who is currently a Starlink user.
It makes your users ``visible" to others in the astronomical community, and
the equivalent file at other Starlink sites make their users visible to
yours.

Every month, RAL merges the Local Usernames File from every Starlink site to
produce a complete list of all currently registered Starlink users.
This list is then distributed to sites via the network in the form of two files:

\begin{quote}
\begin{description}
\item [\htmladdnormallink{star/admin/usernames}
 {http://www.starlink.ac.uk/usernames1}]
 -- contains username, name, category, and
 site code for every user at every site, together with a count of the
 users (Primary and Secondary) and user-codes at each site.
\item [star/admin/unixnames]
 -- contains name, location code, and e-mail address for every user.
\end{description}
\end{quote}

These files enable the Project to keep accurate current and historical records
of Starlink's user population.
They are also the primary sources of information for people who wish to know a
Starlink user's e-mail address, site, or location.

Merging the Local Usernames Files usually throws up inconsistencies.
The {\em Document Librarian}\, then sends a ``Username queries" message to the
sites involved to try to resolve them.
Please read and act promptly on this message.

The correct format of your Local Usernames File is described in
{\em Appendix~\ref{usernamesfile}}.
The file should be maintained by the {\bf staradmin} procedure
(see section~\ref{staradmin}).

\subsubsection{\label{SYLUF}Sending your local usernames File to RAL}

You {\em must}\, send a copy of your Local Usernames File to RAL every
month for merging with the others.
This is done by a ``cron job" -- an automatic procedure that copies the file
from your site to the RAL ftp machine.
After all the cron jobs have finished, the copied files are then copied into
the {\em Document Librarian}'s file space.
(This only happens after the automatic procedure has run -- the {\em Document
Librarian}\, doesn't get files sent by re-run cron jobs.)
A description of how to set up such a job is given in
{\em Appendix~\ref{cronjob}}.

\subsubsection{\label{staradmin}A Database Maintainer --  staradmin}

Installing information in your local Starlink User Database and keeping it
up-to-date is a rather tiresome process.
Fortunately, Adrian Fish (a former Site Manager at UCL) wrote a
perl script to do it automatically (apart from putting the data in).
This has been released as a Starlink Software Item called {\bf staradmin}
(SSN/27) and is maintained Tim Gledhill ({\tt star@star.herts.ac.uk}).
Refer to
\xref{SSN/27}{ssn27}{} to find out how to make your life easier.

\subsection{\label{DPA}Data Protection Act}

In the UK, information about people that is kept on a computer is subject to the
Data Protection Act.
This demands that such information be registered and places restrictions on its
use.

Starlink's files are believed to be covered {\em at Research Council sites
only}\, by the general RAL registration, and we have given full details of the
relevant files to the appropriate RAL contact.
However, this does not cover Universities.
The manager of every Starlink university site should, therefore, inform
their Data Protection Officer of the existence and contents of any
files held on their Starlink computer(s) which contain information about
specific people.
Your DP officer will probably have a form for you to fill in.

Users of Starlink facilities are also responsible for registering their
own files containing information on other people with their DP registrar.
However, we believe this does not involve the Site Manager.

An entry on the {\em Data Protection Register}\, gives the data user's name and
address, together with broad descriptions of:

\begin{itemize}
\item The personal data held.
\item The purposes for which it is used.
\item The sources from which the information may be obtained.
\item The people to whom the information may be disclosed ({\em i.e.}\ shown or
 passed on to).
\item Any overseas countries or territories to which the data may be transferred.
\end{itemize}

Once registered, the data users must comply with the eight {\em Data Protection
Principles}\, of good practice contained within the Act.
Broadly, these state that personal data must be:

\begin{enumerate}
\item Obtained and processed fairly and lawfully.
\item Held only for the lawful purposes described in the data user's register
 entry.
\item Used only for those purposes, and disclosed only to those people,
 described in the register entry.
\item Adequate, relevant, and not excessive in relation to the purpose for which
 they are held.
\item Accurate and, where necessary, kept up-to-date.
\item Held no longer than is necessary for the registered purpose.
\item Accessible to the individual concerned who, where appropriate, has the
 right to have information about themselves corrected or erased.
\item Surrounded by proper security.
\end{enumerate}

\newpage

\section {\label{references}References}

\begin{quote}
\begin{description}
\item [LSN/28 (PRO)] -- STICKY: Produce address labels from an address list.
\item [\xref{SG/4}{sg4}{}] -- ADAM: The Starlink software environment.
\item [\xref{SGP/4}{sgp4}{}] -- Starlink C programming standard.
\item [\xref{SGP/16}{sgp16}{}] -- Starlink application programming standard.
\item [\xref{SGP/21}{sgp21}{}] -- Starlink software distribution policy.
\item [\xref{SGP/31}{sgp31}{}] -- Starlink.
\item [\xref{SGP/41}{sgp41}{}] -- Local management arrangements.
\item [\xref{SGP/42}{sgp42}{}] -- Starlink software strategy.
\item [\xref{SGP/46}{sgp46}{}] -- Starlink's software work during last year.
\item [\xref{SGP/51}{sgp51}{}] -- Starlink software maintenance procedures.
\item [\xref{SSN/9}{ssn9}{}] -- Installing the Unix Starlink software.
\item [\xref{SSN/18}{ssn18}{}] -- Administering Solaris Sun systems.
\item [\xref{SSN/23}{ssn23}{}] -- Starlink benchmarking utility.
\item [\xref{SSN/27}{ssn27}{}] -- staradmin: Starlink user database maintainer.
\item [\xref{SSN/33}{ssn33}{}] -- FORUM: Installation guide.
\item [\xref{SUG}{sug}{}] -- Starlink user's guide.
\item [\xref{SUN/1}{sun1}{}] -- The Starlink software collection.
\item [\xref{SUN/145}{sun145}{}] -- Unix: An introduction.
\item [\xref{SUN/205}{sun205}{}] -- FORUM: Starlink conferencing system.
\item [\xref{SUN/212}{sun212}{}] -- Starlink software CD-ROMs, user's guide.
\item [Glossies] -- a folder (available from the Document Librarian) containing
glossy A4 sheets on Starlink software.  There are 9 sheets available at present
(also available from the Document Librarian).
\end{description}
\end{quote}

\appendix

\newpage

\section {\label{starlinkcontacts}Starlink Contacts}

\begin{quote}
\begin{tabbing}
Contactxxxxxxxxxxxxxxxxxxxxxxxxxxxx\=Namexxxxxxxxxxxxxxx\=\kill
{\em R\^{o}le}                   \> {\em Contact}  \>{\em E-mail address}\\
\\
{\bf Project Manager}            \> Pat Wallace    \>{\tt ptw@star.rl.ac.uk} \\
{\bf Project Scientist}          \> Alan Penny     \>{\tt A.J.Penny@rl.ac.uk} \\
{\bf Head of Operations}         \> John Sherman   \>{\tt jcs@star.rl.ac.uk} \\
{\bf Software Librarian}         \> Martin Bly     \>{\tt ussc@star.rl.ac.uk} \\
{\bf Site Manager (PRO)}         \> Hiten Patel    \>{\tt hiten@star.rl.ac.uk} \\
{\bf Maintenance}                \> Andrea Roberts \>{\tt avr@star.rl.ac.uk} \\
{\bf Inventories}                \> Andrea Roberts \>{\tt avr@star.rl.ac.uk} \\
{\bf SPARC/Solaris support}      \> Chris Clayton  \>{\tt cac@star.rl.ac.uk} \\
{\bf Alpha/Digital Unix support} \> David Rawlinson\>{\tt djr@star.rl.ac.uk} \\
{\bf PC/Linux support}           \> David Rawlinson\>{\tt djr@star.rl.ac.uk} \\
{\bf Peripheral support}         \> Kevin Duffey   \>{\tt kpd@star.rl.ac.uk} \\
{\bf Hardware purchasing}        \> Chris Clayton  \>{\tt cac@star.rl.ac.uk} \\
{\bf Site contract renewals}     \> Kevin Duffey   \>{\tt kpd@star.rl.ac.uk} \\
                                 \> Andrea Roberts \>{\tt avr@star.rl.ac.uk} \\
{\bf Programmer contracts}        \> David Rawlinson \>{\tt djr@star.rl.ac.uk} \\
{\bf Applications software development}\> Rodney Warren-Smith \>{\tt rfws@star.rl.ac.uk} \\
{\bf Graphics and networking}    \> David Terrett  \>{\tt dlt@star.rl.ac.uk} \\
{\bf Communications}             \> Chris Clayton  \>{\tt cac@star.rl.ac.uk}
\end{tabbing}
\end{quote}

\newpage

\section{\label{sitecodes}Site Codes}

\begin{quote}
\begin{tabbing}
Codexxxxx\=Namexxxxxxxxxxxx\kill
{\bf ARM} \> Armagh \\
{\bf BEL} \> Belfast \\
{\bf BIR} \> Birmingham \\
{\bf BRI} \> Bristol \\
{\bf CAM} \> Cambridge: IOA/RGO \\
{\bf MRA} \> Cambridge: MRAO \\
{\bf CAR} \> Cardiff \\
{\bf DUR} \> Durham \\
{\bf EDI} \> Edinburgh \\
{\bf GLA} \> Glasgow \\
{\bf HAT} \> Herts (Hatfield) \\
{\bf IMP} \> Imperial College \\
{\bf JOD} \> Jodrell Bank \\
{\bf KEE} \> Keele \\
{\bf KEN} \> Kent (Canterbury) \\
{\bf LEE} \> Leeds \\
{\bf LEI} \> Leicester \\
{\bf LJM} \> Liverpool John Moores \\
{\bf MAN} \> Manchester \\
{\bf OXF} \> Oxford \\
{\bf PRE} \> Central Lancashire (Preston) \\
{\bf QMW} \> Queen Mary \& Westfield College \\
{\bf PRO} \> RAL: Project \\
{\bf RAL} \> RAL: Astrophysics \\
{\bf SHE} \> Sheffield \\
{\bf STA} \> St Andrews \\
{\bf SOU} \> Southampton \\
{\bf SUS} \> Sussex (Brighton) \\
{\bf UCL} \> University College London \\
\\
{\bf Remote User Groups:} \\
\\
{\bf BRA} \> Bradford \\
{\bf NOT} \> Nottingham \\
{\bf OPU} \> Open University \\
{\bf ULO} \> UCL Observatory, Mill Hill\\
{\bf ULS} \> Ulster\\
{\bf UMI} \> UMIST
\end{tabbing}
\end{quote}

\newpage

\section{\label{usernamesfile}Format of usernames.lis File}

\begin{itemize}

\item {\bf Header section}.

This should start off with a Title on the first line which identifies your site,
followed by a line containing the date of the last revision and the file name.
An example of what is required is:

\begin{verbatim}
                    LOCAL STARLINK USERNAMES (PRO)
Current date: 15-Nov-95                    /star/local/admin/usernames.lis
\end{verbatim}
Put your 3-character site code (see {\em Appendix~\ref{sitecodes}}) at the end
of the Title to identify your site.

This can be followed by any other information you like, such as the meaning
of user codes, or lists of generic usernames in use at your site.
Any information you put in the Header section will be ignored by the merging
programs.
Only information in the User section will be used.

\item {\bf User section}.

This is the most important section of the file.
It contains the information that is extracted and merged at RAL to produce
complete user lists.
{\em Do not use tabs to separate fields, in fact don't use tabs anywhere in
the file -- they derange the merging programs.}

{\bf The section must start with a line containing the characters
{\tt +++} in columns 3 to 5.}
This is important because it is what the merging programs look for to find
the start of the User section.

This is followed by one line for each user containing the following
information in the following format:

\begin{itemize}
\item Column 1: {\bf User code:}
 \begin{quote}
 \begin{description}
  \item [r] -- Research astronomer (UK).
  \item [t] -- Technical support (UK).
  \item [o] -- Other (UK).
  \item [u] -- Undergraduate (UK).
  \item [a] -- Associate (UK).
  \item [f] -- Foreign (non-UK).
 \end{description}
 \end{quote}
 Specify a user code {\em only}\, for your {\em Primary}\, users.
 Leave the field blank for your {\em Secondary}\, users (it will be ignored
 in this case by the merge programs).
\item Column 2: {\bf Secondary user.}\\
 Type an {\tt *} if the user is a Secondary user; leave blank otherwise.
 Thus, if column 1 contains a code, column 2 should be blank, and if column 2
 contains an {\tt *}, column 1 should be blank.
\item Columns 3-26: {\bf Username(s).}\\
 If more than one is specified, separate them with a {\tt /} character.
\item Columns 27-49: {\bf Surname.}
\item Columns 51-68: {\bf First name} or familiar name.
\item Columns 70-74: {\bf Location code.}\\
 Use the codes specified in file {\tt /star/admin/location}.
 If an appropriate code isn't listed, invent one and tell the
 {\em Document Librarian}\, about it.
\item Columns 82-132: {\bf E-mail address.}\\
 Include the username, {\em e.g.}\ {\tt pma@star.rl.ac.uk}.
 If a user has more than one, then he will have to pick one to have
 advertised in this slot.
\end{itemize}
\end{itemize}

\newpage

\section{\label{cronjob}Cron Job to send usernames.lis file to RAL once a month}

{\em Procedure:}

\begin{itemize}

\item Select a machine on which to run the cron job and put an entry in the
crontab file, as described below, to run the ftp job at the time specified
below for your site.

\item Login as root.

\item The entry in the crontab file should be of the form:
\begin{quote}
{\tt <minute> 0 14 * * ftp starlink-ftp.rl.ac.uk}
\end{quote}
This tells your computer to open an ftp session to {\tt starlink-ftp.rl.ac.uk}
at {\tt <minute>} past midnight on the 14th day of each month.
The actual ftp of the file is carried out using a {\tt /.netrc} file
owned by root and with 600 permission.

\item The {\tt /.netrc} file should contain the following lines:
\begin{quote}
----------------------------------------------\\
{\tt
machine starlink-ftp.rl.ac.uk\\
login anonymous\\
password usernames\\
macdef init\\
cd /admin/mdl\\
put /star/local/admin/usernames.lis namesXXX.lis\\
quit\\
\\
}
----------------------------------------------
\end{quote}
The blank line at the end is {\em required}.
In the ``put" command, use the full name of your current Local Usernames File.
(Beware: in the past, some sites have sent obsolete versions of this file by
mistake.)
Replace {\tt XXX} with your 3-letter site code (see {\em
Appendix~\ref{sitecodes}}).
Use uppercase characters for your site code -- for example the correct name for
the Project site would be {\tt namesPRO.lis}.

\item Now set up your crontab file:

\begin{itemize}

\item Make a copy of your current crontab file:
\begin{quote}
{\tt \# crontab -l {\verb+>+} cron.tmp}
\end{quote}

\item Edit this file to add the entry shown previously.

\item Resubmit your crontab file.
\begin{quote}
{\tt \# crontab cron.tmp}
\end{quote}

\item Check that it is there with:
\begin{quote}
{\tt \# crontab -l}
\end{quote}

\end{itemize}
\end{itemize}

Each site should copy their file at a different time to avoid overloading
RAL's ftp machine.
The correct value of {\tt <minute>} at your site is specified in the following
table:

\begin{table}[ht]
\begin{center}
\begin{tabular}{||lr|lr|lr|lr||} 	\hline
%CODExxx\=\kill
ARM &  2 & BEL &  4 & BIR &  6 & CAM &  8 \\
MRA & 10 & CAR & 12 & DUR & 14 & EDI & 16 \\
GLA & 18 & HAT & 20 & IMP & 22 & JOD & 24 \\
KEE & 26 & KEN & 28 & LEI & 30 & LJM & 32 \\
MAN & 34 & OXF & 36 & PRE & 38 & QMW & 40 \\
PRO & 42 & RAL & 44 & STA & 46 & SOU & 48 \\
SUS & 50 & UCL & 52 & SHE & 54 & BRI & 56 \\
LEE & 58 &     &    &     &    &     &    \\
\hline
\end{tabular}
\end{center}
\end{table}

Sometimes the cron job fails to copy the usernames file to RAL successfully.
When this happens, you may get a request from Kevin Duffey for you to
e-mail the file to him directly.
It is important that you send the file directly to him by e-mail, rather than
just re-running the cron job.
This is because the cron job only copies the file to a system directory at
RAL.
Another step is required to copy this file from here to Mike's working
directory, and this only happens once a month (after the original cron jobs
have run).
If you just re-run the cron job, Mike won't get the file.

\end{document}
