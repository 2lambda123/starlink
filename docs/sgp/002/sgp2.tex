\documentstyle{article}
\pagestyle{myheadings}

%------------------------------------------------------------------------------
\newcommand{\stardoccategory}  {Starlink General Paper}
\newcommand{\stardocinitials}  {SGP}
\newcommand{\stardocnumber}    {2.2}
\newcommand{\stardocauthors}   {P T Wallace}
\newcommand{\stardocdate}      {26th September 1990}
\newcommand{\stardoctitle}     {Starlink Software Support Responsibilities}
%------------------------------------------------------------------------------

\newcommand{\stardocname}{\stardocinitials /\stardocnumber}
\markright{\stardocname}
\setlength{\textwidth}{160mm}
\setlength{\textheight}{240mm}
\setlength{\topmargin}{-5mm}
\setlength{\oddsidemargin}{0mm}
\setlength{\evensidemargin}{0mm}
\setlength{\parindent}{0mm}
\setlength{\parskip}{\medskipamount}
\setlength{\unitlength}{1mm}

\newcounter{sectno}
\newcommand{\sect}[1]   {\addtocounter{sectno}{1}
                         \goodbreak\vspace{4ex}\noindent
                         {\thesectno}.\hspace*{1em}{\bf #1}\\[3ex]}

\begin{document}
\thispagestyle{empty}
SCIENCE \& ENGINEERING RESEARCH COUNCIL \hfill \stardocname\\
RUTHERFORD APPLETON LABORATORY\\
{\large\bf Starlink Project\\}
{\large\bf \stardoccategory\ \stardocnumber}
\begin{flushright}
\stardocauthors\\
\stardocdate
\end{flushright}
\vspace{-4mm}
\rule{\textwidth}{0.5mm}
\vspace{5mm}
\begin{center}
{\Large\bf \stardoctitle}
\end{center}
\vspace{5mm}

\sect {BACKGROUND}
In a series of papers written by the Project during 1988,
the Starlink Users' Committee's attention was drawn
to a mismatch between
(i)~the manpower resources available to Starlink and
(ii)~the tasks for
which the community
assumed Starlink was responsible.
This mismatch meant that many users had
over-optimistic expectations of what
software and services could be provided by the
Project, leading in the long term to inefficiency and waste
as well as disappointment and mistrust.  At its meeting in
February 1989, SUC decided to set up a small panel
to look into the choices faced by the Project.
The members of this panel
(Dr.\,J.\,Baruch, Prof.\,G.\,Efstathiou and
Dr.\,G.\,Skinner)
visited RAL in June~1989 and were able
to inspect detailed costings of Starlink's programme
and to talk to Project staff.  The Panel concluded:
\begin{itemize}
\item There was indeed a serious manpower/workload mismatch, and it was
likely to get worse.
\item Starlink's activities and objectives had never been properly defined.
\item It was necessary to formulate and publicise a new definition so
that the astronomical
community would be in no doubt about what the Project is expected to achieve.
\end{itemize}
The panel went on to recommend the following:
\begin{itemize}
\item As far as possible
the present range of activities must continue,
with no radical reductions in
the number of nodes or types of astronomy.
\item Starlink should limit its development and support
of {\it applications} software.
Support for specialised
applications and instrument specific software
should be left to the astronomical community, though
the development of such software should continue to
be compatible with Starlink, and this might be a condition for
grant awards.
\item Starlink should concentrate instead on developing
{\it environment} software, so as to provide adequate tools for
applications programming done by the
astronomical community.
\item Starlink's important and
time-consuming advisory r\^{o}le to
SERC committees and to grant applicants on new
hardware ({\it e.g.}\ whether specific hardware
is value for money) should be more fully appreciated
by the astronomical community.
\item Starlink's r\^{o}le in assessing new
developments ({\it e.g.}\ UNIX workstations)
should be placed at higher priority.
\item The distinctions between software that is
(i)~fully documented and supported, (ii)~partially
documented and supported and (iii)~unsupported
should be made clear to users.
\end{itemize}
Readers of the present document who, despite the
findings of the SUC
Panel, remain sceptical that there really is a
problem to be solved, or who believe that there must be some easy
``fix'', should reflect that the Panel's conclusions
were reached (i)~by astronomers who were themselves
sceptical at the start and (ii)~after long and
detailed consultations.  These conclusions should not be
dismissed lightly or
because some of the consequences are unwelcome.  The
very detailed quantitative and factual evidence which led the
SUC Panel to its decisions will not be presented again here.

In complying with SUC's recommendations, the Project has revised and
clarified its software support policies,
a description of which is the main purpose of the present document.
In addition, efforts are to be made to avoid accepting future commitments
that the Project cannot meet at prevailing levels
of manpower; for example, although it may be
possible to identify and purchase interesting new hardware options,
the job of providing new software for such devices may fall to
interested users at individual sites.

As well as setting out limits to Starlink's software
responsibilities, the Project's
plans try to compensate by identifying areas in which {\it firmer}
commitments
can be made than was customary in the past.  It is, in particular,
hoped that the new policies will enable
more realistic guarantees on timescales to be given.

The proposed new policies were circulated in the community during
Spring~1990, in the form of an early draft of the
present document, and suggestions made during those consultations were
taken into account before the document was first released generally.

\sect {SOFTWARE SUPPORT}
It is well-established in the software industry that
most -- perhaps 70\% -- of the total effort expended on
a program during its life cycle is spent on maintenance,
and that the initial design, coding and testing phases
form an unexpectedly small part of the
whole.\footnote{See for example pp3-5 of {\it Software Evolution}, Lowell~Jay~Arthur,
John~Wiley and Sons (1988), and the bibliography on pp249-250.
This author attributes 3\% of the life-cycle costs to requirements
definition, 3\% to preliminary design, 5\% to detailed design,
7\% to implementation, and 15\% to testing, leaving 67\% for the operations
and maintenance phase.  Some other authors quote an even higher
percentage for this last phase.}
Many users are
unaware of this, and prefer to cling to the comforting but na\"{\i}ve
notion that
once a program is working it will continue to do so for
ever without human intervention.  But what does
{\it maintenance} consist of?
\begin{itemize}
\item Maintaining an awareness of how the software works so that bugs can
be fixed, and reasonably quickly;  fixing bugs.
\item Solving users' problems (which are usually not due to bugs).
\item Making appropriate amendments when changes in other
software, notably the operating system, cause the application
to stop working.
\item Responding to requests for changes and enhancements.
\item Looking after the program source, development tools,
and documentation;  maintaining the capacity to
rebuild the application when necessary and to prepare
revised documentation.
\item Integrating within the Starlink Software Collection and
preparing new releases;  supplying software and documentation on request.
\item Keeping up-to-date the summary and index documentation referring
to the item in question.
\item Keeping the software up-to-date,
to fit in with other software, to reflect changing practices or
to take advantage of new graphics devices {\it etc}.
\end{itemize}
Starlink is able to commit itself to providing support in
all these respects, but only for a subset of the existing software.

In many ways, the Project's revised software support policy
is simply a restatement of the original ideas for Starlink,
{\it circa} 1979.  Nonetheless, the most controversial
aspect of the plan, and the one that has been hardest for
the community to understand and accept, is the Project's inability
to commit itself to {\it full} support of more than a
small fraction of the total volume of applications software required.
It is commonly assumed at many Starlink sites
that local responsibility for writing software is limited to
those specialised applications needed by that site alone, and that
Starlink must therefore be responsible for all general-purpose
applications packages.  This was, in fact, never the case;
from the start, Starlink's funding bodies expected most of this software to
come from the community:
\begin{quotation}
``\ldots there is no way that the {\small STARLINK} system itself
can provide more than a very modest fraction of the necessary
effort.  The {\small STARLINK} management's job is to
encourage, advise and support the astronomers who are willing and
able to contribute their own efforts and their own expertise to the
task.''\footnote{{\it STARLINK}, M.J.Disney \& P.T.Wallace,
Q.Jl.R.Astr.Soc.(1982) {\bf 23}, p501}
\end{quotation}
Although manpower levels in Starlink's central
team have been increased since
the early days, most of the additional capacity has been absorbed
by the vigorous growth in the number of nodes and
users, and by the widening scope of wavelengths and types
of astronomy covered by Starlink.  The need for contributions of
software and support effort from astronomers themselves
is even more pronounced today than when the Project was first
set up.  There is, regrettably, no sign that this volunteer
effort is any more available now than in earlier times.

\sect{WHICH SOFTWARE?}
The software required by astronomers covers a broad range, from
operating systems such as VMS to real-time control of
data-acquisition instrumentation.  Within this range, we
can identify the following categories (all required if the gap between
the instrument and the data analysis computer is to be bridged --
{\it i.e.}\ they are {\it not} priorities):
\begin{enumerate}
\item Operating systems.
\item I/O device drivers.
\item Application support environment software (parameter
      systems, command languages, {\it etc}.).
\item Kernel graphics packages ({\it e.g.}\ GKS).
\item High-level graphics utilities ({\it e.g.}\ PGPLOT, NCAR).
\item General-purpose applications ({\it e.g.}\ FIGARO, KAPPA).
\item Astronomical subroutine libraries ({\it e.g.}\ SLALIB).
\item Applications for one type of astronomy ({\it e.g.}\ GASP).
\item Data reduction systems for one class of detector ({\it e.g.}\ CCD).
\item Data reduction systems for one instrument ({\it e.g.}\ TAURUS).
\item Data acquisition software for telescope and satellite-borne
instruments.
\end{enumerate}
There is little dispute about who is
responsible for software and support in categories~1 and 2, which
is purchased from the computer
vendor, supplemented occasionally with locally developed
device drivers.  It is likewise universally accepted that all
category~11 software is provided by instrument builders.
Regarding categories~3-5, Starlink undertakes to provide a suite
of software covering these areas, and also participates to various extents in
the distribution and support of selected ``foreign'' software
environments (AIPS, IDL, IRAF {\it etc}.).
There have, in the past, been
misunderstandings over category~10 (for historical reasons mainly
in ground-based optical circles rather than in radio or space), but
it now seems to be generally accepted that this software
must be written by those institutions
responsible for the instrument, and maintained
by them for as long as they wish the software to remain in service.

However, this still leaves categories~6-9.  What can and should Starlink
contribute here?

The answer to this question is that Starlink's central team at RAL will
develop and support various category~6 items, and one or two specialist
items in categories~7 and 8, but development and
maintenance of most software in these categories
must come from a combination of
(a)~the Starlink contract programmers, and (b)~the user community.

On those occasions when the Project's central team,
with due circumspection, carries out
changes within the Starlink software,
it does not accept automatic responsibility
for fixing any problems in non-Starlink-supported
software that occur as a consequence.
Similar remarks apply where specific hardware items
become obsolete and are (after due notice) withdrawn from service.
Naturally, where problems of this sort occur, the Project will try
to be as helpful as it can under the circumstances, taking into
account the extent to which contributed software has been written
according to Starlink standards (see SGP/16).

\sect {MANAGEMENT IMPLICATIONS}
Within the Starlink Project, the agreed policies on software
mainly affect the work of three groups:
(i)~the RAL applications staff, (ii)~the RAL
user support and software distribution staff, and (iii)~the
contract applications programmers.

{\it RAL APPLICATIONS:} \hspace{2ex} The {\bf Head
of Applications} (HoA), who is responsible for
managing the SIGs and the contract programmers, is expected in
his own software work to concentrate on programming standards
and techniques and on the provision of tools for program
development and testing rather than on applications themselves.
The HoA has an important r\^{o}le in the
education of user/programmers, and spends time encouraging compliance
with Starlink standards and the
use of supported software products (like the application
support environment package ADAM, now mature and
well-supported), with
the objective of minimising the software support
burden falling on the community.  This is a crucial point.
A key purpose of a system like ADAM is to {\it minimise} the
total software effort required to provide astronomers with
the applications they need.  Writing independent
one-off applications may appear to offer faster results, but
soon the lack of uniformity, the difficulties of inter-package
data exchange, and above all the constant re-invention of the
wheel mean that the total effort has gone up, not down, compared
with the environment approach.
Similarly, the {\bf RAL programmers} tend to work more on infrastructure
software than on applications {\it per se}, typically developing
subroutine libraries rather than complete astronomical application
programs.  Nonetheless, some development of applications
proper does go on within Starlink.  This is
important in order to exploit the special skills and interests of
individual staff, and to ensure that the systems work is
maintaining its relevance.  Such software will, of course, always be
developed according to Starlink programming standards and
will be designed with long-term support in mind.

{\it RAL USER SUPPORT:}  \hspace{2ex} The
{\bf User Support and Software Distribution Group}
adds to the Starlink Software Collection
items which astronomers have submitted for release.
In the case of new items, the Head of Applications is
asked to judge whether the software is a useful
addition to the Collection;  the HoA also advises
on when obsolete items may be withdrawn.  The
pre-release checking and validation process is relatively uncritical,
consisting of just a few elementary ``health'' checks.
Is the package all there?  Are
there any gross wastages of disc space?  Can the package
be run in the context of a normal Starlink user account?
Are there any violations of Starlink
security or system management conventions?
However, these checks are all that Starlink can promise,
and it is up to individual software authors to give
assurances about trustworthiness and long-term support.
Another r\^{o}le of the Group is to provide various forms
of user education, including the production of new and improved
documentation and the development of course and display materials.

{\it HEI CONTRACTORS:} \hspace{2ex} Work directed by {\bf SIGs} and carried out
by the fixed-term {\bf contract programmers}
will not, after the programmer's
contract comes to an end, be supported
automatically by Starlink's central RAL team,
though certain items may be.  The bulk of the
support effort must come from successive generations of contractors, or from
interested users, and it is the responsibility of the
SIG to make appropriate provisions.  In planning and providing the required
level of support, the SIGs may find that it is
difficult to recruit and retain talented
programmers on temporary contracts if too large a proportion of
their workload is support of existing software, rather than
developing new applications.  This difficulty is
perhaps best avoided by tolerating a degree of
otherwise unnecessary duplication and rewriting, and by
retiring obsolescent software without delay.  It should be noted
that there is financial provision for maintaining the numbers of
contract programmers at HEIs well into the future, and some prospect
that their numbers will increase, beyond the current level of 8.

\sect {EDUCATION}
The different software products available through Starlink have
vastly different levels of support and longevity, and
the Project is devising methods to make users more aware of these
characteristics.  (See also the next section.)
Direct education, through Node Managers
and by giving presentations at SLUG meetings, will
be improved and a clearer picture will be painted of what
products Starlink recommends and for which it can offer long-term support.
The production of
posters advertising supported software and recommended
techniques is also being considered.  A further intention
is to enclose Starlink documentation in
colour-coded covers bearing printed statements as to the
level of support available, including specific details of
who is looking after the product and the long-term prognosis.

\sect {SOFTWARE SUPPORT CATEGORIES}
The file RLVAD::LADMINDIR:SUPPORT.LIS contains details of the support
arrangements for each item of Starlink software.  A listing of this
file is appended to the present document as a guide;  for the latest
information it is essential to refer to the file itself.

The support for each named item is classified in two ways.  There are
five {\it support levels}:
\begin{enumerate}
\item[A] Unconditional support, by Starlink staff, at short
notice;  compatibility with other software will be maintained.
\item[B] High level of support, but with reservations based on
geographical, response time or control factors.
\item[C] Best efforts, by a named individual.
\item[D] Little or no support available.
\item[E] Obsolescent.
\end{enumerate}
and four {\it support classes}:
\begin{enumerate}
\item[C] Commercially supported.  (All such software is of support level B.)
\item[S] Starlink-supported.
\item[N] Non-Starlink-supported.
\item[U] Unsupported.  (All such software is of support level D or E.)
\end{enumerate}
The important parameter is the {\it support level};  this is what an
astronomer should take into account when deciding whether or not
to employ any given software item in his or her work.

Starlink's recommendation is to use only software of support level
A or B in long-term projects, to use level C software only for
one-off jobs, and not to use software of level D or E at all.
Level A software is entirely Starlink-supported, and a rapid
response to problem reports can be expected.  Level B software
includes items supported commercially or by volunteers outside
the Project, and there may be delays.

Movement of a given item from one support level to another depends
on the available effort and expertise.  Only rarely will
a downward migration of more than one level occur over a short
period.  Upward migration for software of special importance
to the community will depend on the ability of
existing staff to take on the increased support commitment;
community pressure alone will not be enough.  Where the
Project does agree to raise the support level of an item,
other items may have to be reduced in support level to compensate.
\end{document}
