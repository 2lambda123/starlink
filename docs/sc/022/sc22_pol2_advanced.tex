\chapter{\xlabel{pol2_advanced}POL-2 - Advanced Data Reduction}
\label{sec:advanced}


The pol2map tool for reducing POL-2 data was released to the science
community for the start of 17B observing. As with all newly commissioned
instrumentation the "ideal" reduction has yet to be finalised. This advanced 
section of the POL-2 data reduction documentation aims to provide the interested
astronomer with tools for expanding and examining the POL-2 reduction process further and 
in more detail.


\section{\xlabel{addingdata}Adding new observations}

This section describes the six step process of how to add 
data for one or more new POL2 observations into existing I,
Q and U maps and vector catalogue created by an earlier run of pol2map.

1) Create a text file listing all the existing auto-masked I maps for individual observations
stored in the directory specified by parameter mapdir, and then add in the raw data files
for the new observations. The auto-masked I maps have names that end in "_imap.sdf".

\begin{terminalv}
% ls maps/*imap.sdf > infiles.list
% ls rawdata/*.sdf  >> infiles.list
\end{terminalv}


2) Create a new auto-masked co-added I map including the new observation. The
calcqu  and makemap
commands will be run on the new data and the resulting maps co-added
with the existing maps for the older observations to create the new map:

\begin{terminalv}
% pol2map in=^infiles iout=iauto_new qout=! uout=! mapdir=maps qudir=qudata
\end{terminalv}


3) You will need to decide whether to re-create all the externally masked maps using
external masks defined by the new auto-masked map. This is the case if the auto-masked
map has been changed significantly by the addition of the new observation. To do this it is
necessary to compare the old and new mask.  The old masks should have been created
earlier using the MASKOUT1 and MASKOUT2 parameters (see step 3 in section 2.2). So
now create the new masks that would be generated from the new auto-masked map.

\begin{terminalv}
% pol2map  in=^infiles iout=! qout=! uout=! mapdir=maps mask=iauto_new \
maskout1=astmask_new           maskout2=pcamask_new
\end{terminalv}


4) Decide if the addition of the new data has changed the masks significantly. This involves
comparing astmask.sdf and astmask_new.sdf (and also pcamask.sdf and pcamask_new.sdf).


5) If you decide that the mask has changed significantly and you therefore want to
reprocess all observations using the new mask, remove the existing externally-masked 
maps so that they will be re-created by the next invocation of pol2map.  Note - this will
increase the length of time taken by step 6 enormously! Also, ensure the new auto-masked
co-add is used in place of the old one to define the masks in future.

\begin{terminalv}
% rm mapdir/*Qmap.sdf mapdir/*Umap.sdf mapdir/*Imap.sdf
% mv iauto.sdf iauto_old.sdf
% mv iauto_new.sdf iauto.sdf
\end{terminalv}

6) Re-create the necessary externally masked maps and co-adds, and then create the new
vector catalogue:

\begin{terminalv}
% pol2map in=qudata/\* iout=iext_new qout=! uout=! mapdir=maps \
     mask=iauto
% pol2map in=qudata/\* iout=! qout=qext_new uout=uext_new mapdir=maps \
     mask=iauto ipref=iext_new cat=mycat_new debias=yes
\end{terminalv}

\section{\xlabel{pixelsize}Experimenting with Pixel sizes}

Currently, as with SCUBA-2 reductions, the default reduction pixel size is 4".
The pixel size is controlled by the pixsize parameter in the the \smurf\ pol2map:

\begin{terminalv}
% pol2map pixsize=12
\end{terminalv}


The following four step example shows how to investigate the impact of changing pixel size.
In this example we compare 12" pixels and 7" pixels.


1) Begin with an auto-masked total intensity map from the raw data. For instance:

\begin{terminalv}
% pol2map in=^myfiles.list iout=iauto12 pixsize=12 qout=! uout=! mapdir=maps12 \
                    qudir=qudata
\end{terminalv}


2) Create AST and PCA masks with 12" pixels from the iauto12.sdf file:


\begin{terminalv}
% pol2map in=qudata/\* iout=! qout=! uout=! mapdir=maps12 mask=iauto12 \
                   maskout1=astmask12 maskout2=pcamask12
\end{terminalv}

3) Create masks with 7" pixels by resampling the 12" masks created at
step 2. This is done using the \Kappa\ sqorst command:

\begin{terminalv}
% sqorst  mode=pixelscale pixscale=\'7,7,7E-05\' in=astmask12 out=astmask7
% sqorst  mode=pixelscale pixscale=\'7,7,7E-05\' in=pcamask12 out=pcamask7
\end{terminalv}

4) Create the 7" externally masked I, Q and U maps using the above 7"
masks (note the "mask" parameter value  is enclosed in single AND
double quotes):

\begin{terminalv}
% pol2map in=qudata/\* iout=iext7 qout=qext7 uout=uext7 masktype=mask \
                  mask="'astmask7,pcamask7'" mapdir=maps7 ipref=iext7  \
                  cat=cat7 debias=yes
\end{terminalv}


\begin{tip}
Using larger pixels usually produces slower convergence, so the
above process will take longer than usual - be patient!.

Using larger pixels can sometimes encourage smooth blobs and other
artificial features to appear in the map. You should display the
iauto12.sdf file to check that it does not have such artificial
features.

Check the masks (astmask12.sdf and pcamask12.sdf) to make sure
they look reasonable.
\end{tip}

\section{\xlabel{IPerror}Investigating systematic error in IP}


The error on the IP is reported to be on the order of 0.5\%.
It is possible to investigate the effects of the systematic
error in IP by creating maps using the upper and lower limits on the
IP value. To do this it is needed to add "ipoffset" a configuration 
parameter in makemap which can be used to do such an investigation. To
use it, tun pol2map twice as follows:

\begin{terminalv}
% pol2map config="ipoffset=-0.25"
% pol2map config="ipoffset=0.25"
\end{terminalv}

This produces maps using the upper and lower IP limits (a range of
0.5\%). If pol2map has already been run on POL-2 data then a file 
will already exist that was created using the 
mean IP (the mean IP is used if you omit ipoffset from the config
value, or omit the  config parameter completely). 

%\section{\xlabel{simulations}Simulated data}



