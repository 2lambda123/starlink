\documentclass[twoside,11pt,nolof]{starlink}

% ? Specify used packages
% ? End of specify used packages

% -----------------------------------------------------------------------------
% ? Document identification
\stardoccategory    {Starlink Cookbook}
\stardocinitials    {SC}
\stardocsource      {sc4.4}
\stardocnumber      {4.4}
\stardocauthors     {Malcolm J. Currie}
\stardocdate        {2006 November 26}
\stardoctitle       {C-shell Cookbook}
\stardocversion     {Version 1.3}
\stardocabstract  {
This cookbook describes the fundamentals of writing scripts using
the UNIX C shell.  It shows how to combine Starlink and private applications
with shell commands and constructs to create powerful and time-saving
tools for performing repetitive jobs, creating data-processing
pipelines, and encapsulating useful recipes.  The cookbook aims to give
practical and reassuring examples to at least get you started without
having to consult a UNIX manual.  However, it does not offer a
comprehensive description of C-shell syntax to prevent you from
being overwhelmed or intimidated.  The topics covered are: how to run a
script, defining shell variables, prompting, arithmetic and string
processing, passing information between Starlink applications, obtaining
dataset attributes and FITS header information, processing multiple
files and filename modification, command-line arguments and options, and
loops.  There is also a glossary.}

% ? End of document identification
% -----------------------------------------------------------------------------
% -----------------------------------------------------------------------------
% ? Document-specific \providecommand or \newenvironment commands.

% Also define the html equivalents.

% degrees symbol
\providecommand{\dgs}{\hbox{$^\circ$}}
%


% arcminute symbol
\providecommand{\arcm}{\hbox{$^\prime$}}
%


% arcsec symbol
\providecommand{\arcsec}{\arcm\hskip -0.1em\arcm}
%


% hours symbol
\providecommand{\hr}{\hbox{$^{\rm h}$}}



% minutes symbol
\providecommand{\mn}{\hbox{$^{\rm m}$}}



% seconds symbol
\providecommand{\scn}{\hbox{$^{\rm s}$}}



% decimal-minutes symbol
\providecommand{\um}{\hskip-0.3em\hbox{$^{\rm m}$}\hskip-0.08em}
%


% decimal-degree symbol
\providecommand{\udeg}{\hskip-0.3em\dgs\hskip-0.08em}
%


% decimal-second symbol
\providecommand{\us}{\hskip-0.27em\hbox{$^{\rm s}$}\hskip-0.06em}



% decimal-arcminute symbol
\providecommand{\uarcm}{\hskip-0.28em\arcm\hskip-0.04em}
%


% decimal-arcsecond symbol
\providecommand{\uarcs}{\hskip-0.27em\arcsec\hskip-0.02em}
%


% centre an asterisk
\providecommand{\lsk}{\raisebox{-0.4ex}{\rm *}}

% conditional text
\providecommand{\latexelsehtml}[2]{#1}



\hyphenation{which-ever}

% Lines for breaking up Appendices A and B.
\providecommand{\jrule}{\noindent\rule{\textwidth}{0.45mm}}
\providecommand{\krule}{\vspace*{-1.5ex}
                    \item [\rm \rule{\textwidth}{0.15mm}]}

% Shorthands for hypertext links.
% -------------------------------
\providecommand{\AGIref}{\xref{AGI}{sun48}{}}
\providecommand{\ARDref}{\xref{ARD}{sun183}{}}
\providecommand{\ASTERIXref}{\xref{{\footnotesize ASTERIX}}{sun98}{}}
\providecommand{\CCDPACKref}{\xref{{\footnotesize CCDPACK}}{sun139}{}}
\providecommand{\CGSDRref}{\xref{{\footnotesize CGS4DR}}{sun27}{}}
\providecommand{\CONVERTref}{\xref{{\footnotesize CONVERT}}{sun55}{}}
\providecommand{\ECHOMOPref}{\xref{{\footnotesize ECHOMOP}}{sun152}{}}
\providecommand{\ESPref}{\xref{{\footnotesize ESP}}{sun180}{}}
\providecommand{\Figaroref}{\xref{{\footnotesize FIGARO}}{sun86}{}}
\providecommand{\FITSref}{\htmladdnormallink{FITS}{http://fits.gsfc.nasa.gov/fits\_{}home.html}}
\providecommand{\GKSref}{\xref{GKS}{sun83}{}}
\providecommand{\GWMref}{\xref{GWM}{sun130}{}}
\providecommand{\HDSref}{\xref{HDS}{sun92}{}}
\providecommand{\HDSTRACEref}{\xref{{\footnotesize HDSTRACE}\normalsize}{sun102}{}}
\providecommand{\ICLref}{\xref{{\footnotesize ICL}}{sg5}{}}
\providecommand{\IDIref}{\xref{IDI}{sun65}{}}
\providecommand{\IRCAMPACKref}{\xref{{\footnotesize IRCAMPACK}}{sun177}{}}
\providecommand{\IRAFref}{\htmladdnormallink{{\footnotesize IRAF}}{http://iraf.noao.edu/iraf-homepage.html}}
\providecommand{\IRASref}{\xref{{\footnotesize IRAS90}}{sun163}{}}
\providecommand{\JCMTDRref}{\xref{{\footnotesize JCMTDR}}{sun132}{}}
\providecommand{\KAPPAref}{\xref{{\footnotesize KAPPA}}{sun95}{}}
\providecommand{\NDFrefa}[1]{\xref{#1}{sun33}{}}
\providecommand{\PDAref}{\xref{PDA}{sun194}{}}
\providecommand{\PGPLOTref}{\htmladdnormallink{PGPLOT}{http://astro.caltech.edu/\~{}tjp/pgplot/}}
\providecommand{\PHOTOMref}{\xref{{\footnotesize PHOTOM}}{sun45}{}}
\providecommand{\PISAref}{\xref{{\footnotesize PISA}}{sun109}{}}
\providecommand{\PONGOref}{\xref{{\footnotesize PONGO}}{sun137}{}}
\providecommand{\PSMERGEref}{\xref{{\footnotesize PSMERGE}}{sun164}{}}
\providecommand{\SGSref}{\xref{SGS}{sun85}{}}
\providecommand{\SPECDREref}{\xref{{\footnotesize SPECDRE}}{sun140}{}}
\providecommand{\TSPref}{\xref{{\footnotesize TSP}}{sun66}{}}
\providecommand{\TWODSPECref}{\xref{{\footnotesize TWODSPEC}}{sun16}{}}

% ? End of document-specific commands
%------------------------------------------------------------------------------


%  Title Page.
%  ===========
\begin{document}
\scfrontmatter
\thispagestyle{empty}

% The main text begins here.
% -----------------------------------------------------------------------------

\section{\xlabel{sc4_se_intro}Introduction\label{sc4_se_intro}}

Scripting is a powerful and time-saving tool for performing repetitive
jobs, creating data-processing pipelines, and encapsulating useful
recipes.  A script is a text file containing a set of
\htmlref{\textsf{shell}}{sc4_gl_she} commands and constructs that perform
a routine task.  The former can include UNIX commands like \textbf{mv},
\textbf{cd}, \textbf{awk}, \textbf{sed}; other scripts; and private and
Starlink applications.  You can create and modify scripts with a text
editor.

Although UNIX purists recommend the Bourne or \textbf{bash} shell for
scripts and indeed some consider \htmladdnormallink{C-shell
programming
harmful}{http://www.unix.org.ua/orelly/unix/upt/ch47_02.htm}, many
scientists who are occasional programmers find the familiar C-like syntax
more approachable and easier to learn for scripting.  Whereas young
Turks would advocate the increasingly popular \htmladdnormallink{{\bf
perl}}{http://www.perl.com/} and
\htmladdnormallink{Python}{http://www.python.org/} languages, a 1996
survey of Starlink users placed C-shell scripting near the head of
required cookbooks.  In addition most Starlink commands are available
as C-shell aliases.  This cookbook applies to both the C-shell {\bf
csh} and its variants like the tc-shell \textbf{tcsh}.

This manual illustrates some relevant techniques for creating C-shell
scripts that combine Starlink software to improve your productivity,
without you having to read a UNIX manual.  It aims to give practical and
reassuring examples to at least get you started.  It does \emph{not\/}
offer a comprehensive description of C-shell syntax and facilities.

This is not like other cookbooks in the Starlink series as the
possible recipes are limitless.  Instead of concentrating on recipes,
it therefore focuses on the various ingredients you may need for your
own creations.  Thus it has a tutorial style more akin to a guide.
However, it has been structured with the aim of letting you dip in to
find the desired ingredient.

The author welcomes your comments.  If you have ``How do I do \ldots
in the C-shell'' type questions, or suggestions for further recipes
and ingredients, please contact the Starlink Software Librarian
({\texttt{starlink@jiscmail.ac.uk}}) or the author ({\texttt{mjc@star.rl.ac.uk}}), so that
important techniques omitted from this version may be included in future
editions.

\section{Further Reading}

If you want a comprehensive description of C-shell syntax and
facilities; check the \textbf{man} pages for \textbf{csh}.  Books
exclusively on the C-shell are not as commonplace as you might expect;
one such is {\sl Teach Yourself the Unix C shell in 14 days} by David
Ennis \& James C. Armstrong (SAMS Publishing, Indianapolis, 1994).
While there are plenty of UNIX books, they tend to give spartan
coverage of the C-shell, often concentrating on the interactive
aspects; in many the examples are sparse.  One that bucks the trend is
{\sl UNIX Shells by Example} by Ellie Quigley (Prentice-Hall, New
Jersey, 1997).  This has numerous examples, where the function of each
line is explained.  C-shell is included, and there are chapters on the
tools of the trade like \textbf{awk} and \htmlref{regular
expressions}{sc4_gl_reg_exp}.  The chapter entitled \textit{Shell
Programming\/} in {\sl UNIX for VMS Users\/} by Philip E. Bourne
(Digital Press, 1990) is well worth a read, especially for those of
you who developed command procedures in Starlink's VMS era, and want
to convert them to UNIX scripts.  Chapter 49 of {\sl UNIX Power Tools}
by Jerry Peek, Tim O'Reilly, \& Mike Loukides (O'Reilly \& Associates,
1993) has useful summaries and describes some problems with the
C-shell.

\newpage

\section{How to use this manual}

It is not necessary to read this manual from cover to cover.  Indeed
some of it will be repetitive if you do.  That's deliberate so that
once you have the basics, you can then look up a particular ingredient
without having to read lots of other parts of the cookbook.

Before you write any scripts you should look at
\latexelsehtml{Sections~\ref{sc4_se_running} to \ref{sc4_se_starlink_app}
(except perhaps \ref{sc4_se_package})}{\htmlref{Running a
script}{sc4_se_running}, \htmlref{Some simple examples}{sc4_se_simple},
\htmlref{Shell Variables}{sc4_se_variables}, and
\htmlref{Executing a Starlink Application}{sc4_se_starlink_app}}.

\section{Conventions}
\begin{tabular}{lp{120mm}}
\texttt{verbatim}   &  Commands you enter, filenames, parameters values
                    you supply to scripts or tasks appear in teletype
                    fount. \\
\textbf{command}    &  The names of UNIX or Starlink commands appear
                    in bold. \\
\textsf{term}       &  Special terms explained in the glossary appear
                    in the sans-serif fount. \\
\end{tabular}

Commands to be entered from the terminal are prefixed with a \texttt{\%}
prompt string.  You should not type the \texttt{\%}.

\newpage
\section{\xlabel{sc4_se_running}Running a script\label{sc4_se_running}}

Unfortunately, we must start with some boring, but important technical
stuff.  This will tell you how to run your script using the C-shell.  The
stages are to select the shell, give execute access to the script, and
then actually invoke the script using one of three ways.

\subsection{\xlabel{sc4_se_cshselect}Shell selection\label{sc4_se_cshselect}}

The \emph{first line\/} of your script tells UNIX which shell you want
to use.  The following selects the C-shell.

\begin{small}
\begin{terminalv}
     #!/bin/csh
\end{terminalv}
\end{small}
Somewhat perversely, the \texttt{\#} is actually the comment character.
That's UNIX for you.  However, its presence alone starting the first
line will normally be sufficient to run a script in the C shell.

Most of the examples in this document are script excerpts, and so do not
include the \mbox{\texttt{\#!/bin/csh}}.

\subsection{\xlabel{sc4_se_executable}Making a script executable
\label{sc4_se_executable}}

If you want to run your script by name or with an \htmlref{{\sf
alias}}{sc4_gl_alias}\latex{ (see below)}, you must make your
script executable like this

\begin{small}
\begin{terminalv}
     % chmod +x myscript
\end{terminalv}
\end{small}
where \texttt{myscript} is your C-shell script. \latexelsehtml{Remember}{Note}
that the \texttt{\%} is a convention for the shell prompt; you do not type
it yourself.  You can edit the script without having to make the file
executable again.

\subsection{\xlabel{sc4_se_byname}Executing a script by name
\label{sc4_se_by_name}}

Once your script has execute privilege, thereafter you can
run it like this:

\begin{small}
\begin{terminalv}
     % ./myscript
\end{terminalv}
\end{small}
or if it is situated in directory \texttt{/home/user1/dro/bin} say

\begin{small}
\begin{terminalv}
     % /home/user1/dro/bin/myscript
\end{terminalv}
\end{small}
would execute your script.  This is long-winded if you want to run the
the script frequently.  To omit the directory \htmlref{{\sf
path}}{sc4_gl_path} you need to add the current ({\texttt{.}}) or better,
the specific directory to your PATH
\htmlref{\textsf{environment variable}}{sc4_gl_env}.


\subsection{\xlabel{sc4_se_aliases}Using aliases to run a script
\label{sc4_se_aliases}}

The second method is to define an \htmlref{\textsf{alias}}{sc4_gl_alias}.
An alias is a shorthand for a long command to save typing.
So in fact this method is just a variant of executing a script by name.

\begin{small}
\begin{terminalv}
     % alias myscript /home/user1/dro/bin/myscript
\end{terminalv}
\end{small}
Thereafter you would only enter \texttt{myscript} to run your script. Now
this alias only applies to your \htmlref{\textsf{current
process}}{sc4_gl_cur}. Even child \htmlref{\textsf{processes}}{sc4_gl_pro}
derived from the current process will not inherit the alias.  The way
to make an alias global, so each new C-shell process recognises the
alias, is to define the alias in your \texttt{\$HOME/.cshrc} file.  This
file in your login directory defines the `environment' for each
C-shell process.

Here is a simple {\texttt{.cshrc}} file.  It sets some global variables:
\texttt{noclobber} prevents accidentally overwriting an existing file and
\texttt{history} sets the number of previous commands you can recall.  It
defines three aliases.  The first lets you browse through a
directory listing.  The second reformats the output from \textbf{df}
on Digital UNIX so that lines do not wrap.  The third lists the largest
20 files in a directory.  Finally \texttt{.cshrc} runs the script
\texttt{/star/etc/cshrc} to define Starlink aliases (the \textbf{source}
command is explained \latexelsehtml{in
Section~\ref{sc4_se_source}).}{\htmlref{here}{sc4_se_source}).}

\begin{small}
\begin{terminalv}
     set noclobber
     set history = 50
     alias dsd 'ls -l \!* | more'
     alias dff "df \!* |awk \
      '{printf("\"%-20.20s%9.8s%9.8s%9.8s%9.8s %s\\n\",'$1,$2,$3,$4,$5,$6)}' "'"
     alias big20 'ls -l | sort -k 5 | tail -n 20'
     source /star/etc/cshrc
\end{terminalv}
\end{small}

\subsection{\xlabel{sc4_se_unalias}Removing aliases
\label{sc4_se_unalias}}

There are two methods.  One is permanent; the other overrides an alias
for a single command.  To remove an alias permanently use the {\tt
unalias} command.  This accepts \texttt{*?[~]}
\htmlref{\textsf{wildcards}}{sc4_gl_wild} to \htmlref{\textsf{match}}{sc4_gl_match} the
defined aliases.

\begin{small}
\begin{terminalv}
                             # Removes aliases called:
     % unalias myscript      # myscript
     % unalias kap_*         # kap_ followed by zero or more characters
     % unalias compres?      # compres followed by a single character
     % unalias [f-hz][2-5]   # One of f, g, h, or z then an integer
                             # between 2 and 5 inclusive
\end{terminalv}
\end{small}

To override an alias, precede the alias with backslash.  So suppose you
have an alias to prevent you accidently removing a file as shown below.

\begin{small}
\begin{terminalv}
     % alias rm rm -i
\end{terminalv}
\end{small}
In a script where you know that you want to remove a file, you don't
want the script to seek confirmation.  So suppose you want to delete
file \texttt{myfile}, then you would have a line like

\begin{small}
\begin{terminalv}
     \rm myfile
\end{terminalv}
\end{small}
in your script.


\subsection{\xlabel{sc4_se_source}Executing a script in the current process
\label{sc4_se_source}}

The final option is to \textbf{source} your script so that it runs in the
\htmlref{\textsf{current process}}{sc4_gl_cur}.  The benefit of this technique is
that any aliases defined in the current process will be known to your
script.  For example,

\begin{small}
\begin{terminalv}
     % source myscript
\end{terminalv}
\end{small}
runs \texttt{myscript}.  Any aliases created in \texttt{myscript} will be
available to the current process, unlike invocation by name.
To save typing you can define an alias to source a script.  See the next
section on package aliases for an example.

\subsection{\xlabel{sc4_se_package}Package aliases\label{sc4_se_package}}

While convenient, the creation of aliases in your \texttt{.cshrc} file
does have a drawback: if you define many aliases in the \texttt{.cshrc}
file, it will decelerate \htmlref{\textsf{process}}{sc4_gl_pro} activation.
One way around that is to define a few aliases that run other scripts,
each of which in turn define many related aliases.  Thus you only
create the definitions when they are required.  This is how most
Starlink packages define their commands.  Here is an example.  Suppose
you had a package or a set of related commands called COSMIC installed
in directory \texttt{/home/user2/dro/cosmic}, you would first place the
following line in your \texttt{.cshrc} file.

\begin{small}
\begin{terminalv}
     alias cosmic `source /home/user2/dro/cosmic/cosmic.csh`
\end{terminalv}
\end{small}
The left quotes mean execute the command between them.  So when you
enter

\begin{small}
\begin{terminalv}
     % cosmic
\end{terminalv}
\end{small}
the script \texttt{/home/user2/dro/cosmic/cosmic.csh} is run.  This file
might look like the following.

\begin{small}
\begin{terminalv}
     #!/bin/csh
     alias abund /home/user2/dro/cosmic/abundance
     alias filter `source /home/user1/abc/bin/garbage.csh`
     alias kcorr /home/user2/dro/cosmic/k-correction.csh
     alias radial /home/user2/drmoan/noddy/radial
     alias seeing $KAPPA_DIR/psf isize=21 psf=psf21
     alias zcol kcorr bands=UJR noevol
\end{terminalv}
\end{small}

\newpage
\section{\xlabel{sc4_se_simple}Some simple examples\label{sc4_se_simple}}

Let's start with a few elementary scripts to show that script writing
isn't difficult, and to illustrate some syntax and constructs.

Suppose that we have an editor such as \textbf{jed} which retains a copy
of the previous version of a file by appending \texttt{$\sim$} to the
filename.  If at some time we realise that we really want the former
version and to erase the new one we could have a script called
\texttt{restore} that would perform the operation for an arbitrary file.
It might look something like this.

\begin{small}
\begin{terminalv}
     #!/bin/csh
     rm $1
     mv $1~ $1
\end{terminalv}
\end{small}
The first line demands that the C-shell be used.
The \texttt{\$1} represents the first argument supplied on the command line.
(There is more on this in
\latexelsehtml{Section~\ref{sc4_se_arguments}.)}{\htmlref{Script
arguments and other special variables}{sc4_se_arguments}.)}
If the command \texttt{restore} runs the script, then entering

\begin{small}
\begin{terminalv}
     % restore splat.f
\end{terminalv}
\end{small}
would remove \texttt{splat.f} and replace it with \texttt{splat.f$\sim$}.

The above script makes some assumptions.  First it does not check
whether or not the files exist.  Let's correct that.

\begin{small}
\begin{terminalv}
     #!/bin/csh
     if ( -e $1 && -e $1~ ) then
        rm $1
        mv $1~ $1
     endif
\end{terminalv}
\end{small}

Here we have introduced the \textbf{if...then...endif} construct.  When
the expression inside the parentheses is true, the following statements
are executed until there is an \textbf{else} or \textbf{endif} statement.
In this case \texttt{-e \$1 \&\& -e \$1~} is true only when both files exist.
The \texttt{-e} is one of eight \textsf{file operators} for such things as
file access and type, and \texttt{\&\&} is a logical AND.

The second assumption in \texttt{restore} is that you'll always remember
to supply a file name.

\begin{small}
\begin{terminalv}
     #!/bin/csh
     if ( $#argv == 0 ) then
        echo Error: no file name supplied
     else if ( -e $1 && -e $1~ ) then
        rm $1
        mv $1~ $1
     endif
\end{terminalv}
\end{small}
The script now reports an error message ~\mbox{\texttt{Error:~no file name
supplied}} if you forget the argument.  \texttt{\$\#argv} is the number of
arguments supplied on the command line.  Also notice the \textbf{else if}
construct.  The \texttt{==} tests for equality.

Instead of an error message you could make the script prompt for a
filename.

\begin{small}
\begin{terminalv}
     #!/bin/csh
     if ( $#argv == 0 ) then
        echo -n "The file to restore >"
        set file = $<
     else
        set file = $1
     endif
     if ( -e $file && -e $file~ ) then
        rm $file
        mv $file~ $file
     else
        if ( ! -e $file ) then
           echo Error: $file does not exist
        endif
        if ( ! -e $file~ ) then
           echo Error: $file~ does not exist
        endif
     endif
\end{terminalv}
\end{small}

When the number of arguments is zero, this script issues the prompt
\texttt{The file to restore >}.  The \texttt{echo -n} prevents a skip to a
new line, and so allows you to enter your response after the prompt
string.  \texttt{set file \$<} equates your response to a \htmlref{\textsf{shell
variable}}{sc4_gl_she_var} called \texttt{file}.  If on the other hand you gave the
filename on the command line \texttt{set file = \$1} assigns that name to
variable \texttt{file}.  When we come to use the variable in other commands
it is prefixed with a dollar, meaning ``the value of the variable''.
Finally, the script now tells you which file or files are missing.
Notice how the variable value can be included in the error message.

For your private scripts you need not be as rigorous as this, and in
many cases the simplest script suffices.  If you intend to use a script
frequently and/or sharing with colleagues, it's worth making a little
extra effort such as adding commentary.  Comments will help \emph{you\/}
remember what a lengthy script does after you've not run it for a while.
It might seem tedious at the time, but you will thank yourself when you
come to run it after a long interval.  Comment lines begin with {\tt
\#}.  There are comment examples in some of the longer scripts later in
the cookbook.

\newpage
\section{\xlabel{sc4_se_variables}Shell Variables\label{sc4_se_variables}}

The shell lets you define variables to perform calculations, and to
pass information between commands and applications.  They are either
integers or strings, however it is possible to perform floating-point
arithmetic by using certain applications in your script (see
\latexelsehtml{Section~\ref{sc4_se_real}).}{\htmlref{here}{sc4_se_real}).}
You don't need to declare the data type explicitly; the type is
determined when you assign a value to a variable.

Variables are only defined in the \htmlref{\textsf{current
process}}{sc4_gl_cur}.  If you want global variables, \textit{i.e.}\ ones
that are available to all processes, you should assign them in your
\texttt{.cshrc} file.

Variable names comprise up to 20 letters, digits, and underscores;
and should begin with a letter or underscore.

\subsection{\xlabel{sc4_se_assign_scalar}Assigning scalar values
\label{sc4_se_assign_scalar}}

You assign a value to a variable using the \textbf{set} command.

\begin{small}
\begin{terminalv}
     set colour = blue
     set title = "Chiaroscuro 1997"
     set caption = "The most distant quasar observed is at redshift "
     set flux_100 = 1.2345E-09
     set n = 42
     set _90 = -1
\end{terminalv}
\end{small}
The first four examples assign strings, and the last two assign integer
values.  Yes the value of \texttt{flux\_100} is not a real number.
Multi-word strings should be enclosed in \texttt{"} quotes.  The spaces
around the equals sign are necessary.

You can also remove variables with the \textbf{unset} command.  This
accepts \texttt{*?[~]} \htmlref{\textsf{wildcards}}{sc4_gl_wild} to
\htmlref{\textsf{match}}{sc4_gl_match} the names of the
\htmlref{shell variables}{sc4_se_variables}.

\begin{small}
\begin{terminalv}
     unset colour
     unset iso_*             # iso_ followed by zero or more characters
     unset flux_1??          # flux_1 followed by any pair of characters
     unset [nx-z][0-9]*      # One of n, x, y, or z then an integer followed
                             # by zero or more characters
\end{terminalv}
\end{small}


\subsection{\xlabel{sc4_se_assign_array}Assigning arrays
\label{sc4-se:assign_array}}

The \textbf{set} command is again used but the elements are space-separated
lists enclosed by parentheses.  Somewhat surprisingly for a shell that
mimics C, array elements start at 1, like Fortran.
Here are some illustrations.

\begin{small}
\begin{terminalv}
     set colours = (blue yellow green red pink)
     set label = ("Count rate" Frequency "Integration time (sec)")
     set prime = (2 3 5 7 11 13 17 19 23)
\end{terminalv}
\end{small}
The first element of \texttt{colours} is \texttt{"blue"}, the second is {\tt
"yellow"} and so on.  Multi-word elements must be in \texttt{"} quotes.  So
the first element of \texttt{label} is \texttt{"Count rate"}, and the second
is \texttt{"Frequency"}.  The seventh element of \texttt{prime} is \texttt{17}.

\subsection{\xlabel{sc4_se_values}Using the values of variables
\label{sc4_se_values}}

To obtain the value of a variable, you prefix the variable's name with
the dollar sign.

\begin{small}
\begin{terminalv}
     set count = 333
     echo Number of runs is $count
\end{terminalv}
\end{small}
would write ~\texttt{Number of runs is 333}~ to \htmlref{\textsf{standard
output}}{sc4_gl_std_out}.

\begin{small}
\begin{terminalv}
     set caption = "The most distant quasar observed is at redshift "
     set z = 5.1
     echo $caption$z
\end{terminalv}
\end{small}
This will echo ~\texttt{The most distant quasar observed is at redshift 5.1}.

\begin{small}
\begin{terminalv}
     if ( $n > 10 ) then
        mem2d niter=$n out=deconv
        display style="'title=Result of deconvolution after $n iterations'" accept
     endif
\end{terminalv}
\end{small}
This tests whether the value of variable \texttt{n} is greater than ten, and
if it is, executes the statement within the \textbf{if\ldots endif}
structure.  The value is also passed into an integer parameter of the
application \xref{\textbf{mem2d}}{sun95}{MEM2D}; and a string parameter of application
\xref{\textbf{display}}{sun95}{DISPLAY}, so if \texttt{n} were forty, the resultant value would
be \texttt{"Result of deconvolution after 40 iterations"}.

Arithmetic, logical and string operations are presented in
\latexelsehtml{Sections~\ref{sc4_se_csharith} and \ref{sc4_se_string_proc}.
}{
\htmlref{Arithmetic}{sc4_se_csharith} and
\htmlref{String Processing}{sc4_se_string_proc}.}

The values of array elements are selected by appending the element
index in brackets.  This would echo to \htmlref{\textsf{standard
output}}{sc4_gl_std_out} the string ~\texttt{Starting search at
co-ordinates x=300, y=256.}

\begin{small}
\begin{terminalv}
     set coords = (300 256)
     echo "Starting search at co-ordinates x=$coords[1], y=$coords[2]."
\end{terminalv}
\end{small}

The following illustrates use of a text array to specify the
annotations of a plot created with application
\xref{\textbf{linplot}}{sun95}{LINPLOT}.
\begin{small}
\begin{terminalv}
     set labels = ("Elapsed time" Flux "Light curve of 3C273")
     linplot style="'title=$labels[3],Label(1)=$labels[1],Label(2)=$labels[2]'"
\end{terminalv}
\end{small}

\medskip
There are some shorthands for specifying subsets of an array.  The best
way to describe them is through some examples.
All the values assume \texttt{set prime = (2 3 5 7 11 13 17 19 23)}.

\begin{center}
\begin{tabular}{lp{80mm}l}
Syntax            & Meaning                    & Value \\ \hline
\\
\texttt{\$\#prime}   & Number of elements of variable \texttt{prime} & \texttt{9} \\
\texttt{\$prime[*]}  & All elements of variable \texttt{prime} & \texttt{2 3 5 7 11 13
                                                           17 19 23} \\
\texttt{\$prime[\$]} & The last element of variable \texttt{prime} & \texttt{23} \\
\texttt{\$prime[3-5]} & The third to fifth elements of variable \texttt{prime} &
                                                           \texttt{5 7 11} \\
\texttt{\$prime[8-]} & The eighth to last elements of variable \texttt{prime} &
                                                           \texttt{21 23} \\
\\ \hline
\end{tabular}
\end{center}
\bigskip

Here we bring some of these ingredients together.  Suppose we are
experimenting trying to find the most suitable reddish background
({\texttt{palnum=0}}) colour for image display using the
\xref{\textbf{palentry}}{sun95}{PALENTRY} command
from \KAPPAref\@.  The script below loops using a \textbf{while...end}
construct: all the commands inside the construct are repeated until the
condition following the \textbf{while} is satisfied.  That might seem a
little odd here as we appear to repeat the same commands \emph{ad
infinitum} because the number of colours is fixed.  That doesn't happen
because of the C-shell \textbf{shift} command.  It discards the first
element of an array and reduces the indices of the remaining elements by
one, so the original \texttt{\$colours[2]} becomes \texttt{\$colours[1]} and
so on.  This means that the chosen background colour is always {\tt
\$colours[1]}.  The \textbf{shift} also decrements the number of colours by
one.  So the script changes the background colour and pauses for five
seconds for you to consider the aesthetics.

\begin{small}
\begin{terminalv}
     set colours = (red coral hotpink salmon brown sienna tan)
     while ( $#colours > 0 )
        echo "Trying $colours[1]"
        palentry palnum=0 colour=$colours[1]
        sleep 5
        shift colours
     end
\end{terminalv}
\end{small}


\subsection{\xlabel{sc4_se_spec_char}Special characters
\label{sc4_se_spec_char}}

The shell has a number of special characters otherwise known as
\htmlref{\textsf{metacharacters}}{sc4_gl_met}.  These include
\htmlref{\textsf{wildcard}}{sc4_gl_wild} characters such as \texttt{*?[~]},
single and double quotes, parentheses, and the \verb+\+ line
continuation.

If you assign a value to a variable or application parameter which
includes one or more of these metacharacters, you need to switch off
the special meanings.  This operation is called escaping.
Single characters may be escaped with a backslash.  For multiple
occurrences single quotes \texttt{'~'} will pass the enclosed text
verbatim.  Double quotes \texttt{"~"} do the same as single quotes except
that \texttt{\$}, \verb+\+, and left quote \texttt{`} retain their special
meaning.

\begin{small}
\begin{terminalv}
     setlabel label=\"Syrtis Major\" \\
     set metacharacters = '[]()/&><%$|#`@''"'
     stats europa'(200:299,~120)'
     stats europa"(200:299,~$y)"
\end{terminalv}
\end{small}
In the first example the double quotes are part of parameter PLTITL
(needed because of the embedded space) so are escaped individually.  On
its own \verb+\+ means continue a line, but for Starlink tasks it is
shorthand for the \texttt{accept} keyword.  So we have to tell the shell to
treat the backslash literally by preceding it with backslash!

In the last pair of examples an \xref{\textsf{NDF section}}{sun95}{se_ndfsect}
\latex{(see SUN/95's chapter called ``NDF Sections'')}
is specified.  As the last contains a variable value to be
substituted, the \texttt{\$} retains its normal special meaning but the
parentheses are escaped by surrounding the section in double quotes.

\newpage
\subsection{\xlabel{sc4_se_prompting}Prompting\label{sc4_se_prompting}}

Some scripts might have parameters that cannot be defaulted, and so
if their values are not given on the command line, the script
needs to prompt for them.

\begin{small}
\begin{terminalv}
     echo -n "Give the smoothing size"
     set size = $<
\end{terminalv}
\end{small}
This will prompt for the parameter and store the entered value into
\htmlref{shell variable}{sc4_se_variables} \texttt{size}.

\subsection{\xlabel{sc4_se_arguments}Script arguments and other special
variables\label{sc4_se_arguments}}

The C-shell has some special forms of variable.  We have seen some
already: the \texttt{\$<} for prompting, and how to specify array
elements.  The remainder largely concern command-line arguments.
Here they are tabulated.

\begin{center}
\begin{tabular}{lp{90mm}}
Syntax            & Meaning \\ \hline
\\
\texttt{\$\{0\}}     & The name of the script being run \\
\texttt{\$?name}     & Returns \texttt{1} if the variable \texttt{name} is
                    defined, or \texttt{0} if it is not defined \\
\texttt{\$n}         & The value of the \texttt{n}$^{\rm th}$ argument passed
                    to the script \\
\texttt{\$argv[n]}   & The value of the \texttt{n}$^{\rm th}$ argument passed
                    to the script \\
\texttt{\$\#argv}    & The number of arguments passed to the script \\
\texttt{\$*}         & All the arguments supplied to the script \\
\texttt{\$\$}        & \htmlref{\textsf{Process identification number}}{sc4_gl_pid}
                    (useful for making temporary files with unique names) \\
\\ \hline
\end{tabular}
\end{center}
Thus inside a script, any command-line arguments are accessed as
\htmlref{shell variables}{sc4_se_variables} \texttt{\$1}, \texttt{\$2} \ldots \texttt{\$\#argv}.
There is no practical limit to the number of arguments you can supply.

Let's look at a few examples.  Below is a script \texttt{argex.csh}.
\begin{small}
\begin{terminalv}
     #! /bin/csh
     echo ${0}
     echo "Number of arguments is $#argv"
     echo $2
     echo $argv[2-3]
     echo $argv[$]
     exit
\end{terminalv}
\end{small}
Then we run it with four arguments.

\begin{small}
\begin{terminalv}
     % argex.csh "hello world" 42 3.14159 "(300:400,~100)"
     argex.csh
     Number of arguments is 4
     42
     42 3.14159
     (300:400,~100)
\end{terminalv}
\end{small}
Note that you must enclose any argument passed to a script that contains
a space or shell \htmlref{\textsf{metacharacter}}{sc4_gl_met} in double quotes ({\texttt{"}}).

You can see an example using \texttt{\$*} to process a list of files
\latexelsehtml{in Section~\ref{sc4_se_wildcard_lists}.
}{\htmlref{here}{sc4_se_wildcard_lists}.}  There are other examples of script
arguments in \latexelsehtml{Section~\ref{sc4_se_unix_options}.
}{\htmlref{UNIX options}{sc4_se_unix_options}.}

\subsection{\xlabel{sc4_se_predefined}Predefined variables
\label{sc4_se_predefined}}

The C-shell has some predefined variables.  We've already met {\tt
argv}.  The most useful other ones are listed below.

\begin{center}
\begin{tabular}{lp{100mm}}
Shell variable & Meaning \\ \hline
\\
\texttt{cwd}      & Current working directory \\
\texttt{home}     & Home directory \\
\texttt{path}     & List of directories in which to search for commands \\
\texttt{status}   & The status returned by the last command, where \texttt{0}
                 means a successful completion, and a positive integer
                 indicates an error, the higher the value, the higher
                 the severity. \\
\texttt{user}     & Username \\
\\ \hline
\end{tabular}
\end{center}

So if you wanted to include details of the user and directory at the
head of some data-processing log you might write this in your script.

\begin{small}
\begin{terminalv}
     % echo "Processed by:        $user" > logfile
     % echo "Directory processed: $path" >> logfile
\end{terminalv}
\end{small}
This writes a heading into file \texttt{logfile} saying who ran the
script, and then appends details of the directory in which processing
occurred. Note the different \htmlref{\textsf{metacharacters}}{sc4_gl_met},
namely \texttt{>} redirects the output to a file, and \texttt{>>} appends to
an existing file.

\newpage
\section{\xlabel{sc4_se_starlink_app}Executing a Starlink Application
\label{sc4_se_starlink_app}}

Running Starlink tasks from a script is much the same as running them
interactively from the shell prompt.  The commands are the same.  The
difference for shell use is that you should provide values on the
command line (directly or indirectly) for parameters for which you
would normally be prompted.  You may need to rehearse the commands
interactively to learn what parameter values are needed.
Although there is less typing to use a positional parameter for the
expression, it's prudent to give full parameter names in scripts.
Positions might change and parameter names are easier to follow.
\CURSAref\ is an exception.  For this package you should list the
answers to prompts in a file as described in
\latexelsehtml{Section~\ref{sc4_se_reading_files}.}{\htmlref{reading
dynamic text files}{sc4_se_reading_files}.}

The script must recognise the package commands.  The options for
enabling this are described below.  Then you can run Starlink
applications from the C-shell script by just issuing the commands as if
you were prompted.  You do \emph{not} prefix them with any special
character, like the \texttt{\%} used throughout this manual.

If you already have the commands defined in your current shell, you
can \textbf{source} your script so that it runs in that shell, rather than
in a child \htmlref{\textsf{process}}{sc4_gl_pro} derived from it.  For instance,

\begin{small}
\begin{terminalv}
     % source myscript test
\end{terminalv}
\end{small}
will run the script called \texttt{myscript} with argument \texttt{test} using
the current shell environment; any package definitions currently
defined will be known to your script.  This method is only suitable
for quick one-off jobs, as it does rely on the definition aliases being
present.

The recommended way is to invoke the startup scripts, such as {\tt
kappa}, \texttt{ccdpack} within the script.  The script will take a
little longer to run because of these extra scripts, but it will be
self-contained.  To prevent the package startup message appearing you
could temporarily redefine \textbf{echo} as shown here.

\begin{small}
\begin{terminalv}
     alias echo "echo > /dev/null"
     kappa
     ccdpack
     unalias echo
\end{terminalv}
\end{small}
In traditional UNIX style there is a third option: you could add the
various directories containing the executables to your PATH
\htmlref{\textsf{environment variable}}{sc4_gl_env}, however this will not
pick up the synonym commands.

\begin{small}
\begin{terminalv}
     setenv PATH $PATH:/home/user1/dro/bin:/home/user2/drmoan/noddy
\end{terminalv}
\end{small}
As most of the examples in this document are script excerpts, and for
reasons of brevity, most do not define the package commands explicitly.

\newpage
\subsection{\xlabel{sc4_se_multi_param}Parameter files and the
graphics database}\label{sc4_se_multi_param}

If you run simultaneously more than one shell script executing
Starlink applications, or run such a script in the background while
you continue an interactive session, you may notice some strange
behaviour with parameters.   Starlink applications uses files in
the directory \texttt{\$ADAM\_USER} to store parameter values.  If you don't
tell your script or interactive session where this is located,
tasks will use the same directory.  To prevent sharing of the
parameter files use the following tip.

\begin{terminalv}
     #!/bin/csh
     mkdir /user1/dro/vela/junk_$$
     setenv ADAM_USER /user1/dro/vela/junk_$$

     <main body of the script>

     \rm -r /user1/dro/vela/junk_$$
     # end of script
\end{terminalv}
This creates a temporary directory (\texttt{/user1/dro/vela/junk\_\$\$})
and redefines \texttt{\$ADAM\_USER} to point to it.  Both exist only while the
script runs.  The \$\$ substitutes the process identification number
and so makes a unique name.  The backslash in \verb+\rm+ overrides
any alias \texttt{rm}.

If you are executing graphics tasks which use the \xref{graphics
database}{sun95}{se_agitate}, you may also need to redefine {\tt
\$AGI\_USER} to another directory.  Usually, it is satisfactory to
equate \texttt{\$AGI\_USER} to the \texttt{\$ADAM\_USER} directory.

\subsection{\xlabel{sc4_se_exit_status}How to test whether or not a
Starlink task has failed}\label{sc4_se_exit_status}

In a typical script involving Starlink software, you will invoke
several applications.  Should any of them fail, you normally do not
want the script to continue, unless an error is sometimes expected and
your shell script can take appropriate action.  Either way you want a
test that the application has succeeded.

If you set the ADAM\_EXIT \htmlref{\textsf{environment
variable}}{sc4_gl_env} to \texttt{1} in your script before calling
Starlink applications then the \texttt{status} variable after each task,
will indicate whether or not the task has failed, where 1 means
failure and 0 success.

\begin{small}
\begin{terminalv}
     setenv ADAM_EXIT 1
         .    .    .
         .    .    .
         .    .    .
     stats allsky > /dev/null
     echo $status
     1
     stats $KAPPA_DIR/comwest > /dev/null
     echo $status
     0
\end{terminalv}
\end{small}

The NDF allsky is absent from the current directory, so \textbf{stats}
fails, reflected in the value of \texttt{status}, whereas
\$KAPPA\_DIR/comwest does exist.

Here's an example in action.
\begin{small}
\begin{terminalv}
     setenv ADAM_EXIT 1
         .    .    .
         .    .    .
         .    .    .
     normalize in1=$ndfgen in2=$ndfin out=! device=! > /dev/null
     if ( $status == 1 ) then
        echo "normalize failed comparing $ndf1 and $ndf2."
        goto tidy
     else
        set offset = `parget offset normalize`
        set scale = `parget slope normalize`
     endif
         .    .    .
         .    .    .
         .    .    .
     tidy:
     \rm ${ndfgen}.sdf
\end{terminalv}
\end{small}

The script first switches on the ADAM\_EXIT facility.  A little later
you create an NDF represented by \texttt{\$ndfgen} and then compare it with
the input NDF \texttt{\$ndfin} using \xref{\textbf{normalize}}{sun95}{NORMALIZE}.
If the task fails, you issue an error message and move to a block of
code, normally near the end of the script, where various cleaning
operations occur.  In this case it removes the generated NDF.

When \textbf{normalize} terminates successfully, the script accesses the
\htmlref{\textsf{output parameters}}{sc4_gl_opar} for later processing
with \xref{\textbf{parget}}{sun95}{PARGET}.  This is explained in
\begin{htmlonly}
\htmlonly{\htmlref{Passing information between Starlink applications}{sc4_se_info_parameter}.}
\end{htmlonly}
\latex{Section~\ref{sc4_se_passing_info}.}

\newpage
\section{\xlabel{sc4_se_passing_info}Passing information between
Starlink applications\label{sc4_se_passing_info}}

In scripts you will often want to take some result produced or
calculated by one application, and to pass that information to another.
Two techniques are available: \htmlref{\textsf{piping}}{sc4_gl_pipe} or
\htmlref{\textsf{output parameters}}{sc4_gl_opar}.

\subsection{\xlabel{sc4_se_info_parse}Parsing text output
\label{sc4_se_info_parse}}

If the first task prints the required value, it is possible to use one
or more of \textbf{grep}, \textbf{sed}, and \textbf{awk} to locate and isolate
the value.  For example, to obtain the mean value of a data array you
could have the following lines in your script.

\begin{small}
\begin{terminalv}
     set dummy = `stats accept | grep "Pixel mean"`
     set mean = $dummy[4]
\end{terminalv}
\end{small}

The \texttt{accept} keyword tells \xref{\textbf{stats}}{sun95}{STATS} to use
the current values for any parameters for which it would otherwise
prompt you.  The back quotes (\texttt{`~`} are important.  They tell the
shell to execute the expression they enclose, in this case \mbox{{\tt
stats accept | grep "Pixel mean"}}, and assign the result to variable
\texttt{dummy}. So the \KAPPAref\normalsize\ \textbf{stats} task is run on
the current dataset, and the output is passed to \textbf{grep} which
searches for the line containing the string \texttt{Pixel mean}, and the
mean value.  When you equate a variable to a multi-word string not
enclosed in quotes---words being separated by spaces---the variable
becomes an array, whose elements are each assigned to a word.  So in
this case the fourth word or element is the mean value.

Besides being inelegant, this method demands that the format and
text of the output be fixed.  So it should be avoided where the
first task writes results to \htmlref{\textsf{output
parameters}}{sc4_gl_opar}.

\subsection{\xlabel{sc4_se_info_parameter}Via Parameters
\label{sc4_se_info_parameter}}

There is a more-robust technique for passing information between
Starlink applications; it does not rely on the formatting of the
output. Where an application writes \htmlref{\textsf{output
parameters}}{sc4_gl_opar} (otherwise called results parameters), you
may use the \xref{\textbf{parget}}{sun95}{PARGET} command of \KAPPAref\
to write to \htmlref{\textsf{standard output}}{sc4_gl_std_out} the value
or values associated with a parameter for a named application. In a
script we want to assign the value to a \htmlref{shell
variable}{sc4_se_variables}.  Let's see how that compares with
obtaining the same mean value as before.

\begin{small}
\begin{terminalv}
     stats accept > /dev/null
     set average = `parget mean stats`
\end{terminalv}
\end{small}
This runs \xref{\textbf{stats}}{sun95}{STATS}, but redirects the output
to the null device as the output is not wanted.  However, this does
not prevent the output parameter called MEAN from being assigned the
mean value.  \textbf{parget} retrieves this value, which is then assigned
to variable \texttt{average}.  Note that \textbf{parget} retrieves parameter
values for the last invocation of the task.  Thus if you wanted the
standard deviation too, you would only need to issue the \textbf{stats}
command once (as shown below).

\begin{small}
\begin{terminalv}
     stats \\ > /dev/null
     histpeak use=a sfact=0 device=! accept > /dev/null
     set mean = `parget mean stats`
     set stdev = `parget sigma`
     set kurtosis = `parget kurt histpeak`
\end{terminalv}
\end{small}
The kurtosis is obtained via the KURT output parameter of the \xref{{\bf
histpeak}}{sun180}{HISTPEAK} command of \ESPref.

\newpage
\section{\xlabel{sc4_se_csharith}Arithmetic\label{sc4_se_csharith}}

\subsection{\xlabel{sc4_se_integer}Integer\label{sc4_se_integer}}

You may include integer arithmetic in your scripts.  First you must
assign a value to each variable used with the \textbf{set} command.

\begin{small}
\begin{terminalv}
     set count = 1
     set data = (10 24 30 17)
     set exception = -999
     set list
\end{terminalv}
\end{small}

Notice that arrays are defined as spaced-separated values in
parentheses.  So \texttt{\$data[3]} is 30.  You can also set a variable
to a null value, such as \texttt{list} above, so that the variable is
ready to be assigned to an expression.

To perform arithmetic the expression is prefixed with the special {\tt
@} character like this.

\begin{small}
\begin{terminalv}
     @ count = $count + 1
     @ count++
\end{terminalv}
\end{small}
Both of these add one to the value of \texttt{count}.  Note that the space
following the \texttt{@} and around the \texttt{=} and \texttt{+} operator are
needed.  Likewise the examples below both subtract one from the value of
\texttt{count}.

\begin{small}
\begin{terminalv}
    @ count = $count - 1
    @ count--
\end{terminalv}
\end{small}

There are several other operators, the most important ones are
illustrated below.

\begin{small}
\begin{terminalv}
     @ ratio = $count / 5
     @ octrem = $data[2] % 8
     @ numelement = $i * $j
\end{terminalv}
\end{small}
The first divides the value of \texttt{count} by 5, rounding down, so
if \texttt{count} was 11, \texttt{ratio} would be 2.  The second assigns
the remainder of the second element of the array called \texttt{data}
after division by 8.  So if \texttt{data[2]} is 30, \texttt{octrem} would be
set to 6.  Finally \texttt{numelement} is equated to the product of
variables \texttt{i} and \texttt{j}.

The precedence order is \texttt{* / \%}~~ followed by \texttt{+ -}.  If
you are in doubt, use parentheses to achieve the desired order.

\newpage
\subsection{\xlabel{sc4_se_logical}Logical\label{sc4_se_logical}}

As variables are either integers or strings, you have to find an
alternative route to have logical variables in your script.  One way is
to define string values \texttt{"true"} and \texttt{"false"} (or uppercase
equivalents) and test for these.  For instance, this checks the value of
variable \texttt{smooth};  when it is true the current dataset is Gaussian
smoothed.

\begin{small}
\begin{terminalv}
     if ( $smooth == "true" ) then
        gausmooth \\
     end
\end{terminalv}
\end{small}
Remember that UNIX is case sensitive, so for instance, \texttt{TRUE} and
\texttt{true} are not the same.

Another popular way is to use an integer, which is assigned to 0 for
a false value, and 1 for
\begin{minipage}{80mm}
\smallskip
true.  Indeed this is the way logical expressions are evaluated by the
shell.  For instance in the following logical expression

\begin{small}
\begin{terminalv}
     @ x = ($n < 5 || 20 <= $n)
     @ y = ( !($n < 5 || 20 <= $n) )
\end{terminalv}
\end{small}
variable \texttt{x} is \texttt{0} (false) when \texttt{n} lies between 5 and 19
inclusive, but is \texttt{1} (true) otherwise.  Variable \texttt{y} is the
negation of \texttt{x}.  Note the parentheses surrounding the logical
expressions; they are needed when an expression contains \texttt{>}, {\tt
<}, \texttt{|}, or \texttt{\&}.  A list of the comparison operators is given
in the table \latexelsehtml{to the right}{below}.
\smallskip

Here are a few more examples.
\end{minipage}
\ \hfill \
\begin{minipage}{64mm}
\vspace*{3mm}
\label{sc4_tab_comp_oper}
\begin{tabular}{ll}
\hline
\multicolumn{2}{c}{Comparison operators} \\ \hline
Operator & Meaning \\ \hline
\vspace*{-\medskipamount} \\
\texttt{==} & Equal to \\
\texttt{!}  & Boolean NOT \\
\texttt{!=} & Not equal \\
\texttt{\&\&} & Boolean AND \\
\texttt{||} & Boolean OR \\
\texttt{>} & Greater than \\
\texttt{<} & Less than \\
\texttt{>=} & Greater or equal to \\
\texttt{<=} & Less than or equal to \\
\vspace*{-\medskipamount} \\ \hline
\end{tabular}
\end{minipage}
\vspace*{1mm}

\begin{small}
\begin{terminalv}
     @ flag = ( $name != "abc" )
     @ integrate = ( $flux[4] > 20 && $band != 5 )
     @ finish = ( $finish && ( $count > $number || $flag == 0 ) )
\end{terminalv}
\end{small}

The additional parentheses in the expression for \texttt{finish} indicate
the evaluation order starting from the innermost parentheses.

Observant readers will have noticed there are no comparisons involving
floating-point numbers.  Although you can do some limited string comparison,
another means is needed.  One such is to use the \textbf{bc} calculator.

\begin{small}
\begin{terminalv}
     @ bound = `echo "if ( $lower > -0.5 && $upper < 0.5 ) 1" | bc`

     set current = 1.234*10^03
     set upper = 1234.56
     while ( `echo "if ( $current <= $upper) 1" | bc` )
         <statements when $current <= 1234.56>
     end

     if ( `echo "if ( ${asp} <= 1.1 ) 1" | bc` ) then
         <statements when $asp is less than or equal to 1.1>
     end

\end{terminalv}
\end{small}

The {\texttt{"}} quoted if statement involving the desired conditional
statements is passed to bc.  If \textbf{bc} evaluates the statement true,
it returns the 1, which is equivalent to true in C-shell.  Note since
the \htmlref{\textsf{piping}}{sc4_gl_pipe} into \textbf{bc} is a command,
the whole assignment is enclosed in backquotes.  In the first example
above, the parentheses are not required, as the \texttt{>} and \texttt{<}
are part of a string, not operators, as far as the shell is concerned.
The other two examples show this technique in the context of a loop
control and an if \ldots then block.

Note \textbf{bc} uses an arbitrary and controllable precision (\emph{cf.}
the length and scale attributes).  Be aware it also uses a
\xref{non-standard syntax for scientific
notation}{sc16}{sc16_writingcsh}, as seen defining variable {\tt
current} above.

The C-shell also offers some operators that test the characteristics
of a file, such as file existence.  See
\latexelsehtml{Section~\ref{sc4_se_file_operators}}{\htmlref{File
Operators}{sc4_se_file_operators}} for details.

\newpage
\subsection{\xlabel{sc4_se_real}Floating Point\label{sc4_se_real}}

The shell cannot perform non-integer arithmetic.  Thus you need to
employ some other utility.  The standard UNIX \textbf{bc}
arbitrary-precision calculator is one.  You can also define
an algebraic calculator with \htmladdnormallink{\textbf{awk} or
\textbf{gawk}}{http://home3.inet.tele.dk/frda/src/calc_tip.html} as shown
below.

\begin{small}
\begin{terminalv}
     alias acalc '       awk "BEGIN{ print \!* }" '

     set z = `acalc (sqrt(3.2)+exp(5.4))*4.1/5.2`
     set circle_area = `acalc atan2(0,-1)*${rad}*${rad}`
\end{terminalv}
\end{small}
It is inadvisable to name the alias \texttt{calc} for reasons that will
soon be evident.  Note that in expressions you don't escape the
multiplication sign and the expression is \emph{not} in {\texttt{"}} quotes.
The small set of about ten mathemetical functions available in {\bf
awk}, including arctangent shown above that evaluates $\pi$, limits
what you can achieve.  Another option is the {\KAPPAref} \xref{{\bf
calc}}{sun95}{CALC} command.  \textbf{calc} has most of the Fortran
intrinsic functions available.  For the remainder of this section, we
shall just deal with \textbf{calc}.

The \textbf{calc} tool evaluates Fortran-like expressions, sending the
result to \htmlref{\textsf{standard output}}{sc4_gl_std_out}.  So suppose
that we wish to add two real values.

\begin{small}
\begin{terminalv}
     set a = 5.4321
     set b = 1.2345
     set c = `calc $a+$b`
\end{terminalv}
\end{small}
The variable \texttt{c} takes the value of adding variables \texttt{a} and
\texttt{b}.  Note the back quotes that cause \textbf{calc} to be executed.

This is fine if you know that the values will always be positive.
However, \textbf{calc} will object if there are adjacent mathematical
operators, such as \texttt{+-} as would happen if \texttt{b} were negative.
So surround variables in parentheses, remembering that the \texttt{(~)} are
\htmlref{\textsf{metacharacters}}{sc4_gl_met}.
\latexelsehtml{See Section~\ref{sc4_se_spec_char}}{Click
\htmlref{here}{sc4_se_spec_char}} for more details.

Let's redo the first example along with a few more to illustrate the
recommended syntax.  This time we'll specify the expression by parameter
name rather than position.

\begin{small}
\begin{terminalv}
     set a = 5.4321
     set b = -1.2345
     set c = `calc exp="($a)+($b)"`                     # c = 4.1976
     set d = `calc exp="'( $a - 0.5 * ($b) ) / 100 '"`  # d = 0.0481485
     set e = `calc exp="(($a)+($b))**3"`                # e = 296.2874
     set f = `calc exp="'($a + ($b)) ** 3'"`            # f = e
\end{terminalv}
\end{small}
Don't be intimidated by the surfeit of quotes.  The \texttt{"~"} are needed
because we want the dollar to retain its variable-substitution meaning.
If the expression contains embedded spaces (usually for clarity) it
should be enclosed in single quotes as shown in the assignment to
variable \texttt{f}.  So in general the recipe for using \textbf{calc} is

\verb#     set #\emph{variable} \verb#= `calc exp="'#\emph{expression}\verb#'"`#

Don't try to escape all the metacharacters individually because it can
soon become very messy, error prone, and hard to follow.
\medskip

The default precision is single precision.  If you need more
significant figures then append \texttt{prec=\_double} to the \textbf{calc}
command line.  The special symbol \texttt{pi} has the value of $\pi$.  So

\begin{small}
\begin{terminalv}
     set arc = 0.541209
     set darc = `calc exp=$arc*180.0/pi prec=_double`
\end{terminalv}
\end{small}
converts an angle 0.541209 radians (stored in \texttt{arc}) to degrees
using double-precision arithmetic.

It is sometimes easier to assemble the expression in another variable.
Here's another recipe which demonstrates this approach.  Suppose that you
want to find the average median value for a series of \NDFrefa{\textsf{NDF}s}.

\begin{small}
\begin{terminalv}
     # Estimates the median of each sky frame and subtracts it from that image

     # Enable KAPPA commands.  Initialise variables.
     kappa > /dev/null
     set first = 0
     set numndf = 0

     # Loop for all NDFs
     foreach file (*_fl.sdf)

     # Obtain the NDF's name.
        set file1 = $file:r

     # Assign the median value of the frame.  HISTAT computes the median,
     # and PARGET extracts it from HISTAT's MEDIAN output parameter.
     # Dispose of the HISTAT text output to the null file.
        histat $file1 \\ > /dev/null
        set median = `parget median histat`

     # Subtract the median from the input NDF.
        csub $file1 $median  $file1"_m"

     # Count the number of NDFs.
        @ numndf = $numndf + 1

     # Form the string to sum the median values using CALC.  The shell
     # only permits integer arithmetic.  The expression is different
     # for the first file.
        if ( $first == 0 ) then
           set expr = "( "$median
           set first = 1

     # Append a plus sign and the next median value.
        else
           set expr = `echo $expr" + "$median`
        endif

     end

     # Complete the expression for evaluating the average of the frame
     # medians.
     set expr = `echo $expr" ) / " $numndf`

     # Evaluate the expression and report the result.
     echo "Average median is "`calc $expr`
\end{terminalv}
\end{small}
The script builds an expression to evaluate the average of the median
values.  So the first time it has to write the left parenthesis and
first median value into string variable \texttt{expr}, but otherwise each
new value is appended to the expression.  Once all the medians are
evaluated, the remainder of the expression, including division by the
number of NDFs is appended.  Finally, the expression is evaluated using
\xref{\textbf{calc}}{sun95}{CALC}.  If you want to learn more about the
\mbox{\texttt{set median = `parget median histat`}} command
\latexelsehtml{see Section~\ref{sc4_se_info_parameter}.}{click \htmlref{here}{sc4_se_info_parameter}.}

\subsection{\xlabel{sc4_se_intrinsic}Intrinsic Functions
\label{sc4_se_intrinsic}}

If you want to include intrinsic functions, such as logarithms and
trigonometric functions in your calculations, or perhaps you need some
function for an integer expression that is not supplied by the
C-shell, such as finding the maximum or the absolute value,
\xref{\textbf{calc}}{sun95}{CALC} may be the solution.  It offers the
28 functions tabulated below.

Here are a few examples.

\begin{small}
\begin{terminalv}
     set a = 5
     set b = (3 4 6 -2)
     set c = `calc exp="'min( $a, $b[1], $b[2], $b[3] )'"`
     set d = `calc exp="'dim( $b[4], $a ) + dim( $b[3], $a )'"`
\end{terminalv}
\end{small}
The first expression finds the minimum of the four integers, namely
\texttt{3} and assigns it to variable \texttt{c}.  The second expression
sums two positive differences: \texttt{0} and \texttt{1} respectively.

\begin{small}
\begin{terminalv}
     set mag = `calc exp="'-2.5 * LOG10( $counts )'"`
     set separation = `calc exp="atand2(35.3,$dist)"`
     echo The nearest index is `calc exp="nint($x+0.5)"`
\end{terminalv}
\end{small}
Variable \texttt{mag} is the magnitude derived from the flux
stored in variable \texttt{counts}.  \texttt{separation} is assigned
the inverse tangent of 35.3 divided by the value of variable
\texttt{dist} measured between $-$180 and 180 degrees.

\bigskip

\newpage
\begin{center}
\begin{tabular}{lcl} \hline
%\latex{\emph{Function} & \begin{tabular}{c}Number of \\ arguments \end{tabular} & Description \\}
\emph{Function} & Number of arguments & Description \\
\hline
\\
SQRT & 1 & square root: $\sqrt{\mbox{arg}}$ \\
LOG & 1 & natural logarithm: $\ln (\mbox{arg})$ \\
LOG10 & 1 & common logarithm: $\log _{10}(\mbox{arg})$ \\
EXP & 1 & exponential: $\exp (\mbox{arg})$ \\
ABS & 1 & absolute (positive) value: $\left| \mbox{arg} \right|$ \\
NINT & 1 & nearest integer value to \mbox{arg} \\
MAX & 2 or more & maximum of arguments \\
MIN & 2 or more & minimum of arguments \\
DIM & 2 & Fortran DIM (positive difference) function \\
MOD & 2 & Fortran MOD (remainder) function \\
SIGN & 2 & Fortran SIGN (transfer of sign) function \\
SIN* & 1 & sine function: $\sin (\mbox{arg})$ \\
COS* & 1 & cosine function: $\cos (\mbox{arg})$ \\
TAN* & 1 & tangent function: $\tan (\mbox{arg})$ \\
ASIN* & 1 & inverse sine function: $\sin ^{-1}(\mbox{arg})$ \\
ACOS* & 1 & inverse cosine function: $\cos ^{-1}(\mbox{arg})$ \\
ATAN* & 1 & inverse tangent function: $\tan ^{-1}(\mbox{arg})$ \\
ATAN2* & 2 & inverse tangent function: $\tan ^{-1}(\mbox{arg1/arg2})$ \\
SINH* & 1 & hyperbolic sine function: $\sinh (\mbox{arg})$ \\
COSH* & 1 & hyperbolic cosine function: $\cosh (\mbox{arg})$ \\
TANH* & 1 & hyperbolic tangent function: $\tanh (\mbox{arg})$ \\
SIND* & 1 & sine function: $\sin (\mbox{arg})$ \\
COSD* & 1 & cosine function: $\cos (\mbox{arg})$ \\
TAND* & 1 & tangent function: $\tan (\mbox{arg})$ \\
ASIND* & 1 & inverse sine function: $\sin ^{-1}(\mbox{arg})$ \\
ACOSD* & 1 & inverse cosine function: $\cos ^{-1}(\mbox{arg})$ \\
ATAND* & 1 & inverse tangent function: $\tan ^{-1}(\mbox{arg})$ \\
ATAN2D* & 2 & inverse tangent function: $\tan ^{-1}(\mbox{arg1/arg2})$ \\
\multicolumn{1}{r}{} &
\multicolumn{2}{l}{\footnotesize *Function does not support \emph{integer}
arithmetic.}\\
\\ \hline
\end{tabular}
\end{center}
The intrinsic functions recognised by \textbf{calc} (adapted from SUN/61).
The angular arguments/results of the trigonometric functions are in
radians, except those suffixed with a D, which are in degrees.
\label{sc4_table:intrinsics}

\newpage
\section{\xlabel{sc4_se_string_proc}String Processing
\label{sc4_se_string_proc}}

The C-shell has no string-manipulation tools.  Instead we mostly use
the \textbf{echo} command and \textbf{awk} utility.  The latter has its own
language for text processing and you can write \textbf{awk} scripts to
perform complex processing, normally from files.  There are even books
devoted to \textbf{awk} such as the succinctly titled {\sl sed \& awk} by
Dale Dougherty (O'Reilly \& Associates, 1990).  Naturally enough a
detailed description is far beyond the scope of this cookbook.

Other relevant tools include \textbf{cut}, \textbf{paste}, \textbf{grep} and
\textbf{sed}.  The last two and \textbf{awk} gain much of their power from
\htmlref{regular expressions}{sc4_gl_reg_exp}.  A regular expression
is a pattern of characters used to match the same characters in a
search through text.  The pattern can include special characters to
refine the search.  They include ones to anchor to the beginning or
end of a line, select a specific number of characters, specify any
characters and so on.  If you are going to write lots of scripts which
manipulate text, learning about regular expressions is time well
spent.

Here we present a few one-line recipes.  They can be included inline,
but for clarity the values derived via \textbf{awk} are assigned to
variables using the \textbf{set} command.  Note that these may not be the
most concise way of achieving the desired results.

\subsection{\xlabel{sc4_se_string_conc}String concatenation
\label{sc4_se_string_conc}}

To concatenate strings, you merely abut them.

\begin{small}
\begin{terminalv}
     set catal = "NGC "
     set number = 2345
     set object = "Processing $catal$number."
\end{terminalv}
\end{small}
So \texttt{object} is assigned to \texttt{"Processing NGC 2345."}.
Note that spaces must be enclosed in quotes.  If you want to embed
variable substitutions, these should be double quotes as above.

On occasions the variable name is ambiguous.  Then you have to break up
the string.  Suppose you want \texttt{text} to be \texttt{"File cde1 is not
abc"}.  You can either make two abutted strings or encase the variable
name in braces (\verb+{+~\verb+}+), as shown below.

\begin{small}
\begin{terminalv}
     set name1 = abc
     set name = cde
     set text = "File $name""1 is not $name1"     # These two assignments
     set text = "File ${name}1 is not $name1"     # are equivalent.
\end{terminalv}
\end{small}

Here are some other examples of string concatenation.

\begin{small}
\begin{terminalv}
     echo 'CIRCLE ( '$centre[1]', '$centre[2]', 20 )' > $file1".ard"
     gausmooth in=$root"$suffix" out=${root}_sm accept
     linplot ../arc/near_arc"($lbnd":"$ubnd)"
\end{terminalv}
\end{small}

\newpage
\subsection{\xlabel{sc4_se_string_length}Obtain the length of a string
\label{sc4_se_string_length}}

This requires either the \textbf{wc} command in an
\htmlref{expression}{sc4_se_csharith}\latex{ (see
Section~\ref{sc4_se_csharith})}, or the \textbf{awk} function \textbf{length}.
Here we determine the number of characters
in variable \texttt{object} using both recipes.

\begin{small}
\begin{terminalv}
     set object = "Processing NGC 2345"
     set nchar = `echo $object | awk '{print length($0)}'`      # = 19
     @ nchar = `echo $object | wc -c` - 1
\end{terminalv}
\end{small}

If the variable is an array, you can either obtain the length of the
whole array or just an element.  For the whole the number of
characters is the length of a space-separated list of the elements.
The double quotes are delimiters; they are not part of the string so
are not counted.

\begin{small}
\begin{terminalv}
     set places = ( Jupiter "Eagle Nebula" "Gamma quadrant" )
     set nchara = `echo $places | awk '{print length($0)}'`     # = 35
     set nchar1 = `echo $places[1] | awk '{print length($0)}'`  # =  7
     set nchar2 = `echo $places[2] | awk '{print length($0)}'`  # = 12
\end{terminalv}
\end{small}

\subsection{\xlabel{sc4_se_string_substr}Find the position of a
substring\label{sc4_se_string_substr}}

This requires the \textbf{awk} function \textbf{index}.  This returns zero if
the string could not be located.  Note that comparisons are case
sensitive.

\begin{small}
\begin{terminalv}
     set catal = "NGC "
     set number = 2345
     set object = "Processing $catal$number."
     set cind = `echo $object | awk '{print index($0,"ngc")}'`      # =  0
     set cind = `echo $object | awk '{print index($0,"NGC")}'`      # = 12

     set places = ( Jupiter "Eagle Nebula" "Gamma quadrant" )
     set cposa = `echo $places | awk '{print index($0,"ebu")}'`     # = 16
     set cposn = `echo $places | awk '{print index($0,"alpha")}'`   # =  0
     set cpos1 = `echo $places[1] | awk '{print index($0,"Owl")}'`  # =  0
     set cpos2 = `echo $places[2] | awk '{print index($0,"ebu")}'`  # =  8
     set cpos3 = `echo $places[3] | awk '{print index($0,"rant")}'` # = 11
\end{terminalv}
\end{small}
An array of strings is treated as a space-separated list of the elements.
The double quotes are delimiters; they are not part of the string so
are not counted.

\newpage
\subsection{\xlabel{sc4_se_string_ex_substr}Extracting a substring
\label{sc4_se_string_ex_substr}}

One method uses the \textbf{awk} function \textbf{substr}($s$,$c$,$n$).  This
returns the substring from string $s$ starting from character position
$c$ up to a maximum length of $n$ characters.  If $n$ is not supplied,
the rest of the string from $c$ is returned.  Let's see it in action.

\begin{small}
\begin{terminalv}
     set caption = "Processing NGC 2345."
     set object = `echo $caption | awk '{print substr($0,12,8)}'` # = "NGC 2345"
     set objec_ = `echo $caption | awk '{print substr($0,16)}'`   # = "2345."

     set places = ( Jupiter "Eagle Nebula" "Gamma quadrant" )
     set oba = `echo $places | awk '{print substr($0,28,4)}'` # = "quad"
     set ob1 = `echo $places[3] | awk '{print substr($0,7)}'` # = "quadrant"
\end{terminalv}
\end{small}
An array of strings is treated as a space-separated list of the elements.
The double quotes are delimiters; they are not part of the string so
are not counted.

Another method uses the UNIX \textbf{cut} command.  It too can specify a
range or ranges of characters.  It can also extract fields separated
by nominated characters.  Here are some examples using the same values
for the array \texttt{places}

\begin{small}
\begin{terminalv}
     set cut1 = `echo $places | cut -d ' ' -f1,3`  # = "Jupiter Nebula"
     set cut2 = `echo $places[3] | cut -d a -f2`   # = "mm"
     set cut3 = `echo $places | cut -c3,11`        # = "pg"
     set cut4 = `echo $places | cut -c3-11`        # = "piter Eag"
     set cut5 = `cut -d ' ' -f1,3-5 table.dat`     # Extracts fields 1,3,4,5
                                                   # from file table.dat

\end{terminalv}
\end{small}

The \texttt{-d} qualifier specifies the delimiter between associated data
(otherwise called fields).  Note the the space delimiter must be quoted.
The \texttt{-f} qualifier selects the fields.  You can also select
character columns with the \texttt{-c} qualifier.  Both \texttt{-c} and
\texttt{-f} can comprise a comma-separated list of individual values and/or
ranges of values separated by a hyphen.  As you might expect,
\textbf{cut} can take its input from files too.


\subsection{\xlabel{sc4_se_string_split}Split a string into an
array\label{sc4_se_string_split}}

The \textbf{awk} function \textbf{split}($s$,$a$,\emph{sep}) splits a string
$s$ into an \textbf{awk} array $a$ using the delimiter \emph{sep}.

\begin{small}
\begin{terminalv}
     set time = 12:34:56
     set hr = `echo $time | awk '{split($0,a,":"); print a[1]}'` # = 12
     set sec = `echo $time | awk '{split($0,a,":"); print a[3]}'` # = 56

     # = 12 34 56
     set hms = `echo $time | awk '{split($0,a,":"); print a[1], a[2], a[3]}'`
     set hms = `echo $time | awk '{split($0,a,":"); for (i=1; i<=3; i++) print a[i]}'`
     set hms = `echo $time | awk 'BEGIN{FS=":"}{for (i=1; i<=NF; i++) print $i}'`
\end{terminalv}
\end{small}
Variable \texttt{hms} is an array so \texttt{hms[2]} is \texttt{34}.
The last three statements are equivalent, but the last two more
convenient for longer arrays.  In the second you can specify the start
index and number of elements to print.  If, however, the number of
values can vary and you want all of them to become array elements,
then use the final recipe; here you specify the field separator with
awk's \texttt{FS} built-in variable, and the number of values with the
\texttt{NF} built-in variable.

\subsection{\xlabel{sc4_se_string_case}Changing case
\label{sc4_se_string_case}}

Some implementations of \textbf{awk} offer functions to change case.

\begin{small}
\begin{terminalv}
     set text = "Eta-Aquarid shower"
     set utext = `echo $text | awk '{print toupper($0)}'` # = "ETA-AQUARID SHOWER"
     set ltext = `echo $text | awk '{print tolower($0)}'` # = "eta-aquarid shower"
\end{terminalv}
\end{small}


\subsection{\xlabel{sc4_se_string_sub}String substitution
\label{sc4_se_string_sub}}

Some implementations of \textbf{awk} offer substitution functions {\bf
gsub}($e$,$s$) and \textbf{sub}($e$,$s$).  The latter substitutes the $s$
for the first match with the \htmlref{\textsf{regular
expression}}{sc4_gl_reg_exp} $e$ in our supplied text.  The former
replaces every occurrence.

\begin{small}
\begin{terminalv}
     set text = "Eta-Aquarid shower"
     # = "Eta-Aquarid stream"
     set text = `echo $text | awk '{sub("shower","stream"); print $0}'`

     # = "Eta-Aquxid strex"
     set text1 = `echo $text | awk '{gsub("a[a-z]","x"); print $0}'`

     # = "Eta-Aquaritt stream"
     set text2 = `echo $text | awk '{sub("a*d","tt"); print $0}'`

     set name = "Abell 3158"
     set catalogue = `echo $name | awk '{sub("[0-9]+",""); print $0}'`  # = Abell
     set short = `echo $name | awk '{gsub("[b-z]",""); print $0}'`      # = "A 3158"
\end{terminalv}
\end{small}

There is also \textbf{sed}.
\begin{small}
\begin{terminalv}
     set text = `echo $text | sed 's/shower/stream/'`
\end{terminalv}
\end{small}
is equivalent to the first \textbf{awk} example above.  Similarly you could
replace all occurrences.

\begin{small}
\begin{terminalv}
     set text1 = `echo $text | sed 's/a[a-z]/x/g'`
\end{terminalv}
\end{small}
is equivalent to the second example.  The final \texttt{g} requests that the
substitution is applied to all occurrences.

\subsection{\xlabel{sc4_se_form_print}Formatted Printing
\label{sc4_se_form_print}}

A script may process and analyse many datasets, and the results from
its calculations will often need presentation, often in tabular form
or some aligned output, either for human readability or to be read by
some other software.

The UNIX command \textbf{printf} permits formatted output.  It is
analogous to the C function of the same name.  The syntax is

\begin{small}
\begin{terminalv}
     printf "<format string>" <space-separated argument list of variables>
\end{terminalv}
\end{small}

\begin{minipage}{73mm}
The format string may contain text, conversion codes, and interpreted
sequences.

The conversion codes appear in the same order as the arguments
they correspond to.  A conversion code has the form

\begin{small}
\begin{terminalv}
%[flag][width][.][precision]code
\end{terminalv}
\end{small}

where the items in brackets are optional.
\begin{itemize}
\item
The \emph{code} determines how the output is is converted for printing.
The most commonly used appear in the upper table.
\item
The \emph{width} is a positive integer giving the minimum field width.
A value requiring more characters than the width is still written in
full.  A datum needing few characters than the width is right
justified, unless the flag is \texttt{-}.  \texttt{*} substitutes the next
variable in the argument list, allowing the width to be programmable.
\item
The \emph{precision} specifies the number of decimal places for
floating point; for strings it sets the maximum number of characters
to print.  Again \texttt{*} substitutes the next variable in the argument
list, whose value should be a positive integer.
\item
The \emph{flag} provides additional format control.  The main functions
are listed in the lower table.
\end{itemize}

The interpreted sequences include:\\
\texttt{\textbackslash n} for a new line, \texttt{\textbackslash "} for a double quote,
\texttt{\textbackslash\%} for a percentage sign, and \texttt{\textbackslash\textbackslash} for a backslash.

\end{minipage}
\ \hfill \
\begin{minipage}{75mm}
\vspace*{3mm}
\begin{tabular}{lp{61mm}}
\hline
\multicolumn{2}{c}{Format codes} \\ \hline
Code & Interpretation \\ \hline
\vspace*{-\medskipamount} \\
\texttt{c} & single character \\
\texttt{s} & string \\
\\
\texttt{d}, \texttt{i} & signed integer \\
\texttt{o} & integer written as unsigned octal \\
\texttt{x}, \texttt{X} & integer written as unsigned hexadecimal, the
           latter using uppercase notation \\
\\
\texttt{e}, \texttt{E} & floating point in exponent form \texttt{m.nnnnne$\pm$xx}
                       or \texttt{m.nnnnnE$\pm$xx} respectively \\
\texttt{f} & floating point in \texttt{mmm.nnnn} format\\
\texttt{g} & uses whichever format of \texttt{d}, \texttt{e}, or \texttt{f}
          is shortest \\
\texttt{G} & uses whichever format of \texttt{d}, \texttt{E}, or \texttt{f}
          is shortest \\
\vspace*{-\medskipamount} \\ \hline
\end{tabular}

\vspace*{10mm}
\begin{tabular}{lp{61mm}}
\hline
\multicolumn{2}{c}{Flags} \\ \hline
Code & Purpose \\ \hline
\vspace*{-\medskipamount} \\
\texttt{-} & left justify \\
\texttt{+} & begin a signed number with a \texttt{+} or \texttt{-} \\
blank & Add a space before a signed number that does not begin with a
          \texttt{+} or \texttt{-} \\
\texttt{0} & pad a number on the left with zeroes \\
\texttt{\#} & use a decimal point for the floating-point conversions, and
do not remove trailing zeroes for \texttt{g} and \texttt{G} codes \\
\vspace*{-\medskipamount} \\ \hline
\end{tabular}

\end{minipage}
\vspace*{2mm}

If that's computer gobbledygook here are some examples to make it
clearer.  The result of follows each \textbf{printf} command, unless it
is assigned to a variable through the set mechanism.  The commentary
after the \texttt{\#} is neither part of the output nor should it be
entered to replicate the examples.  Let us start with some integer values.
\begin{small}
\begin{terminalv}
     set int = 1234
     set nint = -999
     printf "%8i\n" $int            # 8-character field, right justified
         1234
     printf "%-8d%d\n" $int $nint   # Two integers, the first left justified
     1234    -999
     printf "%+8i\n" $int
        +1234
     printf "%08i\n" $int
     00001234
\end{terminalv}
\end{small}
Now we turn to some floating-point examples.
\begin{small}
\begin{terminalv}
     set c = 299972.458
     printf "%f %g %e\n" $c $c $c           # The three main codes
     299972.458000 299972 2.999725e+05
     printf "%.2f %.2g %.2e\n" $c $c $c     # As before but set a precision
     299972.46 3e+05 +3.00e+05
     printf "%12.2f %.2G %+.2E\n" $c $c $c  # Show the effect of some flags,
        299972.46 3.0E+05 +3.00E+05         # a width, and E and G codes
\end{terminalv}
\end{small}
Finally we have some character examples of \textbf{printf}.
\begin{small}
\begin{terminalv}
     set system = Wavelength
     set confid = 95
     set ndf = m31
     set acol = 12
     set dp = 2

     printf "Confidence limit %d%% in the %s system\n" $confid $system
     Confidence limit 95% in the Wavelength system  # Simple case, percentage sign
     printf "Confidence limit %f.1%% in the %.4s system\n" $confid $system
     Confidence limit 95.0% in the Wave system      # Truncates to four characters
     set heading = `printf "%10s: %s\n%10s: %s\n\n" "system" $system "file" $ndf`
     echo $heading                                  # Aligned output, saved to a
         system: Wavelength                         # variable
           file: m31

     printf "%*s: %s\n%*s: %.*f\n\n" $acol "system" \
            $system $acol "confidence" 2 $confid
           system: Wavelength                       # Aligned output with a variable
       confidence: 95.00                            # width and precision

     set heading = ""                               # Form a heading by appending to
     foreach k ( $keywords )                        # variable in a loop.  Note the
        set heading = $heading `printf "%8s  " $k`  # absence of \n.
     end
     echo `printf "%s\n" $heading
   \end{terminalv}
\end{small}

Note that there are different implementations.  While you can check
your system's man pages that the desired feature is present, a better
way is to experiment on the command line.

\newpage
\section{\xlabel{sc4_se_files}Dealing with Files\label{sc4_se_files}}


\subsection{\xlabel{sc4_se_filename_ext}Extracting parts of filenames
\label{sc4_se_filename_ext}}

Occasionally you'll want to work with parts of a filename, such as the
path or the file type.  The C-shell provides \textsf{filename modifiers}
that select the various portions.  A couple are shown in the example
below.

\begin{small}
\begin{terminalv}
     set type = $1:e
     set name = $1:r
     if ( $type == "bdf" ) then
        echo "Use BDF2NDF on a VAX to convert the Interim file $1"
     else if ( $type != "dst" ) then
        hdstrace $name
     else
        hdstrace @'"$1"'
     endif
\end{terminalv}
\end{small}

Suppose the first argument of a script, \texttt{\$1}, is a filename called
galaxy.bdf.  The value of variable \texttt{type} is \texttt{bdf} and {\tt
name} equates to \texttt{galaxy} because of the presence of the filename
modifiers \texttt{:e} and \texttt{:r}.  The rest of the script uses the
file type to control the processing, in this case to provide a listing
of the contents of a data file using the \HDSTRACEref\ utility.

The complete list of modifiers, their meanings, and examples is
presented in the table below.
\bigskip

\begin{center}
\begin{tabular}{lp{62mm}l}
Modifier & Value returned & Value for filename \\
 & & \texttt{/star/bin/kappa/comwest.sdf} \\ \hline
\\
\textbf{:e} & Portion of the filename following a full stop; if the filename does
not contain a full stop, it returns a null string & \texttt{sdf} \\
\textbf{:r} & Portion of the filename preceding a full stop; if there is no full stop
present, it returns the complete filename & \texttt{comwest} \\
\textbf{:h} & The path of the filename (mnemonic: h for head) & \texttt{/star/bin/kappa} \\
\textbf{:t} & The tail of the file specification, excluding the path &
                                         \texttt{comwest.sdf} \\
\\ \hline
\end{tabular}
\end{center}
\medskip

\newpage
\subsection{\xlabel{sc4_se_series_files}Process a Series of Files
\label{sc4_se_series_files}}

One of the most common things you'll want to do, having devised
a data-processing path, is to apply those operations to a series
of data files.   For this you need a \textbf{foreach...end} construct.

\begin{small}
\begin{terminalv}
     convert               # Only need be invoked once per process
     foreach file (*.fit)
        stats $file
     end
\end{terminalv}
\end{small}
This takes all the \htmlref{\textsf{FITS}}{sc4_gl_fits} files in the
current directory and computes the statistics for them using the
\xref{\textbf{stats}}{sun95}{STATS} command from \KAPPAref\@.  \texttt{file}
is a \htmlref{\textsf{shell variable}}{sc4_gl_she_var}.  Each time in the
loop \texttt{file} is assigned to the name of the next file of the list
between the parentheses.  The \texttt{*} is the familiar
\htmlref{\textsf{wildcard}}{sc4_gl_wild} which
matches any string.  Remember when you want to use the shell
variable's value you prefix it with a \texttt{\$}.  Thus {\tt
\$file} is the filename.

\subsubsection{\xlabel{sc4_se_series_ndfs}NDFs\label{sc4_se_series_ndfs}}

Some data formats like the \NDFrefa{\textsf{NDF}} demand that only the file
name (\emph{i.e.}\ what appears before the last dot) be given in
commands.  To achieve this you must first strip off the remainder (the
file extension or type) with the \texttt{:r} \textsf{file modifier}.

\begin{small}
\begin{terminalv}
     foreach file (*.sdf)
        histogram $file:r accept
     end
\end{terminalv}
\end{small}

This processes all the \textsf{\HDSref} files in the current directory
and calculates an histogram for each of them using the
\xref{\textbf{histogram}}{sun95}{HISTOGRAM} command from \KAPPAref.  It
assumes that the files are NDFs.  The \texttt{:r}
instructs the shell to remove the file extension (the part of the name
following the the rightmost full stop).  If we didn't do this, the {\bf
histogram} task would try to find NDFs called SDF within each of the
HDS files.

\subsubsection{\xlabel{sc4_se_wildcard_lists}Wildcarded lists of files
\label{sc4_se_wildcard_lists}}

You can give a list of files separated by spaces, each of which can
include the various UNIX \htmlref{\textsf{wildcards}}{sc4_gl_wild}.  Thus
the code below would report the name of each \NDFrefa{\textsf{NDF}} and
its standard deviation. The NDFs are called `Z' followed by a single
character, ccd1, ccd2, ccd3, and spot.

\begin{small}
\begin{terminalv}
     foreach file (Z?.sdf ccd[1-3].sdf spot.sdf)
        echo "NDF:" $file:r"; sigma: "`stats $file:r | grep "Standard deviation"`
     end
\end{terminalv}
\end{small}

\textbf{echo} writes to \htmlref{\textsf{standard output}}{sc4_gl_std_out},
so you can write text including values of \htmlref{shell
variables}{sc4_se_variables} to the screen or redirect it to a file.
Thus the output produced by \xref{\textbf{stats}}{sun95}{STATS} is
\htmlref{\textsf{piped}}{sc4_gl_pipe} (the \texttt{|} is the \htmlref{{\sf
pipe}}{sc4_gl_pipe}) into the UNIX \textbf{grep} utility to search for
the string \texttt{"Standard deviation"}.  The \texttt{`~~`} invokes the
command, and the resulting standard deviation is substituted.
\medskip

You might just want to provide an arbitrary list of NDFs as arguments to
a generic script.  Suppose you had a script called \texttt{splotem}, and
you have made it executable with \mbox{\texttt{chmod +x splotem}}.

\begin{small}
\begin{terminalv}
     #!/bin/csh
     figaro                 # Only need be invoked once per process
     foreach file ($*)
        if (-e $file) then
           splot $file:r accept
        endif
     end
\end{terminalv}
\end{small}

Notice the \texttt{-e} file-comparison operator.  It tests whether the
file exists or not.  (\latexelsehtml{Section~\ref{sc4_se_file_operators}
has a}{Click \htmlref{here}{sc4_se_file_operators} for a} full list of
the file operators.) To plot a series of spectra stored in NDFs, you
just invoke it something like this.

\begin{small}
\begin{terminalv}
     % ./splotem myndf.sdf arc[a-z].sdf hd[0-9]*.sdf
\end{terminalv}
\end{small}

See the glossary for a list of the available
\htmlref{\textsf{wildcards}}{sc4_gl_wild} such as the \texttt{[a-z]} in the
above example.

\subsubsection{\xlabel{sc4_se_wildcard_nosdf}Exclude the .sdf for NDFs
\label{sc4_se_wildcard_nosdf}}

In the \latexelsehtml{\texttt{splotem} example from the previous
section}{\htmlref{splotem example}{sc4_se_wildcard_lists}} the list of
\NDFrefa{\textsf{NDF}s} on the command line required the inclusion of
the \texttt{.sdf} file extension.  Having to supply the \texttt{.sdf} for an
NDF is abnormal.  For reasons of familiarity and ease of use, you
probably want your relevant scripts to accept a list of NDF names and
to append the file extension automatically before the list is passed to
\textbf{foreach}.  \latexelsehtml{So let's modify the
previous example to do this.}{Here is an example.  It is a modified
version of the \texttt{splotem} script.}

\begin{small}
\begin{terminalv}
     #!/bin/csh
     figaro                 # Only need be invoked once per process

     #  Append the HDS file extension to the supplied arguments.
     set ndfs
     set i = 1
     while ( $i <= $#argv )
        set ndfs = ($ndfs[*] $argv[i]".sdf")
        @ i = $i + 1
     end

     #  Plot each 1-dimensional NDFs.
     foreach file ($ndfs[*])
        if (-e $file) then
           splot $file:r accept
        endif
     end
\end{terminalv}
\end{small}
This loops through all the arguments and appends the {\sf
\HDSref}-file extension to them by using a work array \texttt{ndfs}.  The
\textbf{set} defines a value for a \htmlref{\textsf{shell
variable}}{sc4_gl_she_var}; don't forget the spaces around the \texttt{=}.
\texttt{ndfs[*]} means all the elements of variable \texttt{ndfs}.  The loop
adds elements to \texttt{ndfs} which is initialised without a value.
Note the necessary parentheses around the expression \texttt{(\$ndfs[*]
\$argv[i]".sdf")}.

On the command line the wildcards have to be passed verbatim, because
the shell will try to match with files than don't have the \texttt{.sdf}
file extension.  Thus you must protect the wildcards with quotes.
It's a nuisance, but the advantages of wildcards more than compensate.

\begin{small}
\begin{terminalv}
     % ./splotem myndf 'arc[a-z]' 'hd[0-9]*'
     % ./noise myndf 'ccd[a-z]'
\end{terminalv}
\end{small}
If you forget to write the \texttt{'~'}, you'll receive a ~\texttt{No match}
error.

\subsubsection{\xlabel{sc4_se_display_series}Examine a series of NDFs
\label{sc4_se_display_series}}

A common need is to browse through several datasets, perhaps to locate
a specific one, or to determine which are acceptable for further
processing.  The following presents images of a series of \NDFrefa{{\sf
NDF}s} using the \xref{\textbf{display}}{sun95}{DISPLAY} task of
\KAPPAref\@.  The title of each plot tells you which NDF is currently
displayed.

\begin{small}
\begin{terminalv}
     foreach file (*.sdf)
        display $file:r axes style="'title==$file:r'" accept
        sleep 5
     end
\end{terminalv}
\end{small}
\textbf{sleep} pauses the \htmlref{\textsf{process}}{sc4_gl_pro} for a given
number of seconds, allowing you to view each image.  If this is too
inflexible you could add a prompt so the script displays the image
once you press the return key.

\begin{small}
\begin{terminalv}
     set nfiles = `ls *.sdf | wc -w`
     set i = 1
     foreach file (*.sdf)
        display $file:r axes style="'title==$file:r'" accept

     # Prompt for the next file unless it is the last.
        if ( $i < $nfiles ) then
           echo -n "Next?"
           set next = $<

     # Increment the file counter by one.
           @ i++
        endif
     end
\end{terminalv}
\end{small}
The first lines shows a quick way of counting the number of files. It
uses \textbf{ls} to expand the \htmlref{\textsf{wildcards}}{sc4_gl_wild}, then
the command \textbf{wc} to count the number of words.  The back quotes
cause the instruction between them to be run and the values generated
to be assigned to variable \texttt{nfiles}.

You can substitute another visualisation command for \textbf{display}
as appropriate.  You can also use the graphics database to plot more
than one image on the screen or to hardcopy.  The script {\tt
\$KAPPA\_DIR/multiplot.csh} does the latter.

\subsection{\xlabel{sc4_se_filename_mod}Filename modification
\label{sc4_se_filename_mod}}

\latex{Thus far the examples have not created a new file.} When you
want to create an output file, you need a name for it.  This could be an
explicit name, one derived from the \htmlref{\textsf{process identification
number}}{sc4_gl_pid}, one
generated by some counter, or from the input filename.  Here we deal
with all but the trivial first case.

\subsubsection{\xlabel{sc4_se_append_filename}Appending to the input
filename\label{sc4_se_append_filename}}

To help identify datasets and to indicate the processing steps used to
generate them, their names are often created by appending suffices to
the original file name.  This is illustrated below.

\begin{small}
\begin{terminalv}
     foreach file (*.sdf)
        set ndf = $file:r
        block in=$ndf out=$ndf"_sm" accept
     end
\end{terminalv}
\end{small}
This uses \xref{\textbf{block}}{sun95}{BLOCK} from \KAPPAref\ to perform
block smoothing on a series of \NDFrefa{\textsf{NDF}s}, creating new NDFs,
each of which takes the name of the corresponding input NDF with a
\texttt{\_sm} suffix.  The \textbf{accept} keyword accepts the suggested
defaults for parameters that would otherwise be prompted.  We use the
\textbf{set} to assign the NDF name to variable \texttt{ndf} for clarity.

\subsubsection{\xlabel{sc4_se_append_counter}Appending a counter to the input
filename\label{sc4_se_append_counter}}

If a counter is preferred, this example

\begin{small}
\begin{terminalv}
     set count = 1
     foreach file (*.sdf)
        set ndf = $file:r
        @ count = $count + 1
        block in=$ndf out=smooth$count accept
     end
\end{terminalv}
\end{small}
would behave as the previous one except that the output NDFs would be
called \texttt{smooth1}, \texttt{smooth2} and so on.

\subsubsection{\xlabel{sc4_se_filename_substitution}Appending to the input
filename\label{sc4_se_filename_substitution}}

Whilst appending a suffix after each data-processing stage is
feasible, it can generate some long names, which are tedious to
handle.  Instead you might want to replace part of the input name with
a new string.  The following creates another \htmlref{shell
variable}{sc4_se_variables}, \texttt{ndfout} by replacing the string {\tt
\_flat} from the input NDF name with \texttt{\_sm}.  The script
\htmlref{\textsf{pipes}}{sc4_gl_pipe} the input name into the \textbf{sed}
editor which performs the substitution.

\begin{small}
\begin{terminalv}
     foreach file (*_flat.sdf)
        set ndf = $file:r
        set ndfout = `echo $ndf | sed 's#_flat#_sm#'`
        block in=$ndf out=$ndfout accept
     end
\end{terminalv}
\end{small}
The \texttt{\#} is a delimiter for the strings being substituted; it
should be a character that is not present in the strings being
altered.  Notice the \texttt{` `} quotes in the assignment of {\tt
ndfout}.  These instruct the shell to process the expression
immediately, rather than treating it as a literal string.  This is how
you can put values output from UNIX commands and other applications into
shell variables.

\subsection{\xlabel{sc4_se_file_operators}File operators
\label{sc4_se_file_operators}}

\begin{minipage}{62mm}
There is a special class of C-shell operator that lets you test the
properties of a file.  A \textsf{file operator} is used in comparison
expressions of the form \mbox{\texttt{if (file\_operator file) then}}.
A list of file operators is tabulated \latexelsehtml{to the right}{below}.
\medskip

The most common usage is to test for a file's existence.  The following
only runs \textbf{cleanup} if the first argument is an existing file.

\vspace*{9mm}
\end{minipage}
\ \hfill \
\begin{minipage}{74mm}
\vspace*{-16mm}
\begin{tabular}{ll}
\hline
\multicolumn{2}{c}{File operators} \\ \hline
Operator & True if: \\ \hline
\\
\texttt{-d} & file is a directory \\
\texttt{-e} & file exists \\
\texttt{-f} & file is ordinary \\
\texttt{-o} & you are the owner of the file \\
\texttt{-r} & file is readable by you \\
\texttt{-w} & file is writable by you \\
\texttt{-x} & file is executable by you \\
\texttt{-z} & file is empty \\
\\ \hline
\end{tabular}
\end{minipage}
\vspace*{-8mm}

\begin{small}
\begin{terminalv}
     # Check that the file given by the first
     # argument exists before attempting to
     # use it.
     if ( -e $1 ) then
        cleanup $1
     endif
\end{terminalv}
\end{small}

Here are some other examples.

\begin{small}
\begin{terminalv}
     # Remove any empty directories.
     if ( -d $file && -z $file ) then
        rmdir $file

     # Give execute access to text files with a .csh extension.
     else if ( $file:e == "csh" && -f $file ) then
        chmod +x $file
     endif
\end{terminalv}
\end{small}

\newpage
\subsection{\xlabel{sc4_se_create_textfiles}Creating text files
\label{sc4_se_create_textfiles}}

A frequent feature of scripts is redirecting the output from tasks to a
text file.  For instance,

\begin{small}
\begin{terminalv}
     hdstrace $file:r > $file:r.lis
     fitshead $fits > $$.tmp
\end{terminalv}
\end{small}
directs the output of the \xref{\textbf{hdstrace}}{sun102}{} and {\bf
fitshead} to text files. The name of the first is generated from the
name of the file whose contents are being listed, so for \textsf{\HDSref}
file \texttt{cosmic.sdf} the trace is stored in \texttt{cosmic.lis}.  In the
second case, the \htmlref{\textsf{process identification
number}}{sc4_gl_pro} is the name of the text file.  You can include
this special variable to generate the names of temporary files.  (The
\texttt{:r} is described \latexelsehtml{in
Section~\ref{sc4_se_filename_ext}.)}{\htmlref{here}{sc4_se_filename_ext}.)}

If you intend to write more than once to a file you should first create
the file with the \textbf{touch} command, and then append output to the file.

\begin{small}
\begin{terminalv}
     touch logfile.lis
     foreach file (*.sdf)
        echo "FITS headers for $file:r:"  >> logfile.lis
        fitslist $file:r >> logfile.lis
        echo " "
     end
\end{terminalv}
\end{small}
Here we list \htmlref{\textsf{FITS}}{sc4_gl_fits} headers from a series of \NDFrefa{\textsf{NDF}s}
to file \texttt{logfile.lis}.  There is a heading including the dataset
name and blank line between each set of headers.  Notice this time
we use \texttt{>>} to append.  If you try to redirect with \texttt{>}
to an existing file you'll receive an error message whenever you have
the \texttt{noclobber} variable set.  \texttt{>!} redirects regardless
of \texttt{noclobber}.
\bigskip\medskip

There is an alternative---write the text file as part of the script.
This is often a better way for longer files.  It utilises
the \textbf{cat} command to create the file.

\begin{small}
\begin{terminalv}
   cat >! catpair_par.lis   <<EOF
   ${refndf}.txt
   ${compndf}.TXT
   ${compndf}_match.TXT
   C
   XPOS
   YPOS
   XPOS
   YPOS
   $distance
   `echo $time | awk '{print substr($0,1,5)}'`
   EOF
\end{terminalv}
\end{small}
The above writes the text between the two \texttt{EOF}s to file {\tt
catpair\_par.lis}.  Note the second \texttt{EOF} must begin in column 1.
You can choose the delimiting words; common ones are \texttt{EOF}, {\tt
FOO}.  Remember the \texttt{>!} demands that the file be written
regardless of whether the file already exists.

A handy feature is that you can embed \htmlref{\textsf{shell
variables}}{sc4_gl_she_var}, such as \texttt{refndf} and \texttt{distance}
in the example.  You can also include commands between left quotes
(\texttt{`~`}); the commands are evaluated before the file is written.
However, should you want the special characters \texttt{\$}, \verb+\+,
and \texttt{`~`} to be treated literally insert a \verb+\+ before the
delimiting word or a \verb+\+ before each special character.

\subsubsection{\xlabel{sc4_se_write_script}Writing a script within a
script\label{sc4_se_write_script}}

The last technique might be needed if your script writes another
script, say for overnight batch processing, in which you want a command
to be evaluated when the second script is run, not the first.
You can also write files within this second script, provided you choose
different words to delimit the file contents.  Here's an example which
combines both of these techniques.

\begin{small}
\begin{terminalv}
   cat >! /tmp/${user}_batch_${runno}.csh    <<EOF
   #!/bin/csh

   # Create the data file.
   cat >! /tmp/${user}_data_${runno}  <<EOD
   $runno
   \`date\`
   `star/bin/kappa/calc exp="LOG10($C+0.01*$runno)"`
   EOD

   <commands to perform the data processing using the data file>

   # Remove the temporary script and data files.
   rm /tmp/${user}batch_${runno}.csh
   rm /tmp/${user}_data_${runno}

   exit
   EOF
   chmod +x  /tmp/${user}_batch_${runno}.csh
\end{terminalv}
\end{small}

This excerpt writes a script in the temporary directory.  The
temporary script's filename includes our username (\texttt{\$user}) and
some run number stored in variable \texttt{runno}.  The temporary script
begins with the standard comment indicating that it's a C-shell
script.  The script's first action is to write a three-line data file.
Note the different delimiter, \texttt{EOD}.  This data file is created
only when the temporary script is run.  As we want the time and date
information at run time to be written to the data file, the command
substitution backquotes are both escaped with a \verb+\+.  In
contrast, the final line of the data file is evaluated before the
temporary script is written.  Finally, the temporary script removes
itself and the data file.  After making a temporary script, don't
forget to give it execute permission.

\subsection{\xlabel{sc4_se_read_lines}Reading lines from a text file\label{sc4_se_read_lines}}

There is no simple file reading facility in the C-shell.  So we call
upon \textbf{awk} again.

\begin{small}
\begin{terminalv}
     set file = `awk '{print $0}' sources.lis`
\end{terminalv}
\end{small}
Variable \texttt{file} is a space-delineated array of the lines in file
\texttt{sources.lis}.  More useful is to extract a line at a time.

\begin{small}
\begin{terminalv}
     set text = `awk -v ln=$j '{if (NR==ln) print $0}' sources.lis`
\end{terminalv}
\end{small}
where \htmlref{\textsf{shell variable}}{sc4_gl_she} \texttt{j} is a positive
integer and no more than the number of lines in the file, returns the $j$th
line in variable \texttt{text}.

\newpage
\subsection{\xlabel{sc4_se_read_table}Reading tabular
data\label{sc4_se_read_table}}

When reading data into your script from a text file you will often
require to extract columns of data, determine the number of lines
extracted, and sometimes the number columns and selecting columns by
heading name.  The shell does not offer file reading commands, so we
fall back heavily on our friend \textbf{awk}.

\subsubsection{\xlabel{sc4_se_find_nf}Finding the number of fields\label{sc4_se_find_nf}}

The simple recipe is
\begin{small}
\begin{terminalv}
     set ncol = `awk '{if (FNR==1) print NF}' fornax.dat`
\end{terminalv}
\end{small}
where \texttt{FNR} is the number of the records read.  \texttt{NF} is the number of
space-separated fields in that record.  If another character delimits
the columns, this can be set by assigning the \texttt{FS} variable without
reading any of the records in the file (because of the \texttt{BEGIN}
pattern or through the \texttt{-F} option).
\begin{small}
\begin{terminalv}
    set nlines = `wc fornax.dat`

    set ncol = `awk 'BEGIN { FS = ":" }{if (FNR==1) print NF}' fornax.dat`
    set ncol = `awk -F: '{if (FNR==1) print NF}' fornax.dat`
\end{terminalv}
\end{small}
\texttt{FNR}, \texttt{NF}, and \texttt{FS} are called \emph{built-in variables}.

There may be a header line or some schema before the actual data, you
can obtain the field count from the last line.
\begin{small}
\begin{terminalv}
     set ncol = `awk -v nl=$nlines[1] '{if (FNR==nl) print NF}' fornax.dat`
\end{terminalv}
\end{small}
First we obtain the number of lines in the file using \textbf{wc} stored
in \texttt{\$lines[1]}.  This shell variable is passed into \textbf{awk}, as
variable \texttt{nl}, through the \texttt{-v} option.

If you know the comment line begins with a hash (or can be recognised
by some other \htmlref{\textsf{regular expression}}{sc4_gl_reg_exp}) you
can do something like this.
\begin{small}
\begin{terminalv}
     set ncol = `awk -v i=0 '{if ($0 !~ /^#/) i++; if (i==1) print NF}' fornax.dat`
\end{terminalv}
\end{small}
Here we initialise \textbf{awk} variable \texttt{i}.  Then we test the record {\tt
\$0} does not match a line starting with \texttt{\#} and increment \texttt{i},
and only print the number of fields for the first such occurrence.


\subsubsection{Extracting columns}

For a simple case without any comments.
\begin{small}
\begin{terminalv}
     set col1 = `awk '{print $1}' fornax.dat`
\end{terminalv}
\end{small}
Variable \texttt{col1} is an array of the values of the first column.  If
you want an arbitrary column
\begin{small}
\begin{terminalv}
     set col$j = `awk -v cn=$j '{print $cn}' fornax.dat`
\end{terminalv}
\end{small}
where \htmlref{\textsf{shell variable}}{sc4_gl_she} \texttt{j} is a positive
integer and no more than the \htmlref{number of
columns}{sc4_se_find_nf}, returns the $j$th column in the shell array
\texttt{col}$j$.

If there are comment lines to ignore, say beginning with \texttt{\#} or
\texttt{*}, the following excludes those from the array of values.
\begin{small}
\begin{terminalv}
     set col$j = `awk -v cn=$j '$0!~/^[#*]/ {print $cn}' fornax.dat`
\end{terminalv}
\end{small}
or you may want to exclude lines with alphabetic characters.
\begin{small}
\begin{terminalv}
     set col2 = `awk '$0!~/[A-DF-Za-df-z]/ {print $2}' fornax.dat`
\end{terminalv}
\end{small}
Notice that \texttt{E} and \texttt{e} are omitted to allow for exponents.

\subsubsection{Selecting a range}

\textbf{awk} lets you select the lines to extract through boolean
expressions, that includes involving the column data themselves, or
line numbers through \texttt{NR}.
\begin{small}
\begin{terminalv}
     set col$j = `awk -v cn=$j '$2 > 0.579 {print $cn}' fornax.dat`
     set col$j = `awk -v cn=$j '$2>0.579 && $2<1.0 {print $cn}' fornax.dat`
     set col4 = `awk  'sqrt($1*$1+$2*$2) > 1 {print $4};' fornax.dat`
     set col2 = `awk 'NR<=5 || NR>10 {print $2}' fornax.dat`
     set col2 = `awk '$0!~/-999/ {print $2}' fornax.dat`

     set nrow = $#col2.
\end{terminalv}
\end{small}
The first example only includes those values in column 2 that exceed
0.579.  The second further restricts the values to be no greater than 1.0.
The third case involves the square-root of the sum of the squares of
the first two columns.  The fourth omits the sixth to tenth rows.
The fifth example tests for the absence of a null value, \texttt{-999}.

You can find out how many values were extracted through \texttt{\$\#}{\em
var}, such as in the final line above.

You have the standard \htmlref{relational and boolean operators}{sc4_tab_comp_oper}
available, as well as \texttt{$\sim$} and \texttt{!$\sim$} for match and does not match
respectively.  These last two con involve \htmlref{regular expressions}{sc4_gl_reg_exp}
giving powerful selection tools.

\subsubsection{Choosing columns by name}

Suppose your text file has a heading line listing the names of the
columns.
\begin{small}
\begin{terminalv}
    set name = Vmag
    set cn = `awk -v col=$name '{if (NR==1) {for(i=1;i<=NF;\
              i++) {if ($i==col) {print i; break}}}}' fornax.dat`
\end{terminalv}
\end{small}
That looks complicated, so let's go through it step by step.  We
supply the required column name \texttt{name} into the \textbf{awk} variable
\texttt{col} through to \texttt{-v} command-line option.  For the first
record \texttt{NR==1}, we loop through all the fields (\texttt{NF} starting
at the first, and if the current column name (\texttt{\$i}) equals the
requested name, the column number is printed and we break from the
loop.  If the field is not present, the result is null.  The extra
braces associate commands in the same \textbf{for} or \textbf{if} block. Note that
unlike C-shell, in \textbf{awk} the line break can only appear
immediately after a semicolon or brace.

The above can be improved upon using the \texttt{toupper} function to avoid
case sensitivity.
\begin{small}
\begin{terminalv}
    set name = Vmag
    set cn = `awk -v col=$name '{if (NR==1) {for(i=1;i<=NF;\
              i++) {if (toupper($i)==toupper(col)) {print i; break}}}}' fornax.dat`
\end{terminalv}
\end{small}
Or you could attempt to \htmlref{match}{sc4_gl_match} a
\htmlref{regular expression}{sc4_gl_reg_exp}.

\subsection{\xlabel{sc4_se_reading_files}Reading from dynamic text files
\label{sc4_se_reading_files}}

You can also read from a text file created dynamically from within
your script.

\begin{small}
\begin{terminalv}
     ./doubleword < mynovel.txt

     myprog <<BAR
     320 512
     $nstars
     `wc -l < brightnesses.txt`
     testimage
     BAR

     commentary <<\foo
        The AITCH package offers unrivalled facilities.
        It is also easy to use because of its GUI interface.

                   Save $$$ if you buy now.
     foo
\end{terminalv}
\end{small}
Command \texttt{./doubleword} reads its \htmlref{\textsf{standard
input}}{sc4_gl_std_out} from the file \texttt{mynovel.txt}.  The {\tt
<<}\emph{word\/} obtains the input data from the script file itself
until there is line beginning \emph{word}.  You may also include
variables and commands to execute as the \texttt{\$},
\verb+\+, and \texttt{`~`} retain their special meaning.  If you want these
characters to be treated literally, say to prevent substitution, insert
a \verb+\+ before the delimiting \emph{word}.  The command \texttt{myprog}
reads from the script, substituting the value of variable \texttt{nstars}
in the second line, and the number of lines in file {\tt
brightnesses.txt} in the third line.

The technical term for such files are \emph{here documents}.

\subsection{\xlabel{sc4_se_junk_output}Discarding text output
\label{sc4_se_junk_output}}

The output from some routines is often unwanted in scripts.  In these
cases redirect the \htmlref{\textsf{standard output}}{sc4_gl_std_out} to a
null file.

\begin{small}
\begin{terminalv}
     correlate in1=frame1 in2=frame2 out=framec > /dev/null
\end{terminalv}
\end{small}
Here the text output from the task \textbf{correlate} is disposed of to
the \texttt{/dev/null} file.  Messages from Starlink tasks and usually
Fortran channel 6 write to \textsf{standard output}.

\newpage
\subsection{\xlabel{sc4_se_dataset_attributes}Obtaining dataset
attributes\label{sc4_se_dataset_attributes}}

When writing a data-processing pipeline connecting several applications
you will often need to know some attribute of the data file, such as
its number of dimensions, its shape, whether or not it may contain bad
pixels, a variance array or a specified extension.  The way to
access these data is with \xref{\textbf{ndftrace}}{sun95}{NDFTRACE} from
\KAPPAref\ and \xref{\textbf{parget}}{sun95}{PARGET} commands.
\textbf{ndftrace} inquires the data, and \textbf{parget} communicates the
information to a \htmlref{shell variable}{sc4_se_variables}.

\subsubsection{\xlabel{sc4_se_dataset_shape}Obtaining dataset shape
\label{sc4_se_dataset_shape}}

Suppose that you want to process all the two-dimensional \NDFrefa{{\sf
NDF}s} in a directory.  You would write something like this in your
script.

\begin{small}
\begin{terminalv}
     foreach file (*.sdf)
        ndftrace $file:r > /dev/null
        set nodims = `parget ndim ndftrace`
        if ( $nodims == 2 ) then
           <perform the processing of the two-dimensional datasets>
        endif
     end
\end{terminalv}
\end{small}

Note although called \xref{\textbf{ndftrace}}{sun95}{NDFTRACE}, this
function can determine the properties of foreign data formats through
the automatic conversion system (\xref{SUN/55}{sun55}{},
\xref{SSN/20}{ssn20}{}).  Of course, other formats do not have all the
facilities of an NDF.

If you want the dimensions of a \htmlref{\textsf{FITS}}{sc4_gl_fits} file
supplied as the first argument you need this ingredient.

\begin{small}
\begin{terminalv}
     ndftrace $1 > /dev/null
     set dims = `parget dims ndftrace`
\end{terminalv}
\end{small}
Then \texttt{dims[$i$]} will contain the size of the $i^{\rm th}$
dimension.  Similarly

\begin{small}
\begin{terminalv}
     ndftrace $1 > /dev/null
     set lbnd = `parget lbound ndftrace`
     set ubnd = `parget ubound`
\end{terminalv}
\end{small}
will assign the pixel bounds to arrays \texttt{lbnd} and \texttt{ubnd}.

\subsubsection{\xlabel{sc4_se_dataset_atlist}Available attributes
\label{sc4_se_dataset_atlist}}

Below is a complete list of the results parameters from
\xref{\textbf{ndftrace}}{sun95}{NDFTRACE}.  If the parameter is an array,
it will have one element per dimension of the data array (given by
parameter NDIM); except for EXTNAM and EXTTYPE where there is one
element per extension (given by parameter NEXTN). Several of the axis
parameters are only set if the \textbf{ndftrace} input keyword {\tt
fullaxis} is set (not the default).  To obtain, say, the data type of
the axis centres of the current dataset, the code would look like
this.

\begin{small}
\begin{terminalv}
     ndftrace fullaxis accept > dev/null
     set axtype = `parget atype ndftrace`
\end{terminalv}
\end{small}

\newpage
\begin{center}
\begin{tabular}{lcp{112mm}}
Name & Array? & Meaning \\ \hline
\\
AEND & Yes & The axis upper extents of the NDF.  For non-monotonic axes,
             zero is used.  See parameter AMONO.  This is not assigned if
             AXIS is \texttt{FALSE}. \\
AFORM & Yes & The storage forms of the axis centres of the NDF.  This is
              only written when parameter FULLAXIS is \texttt{TRUE} and AXIS
              is \texttt{TRUE}. \\
ALABEL & Yes & The axis labels of the NDF.  This is not assigned if AXIS is
               \texttt{FALSE}. \\
AMONO & Yes &  These are \texttt{TRUE} when the axis centres are monotonic, and \texttt{FALSE}
               otherwise.  This is not assigned if AXIS is \texttt{FALSE}. \\
ANORM & Yes &  The axis normalisation flags of the NDF.  This is only written
               when FULLAXIS is \texttt{TRUE} and AXIS is \texttt{TRUE}. \\
ASTART & Yes &  The axis lower extents of the NDF.  For non-monotonic axes,
                zero is used.  See parameter AMONO.  This is not assigned if
                AXIS is \texttt{FALSE}. \\
ATYPE & Yes & The data types of the axis centres of the NDF.  This is only
              written when FULLAXIS is \texttt{TRUE} and AXIS is \texttt{TRUE}. \\
AUNITS & Yes & The axis units of the NDF.  This is not assigned if AXIS is
               \texttt{FALSE}. \\
AVARIANCE & Yes & Whether or not there are axis variance arrays present in the
                  NDF.  This is only written when FULLAXIS is \texttt{TRUE} and AXIS is
                  \texttt{TRUE}.\\
AXIS & & Whether or not the NDF has an axis system. \\
BAD & & If \texttt{TRUE}, the NDF's data array may contain bad values. \\
BADBITS & & The BADBITS mask.  This is only valid when QUALITY is \texttt{TRUE}. \\
CURRENT & &  The integer Frame index of the current co-ordinate Frame
             in the WCS component. \\
DIMS & Yes & The dimensions of the NDF. \\
EXTNAME & Yes & The names of the extensions in the NDF.  It is only written
                when NEXTN is positive. \\
EXTTYPE & Yes & The types of the extensions in the NDF.  Their order
                corresponds to the names in EXTNAME.  It is only written when
                NEXTN is positive. \\
FDIM & Yes & The numbers of axes in each co-ordinate Frame stored in the WCS
             component of the NDF. The elements in this parameter correspond to
             those in FDOMAIN and FTITLE. The number of elements in
             each of these parameters is given by NFRAME. \\
FDOMAIN & Yes & The domain of each co-ordinate Frame stored in the WCS component
                of the NDF. The elements in this parameter correspond to those in
                FDIM and FTITLE. The number of elements in each of these
                parameters is given by NFRAME. \\
FLABEL & Yes & The axis labels from the current WCS Frame of the NDF. \\
FLBND  & Yes & The lower bounds of the bounding box enclosing the NDF in the
               current WCS Frame. The number of elements in this parameter is equal
               to the number of axes in the current WCS Frame (see FDIM).\\
FORM & & The storage form of the NDF's data array. \\
FTITLE & Yes & The title of each co-ordinate Frame stored in the WCS component
                of the NDF. The elements in this parameter correspond to those in
                FDOMAIN and FDIM.  The number of elements in each of these
                parameters is given by NFRAME. \\
\\ \hline
\end{tabular}
\end{center}

\newpage
\begin{center}
\begin{tabular}{lcp{112mm}}
Name & Array? & Meaning \\ \hline
\\
FUBND & Yes & The upper bounds of the bounding box enclosing the NDF in the
               current WCS Frame. The number of elements in this parameter is equal
               to the number of axes in the current WCS Frame (see FDIM). \\
FUNIT & Yes & The axis units from the current WCS Frame of the NDF.\\
HISTORY & & Whether or not the NDF contains HISTORY records. \\
LABEL & & The label of the NDF. \\
LBOUND & Yes & The lower bounds of the NDF. \\
NDIM & & The number of dimensions of the NDF. \\
NEXTN & & The number of extensions in the NDF. \\
NFRAME & & The number of WCS domains described by FDIM, FDOMAIN and
            FTITLE. Set to zero if WCS is \texttt{FALSE}. \\
QUALITY & & Whether or not the NDF contains a QUALITY array. \\
TITLE & & The title of the NDF. \\
TYPE & & The data type of the NDF's data array. \\
UBOUND & Yes & The upper bounds of the NDF.\\
UNITS & & The units of the NDF. \\
VARIANCE & & Whether or not the NDF contains a VARIANCE array. \\
WCS & & Whether or not the NDF has any WCS co-ordinate Frames, over and
        above the default GRID, PIXEL and AXIS Frames. \\
WIDTH & Yes & Whether or not there are axis width arrays present in
              the NDF.  This is only written when FULLAXIS is \texttt{TRUE} and AXIS is
              \texttt{TRUE}. \\
\\ \hline
\end{tabular}
\end{center}

\newpage
\subsubsection{\xlabel{sc4_se_dataset_variance}Does the dataset have
variance/quality/axis/history information?\label{sc4_se_dataset_variance}}

Suppose you have an application which demands that variance information
be present, say for optimal extraction of spectra, you could test for
the existence of a variance array in your \htmlref{{\sf
FITS}}{sc4_gl_fits} file called \texttt{dataset.fit} like this.

\begin{small}
\begin{terminalv}
     #  Enable automatic conversion
     convert         # Needs to be invoked only once per process

     set file = dataset.fit
     ndftrace $file > /dev/null
     set varpres = `parget variance ndftrace`
     if ( $varpres == "FALSE" ) then
        echo "File $file does not contain variance information"
     else
        <process the dataset>
     endif
\end{terminalv}
\end{small}
The logical results parameters have values \texttt{TRUE} or \texttt{FALSE}\@.
You merely substitute another component such as quality or axis in the
\xref{\textbf{parget}}{sun95}{PARGET} command to test for the presence of
these components.

\subsubsection{\xlabel{sc4_se_dataset_badpix}Testing for bad pixels
\label{sc4_se_dataset_badpix}}

Imagine you have an application which could not process bad pixels.
You could test whether a dataset \emph{might\/} contain bad pixels, and
run some pre-processing task to remove them first.  This attribute
could be inquired via \xref{\textbf{ndftrace}}{sun95}{NDFTRACE}.  If you
need to know whether or not any were actually present, you should run
\xref{\textbf{setbad}}{sun95}{SETBAD} from \KAPPAref\ first.

\begin{small}
\begin{terminalv}
     setbad $file
     ndftrace $file > /dev/null
     set badpix = `parget bad ndftrace`
     if ( badpix == "TRUE" ) then
        <remove the bad pixels>
     else
        goto tidy
     endif
     <perform data processing>

     tidy:
     <tidy any temporary files, windows etc.>
     exit
\end{terminalv}
\end{small}
Here we also introduce the \textbf{goto} command---yes there really is
one.  It is usually reserved for exiting (\texttt{goto exit}), or, as here,
moving to a named label.  This lets us skip over some code, and move
directly to the closedown tidying operations.  Notice the colon
terminating the label itself, and that it is absent from the \textbf{goto}
command.

\newpage
\subsubsection{\xlabel{sc4_se_dataset_spectrum}Testing for a spectral
dataset\label{sc4_se_dataset_spectrum}}

One recipe for testing for a spectrum is to look at the axis labels.
(whereas a modern approach might use WCS information). Here is a
longer example showing how this might be implemented.
Suppose the name of the dataset being probed is stored in variable
\texttt{ndf}.

\begin{small}
\begin{terminalv}
     # Get the full attributes.
     ndftrace $ndf fullaxis accept > /dev/null

     # Assign the axis labels and number of dimensions to variables.
     set axlabel = `parget atype ndftrace`
     set nodims = `parget ndim`

     # Exit the script when there are too many dimensions to handle.
     if ( $nodims > 2 ) then
        echo Cannot process a $nodims-dimensional dataset.
        goto exit
     endif

     # Loop for each dimension or until a spectral axis is detected.
     set i = 1
     set spectrum = FALSE
     while ( $i <= nodims && $spectrum == FALSE )

     # For simplicity the definition of a spectral axis is that
     # the axis label is one of a list of acceptable values.  This
     # test could be made more sophisticated.  The toupper converts the
     # label to uppercase to simplify the comparison.  Note the \ line
     # continuation.
        set uaxlabel = `echo $axlabel[$i] | awk '{print toupper($0)}'`
        if ( $uaxlabel == "WAVELENGTH" || $uaxlabel == "FREQUENCY" \
             $uaxlabel == "VELOCITY" ) then

     # Record that the axis is found and which dimension it is.
           set spectrum = TRUE
           set spaxis = $i
        endif
        @ i++
     end

     # Process the spectrum.
     if ( $spectrum == TRUE ) then

     # Rotate the dataset to make the spectral axis along the first
     # dimension.
        if ( $spaxis == 2 ) then
           irot90 $file $file"_rot" accept

     # Fit the continuum.
           sfit spectrum=$file"_rot" order=2 output=$file"_fit" accept
        else
           sfit spectrum=$file order=2 output=$file"_fit accept
        end if
     endif
\end{terminalv}
\end{small}

\newpage
\subsection{\xlabel{sc4_se_FITS_headers}FITS Headers\label{sc4_se_FITS_headers}}

Associated with \htmlref{\textsf{FITS}}{sc4_gl_fits} files and many
\NDFrefa{\textsf{NDF}s} is header information stored in 80-character
`cards'.  It is possible to use these ancillary data in your script.
Each non-comment header has a keyword, by which you can reference it;
a value; and usually a comment.  \KAPPAref\ from V0.10 has a few
commands for processing \textsf{FITS} header information described in the
following sections.

\subsubsection{Testing for the existence of a FITS header value}

Suppose that you wanted to determine whether an NDF called image123
contains an AIRMASS keyword in its \htmlref{\textsf{FITS}}{sc4_gl_fits}
headers (stored in the FITS extension).

\begin{small}
\begin{terminalv}
     set airpres = `fitsexist image123 airmass`
     if ( $airpres == "TRUE" ) then
        <access AIRMASS FITS header>
     endif
\end{terminalv}
\end{small}
Variable \texttt{airpres} would be assigned \texttt{"TRUE"} when the AIRMASS
card was present, and \texttt{"FALSE"} otherwise.  Remember that the
\texttt{`~`} quotes cause the enclosed command to be executed.

\subsubsection{Reading a FITS header value}

Once we know the named header exists, we can then assign its value
to a \htmlref{shell variable}{sc4_se_variables}.

\begin{small}
\begin{terminalv}
     set airpres = `fitsexist image123 airmass`
     if ( $airpres == "TRUE" ) then
        set airmass = `fitsval image123 airmass`
        echo "The airmass for image123 is $airmass."
     endif
\end{terminalv}
\end{small}

\subsubsection{Writing or modifying a FITS header value}

We can also write new headers at specified locations (the default being
just before the END card), or revise the value and/or comment of existing
headers.  As we know the header AIRMASS exists in image123, the
following revises the value and comment of the AIRMASS header.
It also writes a new header called FILTER immediately preceding the
AIRMASS card assigning it value \texttt{B} and comment \texttt{Waveband}.

\begin{small}
\begin{terminalv}
     fitswrite image123 airmass value=1.062 comment=\"Corrected airmass\"
     fitswrite image123 filter position=airmass value=B comment=Waveband
\end{terminalv}
\end{small}

As we want the \htmlref{\textsf{metacharacters}}{sc4_gl_met} \texttt{"} to be
treated literally, each is preceded by a backslash.

\subsection{\xlabel{sc4_se_other_objects}Accessing other objects
\label{sc4_se_other_objects}}

You can manipulate data objects in \HDSref\ files, such as components
of an \NDFrefa{\textsf{NDF}'s} extension.  There are several Starlink
applications for this purpose including the \Figaroref\ commands
\xref{\textbf{copobj}}{sun86}{COPOBJ},
\xref{\textbf{creobj}}{sun86}{CREOBJ}, \xref{\textbf{delobj}}{sun86}{DELOBJ},
\xref{\textbf{renobj}}{sun86}{RENOBJ}, \xref{\textbf{setobj}}{sun86}{SETOBJ}; and
the \KAPPAref\ commands \xref{\textbf{setext}}{sun95}{SETEXT}, and
\xref{\textbf{erase}}{sun95}{ERASE}.

For example, if you wanted to obtain the value of the EPOCH object
from an extension called IRAS\_ASTROMETRY in an NDF called lmc,
you could do it like this.

\begin{small}
\begin{terminalv}
     set year = `setext lmc xname=iras_astrometry option=get \
                 cname=epoch noloop`
\end{terminalv}
\end{small}
The \texttt{noloop} prevents prompting for another extension-editing
operation.  The single backslash is the line continuation.

\subsection{\xlabel{sc4_se_NDF_section}Defining NDF sections with
variables\label{sc4_se_NDF_section}}

If you want to define a subset or superset of a dataset, most Starlink
applications recognise \xref{\textsf{NDF sections}}{sun95}{se_ndfsect}
\latex{(see SUN/95's chapter called ``NDF Sections'')}
appended after the name.

A na\"{\i}ve approach might expect the following to work

\begin{small}
\begin{terminalv}
     set lbnd = 50
     set ubnd = 120
     linplot $KAPPA_DIR/spectrum"($lbnd:$ubnd)"
     display $KAPPA_DIR/comwest"($lbnd:$ubnd",$lbnd~$ubnd)"
\end{terminalv}
\end{small}
however, they generate the ~~\texttt{Bad : modifier in \$ (\$).}~ error.
That's because it is stupidly looking for a \htmlref{filename
modifier}{sc4_se_filename_ext} \texttt{:\$}\latex{ (see
Section~\ref{sc4_se_filename_ext})}.

Instead here are some recipes that work.

\begin{small}
\begin{terminalv}
     set lbnd = 50
     set ubnd = 120
     set lrange = "101:150"

     linplot $KAPPA_DIR/spectrum"($lbnd":"$ubnd)"
     stats abc"(-20:99,~$ubnd)"
     display $KAPPA_DIR/comwest"($lbnd":"$ubnd",$lbnd":"$ubnd")"
     histogram hale-bopp.fit'('$lbnd':'$ubnd','$lbnd':'$ubnd')'
     ndfcopy $file1.imh"("$lbnd":"$ubnd","$lrange")" $work"1"
     splot hd102456'('$ubnd~60')'
\end{terminalv}
\end{small}
An easy-to-remember formula is to enclose the parentheses and colons
in quotes.

\newpage
\section{\xlabel{sc4_se_loop_times}Loop a specified number of times
\label{sc4_se_loop_times}}
Suppose that you want to loop a specified number of times.  For this
you need a \textbf{while} statement, as in this example.

\begin{small}
\begin{terminalv}
     set count = 1
     set tempin = myndf
     set box = 23
     while ($count < 5)
        @ box = $box - $count * 2
        block in=$tempin out=tempndf box=$box
        mv tempndf.sdf myndf_sm4.sdf
        set tempin = myndf_sm4
        @ count = $count + 1
     end
\end{terminalv}
\end{small}
This performs four block smoothing operations, with a decreasing box
size, on a \NDFrefa{\textsf{NDF}} called myndf, the final result being
stored in the NDF called myndf\_sm4.  The \textbf{while} statement tests
a conditional statement and loops if it is true.  So here it loops
whenever the value of \texttt{count} is less than 5.
\latexelsehtml{For a list of the available operators
see Section~\ref{sc4_se_integer} or the \textbf{man}
pages.}{Click \htmlref{here}{sc4_se_integer} to see examples of other
operators.}

\begin{small}
\begin{terminalv}
     % man csh
     /Expressions
\end{terminalv}
\end{small}

The box size for the smoothing is evaluated in an expression
\mbox{\texttt{@ box = \$box - \$count * 2}}.  Note the space between the
\texttt{@} and the variable, and the spaces around the \texttt{=}.  Thus
the box sizes will be 21, 17, 11, 3.  Further details are
\latexelsehtml{in
Section~\ref{sc4_se_integer}.}{\htmlref{here}{sc4_se_integer}.}
You should also give the variable a value with \textbf{set} before
assigning it an expression.  Another expression increments the counter
\texttt{count}.  The NDF called tempndf is overwritten for each smooth
using the standard UNIX command \textbf{mv}.

\newpage
\section{\xlabel{sc4_se_unix_options}UNIX-style options
\label{sc4_se_unix_options}}

If you want other arguments, or a UNIX-like set of \emph{options\/}
for specifying arguments with defaults, you'll need to use \textbf{switch}
and \textbf{shift} statements, rather than the simple \textbf{while} loop.

Suppose that we want to obtain photometric details of a galaxy, known
to be the brightest object in a series of \NDFrefa{\textsf{NDF}s}.  Before
that's possible we must first measure the background in each image.
To reduce contamination that could bias the determination of the sky
we mask the region containing the galaxy.  Here we obtain the values
for the shape, orientation, and the names of the NDFs, and assign them
to \htmlref{shell variables}{sc4_se_variables}.  There are also default values.

\begin{small}
\begin{terminalv}
     #  Initialise some shell variables, including the default shape
     #  and orientation.
     set major = 82
     set minor = 44
     set orient = 152
     set args = ($argv[1-])
     set ndfs

     #  Check that there are some arguments present.
     if ( $#args == 0 ) then
        echo "Usage: galmask [-a semi-major] [-b semi-minor] " \
                            "[-o orientation] ndf1 [ndf2...]"
        exit
     endif

     #  Process each of the arguments to the script.
     while ( $#args > 0 )
        switch ($args[1])
        case -a:        #  Semi-major axis
           shift args
           set major = $args[1]
           shift args
           breaksw
        case -b:        #  Semi-minor axis
           shift args
           set minor = $args[1]
           shift args
           breaksw
        case -o:        #  Orientation
           shift args
           set orient = $args[1]
           shift args
           breaksw
        case *:         #  The NDFs
           set ndfs = ($ndfs[1-] $args[1])
           shift args
           breaksw
        endsw
     end

     # Loop through the remaining arguments, assuming that these are NDFs.
     foreach file ($ndfs[*])
        :      :      :
\end{terminalv}
\end{small}
Some defaults are defined so that if the argument is not present,
there is a suitable value.  So for instance, the ellipse orientation
is 152\dgs\ unless there is a \mbox{\texttt{-o} {\em
orientation-value\/}} on the command line when invoking the script.

The \textbf{switch} looks for specific arguments.  If the first argument
matches one of the allowed cases: \texttt{-a}, \texttt{-b}, \texttt{-o}, or any
string ({\texttt{*}}) in that order, the script jumps to that case, and
follows the commands there until it encounters the \textbf{breaksw}, and
then the script moves to the \textbf{endsw} statement.  You may be
wondering why only the first element of the arguments is tested against
the cases.  This is where the \textbf{shift} statement comes in.  This
decrements the element indices of an array by one, so that {\tt
\$argv[1]} is lost, \texttt{\$argv[2]} becomes \texttt{\$argv[1]}, {\tt
\$argv[3]} becomes \texttt{\$argv[2]} and so on.  If we look at the
\texttt{-a} case.  The arguments are shifted, so that the first argument
is now the value of the semi-major axis.  This in turn is shifted out
of the arguments.  The \htmlref{\texttt{*} case}{sc4_se_arguments} does
the equivalent of the earlier \textbf{while} loop \latex{(see
Section~\ref{sc4_se_arguments})} as it appends the NDF filenames.  The
\texttt{[1-]} means the first element until the last.

To actually perform the masking we could write an \htmlref{{\sf
ARD}}{sc4_gl_ard} ellipse shape using the variables, and
\htmlref{\textsf{pipe}}{sc4_gl_pipe} it into a
file, in this case \texttt{biggal.ard}.

\begin{small}
\begin{terminalv}
     echo 'ELLIPSE( '$centre[1]', '$centre[2]', '$major', '$minor', '$orient' )' \
          > biggal.ard
\end{terminalv}
\end{small}

On the command line we could enter

\begin{small}
\begin{terminalv}
     % galmask -o 145 -a 78.4 myndf 'ccd*o'
\end{terminalv}
\end{small}
if the script was called \texttt{galmask}.  This would set the semi-major
axis to 78.4 pixels, the orientation to 145\dgs, leaving the
semi-minor axis at its default of 44 pixels.  It would operate on
the NDF called myndf and those whose names began with \texttt{ccd} and
ended with \texttt{o}.

There is a \htmlref{related example}{sc4_se_long1} \latex{in
Section~\ref{sc4_se_long1}} where the ellipse axis lengths and
orientation are fixed for two galaxies.  The above options code could
replace that recipe's first two lines, if we wanted to alter the
spatial parameters for the brighter galaxy to see which gives best
results.

In \texttt{\$KAPPA\_DIR/multiplot.csh} you can find a complete example.
For a series of NDFs, this produces a grid of displayed images with
axes and titles, prints them to a nominated device, once each page is
full.  You can modify it for other operations, where a single page of
graphics output is produced by running several tasks, and joining them
together with \PSMERGEref\latex{ (SUN/164)}.

\newpage
\section{\xlabel{sc4_se_debugging}Debugging scripts
\label{sc4_se_debugging}}

\htmlref{Earlier}{sc4_se_running} \latex{in
Section~\ref{sc4_se_running}} a number of ways to run a
script were presented.  There is yet another, which makes debugging
easier.  If you invoke the script with the \textbf{csh} command you can
specify none or one or more options.  The \textbf{command} takes the form

\begin{small}
\begin{terminalv}
     % csh [options] script
\end{terminalv}
\end{small}
where \texttt{script} is the file.  There is a choice of two debugging
options.

\begin{small}
\begin{terminalv}
     % csh -x myscript
     % csh -v myscript
\end{terminalv}
\end{small}
The \textbf{x} option echoes the command lines after variable substitution,
and the \textbf{v} option echoes the command lines before variable
substitution.  The \textbf{v} is normally used to identify the line at
which the script is failing, and the \textbf{x} option helps to locate
errors in variable substitution.  As these options can produce lots of
output flashing past your eyes, it is often sensible to redirect the
output to a file to be examined at a comfortable pace.

If you prefer, you can run the script in the normal way by putting the
option into the first comment line.

\begin{small}
\begin{terminalv}
     #!/bin/csh -x
\end{terminalv}
\end{small}

\section{\xlabel{sc4_se_break_in}Breaking-in
\label{sc4_se_break_in}}

Sometimes you will realise you have made a mistake after starting a
script, and therefore want to stop it to save time or damage.  You can
break-in using \texttt{CTRL/C} (hitting C while depressing the CTRL key),
but doing this can leave unwanted intermediate files, graphics
windows, and other mess.  However, if you have

\begin{small}
\begin{terminalv}
     onintr label

     <the main body of the script>

     label:
     <perform housekeeping>

     exit
\end{terminalv}
\end{small}
near the head and tail of your script, whenever \texttt{CTRL/C} is
detected, the script will go to the specified label.  There you can
close down the script in a controlled fashion, removing the garbage.  If
you want to prevent \texttt{CTRL/C} interrupts, you should include the line

\begin{small}
\begin{terminalv}
     % onintr -
\end{terminalv}
\end{small}
instead.

\newpage
\section{\xlabel{sc4_se_long_recipes}Longer recipes
\label{sc4_se_long_recipes}}

This section brings together a number of the individual ingredients
seen earlier into lengthier scripts.  So far there is only one.

\subsection{\xlabel{sc4_se_long1}Recipe for masking and background-fitting
\label{sc4_se_long1}}

The following script fits a surface to the background in a series of
\NDFrefa{\textsf{NDF}s}.
First two bright galaxies are to be excluded from the fitting process by
being \xref{masked using an ARD region}{sun95}{se_ardwork}\latex{ (see
SUN/95, ``Doing it the ARD way'')}.  It is known that the brighter
galaxy is the brightest object in the field and the relative
displacement of the two galaxies is constant.

\begin{small}
\begin{terminalv}
     # Loop through the remaining arguments, assuming that these are NDFs.
     foreach file ($ndfs[1-])

     # Obtain the NDF's name.
        set file1=$file:r

     # Obtain the centre of the galaxy, assuming it is the pixel with the
     # largest value.
        stats $file1 > /dev/null

     # Store the maximum (centre) co-ordinates values in shell variables.
        set centre = `parget maxcoord stats`

     # Create a two-line ARD file.  The first is for the bright galaxy, and
     # the other is the second brightest galaxy.  We assume a constant offset.
     # Use CALC to evaluate the expressions, as the centres are strings and
     # might be floating point.
        echo 'ELLIPSE( '$centre[1]', '$centre[2]', 82, 44, 152 )' > $file1".ard"

        set aa = `calc exp=\"$centre[1] + 68\"`
        set bb = `calc exp=\"$centre[2] - 59\"`

        echo 'ELLIPSE( '$aa', '$bb', 30, 25, 105 )' >> $file1".ard"

     # Mask the NDF.
        ardmask $file1 $file1".ard" $file1"_masked" cosys=w
        \rm $file1.ard

     # Do the surface fit.
        echo " "
        echo $file:r
        surfit in=$file1"_masked" out=$file1"_bg" estimator=median \
               fittype=polynomial nxpar=4 nypar=4 ix=16 iy=16 \
               fitclip=\[2,2.5,3\] evaluate=interpolate

     # Perform the sky subtraction.
        sub $file1 $file1"_bg" $file1"_ss"

     # Remove work files.
         \rm $file1"_bg.sdf" $file1"_masked.sdf"
     end

     exit
\end{terminalv}
\end{small}
\xref{\textbf{parget}}{sun95}{PARGET} obtains the $x$-$y$ co-ordinates of
the maximum value in each NDF, and these are stored in variable \texttt{centre}.
Since there are two values, \texttt{centre} is an array.  The value of
the first element \texttt{\$centre[1]} is the $x$ co-ordinate from
\xref{\textbf{stats}}{sun95}{STATS}, and the second \texttt{\$centre[2]} is
the $y$ co-ordinate.  These values are placed in an \htmlref{{\sf
ARD}}{sc4_gl_ard} expression and output to \texttt{\$file".ard"}.
\xref{\textbf{calc}}{sun95}{CALC} applies the offset between the galaxies.  The
second ARD expression is appended to the same text file.  \xref{{\bf
ardmask}}{sun95}{ARDMASK} then masks the galaxies, \xref{{\bf
surfit}}{sun95}{SURFIT} performs the background fit, and subtracts it from the
original NDF.  The intermediate files are removed.  The \textbf{exit}
command meaning exit the script is implied at the end of the file, so it
can usually be omitted.  You might want to call it to leave the script,
when some error is encountered.

If you want to make the shape and orientation arguments of the script,
see the example of \htmlref{UNIX-style options}{sc4_se_unix_options}
\latex{in Section~\ref{sc4_se_unix_options}}.

\newpage
\section{\xlabel{sc4_se_glossary}Glossary\label{sc4_sc:glossary}}

\begin{itemize}

\item {\bf\label{sc4_gl_alias}Alias}\\
      A mechanism for abbreviating a C-shell command line.

\item {\bf\label{sc4_gl_ard}ARD}\\
      ASCII Region Definition.  A syntax for specifying various
      regions of an image in a text file.  Used for masking image data.
      It is described in \xref{SUN/183}{sun183}{}.

\item {\bf\label{sc4_gl_awk}awk}\\
      A pattern-scanning and text-processing language.  In other words
      it is a programmable report generator.  Its name comes from the
      initials of the authors.

\item {\bf\label{sc4_gl_convert}CONVERT}\\
      A Starlink utility package for converting between
      \htmlref{NDF}{sc4_gl_ndf} and various data formats such as
      \htmlref{FITS}{sc4_gl_fits}.  It is described in
      \xref{SUN/55}{sun55}{}.

\item {\bf\label{sc4_gl_cur}Current process}\\
      The task currently running the shell (in the context of this
      document).  C-shell scripts invoked from the current process
      are each run in new (child) processes.

\item {\bf\label{sc4_gl_env}Environment variable}\\
      A global variable to define the characteristics of a UNIX
      environment such as the terminal type or home directory.  It is
      defined with the \textbf{setenv} command.  By convention, environment
      variables appear in uppercase.  Environment variables are often
      used to refer to a directory or to tune software.

\item {\bf\label{sc4_gl_figaro}Figaro}\\
      A general astronomical data-reduction package but concentrating
      on spectroscopy.  It is available in several flavours.  The Starlink
      version is described in \xref{SUN/86}{sun86}{}.

\item {\bf\label{sc4_gl_filmod}File modifier}\\
      Syntax for specifying a part of a filename.  They appear following
      a filename and have the form colon followed by a letter.  For
      example, \texttt{:t} excludes the path.

\item {\bf\label{sc4_gl_fits}FITS}\\
      Flexible Image Transport System (\FITSref).  The most commonly used
      format for astronomical data storage.  It comprises a series of
      text headers followed by image or tabular data.

\item {\bf\label{sc4_gl_hds}HDS}\\
      Hierarchical Data System.  The underlying data system for Starlink
      data files.  It is used to create and access \htmlref{NDF}{sc4_gl_ndf}
      datasets.

\item {\bf\label{sc4_gl_kappa}KAPPA}\\
      The Starlink Kernel Application Package.  A suite of facilities
      for processing and viewing astronomical images.
      Described in \xref{SUN/95}{sun95}{}.

\item {\bf\label{sc4_gl_match}Match}\\
      A string in a file that is successfully specified by a
      \htmlref{regular expression}{sc4_gl_reg_exp}.  It is also
      a filename, \htmlref{shell variable}{sc4_gl_she_var}, or
      \htmlref{alias}{sc4_gl_alias} successfully specified by
      \htmlref{wildcards}{sc4_gl_wild} within the shell.

\item {\bf\label{sc4_gl_met}Metacharacter}\\
      Characters which have special meanings.  For the shell these
      include \htmlref{wildcards}{sc4_gl_wild}, quotes, and logical
      operators.  In \htmlref{regular expressions}{sc4_gl_reg_exp},
      metacharacters are used to specify strings to match.

\item {\bf\label{sc4_gl_ndf}NDF}\\
      The standard Starlink data-storage format.  An hierarchical format for
      multi-dimensional data storage.  Accessed using libraries supported
      by Starlink.  Use of NDF is described in \NDFrefa{SUN/33}.

\item {\bf\label{sc4_gl_ndf_se}NDF Section}\\
      A subset or superset of a dataset (originally applied to just
      \htmlref{NDFs}{sc4_gl_ndf}) defined by specifying the pixel bounds
      or co-ordinate limits along each axis following the dataset's name.
      See \latexelsehtml{SUN/95's chapter called ``NDF
      Sections''}{\xref{here}{sun95}{se_ndfsect}} for a description
      and many examples.

\item {\bf\label{sc4_gl_opar}Output parameters}\\
      A channel through which some Starlink applications write their
      results.  These tend to be applications that do not create an
      output dataset, such as statistics and dataset attributes.  They
      are sometimes called ``Results parameters''.  For examples see
\begin{htmlonly}
      \htmlonly{\htmlref{Passing information between
      Starlink applications}{sc4_se_info_parameter}.}
\end{htmlonly}
      \latex{Section~\ref{sc4_se_passing_info}.}

\item {\bf\label{sc4_gl_path}Path}\\
      A list of directories which the system searches in turn to
      resolve command requests.

\item {\bf\label{sc4_gl_pipe}Pipe, piping}\\
      Mechanism by which the \htmlref{standard output}{sc4_gl_std_out}
      of one programme is passed to the \htmlref{standard
      input}{sc4_gl_std_inp} of another.  It allows sophisticated tools
      to be made from simple commands.  The \texttt{|} character
      represents the pipe.

\item {\bf\label{sc4_gl_pro}Process}\\
      A task being performed by the computer.

\item {\bf\label{sc4_gl_pid}Process identification number}\\
      A positive integer that uniquely identifies a
      \htmlref{process}{sc4_gl_pro} within the system.

\item {\bf\label{sc4_gl_reg_exp}Regular expression}\\
      A pattern of characters used to match against the same
      characters in a search.  They usually include \htmlref{special
      characters}{sc4_gl_met}, which represent things other than
      themselves, to refine the search.  Regular expressions empower
      utilities like \textbf{grep}, \textbf{sed} and \textbf{awk}.  Although
      similar to shell \htmlref{wildcards}{sc4_gl_wild} there are
      differences, so be careful.

%      \begin{tabular}{lp{100mm}}
%      Regular expression & Meaning \\ \hline
%      \texttt{.} & Any single character  \\
%      \texttt{*} & Zero or more characters \\
%      \verb+^+ & Matches only if the string is at the beginning of
%                 a line \\
%      \texttt{[~]} & \\
%      \texttt{\$} & End of line \\
%      \\ \hline
%      \end{tabular}

\item {\bf\label{sc4_gl_sed}sed}\\
      The stream editor.  It is useful for editing large files or
      applying the same edits to a series of files.

\item {\bf\label{sc4_gl_she}Shell}\\
      A programme that listens to your terminal, and accepts and
      interprets the commands you type.  There are several UNIX
      shells including the Bourne (sh), Bourne-again (bash),
      Korn (ksh), as well as the C shell (csh).

\item {\bf\label{sc4_gl_she_var}Shell variable}\\
      An identifier that can store one or more strings.  Variables
      enable string processing, and integer and logical expressions
      in the \htmlref{shell}{sc4_gl_she}.  See
\begin{htmlonly}
      \htmlref{Shell Variables}{sc4_se_variables}
\end{htmlonly}
      \latex{Section~\ref{sc4_se_variables}} for more details.

\item {\bf\label{sc4_gl_std_inp}Standard input}\\
      The file from which most UNIX programmes read their input data.
      It defaults to the terminal if you do not supply the input on
      the command line or a file.

\item {\bf\label{sc4_gl_std_out}Standard output}\\
      File to which most UNIX programmes output their results.  Text
      output from Starlink applications are also routed there.
      Standard output defaults to your terminal.  It can be
      \htmlref{piped}{sc4_gl_pipe} into commands that accept
      \htmlref{standard input}{sc4_gl_std_inp}.

\item {\bf\label{sc4_gl_starlink}Starlink}\\
      UK network of computers for astronomers; a collection of
      software to reduce and analyse astronomical data; and the team
      of people supporting this hardware and software.

\item {\bf\label{sc4_gl_wild}Wildcards}\\
      A shorthand notation to specify filenames,
      \htmlref{aliases}{sc4_gl_alias} or \htmlref{shell
      variables}{sc4_gl_she_var} by supplying a certain special
      characters that represent things other than themselves.
      \medskip


      \begin{tabular}{lp{100mm}}
      Wildcard expression & Matches \\ \hline
      \texttt{*} & Zero or more characters \\
      \texttt{?} & Exactly one character \\
      \texttt{[xyz]} & One character in the set \texttt{x}, \texttt{y}, or \texttt{z} \\
      \texttt{[a-m]} & One character in the range from \texttt{a} to \texttt{m} \\
      \texttt{[A-Za-z]} & All alphabetic characters \\
      \texttt{[0-9]} & All numeric characters \\
      \texttt{\{alpha,beta,a,b\}} & A set of options, \texttt{alpha}, \texttt{beta}, \texttt{a}, or \texttt{b} \\
      \texttt{\{aa,bb[1-3]\}} & \texttt{aa}, \texttt{bb1}, \texttt{bb2}, or \texttt{bb3} \\
      \\ \hline
      \end{tabular}


\end{itemize}

\end{document}
