\chapter{\xlabel{mapstats}\drrecipe{SCUBA2\_MAPSTATS}}
\label{app:mapstats}

The \picard\ recipe \drrecipe{SCUBA2\_MAPSTATS} estimates the RMS
and NEFD for a given observation.

The average NEP is calculated from the QL pipeline log file
(\file{log.nep}) corresponding to the date and wavelength. The FCF
(either from the file or the standard value) and mean transmission
over the observation is used to convert this to a zenith NEFD
(NEFD).

The input images are cropped to the given size (as specified in the
FITS headers or via the MAP\_HEIGHT and MAP\_WIDTH recipe parameters)
before the mean exposure time is derived, along with the mean/median
noise and NEFD (RMS and NEFDs).

The parameters written out to \file{log.mapstats} are listed below:
\begin{table}[h!]
  \begin{center}
    \begin{tabular}{|p{2.5cm}|p{12cm}|}
      \hline
      \textbf{Parameter} & \textbf{Description}\\
      \hline
      UT & UT date including day fraction\\
      HST & HST date and time stamp \\
      Obs & observation number\\
      Source & object name\\
      Mode & observation mode (either daisy or pong)\\
      FILTER & filter (wavelength)\\
      El & mean elevation of observation\\
      Airmass & mean airmass of observation\\
      Trans & mean line-of-sight transmission\\
      Tau225 & mean zenith optical depth at 225 GHz, derived from WVM\\
      telapsed & elapsed time of observation in seconds\\
      texp & mean exposure time, derived from EXP\_TIME NDF component (sec)\\
      rms & RMS noise in map, obtained from median of error array\\
      rms\_units & RMS units\\
      nefd & NEFD derived from combination of variance and exposure time images\\
      nefd\_units & NEFD units\\
      mapsize & size (requested diameter) of input map (arcsec)\\
      pixscale & pixel scale in arcsec\\
      project & project ID\\
      recipe & reduction recipe used\\
      filename & name of input file\\
      \hline
    \end{tabular}
  \end{center}
\end{table}



