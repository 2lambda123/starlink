\chapter{\xlabel{fits}Convert format from FITS to NDF}
\label{app:fits}

It is often useful to utilise data from other wavelengths or
instruments (either for a comparison or for an external mask). In the
following example, the FITS file called \file{file.fits} is
converted to NDF format as \file{file.sdf} using the \starlink\
package \convert. Note that the \file{.sdf} file extension NDF may
be omitted to save typing.

\begin{terminalv}
% convert
% fits2ndf file.fits file.sdf
\end{terminalv}

Unless your FITS file is from a recognised source, \fitstondf\ puts the
various FITS extensions into NDF extensions called FITS\_EXT\_$<n>$,
where $n$ counts from one for the first FITS extension. These are
located in the top-level MORE component of the NDF. If one of these is
the array you want---it should be an NDF too---it will be easier to
handle if you copy out of the NDF extension to its own NDF.

\begin{terminalv}
% ndfcopy in=file.more.fits_ext_1 out=file_data
\end{terminalv}

If there is a variance array stored in another extension that you want
to attach to your new NDF, a command like the following will do that.

\begin{terminalv}
% setvar ndf=file_data from=file.more.fits_ext_2
\end{terminalv}

\task{fits2ndf} does offer a way of mapping FITS extensions to familiar NDF
array components DATA, VARIANCE, and QUALITY through the \param{EXTABLE} file,
avoiding the \ndfcopy\ and possible \setvar\ steps.




