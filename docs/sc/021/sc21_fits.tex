\chapter{\xlabel{fits}Convert Format from FITS to NDF}
\label{app:fits}

It is often useful to utilise data from other wavelengths or
instruments (either for a comparison or for an external mask). In the
following example, the FITS file called \file{file.fits} is
converted to NDF format as \file{file.sdf} using the \starlink\
package \convert. Note that the \file{.sdf} file extension NDF may
be omitted to save typing.

\begin{terminalv}
% convert
% fits2ndf file.fits file.sdf
\end{terminalv}

FITS files from certain recognised sources have special rules applied
when converting from FITS to NDF, as described in the documentation for
\xref{\task{fits2ndf}}{sun55}{FITS2NDF}. For FITS files from other sources, the
primary array in the FITS file is stored as the main NDF in the output
file. Any FITS extensions present in the FITS file will be placed into
NDF extensions called FITS\_EXT\_$<n>$, where $n$ counts from one for
the first FITS extension.  To see a list of the extension NDFs in
\file{fred.sdf}, do:

\begin{terminalv}
% ndfecho fred.more
fred.MORE.FITS_EXT_1
fred.MORE.FITS_EXT_2
fred.MORE.FITS_EXT_3
\end{terminalv}

When running a \textsc{Kappa} ot \textsc{smurf} command, you can refer
to these extension just as they are listed above. So for instance:

\begin{terminalv}
% ndftrace fred.MORE.FITS_EXT_1

   NDF structure /home/dsb/fred.MORE.FITS_EXT_1:
      Units:  COUNTS/S

   Shape:
      No. of dimensions:  2
      Dimension size(s):  270 x 263
      Pixel bounds     :  1:270, 1:263
      Total pixels     :  71010
...
...
\end{terminalv}

Alternatively, you can copy the NDF into its own separate file:

\begin{terminalv}
% ndfcopy in=fred.MORE.FITS_EXT_1 out=new_file
\end{terminalv}

If one of the extensions contains a variance array that you would like to
as the \texttt{Variance} component of the main NDF, a command like the
following will do that:

\begin{terminalv}
% setvar ndf=new_file from=fred.MORE.FITS_EXT_2
\end{terminalv}

The \task{fits2ndf} command offers a way of mapping FITS extensions to
familiar NDF array components DATA, VARIANCE, and QUALITY
through the \param{EXTABLE} file, avoiding the \ndfcopy\ and possible
\setvar\ steps.

You can convert an NDF to a FITS file using the command
\xref{\task{ndf2fits}}{sun55}{NDF2FITS}:

\begin{terminalv}
% convert
% ndf2fits file.sdf file.fits
\end{terminalv}




