% This file was auto-generated and should not be edited!!!
% It was created using:
%
%	applications/smurf/defaults/make_pardocs \
%		selected_params.lis \
%		selected_params.tex \
%		../../../applications/smurf/defaults 
%
\sstroutine{
   AST.MAPSPIKE
}{
   Removes spikes from the map
}{
   \sstdescription{
      If ast.mapspike is non-zero, spikes in the time-series
      residuals will be identified by looking at the spread of
      residual values that contribute to each map pixel. Any
      residuals that are above ast.mapspike standard deviations
      from the mean value in the pixel are flagged as spikes. [10.0]
   }
   \sstattributetype{
      real
   }
   \sstdiytopic{
      SMURF Usage
   }{
      MAKEMAP, CALCQU
   }
}
\sstroutine{
   AST.SKIP
}{
   Skip subtraction of astronomical signal
}{
   \sstdescription{
      If ast.skip is non-zero, it gives the number of initial
      iterations for which no AST model (astronomical signal)
      should be subtracted. A map is still formed at the end of
      these iterations but the astronomical signal implied by this
      map is not removed from the residuals (neither is it added
      back prior to forming the next map). This means that the
      residuals at the start of each of these iterations are
      unchanging and essentially equal to the cleaned raw data.
      The value supplied for parameter {\tt{"}}\xref{numiter}{sun258}{NUMITER}{\tt{"}} should therefore
      be greater than the value of {\tt{"}}ast.skip{\tt{"}}.

      This option is useful, for instance, when using SNR-based
      masking for the FLT and/or COM models, since it allows a
      reasonable mask to be formed before subtracting off the
      first estimate of the astronommical signal.

      If ast.skip is set to a negative value, it gives the
      largest number of iterations to perform and indicates
      that the AST model should be skipped on all of them. In
      this case the value supplied for parameter {\tt{"}}\xref{numiter}{sun258}{NUMITER}{\tt{"}} is
      ignored. [0]
   }
   \sstattributetype{
      integer
   }
   \sstdiytopic{
      SMURF Usage
   }{
      MAKEMAP, CALCQU
   }
}
\sstroutine{
   AST.ZERO\_CIRCLE
}{
   Reduces spurious large scale structure in the final map
   outside a circle of given radius
}{
   \sstdescription{
      Using ast.zero\_circle defines a circle on the map outside
      of which the map will be constrained to zero on each
      iteration (but see parameter {\tt{"}}\xref{ast.zero\_notlast}{sun258}{AST.ZERO_NOTLAST}{\tt{"}}). If a value
      is supplied for this parameter, it can be a single real value,
      or a comma-separated list of three real values in parentheses.
      If one value is supplied, it should be the radius of the
      circle in decimal degrees (the centre of the circle defaults
      to the coordinates at the tangent point of the map). If three
      values are supplied they should be the central longitude,
      latitude and radius of the circle, in decimal degrees, in the
      coordinate system of the map (e.g., RA and Dec.). [{\tt $<$undef$>$}]
   }
   \sstattributetype{
      real
   }
   \sstdiytopic{
      SMURF Usage
   }{
      MAKEMAP, CALCQU
   }
}
\sstroutine{
   AST.ZERO\_FREEZE
}{
   Prevent the AST mask from changing after a given number
   of iterations. This can help convergence
}{
   \sstdescription{
      If ast.zero\_freeze is non-zero, the AST mask will be frozen
      after the specified number of iterations. A value of zero
      means that the mask is never frozen. Note, any initial
      iterations specified by parameter {\tt{"}}\xref{ast.skip}{sun258}{AST.SKIP}{\tt{"}} are not
      included in the count of iterations. [0]
   }
   \sstattributetype{
      integer
   }
   \sstdiytopic{
      SMURF Usage
   }{
      MAKEMAP, CALCQU
   }
}
\sstroutine{
   AST.ZERO\_LOWHITS
}{
   Reduces spurious large scale structure in the final map
   in regions containing few data samples
}{
   \sstdescription{
      Using ast.zero\_lowhits causes the map to be forced to
      zero in regions where the number of samples falling in
      each pixel is less than ast.zero\_lowhits times the mean
      number of samples per pixel, averaged over the map. A
      value of zero means that no masking of low hits regions
      is performed. The mask is updated on each iteration. [0]
   }
   \sstattributetype{
      real
   }
   \sstdiytopic{
      SMURF Usage
   }{
      MAKEMAP, CALCQU
   }
}
\sstroutine{
   AST.ZERO\_MASK
}{
   Reduces spurious large scale structure in the final map
   within fixed regions specified by an external mask
}{
   \sstdescription{
      If ast.zero\_mask is set to one of {\tt{"}}REF{\tt{"}}, {\tt{"}}MASK2{\tt{"}} or
      {\tt{"}}MASK3{\tt{"}} then an NDF will be obtained using the specified
      ADAM parameter (REF, MASK2 or MASK3) and used as a
      user-defined mask. Setting ast.zero\_mask to an integer value
      larger than zero has the same effect as setting it to {\tt{"}}REF{\tt{"}}.
      Setting it to an integer less than or equal to zero results
      in no external mask being used with the AST model. Note,
      using {\tt{"}}REF{\tt{"}} ensures that the mask and the output image of
      MAKEMAP are on the same pixel grid - using {\tt{"}}MASK2{\tt{"}} or {\tt{"}}MASK3{\tt{"}}
      does not provide this guarantee (it is then the users
      responsibility to ensure that the supplied masks are aligned
      with the output image in pixel coordinates). The pixels in
      the map that are to be constrained to 0 should be set to the
      bad value in the mask. All other pixels will be allowed to
      vary during map-making. [0]
   }
   \sstattributetype{
      integer or string
   }
   \sstdiytopic{
      SMURF Usage
   }{
      MAKEMAP, CALCQU
   }
}
\sstroutine{
   AST.ZERO\_NITER
}{
   Allows AST masking to be switched off after a given
   number of iterations
}{
   \sstdescription{
      If ast.zero\_niter is non-zero, it gives the number of
      iterations for which the AST model should be masked.
      Subsequent iterations are not masked. A value of zero
      means {\tt{"}}mask on all iterations{\tt{"}}. However, if parameter
      {\tt{"}}\xref{ast.zero\_notlast}{sun258}{AST.ZERO_NOTLAST}{\tt{"}} is set, the mask will will not be
      applied on the last iteration, even if ast.zero\_niter
      is zero. This feature will probably be useful for deep
      point-source observations for which the large-scale noise
      is not as important, but keeping as much data around the
      edges of the map is. Note, any initial iterations
      specified by parameter {\tt{"}}\xref{ast.skip}{sun258}{AST.SKIP}{\tt{"}} are not included in
      the count of iterations. [0]
   }
   \sstattributetype{
      int
   }
   \sstdiytopic{
      SMURF Usage
   }{
      MAKEMAP, CALCQU
   }
}
\sstroutine{
   AST.ZERO\_NOTLAST
}{
   Allows flux to be present in masked areas in the final map
}{
   \sstdescription{
      If ast.zero\_notlast is 1, then the map is not masked on
      the final iteration. This means that data samples that fall
      outside the masked areas are allowed to remain in the final
      map. If ast.zero\_notlast is 0, then the map is masked even on
      the final iteration, meaning that the masked areas will
      be zero in the final map. [1]
   }
   \sstattributetype{
      int
   }
   \sstdiytopic{
      SMURF Usage
   }{
      MAKEMAP, CALCQU
   }
}
\sstroutine{
   AST.ZERO\_SNR
}{
   Reduces spurious large scale structure in the final map
   within regions of low signal-to-noise
}{
   \sstdescription{
      The ast.zero\_snr parameter will mask the map after each
      iteration based on the the signal to noise ratio within
      each map pixel. For example, if it is set to 5, after each
      iteration all map pixels with an SNR below this threshold
      will be forced to zero. the mask is re-evaluated on each
      iteration. An ast.zero\_snr value of zero means no SNR mask
      is used. See also parameter {\tt{"}}\xref{ast.zero\_snr\_ffclean}{sun258}{AST.ZERO_SNR_FFCLEAN}{\tt{"}}. [0.0]
   }
   \sstattributetype{
      real
   }
   \sstdiytopic{
      SMURF Usage
   }{
      MAKEMAP, CALCQU
   }
}
\sstroutine{
   AST.ZERO\_SNR\_FFCLEAN
}{
   Provides alternative method for SNR masking
}{
   \sstdescription{
      Setting this parameter to a non-zero value causes the
      SNR mask requested by parameter {\tt{"}}\xref{ast.zero\_snr}{sun258}{AST.ZERO_SNR}{\tt{"}} to be
      created using an algorithm like that used by the KAPPA
      command {\tt{"}}FFCLEAN{\tt{"}} (see SUN/95), instead of using a simple
      thresholding of the SNR map. The parameter {\tt{"}}\xref{ast.zero\_snr}{sun258}{AST.ZERO_SNR}{\tt{"}}
      gives the clipping level of the ffclean algorithm, and the
      parameter {\tt{"}}\xref{ast.zero\_snr\_hipass}{sun258}{AST.ZERO_SNR_HIPASS}{\tt{"}} gives the box size. Using
      an ffclean algorithm prevents the source regions within
      the mask being larger than the box size, and may thus
      produce faster convergence and avoid blobs developing. [0]
   }
   \sstattributetype{
      int
   }
   \sstdiytopic{
      SMURF Usage
   }{
      MAKEMAP, CALCQU
   }
}
\sstroutine{
   AST.ZERO\_SNR\_FWHM
}{
   Can help to remove bowls around sources
}{
   \sstdescription{
      If ast.zero\_snr\_fwhm is non-zero, the map-maker produces
      two maps: the first is created normally using the mask
      specified by parameter {\tt{"}}\xref{ast.zero\_snr}{sun258}{AST.ZERO_SNR}{\tt{"}}. The final SNR-based
      mask associated with this map is then smoothed using a
      Gaussian with FWHM equal to the ast.zero\_snr\_fwhm value (in
      arcsec). The whole iterative map-making process is then
      run again from the start, using this smoothed mask on every
      iteration, to create the final map. Consequently, setting
      ast.zero\_snr\_fwhm causes the time taken to create the final
      map to nearly double. [0.0]
   }
   \sstattributetype{
      real
   }
   \sstdiytopic{
      SMURF Usage
   }{
      MAKEMAP, CALCQU
   }
}
\sstroutine{
   AST.ZERO\_SNR\_LOW
}{
   Can help to remove bowls around sources
}{
   \sstdescription{
      The ast.zero\_snr\_low parameter gives the value (in the range
      0.0 to 1.0) at which to threshold the smoothed mask specified
      by parameter {\tt{"}}\xref{ast.zero\_snr\_fwhm}{sun258}{AST.ZERO_SNR_FWHM}{\tt{"}}. If it is negative, the value
      is taken as the max smoothed value of a blob containing
      {\tt{"}}ast.zero\_snr\_low{\tt{"}} pixels. Thus a value of {\tt{"}}-1.1{\tt{"}} will cut at
      a height just sufficient to remove blobs of a single pixel
      form the mask. A value of {\tt{"}}-2.1{\tt{"}} would remove blobs of two
      pixels form the mask, etc. [-1.1]
   }
   \sstattributetype{
      real
   }
   \sstdiytopic{
      SMURF Usage
   }{
      MAKEMAP, CALCQU
   }
}
\sstroutine{
   AST.ZERO\_SNRLO
}{
   Can help to remove bowls around sources by increasing the
   size of the SNR mask without introducing noise
}{
   \sstdescription{
      If non-zero values are supplied for ast.zero\_snrlo and
      parameter {\tt{"}}\xref{ast.zero\_snr}{sun258}{AST.ZERO_SNR}{\tt{"}}, then the basic mask created by
      thresholding at the SNR value specified by ast.zero\_snr is
      modified by expanding each un-masked {\tt{"}}source{\tt{"}} area down to
      an SNR equal to ast.zero\_snrlo, without introducing any new
      isolated source areas. The ast.zero\_snrlo should be lower
      than the ast.zero\_snr value. [0]
   }
   \sstattributetype{
      real
   }
   \sstdiytopic{
      SMURF Usage
   }{
      MAKEMAP, CALCQU
   }
}
\sstroutine{
   AST.ZERO\_UNION
}{
   Controls how multiple AST masks are combined
}{
   \sstdescription{
      If more than one AST mask is specified (for instance, if
      values are supplied for both parameter {\tt{"}}\xref{ast.zero\_lowhits}{sun258}{AST.ZERO_LOWHITS}{\tt{"}}
      and parameter {\tt{"}}\xref{ast.zero\_snr}{sun258}{AST.ZERO_SNR}{\tt{"}}), then they are combined
      into a single mask. If ast.zero\_union is true (i.e.
      non-zero), then the source region in the combined mask is
      the union of the source regions in the individual masks.
      If ast.zero\_union is false (i.e. zero), then the source
      region in the combined mask is the intersection of the
      source regions in the individual masks. [1]
   }
   \sstattributetype{
      integer
   }
   \sstdiytopic{
      SMURF Usage
   }{
      MAKEMAP, CALCQU
   }
}
\sstroutine{
   BADFRAC
}{
   Ensures that bad data from DA system are ignored
}{
   \sstdescription{
      The fraction of samples to be bad to flag entire bolo as
      dead. [0.05]
   }
   \sstattributetype{
      real
   }
   \sstdiytopic{
      SMURF Usage
   }{
      SC2CLEAN, CALCQU, MAKEMAP
   }
}
\sstroutine{
   BOLOMAP
}{
   Create NDFs holding the map made form each bolometer
}{
   \sstdescription{
      If non-zero, a separate map will be created from each
      individual bolometer. These maps are placed in the BOLOMAPS
      component of the SMURF extension in the main output map. [0]
   }
   \sstattributetype{
      integer
   }
   \sstdiytopic{
      SMURF Usage
   }{
      MAKEMAP, CALCQU
   }
}
\sstroutine{
   COM.CORR\_ABSTOL
}{
   Controls the rejection of bad samples from the COM estimate
}{
   \sstdescription{
      Gives the absolute lower limit of acceptable correlation
      between a bolometer time-stream and the common-mode. This
      is the first of a set of {\tt{"}}com.$<$xxx$>${\tt{"}} parameters that control
      the rejection of bad detectors based on the gain and
      correlation coefficients for the fit of the common-mode
      signal to each detector (good at identifying bolo signals
      with bizarre gains, or shapes if they have for example steps
      in them). These are basically sigma-clippers; outliers are
      removed at the given threshold and then new means and sample
      standard deviations are measured until convergence. The time
      axis is divided up into one or more equal sized boxes, and
      a separate fit is performed for each box. If you wish to
      completely disable the flagging of outlier bolometers
      compared with the common-mode, simply set com.noflag=1.
      The flags may be frozen after a specified number of
      iterations - see parameter {\tt{"}}\xref{com.freeze\_flags}{sun258}{COM.FREEZE_FLAGS}{\tt{"}}. [0.2]
   }
   \sstattributetype{
      real
   }
   \sstdiytopic{
      SMURF Usage
   }{
      SC2CLEAN, CALCQU, MAKEMAP
   }
}
\sstroutine{
   COM.CORR\_TOL
}{
   Controls the rejection of bad samples from the COM estimate
}{
   \sstdescription{
      The maximum number of standard deviations away from the
      mean correlation coefficient that a bolometer can be
      without being rejected. See parameter {\tt{"}}\xref{com.corr\_abstol}{sun258}{COM.CORR_ABSTOL}{\tt{"}}. [5.0]
   }
   \sstattributetype{
      real
   }
   \sstdiytopic{
      SMURF Usage
   }{
      SC2CLEAN, CALCQU, MAKEMAP
   }
}
\sstroutine{
   COM.FREEZE\_FLAGS
}{
   Controls flagging of samples that differ from the common-mode
}{
   \sstdescription{
      If non-zero, the flags marking bolometers that differ
      from the common-mode are frozen after the specified number
      of iterations. this can help convergence. See parameter
      {\tt{"}}\xref{com.corr\_abstol}{sun258}{COM.CORR_ABSTOL}{\tt{"}}. Note - any initial iterations
      specified by parameter {\tt{"}}\xref{ast.skip}{sun258}{AST.SKIP}{\tt{"}} are not included in the
      count of iterations. [0]
   }
   \sstattributetype{
      integer
   }
   \sstdiytopic{
      SMURF Usage
   }{
      SC2CLEAN, CALCQU, MAKEMAP
   }
}
\sstroutine{
   COM.GAIN\_ABSTOL
}{
   Controls the rejection of bad samples from the COM estimate
}{
   \sstdescription{
      The maximum absolute ratio between a bolometer's gain
      coefficient, and the mean gain coefficient for the bolometer
      not to be rejected. See parameter {\tt{"}}\xref{com.corr\_abstol}{sun258}{COM.CORR_ABSTOL}{\tt{"}}. [3.0]
   }
   \sstattributetype{
      real
   }
   \sstdiytopic{
      SMURF Usage
   }{
      SC2CLEAN, CALCQU, MAKEMAP
   }
}
\sstroutine{
   COM.GAIN\_BOX
}{
   Controls the rejection of bad samples from the COM estimate
}{
   \sstdescription{
      The number of time slices (or seconds if negative) in
      a box. The gain, offset and correlation coefficient
      describing the relationship between a bolometer time
      stream and the common-mode is re-evauated for each such
      box of time slices. See parameter {\tt{"}}\xref{com.corr\_abstol}{sun258}{COM.CORR_ABSTOL}{\tt{"}}. [-30.0]
   }
   \sstattributetype{
      real
   }
   \sstdiytopic{
      SMURF Usage
   }{
      SC2CLEAN, CALCQU, MAKEMAP
   }
}
\sstroutine{
   COM.GAIN\_FGOOD
}{
   Controls the rejection of bad samples from the COM estimate
}{
   \sstdescription{
      The minimum fraction of good gain boxes for a usable bolometer.
      See parameter {\tt{"}}\xref{com.corr\_abstol}{sun258}{COM.CORR_ABSTOL}{\tt{"}}. [0.25]
   }
   \sstattributetype{
      integer
   }
   \sstdiytopic{
      SMURF Usage
   }{
      SC2CLEAN, CALCQU, MAKEMAP
   }
}
\sstroutine{
   COM.GAIN\_RAT
}{
   Controls the rejection of bad samples from the COM estimate
}{
   \sstdescription{
      The ratio of the largest usable gain to the mean gain for a
      bolometer not to be rejected. See parameter {\tt{"}}\xref{com.corr\_abstol}{sun258}{COM.CORR_ABSTOL}{\tt{"}}. [4.0]
   }
   \sstattributetype{
      real
   }
   \sstdiytopic{
      SMURF Usage
   }{
      SC2CLEAN, CALCQU, MAKEMAP
   }
}
\sstroutine{
   COM.GAIN\_TOL
}{
   Controls the rejection of bad samples from the COM estimate
}{
   \sstdescription{
      The maximum number of standard deviations away from the
      mean gain coefficient that a bolometer can be without being
      rejected. See parameter {\tt{"}}\xref{com.corr\_abstol}{sun258}{COM.CORR_ABSTOL}{\tt{"}}. [5]
   }
   \sstattributetype{
      real
   }
   \sstdiytopic{
      SMURF Usage
   }{
      SC2CLEAN, CALCQU, MAKEMAP
   }
}
\sstroutine{
   COM.NOFLAG
}{
   Controls the rejection of bad samples from the COM estimate
}{
   \sstdescription{
      If non-zero, disable flagging of bad bolometers using the
      common-mode. See parameter {\tt{"}}\xref{com.corr\_abstol}{sun258}{COM.CORR_ABSTOL}{\tt{"}}. [0]
   }
   \sstattributetype{
      integer
   }
   \sstdiytopic{
      SMURF Usage
   }{
      SC2CLEAN, CALCQU, MAKEMAP
   }
}
\sstroutine{
   COM.PERARRAY
}{
   Controls the estimate of the common-mode signal
}{
   \sstdescription{
      If non-zero, calculate a separate common-mode signal
      for each subarray. If zero, a single common-mode signal
      will be calculated from all subarrays at a given
      wavelength simultaneously. [0]
   }
   \sstattributetype{
      integer
   }
   \sstdiytopic{
      SMURF Usage
   }{
      SC2CLEAN, CALCQU, MAKEMAP
   }
}
\sstroutine{
   COM.SIG\_LIMIT
}{
   Controls the rejection of time slices with inconsistent
   common-modes
}{
   \sstdescription{
      This value is only used if data from more than one
      sub-array is being included in the map, and parameter
      {\tt{"}}\xref{com.perarray}{sun258}{COM.PERARRAY}{\tt{"}} is zero, causing a single mean COM model to
      be used for all sub-arrays. In such cases, if the
      common-mode signals estimated from the individual
      sub-arrays show significantly different structure,
      subtracting a mean COM model will leave large residuals in
      the bolometer time streams. If a FLT model is being used,
      the high pass filter will remove much of this residual
      signal, but any high frequency component will remain,
      often causing strong {\tt{"}}blobs{\tt{"}} in the map.

      The COM model can identify and flag time slices where the
      individual common-mode signals appear to be significantly
      different to the mean common-mode signal. The individual
      common-mode signals are first high-pass filtered using the
      same filter as the FLT model. At each time slice, the
      mean and standard deviation of the remaining high frequency
      common-modes are found. The RMS of these standard
      deviations, taken over the whole time stream, is then found.
      All bolometers within individual time slices for which the
      standard deviation exceeds {\tt{"}}com.sig\_limit{\tt{"}} times the RMS value
      are flagged in all sub-arrays, causing them to be excluded
      from the FLT model and the map. Note, this flagging process is
      performed from scratch on each iteration - that is, samples
      flagged on a previous iteration are tested again on each
      subsequent iteration and may be {\tt{"}}unflagged{\tt{"}} if the differences
      between individual common-mode signal become smaller.

      Setting this parameter to 3 is a good starting point.
      Setting it to zero (the default) causes this consistency
      filter to be skipped. See also parameter {\tt{"}}\xref{com.sig\_wing}{sun258}{COM.SIG_WING}{\tt{"}}. [0]
   }
   \sstattributetype{
      real
   }
   \sstdiytopic{
      SMURF Usage
   }{
      SC2CLEAN, CALCQU, MAKEMAP
   }
}
\sstroutine{
   COM.ZERO\_CIRCLE
}{
   Improves common-mode estimation by excluding sources
   within a circle of given radius from the COM estimate
}{
   \sstdescription{
      Using com.zero\_circle causes any samples falling within a
      specified circle on the map to be excluded from the
      estimate of the mean signal at each time slice (the
      common mode, or {\tt{"}}COM{\tt{"}}, signal). [{\tt $<$undef$>$}]
   }
   \sstattributetype{
      real
   }
   \sstdiytopic{
      SMURF Usage
   }{
      MAKEMAP, CALCQU
   }
}
\sstroutine{
   COM.ZERO\_FREEZE
}{
   Prevent the COM mask from changing after a given number
   of iterations. This can help convergence
}{
   \sstdescription{
      If com.zero\_freeze is non-zero, the COM mask will be frozen
      after the specified number of iterations. A value of zero
      means that the mask is never frozen. Note, any initial
      iterations specified by parameter {\tt{"}}\xref{ast.skip}{sun258}{AST.SKIP}{\tt{"}} are not
      included in the count of iterations. [0]
   }
   \sstattributetype{
      integer
   }
   \sstdiytopic{
      SMURF Usage
   }{
      MAKEMAP, CALCQU
   }
}
\sstroutine{
   COM.ZERO\_LOWHITS
}{
   Improves common-mode estimation by excluding sources
   in regions containing many data samples
}{
   \sstdescription{
      Using com.zero\_lowhits causes samples to be excluded from
      the estimation of the common mode if they fall in regions
      of the map where the number of samples falling in each
      pixel is higher than com.zero\_lowhits times the mean number
      of samples per pixel, averaged over the map. A value of
      zero means that no masking of low hits regions is
      performed. The mask is updated on each iteration. [0]
   }
   \sstattributetype{
      real
   }
   \sstdiytopic{
      SMURF Usage
   }{
      MAKEMAP, CALCQU
   }
}
\sstroutine{
   COM.ZERO\_MASK
}{
   Provides a  better estimate of the common-mode ({\tt{"}}COM{\tt{"}})
   signal, by excluding samples that fall within fixed
   regions on the sky specified by an external mask
}{
   \sstdescription{
      If com.zero\_mask is set to one of {\tt{"}}REF{\tt{"}}, {\tt{"}}MASK2{\tt{"}} or
      {\tt{"}}MASK3{\tt{"}} then an NDF will be obtained using the specified
      ADAM parameter (REF, MASK2 or MASK3) and used as a
      user-defined mask. Setting com.zero\_mask to an integer value
      larger than zero has the same effect as setting it to {\tt{"}}REF{\tt{"}}.
      Setting it to an integer less than or equal to zero results
      in no external mask being used with the COM model. Note,
      using {\tt{"}}REF{\tt{"}} ensures that the mask and the output image of
      MAKEMAP are on the same pixel grid - using {\tt{"}}MASK2{\tt{"}} or {\tt{"}}MASK3{\tt{"}}
      does not provide this guarantee (it is then the users
      responsibility to ensure that the supplied masks are aligned
      with the output image in pixel coordinates). The pixels in
      the map that are to be included in the common-mode
      estimation should be set to the bad value in the mask. All
      other pixels will be excluded from the COM estimation. [0]
   }
   \sstattributetype{
      integer or string
   }
   \sstdiytopic{
      SMURF Usage
   }{
      MAKEMAP, CALCQU
   }
}
\sstroutine{
   COM.ZERO\_SNR
}{
   Improve the estimate of the common-mode by excluding samples
   that correspond to high SNR pixels in the map
}{
   \sstdescription{
      Setting the com.zero\_snr parameter will exclude samples from
      the COM estimate that fall within map pixels with SNR values
      greater than com.zero\_snr. A com.zero\_snr value of zero means
      no SNR mask is used. See also parameter {\tt{"}}\xref{com.zero\_snr\_ffclean}{sun258}{COM.ZERO_SNR_FFCLEAN}{\tt{"}}.

      Note, the SNR values are only available once a map has been
      created, and so using this parameter results in no COM masking
      on the first iteration. Consequently the map at the end of the
      first iteration will have a bowl around any bright sources,
      since no COM masking was done. Normally, these rings
      would polute the AST model derived from the map, and thus
      polute the residuals on the next iteration, resulting in
      the bowls remaining in later maps. To avoid this, parameter
      {\tt{"}}\xref{ast.skip}{sun258}{AST.SKIP}{\tt{"}} can be set to a positive value. This causes the
      AST model to be skipped (i.e. no AST signal is subtracted from
      the residuals) for the first {\tt{"}}ast.skip{\tt{"}} iterations. This means
      that a good COM mask can be formed from these initial iterations
      before any AST model is calculated and used. [0.0]
   }
   \sstattributetype{
      real
   }
   \sstdiytopic{
      SMURF Usage
   }{
      MAKEMAP, CALCQU
   }
}
\sstroutine{
   COM.ZERO\_SNR\_FFCLEAN
}{
   Provides alternative method for SNR masking
}{
   \sstdescription{
      Setting this parameter to a non-zero value causes the
      SNR mask requested by parameter {\tt{"}}\xref{com.zero\_snr}{sun258}{COM.ZERO_SNR}{\tt{"}} to be
      created using an algorithm like that used by the KAPPA
      command {\tt{"}}FFCLEAN{\tt{"}} (see SUN/95), instead of using a simple
      thresholding of the SNR map. The parameter {\tt{"}}\xref{com.zero\_snr}{sun258}{COM.ZERO_SNR}{\tt{"}}
      gives the clipping level of the ffclean algorithm, and the
      parameter {\tt{"}}\xref{com.zero\_snr\_hipass}{sun258}{COM.ZERO_SNR_HIPASS}{\tt{"}} gives the box size. Using
      an ffclean algorithm prevents the source regions within
      the mask being larger than the box size. [0]
   }
   \sstattributetype{
      int
   }
   \sstdiytopic{
      SMURF Usage
   }{
      MAKEMAP, CALCQU
   }
}
\sstroutine{
   COM.ZERO\_SNRLO
}{
   Improve estimate of the common-mode by increasing the size
   of the SNR mask without introducing noise
}{
   \sstdescription{
      If non-zero values are supplied for com.zero\_snrlo and
      parameter {\tt{"}}\xref{com.zero\_snr}{sun258}{COM.ZERO_SNR}{\tt{"}}, then the basic mask created by
      thresholding at the SNR value specified by com.zero\_snr is
      modified by expanding each un-masked {\tt{"}}source{\tt{"}} area down to
      an SNR equal to com.zero\_snrlo, without introducing any new
      isolated source areas. The com.zero\_snrlo should be lower than
      the com.zero\_snr value. [0]
   }
   \sstattributetype{
      real
   }
   \sstdiytopic{
      SMURF Usage
   }{
      MAKEMAP, CALCQU
   }
}
\sstroutine{
   COM.ZERO\_UNION
}{
   Controls how multiple COM masks are combined
}{
   \sstdescription{
      If more than one COM mask is specified (for instance, if
      values are supplied for both parameter {\tt{"}}\xref{com.zero\_lowhits}{sun258}{COM.ZERO_LOWHITS}{\tt{"}}
      and parameter {\tt{"}}\xref{com.zero\_snr}{sun258}{COM.ZERO_SNR}{\tt{"}}), then they are combined
      into a single mask. If com.zero\_union is true (i.e.
      non-zero), then the source region in the combined mask is
      the union of the source regions in the individual masks.
      If com.zero\_union is false (i.e. zero), then the source
      region in the combined mask is the intersection of the
      source regions in the individual masks. [1]
   }
   \sstattributetype{
      integer
   }
   \sstdiytopic{
      SMURF Usage
   }{
      MAKEMAP, CALCQU
   }
}
\sstroutine{
   COMPREPROCESS
}{
   Remove common-mode before the iterative algorithm begins
}{
   \sstdescription{
      If non-zero, the common-mode will be estimated and removed
      additionally as a pre-processing step. All the {\tt{"}}com.$<$xxx$>${\tt{"}} and
      {\tt{"}}gai.$<$xxx$>${\tt{"}} parameters are parsed and used (e.g., to also
      flag bad data and optionally flatfield off the relative
      response to the common-mode signal). If this pre-processing
      step is chosen, it is still possible to specify COM/GAI
      as model components in the iterative solution. [0]
   }
   \sstattributetype{
      integer
   }
   \sstdiytopic{
      SMURF Usage
   }{
      SC2CLEAN, CALCQU, MAKEMAP
   }
}
\sstroutine{
   DCFITBOX
}{
   Control the cleaning of DC steps in bolometer time streams
}{
   \sstdescription{
      This gives the box size over which to fit data with a
      straight line on either side of a potential DC jump,
      prior to estimating the bolometer noise levels when doing
      initial data cleaning. If positive, in units of samples. If
      negative, in units of seconds. If zero, do not perform step
      correction during initial data cleaning. [30]
   }
   \sstattributetype{
      integer
   }
   \sstdiytopic{
      SMURF Usage
   }{
      SC2CLEAN, CALCQU, MAKEMAP
   }
}
\sstroutine{
   DCLIMCORR
}{
   Control the cleaning of DC steps in bolometer time streams
}{
   \sstdescription{
      If more than DCLIMCORR bolometer have a step at a given
      time, then all bolometers are corrected for a step at that
      time, using lower thresholds. A value of zero switches off
      the correction of correlated steps within the initial
      data cleaning phase. Only used if parameter {\tt{"}}\xref{dcfitbox}{sun258}{DCFITBOX}{\tt{"}} is
      non-zero. [0]
   }
   \sstattributetype{
      integer
   }
   \sstdiytopic{
      SMURF Usage
   }{
      SC2CLEAN, CALCQU, MAKEMAP
   }
}
\sstroutine{
   DCMAXSTEPS
}{
   Control the cleaning of DC steps in bolometer time streams
}{
   \sstdescription{
      The maximum number of steps that can be corrected in each
      minute of good data (i.e. per 12000 samples) from a bolometer
      before the entire bolometer is flagged as bad. A value of
      zero will cause a bolometer to be rejected if any steps are
      found in the bolometer data stream. Only used if parameter
      {\tt{"}}\xref{dcfitbox}{sun258}{DCFITBOX}{\tt{"}} is non-zero. [10]
   }
   \sstattributetype{
      integer
   }
   \sstdiytopic{
      SMURF Usage
   }{
      SC2CLEAN, CALCQU, MAKEMAP
   }
}
\sstroutine{
   DCSMOOTH
}{
   Control the cleaning of DC steps in bolometer time streams
}{
   \sstdescription{
      The width of the median filter used to smooth a bolometer
      data stream prior to finding DC jumps. If positive, in units
      of samples. If negative, in units of seconds. Only used if
      parameter {\tt{"}}\xref{dcfitbox}{sun258}{DCFITBOX}{\tt{"}} is non-zero. [50]
   }
   \sstattributetype{
      integer
   }
   \sstdiytopic{
      SMURF Usage
   }{
      SC2CLEAN, CALCQU, MAKEMAP
   }
}
\sstroutine{
   DCTHRESH
}{
   Control the cleaning of DC steps in bolometer time streams
}{
   \sstdescription{
      The SNR threshold at which to detect DC steps. Note, this
      refers to the noise level in the bolometer data after it
      has been smoothed with a median filter of width given by
      parameter {\tt{"}}\xref{dcsmooth}{sun258}{DCSMOOTH}{\tt{"}}. In order to find the equivalent
      threshold in the unsmoothed data, multiply the dcthresh
      value by 1.25/sqrt(dcsmooth). For instance, the default
      values for dcsmooth (50) and dcthresh (25) correspond
      to a threshold of 25$*$1.25/sqrt(50) = 4.4 sigma in the
      unsmoothed data. Only used if parameter {\tt{"}}\xref{dcfitbox}{sun258}{DCFITBOX}{\tt{"}} is
      non-zero. [25.0]
   }
   \sstattributetype{
      integer
   }
   \sstdiytopic{
      SMURF Usage
   }{
      SC2CLEAN, CALCQU, MAKEMAP
   }
}
\sstroutine{
   DIAG.OUT
}{
   Switches on the dumping of various diagnostic information
}{
   \sstdescription{
      The full path/name for the HDS container file in which to
      store the diagnostic info. This will contain components for
      each requested model (see parameter {\tt{"}}\xref{diag.models}{sun258}{DIAG.MODELS}{\tt{"}}), with
      names like {\tt{"}}COM{\tt{"}}, {\tt{"}}FLT{\tt{"}}, etc. Each of these components will
      contain multiple NDFs with names in the following format:
      {\tt{"}}$<$where$>$\_$<$chunk$>$\_$<$what$>${\tt{"}}, where $<$what$>$ is {\tt{"}}power{\tt{"}} or {\tt{"}}time{\tt{"}},
      $<$chunk$>$ is the integer chunk index, and $<$where$>$ is one of:

      \sstitemlist{

         \sstitem
         {\tt{"}}bef{\tt{"}}: the NDF contains the residuals as they were
         before the model was subtracted.

         \sstitem
         {\tt{"}}mod{\tt{"}}: the NDF contains the model values themselves.

         \sstitem
         {\tt{"}}aft{\tt{"}}: the NDF contains the residuals as they were after
         the model was subtracted.

      }
      Each NDF will be 2-dimensional, with the first pixel axis
      representing time or frequency, and the second pixel axis
      representing iteration number. [{\tt $<$undef$>$}]
   }
   \sstattributetype{
      string
   }
   \sstdiytopic{
      SMURF Usage
   }{
      MAKEMAP, CALCQU
   }
}
\sstroutine{
   DOCLEAN
}{
   Allows pre-cleaned data to be used
}{
   \sstdescription{
      Set this to 0 to turn off all data cleaning operations
      prior to the start of iterative map-making. [1]
   }
   \sstattributetype{
      integer
   }
   \sstdiytopic{
      SMURF Usage
   }{
      MAKEMAP, CALCQU
   }
}
\sstroutine{
   DOWNSAMPSCALE
}{
   Speeds up map-making, and reduces memory requirements
}{
   \sstdescription{
      If the telescope is scanning slowly the data may be
      safely down-sampled to save memory and time. This parameter
      controls the minimum angular scale on the sky. The new
      sample frequency is chosen such that this scale will be
      preserved taking into account the average slew speed and
      the sample rate of the input files. If a positive value is
      selected, this gives the angular scale (in arcsec) to which
      the new sample rate will be matched. Alternatively, if a
      negative value is supplied, its magnitude will be multiplied
      by the PIXSIZE for the requested map. For example, the default
      here is to set it to -1 such that the time-series sample rate
      matches the pixel grid (in practice, a factor of 2 might make
      more sense as this would correspond to the Nyquist frequency
      of the map pixel grid). [-1]
   }
   \sstattributetype{
      real
   }
   \sstdiytopic{
      SMURF Usage
   }{
      MAKEMAP, CALCQU
   }
}
\sstroutine{
   EXPORTCLEAN
}{
   Allows the initial cleaned data to examined or saved for
   later use
}{
   \sstdescription{
      If non-zero, the data will be saved to an NDF immediately
      after data cleaning and before map-making. The NDF name will
      be the same as model components, except with the suffix
      {\tt{"}}\_cln{\tt{"}}. Even if parameter {\tt{"}}\xref{doclean}{sun258}{DOCLEAN}{\tt{"}} ise set to zero, the
      data will be exported immediately before map-making. [0]
   }
   \sstattributetype{
      integer
   }
   \sstdiytopic{
      SMURF Usage
   }{
      MAKEMAP, CALCQU
   }
}
\sstroutine{
   EXPORTNDF
}{
   Create NDFs holding final model values
}{
   \sstdescription{
      Specify a value of 1 or 0 to export all or none of the
      model components after the final iteration. You can also
      specify a comma-separated list of component names, enclosed
      in parentheses, to be exported. Note that you can specify
      additional components RES and QUA to what may be provided to
      parameter {\tt{"}}\xref{modelorder}{sun258}{MODELORDER}{\tt{"}} if you wish to export the residual
      model or quality arrays respectively. Exportation of RES is
      implied if NOI is specified as it becomes the variance
      component of the resulting NDF for RES. QUA will become the
      quality component of any full 3-dimensional model (e.g.
      RES, AST, FLT, EXT), but no quality will be written to model
      components with different dimensions. [0]
   }
   \sstattributetype{
      integer or list of strings
   }
   \sstdiytopic{
      SMURF Usage
   }{
      MAKEMAP, CALCQU
   }
}
\sstroutine{
   EXT.CSOTAU
}{
   Controls the extinction values used in the EXT model
}{
   \sstdescription{
      Specifies the CSO tau value to be used by the EXT model.
      If {\tt $<$undef$>$}, the default value to use is derived from the
      FITS headers. See parameter {\tt{"}}\xref{ext.tausrc}{sun258}{EXT.TAUSRC}{\tt{"}}. [{\tt $<$undef$>$}]
   }
   \sstattributetype{
      real
   }
   \sstdiytopic{
      SMURF Usage
   }{
      MAKEMAP, CALCQU
   }
}
\sstroutine{
   EXT.FILTERTAU
}{
   Controls the extinction values used in the EXT model
}{
   \sstdescription{
      Used if parameter {\tt{"}}\xref{ext.tausrc}{sun258}{EXT.TAUSRC}{\tt{"}} is set to {\tt{"}}filtertau{\tt{"}}. If
      {\tt $<$undef$>$}, the default value to use is derived from the
      FITS headers. [{\tt $<$undef$>$}]
   }
   \sstattributetype{
      real
   }
   \sstdiytopic{
      SMURF Usage
   }{
      MAKEMAP, CALCQU
   }
}
\sstroutine{
   EXT.TAUMETHOD
}{
   Controls the extinction values used in the EXT model
}{
   \sstdescription{
      The method to use for determing tau. Can be {\tt{"}}adaptive{\tt{"}},
      {\tt{"}}full{\tt{"}} or {\tt{"}}quick{\tt{"}}. See parameter {\tt{"}}\xref{ext.tausrc}{sun258}{EXT.TAUSRC}{\tt{"}}. [adaptive]
   }
   \sstattributetype{
      string
   }
   \sstdiytopic{
      SMURF Usage
   }{
      MAKEMAP, CALCQU
   }
}
\sstroutine{
   EXT.TAURELATION.450
}{
   Controls the 450 um extinction values used in the EXT model
   and the EXTINCTION task
}{
   \sstdescription{
      Each tau relation is parameterised in the form:
      {\tt{"}}tau\_filt = a ( tau\_cso $+$ b ){\tt{"}}, where {\tt{"}}a{\tt{"}} and {\tt{"}}b{\tt{"}} are the
      two values supplied for this parameter. See also parameter
      {\tt{"}}\xref{ext.taurelation.850}{sun258}{EXT.TAURELATION.850}{\tt{"}}. Here, the trailing {\tt{"}}.450{\tt{"}} in the
      parameter name is the filter name, not the sub-instrument
      name as would be the case if it appeared at the start of
      parameter name. [(26.0,-0.012)]
   }
   \sstattributetype{
      (real,real)
   }
   \sstdiytopic{
      SMURF Usage
   }{
      EXTINCTION, CALCQU, MAKEMAP
   }
}
\sstroutine{
   EXT.TAURELATION.850
}{
   Controls the 850 um extinction values used in the EXT model
   and the EXTINCTION task
}{
   \sstdescription{
      The 850 um equivalent to the parameter {\tt{"}}\xref{ext.taurelation.450}{sun258}{EXT.TAURELATION.450}{\tt{"}}. [(4.6,-0.0043)]
   }
   \sstattributetype{
      (real,real)
   }
   \sstdiytopic{
      SMURF Usage
   }{
      EXTINCTION, CALCQU, MAKEMAP
   }
}
\sstroutine{
   EXT.TAUSRC
}{
   Controls the extinction values used in the EXT model
}{
   \sstdescription{
      Best is to use WVM, uses continuously varying measurements
      as a function of time stored with each observation. Allowed
      values are  {\tt{"}}auto{\tt{"}}, {\tt{"}}wvmraw{\tt{"}}, {\tt{"}}csotau{\tt{"}} and {\tt{"}}filtertau{\tt{"}}. See
      EXTINCTION task for further information. [auto]
   }
   \sstattributetype{
      string
   }
   \sstdiytopic{
      SMURF Usage
   }{
      MAKEMAP, CALCQU
   }
}
\sstroutine{
   FAKEMAP
}{
   Diagnostic tool to explore the effects of the map-making
   process on known sources
}{
   \sstdescription{
      To test the response of the map-maker to different known
      astronomical sources, an external {\tt{"}}fakemap{\tt{"}} can be specified
      to provide an image of the sky that will produce additional
      astronomical signal to the time series. At present, the
      dimensions of this map must be identical to that of the real
      map. A typical procedure may involve: (i) produce a map with
      makemap; (ii) produce an image with simulated data with the
      same pixel dimensions; (iii) specify this new map for the
      {\tt{"}}fakemap{\tt{"}} parameter below. Note that this is a fully-parsed
      ndf identifier, so you can do things like:

      \sstitemlist{

         \sstitem
         fakemap = fakesky.sdf

         \sstitem
         fakemap = fakesky[1:300,100:450] [{\tt $<$undef$>$}]
      }
   }
   \sstattributetype{
      string
   }
   \sstdiytopic{
      SMURF Usage
   }{
      MAKEMAP, CALCQU
   }
}
\sstroutine{
   FAKESCALE
}{
   Control the use oif the supplied fake map
}{
   \sstdescription{
      Each pixel in the supplied fake map (see paramater {\tt{"}}fakemap{\tt{"}})
      will be multiplied by this scaling factor before being added
      to the time stream data. [1]
   }
   \sstattributetype{
      real
   }
   \sstdiytopic{
      SMURF Usage
   }{
      MAKEMAP, CALCQU
   }
}
\sstroutine{
   FILT\_EDGE\_LARGESCALE
}{
   Specifies the largest scale size to be retained by the
   initial data cleaning
}{
   \sstdescription{
      Together with parameter {\tt{"}}\xref{filt\_edge\_smallscale}{sun258}{FILT_EDGE_SMALLSCALE}{\tt{"}}, this
      specifies the frequencies which the initial data cleaning is
      to remove from the data streams, based on a range of requested
      spatial scales (in arcsec), and using internal measurements of
      the average slew speed. These will override parameter
      {\tt{"}}\xref{filt\_edgehigh}{sun258}{FILT_EDGEHIGH}{\tt{"}} and parameter {\tt{"}}\xref{filt\_edgelow}{sun258}{FILT_EDGELOW}{\tt{"}}. For example,
      suppose the slew speed is 100 arcsec/sec. We want to ensure
      that the beam is fully sampled, say 2 arcsec at 450um.
      That scale is crossed in 2/100 = 0.02 s, so we don't need
      frequencies in the data above 1/0.02 = 50 Hz in this case
      (i.e. internally it will set filt\_edgelow to 50Hz if
      filt\_edge\_smallscale is set to 2 arcsec). Similarly, if
      we would like to attempt to preserve scales of 10 arcmin =
      600 arcsec, we would want to keep frequencies that are
      greater than 1/(600/100.) = 0.17 Hz (i.e. setting
      filt\_edge\_largescale=600 would translate into
      filt\_edgehigh = 0.17 Hz). [0]
   }
   \sstattributetype{
      real
   }
   \sstdiytopic{
      SMURF Usage
   }{
      SC2CLEAN, CALCQU, MAKEMAP
   }
}
\sstroutine{
   FILT\_EDGE\_SMALLSCALE
}{
   Specifies the smallest largest scale size to be retained by
   the initial data cleaning
}{
   \sstdescription{
      If non-zero, features with spatial sizes less than this
      value (in arcsec) will be removed by the initial data
      cleaning. See parameter {\tt{"}}\xref{filt\_edge\_largescale}{sun258}{FILT_EDGE_LARGESCALE}{\tt{"}}. [0]
   }
   \sstattributetype{
      real
   }
   \sstdiytopic{
      SMURF Usage
   }{
      SC2CLEAN, CALCQU, MAKEMAP
   }
}
\sstroutine{
   FILT\_EDGELOW
}{
   Specifies the highest frequency to be retained by the
   initial data cleaning
}{
   \sstdescription{
      If non-zero, this is the cut-off frequency of a
      low pass filter that is applied to the data stream as
      part of the initial data cleaning. See also parameter
      {\tt{"}}\xref{filt\_edge\_largescale}{sun258}{FILT_EDGE_LARGESCALE}{\tt{"}}. [0]
   }
   \sstattributetype{
      real
   }
   \sstdiytopic{
      SMURF Usage
   }{
      SC2CLEAN, CALCQU, MAKEMAP
   }
}
\sstroutine{
   FILT\_ORDER
}{
   Indicates the shape of the filter
}{
   \sstdescription{
      If filt\_order is zero or negative, a the edge filters
      defined by the other {\tt{"}}filt\_...{\tt{"}} parameters are hard-edged.
      That is, they change from zero to one without any intermediate
      values. If filt\_order is larger than zero, the edge
      filters are soft-edged Butterworth filters with order
      given by the value of this parameter. An order of 1 is
      the softest, and will thus produce least ringing, at the
      expense of poorer frequency response. Higher orders
      produce sharper filters that have better frequency response
      but at the expense of greater ringing. [0]
   }
   \sstattributetype{
      integer
   }
   \sstdiytopic{
      SMURF Usage
   }{
      SC2CLEAN, CALCQU, MAKEMAP
   }
}
\sstroutine{
   FLAGFAST
}{
   Flag data when we're moving too fast
}{
   \sstdescription{
      Data taken when the telescope was moving too fast such
      that sources are smeared can be flagged using this
      parameter. The value is a threshold slew velocity (arcsec/sec)
      measured in tracking coordinates. Assuming a sample rate of
      200 Hz, we want to be able to fully-sample the 450 and 850
      beams. For now just set it to something that is bigger than
      we need, but be warned that point-sources may be smeared-out. [1000]
   }
   \sstattributetype{
      integer
   }
   \sstdiytopic{
      SMURF Usage
   }{
      SC2CLEAN, CALCQU, MAKEMAP
   }
}
\sstroutine{
   FLAGSLOW
}{
   Flag data when we're moving too slowly
}{
   \sstdescription{
      Data taken when the telescope was moving too slowly
      such that sources are buried in 1/f noise, can be flagged
      using this parameter. The value is a threshold slew velocity
      (arcsec/sec) measured in tracking coordinates. Assuming we
      would like to be able to sample scales of at least 30 arcsec
      (at least two 15 arcsec beams at 850), and assuming a typical
      1/f knee of 1 Hz, the telescope needs to slew at least 30
      arcsec/sec to place sources in the signal band above the
      knee. [30]
   }
   \sstattributetype{
      integer
   }
   \sstdiytopic{
      SMURF Usage
   }{
      SC2CLEAN, CALCQU, MAKEMAP
   }
}
\sstroutine{
   FLT.FILT\_EDGE\_LARGESCALE
}{
   Specifies the largest scale size to be retained by the
   FLT model
}{
   \sstdescription{
      Together with parameter {\tt{"}}\xref{flt.filt\_edge\_smallscale}{sun258}{FLT.FILT_EDGE_SMALLSCALE}{\tt{"}}, this
      specifies the frequencies which the FLT model is to remove
      from the data streams, based on a range of requested spatial
      scales (in arcsec), and using internal measurements of
      the average slew speed. These will override parameter
      {\tt{"}}\xref{flt.filt\_edgehigh}{sun258}{FLT.FILT_EDGEHIGH}{\tt{"}} and parameter {\tt{"}}\xref{flt.filt\_edgelow}{sun258}{FLT.FILT_EDGELOW}{\tt{"}}. For
      example, suppose the slew speed is 100 arcsec/sec. We want to
      ensure that the beam is fully sampled, say 2 arcsec at 450um.
      That scale is crossed in 2/100 = 0.02 s, so we don't need
      frequencies in the data above 1/0.02 = 50 Hz in this case
      (i.e. internally it will set flt.filt\_edgelow to 50Hz if
      flt.filt\_edge\_smallscale is set to 2 arcsec). Similarly, if
      we would like to attempt to preserve scales of 10 arcmin =
      600 arcsec, we would want to keep frequencies that are
      greater than 1/(600/100.) = 0.17 Hz (i.e. setting
      flt.filt\_edge\_largescale=600 would translate into
      flt.filt\_edgehigh = 0.17 Hz). [600 (for 450 um), 300 (for 850 um)]
   }
   \sstattributetype{
      real
   }
   \sstdiytopic{
      SMURF Usage
   }{
      MAKEMAP, CALCQU
   }
}
\sstroutine{
   FLT.FILT\_EDGE\_LARGESCALE\_LAST
}{
   Specifies the largest scale size to be retained by the
   FLT model on the last iteration
}{
   \sstdescription{
      This is the value to be used for parameter {\tt{"}}\xref{flt.filt\_edge\_largescale}{sun258}{FLT.FILT_EDGE_LARGESCALE}{\tt{"}}
      on the last iteration. If it is {\tt{"}}{\tt $<$undef$>$}{\tt{"}}, then the same value
      will be used as for earlier iterations. [{\tt $<$undef$>$}]
   }
   \sstattributetype{
      real
   }
   \sstdiytopic{
      SMURF Usage
   }{
      MAKEMAP, CALCQU
   }
}
\sstroutine{
   FLT.FILT\_ORDER
}{
   Indicates the shape of the filter
}{
   \sstdescription{
      If flt.filt\_order is zero or negative, a the edge filters
      defined by the other {\tt{"}}flt.filt\_...{\tt{"}} parameters are hard-edged.
      That is, they change from zero to one without any intermediate
      values. If flt.filt\_order is larger than zero, the edge
      filters are soft-edged Butterworth filters with order
      given by the value of this parameter. An order of 1 is
      the softest, and will thus produce least ringing, at the
      expense of poorer frequency response. Higher orders
      produce sharper filters that have better frequency response
      but at the expense of greater ringing. [0]
   }
   \sstattributetype{
      integer
   }
   \sstdiytopic{
      SMURF Usage
   }{
      MAKEMAP, CALCQU
   }
}
\sstroutine{
   FLT.NOTFIRST
}{
   May improve convergence by avoiding the filtering of
   strong sources on the first iteration
}{
   \sstdescription{
      If this is non-zero, then low frequencies will not be
      removed from the time streams on the first iteration. [0]
   }
   \sstattributetype{
      integer
   }
   \sstdiytopic{
      SMURF Usage
   }{
      MAKEMAP, CALCQU
   }
}
\sstroutine{
   FLT.RING\_BOX1
}{
   Controls the flagging of samples that suffer from ringing
}{
   \sstdescription{
      If this is non-zero, then a ringing filter is applied to
      the residuals once the FLT model has been removed. This
      filter attempts to locate and flag residuals that suffer
      from ringing. It gives the size of the box used to smooth
      the residuals in order to determine the background. It
      is specified as a multiple of the filter size. A value of
      0.5 could be a good starting point. Note, the ringing
      filter is not applied on any initial iterations specified
      by parameter {\tt{"}}\xref{ast.skip}{sun258}{AST.SKIP}{\tt{"}}. [0]
   }
   \sstattributetype{
      float
   }
   \sstdiytopic{
      SMURF Usage
   }{
      MAKEMAP, CALCQU
   }
}
\sstroutine{
   FLT.ZERO\_CIRCLE
}{
   Speeds up convergences and reduces ringing by excluding
   sources within a circle of given radius from the FLT estimate
}{
   \sstdescription{
      Using flt.zero\_circle causes any samples falling within a
      specified circle on the map to be excluded from the
      filtering performed by the FLT model. [{\tt $<$undef$>$}]
   }
   \sstattributetype{
      real
   }
   \sstdiytopic{
      SMURF Usage
   }{
      MAKEMAP, CALCQU
   }
}
\sstroutine{
   FLT.ZERO\_FREEZE
}{
   Prevent the FLT mask from changing after a given number
   of iterations. This can help convergence
}{
   \sstdescription{
      If flt.zero\_freeze is non-zero, the FLT mask will be frozen
      after the specified number of iterations. A value of zero
      means that the mask is never frozen. Note, any initial
      iterations specified by parameter {\tt{"}}\xref{ast.skip}{sun258}{AST.SKIP}{\tt{"}} are not
      included in the count of iterations. [0]
   }
   \sstattributetype{
      integer
   }
   \sstdiytopic{
      SMURF Usage
   }{
      MAKEMAP, CALCQU
   }
}
\sstroutine{
   FLT.ZERO\_LOWHITS
}{
   Experimental
}{
   \sstdescription{
      Using flt.zero\_lowhits causes samples to be excluded from
      the filtering performed by the FLT model if they fall in
      regions of the map where the number of samples falling in each
      pixel is higher than flt.zero\_lowhits times the mean number
      of samples per pixel, averaged over the map. A value of
      zero means that no masking of low hits regions is
      performed. The mask is updated on each iteration. [0]
   }
   \sstattributetype{
      real
   }
   \sstdiytopic{
      SMURF Usage
   }{
      MAKEMAP, CALCQU
   }
}
\sstroutine{
   FLT.ZERO\_MASK
}{
   Speeds up convergences and reduces ringing by excluding
   sources within a region specified by an external mask file
   from the filtering performed by the FLT model
}{
   \sstdescription{
      If flt.zero\_mask is set to one of {\tt{"}}REF{\tt{"}}, {\tt{"}}MASK2{\tt{"}} or
      {\tt{"}}MASK3{\tt{"}} then an NDF will be obtained using the specified
      ADAM parameter (REF, MASK2 or MASK3) and used as a
      user-defined mask. Setting flt.zero\_mask to an integer value
      larger than zero has the same effect as setting it to {\tt{"}}REF{\tt{"}}.
      Setting it to an integer less than or equal to zero results
      in no external mask being used with the COM model. Note,
      using {\tt{"}}REF{\tt{"}} ensures that the mask and the output image of
      MAKEMAP are on the same pixel grid - using {\tt{"}}MASK2{\tt{"}} or {\tt{"}}MASK3{\tt{"}}
      does not provide this guarantee (it is then the users
      responsibility to ensure that the supplied masks are aligned
      with the output image in pixel coordinates). The pixels in
      the map that are to be included in filtering performed by
      the FLT model should be set to the bad value in the mask. All
      other pixels will be excluded from the filtering (i.e.
      they will be replaced by artifical data interpolated from the
      adjacent data). [0]
   }
   \sstattributetype{
      integer or string
   }
   \sstdiytopic{
      SMURF Usage
   }{
      MAKEMAP, CALCQU
   }
}
\sstroutine{
   FLT.ZERO\_NITER
}{
   Allows FLT masking to be switched off after a given
   number of iterations
}{
   \sstdescription{
      If flt.zero\_niter is non-zero, it gives the number of
      iterations for which the FLT model should be masked.
      Subsequent iterations are not masked. A value of zero
      means {\tt{"}}mask on all iterations{\tt{"}}. However, if parameter
      {\tt{"}}\xref{flt.zero\_notlast}{sun258}{FLT.ZERO_NOTLAST}{\tt{"}} is set, the mask will will not be applied
      on the last iteration, even if flt.zero\_niter is zero.
      Note, using FLT masking on many iterations can inhibit
      convergence. Also, any initial iterations specified by
      parameter {\tt{"}}\xref{ast.skip}{sun258}{AST.SKIP}{\tt{"}} are not included in the count of
      iterations. [2]
   }
   \sstattributetype{
      int
   }
   \sstdiytopic{
      SMURF Usage
   }{
      MAKEMAP, CALCQU
   }
}
\sstroutine{
   FLT.ZERO\_SNR
}{
   Speeds up convergences and reduces ringing by excluding samples
   that correspond to high SNR pixels in the map
}{
   \sstdescription{
      Setting the flt.zero\_snr parameter will prevent samples
      contributing to the FLT model if they fall within map pixels
      that have SNR values greater than flt.zero\_snr. A flt.zero\_snr
      value of zero means no SNR mask is used. See also parameter
      {\tt{"}}\xref{flt.zero\_snr\_ffclean}{sun258}{FLT.ZERO_SNR_FFCLEAN}{\tt{"}}.

      Note, the SNR values are only available once a map has been
      created, and so using this parameter results in no FLT masking
      on the first iteration. Consequently the map at the end of the
      first iteration will have deep rings around bright sources,
      since no FLT masking was done. Normally, these rings
      would polute the AST model derived from the map, and thus
      polute the residuals on the next iteration, resulting in
      the rings remaining in later maps. To avoid this, parameter
      {\tt{"}}\xref{ast.skip}{sun258}{AST.SKIP}{\tt{"}} can be set to a positive value. This causes the
      AST model to be skipped (i.e. no AST signal is subtracted from
      the residuals) for the first {\tt{"}}ast.skip{\tt{"}} iterations. This means
      that a good FLT mask can be formed from these initial iterations
      before any AST model is calculated and used. [0.0]
   }
   \sstattributetype{
      real
   }
   \sstdiytopic{
      SMURF Usage
   }{
      MAKEMAP, CALCQU
   }
}
\sstroutine{
   FLT.ZERO\_SNR\_FFCLEAN
}{
   Provides alternative method for SNR masking
}{
   \sstdescription{
      Setting this parameter to a non-zero value causes the
      SNR mask requested by parameter {\tt{"}}\xref{flt.zero\_snr}{sun258}{FLT.ZERO_SNR}{\tt{"}} to be
      created using an algorithm like that used by the KAPPA
      command {\tt{"}}FFCLEAN{\tt{"}} (see SUN/95), instead of using a simple
      thresholding of the SNR map. The parameter {\tt{"}}\xref{flt.zero\_snr}{sun258}{FLT.ZERO_SNR}{\tt{"}}
      gives the clipping level of the ffclean algorithm, and the
      parameter {\tt{"}}\xref{flt.zero\_snr\_hipass}{sun258}{FLT.ZERO_SNR_HIPASS}{\tt{"}} gives the box size. Using
      an ffclean algorithm prevents the source regions within
      the mask being larger than the box size, and may thus
      produce faster convergence and avoid blobs developing. [0]
   }
   \sstattributetype{
      int
   }
   \sstdiytopic{
      SMURF Usage
   }{
      MAKEMAP, CALCQU
   }
}
\sstroutine{
   FLT.ZERO\_SNRLO
}{
   Speeds up convergences and reduces ringing by increasing the
   size of the SNR mask without introducing noise
}{
   \sstdescription{
      If non-zero values are supplied for flt.zero\_snrlo and
      parameter {\tt{"}}\xref{flt.zero\_snr}{sun258}{FLT.ZERO_SNR}{\tt{"}}, then the basic mask created by
      thresholding at the SNR value specified by flt.zero\_snr is
      modified by expanding each un-masked {\tt{"}}source{\tt{"}} area down to
      an SNR equal to flt.zero\_snrlo, without introducing any new
      isolated source areas. The flt.zero\_snrlo should be lower than
      the flt.zero\_snr value. [0]
   }
   \sstattributetype{
      real
   }
   \sstdiytopic{
      SMURF Usage
   }{
      MAKEMAP, CALCQU
   }
}
\sstroutine{
   FLT.ZERO\_UNION
}{
   Controls how multiple FLT masks are combined
}{
   \sstdescription{
      If more than one FLT mask is specified (for instance, if
      values are supplied for both parameter {\tt{"}}\xref{flt.zero\_lowhits}{sun258}{FLT.ZERO_LOWHITS}{\tt{"}}
      and parameter {\tt{"}}\xref{flt.zero\_snr}{sun258}{FLT.ZERO_SNR}{\tt{"}}), then they are combined
      into a single mask. If flt.zero\_union is true (i.e.
      non-zero), then the source region in the combined mask is
      the union of the source regions in the individual masks.
      If flt.zero\_union is false (i.e. zero), then the source
      region in the combined mask is the intersection of the
      source regions in the individual masks. [1]
   }
   \sstattributetype{
      integer
   }
   \sstdiytopic{
      SMURF Usage
   }{
      MAKEMAP, CALCQU
   }
}
\sstroutine{
   HITSLIMIT
}{
   Rejects map pixels that receive very few samples
}{
   \sstdescription{
      If non-zero, pixels that receive very few bolometer samples
      are set to bad in the final map. The limiting number of
      bolometer samples is equal to {\tt{"}}hitslimt{\tt{"}} times the mean
      number of hits per pixel, averaged over the map pixels
      that recieve at least one bolometer sample. [0.01]
   }
   \sstattributetype{
      real
   }
   \sstdiytopic{
      SMURF Usage
   }{
      MAKEMAP, CALCQU
   }
}
\sstroutine{
   ITERMAP
}{
   Create NDFs holding the map created by each iteration
}{
   \sstdescription{
      If itermap is set to a positive value, the map from each iteration
      of each chunk will be stored in an output NDF. If itermap is set
      to a negative value, only the final iteration will be written from
      each chunk. If its absolute value is larger than 1, then each
      itermap will include a quality component that reflects the AST
      mask in use.

      By default, each itermap NDF will be stored in an extension
      called .MORE.SMURF.ITERMAPS in the main output NDF. However,
      an alternative location can be specified by supplying a value
      for ADAM parameter ITERMAPS. This is useful as it allows you
      to look at earlier itermaps whilst makemap is still running. [0]
   }
   \sstattributetype{
      integer
   }
   \sstdiytopic{
      SMURF Usage
   }{
      MAKEMAP, CALCQU
   }
}
\sstroutine{
   MAPTOL
}{
   Specifies when to stop iterating
}{
   \sstdescription{
      If the normalised change (either the mean or maximum change -
      see parameter {\tt{"}}\xref{maptol\_mean}{sun258}{MAPTOL_MEAN}{\tt{"}}) between the maps created on
      subsequent iterations falls below the value of maptol, then the
      map-maker performs one more iteration and then terminates. Only
      used if parameter {\tt{"}}\xref{numiter}{sun258}{NUMITER}{\tt{"}} is negative. The normalised mean
      (or maximum) change between maps is defined as the mean (or
      maximum) of the absolute change in map pixel value, taken
      over all pixels within the region of the AST mask (if any,
      see parameter {\tt{"}}\xref{ast.zero\_mask}{sun258}{AST.ZERO_MASK}{\tt{"}}, etc), and normalised by the RMS
      of the square root of the pixel variances. Compared to parameter
      {\tt{"}}\xref{chitol}{sun258}{CHITOL}{\tt{"}}, this is much more like a {\tt{"}}by eye{\tt{"}} test, that will stop
      the solution when the map stops changing. [0.05]
   }
   \sstattributetype{
      real
   }
   \sstdiytopic{
      SMURF Usage
   }{
      MAKEMAP, CALCQU
   }
}
\sstroutine{
   MAXLEN
}{
   Determines how the input time-series data is split into chunks
}{
   \sstdescription{
      The maximum length (in seconds) for a single chunk of
      concatenated data. If 0 is supplied, attempt to concatenate
      entire continuous chunks. [0]
   }
   \sstattributetype{
      integer
   }
   \sstdiytopic{
      SMURF Usage
   }{
      MAKEMAP, CALCQU
   }
}
\sstroutine{
   MODELORDER
}{
   Specifies which models to include in the iterative
   process, and the order in which they are evaluated
}{
   \sstdescription{
      This should be a comma-separated list, in parentheses,
      containing one or more of the following model names, in the
      order in which they should be evaluated. Note: components
      specified AFTER 'ast' will not be calculated for the first
      time until the second iteration:

      \sstitemlist{

         \sstitem
         dks: fit and remove dark squid for the column

         \sstitem
         com: remove common-mode signal

         \sstitem
         gai: if com specified, fit gain/offset of common mode

         \sstitem
         ext: apply extinction correction

         \sstitem
         ast: estimate the map and astronomical signal

         \sstitem
         flt: apply filter to time streams

         \sstitem
         noi: estimate time-domain variance

         \sstitem
         smo: time series smoothing using a median or mean boxcar filter

         \sstitem
         ssn: scan-synchronous (i.e. azimuth dependent) noise removal

         \sstitem
         pln: remove plane from each time slice

         \sstitem
         tmp: remove externally define template such as azimuth [(com,gai,ext,flt,ast,noi)]
      }
   }
   \sstattributetype{
      list of strings
   }
   \sstdiytopic{
      SMURF Usage
   }{
      MAKEMAP, CALCQU
   }
}
\sstroutine{
   NOI.BOX\_SIZE
}{
   Allow finer estimation of the noise levels in the time-series
   data
}{
   \sstdescription{
      Specifies the number of time slices used to determine the
      noise level in a section of a bolometer time stream. If zero,
      then the whole bolometer time stream is used, and each
      bolometer has only one variance value. If non-zero, each
      bolometer time stream is divided up into boxes containing the
      specified number of time slices, and a separate variance
      is found for each box. This variance is then used for each
      sample in the box, so each bolometer ends up with a variance
      for every time slice. Negative values are interpreted as number
      of seconds, and positive values as a number of down-sampled
      time slices. Note, very small box sizes may produce
      unrepresentative noise levels, and there is a hard minimum
      of 101 on the number of downsampled time slices in a noise
      box. Also, if the number of time slices in the data is smaller
      than two times the requested box size, then a single noise
      value is used for each bolometer. [-15]
   }
   \sstattributetype{
      real
   }
   \sstdiytopic{
      SMURF Usage
   }{
      MAKEMAP, CALCQU
   }
}
\sstroutine{
   NOI.BOX\_TYPE
}{
   Determines how the noise in each box is found
}{
   \sstdescription{
      If this is zero, the noise in each box (see parameter
      {\tt{"}}\xref{noi.box\_size}{sun258}{NOI.BOX_SIZE}{\tt{"}}) is found by taking Fourier transform of the
      residuals in each box and then using the mean power in the
      range 2 to 10 Hz as the noise. If it is non-zero, the noise
      for each residual is set to the variance of the neighbouring
      residuals in a box centred on the residual. Using this
      scheme causes the noise values to vary continuously with
      time, whereas the FFT scheme produced blocks of equal
      noise values. When using a small box size, {\tt{"}}noi.box\_type=1{\tt{"}}
      will often result in far fewer samples being flagged as
      unusable. Note, if {\tt{"}}noi.box\_size{\tt{"}} is set to zero, then the
      value of {\tt{"}}noi.box\_type{\tt{"}} is ignored and the noise is always
      calculated on the basis of the mean power in the 2 to 10
      Hz band. [1]
   }
   \sstattributetype{
      integer
   }
   \sstdiytopic{
      SMURF Usage
   }{
      MAKEMAP, CALCQU
   }
}
\sstroutine{
   NOI.CALCFIRST
}{
   Determines when the noise in each bolometer is estimated
}{
   \sstdescription{
      If a non-zero value is supplied, the bolometer noise levels are
      calculated immediately after pre-conditioning. Otherwise, they
      calculated at the end of the first iteration. The former can
      reduce execution time if parameter {\tt{"}}noiseclip{\tt{"}} is also set
      since both operations share a single FFT. [0]
   }
   \sstattributetype{
      integer
   }
   \sstdiytopic{
      SMURF Usage
   }{
      MAKEMAP, CALCQU
   }
}
\sstroutine{
   NOISECLIPHIGH
}{
   Reject bolometers based on their noise
}{
   \sstdescription{
      This step will remove any bolometers noisier than
      noisecliphigh standard deviations above the median, or
      noisecliplow standard deviations below the median. Normally
      the noise clipping happens at the end of the cleaning stage,
      but if you set noiseclipprecom it will instead occur
      immediately prior to common-mode subtraction (see
      parameter {\tt{"}}comppreprocess{\tt{"}}). [4]
   }
   \sstattributetype{
      real
   }
   \sstdiytopic{
      SMURF Usage
   }{
      SC2CLEAN, CALCQU, MAKEMAP
   }
}
\sstroutine{
   NOISECLIPLOW
}{
   Reject bolometers based on their noise
}{
   \sstdescription{
      See parameter {\tt{"}}\xref{noisecliphigh}{sun258}{NOISECLIPHIGH}{\tt{"}}. [0]
   }
   \sstattributetype{
      real
   }
   \sstdiytopic{
      SMURF Usage
   }{
      SC2CLEAN, CALCQU, MAKEMAP
   }
}
\sstroutine{
   NOISECLIPPRECOM
}{
   Reject bolometers based on their noise
}{
   \sstdescription{
      See parameter {\tt{"}}\xref{noisecliphigh}{sun258}{NOISECLIPHIGH}{\tt{"}}. [0]
   }
   \sstattributetype{
      integer
   }
   \sstdiytopic{
      SMURF Usage
   }{
      SC2CLEAN, CALCQU, MAKEMAP
   }
}
\sstroutine{
   NUMITER
}{
   Specifies when to stop iterating
}{
   \sstdescription{
      If a positive number is supplied, the specified number of
      iterations will always be performed. If a negative number is
      supplied, the absolute value gives the maximum number of
      iterations to perform. Fewer iterations will be performed if
      the termination criteria specified by parameter {\tt{"}}\xref{maptol}{sun258}{MAPTOL}{\tt{"}} and
      parameter {\tt{"}}\xref{chitol}{sun258}{CHITOL}{\tt{"}} are both met before {\tt{"}}-numiter{\tt{"}}
      iterations have been performed. [-5]
   }
   \sstattributetype{
      real
   }
   \sstdiytopic{
      SMURF Usage
   }{
      MAKEMAP, CALCQU
   }
}
\sstroutine{
   ORDER
}{
   Baseline removal
}{
   \sstdescription{
      Subtract a baseline polynomial of this order as part of
      the initial cleaning phase. [1]
   }
   \sstattributetype{
      integer
   }
   \sstdiytopic{
      SMURF Usage
   }{
      SC2CLEAN, CALCQU, MAKEMAP
   }
}
\sstroutine{
   SHORTMAP
}{
   Create NDFs holding the map made form a short chunk of data
}{
   \sstdescription{
      If non-zero, then an extension called .MORE.SMURF.SHORTMAPS
      is added to the output map NDF, holding maps made from every
      group of {\tt{"}}shortmap{\tt{"}} adjacent time slices. Alternatively, set
      to -1 to produce a map each time the TCS\_INDEX value within
      the JCMTSTATE extension is incremented (i.e., each time a full
      pass through the scan pattern has been completed). Any other
      negative value is interpreted as a duration in seconds, and is
      converted to time slices using the (possibly down-sampled)
      sample frequency of the data being mapped. [0]
   }
   \sstattributetype{
      integer
   }
   \sstdiytopic{
      SMURF Usage
   }{
      MAKEMAP, CALCQU
   }
}
\sstroutine{
   SPIKEBOX
}{
   Controls time-based spike detection within initial data
   cleaning
}{
   \sstdescription{
      Size of filter box for the sigma-clipper within the
      initial data cleaning, in units of samples if positive and
      seconds if negative. For instance, setting spikebox to 50
      will check for excursions from a rolling median filter in a
      box of length 50 samples. Also see parameter {\tt{"}}\xref{spikethresh}{sun258}{SPIKETHRESH}{\tt{"}}. [50]
   }
   \sstattributetype{
      integer
   }
   \sstdiytopic{
      SMURF Usage
   }{
      SC2CLEAN, CALCQU, MAKEMAP
   }
}
\sstroutine{
   SPIKETHRESH
}{
   Controls time-based spike detection within initial data
   cleaning
}{
   \sstdescription{
      The SNR value at which to flag spikes within the
      sigma-clipper used within initial data cleaning. Also see
      parameter {\tt{"}}\xref{spikebox}{sun258}{SPIKEBOX}{\tt{"}}. No de-spiking is performed in the
      initial data cleaning if a value of zero is supplied. [0]
   }
   \sstattributetype{
      integer
   }
   \sstdiytopic{
      SMURF Usage
   }{
      SC2CLEAN, CALCQU, MAKEMAP
   }
}
\sstroutine{
   WHITEN
}{
   Experimental
}{
   \sstdescription{
      If non-zero, then a whitening filter will be applied to
      each time stream prior to any filtering that is done as
      part of the initial data cleaning, as specified by
      parameter {\tt{"}}flt\_edge\_largescale{\tt{"}}, etc. [0]
   }
   \sstattributetype{
      integer
   }
   \sstdiytopic{
      SMURF Usage
   }{
      SC2CLEAN, CALCQU, MAKEMAP
   }
}
