\chapter{Introduction}
\label{sec:intro}

% set up page numbers in arabic numerals and restart from 1
\renewcommand{\thepage}{\arabic{page}}
\setcounter{page}{1}

\section{This cookbook}

This guide is designed to instruct SCUBA-2 users on the best ways to
reduce and visualise their data using \starlink\ packages:
\smurf \cite{smurf}, \Kappa \cite{kappa}, \gaia \cite{gaia}, \oracdr\
\cite{oracdr} and \picard \cite{picard}.

This guide covers the following topics.
\begin{itemize}
\itemsep0em
\item \cref{Chapter}{sec:intro}{Chapter 1} -- Computer resources you'll need before getting started.
\item \cref{Chapter}{sec:s2}{Chapter 2} -- A description of SCUBA-2 and its observing modes.
\item \cref{Chapter}{sec:dimm}{Chapter 3}  -- An introduction to the Dynamic Iterative Map-Maker and
a description of the configuration files.
\item \cref{Chapter}{sec:pipe}{Chapter 4}  -- Instructions for using the \oracdr\ pipeline to reduce
your data ``the easy way'', along with details on data retrieved from the JSA.
\item \cref{Chapter}{sec:manual}{Chapter 5}  -- Instructions for using individual \smurf\ commands to
reduce your data---useful if you need extra flexibility.
\item \cref{Chapter}{sec:tweak}{Chapter 6}  -- Options for tailoring the configuration parameters to
improve your final map.
\item \cref{Chapter}{sec:eg}{Chapter 7}  -- Two worked examples covering a \htmlref{blank cosmology
field}{sec:cosmology} and a \htmlref{galactic field}{sec:bright_ex}.
\item \cref{Chapter}{sec:postprocess}{Chapter 8}  -- Post-processing reduction steps such as applying
the FCF, co-adding multiple maps and estimating the noise.
\item \cref{Chapter}{sec:raw}{Chapter 9}  -- SCUBA-2 Diagnostic Tools.


\end{itemize}

Throughout this document, a percent sign (\texttt{\%}) is used to
represent the Unix shell prompt. What follows each \texttt{\%} will be
the text that you should type to initiate the described action.

\section{\xlabel{computing} Before you start: computing resources}
\label{sec:computing}

Before reducing SCUBA-2 data using the Dynamic Iterative Map-Maker you should confirm you have
sufficient computing resources for your type of map.

We recommend the following:
\begin{table}[h!]
  \centering
  \begin{tabular}{ll}
    \hline
    \textbf{Reduction type} &\textbf{Memory} \\
    \hline
    Large maps (\textsc{pong})& 96 GB\\
    Small maps (\textsc{daisy})&32--64 GB\\
    Blank fields&32--64 GB\\
    \hline
  \end{tabular}
\end{table}

\subsubsection*{Why these recommendations?}

For large-area maps it is important to process a full observation in a
single chunk. See the text box on
\latexhtml{Page~\pageref{box:chunk}}{\htmlref{Chunking}{box:chunk}}\ for
an explanation of chunking. For large maps, using normal map-maker
parameters a machine having 96\,GB is acceptable. It is important that
the memory is as fast as can be afforded, as RAM speed has a direct
linear effect on processing time given that the time-series data are
continually being funnelled through the CPU.

For blank-field surveys or smaller
regions of the sky you can usefully run the map-maker with less memory
and 32 to 64\,GB is reasonable depending on the specifics of your data
set. \textsc{smurf} is multi-threaded so multiple cores do help
although above eight cores the price/performance gains tend to drop
off.

If you have a very large machine (128\,GB and 24 cores) you may be able
to run two instances of the map-maker in parallel without chunking,
depending on the nature of the data. Use
the \envvar{SMURF\_THREADS}\footnote{\envvar{SMURF\_THREADS} should be
set to an integer value indicating the number of threads to be used by
each process.} environment variable to restrict each map-maker to half
the available cores.

\section{\xlabel{software}Before you start: software}

This manual uses software from the \starlink\ packages: \smurf\
\cite{smurf}, \Kappa\ \cite{kappa}, \gaia\ \cite{gaia}, \oracdr\
\cite{oracdr} and \picard\ \cite{picard}. Starlink software must be
installed on your system, and Starlink aliases and environment
variables must be defined before attempting to reduce any SCUBA-2
data.

\subsection{Data formats}
\label{sec:ndf}

Data files for SCUBA-2 use the Starlink N-dimensional Data Format (NDF,
see Jenness et al.\ 2014\cite{ndf}), a hierarchical format which allows
additional data and metadata to be stored within a single file. \Kappa\
contains \xref{many commands}{sun95} {ap_classified}\ for examining and
manipulating NDF structures. The introductory sections of the \Kappa\
document (\xref{SUN/95}{sun95}{}) contain much useful information on
the contents of an NDF structure and how to manipulate them.

A single NDF structure describes a single data array with associated
meta-data. NDFs are usually stored within files of type ``\verb+.sdf+''.
In most cases (but not all), a single \verb+.sdf+ file will contain just
one top-level NDF structure, and the NDF can be referred to simply by
giving the name of the file (with or without the ``\verb+.sdf+'' prefix).
In many cases, a top-level NDF containing JCMT data will contain other
``extension'' NDFs buried inside them at a lower level. For instance, raw
files contain a number of NDF components which store observation-specific
data necessary for subsequent processing. The contents of these (and
other NDF) files may be listed with \HDSTRACEref. Each file holding raw
JCMT data on disk is also known as a `sub-scan'.

The main components of any NDF structure are:
\begin{itemize}
\item An array of numerical data (may have up to seven
dimensions---usually three for JCMT data);
\item An array of variance values corresponding to the numerical data
values;
\item An array holding up to eight boolean flags (known as ``quality
flags'', see \cref{Section}{sec:maskshow}{Displaying masks}) for each pixel;
\item World Coordinate System information;
\item History;
\item Data units
\item Other extensions items. These are defined by particular packages,
but usually include a list of FITS-like headers together with provenance
information that indicates how the NDF was created. Raw JCMT files also
include extensions that define the state of the telescope and instrument
at each time slice within the observation.
\end{itemize}

The \convert\ package contains commands \xref{\task{fits2ndf}}{sun55}{FITS2NDF} and
\xref{\task{ndf2fits}}{sun55}{NDF2FITS} that allow interchange between FITS
and NDF format.

\subsection{Initialising Starlink}
\label{sec:starinit}

The commands and environment variables needed to start up the required
Starlink packages (\smurf \cite{smurf}, \Kappa, \emph{etc.}) must first
be defined. For C shells (csh, tcsh), do:

\begin{terminalv}
% setenv STARLINK_DIR <path to the starlink installation>
% source $STARLINK_DIR/etc/login
% source $STARLINK_DIR/etc/cshrc
\end{terminalv}

before using any Starlink commands. For Bourne shells (sh, bash, zsh), do:

\begin{terminalv}
% export STARLINK_DIR=<path to the starlink installation>
% source $STARLINK_DIR/etc/profile
\end{terminalv}

\subsection{KAPPA and SMURF for data processing}
\label{sec:packinit}

The Sub-Millimetre User Reduction Facility, or \textsc{Smurf},
contains the Dynamic Iterative Map-Maker, which will process raw
SCUBA-2 data into images (see \smurfsun). \textsc{Kappa} meanwhile is
an application package comprising general-purpose commands mostly for
manipulating and visualising NDF data (see \kappasun). Before starting
any data reduction you will want to initiate both \textsc{Smurf} and
\textsc{Kappa}.

\begin{terminalv}
% smurf
% kappa
\end{terminalv}

After entering the above commands, you can access the help information
for either package by typing \texttt{smurfhelp} or
\texttt{kaphelp} respectively in a terminal, or by using the
\task{showme} facility to access the hypertext documentation. See
\cref{Section}{sec:help}{How to get help} for more information.



\begin{tip}
The .sdf extension on filenames need not be specified when running most
Starlink commands (the exception is \picard).
\end{tip}


\subsection{GAIA for viewing your map}

Image visualisation can be done with \gaia\ (see
\gaiasun). \textsc{Gaia} is a GUI-driven image and data-cube display and
analysis tool, which incorporates facilities such as source detection,
three-dimensional visualisation, photometry and the ability to query
and overlay on-line or local catalogues.
\begin{terminalv}
% gaia map.sdf
\end{terminalv}

Alternatively, the \Kappa\ package includes many visualisation commands
that can be run from the shell comand-line or incorporated easily into your
own scripts---see Appendix ``\xref{Classified KAPPA commands}{sun95}{cl_datadisplay}''
in SUN/95. These tools are particularly useful for creating more complex
composite plots including multiple images, line-plots, \emph{etc}, such
as the multi-image plots in \cref{Section}{sec:itermaps}{Monitoring the
map at the end of each iteration}.

\subsection{ORAC-DR for running the pipeline}

The \oracdr\ Data-Reduction Pipeline \cite{oracdr} (hereafter just
\textsc{Orac-dr}) is an automated reduction pipeline. \textsc{Orac-dr}
uses \smurf\ and \Kappa\ (along with other Starlink tools) to perform
an automated reduction of the raw data following pre-defined recipes
to produce calibrated maps.  The following commands initialise
\textsc{Orac-dr} ready to process 850\,$\mu$m and 450\,$\mu$m data
respectively.
\begin{terminalv}
% oracdr_scuba2_850
% oracdr_scuba2_450
\end{terminalv}
For more information on available recipes and intructions for running the pipeline
see \cref{Chapter}{sec:pipe}{The SCUBA-2 Pipeline}.

\subsection{PICARD for post-reduction processing}

\textsc{Picard} uses a pipeline system similar to \oracdr\ for
post-processing and analysis of reduced data. \textsc{Picard}
documentation can be found at
\href{http://www.oracdr.org/oracdr/PICARD}{\textsc{Orac-dr} web page},
or at \picardsun. All \textsc{Picard} recipes follow the same
structure and are run like so:
\begin{terminalv}
% picard -recpars <recipe_params_file> RECIPE <input_files>
\end{terminalv}
where \param{<recipe\_param\_file>} is a text file containing the
relevant recipe parameters. \param{RECIPE} is the name of the recipe
to be run (note the caps). The list of files to be processed is given
by  \param{<input\_files>}. These must be in the current directory or a
directory defined by the environment variable \envvar{ORAC\_DATA\_IN}. A
number of \textsc{Picard} recipes will be demonstrated in
\cref{Chapter}{sec:postprocess}{Post-processing Reduction Steps}.

Other command-line options include \texttt{-log xsf} where the log
file is written to any combination of the screen [\texttt{s}], a file
[\texttt{f}] or an X-window [\texttt{x}]. \texttt{s} or \texttt{sf} is
recommended as the recipes are short and the X-window automatically
closes upon completion. To ensure an X-window does not open, you
can use the command-line option: \texttt{-nodisplay}.

You do not specify an output filename for \picard, instead the output
is generated by adding a recipe depending suffix to the input
filename. If there is more than one input file then the name of the
last file is used.

You can create a file which lists the input files to be passed to
\picard\ for processing. this file is read by \picard\ via the
Linux/Unix \task{cat} command. For example:

\begin{terminalv}
% picard -log s RECIPE_NAME `cat myfilestoprocess.lis`
\end{terminalv}

To execute the \task{cat} command you must enclose it in back
quotes. You must also include the \file{.sdf} extension on any files
passed to \picard.

\begin{tip}
  Unlike other Starlink packages, the .sdf extension must be included
  when supplying the names of Starlink data files to \textsc{Picard}.
\end{tip}

\begin{tip}
  If the environment variable \envvar{ORAC\_DATA\_OUT} is defined, any
  files created by \textsc{Picard} will be written in that
  location. Check there if new files are expected but do not appear in
  your working directory.
\end{tip}

\newpage

\subsection{\xlabel{help}How to get help}
\label{sec:help}

\begin{table}[h!]
\begin{tabular}{p{2.3cm}|p{7.3cm}|p{5cm}}
\hline
\textbf{Help\newline command} & \textbf{Description} & \textbf{Usage}\\
\hline
\task{showme} & If you know the name of the Starlink document you want to view,
                use \task{showme}. When run, it launches a new webpage or tab
                displaying the hypertext version of the document. &
\texttt{\%~showme~sun95}\\
\hline
\task{findme} & \task{findme} searches Starlink documents for a keyword. When
                run, it launches a new webpage or tab listing the results. &
                \texttt{\% findme~kappa}\\
\hline
\task{docfind} & \task{docfind} searches the internal list files for keywords. It then
                 searches the document titles. The result is displayed using the
                 Unix \task{more} command. & \texttt{\%~docfind~kappa}\\
\hline
Run routines with prompts & You can run any routine with the option
                            \texttt{prompt} after the command. This will
                            prompt for every parameter available. If you
                            then want a further description of any parameter
                            type  \texttt{?} at the relevant prompt. &
                            \texttt{\%~makemap~prompt~\newline\~\%~REF~-~Ref.~NDF~/!/$>$~?}\\
\hline
Google & A simple Google search such as ``\texttt{starlink kappa fitslist}''
will usually return links to the appropriatre documents. However, be
aware that the results may include links to out of date versions of the
document hosted at non-Starlink sites. Always look for results in
\texttt{"www.starlink.ac.uk/docs} (or \texttt{"www.starlink.ac.uk/devdocs}
for the cutting-edge development version of the document). & \\
\hline
\end{tabular}
\end{table}

\section{Processing options}

You have two options for processing your data:

\begin{enumerate}
\item running the automated pipeline (\textsc{Orac-dr}), or
\item performing each step manually.
\end{enumerate}

The pipeline approach is simpler and works well if you have a
lot of data to process.
Performing each step by hand allows more fine-tuned control of certain
processing and analysis steps, and is especially useful for refining
the parameters used by the map-maker. However, once the optimal
parameters have been determined, it is possible to pass them to the
pipeline to process other observations using the same configuration.
\cref{Chapter}{sec:dimm}{The Dynamic Iterative Map-maker
Explained} and \cref{Chapter}{sec:manual}{Running makemap Outside the
Pipeline} discuss the manual approach; to use the science pipeline,
skip straight to \cref{Chapter}{sec:pipe}{The SCUBA-2 Pipeline}.

The JCMT will produce pipeline-reduced files for each observation and
group of repeat observations for each night. These are reduced using
the \oracdr\ pipeline with the recipe specified in the MSB.
\cref{Chapter}{sec:pipe}{The SCUBA-2 Pipeline} gives instruction on
retrieving reduced data from the \htmladdnormallink{JCMT Science
  Archive}{http://www3.cadc-ccda.hia-iha.nrc-cnrc.gc.ca/jcmt/} at
CADC.
