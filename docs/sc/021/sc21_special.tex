\chapter{\xlabel{special}Specialised configuration files}
\label{app:special}

\section{\file{dimmconfig\_jsa\_generic.lis}}
\begin{terminalv}
^$STARLINK_DIR/share/smurf/dimmconfig.lis

#  Less aggressive cleaning to cope with bright sources
noisecliphigh=10.0
dcthresh = 100

#  Don't want extended structure, so avoid problems with COM model by using
#  individual common-mode models for each subarray.
com.perarray = 1

#  Aggressive filtering.
flt.filt_edge_largescale=200

#  Allow bolometer noise to vary with time, and using a box filter to
#  determine mean noise in each box, in order to presevre as many samples
#  as possible.
noi.box_size=-15
noi.box_type=1

#  Use a maximum of 20 iterations
numiter=-25
maptol_mean=1
maptol=0.01

# new recommendations and using an ast model
ast.zero_snr = 5
ast.zero_snrlo = 3

ast.skip=5
flt.zero_snr=5
flt.zero_snrlo=3
\end{terminalv}


\section{\file{dimmconfig\_bright\_extended.lis}}
\begin{terminalv}
^$STARLINK_DIR/share/smurf/dimmconfig.lis
   numiter=-40
   flt.filt_edge_largescale=480
   ast.zero_snr=3
   ast.zero_snrlo=2

   ast.skip=5
   flt.zero_snr=5
   flt.zero_snrlo=3

\end{terminalv}

\section{\file{dimmconfig\_bright\_compact.lis}}
\begin{terminalv}

^$STARLINK_DIR/share/smurf/dimmconfig_bright.lis
   numiter=-40

#  Per array common-mode should be fine here since we are dealing with
#  a compact source. It seems to make things more stable.
   com.perarray = 1

#  We can get away with harsher filtering since the boundary conditions are
#  quite tight
   flt.filt_edge_largescale=200

#  Use boundary constraints since the source is assumed to be isolated
   ast.zero_circle = (0.0166666666)

#  Mask the data when forming th FLT model in order to exclude the
#  source. This only happends on the first two iterations. This usually
#  speeds up convergence.
   flt.zero_circle = (0.0166666666)

\end{terminalv}


\section{\file{dimmconfig\_blank\_field.lis}}
\begin{terminalv}
   ^$STARLINK_DIR/share/smurf/dimmconfig.lis

   numiter = 4

#  No FLT model is needed since we are filtering out low frequencies as
#  part of the initial cleaning process.
   modelorder = (com,ext,ast,noi)

#  Use time-domain de-spiker first because we are only doing a single
#  FFT-based high-pass filter before the iterations start.
   spikethresh = 10

#  Heavier high-pass filtering. Note that the default padding/apodization
#  that will be used is the number of samples that corresponds to the
#  period of the knee frequency in the high-pass filter (i.e. 200*(1/freq) )
   filt_edge_largescale = 200

#  Large-scale structure is not an issue so treat each sub-array
#  independently for common-mode removal
   com.perarray = 1

\end{terminalv}



\section{\file{dimmconfig\_bright.lis}}

This file is included as although it is not used in its own right it
form the basis (on top of \file{dimmconfig.lis} for both
\file{dimmconfig\_bright\_compact.lis} and
\file{dimmconfig\_bright\_extended.lis}
\begin{terminalv}

   ^$STARLINK_DIR/share/smurf/dimmconfig.lis

   numiter = 20

#  Much weaker bolometer noise clip
   noisecliphigh = 10.0

#  Less aggressive DC step finder to avoid problems with bright sources
   dcthresh = 100

#  Less aggressive bolo flagging.
   com.corr_tol = 7
   com.gain_tol = 7
   com.gain_abstol = 5
\end{terminalv}

