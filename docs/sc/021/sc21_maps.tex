\chapter{\xlabel{maps}Reducing Your Data}
\label{sec:maps}

This chapter describes how to run the map-maker and what to look out
for during processing. We also discuss reducing your data using the
\oracdr\ science pipeline and why you might want to chose this option.

As discussed in \cref{Chapter}{sec:dimm}{The
Dynamic Iterative Map-Maker}, all of the settings for the map-maker
are stored in configuration files. In this chapter we use
the default configuration file, \file{dimmconfig.lis} to reduce CRL~2688,
one of SCUBA-2's secondary calibrators. For an
overview of the specialised configuration files available see
\cref{Section}{sec:config}{this section}.

\section{\xlabel{running_dimm}Running the iterative map-maker}
\label{sec:running}

When running the map-maker, you can call any of the provided
configuration files directly from the Starlink path (e.g.
\file{\^{}\$STARLINK\_DIR/share/smurf/dimmconfig.lis}). Alternatively,
a local copy can be made and edited. Advice on which parameters to
edit can be found in \cref{Section}{sec:tweak}{Tweaking the
configuration file}.

\begin{tip}
  An up-caret (\,\texttt{\^{}}\,) is required any time you are reading
  in a group text file in \starlink. For the map-maker this includes
  the configuration file (a group of parameters) and a list of your
  input files (e.g. \texttt{in=\^{}\,myrawfiles.txt}).
\end{tip}


The default pixel sizes are:
\begin{itemize}
\item 2\,arcsec at 450$\mu$m
\item 4\,arcsec at 850$\mu$m
\end{itemize}

These can be changed by adding the \texttt{pixsize=}$x$ to the
command string\footnote{The default sizes are defined as one quarter
of the Airy disk rounded up to the nearest half arcsecond.}, where $x$
is your desired pixel size in arcsecs. We advise that you do not
increase the pixel size at this stage as it will compromise model
fitting---instead regrid your map as a post-processing step.

% Other ADAM parameter available with \makemap\ include ref, maxmem
% and msg\_filter. Some of these appear in the example below but see
% \cref{Appendix}{app:adam}{MAKEMAP ADAM Parameters} for
% descriptions. A complete list of all ADAM parameters can be found in
% \smurfsun.

\begin{tip}
  Map-maker not finding your raw files from a path? Check you have
  double quotes around your `in' option and are not using
  \texttt{*sdf}---use either \texttt{*.sdf} or just \texttt{*}.
\end{tip}



\begin{terminalv}
% makemap in="/jcmtdata/raw/scuba2/s8*/20120720/00030/*.sdf" out=850_crl2688 \
  config=^$STARLINK_DIR/share/smurf/dimmconfig.lis

Out of 32 input files, 4 were darks, 8 were fast flats and 20 were science
Processing data from instrument 'SCUBA-2' for object 'CRL2688' from the
following observation  :
  20120720 #30 scan  /shutter

MAKEMAP: Map-maker will use no more than 68401 MiB of memory

Projection parameters used:
CRPIX1 = 0
CRPIX2 = 0
CRVAL1 = 315.578333333333 ( RA = 21:02:18.800 )
CRVAL2 = 36.6938055555556 ( Dec = 36:41:37.70 )
CDELT1 = -0.00111111111111111 ( -4 arcsec )
CDELT2 = 0.00111111111111111 ( 4 arcsec )
CROTA2 = 0

Output map pixel bounds: ( -132:122, -126:129 )

Output map WCS bounds:
Right ascension: 21:01:38.318 -> 21:03:03.280
Declination: 36:33:07.19 -> 36:50:11.70

smf_iteratemap: will down-sample data to match angular scale of 4 arcsec
smf_iteratemap: Iterate to convergence (max 5)
smf_iteratemap: stop when change in chi^2 < 0.001
smf_iteratemap: provided data are in 1 continuous chunks, the largest of which
has 5957 samples (153.729 s)
smf_iteratemap: map-making requires 1376 MiB (map=3 MiB model calc=1372 MiB)
smf_iteratemap: Continuous chunk 1 / 1 =========
smf_calc_smoothedwvm: 0.977444 s to calculate unsmoothed WVM tau values
smf_iteratemap: Iteration 1 / 5 ---------------
--- Size of the entire data array ------------------------------------------
bolos  : 5120
tslices: bnd:0(0.0 min), map:5957(2.6 min), tot:5957(2.6 min)
Total samples: 30499840
--- Quality flagging statistics --------------------------------------------
 BADDA:   10972794 (35.98%),        1842 bolos
BADBOL:   11818688 (38.75%),        1984 bolos
DCJUMP:      38809 ( 0.13%),
  STAT:      71680 ( 0.24%),          14 tslices
 NOISE:     810152 ( 2.66%),         136 bolos
Total samples available for map:   18634826, 61.10% of max (3128.22 bolos)
smf_iteratemap: Calculate time-stream model components
smf_iteratemap: Rebin residual to estimate MAP
smf_iteratemap: Calculate ast
--- Quality flagging statistics --------------------------------------------
 BADDA:   10972794 (35.98%),        1842 bolos  ,change          0 (+0.00%)
BADBOL:   11925914 (39.10%),        2002 bolos  ,change     107226 (+0.91%)
DCJUMP:      38809 ( 0.13%),                    ,change          0 (+0.00%)
  STAT:      71680 ( 0.24%),          14 tslices,change          0 (+0.00%)
   COM:     323165 ( 1.06%),                    ,change     323165 (+0.00%)
 NOISE:     810152 ( 2.66%),         136 bolos  ,change          0 (+0.00%)
Total samples available for map:   18312771, 60.04% of max (3074.16 bolos)
     Change from last report:    -322055, -1.73% of previous
smf_iteratemap: Will calculate chi^2 next iteration
smf_iteratemap: *** NORMALIZED MAP CHANGE: 0.874979 (mean) 73.7106 (max)
smf_iteratemap: Iteration 2 / 5 ---------------
smf_iteratemap: Calculate time-stream model components
smf_iteratemap: Rebin residual to estimate MAP
smf_iteratemap: Calculate ast
--- Quality flagging statistics --------------------------------------------
 BADDA:   10972794 (35.98%),        1842 bolos  ,change          0 (+0.00%)
BADBOL:   11949742 (39.18%),        2006 bolos  ,change      23828 (+0.20%)
 SPIKE:         34 ( 0.00%),                    ,change         34 (+0.00%)
DCJUMP:      38809 ( 0.13%),                    ,change          0 (+0.00%)
  STAT:      71680 ( 0.24%),          14 tslices,change          0 (+0.00%)
   COM:     357816 ( 1.17%),                    ,change      34651 (+10.72%)
 NOISE:     810152 ( 2.66%),         136 bolos  ,change          0 (+0.00%)
Total samples available for map:   18278374, 59.93% of max (3068.39 bolos)
     Change from last report:     -34397, -0.19% of previous
smf_iteratemap: *** CHISQUARED = 0.983228126551834
smf_iteratemap: *** NORMALIZED MAP CHANGE: 1.29181 (mean) 15.0552 (max)
smf_iteratemap: Iteration 3 / 5 ---------------
.....
.....
.....
smf_iteratemap: Iteration 5 / 5 ---------------
smf_iteratemap: Calculate time-stream model components
smf_iteratemap: Rebin residual to estimate MAP
smf_iteratemap: Calculate ast
--- Quality flagging statistics --------------------------------------------
 BADDA:   10972794 (35.98%),        1842 bolos  ,change          0 (+0.00%)
BADBOL:   11949742 (39.18%),        2006 bolos  ,change          0 (+0.00%)
 SPIKE:         34 ( 0.00%),                    ,change          0 (+0.00%)
DCJUMP:      38809 ( 0.13%),                    ,change          0 (+0.00%)
  STAT:      71680 ( 0.24%),          14 tslices,change          0 (+0.00%)
   COM:     362902 ( 1.19%),                    ,change          0 (+0.00%)
 NOISE:     810152 ( 2.66%),         136 bolos  ,change          0 (+0.00%)
Total samples available for map:   18273302, 59.91% of max (3067.53 bolos)
     Change from last report:          0, +0.00% of previous
smf_iteratemap: *** CHISQUARED = 0.952708604771402
smf_iteratemap: *** change: -0.000109049487216351
smf_iteratemap: *** NORMALIZED MAP CHANGE: 0.107427 (mean) 2.96138 (max)
smf_iteratemap: ****** Completed in 5 iterations
smf_iteratemap: ****** Solution CONVERGED
Total samples available from all chunks: 18273302 (3067.53 bolos)
%%%%\end{}
\end{terminalv}



\section{\xlabel{look_for}What to look out for}
\flushbottom

Once the map-maker has completed you can open your output map using
\gaia\ (see \cref{Figure}{fig:itermap}{this example}). The excerpt in
\cref{Section}{sec:running}{Running the iterative map-maker} shows the
output written to the terminal as you run the map-maker. There are a
number of clues in this output that indicate the status of the
reduction.

\starfig{sc21_crl2688}{[t!]}{width=0.7\linewidth}{fig:itermap}{
  CRL~2688 produced with \makemap}{
  Map of CRL~2688 produced with the \smurf\ task \makemap\ using the
  iterative algorithm with default parameters.
}


\begin{description}
\item[The number of input files] The first to note is the number of
  input files; it is worth checking this matches your expected
  number. Also summarised are the source name, UT date and scan
  number.

\item[Map dimension] Next the basic dimensions of the data being
  processed are listed near the start of the first iteration. The
  example above has 4\,arcsec pixels---the default at 850$\mu$m.


\item[Chunking] The map-maker then determines if the raw data should
  be split and processed in more than one chunk. In this map the data
  are reduced in one continuous piece: \param{Continuous chunk 1 /
    1}. Chunking is where the map-maker processes sub-sections of the
  time-series data independently and should be avoided if
  possible---see the text box below.
\end{description}


\subsubsection*{Quality statistics}

At the beginning of the reduction, the main purpose of QUALITY
flagging is to indicate how many bolometers are being used. In the
example above you can see that from a total of 5120 bolometers, 1842
were turned off during data acquisition (\texttt{BADDA}). In addition,
136 bolometers exceeded the acceptable noise threshold
(\texttt{NOISE}), while tiny fractions of the data were flagged
because the telescope was moving too slowly (\texttt{STAT}) or the
sample are adjacent to a step that was removed (\texttt{DCJUMP}).

The total number of bad bolometers (\texttt{BADBOL}) is 1984.
Accounting for these, and the small numbers of additionally flagged
samples, 3128.22 effective bolometers are available after initial
cleaning\footnote{The fractional number is due to time-slices being
removed during cleaning. The number of bolometers is then
reconstructed from the number of remaining time-slices.}.


\begin{sltextbox}{Data Chunking}
  \label{box:chunk}
  Chunking occurs when there is insufficient computer memory available
  for the map-maker or when there is a gap in the time-series data
  (e.g.  from a missing sub-scan). In these cases, the map-maker
  divides up the time-series data and reduces each sub-portion
  independently, before re-combining all the outputs at a later
  stage. Ideally you want your data reduced in a single chunk, however
  this can be unfeasible for large maps.

  The more data the map-maker processes at once, the better chance it
  has of determining the difference between sky signal and background
  noise.

  In \textsc{daisy} mode chunking is less of a concern as the entire
  map area is covered many times in the space of a single observation.

  For \textsc{pong} maps chunking is a bigger concern, with the
  maximum number of chunks that can be tolerated dependent on the
  number of map rotations. For example, a 40-minute \textsc{pong} map
  with eight rotations may get divided into three or four
  chunks. Although not ideal, this will mean that each point is still
  covered by two or three passes. Fewer passes than this however and
  the map-maker become less effective.
\end{sltextbox}

After each subsequent iteration a new `Quality' report is produced,
indicating how the flags have changed. An important flag that appears
in the `Quality' report following the first iteration is \model{COM}:
the DIMM rejects bolometers (or portions of their time series) if they
differ significantly from the common-mode (average) of the remaining
bolometers.

You may note that compared with the initial report, the total number
of samples with good `Quality' (\texttt{Total samples available for
  map}) has dropped from 18634826 to 18273302 (about a 2 per cent
decrease) as additional samples were flagged in each iteration.

Be aware that some large reductions may take many iterations to reach
convergence and you may find significantly fewer bolometers remaining
resulting in higher noise than expected.

\subsubsection*{Convergence}

The convergence criteria \param{maptol} is updated for each
iteration. The convergence can be checked from the line reporting\\*
\hspace*{0.5cm} \texttt{smf\_iteratemap: *** NORMALIZED MAP CHANGE:
  0.10559 (mean) 2.81081 (max)}

The number to look out for is the mean value. This will have to drop
below your required \param{maptol} for convergence to be achieved.

The default configuration file used in this example executes a maximum
of five iterations, but stops sooner if the change in \param{maptol}
drops below 0.05 (i.e. \param{numiter~=$-$5}). In this example it
stops after five iterations.



\begin{tip}
  You can interrupt the processing at any stage with a single
  \texttt{Ctrl-C}. The map-maker will complete the iteration then
  write out a final science map. Entering \texttt{Ctrl-C} twice will
  kill the process immediately.\widowpenalty=100000
\end{tip}

\section{\xlabel{sciencepl}Using the science pipeline}

You can also reduce your data using the \oracdr\ science pipeline on a
local computer. There are advantages to running the map-maker using
the pipeline. You can feed the pipeline observations of multiple
sources rather than feed in a single source at a time. The pipeline
will recognise the different sources and make a separate map for each,
whereas the map-maker would make a single large map that would include
all your sources (no matter how widely spaced!).

Another useful feature is that the pipeline will generate a log files
to record various useful quantities. The standard log files from
reducing science data are:

\begin{itemize}
\item \file{log.noise}---noise in the map for each observation and the co-add
(calculated from the median of the error array), and
\item \file{log.nefd}---NEFD calculated for each observation and for
the co-added map(s).
\end{itemize}
The pipeline will produce calibrated maps; by default these are
calibrated using the standard FCFs, although you can specify your
own if you wish.

Running the science pipeline is very straightforward and can be as
simple as the example below. Here a list is made of all your raw data,
the pipeline is then initiated, finally the reduction is started with
instructions to loop through all data listed in the supplied file and
the filename in question is given.

\begin{terminalv}
% ls s8*.sdf > myfiles.lis
% oracdr_scuba2_850
% oracdr -loop file -files myfiles.lis
\end{terminalv}

\textbf{For a more in-depth discussion on running the pipeline and a
  discussion of the various outputs see \cref{Chapter}{sec:pipe}{The
    SCUBA-2 Pipeline} or \pipelinesun.}



