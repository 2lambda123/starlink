\chapter{\xlabel{fcfsred}FCFs by reduction date}
\label{app:fcfs}

Ongoing development of the SCUBA-2 analysis has resulted in ongoing
changes to the atmospheric opacity relationships and the FCFs. 
Depending on when your data were reduced you will need to apply 
different calibration values. \textsc{orac-dr} will automatically apply
the appropriate values. 

\textbf{Note:} As of the 2021A Starlink release, the
opacity relations and FCF values have been updated following the
results presented by Mairs et al. 2021 \cite{mairs21}. Historical values derived by Dempsey et al. 2013
\cite{dempsey12} are presented below the updated values for comparison
with previously-reduced data.

\vspace{1cm}

\textbf{Explanation of parameters} (see also Appendix \ref{app:cal}):

Starlink currently applies the following multiplicative extinction correction to SCUBA-2 data:

\begin{equation}
\mathrm{Extinction\:Correction} = \frac{1}{\mathrm{exp}[-\tau_{\nu}\times\mathrm{Airmass}]}
\end{equation}
where $\tau_{\nu}$ is the atmospheric opacity at the given frequency, $\nu$. ``Opacity relations''  
relate the measured $\tau_{225}$ to the atmospheric opacity at the operating frequencies of 
SCUBA-2, $\tau_{666}$ (450~$\mu$m) and $\tau_{345}$ (850~$\mu$m). Their form is:
\begin{equation}
\label{eq:2021taurelation}
\tau_{\nu} = \mathrm{a}\times(\tau_{225,\mathrm{zen}} + \mathrm{b} + \mathrm{c}\times\sqrt{\tau_{225,\mathrm{zen}}}),
\end{equation}
where a, b, and c are empirically derived coefficients (see Mairs et al 2021 \ref{mairs21}).
You can find out when your data was reduced and hence what tau
relation was applied by using the \Kappa\ command \hislist.

\begin{terminalv}
%  hislist file | grep EXT.TAURELATION
\end{terminalv}

This will return something like the following,
\begin{terminalv}
      EXT.TAURELATION.450=(23.3,-0.018,0.05),
      EXT.TAURELATION.850=(3.71,-0.040,0.202), EXT.TAUSRC=auto, FAKESCALE=1,
\end{terminalv}
indicating, for example, that at 850~$\mu$m, a=3.7, b=-0.040, and c=0.202.

\vspace{5mm}

There are two commonly used FCF types:
\begin{itemize}
\item Peak FCF (Jy/pW/beam)---multiply your map by this when you wish
to measure absolute peak fluxes of discrete sources.
\item Arcsec FCF (Jy/pW/arcsec$^2$)---multiply your map by this if
you wish to use the calibrated map to do aperture photometry/extended source flux recovery.
\end{itemize}

\textbf{Note:} The FCFs are applied after the extinction correction, so the values are intrinsically
related to one another. The proper extinction correction must be applied before using the FCF
values presented below.

\newpage

\textbf{Results from Mairs et al 2021 \cite{mairs21}. Employed by Starlink versions 2021A and after. Dates are inclusive.}\\
\begin{table}[h!]
\begin{center}
\begin{tabular}{|l|c|c|c|c|}
 \hline
 \multicolumn{1}{|c|}{Date} &
 \multicolumn{2}{c|}{FCF - 450$\mu$m} &
 \multicolumn{2}{c|}{FCF - 850$\mu$m} \\
\cline{2-5}
& Jy/pW/beam &Jy/pW/arcsec$^2$ & Jy/pW/beam &Jy/pW/arcsec$^2$ \\
 \hline
until 30 April 2011 &383  & 4.9 &1080 &5.0 \\
1 May 2011 - 1 Nov 2016 & $531\pm93$ & $4.61\pm0.60$ & $525\pm37$ & $2.25\pm0.13$ \\
1 Nov 2016 - 29 June 2018 & $531\pm93$ & $4.61\pm0.60$ & $516\pm42$ & $2.13\pm0.12$ \\
30 June 2018 Onwards & $472\pm76$ & $3.87\pm0.53$ & $495\pm32$ & $2.07\pm0.12$ 
 \\
\hline
\end{tabular}
\end{center}
\end{table}

From 1 May 2011 onwards, the a, b, and c values presented in the table below correspond to Equation \ref{eq:2021taurelation}:\\
\begin{table}[h!]
\begin{center}
\begin{tabular}{|l|c|c|}
 \hline
 \multicolumn{1}{|c}{Date} & \multicolumn{2}{|c|}{Opacity Relation}  \\ \cline{2-3}
                           & 450$\mu$m  & 850$\mu$m \\ \hline
until 30 April 2011 & 32 $\times$ ($\tau_{225}$ - 0.02)    & 5.2 $\times$ ($\tau_{225}$ - 0.013)  \\
1 May 2011 Onwards & \begin{tabular}[c]{@{}c@{}c@{}c@{}}\\a=23.3$\pm$1.5\\b=-0.018$\pm$0.006\\c=0.05$\pm$0.04\end{tabular}& \begin{tabular}[c]{@{}c@{}c@{}c@{}}\\a=3.71$\pm$0.18\\b=-0.040$\pm$0.008\\c=0.202$\pm$0.044\end{tabular}\\
\hline
\end{tabular}
\end{center}
\end{table}

\rule{1.0\textwidth}{2pt}

\textbf{Results from Dempsey et al. 2013 \cite{dempsey12}: employed by Starlink versions 2018A and previous.}\\
\begin{table}[h!]
\begin{center}
\begin{tabular}{|l|c|c|c|c|}
 \hline
 \multicolumn{1}{|c|}{Date} &
 \multicolumn{2}{c|}{FCF - 450$\mu$m} &
 \multicolumn{2}{c|}{FCF - 850$\mu$m} \\
\cline{2-5}
& Jy/pW/beam &Jy/pW/arcsec$^2$ & Jy/pW/beam &Jy/pW/arcsec$^2$ \\
 \hline
until January 2012 &383  & 4.9&1080 &5.0 \\
January 2012 - July 2012&606&6.06 &556 &2.42 \\
July 2012 onwards&491 &4.71 &537 &2.34 \\
\hline
\end{tabular}
\end{center}
\end{table}
\vspace{-2mm}
\begin{table}[h!]
\begin{center}
\begin{tabular}{|l|c|c|}
 \hline
 \multicolumn{1}{|c}{Date} & \multicolumn{2}{|c|}{Opacity Relation}  \\ \cline{2-3}
                           & 450$\mu$m  & 850$\mu$m \\ \hline
until January 2012       & 32 $\times$ ($\tau_{225}$ - 0.02)    & 5.2 $\times$ ($\tau_{225}$ - 0.013)  \\
January 2012 - July 2012 & 26 $\times$ ($\tau_{225}$- 0.01923)  & 4.6 $\times$ ($\tau_{225}$ - 0.00435)  \\
July 2012 onwards        & 26 $\times$ ($\tau_{225}$ - 0.01196) & 4.6 $\times$ ($\tau_{225}$ - 0.00435)  \\
\hline
\end{tabular}
\end{center}
\end{table}
