\chapter{\xlabel{app_clean}Cleaning the Raw Data}
\label{app:clean}

There are two ways to clean time-series data:

\begin{enumerate}
\item Run \makemap\ and add \setparam{DOCLEAN}{doclean}{1} to your
configuration (see \cref{Section}{sec:reuse}{Re-using previously cleaned
data to speed up map-making}).
\item Run \clean\ to clean the time-series data without making a map.
\end{enumerate}

The rest of this appendix gives more details on using \clean.

\task{sc2clean} can be used to do two basic tasks in one go:
concatenate data (with or without applying a flatfield); and cleaning
(fix up steps and spikes, remove the means, filter, remove common-mode
etc.). It uses the same configuration files as the iterative map-maker
(though ignoring the map-making specific items).

In this first basic example, we just want to clean up some data enough
to see whether the bolometers have been flat-fielded correctly, and
more-or-less exhibit the same behaviour over time. The pre-processing
or cleaning steps used by default (\emph{i.e.} if ``\texttt{config=def}''
is included on the command line) are summarised in
\cref{Section}{cat:preproc}{this table}. Note, whilst it is not
recommended to run \makemap\ in this way (\emph{i.e.} without a
configuration file), it is not so critical when running \clean.

\begin{terminalv}
% sc2clean ^files.lis clean config=def
\end{terminalv}

Here \file{files.lis} can just contain a single file from a sub-array, or a
subset, e.g. \file{s8a20110417\_00051\_0003.sdf} (the first file
containing science data), \file{s8a20110417\_00051\_000"[1234]"} (File
1 is a noise observation with shutter closed that gets ignored, File 2
is a flatfield observation that will be used to override the flatfield
stored in the subsequent Files 3 and 4 which are concatenated
together, the \file{.sdf} is optional),
\file{s8a20110417\_00051\_000\textbackslash?} (Files 1 through 9),
\file{s8a20110417\_00051\_\textbackslash*} (the whole observation).

If you inspect the resulting \file{clean.sdf} in \gaia\
(\cref{Section}{sec:gaiacube}{Displaying time-series data}) and flip
through the data cube you should see all of the bolometers signals go
up and down together with about the same amplitude: the hope is that
for a well-behaved instrument you are mostly seeing sky noise
variations that are seen with roughly the same amplitude by all
bolometers.

Another common feature, if the scans are particularly long and/or fast
(e.g. 1\,degree across), is strong periodic signals that are
correlated with the scan pattern. See
\cref{Section}{sec:scan}{Displaying scan patterns}---in particular you
will want to plot \texttt{az} and \texttt{el} (the absolute azimuth
and elevation), and also \texttt{daz} and \texttt{del} (the azimuth
and elevation offsets from the map centre). This signal is usually
azimuth-correlated due to magnetic-field pickup. It only shows up in
azimuth, because the instrument is on a Nasmyth platform and therefore
does not move in elevation.

Part of the reason the signals look the same is because they have been
flatfielded. You can turn off flatfielding using the \param{noflat}
option to \task{sc2clean}, and you should then see that all of the
detector amplitudes vary.

Another very useful option is to remove the common signal observed by
all of the bolometers. This may be accomplished by

\begin{terminalv}
% sc2clean ^files.lis clean config='"compreprocess=1"'
\end{terminalv}

This \texttt{config} setting causes the default values to be used for all
configuration parameters except \xparam{COMPREPROCESS}{compreprocess},
which is set to 1 (the default is 0).  The residual signal left by this
command will exhibit second-order time-varying correlated signals across
the focal plane.  Usually these are not very large, but in some cases some
very large localized signals have been detected, particularly in the
850\,$\mu$m arrays in early 2011.

Another variation on this is to accentuate the residual low-frequency
noise by low-pass filtering the result. This can again be accomplished
by simply adding a filter command in the \param{config} parameter,
which in this case low-pass filters with a cutoff at 10\,Hz:

\begin{terminalv}
% sc2clean ^files.lis clean config='"compreprocess=1,filt_edgelow=10"'
\end{terminalv}

Finally, in some cases you might just want to fit and remove
polynomial baselines from the bolometers (by default only the mean is
removed). This example will remove a line, but you can increase the
value of \xparam{ORDER}{order} to remove higher-order polynomials

\begin{terminalv}
% sc2clean ^files.lis clean config='"order=1"'
\end{terminalv}

Non-default values for any of the cleaning parameter can be specified like so:
\begin{terminalv}
% sc2clean ^files.lis clean config='"order=1,dcfitbox=30,dcthresh=25,dcsmooth=50"'
\end{terminalv}
Or you can create your own customised configuration file. For instance:

\begin{terminalv}
% cat myconf
order=1
dcfitbox=30
dcthresh=25
dcsmooth=50
% sc2clean ^files.lis clean config=^myconf
\end{terminalv}

The more interesting pre-processing options that may be specified are listed
and described in \cref{Appendix}{app:parameters}{here}.




