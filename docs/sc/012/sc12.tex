\documentstyle[11pt]{article}
\pagestyle{myheadings}

% -----------------------------------------------------------------------------
% ? Document identification
\newcommand{\stardoccategory}  {Starlink Cookbook}
\newcommand{\stardocinitials}  {SC}
\newcommand{\stardocsource}    {sc\stardocnumber}
\newcommand{\stardocnumber}    {12.1}
\newcommand{\stardocauthors}   {Jon Lockley}
\newcommand{\stardocdate}      {28 November 1997}
\newcommand{\stardoctitle}     {Writing your own Data Reduction Software}
\newcommand{\stardocabstract}  {This Cookbook shows you how to write your
own software for astronomical data reduction.
There are two parts.
The first part introduces you to simple data reduction using Starlink's
IMG subroutine library.
This hides a lot of the underlying complications.
The second part introduces you to the more powerful (but more complex)
NDF subroutine library for handling data.
}
% ? End of document identification

% -----------------------------------------------------------------------------

\newcommand{\stardocname}{\stardocinitials /\stardocnumber}
\markright{\stardocname}
\setlength{\textwidth}{160mm}
\setlength{\textheight}{230mm}
\setlength{\topmargin}{-2mm}
\setlength{\oddsidemargin}{0mm}
\setlength{\evensidemargin}{0mm}
\setlength{\parindent}{0mm}
\setlength{\parskip}{\medskipamount}
\setlength{\unitlength}{1mm}

% -----------------------------------------------------------------------------
%  Hypertext definitions.
%  ======================
%  These are used by the LaTeX2HTML translator in conjunction with star2html.

%  Comment.sty: version 2.0, 19 June 1992
%  Selectively in/exclude pieces of text.
%
%  Author
%    Victor Eijkhout                                      <eijkhout@cs.utk.edu>
%    Department of Computer Science
%    University Tennessee at Knoxville
%    104 Ayres Hall
%    Knoxville, TN 37996
%    USA

%  Do not remove the %begin{latexonly} and %end{latexonly} lines (used by 
%  star2html to signify raw TeX that latex2html cannot process).
%begin{latexonly}
\makeatletter
\def\makeinnocent#1{\catcode`#1=12 }
\def\csarg#1#2{\expandafter#1\csname#2\endcsname}

\def\ThrowAwayComment#1{\begingroup
    \def\CurrentComment{#1}%
    \let\do\makeinnocent \dospecials
    \makeinnocent\^^L% and whatever other special cases
    \endlinechar`\^^M \catcode`\^^M=12 \xComment}
{\catcode`\^^M=12 \endlinechar=-1 %
 \gdef\xComment#1^^M{\def\test{#1}
      \csarg\ifx{PlainEnd\CurrentComment Test}\test
          \let\html@next\endgroup
      \else \csarg\ifx{LaLaEnd\CurrentComment Test}\test
            \edef\html@next{\endgroup\noexpand\end{\CurrentComment}}
      \else \let\html@next\xComment
      \fi \fi \html@next}
}
\makeatother

\def\includecomment
 #1{\expandafter\def\csname#1\endcsname{}%
    \expandafter\def\csname end#1\endcsname{}}
\def\excludecomment
 #1{\expandafter\def\csname#1\endcsname{\ThrowAwayComment{#1}}%
    {\escapechar=-1\relax
     \csarg\xdef{PlainEnd#1Test}{\string\\end#1}%
     \csarg\xdef{LaLaEnd#1Test}{\string\\end\string\{#1\string\}}%
    }}

%  Define environments that ignore their contents.
\excludecomment{comment}
\excludecomment{rawhtml}
\excludecomment{htmlonly}

%  Hypertext commands etc. This is a condensed version of the html.sty
%  file supplied with LaTeX2HTML by: Nikos Drakos <nikos@cbl.leeds.ac.uk> &
%  Jelle van Zeijl <jvzeijl@isou17.estec.esa.nl>. The LaTeX2HTML documentation
%  should be consulted about all commands (and the environments defined above)
%  except \xref and \xlabel which are Starlink specific.

\newcommand{\htmladdnormallinkfoot}[2]{#1\footnote{#2}}
\newcommand{\htmladdnormallink}[2]{#1}
\newcommand{\htmladdimg}[1]{}
\newenvironment{latexonly}{}{}
\newcommand{\hyperref}[4]{#2\ref{#4}#3}
\newcommand{\htmlref}[2]{#1}
\newcommand{\htmlimage}[1]{}
\newcommand{\htmladdtonavigation}[1]{}

% Define commands for HTML-only or LaTeX-only text.
\newcommand{\html}[1]{}
\newcommand{\latex}[1]{#1}

% Use latex2html 98.2.
\newcommand{\latexhtml}[2]{#1}

%  Starlink cross-references and labels.
\newcommand{\xref}[3]{#1}
\newcommand{\xlabel}[1]{}

%  LaTeX2HTML symbol.
\newcommand{\latextohtml}{{\bf LaTeX}{2}{\tt{HTML}}}

%  Define command to re-centre underscore for Latex and leave as normal
%  for HTML (severe problems with \_ in tabbing environments and \_\_
%  generally otherwise).
\newcommand{\setunderscore}{\renewcommand{\_}{{\tt\symbol{95}}}}
\latex{\setunderscore}

% -----------------------------------------------------------------------------
%  Debugging.
%  =========
%  Remove % from the following to debug links in the HTML version using Latex.

% \newcommand{\hotlink}[2]{\fbox{\begin{tabular}[t]{@{}c@{}}#1\\\hline{\footnotesize #2}\end{tabular}}}
% \renewcommand{\htmladdnormallinkfoot}[2]{\hotlink{#1}{#2}}
% \renewcommand{\htmladdnormallink}[2]{\hotlink{#1}{#2}}
% \renewcommand{\hyperref}[4]{\hotlink{#1}{\S\ref{#4}}}
% \renewcommand{\htmlref}[2]{\hotlink{#1}{\S\ref{#2}}}
% \renewcommand{\xref}[3]{\hotlink{#1}{#2 -- #3}}
%end{latexonly}
% -----------------------------------------------------------------------------
% ? Document-specific \newcommand or \newenvironment commands.
% ? End of document-specific commands
% -----------------------------------------------------------------------------
%  Title Page.
%  ===========
\renewcommand{\thepage}{\roman{page}}
\begin{document}
\thispagestyle{empty}

%  Latex document header.
%  ======================
\begin{latexonly}
   CCLRC / {\sc Rutherford Appleton Laboratory} \hfill {\bf \stardocname}\\
   {\large Particle Physics \& Astronomy Research Council}\\
   {\large Starlink Project\\}
   {\large \stardoccategory\ \stardocnumber}
   \begin{flushright}
   \stardocauthors\\
   \stardocdate
   \end{flushright}
   \vspace{-4mm}
   \rule{\textwidth}{0.5mm}
   \vspace{5mm}
   \begin{center}
   {\Huge\bf  \stardoctitle \\ [2.5ex]}
   \end{center}
   \vspace{5mm}

% ? Heading for abstract if used.
   \vspace{10mm}
   \begin{center}
      {\Large\bf Abstract}
   \end{center}
% ? End of heading for abstract.
\end{latexonly}

%  HTML documentation header.
%  ==========================
\begin{htmlonly}
   \xlabel{}
   \begin{rawhtml} <H1> \end{rawhtml}
      \stardoctitle
%      \stardocversion\\
%      \stardocmanual
   \begin{rawhtml} </H1> \end{rawhtml}

% ? Add picture here if required.
% ? End of picture

   \begin{rawhtml} <P> <I> \end{rawhtml}
   \stardoccategory\ \stardocnumber \\
   \stardocauthors \\
   \stardocdate
   \begin{rawhtml} </I> </P> <H3> \end{rawhtml}
      \htmladdnormallink{CCLRC}{http://www.cclrc.ac.uk} /
      \htmladdnormallink{Rutherford Appleton Laboratory}
                        {http://www.cclrc.ac.uk/ral} \\
      \htmladdnormallink{Particle Physics \& Astronomy Research Council}
                        {http://www.pparc.ac.uk} \\
   \begin{rawhtml} </H3> <H2> \end{rawhtml}
      \htmladdnormallink{Starlink Project}{http://star-www.rl.ac.uk/}
   \begin{rawhtml} </H2> \end{rawhtml}
   \htmladdnormallink{\htmladdimg{source.gif} Retrieve hardcopy}
      {http://star-www.rl.ac.uk/cgi-bin/hcserver?\stardocsource}\\

%  HTML document table of contents. 
%  ================================
%  Add table of contents header and a navigation button to return to this 
%  point in the document (this should always go before the abstract \section). 
  \label{stardoccontents}
  \begin{rawhtml} 
    <HR>
    <H2>Contents</H2>
  \end{rawhtml}
  \htmladdtonavigation{\htmlref{\htmladdimg{contents_motif.gif}}
        {stardoccontents}}

% ? New section for abstract if used.
  \section{\xlabel{abstract}Abstract}
% ? End of new section for abstract
\end{htmlonly}

% -----------------------------------------------------------------------------
% ? Document Abstract. (if used)
%  ==================
\stardocabstract
% ? End of document abstract
% -----------------------------------------------------------------------------
% ? Latex document Table of Contents (if used).
%  ===========================================
\newpage
\begin{latexonly}
   \setlength{\parskip}{0mm}
   \tableofcontents
   \setlength{\parskip}{\medskipamount}
   \markright{\stardocname}
\end{latexonly}
% ? End of Latex document table of contents
% -----------------------------------------------------------------------------
\newpage
~
\newpage
\renewcommand{\thepage}{\arabic{page}}
\setcounter{page}{1}

\section{\xlabel{introduction}Introduction}

\subsection{What is this book about?}

Put simply, this book is about helping you to write 
your own data reduction software. This is achieved
by showing you how to access and manipulate astronomical
data files. 

\subsection{Why write my own software?}

This book is partly for people who haven't got the tool to do the job. For
example, existing software may not work on a particularly non-standard
data set.  They are in the position of having to write their own data
reduction software to get around the problem. This isn't always an easy
task, especially for the inexperienced programmer.

Alternatively, this book can be used by people who have an idea for a new
data reduction algorithm.  They might even want to pass it on to a larger
audience and want to be sure that it will work on a variety of machines
with a wide variety of data formats. 

Before you go any further:

\begin{quote}
{\fbox {\bf Make absolutely sure you {\em need}\, to write new software.}}
\end{quote}

Use the {\sf findme} command \xref{(SUN/188)}{sun188}{} to see if the job 
has already been done.
Ask your site manager. Ask a Starlink programmer (you'll find their email
addresses on the back of the Starlink Bulletin or on the Starlink home
page). Above all, don't waste {\em your}\, valuable time on writing software
that already exists.
 
\subsection{How this book works}

This cookbook comes in two parts.

The first part leads you through a series of examples of very simple
programs which handle data files. 
It tries to expose you to the absolute minimum
of ``inside trickery'', 
Much non-essential information is sacrificed 
to help you get basic jobs done quickly.

The second part of this book delves a little deeper.  It takes an approach
which involves writing more complex code than that seen in part one, but
it's code that can do more stuff. We'll also start looking at how to write
software {\em packages}\, i.e. a toolbox to keep your tools in.

This book is {\em not}\, for people who want to know the in-depth workings
of data mapping, format conversion and the like (although you will
certainly find the references to the appropriate documents here).  It is
therefore not recommended that you use this guide as a manual.

\subsection{What will I be able to do after reading this?}

Hopefully, this book will show you how to do stuff you didn't know how to do
before. For example, after part one you'll be able to write simple
software which can:

\begin{itemize}
\item Read, manipulate, and create your own NDF (e.g. .sdf) format files.
\item Gain access to other data formats.
\item Read FITS header information.
\end{itemize}

After reading part two of this book, you'll be able to:

\begin{itemize}
\item Access quality and variance information.
\item Access and create extensions.
\item Make your own data reduction packages (monoliths).
\end{itemize}

\newpage

\part{SIMPLE ACCESS TO DATA}

\markright{\stardocname}

This first section of the cookbook follows a tutorial style. Its aim is
to get you creating very simple applications. Because of its tutorial
style, more confident programmers might find it more useful to skip the
explanatory text and go straight to the sample source codes.  

However, I recommend that even the experienced programmer should read the
sections on {\sf interface files} and the {\sf alink} command. 

\newpage

\section{\xlabel{getting_started}Getting started}

\subsection{Introduction}

The programs used in this book, and those you will write yourself in the
same style, make use of the {\it ADAM software environment}.  In a
nutshell, this is a mechanism for bringing  lots of different
facilities and giving the result a common look and feel. 

ADAM does a lot more than make things look pretty, but you'll be glad to
know that there is no need to understand ADAM inside out to use this book.
In fact, we'll hardly discuss ADAM at all. Suffice it to say that ADAM is
there, providing the infrastructure you need to manipulate data files. 

       A program using the ADAM environment is usually referred
       to as an ADAM {\it task}. In practice, there are two types
       of ADAM task. {\it I-tasks} are used to perform jobs such
       as instrumentation control. They will not be referred to again in
       this book. {\it A-tasks} are concerned with data reduction and 
       analysis. It is these we will be concentrating on.

       A-tasks are often also called {\it applications}. This is the
       convention we will use for the rest of this book. After all
       we are in the business of writing software which 
       does {\em stuff}.

       It is possible to group a collection of applications together in a 
       single package known as a {\it monolith}. Starlink packages
       are often made in this way (e.g. Figaro). This helps to keep related 
       applications together. In the above example, one could package
       both the {\sf log} and {\sf alog} applications together 
       into a simple package called {\sf logarithms}. Both {\sf log} and
       {\sf alog} would be used separately in the same way that each
       application in {\sf Figaro} (or {\sf Kappa} etc) is run 
       separately. We will return to monoliths in part two of this book.

\subsection{Interface files}

       Most of the applications you write will require information to
       be given them by you or another user. Each piece of information
       required by the code is called a {\it parameter}.

       Every ADAM application requires an {\it interface file} to provide 
       information about the parameters required to 
       perform the job requested. This file also specifies the
       prompts the user receives when asked to input parameter values,
       and the type of value required.

       The name of the interface file for an application ends with the
       extension {\sf .ifl}. In other words, an application called {\sf prog}
       has a source code file {\sf prog.f} and a corresponding interface file
       {\sf prog.ifl}.

\subsection{Compiling your applications -- The alink command}

ADAM A-tasks are easily compiled with the facility {\sf alink}.
A typical use of {\sf alink} might be:

\begin{verbatim}
   % alink prog.f -L/star/lib `img_link_adam`
\end{verbatim}

This command compiles and links the application source code. It 
also makes use of a Starlink library (in this case the IMG library).
It's not too important that you understand how all this works just 
now. 

\subsection{Section summary}

We've covered quite a bit of terminology in this section, so rather than 
allow confusion to set in, let's review what's essential to know in 
what we've seen so far:

\begin{itemize}
\item You will be writing applications.
\item Applications can be individual or part of larger packages
called monoliths.
\item All applications require a source code file and an interface file.
\item Applications are compiled (and linked) using {\sf alink}.
\end{itemize}

\section{\xlabel{the_basics_of_the_img_library}The basics of the IMG library}

      {\em Note}\, -- The more experienced programmer might find the first
      couple of examples in this section worth skipping.
      In this section, you'll begin to write your own applications. We'll
      start very slowly by looking at a traditional ``do nothing'' code. 

\subsection{Example 1 -- A ``do nothing" code}

      Let's get started on your first application. Enter the following
      code via your choice of text editor. Save it as {\sf nowt.f}.

\begin{quote}
{\small
\begin{verbatim}
      SUBROUTINE NOWT(STATUS)
C This application does nothing. It is an inapplicable application.
      CONTINUE
      END
\end{verbatim}
}
\end{quote}

      You'll need an interface file as well,
      despite the lack of code, as it is required by {\sf alink}.
      Again, enter this with your text editor, but this time save it as {\sf 
      nowt.ifl}

\begin{quote}
{\small
\begin{verbatim} 
interface nowt
endinterface
\end{verbatim}
}
\end{quote}

      Now we can compile and link the code in the following way. From the
      UNIX prompt type:

\begin{verbatim}
   % alink nowt.f
\end{verbatim}

      This should create an executable file called {\sf nowt}. Try running
      {\sf nowt}. If you've done everything correctly, nowt happens. While
      this is a good sign, it's hardly gripping stuff. You could
      always edit the code so it reads:

\begin{quote}
{\small
\begin{verbatim}
      SUBROUTINE NOWT(STATUS)
C This application does nothing. But at least you know it's friendly.
      WRITE (*,*) 'Hello World!'
      CONTINUE
      END
\end{verbatim}
}
\end{quote}

      {\bf Note that you can run {\sf alink} to 
      recompile {\sf nowt.f} without changing the interface file. This is
      because the {\sf nowt} application still doesn't require any
      parameters to work.}

\subsection{Example 2 -- Opening and closing files}

Let's now progress to a code which opens a Starlink (.sdf) 
data file. Enter the following 
code and save it as {\sf sesame.f}

\begin{quote}
{\small
\begin{verbatim}
       SUBROUTINE SESAME(STATUS)
C This application opens a file and promptly closes it again. 
       INTEGER NX, NY, IP, STATUS
       CALL IMG_IN('IN',NX,NY,IP,STATUS)
       CALL IMG_FREE('IN',STATUS)
       END
\end{verbatim}
}
\end{quote}

And of course we'll need the obligatory interface file {\sf sesame.ifl}

\begin{quote}
{\small
\begin{verbatim}
interface SESAME
  parameter IN
    prompt 'Input Image'
  endparameter
endinterface 
\end{verbatim}       
}
\end{quote}

Now, to compile sesame use:

\begin{verbatim}
   % alink sesame.f -L/star/lib `img_link_adam`
\end{verbatim}

If you get error messages, first make sure you have started the Starlink
software on your session.  If you still have trouble, try the command {\sf
star\_dev}. (This sets up a link to the Starlink libraries -- don't worry too
much about how it works). Note the change in the {\sf alink} command from the
previous ones.

Now try running your code. You should get:

\begin{verbatim}
   % sesame
   IN - Input image >
\end{verbatim}

Assuming your code works(!), you can give the name of an NDF file which it
will read and promptly do nothing. This code appears to be another ``do
nothing'' code. It isn't, so let's take a closer look at what's going on.

Firstly, we start the application with the usual

\begin{verbatim}
   SUBROUTINE SESAME(STATUS)
\end{verbatim}

It is a convention when writing ADAM applications to always have 
a subroutine (of the same name as the application) with
an integer argument instead of the more familiar

\begin{verbatim}
   PROGRAM SESAME
\end{verbatim}

statement. Don't worry too much about why this should be the case. 
The next line merely declares some variables. Now 
we get to:

\begin{verbatim}
   CALL IMG_IN('IN',NX,NY,IP,STATUS)
\end{verbatim}

This line opens the file you told it to. It associates the file with a
{\it parameter} called {\sf IN}. In order to get information about the
parameter {\sf IN}, the application looked in the interface file. It was
told that for this parameter it would need to prompt the user into
providing a file name by using a message. 

{\sf IMG\_IN} also found out some information about the image. Its size
is returned via the arguments {\sf NX} and {\sf NY}.  The next variable
({\sf IP}) is a {\it pointer}. We will return to this concept in the next
section.  {\sf STATUS} is used to check all is well. 

The next call,

\begin{verbatim}
   CALL IMG_FREE('IN',STATUS)
\end{verbatim}

simply closes the image associated with the parameter {\sf IN} and tidies
up. Applications which call more than one image must apply this call to
all the images used. 

\subsection{About the IMG library}

In the last section, we used two subroutine calls of the form {\sf CALL
IMG\_} These subroutines are part of the {\it IMG library}. You can find
out all about the facilities offered by IMG by reading
\xref{SUN/160}{sun160}{}. 

IMG is a collection of software designed to make the task of reading 
astronomical data easy. It allows access to both the data itself and the
header information contained within the files. It can be called from both
Fortran and C codes (we will stick to the former) and forms part
of the \xref{NDF library}{sun33}{}. The NDF library is somewhat more
complex (and somewhat more powerful) than the IMG library, so let's save
any more discussion on it until later sections.

When the {\sf alink} command was used in the last section,

\begin{verbatim} 
   % alink sesame.f -L/star/lib `img_link_adam`
\end{verbatim}

you explicitly told alink that you needed to link {\sf sesame.f} to
the IMG library. {\bf You must make absolutely sure that every
time you use the IMG library, you inform alink in this way!} This
also goes for any other Starlink libraries you use (as we will do later 
on in this book).
 
\subsection{Really getting to your data (or data mapping)}

Opening and closing files is fine, but what you really want to be able to
do is get to your data. ADAM libraries have a particular way of handling
arrays of data. Remember how in {\sf sesame.f} you used:

\begin{verbatim}
   CALL IMG_IN('IN',NX,NY,IP,STATUS)
\end{verbatim}

but we didn't discuss the integer variable {\sf IP}? Well, this is
related to data access. Specifically, {\sf IP} acts as a {\it
pointer} to the array of data in the file. Users of the C programming
language will be familiar with the concept of pointers, but Fortran
programmers will generally have little or no experience of them. No problem,
you don't need to know the ins and outs of how they work. Put simply, a
pointer tells a piece of code where some information is kept in the memory
of the machine. In the case of the applications we're going to be looking
at in a minute, we'll use pointers to ``map'' how the data contained in a
file is held in the memory of your machine. 

\subsection{Putting it all together -- Complete applications}

In this section, we'll go through two applications. The first demonstrates
how data in a file is mapped into an array of values. The second covers
the methods of creating new files and updating old files.

\subsection{Example 3 -- Processing data values}

Type in the following code and save it as {\sf stats.f}

\begin{quote}
{\small
\begin{verbatim}
       SUBROUTINE STATS(STATUS)
C This program works out the mean, the minimum and the maximum
C values in the data array of a file.
       INTEGER NX, NY, IP, STATUS, I, J
       REAL MIN, MAX, MEAN, SUM
C       
       CALL IMG_IN('IN',NX,NY,IP,STATUS)
       CALL DOSTAT(%VAL(IP),NX,NY,STATUS)
       CALL IMG_FREE('IN',STATUS)
       END

       SUBROUTINE DOSTAT(IMAGE,NX,NY,STATUS)
C Make sure SAE_PAR is available
       INCLUDE 'SAE_PAR'
       REAL IMAGE(NX,NY), SUM, MIN, MAX
C
       IF (STATUS .NE. SAI__OK) RETURN
C Initialise the variables we'll need.
       SUM = 0.0
       MAX = IMAGE(1,1)
       MIN = IMAGE(1,1)
C Now loop around all the points in the data
       DO I = 1, NX
         DO J = 1, NY
           SUM = SUM + IMAGE (I,J)
           IF (IMAGE(I,J) .GT. MAX) MAX = IMAGE(I,J)
           IF (IMAGE(I,J) .LT. MIN) MIN = IMAGE(I,J)
         END DO
       END DO
C Print the results
       WRITE (*,*) 'The size of the image is ',NX,' by ',NY,' pixels'
       WRITE (*,*) ' '
       WRITE (*,*) 'The mean of the image is ', SUM/REAL(NX*NY) 
       WRITE (*,*) ''
       WRITE (*,*) 'The minimum and maximum are ',MIN,' and ',MAX
       END
\end{verbatim}
}
\end{quote}

Have a go at writing your own interface file for {\sf stats} and
compile it with {\sf alink}. Finally, try running it on an
NDF (i.e. .sdf) file of your choice.

Let's have a careful look at how this code is constructed. The main body
of the code consists of opening an image, calling a subroutine which does the
actual work, followed by a call to close the image and clean up. You'll
find this structure in a lot of Starlink software. It helps to keep the 
code uncluttered and is easy to maintain, so we recommend that
you try to stick to it.

The call to the {\sf DOSTAT} subroutine shows how pointers are used to
access the data in the file. The {\sf {\%VAL}} statement is a VAX
extension to Fortran.  Normally, when a Fortran code uses arrays, it needs
to have some information about the size of the array right from the outset.
There is no method of doing memory allocation ``on the fly''. (This is not
true of Fortran~90, but let's restrict ourselves to using Fortran~77 for
this guide). This just isn't practical for applications which have to deal
with many different sizes of images. 
 
The {\sf \%VAL} method makes the pointer (in this case {\sf IP}) look
just like an array. What's more, it makes sure that the array is of the
right size, in this case {\sf NX} by {\sf NY} pixels. This is not a
feature of ``standard'' Fortran~77, but unless you hear otherwise, it
should be OK on your machine. 

You'll see that when the subroutine \latex{\sf DOSTAT} inherits the
{\sf \%VAL}, it does so with a ready-made array of real values. This way,
there is no need to pre-define your array sizes. 

\subsection{Output from applications}

When you want to write out a data file you want to do one of two things. 
Either you want to write out the same data file you read in, presumably
with some modification, or you want to create an entirely new file.
The IMG library quite happily lets you do both. The next two codes
describe each of these processes in turn.

\subsection{Example 4 -- Updating a file}

Enter the code for {\sf clip.f}:

\begin{quote}
{\small
\begin{verbatim}
       SUBROUTINE CLIP(STATUS)
C This application sets all the data values above and
C below two thresholds to zero.
       CALL IMG_IN(IMAGE,NX,NY,IP,ISTAT)
       WRITE (*,*) 'Enter the maximum data value permitted >'
       READ (*,*) MAX
       WRITE (*,*) 'Enter the minimum data value permitted >'
       READ (*,*) MIN
C 
C Make sure we haven't done anything really dumb
C
       IF (MAX .LE. MIN) THEN
C
C Oh dear...
C
         WRITE (*,*) 'Doh!!!!'
         RETURN
       END IF
C
C Open the image, but do so in such a way that it can be
C modified rather than kept the same.
C
       CALL IMG_MOD('IN',NX,NY,IP,STATUS)
C
C Now clip off the values which are higher and lower than
C the requested values.
C
       CALL CLIPIT(%VAL(IP),NX,NY,MAX,MIN,NCLIP,STATUS)
C
C Let the user know how many pixels were reset. 
C
       WRITE (*,*) NCLIP,' values were reset.'
C
C Tidy up. 
C
       CALL IMG_FREE('IN',STATUS)
       END
    
       SUBROUTINE CLIPIT(IMAGE,NX,NY,MIN,MAX,NCLIP,STATUS)
C
       REAL IMAGE (NX,NY)
       REAL MIN, MAX
       INTEGER NX, NY, STATUS, NCLIP
C
C Make sure SAE_PAR is available
C
       INCLUDE 'SAE_PAR'
C
C Is everything OK?
C
       IF (STATUS .NE. SAI__OK) RETURN
C
C Get clipping!
C
       NCLIP = 0
       DO I = 1, NX
         DO J = 1, NY
           IF (IMAGE(I,J).LT.MIN .OR. IMAGE(I,J).GT.MAX) THEN
           IMAGE (I,J) = 0.0
           NCLIP = NCLIP + 1
         END DO
       RETURN
       END
\end{verbatim}
}
\end{quote}

and the corresponding interface file {\sf clip.ifl}:

\begin{quote}
{\small
\begin{verbatim}
interface CLIP
  parameter IN
    prompt 'Image to clip'
  endparameter
endinterface
\end{verbatim}
}
\end{quote}

and compile it using {\sf alink}, remembering to link to the IMG library.

You can check that your code has worked by using the {\sf stats} application
from the last section. You should see how the reported maximum and 
minimum data values have changed if you chose your clipping limits 
appropriately.

This code {\em modifies}\, your input file, i.e. no new file is created. This
is often not the most useful form of output as, especially in data
reduction, you might want to repeat a step with the original data many
times. In such a case, it is more useful to preserve the input data and 
create a new file from scratch. This next code demonstrates how this is 
done using the {\sf IMG} library.

\subsection{Example 5 -- Creating a new file}

Enter the following code for {\sf bigger.f}.

\begin{quote}
{\small
\begin{verbatim}
      SUBROUTINE BIGGER(STATUS)
C
C This program goes through two images pixel by pixel. It writes
C the largest value of two pixels in the same parts of the two 
C images to a third image. If the first two images are not the
C same size then the program quits.
C
C Read in the two input images.
C
      CALL IMG_IN('IN1,IN2',NX1,NY1,IP1,STATUS)
C
C Make an output image, modelled on the first input image
C
      CALL IMG_OUT('IN1','OUT',IPOUT,ISTAT)
C
C Do the comparison
C
      CALL BIG(%VAL(IP1),%VAL(IP2),%VAL(IPOUT),NX,NY,STATUS)
C
C Tidy up and close
C
      CALL IMG_FREE('*',STATUS)
      END

      SUBROUTINE BIG(IMAGE1,IMAGE2,IMAGE3,NX,NY,STATUS)
C
      REAL IMAGE1(NX,NY)
      REAL IMAGE2(NX,NY)
      REAL IMAGE3(NX,NY)
      INTEGER NX, NY, STATUS, I, J
C
      DO I = 1, NX
        DO J = 1, NY
          IF (IMAGE1(I,J) .GT. IMAGE2(I,J)) THEN
            IMAGE3(I,J) = IMAGE1(I,J)
          ELSE
            IMAGE3(I,J) = IMAGE2(I,J)
          ENDIF
        END DO
      END DO
      RETURN
      END
\end{verbatim}
}
\end{quote}

Notice how the code reads two distinct images in with a single statement.

The interface file, {\sf bigger.ifl}, looks a little different to that 
which you have already seen:

\begin{quote}
{\small
\begin{verbatim}
interface BIGGER

  parameter IN1
    prompt 'Enter first file name'
  end parameter

  parameter IN2
    prompt 'Enter second file name'
  end parameter

  parameter OUT
    prompt 'Enter output file name'
  end parameter

endinterface
\end{verbatim}
}
\end{quote}

\section{\xlabel{accessing_header_information}Accessing header information}

It is often very useful for an application to be able to access the header
information within an NDF, e.g. to get the airmass, time, name of the
source etc. The IMG library provides a number of simple routines which
allow you to gain access to the header. The following example shows how an
integer header item called {\sf ALT\_OBS} is read from the header. 

\subsection{Example 6 -- Reading a header item}

Code:

 \begin{quote}
{\small
\begin{verbatim}
        SUBROUTINE HEADER(STATUS)
C This simple code reads a header item.
        INCLUDE 'SAE_PAR'
        INTEGER STATUS     ! Global Status
        INTEGER NX         ! Number of X pixels
        INTEGER NY         ! Number of Y pixels
        INTEGER IP         ! Image pointer
        INTEGER ALTOBS     ! Item to get 
C
        CALL IMG_IN('IN',NX,NY,IP,STATUS)
        CALL HDR_INI('IN','FITS','ALT_OBS',1,ALTOBS,STATUS)
        IF (STATUS .NE. SAI__OK) RETURN
        WRITE (*,*) ALTOBS
        CALL IMG_FREE('IN',ISTAT)
        END
\end{verbatim}
}
\end{quote}

Interface file:
  
\begin{quote}
{\small
\begin{verbatim}
interface HEADER
  parameter IN
  prompt 'Input NDF'
  endparameter
endinterface
\end{verbatim}
}
\end{quote}

Note that this assumes that there really is an item called {\sf ALT\_OBS}
in the header. To check this (and to find if you need to substitute
something else in as a test) use the {\sf fitskeys} command in {\sf
Figaro} to find out which items exist within the FITS header of your test
files. 

The above example read in an integer value.  {\sf HDR\_INR}, {\sf
HDR\_INC}, {\sf HDR\_IND} and {\sf HDR\_INL} read in real, character,
double precision and logical header items respectively. 

Other {\sf HDR}routines include:

\begin{itemize}
\item {\sf HDR\_DELET}\ {\em Delete a header item}.
\item {\sf HDR\_MOD} {\em Read and/or modify a header item}.
\item {\sf HDR\_NAME} {\em Return the name of a header item}.
\item {\sf HDR\_NUMB} {\em Return number of items in header or
number of occurrences of a named item}.
\item {\sf HDR\_OUT[I,R,C,D,L]} {\em Write a header item}.
\end{itemize}

A full description of how to use these items can be found in the IMG manual
\xref{(SUN/160)}{sun160}{}.

\section{\xlabel{accessing_different_file_formats}Accessing different file formats}

\subsection{The Convert package}

One major difficulty encountered by all astronomers at one time or another is
the number of different formats data gets stored in. So far you have been 
using NDF files. It would be very useful to have code which reads not 
only NDF files, but IRAF format files, GIFS, TIFFS, ASCII, FITS etc.

Starlink has a conversion package (called Convert surprisingly) which 
allows you to move either directly or indirectly from one format to another.
See \xref{SUN/55}{sun55}{} for more information on the Convert 
package.

\subsection{Getting your code to read different file formats}

Wouldn't it be great if you could write a piece of code like Convert 
which could handle all these different formats? Guess what -- you already 
have!

{\bf By using the IMG library, you have unknowingly given yourself the 
ability to access a whole range of other data formats.}

\noindent
All you need to do now to get to these other formats is to type

\begin{verbatim}
   % convert
\end{verbatim}

and then you can immediately apply all the codes you've written with this
book to the following format of files:

\begin{itemize}
\item ASCII
\item DST
\item FITS
\item GASP
\item GIF
\item IRAF
\item NDF
\item TIFF
\end{itemize}

Try one of your codes on a file in a different format (after starting
Convert of course). If you don't have any other format files, pick up some
random junk from the World Wide Web. Alternatively, use Convert to change
the format of a test NDF file into, for example, a GIF format file. 

If you have your own special format of data, \xref{SUN/55}{sun55}{} 
shows you how to add it to those which Convert can handle.

\section{\xlabel{summary_of_part_one}Summary of part one}

We've now finished part one of this book. You are now in a position 
to do the following with a whole range of data formats:

\begin{itemize}
\item Open a data file.
\item Manipulate the data within the file.
\item Modify existing data files.
\item Create new files.
\item Read FITS header items.
\end{itemize}

This might be sufficient for your needs. On the other hand, you will
probably be aware that there is a lot more information contained within an
NDF than ``just'' data. For example, we have not dealt with access to 
statistical error and quality information stored within the file. 

The IMG library is part of a much larger collection of data access routines:
the NDF library. In part two of this book, we switch to the latter method
of data access so that we can deal with the above issues. If you already have
enough information to complete your job, there is no strict requirement 
to go further -- go get your stuff done. 

\newpage

\part{ADVANCED ACCESS TO DATA}

\markright{\stardocname}

In the second part of this book, we switch from the IMG library to the
more advanced NDF data access library. We explore how the NDF library
is able to access many more articles of information which can be
stored within an NDF data file, such as error information and 
objects known as {\em extensions}.

Later in this half of the book, a list of helpful resources for
programmers is provided, along with some helpful hints for 
writing applications.

\newpage

\section{\xlabel{using_the_ndf_library_calls}Using the NDF library calls}

To illustrate the NDF library calls, let's change the code {\sf clip.f} so
that the {\sf IMG} calls are replaced by {\sf NDF} calls. Enter the
following code and save it as {\sf clip2.f}. 

\subsection{Example 7 -- Using NDF library calls to update an NDF}

Code:

\begin{quote}
{\small
\begin{verbatim}
      SUBROUTINE CLIP2(STATUS)
C
      IMPLICIT NONE
      INCLUDE 'SAE_PAR'
      INTEGER STATUS, NDF1, NELM, PTR1
      REAL MIN, MAX
C
C Enable the NDF calls
C
      CALL NDF_BEGIN
C
C Associate the input NDF with some convenient label.
C In this case, let's call it NDF1. We're also going to 
C access the file in such a way that it is UPDATEd.
C
      CALL NDF_ASSOC('INPUT','UPDATE',NDF1,STATUS)
C
C Map the NDF data array
C
      CALL NDF_MAP(NDF1,'Data','_REAL','UPDATE',PTR1,NELM,STATUS)
C
C Now get the threshold values
C
      WRITE (*,*) 'Enter min and max values >'
      READ (*,*), MIN, MAX
C
C Do the clipping.
C
      CALL CLIPIT(NELM,%VAL(PTR1),MIN,MAX,STATUS)
C
C Close the NDF 
C
      CALL NDF_END(STATUS)
      END

      SUBROUTINE CLIPIT(NELM,VALUE,MIN,MAX,STATUS)
C
      IMPLICIT NONE
      INCLUDE 'SAE_PAR'
      INTEGER NELM, STATUS, COUNTER, NCHANGE
      REAL VALUE(NELM), MIN, MAX
C
C Check everything is OK
C
      IF (STATUS .NE. SAI__OK) RETURN
      DO COUNTER = 1, NELM
        IF (VALUE(COUNTER).GT.MAX .OR. VALUE(COUNTER).LT.MIN) THEN
        VALUE (COUNTER) = 0.0
        NCHANGE = NCHANGE + 1
        ENDIF
      END DO
      WRITE (*,*) NCHANGE, ' points were changed.'
      END
\end{verbatim}
}
\end{quote}

Interface file {\sf clip2.ifl}:

\begin{quote}
{\small
\begin{verbatim}
interface clip2
   parameter INPUT
      prompt 'Input NDF'
   endparameter
endinterface
\end{verbatim}
}
\end{quote}

and compile it using 

\begin{quote}
{\small
\begin{verbatim}
   % alink clip2.f -L/star/lib `ndf_link_adam`
\end{verbatim}
}
\end{quote}

noting the change in the library you're linking your code to. Try running
the application and compare its performance with the one that used the IMG 
library. You shouldn't notice any difference. 

{\bf Note that you can still access GIF, IRAF, ASCII etc formats just as
you could with the IMG library. All you have to remember is to start
``Convert'' first. The NDF library is more versatile in terms of available
file formats than its name would suggest.}

You'll probably notice that a lot of the methodology behind the
NDF library is similar to that which we used for the IMG library in part one.
We still label NDFs. We still map them with pointers. One change you 
might have noticed is that when we mapped the NDF, we explicitly stated
it was the {\em `Data'}\, structure we wanted to map. As we'll see
later, we are not limited to this.

\subsection{Creating a new NDF}

In the last example, an NDF was made by updating an old one. In this
section, there are three examples of creating a new NDF. The third
example is the most complete, so the reader who is in a hurry could
skip the first two examples.

\subsection{Example 8a -- Creating a new NDF (the bare essentials)}

Code:

\begin{quote}
{\small
\begin{verbatim}
       SUBROUTINE CREATE(STATUS)
*
* This code just makes an empty, square NDF 
*
* Parameters
* 
* LBND = _INTEGER (Read)
*  Lower Bounds
*
* UBND =  _INTEGER (Read)
*  Upper Bounds
*
* OUT = NDF (Read)
*  Output NDF
*
       IMPLICIT NONE     ! No implicit typing
       INCLUDE 'SAE_PAR' ! SAE constants
       INTEGER STATUS    ! Global status
       INTEGER NDIM      ! Number of dimensions of NDF
       INTEGER LBND      ! Lower bound
       INTEGER UBND      ! Upper bound
       INTEGER INDF      ! NDF Identifier
       PARAMETER(NDIM=2) ! Set Number of Dimensions = 2
*
* Start NDF context
*
       CALL NDF_BEGIN
*
* Check status
*
       IF (STATUS .NE. SAI__OK) RETURN
*
* Read in the bounds of the NDF
* 
       CALL PAR_GET0I('LBND',LBND,STATUS)       
       CALL PAR_GET0I('UBND',UBND,STATUS)       
*
* Create the new NDF
*
       CALL NDF_CREAT('OUT','_REAL',NDIM,LBND,UBND,INDF,STATUS)
*
* Finish up
*
       CALL NDF_END(STATUS)
       END
\end{verbatim}
}
\end{quote}

Interface file:

\begin{quote}
{\small
\begin{verbatim}
interface CREATE

  parameter LBND
    position 1
    type _INTEGER
    prompt 'Lower bound'
  endparameter

  parameter UBND
    position 2
    type _INTEGER
    prompt 'Upper bound'
  endparameter

  parameter OUT
    position 3
    prompt 'Output NDF'
  endparameter

endinterface
\end{verbatim}
}
\end{quote}

Running this code gives:

\begin{quote}
{\small
\begin{verbatim}
% create
LBND - Lower bound > 0 
UBND - Upper bound > 100
OUT - Output NDF > jon
!! The NDF structure /home/TEST/jon has been
!     released from the NDF_ system with its data component in an undefined
!     state (possible programming error).
!  NDF_END: Error ending an NDF context.
!  Application exit status NDF__DUDEF, data component undefined    
\end{verbatim}
}
\end{quote}

This rather ghastly error merely means that the file is empty. Using {\sf
hdstrace} to list the contents of the NDF gives: 

\begin{quote}
{\small
\begin{verbatim}
JON  <NDF>
 
   DATA_ARRAY     <ARRAY>         {structure}
      DATA(101,1)    <_REAL>         {undefined}
      ORIGIN(2)      <_INTEGER>      0,0
 
End of Trace.
\end{verbatim}
}
\end{quote}

Now let's see how we can begin to put information into the NDF. 

\subsection{Example 8b -- Putting a title into the NDF}

Code:

\begin{quote}
{\small
\begin{verbatim}
       SUBROUTINE CREATE2(STATUS)
*
* This code just makes a square NDF with a Title Component 
*
* Parameters   
*
* LBND = _INTEGER (Read)
*  Lower Bounds
*
* UBND =  _INTEGER (Read)
*  Upper Bounds
*      
* OUT = NDF (Read)
*  Output NDF
*
* TITLE = LITERAL (Read)
*  The NDF title
*
       IMPLICIT NONE     ! No implicit typing
       INCLUDE 'SAE_PAR' ! SAE constants
       INTEGER STATUS    ! Global status
       INTEGER NDIM      ! Number of dimensions of NDF
       INTEGER LBND      ! Lower bound
       INTEGER UBND      ! Upper bound
       INTEGER INDF      ! NDF Identifier
       PARAMETER(NDIM=2) ! Set Number of Dimensions = 2
*
* Start NDF context
*
       CALL NDF_BEGIN
*
* Check status
*
       IF (STATUS .NE. SAI__OK) RETURN
*
* Read in the bounds of the NDF
* 
       CALL PAR_GET0I('LBND',LBND,STATUS)       
       CALL PAR_GET0I('UBND',UBND,STATUS)       
*
* Create the new NDF
*
       CALL NDF_CREAT('OUT','_REAL',NDIM,LBND,UBND,INDF,STATUS)
*
* Reset the old title if there is one
*
       CALL NDF_RESET(INDF,'Title',STATUS)
*
* Get the new title
*
       CALL NDF_CINP('TITLE',INDF,'Title',STATUS)
*
* Tidy up
*
       CALL NDF_END(STATUS)
       END
\end{verbatim}
}
\end{quote}

Interface file:

\begin{quote}
{\small
\begin{verbatim}
interface CREATE2

  parameter LBND
    position 1
    type _INTEGER
    prompt 'Lower bound'
  endparameter

  parameter UBND
    position 2
    type _INTEGER
    prompt 'Upper bound'
  endparameter

  parameter OUT
    position 3
    prompt 'Output NDF'
  endparameter

  parameter TITLE
    position 4
    type 'Literal'
    prompt 'Title'
  endparameter

endinterface
\end{verbatim}
}
\end{quote}
    
Again, on running this code, a warning is reported. However, 
{\sf hdstrace} reports:

\begin{quote}
{\small
\begin{verbatim}
JON2  <NDF>
 
   DATA_ARRAY     <ARRAY>         {structure}
      DATA(101,1)    <_REAL>         {undefined}
      ORIGIN(2)      <_INTEGER>      0,0
 
   TITLE          <_CHAR*4>       'test'
 
End of Trace.
\end{verbatim}
}
\end{quote}

\subsection{Example 8c -- Creating a data and variance array in an NDF}

In this example, we put both a data array and variance array (we'll 
mention these in more more detail in the next section) into the file. 
This is done purely by mapping them:

\begin{quote}
{\small
\begin{verbatim}
       SUBROUTINE CREATE3(STATUS)
*
* This code just makes a square NDF with a Title Component 
*
* Parameters   
*
* LBND = _INTEGER (Read)
*  Lower Bounds
*
* UBND =  _INTEGER (Read)
*  Upper Bounds
*      
* OUT = NDF (Read)
*  Output NDF
*
* TITLE = LITERAL (Read)
*  The NDF title
*
       IMPLICIT NONE     ! No implicit typing
       INCLUDE 'SAE_PAR' ! Global SAE constants
       INTEGER STATUS    ! Global status
       INTEGER NDIM      ! Number of dimensions of NDF
       INTEGER LBND      ! Lower bound
       INTEGER UBND      ! Upper bound
       INTEGER INDF      ! NDF identifier
       INTEGER IPNTR     ! Data pointer
       INTEGER VPNTR     ! Variance pointer
       INTEGER NPIX      ! Number of pixels
       PARAMETER(NDIM=2) ! Set Number of Dimensions = 2
*
* Start NDF context
*
       CALL NDF_BEGIN
*
* Check status
*
       IF (STATUS .NE. SAI__OK) RETURN
*
* Read in the bounds of the NDF
*
       CALL PAR_GET0I('LBND',LBND,STATUS)       
       CALL PAR_GET0I('UBND',UBND,STATUS)       
*
* Create the new NDF
*
       CALL NDF_CREAT('OUT','_REAL',NDIM,LBND,UBND,INDF,STATUS)
*
* Reset the old title if there is one
* 
       CALL NDF_RESET(INDF,'Title',STATUS)
*
* Get the new title
*
       CALL NDF_CINP('TITLE',INDF,'Title',STATUS)
* 
* Work out the number of pixels
*
       NPIX = (ABS(UBND-LBND))**2
*
* Map data and variance arrays to force them into existence
*
       CALL NDF_MAP(INDF,'Data','_REAL','WRITE',IPNTR,NPIX,STATUS)
       CALL NDF_MAP(INDF,'Variance','_REAL','WRITE',VPNTR,NPIX,STATUS)
*
* Tidy up
*
       CALL NDF_END(STATUS)
       END
\end{verbatim}
}
\end{quote}

Interface file:

\begin{quote}
{\small
\begin{verbatim}
interface CREATE3

  parameter LBND
    position 1
    type _INTEGER
    prompt 'Lower bound'
  endparameter

  parameter UBND
    position 2
    type _INTEGER
    prompt 'Upper bound'
  endparameter

  parameter OUT
    position 3
    prompt 'Output NDF'
  endparameter

  parameter TITLE
    position 4
    type 'Literal'
    prompt 'Title'
  endparameter

endinterface
\end{verbatim}
}
\end{quote}

    
Obviously an additional subroutine is needed to put useful values into 
the arrays.

\section{\xlabel{variances}Variances, bad pixels, and quality -- The art of error propagation}

NDFs contain a lot more than just the data values. They contain so called
{\em headers}, along with values such as Quality and Variance associated with
each pixel. For some NDFs there is also axes information. 
 
The easiest way to take a look at what structures exist within an
NDF is to use {\sf hdstrace}. 
A sample trace might produce an output such as:

\begin{quote}
{\small
\begin{verbatim}
F2  <NDF>
 
   DATA_ARRAY     <ARRAY>         {structure}
      DATA(1021,200)  <_REAL>        41.46541,69.59,51.00168,48.32741,
                                     ... 169.1622,151.1605,150.1822,166.3774
      ORIGIN(2)      <_INTEGER>      1,1
      BAD_PIXEL      <_LOGICAL>      FALSE
 
   TITLE          <_CHAR*8>       'FEIGE 34'
   UNITS          <_CHAR*9>       'ELECTRONS'
   VARIANCE       <ARRAY>         {structure}
      DATA(1021,200)  <_REAL>        521.5062,571.9399,534.9544,532.3478,
                                     ... 587.9011,556.7531,627.0024,590.9243
      ORIGIN(2)      <_INTEGER>      1,1
 
   MORE           <EXT>           {structure}
      FITS(174)      <_CHAR*80>      'SIMPLE  =                    T','BI...'
                                     ... '15TELE  =       
...','PACKEND','END'
      CCDPACK        <CCDPACK_EXT>   {structure}
         DEBIAS         <_CHAR*24>      'Tue Apr 29 21:11:30 1997'End of 
Trace.
\end{verbatim}
}
\end{quote}

We can see from this that there are arrays for both the data and the
variances. There is also some FITS information, and the file has at some
point been processed by CCDPACK. Let's write a new code that processes the
variances in a similar way to how clip processed the data array. 

\subsection{Processing a variance array}
 
Let's illustrate the way in which the NDF library can handle variance 
information by developing a simple application. In the following example, 
an image is searched for ``spikes'' which are potential cosmic ray events.

When spikes are detected, the pixels are set to what is known as a
{\em BAD}\, value. BAD pixels arise in situations where the values of the pixels
are either unknown, useless, or meaningless. In this simple routine, we
assign BAD values to pixels whose information is corrupted by cosmic rays.

In this example, any variance array present in the NDF is left untouched.
This means that the variance array {\em is no longer valid}. It is,
therefore, not passed on to the output image. Guidelines for when
components of an NDF should and should not be propagated are listed in
\xref{SUN/33}{sun33}{}. We will deal with a case where the variance
information {\em is}\, propagated later in this section. 

\subsection{Example 9 -- A code which does not propagate the variance array}

Code:

\begin{quote}
{\small
\begin{verbatim}
      SUBROUTINE ZAPPER(STATUS)
*
* This routine zaps all pixels whose value goes above a certain
* threshold defined by the user. A new NDF is propagated from the
* old one. In this example the variance array (if present) is
* not modified.
*
* Parameters
*
* IN = NDF (Read)
*  Input NDF
*
* OUT = NDF (Write)
*  Output NDF
*
* THRESH = _REAL (Read)
*  Threshold for zapping
*
      IMPLICIT NONE     ! No implicit typing
      INCLUDE 'SAE_PAR' ! SAE constants
      INTEGER STATUS    ! Global Status
*
* Local Variables
*
      INTEGER DIM(2)    ! Dimensions of image
      INTEGER NPIX      ! Number of pixels
      INTEGER INDF      ! Input NDF identifier
      INTEGER ONDF      ! Output NDF identifier
      INTEGER IPNTR     ! Input NDF pointer 
      INTEGER OPNTR     ! Output NDF pointer
      INTEGER NDIM      ! Number of dimensions
      REAL THRESH       ! Threshold value
*
* Check the "inherited status" before starting the code proper
*
      IF (STATUS .NE. SAI__OK) RETURN
*
* Obtain the input NDF. Find out how big it is in both dimensions
*
      CALL NDF_ASSOC('IN','READ',INDF,STATUS)
      CALL NDF_DIM(INDF,2,DIM,NDIM,STATUS)
*
* Read in the threshold value
*
      CALL PAR_GETOR('THRESH',THRESH,STATUS)
*
* Create a new output NDF. Model it on the old one. Propagate and 
* DATA information. The variances will NOT be propagated.
*

      CALL NDF_PROP(INDF,'Data,NoVariance','OUT',ONDF,STATUS)
*
* Map the data arrays in both the input and the output files
*
      CALL NDF_MAP(INDF,'Data','_REAL','READ',IPNTR,NPIX,STATUS)
      CALL NDF_MAP(ONDF,'Data','_REAL','WRITE',OPNTR,NPIX,STATUS)
*
* Call the main working subroutine. Write new values to output
*
      CALL ZAP(THRESH,NPIX,%VAL(IPNTR),%VAL(OPNTR),STATUS)
*
* Close down
*
      CALL NDF_END(STATUS)
      END

      SUBROUTINE ZAP(THRESH,NPIX,IMAGE,OUT,STATUS)
*
* Zap pixels with more counts than THRESH by assigning BAD values to them
*
      IMPLICIT NONE       ! No implicit typing
      INCLUDE 'SAE_PAR'   ! Standard SAE constants
      INCLUDE 'PRM_PAR'   ! Define BAD constants
* 
* (>) - Given,  (<) - Output
*
      REAL THRESH         ! (>) Threshold value
      INTEGER NPIX        ! (>) Number of pixels in images
      REAL IMAGE(NPIX)    ! (>) Array of input pixel values
      REAL OUT(NPIX)      ! (<) Array of output pixel values
      INTEGER STATUS      ! (<>) Global Status
*
* Local variables
*
      INTEGER N
*
* Go through image and find excess values
*
      DO 1, N = 1, NPIX
        IF (IMAGE(N).NE.VAL__BADR .AND. IMAGE(N).GT.THRESH) THEN
          OUT(N) = VAL__BADR
        ELSE
          OUT(N) = IMAGE(N)
        ENDIF
 1    CONTINUE
      END
\end{verbatim}
}
\end{quote}

Interface file:

\begin{quote}
{\small
\begin{verbatim}
interface ZAPPER

 parameter IN
  position 1
  prompt 'Input file'
 endparameter

 parameter OUT
  position 2
  prompt 'Output file'
 endparameter

 parameter THRESH
  position 3
  type _REAL
  prompt 'Threshold value'
 end parameter
 
endinterface
\end{verbatim}
}
\end{quote}

Note how the interface file now includes a parameter which is a {\em REAL}\,
number as well as filenames.

The \latex{\sf hdstrace} application
reveals that if the input NDF contained a variance array, it does not
exist in the output NDF. This is because the 
{\sf NDF\_PROP} call did
not explicitly list it. In order to get the variance array propagated
through to the output, let's adapt the code and call it {\sf zapper2}. 

\subsection{Example 10 -- A code which DOES propagate the variance array}

This code propagates the variance array. Note you will need to copy the
{\sf zapper.ifl} file to a {\sf zapper2.ifl} file.

\begin{quote}
{\small
\begin{verbatim}
      SUBROUTINE ZAPPER2(STATUS)
*
* This routine zaps all pixels whose value goes above a certain
* threshold defined by the user. A new NDF is propagated from the
* old one. 
*
* Parameters
*
* IN = NDF (Read)
*  Input NDF
*
* OUT = NDF (Write)
*  Output NDF
*
* THRESH = _REAL (Read)
*  Threshold for zapping
*
      IMPLICIT NONE     ! No implicit typing
      INCLUDE 'SAE_PAR' ! SAE constants
      INTEGER STATUS    ! Global Status
*
* Local Variables
*
      INTEGER DIM(2)    ! Dimensions of image
      INTEGER NPIX      ! Number of pixels
      INTEGER INDF      ! Input NDF identifier
      INTEGER ONDF      ! Output NDF identifier
      INTEGER IPNTR     ! Input NDF data pointer 
      INTEGER OPNTR     ! Output NDF data pointer
      INTEGER VIPNTR    ! Input NDF variance pointer 
      INTEGER VOPNTR    ! Output NDF variance pointer
      INTEGER NDIM      ! Number of dimensions
      REAL THRESH       ! Threshold value
*
* Check the "inherited status" before starting the code proper
*
      IF (STATUS .NE. SAI__OK) RETURN
*
* Begin the NDF context
* 
      CALL NDF_BEGIN
*
* Obtain the input NDF. Find out how big it is in both dimensions
*
      CALL NDF_ASSOC('IN','READ',INDF,STATUS)
      CALL NDF_DIM(INDF,2,DIM,NDIM,STATUS)
*
* Read in the threshold value
*
      CALL PAR_GET0R('THRESH',THRESH,STATUS)
*
* Create a new output NDF. Model it on the old one. Propagate  
* DATA and VARIANCES.
*
      CALL NDF_PROP(INDF,'Data,Variance','OUT',ONDF,STATUS)
*
* Map the data arrays in both the input and the output files
*
      CALL NDF_MAP(INDF,'Data','_REAL','READ',IPNTR,NPIX,STATUS)
      CALL NDF_MAP(INDF,'Variance','_REAL','READ',VIPNTR,NPIX,STATUS)
      CALL NDF_MAP(ONDF,'Data','_REAL','WRITE',OPNTR,NPIX,STATUS)
      CALL NDF_MAP(ONDF,'Variance','_REAL','WRITE',VOPNTR,NPIX,STATUS)
*
* Call the main working subroutine. Write new values to output
*
      CALL ZAP(THRESH,NPIX,%VAL(IPNTR),%VAL(OPNTR),STATUS)
*
* Close down
*
      CALL NDF_END(STATUS)
      END

      SUBROUTINE ZAP(THRESH,NPIX,IMAGE,OUT,STATUS)
*
* Zap pixels with more counts than THRESH by assigning BAD values to them
*
      IMPLICIT NONE       ! No implicit typing
      INCLUDE 'SAE_PAR'   ! Standard SAE constants
      INCLUDE 'PRM_PAR'   ! Define BAD constants
* 
* (>) - Given,  (<) - Output
*
      REAL THRESH         ! (>) Threshold value
      INTEGER NPIX        ! (>) Number of pixels in images
      REAL IMAGE(NPIX)    ! (>) Array of input pixel values
      REAL OUT(NPIX)      ! (<) Array of output pixel values
      INTEGER STATUS      ! (<>) Global Status
*
* Local variables
*
      INTEGER N
*
* Go through image and find excess values
*
      DO 1, N = 1, NPIX
        IF (IMAGE(N).NE.VAL__BADR .AND. IMAGE(N).GT.THRESH) THEN
          OUT(N) = VAL__BADR
        ELSE
          OUT(N) = IMAGE(N)
        ENDIF
 1    CONTINUE
      END
\end{verbatim}
}
\end{quote}

Note that although the variance array was mapped, none of its values were
changed in this simple application. The variance array was mapped and
therefore allocated pointers (although this wasn't necessary to force the
variance propagation). The interested reader might wish to try to adapt
the call to {\sf ZAP} to fix the variance values using these pointers. 

\subsection{The Quality component}

In addition to the Data and Variance components, some NDFs also contain a
{\em Quality}\, component.  The quality array is not usually accessed
directly by the user. Instead, the processing of the Data and Variance
arrays {\em automatically}\, takes into account the values in the Quality
array unless the application explicitly accesses and maps the Quality 
array.

Why bother with a Quality array when there is a variance array? The
Quality value of a pixel can be used to describe its suitability for
performing a particular task. For example, imagine a situation where
photometry is performed on an object. The quality array could be set to
one value for the object and another for the sky pixels. 

Manipulation of the Quality array is not commonly done in most 
applications and so there will be no further discussion of it here. 
\xref{SUN/33}{sun33}{} contains a full discussion of how to 
access the Quality array.

\subsection{Rules for propagation of variance arrays}

It is not appropriate to propagate the values in the variance array in all
circumstances. For example, adding two NDFs together will result in new
variances for the result. If these new variances are not calculated, the output
NDF must not be allowed to contain a variance array. On the other hand, if a
constant is added to an NDF, the variances can be passed on to the output with
no calculations necessary. 

The default used by the routine {\sf NDF\_PROP} is {\em not}\, to propagate
the data, quality, or variance arrays. This explains why is was necessary
to explicitly list what we wanted to propagate in the last example. 

\section{\xlabel{ndf_extensions}NDF extensions}

The NDF format is an example of the HDS (Hierarchical Data Structure) format.
Essentially, this means that lots of different items of
information can be put into the same file. For example, we have
already seen how the Data and Variance arrays come as distinct pieces
of information.

All NDFs have a number of standard components, many of which (but not
all) we have dealt with in this book. HDS (and therefore NDF) files
may
also contain {\em extensions}. These may be so commonly used they are
virtually standard (e.g. a FITS header), or be specific to a particular
instrument or software package (e.g. the IRAS extension).

In this section, we deal with how to access and create extensions to
NDF files. 

\subsection{Example 11 -- Reading an extension}

The following example reads in an IRAS extension:

\begin{quote}
{\small
\begin{verbatim}
       SUBROUTINE EXTEND(STATUS)
*
* This application looks for an IRAS extension to an NDF.
* If it is present, it reads and prints the OFFSET value
*
       INTEGER STATUS               ! Global Status
       INTEGER INDF                 ! NDF identifier 
       INTEGER OFFSET               ! Item from the IRAS extension
       INCLUDE 'DAT_PAR'            ! Define DAT__SZLOC
       CHARACTER * (DAT__SZLOC) LOC ! Locator for the IRAS extension
       LOGICAL EXIST                ! TRUE if the IRAS extension exists
       INCLUDE 'SAE_PAR'            ! Define the SAE constants
*
* Begin the NDF context
*
       CALL NDF_BEGIN
*
* Get the name of the NDF
*
       CALL NDF_ASSOC('IN','READ',INDF,STATUS)
*
* Find out if the IRAS extension exists or not
*
       CALL NDF_XSTAT(INDF,'IRAS',EXIST,STATUS)
*
* If it does, print it to the screen. If not, finish up.
*
       IF (EXIST) THEN
         CALL NDF_XLOC(INDF,'IRAS','READ',LOC,STATUS)
         CALL CMP_GET0I(LOC,'OFFSET',OFFSET,STATUS)
         WRITE (*,*) 'Offset = ', OFFSET
*
* Annul the locator
*
         CALL DAT_ANNUL(LOC,STATUS)
       ELSE
         WRITE (*,*) 'No IRAS extension exists in this file'
       ENDIF
*
* Tidy up
*
       CALL NDF_END(STATUS)
       END
\end{verbatim}
}
\end{quote}

Interface file:

\begin{quote}
{\small
\begin{verbatim}
interface EXTEND
  parameter IN
  prompt 'Input NDF'
  endparameter
endinterface
\end{verbatim}
}
\end{quote}

Note that the call to {\sf CMP\_GET0I} should be replaced by {\sf
CMP\_GET0R} for real numbers, {\sf CMP\_GET0D} for double precision numbers,
{\sf CMP\_GET0L} for logical values, {\sf CMP\_GET0C} for character values. 

\subsection{Example 12 -- Creating a new extension}

In the following example, a code is used to place the temperature of the CCD
chip used during the observations into a new extension. In practice, many 
more items could also be put there such as pixel scale, size, gain etc.
These items can then be read and used by later applications as part of 
the data reduction process.

\begin{quote}
{\small
\begin{verbatim}
       SUBROUTINE EXTEND2(STATUS)
*
* This application puts a value for the CCD temperature into
* an extension designed to hold information about the CCD's
* run time properties.
*
       INTEGER INDF                 ! NDF identifier
       INTEGER STATUS               ! Global Status
       REAL TEMP                    ! CCD Temperature
       LOGICAL EXIST                ! TRUE if the extension exists
       INCLUDE 'DAT_PAR'            ! DAT constants
       CHARACTER * (DAT__SZLOC) LOC ! Extension Locator
       INCLUDE 'SAE_PAR'            ! SAE constants
*
* Begin the NDF context
*
       CALL NDF_BEGIN
*
* Get the name of the NDF
*
       CALL NDF_ASSOC('IN','UPDATE',INDF,STATUS)
*
* Find out if the CCD extension exists or not
*
       CALL NDF_XSTAT(INDF,'CCD',EXIST,STATUS)
       IF (.NOT. EXIST) THEN
*
* Create the new extension
*
         CALL NDF_XNEW(INDF,'CCD','CCD_EXTENSION',0,0,LOC,STATUS)
*
* If the temperature component doesn't exist then make it
*
         CALL CMP_MOD(LOC,'TEMPERATURE','_REAL',0,0,STATUS)
*
* Get the _REAL value for the temperature
*
         CALL PAR_GET0R('TEMP',TEMP,STATUS)
*
* Put the _REAL value into the extension
*
         CALL CMP_PUT0R(LOC,'TEMPERATURE',TEMP,STATUS)
*
* Annul the locator
*
         CALL DAT_ANNUL(LOC,STATUS)
       ELSE
         WRITE (*,*) 'CCD extension already present' 
       ENDIF
*
* Tidy up
*
       CALL NDF_END(STATUS)
       END
\end{verbatim}
}
\end{quote}

Interface file:

\begin{quote}
{\small
\begin{verbatim}
interface EXTEND2

  parameter IN
  prompt 'Name of NDF'
  endparameter

  parameter TEMP
  type _REAL
  prompt 'Temperature'
  endparameter

endinterface
\end{verbatim}
}
\end{quote}

\section{\xlabel{making_your_own_software_package}Making your own software package}

\subsection{Monoliths}

Sometimes it can be useful to group several applications together into 
one package. The result is called a {\em monolith}. A monolith consists 
of a ``top level'' code which then calls the various applications as needed.
The monolith also has its own interface file (whose structure is slightly 
different from those of applications). The following monolith brings 
together examples 8a, 8b, and 8c into a single package called {\sf newndf}.

\subsection{Example 13 -- A monolith}
 
Code:

\begin{quote}
{\small
\begin{verbatim}
       SUBROUTINE NEWNDF(STATUS)
*
* This monolith puts together all three create applications
* into a single package.
*
       INCLUDE 'SAE_PAR'
       INCLUDE 'PAR_PAR'
       INTEGER STATUS
       CHARACTER * (PAR__SZNAM) ACTION
*     
       IF (STATUS .NE. SAI__OK) RETURN
*
* Get the action name
*
       CALL TASK_GET_NAME(ACTION,STATUS)
*
* Go through the commands
*
       IF (ACTION .EQ. 'CREATE') THEN
         CALL CREATE(STATUS)
       ELSEIF (ACTION .EQ. 'CREATE2') THEN
         CALL CREATE2(STATUS)
       ELSEIF (ACTION .EQ. 'CREATE3') THEN
         CALL CREATE3(STATUS)
       ELSE
         WRITE (*,*) 'Dunno!'
       ENDIF
       END
\end{verbatim}
}
\end{quote}

Note how the names of the three separate tasks are in CAPITAL LETTERS. 
Lower case will not work, even though the filenames of your applications
usually are lower case. The corresponding interface file is:

\begin{quote}
{\small
\begin{verbatim}
monolith NEWNDF

interface CREATE
  parameter LBND
    position 1
    type _INTEGER
    prompt 'Lower bound'
  endparameter
  parameter UBND
    position 2
    type _INTEGER
    prompt 'Upper bound'
  endparameter
  parameter OUT
    position 3
    prompt 'Output NDF'
  endparameter
endinterface

interface CREATE2
  parameter LBND
    position 1
    type _INTEGER
    prompt 'Lower bound'
  endparameter
  parameter UBND
    position 2
    type _INTEGER
    prompt 'Upper bound'
  endparameter
  parameter OUT
    position 3
    prompt 'Output NDF'
  endparameter
  parameter TITLE
    position 4
    type 'Literal'
    prompt 'Title'
  endparameter
endinterface

interface CREATE3
  parameter LBND
    position 1
    type _INTEGER
    prompt 'Lower bound'
  endparameter
  parameter UBND
    position 2
    type _INTEGER
    prompt 'Upper bound'
  endparameter
  parameter OUT
    position 3
    prompt 'Output NDF'
  endparameter
  parameter TITLE
    position 4
    type 'Literal'
    prompt 'Title'
  endparameter
endinterface

endmonolith    
\end{verbatim}
}
\end{quote}

To compile the code, {\sf alink} is used thus:

\begin{verbatim}
   alink newndf.f create.f create2.f create3.f -L/star/lib `ndf_link_adam`  
\end{verbatim}

but one further stage is still required to get the monolith to work. Soft 
links must be made from the monolithic executable to the names of the 
tasks it contains. In other words, to get the above example to work you 
must type:

\begin{verbatim}
   ln -s newndf create
   ln -s newndf create2
   ln -s newndf create3
\end{verbatim}

Most Starlink packages also use a script to set up aliases to these 
soft links. For example, when Figaro is started, the user is really 
running a script called:

\begin{verbatim}
   /star/bin/figaro/figaro.csh
\end{verbatim}
 
\section{\xlabel{compiling_code_without_adam}Compiling code without ADAM}

Throughout this book, the codes have been compiled using the ADAM software
environment. While useful, it does not necessarily {\em have}\, to be the
case. So called Standalone applications are independent of ADAM and have
no interface files. They are compiled with the standard {\sf f77} command
and flags rather than the {\sf alink} command. For example: 

\begin{verbatim}
   % alink myprog.f -L/star/lib `ndf_link_adam`
\end{verbatim}

would be replaced by:

\begin{verbatim}
   % f77 myprog.f -L/star/lib `ndf_link` -o myprog
\end{verbatim}

Note, however, that there are some NDF routines which rely on the parameter 
system. For more details, see Appendix 3 of \xref{SUN/33}{sun33}{}.

\section{\xlabel{miscellaneous}Miscellaneous}

\subsection{A brief word about axes}

Some NDFs contain information about axes other than just the pixel
coordinate numbers of each pixel in the image, e.g. the wavelength scale
of a spectrum. The issue of manipulating axes is complex and somewhat
beyond the scope of this cookbook. There are already applications in
existence which modify axis scales for tasks such as wavelength and flux
calibration (such as those in Figaro). 

It is suggested that the reader who really {\em needs}\, to manipulate axes 
consults the NDF programming manual directly \xref{(SUN/33)}{sun33}{}.

\subsection{Starlink libraries}

One of the reasons Starlink came into existence was to try to reduce the
duplication of effort amongst astronomers. Why should there be an
individual flat fielding routine for every astronomer? One of the best
ways to reduce the amount of duplication of effort is by using software
{\em libraries}. These contain large amounts of code already written for
you. For example, if you need to write a code which works out the Julian
Date of an object, you can use the SLALIB Library. 

You can think of these libraries as a bunch of subroutines scrunched up
together in one big file. All you have to do to use them is to write a
``top layer'' of Fortran (or sometimes C) code to call them. In fact, this
is what you've been doing throughout this book. The IMG and NDF libraries
are called by the top layer of code which you write. 

{\bf Just like you had to link your application to the NDF and IMG
libraries, you also have to link to any other libraries you call. Check
the reference manual of the libraries you need to use to find out how to
use alink appropriately.}

There are a number of libraries available for you to use when writing your
applications or packages. Starlink provides access to a number of them not
covered in this book and they are listed in Appendix A of this book. 

Note that Starlink no longer uses NAG libraries. The reason for this is
that NAG is a commercial package and as such, cannot be freely
distributed. The PDA library {\em can}\, be distributed freely. If you
intend to pass your software on to other users, you might need to take the
same approach. 

\subsection{Hints for writing good applications}

There are two groups of people you need to think about in writing a
``public'' application. The first group is the most obvious: the people
who are going to use it. A code containing the slickest algorithms ever
seen is no use to anyone if they can't understand the instructions! This
can be avoided by paying attention to both your run time prompts and
user documentation. 

The second, less obvious, group of people you need to cater for are those
who support the code. If that person is you the author, then it is all too
tempting not to comment your code adequately. On the other hand the code
might need to be passed on to a colleague. There is nothing worse for the
useful lifetime of a package than giving the support team thousands of
lines of uncommented, poorly structured code. Write your code to be as
friendly to the programmer as it is to the user. 

\subsection{Documenting your code}

There are three ways to document a code. Firstly, write a manual or
manuals. You certainly need a ``User Guide'' telling the user how to use
the package, what it does, and to some small extent how the inner workings
function. If the latter is getting too technical i.e. full of ``this
algorithm incorporates bivariate Chebyshev polynomial solutions'', perhaps
it's time to write the ``Reference Guide''. There's no hard and fast
rule as to how to do this. It's partly a matter of taste but more a case
of knowing your audience.
Some advice is given in \xref{SGP/28}{sgp28}{}.

Secondly, comment the source code. 
Firstly and most importantly, the programmer must be shown how the code 
works, i.e. ``what goes where'', ``this line does'' and ``this piece of
code is commented out because it was both too slow and unaesthetic''.
Don't be verbose (like some of the examples in this book have been),
but make sure it's clear.

Thirdly, there is online help. There are a number of Starlink facilities
to help implement this. The {\sf HLP} package 
helps develop your online support facilities.

\subsection{Portability}

At the time of writing, Fortran 90 is not yet easily portable from one
platform to the next. It is recommended that the established  Fortran~77
is used for writing applications for the time being. 

\subsection{Submitting code to Starlink}

Starlink welcomes software contributions from the astronomical community.
If you feel you may have software to contribute to the Starlink Project, 
please contact the Starlink Librarian ({\sf ussc@star.rl.ac.uk}) who 
will be glad to hear from you.

\newpage

\appendix

\section{\xlabel{useful_resources}Useful resources}

\subsection{Useful libraries and commands}

Convert -- {\em (Automatic) format conversion} - \xref{(SUN/55)}{sun55}{}

Findme -- {\em Documentation search via a Web browser} - 
\xref{(SUN/188)}{sun188}{}

FIO -- {\em Fortran file handling} - \xref{(SUN/143)}{sun143}{}

HLP -- {\em On-line documentation support} - \xref{(SUN/124)}{sun124}{}.

IMG -- {\em Simple data access library} - \xref{(SUN/160)}{sun160}{}

MEMSYS -- {\em Maximum entropy image reconstruction} -
\xref{(SUN/117)}{sun117}{}

NDF -- {\em More advanced data access library} - \xref{(SUN/33)}{sun33}{}

PDA -- {\em Public domain mathematical algorithms (as used by Starlink 
Software)} - \xref{(SUN/194)}{sun194}{}

Showme -- {\em Web-based document retriever} - \xref{(SUN/188)}{sun188}{} 

SLALIB -- {\em Positional and time information} - \xref{(SUN/67)}{sun67}{}

\subsection{Further reading}

ADAM -- {\em The Starlink software environment} - \xref{(SG/4)}{sg4}{}

ADAM -- {\em Introduction to ADAM programming} - \xref{(SUN/101)}{sun101}{}

ADAM -- {\em Unix version} - \xref{(SUN/144)}{sun144}{}

ADAM -- {\em Programmer's facilities and documentation guide} - \xref{(SG/6)}{sg6}{}

ADAM -- {\em Graphics programmer's guide} - \xref{(SUN/113)}{sun113}{}

ADAM -- {\em Interface module, reference module} - \xref{(SUN/115)}{sun115}{}

ADAM -- {\em Guide to writing instrumentation tasks} - 
\xref{(SUN/134)}{sun134}{}

NDF -- {\em Adding format conversion facilities to NDF data} -
\xref{(SSN/20)}{ssn20}{}

CNF -- {\em C and Fortran mixed programming} - \xref{(SUN/209)}{sun209}{}

FITSIO -- {\em Disk FITS input/output subroutines} - \xref{(SUN/136)}{sun136}{}

FIO/RIO -- {\em Fortran file I/O routines} - \xref{(SUN/143)}{sun143}{}

FTNCHEK -- {\em A Fortran77 source-code checker} - \xref{(SUN/172)}{sun172}{}

FORCHECK -- {\em Fortran verifier and programming aid} - 
\xref{(SUN/73)}{sun73}{}

HDSTRACE -- {\em HDS data file listing}  - \xref{(SUN/102)}{sun102}{}

{\em How to write good documents for Starlink} - \xref{(SGP/28)}{sgp28}{}

IRAFSTAR -- {\em The IRAF/Starlink inter-operability infrastructure} -
\xref{(SSN/35)}{ssn35}{}

MERS (MSG and ERR) -- {\em Message and error reporting systems} -
\xref{(SUN/104)}{sun104}{}

{\em Starlink applications programming standard} -- \xref{(SGP/16)}{sgp16}{}

{\em Starlink C programming standard} -- \xref{(SGP/4)}{sgp4}{}

\section{\xlabel{glossary}Glossary}

{\sf Application} -- A piece of software which operates on data to
produce some desired result.

{\sf C} -- A programming language used in some applications and libraries. 

{\sf Extensions (or HDS extensions)} -- Components of an HDS (e.g. an
NDF) file used to store additional information about data.

{\sf f77} -- the Fortran 77 compiler command

{\sf f90} -- the Fortran 90 compiler command

{\sf Fortran} -- The Starlink preferred language for applications programming.

{\sf Header} --
Information at the start of a file which describes the origins, components,
and sometimes the history of the file. 

{\sf HDS} --
The Hierarchical Data Structure file format. This allows many pieces of
distinct information to be stored in one file. NDFs are an example of
the HDS format.

{\sf IMG library} --
A simple library used to gain access to NDF files.

{\sf Interface file} -- A file which the ADAM system uses to find out
information about the various parameters needed to use a code.

{\sf Monolith} --
A code which groups many applications together into a single package.

{\sf NDF} -- N-dimensional data format. This is a way of storing many 
different types of information in the same file. All Starlink Data Format 
files (.sdf files) are NDFs.

{\sf NDF library} --
A collection of routines used to access and manipulate all the various
components of an NDF file.

{\sf Package} -- A larger piece of software consisting of many small 
applications.

{\sf Parameters} -- Items of information which the code needs to run. These
are usually provided by the user.

{\sf Mapping} -- A way of describing the layout of an array within an NDF
to the codes that manipulate them.

{\sf Quality array} -- An array of values associated with all the pixels in
a file which report on the status of the pixel e.g. good/bad. 

{\sf Starlink} -- The RAL-based organisation responsible for
coordinating the UK astronomical computing effort.

{\sf Variance array} -- An array of values associated with all the pixels in
a file which correspond to the squared standard deviation of the data 
value of those pixels.

\end{document}
