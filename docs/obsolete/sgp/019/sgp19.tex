\documentstyle {article} 
\markright{SGP/19.5}
\setlength{\textwidth}{153mm}
\setlength{\textheight}{220mm}
\setlength{\oddsidemargin}{3mm}
\setlength{\evensidemargin}{3mm}
\pagestyle{myheadings}

\begin{document}
\thispagestyle{plain}
\noindent
SCIENCE \& ENGINEERING RESEARCH COUNCIL \hfill SGP/19.5\\
RUTHERFORD APPLETON LABORATORY\\
{\large\bf Starlink Project\\}
{\large\bf Starlink General Paper 19.5}
\begin{flushright}
M D Lawden\\
30 March 1987
\end{flushright}
\vspace{-4mm}
\rule{\textwidth}{0.5mm}
\vspace{10mm}
\begin{center}
{\Large \bf
STARLINK SOFTWARE SUBMISSION}
\end{center}
\vspace{10mm}
\section {INTRODUCTION}
The definition, structure and management of the Starlink Software Collection is
described in SGP/20 which should be read by everyone involved in the
production of software for the Starlink project.
The Collection is managed by the Starlink Software Librarian (username STAR) who
decides where new software should be stored in the existing structure and who
has editorial control of Starlink documentation.
This paper describes the principles governing the preparation and submission of
software for inclusion in the Collection.
Such software should meet appropriate standards of programming, structure,
completeness and documentation, and should be submitted by following the
procedure described in section 4.
\section {ELIGIBILITY FOR RELEASE}
Everyone is encouraged to submit software of general utility for consideration
for inclusion in the Collection.
However, plans for writing such software should first be discussed with the
Project Manager so that they can be coordinated with the overall plans.
Software is selected for inclusion in the Collection by the Project Manager and
the Librarian.
Their selection is strongly influenced by the various Special Interest Groups
who advise on the requirements of specific application areas.
It is also desirable for the Collection to be limited to a reasonable size and
to avoid unnecessary duplication.
\section {DEVELOPMENT STANDARDS}
Software should conform to the programming standards specified in SGP/16 and
should be as modular as possible.
It should be possible to store it at a single point in the STARLINK directory
tree (see RLVAD::LADMINDIR:DIRMAP.LIS) although large or specialised items may
be installed independently in their own top level directory.
Associated Starlink Notes or Papers will be stored in DOCSDIR.
The Librarian is responsible for deciding where your software is placed in the
Starlink directory structure.
He may wish to put it in a new subdirectory so do not assume it will go in an
existing directory.
He is also responsible for coordinating the use of logical names and global
symbols defined in the Collection.
You should discuss these topics with him before finalising software design and
writing extensive documentation as they will affect your instructions for use
and the names of directories and files.

Software should be adequately documented and should include at least one
Starlink User Note (this may refer to other documentation).
Produce documentation in the form of RUNOFF files since these can be changed
easily to provide a standard output format or to incorporate future changes
(see SGP/28).
Use the standard headers stored in DOCSDIR:SUN.RNO, etc.
RUNOFF is described in the VAX RUNOFF manual and the Guide to Text Processing.
Start every new sentence on a new line as this makes editing much easier.
Although RUNOFF files are the preferred form of text documentation, do not be
discouraged from including diagrams, illustrations, mathematical formulae or
special notation in your documentation.
You can leave space for anything that is more easily produced by other methods
and the final master copy produced by hand by putting in the missing bits on
the printed text.
An alternative to RUNOFF is TEX/LATEX.
These can produce a more elaborate output style and higher quality masters.
Starlink recommends that you use LATEX and base your document on the standard
headers in DOCSDIR called SUN.TEX, SGP.TEX and SSN.TEX.
A disadvantage of TEX/LATEX is that documentation in this form is not easily
viewed on a terminal.
However, there is a clear trend towards using them and I think the advantages
outweigh the disadvantages.
The Librarian reserves the right to carry out minor editorial changes to any
Starlink documentation you submit in order to make it conform to general
Starlink standards.
If you revise any existing Starlink documentation, copy the existing version
from DOCSDIR and edit that so as to preserve any existing editorial changes.
Do not use a private copy.
\section {SUBMISSION STANDARDS}
Installation instructions should be clear, simple, complete and unambiguous.
Do not assume any special knowledge or insight on the part of the Librarian
into the structure, relationships, dependencies or operation of your software.
The installation should be a simple clerical operation as the Librarian is
performing an administrative function, not a software development task.
If there are any difficulties or complexities associated with the installation
of your software, it is your responsibility to resolve them and present the
Librarian with a simple task.

Supply the Librarian with a list of everything that constitutes the package
(including both disc files and external items such as paper diagrams), together
with a clear breakdown into the following categories:
\begin{itemize}
\item required for use
\item documentation
\item required for support and development
\end{itemize}
For anything other than a simple program, explain the structure, relationships
and dependencies of your software.
Describe how to create an executable program or system from the source code.
Everything needed must be provided or already be available in the Collection.
If it is a simple `Compile-Link' operation, a set of clear and complete
instructions is adequate.
Otherwise, provide a command procedure to generate an executable system.
Thus, the package submitted should be complete in the sense that someone in
Australia (for example) can recreate the executable files from the source and
libraries in the Starlink directories.

Supply software in a form that is independent of private source and
documentation development tools.

Indicate the location of all relevant documentation and provide master copies of
any supporting documentation, unless you provide a file which can easily be
printed on a lineprinter or processed by LATEX for output on a laser printer.
Include enough systems documentation to enable the Librarian to understand the
structure and operation of your software and how to create a working system
from your source code.

A form has been designed to help you specify the required information.
This is the STARLINK SOFTWARE ITEM SUBMISSION FORM (SSISF).
A specimen is shown in Appendix A.
They can be obtained from your local Site Manager or from the Librarian at RAL.
The software you submit for release will comprise one or more software items.
Fill in a separate SSISF for each item and send them, together with any
necessary tapes, manuals or master copies, to the Librarian.
Some notes to help you fill in this form are show in Appendix B.

The Librarian will release your software by carrying out the procedure described
in SSN/41.
\section {SUPPORT STANDARDS}
You should be prepared to support any software you submit for release.
If you are unable to do this yourself, nominate someone else who is.
Starlink staff cannot support other people's software except at an elementary
level.
If no support is provided, an item will be released (if at all) `as is' and
designated `not supported' in the Starlink Software Index.
Sometimes the importance of an item will force its release without support.
\section {INSTALLATION PROBLEMS}
Installation problems are usually caused by inaccurate, incomplete or ambiguous
information.
Please carefully check your own submissions for these faults.

The problems which have caused more time-delay and frustration than any others
have been:
\begin{itemize}
\item dependence on software which is not in the Collection.
\item use of devices which are not available at all Starlink nodes.
\item use of devices in a non-standard way.
\item use of private documentation production systems.
\item no instructions about how to generate a working program.
\end{itemize}
Often there are good reasons why the software was developed in this way, but you
must understand that such software may simply not work when installed as
Starlink software.
Furthermore, the Collection is routinely distributed to non-Starlink VAX's.
If your software does depend on some local system, submit this for installation
in the Collection as a separate item in conjunction with your primary software.
\section {SUMMARY}
The steps required to prepare and submit software for the Collection are
summarised below:
\begin{enumerate}
\item Coordinate your software development with the Starlink Project Manager and
Software Librarian.
\item Develop and implement your software in private directories, but be aware of
the changes that will be required when it is installed in Starlink directories.
\item Gather together all the information and material required by the Librarian.
\item Complete a Starlink Software Item Submission Form for every item submitted.
\item Submit the software and SSISF to the Librarian for release.
\item Resolve any installation problems encountered by the Librarian.
\item Support the software.
\end{enumerate}
\newpage
\appendix
\section {STARLINK SOFTWARE ITEM SUBMISSION FORM}
Please read SGP/19 and the notes in Appendix B before attempting to complete
this form.
If the answer to a question is given in a supplied document, give a reference
as the answer.
If an answer is too long for the space provided, attach a separate sheet and
identify the answer by the question number.
Send the completed form to the Starlink Software Librarian at RAL.
**************************************************************************************
\begin{tabbing}
10.X\=XXX\=\kill
1.\>Title:\\
\>Acronym:\\
\\
2.\>Names of:\\
\>\>person submitting this item:\\
\>\>anyone else who can answer technical queries:\\
\>\>person responsible for support:\\
\\
3.\>With which Starlink staff have you been coordinating development of this
software?\\
\\
\\
4.\>Is it a New item:\hspace{30mm}Modification to old item:\\
\>If it is a modification, what modifications have been made:\\
\\
\\
\>Are there any other items which should be released in conjunction with this
one?\\
\\
\\
5.\>Specify completely the files/directories to be copied into Starlink
directories:\\
\\
\\
\\
\\
6.\>Specify any special environmental requirements:\\
\>* Logical Name definitions:\\
\\
\\
\\
\\
\>* Global symbol definitions:\\
\\
\\
\\
\\
\>* Does this program run under a Starlink environment?\\
\>* What version of VMS is it running under?\\
\>* Does it use any special device drivers? (state)\\
\\
\\
\\
\>* Does it use any special devices, eg. Calcomp plotter? (state)\\
\\
\\
\\
\>* Any other unusual requirements?\\
7.\>Does this item make use of any software (apart from VMS) which has not\\
\>been installed in the Starlink Software Collection? (specify)\\
\\
\\
\\
\\
8.\> Does this item use any graphics system (eg. GKS, FINGS, etc)? (specify)\\
\\
\\
\\
\\
9.\>Is the ARGS or Ikon necessary for the proper use of this item?\\
10.\>Is the Versatec necessary for the proper use of this item?\\
11.\>Specify how to create the executable program or system:\\
\\
\\
\\
\\
12. Specify how to execute the program or system:\\
\\
\\
\\
\\
13. Specify any unusual or large resource requirements:\\
\\
\\
\\
\\
14.\>Where is the documentation which supports this software?\\
\>(eg. name of file holding this document, paper documents attached)\\
\\
\\
\\
\\
15.\>Specify a test which can be carried out to verify that this\\
\>item has been installed correctly and any test data needed.\\
\\
\\
\\
\\
16.\>If relevant, specify a demonstration procedure that can be used to\\
\>show off this item's capabilities to visitors.\\
\\
\\
\\
\\
17.\>Installation procedure:
\end{tabbing}
\newpage
\section {SSISF --- NOTES}
These notes are meant help you complete the STARLINK SOFTWARE ITEM SUBMISSION
FORM.
They explain the purpose of the non-trivial sections and the type of
information required.

1. Specify a short descriptive title.
Every item should also have a short single word descriptive name associated
with it; this is the `acronym'.

4. If the item has been released before, describe the changes made in this
release.
We also want to know any dependencies this item may have on any software not in
the Collection.

5. A key part of the installation process is the copying of files into Starlink
directories from where you store them.
Specify the location of the files you want released.
The Librarian will normally decide which Starlink directories should receive
them.
He may decide to create one or more new Starlink directories, particularly for
packages.

6. The Librarian may not be aware of special environmental requirements of
your software.
If any other actions are required besides copying the files specified in
section 5 in order to make your program usable, specify them here; examples of
what we have in mind are specified on the form.

7. If your software depends on other software that is not in the Collection it
may not work when installed.
If it does use such software, consider releasing it for the Collection.
Remember that if your software proves difficult to install, it will be returned
to you for action.
The software contained in the Collection is specified in ADMINDIR:SSI.LIS and
ADMINDIR:SSF.LIS.

11. Give clear explicit instructions on how to create a working version of your
program or system.
You can write this explicitly, specify a command procedure, or point to some
documentation that contains this information.

12. You can point to supplied documentation.

13. Can a normal user with normal quotas and privileges run your program or
package?
If not, what does he need?

14. Software should not be included in the Collection unless it is documented
to a standard such that another user can use your software successfully with no
help other than the documentation or on-line support.
If possible, your documentation should exist in one or more computer files in
RUNOFF or LATEX format.
These are the easiest forms to change and edit.
The Librarian will normally generate a master copy and distribute copies of this
as part of the release process.
Do not submit files which depend on private libraries of macros (for instance). 
Your file should be usable by the Librarian without knowledge of any other of
your files.
If your documentation cannot be prepared in this way (eg. it needs diagrams or
the text is not available in a computer file), send a master of anything that
needs to be added or copied.
The Librarian should normally be the person to issue documentation associated
with software items being released.

15. It is comforting to be able to carry out a simple test on your software
after it has been installed in Starlink directories to show that it gives the
same results as it did when you ran it.
Please provide us with such a test if you can, otherwise your software will be
released (if at all) untested and marked `No Test Supplied' in our records.

16. Visitors sometimes turn up and ask to be given a demonstration of
`Starlink'.
Usually what they want is a colourful, photogenic display of graphs or pictures
of the `Tomorrows World' variety.
Sometimes they actually want to see some serious analysis of astronomical data.
The people who are caught on the spot on these occasions may have very little
practical experience in actually running this kind of thing.
The visitors are likely to go away wondering why such an expensive and powerful
system doesn't seem able to do that much.
If your program or package is suitable for such demonstrations we would
appreciate it if you would take the trouble to put together some kind of
demonstration that would be suitable for showing to visitors.
It doesn't have to be exhaustive.
Please help us to give Starlink a good image on these occasions.

17. If no instructions are given the Librarian will copy the files you
specified in section 5 into likely looking Starlink directories and expect
your program to work immediately.
If there is more to it than that and the problems have not been covered by
answers to previous sections, specify here what the Librarian must do to
install your software successfully.
\end{document}
