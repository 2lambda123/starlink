\documentstyle[11pt]{article} 
\pagestyle{myheadings}

%------------------------------------------------------------------------------
\newcommand{\stardoccategory}  {Starlink General Paper}
\newcommand{\stardocinitials}  {SGP}
\newcommand{\stardocnumber}    {21.8}
\newcommand{\stardocauthors}   {M D Lawden, M J Bly, J C Sherman}
\newcommand{\stardocdate}      {24 November 1992}
\newcommand{\stardoctitle}     {Starlink Software Distribution Policy}
%------------------------------------------------------------------------------

\newcommand{\stardocname}{\stardocinitials /\stardocnumber}
\markright{\stardocname}
\setlength{\textwidth}{160mm}
\setlength{\textheight}{230mm}
\setlength{\topmargin}{-2mm}
\setlength{\oddsidemargin}{0mm}
\setlength{\evensidemargin}{0mm}
\setlength{\parindent}{0mm}
\setlength{\parskip}{\medskipamount}
\setlength{\unitlength}{1mm}

\begin{document}
\thispagestyle{empty}
SCIENCE \& ENGINEERING RESEARCH COUNCIL \hfill \stardocname\\
RUTHERFORD APPLETON LABORATORY\\
{\large\bf Starlink Project\\}
{\large\bf \stardoccategory\ \stardocnumber}
\begin{flushright}
\stardocauthors\\
\stardocdate
\end{flushright}
\vspace{-4mm}
\rule{\textwidth}{0.5mm}
\vspace{5mm}
\begin{center}
{\LARGE \bf \stardoctitle}
\end{center}
\vspace{5mm}

%-------------------------------------------------------------------------------
\setlength{\parskip}{0mm}
\tableofcontents
\setlength{\parskip}{\medskipamount}
\markright{\stardocname}
%-------------------------------------------------------------------------------

\newpage

\section {Introduction}
\label{se:intro}

This paper specifies Starlink's policy for distributing its software both
within Starlink, and to non-Starlink sites. It applies to anyone (Site Manager
or user) who is considering installing Starlink software at a Starlink site, or
sending it to a non-Starlink site. It also provides information about how to
obtain commercial software used by Starlink. The policy outlined herein applies 
to the VAX/VMS and Unix implementations of Starlink software, and any other
implementations on other systems. 

\begin{itemize}
\item SGP/20 defines the meaning of the term `Starlink Software'.
\item SSN/15 specifies the procedure for physically distributing and installing
Starlink software.
\item SSN/41 specifies the procedure for updating Starlink software.
\end{itemize}

This paper distinguishes between {\em Non-commercial Software,} usually written
by an astronomer or Starlink programmer, and {\em Commercial Software,} which
has been obtained from a supplier such as NAG on payment of a license or
purchase fee.

{\em Non-commercial Software} can be supplied to bona-fide astronomers at
non-Starlink sites from any Starlink site with the consent of the Site Manager,
provided the restrictions described in Appendix \ref{se:cons} are adhered to,
and that full information about the recipient and the software items supplied
are sent to the Starlink Software Librarian (currently Martin Bly) at RAL.

{\em Commercial Software} {\bf cannot} be supplied to non-Starlink sites,
except under the specific circumstances described in Section \ref{se:comm}.

The currently available Starlink software is specified in file  {\tt
ADMINDIR:SSI.LIS} (VAX/VMS systems) which should be consulted for the latest
situation.

The normal point of distribution for the Starlink Software Collection is the
Starlink Software Librarian at RAL, who may be contacted at the following
address:\label{pa:add} 

\begin{verse}
Starlink Software Librarian,\\
Rutherford Appleton Laboratory \\
Room 1-28, R68 \\
Chilton, \\
DIDCOT, \\
Oxfordshire \\
OX11 0QX \\
UNITED KINGDOM.


Tel: 0235 445363 (UK) \quad (+44 235 445363 overseas) \\
Fax: 0235 445848 (UK) \quad (+44 235 445848 overseas) \\
email: RLVAD::STAR \quad (DECnet/SPAN) \\
email: STAR@UK.AC.RL.STAR \quad (JANET) \\
email: star@star.rl.ac.uk \quad (Internet)
\end{verse}

\newpage

\section {Non-commercial Software}
\label{se:nonc}

Appendix \ref{se:lsco} contains a letter from SERC Central Office concerning 
the exchange of Starlink software.
The conditions under which non-commercial Starlink software may be distributed
to non-Starlink sites is set out in Appendix \ref{se:cons}.
They have been influenced by the document reproduced in Appendix
\ref{se:lsco}.

Appendix \ref{se:gdf} contains a form devised by the licenser of GKS 7.2 and
7.4 (NRDC) which states the conditions under which this software can be
distributed by Starlink.

Any Starlink software distributed to a non-Starlink site should be accompanied
by a copy of SUG, SUN/1, SGP/25, SGP/31, SSN/15 and the policy statement shown
in Appendix \ref{se:cons}. The Starlink Software Librarian must agree to the
distribution before it is made. Details of known software distributions to
non-Starlink sites are stored in file {\tt RLVAD::LADMINDIR:EXTERNAL.LIS}.

\section {Commercial Software}
\label{se:comm}

There are two items of commercial software\footnote{Another item (NCAR) may
change its status at some time in the future to become licensed software; if
this happens, it will affect distribution to new sites but will not affect
existing use.} which can be distributed to non-Starlink sites under special
conditions.

These are:
\begin{tabbing}
XXXX\=XXXXX\=\kill
\>{\bf GKS}:\>Graphical Kernel System (version 7.2 on VMS or 7.4 on Unix).\\
\>{\bf NAG}:\>Mathematical subroutine library.
\end{tabbing}

The status of currently used commercial software is shown in Table
\ref{tab:comm}.

\begin{table}[h]
\begin{center}
\begin{tabular}{||l|l|l|l||} \hline
{\em Item} & {\em Proprietor} & {\em Installed} & {\em Support} \\ \hline \hline
GENSTAT   & NAG      & All sites & Software Librarian\\
GKS       & NRDC     & All sites & David Terrett \\
MEMSYS    & MEDC Ltd & All sites & David Berry \\
NAG       & NAG Ltd  & All sites & Software Librarian \\
TOOLPACK  & NAG Ltd  & All sites & Software Librarian \\ \hline
CLUSTAN   & Clustan Ltd & BIR CAM DUR & Malcolm Currie \\
          &             & LEI MAN PRE & \\
          &             & QMW SOU RAL ROE & \\
IDL       & Research Systems Inc & ARM BEL BIR & Jeff Payne \\
          &             & CAM RAL OXF QMW & \\ \hline
FORCHECK  & Polyhedron Software & STADAT & Software Librarian \\
MAPLE     & Waterloo Maple Software & STADAT & Software Librarian\\
SPAG      & Polyhedron Software & STADAT & Paul Rees \\ \hline
\end{tabular}
\caption{Commercial software in the Starlink Software Collection.}
\label{tab:comm}
\end{center}
\end{table}

The licensing details for each commercial software item in the Starlink
Software Collection is described in more detail below.

\subsection {CLUSTAN}

{\em VMS only}

This software must only be installed at Starlink sites as specified in Table 
\ref{tab:comm}.
It must not be installed at other Starlink sites, or be distributed to
non-Starlink sites.
Purchase enquiries should be directed to:

\begin{verse}
CLUSTAN Ltd,\\
16 Kingsburgh Road,\\
EDINBURGH,\\
EH12 6DZ\\
UNITED KINGDOM.

Tel: 031-337-1448; \hspace{10mm} Tx: 72165 (TELEXG)
\end{verse}

\subsection {FORCHECK}

{\em VMS only}

This software must only be installed on the STADAT Starlink node.
It must not be installed on other Starlink nodes, or be distributed to
non-Starlink sites.
Purchase enquiries should be directed to:

\begin{verse}
Polyhedron Software Ltd,\\
Linden House,\\
93 High Street,\\
Standlake,\\
WITNEY,\\
OX8 7RH\\
UNITED KINGDOM

Tel: 0865-300-579;  \hspace{10mm} Fax: 0865-300-232
\end{verse}

\subsection {GENSTAT}

{\em VMS only}

This software is licensed under the same terms as NAG. It must not be
distributed to non-Starlink sites but can be provided to new Starlink sites.
Starlink may have up to 5 installations of GENSTAT, one each on any of the 
Starlink VAX Clusters.

See under NAG for further details.

\subsection {GKS}

GKS can be distributed to a new Starlink site without charge or notice.
It can be distributed to non-Starlink sites only under special conditions.
For these cases, send the following letter together with the GKS Declaration
Form shown in Appendix \ref{se:gdf} to the site requesting the software:

\begin{quotation}
`The Starlink project has adopted version 7.2 of the GKS graphics software as
part of the Starlink Software Collection. Many items in this Collection depend
on GKS 7.2 so you may well need to install it. If you wish to receive it, you
must complete the enclosed form and return it to me. If you wish to distribute
it to any other sites, each such site must complete a copy of this form and
return it to me. Licensing agreements with the owners of the software do not
permit me to supply it unless this procedure is followed.'
\end{quotation}

The Declaration Form specifies the conditions under which the software must be
used. The Declaration Form should be completed and returned to the Starlink
Software Librarian at RAL, who will file it. GKS may then be sent to the person
who requested it.

GKS is also obtainable from:

\begin{verse}
The Secretary (Ref: KWC/123855),\\
National Research Development Corporation,\\
101 Newington Causeway,\\
LONDON\\
SE1 6BU
\end{verse}

It is possible to use Starlink software in conjunction with other GKS
implementations. Starlink actually uses GKS 7.4 in the Unix Software
Collection. However, there are likely to be slight differences in behaviour
which will require adaptation of applications code and may prevent some
applications from running satisfactorily.

\subsection {IDL}

This software must only be installed at Starlink sites as specified in Table 1.
It must not be installed at other Starlink sites, or be distributed to
non-Starlink sites.
Purchase enquiries should be directed to:

\begin{verse}
Research Systems Inc,\\
777 29th Street, Suite 302,\\
Boulder,\\
CO 80303\\
USA

Tel: 010-1-303-786-9900; \hspace{10mm} Fax: 010-1-303-786-9909
\end{verse}

\subsection {MAPLE}

{\em VMS only}

This software must only be installed on the STADAT Starlink node.
It must not be installed on other Starlink nodes, or be distributed to
non-Starlink sites.
Purchase enquiries should be directed to:
\begin{verse}
Waterloo Maple Software,\\
160 Columbia Street West,\\
Waterloo,\\
Ontario,\\
N2L 3L3,\\
CANADA

Tel: 010-1-519-747-2373;  \hspace{10mm} Fax: 010-1-519-747-5284
\end{verse}

\subsection {MEMSYS}

{\em VMS only}

This software must only be installed at Starlink sites.
It must not be distributed to non-Starlink sites.
In addition, any person who wishes to use this software must complete a copy of
the {\em `Memsys Starlink Academic User's Licence Agreement'}\/ and return it 
to the Starlink Software Librarian at RAL.
Access to MEMSYS is controlled by Access Control Lists (ACLs).

Purchase enquiries should be directed to:
\begin{verse}
Maximum Entropy Data Consultants Ltd,\\
20 Cockcroft Place,\\
CAMBRIDGE\\
CB3 0HF\\
UNITED KINGDOM

Tel: 0223-312562;  \hspace{10mm} Fax: 0223-354599
\end{verse}

\subsection {NAG}

The SERC and NAG have negotiated a deal for the period 1st April 1991 to 31st
March 1996, which allows many implementations of the NAG Fortran and Graphics 
Libraries to be run on any SERC owned CPU. This includes Starlink CPUs, and
Starlink contributes towards the cost of the deal. 

The software currently available and applicable to Starlink is shown in 
Table~\ref{tab:nag}.

\begin{table}[h]
\begin{center}
\begin{tabular}{|l|l|c|c|c|} \hline
{\em Product}  & {\em Version} & {\em VAX/VMS} & {\em SPARCstation} & 
{\em DECstation} \\ \hline \hline
Fortran & Double   &    X    &     X        &    X       \\
        & Single   &    X    &     X        &    -       \\
        & G\_Float &    X    &     -        &    -       \\ \hline
Graphics & Double  &    X    &     X        &    X       \\
        & Single   &    X    &     -        &    -       \\ \hline
Online Info & ~~~- &    X    &     X        &    X       \\ \hline
\end{tabular}
\caption{NAG Fortran and Graphics Implementations available to Starlink.}
\label{tab:nag}
\end{center}
\end{table}

Purchase enquiries should be directed to:

\begin{verse}
NAG Ltd\\
Wilkinson House\\
Jordan Hill Road\\
OXFORD, OX2 8DR\\
UNITED KINGDOM

Tel: 0865-511245;\hspace{10mm}Tx: 83354 NAG UK G;\hspace{10mm}Fax:
0865-310139\\
\end{verse}

When a new version of the NAG library is released, RAL receives tapes
containing the new versions together with updates for the NAG Fortran Library
Manual and a new Introductory Guide. RAL then distributes the object libraries
to the other Starlink sites as part of a Starlink Software Change (SSC or
USSC). The other Starlink sites do not automatically receive any documentation
updates so it is necessary to buy them from NAG. Please contact the Starlink
Documentation Librarian regarding documentation purchase.

Some Starlink software contains NAG routines which have been incorporated into
executable modules during the linking process.
The specific modules involved are unknown, but we have obtained an informal
agreement with NAG to distribute such Starlink software under the following
conditions:
\begin{enumerate}
\item Sites which are sent this software must satisfy the general conditions
laid out in Appendix~\ref{se:cons}.
\item Distribution must be only occasional.
\item A disclaimer (see Appendix~\ref{se:cons}) must be sent with the software.
\end{enumerate}
NAG have asked that the following paragraph be included in the documentation
with the software and/or in a covering letter:

\begin{quote} 
`Fortran subroutines are included in this program by kind permission of the
Numerical Algorithms Group Ltd (NAG), Jordan Hill Road, Oxford, OX2 8DR who
retain all intellectual property rights to the routines. They are not to be
used separately from this program or any charge to be made for this program
without the express written permission of NAG.' 
\end{quote}

\subsection {NCAR}

The NCAR graphics library was originally distributed as public domain software.
This appears to have changed.
A release was received at RAL in August 1986 accompanied by a letter (dated
July 25th 1986) containing the following paragraph:

\begin{quote}
`Under new guidelines from the NSF, this NCAR software package is not in the
public domain. Distribution by NCAR does not include the right of the recipient
or the user to distribute or use this software for commercial purposes.
Commercial distribution is permitted only under agreement with NCAR.'
\end{quote}

Starlink's current view is that its software distribution is not `commercial
distribution' and so we can still distribute NCAR along with the rest of the
Starlink software. However, this position is under consideration and may
change.

The original source of NCAR is:
\begin{verse}
National Center for Atmospheric Research,\\
Scientific Computing Division\\
P O Box 3000,\\
Boulder,\\
Co 80307,\\
USA

Tel: 010-1-303-497-1000
\end{verse}

\subsection {SPAG}

{\em VMS only}

This software must only be installed on the STADAT Starlink node.
It must not be installed on other Starlink nodes, or be distributed to
non-Starlink sites.

SPAG is part of the PlusFORT suite from Polyhedron Software. For details see
FORCHECK.

\subsection {TOOLPACK}

{\em VMS only}

This software may be installed and used at any Starlink site but must not
be distributed to non-Starlink sites.
It may be purchased from NAG.

\subsection {Other licensed software}

Other licensed software used at Starlink sites (VMS, Fortran, PSI, DECnet,
Coloured Book, JTMP, Rabbit-7, Auditor, etc) is not part of the Starlink
Software Collection and distribution of such software to non-Starlink sites is
{\em not allowed}.
Please refer any questions regarding this software to the Starlink Network
Manager (John Sherman).

\section {References}

\begin{description}
\begin{description}
\item [SGP/20]: Starlink Software Management
\item [SGP/25]: Starlink Site Manager's Guide --- Major Nodes
\item [SGP/37]: Starlink Site Manager's Guide --- Minor Nodes
\item [SSN/15]: Starlink Software Installation
\item [SSN/41]: Starlink Software Changes
\item [SUG]: Starlink User Guide
\end{description}
\end{description}

\newpage

\appendix

\section {Letter from SERC Central Office}
\label{se:lsco}

This letter from SERC Central Office regarding Software Distribution was sent by
Peter Davies (SERC AII Secretary) to Keith Tritton (then Starlink Project
Scientist) on 12th May 1982:

\vspace{25mm}

\begin{quote}
\begin{center}
EXCHANGE OF STARLINK SOFTWARE
\end{center}

You will recall that I have been seeking the `official line' on how we should
tackle exchange and distribution of Starlink software.
SSAG took a very sensible line on this about 18 months ago but it was not
apparently cleared then with Finance Division here.
The step is required because public resources have been invested in the
software and the question of exploitation must be considered.

Happily the outcome recognises that scientific collaboration in modern astronomy
implies exchange of software.
Thus provided that the distribution of our software is done as part of a
scientific exchange whereby we also expect to benefit in the long term, it can
go ahead within the following guidelines:
\begin{itemize}
\item Licensed software is not distributed.
\item The recipients have a bona-fide astronomical interest and will not
 undertake any commercial exploitation of the software, and therefore will
 retain it for their own use and not themselves distribute it without the
 approval of Starlink management
\item Individual Starlink users seek the approval of Starlink management
 before distributing software developed at SERC's expense (i.e.\ to ensure that
 the two points above can be safeguarded).
\end{itemize}
If there is to be a formal agreement with any organisation then Finance Division
will need to be consulted and they will keep the interests of British Technology
Group\footnote{Now NRDC.} in mind.
This should be done through ASR Division, either by contacting me or Vince
Foley.
\end{quote}

\newpage

\section{Starlink Software --- Conditions of Supply}
\label{se:cons}

The {\em General Starlink Software Distribution Policy} is as follows:

\begin{quote}
Starlink is prepared to distribute Starlink software to anyone with a bona-fide
astronomical interest, within the following guidelines:

\begin{enumerate}

\item Only non-licensed software can be distributed, unless special permission
has been obtained to do so from the licensee.

\item The recipients must not undertake any commercial exploitation of the
software, and must not themselves distribute it without the agreement of
Starlink management.

\item Starlink does not undertake to support or maintain distributed software.
(There are exceptions to this for particular sites which are closely associated
with Starlink; namely the AAO, SAAO and some UK universities running Starlink
software on their own VAXs.)

\item Recipients of Starlink software must give the following details of their
site: full postal and computer network addresses, telephone, fax and/or telex
number, name of a personal contact, tape type and density preferred.

\item Anyone intending to distribute Starlink software must first obtain the
permission of the Starlink Software Librarian (Martin Bly), or the Site
Manager, and provide him  with details of the software and site involved.

\item Starlink software is supplied on an `as is' basis and is not guaranteed
for any particular purpose. The Starlink project does not offer any warranties
or representations, nor does it accept any liabilities with respect to the
supplied software.

\item Any site requesting the GKS software must complete and return to the
Starlink Software Librarian the GKS form shown in Appendix \ref{se:gdf}. 

\end{enumerate}

\end{quote}

Some Starlink software makes use of routines which have been linked with the
NAG libraries.
NAG have asked for the following paragraph to apply to such software:

\begin{quote}
`Fortran subroutines are included in this program by kind permission of the
Numerical Algorithms Group Ltd (NAG), Jordan Hill Road, Oxford, OX2 8DR who
retain all intellectual property rights to the routines. They are not to be
used separately from this program or any charge to be made for this program
without the express written permission of NAG.'
\end{quote}

\newpage

\section{GKS Declaration Form} 
\label{se:gdf} 
To be returned to the Starlink Software Librarian, at the address on 
page~\pageref{pa:add}.

\begin{footnotesize}
\begin{center}
{\bf The University or Institution receiving the software:}
\end{center}
\begin{verbatim}
      FROM: ..................................................................

            ..................................................................

            ..................................................................

      To: The Secretary (Ref: KWC/123855)       Copy to: Dr C D Osland
          National Research Development Corp.            Rutherford Appleton Lab
          101 Newington Causeway                         Chilton, DIDCOT, Oxon
          LONDON, SE1 6BU                                OX11 0QX
\end{verbatim}

Dear Sirs,
\begin{center}
COMPUTER SOFTWARE IN WHICH NRDC HAS EXPLOITATION INTERESTS
\end{center}
My Institution is interested in obtaining from the Rutherford Appleton
Laboratory (RAL) a copy of the computer software known as RAL-GKS which is
more particularly described below.

I understand that the software has been developed with financial support from
the Science and Engineering Research Council and that, consequently, the
National Research Development Corporation (NRDC) has an interest in its
commercial exploitation. In recognition of this, I agree that the programs and
other information (as described below) when received by my Institution will be
kept confidential and will be used by my Institution for research development
and educational purposes only and not for commercial purposes with or without
any alterations. The originals and any further copies made by me will be
returned on request. I also agree not to provide copies for use outside my
Institution and not to publish details of the programs without the consent of
NRDC. If any modifications are made to this software by anyone in this
Institution, the modified software including any new workstation handler or
device driver code will become part of the original software covered by this
undertaking and a free copy supplied as soon as possible to the Rutherford
Appleton Laboratory. I will advise any enquirers of this software to contact
NRDC quoting the above reference.

Yours faithfully
\begin{verbatim}
      Research Investigator ................................................... (signature)

      ............. (Date) ........................................ (Name - block capitals)

      University Secretary or Nominee ......................................... (signature)

      ............. (Date) ........................................ (Name - block capitals)

      for and on behalf of .................................................. (Institution)
\end{verbatim}
\begin{center}
DESCRIPTION OF PROGRAMS AND OTHER INFORMATION
\end{center}

The RAL-GKS software package is the software developed at the Rutherford
Appleton Laboratory by a team which has been supported jointly by the Science
and Engineering Research Council and International Computers Limited, as
described in the RAL-GKS User Guide dated May 1985. The software to be supplied
will consist of the object code prepared for a given machine range but will not
include the PERQ system interface nor the PERQ screen device handler modules.
Source code for workstation interface modules, and source code for an example
of a workstation handler may also be included.

\end{footnotesize}
\end{document}
