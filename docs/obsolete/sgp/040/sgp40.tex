\documentstyle{article} 
\pagestyle{myheadings}

%------------------------------------------------------------------------------
\newcommand{\stardoccategory}  {Starlink General Paper}
\newcommand{\stardocinitials}  {SGP}
\newcommand{\stardocnumber}    {40.1}
\newcommand{\stardocauthors}   {J C Sherman}
\newcommand{\stardocdate}      {28 August 1990}
\newcommand{\stardoctitle}     {The Role of Wish-lists in Starlink's Purchasing}
%------------------------------------------------------------------------------

\newcommand{\stardocname}{\stardocinitials /\stardocnumber}
\renewcommand{\_}{{\tt\char'137}}     % re-centres the underscore
\markright{\stardocname}
\setlength{\textwidth}{160mm}
\setlength{\textheight}{240mm}
\setlength{\topmargin}{-5mm}
\setlength{\oddsidemargin}{0mm}
\setlength{\evensidemargin}{0mm}
\setlength{\parindent}{0mm}
\setlength{\parskip}{\medskipamount}
\setlength{\unitlength}{1mm}

%------------------------------------------------------------------------------
% Add any \newcommand or \newenvironment commands here
%------------------------------------------------------------------------------

\begin{document}
\thispagestyle{empty}
SCIENCE \& ENGINEERING RESEARCH COUNCIL \hfill \stardocname\\
RUTHERFORD APPLETON LABORATORY\\
{\large\bf Starlink Project\\}
{\large\bf \stardoccategory\ \stardocnumber}
\begin{flushright}
\stardocauthors\\
\stardocdate
\end{flushright}
\vspace{-4mm}
\rule{\textwidth}{0.5mm}
\vspace{5mm}
\begin{center}
{\Large\bf \stardoctitle}
\end{center}
\vspace{5mm}

%------------------------------------------------------------------------------
%  Add this part if you want a table of contents
%  \setlength{\parskip}{0mm}
%  \tableofcontents
%  \setlength{\parskip}{\medskipamount}
%  \markright{\stardocname}
%------------------------------------------------------------------------------

\section  {Introduction}

Wish-lists are the lists of desirable node enhancements drawn
up by Area Management Committees (AMCs) at each AMC meeting and
documented as part of the AMC minutes.  The wish-lists include new items
of hardware and software and also replacements for worn out or obsolete
hardware and software.  They are usually lists of hardware items, but
proprietary software, for example, NAG products, Rabbit products or IDL, can
also be included.  Desirable enhancements to Starlink's applications software
are best dealt with via the Special Interest Groups (SIGs).
 
Where there are several Minor Nodes in an AMC's catchment area
and when the wish-list includes replacements as well as new items,
it is difficult to construct a list in strict priority order.
Nevertheless, some indication of priorities enhances the value of the list.
 
The purpose of this paper is to clarify the role played by wish-lists
in Starlink's purchasing procedures, in particular the influence wish-lists
have on the choice of what to buy next.  It is expected that this paper will
be of interest to Site Managers, AMC members and those users who attend
SLUG meetings.
 
The role of wish-lists has not been documented hitherto.  However, it has
become clear recently that some sites have misunderstood how wish-lists
are used at present --- hence
the need for this paper.  This paper documents what has been the situation for
several years; it does not announce any recent change of policy.  It is of
historical interest to note that wish-lists were first introduced as contingency
planning, in preparation for a possible ``windfall'' increase in Starlink's
funds at the end of a financial year.  Initially, wish-lists did not have any
role in Starlink's normal purchasing plans but their importance
has grown with time.
 
\section  {The Present Role of Wish-lists}

One of Starlink's primary aims is to provide a computing service
which is relevant
to the needs of Starlink's users.  Consequently, the users' views of
enhancements
they would like to see at their node, as formulated in AMC wish-lists,
is very important information.  However, wish-lists are not the only
factor which influences or constrains purchasing decisions.
 
Taking these other factors into account will often have the effect that
wish-list items are not purchased in the same order as the wish-list
is presented.  They may also, occasionally, have the effect that a
wish-list item is delayed or not purchased at all.
 
Among the other factors are the following:
\begin{itemize}
\item
Strategic changes, as agreed by the Starlink Users' Committee (SUC), must be
implemented.  Recent examples of this are (a) the replacement of all
VAX 780s and VAX 750s and (b) the move towards workstations and X-windows.
\item
``Special Offers'' become available from time to time which Starlink
may wish to exploit.  These cannot be anticipated in wish-lists.
\item
Upgrades sometimes result in valuable equipment being displaced and
Starlink may wish to reuse this equipment at another site.
In addition, old equipment is replaced from time to time
to reduce maintenance costs.  Hardware compatibility requirements
sometimes make it impossible to make these activities completely
consistent with the wish-lists.
\item
Some wish-list items may be for types of equipment which Starlink has
not previously purchased.  There will then be a delay while a
suitable product is identified.  A recent example of this is
A4 colour hardcopy, which has been purchased after the Hardware
Advisory Group (HAG) considered alternative devices.
\item
AMCs may be unaware of the situation at other Starlink nodes and may
adopt different criteria from others when considering their own requirements.
For example, a level of disc space provision which one AMC may regard as
tight but acceptable may be regarded by another AMC as totally
inadequate.  One consequence of this ``calibration'' difference between AMCs
is that some wish-list items can be delayed or not purchased at all.
\end{itemize}
Wish-lists and minutes of AMC meetings are filed at RAL and are
referred to when upgrades for a particular node are considered.
It is important that Site Managers ensure that RAL has an up-to-date
wish list (and an up-to-date configuration diagram).  Changes
to the wish-lists can be made at any time but in practice they are usually
made twice a year, following AMC meetings.  To facilitate rapid reference,
Starlink has started to store wish-lists on-line at RAL.
 
To guard against wish-lists being out-of-date,
most proposed purchases are discussed with
Site Managers before any orders are placed.  Any purchase
which is not on the wish-list is always discussed with the site concerned.
 
\section {Misconceptions}
 
A common misconception is that Starlink sites have an earmarked share
of Starlink's budget each year and that Starlink's job is to spend that share
on the particular site's wish-list, starting at the top and working down
until all the funds are spent.  This is quite wrong in two respects.  Firstly,
the share-per-site-each-year approach would make major changes impossible ---
by dividing the available funds between a large number of sites there would
not be sufficient at any one site to make the fundamental changes that are
required from time to time.  To overcome this problem, funds can be
concentrated on a subset of sites in one year with other sites benefitting
in subsequent years.
An example of this approach is the replacement of the VAX 780s and 750s;
when the first batch of VAX 780 replacements were made, the sites
concerned absorbed most of the funds available for hardware purchases that
year with very little left for any other sites.  
Secondly, the factors noted in section 2 make it impossible
to start at the top of a wish-list and work down.
 
\section {Timing}
 
AMC meetings and wish-list updates usually take place twice a year.
However, because SERC finances are arranged on a strict financial year by
financial year basis, as dictated by Treasury rules, the pattern of Starlink's
purchasing tends to be year by year.  AMCs should not, therefore, be surprised
to find that an item added to their wish-list six months previously
has not been purchased by the time they next review their wish-list.
A time scale of six months to a year for Starlink enhancements via the
wish-lists is similar to to the time scale of SERC grant funding, measured
from application drafting to equipment installation.
 
On the other hand, an item that has been on the wish-list for
two years or more with no sign of action is either an oversight or suffers
from a fundamental problem, such as not being consistent with Starlink
hardware or software policy.  Site Managers, AMCs and users should bring
such items to Starlink's attention at AMC meetings and in the AMC minutes,
or, between meetings, via e-mail.
\end{document}
