\documentstyle{article}
\pagestyle{myheadings}

%------------------------------------------------------------------------------
\newcommand{\stardoccategory}  {Starlink General Paper}
\newcommand{\stardocinitials}  {SGP}
\newcommand{\stardocnumber}    {20.6}
\newcommand{\stardocauthors}   {M D Lawden}
\newcommand{\stardocdate}      {16 November 1990}
\newcommand{\stardoctitle}     {Starlink Software Management}
%------------------------------------------------------------------------------

\newcommand{\stardocname}{\stardocinitials /\stardocnumber}
\markright{\stardocname}
\setlength{\textwidth}{160mm}
\setlength{\textheight}{240mm}
\setlength{\topmargin}{-5mm}
\setlength{\oddsidemargin}{0mm}
\setlength{\evensidemargin}{0mm}
\setlength{\parindent}{0mm}
\setlength{\parskip}{\medskipamount}
\setlength{\unitlength}{1mm}

\begin{document}
\thispagestyle{empty}
SCIENCE \& ENGINEERING RESEARCH COUNCIL \hfill \stardocname\\
RUTHERFORD APPLETON LABORATORY\\
{\large\bf Starlink Project\\}
{\large\bf \stardoccategory\ \stardocnumber}
\begin{flushright}
\stardocauthors\\
\stardocdate
\end{flushright}
\vspace{-4mm}
\rule{\textwidth}{0.5mm}
\vspace{5mm}
\begin{center}
{\Large\bf \stardoctitle}
\end{center}
\vspace{5mm}

The term {\em Starlink Software} is ambiguous.
In its broadest sense, it refers to all the software on Starlink machines.
In this paper the term is used in a narrower sense; it refers to a formally
defined set of software which is looked after by a single individual.
This set of software is called the {\em Starlink Software Collection}, and the
individual who looks after it is called the {\em Starlink Software Librarian}.
The abbreviations `Collection' and `Librarian' will be used in the rest of this
paper.

The overall development of Starlink Software is planned by the Project Manager
in conjunction with the Head of Applications, the Project Scientist, the
Starlink Users' Committee (SUC), and the Special Interest Groups (SIG).
This paper deals only with the management of the Collection and does not
consider the overall software development problem (see SGP/13).

The Librarian is responsible for the management of the Collection.
This includes the following aspects:
\begin{itemize}
\item Directory organisation
\item Installation
\item Maintenance, in conjunction with software supporters
\item Distribution
\item Security
\item Bug recording and forwarding to software supporters
\item Documentation standards and organisation
\end{itemize}
The current identity of the Librarian is specified in the Starlink Who's Who
(ADMIN\-DIR:\-WHOS\-WHO.\-LIS).

The items comprising the Collection are specified in the Starlink Software
Index (ADMINDIR:SSI.LIS).
The Librarian also maintains various information summaries of the software,
such as a directory map, functional analysis, and size analysis, which can be
obtained from him on demand.

The Collection is divided into six sets:
\begin{description}
\item [Standard set]:
This is a kernel of software items which should be installed at every Starlink
node.
It is also the software which is sent to a non-Starlink site which requests the
`Starlink software'.
It includes the directory [STARLINK] which holds small items that can be
distributed on request, and also administrative information such as lists of
users and document indexes.
\item [Option set]:
This comprises software items which can be installed at Starlink nodes as
options if required.
They tend to be large and specialised and are stored outside the [STARLINK]
directory in their own top level directories.
They are only distributed on request.
Some are subject to licence restrictions (see SGP/21).
\item [Restricted set]:
This comprises software items which can only be installed on specific Starlink
nodes; usually only on STADAT, the central data and software facility.
The restriction is usually part of a licence agreement.
\item [Secure set]:
This comprises software items which are concerned with system management and
are not required by general users.
\item [Devolved set]
This comprises important software items (usually obtained from overseas) which
are important to some Starlink users but which are not directly supported by
Starlink.
The management, maintenance, distribution, and support is devolved to the
individual or organisation specified in the Starlink Software Index.
They are listed as part of the Collection in order to make them `visible'.
\item [Data set]:
These are large data files, such as star catalogues, which are stored on STADAT
but which may also be installed locally if desired.
\end{description}
In addition, each Starlink node may install its own software or variants of
Starlink software and index and manage it in a similar fashion.
Some of these items may be stored in a [STARLOCAL] directory having a similar
structure to [STARLINK] but containing files which are local to that site.
This enables people to install local modifications of Starlink software
without having to alter the released files.
The items comprising the local software at each site should be specified in
a Local Software Index (LADMINDIR:SSI.LIS).

The term {\em software item} is used to refer to a piece of software which is
described by a single entry in the Starlink Software Index.
An item can range from a single program to a large package.
Its distinguishing characteristic is that it is fairly autonomous and can be
used and managed as a single entity.
The concept is really just a management tool, and large packages can contain
programs or facilities that a user may regard as a separate item, even when
they are not classified as such by Starlink.
Ultimately, the Librarian decides what is or is not regarded as an item.

Separate items can be related in the sense that one item can make use of another
item, but each item should have some independent function.
Unfortunately, the interrelationships between items are not fully documented, so
no guarantee can be given that subsets of the Collection will work properly.

It is Starlink policy to restrict the Collection to a manageable size.
In particular, each item should be of wide use and duplication of items
which do the same sort of thing (e.g.\ many different graphics systems) is
discouraged.
Nevertheless, the Collection continues to grow in size fairly rapidly.

Several Starlink papers are available which deal with various aspects of
Starlink software.
The following provide a general background to Starlink software development:
\begin{description}
\item [SGP/13]: Starlink applications software development
\item [SGP/14]: Starlink Special Interest Groups
\item [SGP/16]: Starlink applications programming standards
\item [SGP/21]: Starlink software distribution policy
\item [SGP/25]: Starlink Site Manager's Guide --- Major Nodes
\item [SGP/28]: Starlink documentation production
\item [SGP/37]: Starlink Site Manager's Guide --- Minor Nodes
\item [SUN/1]: Starlink software --- An introduction
\end{description}
The following deal with specific details of the management of Starlink software:
\begin{description}
\item [SGP/19]: Starlink software submission
\item [SSN/15]: Starlink software installation
\item [SSN/41]: Starlink software changes
\end{description}
\end{document}
