\documentclass[twoside,11pt]{article}
\pagestyle{myheadings}

% -----------------------------------------------------------------------------
\newcommand{\stardoccategory}  {Starlink General Paper}
\newcommand{\stardocinitials}  {SGP}
\newcommand{\stardocnumber}    {44.1}
\newcommand{\stardocsource}    {sgp\stardocsource}
\newcommand{\stardocauthors}   {R.F. Warren-Smith}
\newcommand{\stardocdate}      {7th June 1995}
\newcommand{\stardoctitle}     {Guidelines for Starlink\\
                                Software Strategy Groups}
% -----------------------------------------------------------------------------

\newcommand{\stardocname}{\stardocinitials /\stardocnumber}
\markright{\stardocname}
\setlength{\textwidth}{160mm}
\setlength{\textheight}{230mm}
\setlength{\topmargin}{-2mm}
\setlength{\oddsidemargin}{0mm}
\setlength{\evensidemargin}{0mm}
\setlength{\parindent}{0mm}
\setlength{\parskip}{\medskipamount}
\setlength{\unitlength}{1mm}

% -----------------------------------------------------------------------------
% Hypertext definitions.
% These are used by the LaTeX2HTML translator in conjuction with star2html.

% Comment.sty: version 2.0, 19 June 1992
% Selectively in/exclude pieces of text.
%
% Author
%    Victor Eijkhout                                      <eijkhout@cs.utk.edu>
%    Department of Computer Science
%    University Tennessee at Knoxville
%    104 Ayres Hall
%    Knoxville, TN 37996
%    USA

%  Do not remove the %\begin{latexonly} and %end{latexonly} lines (used by
%  star2html to signify raw TeX that latex2html cannot process).
%begin{latexonly}
\makeatletter
\def\makeinnocent#1{\catcode`#1=12 }
\def\csarg#1#2{\expandafter#1\csname#2\endcsname}

\def\ThrowAwayComment#1{\begingroup
    \def\CurrentComment{#1}%
    \let\do\makeinnocent \dospecials
    \makeinnocent\^^L% and whatever other special cases
    \endlinechar`\^^M \catcode`\^^M=12 \xComment}
{\catcode`\^^M=12 \endlinechar=-1 %
 \gdef\xComment#1^^M{\def\test{#1}
      \csarg\ifx{PlainEnd\CurrentComment Test}\test
          \let\html@next\endgroup
      \else \csarg\ifx{LaLaEnd\CurrentComment Test}\test
            \edef\html@next{\endgroup\noexpand\end{\CurrentComment}}
      \else \let\html@next\xComment
      \fi \fi \html@next}
}
\makeatother

\def\includecomment
 #1{\expandafter\def\csname#1\endcsname{}%
    \expandafter\def\csname end#1\endcsname{}}
\def\excludecomment
 #1{\expandafter\def\csname#1\endcsname{\ThrowAwayComment{#1}}%
    {\escapechar=-1\relax
     \csarg\xdef{PlainEnd#1Test}{\string\\end#1}%
     \csarg\xdef{LaLaEnd#1Test}{\string\\end\string\{#1\string\}}%
    }}

%  Define environments that ignore their contents.
\excludecomment{comment}
\excludecomment{rawhtml}
\excludecomment{htmlonly}

%  Hypertext commands etc. This is a condensed version of the html.sty
%  file supplied with LaTeX2HTML by: Nikos Drakos <nikos@cbl.leeds.ac.uk> &
%  Jelle van Zeijl <jvzeijl@isou17.estec.esa.nl>. The LaTeX2HTML documentation
%  should be consulted about all commands (and the environments defined above)
%  except \xref and \xlabel which are Starlink specific.

\newcommand{\htmladdnormallinkfoot}[2]{#1\footnote{#2}}
\newcommand{\htmladdnormallink}[2]{#1}
\newcommand{\htmladdimg}[1]{}
\newenvironment{latexonly}{}{}
\newcommand{\hyperref}[4]{#2\ref{#4}#3}
\newcommand{\htmlref}[2]{#1}
\newcommand{\htmlimage}[1]{}
\newcommand{\htmladdtonavigation}[1]{}

% Define commands for HTML-only or LaTeX-only text.
\newcommand{\html}[1]{}
\newcommand{\latex}[1]{#1}

% Use latex2html 98.2.
\newcommand{\latexhtml}[2]{#1}

% Starlink cross-references and labels.
\newcommand{\xref}[3]{#1}
\newcommand{\xlabel}[1]{}

%  Define commands to recentre underscore for Latex and leave as normal
%  for HTML (problems with _ in tabbing environments and __ generally
%  otherwise).
\newcommand{\setunderscore}{\renewcommand{\_}{{\tt\symbol{95}}}}
\latex{\setunderscore}

%  LaTeX2HTML symbol.
\newcommand{\latextohtml}{{\bf LaTeX}{2}{\tt{HTML}}}

% -----------------------------------------------------------------------------
%  Debugging.
%  =========
%  Un-comment the following to debug links in the HTML version using Latex.

% \newcommand{\hotlink}[2]{\fbox{\begin{tabular}[t]{@{}c@{}}#1\\\hline{\footnotesize #2}\end{tabular}}}
% \renewcommand{\htmladdnormallinkfoot}[2]{\hotlink{#1}{#2}}
% \renewcommand{\htmladdnormallink}[2]{\hotlink{#1}{#2}}
% \renewcommand{\hyperref}[4]{\hotlink{#1}{\S\ref{#4}}}
% \renewcommand{\htmlref}[2]{\hotlink{#1}{\S\ref{#2}}}
% \renewcommand{\xref}[3]{\hotlink{#1}{#2 -- #3}}
%end{latexonly}
% -----------------------------------------------------------------------------
% Add any document-specific \newcommand or \newenvironment commands here

\newcommand{\planref}[1]{\htmladdnormallink{#1}{http://www.starlink.ac.uk/\~{}rfws/projects/planitems.html}}
\newcommand{\projectsref}[1]{\htmladdnormallink{#1}{http://www.starlink.ac.uk/\~{}rfws/projects/index.html}}
\newcommand{\sitesref}[1]{\htmladdnormallink{#1}{http://www.starlink.ac.uk/sites.html}}
\newcommand{\apref}[1]{\htmladdnormallink{#1}{http://www.starlink.ac.uk/programmers.html}}
\newcommand{\staffref}[1]{\htmladdnormallink{#1}{http://www.starlink.ac.uk/pro.html}}
\newcommand{\ralref}[1]{\htmladdnormallink{#1}{http://www.clrc.ac.uk/ral/index.html}}
\newcommand{\cclref}[1]{\htmladdnormallink{#1}{http://www.clrc.ac.uk/}}

\newcommand{\st}[1]{{\em{#1}}}
\newcommand{\qt}[1]{``#1''}
\begin{htmlonly}
   \newcommand{\qt}[1]{{\tt{"}}#1{\tt{"}}}
\end{htmlonly}
% -----------------------------------------------------------------------------
%  Title Page.
%  ===========
\begin{document}
\thispagestyle{empty}

%  Latex document header.
\begin{latexonly}
   CCL / {\sc Rutherford Appleton Laboratory} \hfill {\bf \stardocname}\\
   {\large Particle Physics \& Astronomy Research Council}\\
   {\large Starlink Project\\}
   {\large \stardoccategory\ \stardocnumber}
   \begin{flushright}
   \stardocauthors\\
   \stardocdate
   \end{flushright}
   \vspace{-4mm}
   \rule{\textwidth}{0.5mm}
   \vspace{5mm}
   \begin{center}
   {\Large\bf \stardoctitle}
   \end{center}
   \vspace{5mm}

%  Add heading for abstract if used.
%   \vspace{10mm}
%   \begin{center}
%      {\Large\bf Description}
%   \end{center}
\end{latexonly}

%  HTML documentation header.
\begin{htmlonly}
   \xlabel{}
   \label{stardoctoppage}
   \begin{rawhtml} <H1> \end{rawhtml}
      \stardoctitle
   \begin{rawhtml} </H1> \end{rawhtml}

%  Add picture here if required.

   \begin{rawhtml} <P> <I> \end{rawhtml}
   \stardoccategory\ \stardocnumber \\
   \stardocauthors \\
   \stardocdate
   \begin{rawhtml} </I> </P> <H3> \end{rawhtml}
      \htmladdnormallink{CCL}{http://www.clrc.ac.uk} /
      \htmladdnormallink{Rutherford Appleton Laboratory}
                        {http://www.clrc.ac.uk/ral} \\
      Particle Physics \& Astronomy Research Council \\
   \begin{rawhtml} </H3> <H2> \end{rawhtml}
      \htmladdnormallink{Starlink Project}{http://www.starlink.ac.uk/}
   \begin{rawhtml} </H2> \end{rawhtml}
   \htmladdnormallink{\htmladdimg{source.gif} Retrieve hardcopy}
      {http://www.starlink.ac.uk/cgi-bin/hcserver?\stardocsource}\\

%  Start new section for abstract if used.
%  \section{\xlabel{abstract}Abstract}

\end{htmlonly}

% -----------------------------------------------------------------------------
%  Document Abstract. (if used)
%  ==================
% -----------------------------------------------------------------------------
%  Table of Contents. (if used)
%  ==================
\begin{htmlonly}
   \htmladdtonavigation{\htmlref{\htmladdimg{contents_motif.gif}}
                                            {stardoctoppage}}
\end{htmlonly}
% \begin{latexonly}
%    \setlength{\parskip}{0mm}
%    \tableofcontents
%    \setlength{\parskip}{\medskipamount}
%    \markright{\stardocname}
% \end{latexonly}
% -----------------------------------------------------------------------------

\section{\xlabel{introduction}INTRODUCTION}

Software Strategy Groups (SSGs) are Starlink user groups which have
been set up by the Project to allow it to consult with expert software
users and to obtain advice about those software issues which affect
the use of its computing service.  The SSGs also form an important
part of a \htmlref{procedure}{sect:procedure} for annually
re-assessing the software priorities of the Project and for
identifying future objectives.

This document describes the nature of the SSGs and the role which they
play in this procedure. It also covers some \htmlref{administrative
arrangements}{sect:admin}.  Its primary purpose is to give
\htmlref{guidance}{sect:guidance} to those participating in SSGs, so
that they can operate as effectively as possible.

\section{\xlabel{historical}HISTORICAL BACKGROUND}

The formation of the SSGs arose out of the recommendations of a major
review of Starlink which took place during 1991/92 under the
Chairmanship of Prof.\ Peter Wilmore of Birmingham University.  In its
report, the Wilmore Review examined the role of the \st{Special
Interest Groups} (SIGs) which had until then advised Starlink on its
applications programming policy. The review identified a number of
deficiencies in the way the SIGs operated and asked the Starlink
Panel (also set up as a result of this review) to examine their future
role.

The Starlink Panel subsequently asked the Project to prepare a policy
paper proposing new ways of identifying applications software
priorities and managing the associated software development by
\apref{contract programmers} located at \sitesref{Starlink
sites}. This paper was discussed and approved at the July 1993 meeting
of the Panel.

In making these proposals, the Project examined the way that the SIGs
had operated in great detail. It shared the view of the Wilmore Review
that there was scope for reform, but also felt that some of the SIGs'
functions were very important to Starlink and should be preserved in
some manner. It also identified some new requirements that had not
previously been recognised.

As a result, a considerable number of changes have been introduced.
These include the disbanding of the SIGs and the formation of a
smaller number of SSGs with a somewhat \htmlref{different
purpose}{sect:comparison}. The intention has been to transfer to
others those functions which the SIGs were not able to perform well,
but to retain in the SSGs their most useful function of allowing the
Project to consult with expert software users.

In transferring responsibilities to others, the Project was aware that
each group of people will have different perspectives and priorities,
and that their advice will therefore have different strengths and
weaknesses. A software review \htmlref{procedure}{sect:procedure} was
therefore designed to ensure that contributions from different
quarters are balanced and complementary.

\section{\xlabel{review_procedure}\label{sect:procedure}THE SOFTWARE REVIEW PROCEDURE}

The procedure for reviewing and updating Starlink's software strategy
consists of a number of steps which are intended to be repeated
annually, as follows:

\begin{enumerate}
\item A {\bf Software Questionnaire,}\footnote{Note that a full
questionnaire may not be carried out every year if
previously-collected data is still relevant. The intention is simply
to keep our information up to date by asking appropriate questions as
they become necessary.}  circulated to all users, will sample opinion
and identify those trends and problems that affect substantial numbers
of users.  It will also measure software usage and identify which
users are interested in particular types of software. These results
will be a valuable resource in planning future software strategy and
will also facilitate more direct contact with interested users,
allowing them to be targeted when advice is needed on more detailed
matters that do not warrant a place in the questionnaire itself.

\begin{quote}
\st{Note: It is expected that these results will most usefully reveal
deficiencies in the existing service and that those replying to the
questionnaire will not usually examine future options in much
detail. Nevertheless, the questionnaire is likely to be as
\qt{unbiassed} a sample of user opinion as can easily be obtained. The
questionnaire results and the problems and preferences it identifies
should therefore be given considerable weight and form a background to
all SSG discussions.}
\end{quote}

\item An {\bf Open Meeting} will provide an opportunity to
describe the Project's status and report on major developments. It
will also allow early results from the questionnaire to be presented,
with face-to-face discussion to clarify particular points.

\item The {\bf Software Strategy Groups} will be asked to provide
advice to the Project on the strategy that should be adopted over the
following 3 years in order to satisfy (and continue to satisfy) the
needs identified by users in the questionnaire, having regard for the
numbers of users involved. They will also be asked to identify in more
detail those software projects that should be carried out over the
next year in order to implement the strategy.

\begin{quote}
\st{Note: SSG discussions are expected to complement the questionnaire
results by being more forward-looking, examining future options in
some detail and relating it to actual work required. At the same time,
SSGs must recognise the same objectives as the Project -- those of
providing the best possible computing service given the resources
available by satisfying the needs expressed by typical users in the
questionnaire. Where these two perspectives produce conflicting
advice, the Project and the Starlink Panel will need to ensure
balance.}
\end{quote}

\item The Project will also invite similar input from {\bf other
interested groups or individuals} in the astronomical community.  While
the SSGs are the bodies \qt{officially sponsored} to provide this
information, this should not be considered their right
alone. Additional input might be valuable, for instance, where
particular specialised requirements cannot be well represented by the
broader interests of the SSGs, or where an SSG has failed to
adequately address the needs of typical users.

\begin{quote}
\st{Note: To some extent this is an insurance against SSGs which do
not function well (their input may be compared with that from other
sources and discounted if better advice is available) but it also
provides an opportunity to broaden the range of options considered, in
case anything promising has been overlooked.}
\end{quote}

\item The {\bf Starlink Project} will be responsible for producing a
balanced and coherent overall \xref{strategy}{sgp42}{} for Starlink
software work over the next 3 years, and a more detailed
\planref{plan} for the first 12 months of that period. This will be
based on the questionnaire results and all the input received
subsequently, and will attempt to strike a balance between the advice
received from different quarters. It will also include contributions
from the Project on matters where it has expertise (for example,
technical feasibility, new software and hardware products,
\st{etc.}). Any software needs that have been overlooked (because they
do not fall clearly within the domain of any particular group, for
instance) will also be included at this stage. The strategy will be
designed to meet all the constraints imposed by available resources
(particularly manpower) and will identify timescales whenever
possible.

\begin{quote}
\st{Note: Apart from adding its own expertise, the Project's main
problems here are likely to be in resolving conflicting advice and in
balancing users' expectations against the resources available. It is,
of course, probable that a good deal of what is asked for will not be
achievable, and the Project will aim to make it clear where goals
cannot be met because of resource constraints.}
\end{quote}

\item The {\bf Starlink Panel} will discuss the proposed strategy, which
will be presented along with all the supporting input. The Panel will be
responsible for making any changes it feels are necessary and for
approving it along with appropriate levels of manpower to carry it out.

\item The final strategy will be \projectsref{\bf published} so that
all users can see what software work is to be conducted in the next
year and what the longer-term objectives are. They will also be able
to see what work will not be carried out.

\item Starlink's \apref{\bf Applications Programmers} (and
software \staffref{staff} at \ralref{RAL}, when appropriate) will be
responsible for carrying out the software work, under the management
of the Starlink Project which will supervise the scheduling of work to
ensure that timescales and quality standards are met.  The Project
will also be responsible for ensuring that its programming staff
maintain adequate contact with users and any experts needed to advise
them during their work. This may include e-mail contact based on
address information obtained via the questionnaire, and visits and
technical meetings arranged to tackle particular problems, according
to the requirements of each project.

\begin{quote}
\st{Note: In the interests of flexibility, Starlink will retain the
ability to depart from its published strategy where unanticipated
developments require it to do so. If such changes are significant they
will be raised with the Starlink Panel who may wish to discuss them
further.}
\end{quote}

\item The success of the work carried out in previous years will be
reviewed annually at each stage of this procedure, and appropriate
changes made.  If it appears necessary (as a result of the
questionnaire, for example), the composition of the SSGs and the
subjects they cover may be modified so that they continue to reflect
user opinion.

\end{enumerate}

\section{\xlabel{being_effective}\label{sect:guidance}HOW CAN SSGS WORK MOST EFFECTIVELY?}

The following points are intended to help new SSG members by outlining
the approaches they can take to help the SSGs function most
effectively:

\begin{description}

\item [{\bf Be Informal.}] One of the main purposes of the SSGs is to
provide an opportunity for the Project to consult with expert software
users and to exchange ideas and share experience and expertise. The
idea is to reach a consensus about where we are going in the future
and the steps needed to get there. This is likely to work best if SSG
meetings have a \qt{workshop} atmosphere and a flexible agenda so that
a wide range of possibilities can be explored before deciding on the
options that need pursuing.

\item[{\bf Be Well Informed.}] Starlink needs to stay well informed
about all kinds of developments that may affect its plans. Having
dozens of SSG members looking out for relevant information is a good
way of doing this. If you see or hear of anything interesting, then be
prepared to raise it at an SSG meeting. It may be common knowledge
to some people, but it could be news to others. Many things are
potentially interesting: new instrumentation, software, hardware,
projects, staff, data, books, \st{etc.}, \st{etc.}

\item[{\bf Look Ahead.}] A particular responsibility of the SSGs is to
look ahead at what might be needed in future (a minimum of 3 years is
suggested). We all know that computing needs and expectations change
rapidly and that prediction is an uncertain business, but SSG members
should be in a far better position to look ahead in many areas of
astronomy than are members of the Project.  Anticipated increases or
decreases in data processing requirements as a result of
instrumentation developments are a good example, and of considerable
practical importance.

This is a newly-introduced responsibility, but a specific and
important one. Software takes a long time to develop and major
projects must be planned in plenty of time.

\item[{\bf Be Responsible.}] When planning for the future, remember
that Starlink already has a large amount of software in regular use by
many astronomers and that they may not be happy if we simply abandon
them to free up manpower for new projects, or ditch their software and
start again! The problem we have to address is not simply identifying
what we want in future, it also means planning how to get there from
where we are now, and doing so within the resources available.

\item[{\bf Be Specific.}] Be prepared to identify the results that are
needed from a particular software project. Avoid vague terms like
\qt{support package X} which often covers a multitude of unconvincing
possibilities and will, quite rightly, be viewed with suspicion by
others. Instead, be prepared to say what needs doing to it in
sufficient detail for the cost in terms of manpower to be assessed.
The value in terms of scientific productivity should also be
addressed.

\item[{\bf Be Realistic.}] There are usually many more things we would
like to do to software than can realistically be achieved --
over-optimism is a major cause of failure in software projects. This
means we cannot always strive for perfection and may frequently have
to accept compromise solutions.  Be prepared to identify the things
that really count and examine what would happen if they didn't get
done. Make sure you have a backup plan.

\end{description}

\section{\xlabel{admin_arrangements}\label{sect:admin}ADMINISTRATIVE ARRANGEMENTS}

\subsection{Membership}

SSGs will normally consist of around 8 to 10 people, including a Chair
and a Secretary.

Membership is open to all registered Starlink users (but also see
below) and will normally be selected by the Chair, subject to
agreement with the Project. Members should, as a rule, be drawn from a
variety of \sitesref{Starlink sites} and be capable of representing a
wide range of opinion and expertise.

To encourage new users to participate, SSG membership will normally
last for two years. However, contributions for a shorter period (even
just a single meeting) are equally welcome. Particular individuals may
be asked to stay on for a further year if this is needed to help stagger
membership changes.

As far as possible, members should be \qt{typical users} of the
software under consideration. Where they possess special expertise (as
many will), \st{it is expertise in research (\st{i.e.}\ astronomy) and
the associated extensive and/or demanding use of software} which
should be preferred over technical competence in its design or
implementation.

For this reason, \staffref{Starlink Project staff} and their
contractors should not normally be members of SSGs, since they would
otherwise displace more typical users. Users who are primarily
providers of software may also be discouraged from SSG membership for
the same reason. Such people should, however, be encouraged to attend
SSG meetings as invited experts where their technical knowledge will
assist the SSG in its discussions (normally this would be at the
invitation of either the Chair or the Project).

So as to involve as many people as possible, users may be discouraged
from joining more than one SSG at the same time unless there are
particularly good reasons for doing so.

Chairs of SSGs are requested to inform the Project of all changes of
membership as soon as possible so that records can be kept up to date.

\subsection{Meetings}

SSGs should meet once or twice per year -- those that do not may be
disbanded.

The Chair is responsible (with the Secretary's assistance) for
convening meetings and selecting a suitable time and location for
them. It is expected that they will normally be held at a convenient
Starlink site. Meeting arrangements should be made by the Secretary
working with the Chair. Care should be taken to ensure that travel to
meetings is convenient for those attending, and that travel costs are
no higher than necessary.

When planning a meeting, the Chair or Secretary should contact the
Project in the first instance to:
\begin{itemize}
\item Agree the meeting (this is simply a formality, as expenditure may be
involved), and

\item Ensure there is no clash with other Starlink events.
\end{itemize}

\subsection{Keeping Records}

The SSGs will provide formal input to the Project in their annual
software strategy proposal, and this should document the outcome of
the discussions which take place at SSG meetings.  Because of this,
and to keep meetings relatively informal, the Project does not wish to
impose a further burden of keeping detailed minutes where this would
lead to unnecessary effort and duplication.

SSGs should discuss how they wish to keep records of their meetings
and the possible audience these might have. If the dissemination of
information to other Starlink users is felt to be important, then a
suitable format (such as the World Wide Web) should be considered.

\subsection{Expenses}

In most cases (see below), Starlink will reimburse necessary expenses
for SSG members and invited experts attending meetings. For this
purpose, you should retain all relevant receipts and make a claim on a
standard \ralref{RAL} visitors' claim form,\footnote{These forms are
identified by a large \qt{V} at the top. Please do not use \qt{staff}
claim forms (identified by an \qt{S}) or payment of your claim may be
delayed.}  which should then be sent to:

\begin{quote}
Andrea Roberts,\\
Starlink Project,\\
Rutherford Appleton Laboratory,\\
Chilton, DIDCOT,\\
Oxon OX11 0QX
\end{quote}

Claim forms should be available at SSG meetings and many Starlink Site
Managers also keep a stock. If all else fails, e-mail to any
\staffref{Project member} at RAL will secure a supply.

Different arrangements exist, however, for the following people:

\begin{itemize}

\item {\bf Employees of PPARC establishments,} whose expenses should
normally be paid by their own establishment.

\item {\bf Starlink contractors,} who should claim from their home
institution (who will, in turn, invoice Starlink).

\item {\bf \cclref{CCL} staff,} who should normally submit a claim
against their own project's funds.

\end{itemize}

\section{\xlabel{annual_input}ANNUAL INPUT TO THE PROJECT}

SSGs will normally be asked to provide written input to the Project
once a year at the end of August. The following checklist outlines the
items that should be included in this submission:

\begin{enumerate}
\item The name of your SSG and the date of your submission.

\item A full list of SSG members, indicating if any have not
contributed (for example, some people may have missed a vital
meeting).

\item A description of the main developments you expect to take place
over the next 3 years in your subject area, and the impact you
anticipate these having on data processing software requirements.

\item A broad strategy, covering at least the next 3 years, that will
allow your expected software requirements to be met, given anticipated
resources. For instance, you should identify aspects of Starlink's
software provision that will need to be developed, and may also need
to identify activities that should be scaled down to compensate. In
doing this, you may want to refer to Starlink's overall Software
Strategy document (\xref{SGP/42}{sgp42}{}) and suggest any changes
that you think should be made.

\item A list, in priority order, of the most important specific
objectives that need to be accomplished during the next year in order
to implement your strategy. In doing this, you should not restrict
yourself too narrowly to your SSG's subject area. Many developments
({\em e.g.}\ in documentation and infrastructure software) are
relevant to many SSGs, and the Project would appreciate knowing how
your SSG would prioritise them.

\item For each objective (above), provide justification in terms of
the astronomical benefits expected and the number of anticipated
users, and briefly outline the software work required (technical
detail is not expected). Also say when the resulting products will be
needed and for how long ({\em i.e.}\ estimate their useful
lifetime). This information will greatly assist the Project in knowing
how sophisticated a solution must be adopted.

\end{enumerate}

\newpage
\appendix
\section{\label{sect:comparison}COMPARISON WITH THE OLD SIGS}

Although \st{Software Strategy Groups} have no particular connection
with the earlier \st{Special Interest Groups} (SIGs), it is inevitable
that comparisons will be made, especially as some individuals will
have been members of both types of group.  There may even be a
tendency for SSGs to evolve in similar ways. Great care should be
taken to ensure that this does not happen, however, as future reviews
of Starlink are unlikely to be sympathetic to the continuation of
these user groups if the problems associated with the SIGs are seen to
persist.  To help guard against this, the following points highlight
some of the most important differences between the SSGs and the SIGs:

\begin{enumerate}

\item SSG members are not necessarily expected to be
\qt{typical} Starlink users (\st{i.e.}\ an unbiassed sample).  Being
relatively expert, they are not likely (for instance) to suffer the
same difficulties as novice users, for whom Starlink must also cater.
They should, however, be in a good position to suggest possible
solutions to these problems once they have been identified.  Because
of this, SSGs are not intended to be the sole representatives of user
opinion on software matters.

Emphasis will instead be placed on other sources of information, such
as the questionnaire and other surveys of individuals, when
questions arise about the balance of opinion amongst users as a
whole. A wide range of opinion and representation on the SSGs is, of
course, still desirable, but its limitations and the impossibility of
being both \qt{typical} and \qt{expert} are recognised.

\item The SSGs will not manage any of Starlink's software development
work and therefore do not have an allocation of manpower to call upon
-- neither should they place actions on \staffref{Starlink staff} or
\apref{contractors}.  While it should be possible to estimate the
amount of work that might be achieved in any year, this will not be
guaranteed and will depend on competing demands from many other
quarters.

This change recognises that very few SSG members will
have either the expertise or time required to manage software
work in the detail required, and that this is better performed by the
Project staff for whom it is a full-time occupation. The Project
is also in a better position to adopt a unified approach across
different subject areas, given that this may involve individual
programmers working on a variety of projects.

\item Starlink programmers will not be assigned permanently to subject
areas, nor associated with particular specialist groups (SIGs in the
past). Instead, they will constitute a \qt{pool} on which the Project
will draw in order to carry out its highest priority objectives,
assessed annually over all subject areas.  In the interests of
flexibility, this will also give the Project the opportunity to switch
manpower between different projects as required. The projects
undertaken need not necessarily be restricted to those identified by
the SSGs.

As a result, the number of SSGs and programmers will not necessarily
be equal. To further reinforce this change, Starlink programmers will
not act as secretaries to SSG meetings. However, they will usually
attend such meetings as seem relevant to their anticipated work
schedule.

\item The degree of computing expertise required of SSG members should
not generally exceed that of an experienced astronomical
researcher. Although SSG members are expected to be software
\qt{experts}, this should generally be as a result of their
particularly heavy or demanding \underline{use} of the software,
rather than because of technical expertise in its design or
implementation. Members may also be selected because of their
knowledge of relevant astronomical developments, for example advances
in instrumentation.

This change recognises the fact that the SIGs had become dominated by
expert software developers, making the resulting technical discussions
inaccessible to more typical users of their software. SSGs must remain
more open and accessible to participation by ordinary users if they
are to be accepted and not mistrusted by them. For this reason, a more
rapid turnover of SSG membership will also be promoted.

It is recognised, of course, that technical expertise may be required
at SSG meetings, and it is expected that Chairs will invite relevant
experts to complement the SSG membership and provide any technical
advice needed in the discussions. The Project will also contribute its
expertise as required.

\item SSGs are not expected to provide detailed technical advice
(\st{e.g.}\ on algorithms), nor to evaluate or design software in
technical detail. The main objectives should normally be to identify
the projects required, the objectives they need to fulfil and the
timescales on which they are required.

Detailed technical work (which may include feasibility, evaluation and
design studies) can be better performed outside formal meetings by
Starlink programmers working with expert individuals (and other
programmers where relevant). The SSGs have a role in helping to
identify these experts.

\end{enumerate}

\end{document}
