\documentstyle[11pt]{article}
\pagestyle{myheadings}

%------------------------------------------------------------------------------
\newcommand{\stardoccategory}  {Starlink General Paper}
\newcommand{\stardocinitials}  {SGP}
\newcommand{\stardocnumber}    {7.1}
\newcommand{\stardocauthors}   {P.T.Wallace}
\newcommand{\stardocdate}      {6th August 1991}
\newcommand{\stardoctitle}     {Unix and Starlink}
%------------------------------------------------------------------------------

\newcommand{\stardocname}{\stardocinitials /\stardocnumber}
\renewcommand{\_}{{\tt\char'137}}     % re-centres the underscore
\markright{\stardocname}
\setlength{\textwidth}{160mm}
\setlength{\textheight}{240mm}
\setlength{\topmargin}{-5mm}
\setlength{\oddsidemargin}{0mm}
\setlength{\evensidemargin}{0mm}
\setlength{\parindent}{0mm}
\setlength{\parskip}{\medskipamount}
\setlength{\unitlength}{1mm}

\begin{document}
\thispagestyle{empty}
SCIENCE \& ENGINEERING RESEARCH COUNCIL \hfill \stardocname\\
RUTHERFORD APPLETON LABORATORY\\
{\large\bf Starlink Project\\}
{\large\bf \stardoccategory\ \stardocnumber}
\begin{flushright}
\stardocauthors\\
\stardocdate
\end{flushright}
\vspace{-4mm}
\rule{\textwidth}{0.5mm}
\vspace{4mm}
\begin{center}
{\Large\bf \stardoctitle}
\end{center}
\vspace{1mm}
\begin{quote}
{\it In order to exploit the most cost-effective hardware and to
maintain compatibility with the international community, Starlink has
begun introducing Unix-based computers.  A mixed Unix/VMS service, with
its extra system management and software support costs, will not be
efficient in the long term, and over the next few years the existing
VMS service will run down and a complete change to Unix will occur.
The details and timing of this transition will vary from site to site in
accordance with local needs and conditions.  However, as far as possible
all future hardware purchases will be Unix-based.  As representatives of
the user community, the Starlink Users' Committee has worked closely
with the Project in formulating these plans and firmly endorses them.}
\end{quote}

\vspace{0.5mm}

%------------------------------------------------------------------------------
\setlength{\parskip}{0mm}
\tableofcontents
\setlength{\parskip}{\medskipamount}
\markright{\stardocname}
%------------------------------------------------------------------------------

\pagebreak

\section{INTRODUCTION}
An important feature of Starlink since its creation in 1980, to which
the Project owes much of its success, is the compatibility of the
computers at each site.  This uniformity has meant that programs
developed at one site run, usually without any changes at all, on all
other Starlink machines and on many VAXs throughout the World.  That
this has been possible despite profound changes in the computer hardware
itself (from VAX-11/780s in the early days to MicroVAXs and VAXstations
now) is, of course, due to the constancy of the VMS operating system and
the scalability of the VAX architecture.  However, the VMS era is
drawing to a close;  new Starlink machines will be based instead on the
Unix operating system.  This is clearly a profound change, and one which
will bring costs as well as benefits to UK astronomers.  The present
paper gives the reasons for making the switch and explains some of the
consequences.

\section{UNIX}
\subsection{Unix and Other Operating Systems}
Unix is the name of a family of operating systems, two specific examples
of which are (i)~SunOS, the implementation of Unix found on the Starlink
Sun workstations, and (ii)~Ultrix, the implementation of Unix found on
the Starlink DECstations.  Two other operating systems well-known to UK
astronomers are VMS, which runs on VAX, and MS-DOS, which runs on
IBM-PC.  Unix and VMS are both multi-user, interactive, multi-tasking,
virtual memory operating systems.  In contrast, MS-DOS is a single-user
system, and offers neither (serious) multi-tasking nor virtual memory.

\subsection{Unix versus VMS}
The parts of an operating system which give it a distinct character to
users are the user interface and the utilities.  Programmers are in
addition conscious of the underlying {\it system services}\, and the
range of low-level facilities, especially the input/output functions, to
which these services give access.  In all these respects Unix and VMS
differ significantly.

To typical Starlink users, familiar with the VMS system, Unix appears as
another command language (or {\it shell}\, in Unix parlance) to learn, a
different file and directory system to cope with, and another set of
tools and utility programs to master.  Compared with VMS, the commands
are terse and oddly-named, and rather inconsistent.  Guessing can be
hazardous;  there is great power and freedom, but attendant danger.  The
programmer gets a powerful and comprehensive, but unpolished, set of
tools, many oriented towards the systems programmer (and other users
of the C language).  The computer-science community has generated an
enormous collection of Unix-based utility software, certain items of
which have passed into what is generally accepted to be ``standard''
Unix.  Many of these items have no counterparts under VMS.

In summary, when VMS and Unix are compared, VMS comes across as a
well-finished and fully standardised product which carries out a
specific range of tasks extremely effectively, whereas Unix comes in kit
form and in various flavours, is never as tidy and safe as VMS, but is
ultimately more versatile.  More importantly, Unix has the great
advantage of running on many different sorts of computer while VMS is
restricted to VAXs.

Both Unix and VMS are capable of running typical Fortran astronomical
application programs equally efficiently.

\subsection{Why has Unix become popular?}
Computing enthusiasts attribute the popularity of Unix to its various
technical strengths -- pipes and I/O redirection, ``lightweight''
processes, multiple shells and so on -- and from its ``toolbox'' style,
which allows complicated and powerful new facilities to be constructed
quickly just by plugging together simple existing tools.  However, the
real secret of its success is that it was available to hardware
manufacturers at low cost, so that there is now a huge range of machines
on which some variant of Unix runs.  This sets Unix apart fundamentally
from any other operating system:  almost all interesting new equipment
is Unix-based, from workstations to minisupercomputers, and few computer
manufacturers today would seriously contemplate developing their own
operating system from scratch.

Thus, the purely technical strengths of Unix are {\it not}\, the
reason why it is set to become the principal Starlink operating system;
indeed, it would even be possible to argue that, overall, VMS is the
better operating system of the two.  Such considerations are irrelevant:
the current move to Unix is because it is the {\it de facto}\, standard
(just as VAX/VMS was in its day), that will enable Starlink users to run
a wealth of applications software on the latest and fastest hardware.

\section{UNIX AND STARLINK}
\subsection{Why Starlink is Switching to Unix}
Unix will completely replace VMS at all Starlink sites over the next
few years.  There are two key reasons why this policy has to be pursued:
\begin{itemize}
\item Starlink users need to run software imported from astronomical
communities abroad.  At one time such software was always VAX/VMS-based,
but in recent years the balance has swung decisively towards Unix, most
often on Sun or Convex.  For this reason alone, therefore, Starlink
users are certain to be confronted increasingly by Unix, and will expect
support from Starlink in this area.
\item To get the best value for money, we must use RISC\footnote{RISC
stands for {\it Reduced Instruction Set Computer}, a design of processor
which achieves very high speed by having a small instruction set
designed to be compatible with pipelining and other speed-enhancing
techniques.  Many Unix computers use RISC chips;  present VAX processors
are all CISC {\it (Complex Instruction Set Computer)} machines.  CISC
designs have properties which are less advantageous now than in former
times when, for example, memory was expensive.} hardware.  Providing
Unix-based forms of our applications software are available, we have the
freedom to select new computers from a wide range of vendors, in the
knowledge that the vigorous competition between vendors is securing
leading-edge performance at low prices.
\end{itemize}

\subsection{What will happen if Starlink ignores Unix?}
At the time of writing, considerable use of Unix within Starlink is
already occurring, mainly as a consequence of the introduction of the
SUN workstations, and this will grow even without any Starlink-led
replacement of VAX hardware with Unix equipment.

However, there are those who argue that Starlink's manpower resources
are already too thinly-spread, and that the introduction of a completely
new operating system is madness.  After all, the VAXs are providing a
satisfactory service, DEC are unlikely to go out of business in the
foreseeable future (and in a couple of years will be offering RISC-based
VAX/VMS systems which have much better price/performance than present
VAXs) and almost all of Starlink's software is VMS-based.  Many existing
Starlink users, satisfied with VMS and reluctant to come to grips with a
new and initially unfriendly system, will feel that a diversion of
scarce effort away from VMS-related work into Unix is wasteful and
unnecessary.

Despite these understandable views, the Project and the Starlink Users'
Committee have concluded that there is an urgent need to break out of
the VMS proprietary straitjacket and into ``open systems''.  If Starlink
and its user community were not to rise to the Unix challenge, the
following undesirable consequences would follow:
\begin{itemize}
\item Money would be wasted on slow VAXs.  A large gap in performance
exists between RISC and VAX processors at the same price.  At the time
of writing, for example, a Sun SPARCstation~2 offers 2.5 times better
CPU price performance than a comparably-priced VAXstation, while a
DECstation~5000/200 is even better value at 3.2 times the VAX's
price/performance.\footnote{The models compared were
(i)~the VAXstation~3100/76, 12Mb, 19-inch colour display, 104Mb~disc,
6.8~SPECmarks for \pounds10257,
(ii)~the Sun SPARCstation~2, 16Mb, 16-inch colour display, 207Mb~disc,
21~SPECmarks for \pounds12485,
(iii)~DECstation~5000/120, 16Mb, 16-inch colour display, 330Mb~disc, CD,
14~SPECmarks for \pounds6482, and
(iv)~DECstation~5000/200, 16Mb, 16-inch colour display, 330Mb~disc, CD,
20~SPECmarks for \pounds9348.  One SPECmark is about the CPU power
of a VAX-11/780.
All prices are one-off, include discounts and VAT, and are for
July~1991.}  A version of VMS running on a RISC processor is under
development by DEC, but this will do no more than narrow the present
price/performance gap, and at some unknown time in the future.
\item Hardware and software provision in UK astronomy would become
uncoordinated.  If Starlink had nothing to do with Unix, the
introduction of Unix hardware via grants could result in
diversification, duplication and waste.
\item We would risk cutting ourselves off from international
collaborators.  If the absence of Unix support and applications were to
cause Starlink sites to reject the introduction of Unix hardware,
valuable software provided by overseas groups would not be exploited.
Conversely, astronomers overseas are already losing interest in
VMS-based applications developed by UK astronomers.  A related issue is
the difficulties that will be experienced by VMS-only Starlink users
working temporarily overseas (especially in the USA), and by the new
breed of Unix-only visitors to the UK encountering old-fashioned
VMS-only Starlink sites.
\item We would not be able to use commercial packages, which are
increasingly based on Unix (and/or MS-DOS), and which in some areas
(for example graphics) already offer far more than the equivalent
Starlink facilities ever will.
\end{itemize}

\subsection{Starlink's VAX/VMS Service in the Future}
It is always possible that the promised future migration onto RISC
hardware of the VMS operating system would revitalise VMS usage in
astronomy.  However, the pressure on Starlink to eliminate the costs of
supporting multiple hardware types and software versions means that it
is only a matter of time before Starlink sites switch over completely
from VMS to Unix.  This has already happened in important institutions
overseas, and UK astronomers have fallen noticeably behind international
practice in this regard.  We do not have the resources to support
multiple operating systems;  if we pick one, it has to be Unix.

Despite the inevitability of the move to Unix, it is unlikely that any
Starlink site will be ready to make a complete change within less than
about 2~years and some will still be offering limited VMS capability in
5~years.  The lifetime of the existing VAX hardware, the availability of
money for new hardware, and the existence of VMS-only software are all
factors which will limit the speed of change.

At any particular Starlink site, when the switch does come the Project
naturally hopes that the change will have the blessing of the majority
of users.  However, cost considerations will have to be allowed to
outweigh the opposition to change that will inevitably linger on in
places.

One of the first consequences of the Project's Unix policy to be seen at
the Starlink sites is that it will be much more difficult to justify, in
the future, the purchase of VAX hardware.  Only if there is a very
strong case for doing so will such hardware be purchased.

\subsection{Effects Outside Starlink}
In supporting the UK astronomy community, the Project must consider the
likely impact of its Unix policy on SERC astronomy activities not
directly served by Starlink but within its sphere of influence.
For example, we have to take into account various space activities
and, in particular the SERC's overseas observatories.

Will the present excellent level of compatibility between the
observatories and Starlink be damaged?
Contact between the observatories and the Starlink Project has always
been good, in terms of e-mail accessibility and the exchange of ideas
if not in person.  When the initiative for hardware changes has come
from Starlink, the observatories have always been able to follow suit
in order to maintain compatibility;  conversely, innovations at the
observatories have stimulated matching changes in Starlink.  The need
for uniformity is widely accepted and so major incompatibilities are
unlikely to develop.  There are no indications at present that the
observatories will be slower to phase out their VAXs than the most
conservative of the Starlink nodes.

At all the observatories, software portability is now regarded as
crucial, and Starlink-compatible Unix equipment is either already
present or planned.  The same remarks apply to various space activities,
where international pressures are already forcing the move to Unix and
where Unix-compatible products from Starlink will be as influential in
the future as the VMS-based products have been in the past.

\section{THE IMPACT OF UNIX ON STARLINK}
\subsection{Disadvantages}
Despite the advantages of moving to Unix described so far --
compatibility with the international community, and access to the latest
and fastest computer hardware -- there is a price to pay:
\begin{itemize}
\item Most users will initially dislike Unix, finding it terse and
unforgiving, and will resent having to grapple with it merely to do what
they can already do on the VAX.  However, experience so far shows that
in about a month of regular use Starlink users will begin to appreciate
the flexibility of the Unix operating system and the concise commands.
Overseas groups report that astronomers who encounter Unix
first and subsequently are obliged to use VMS find the Unix-to-VMS
transition just as painful as the converse, and moreover continue to
prefer Unix to VMS afterwards.  This indicates that any hostility
towards Unix from existing Starlink users will be due at least in part
to lack of familiarity and fear of the unknown, albeit understandable
for those many users who will not see clear and immediate benefits from
making the change.
\item Much of the existing Starlink software is unashamedly
VMS-specific (despite forthright advice given by the Project over the
years) and adapting such software for use under Unix will take a long
time, absorb large amounts of skilled software effort and may even in
some cases be impractical.  (ADAM-based applications will, in contrast,
port readily once the ADAM environment itself is available in
Unix-compatible form.)
\item There are differences between the various implementations of
Unix.  These tend to be headaches for software developers and
system managers rather than users.  The difficulties can be minimised
by (i)~sticking to one type of equipment at one site and (ii)~using
Starlink-provided portability tools.  An example of the latter is the
set of C macros already released which allows C and Fortran code to
be mixed on all current Starlink hardware types.  In any case, there is
a strong convergence taking place into two principal Unix strains, the
{\it Unix International}\, (System~V) and the {\it Open Systems
Foundation}\, (Berkeley) versions, and in addition each of these
implementations is steadily being enhanced to incorporate the
features of the other.  Further impetus to the convergence is provided
by the {\it POSIX}\, standards, which both major Unix offerings (and a
number of other operating systems including VMS) support.  For users and
programmers, the best policy is to be aware of those areas in which
there are differences, and to avoid techniques which are peculiar to one
flavour of Unix.
\item Unix takes longer to learn and is harder to master than VMS, and
new Unix users will expect a lot of support from Starlink, in particular
from their site managers.  The extra site management effort that is
needed is, in fact, one of the Project's biggest worries.  In addition
to the increased workload from new hardware and system software, the
hard-pressed site managers will also have to deal with the wider range
of astronomy applications software that will be available on Unix
systems.
\end{itemize}

\subsection{Hardware}
Starlink has, for some time, been buying peripheral devices which
are compatible with the non-VAX-dependent SCSI interface standard.
Small sums spent on SCSI interfaces will enable these to be transferred
to future Unix systems.  Peripherals which cannot be transferred will
either be retired along with their host VAX, some years from now, or
will be replaced with SCSI devices as they get old.  It should not be
forgotten that several generations of Starlink peripheral have come and
gone during the VAX era, and even in a period of financial restraint it
will still be cost-effective to phase out obsolescent equipment, to save
on maintenance for example.

In the future all CPU and peripherals purchased will, as far as
possible, be Unix-compatible.  The hardware mix will thus move from the
present VAX/VMS dominance (110~VAX/VMS CPUs and 20~Unix CPUs) towards
Unix dominance, as existing VAX/VMS equipment reaches the end of its
useful life and is eliminated.  The pace of this change is limited by
several factors, two important ones being:
\begin{itemize}
\item The availability of suitable software to run on the new Unix
hardware.  This varies from site to site, some sites being able to do
almost all their processing on Unix platforms at present whereas others,
regrettably, will not have suitable software available for some time.
\item The availability of funds.  After the severe cuts in Starlink's
funding introduced in 1990/91 there is very little, if any, money to
spend on new hardware.  Unless this state of affairs can be corrected,
the pace of change in hardware will be much slower than is required to
keep abreast of international competition.  Furthermore, a slow change
from VAX/VMS to Unix equipment will increase costs and/or reduce the
levels of service to astronomers since it will lengthen the expensive
period when both systems are run in parallel.
\end{itemize}

\subsection{System Management}
The Project recognises the need for extra site management effort during
the transition from VAX/VMS to Unix and would ideally like to provide it
through additional contract staff.  However, at present funding levels
this is unlikely so that fully effective use of new Unix hardware will
require increased levels of ``self-help''.  It should of course be borne
in mind that self-help by research astronomers may give poorer
value-for-money to the community than employing and training appropriate
technical staff, when lost scientific output is included in the analysis.

\subsection{Applications Software}
Throughout the history of Starlink, astronomers have preferred to see
resources concentrated on providing new hardware and in order to do so
have been willing to forego the development of integrated and
machine-independent software.  The consequence of this policy is that
Starlink software has lagged behind developments in hardware and has
some catching-up to do.  Until all key Starlink software has been made
to run on the new hardware, users will naturally turn to whichever
existing package suits their work best -- IRAF, AIPS, Sun-Figaro, IDL,
MIDAS {\it etc}.  There is clearly a possibility that such usage will
become permanent, with the adverse consequences that the user community
will remain fragmented, software support effort will be diluted, and the
existing investment in Starlink software made by the Special Interest
Groups will be thrown away.

Porting of many existing Starlink applications is already underway,
using the infrastructure software that has already been provided in
SunOS- and Ultrix-compatible form (for example PGPLOT and HDS), and
quick results are being achieved.  Conversion of the various libraries
and tools which make up the ADAM environment is in hand, and the easy
porting of large quantities of application code will begin in less than
a year.  Adaptation of individual applications by astronomer/programmers
is usually straightforward, especially since both the Sun and DECstation
machines support VAX Fortran extensions.

Until Unix versions of popular applications appear, experience suggests
that most users will not bother to change from VMS.  (An exception to
this may be Figaro, where some users will be prepared to adapt their
work to the Caltech Sun version.  This is not a worry, because Figaro
{\it applications}\, have been an important part of Starlink's plans
for several years, and their compatibility with Starlink's
{\it infrastructure}\, software is steadily increasing.)

Another factor which blurs the switch to Unix and stretches the time
available for application porting is that the portable HDS system,
together with the NFS file-serving facilities, will enable users to work
on the same set of files simultaneously from a VAX and a Unix machine
(from the same screen if an X-terminal is available).  This means that a
user will be able to exploit a Sun (say) for CPU-intensive parts of his
work long before every last VAX application that he might need has been
ported.

But as Starlink's investment in Unix equipment grows, why not simply run
IRAF, AIPS and so on, letting the present Starlink software die with the
VAXs?  One reason not to pursue such a policy is that in comparison with
competing packages, the Starlink products offer significant advantages,
including:
\begin{itemize}
\item A great deal of application software is already available under
the ADAM environment, much of it superior to corresponding programs in
the older packages.  For example, the ADAM 2-D photometry package
PHOTOM is far better than the corresponding set of facilities in Figaro.
\item Separate specialised application packages retain their advantages,
yet can share data with each other using HDS and have a common
user-interface.
\item The data structuring facilities are much better than in most other
packages, offering a rational approach to extension, the ability to
process sub-arrays, handling of ``data quality'' and ``error-bar''
information, and so on.
\item The applications are written in industry-standard languages
(notably Fortran) and are easy to develop.
\item There is technical support from Starlink and a development
programme under UK control.
\item An extensive suite of kernel applications makes expensive
duplication between separate packages avoidable.
\end{itemize}
None of the existing software environments from sources outside Starlink
offers all of these advantages.

\subsection{Networking}
With the eclipse of the VAXs, the present DECnet service will be
overtaken by one based on TCP/IP protocols.  TCP/IP is the natural
networking to use with UNIX systems, and is used on those networks in
the USA and elsewhere commonly referred to as the {\it Internet}.  The
TCP/IP protocols offer broadly the same capabilities and performance as
DECnet but are less integrated with the rest of the operating system.
The Joint Academic Network JANET, which supports Starlink's present
DECnet service, will soon be running TCP/IP and will be integrated with
the Internet in the same way that Starlink's current DECnet is
integrated with SPAN.

Starlink's present high levels of network security will need to be
maintained throughout the transition to Unix.  Most present Unix
implementations are less comprehensive than VMS in this area, and so
care and circumspection will be needed if existing levels of protection
against unauthorised use are to continue.

\subsection{Why Not Scrap VMS Immediately?}
Having accepted that a transition to Unix is coming, many users will
naturally wish to get on with the change.  Why not suspend maintenance
on the VMS software and the VAX hardware, switch off individual VAXs as
they become unserviceable, or run Unix on the existing VAXs?  Here are
some of the factors that will affect the timing of the transition and
the level of any residual VMS service:
\begin{itemize}
\item There is a large body of more or less VAX-specific application
software which astronomers need to use, both private and
Starlink-distributed.
\item Users need time to adapt themselves and their software to the new
ways.  The body of VMS expertise that the community has built up over
the last decade is a valuable resource that should not be sacrificed
with unnecessary haste.
\item There are vital DEC utilities which need to continue (for example
VAX Notes).  Time will be needed to identify and install Unix
equivalents.
\item Compatibility with wide-area-networks needs to be maintained,
and it will take time to make a full transition to TCP/IP-based
protocols.
\item It is important to remain compatible with the overseas
observatories, various space missions and other existing activities.
\item The pace of change is limited by the funds available for new
hardware.
\end{itemize}

\section{WORK IN PROGRESS}
\subsection{Software}
Very little of Starlink's current software work is VAX-specific.

Open standards are now accepted by everyone, from the users (most
of whom no longer regard software portability as a frivolous
computer-science distraction) to manufacturers (who are finding it
ever harder to ensnare their customers through machine-specific
software).  It is now possible for almost every line of code being
written to be platform-independent (non-Unix as well as non-VMS);
consequently, most of the VMS-specific effort now being expended by
Starlink is in the areas of system support and software distribution, as
opposed to developing new software.

Most of the applications software currently available on Starlink's Unix
hardware runs under portable environments from overseas sources where
the switch from VMS to Unix was made some years ago:  AIPS, MIDAS, IRAF,
IDL.  There are also various freestanding applications in use, including
the SAOIMAGE image display and manipulation utility.  A Caltech version
of the popular FIGARO package is available for Sun.  (The applications
from the AAO-sourced Figaro can be run under the ADAM environment and
will therefore in due course operate on Unix platforms, hopefully
eliminating the need for a separate and less up-to-date Caltech version.)
Most of the completed Starlink Project porting work to date has
concentrated on infrastructure software (GKS, SGS, GNS, PGPLOT, and
various ADAM libraries) to enable programmers to begin porting
applications, both ADAM and non-ADAM.  Work is almost complete on Sun
and DECstation versions of the HDS package, a vital part of Starlink
applications and unmatched in rival environments.  Applications software
which conforms to the Starlink programming standard (SGP/16, which has
been available since 1981), is likely to run unchanged on Unix
platforms, or with minimal changes.  For example all but one or two
minor routines from the SLALIB library run without change on DECstation
and Sun, and DECstation versions of the ASTROM and COCO applications
have been demonstrated (also MS-DOS versions on PC).  Portable
adaptations of these and other existing Starlink applications will be
produced as dictated by user demand and the availability of manpower.

Much of the work on ADAM applications in the past has been to adapt
pre-existing packages, providing a more consistent user interface,
allowing easy and efficient data interchange, and giving access to
common graphical tools, all of which are, directly or indirectly,
major benefits for the user.  The main thrust of this integration
programme has been in the important FIGARO system, particularly through
the use of standard Starlink NDF/HDS data structures.  Another package
brought into the ADAM-based set is DAOPHOT, and an ADAM version of NOD2
is in preparation.  In addition to these adapted applications, many
packages have been developed specially for ADAM, or are now offered
only in ADAM form.  These include KAPPA, PHOTOM, PISA, ASTERIX, IRCAM,
SCAR, TSP, TRACE and IKONPAINT.  The bulk of the work on new
applications being carried out by Starlink is ADAM-based.  Packages in
preparation include IRAS90, PONGO, CCDPACK and a galaxy photometry
suite.  There are also ADAM-based supporting utilities
for data translation, program metrics and documentation production --
CONVERT and SST.  All of these applications will be readily portable to
Unix machines once the Unix-compatible ADAM is available.  The same goes
for the private software which has been developed within the ADAM
framework, or using major ADAM subsystems like the graphics and image
display libraries and the HDS hierarchical data system.

\subsection{Hardware}
Most discussions on Unix-based hardware for Starlink have tended to talk
of ``workstations'', simply because this class of device is the one most
sites are asking for and the only one currently represented on Starlink.
However, in addition to workstations, the Project is looking actively at
several other types of equipment, from minisupercomputers like the
Convex to board-level processors like the i860.  Nevertheless, in spite
of the obvious fact that processing tasks requiring the fastest CPUs or
the largest memories can only be run on specialised devices of one sort
or another, experience shows that workstations are useful for a
surprisingly wide variety of tasks.  For example, at Jodrell~Bank the
Sun SPARCstation~2 has proved to be faster than the Alliant
minisupercomputer on scalar work and, unexpectedly, comparable to the
Alliant on AIPS mapping job-mixes where the Alliant was expected to
excel.  Also, multi-user access to Sun and DEC workstations from X- and
character-terminals proves very successful, allowing the workstations to
double as compute servers.

Another point to be considered is the extent to which the existing
Starlink workstations, most of which are Suns, will dictate future
purchases.  Are we to be locked into Sun equipment?  The answer to this
question is no.  We already have a mixture of equipment on Starlink,
including DECstations as well as Suns, and it is the Project's intention
to continue developing truly portable forms of the Starlink software,
not Sun-only (or even Unix-only) versions.  However, system management
workload has to be taken into account when deciding what equipment to
use, and at a site which already had, say, Suns there would be a strong
incentive to continue using them.

\section{ADVICE TO STARLINK USERS}
Resource constraints, both in terms of money and manpower, mean that
though Starlink will do its best to extend Unix capabilities to all
sites soon, it cannot provide the large numbers of Unix computers which
users want, nor the user-training and the documentation to bring the
equipment into effective use.  Several sites still do not have a
single Starlink-provided Unix machine, and this may be true for another
year or two in places.  Even at sites where such equipment has been
provided, present budgets will not allow elaborate education programmes
to be carried out by Starlink, nor is it clear that this is needed.  It
should be possible to obtain the required growth in Unix expertise from
local resources, as follows:

\begin{itemize}
\item On the hardware side, users should make every effort to gain
access to HEI-provided Unix computers.  This may mean taking up local
computing departments' offers of equipment (as has happened in some
cases already) or making direct approaches and asking for such hardware.
\item Site managers and users should investigate what training courses
are locally available.  Most HEI computing departments have several
years' experience in Unix, and the quality of training they can provide
is generally very good.
\item Having gained access both to the hardware (whether
Starlink-provided or not) and to training, users should be prepared to
transfer some part of their work from the VAX to the Unix equipment --
perhaps one-off Fortran development, or text-processing -- in order to
develop and retain expertise in Unix commands and utilities.
\item Programmers planning to move private code from the VAX to a Unix
platform would be well advised to check their programs for compliance
with the Starlink Application Programming Standard (see SGP/16).  Even
though both of the present favourite machines (Sun and DECstation)
support DEC Fortran extensions, these extensions may not be available on
HEI-provided equipment or on hardware which the Project may purchase in
the future.  A further piece of advice is to adapt applications to run
in the Starlink ADAM environment -- in other words to call Starlink
libraries to perform input/output, error-handling, parameter input and
so on.  Such applications can be run now on the VAXs, and later will
move to Unix platforms without significant extra effort.
\item Agreeing on an acceptable Unix-compatible text editor is a
potential problem.  Users have the choice of (i)~learning Unix's
hairshirt screen editor {\it vi}, (ii)~mastering the comprehensive
portable editor EMACS, or (iii)~giving in to temptation and using
whatever machine-specific screen editor comes with their Unix computer.
For most users EMACS is probably the best choice, and it is being provided
on all Starlink processor types.  Programming specialists tend to use
{\it vi}, which they can rely on finding on all Unix systems and with
which they can achieve high typing speeds.
\item Many Unix features are most accessible from the C programming
language, and some astronomer/programmers will find proficiency in C of
growing benefit.  C compilers are provided on Starlink VAXs, and there
is a Starlink C programming standard (SGP/4).  However, C is less suited
to applications involving intensive numerical work, where Fortran is
still the recommended language, and in the future Fortran~90 is expected
to take over from Fortran~77 as the language of choice for Starlink
applications.  The document SGP/5 shows how to write applications in a
mixture of Fortran and C.
\end{itemize}
\end{document}
