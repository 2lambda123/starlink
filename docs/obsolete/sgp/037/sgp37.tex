\documentstyle{article} 
\pagestyle{myheadings}

%------------------------------------------------------------------------------
\newcommand{\stardoccategory}  {Starlink General Paper}
\newcommand{\stardocinitials}  {SGP}
\newcommand{\stardocnumber}    {37.5}
\newcommand{\stardocauthors}   {M D Lawden, J C Sherman}
\newcommand{\stardocdate}      {24 May 1990}
\newcommand{\stardoctitle}     {Starlink Site Manager's Guide --- Minor Nodes}
%------------------------------------------------------------------------------

\newcommand{\stardocname}{\stardocinitials /\stardocnumber}
\markright{\stardocname}
\setlength{\textwidth}{160mm}
\setlength{\textheight}{240mm}
\setlength{\topmargin}{-5mm}
\setlength{\oddsidemargin}{0mm}
\setlength{\evensidemargin}{0mm}
\setlength{\parindent}{0mm}
\setlength{\parskip}{\medskipamount}
\setlength{\unitlength}{1mm}

\begin{document}
\thispagestyle{empty}
SCIENCE \& ENGINEERING RESEARCH COUNCIL \hfill \stardocname\\
RUTHERFORD APPLETON LABORATORY\\
{\large\bf Starlink Project\\}
{\large\bf \stardoccategory\ \stardocnumber}
\begin{flushright}
\stardocauthors\\
\stardocdate
\end{flushright}
\vspace{-4mm}
\rule{\textwidth}{0.5mm}
\vspace{5mm}
\begin{center}
{\Large\bf \stardoctitle}
\end{center}
\vspace{5mm}

\setlength{\parskip}{0mm}
\tableofcontents
\setlength{\parskip}{\medskipamount}
\markright{\stardocname}

\newpage

\section {Introduction}

This paper is for managers of Starlink's Minor Nodes, some of which have a
Site Contract and a site manager paid 50\% by Starlink.
Where there is such a contract, managers should make themselves familiar with
its terms (copies are available from the University or from Starlink).
Some of the Minor Nodes without Site Contracts at present may agree contracts
in due course but others may prefer to remain without contracts indefinitely.
 
Minor Nodes were set up with funding from SERC grants and support users who
previously made up a Starlink Remote User Group (RUG).
Many of the grants which set up Minor Nodes had special conditions attached
to them and managers should make themselves familiar with these conditions
(copies of the announcement letter and the conditions are available from
the grant holder or from Starlink).
In particular, managers should note the special conditions that:
\begin{itemize}
\item Ownership of the equipment funded by the grant should remain with SERC
and not transfer to the University.
\item The Area Management Committee of the appropriate Catchment Area should
oversee the use of the system.
\end{itemize}
 
The following recommendations on running a Starlink Minor Node apply fully
at those sites were there is a Site Contract.
For other Minor Nodes, however, the recommendations do not apply with the same
force.
Nevertheless, these other nodes and especially those which hope to agree
contracts in due course would be well advised to follow Starlink conventions
for the following reasons:
\begin{itemize}
\item If security and/or usage practices are too different from Starlink's,
it may become necessary to prohibit use of shared Starlink resources,
such as the database machine.
In extreme cases, it may even become necessary to remove the non-standard site
from Starlink's DECnet.
\item If the recommendations are to be followed eventually, disruptive changes
in the future can be avoided by following Starlink practices now.
\end{itemize}
The intention of this Guide is to help you understand the Starlink project and
your role within Starlink.
It is assumed that you are aware of the documentation DEC provide for managers
and operators of VAX computers, so the Guide contains only the extra information
you need for Starlink purposes.
You should visit the central Starlink site at RAL to see how that site is
run and to get to know the project staff.
You can propose such a visit at any time --- don't wait to be asked.

There are a number of people you can turn to for help.
These include the other site managers, and project staff with particular areas
of expertise.
Their names, addresses (network and postal) and functions are specified in
ADMINDIR:WHOSWHO.LIS.
In general, use the VMS MAIL utility to communicate with people as your message
can then be dealt with at a convenient time and you have a printed record.
VMS MAIL works only if you and the person you are contacting are both on nodes
of Starlink's DECnet network; otherwise, you should use POST (see SUN/36).
 
The Guide for managers of Starlink's Major Nodes is a separate paper -- SGP/25,
``Starlink Site Manager's Guide -- Major Nodes''.
The present Guide refers frequently to SGP/25 and should be read together with
SGP/25.
(References within SGP/25 to off-site users, i.e.\ references to Remote User
Groups, do not apply here since no Minor Nodes support off-site users.)

\section {Site Manager's Responsibilities}

The section on Site Manager's Responsibilities in SGP/25 applies here
with the following additions and changes.
\begin{itemize}
\item Managers at Minor Nodes with contracts, i.e.\ those managers paid partly
by Starlink, may be called on from time to time to carry out duties at Starlink
sites within the appropriate Catchment Area but away from their home
institution.
For example, if the manager at UCL were absent for a prolonged period, Starlink
may ask the QMC manager to spend a day a week at UCL.
\item Acting as secretary to the Area Management Committee does not apply;
this job is done by the manager of the Major Node.
\item Helping to manage the finances of the node does not apply since
Starlink does not fund consumables for Minor Nodes.
\end{itemize}

\section {Administration}

The section on Administration in SGP/25 applies here, except for the following
points:
\begin{itemize}
\item Your Annual Report will form part of the AMC report for your Catchment
Area.
\item If your users need to use a Major Node they can claim travelling expenses,
just as they did when they were a Remote User Group.
The claim form should be sent to the manager of the Major Node.
\item The managers of those Minor Nodes which do not yet have a site contract
cannot use the mechanism for payment of travel and subsistence expenses
described in SGP/25.
Instead, they should complete RAL form N2(S), which can be obtained from
Andrea Roberts, and return it to Andrea, R68, RAL.
Please don't send these forms directly to the RAL Claims Office as this will
delay payment.
There is no {\it per diem} subsistence rate for non-SERC employees so actual
expenses must be declared.
It is important to include all receipts with the claim; a modest mileage rate
is paid if you use your own car.
Please contact Andrea Roberts in advance of the visit if the subsequent travel
and subsistence expenses claim is likely to be unusual.

\end{itemize}

\section {Meetings}

The meetings described in section 4 of SGP/25 are all relevant here but the
recommendations on attendance at SMM, AGM and AMC/SLUG meetings are different
from SGP/25.
 
Starlink has 11 Major Nodes, 9 Minor Nodes, and 1 Remotely Managed Node.
If all site managers were to attend every SMM, the large number of managers
(more than 20) together with appropriate Starlink staff would make the meetings
longer and less effective.
The attendance at SMMs is therefore restricted to:
\begin{itemize}
\item All managers of Major Nodes
\item Appropriate Starlink staff
\item Managers of those Minor Nodes located within the Catchment Area of the
site where the meeting is held.
However, all managers of Minor Nodes are invited to attend the larger AGMs.
\end{itemize}
 
Starlink expects managers of Minor Nodes with contracts, who are paid partly 
by Starlink, to attend every AGM and all relevant SMM and AMC/SLUG meetings.
Starlink cannot insist that managers of other Minor Nodes attend these meetings,
but nevertheless they are urged to do so since:
\begin{itemize}
\item You can discover the latest changes in Starlink policy and can, through
your comments, influence that policy.
You will also gain the knowledge which will enable you to answer questions on
Starlink from your users.
\item You can represent the interests of your node when important topics
such as AMC wish-lists are discussed.
\end{itemize}
Travelling expenses for attendance at these meetings will be reimbursed by
Starlink, either directly or via the University.

\section {Operating Standards}

Section 5 of SGP/25, Operating Standards, applies here with a few modifications
and notes as follows.
\begin{itemize}
\item 5.2 Disc backup: Users of Minor Nodes are no more and no less careless
than users of Major Nodes and the disc hardware in use at Minor Nodes is very
similar to that at Major Nodes.  Consequently, the risk of accidental file
loss, per user and per Mbyte, is the same for Minor Nodes as for
Major Nodes and the recommendations on disc backup in SGP/25 also apply here.
In particular, backup procedures should be carried out with the same frequency
as for Major Nodes.
(Because there are fewer discs at Minor Nodes the time taken for each backup
run will be less than at a Major Node and because there are fewer discs and
fewer users the intervals between file losses will be longer.  The time spent
on backup per file loss will therefore be much the same as at a Major Node.)
\item 5.3 Special usernames: Managers should have usernames on the Major Node in
their Catchment Area but do not normally have usernames at other Starlink sites.
They should also have usernames on the database machine, STADAT.
\item 5.4 Use of other nodes: Users of Minor Nodes can apply for usernames on
the Major Node in their Catchment Area.
This can be done to gain access to a particular peripheral, for example.

The STARLINK captive account should be available on your system.
\item 5.5 User authorization file: The single letter site codes for Minor
Nodes are given in SGP/25, Appendix B.
\end{itemize}

\section {Conferences}

The recommendations in SGP/25 all apply to Minor Nodes as well as Major Nodes.

\section {User Registration}

The section on User Registration in SGP/25 applies here without any major
changes.
 
The use of your node and the registration of users is a particularly difficult
policy area for some Minor Nodes so be sure to consult Gordon Bromage if there
is any doubt at all whether work falls within Starlink's terms of reference
or not.
 
For those of your users with usernames on the appropriate Major Node, specify
your own node as the `Primary Node' and the Major Node as the `Secondary Node'.
If such a user leaves, please inform the manager of the Major Node.  

\section {Starlink Software}

The corresponding section of SGP/25 applies here without change.

\section {On-line Information Summaries}

The corresponding section of SGP/25 applies here.
The file AMC.LIS should be copied from your Major Node whenever there are
changes in the AMC membership.
\end{document}
