\documentstyle{article}
\pagestyle{myheadings}
\markright{SGP/39.1}
\setlength{\textwidth}{160mm}
\setlength{\textheight}{240mm}
\setlength{\topmargin}{-5mm}
\setlength{\oddsidemargin}{0mm}
\setlength{\evensidemargin}{0mm}
\setlength{\parindent}{0mm}
\setlength{\parskip}{\medskipamount}
\setlength{\unitlength}{1mm}

\begin{document}
\thispagestyle{empty}
SCIENCE \& ENGINEERING RESEARCH COUNCIL \hfill SGP/39.1\\
RUTHERFORD APPLETON LABORATORY\\
{\large\bf Starlink Project\\}
{\large\bf Starlink General Paper 39.1}
\begin{flushright}
J C Sherman\\
28th September 1987
\end{flushright}
\vspace{-4mm}
\rule{\textwidth}{0.5mm}
\vspace{5mm}
\begin{center}
{\Large\bf Upgrades to Starlink Hardware and Software}
\end{center}
\vspace{5mm}

\section{Hardware Upgrades}

\subsection{Starlink funded}
Any suggestions for improvements to Starlink nodes to be funded by Starlink
should be made at SLUG or LMC meetings.
Between meetings, urgent matters can be raised with the Site Manager (who is
also the LMC secretary) and with the LMC Chairman.
The LMC can then place the various suggestions in priority order, and make
representations to Starlink on behalf of the node as a whole.
Ex-RUG's should also make their requests for hardware upgrades (usually upgrades
to their MicroVAX) via their catchment-area LMC.
An important channel of communication is the ``wish-list" section of each LMC's
Annual Report, due on the 1st August each year.

\subsection{Non-Starlink funded}
SERC grants and UGC funds are the usual sources of non-Starlink funding
available to Universities.
SERC Establishments can make funds available from existing Project lines.
Anyone (either University or Establishment) intending to enhance their Starlink
node using non-Starlink funds should discuss their plans with their Site
Manager, with their LMC Chairman, and with the project team at RAL at an early
stage.
For example, plans should be discussed WELL BEFORE any grant application
deadline.
These discussions should cover:
\begin{itemize}
\item technical feasibility
\item compatibility with other (Starlink) plans
\item maintenance arrangements for the new item.
\end{itemize}
The Starlink Users' Committee (SUC) and the project team at RAL are invited to
make technical comments on all SERC grant applications related to Starlink.
Adequate discussion before the application is drafted will help to avoid
feasibility, compatibility or maintenance problems and corresponding adverse
comments.

\section {Software Upgrades}

\subsection{Local Software (provided by Starlink)}
Suggestions for software work which is unlikely to be of interest at any other
Starlink site should be made as for Starlink funded hardware upgrades (section
1.1).
In principle, requests for effort to develop local software could be included in
the LMC's ``wish list" but, since the project team at RAL has no effort
available for such work, this is not done.
Local software is developed by the Site Manager and volunteers, half the Site
Managers' time being devoted to user support in general, including local
software development.
Suggestions for purchase of proprietary software (for example, languages
or tape archiving systems) can be included in the ``wish-lists" mentioned in
section 1.1.

\subsection{Starlink-wide Software (provided by Starlink)}
Suggestions should be made at Special Interest Group (SIG) meetings.
Anyone who does not normally attend SIG meetings but would like to do so should
contact the appropriate SIG Chairman (see ADMINDIR:WHOSWHO.LIS).
If the proposed software could be of interest to more than one SIG, please
consult the Project Scientist, Gordon Bromage (RLVAD::GEB).
Starlink now has 6 application programmers working on contract at Universities
so suggestions which are given high priority by the SIG and the Project
Scientist will be implemented.

\subsection{Non-Starlink funded Software}
The points made in section 1.2 regarding early discussion of plans also apply
here.
As for hardware grant applications (section 1.2), SERC grant applications
for PDRA's to develop Starlink-related software will be reviewed by SUC.

\section{Time scale of upgrades provided by Starlink}

Major new hardware upgrades to a node can take more than a year from first
suggestion to installation.
The suggestion has to be adopted by the LMC, included in Starlink's expenditure
plans, reviewed by SUC, ordered and finally delivered.
Modifications to existing plans or minor items, however, can be dealt with in
a matter of months.

Initiatives involving large changes to Starlink policy and funding can take even
longer to come to fruition.
Such initiatives are usually reviewed via the Five Year Forward Look mechanism
and consequently will take an absolute minimum of 1.5 years to implement and
probably 2 to 3 years.

Software initiatives adopted by SIG's can take from weeks to years to implement,
depending on the scale of the work.
If the new software is already available with suitable documentation, perhaps
as a result of local efforts, it can be distributed quickly by Starlink;
distribution taking as little as one week.
\end{document}
