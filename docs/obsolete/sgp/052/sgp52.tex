\documentclass[twoside,11pt]{article}
\pagestyle{myheadings}

% -----------------------------------------------------------------------------
% Command string to write DRAFT on each page in grey.
% \special{!userdict begin /bop-hook{gsave 200 30 translate
% 65 rotate /Times-Roman findfont 216 scalefont setfont
% 0 0 moveto 0.7 setgray (DRAFT) show grestore}def end}

% -----------------------------------------------------------------------------
% ? Document identification
\newcommand{\stardoccategory}  {Starlink General Paper}
\newcommand{\stardocinitials}  {SGP}
\newcommand{\stardocsource}    {sgp\stardocnumber}
\newcommand{\stardocnumber}    {52.1}
\newcommand{\stardocauthors}   {M.\,J.\,Bly}
\newcommand{\stardocdate}      {5 August 1997}
\newcommand{\stardoctitle}     {Plan for Starlink Software Distribution}
\newcommand{\stardocabstract}  {%

This paper outlines changes to the Starlink Software Distribution system.

The new scheme is based on a twice-yearly release of the whole Unix
Starlink Software Collection (USSC) on CD-ROM backed up by releases of
bug fixes using the Starlink Software Store.  Development versions of
the USSC will be maintained at RAL and made available to sites hosting
contract programmers.  Information files and management documentation
will be distributed monthly using a simplified version of the current
distribution system.  Commercial software will be distributed using a
modified version of the current distribution system, to coincide with
CD-ROM releases.

}

% ? End of document identification
% -----------------------------------------------------------------------------

\newcommand{\stardocname}{\stardocinitials /\stardocnumber}
\markboth{\stardocname}{\stardocname}
\setlength{\textwidth}{160mm}
\setlength{\textheight}{230mm}
\setlength{\topmargin}{-2mm}
\setlength{\oddsidemargin}{0mm}
\setlength{\evensidemargin}{0mm}
\setlength{\parindent}{0mm}
\setlength{\parskip}{\medskipamount}
\setlength{\unitlength}{1mm}

% -----------------------------------------------------------------------------
%  Hypertext definitions.
%  ======================
%  These are used by the LaTeX2HTML translator in conjunction with star2html.

%  Comment.sty: version 2.0, 19 June 1992
%  Selectively in/exclude pieces of text.
%
%  Author
%    Victor Eijkhout                                      <eijkhout@cs.utk.edu>
%    Department of Computer Science
%    University Tennessee at Knoxville
%    104 Ayres Hall
%    Knoxville, TN 37996
%    USA

%  Do not remove the %begin{latexonly} and %end{latexonly} lines (used by
%  star2html to signify raw TeX that latex2html cannot process).
%begin{latexonly}
\makeatletter
\def\makeinnocent#1{\catcode`#1=12 }
\def\csarg#1#2{\expandafter#1\csname#2\endcsname}

\def\ThrowAwayComment#1{\begingroup
    \def\CurrentComment{#1}%
    \let\do\makeinnocent \dospecials
    \makeinnocent\^^L% and whatever other special cases
    \endlinechar`\^^M \catcode`\^^M=12 \xComment}
{\catcode`\^^M=12 \endlinechar=-1 %
 \gdef\xComment#1^^M{\def\test{#1}
      \csarg\ifx{PlainEnd\CurrentComment Test}\test
          \let\html@next\endgroup
      \else \csarg\ifx{LaLaEnd\CurrentComment Test}\test
            \edef\html@next{\endgroup\noexpand\end{\CurrentComment}}
      \else \let\html@next\xComment
      \fi \fi \html@next}
}
\makeatother

\def\includecomment
 #1{\expandafter\def\csname#1\endcsname{}%
    \expandafter\def\csname end#1\endcsname{}}
\def\excludecomment
 #1{\expandafter\def\csname#1\endcsname{\ThrowAwayComment{#1}}%
    {\escapechar=-1\relax
     \csarg\xdef{PlainEnd#1Test}{\string\\end#1}%
     \csarg\xdef{LaLaEnd#1Test}{\string\\end\string\{#1\string\}}%
    }}

%  Define environments that ignore their contents.
\excludecomment{comment}
\excludecomment{rawhtml}
\excludecomment{htmlonly}

%  Hypertext commands etc. This is a condensed version of the html.sty
%  file supplied with LaTeX2HTML by: Nikos Drakos <nikos@cbl.leeds.ac.uk> &
%  Jelle van Zeijl <jvzeijl@isou17.estec.esa.nl>. The LaTeX2HTML documentation
%  should be consulted about all commands (and the environments defined above)
%  except \xref and \xlabel which are Starlink specific.

\newcommand{\htmladdnormallinkfoot}[2]{#1\footnote{#2}}
\newcommand{\htmladdnormallink}[2]{#1}
\newcommand{\htmladdimg}[1]{}
\newenvironment{latexonly}{}{}
\newcommand{\hyperref}[4]{#2\ref{#4}#3}
\newcommand{\htmlref}[2]{#1}
\newcommand{\htmlimage}[1]{}
\newcommand{\htmladdtonavigation}[1]{}

%  Starlink cross-references and labels.
\newcommand{\xref}[3]{#1}
\newcommand{\xlabel}[1]{}

%  LaTeX2HTML symbol.
\newcommand{\latextohtml}{{\bf LaTeX}{2}{\tt{HTML}}}

%  Define command to re-centre underscore for Latex and leave as normal
%  for HTML (severe problems with \_ in tabbing environments and \_\_
%  generally otherwise).
\newcommand{\latex}[1]{#1}
\newcommand{\setunderscore}{\renewcommand{\_}{{\tt\symbol{95}}}}
\latex{\setunderscore}

% -----------------------------------------------------------------------------
%  Debugging.
%  =========
%  Remove % from the following to debug links in the HTML version using Latex.

% \newcommand{\hotlink}[2]{\fbox{\begin{tabular}[t]{@{}c@{}}#1\\\hline{\footnotesize #2}\end{tabular}}}
% \renewcommand{\htmladdnormallinkfoot}[2]{\hotlink{#1}{#2}}
% \renewcommand{\htmladdnormallink}[2]{\hotlink{#1}{#2}}
% \renewcommand{\hyperref}[4]{\hotlink{#1}{\S\ref{#4}}}
% \renewcommand{\htmlref}[2]{\hotlink{#1}{\S\ref{#2}}}
% \renewcommand{\xref}[3]{\hotlink{#1}{#2 -- #3}}
%end{latexonly}
% -----------------------------------------------------------------------------
% ? Document-specific \newcommand or \newenvironment commands.
% ? End of document-specific commands
% -----------------------------------------------------------------------------
%  Title Page.
%  ===========
\renewcommand{\thepage}{\roman{page}}
\begin{document}
\thispagestyle{empty}

%  Latex document header.
%  ======================
\begin{latexonly}
   CCLRC / {\sc Rutherford Appleton Laboratory} \hfill {\bf \stardocname}\\
   {\large Particle Physics \& Astronomy Research Council}\\
   {\large Starlink Project\\}
   {\large \stardoccategory\ \stardocnumber}
   \begin{flushright}
   \stardocauthors\\
   \stardocdate
   \end{flushright}
   \vspace{-4mm}
   \rule{\textwidth}{0.5mm}
   \vspace{5mm}
   \begin{center}
   {\Large\bf \stardoctitle}
   \end{center}
   \vspace{5mm}

% ? Heading for abstract if used.
   \vspace{10mm}
   \begin{center}
      {\Large\bf Abstract}
   \end{center}
% ? End of heading for abstract.
\end{latexonly}

%  HTML documentation header.
%  ==========================
\begin{htmlonly}
   \xlabel{}
   \begin{rawhtml} <H1> \end{rawhtml}
      \stardoctitle
   \begin{rawhtml} </H1> \end{rawhtml}

% ? Add picture here if required.
% ? End of picture

   \begin{rawhtml} <P> <I> \end{rawhtml}
   \stardoccategory\ \stardocnumber \\
   \stardocauthors \\
   \stardocdate
   \begin{rawhtml} </I> </P> <H3> \end{rawhtml}
      \htmladdnormallink{CCLRC}{http://www.cclrc.ac.uk} /
      \htmladdnormallink{Rutherford Appleton Laboratory}
                        {http://www.cclrc.ac.uk/ral} \\
      \htmladdnormallink{Particle Physics \& Astronomy Research Council}
                        {http://www.pparc.ac.uk} \\
   \begin{rawhtml} </H3> <H2> \end{rawhtml}
      \htmladdnormallink{Starlink Project}{http://www.starlink.ac.uk/}
   \begin{rawhtml} </H2> \end{rawhtml}
   \htmladdnormallink{\htmladdimg{source.gif} Retrieve hardcopy}
      {http://www.starlink.ac.uk/cgi-bin/hcserver?\stardocsource}\\

%  HTML document table of contents.
%  ================================
%  Add table of contents header and a navigation button to return to this
%  point in the document (this should always go before the abstract \section).
  \label{stardoccontents}
  \begin{rawhtml}
    <HR>
    <H2>Contents</H2>
  \end{rawhtml}
  \htmladdtonavigation{\htmlref{\htmladdimg{contents_motif.gif}}
        {stardoccontents}}

% ? New section for abstract if used.
  \section{\xlabel{abstract}Abstract}
% ? End of new section for abstract

\end{htmlonly}

% -----------------------------------------------------------------------------
% ? Document Abstract. (if used)
%  ==================
\stardocabstract
% ? End of document abstract
% -----------------------------------------------------------------------------
% ? Latex document Table of Contents (if used).
%  ===========================================
\newpage
\begin{latexonly}
   \setlength{\parskip}{0mm}
   \tableofcontents
   \setlength{\parskip}{\medskipamount}
   \markboth{\stardocname}{\stardocname}
\end{latexonly}
% ? End of Latex document table of contents
% -----------------------------------------------------------------------------
\cleardoublepage
\renewcommand{\thepage}{\arabic{page}}
\setcounter{page}{1}

\section{\label{introduction}\xlabel{introduction}Introduction}

Starlink plans to change to a software distribution scheme based
on CD-ROMs.  Compared with the current network-based distribution
scheme, the new method has the advantages that it will be more
convenient for users of Starlink software and will require less
effort to operate.

\section{\label{distribution_method}\xlabel{distribution_method}Distribution Method}

The proposed distribution method is based on a twice yearly issue of
CD-ROMs, backed up by the Starlink Software Store.

\subsection{\label{cdroms}\xlabel{cdroms}CD-ROMs}

Each of the twice yearly issue of CD-ROMs will consist of complete
rebuilds of all libraries, packages\footnote{Packages that use NAG
libraries will not be included on the CD-ROM (or the Software Store).
Such packages will continue in use using the existing copies until
versions that do not use NAG libraries are available.} and utilities
from scratch, for all supported systems, and be accompanied by the
latest stable versions of all base set packages.  The CD-ROMs will be
issued in March (the Spring release) and September (the Autumn release).  The
issue of each CD-ROM will coincide with the issue of a Starlink
Bulletin with suitable publicity material for the CD-ROM and the new
and modified products it carries.

The Spring release will include all products from the Software Plan for the
previous year.  The deadline for submission of products for the
Spring release will be the closest working date to 15th November of a
Software Plan year (\emph{e.g.},\/ 14/11/1997 for plan year 1997 and
CD-ROM issue Spring 1998 in March 1998).  The long period between deadline
and issue of the Spring release allows for any required beta-testing and
integration with existing software, and preparation of the Bulletin.

The Autumn release will include all products from the Software Plan for the
current year, scheduled for delivery in the first half of the plan
year, and any bug fixes required from the Spring release.  The deadline for
submission of products and fixes for the Autumn release will be the closest
working date to 15th June (\emph{e.g.},\/ the submission deadline was
16/06/1997 for the Autumn 1997 release).

Any products that miss the deadlines will be held until the next CD-ROM
release.

\subsection{\label{software_store}\xlabel{software_store}The Software Store}

The Starlink Software Store will continue in use to provide convenient
access to Starlink Software for astronomers around the world.  For the
most part, it will be updated twice a year, at the same time as the
CD-ROM releases.

It is of course inevitable that the software issued on each CD-ROM will
have bugs in it, some of them serious.  To allow for fixes to serious
bugs, the Software Store will be used to provide corrected copies of
any packages affected. Only serious bug fixes will be made available
via the Software Store, and may be submitted at any time.  However it
is expected that the number of urgent / serious bug fixes will be low.

Apart from serious bug fixes, the Software Store will be updated only
twice a year, \emph{i.e.},\/ normal releases will not be made via the
Software Store.

\subsection{\label{updating_copies_of_the_ussc_at_sites}\xlabel{updating_copies_of_the_ussc_at_sites}Updating copies of the Starlink Software at sites}

Sites will not be \emph{required}\/ to keep up with bug fixes, although
of course they may if they wish.  However Starlink sites will be
expected to be not more than one CD-ROM issue out of date with their
local software installation.

Complete runnable versions of the USSC for each platform will be
provided on CD-ROM, so that sites may use the USSC direct from the
CD-ROM, though for reasons of speed, this is not recommended for more than
short term single machine use.

The CD-ROM will also provide a method to update installed versions of
packages \emph{etc.}, by copying from CD-ROM to disk.

Sites will not be expected to update the whole of their installation in
one go though they may do so if they wish.  Tools will be provided to
enable the site manager to update selected items one by one to suit
their local users, or to update a whole installation automatically by
detecting and updating individual packages.

All update options will be clearly described in the release notes that
will accompany the CD-ROM release.

\subsection{\label{cdrom_inventories}\xlabel{cdrom_inventories}CD-ROM inventories}

Due to the size of the complete USSC, it is obvious that a single CD-ROM
cannot hold the necessary components for all three systems.  Hence it is
likely that each CD-ROM distribution will consist of a 3 or 4 CD set.

Space restrictions on the CD-ROMs may prevent the provision of all
sources on the CD-ROMs.  Sources will usually be available via the
Software Store.

\section{\label{software_deadlines}\xlabel{software_deadlines}Software Deadlines}

It is important that programmers meet the prescribed deadlines.  The aim
of imposing the strict product deadlines is to concentrate effort on
integration, testing and distribution into predictable blocks of effort
twice a year.  It is therefore vital that programmers make every effort to
complete assigned projects on time.

\subsection{\label{the_annual_software_plan}\xlabel{the_annual_software_plan}The annual Software Plan}

The annual Software Plan and the assignment of projects for the year is
vital to the success of the new scheme.

To this end, the annual Software Plan will be created with appropriate
checks part way through to ensure that projects are on time, and to
allow revision of any projects that are not likely to meet the delivery
deadlines.  The Head of Software will monitor the Software Plans in the
early and mid stages to allow for mid-plan adjustments.  The Software
Librarian will monitor the period towards completion of each project to
make sure that the deadlines are met.

\section{\label{software_development_systems}\xlabel{software_development_systems}Software Development Systems}

One disadvantage of using CD-ROM and the Software Store as described is
that developments in software, particularly infrastructure, needed by
programmers at sites for their own development work will not be available
quickly.

To provide up-to-date development environments for the programmers not at
RAL, a system for mirroring the development systems maintained at RAL is to
be implemented.  Starlink at RAL will maintain three development systems
(one for each supported system) which will be matched by mirroring
software at each programmer site.   The development systems will use
slightly different root configurations from the standard distributions so
that a complete development system may coexist on a machine with the
standard distribution without interference.

To economise on storage and management requirements, each programmer
will be able to have local access to a mirrored development system for
only one of the supported systems.  Access to the development systems
of the other supported systems will be provided centrally at RAL
through existing arrangements.

Development versions of software may be submitted at any time they are
ready for use by development staff, though it should be noted
that submissions of development versions during the preparations
for a CD-ROM issue are likely to be held until the CD-ROM is released.
Development software submissions must be made in accordance with the
provisions of SGP/51 and meet the normal software standards.

The necessary hardware to support the mirrored development version of the
USSC will be provided by Starlink.  The exact method of mirroring will be
determined after discussion with site managers.  Updates to the mirrored
development systems will be controlled by RAL.

\section{\label{implementation}\xlabel{implementation}Implementation}

The new distribution strategy will be started with the Autumn 1997 CD-ROM, which
will consist of three rebuilt USSC distributions, made from a single
integrated source code set.

\subsection{\label{commercial_software}\xlabel{commercial_software}Commercial software}

There are two parts of the Starlink Software which are commercial
software: NAG and MEMSYS\@.  Commercial software may not be placed on
the CD-ROM or Software Store unless Starlink is granted permission to
do so.

Updates to commercial software such as NAG and MEMSYS will be made by a
modified version of the current  update mechanism, to coincide with the
CD-ROM releases.

\subsection{\label{documents_and_information_files}\xlabel{documents_and_information_files}Documents and information files}

Most documentation is attached to the appropriate packages and will be
made available with those packages via the CD-ROM or the Software Store.

However, some documents are `homeless' in that they do not refer to a
particular package.  They are usually management documents or refer to
hardware, Starlink policy or system maintenance.

Starlink also maintains information files such as user lists, the
Software Index or the lists of documents.  These are not attached to a
software package either.

In order that these files may be kept up-to date at sites, a simplified
update mechanism will be used once a month to distribute updated
and new `homeless' documents and information files.  It is also possible
that this process my be automated using the \texttt{rdist} program.

\subsection{\label{local_systems}\xlabel{local_systems}Local systems}

Many sites run with local configurations in addation to their main Starlink
installation, \emph{i.e.},\/ the \texttt{/star/local} directories.

To allow this to continue even when running off a CD-ROM, a link will
be provided in the top level of the on-line CD-ROM distribution which
will link to a fixed point such as \texttt{/star-local} so that when you set
\texttt{/star} to link to the CD-ROM, \texttt{/star/local} will point to
\texttt{/star-local}.  \texttt{/star-local} can then be a link to a local
system.

Similar provision will not be made for the NAG and MEMSYS
distributions.  Instead, the NAG and MEMSYS distribution instructions
will contain details of how they may be installed, but the
distributions will not be available for use with systems running direct
from a CD-ROM.

\newpage
\appendix

\section{\label{timetable}\xlabel{timetable}Timetable}

Work is already underway to implement the new distribution system, with
beta tests to various software packages in progress for the Autumn 1997 CD-ROM
distribution.

The following timetable describes the planned schedule of events leading to
the Autumn 1997 and Spring 1998 CD-ROM releases.

\begin{description}

\item[31 July 1997]
   Last scheduled date for Starlink Software updates using current system.
   Beta testing completed for Autumn 1997 CD-ROM products.

\item[29 August 1997]
   CD-ROM Autumn 1997 and Starlink Bulletin ready for production.

\item[15 September 1997]
   Scheduled issue date for Autumn 1997 CD-ROM.  Software Store updated to
   provide Autumn 1997 build sets.

\item[30 September 1997]
   Development mirror systems available.

\item[14 November 1997]
   Submission deadline for products and publicity material for products
   scheduled for delivery in second half of 96/97 Software Plan.

\item[1 December 1997]
   Begin integration and beta-testing (if required) for products submitted.

\item[December 1997 \& January 1998]
   Beta testing products for Spring 1998 CD-ROM release.
   Preparation of publicity material and Bulletin.

\item[February/March 1998]
   Building and testing of Spring 1998 CD-ROM.

\item[31 March 1998]
   Issue date of Spring 1998 CD-ROM.

\end{description}

\end{document}
