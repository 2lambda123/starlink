\documentstyle[11pt]{article}
\pagestyle{myheadings}

% -----------------------------------------------------------------------------
% ? Document identification
\newcommand{\stardoccategory}  {Starlink General Paper}
\newcommand{\stardocinitials}  {SGP}
\newcommand{\stardocsource}    {sgp46.2}
\newcommand{\stardocnumber}    {46.2}
\newcommand{\stardocauthors}   {R.F. Warren-Smith}
\newcommand{\stardocdate}      {9th April 1997}
\newcommand{\stardoctitle}     {Starlink's Software Work During 1996}
% ? End of document identification
% -----------------------------------------------------------------------------

\newcommand{\stardocname}{\stardocinitials /\stardocnumber}
\markright{\stardocname}
\setlength{\textwidth}{160mm}
\setlength{\textheight}{230mm}
\setlength{\topmargin}{-2mm}
\setlength{\oddsidemargin}{0mm}
\setlength{\evensidemargin}{0mm}
\setlength{\parindent}{0mm}
\setlength{\parskip}{\medskipamount}
\setlength{\unitlength}{1mm}

% -----------------------------------------------------------------------------
%  Hypertext definitions.
%  ======================
%  These are used by the LaTeX2HTML translator in conjunction with star2html.

%  Comment.sty: version 2.0, 19 June 1992
%  Selectively in/exclude pieces of text.
%
%  Author
%    Victor Eijkhout                                      <eijkhout@cs.utk.edu>
%    Department of Computer Science
%    University Tennessee at Knoxville
%    104 Ayres Hall
%    Knoxville, TN 37996
%    USA

%  Do not remove the %\begin{rawtex} and %\end{rawtex} lines (used by 
%  star2html to signify raw TeX that latex2html cannot process).
%\begin{rawtex}
\makeatletter
\def\makeinnocent#1{\catcode`#1=12 }
\def\csarg#1#2{\expandafter#1\csname#2\endcsname}

\def\ThrowAwayComment#1{\begingroup
    \def\CurrentComment{#1}%
    \let\do\makeinnocent \dospecials
    \makeinnocent\^^L% and whatever other special cases
    \endlinechar`\^^M \catcode`\^^M=12 \xComment}
{\catcode`\^^M=12 \endlinechar=-1 %
 \gdef\xComment#1^^M{\def\test{#1}
      \csarg\ifx{PlainEnd\CurrentComment Test}\test
          \let\html@next\endgroup
      \else \csarg\ifx{LaLaEnd\CurrentComment Test}\test
            \edef\html@next{\endgroup\noexpand\end{\CurrentComment}}
      \else \let\html@next\xComment
      \fi \fi \html@next}
}
\makeatother

\def\includecomment
 #1{\expandafter\def\csname#1\endcsname{}%
    \expandafter\def\csname end#1\endcsname{}}
\def\excludecomment
 #1{\expandafter\def\csname#1\endcsname{\ThrowAwayComment{#1}}%
    {\escapechar=-1\relax
     \csarg\xdef{PlainEnd#1Test}{\string\\end#1}%
     \csarg\xdef{LaLaEnd#1Test}{\string\\end\string\{#1\string\}}%
    }}

%  Define environments that ignore their contents.
\excludecomment{comment}
\excludecomment{rawhtml}
\excludecomment{htmlonly}
%\end{rawtex}

%  Hypertext commands etc. This is a condensed version of the html.sty
%  file supplied with LaTeX2HTML by: Nikos Drakos <nikos@cbl.leeds.ac.uk> &
%  Jelle van Zeijl <jvzeijl@isou17.estec.esa.nl>. The LaTeX2HTML documentation
%  should be consulted about all commands (and the environments defined above)
%  except \xref and \xlabel which are Starlink specific.

\newcommand{\htmladdnormallinkfoot}[2]{#1\footnote{#2}}
\newcommand{\htmladdnormallink}[2]{#1}
\newcommand{\htmladdimg}[1]{}
\newenvironment{latexonly}{}{}
\newcommand{\hyperref}[4]{#2\ref{#4}#3}
\newcommand{\htmlref}[2]{#1}
\newcommand{\htmlimage}[1]{}
\newcommand{\htmladdtonavigation}[1]{}

%  Starlink cross-references and labels.
\newcommand{\xref}[3]{#1}
\newcommand{\xlabel}[1]{}

%  LaTeX2HTML symbol.
\newcommand{\latextohtml}{{\bf LaTeX}{2}{\tt{HTML}}}

%  Define command to re-centre underscore for Latex and leave as normal
%  for HTML (severe problems with \_ in tabbing environments and \_\_
%  generally otherwise).
\newcommand{\latex}[1]{#1}
\newcommand{\setunderscore}{\renewcommand{\_}{{\tt\symbol{95}}}}
\latex{\setunderscore}

%  Redefine the \tableofcontents command. This procrastination is necessary 
%  to stop the automatic creation of a second table of contents page
%  by latex2html.
\newcommand{\latexonlytoc}[0]{\tableofcontents}

% -----------------------------------------------------------------------------
%  Debugging.
%  =========
%  Remove % on the following to debug links in the HTML version using Latex.

% \newcommand{\hotlink}[2]{\fbox{\begin{tabular}[t]{@{}c@{}}#1\\\hline{\footnotesize #2}\end{tabular}}}
% \renewcommand{\htmladdnormallinkfoot}[2]{\hotlink{#1}{#2}}
% \renewcommand{\htmladdnormallink}[2]{\hotlink{#1}{#2}}
% \renewcommand{\hyperref}[4]{\hotlink{#1}{\S\ref{#4}}}
% \renewcommand{\htmlref}[2]{\hotlink{#1}{\S\ref{#2}}}
% \renewcommand{\xref}[3]{\hotlink{#1}{#2 -- #3}}
% -----------------------------------------------------------------------------
% ? Document specific \newcommand or \newenvironment commands.
% ? End of document specific commands
% -----------------------------------------------------------------------------
%  Title Page.
%  ===========
\renewcommand{\thepage}{\roman{page}}
\begin{document}
\thispagestyle{empty}

%  Latex document header.
%  ======================
\begin{latexonly}
   CCLRC / {\sc Rutherford Appleton Laboratory} \hfill {\bf \stardocname}\\
   {\large Particle Physics \& Astronomy Research Council}\\
   {\large Starlink Project\\}
   {\large \stardoccategory\ \stardocnumber}
   \begin{flushright}
   \stardocauthors\\
   \stardocdate
   \end{flushright}
   \vspace{-4mm}
   \rule{\textwidth}{0.5mm}
   \vspace{5mm}
   \begin{center}
   {\Large\bf \stardoctitle}
   \end{center}
   \vspace{5mm}

% ? Heading for abstract if used.
   \vspace{10mm}
   \begin{center}
      {\Large\bf Abstract}
   \end{center}
% ? End of heading for abstract.
\end{latexonly}

%  HTML documentation header.
%  ==========================
\begin{htmlonly}
   \xlabel{}
   \begin{rawhtml} <H1> \end{rawhtml}
      \stardoctitle
   \begin{rawhtml} </H1> \end{rawhtml}

% ? Add picture here if required.
% ? End of picture

   \begin{rawhtml} <P> <I> \end{rawhtml}
   \stardoccategory \stardocnumber \\
   \stardocauthors \\
   \stardocdate
   \begin{rawhtml} </I> </P> <H3> \end{rawhtml}
      \htmladdnormallink{CCLRC}{http://www.cclrc.ac.uk} /
      \htmladdnormallink{Rutherford Appleton Laboratory}
                        {http://www.cclrc.ac.uk/ral} \\
      \htmladdnormallink{Particle Physics \& Astronomy Research Council}
                        {http://www.pparc.ac.uk} \\
   \begin{rawhtml} </H3> <H2> \end{rawhtml}
      \htmladdnormallink{Starlink Project}{http://www.starlink.ac.uk/}
   \begin{rawhtml} </H2> \end{rawhtml}
   \htmladdnormallink{\htmladdimg{source.gif} Retrieve hardcopy}
      {http://www.starlink.ac.uk/cgi-bin/hcserver?\stardocsource}\\

%  HTML document table of contents. 
%  ================================
%  Add table of contents header and a navigation button to return to this 
%  point in the document (this should always go before the abstract \section). 
  \label{stardoccontents}
  \begin{rawhtml} 
    <HR>
    <H2>Contents</H2>
  \end{rawhtml}
  \renewcommand{\latexonlytoc}[0]{}
  \htmladdtonavigation{\htmlref{\htmladdimg{contents_motif.gif}}
        {stardoccontents}}

% ? New section for abstract if used.
  \section{\xlabel{abstract}Abstract}
% ? End of new section for abstract

\end{htmlonly}

% -----------------------------------------------------------------------------
% ? Document Abstract. (if used)
%  ==================
This paper presents a summary of the software work carried out by
Starlink during 1996.  It is based on the software plan approved by
the Starlink Panel in November 1995.\footnote {In fact, there was a
delay (because of on-going reviews of Starlink) in establishing the
subsequent software plan at the end of 1996 that would normally have
followed on from the 1995 plan. This paper therefore covers a slightly
longer period than originally intended, from mid-November 1995 to the
end of January 1997. Because of staff shortages during this period,
however, the actual work completed still falls a little short of that
originally planned.}

With some exceptions (marked with an asterisk in the table of
contents), most of the planned work has been completed and the
resulting products have either been distributed or are queued for
distribution shortly ({\em{i.e.}}\ early in 1997).

Note that this paper concentrates on the work of Starlink's
Applications Programmers, software staff at RAL and those Site
Managers with specific software responsibilities. It does not cover
related work carried out ({\em{e.g.}}) by other Site Managers, nor
does it describe routine ``baseline support'' of established software
packages ({\em{e.g.}}\ minor bug fixes) unless this has resulted in
significant new features.

% ? End of document abstract
% -----------------------------------------------------------------------------
% ? Latex document Table of Contents (if used).
%  ===========================================
% \newpage
 \begin{latexonly}
   \setlength{\parskip}{0mm}
   \latexonlytoc
   \setlength{\parskip}{\medskipamount}
   \markright{\stardocname}
 \end{latexonly}
% ? End of Latex document table of contents
% -----------------------------------------------------------------------------
\newpage
\renewcommand{\thepage}{\arabic{page}}
\setcounter{page}{1}

\section{\xlabel{SPECTROSCOPY}SPECTROSCOPY}

\subsection{\label{spec:figaro}Transparent Foreign Data Access for Figaro}

The routines which Figaro (\xref{SUN/86}{sun86}{}) uses to access data
have been completely replaced this year by a new internal Figaro
library (FDA) which, in turn, calls the Starlink NDF data access
library (\xref{SUN/33}{sun33}{}). The main advantage of this change is
that Figaro now has access to the NDF library's facilities for
transparently accessing a \xref{range of ``foreign'' data
formats}{sun55}{sect_auto}, including FITS and IRAF files (in addition
to the NDF and Figaro formats that were previously used). Automatic
data compression/decompression is also possible.

This was an essential step before the project to integrate Figaro with
the IRAF command language (see \S\ref{infra:IRAF}) could be
undertaken.

This move also gives Figaro applications access to several other
services provided by the NDF library (and already used by packages
such as \xref{KAPPA}{sun95}{}), including automatic \xref{recording of
data processing history}{sun95}{se_ndfhistory} and the ability to
operate on \xref{sub-sets of spectra and images}{sun95}{se_ndfsect}.

Details of the changes and new features were described by an article
in the \htmladdnormallinkfoot{Starlink
Bulletin}{http://www.starlink.ac.uk/bulletin/96sep/a16.html} (September
1996).

\subsection{Echelle Data Reduction}

The \xref{ECHOMOP}{sun152}{} Echelle reduction package has undergone a
period of ``enhanced support'' this year, as recommended by the
Spectroscopy Software Strategy Group in its 1995 input to
Starlink. This has involved a period of extensive bug fixing and
testing, resulting in a new version (V3.2) which runs approximately
twice as fast as the previous version. (For Digital Unix users, this
represents a factor 16 improvement in speed over the last two years.)

Calls to the NAG library have been removed from ECHOMOP in preparation
for distribution via the Starlink \htmladdnormallinkfoot{Software
Store}{http://www.starlink.ac.uk/cgi-store/storetop} and on CDROM.

The Starlink Guide ``Introduction to Echelle Spectroscopy''
(\xref{SG/9}{sg9}{}) has been updated and some extra illustrations
added.

\subsection{IUE Final Archive Access}

A new FITS reader (\xref{FITS2NDF}{sun55}{FITS2NDF}) has been added to
the CONVERT package (\xref{SUN/55}{sun55}{}). Amongst the various
flavours of FITS format that it can read are the IUE Final Archive
MXLO, LILO, LIHI, RILO, RIHI, SILO and SIHI products.

A consequence of this will be that in future most Starlink
applications will be able to read these file formats transparently.

\section{\xlabel{IMAGE_PROCESSING}IMAGE PROCESSING}

\subsection{ISO Data Access}

The new \xref{FITS2NDF}{sun55}{FITS2NDF} FITS reader in the CONVERT
package (\xref{SUN/55}{sun55}{}) is now able to read the ISO
auto-analysis datasets for SWS, LWS, CAM and some PHOT products.

A consequence of this will be that in future most Starlink
applications will be able to read these file formats transparently.
Other products can be accessed using CURSA (\xref{SUN/190}{sun190}{}).

\subsection{Handling Large Datasets}

A study has been conducted into the software and system administration
implications of handling extremely large image datasets, such as
images from CCD mosaic cameras. A report on this is being prepared to
guide future technical developments.

\subsection{\label{ip:GAIA}Graphical Image Manipulation Tool (GAIA)}

Work has been conducted to adapt a graphical Real Time Display tool
(RTD), developed at ESO, for use as a toolbox for performing
astronomical manipulation of images.

RTD comes with interactive image display facilities broadly similar to
(but more extensive than) those of SAOIMAGE, and it also provides
graphical annotation facilities suitable for publication or
presentation work.  Our work has concentrated, however, not on image
display, but on extending these capabilities so that interactive
processing of astronomical images becomes possible.\footnote{Various
other display tools were also considered, but did not prove suitable
at the time because they stored only a scaled version of the data
internally.} The resulting software is called GAIA (Graphical
Astronomy and Image Analysis) and is described in
\xref{SUN/214}{sun214}{}.

The facilities that we have added to RTD include:

\begin{itemize}
\item Selection of arbitrary image regions (boxes, circles, rotating
ellipses, {\em etc.}) with output in a format compatible with other
applications.

\item Highly interactive aperture photometry, including measurement of
more than one object at a time with annular (or detached) sky
regions. Apertures may be placed and resized individually and results
saved and restored, {\em etc.} Full handling of statistical error
information is included.

\item Interactive editing of image features such as defects, cosmic
ray hits, stars, {\em etc.}, with replacement with an interpolated
background. This facility is an enhanced form of the very popular
PATCH application originally provided on the old ARGS image displays,
but unavailable since their demise.

\item Blinking between multiple images.
\end{itemize}

An article on GAIA has recently appeared in the
\htmladdnormallinkfoot{Starlink
Bulletin}{http://www.starlink.ac.uk/bulletin/97mar/a03.html} (March
1997).

\section{\xlabel{THEORY_AND_STATISTICAL_ANALYSIS}THEORY AND STATISTICAL ANALYSIS}

\subsection{Advertise Computer Algebra Software}

A new document ``Computer Algebra Software'' (\xref{SGP/47}{sgp47}{})
has been produced, outlining the major features of the two leading
packages in this area -- Maple and Mathematica -- and comparing
them. It includes a list of other packages available and pointers to
related information.

\subsection{Enhanced User Support For Visualisation Software}

As planned, we have provided some special support this year to assist
users with visualising multi-dimensional data (in particular, there is
interest in visualising spectral data from the JCMT). Overall,
however, the demand has been much lower than expected. This is due to
problems with the supplier of IBM's Data Explorer software which has
substantially delayed the release of the related products developed by
Starlink during 1995 ({\em e.g.}\ see \xref{SUN/203}{sun203}{}). Most
of these problems have now been resolved.

\subsection{Ease the Transition to Fortran~90}

Fortran~90 compilers have been made available to sites requesting
them.  There have, as yet, been no significant Fortran~90 related
requests for software support.

\section{\xlabel{INFORMATION_SERVICES_AND_DATABASES}INFORMATION SERVICES AND DATABASES}

\subsection{A UK Astronomy ftp Archive}

The Starlink UK IRAF
\htmladdnormallinkfoot{mirror}{http://www.starlink.ac.uk/iraf/} has
been set up and a poster paper was presented on this at the 1996 ADASS
meeting in Charlottesville. In the first 5 months of operation, the
mirror had distributed more than 4300 files to 120 different nodes in
more than 10 countries.

\subsection{A Gripe Command*}

No significant progress has been made this year towards developing a
``gripe'' command to permit users to submit complaints and suggestions
about software in a more automated way.

\subsection{Observing Preparation Software}

A new Starlink guide ``Preparing to Observe'' (\xref{SG/10}{sg10}{})
has been prepared to offer advice on choice of software to those
making preparations to apply for (or take up) observing time on a
variety of instruments.

\subsection{Upgrades to Catalogue Access Software}

A number of enhancements have been made to the CURSA package
(\xref{SUN/190}{sun190}{}) to improve its ability to handle
astronomical tables and catalogues.

These include a ``Small Text List'' catalogue format that may be used
to access normal ASCII text files as tables, compatibility with
\xref{KAPPA}{sun95}{} tables, and a new {\tt catselect} application
which can generate more sophisticated selections from catalogues.

An article giving further details has recently appeared in the
\htmladdnormallinkfoot{Starlink
Bulletin}{http://www.starlink.ac.uk/bulletin/97mar/a05.html} (March
1997).

\section{\xlabel{GRAPHICS_AND_INFRASTRUCTURE}GRAPHICS AND INFRASTRUCTURE}

\subsection{\label{infra:astrom}Astrometric Coordinate Systems}

A new library has been developed to support the representation of
World Coordinate System (WCS) information, for example celestial
coordinate systems, in data reduction software. It is written in C and
supports most of the concepts in the proposed FITS WCS
standard.\footnote{The FITS proposal is still evolving, so some of its
latest ideas have yet to be included.} For example, it supports a wide
range of sky projections and astronomical coordinate systems.

The library caters for the storage and retrieval of WCS data from a
variety of sources ({\em i.e.}\ it is not tied to any particular data
system or environment) and for its manipulation within programs, using
a high-level model of inter-related coordinate systems. It also
provides graphics facilities which allow applications to draw ({\em
e.g.}) arbitrary sky coordinate grids with a single call.

At present the library will be of limited interest to most users, but
is being made available to software developers so that its facilities
can start to be included in applications (which we expect will happen
during 1997). We expect to make a general release of the library once
the FITS standard has stabilised a little more and any feedback from
developers has been included.

\subsection{\label{infra:IRAF}Integration with IRAF}

We have pursued the integration of Starlink software with IRAF this
year primarily through enhancements to the Figaro package. We see this
as the forerunner of similar work on other packages in future.

As an initial part of this work, the data access system used by Figaro
has been completely replaced (see \S\ref{spec:figaro}). A prototype
IRAF interface, produced during 1995, has also been upgraded ready for
general release. As part of this, we have developed a high-level
language to describe packages, from which the interface to IRAF (and
possibly other environments or command languages in future) can be
generated automatically.

The resulting IRAF interface for Figaro means that Figaro is now
available as an ``IRAF package'' which runs from the IRAF {\tt cl} and
has full access to the IRAF parameter system. It can also exchange
data transparently with other IRAF tasks (although some Figaro
facilities, such as error and quality handling, are not then available
because of the limitations of the IRAF data format).

The product of this work has been available for test purposes for
several months and we are now preparing a full release. This will
include both the IRAF interface software (which can be used by other
packages in future) and of Figaro -- with added IRAF
compatibility. The same version of Figaro will continue to function as
normal outside IRAF.

\subsection{Removal of NAG dependence}

Our programme to remove calls to the NAG library
(\xref{SUN/28}{sun28}{}) from Starlink-supported applications packages
is now essentially complete. This was the second year of a long-term
development aimed at allowing these packages to be distributed freely
({\em e.g.}\ on CDROM and \htmladdnormallink{{\em via} the
WWW}{http://www.starlink.ac.uk/cgi-store/storetop}) without license
restrictions. There are still a very small number of NAG calls
remaining in Starlink software, but these fall into the following
categories:

\begin{itemize}
\item Software which is supported primarily outside of Starlink. In
these cases we have provided replacements for many of the NAG routines
{\em via}\ the PDA library (\xref{SUN/194}{sun194}{}) and are awaiting
updated versions of the applications.

\item NAG routines where no replacement could be found. These are all
graphics routines and we have simply omitted the (very few) affected
applications from distributions made to non-Starlink sites.
\end{itemize}

For most purposes, Starlink software is therefore now free of license
restrictions associated with the NAG library, but note that we have no
plans to withdraw NAG as a valuable tool for users writing their own
software.

\subsection{Investigate New Command Language Possibilities}

We have not been able to conduct as wide a survey of possible future
command language options as we would have liked this year, but have
studied the possibilities of running Starlink software using IDL as a
``command language'' in a little detail. This is an important example
to examine, both because of IDL's intrinsic attractiveness, and
because it illustrates the sort of problems that might arise when
considering other command languages.

The main problems that have been identified are:

\begin{itemize}
\item IDL is basically a ``single tasking'' environment. Although it
has facilities for running other tasks, it is not designed with the
intention that multiple tasks should be running simultaneously, and
that it might need to respond to them. Its inter-process communication
mechanisms are therefore fairly limited.

\item Because IDL primarily stores data in memory, there is a problem
of interfacing transparently to any data reduction applications that
store data in files (this is likely to be true of many command
languages). While explicit read/write operations can easily be
performed to transfer the data, it is more difficult to make these
happen automatically when they are needed.  The solution is fairly
straightforward in concept, but ideally would use inter-process
communication facilities which IDL lacks.
\end{itemize}

In conclusion, integrating Starlink applications into IDL seems to be
possible, and probably remains an attractive option for the future,
but would involve rather more work than was initially hoped. Other
command languages with better inter-process communication might prove
rather easier.

\subsection{\label{infra:IMG}Simplified Interface to N-Dimensional Data}

We have completed and released a library (IMG, described in
\xref{SUN/160}{sun160}{}) which provides a simplified high-level
programming interface to astronomical data with between 1 and 3
dimensions ({\em i.e.}\ spectra, images and ``cubes''). It also
provides a simple interface to attached header information. It is
possible to call the new library from both Fortran and C.

Because of Starlink's automatic format conversion facilities (see
\xref{SUN/55}{sun55}{sect_auto}), this single interface also gives
programmers straightforward access to a wide range of astronomical
data formats, including FITS and IRAF files as well as Starlink NDF
data.

An article about the IMG library has appeared in the
\htmladdnormallinkfoot{Starlink
Bulletin}{http://www.starlink.ac.uk/bulletin/96sep/a20.html} (September
1996).

\subsection{Port of Starlink Software to Linux}

We have now essentially completed the port of Starlink-supported
applications to run on PC hardware under the Linux operating
system. The resulting Linux software collection is being distributed
on CDROM.

Because of the timing of the latest (2nd) release of this CDROM, not
all the ported applications have yet been included on it. However, all
those which have been ported can be obtained from the
\htmladdnormallinkfoot{Starlink Software
Store}{http://www.starlink.ac.uk/cgi-store/storetop} on the WWW, and a
new release of the CDROM which will contain these is planned for April
1997.

There remain a few applications packages that have not yet been ported
in the following categories:

\begin{itemize}
\item Packages that we know are currently undergoing active
development outside Starlink have not been ported. We are instead
waiting for the new versions of the software to avoid having to repeat
any of the porting work.

\item In some cases, support staff outside Starlink are carrying out
the port. We are still waiting for several packages in this category.
\end{itemize}

An article giving further details has recently appeared in the
\htmladdnormallinkfoot{Starlink
Bulletin}{http://www.starlink.ac.uk/bulletin/97mar/a01.html} (March
1997).

\subsection{Library Interfaces for the C Language*}

We have developed a preliminary tool which allows C interfaces to
Fortran libraries to be generated semi-automatically, starting from a
description of the library interface. Some manual intervention is
still required, but far less than would otherwise be the case.

We have used this technique to produce a C interface to the new IMG
library (\xref{SUN/160}{sun160}{} -- see \S\ref{infra:IMG}) which has
now been released. An article describing this has appeared in the
\htmladdnormallinkfoot{Starlink
Bulletin}{http://www.starlink.ac.uk/bulletin/96sep/a20.html} (September
1996).

Unfortunately, due to lack of time, we have not yet been able to apply
the same technique to any other libraries, although it has been tested
on several others.

\section{\xlabel{RADIO_AND_MM_ASTRONOMY}RADIO AND MM ASTRONOMY}

\subsection{Fix Problems with SPECX}

Problems affecting the reliability of HDS (\xref{SUN/92}{sun92}{})
file usage by SPECX (\xref{SUN/17}{sun17}{}) and slowness in the use
of large data cubes have now been fixed. We are grateful to staff at
JACH who performed most of the work on this once the causes of the
problems had been identified.

\subsection{Importing Datasets into IRAS90}

We have verified that the reported problems with importing astrometric
data into the IRAS90 package (\xref{SUN/163}{sun163}{}) can be
overcome by using recently-added facilities
({\em e.g.}\ \xref{SETSKY}{sun95}{SETSKY}) in KAPPA
(\xref{SUN/95}{sun95}{}). We have not attempted any upgrades to IRAS90
itself because of (a) the probable short life of this package now that
ISO data are becoming available and (b) new parallel developments in
the area of astrometric coordinate handling (see
\S\ref{infra:astrom}).

\subsection{2-D Gaussian Fitting Program}

We have upgraded the ESP package (\xref{SUN/180}{sun180}{}) to provide
a very flexible application for fitting multiple 2-dimensional
Gaussian profiles to imaging data.

An article on this has recently appeared in the
\htmladdnormallinkfoot{Starlink
Bulletin}{http://www.starlink.ac.uk/bulletin/97mar/a02.html} (March
1997).

\subsection{Export Copies of SPECX and JCMTDR}

SPECX (\xref{SUN/17}{sun17}{}) is now available for worldwide
distribution from the \htmladdnormallinkfoot{Starlink Software
Store}{http://www.starlink.ac.uk/cgi-store/storetop}. JCMTDR
(\xref{SUN/132}{sun132}{}) will also be distributed in this way once a
new version is received (JCMTDR is supported outside Starlink and we
cannot distribute the current version as it is still covered by the
license terms of the NAG library).

\subsection{Distribution of the DIFMAP Package*}

We have not made any progress with this project this year, although an
article on DIFMAP has recently appeared in the
\htmladdnormallinkfoot{Starlink
Bulletin}{http://www.starlink.ac.uk/bulletin/97mar/a07.html} (March
1997).

\subsection{UNIX Version of the FLUXES Program}

The FLUXES program, which calculates infrared fluxes and positions of
the planets, for use in calibrating JCMT data, has been ported to
Starlink operating systems, upgraded to use the JPL ephemeris, and
fully documented. During this process some problems with the choice of
time system came to light and have been corrected. FLUXES is
documented in \xref{SUN/213}{sun213}{}.

\subsection{AIPS Cookbook On-Line}

An HTML version of the \htmladdnormallinkfoot{AIPS
cookbook}{http://axp2.ast.man.ac.uk:8000/\~{}jwp/COOKBOOK/cook.html}
has been produced (from the original Latex) and is now maintained
on-line by Starlink's AIPS support contact, John Palmer.\footnote{John
Palmer, who was also the part-time Starlink site manager at
Manchester, has recently left. A replacement is being recruited.}

\subsection{Fix Figaro/KAPPA Parameter Problem}

Some minor parameter system problems with Figaro
(\xref{SUN/86}{sun86}{}) and KAPPA (\xref{SUN/95}{sun95}{}),
identified by the Radio/(sub)mm Software Strategy Group, have been
fixed.

\section{\xlabel{X-RAY_ASTRONOMY}X-RAY ASTRONOMY}

\subsection{Maintaining FTOOLS/XSELECT Software}

A \htmladdnormallinkfoot{UK
copy}{http://www.astro.cf.ac.uk/pub/Grant.Privett/ftools.html} of the
FTOOLS software and related documentation is now maintained by Grant
Privett in Cardiff, who has recently published an article in the
\htmladdnormallinkfoot{Starlink
Bulletin}{http://www.starlink.ac.uk/bulletin/97mar/a19.html} to
announce this service. Grant also acts as a support contact for UK
users.\footnote{Grant Privett has recently left. A replacement is
being recruited, although this may not necessarily be at Cardiff.}

\subsection{\label{xray:catalogue}Sky Map/Catalogue Plotting Software*}

As part of a project to provide facilities for plotting data from
remote astronomical catalogues, we have developed software to perform
the following functions:

\begin{enumerate}
\item Locate a range of remote astronomical catalogues on the WWW and
allow object selections to be made ({\em e.g.}\ by specifying position
criteria) in a standard way. The data for the selected objects are
then returned.

\item Browse the returned data and make further selections, {\em
etc.}\ -- for example, select by object brightness or size. The data
can also be combined with data from local catalogues.

\item Convert the selected data into plotting information (for
example, to allow it to be plotted over a related image with circles
whose colour and size is dictated by the attributes of each
object). The way in which object attributes are converted into a plot
can be changed very easily.
\end{enumerate}

We had expected to make this software available through a new
graphical interface to the ASTERIX package (\xref{SUN/98}{sun98}{})
during 1996, but unfortunately (due to the departure of David Allan
who was supporting ASTERIX at Birmingham), this has not been possible.

We are still pursuing this possibility for a future ASTERIX release,
but may also attempt to provide these capabilities through another
route during 1997. The GAIA graphical image analysis tool
(see \S\ref{ip:GAIA}) is a possibility.

\section{\xlabel{DOCUMENTATION}\label{sect:documentation}DOCUMENTATION}

\subsection{Cookbooks}

We have completed four new Starlink Cookbooks this year, covering the
following topics:

\begin{itemize}
\item \xref{SC/4}{sc4}{} -- Writing C shell scripts. 

\item \xref{SC/5}{sc5}{} -- Introduction to CCD reductions.

\item \xref{SC/6}{sc6}{} -- CCD photometric calibration.

\item \xref{SC/7}{sc7}{} -- Introduction to spectroscopic reductions.
\end{itemize}

These are undergoing final editorial review and will be released shortly.

\end{document}
