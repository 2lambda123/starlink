\documentstyle{article} 
\pagestyle{myheadings}

%------------------------------------------------------------------------------
\newcommand{\stardoccategory}  {Starlink General Paper}
\newcommand{\stardocinitials}  {SGP}
\newcommand{\stardocnumber}    {14.5}
\newcommand{\stardocauthors}   {G E Bromage}
\newcommand{\stardocdate}      {16 December 1988}
\newcommand{\stardoctitle}     {Starlink Special Interest Groups --- SIGs}
%------------------------------------------------------------------------------

\newcommand{\stardocname}{\stardocinitials /\stardocnumber}
\markright{\stardocname}
\setlength{\textwidth}{160mm}
\setlength{\textheight}{240mm}
\setlength{\topmargin}{-5mm}
\setlength{\oddsidemargin}{0mm}
\setlength{\evensidemargin}{0mm}
\setlength{\parindent}{0mm}
\setlength{\parskip}{\medskipamount}
\setlength{\unitlength}{1mm}

\begin{document}
\thispagestyle{empty}
SCIENCE \& ENGINEERING RESEARCH COUNCIL \hfill \stardocname\\
RUTHERFORD APPLETON LABORATORY\\
{\large\bf Starlink Project\\}
{\large\bf \stardoccategory\ \stardocnumber}
\begin{flushright}
\stardocauthors\\
\stardocdate
\end{flushright}
\vspace{-4mm}
\rule{\textwidth}{0.5mm}
\vspace{5mm}
\begin{center}
{\Large\bf \stardoctitle}
\end{center}
\vspace{5mm}

\section{SIGs}

Starlink has set up Special Interest Groups (SIGs) in a number of software
areas, primarily composed of astronomers actively involved in the computer
reduction of astronomical data.

\section{Terms of Reference}

Their terms of reference are:
\begin{itemize}
\item to identify the software that is most likely to be required by Starlink
users in a specific subject area;
\item to establish the priorities for the development of such software;
\item to identify and coordinate the programming effort available in this area;
\item to strongly encourage the development of such software to Starlink
standards and within the Starlink software environment;
\item to encourage dissemination of information to, and feedback from, other
SIGs and the general Starlink community, on such software and its development;
\item to make appropriate recommendations to Starlink central management and the
Starlink Users' Committee, in the light of its discussions.
\end{itemize}

\section{Applications Programmers}

Where an Applications Programmer has been appointed in the area of
interest of a SIG, he or she is expected to collaborate closely with that SIG.

\section{Rules}

A number of rules govern the functioning of each SIG.
These are:
\begin{itemize}
\item The Chairman and members must be accredited by the Starlink Project
Scientist;
\item A Starlink central management representative will attend each meeting;
\item Travelling expenses and day subsistence will be paid by Starlink for
members travelling to no more than five meetings per year per SIG.
Meetings should take place at a location that minimises travelling expenses,
unless authorization is obtained otherwise.
A delegate may be sent in place of a member who cannot attend.
\item Each SIG must produce a short Annual Report, due on 1st August each year,
for presentation to the Starlink Users' Committee.
\item The status and aims of each SIG will be reviewed annually by SUC.
All SIGs must have clearly-defined long-term aims and annual targets, and these
should be debated and agreed at the first SIG meeting of each reporting year.
\item Each SIG should aim to meet between two and four times a year, unless
there is a special reason to be in a ``dormant" state.
\end{itemize}

\section{Current Status}

The current names and chairmen (with e-mail addresses) of the SIGs are
kept on-line in the file ADMINDIR:WHOSWHO.LIS.
Details of the full membership are kept in the file ADMINDIR:SIG.LIS.
The current (1988 Dec) list of SIGs, together with their Chairmen, is as
follows:
\begin{quote}
\begin{tabbing}
SPECSIGxxx\=Image Processing Environmentxxxxxx\=\kill
DBSIG\>Database\>Dr Dennis Kelly (ROE)\\
HSTSIG\>HST software\>Prof Mike Disney (CAR)\\
IPESIG\>Image Processing Environment\>Dr Rodney Warren-Smith (DUR)\\
IRASIG\>IRAS software\>Dr Jim Emerson (QMC)\\
IUESIG\>IUE software\>Dr Linda Smith (UCL)\\
RASIG\>Radio Astronomy\>Dr Robert Laing (RGO)\\
SASIG\>Solar Astrophysics\>Dr Helen Mason (DAMTP,CAM)\\
SPECSIG\>Spectroscopy\>Dr Phil Hill (St.Andrews)\\
VIMSIG\>Volumetric Imaging\>Dr Steve Unger (RGO)\\
XRASIG\>X-ray Astronomy\>Dr Trevor Ponman (BIR)
\end{tabbing}
\end{quote}

\end{document}
