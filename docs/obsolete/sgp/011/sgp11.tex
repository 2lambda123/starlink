\documentstyle[11pt]{article}
\pagestyle{myheadings}

%------------------------------------------------------------------------------
\newcommand{\stardoccategory}  {Starlink General Paper}
\newcommand{\stardocinitials}  {SGP}
\newcommand{\stardocnumber}    {11.1}
\newcommand{\stardocauthors}   {C A Clayton\\ M J Bly}
\newcommand{\stardocdate}      {17 October 1991}
\newcommand{\stardoctitle}     {Starlink Software Submission for UNIX Systems}
%------------------------------------------------------------------------------

\newcommand{\stardocname}{\stardocinitials /\stardocnumber}
\renewcommand{\_}{{\tt\char'137}}     % re-centres the underscore
\markright{\stardocname}
\setlength{\textwidth}{160mm}
\setlength{\textheight}{230mm}
\setlength{\topmargin}{-2mm}
\setlength{\oddsidemargin}{0mm}
\setlength{\evensidemargin}{0mm}
\setlength{\parindent}{0mm}
\setlength{\parskip}{\medskipamount}
\setlength{\unitlength}{1mm}

%------------------------------------------------------------------------------
% Add any \newcommand or \newenvironment commands here
%------------------------------------------------------------------------------

\begin{document}
\thispagestyle{empty}
SCIENCE \& ENGINEERING RESEARCH COUNCIL \hfill \stardocname\\
RUTHERFORD APPLETON LABORATORY\\
{\large\bf Starlink Project\\}
{\large\bf \stardoccategory\ \stardocnumber}
\begin{flushright}
\stardocauthors\\
\stardocdate
\end{flushright}
\vspace{-4mm}
\rule{\textwidth}{0.5mm}
\vspace{5mm}
\begin{center}
{\LARGE\bf \stardoctitle}
\end{center}
\vspace{5mm}

%------------------------------------------------------------------------------
%  Add this part if you want a table of contents
%  \setlength{\parskip}{0mm}
%  \tableofcontents
%  \setlength{\parskip}{\medskipamount}
%  \markright{\stardocname}
%------------------------------------------------------------------------------

\section {Introduction}

This paper describes the procedure for submission of a piece of
software for inclusion in the Unix Starlink Software Collection (USSC).

The organization of the USSC is described
in SSN/66, which must be read by anyone considering submitting
software for inclusion in the Collection. The present document assumes that
the reader is familiar with the contents of SSN/66, in particular the
layout of the Starlink directories and the use and purpose of make files.
SUN/118 describes the current USSC and how to use it. Some of the material in
the introduction to SUN/118 may also be relevant to your software submission.

The USSC is managed by the Starlink Software Librarian who for the USSC only
is temporarily Chris Clayton (RLVAD::CAC).

\section {Development and Submission Standards}

SGP/19 contains instructions and advice for anyone submitting software for
inclusion in the Starlink Software Collection for VMS. Most of the advice in
that document on topics such as programming standards and documentation is
equally applicable to software for UNIX systems and is not repeated here.
Anyone wishing to submit software for the USSC should first read Sections 3, 4
\& 5 of SGP/19.

\section{Submission Procedure}

In order to submit a piece of software for inclusion in the USSC, you must fill
in the {\em ``UNIX STARLINK SOFTWARE ITEM SUBMISSION FORM''\/} (USSISF) at the
end of this document and send it to the Starlink Software Librarian. The most
important items in the form are discussed below. The item numbers refer to the
section numbers on  the form.

\begin{enumerate}

\item Full title and Acronym for the Software.
Every item should have a short single word descriptive name associated
with it; this is the `acronym'.
For example, ``Hierarchical Data System'' is a full title and ``HDS'' is
an acronym.

\item The level of support that this software will receive.
The five support levels that Starlink recognizes are listed in SGP/2.
The names of those individuals providing the support should also be included.

\item An indication of which Starlink Staff (if any) you have been
coordinating the development of this software with.

\item If this is a modification to a previous release of the software, a list
of modifications must be provided.

\item A statement of which architectures the software is supported on. At present
Starlink will distribute software for SPARC machines and
DECstations/DECsystems.  The versions of the operating systems that the software
has been tested against must also be given (e.g. SunOS 4.1.1,  Ultrix 4.2).

\item A list of which compilers the software can be built with. At present
Starlink will accept software built with SUN Fortran, DEC Fortran for RISC,
GNU C and
DEC C. Version numbers of the compilers used should be stated (e.g. SUN FORTRAN
1.4, GNU C 1.35).

A list of environment variables that need to be set before the make can
take place must be supplied. Ideally, compilation f
lags should be set within the make file
itself and only the environment variable SYSTEM should need to be set.
The following values for SYSTEM should be used:

\begin{itemize}

\item {\bf sun4} -- SPARC architecture

\item {\bf mips} -- DECstations and DECsystems


\end{itemize}

A list of actions that might need to take place before the
built software can be run. These should be listed on the form under
``Any other unusual requirements''. For example, the package may require a
set--up file to be run first or some environment variables may need to
be set.

It should be noted that the USSC is in an early stage of evolution and it
is likely that the mechanisms with which we support multiple architectures
may have to change.

\item The source code of the files to be released and the expected location
of each source file in the {\tt /star} tree. This location may be changed by
the  Starlink  Software Librarian and hence only path names for source
directories relative to the make file should be used in a make file. The
makefile will usually  be stored in the same directory as the source code.
Executables are not required. In general, the UNIX Starlink software
distribution  service only distributes source code and executables appropriate
for the local machines are built at each target site. This allows Starlink to
support multiple architectures with minimal overheads.

A make file that will build the software from the source files and
install it in the appropriate Starlink directories must be supplied.
The make file should have three standard targets that do the following:

\begin{itemize}

\item {\bf build} --- Builds the package from its source files but does not
install it. The {\bf build} target, should be the first target
in the make file. This is useful for developing software and testing it
in a local directory.

\item {\bf install} --- Installs the files created with the {\bf build}
target into {\tt /star/bin} etc.

\item {\bf clean} --- Deletes intermediate files created during the build
operation.

\end{itemize}

You may also wish to have a test target:

\begin{itemize}


\item {\bf test} --- Tests that the installation of the package has been
successful. The nature of this test will vary from package to package but as
an example, the {\bf test} target might build a piece of software that is
dependent on the software that has just been installed and display a
message of the form ``{\bf Installation of xyz successful}'' when run
(see item 12).

\end{itemize}

Full details of the usage and structure of Starlink make files (including an
annotated example make file for packages) can be found in SSN/66.

\item A list of dependencies on any software that is not a part
of the USSC.
If your software depends on other software that is not in the Collection it
may not work when installed.
If it does use such software, consider releasing it for the Collection.
Remember that if your software proves difficult to install, it will be returned
to you for action.
The software contained in the USSC is specified in SUN/118.

\item A list of dependencies on other pieces of Starlink software (e.g.
SAE\_PAR, EMS, HDS, etc.).

\item A list of the make commands that should be executed to build and install
the software and clean up afterwards. The required sequence is shown in the
following example:

\begin{quote}
{\tt
\% cd /star/starlink/lib/hds

\% make build

\% make install

\% make clean
}
\end{quote}

\item A description of how to run the software. You may point to supplied
documentation.

\item A description of some sort of test that can be carried out to ensure
that the software has successfully been installed. For example, you could
use the test target in the make file to perform the test. An example
test target is illustrated below:

\begin{verbatim}

#  Target for performing an installation test.
test:
        f77 hds_test.f `/star/bin/hdslink` `emslink` -o hds_test.out
        hds_test.out
        rm hds_test.out

\end{verbatim}


Such a test would then be executed with the command:

\begin{quote}
{\tt
\% make test
}
\end{quote}

It is comforting to be able to carry out a simple test on your software
after it has been installed in Starlink directories to show that it gives the
same results as it did when you ran it.
Remember to say what to expect if the test is a success and to supply any test
data that might be needed.

\item A list of any unusual or large resource requirements.

\item Relevant documentation. At present, this is not sent out with the
UNIX software but instead is still distributed via the VMS software
distribution mechanism. Advice on preparing documentation can be found in
SGP/20. A Starlink User Note (SUN) is the preferred type of documentation.
A news item may also be submitted, if appropriate.

Software should not be included in the Collection unless it is documented
to a standard such that another user can use your software successfully with no
help other than the documentation or on-line support.
If possible, your documentation should exist in one or more computer files in
\LaTeX\ format.
These are the easiest forms to change and edit.
The Librarian will normally generate a master copy and distribute copies of this
as part of the release process.
Do not submit files which depend on private libraries of macros (for instance).
Your file should be usable by the Librarian without knowledge of any other of
your files.
If your documentation cannot be prepared in this way (e.g. it needs diagrams or
the text is not available in a computer file), send a master of anything that
needs to be added or copied.
The Librarian should normally be the person to issue documentation associated
with software items being released.


\item If applicable, a recipe for a demonstration of the package would be
useful. There are two types of demonstration that would be useful.

\begin{itemize}

\item Demonstration for astronomers --- This type of demonstration should
show off the capabilities of the package to would--be users.

\item Demonstration for visitors ---
Visitors sometimes turn up and ask to be given a demonstration of
``Starlink''. Usually what they want is a colourful, photogenic display of
graphs or pictures. The people who are caught on the spot on these occasions
may have very little practical experience in actually running this kind of
thing and the visitors are likely to go away wondering why such an expensive
and powerful system doesn't seem able to do that much. An easy--to--run
demonstration of your software would help avoid this situation.

\end{itemize}

\item A description of the expected installation procedure. Please make this
as detailed as possible to minimize further interaction between the
Starlink Software librarian and yourself.

\end{enumerate}


\newpage
\small
\appendix
\section {UNIX STARLINK SOFTWARE ITEM SUBMISSION FORM}
Please read SGP/11 before attempting to complete
this form.
If the answer to a question is given in a supplied document, give a reference
as the answer.
If an answer is too long for the space provided, attach a separate sheet and
identify the answer by the question number.
Send the completed form to the Starlink Software Librarian at RAL.

**************************************************************************************
\begin{tabbing}
10.X\=XXX\=\kill
1.\>Title:\\
\>Acronym:\\
\\
2.\>Names of:\\
\>\>person submitting this item:\\
\>\>anyone else who can answer technical queries:\\
\>\>person(s) responsible for support:\\
\>\>support level:\\
\\
3.\>With which Starlink staff have you been coordinating development of this
software?\\
\\
\\
4.\>Is it a new item:\hspace{30mm}Modification to old item:\\
\>If it is a modification, what modifications have been made:\\
\\
\\
\>Are there any other items which should be released in conjunction with this
one?\\
\\
\\
5.\>Which hardware platforms does this software run on? (specify)
\\
\\
\\
6.\>Specify any special environmental requirements:\\
\>* What version of SunOS does it run under (if applicable)?\\
\>* What version of Ultrix does it run under (if applicable)?\\
\>* Which compilers can be used to build the software? (include version)\\
\\
\\
\>* Which environment variables must be set to build the software? (specify):\\
\\
\\
\\
\\
\>* Does this program run under ADAM?\\
\>* Does it use any special device drivers? (state)\\
\>* Does it use any special devices, eg. X-Windows device? (state)\\
\\
\\
\\
\>* Any other unusual requirements?\\
\\
7.\>Specify completely the files/directories to be copied into Starlink
directories:\\
\\
\\
\\
\\
8.\>Does this item make use of any software (apart from the operating system) \\
\>which has not been installed in the UNIX Starlink Software Collection? (specify)\\
\\
\\
\\
\\
9.\> Does this item have any dependencies on any other piece of Starlink
software \\ (e.g. GKS, EMS, HDS, etc.)? (specify)\\
\\
\\
\\
\\
10.\>Specify how to create the executable program or system:\\
\\
\\
\\
\\
11. Specify how to execute the program or system:\\
\\
\\
\\
\\
12.\>Specify a test which can be carried out to verify that this\\
\>item has been installed correctly and any test data needed.\\
\\
\\
\\
\\
13. Specify any unusual or large resource requirements:\\
\\
\\
\\
\\
14.\>Where is the documentation which supports this software?\\
\>(eg. name of file holding this document, paper documents attached)\\
\\
\\
\\
\\
15.\>If relevant, specify a demonstration procedure that can be used to\\
\>show off this item's capabilities to visitors.\\
\\
\\
\\
\\
16.\>Installation procedure:
\end{tabbing}

\end{document}
