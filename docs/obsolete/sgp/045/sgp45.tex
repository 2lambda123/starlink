\documentclass[twoside,11pt]{article}
\pagestyle{myheadings}

% -----------------------------------------------------------------------------
\newcommand{\stardoccategory}  {Starlink General Paper}
\newcommand{\stardocinitials}  {SGP}
\newcommand{\stardocnumber}    {45.1}
\newcommand{\stardocsource}    {sgp\stardocsource}
\newcommand{\stardocauthors}   {B K McIlwrath}
\newcommand{\stardocdate}      {13 September 1995}
\newcommand{\stardoctitle}     {ADAM Review Panel Report}
% -----------------------------------------------------------------------------

\newcommand{\stardocname}{\stardocinitials /\stardocnumber}
\markright{\stardocname}
\setlength{\textwidth}{160mm}
\setlength{\textheight}{230mm}
\setlength{\topmargin}{-2mm}
\setlength{\oddsidemargin}{0mm}
\setlength{\evensidemargin}{0mm}
\setlength{\parindent}{0mm}
\setlength{\parskip}{\medskipamount}
\setlength{\unitlength}{1mm}

% -----------------------------------------------------------------------------
% Hypertext definitions.
% These are used by the LaTeX2HTML translator in conjuction with star2html.

% Comment.sty: version 2.0, 19 June 1992
% Selectively in/exclude pieces of text.
%
% Author
%    Victor Eijkhout                                      <eijkhout@cs.utk.edu>
%    Department of Computer Science
%    University Tennessee at Knoxville
%    104 Ayres Hall
%    Knoxville, TN 37996
%    USA

%  Do not remove the %begin{latexonly} and %end{latexonly} lines (used by
%  star2html to signify raw TeX that latex2html cannot process).
%begin{latexonly}
\makeatletter
\def\makeinnocent#1{\catcode`#1=12 }
\def\csarg#1#2{\expandafter#1\csname#2\endcsname}

\def\ThrowAwayComment#1{\begingroup
    \def\CurrentComment{#1}%
    \let\do\makeinnocent \dospecials
    \makeinnocent\^^L% and whatever other special cases
    \endlinechar`\^^M \catcode`\^^M=12 \xComment}
{\catcode`\^^M=12 \endlinechar=-1 %
 \gdef\xComment#1^^M{\def\test{#1}
      \csarg\ifx{PlainEnd\CurrentComment Test}\test
          \let\html@next\endgroup
      \else \csarg\ifx{LaLaEnd\CurrentComment Test}\test
            \edef\html@next{\endgroup\noexpand\end{\CurrentComment}}
      \else \let\html@next\xComment
      \fi \fi \html@next}
}
\makeatother

\def\includecomment
 #1{\expandafter\def\csname#1\endcsname{}%
    \expandafter\def\csname end#1\endcsname{}}
\def\excludecomment
 #1{\expandafter\def\csname#1\endcsname{\ThrowAwayComment{#1}}%
    {\escapechar=-1\relax
     \csarg\xdef{PlainEnd#1Test}{\string\\end#1}%
     \csarg\xdef{LaLaEnd#1Test}{\string\\end\string\{#1\string\}}%
    }}

%  Define environments that ignore their contents.
\excludecomment{comment}
\excludecomment{rawhtml}
\excludecomment{htmlonly}

%  Hypertext commands etc. This is a condensed version of the html.sty
%  file supplied with LaTeX2HTML by: Nikos Drakos <nikos@cbl.leeds.ac.uk> &
%  Jelle van Zeijl <jvzeijl@isou17.estec.esa.nl>. The LaTeX2HTML documentation
%  should be consulted about all commands (and the environments defined above)
%  except \xref and \xlabel which are Starlink specific.

\newcommand{\htmladdnormallinkfoot}[2]{#1\footnote{#2}}
\newcommand{\htmladdnormallink}[2]{#1}
\newcommand{\htmladdimg}[1]{}
\newenvironment{latexonly}{}{}
\newcommand{\hyperref}[4]{#2\ref{#4}#3}
\newcommand{\htmlref}[2]{#1}
\newcommand{\htmlimage}[1]{}
\newcommand{\htmladdtonavigation}[1]{}

% Define commands for HTML-only or LaTeX-only text.
\newcommand{\html}[1]{}
\newcommand{\latex}[1]{#1}

% Use latex2html 98.2.
\newcommand{\latexhtml}[2]{#1}

% Starlink cross-references and labels.
\newcommand{\xref}[3]{#1}
\newcommand{\xlabel}[1]{}

%  LaTeX2HTML symbol.
\newcommand{\latextohtml}{{\bf LaTeX}{2}{\tt{HTML}}}

%  Define command to recentre underscore for Latex and leave as normal
%  for HTML (severe problems with \_ in tabbing environments and \_\_
%  generally otherwise).
\newcommand{\setunderscore}{\renewcommand{\_}{{\tt\symbol{95}}}}
\latex{\setunderscore}

% -----------------------------------------------------------------------------
%  Debugging.
%  =========
%  Un-comment the following to debug links in the HTML version using Latex.

% \newcommand{\hotlink}[2]{\fbox{\begin{tabular}[t]{@{}c@{}}#1\\\hline{\footnotesize #2}\end{tabular}}}
% \renewcommand{\htmladdnormallinkfoot}[2]{\hotlink{#1}{#2}}
% \renewcommand{\htmladdnormallink}[2]{\hotlink{#1}{#2}}
% \renewcommand{\hyperref}[4]{\hotlink{#1}{\S\ref{#4}}}
% \renewcommand{\htmlref}[2]{\hotlink{#1}{\S\ref{#2}}}
% \renewcommand{\xref}[3]{\hotlink{#1}{#2 -- #3}}
%end{latexonly}
% -----------------------------------------------------------------------------
% Add any document-specific \newcommand or \newenvironment commands here

% -----------------------------------------------------------------------------
%  Title Page.
%  ===========
\begin{document}
\thispagestyle{empty}

%  Latex document header.
\begin{latexonly}
   CCLRC / {\sc Rutherford Appleton Laboratory} \hfill {\bf \stardocname}\\
   {\large Particle Physics \& Astronomy Research Council}\\
   {\large Starlink Project\\}
   {\large \stardoccategory\ \stardocnumber}
   \begin{flushright}
   \stardocauthors\\
   \stardocdate
   \end{flushright}
   \vspace{-4mm}
   \rule{\textwidth}{0.5mm}
   \vspace{5mm}
   \begin{center}
   {\Large\bf \stardoctitle}
   \end{center}
   \vspace{5mm}

%  Add heading for abstract if used.
%   \vspace{10mm}
%   \begin{center}
%      {\Large\bf Description}
%   \end{center}
\end{latexonly}

%  HTML documentation header.
\begin{htmlonly}
   \xlabel{}
   \begin{rawhtml} <H1> \end{rawhtml}
      \stardoctitle
   \begin{rawhtml} </H1> \end{rawhtml}

%  Add picture here if required.

   \begin{rawhtml} <P> <I> \end{rawhtml}
   \stardoccategory\ \stardocnumber \\
   \stardocauthors \\
   \stardocdate
   \begin{rawhtml} </I> </P> <H3> \end{rawhtml}
      \htmladdnormallink{CCLRC}{http://www.cclrc.ac.uk} /
      \htmladdnormallink{Rutherford Appleton Laboratory}
                        {http://www.cclrc.ac.uk/ral} \\
      Particle Physics \& Astronomy Research Council \\
   \begin{rawhtml} </H3> <H2> \end{rawhtml}
      \htmladdnormallink{Starlink Project}{http://www.starlink.ac.uk/}
   \begin{rawhtml} </H2> \end{rawhtml}
   \htmladdnormallink{\htmladdimg{source.gif} Retrieve hardcopy}
      {http://www.starlink.ac.uk/cgi-bin/hcserver?\stardocsource}\\

% HTML document table of contents (if used).
% ==========================================
% Add table of contents header and a navigation button to return
% to this point in the document (this should always go before the
% abstract \section). This places the table of contents on the title
% page. Do not use this if you want the normal behaviour.
   \label{stardoccontents}
   \begin{rawhtml}
     <HR>
     <H2>Contents</H2>
   \end{rawhtml}
   \htmladdtonavigation{\htmlref{\htmladdimg{contents_motif.gif}}
                                            {stardoccontents}}

%  Start new section for abstract if used.
%  \section{\xlabel{abstract}Abstract}

\end{htmlonly}

% -----------------------------------------------------------------------------
%  Document Abstract. (if used)
%  ==================
% -----------------------------------------------------------------------------
%  Latex document Table of Contents. (if used)
%  ===========================================
% \begin{latexonly}
%    \setlength{\parskip}{0mm}
%    \tableofcontents
%    \setlength{\parskip}{\medskipamount}
%    \markright{\stardocname}
% \end{latexonly}
% -----------------------------------------------------------------------------


\begin{quote}
{\it The ADAM Review Panel was constituted by SERC in early 1994 under
the Chairmanship of Professor R. Hills (MRAO, Cambridge) and met for the
first time in March of that year. Its final report was
issued on 16th March 1995 and is reproduced here.
}
\end{quote}


\section{Summary}

The Panel reviewed the work of the ADAM Support Group (ASG).  This group
provides the software infrastructure which underpins much of the data
processing activity carried out by the UK community.  The ADAM `environment'
is used by the applications programmers who produce and maintain the data
reduction packages written for Starlink and, in varying degrees, by other
programmers for a range of astronomical purposes.  The ADAM system is also
used for control and data aquisition at the telescopes in Hawaii and La
Palma.

It should be appreciated that the Panel was not, in general, reviewing the
whole suite of software distributed through the Starlink system (which is
sometimes given the name `ADAM'), but was mainly focused on the lower-level
infrastructure.  This is based on the `Astronomical Data Aquisition and
Management' system, which was created (largely in the Royal Observatories)
for use at the telescopes but was adopted by Starlink at the end of the
1980's.  The development of this system, and especially its transfer to a
range of Unix machines, has taken up a great deal of effort over the last
several years.  These tasks are now, however, largely complete and the level
of staffing has fallen to about 3 posts, which is a relatively small fraction
of the total Starlink effort.

The Panel found that the project had essentially met the goals that had been
set when the ADAM Support Group was set up in 1990 and that good progress had
been made towards the additional targets set by the Starlink Review in 1992.

The use of ADAM for real-time control and data-aquisition purposes was
examined in some detail and submissions were received from observatories and
instrumentation groups about their future plans in this area.  The Panel
concluded that although some effort would be needed to support the existing
real-time uses of ADAM, the future developments were likely to take other
directions.  They recommend that the Observatories and the AAO develop a
coordinated strategy for the development of the software infrastructure
needed for control and data-aquisition, and that the real-time applications
should not be a major focus of the ADAM Support Group in the future.

In the data-reduction area the ADAM environment is used extensively by
programmers in the UK and in particular it underpins the work of the
Starlink applications programmers.  It provides the User Interface that is
seen by the astronomer reducing data and the underlying data-storage
routines, etc., which ensure easy transfer of data between different
packages.  The Panel were presented with the results of the Starlink Software
Survey and also solicited views from the community on further points
concerning the software infrastructure specifically.  They received from the
Project an outline of their future plans for further development of the ADAM
system.

The Panel found that the ADAM environment plays a key role in supporting
a number of major data reduction packages which are widely used in the UK.
It provides an effective solution to the problems of managing large software
systems and there is strong support for it in the community.  So long as the
present concept is continued where a suite of data-analysis programmes is
provided on a national basis and conforming to certain standards, it is
essential that a central group exists to provide the core software
environment.  The Panel noted that there would have been a case for such a
support group in the UK even if Starlink had adopted an environment from an
foreign source.  The Panel accepted that because of the continued
developments in the field it was not practical simply to maintain the
software in its present state; further enhancements to the infrastructure
will need to be undertaken but at a substantially lower level of effort than
had been needed for the transfer to Unix.  The Panel made specific
suggestions in response to the plans proposed by the Project (see below) but
emphasised that flexibility was needed and that the detailed programme of the
ADAM Support Group should be kept under review by the Starlink Panel and the
appropriate Software Strategy Groups.

The Panel also found that a significant number of users prefer to use
environments other than ADAM for their data analysis and find them more
powerful or more convenient for their applications.  This is particularly
true of packages which are de facto international standards, such as AIPS and
IRAF.  The Panel believes that support should be provided for other
environments, especially IRAF, AIPS and in the future perhaps AIPS++, so that
the community is allowed more flexibility.  The Support Group should in
future work on these other environments, in parallel with supporting ADAM,
especially where this is the most cost-effective approach.

The Panel do not feel that it would be desirable or practical to drop ADAM,
especially given that it represents such a large investment of intellectual
effort and money.  Nor did they wish to impose a change to IRAF, particularly
so soon after the move from VMS to Unix.  Instead, the goal should be to
exploit the best features of ADAM but also to provide users with a choice of
environments and for the levels of support for these to be determined by the
demand from the community as monitored by the Starlink Panel.

To reflect this wider role, the Panel suggests renaming the ADAM Support
Group the Applications Support Group.  (It was noted that this name might
lead some to assume that the group actually writes the applications software
but on the whole the Panel preferred this to alternatives such as Environment
Support Group and Infrastructure Support Group.)

The Panel recommends that the current level of staffing (3 staff years per
year) should be retained in order to meet this programme.  With the reduced
emphasis on the ``on-line'' applications and the dropping of some other items
of work on ADAM, which the panel felt were not high priority, it should be
possible for the Group to extend its role to cover other environments (the
main support for which would, of course, come from elsewhere) without an
increase in resources.  A substantial reduction in resources would be
counter-productive in that the much larger amounts of effort used in carrying
out analysis and writing applications would become less efficient, probably
in quite a short time.

The Panel saw no need for future specific reviews of the ASG. The Starlink
Panel should be the body through which changes of direction or emphasis
should be input in future, as part of its remit over the whole Starlink
project.

\section{The Future of ADAM for Real-time Applications}

The panel received detailed reports from groups operating telescopes and
building instruments on their plans for control and data-taking software (which
was the original purpose of the ADAM system).  At present the ING, JCMT and
UKIRT all use ADAM running on VAXes under VMS for controlling telescopes and
instruments.  A Unix version is to be used for SCUBA and SuperCosmos.

There was a general consensus between the sites using ADAM for control
applications, both in regarding the existing system as having provided a
satisfactory environment and in how things might develop in future.  The
three main questions for the future are: the future of VMS ADAM, the support
requirements specific to on-line applications of Unix ADAM and the
anticipated developments in control software over the next few years.

All of the sites currently using VMS ADAM for control applications (JACH, AAO,
LPO) support a large amount of applications software.  It is unrealistic to
expect this to be ported to another environment immediately, and all of the
sites anticipate that VMS systems will be in use for at least 5 years. Opinions
differ on the advisability of freezing the VMS version, however.  It will be
necessary to verify that new releases of ADAM run under all relevant versions of
the operating system.  Given that VMS ADAM is a mature system, the support
required from the Starlink project is likely to be minimal.

The role of Unix ADAM for on-line work is at present restricted to the SCUBA
and SuperCosmos projects. These are both being developed at ROE; the former will
be commissioned on the JCMT shortly.  JACH and ROE therefore require support
for these applications.  In the past, the requirements specific to on-line work
(e.g. the message system) have taken second place to those needed for data
reduction, and some effort has had to be expended by ROE in order to make good
the deficiency. These problems are now largely solved, but support
will be needed over the next 3 years until Unix ADAM can be regarded as a
mature product.  It is unlikely that AAO or LPO/RGO will use Unix ADAM for
control applications in the near future, although some parts of it (e.g. HDS)
are likely to be used in specific cases.

All of the Observatories agreed that Unix ADAM was unlikely to form the
environment for future control systems.  There were two main reasons for this.
Firstly, the AAO DRAMA system addresses many of the deficiencies of the ADAM
environment (for example, the restriction to fixed-length messages and the
over-complex parameter system), runs under VMS, Unix and VxWorks and has high
performance. It can also be used with the emerging standards for command
language and GUI: Tcl and TK.  All of the Royal Observatory sites were
considering the adoption of DRAMA.  Secondly, the control software environment
for the Gemini project does not include ADAM.  The operating systems are Unix
and VxWorks, the user interface is Tcl/TK and the majority of the real-time
software will be written using the EPICS system developed for accelerator
control and running under VxWorks.  Some portions of DRAMA (although not the
entire system) will also be used.  The Royal Observatories are heavily
involved in software development for Gemini, and will therefore use EPICS for
other projects, at least for low-level control.  It is possible to adopt a
hybrid approach by interfacing EPICS either to existing Vax ADAM systems
(as has been done for the WHT WYFFOS spectrograph) or to DRAMA.

It is likely that future systems will use a mixture of EPICS and DRAMA,
initially interfacing to existing ADAM systems, but gradually reducing the
dependence.  Tcl and TK will probably be used to develop user interfaces
(although there are other good alternatives).
The use of Unix ADAM for SCUBA and SuperCosmos was a sensible choice at the
start of these projects, but the arrival of EPICS and DRAMA has changed the
picture considerably and it may well be that these are the only control
applications written in the environment.  We recognize that additional effort
will be required at the Observatories to support DRAMA for use external to
AAO, and to provide local expertise in EPICS.

We recommend that the Royal Observatories and the AAO develop a coordinated
policy for future control software and identify the necessary support
requirements for Unix and VMS ADAM, DRAMA and EPICS.  We anticipate that AAO
and RGO will be best placed to support DRAMA and EPICS, respectively, for UK
users, whilst the Starlink project will retain responsibility for ADAM under
both operating systems.


\section{Priorities of ASG}

This section sketches a recommended future for the data reduction side
of ADAM.

The centrally coordinated support for on-line ADAM, addressed in the previous
section, is assumed to be carried out within the ASG's 3 sy/yr but should not
consume a substantial fraction of that resource.

We have used the `Plans for next three years' handout presented
at the 17/4/94 Panel meeting together with the `Further Development'
section of paper ARP(94)02 presented at the 23/5/94 meeting and
combined them into what we hope is a program for a healthily developing
ASG.  The Panel recommendations are only guidelines for Starlink management,
who are encouraged to continue to exercise their management discretion, and
to report in the normal way to the Starlink Panel, seeking its guidance where
appropriate.

Although our recommendation of 3 sy/yr is the same as the request from the
Starlink project,
our priorities are different.  The Panel consider that on-line
ADAM should be supported on a care and maintenance basis to cope with the
demands of current instruments and those shortly to be delivered
which expect to rely on ADAM (e.g.\, SCUBA). The Panel consider that
development of on-line environments is best carried out by the Royal
Observatories where there is much closer contact with developers and
users of instruments and telescopes.
This change of emphasis should reduce the manpower needed to
maintain a `Status Quo', and we are keen to see some of the developments
suggested in ARP(94)02 occurring and therefore consider that with the
removal of the requirement to develop on-line environment further a
3 sy/yr level should produce a healthily developing ASG.

The Panel felt that as systems and environments evolved it was important to
be in a position to make full use of what was available internationally.
In particular if an application exists, perhaps produced in another
country, which runs in an internationally used environment (e.g.\,
IRAF) significant Starlink effort should not go into duplicating it
within other environments, such as ADAM, unless there was a very clearly
defined reason why this was essential.  In some cases it might even be more
cost effective to produce new applications that run in some other
environment. We note that work is already in progress to allow ADAM
applications to access files in ``foreign'' formats transparently, which
should greatly facilitate the use by astronomers of more than software
environment.

Thus, depending on the relevant circumstances, the ASG might be called on
to help Applications Programmers (APs) working in the ADAM, IRAF, or
AIPS++ environments, or even others.  The ASG (and APs) are encouraged to
widen their remits to use and contribute to international software
environments where, for particular applications, these were more
appropriate than ADAM. Equally the ASG are encouraged to collaborate with
those supporting other environments with the aim of making
appropriate features of ADAM and its applications available to users of other
environments. With the high mobility of astronomers the ultimate goal should
be a common international environment under which all data reduction
applications can be run.  This is not yet in prospect, but the steady
adoption of common standards (e.g.\, FITS for data files, POSIX for the O/S
interface, X-windows and Postscript for graphics) suggests that we are at
least going in the right direction.

The Panel strongly supported the concept of porting software to a PC
environment (such as the free LINUX system) which offers the possibility of
running applications on hardware not paid for by Starlink and without the
need for expensive software licences.

Of the plans for the next three years, the Panel supported the following
using the same order as given by the project. (The staff levels are
notional. We note that the level in ARP(94)02 did not add up to 9.0.)

\begin{itemize}
\item Continue to prepare and test new releases of ADAM
(0.3 sy/yr total 0.9 sy)
\item Continue to provide support for existing software in terms of bug fixes
and requested improvements (0.4 sy/yr total 1.2 sy or less with reducing on-line
importance)
\item Development of existing libraries
(0.6 sy)
\item Provide a library that makes it easier to access data files
(0.3 sy)
\item Provide transparent access to data files from other data reduction systems
(0.3 sy)
\item Further work on GUIs using emerging standards (e.g. Tcl/Tk), if
possible coordinated with other astronomical projects, especially IRAF.
Provide support for a
few specific template applications with built in GUIs (0.4 sy)
\item Provide support for writing applications in C
(0.5 sy)
\item Re-evaluate the parameter system in the light of other systems
(0.3 sy)
\item Provide a new parameter system if a viable and effective way forward
emerges from the above
(1.0 sy)
\item Port to other architectures (especially LINUX, see above)
(0.2 sy /architecture) + 0.5 sy/y long term support (operations) (total 1.5 sy)
\end{itemize}

It was felt that, although documentation was important, user education should be
the responsibility of the site managers not the ASG.

\section{Further Development}

The following initiatives were supported \underline{in a rough order of
priority}.  In each case we reproduce the Project's very rough estimates
of the effort required in the first year and in subsequent years.  Whilst
further initiatives would also be supported it was not considered that
within the 3 sy/yr they would be feasible over the next 3 years, but
encouraged the Starlink Project and Panel to keep these priorities under
review to be sure to respond to developments in the fields.

\begin{itemize}
\item Increased participation in international software consortia
   ($0.0+0.5$).
\item Smooth interoperation with other environments (as opposed simply to data
exchange via FITS files).  This could include the ability to run ADAM
applications within other environments, for example IRAF and AIPS++, but
such a project should only be undertaken if a truly cost-effective way of
achieving it can be found
      ($1.0+0.3$).
\item ADAM enhanced to allow full GUI-based operation via a simple programming
interface
      ($1+0.3$).
\item Proper handling of astrometric information in data files, building on the
IRAS90 work in this area (unless other environments already cover this
adequately)
      ($0.6+0.1$).
\item Improvements to ICL to make it more IDL-like
      ($0.5+0.1$).
\item Replacement programming interfaces using Fortran~90 and/or C++
      ($2+0$).
\end{itemize}

The Panel expects that a support group working on these topics will make a
valuable contribution to the overall productivity of data processing
activities by the astronomical community and is a necessary complement to
effort in other areas such as improved instrumentation.

\section{Panel Membership}

R E Hills (Cambridge, Chairman)

P L Dufton (Belfast)

J P Emerson (QMW,London)

R Laing (RGO)

J Meaburn (Manchester)

J Noordham (Dwingeloo, Netherlands)

C G Page (Leicester)

A Pedlar (Jodrell Bank)

Secretary:  A D le Masurier  (PPARC)

\end{document}
