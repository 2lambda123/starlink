\documentstyle[11pt]{article}
\pagestyle{myheadings}

%------------------------------------------------------------------------------
\newcommand{\stardoccategory}  {Starlink General Paper}
\newcommand{\stardocinitials}  {SGP}
\newcommand{\stardocnumber}    {10.1}
\newcommand{\stardocauthors}   {C.S. Jeffery \\
                                P.C.T. Rees}
\newcommand{\stardocdate}      {1 October 1991}
\newcommand{\stardoctitle}     {Theoretical Astrophysics Subroutine Library}
%------------------------------------------------------------------------------

\newcommand{\stardocname}{\stardocinitials /\stardocnumber}
\renewcommand{\_}{{\tt\char'137}}     % re-centres the underscore
\markright{\stardocname}
\setlength{\textwidth}{160mm}
\setlength{\textheight}{230mm}
\setlength{\topmargin}{-2mm}
\setlength{\oddsidemargin}{0mm}
\setlength{\evensidemargin}{0mm}
\setlength{\parindent}{0mm}
\setlength{\parskip}{\medskipamount}
\setlength{\unitlength}{1mm}

%------------------------------------------------------------------------------
% Add any \newcommand or \newenvironment commands here
%------------------------------------------------------------------------------

\begin{document}
\thispagestyle{empty}
SCIENCE \& ENGINEERING RESEARCH COUNCIL \hfill \stardocname\\
RUTHERFORD APPLETON LABORATORY\\
{\large\bf Starlink Project\\}
{\large\bf \stardoccategory\ \stardocnumber}
\begin{flushright}
\stardocauthors\\
\stardocdate
\end{flushright}
\vspace{-4mm}
\rule{\textwidth}{0.5mm}
\vspace{5mm}
\begin{center}
{\Large\bf \stardoctitle}
\end{center}
\vspace{5mm}

%------------------------------------------------------------------------------
%  Add this part if you want a table of contents
%  \setlength{\parskip}{0mm}
%  \tableofcontents
%  \setlength{\parskip}{\medskipamount}
%  \markright{\stardocname}
%------------------------------------------------------------------------------

\begin {abstract}
It is proposed that a well-documented subroutine library be written in the area
of theoretical astrophysics.
A questionnaire has been circulated to the UK astronomical community.
The response indicated interest in a library of subroutines mainly covering
the area of astronomical spectroscopy.
A preliminary discussion is presented of the intended contents of the
library, in terms of its functionality, design and documentation.
Targets for the development of the initial version of the library are given.
Comments are invited.
\end {abstract}

\section {Introduction}

It has been proposed that a library of proven Fortran subroutines might be
established for theoretical astrophysics calculations used frequently in the
analysis of astronomical spectra.
The aim of the library would be to provide an effective and reliable toolkit for
program development, based on well documented algorithms and coded for a
high degree of portability ({\em e.g.}\ using the STARLINK coding standard --
SGP/16).
Each subroutine in the library would be proven to a documented accuracy for
a minimum of VAX/VMS and UNICOS operating systems.

To implement a theoretical astrophysics subroutine library involves identifying
specific areas of astrophysical computation which can benefit from such a
library and for which existing software may already be available.
Each area may then be looked at in turn by collecting together all the available
code and evaluating the performance of each algorithm.
Because of the amount of manpower available this process has to be a
collaborative one.
Given the number of suitable public-domain routines within the CCP7
program library, an approach was made to CCP7 to collaborate on
such a project and it was agreed by the CCP7 Working Group that Simon
Jeffery (CCP7 secretary) and Paul Rees (STARLINK Project) should proceed
with further steps to establish a library.
A questionnaire was distributed to determine the level and areas of existing
demand and to determine what initial contributions might be available
({\em e.g.}\ from CCP7 Working Group members) in addition to the public-domain
routines within the existing CCP7 program library.

This document discusses the response to the questionnaire and provides an
outline of the initial implementation of the library.

\section {Questionnaire Response}

\subsection {Circulation}
The text of the questionnaire distributed to the UK astronomical
community is given in Appendix \ref{quest_sect}.
This questionnaire was distributed over the STARLINK network and mailed to
all members of CCP7: a total of approximately 1300 professional and graduate
astronomers.
% From this group, a disappointing ten responses were received by the six week
% response deadline.
% and four of these respondents did not fill in the questionnaire.
%
% \subsection {Level of interest}
All respondents answered that they or other members of their research group
would be likely to use a theoretical astrophysics subroutine library as a
programming tool, if such a library were available.
Most were willing to consider replacing their own subroutines with
proven third-party software.

\subsection {Area of interest}
Most respondents listed optical, ultraviolet, infra-red, millimetre and
radio spectroscopy as areas in which they need subroutines now or
will need soon.
X-ray spectroscopy was also identified by several respondents as an area
in which they will need theoretical astrophysics software soon.
Other areas mentioned included interstellar chemistry, plasma physics and
interstellar scattering calculations.

\subsection {Specific applications}

\subsubsection {Calculations}
In order of decreasing frequency, the list of specific calculations required
by the questionnaire respondents is as follows:

\begin {itemize}
\item Voigt function;
\item Planck function;
\item Fermi-Dirac integrals, partition functions, continuous opacities,
equation of state, Rayleigh scattering, particle size integration, Mie
scattering.
\end {itemize}


\subsubsection {Data}
Four of the respondents noted that they require atomic data, noticeably in
the X-ray wavelength region.
This is clearly a large and important area which requires further
detailed investigation.

\subsubsection {Performance}
No clear performance requirements ({\em e.g.}\ speed, accuracy) were
identified from the questionnaire response.

\subsubsection {Contributions}
Fortran code has been offered for the following calculations:

\begin {itemize}
\item Voigt function;
\item Gaunt factors (free-free and bound-free);
\item Rayleigh scattering;
\item dust particle size integration;
\item Mie scattering.
\end {itemize}

\subsection {Evaluation}

Although numerically disappointing, the questionnaire response was of the order
anticipated and established that interest in a theoretical astrophysics
library does exist -- at least within those sections of the UK community
concerned with astronomical spectroscopy.
However, the requirements indicated be the questionnaire responses
are diverse, making the identification of specific areas of demand difficult.
Previous experience suggests that a higher level of interest is likely to be
generated by the existence of a software product, especially over an extended
period of time.
With this in mind, it was decided to design and code an initial version of
the theoretical astrophysics (TAP) subroutine library, with emphasis on
low-level and infra-structure facilities which may subsequently be built
into larger applications at a later date.
There follows a discussion of various design aspects of the initial
implementation of the TAP library.

\section {Functionality}
The question of what an initial implementation of the TAP subroutine library
should provide is difficult to clarify because of the breadth of astrophysics,
even within the field of spectroscopy.
However, it is considered that a minimum requirement must address the
following points:

\begin {description}
\item [Units] -- From the point of view of the development and
maintenance of software, a single standard system of units is highly
attractive.
However, in astrophysics SI, cgs, atomic and other units peculiar to astronomy
are in current use.
Clearly some means of unit conversion is required, preferably with a very
simple user interface using a single subroutine call.

\item [Constants] -- The standard initialisation of constants is fundamental
to the repeatability of computational astrophysics.
Here, three types of constants can be identified:

\begin {itemize}
\item mathematical constants;
\item astronomical constants;
\item physical constants.
\end {itemize}

In addition to establishing the range of constants required, it is also
important to identify the precision and accuracy to which these constants
are initialised.
Also, standard sources of these data ({\em e.g.}\ CODATA, IAU) need to be
identified and the data maintained -- {\em i.e.}\ it is reasonable to assume
that some of these data will require updating as more accurate values are
adopted.

\item [Atomic and molecular data] -- All but the most trivial theoretical
astrophysics calculations require atomic and molecular data.
However, unlike the fundamental constants, there is usually no standard source
for these data.
Also, the types of data available vary substantially, from
lists of wavelengths through compilations of semi-empirical values
to the results of detailed quantum-mechanical calculations.
Moreover, the quality of these data also vary substantially.

It is suggested that only the most reliable data are of interest to the
average user.
Maintaining these data in a library requires regular review
and, ideally, a mechanism for obtaining some measure of quality control
(via consensus from the user community).
Although the continued maintenance overhead for these data must not be
underestimated, the benefits of universal access to a standard source of
reliable atomic and molecular data are considerable.

The mechanism by which such a database could be constructed and maintained
needs to be investigated further.
In this respect, it is noted that this topic is exercising a
number of current research programs ({\em e.g.}\ the Opacity Project and CCP2)
and it is recognised that substantial collaboration will be necessary.

\item [Elementary functions] -- The more complex astrophysical calculations
presume the existence of elementary software, {\em e.g.}

\begin {itemize}
\item Voigt function;
\item exponential integrals;
\item Fermi-Dirac integrals;
\item Planck radiation laws.
\end {itemize}

Clearly, these calculations have to be provided before more complex
calculations are encoded.

\item [Mathematical techniques] -- A number of modern numerical procedures
require specific mathematical techniques ({\em e.g.}\ specific methods for
matrix inversion, integration of simultaneous differential equations)
which are not provided for in standard mathematical subroutine libraries
like NAG.
Code does exist for many of these techniques, but it is frequently embedded
within larger applications.
Extracting these techniques into a documented library would make the
transfer of existing techniques to new applications much more accessible to
the theoretician.

\item [Simple applications] -- Types of calculations which are more specific
than the elementary functions noted previously include:

\begin {itemize}
\item radiative transfer;
\item equation of state;
\item Boltzmann and Saha-Boltzmann distributions;
\item opacity calculations.
\end {itemize}

This type of relatively simple application probably represents the current
limit to which a theoretical astrophysics subroutine library can provide
tools for building larger applications.
Eventually, the provision of code for higher level algorithms as a general
tool is defeated by the law of diminishing returns -- {\em i.e.}\ the
limited number of occasions on which code for such an algorithm is required
for a new application.
\end {description}

\section {Design}
Several issues are important to the design of any subroutine library to
ensure a high level of maintainability.
In the area considered here, some requirements such as numerical accuracy
become increasingly important, whilst others such as interactive
graphics may be overlooked.
Although many design considerations may be obvious, the following minimum
specification is regarded as essential before any coding can begin.

\begin {description}
\item [Language] -- The language which currently dominates computational
physical science is Fortran 77, although both Algol and C are also in use.
Since nearly all existing software relevant to the TAP subroutine library
is written in Fortran, this language continues to be the only realistic option.
Compilers for the new Fortran standard, Fortran 90, will become more widely
available within the next two years.
Fortran 90 will be adopted when it is appropriate to do so.
The STARLINK Fortran coding standard, SGP/16, will be the coding standard used.

\item [Binding] -- The use of third-party software, {\em e.g.}\ the NAG
subroutine library, and infra-structure software, {\em e.g.}\ the STARLINK
character handling routines, need to be considered carefully since their
use has implications for the portability of the TAP subroutine library.
Ideally, all infra-structure code should port to a supported operating
system as though it forms a self-contained component of the TAP library.

\item [Conventions] -- A subroutine naming convention must be adopted, based
upon the recommendations  contained in SGP/16.

\item [Precision] -- All floating point subroutine arguments will be
declared as DOUBLE PRECISION.
In this context, these arguments may be considered reliable to 14 significant
figures unless otherwise stated in the documentation.
All integer subroutine arguments will be assumed four-byte integers, {\em
i.e.}\ lie in the range -2147483647 to 2147483647.
This range takes account of the variation in the representation of negative
integers between machines of different architecture.

\item [Accuracy] -- The accuracy of a calculation is dependent upon the
algorithm used as well as the precision of the arithmetic.
The maximum accuracy of an algorithm can never exceed its precision.
Where necessary, {\em i.e.}\ when returned values may be less accurate than
14 significant figures, a detailed discussion of the accuracy of a calculation
will be documented.

\item [Error checking and flagging] --
The inherited status conventions recommended by STARLINK (SGP/16, SUN/104)
will be adopted.
Use of the STARLINK error reporting software, SUN/104, needs to be investigated.
\end {description}

\section {Documentation}
The documentation of the theoretical astrophysics subroutine library will use
the style adopted by the STARLINK project.
A full discussion of the algorithm used and including
references will be presented in the documentation for each routine.
This aspect of the documentation is regarded of equal importance to the
software itself.

\section {Development}
The development programme outlined here is clearly a considerable task.
However, the emphasis of any development effort must be upon the quality
of the resulting software; {\em i.e.}\ its reliability, accessibility and
continued maintainability.
This emphasis will inevitably result in a slower initial development, but
it should avoid set-backs resulting from maintenance and redesign as the
library becomes more mature.

Prerequisites for the implementation of the astrophysical calculations are
the consistent initialisation of astrophysical constants and the
conversion of units.
These two areas require immediate attention; elementary functions may then be
implemented.
It is intended to begin development of a preliminary version of the library
immediately.
Comments on any relevant aspects will be welcomed by either author.

\newpage
\appendix
\section {Questionnaire} \label{quest_sect}

\small
\begin{verbatim}
        THEORETICAL ASTROPHYSICS SUBROUTINE LIBRARY QUESTIONNAIRE
        =========================================================

        1. NAME AND E-MAIL ADDRESS



        2. LEVEL OF INTEREST

        Given that the use of well-documented subroutine libraries assists
        in the long-term maintenance and portability of applications
        software, would you or other members of your research group be
        likely to use  an astrophysics subroutine library as a programming
        tool?

        Would you consider replacing your own (perfect!) FORTRAN
        subroutines with proven 3rd party software?


        3. AREA OF INTEREST

        In which general areas do/might you use/require subroutines?

                                                Use     Need    Might
                                                now     soon    need

        Optical/UV stellar spectroscopy
        IR/mm-wave/radio spectroscopy
        Interstellar chemistry
        Other (please specify)...............


        4. SPECIFIC APPLICATIONS

        If well-documented and proven FORTRAN subroutines were available,
        which of the following specific task descriptions would you be
        likely to use. Similarly, would you be prepared to make a
        contribution in any of the following areas.

                                                Require         Contribute

        Trivial...
          Planck function                       ........        ........

          Other (specify) .................     ........        ........

        Utilities...
          Voigt function                        ........        ........
          Fermi-Dirac integrals                 ........        ........
          Partition functions                   ........        ........
          Occupation numbers                    ........        ........
          Continuous opacities                  ........        ........
          Atomic data                           ........        ........
          (photoionisation x-sections, etc)

          Other (specify) .................     ........        ........

        Compound...
          Equation of state                     ........        ........

          Other (specify) .................     ........        ........

        Performance...
          Please give a brief account any specific accuracy or other
          performance requirements you consider necessary from the
          applications you have specified.



        5. CONTRIBUTIONS

        Please describe in more detail any subroutines you are prepared
        to contribute. In particular, we would like to know:

          1. the name by which you know the subroutine
          2. the general subject area
          3. the precise function of the routine
          4. approximate size (lines of FORTRAN) and documentation level
          5. references

        If the offered contribution is a suite of routines for a single
        task or set of tasks, the name of the top-level routine and the
        function of the entire suite need only be supplied.

        Example:

          1. PLANCK
          2. Radiation laws
          3. Evaluate Planck function at a single frequency
          4. 4 lines of FORTRAN, 0 comments
          5. Planck,M. 1959. The Theory of Heat Radiation, New York: Dover.



                           Thank you for your response.
\end{verbatim}
\end{document}
