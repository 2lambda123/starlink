\documentstyle{article}
\pagestyle{myheadings}
\markright{SG/1.2}
\setlength{\textwidth}{160mm}
\setlength{\textheight}{240mm}
\setlength{\topmargin}{-5mm}
\setlength{\oddsidemargin}{0mm}
\setlength{\evensidemargin}{0mm}
\setlength{\parindent}{0mm}
\setlength{\parskip}{\medskipamount}
\setlength{\unitlength}{1mm}

\begin{document}
\thispagestyle{empty}
SCIENCE \& ENGINEERING RESEARCH COUNCIL \hfill SG/1.2\\
RUTHERFORD APPLETON LABORATORY\\
{\large\bf Starlink Project\\}
{\large\bf Starlink Guide 1.2}
\begin{flushright}
C R Benn\\
1 April 1988
\end{flushright}
\vspace{-4mm}
\rule{\textwidth}{0.5mm}
\vspace{5mm}
\begin{center}
{\Large\bf ASPIC}
\end{center}
\vspace{5mm}

{\setlength{\parskip}{0mm}
\tableofcontents}

\markright{SG/1.2}
\newpage

\section {INTRODUCTION}

ASPIC is a collection of about 400 image-processing programs.
This guide is a summary of useful information about them, with particular
reference to simple interactive display and analysis of 2-D images.
A complete list of programs is given in Appendix A.
Detailed information is given for 53 of the most commonly-used
programs in Appendix B.

Send information about bugs in ASPIC programs or documentation to the
Starlink Software Librarian (RLVAD::STAR).
These will be recorded in DOCSDIR:BUGS,
and may or may not get fixed.
Software support for ASPIC is in the future likely to
be very limited (see Section 7).

This guide was originally produced at the Royal Greenwich Observatory
in May 1986, and
has been prepared for release in its present form by M D Lawden
at the Rutherford Appleton Laboratory.

\subsection {Documentation}

Starlink User Note 23 (SUN/23)
introduces ASPIC and gives classified lists of programs.
SUN/24 and SUN/27 contain more detailed information on some of the programs.
The most detailed source of information is the on-line help library
- see Section 1.5.
For information on the DSCL command language,
under which ASPIC programs are usually run,
see SUN/74.

For documentation on other Starlink subjects, try typing
{\tt @SSC:LOGIN}
then {\tt FIND subject-name}.
Print file DOCSDIR:SUBJECT.LIS
for a full list of subjects.
For further pointers to documentation,
see section 4 of the Starlink User Guide.

\subsection {ASPIC}

ASPIC stands for AStronomical PICtures.
The programs are written by many different authors, but have in common that they all use the Starlink INTERIM
routines for data handling (SUN/4).

The programs are grouped into the following classes (sometimes referred to
as `packages'):

\begin{description}
\item[ARGS] :
Image display and manipulation programs for the ARGS device.
May be suitable for other image displays such as the IKON.
See also {\bf LUT}.
\item[ARITHMETIC] :
Scalar and array operations.
\item[ASTROMETRY] :
Programs to input and manipulate r.a., dec.\ lists.
\item[COMPRESS] :
Programs to pack images into as few bits as possible.
\item[DETAILS] :
 Extraction of summary information from images.
\item[DISPLAY] :
Miscellaneous programs,
not covered by the ARGS `package',
for displaying images.
\item[DOCS] :
Creation and printing of ASPIC documents.
\item[EDRS] :
An Electronography Data Reduction System; see SUN/24.
Type `GO EDRS' before using this package.
A full manual (90 pages) resides in [STARLINK.PACK.ASPIC.EDRS]\-MANUAL.\-LIS;
a copy may be available at your site.
\item[FC] :
Combination of up to 3 images into a single false-colour picture.
\item[FILTER] :
Non-Fourier noise filtering.
\item[FOURIER] :
Basic Fourier-transform programs.
\item[GALPHOT] :
Bright-galaxy photometry; see SUN/24.
\item[GEOMETRY] :
Geometrical transformations.
\item[GRASP] :
Giant Raster And Stellar Photometry system; an extension of EDRS.
\item[INPUT] :
Creation of Starlink images (.BDF files).
\item[LUT] :
Programs to load or change the lookup table for the ARGS display.
\item[MISCELLANEOUS] :
$\approx$ 50 miscellaneous programs; see SUN/24 and SUN/27.
\item[PERIODS] :
Programs to search for periods in unequally-spaced photometric data; see SUN/24.
Type `GO PER' before using this package.
\item[PHOTOMETRY] :
Photometry programs from ROE; see SUN/24.
\item[POLARIMETRY] :
Polarimetry programs; see SUN/24.
\item[SIZE] :
Programs to expand, contract or select part of an image.
\end{description}

The main photometry programs are in the EDRS, GRASP and PHOTOMETRY classes.

Many sites have also installed local ASPIC programs and packages; ask your
Site Manager about these.

\subsection{ARGS}

The ARGS displays 16 512x512 pixel planes.
Planes 0-7 are picture, 8-15 are overlay.
The least significant 8 bits of any pixel are used as an index (value 0-255)
to the ARGS lookup table, each entry of which consists of a triad of values
in the range 0-255, representing intensities for red, green and blue guns.
The overlay planes are used for vector graphics, labels, etc.
To clear the overlay planes use program {\bf ADISOV}.

For information on ARGSLIB subroutines see SUN/10, LSN/11 (RGO) and
LSN/12 (RGO).

Individual Starlink sites may operate online or offline
booking systems for the ARGS
(e.g. at the RGO, there is an online {\bf BOOK} command).

\subsection{INTERIM}

The INTERIM Starlink software environment is described in SUN/4.
It comprises 36 user-callable routines which interface programs with:
\begin{itemize}
\item the user
\item default parameter settings in the connection file (program.CON) which
accompanies every program file (program.EXE).
\item images, which are stored in Starlink bulk data frames (.BDF files).
\end{itemize}
.BDF files cannot be read via normal Fortran I/O.
If you attempt to type a .BDF file at a terminal, unrelated files may be read out
and the terminal will probably go catatonic.
You will have to power the terminal OFF/ON to recover.
Don't try printing a .BDF file either; it screws up the printer.

Two different command languages are available: RUNSTAR and DSCL.
RUNSTAR is adequate for development work, but DSCL is normally selected for
running INTERIM programs such as those in ASPIC
(but see Section 3).

\subsection {DSCL}
DSCL was written primarily as a command environment for ASPIC but can be used
to run any program.
SUN/23 includes a brief introduction to DSCL; it is  described fully in SUN/74.

DSCL provides on-line help for ASPIC, a stack for intermediate storage of
images, and a convenient way of handling parameters.
The price paid by the user for this flexibility is slowness.
See Section 3 for details of the faster ZAP command environment
available at some sites.

To set parameter values, type `parameter-name = value', e.g.\ ADISP PVHI=3000
PVLO=100.

DSCL is activated by typing {\tt DSCL}.
Here are some examples of DSCL commands:
\begin{description}
\item[DEMO/ECHO]:
runs a demonstration program using a 256x256 PDS scan of the Horsehead Nebula
on an UKSTU plate.
\item[FILEDOC ARGS ADISP]:
sends a copy of the help information on program ADISP in class ARGS to file
ADISP.DOC.
\item[GO EDRS]:
allows you to access commands (including the help library) of the EDRS package.
The GO command is unnecessary for most ASPIC packages (except PER and CHART).
\item[HELP ARGS]:
gives help on the ARGS class.
\item[HELP ARGS AFLASH]:
gives help on program AFLASH in class ARGS.
\item[HELP ASPIC]:
gives brief information on ASPIC.
\item[HELP DOCS]:
gives information on printing documents.
\item[HELP DSCL]:
gives brief information on DSCL.
\item[HELP * STARFIT]:
locates the class containing STARFIT and gives help.
\item[LOOK TBXY]:
looks at the parameter settings for program TBXY.
\item[STOP]:
exits from DSCL in an orderly fashion.
\item[WHERE]:
types a list of ASPIC program titles (without specifications).
\item[WHERE CONTOUR]:
types the name of the class containing the CONTOUR program.
\end{description}

\subsection{Batch jobs}

Create a command file (BATCH.COM say) containing the commands to be
executed, for example:
\begin{verbatim}
     $ @SSC:LOGIN
     $ RUNSTAR GREYSCALE/DEVICE=PRINTRONIX/DEFAULT=T/INPUT=5C12.101/LABEL=5C12.101
\end{verbatim}
and submit it for execution:
\begin{verbatim}
     $ SUBMIT BATCH
\end{verbatim}
Your site may run a fast queue which you can use for short jobs, for example
at RGO type:
\begin{verbatim}
     $ SUBMIT/Q=SYS_SHORT BATCH
\end{verbatim}
WARNING --- {\em Don't} use the command `DSCL' in a command procedure, and
{\em Do} specify all the parameters for which the user would normally be prompted.
Otherwise, the job may enter an infinite loop.

\section {Using ASPIC}

To use ASPIC, login to the Starlink VAX and type {\tt @SSC:LOGIN} (unless
you have sensibly put this into your LOGIN.COM file), then type
{\tt DSCL} which sets up the DSCL command environment.
The prompt {\tt DSCL$>$} is an invitation to enter the name of an
application program (e.g.\ PICK), the name of a DSCL procedure (e.g.\
LUTGREY), a DSCL command (e.g.\ HELP LUT), or a DCL command (e.g. DIR).

\subsection {Copying images to/from tape,
and conversion to/from other formats}

To read FITS images from a tape, {\tt MOUNT/FOR MTA0} (or {\tt MTA1}, etc.)
and use {\bf FITSIN} (LUN/48 (RGO)).
{\bf FITSHEAD} may be used to read the frame-headers.
{\bf FITSINDEX} (LUN/70 (RGO)) produces a summary listing of files on the tape.
It was designed to summarise parameters on the tape generated at the La Palma
telescopes but may work (and can easily be modified to do so) for tapes from
other telescopes.
See SUN/3 for details of {\bf IPCSIN} and {\bf VICARIN}, and LUN/9 (MAN)
for {\bf PDSIN}.

To convert Starlink .BDF files to FITS format, use {\bf FITSOUT}
(LUN/49 (RGO)).
To copy .BDF files to and from tape, use e.g.\ {\tt MOUNT MTA0}, followed by
{\tt COPY *.BDF MTA0}.

To rewind a tape without unloading, {\tt DISMOUNT/NOUNLOAD MTA0}.

To convert .BDF files to KAPPA HDS image format (.SDF files), use the KAPPA program {\bf STARIN} (SUN/96).
At the time of writing, there is no program for the inverse
conversion KAPPA HDS
$\rightarrow$
Starlink .BDF.

To convert to and from FIGARO (SUN/86) HDS format (.DST files),
use the FIGARO programs {\bf STARIN} and {\bf STAROUT}.
To convert a KAPPA HDS data structure to a
FIGARO HDS structure, use
the FIGARO program {\bf RENOBJ}.

\subsection {Displaying images and image information}

Use {\bf DESCR} to look at image headers, {\bf WRDSCR} and {\bf CYDSCR} to
alter them.

The quickest image-display program is {\bf AFLASH}, which does not do any
rescaling.
Use of {\bf AFLASH} before programs like {\bf SLICE} and {\bf APERASP} should be
avoided because {\bf AFLASH} interferes with the overlay planes and strange
colours may result (use {\bf ADISOV}or {\bf LUTGREY} to clean them up).
{\bf APIC} is also fast, and allows interactive manipulation of scale and zero
point.
{\bf ADISP} is preferred, although it is slower --- it requests low
and high pixel values between which to load a greyscale.
Use {\bf STATS} after setting {\tt STATS RMS=TRUE} for a formal estimate of
mean and rms.
{\bf SLICE} and {\bf PEEP} allow a more direct inspection of the mean and rms
levels.
{\bf ADHC} generates a high-contrast display of an image.

To clear the screen, or reset the ARGS completely, use {\bf ACLEAR} or
{\bf ARESET}.

{\bf APAN} pans and zooms the image.

{\bf ABLINK} displays and blinks two superimposed images.
{\bf ALIGN} may be used to align two images under trackerball control.
{\bf ADD} adds two images.
There is a program inside ZAP (Section 3) to add together more than two images.

{\bf ATEXT} places text on the ARGS display.
{\bf AHARDCOPY} generates a Printronix copy of the current ARGS display; the
parameter THRESH determines the boundary between black and white.
{\bf VERGREY} and {\bf GREYSCALE} generate pseudo-greyscale output for the
Versatec and Printronix respectively --- both take about a minute to process a
512x512 image, and both print automatically (see figure~\ref{greyscale}).
{\bf VIEW} may also be used to generate hardcopy.
Note that the Versatec is obsolete.

{\bf PEEP} types a 9x9 rectangle of an image at your terminal.
{\bf LIST} sends a complete listing of an image to a file.
To list part of an image, try {\bf PICK} (which creates a new image from part
of an existing one), then {\bf LIST}.

{\bf LUTGREY} loads the default greyscale lookup table.
{\bf LUTCOL} loads a colour lookup table (in {\tt ASPDIR:\-CLUT.\-BDF}).
{\bf LUTREAD} loads any lookup table.
{\bf LUTROT} cycles the lookup table using the trackerball.
The LUT class contains several commands for manipulating the lookup table.

{\bf CONTOUR} produces a contour map of an image on one of several devices,
from which it can be printed using {\bf AHARDCOPY} (see figure~\ref{contour1}
and figure~\ref{contour2}).

{\bf ASAVE} saves the current ARGS pixel store in a .BDF file.

The figures in this section show examples of output from some of these
programs.
Figure~\ref{greyscale} shows a 150x150 section of a CCD image output on
a Printronix by program {\bf GREYSCALE}.
\begin {figure}[htbp]
\begin {center}
\begin {picture}(118,88)
\put (0,0){\line(1,0){118}}
\put (118,0){\line(0,1){88}}
\put (118,88){\line(-1,0){118}}
\put (0,88){\line(0,-1){88}}
\end {picture}
\caption {{\bf GREYSCALE} --- Sample output}
\label {greyscale}
\end {center}
\end {figure}
Figure~\ref{contour1} shows the same image as in figure~\ref{greyscale},
but produced by displaying the image on the ARGS using {\bf CONTOUR},
and then outputting it on the Printronix using {\bf AHARDCOPY}.
The bottom contour is about 2 sigma above the mean level.
\begin {figure}[htbp]
\begin {center}
\begin {picture}(160,150)
\put (0,0){\line(1,0){160}}
\put (160,0){\line(0,1){150}}
\put (160,150,){\line(-1,0){160}}
\put (0,150){\line(0,-1){150}}
\end {picture}
\caption {{\bf CONTOUR/AHARDCOPY} --- Sample output}
\label {contour1}
\end {center}
\end {figure}
Figure~\ref{contour2} was produced in the same way as figure~\ref{contour1},
except that it shows the central part of the same image.
\begin {figure}[htbp]
\begin {center}
\begin {picture}(160,130)
\put (0,0){\line(1,0){160}}
\put (160,0){\line(0,1){130}}
\put (160,130,){\line(-1,0){160}}
\put (0,130){\line(0,-1){130}}
\end {picture}
\caption {{\bf CONTOUR/AHARDCOPY} --- Sample output (detail)}
\label {contour2}
\end {center}
\end {figure}
Figure~\ref{inspect} shows the same image as that shown in
figure~\ref{contour2}, except that it was produced using {\bf INSPECT} (with
the SOLID option) instead of {\bf CONTOUR}.
\begin {figure}[htbp]
\begin {center}
\begin {picture}(160,100)
\put (0,0){\line(1,0){160}}
\put (160,0){\line(0,1){100}}
\put (160,100){\line(-1,0){160}}
\put (0,100){\line(0,-1){100}}
\end {picture}
\caption {{\bf INSPECT/AHARDCOPY} --- Sample output (SOLID option)}
\label {inspect}
\end {center}
\end {figure}

\subsection {Calibrating and cleaning images}

{\bf ADD}, {\bf DIV}, {\bf MULT} and {\bf SUB} are useful for operating on
every pixel of an image with a constant value (e.g.\ for subtracting a bias
level from a CCD frame).
{\bf CADD}, {\bf CDIV}, {\bf CMULT} and {\bf CSUB} are useful for operating
on every pixel of an image with the value of the corresponding pixel in
another image (e.g.\ to divide a CCD image by a flat field).

There is a simple CCD-reduction routine in the ZAP monolith (see Section 3).

\subsection {Obtaining x,y positions}

To obtain the x,y position of the cursor, use {\bf TBXY}, followed by {\tt
LOOK TBXY} (the position is returned as the X,Y pair of parameters).
{\bf TBXY} running inside ZAP (Section 3) types x,y on the screen; the IKON version (Section 6) gives
pixel values as well.

To create a file of x,y positions using the cursor, e.g.\ as input to
{\bf STARFIT}, type {\tt GO EDRS} to enter the EDRS package, then use
{\bf XYCURA}.

x,y lists can be created, edited, displayed, etc., using about 25 programs
in the EDRS and GRASP classes, all with names beginning `XY'.
The lists are stored as .BDF files in the default directory.
For example:
{\bf XYKEY} allows entry and storage of a list of x,y values from the
keyboard.
{\bf XYLIST} lists an x,y list.
{\bf XYDRAW} plots a graph of the first two parameters in the list (usually
x,y) on any device (use {\bf AHARDCOPY} to send the plot on the ARGS to the
Printronix).
{\bf XYPRINT} sends a neat listing of an x,y list to file XYPRINT.LIS.
{\bf XYEDIT} changes a .BDF x,y list to ASCII format (or vice versa), which
allows editing of the list using the VAX editors.

\subsection {Simple photometry}

There are several different programs for carrying out profile-fitting and
aperture photometry in the EDRS, GRASP and PHOTOMETRY classes.
Examples:
{\bf APERASP} is a fast, interactive aperture-photometry program which
integrates pixels within a cursor of arbitrary size.
The default cursor is circular, but it can also be square or elliptical.
{\bf APERCUR} is a similar program.
{\bf APERFOT} is another interactive aperture-photometry program,
which estimates the sky contribution from an annulus around the
area to be integrated.
{\bf APERMAG} carries out aperture photometry for a series of objects
in an x,y list.

Gaussian profiles may be fitted using programs in the EDRS class.
Programs {\bf STARFIT} and {\bf STARMAG} both require input x,y lists
(see previous section).
{\bf STARFIT} determines the mean seeing for objects at a series of specified
x,y positions.
{\bf STARMAG} fits assumed Gaussian profiles at a series of specified x,y
positions.

\section {ZAP --- an ASPIC monolith}

ZAP is a monolith --- a pile of ASPIC Fortran with a menu at the top.
It is intended primarily for rapid interactive inspection and analysis of
images.
To use ZAP, type {\tt ZAP} where {\tt DSCL}
would normally be typed.
For a command menu, press return in response to the {\tt ZAP>} prompt.

When the VAX is busy, DSCL may take up to 60 seconds to interpret a command,
yet only a few seconds to execute the compiled program.
With the monolith, the user may call up ASPIC routines immediately.
The monolith at present incorporates the following ASPIC commands:
{\bf ADHC, ADISOV, ADISP, AFLASH, AHARDCOPY, APAN, APERASP, APERCUR, ARESET,
ASAVE, ATEXT, CONTOUR, DESCR, GREYSCALE, INCARN, INSPECT, LIST, LUTCOL, LUTE,
LUTGREY, LUTLIN, LUTROT, PEEP, PICK, SLICE, STARFIT, STARMAG, STATS, TBXY,
XYCURA, XYCURB, XYDRAWA, XYEDIT, XYLIST}.
A few minor modifications have been made to the original code, e.g.\
{\bf TBXY} now sends the x,y coordinates to the screen; {\bf APERCUR}
incorporates an {\bf APMEANSKY} command which allows you to average the counts
from several areas of sky before measuring the flux from the star.

To obtain DCL commands inside ZAP, type {\tt \$ command} (e.g. {\tt \$ DIR}).

ZAP uses by default the ARGS image-display device.
Malcolm Yallop has relinked a version of ZAP to display on the IKON (see
LUN/72 (RGO)).

ZAP incorporates a {\em current-image} facility.
To use it, type {\tt CURRENT} in response to the {\tt ZAP>} prompt.
Subsequent commands will not result in a prompt for image-name.
To change the current image, type {\tt IMAGE}.

ZAP includes some new commands:
\begin{description}
\item [ADD10] :
allows 10 frames to be added simultaneously (useful for averaging a number of
exposures).
To add less than 10, give zero weights to the unwanted frames.
\item [BLANK] :
creates a flat frame (all pixels have value = 1).
It is sometimes used in conjunction with the CCD routine.
\item [CCD] :
is based on a program written by Bill Sparkes for reducing AAT CCD data.
The program has been modified for use with data collected by the La Palma RCA
and GEC CCD cameras.
The program reads INPUT, BIAS and FLAT-FIELD frames, and calculates
\begin{verbatim}
       OUTPUT = (INPUT - BIAS) / FLAT-FIELD
\end{verbatim}
The spectroscopic mode allows subtraction of pre-flash frames.
For the La Palma RCA chip, the program establishes the bias level from the
mean value of pixels in the rectangle (X--20,10) to (X--5,Y--10), where the
dimensions of the chip are XxY pixels.
For the La Palma GEC chip, the mean bias level is measured in the rectangle
(2,25) to (8,Y--25).
\item [IAMFILE] :
generates an ASCII (printable) version of the output from the {\bf IAMANAL}
program.
\item [RGO] :
gives access to the RGO local help library.
\item [SEARCHPIX] :
lists the x,y coordinates of all pixels in the frame having values between
specified limits.
\end{description}

ZAP stands for Zero-delay ASPIC Processor.

\section {Gobbling up disc space --- and how not to}

Most ASPIC routines demand {\em real} data.
If the image is {\em integer}, a {\em real} `incarnation' will be created
automatically and added on to the existing image, tripling consumption of
disc space.
For example, a 320x512 CCD frame might load from tape into 700 blocks
(1 block = 512 bytes) of 2-byte pixels.
Use of {\bf AFLASH}, which requires {\em signed-word integer} (SW; 2-byte) data,
will leave the file unchanged.
But use of {\bf ADISP}, which requires {\em real} (R; 4-byte) data, will
increase the size of the file to 2100 (700 + 1400) blocks.

Here are some ways to cope with ASPIC's thirst for disc space:
\begin{itemize}
\item Use scratch space; if your site supports this facility.
Ask your site manager how to do this.
Your files are likely to be deleted automatically after lying unused for
12 hours or so.
\item Use {\bf PICK} to select an area of interest from the image and discard
the larger image.
\item Use {\tt SET PROTECT=OWNER:R}, (undo with {\tt SET PROTECT=OWNER:RWED}).
This will prevent incarnations being created, but will not affect execution
of programs that do not seek to create files.
To prevent writing to an individual file, use {\tt SET PROTECT
file/PROT=OWNER:R}.
Program {\bf REALFR} may be used to delete all but the {\em real} incarnation
of an image.
{\bf INCARN} will create a new frame from an old one, changing the format
(e.g.\ from SW to R).
\end{itemize}

\section {ASPIC source code --- where is it?}

ASPIC source code may be found in text library files (type .TLB) in directory
ASPDIR and its sub-directories; .OLB files contain the corresponding object
modules.
Code for the main programs is held in ASPICSOURCE:SOURCE.TLB.
ASPIC programs make calls to GKS, NAG, and SGS, and also to about 2000 routines
in 12 libraries in the LIBDIR, STARDIR and [STARLINK.PACK.ASPIC.LIB]
directories.
A list of the libraries linked may be found in ASPDIR:SLINK.COM.

Use the DCL command LIBRARY to access these files.
For example, {\tt LIBRARY/LIST} lists the routines stored in a given .TLB file,
while {\tt LIBRARY/EXTRACT=(routine1,routine2,...)} will copy the source code
for the specified routines into a file library-name.TXT.

\section{Using ASPIC with the IKON image-display device}

Starlink plans to phase out the ARGS image-display devices and replace them
with Digisolve IKON pixel engines.
Ralph Martin and Malcolm Yallop at RGO have written a rudimentary library
of display routines, substitutes for ARGSLIB with which ASPIC programs may
be relinked.

At present, the following commands are available in a relinked form:
{\bf ACLEAR, ADHC, ADISP, AFLASH, AHARDCOPY, APAN, APERCUR, ARESET, ASAVE,
LUTCOL, LUTCONT, LUTGREY, SLICE, TBXY}.
They are discussed in LUN/61 (RGO).

\newpage
\section{Alternatives to ASPIC}

ASPIC, running under the INTERIM environment, will at some stage be replaced by a
suite of programs running under the new ADAM environment (SUN/94), and using
the HDS data structures (SUN/92) instead of .BDF files.
The most-used functions in ASPIC are already available in the KAPPA package
(SUN/95), which uses HDS and is able to display to either ARGS or IKON devices.
BDF frames may be converted to HDS data structures using STARIN (SUN/96).
The advantages of using KAPPA are that  it is likely to be well supported and
that it will attract a growing number of users and programmers.
The main disadvantage, at the time of writing, is that there exist no
equivalents of many of the ASPIC routines (KAPPA includes about 60).
For example, there is no
equivalent of FITSIN for reading FITS files from tape, and there are
no programs for obtaining x,y positions from the display.

The FIGARO package (SUN/86) also includes some 2-D image handling routines.
FIGARO uses HDS data structures, but these are not compatible with KAPPA HDS
structures.

See Section 2.1 for advice on converting images between
ASPIC/INTERIM, FITS, KAPPA and FIGARO formats.

\appendix

\section{List of ASPIC commands in alphabetical order}

This appendix lists every documented ASPIC program and procedure.
These correspond to topics in the HELP library and are arranged in alphabetical
order.

Some programs can be run either directly (by specifying the program name) or
indirectly (by specifying a procedure which runs the program).
In such cases the name of the program is made by following the name of the
procedure by the character `P'.
For example, {\bf INDPIX} is a procedure which runs program {\bf INDPIXP}.
In the list below, such procedure/program combinations are indicated by
following the procedure name by the character `*', eg.\ {\bf INDPIX*}.
\begin{description}
\item [ABLINK]: Display two images on the ARGS and allow them to be registered
and blinked under trackerball control.
\item [ABLOCK]: Display a ramp on the ARGS.
\item [ABOXASP]: Define and save a series of rectangular boxes drawn round
selected features in an image.
\item [ACLEAR]: Clear the ARGS display.
\item [ADD]: Add two frames.
\item [ADDMSK]: Combine two images where effect is defined by 2 masks
created by POLIFILLA.
\item [ADHC]: Automatic high-contrast display of an image.
\item [ADISOV]: Disable (thereby making the data invisible) an overlay plane in
the ARGS.
\item [ADISP]: Display an image on the ARGS with automatic scaling of intensity
values if required.
\item [ADISP3]: Variation of ADISP which displays in turn the z-planes in a
3D image.
\item [AEROV]: Erase ARGS overlays in a specified bit plane.
\item [AFLASH]: Faster version of ADISP because it does not do any intensity
scaling.
\item [AFRAME]: Display a frame and graticule round the latest image displayed
on the ARGS.
\item [AGBLINK]: Display two images on the ARGS; just like ABLINK except two
distinct lookup tables may be used, one with each image.
\item [AHARDCOPY]: Copy the current ARGS pixel store onto the Printronix
lineprinter.
\item [ALIGN]: Align an image with a reference image.
\item [ALIST]: List parameters for all images held in the ARGS database and
hence displayed on the ARGS by ASPIC programs.
\item [ANNOTASP]: Annotate an image displayed on the ARGS with text strings,
arrows or scale-length bars.
\item [APAN]: Pan and zoom the ARGS display; return final position of cursor.
\item [APANG]: Pan and zoom the ARGS display with independent scaling in X and
Y; return final position of cursor.
\item [APERASP]: Perform all the tasks related to aperture photometry.
\item [APERCUR]: Aperture photometry using a circular aperture positioned on an
ARGS image by the cursor.
\item [APERFOT]: Simple aperture photometry.
\item [APERMAG]: Aperture photometry of all the stars in a list of positions.
\item [APIC]: Display an I*2 image on the ARGS and allow the scale and zero
point to be changed interactively to produce the best looking image.
It is faster than other display programs and allows quick investigation of the
range of intensity values.
Can pan and zoom.
\item [APLOT*]: Plot an image on the ARGS with suitable scaling so that the whole
image fits onto the whole screen.
\item [APLOTQ]: Similar to APLOT but fits the image into one quadrant of the
ARGS screen.
\item [APLOTRNG]: Plot an intensity wedge on the ARGS with a scale.
\item [ARESET]: Reset all ARGS functions.
\item [ARGPIC]: Display images on the ARGS with autoscaling and choice of
screen position.
\item [ARGSCIR]: Display a circular cursor on the ARGS and return its position
in array units and size.
\item [ARGSCUR]: Display an ARGS cursor and return its position in array units.
\item [ARITH]: Scalar arithmetic.
\item [ASAVE]: Save the current ARGS pixel store in a BDF file.
\item [ASXY]: Calculate mean, standard deviation, normalized standard deviation
and number of valid pixels in a selected area of an image.
\item [ATEXT]: Place text on the ARGS display.
\item [AUCUR]: Allow use of a variable size and shape cursor.
\item [AVERAGE]: Average all valid regions of an image.
\item [AZOOM]: Zoom the ARGS about a defined position.
\item [BATFLAT]: A procedure which runs FLAT as a batch job.
\item [BATIMSTAC]: A procedure which runs IMSTACK as a batch job.
\item [BATLORMUL]: A procedure which runs LORMUL as a batch job.
\item [BATLORSIM]: A procedure which runs LORSIM as a batch job.
\item [BATPDSIM]: A procedure which runs PDSIM as a batch job.
\item [BDFKEY]: Read image data from a terminal.
\item [BIN]: Average an image using rectangular bins.
\item [BITMASK]: Logical AND of a frame with a scalar.
\item [BLANK]: Blank out a range of lines or columns in an image.
\item [BLINKER]: Use after ABLINK to continue blinking with new zoom
factors and centre.
\item [BLURR]: Generate a 5x5 frame containing a Gaussian profile.
\item [BONW]: Set the ARGS background to white with black lines when using the
ARGS in vector rather than image mode.
\item [BOXASP]: Define a single rectangular box round a selected feature in an
image.
\item [BOXFILTER]: Local averaging of an image.
\item [CADD]: Add a scalar to a frame.
\item [CALHELP]: Short help program for {\em CALIBRATION} package.
\item [CALIB]: Convert an image using calibrations in table LOOKUP generated
by programs in {\em CALIBRATION} package.
\item [CATCOPY]: Copy part of the Solar Atlas from tape to disc.
\item [CCDCON]: Change the dimensions of an image by resetting the values of
the NAXIS1 and NAXIS2 descriptors.
\item [CCIRD]: Set the transformation coefficients calculated by CINVERT to the
input for RDTOXY.
\item [CDEMO]: Demonstration of a square cursor; allows zoom of an ARGS image
so that a selected region fills the screen.
\item [CDIV]: Divide a frame by a scalar.
\item [CENTROID]: Find the centre of fiducial marks.
\item [CFICI]: Set the transformation coefficients calculated by XYFIT to the
input for CINVERT.
\item [CFITR]: Set the transformation coefficients calculated by XYFIT to the
input for XYTRAN.
\item [CFIXY]: Set the transformation coefficients calculated by XYFIT to the
input for XYTORD.
\item [CINVERT]: Invert a bi-linear transformation.
\item [CLRLIM]: Clear histogram limits in IAMHIS, SETLIM and GETLIM.
\item [CLRQUAD]: Clear a given quadrant on the ARGS.
\item [CMPRS]: Compress an image by integer factors in X and Y.
\item [CMULT]: Multiply a frame by a scalar.
\item [CNTEXT*]: Store details of contours found in a predefined region of an
image.
They can then be used for fitting ellipses.
\item [COG]: Show the centre of gravity in (X,Y) of an image.
\item [COLCYCLE]: Generate a colour table consisting of cyclic replicas of an
original colour table.
\item [COLDIF]: Generate a colour difference profile from two profiles
extracted along the same axis of a nebular image in different colours.
\item [COLFIX*]: Change a specified column of an image to a given value.
\item [COLLAPSE]: Produce a 1D image by collapsing a polygonal section of an
image along the x or y axis.
\item [COLSEL]: Select colours from a pre-defined palette; a standard palette
is supplied by default.
\item [COLSLICE]: Generate a monochromatic ramp colour table.
\item [COMPAVE]: Compress an image by averaging adjacent elements.
\item [COMPICK]: Compress an image by selecting elements in a regular grid.
\item [CONCAT]: Concatenate two or more input datasets.
\item [CONFLEV]: Rescale an image using the background level and noise variance.
\item [CONT]: Plot contour map of an image leaving areas containing invalid
pixels blank.
\item [CONTOUR]: Draw a contour map of an image on one of several devices.
\item [CONV]: Convolve an image with a second (smaller) one as may have been
generated by BLURR or PSFEST.
\item [COORDS]: Allow up to 500 points in the current image displayed on the
ARGS to be identified using the cursor and stored as an output image.
\item [COSBELL]: Apply a cosine-bell function to a 1D or 2D image.
\item [COSFIT]: Produce COSMOS intensity conversion lookup table LOOKUP from
known conversions in LEVELS.
\item [CPOW]: Raise each element of a frame to a power.
\item [CREDOC]: Guide user through the creation of program documentation
(calls the program DOCIT).
\item [CSUB]: Subtract a scalar from a frame.
\item [CURFIT]: Calculate the astrometric position and photometric parameters
for a stellar image identified using the box-cursor on the ARGS.
\item [CURVAL]: Display the coordinates and value of a cursor selected pixel
within an image.
\item [CUT]: Apply upper and lower intensity cuts to an image.
\item [CVPLOT]: Produce annotated contour map of an image with corresponding
polarization vectors overlayed.
\item [DATIM]: Display current date and time as characters on the ARGS if
needed for recording.
\item [DESCR]: Show one or all of the descriptors of a frame.
\item [DESCRIPT]: Make image descriptor items available as program parameters.
\item [DETREND]: Fit a polynomial to the raw observations to remove any trends;
reduce the mean value to zero.
\item [DFFTASP]: Efficient 2D FFT program.
\item [DISKFIL*]: Remove a disc-shaped area from an FFT modes image.
\item [DIV]: Divide a frame by a second frame.
\item [DIVFF]: Divide a frame by a frame, preserving the scaling.
\item [DOODLE]: Create images on the ARGS suitable for photographing to make
slides.
Allows images, text, lines and arrows to be displayed in a flexible way to
convert a {\em picture} into a {\em diagram}.
\item [E2DCOL]: Generate the default E2D colour table.
\item [EDDOC]: Allow user to edit program documentation before it is put into a
program.
\item [EDITLEV]: Edit existing LEVELS file interactively.
\item [EDITTAB]: Edit existing TABLE file interactively.
\item [ELLPLTP]: Plot previously found contours with superimposed ellipses
which have been fitted using FITELLP.
\item [ELLPRTP]: Print details of fitted ellipses.
\item [ENTERTAB]: Enter 2D table of reals from keyboard.
\item [EQPROFASP]: Compute equivalent profile, with associated photometric
parameters, of a bright galaxy within an image in the system of de Vaucouleurs.
\item [EXP]: Exponentiate each element of a frame.
\item [EXPAND]: Expand part of an image by a {\em sinc} interpolation.
\item [EXTEND]: Unconstrained extrapolation from known data into the unknown
(dangerous).
\item [EXTRACT]: Select part of a dataset by extracting all samples whose epochs
lie between two given values.
\item [FC]: Perform the `standard' false colour and colour enhancement
operations in sequence.
\item [FCDISP]: Display a false colour image with the correct LUT and zoom.
\item [FCPACK]: Form a single false colour image from three (R,G,B) input
images.
\item [FCSAT]: Perform enhancement on the `saturation' image.
\item [FCSCALE]: Allow rescaling of three input images to a common background
and exposure.
\item [FCTHSI]: Convert a set of R,G,B images to a set of H,S,I (Hue,
Saturation, Intensity) images.
\item [FCTRGB]: Convert a set of H,S,I images back to R,G,B.
\item [FFCLEAN]: Remove blemishes from a flat image.
\item [FFT]: Apply a fast Fourier transform to a 1D or 2D image.
\item [FFTPOW]: Compute the power spectrum of data sampled at equal intervals.
\item [FILDEF]: Create a 1D or axi-symmetric 2D filter.
\item [FILEDOC]: Generate the HELP available on any topic in a form
suitable for printing.
\item [FILLGAP]: Fill gaps between datasets which are themselves regularly
sampled and with gaps which are an integral number of samples wide using an
auto-regressive model based on the whole of the data.
\item [FILTSPEC]: Calculate resultant spectral energy distribution when a source
with a known continuum spectrum is observed through combinations of optical
components each of which has its own spectral transmission function.
\item [FINULS]: Filter image noise using local statistics.
\item [FITBAKASP]: Fit a polynomial to the background and divide it into the
original image giving normalized intensities.
\item [FITELLP]: Fit an ellipse to a set of extracted contours and store them
for input to ELLPLTP and ELLPRTP.
\item [FITSHEAD]: Extract header from standard FITS format tape.
\item [FITSIN]: Convert FITS data on magnetic tape into Starlink frames.
\item [FITSOUT]: Write standard FITS format tape.
\item [FLAG]: Flag points which are to be removed from a dataset using a graphics
cursor.
\item [FLAT]: A procedure which runs FLATTEN.
\item [FLATTEN]: Flat fields an image using an un-aligned flat field and a
co-ordinate transformation file to allow for the misalignment.
\item [FLIP]: Invert an image with respect to a horizontal or vertical axis.
\item [FLIPMSK]: Interchange 0's and 1's in a mask image.
\item [FOLD]: Fold the data at a given period, find a mean curve and subtract
it from the data.
\item [FREQ]: Set parameter GLOBAL\_FREQ to mean frequency.
\item [FRINGE]: Remove the fringe pattern which is characteristic of some CCD
images.
\item [FTCONJ]: Complex conjugation of a Hermitian transform.
\item [GAUFIT]: Estimate the Gaussian profile that best fits the stars in an
image.
\item [GAUMAG]: Estimate the magnitudes of stars in an image by reading an
(X,Y) list of positions and fitting a full 2D Gaussian profile to each star.
\item [GAUSFIT]:  Produce astrometric positions and photometric parameters for
a number of stars for which approximate positions are known using a Gaussian
fit.
\item [GBLINKER]: More general version of BLINKER which allows different
lookup tables to be used with the two images being blinked.
\item [GETLEV]: Enter known calibration levels by ARGS box cursor on a
stepwedge frame.
\item [GETLIM*]: Return results to IAMHISP and SETLIMP.
\item [GETPAR]: Set IAM parameter PAR using the cursor.
\item [GETSKY]: Set IAM parameter SKY in master connection file IAMANALP.CON.
\item [GETTHR]: Set IAM parameter THRLD in master connection file IAMANAL.CON.
\item [GREYLASER]: Produce a grey-scale representation of an image file on
a Canon LBP-8 A2 laser printer (see SUN/24).
\item [GREYSCALE]: Extended version of VERGREY which also produces greyscale
pictures on the Printronix printers.
\item [GRID]: Incorporate a black and white grid into an image.
\item [HEATCOL]: Generate a stepped pseudo {\em heat sequence} colour table.
\item [HEATCON]: Generate a continous pseudo {\em heat sequence} colour table.
\item [HIDE]: Draw various forms of hidden-line plot.
\item [HIFREQ*]: Eliminate high frequency components from an FFT mode image
produced by DFFTASP.
\item [HIST*]: Histogram a patch of an array.
\item [HISTMATCH]: Rescale an image so that its histogram has a prescribed form.
\item [HISTPLOT]: 1D histogram plot of a Starlink image.
\item [HSICOL]: User definition of a palette for input to COLSEL if the default
is not acceptable.
\item [IAMANAL]: Analyse image and find all objects and their parameters, given
various thresholds.
\item [IAMEDRS]: Convert a file of IAM parameterised data to EDRS (X,Y) list
format.
\item [IAMHELP]: Short help listing for IAM suite of programs.
\item [IAMHIS*]: Compute intensity histogram of image and plot on given ARGS
quadrant.
\item [IAMPR]: Output image parameters from IAM.
\item [ICBLINK]: A variation on BLINK which blinks between two 16-bit images.
The contrast may be altered at any time.
\item [ICDISP]: Variation on ADISP which specifically displays 16-bit images.
It also allows compression of an image so that it will fit on the ARGS screen.
\item [ICFLASH]: Display an image on the ARGS, scaling it around sky level to
show faint features.
Compresses the image if a side is $>$ 512.
\item [IFLASH]: Display an image on the ARGS.
Does't do any intensity scaling so is much faster than ADISP.
If the image is $>$ 512 it is sampled down to less.
\item [IGJOB]: Produce Versatec plot file of IAM images.
\item [IGPLOT]: Plot IAM image on ARGS quadrant.
\item [IJOIN]: Join up to 25 2D images.
\item [IMANIC]: Select part of a 2D 16-bit image.
\item [IMERGE]: Merge up to 25 16-bit images.
\item [IMGARITH]: Image arithmetic.
\item [IMSIZE]: Determine the dimensionality and size of a frame.
\item [IMSTACK]: Stack images together which need not be aligned; it should be
used before PSTACK and IMGARITH are run, wherever possible.
\item [INCARN]: Extract a specified incarnation from an image.
\item [INDPIX*]: Change an indicated pixel to a defined value.
\item [INSPECT]: Perform similar functions to PEEP, WRHIST and STATS on several
selected regions of an image.
\item [INVARG]: Invert current colour table on the ARGS.
\item [INVCOL]: Invert colour table held as a {\em BDF} file.
\item [IPCSIN]: Convert IPCS data on magnetic tape into Starlink frames.
(see SUN/3).
\item [IPOLYGON]: A 16 bit integer version of POLYFILLA.
\item [ISLICE]: Version of SLICE which works on 16-bit images.
\item [ITFCORR]: Apply intensity transfer function correction to an image.
\item [ITFGEN]: Generate an intensity transfer function.
\item [ITFGENA]: Generate a conversion table to correct for PDS and emulsion
non-linearity.
\item [ITFPLOT]: Plot an intensity transfer function.
\item [ITOR]: Convert an I*2 format image containing a scale factor and zero
level in its descriptor into a R*4 image.
\item [JONESASP]: Smoothing program which uses different degrees of smoothing
for the central parts of a galaxy and the fainter outer parts.
\item [LAPLACE]: Form the difference between an image and some multiple of its
Laplacian.
\item [LIMITS]: Draw a graph of a 1D frame and ask for cursor selection of two
points.
\item [LIN]: Estimate amplitude and phase of up to 20 sine-waves using a
{\em linear} least squares fit.
\item [LINCONU]: Read a Starlink image (as unsigned 16 bit integers) together
with the descriptor items BSCALE and BZERO and convert the data into REAL
values using: \newline OUTPUT=INPUT*BSCALE+BZERO.
\item [LINCONV]: Read a Starlink image (as 16 bit integers) together with the
descriptor items BSCALE and BZERO and convert the data into REAL values
using:\newline OUTPUT=INPUT*BSCALE+BZERO.
\item [LINEFIT]: Fit linear functions to each line in an image.
\item [LINFIX*]: Change an indicated row of an image to a defined value
(cf.\ COLFIX).
\item [LINPLOT]: Draw the graph of a 1D frame on one of several devices.
\item [LIST]: List part of a frame.
\item [LISTLEV]: List LEVELS file on lineprinter.
\item [LISTPAR]: Display a formatted version of a parameter file on the
terminal.
\item [LOFREQ*]: Eliminate low frequencies from an FFT mode image produced by
DFFTASP.
\item [LOG]: Natural logarithm of each element of a frame.
\item [LORCUR]: Measure cursor-selected stars by fitting full 2D Lorentz
profiles, including an option to handle overlapping stars.
\item [LORFIT]: Estimate the Lorentzian profile that best fits the stars in an
image.
\item [LORMUL]: Measure magnitudes of all stars in an (X,Y) list, including
possible overlapping stars, using a full 2D Lorentz profile.
\item [LORSIM]: Measure magnitudes of stars by fitting full 2D Lorentz profiles
to isolated stars whose positions are given in an (X,Y) list.
\item [LSEE]: Simple to use image display package (now almost never used).
\item [LUTCOL]: Write a standard colour LUT to the ARGS.
\item [LUTCONT]: Fill the ARGS colour table with discrete values to give a
contour-like display.
\item [LUTE]: Interactive tuning of a LUT using the push-buttons.
\item [LUTFC]: Load the `false-colour' LUT (as used in the opening
demonstration at RAL).
\item [LUTGREY]: Load the standard grey LUT.
\item [LUTLIN]: Interactive manipulation of a linear (or logarithmic) LUT using
the ARGS cursor.
\item [LUTREAD]: Read any pre-defined LUT and write it to the ARGS.
\item [LUTROT]: Cycle a LUT using a trackerball.
\item [LUTSET]: Fill part of the ARGS colour table with values linearly
interpolated between two RGB sets.
\item [LUTSTORE]: Store the look-up-table currently held in the ARGS as a
Starlink frame.
It can subsequently be read back into the ARGS using LUTREAD.
\item [LUTTWEAK]: Interactive manipulation of a LUT; three different LUTs are
available.
\item [MAGAV]: Average the magnitudes found by any of the GRASP programs,
rejecting stars which have poor fit parameters.
\item [MAGCNTASP]: Produce absolutely calibrated contours in magnitudes per
square arcsec.
\item [MAGCOR]: Correct the magnitudes found by any of the GRASP programs for
the effects of measuring machine aperture density averaging.
\item [MAGDIAG]: Combine several magnitude lists, apply colour equations and
zero points, then plot and store colour-magnitude or colour-colour diagrams.
\item [MAGRMS]: Plot the run of RMS error with magnitude for the results from
any of these photometric programs.
\item [MANIC]: M and N image conversion --- pick part of a 1D, 2D or 3D frame
and convert it into a 1D, 2D or 3D frame.
\item [MANYG]: Create a 2D test image containing a background, as many Gaussian
images as required, and some noise.
\item [MASK]: Mask one image with another so that the first image shows where
it is valid but the second image shows through in the regions where the first
is invalid.
\item [MATHS]: Apply arithmetic and mathematical functions to input images and
constants.
\item [MEAN*]: Calculate the mean from a patch of an array.
\item [MEDBOX*]: Calculate the median from a patch of an array.
\item [MEDIAN]: Apply a median filter at each point in an image.
\item [MEM]: Maximum entropy deconvolution program for 2D images.
\item [MEMPOW]: Compute the maximum entropy estimate of the {\em true} power
spectrum of up to 50 datasets.
\item [MERGE]: Combine two or more images with weighting to form a mosaic.
\item [MODAL]: Apply a modal filter to discrete boxes in an image and then use
linear interpolation for intermediate values.
\item [MODE]: Apply a modal filter at each point of an image.
\item [MODHELP]: Summarize the functions of the pixel modification programs
in E2D (formerly known as MODPIXASP).
\item [MOVE]: Move an image by a non-integral shift in X and Y.
\item [MOVIE]: Display a movie made up of 1D plots of successive rows of a
2D frame.
Plot up to 500 pixels in the X-direction with a delay between each row
controlled by the trackerball.
\item [MSLICE]: Variant of SLICE which allows multiple slices to be obtained
but which stores none of them.
\item [MULCON]: Produce a contour map of a frame, or a series of contour maps
of selected regions of a frame.
Output files produced are suitable for a Versatec.
\item [MULT]: Multiply two frames, element by element.
\item [NITPIK]: Remove small defects from an image.
\item [NON]: Non-linear least squares fit of a set of sine waves starting with
good initial extimates.
\item [NOPROMPT]: Cancel automatic prompting mode set by PROMPT.
\item [NORMALIZE]: Calculate the scaling required between two images of the
same object with different exposure times so that they have (on average) the
same data values.
\item [NORMFILT]: Apply a convolution and normalization to an image as an aid to
star detection when using a matched filter.
\item [NSTACK]: Stack images to improve signal/noise and remove blemishes.
\item [ODDHDC1]: Apply an odd-order hierarchical convolution to a 1D image.
\item [ODDHDC2]: Apply an odd-order hierarchical convolution to a 2D image.
\item [OUTSET*]: Set all pixels in an image outside a circle to a defined value.
\item [PACK]: Pack an image which has been scaled from 0 to 1 (thresholded) or
0 to 3 (confidence levels) by CONFLEV.
\item [PARDIF]: Crude estimate of the partial derivative of an image.
\item [PARPLTP]: Extract any two parameters from the ellipses which have been
fitted by FITELLP and plot one against the other.
\item [PATCH]: Replace selected regions of an ARGS displayed image with a
smooth or noisy patch.
\item [PCT]: Generate the principle component transformation on a set of up to 4
frames.
\item [PDM]: {\em Phase Dispersion Methods} --- several variations on the
{\em string} method for estimating the possible frequencies present on a
dataset are available.
\item [PDSCOR]: Get clear areas measured by a segmented PDS raster and work
out PDS drift.
Fill 1D array with a {\em clear} value for each line of the main raster.
Display the levels of the reference scans and the fit and run of the mode of
the main scan values which are near the central scan values image.
\item [PDSIM]: Take the PDS main image and reference area scans from PDSMULTI,
estimate the variable zero level, subtract it and correct for PDS and emulsion
non-linearities.
\item [PDSIN]: Convert PDS data on magnetic tape into Starlink frames.
\item [PDSMULTI]: Take a PDS tape created by the MULTI-mode and produce Starlink
images.
\item [PDSRASTER]: Read a nine track PDS tape of rasters written in the
Scansalot format from the Forth system and create a Starlink image.
\item [PDSRIPPLE]: Eliminate the zero-point ripple on PDS produced images.
\item [PECALBASP]: Absolute photometric calibration of a 2D image which is held
as intensity normalized to a sky background of 1.0.
\item [PEEP]: Type a 9x9 section of an image.
\item [PERIOD]: Set parameter GLOBAL\_FREQ to mean period.
\item [PICK]: Pick a 2D subset from a 2D image using the ARGS cursor {\em or} a
keyboard.
MANIC handles much more general cases.
\item [PIXDUPE]: Expand an image by pixel duplication.
\item [PIXFILL]: Replace invalid regions of an image with a smooth function.
\item [PIXMAP]: Move the pixels in an image to new positions so as to put them
in a 2D set of bins.
\item [PIXUNMAP]: Generalized resampling of a 1D or 2D image at a set of points
defined by specified coordinate images.
\item [PLOTDAT]: Plot datasets.
\item [PLOTFIT]: Plot data created from a parameter file to be plotted over a
dataset.
\item [PLOTPS]: Plot the power spectrum computed by POWER.
\item [PLOTQTH]: Threshold an image and plot it on an ARGS quadrant.
\item [POLAR]: Generate total polarization and polarization angle from sky
subtracted Q and U polarization frames.
\item [POLFIT]: Produce intensity conversion table LOOKUP from known conversions
in LEVELS using a polynomial fit.
\item [POLYFILLA]: Cursor definition on the ARGS of a new image made up from
one or more polygons.
The interiors are set to 1, the exteriors to 0.
\item [POWER]: Compute (optionally) the power spectrum and/or the window
function of the photometric data.
\item [PRAPERCUR]: Format output from APERCUR for printing.
\item [PRAPERMAG]: Format output from APERMAG for printing.
\item [PRFDECASP]: Determine the parameters of the bulge and disc components in
the observed profile of a galaxy.
\item [PRFLOGASP]: Convert a profile held as normalized intensities into log of
intensity above the sky.
\item [PRFPLTASP]: Plot an extracted galaxy profile.
\item [PRFPRTASP]: Produce a printed listing of a fitted profile and its
associated parameters.
\item [PRFVISASP]: Visual comparison of an extracted profile with profiles
computed from a model with fixed parameters.
\item [PRGAUMAG]: Format output from GAUMAG for printing.
\item [PRIAXEASP]: Extract the profiles along two orthogonal axes centred on a
galaxy.
\item [PRINTTAB]: Print 2D table on lineprinter.
\item [PRLORCUR]: Format output from LORCUR for printing.
\item [PRLORMUL]: Format output from LORMUL for printing.
\item [PRLORSIM]: Format output from LORSIM for printing.
\item [PRMAGS]: Format an EDRS-type list of positions and magnitudes for
printing.
\item [PROFILE]: Extract and plot sections through images.
\item [PROMPT]: Activate automatic prompting.
\item [PRPLOT]: Produce plot of polarization vectors from total polarization
and polarization angle data.
\item [PSFEST]: Point-spread-function (star profile) from an image.
\item [PSPEC]: Power spectrum from the Fourier transform.
\item [PSTACK]: Add images which are mutually aligned and normalized; it is a
variation of NSTACK.
\item [PUTDOC]: Put a documentation file created by CREDOC into the source of
the program in the form of `C+' comments (calls program STUFF and the command
procedure EDITIT).
\item [PUTSTAR]: Create a realistic artificial star and galaxy field containing
profiles and noise which emulate electronic or photographic detectors.
\item [PW]: {\em Pre-Whitening} --- Remove the variation specified by a set of
parameters from an existing dataset.
\item [RDKEY]: Keyboard input of a list of right ascensions and declinations.
\item [RDLIST]: Type/print a stored list of right ascensions and declinations.
\item [RDTOXY]: Convert a list of right ascensions and declinations into crosses
superimposed on an image displayed on the ARGS.
\item [REALFR]: Tidy up an image so that only the REAL incarnation is left (it
needs some work space within your quota).
\item [RECFIL*]: Set pixels inside a selected rectangle to a defined value.
\item [RESAMPLE]: Resample an image according to a bi-linear position
transformation.
\item [RESCALE]: Alter the scale and zero descriptors and reset the invalid
pixel flag.
\item [RESTORE]: Write an unformatted BDF file as a formatted DAT file.
\item [RHDC1]: Apply a reduced odd-order hierarchical convolution to a 1D
image.
\item [RHDC2]: Apply a reduced odd-order hierarchical convolution to a 2D
image.
\item [RINGCOL]: Generate a continuous purple, blue, green, red, purple colour
table.
\item [ROT3D]: Rotate the axes of a 3D image from (X,Y,Z) to (Z,Y,X).
Designed for use in conjunction with STK23 to convert a set of 2 spectra (such
as long slit IPCS) into a TAURUS-compatible cube.
\item [RTOI]: Tiny procedure which may be of use within procedures to
convert parameters from REAL (as written by WRKEY) into INTEGER.
It also sets a global parameter.
\item [SATCOR]: Apply the electronograph saturation law to an image or
de-saturate an image.
\item [SECTOR]: Define, display and store the radial profile in a cursor-defined
sector of an image displayed on the ARGS.
\item [SEGMENT]: Copy polygonal segments of one image to another.
\item [SELECT]: Select one or more subsets from a given dataset and store as
separate datasets.
\item [SETALL*]:  Change IAM parameters.
\item [SETAREA*]: Set IAM area cut.
\item [SETLEV]: Produce calibration LEVELS file by keyboard entry.
\item [SETLIM*]: Set zero point and highest histogram bin value.
\item [SETMAG*]: Set IAM sky background.
\item [SETPER*]: Set threshold parameter of histogram.
\item [SIMPAR]: Generate control parameter file for STARSIMP.
\item [SINCFIL*]: Multiply an FFT image by a {\em sinc} function.
\item [SLAFIT]: Generate intensity conversion table LOOKUP from LEVELS using
a slalom fit devised by John Cooke.
\item [SLICE]: Define, display and store a 1D cursor-defined slice through an
image displayed on the ARGS.
\item [SMOOTH]: Smooth an image by a Gaussian or top hat convolution.
\item [SPECOL]: Generate a continuous blue, green, red colour table.
\item [SPLEQ]: Resample a dataset at equal time intervals.
\item [SPLFIT]: Generate intensity conversion table LOOKUP from LEVELS using a
spline fit.
\item [SQORST]: Either squashes or expands an `unknown' image into a defined
shape by performing bi-linear interpolation.
\item [SQRT]: Replace each element by its square root.
\item [STAR]: Find location, size and intensity of a star assumed to occupy
most of an image.
\item [STAREMASP]: Interactively remove stars and blemishes from an image.
\item [STARFIT]: Determine the best profile parameters for the stars in an
image.
\item [STARMAG]: Perform stellar photometry by fitting a model star profile.
\item [STARSIM*]: Generate a frame containing  an artificial star field with
user defined parameters.
\item [STARXY]: Measure astrometric (X,Y) positions of user selected stars in an
image and save them in a file.
\item [STATS]: Show dimensions, range of values and so on of a frame.
\item [STK23]: Combine a set of 2D images into a 3-D cube.
\item [STOKES]: Generate Stokes parameters Q and U from total polarization
and polarization angle.
\item [STORE]: Convert part of a formatted DAT file into an unformatted BDF
file.
\item [SUB]: Subtract the second frame from the first.
\item [SURFIT]: Produce image defined by a polynomial or bi-cubic spline surface
which is a least-squares fit to an input image.
\item [SWEEP]: Rotate the current ARGS colour table.
\item [SYMDIS*]: Remove 4 symmetrically placed disc-shaped areas from an FFT
image.
\item [SYNTH]: Generate sinusoidal data with user-defined periods and noise
levels at the same epochs as an input data set.
\item [TARGET]: Overlay markers showing the brightest stars in an artificial
image on top of the image displayed on the ARGS.
\item [TBXY]: Return the location of a cursor-defined point in an image
displayed on the ARGS.
\item [TBXY2]: Return the location of a pair of cursor-defined points in an
image displayed on the ARGS.
\item [TESTFIT]: Calculate the astrometric position and photometric parameters
of a stellar image for which approximate coordinates are input by the user.
\item [THRESH]: Define a threshold value using the ARGS trackerball; the effect
is visible on the ARGS.
\item [THRESHOLD]: Zeroise values of an image which lie outside specified
lower and upper limits.
\item [TOSTAR]: Convert RGODR-format images into Starlink frames.
\item [TRCONCAT]: Calculate the combined effect of applying two successive
6-parameter linear position transformations to a set of (X,Y) coordinates.
\item [TRIM]: Change the size of an image by selecting a square or rectangular
region to copy into another image.
\item [TWOTONE]: Take two 2D Starlink images and form a single coded
`pseudo-colour' image.
\item [TYPEIN]: Keyboard input to a 1D frame.
\item [TYPETAB]: Display 2D table on terminal.
\item [UNDOC]: Remove the documentation from the start of a program if major
changes are required (calls the program STRIP).
\item [UNPACK]: Unpack images created by PACK (PACK and UNPACK may be used to
compress images for transmission over the network).
\item [UNZOOM]: Set the ARGS zoom factors to 1 and 1 on the centre of the
screen.
\item [USMASK]: Form an `un-sharp' masked image by subtracting a smoothed
version of an image from the original.
\item [VARGREY]: Generate a grey colour table with user defined end points.
\item [VELINT]: Form a composite image from a velocity and an intensity image.
\item [VERGREY]: Generate pseudo-greyscale output for the Versatec.
\item [VIDISP]: Display an image created by VELINT as a colour coded velocity
map but with intensities represented as well.
\item [VIEW]: Produce a file suitable for printing containing an array of the
integers stored in a square section of an image.
\item [WAVEGET]: Copy part of the Solar Spectrum Atlas into a Starlink image.
\item [WONB]: Set the ARGS to white lines on a black background when it is used
in vector mode.
\item [WRDSCR]: Write new items as descriptors into an existing frame.
\item [WRHIST]: Form, store and/or write out the histogram of a frame.
\item [XYARITH]: Make a new (X,Y) list by performing linear arithmetic
operations on any of the parameters stored in two existing (X,Y) lists.
\item [XYCHART]: Plot a realistic looking star map; star identifications may be
put by the stars.
\item [XYCOEFF]: Generate a set of position transformation coefficients from a
specified shift, rotation and magnification.
\item [XYCUR]: Create a file of (X,Y) positions.
\item [XYCURA]: Create a new list of (X,Y) positions from part or all of an
existing list using the ARGS cursor.
\item [XYCURB]: Create a new list of (X,Y) positions by using the ARGS cursor to
correct the positions in an existing list.
\item [XYCUT]: Extract a portion of an (X,Y) list and write it to a new file.
You select the section of columns to be extracted.
Identifiers and headers are copied over correctly.
\item [XYDRAW]: Draw a graph of the first two parameters (usually X and Y) in
any (X,Y) list.
\item [XYDRAWA]: Draw from (X,Y) lists.
The graph is a smooth curve (using a cubic spline) joining the points.
\item [XYEDIT]: Convert the normal (binary) version of an (X,Y) list into an
ASCII version, and vice versa, so that (X,Y) lists may be created or edited
using one of the standard editors.
\item [XYFIT]: Fit a bi-linear transformation between two sets of positions.
\item [XYFITA]: Generate the coefficients for a linear transformation between
one set of (X,Y) positions and another.
\item [XYJOIN]: Join together two EDRS (X,Y) files.
\item [XYKEY]: Enter and store a list of (X,Y) positions from a keyboard.
\item [XYKEYA]: Extended version of XYKEY allowing keyboard entry of items to an
(X,Y) list.
\item [XYLIST]: Type a set of (X,Y) positions.
\item [XYLISTA]: Create a printable listing of any (X,Y) list in F12.4 format.
\item [XYMATCH]: Take two (X,Y) lists and create two new files containing
either the stars from each list which have corresponding identifiers, sorted
in the same order, or the entries which have no corresponding identifier.
\item [XYMULTA]: Take two (X,Y) lists and makes a third by copying the first
one, replacing the second parameter with the multiple of the second parameter
of the first and second files.
Suitable for multiplying two response functions.
\item [XYPMATCH]: Take two (X,Y) lists and make two new ones which are copies
of the old ones but only contain entries which match in (X,Y) position.
\item [XYPRNT]: Make a neat listing of an (X,Y) file.
\item [XYRENUM]: Take an (X,Y) file and renumber the identifiers from 1 to N.
\item [XYSORT]: Sort an (X,Y) list into ascending or descending order of one of
the parameters contained in it.
\item [XYSTAT]: Find statistics for 1 or 2 sets of numbers taken from (X,Y)
lists.
\item [XYTORD]: Convert cursor defined positions on the ARGS into a list of
right ascensions and declinations.
\item [XYTRAN]: Apply a linear transformation to a set of (X,Y) positions.
\item [XYWEED]: Take an (X,Y) list and extract those entries which have values
for one of the parameters present inside or outside a defined range.
\item [XYWEEDA]: Take an (X,Y) list and extract those entries whose (X,Y) lie
in a pixel of the reference image which has a non-zero value.
\item [ZAPLIN]: Replace a vertical or horizontal strip in an image to remove
defects.
\item [ZEBRA]: Generate a {\em pseudo contour} colour table.
\end{description}

\newpage
\section {Specifications of commonly used programs}
This section gives details of the following 53 commonly used ASPIC programs:

\begin{tabbing}
MMMM\=MMMMMMMMMMMM\=MMMMMMMMMMMM\=MMMMMMMMMMMM\=\kill
\>ABLINK\>ATEXT\>INCARN\>SLICE\\
\>ACLEAR\>CADD\>INSPECT\>STARFIT\\
\>ADD\>CONTOUR\>LIST\>STARMAG\\
\>ADHC\>CSUB\>LUTCOL\>STATS\\
\>ADISOV\>CYDSCR\>LUTCONT\>SUB\\
\>ADISP\>DESCR\>LUTE\>TBXY\\
\>AFLASH\>DIV\>LUTGREY\>VERGREY\\
\>AHARDCOPY\>FC\>LUTLIN\>WRDSCR\\
\>ALIST\>FILEDOC\>LUTREAD\>XYCURA\\
\>APAN\>GAUSFIT\>LUTROT\>XYCURB\\
\>APERASP\>GREYSCALE\>PEEP\>XYDRAWA\\
\>APERCUR\>HISTPLOT\>PICK\>XYEDIT\\
\>ARESET\>IAMANAL\>REALFR\>XYLIST\\
\>ASAVE
\end{tabbing}
The text is based on their on-line help files.\\

The parameter descriptions have three columns which contain the following
information:
\begin{itemize}
\item Parameter name.
\item Default value (if any)
\item Description
\end{itemize}
Parameters which are normally defaulted are listed after other parameters.\\

\goodbreak
\rule{\textwidth}{0.3mm}
{\Large {\bf ABLINK} \hfill {\bf ABLINK}}
\begin{description}
\item [FUNCTION]:
Displays two images, one in the bottom 8 bits of ARGS memory, the other in the
top 8 bits.
\item [COMMAND]:
\begin{quote}
{\tt ABLINK}
\end{quote}
\item [NOTES]:
After running this program, run the BLINKER program which allows the use of the
trackerball buttons to flip between the two images, both manually and
automatically.
\end{description}

\goodbreak
\rule{\textwidth}{0.3mm}
{\Large {\bf ACLEAR} \hfill {\bf ACLEAR}}
\begin{description}
\item [FUNCTION]:
Writes zeros into the ARGS picture memory.
\item [COMMAND]:
\begin{quote}
{\tt ACLEAR}
\end{quote}
\item [NOTES]:
It clears the ARGS, leaving the LUT and zoom factors unchanged.
\end{description}

\goodbreak
\rule{\textwidth}{0.3mm}
{\Large {\bf ADD} \hfill {\bf ADD}}
\begin{description}
\item [FUNCTION]:
The sum IMAGEIN1+IMAGEIN2 is stored in IMAGEOUT.
\item [COMMAND]:
\begin{quote}
{\tt ADD  IMAGEIN1  IMAGEIN2  IMAGEOUT}
\end{quote}
\end{description}

\goodbreak
\rule{\textwidth}{0.3mm}
{\Large {\bf ADHC} \hfill {\bf ADHC}}
\begin{description}
\item [FUNCTION]:
Displays an image on the ARGS at a specified center after automatic scaling.
A nonlinear scaling function is computed from a histogram of image pixel values
to produce a very high contrast display on the ARGS.
(ADHC = Args Display High Contrast)
\item [COMMAND]:
\begin{quote}
{\tt ADHC  [SHADE=1.]}
\end{quote}
\item [PARAMETERS]:
\begin{tabbing}
XXXXXXXX\=XXXXXXXX\=\kill
IMAGE\>\>\begin{minipage}[t]{100mm}
The 2-D Starlink image to be displayed. It may be of any size, but no more than
512*512 pixels will be visible.
\end{minipage}\\
XC\>256\>\begin{minipage}[t]{100mm}
The x co-ordinate on the ARGS where the {\em center} of the image will be
located.
\end{minipage}\\
YC\>256\>\begin{minipage}[t]{100mm}
The y co-ordinate on the ARGS where the {\em center} of the image will be
located.
\end{minipage}\\
SHADE\>0.5\>\begin{minipage}[t]{100mm}
Controls the overall brightness of the ARGS image.
A smaller value makes the image brighter.
Legal range is from 0.0 to 1.0.
\end{minipage}
\end{tabbing}
\item [NOTES]:
Useful for a first look at the image and to examine the faint image structure
near the background level.
The SHADE parameter allows the overall brightness of the ARGS image to be
adjusted.
\end{description}

\goodbreak
\rule{\textwidth}{0.3mm}
{\Large {\bf ADISOV} \hfill {\bf ADISOV}}
\begin{description}
\item [FUNCTION]:
Disables ARGS overlays for a specified bit plane.
\item [COMMAND]:
\begin{quote}
{\tt ADISOV [PLANE]}
\end{quote}
\item [PARAMETERS] :
\begin{tabbing}
XXXXXXXX\=XXXXXXXX\=\kill
(normally defaulted) :\\
\\
PLANE\>8\>\begin{minipage}[t]{100mm}
The bit plane to be disabled (it is actually 8+MOD(PLANE,8) which is used).
\end{minipage}
\end{tabbing}
\item [NOTES]:
It may be used to erase any graphics or text written to any one of the ARGS
overlay planes.
\end{description}

\goodbreak
\rule{\textwidth}{0.3mm}
{\Large {\bf ADISP} \hfill {\bf ADISP}}
\begin{description}
\item [FUNCTION]:
Displays an image on the ARGS at a specified center after suitable scaling.
\item [COMMAND]:
\begin{quote}
{\tt ADISP [TRIM=FALSE] [LOG=TRUE]}
\end{quote}
\item [PARAMETERS]:
\begin{tabbing}
XXXXXXXX\=XXXXXXXX\=\kill
IMAGE\>\>\begin{minipage}[t]{100mm}
The 2-D Starlink image to be displayed. It may be of any size, but no more than
512*512 pixels will be visible.
\end{minipage}\\
XC\>256\>\begin{minipage}[t]{100mm}
The x co-ordinate on the ARGS where the {\em center} of the image will be
located.
\end{minipage}\\
YC\>256\>\begin{minipage}[t]{100mm}
The y co-ordinate on the ARGS where the {\em center} of the image will be
located.
\end{minipage}\\
PVLO\>Min value\>Data with this value are scaled to zero.\\
PVHI\>Max value\>\begin{minipage}[t]{100mm}
Data with this value are scaled to 255, which is the largest value to be
displayed by the ARGS.
\end{minipage}\\
\\
(normally defaulted) :\\
\\
TRIM\>TRUE\>\begin{minipage}[t]{100mm}
If TRUE, values outside the range PVLO to PVHI are set to those values.
If FALSE, PVLO and PVHI are ignored and the values reduced to 16 bits.
This may cause them to wrap around on the ARGS.
\end{minipage}\\
LOG\>FALSE\>\begin{minipage}[t]{100mm}
If TRUE, logarithmic scaling is used between the defined limits, otherwise
linear scaling is used.
\end{minipage}
\end{tabbing}
\item [NOTES]:
This is the major routine used to look at data.
Note the use of the LOG parameter, and remember that by choosing suitable values
for PVLO and PVHI a negative display is generated.
\end{description}

\goodbreak
\rule{\textwidth}{0.3mm}
{\Large {\bf AFLASH} \hfill {\bf AFLASH}}
\begin{description}
\item [FUNCTION]:
Displays an image on the ARGS by simply copying a 2-D Starlink image to the
ARGS memory.
AFLASH does not do any intensity scaling, thus it is much faster than the
similar program ADISP.
\item [COMMAND]:
\begin{quote}
{\tt AFLASH}
\end{quote}
\item [PARAMETERS]:
\begin{tabbing}
XXXXXXXX\=XXXXXXXX\=\kill
IMAGE\>\>\begin{minipage}[t]{100mm}
The 2-D Starlink image (I*2) to  be displayed.
It may be of any size, but no more than 512*512 pixels will be visible.
\end{minipage}\\
XC\>256\>\begin{minipage}[t]{100mm}
The x co-ordinate on the ARGS where the {\em center} of the image will be
located.
\end{minipage}\\
YC\>256\>\begin{minipage}[t]{100mm}
The y co-ordinate on the ARGS where the {\em center} of the image will be
located.
\end{minipage}
\end{tabbing}
\item [NOTES]:
Mainly useful for displaying INTEGER*2 arrays which have pixels in the range
0 $<$ N $<$ 255, which is the extent of the ARGS look up table.
Values outside this range will be displayed with a LUT value of MOD(N,256).
\end{description}

\goodbreak
\rule{\textwidth}{0.3mm}
{\Large {\bf AHARDCOPY} \hfill {\bf AHARDCOPY}}
\begin{description}
\item [FUNCTION]:
Copies the current ARGS display onto the Printronix lineprinter.
\item [COMMAND]:
\begin{quote}
{\tt AHARDCOPY [XLIMITS=I1,I2] [YLIMITS=J1,J2] [SQUARE=logical] [THRESH=value]}
\end{quote}
\item [PARAMETERS]:
\begin{tabbing}
XXXXXXXX\=XXXXXXXX\=\kill
LABEL\>\>Optional label for the lineprinter plot.\\
\\
(normally defaulted) :\\
\\
XLIMITS\>0,511\>Minimum and maximum columns on the ARGS to be plotted.\\
YLIMITS\>0,511\>Minimum and maximum rows on the ARGS to be plotted.\\
SQUARE\>FALSE\>\begin{minipage}[t]{100mm}
If true, every 5th row is printed twice so that the overall image has the
correct proportions in X and Y.
\end{minipage}\\
THRESH\>0\>\begin{minipage}[t]{100mm}
Defines the threshold level at which black becomes white or vice versa
depending on the sign of the threshold value.
\end{minipage}
\end{tabbing}
\item [NOTES]:
By default the entire 512x512 pixel area is copied from the ARGS memory and
plotted on the printer as any array of 512x512 dots.
The sign of the value THRESH determines if pixel values above/below this value
are represented as blank/black dots on the printer.
If the sign is positive then pixel values below THRESH will be blank while
pixel values equal to or above THRESH will be black.
The converse applies if the sign of the value THRESH is negative.
Optionally, any rectangular region of the ARGS screen may be plotted by
specifying XLIMITS and YLIMITS on the command line.
The plot is automatically sent to the line printer queue.
Note that since the line printer pixels do not have equal spacing in the X and
Y directions, the plot is not square unless the SQUARE option is specified on
the command line.

It is useful for getting a hard copy of any graphics currently displayed on the
ARGS.
Since the lineprinter can only display a pixel as either white or black, this
routine does not work well on grey scale images unless the program THRESH has
been run to define a threshold value, in which case set THRESH=THRESH\_OFFSET
on the command line.
\end{description}

\goodbreak
\rule{\textwidth}{0.3mm}
{\Large {\bf ALIST} \hfill {\bf ALIST}}
\begin{description}
\item [FUNCTION]:
Lists details of the size and location of all images held in the ARGS database..
\item [COMMAND]:
\begin{quote}
{\tt ALIST}
\end{quote}
\item [NOTES]:
It should tell you what images are actually displayed on the ARGS.
If any non-ASPIC programs have been used, it is likely that no entry was made
in the ARGS database.
\end{description}

\goodbreak
\rule{\textwidth}{0.3mm}
{\Large {\bf APAN} \hfill {\bf APAN}}
\begin{description}
\item [FUNCTION]:
ARGS trackerball/cursor pan and zoom package.
\item [COMMAND]:
\begin{quote}
{\tt APAN}
\end{quote}
\item [NOTES]:
The trackerball buttons are used as follows:
\begin{quote}
\begin{tabbing}
XXXXXX\=XXXX\=\kill
GREEN\>1\>\begin{minipage}[t]{100mm}
resets the zoom factors and image position.
Zoom factors allowed are 1,2,4,8,16.
\end{minipage}\\
WHITE\>2\>decreases the x and y zoom factors.\\
WHITE\>3\>increases the x and y zoom factors.\\
RED\>4\>causes an exit from the program.
\end{tabbing}
\end{quote}
The value of the IX,IY position of the cursor on the image is returned.
\end{description}

\goodbreak
\rule{\textwidth}{0.3mm}
{\Large {\bf APERASP} \hfill {\bf APERASP}}
\begin{description}
\item [FUNCTION]:
The ASPIC aperture photometer is a set of routines designed for aperture
photometry of images selected interactively by the cursor.
Before using the photometer, the image frame to be measured is displayed on an
ARGS, for example by using ADISP.
The aperture photometer contains several functions, which are obtained by the
commands described below.
\item [COMMAND]:
\begin{quote}
{\tt APERASP}
\end{quote}
\item [OPTIONS]:
\begin{tabbing}
XXXXXXXX\=\kill
SHAPE\>\begin{minipage}[t]{120mm}
Select the shape of the cursor.
The available options are square or ellipse (circular cursors are included as
a special case of an ellipse).
\end{minipage}\\
SIZE\>\begin{minipage}[t]{120mm}
Enter a new aperture size.
In order to keep the magnitudes on the same scale throughout a measuring run,
the size should be kept constant for all the stars measured.
\end{minipage}\\
STAR\>\begin{minipage}[t]{120mm}
Measure the signal from a star in the frame.
The cursor is positioned over the star to be measured.
If a square cursor is selected, the trackerball buttons can be used to increase
or decrease the size of the aperture.
If an elliptical cursor has been selected, the trackerball buttons can be used
to change the ellipticity and orientation of the cursor as well as its size.
\end{minipage}\\
SKY\>\begin{minipage}[t]{120mm}
This is used to measure the signal from a patch of sky neighbouring the star
measured.
The procedure and options are the same as for STAR, however the aperture used
need not be the same size as that used in the STAR measurement.
\end{minipage}\\
MAG\>\begin{minipage}[t]{120mm}
This is used to derive a stellar magnitude from the stellar and sky measures
according to the formula:
\[mag = zero - 2.5 \log_{10}  (star - sky)\]
where zero is a zero point.
\end{minipage}\\
ZERO\>Enter a value for the zero point from the environment.\\
E\>Exit the aperture photometer and return to DSCL.
\end{tabbing}
\end{description}

\goodbreak
\rule{\textwidth}{0.3mm}
{\Large {\bf APERCUR} \hfill {\bf APERCUR}}
\begin{description}
\item [FUNCTION]:
Aperture photometry using a circular aperture positioned on an ARGS image by
the cursor.
\item [COMMAND]:
\begin{quote}
{\tt APERCUR}
\end{quote}
\item [PARAMETERS] :
\begin{tabbing}
XXXXXXXXX\=XXXXXXXX\=\kill
NEWIMAGE\>Yes\>\begin{minipage}[t]{100mm}
Flag for displaying new image or using image already on ARGS.
Choices are YES/NO.
\end{minipage}\\
IMAGE\>\>\begin{minipage}[t]{100mm}
This is the name of the image.
The program likes I*2 files with BSCALE,BZERO headers as in FITS, but deals with
others OK.
\end{minipage}\\
CLEAR\>Yes\>\begin{minipage}[t]{100mm}
Clear any prevoius aperture marks left on image.
Choices are YES,NO.
\end{minipage}\\
OPTION\>APERTURE\>\begin{minipage}[t]{100mm}
Flag for choice of action.
Choices are APERTURE, RMS, MAGNITUDE, ZEROMAG, ZEROTIME, STORE, HELP, IMAGE,
EXIT; defined above.
\end{minipage}\\
ZEROM\>0.00\>\begin{minipage}[t]{100mm}
The zeropoint to be applied to the magnitudes.
\end{minipage}\\
EXPTIME\>1.0\>\begin{minipage}[t]{100mm}
The exposure time which can be used for calculating the zeropoint to be
applied.
\end{minipage}\\
RMSCUR\>1.0\>\begin{minipage}[t]{100mm}
The correction factor to apply to the RMS result which can correct for a
non-Poisson character (see above).
\end{minipage}\\
XYWOUT\>\>\begin{minipage}[t]{100mm}
This is the name of the .BDF file used to store the results.
\end{minipage}\\
XYWTIT\>Output\>\begin{minipage}[t]{100mm}
This is the title to be associated with the output file.
\end{minipage}
\end{tabbing}
\item [NOTES]:
Input the image name to acquire ARGS image, or to display a new image on the
ARGS.
If the first, decide if clearing previous aperture marks, then repeatedly
choose option:
\begin{tabbing}
XXXXXXXXXXXX\=\kill
OPTION\>FUNCTION\\
Aperture (A)\>\begin{minipage}[t]{114mm}
Enables cursor to choose a position, aperture size and star/sky choice.
Star flux in aperture found, or sky level in aperture found, and typed.
If star, then the star and last sky level is taken and calculations done for
the Mag calc:
\[Mag = 30 + Z - 2.5 \log_{10} (star-sky level)\]
N.B. The magnitude is {\em not} stored.
\end{minipage}\\
Magnitude (M)\>Recalc the mag of the last star and sky\\
Store (ST)\>Store the last star magnitude result\\
Zeromag (ZEROM)\>Change the zero point, Z to be used\\
Zerotime (ZEROT)\>\begin{minipage}[t]{120mm}
Change zeropoint by using exposure time, so $Z = 2.5 \log exptime$.
\end{minipage}\\
Rms  (R)\>Change the Poisson scaling factor for RMS\\
Help (H)\>List options\\
Image (I)\>Display a new image for use\\
Exit (E)\>End program
\end{tabbing}
The starting aperture diameter is 20, the starting zero point is 0.00.
The cursor is moved by the trackerball and the size of the aperture is diplayed
as a circle.

The trackerball buttons are used as follows:
\begin{quote}
\begin{tabbing}
XXXXXX\=XXXX\=\kill
GREEN\>1\>Sky here and return\\
WHITE\>2\>Shrink aperture a bit\\
WHITE\>3\>Swell aperture a bit\\
RED\>4\>Star here and return
\end{tabbing}
\end{quote}
A Green or Red circle is left painted at the position chosen.

As well as the magnitude, an estimate of the error is made by taking the sky
estimate and the number of pixels in the sky estimate and then:
\[RMS = \frac {\sqrt{star + \frac{sky * starpixno}{skypixno}}}{star}\]
This assumes that the pixel values are Poisson distributed, but there is an
option to set a correction factor if the values are actually counts divided
by some constant.
The RMS is then divided by the square root of this constant.
This constant is also estimated with each sky level estimate from the scatter
of pixel values inside the sky ap.
It is set to 1 at the start and is changed by keyboard entry only.

At the end of the program all the stored magnitudes with information on
position, sky, etc.\ are stored away in an output file in the EDRS XY format
so they can be used by the GRASP mag handling programs.
\end{description}

\goodbreak
\rule{\textwidth}{0.3mm}
{\Large {\bf ARESET} \hfill {\bf ARESET}}
\begin{description}
\item [FUNCTION]:
Resets the ARGS currently assigned to ARGS\_DEVICE.
\item [COMMAND]:
\begin{quote}
{\tt ARESET}
\end{quote}
\end{description}

\goodbreak
\rule{\textwidth}{0.3mm}
{\Large {\bf ASAVE} \hfill {\bf ASAVE}}
\begin{description}
\item [FUNCTION]:
Saves the currently displayed ARGS image by copying it into a file on disc.
By default the entire 16 bit image is read from the ARGS memory store into an
INTEGER*2 file.
Optionally, any rectangular region of the ARGS screen may be saved by
specifying the X and Y limits of the region.
\item [COMMAND]:
\begin{quote}
{\tt ASAVE [XLIMITS=I1,I2] [YLIMITS=J1,J2]}
\end{quote}
\item [PARAMETERS] :
\begin{tabbing}
XXXXXXXXX\=XXXXXXXX\=\kill
IMAGE\>\>\begin{minipage}[t]{100mm}
Name of Starlink 2-D disc file to be created by ASAVE, containing the ARGS image.
\end{minipage}\\
\\
(normally defaulted) :\\
\\
XLIMITS\>0,511\>Minimum and maximum columns on the ARGS to be saved.\\
YLIMITS\>0,511\>Minimum and maximum rows on the ARGS to be saved.
\end{tabbing}
\item [NOTES]:
Useful for storing an ARGS image for later display.
The image can be redisplayed most simply with the program AFLASH, which simply
copies the array back into the ARGS without any scaling.
Alternatively, the program ADISP may be used if scaling limits of 0 to 255 are
specified.
However, ADISP will not re-display any overlays properly.
\end{description}

\goodbreak
\rule{\textwidth}{0.3mm}
{\Large {\bf ATEXT} \hfill {\bf ATEXT}}
\begin{description}
\item [FUNCTION]:
Displays text on the ARGS screen.
\item [COMMAND]:
\begin{quote}
{\tt ATEXT}
\end{quote}
\item [PARAMETERS] :
\begin{tabbing}
XXXXXXXX\=XXXXXXXX\=XXXXXX\=\kill
FCOL\>\>The colour of the text (W,R,G,B,Y,M,C) where:\\
\>\>\>W  =  WHITE\\
\>\>\>R  =  RED\\
\>\>\>G  =  GREEN\\
\>\>\>B  =  BLUE\\
\>\>\>Y  =  YELLOW\\
\>\>\>M  =  MAGENTA\\
\>\>\>C  =  CYAN\\
SIZE\>\>The size of the characters. Valid range is 0-31.\\
\>\>\>0 draws text approx  10 ARGS units high.\\
\>\>\>31 draws text approx 250 ARGS units high.\\
CENTRE\>\>The position on the ARGS which will be the centre of the text
string.\\
DIR\>\>The direction of the text (help at run time):\\
\>\>\>TEXT: The characters to be displayed.\\
\>\>\>QUIET: LOGICAL Iff true no user prompts given.
\end{tabbing}
\end{description}

\goodbreak
\rule{\textwidth}{0.3mm}
{\Large {\bf CADD} \hfill {\bf CADD}}
\begin{description}
\item [FUNCTION]:
Add SCALAR to every element of IMAGEIN giving IMAGEOUT.
\item [COMMAND]:
\begin{quote}
{\tt CADD  SCALAR  IMAGEIN  IMAGEOUT}
\end{quote}
\end{description}

\goodbreak
\rule{\textwidth}{0.3mm}
{\Large {\bf CONTOUR} \hfill {\bf CONTOUR}}
\begin{description}
\item [FUNCTION]:
It draws contours of a 2-D Starlink image on one of several graphics devices.
Options are available to select part of the image only, and to smooth the data
before contouring is done.
\item [COMMAND]:
\begin{quote}
{\tt CONTOUR}
\end{quote}
\item [PARAMETERS] :
\begin{tabbing}
XXXXXXXX\=XXXXXXXX\=\kill
DEVICE\>\>\begin{minipage}[t]{100mm}
This is the graphics device to be used.
It is specified by a character string which is the GKS device number or a
workstation name as described in SGP/26.
\end{minipage}\\
IN\>\>This is the 2-D Starlink input image.\\
BLOCK\>1\>\begin{minipage}[t]{100mm}
This is the factor to be used for blocking the data before contouring.
\end{minipage}\\
SIZE\>\>\begin{minipage}[t]{100mm}
Only used for workstations 5, 7, 9, 10 (hardcopy devices) and defines the plot
size in centimeters.
\end{minipage}\\
WHOLE\>Yes\>\begin{minipage}[t]{100mm}
Decides if the whole or part of the picture is to be plotted.
\end{minipage}\\
SAMPLES\>\>\begin{minipage}[t]{100mm}
If WHOLE=NO, this defines the range of pixels to be used --- 4 values in the
order Xstart, Xend, Ystart, Yend.
\end{minipage}\\
HEIGHTS\>\>\begin{minipage}[t]{100mm}
Up to 30 contour heights may be entered, separated by commas or spaces.
\end{minipage}\\
\\
(normally defaulted) :\\
\\
PENS\>TRUE\>\begin{minipage}[t]{100mm}
If PENS is true then each contour level will be drawn using a different GKS pen.
The result will depend on the device, i.e.\ the ARGS will produce different
colours but a GOC will produce different line types.
If PENS is false, the default pen will be used for all contour levels.
\end{minipage}\\
BATCH\>FALSE\>\begin{minipage}[t]{100mm}
In interactive mode the program will allow a user to type CONTROL\_C to
terminate plotting of the current contour level.
If CONTOUR is run in BATCH mode then since CONTROL\_C is not valid the program
will fail unless BATCH is set to true.
\end{minipage}
\end{tabbing}
\item [NOTES]:
It is most useful for generating hard copy as an alternative to VERGREY.
\end{description}

\goodbreak
\rule{\textwidth}{0.3mm}
{\Large {\bf CSUB} \hfill {\bf CSUB}}
\begin{description}
\item [FUNCTION]:
Subtracts SCALAR from every element of IMAGEIN giving IMAGEOUT.
\item [COMMAND]:
\begin{quote}
{\tt CSUB  SCALAR  IMAGEIN  IMAGEOUT}
\end{quote}
\end{description}

\goodbreak
\rule{\textwidth}{0.3mm}
{\Large {\bf CYDSCR} \hfill {\bf CYDSCR}}
\begin{description}
\item [FUNCTION]:
Copies the whole of the descriptor part of one frame into another frame.
\item [COMMAND]:
\begin{quote}
{\tt CYDSCR}
\end{quote}
\item [PARAMETERS] :
\begin{tabbing}
XXXXXXXX\=XXXXXXXX\=\kill
FROM\>\>\begin{minipage}[t]{100mm}
The name of the Starlink frame whose descriptor is to be copied.
\end{minipage}\\
TO\>\>\begin{minipage}[t]{100mm}
The name of the Starlink frame which is to have its descriptor overwritten by
that from FROM.
\end{minipage}
\end{tabbing}
\item [NOTES]:
It may be used to transfer specialized information from frame to frame without
changing the data values.
Note that it is only permitted if the total number of pixels in the two images
are the same.
\end{description}

\goodbreak
\rule{\textwidth}{0.3mm}
{\Large {\bf DESCR} \hfill {\bf DESCR}}
\begin{description}
\item [FUNCTION]:
Accesses the descriptors of a BDF file.
\item [COMMAND]:
\begin{quote}
{\tt DESCR}
\end{quote}
\item [PARAMETERS] :
\begin{tabbing}
XXXXXXXX\=XXXXXXXX\=\kill
DESNAME\>\>\begin{minipage}[t]{100mm}
Descriptor name (if response is null then all descriptors will be output).
\end{minipage}\\
IMAGE\>\>The BDF file name.
\end{tabbing}
\end{description}

\goodbreak
\rule{\textwidth}{0.3mm}
{\Large {\bf DIV} \hfill {\bf DIV}}
\begin{description}
\item [FUNCTION]:
The quotient IMAGEIN1/IMAGEIN2 is stored in IMAGEOUT.
An unacceptably small divisor pixel forces a zero result.
\item [COMMAND]:
\begin{quote}
{\tt DIV  IMAGEIN1  IMAGEIN2  IMAGEOUT}
\end{quote}
\end{description}

\goodbreak
\rule{\textwidth}{0.3mm}
{\Large {\bf FC} \hfill {\bf FC}}
\begin{description}
\item [FUNCTION]:
This is the standard sequence of false colour calculations; it may be used as a
black box, or as a guide to the rest of the false colour package.
Each of the programs starts with the letters {\bf FC}.
These notes describe how the package fits together.
\item [COMMAND]:
\begin{quote}
{\tt FC}
\end{quote}
\item [PROGRAMS]:
\begin{tabbing}
XXXXXXXXXX\=\kill
FCSCALE\>\begin{minipage}[t]{110mm}
allows rescaling of the intensities of 3 Starlink 2-D images so that their
histograms agree (approximately).
More more details, see below.
\end{minipage}\\
FCPACK\>\begin{minipage}[t]{110mm}
takes 3 Starlink images (R G B) and forms a single false colour image which,
when displayed with the APE lut, gives a coloured display.
\end{minipage}\\
FCTHSI\>\begin{minipage}[t]{110mm}
takes 3 Starlink images (R G B) and converts them into 3 new (H S I) images.
\end{minipage}\\
FCTRGB\>takes 3 (H S I) images and converts them back into 3 (R G B) images.\\
FCSAT\>\begin{minipage}[t]{110mm}
applies saturation enhancement to the ``S" image generated by FCTHSI.
NB.\ It can also be useful to replace the ``I" image generated by FCTHSI with
its logarithm, as computed by the ASPIC program LOG.
After any of these transformations, FCTRGB should be used, followed by FCPACK.
\end{minipage}\\
LUTFC\>\begin{minipage}[t]{110mm}
this is a convenient procedure for loading a suitable look-up-table into the
ARGS when displaying a false colour image.
The lut used is the old `KAPE' lut, renamed FCLUT in
SYS:\-[STARPACK.\-LOCAL.\-ASPIC.\-BDF].
\end{minipage}\\
FCDISP\>\begin{minipage}[t]{110mm}
displays a falsecolour image with the above look-up-table and the most natural
zoom factor.
Its parameters are IMAGE, X0, Y0.
\end{minipage}
\end{tabbing}
\end{description}

\goodbreak
\rule{\textwidth}{0.3mm}
{\Large {\bf FCSCALE} \hfill {\bf FCSCALE}}
\begin{description}
\item [FUNCTION]:
Reads in three Starlink images, computes and plots their histograms, and
allows the user to define six points, which must consist of two sets of three
points.
These will normally be the location of the peak of each histogram, going from
bottom of the screen to the top, and likewise a sensible upper limit to the
signal in each histogram.
The data are rescaled and the new histograms plotted.
Interaction is via the cursor on any GKS device.
\item [COMMAND]:
\begin{quote}
{\tt FCSCALE}
\end{quote}
\item [PARAMETERS] :
\begin{tabbing}
XXXXXXXX\=XXXXXXXX\=\kill
INA\>\>First input 2-D Starlink image.\\
INB\>\>Second input image.\\
INC\>\>Third input image.\\
OUTA\>\>Output Starlink image corresponding to INA.\\
OUTB\>\>Second output image.\\
OUTC\>\>Third output image.\\
DEVICE\>\>The GKS device, which must have a cursor.
\end{tabbing}
\item [NOTES]:
It was originally designed to allow data which had not been properly calibrated
and reduced to a common scale, to be used to create a False Colour display.
It may also be used before mosaicing images, and so on.
\end{description}

\goodbreak
\rule{\textwidth}{0.3mm}
{\Large {\bf FILEDOC} \hfill {\bf FILEDOC}}
\begin{description}
\item [FUNCTION]:
Creates a permanent copy of ASPIC documentation in your own directory.
\item [COMMAND]:
\begin{quote}
{\tt FILEDOC}
\end{quote}
\item [NOTES]:
The command can be used in various ways:
\begin{verbatim}
    FILEDOC
\end{verbatim}
a classified list of all current ASPIC programs is created in your directory.
\begin{verbatim}
    FILEDOC *...
\end{verbatim}
a complete list of {\em all} the ASPIC documentation will be stored in your
directory (more than 1400 blocks!).
\begin{verbatim}
    FILEDOC par1 [par2 [par3 [par4]]]
\end{verbatim}
the information requested will be written to a file with a suitable name.
Note that the full list of parameters must be entered --- it has no memory.
\begin{verbatim}
    FILEDOC * program
\end{verbatim}
type if you cannot remember the class of a particular program.
\end{description}

\goodbreak
\rule{\textwidth}{0.3mm}
{\Large {\bf GAUSFIT} \hfill {\bf GAUSFIT}}
\begin{description}
\item [FUNCTION]:
Produces astrometric positions and photometric parameters, determined from a
gaussian fit, for a number of stars for which  approximate positions, in the
input Starlink data frame, are given in a second Starlink frame.
This frame could be created by program COORDS, for example.
\item [COMMAND]:
\begin{quote}
{\tt GAUSFIT}
\end{quote}
\item [PARAMETERS] :
\begin{tabbing}
XXXXXXXX\=XXXXXXXX\=\kill
MODE\>\>\begin{minipage}[t]{100mm}
the number of parameters to be fitted.
The range is from 3 to 5 with a default of 3.
\end{minipage}\\
IAREA\>\>\begin{minipage}[t]{100mm}
gives the size of the area over which the fit will be performed for each star.
The range is from 20 to 512 pixels with a default of 40.
\end{minipage}\\
EXP\>\>\begin{minipage}[t]{100mm}
the saturation exponent; if MODE is 5, this value will be the initial value
used by the fitting routine, otherwise it will remain fixed.
\end{minipage}\\
DSAT\>\>\begin{minipage}[t]{100mm}
the saturation density; if MODE is 3, this value will be fixed, otherwise it
will be the starting value used by the fitting routine.
\end{minipage}
\end{tabbing}
\item [NOTES]:
For each image the X and Y positions, and errors on the positions, will be
calculated.
The number of photometric parameters which are fitted to the image is decided
by the user.
Three parameters, the magnitude index, background level and 1 sigma image width
will always be calculated.
The remaining two parameters, the saturation density and saturation exponent,
which determine the shape of the profile as the image saturates, can either be
calculated or used as  constants.
This allows these last two parameters to be determined from a number of bright
images and then applied as constants to all images to be measured.

When GAUSFIT is run the above parameters will be prompted for.
The user will be asked to supply a Starlink frame containing the image data,
followed by a Starlink frame containing the approximate positions for the
objects for which the fit is to be performed.
The results will be output to the terminal and also written to the file
OBJECTS.LIS.
If such a file already exists then a new version will be created.
\end{description}

\goodbreak
\rule{\textwidth}{0.3mm}
{\Large {\bf GREYSCALE} \hfill {\bf GREYSCALE}}
\begin{description}
\item [FUNCTION]:
Generates output which, when plotted on the Printronix or Versatec printers,
gives a greyscale representation of the input image.
The technique used generates a density of dots in the output which is
proportional to the input data value.
The pattern of dots is random.
\item [COMMAND]:
\begin{quote}
{\tt GREYSCALE}
\end{quote}
\item [PARAMETERS] :
\begin{tabbing}
XXXXXXXX\=XXXXXXXXXX\=\kill
INPUT\>\>\begin{minipage}[t]{100mm}
This is the 2-D Starlink image which is to be plotted.
\end{minipage}\\
DEVICE\>PRINTRONIX\>\begin{minipage}[t]{100mm}
Output device.
Default is the Printronix, otherwise the Versatec printer.
\end{minipage}\\
DEFAULT\>T\>\begin{minipage}[t]{100mm}
If TRUE, will assume default values for rotation, window size, pixel size,
grid lines, and intensity scaling.
\end{minipage}\\
ROTATE\>F\>\begin{minipage}[t]{100mm}
If TRUE, will rotate image 90 degrees on the printer page.
\end{minipage}\\
CHANGE\>F\>\begin{minipage}[t]{100mm}
If TRUE, will change the default window size which defines the area of the 2-D
array to be printed.
\end{minipage}\\
IVALUES\>1,MAXCOLS\>\begin{minipage}[t]{100mm}
The first and last column of the 2-D Starlink image to be included in the
window.
\end{minipage}\\
JVALUES\>1,MAXROWS\>\begin{minipage}[t]{100mm}
The first and last row of the 2-D Starlink image to be included in the window.
\end{minipage}\\
GRID\>0\>\begin{minipage}[t]{100mm}
Specifies the spacing (in pixel units) between grid lines to be superposed on
the output plot.
If 0, no grid lines will be generated.
\end{minipage}\\
PIXEL\>WIDTH\>\begin{minipage}[t]{100mm}
Specifies the height of each image pixel.
The maximum width output is always produced (within the constraint that each
image pixel must be an integer number of printer dots wide) and by default
square image pixels are produced.
(Actually the image height is compressed by a factor of 60/72 due to the
spacing of the rows on the Printronix).
\end{minipage}\\
MAXINF\>T\>\begin{minipage}[t]{100mm}
If TRUE, will automatically scale the image, using a histogram equalization
technique, to convey the maximum amount of information in the output image.
\end{minipage}\\
WHITE\>MIN\>\begin{minipage}[t]{100mm}
If MAXINF = FALSE, this defines the array value to correspond to pure white on
the output plot.
\end{minipage}\\
BLACK\>MAX\>\begin{minipage}[t]{100mm}
IF MAXINF = FALSE, this defines the array value to correspond to pure black on
the output plot.
\end{minipage}\\
LOG\>F\>\begin{minipage}[t]{100mm}
If MAXINF = FALSE, this defines whether or not logarithmic intensity scaling is
to be used.
\end{minipage}\\
LABEL\>\>A title, up to 80 characters, to be added at the bottom of the plot.
\end{tabbing}
\item [NOTES]:
It gives a hard copy of image data.
When the hard copy is for the Printronix, it is written to a file called
GREYPLT.LIS which is automatically queued to the printer when the program
finishes.
Additional copies may be obtained by typing the command:
\begin{verbatim}
    $ PRINT/QUEUE=SYS_PRINTRONIX/PASSALL/NOFEED GREYPLT
\end{verbatim}
When the hard copy is for the Versatec, it is written to a file called
DIT.DAT which is automatically queued to the Versatec when the program finishes.
A file called DIT.LIS is also queued to the Versatec and this contains
information on the image.
Additional copies may be obtained by typing the command:
\begin{verbatim}
    $ PRINT/QUEUE=SYS_VERSATEC/PASSALL/NOFEED DIT.DAT,DIT.LIS
\end{verbatim}
\end{description}

\goodbreak
\rule{\textwidth}{0.3mm}
{\Large {\bf HISTPLOT} \hfill {\bf HISTPLOT}}
\begin{description}
\item [FUNCTION]:
Produces a 1-D plot from a Starlink image using descriptor items HMIN and
HMAX, if present, to convert the array subscript into physical units.
\item [COMMAND]:
\begin{quote}
{\tt HISTPLOT}
\end{quote}
\item [PARAMETERS] :
\begin{tabbing}
XXXXXXXX\=XXXXXXXX\=\kill
INPUT\>\>This is the Starlink image used for input. It should be 1-D.\\
DEVICE\>GKS\_1\>\begin{minipage}[t]{100mm}
This is the graphics device to be used.
It may be any GKS device as specified in SGP/26.
\end{minipage}\\
YRANGE\>Min, Max\>This defines the vertical limits for plotting.\\
XRANGE\>Min, Max\>\begin{minipage}[t]{100mm}
This allows selection of part of the horizontal range to be plotted.
\end{minipage}\\
AXES\>Lin,Lin\>\begin{minipage}[t]{100mm}
This allows definition of linear or logarithmic axes in both X (first) and Y.
Note that logs and zeros do not go well together!
\end{minipage}\\
SIZE\>\>\begin{minipage}[t]{100mm}
This is the size of the plot, but is only used if the selected device does
not have a cursor.
\end{minipage}
\end{tabbing}
\item [NOTES]:
It was written to plot meaningful histograms from the file created by WRHIST
(qv).
\end{description}

\goodbreak
\rule{\textwidth}{0.3mm}
{\Large {\bf IAMANAL} \hfill {\bf IAMANAL}}
\begin{description}
\item [FUNCTION]:
COSMOS ANALYSER --- part of IAM suite.
Given an input image and various thresholds, it analyses the image and finds all
objects and their parameters.
\item [COMMAND]:
\begin{quote}
{\tt IAMANAL}
\end{quote}
\item [PARAMETERS] :
\begin{tabbing}
XXXXXXXX\=XXXXXXXX\=\kill
INPUTIMG\>(RA)\>input image to be analysed --- 2d picture array.\\
SKY\>(R)\>sky background level.\\
THRLD\>(R)\>threshold setting.\\
AREA\>(I)\>area cut --- smallest number of pixels accepted in an image.\\
SKYMAG\>(R)\>sky background magnitude in one pixel (default 0).\\
\\
returned --- (program parameters) :\\
\\
PARAMS\>(RA)\>\begin{minipage}[t]{100mm}
list of parameters for each object --- iyext = number of objects found.
\end{minipage}\\
\\
returned --- (descriptors in PARAMS) :\\
\\
SKY\>(R)\>sky background level.\\
THRLD\>(R)\>threshold setting.\\
AREA\>(I)\>area cut --- smallest number of pixels accepted in an image.\\
SKYMAG\>(R)\>sky background magnitude in one pixel (default 0).\\
MAXDIM\>(R)\>\begin{minipage}[t]{100mm}
the larger dimension of INPUTIMG, (IGJOB needs it later for Versatec).
\end{minipage}
\end{tabbing}
\item [NOTES]:
The 14 parameters calculated in this version are:
\begin{tabbing}
XXXXXXXX\=XXXXXXXX\=\kill
1\>X-CENTROID UNWEIGHTED\\
2\>Y-CENTROID UNWEIGHTED\\
3\>SEMI-MAJOR AXIS UNWEIGHTED\\
4\>SEMI-MINOR AXIS UNWEIGHTED\\
5\>ORIENTATION UNWEIGHTED\\
6\>X-CENTROID WEIGHTED\\
7\>Y-CENTROID WEIGHTED\\
8\>SEMI-MAJOR AXIS WEIGHTED\\
9\>SEMI-MINOR AXIS WEIGHTED\\
10\>ORIENTATION WEIGHTED\\
11\>AREA\\
12\>TOTAL INTENSITY\\
13\>MAXIMUM INTENSITY\\
18\>ELLIPTICITY UNWEIGHTED\\
19\>ELLIPTICITY WEIGHTED
\end{tabbing}
\end{description}

\goodbreak
\rule{\textwidth}{0.3mm}
{\Large {\bf INCARN} \hfill {\bf INCARN}}
\begin{description}
\item [FUNCTION]:
Copies an incarnation from an existing bulk data frame into a new frame.
\item [COMMAND]:
\begin{quote}
{\tt INCARN}
\end{quote}
\item [PARAMETERS] :
\begin{tabbing}
XXXXXXXX\=XXXXXXXX\=XXXXXX\=\kill
INPUT\>\>Bulk data frame for input.\\
OUTPUT\>\>Bulk data frame for output.\\
FORMAT\>\>Character string denoting data format of incarnation to be copied.\\
\>\>\>'SB' = Signed Byte\\
\>\>\>'SW' = Signed Word\\
\>\>\>'SL' = Signed Longword\\
\>\>\>'R ' = Real\\
\>\>\>'DP' = Double Precision\\
\>\>\>'UB' = Unsigned Byte\\
\>\>\>'UW' = Unsigned Word
\end{tabbing}
\end{description}

\goodbreak
\rule{\textwidth}{0.3mm}
{\Large {\bf INSPECT} \hfill {\bf INSPECT}}
\begin{description}
\item [FUNCTION]:
Can be used for inspecting areas of an image.
It is designed for INTEGER*2 images.
\item [COMMAND]:
\begin{quote}
{\tt INSPECT}
\end{quote}
\item [PARAMETERS] :
\begin{tabbing}
XXXXXXXX\=XXXXXXXX\=\kill
IMAGE\>\>The input image.\\
CURSOR\>Yes\>\begin{minipage}[t]{100mm}
Flag for defining area by cursor or keyboard.
Choices are YES,NO.
\end{minipage}\\
XAREA\>All\>X limits of area to be chosen.\\
YAREA\>All\>Y limits of area to be chosen.\\
OPTION\>Histogram\>\begin{minipage}[t]{100mm}
Flag for next course of action.
Choices are as described above.
\end{minipage}\\
HIGH\>Top\>Generally top of range to be treated.\\
LOW\>Bottom\>Generally bottom of range to be treated.\\
STEP\>Appropriate\>\begin{minipage}[t]{100mm}
Size of grouping of points in histogram.
Default gives 1000 across output device.
\end{minipage}\\
DEVICE\>ARGS\>\begin{minipage}[t]{100mm}
If the data are plotted out this is the flag for which device to
plot out on.
Choices are ARGS, TEKTRONIX, GOC, VERSATEC, CALCOMP, CC81.
\end{minipage}\\
DEVSIZE\>Various\>The size of the output plot.\\
DEVLIM\>Min,Max\>The minimum and maximum values to be plotted out.
\end{tabbing}
\item [NOTES]:
You choose an image and are then asked to define an area by Cursor or Keyboard
definition.
You can then choose an option from this list:
\begin{quote}
\begin{tabbing}
XXX\=HISTOGRAMXX\=Do 2) for points within 3sigma of the meanXX\=\kill
\>Command\>Function\>Short Command\\
\\
1)\>HISTOGRAM\>Display its histogram\>HI\\
2)\>MEAN\>Calculate its mean and std dev\>M\\
3)\>3MEAN\>Do 2) for points within 3sigma of the mean\>3\\
4)\>VALUES\>Type out a 5x5 grid of values in the area\>V\\
5)\>SOLID\>Display the area as a solid body plot\>S\\
6)\>AREA\>Choose a new area\>A\\
7)\>HELP\>List the options\>HE\\
8)\>EXIT\>Exit from the program\>E
\end{tabbing}
\end{quote}
After an option is completed the choice is repeated.

The trackerball buttons are used as follows:
\begin{quote}
\begin{tabbing}
XXXXXX\=XXXX\=\kill
GREEN\>1\>\begin{minipage}[t]{100mm}
This causes the screen magnification to revert to 1 and the cursor to go to
the centre of the image.
\end{minipage}\\
WHITE\>2\>\begin{minipage}[t]{100mm}
This causes the screen magnification to be divided by two.
If this would cause it to be less than 1, there is no effect.
\end{minipage}\\
WHITE\>3\>\begin{minipage}[t]{100mm}
This causes the screen magnification to be multiplied by two.
If this would make it too large, there is no effect.
\end{minipage}\\
RED\>4\>\begin{minipage}[t]{100mm}
This is used to define a corner of the area, and then define the opposing
corner.
A rectangle is then painted round the area.
\end{minipage}
\end{tabbing}
\end{quote}
\end{description}

\goodbreak
\rule{\textwidth}{0.3mm}
{\Large {\bf LIST} \hfill {\bf LIST}}
\begin{description}
\item [FUNCTION]:
Generates a formatted listing of a 1 or 2-D Starlink image.
The resulting file may be printed/typed in the normal way.
\item [COMMAND]:
\begin{quote}
{\tt LIST}
\end{quote}
\item [PARAMETERS] :
\begin{tabbing}
XXXXXX\=XXXXXXXXXXXX\=\kill
IMAGE\>\>The 1 or 2-D  Starlink image which is to be listed.\\
FILE\>[IMAGELIS].LIS\>\begin{minipage}[t]{100mm}
This is the name of the file to which output is written.
If omitted completely the name IMAGELIS.LIS is used.
If a name only is entered then a default type of .LIS is used.
\end{minipage}\\
FACTOR\>1\>\begin{minipage}[t]{100mm}
The pixel values are multiplied by FACTOR before being printed.
It should be chosen so that a 3-digit representation of data*factor is
meaningful.
\end{minipage}
\end{tabbing}
\item [NOTES]:
This lets you look at your data.
To have a look at a small section of data, program PEEP may be used.
\end{description}

\goodbreak
\rule{\textwidth}{0.3mm}
{\Large {\bf LUTCOL} \hfill {\bf LUTCOL}}
\begin{description}
\item [FUNCTION]:
Reads the standard coloured lookup table using program LUTREAD.
The second and third parameters of LUTREAD may be given as the first and
second of this procedure --- see documentation of LUTREAD.
\item [COMMAND]:
\begin{quote}
{\tt LUTCOL}
\end{quote}
\end{description}

\goodbreak
\rule{\textwidth}{0.3mm}
{\Large {\bf LUTCONT} \hfill {\bf LUTCONT}}
\begin{description}
\item [FUNCTION]:
Fills a look-up-table for the ARGS with the 7 basic colours in turn
(red ,green, blue, yellow, cyan, magenta, white).
The results are seen on the ARGS at once, and may be modified by erasing or
overwriting part of the table.
The resulting LUT may be stored for future use.
\item [COMMAND]:
\begin{quote}
{\tt LUTCONT}
\end{quote}
\item [PARAMETERS] :
\begin{tabbing}
XXXXXXXXX\=XXXXXXXX\=\kill
CONTOURS\>prev value\>\begin{minipage}[t]{100mm}
1, 2 or 3 values may be input.
If the first is positive then LUT entries are set-up, but if negative entries
are set to zero, thereby erasing them.
If only one value is input, that entry is set to the next colour.
If two values are input, then all entries between the first and second are
filled with the next colour.
If three values are input then blocks of LUT from first to first+last, and so on
up to second are filled with the colours in succession.
{\em Exit is by entering no values.}
\end{minipage}\\
OUTPUT\>\>\begin{minipage}[t]{100mm}
This is an optional parameter.
If present, the LUT which has been created will be stored for later use by
LUTREAD.
\end{minipage}
\end{tabbing}
\item [NOTES]:
It was originally designed to set up individual entries in a LUT to generate
contour-like displays, but has now been generalized as an alternative to
LUTSET and LUTE as a means of setting up a LUT.
\end{description}

\goodbreak
\rule{\textwidth}{0.3mm}
{\Large {\bf LUTE} \hfill {\bf LUTE}}
\begin{description}
\item [FUNCTION]:
Uses the trackerball buttons to allow interactive mapping of a predefined
colour table on to an ARGS image.
The image should be displayed, preferably with a set of blocks.
\item [COMMAND]:
\begin{quote}
{\tt LUTE}
\end{quote}
\item [PARAMETERS] :
\begin{tabbing}
XXXXXXXX\=XXXXXXXX\=\kill
LUT\>\>input lookup table.\\
OUTPUT\>\>output lookup table (optional).\\
STEP\>3\>increment in boundary.
\end{tabbing}
\item [NOTES]:
The trackerball buttons are used as follows:
\begin{quote}
\begin{tabbing}
XXXXXX\=XXXX\=\kill
GREEN\>1\>change to next colour.\\
WHITE\>2\>move the boundary to the left.\\
WHITE\>3\>move the boundary to the right.\\
RED\>4\>fill the lut and exit.
\end{tabbing}
\end{quote}
N.B. {\em Ignore the cursor itself.}
\end{description}

\goodbreak
\rule{\textwidth}{0.3mm}
{\Large {\bf LUTGREY} \hfill {\bf LUTGREY}}
\begin{description}
\item [FUNCTION]:
Reads standard grey lookup table using program LUTREAD.
The second and third parameters to LUTREAD may be given as the first and
second to this procedure --- see documentation to LUTREAD.
\item [COMMAND]:
\begin{quote}
{\tt LUTGREY}
\end{quote}
\end{description}

\goodbreak
\rule{\textwidth}{0.3mm}
{\Large {\bf LUTLIN} \hfill {\bf LUTLIN}}
\begin{description}
\item [FUNCTION]:
Uses the ARGS cursor/trackerball to allow interactive modification of the
look-up-table in the ARGS.
Normally a greyscale LUT is used, but any other may be read in.
Values below a lower limit are set to the limit, and those above an upper limit
are set to that limit.
The cursor is used to move these limits, on the assumption that the scale from 0
to 255 is spread out across the full width of the displayed image.
An option is available to replace the (default) linear LUT with a logarithmic
one, between the same limits.
Note that only the x position of the cursor is used, but the cursor must be
within the image currently displayed.
\item [COMMAND]:
\begin{quote}
{\tt LUTLIN [LUT=name] [QUIET=false]}
\end{quote}
\item [PARAMETERS] :
\begin{tabbing}
XXXXXXX\=XXXXXXXXXX\=\kill
PVLO\>0\>\begin{minipage}[t]{100mm}
This is only used if QUIET=FALSE, and is the data value corresponding to the
ARGS value of 0, as used when the image was displayed.
\end{minipage}\\
PVHI\>255\>\begin{minipage}[t]{100mm}
This is only used if QUIET=FALSE, and is the data value corresponding to the
ARGS value of 255, as used when the image was displayed.
\end{minipage}\\
OUTPUT\>\>\begin{minipage}[t]{100mm}
If present it is the name of the .BDF file used to store the LUT which has been
produced.
\end{minipage}\\
\\
(normally defaulted) :\\
\\
LUT\>ASPDIR:GLUT\>\begin{minipage}[t]{100mm}
This is the LUT which is used --- the default is the normal greyscale one.
\end{minipage}\\
QUIET\>TRUE\>\begin{minipage}[t]{100mm}
If FALSE, it permits output to the VDU of the current position of the lower and
upper cut-off points both in terms of LUT entries (0 to 255) and data values
(PVLO to PVHI).
\end{minipage}
\end{tabbing}
\item [NOTES]:
It is designed to allow the user to create the ideal display of an image, and
to store the resulting look-up-table for later use.

The trackerball buttons are used as follows:
\begin{quote}
\begin{tabbing}
XXXXXX\=XXXX\=\kill
GREEN\>1\>\begin{minipage}[t]{100mm}
Changes the mode of the display from linear to logarithmic, and back again.
\end{minipage}\\
WHITE\>2\>\begin{minipage}[t]{100mm}
Reads the current cursor (x) position as the LOWER cutoff (i.e.\ black,
normally).
\end{minipage}\\
WHITE\>3\>\begin{minipage}[t]{100mm}
Reads the current cursor (x) position as the UPPER cutoff (i.e.\ white,
normally).
\end{minipage}\\
RED\>4\>Exits from the program.
\end{tabbing}
\end{quote}
\end{description}

\goodbreak
\rule{\textwidth}{0.3mm}
{\Large {\bf LUTREAD} \hfill {\bf LUTREAD}}
\begin{description}
\item [FUNCTION]:
Sends a colour lookup table to the ARGS.
Optional threshholds and logging of the scale.
\item [COMMAND]:
\begin{quote}
{\tt LUTREAD}
\end{quote}
\item [PARAMETERS] :
\begin{tabbing}
XXXXXXXX\=XXXXXXXX\=\kill
LUT\>\>Colour lookup table.\\
LIMITS\>0,255\>Range in ARGS lookup table onto which lut is to be mapped.\\
LOG\>FALSE\>TRUE iff a log scale is to be used.
\end{tabbing}
\end{description}

\goodbreak
\rule{\textwidth}{0.3mm}
{\Large {\bf LUTROT} \hfill {\bf LUTROT}}
\begin{description}
\item [FUNCTION]:
Rotates entries 1-255 in the LUT by means of the tracker ball.
\item [COMMAND]:
\begin{quote}
{\tt LUTROT}
\end{quote}
\item [NOTES]:
The trackerball buttons are used as follows:
\begin{quote}
\begin{tabbing}
XXXXXX\=XXXX\=\kill
GREEN\>1\>Reset LUT.\\
RED\>4\>Exit.
\end{tabbing}
\end{quote}
The table offset is returned to DSCL.
\end{description}

\goodbreak
\rule{\textwidth}{0.3mm}
{\Large {\bf PEEP} \hfill {\bf PEEP}}
\begin{description}
\item [FUNCTION]:
Displays a  9x9 area of image, centred on (X0,Y0), on terminal.
\item [COMMAND]:
\begin{quote}
{\tt PEEP  IMAGE  X0  Y0  [FACTOR]}
\end{quote}
\item [NOTES]:
(0,0) is the bottom left-hand corner.

Integer representation of up to 7 characters (including minus sign if
appropriate) is used for each displayed pixel.

The pixel values are multiplied by FACTOR before display.
FACTOR defaults to 1.
\end{description}

\goodbreak
\rule{\textwidth}{0.3mm}
{\Large {\bf PICK} \hfill {\bf PICK}}
\begin{description}
\item [FUNCTION]:
May be used to define a rectangular sub-set of a 2-D Starlink image for
storage as a new image.
\item [COMMAND]:
\begin{quote}
{\tt PICK}
\end{quote}
\item [PARAMETERS] :
\begin{tabbing}
XXXXXXXX\=XXXXXXXX\=\kill
INPUT\>\>This is the input 2-D Starlink image.\\
OUTPUT\>\>This is the name of the new image picked out of INPUT.\\
XYPOS1\>\>\begin{minipage}[t]{100mm}
By default the program tries to read the ARGS cursor.
If this is not possible, or the cursor is not moved within 5 secs of appearing,
the user is prompted for keyboard input.
In this case, XYPOS1 is a pair of (x,y) co-ordinates defining one corner of the
sub-image.
\end{minipage}\\
XYPOS2\>\>\begin{minipage}[t]{100mm}
If keyboard input is required (see above), then XYPOS2 are the (x,y)
co-ordinates of a second corner.
The two pairs are sorted out, and printed on the terminal with bottom-left-hand
first.
\end{minipage}\\
XYOUT1\>\>\begin{minipage}[t]{100mm}
Output parameter used to return the (x,y) position of the BLH point of the
selected region.
\end{minipage}\\
XYOUT2\>\>\begin{minipage}[t]{100mm}
Output parameter used to store the (x,y) position of the TRH point of the
selected region.
\end{minipage}
\end{tabbing}
\item [NOTES]:
It may be used with or without the ARGS cursor.
For a more general extraction program use MANIC, which works on 1, 2, and 3-D
images.

The trackerball buttons are used as follows:
\begin{quote}
\begin{tabbing}
XXXXXX\=XXXX\=\kill
GREEN\>1\>\begin{minipage}[t]{100mm}
Read the cursor position --- only valid if the  cursor was moved within 5
seconds, but then there is no time limit.
\end{minipage}\\
WHITE\>2\>As 1.\\
WHITE\>3\>As 1.\\
RED\>4\>As 1.
\end{tabbing}
\end{quote}
\end{description}

\goodbreak
\rule{\textwidth}{0.3mm}
{\Large {\bf REALFR} \hfill {\bf REALFR}}
\begin{description}
\item [FUNCTION]:
Extracts the REAL incarnation from a Starlink bulk data frame, copies it into a
dummy frame, deletes the original, and renames the dummy to the original name.
\item [COMMAND]:
\begin{quote}
{\tt REALFR [frame]}
\end{quote}
\item [PARAMETERS] :
\begin{tabbing}
XXXXXXXX\=XXXXXXXX\=\kill
FRAME\>\>The name of the original (and final) .BDF frame.
\end{tabbing}
\item [NOTES]:
It may be used to delete all but the REALincarnation of a frame, as a way of
saving space when using ASPIC.
Note that space is required for the dummy frame; errors can lose your original
data frame.
\end{description}

\goodbreak
\rule{\textwidth}{0.3mm}
{\Large {\bf SLICE} \hfill {\bf SLICE}}
\begin{description}
\item [FUNCTION]:
Allows cursor definition of two points in an image displayed on the ARGS and
display of the slice through the image with the two points as end points.
Interpolation is used to ensure that the spacing of samples in the slice is
always at the pixel spacing, irrespective of the angle.
\item [COMMAND]:
\begin{quote}
{\tt SLICE}
\end{quote}
\item [PARAMETERS] :
\begin{tabbing}
XXXXXXXX\=XXXXXXXX\=\kill
INPUT\>\>This is the input image ---  it ought to be visible on the ARGS.\\
OUTPUT\>\>This is used, if present, as the name for the stored slice.
\end{tabbing}
\item [NOTES]:
Use to look at sections in any direction through an image.
The resulting slice may be stored as a 1-D Starlink image.

The trackerball buttons are used as follows:
\begin{quote}
\begin{tabbing}
XXXXXX\=XXXX\=\kill
GREEN\>1\>\begin{minipage}[t]{100mm}
This may be used to pick two points from one of the images displayed on the
ARGS.
Both points must be from the same image, and there must be at least one image
displayed.
\end{minipage}\\
WHITE\>2\>\begin{minipage}[t]{100mm}
If this button is pressed after the first point has been entered, the sampling
line that would result from these two points is displayed.
To accept that line, press button 1 again otherwise move the cursor and hit
button 2 to try again.
\end{minipage}\\
WHITE\>3\>Same as 2.\\
RED\>4\>This forces unconditional exit from the program.
\end{tabbing}
\end{quote}
\end{description}

\goodbreak
\rule{\textwidth}{0.3mm}
{\Large {\bf STARFIT} \hfill {\bf STARFIT}}
\begin{description}
\item [FUNCTION]:
Finds a set of parameters describing a model star image, which can be used
when performing stellar photometry using STARMAG.
The model has a radial profile:
\[D = A \exp (-0.5 \frac{r^{\prime}}{sigma} \exp (GAMMA))\]
where $r^{\prime}$ is calculated from the true radial distance from the star
centre, r, allowing for image ellipticity.
The program combines a number of star images specified by the user and
determines a mean seeing disc size (SEEING), radial fall-off parameter (GAMMA),
axis ratio (AXISR) and axis inclination (THETA) of a model star image.
\item [COMMAND]:
\begin{quote}
{\tt STARFIT   (Precede with GO EDRS, to enter EDRS package)}
\end{quote}
\item [PARAMETERS] :
\begin{tabbing}
XXXXXXXX\=XXXXXXXX\=\kill
ILEVEL\>2\>\begin{minipage}[t]{100mm}
An integer from 1 to 3 controlling the printing of results:\\
1: Printing is supressed.\\
2: Results are printed, giving the number of stars used and the mean profile
parameters.\\
3: A table, giving details of the seeing and ellipticity of each star image
used is also printed.
This table indicates if any star could not be used.
\end{minipage}\\
IMAGE\>\>The input image containing the stars to be fitted.\\
INPUT\>\>\begin{minipage}[t]{100mm}
An input list of X,Y positions for the stars to be used in the fit.
\end{minipage}\\
ISIZE\>15\>\begin{minipage}[t]{100mm}
An integer from 3 to 101 specifying the side of the square area to be used when
forming the marginal profiles for a star image.
This should be sufficiently large to contain the entire star image.
If an even value is given, the next largest odd number is used instead.
\end{minipage}\\
RANGE\>4.0\>\begin{minipage}[t]{100mm}
The number of image profile widths (`sigma' in the equation above) out to which
the radial star profile is to be fitted.
(There is an upper limit of 50 pixels to the radius at which data is actually
used.)
\end{minipage}\\
DEVICE\>NONE\>\begin{minipage}[t]{100mm}
A parameter specifying a graphical output device on which to produce a plot of
the mean radial profile of the stars and the fitted function.
Any GKS supported plotting device may be specified.
Abbreviations are acceptable.
The default (NONE) produces no plot.
\end{minipage}\\
SEEING\>\>\begin{minipage}[t]{100mm}
An output parameter giving the seeing disc size: the full width at half
maximum across the minor axis of a star image.
\end{minipage}\\
AXISR\>\>\begin{minipage}[t]{100mm}
An output parameter giving the axis ratio of the star images: the ratio of the
major axis length to that of the minor axis.
\end{minipage}\\
THETA\>\>\begin{minipage}[t]{100mm}
An output parameter giving the inclination of the major axis of the star images
to the X axis (increasing pixel number direction).
This value is in degrees, X through Y being considered positive.
\end{minipage}\\
GAMMA\>\>\begin{minipage}[t]{100mm}
An output parameter giving the radial fall-off parameter in the equation above.
\end{minipage}
\end{tabbing}
\item [METHOD]:
Marginal profiles of each star image are formed in 4 directions, inclined at
45 degree intervals.
A gaussian curve and background is fitted to each profile.
Using the resulting 4 gaussian centres, a mean centre is found for each star.

The 4 gaussian widths of all the stars are combined, using a weighted average
with rejection of erroneous data, and from the 4 average widths the seeing disc
size, axis ratio and axis inclination are calculated.

The data surrounding each star is then binned into isophotal zones which are
elliptical annuli centred on the star --- the ellipse parameters being those
just calculated.
The data in each zone is processed to remove erroneous points and to find an
average value.
A gaussian profile is fitted to these average values and the derived amplitude
is used to normalise the values to an amplitude of unity.
The normalised values are put into bins together with the corresponding data
from all other stars and this binned data represents a weighted average radial
profile for the set of stars, with the image ellipticity removed.
Finally a radial profile is fitted to this data, giving the radial profile
parameter GAMMA and a final re-estimate of the seeing disc size.
\item [NOTES]:
The values of the parameters SEEING, AXISR, THETA and GAMMA are in a form for
direct use with the program STARMAG.

The stars used to determine the mean image parameters should be chosen to
represent those whose magnitudes are to be found using STARMAG, and to be
sufficiently bright, uncrowded and noise-free to allow an accurate fit to be
made.

The execution time is approximately proportional to the number of stars used
and the image area which each occupies.
\item [ERRORS]:
The program may access the following error parameter(s) to indicate error
conditions:
\begin{verbatim}
    NOSTARS
          ***NO STAR IMAGES FOUND
\end{verbatim}
No image features at the positions given resembled star images sufficiently
closely to allow their centres and widths to be determined.
There is thus no data on which to base a fit.
\begin{verbatim}
    NOFIT
           ***RADIAL PROFILE FIT COULD NOT BE OBTAINED
\end{verbatim}
There is insufficient data in the final average radial profile or the profile
differs too much from the form expected (a gaussian) for a fit to be obtained.
\end{description}

\goodbreak
\rule{\textwidth}{0.3mm}
{\Large {\bf STARMAG} \hfill {\bf STARMAG}}
\begin{description}
\item [FUNCTION]:
Determines the integrated brightness and magnitude of star images by fitting
a 2 dimensional surface to the data.
The program is intended for images which have been previously linearised and
spatially calibrated, although it is not necessary for the background to have
been subtracted.
The method assumes that all star images have the same shape, and some
inaccuracy will result if this is not the case.
\item [COMMAND]:
\begin{quote}
{\tt STARMAG  (Precede with GO EDRS, to enter EDRS package)}
\end{quote}
\item [PARAMETERS] :
\begin{tabbing}
XXXXXXXX\=XXXXXXXX\=\kill
ILEVEL\>2\>\begin{minipage}[t]{100mm}
An integer from 1 to 3 to control the printing of results:\\
1: Printing is supressed.\\
2: Results are printed for each star successfully fitted.\\
3: The table also includes stars not successfully fitted, together with
appropriate error messages.
\end{minipage}\\
IMAGE\>\>The input image containing the stars.\\
INPUT\>\>An X,Y list of approximate positions of stars to be fitted.\\
SEEING\>\>\begin{minipage}[t]{100mm}
The seeing disc size: the full width at half maximum across the minor axis of
a star image.
\end{minipage}\\
AXISR\>1.0\>\begin{minipage}[t]{100mm}
The axis ratio of the star images: the ratio of the major axis length to that
of the minor axis.
This value should lie in the range 1.0 to 2.0.
\end{minipage}\\
THETA\>0.0\>\begin{minipage}[t]{100mm}
The angle by which the major axis of a star image is inclined to the X axis
(increasing pixel number direction).
The value should be in the range  -180.0 to +180.0 degrees, X through Y being
considered positive.
\end{minipage}\\
GAMMA\>2.0\>\begin{minipage}[t]{100mm}
The radial profile parameter which determines the rapidity of the radial
fall-off in a star's brightness.
The value 2.0 corresponds to a gaussian profile.
The value should lie in the range 1.0 to 5.0.
\end{minipage}\\
RANGE\>3.5\>\begin{minipage}[t]{100mm}
The number of image profie widths (`sigma' in the equation above) out to which
the radial star profile is fitted, after binning into iso-photal zones.
(There is an upper limit of 50 pixels to the radius at which data is actually
used.)
\end{minipage}\\
ZEROMAG\>0.0\>\begin{minipage}[t]{100mm}
The zero point of the magnitude scale.
The calculated magnitude of a star image is:
\[MAG = -2.5 \log _{10} brightness + ZEROMAG\]
where {\em brightness} is the integrated brightness of the star.
\end{minipage}\\
OUTPUT\>\>\begin{minipage}[t]{100mm}
The output X,Y list containing the accurate centre positions of the stars which
were successfully fitted and their magnitudes.
These results may be listed using XYLIST.
\end{minipage}\\
TITLE\>\>\begin{minipage}[t]{100mm}
A title (30 characters) for the output list.
The run-time default is the title of the input list.
\end{minipage}
\end{tabbing}
\item [METHOD] :
Each star image is located by forming marginal profiles in the X and Y
directions, subtracting a background estimate and fitting a gaussian of a
specified width.
Having thus located the centre of each star image, the surrounding data is
binned into iso-photal zones, which are circular or elliptical annuli centred
on the star.
The data in each zone is processed to remove aberrant values and a radial star
profile and background is then fitted to the binned data.
The integrated brightness is obtained from the integral of the fitted function.
The full form of the fitted star surface is:
\[D = A \exp (-0.5 \frac{r^{\prime}}{sigma} \exp(GAMMA)) + B\]
A is the star central amplitude, B the background and sigma the profile width.
The parameter GAMMA determines the rate of radial fall-off in the star profile
and is normally set around 2, corresponding to a gaussian profile.
The effective radial distance $r^{\prime}$ is derived from the true distance
from the star centre r, allowing for ellipticity in the star image.

The user must supply the axis ratio of the ellipse, the inclination of the major
axis to the X axis, the FWHM seeing across the minor ellipse axis and the
radial profile parameter GAMMA.
These parameters remain constant throughout a run.
The program calculates the centre, the central amplitude, the background and the
integrated magnitude for each star.
\item [NOTES]:
Values for the parameters SEEING, AXISR, THETA and GAMMA may be determined
using the program STARFIT.

Although the program will handle invalid pixels, the results may be unreliable
if there is a significant number of these within a star image.
The program also makes an attempt to reject blemishes during the radial
binning process, but this process will not be efficient if many blemishes occur
close to a star centre.

The execution time is approximately proportional to the number of stars fitted
and the area of image which each star occupies.
\item [ERRORS]:
The program may access the following error parameter(s) to indicate error
conditions:
\begin{verbatim}
    NOLIST
          ***OUTPUT LIST CONTAINS NO DATA
\end{verbatim}
No stars were successfully fited, so there are no results in the output list.
\item [MESSAGES]:
In addition to Starlink error parameters, the list of results may contain (if
ILEVEL=3) the following error messages, indicating why stars could not be
fitted:
\begin{verbatim}
    STAR CENTRE NOT FOUND
\end{verbatim}
The star centre was not found because the attempt to fit a gaussian to its
marginal profiles did not converge.
The image feature does not resemble a gaussian sufficiently closely.
\begin{verbatim}
    TOO LITTLE DATA TO FIND STAR
\end{verbatim}
There is insufficient valid data at the initial position given to allow the
star centre to be found.
\begin{verbatim}
    TOO LITTLE DATA TO FIT STAR
\end{verbatim}
Having binned the star data into isophotal zones, there are insufficient zones
containing valid data to fit the radial profile successfully.
\begin{verbatim}
    FIT GIVES NEGATIVE BRIGHTNESS
\end{verbatim}
After fitting the radial profile, the integrated intensity of the star appears
to be negative.
\end{description}

\goodbreak
\rule{\textwidth}{0.3mm}
{\Large {\bf STATS} \hfill {\bf STATS}}
\begin{description}
\item [FUNCTION]:
Computes and returns to the environment some key statistical parameters for a
Starlink frame.
The basic information is also displayed on the terminal; the rest may be
inspected (within DSCL) by typing LOOK STATS.
\item [COMMAND]:
\begin{quote}
{\tt STATS  [RMS=true]  [QUIET=true]}
\end{quote}
\item [PARAMETERS] :
\begin{tabbing}
XXXXXXXX\=XXXXXXXX\=\kill
IMAGE\>\>The input Starlink frame (any dimension).\\
NAXIS\>\>Returned as the number of dimensions of the image.\\
NAXIS1\>\>Returned as the size of the first dimension.\\
NAXIS2\>\>Returned as the size of the second dimension.\\
NAXIS3\>\>Returned as the size of the third dimension.\\
TOTAL\>\>Returned as the sum of all the data values in the frame.\\
MEAN\>\>Returned as the average data value.\\
MIN\>\>Returned as the smallest data value found.\\
MAX\>\>Returned as the largest data value found.\\
SIGM1\>\>Returned as MEAN - SIGMA.\\
SIGP7\>\>Returned as MEAN + 7*SIGMA.\\
STDEV\>\>\begin{minipage}[t]{100mm}
If RMS=TRUE (not the default) this is returned as the RMS deviation from the
mean.
This is usually a more meaningful value than the SIGMA used in SIGM1, SIGP7
{\em which is the square root of the average value}.
\end{minipage}\\
\\
(normally defaulted) :\\
\\
QUIET\>FALSE\>\begin{minipage}[t]{100mm}
By default some of the parameter values are displayed on the terminal.
If QUIET=TRUE then nothing is typed.
\end{minipage}\\
RMS\>FALSE\>\begin{minipage}[t]{100mm}
If RMS=TRUE then a value is returned in STDEV.
The default is false to save time computing its value when not needed.
\end{minipage}
\end{tabbing}
\item [NOTES]:
It is an easy way of getting the basic information about an image as a check on
its state, or prior to display.
In particular the SIGM1 and SIGP7 were designed to give good low and high values
for image display to show data close to the sky limit.
In fact the sigma used is simply the square root of the mean so if there is a
large offset the limits will not be as good as possible.
STDEV gives a much  better estimate of the image noise.
\end{description}

\goodbreak
\rule{\textwidth}{0.3mm}
{\Large {\bf SUB} \hfill {\bf SUB}}
\begin{description}
\item [FUNCTION]:
The difference IMAGEIN1-IMAGEIN2 is stored in IMAGEOUT.
\item [COMMAND]:
\begin{quote}
{\tt SUB  IMAGEIN1  IMAGEIN2  IMAGEOUT}
\end{quote}
\end{description}

\goodbreak
\rule{\textwidth}{0.3mm}
{\Large {\bf TBXY} \hfill {\bf TBXY}}
\begin{description}
\item [FUNCTION]:
Uses the ARGS cursor to pick a single point and return to the environment an
identification number for the image selected and the co-ordinates of the point
selected in user units (which are not necessarily the same as array units, but
usually will be).
\item [COMMAND]:
\begin{quote}
{\tt TBXY}
\end{quote}
\item [PARAMETERS] :
\begin{tabbing}
XXXXXXXX\=XXXXXXXX\=\kill
AX\>\>The returned x co-ordinate of the point selected in ARGS units.\\
AY\>\>The returned y co-ordinate of the point selected in ARGS units.\\
ID\>\>\begin{minipage}[t]{100mm}
The identification number of the image (or whatever) selected, as stored in the
ARGS database.
\end{minipage}\\
X\>\>\begin{minipage}[t]{100mm}
The returned x co-ordinate of the selected point in user units (usually array
units, but need not be).
\end{minipage}\\
Y\>\>\begin{minipage}[t]{100mm}
The returned y co-ordinate of the selected point in user units (usually array
units, but need not be).
\end{minipage}
\end{tabbing}
\item [NOTES]:
It is really a test program, not for serious use.

The trackerball buttons are used as follows:
\begin{quote}
\begin{tabbing}
XXXXXX\=XXXX\=\kill
GREEN\>1\>Pick the point --- must lie within one of the visible images.\\
WHITE\>2\>Same as 1.\\
WHITE\>3\>Same as 1.\\
RED\>4\>Same as 1.
\end{tabbing}
\end{quote}
\end{description}

\goodbreak
\rule{\textwidth}{0.3mm}
{\Large {\bf VERGREY} \hfill {\bf VERGREY}}
\begin{description}
\item [FUNCTION]:
Generates output which, when plotted on the Versatec, gives a greyscale
representation of the input image.
The technique used generates a density of dots in the output which is
proportional to the input data value.
The pattern of dots is random.
\item [COMMAND]:
\begin{quote}
{\tt VERGREY  [RATIO=r]}
\end{quote}
\item [PARAMETERS] :
\begin{tabbing}
XXXXXX\=XXXXXXXXXXXX\=\kill
INPUT\>\>This is the 2-D Starlink image which is to be plotted.\\
LIMITS\>Min,max\>\begin{minipage}[t]{100mm}
These are the low and high limits for scaling before output.
\end{minipage}\\
TITLE\>VERGREY output\>\begin{minipage}[t]{100mm}
This is a title which is plotted after the image, up to 80 characters.
\end{minipage}\\
TYPE\>1\>\begin{minipage}[t]{100mm}
This defines the scaling to be used.
The default sets LIMIT(1) to `white' and LIMIT(2) to `black' and uses linear
scaling; 2 gives the same limits but logarithmic scaling between them.
The corresponding -ve values reverse black and white.
\end{minipage}\\
WIDTH\>4.0\>\begin{minipage}[t]{100mm}
This is the width of the plot on the Versatec.
It may be between 1.0 and 10.5 inches, but note that the time to generate the
output is proportional to the square of the width.
The actual plot will be as close to this value as possible, as each pixel must
contain an integral number of dots.
\end{minipage}\\
\\
(normally defaulted) :\\
\\
RATIO\>1.0\>\begin{minipage}[t]{100mm}
This is the ratio of the height (y) of the pixels to their width (x).
A the nearest value is used given the discrete nature of the medium.
\end{minipage}\\
KEEP\>DELETE\>\begin{minipage}[t]{100mm}
This determines if the DIT.DAT file is to be deleted when the output has been
produced.
By default it will be deleted.
If the file is to be kept then this parameter must be set to KEEP.
\end{minipage}
\end{tabbing}
\item [NOTES]:
It gives a satisfactory form of hard copy of image data, but it is rather slow
and sometimes disappointing.
The data are written to a file called DIT.DAT which is printed on the Versatec
automatically when VERGREY terminates.
Additional copies may be obtained by typing the following command:
\begin{verbatim}
    $ PRINT/QUEUE=SYS_VERSATEC/PASSALL/NOFEED DIT.DAT
\end{verbatim}
\end{description}

\goodbreak
\rule{\textwidth}{0.3mm}
{\Large {\bf WRDSCR} \hfill {\bf WRDSCR}}
\begin{description}
\item [FUNCTION]:
It allows descriptor items (specified by NAME) to be overwritten with new
information (VALUE) or inserted if they were not present.
\item [COMMAND]:
\begin{quote}
{\tt WRDSCR   [LOOP=false]}
\end{quote}
\item [PARAMETERS] :
\begin{tabbing}
XXXXXXXX\=XXXXXXXX\=\kill
FRAME\>\>\begin{minipage}[t]{100mm}
The name of the (existing) frame to which the descriptors are to be added.
\end{minipage}\\
NAME\>\>\begin{minipage}[t]{100mm}
The name of the descriptor (up to 20 characters, null response to exit).
\end{minipage}\\
VALUE\>\>\begin{minipage}[t]{100mm}
A character string containing the `value' of the descriptor.
May be up to 72 characters.
\end{minipage}\\
\\
(normally defaulted) :\\
\\
LOOP\>TRUE\>\begin{minipage}[t]{100mm}
If TRUE it allows many items to be inserted; if FALSE then only one is allowed.
\end{minipage}
\end{tabbing}
\item [NOTES]:
It may used to store useful information with image data.
\end{description}

\goodbreak
\rule{\textwidth}{0.3mm}
{\Large {\bf XYCURA} \hfill {\bf XYCURA}}
\begin{description}
\item [FUNCTION]:
Make a list of X,Y positions by using a cursor to define the positions on an
image displayed on the ARGS.
The positions can have identifiers given to them.
Also to display on such an image the X,Y positions of an optional input list.
If this is done and the cursor is used to define more positions, these are
added to the input list instead of a new list being made.
These positions can be displayed as crosses; as ticks displaced alternately by
(-2,0) and (2,0); as 1 pixel spots.
Of the input list, only those positions inside the area of the image displayed
on the ARGS will be put up.
\item [COMMAND]:
\begin{quote}
{\tt XYCURA  [ILEVEL=] [MARKER=] [COLOUR=] [IDENTIFY=] [MAXENTRY=]}
\end{quote}
\item [PARAMETERS] :
\begin{tabbing}
XXXXXXXXX\=XXXXXXXXXXX\=\kill
OVERCL\>NO\>\begin{minipage}[t]{100mm}
Flag for clearing the crosses left on the ARGS from any previous XYCURA.
Choices are NO,YES.
\end{minipage}\\
INPUT\>\>The optional input XY list, return null if there is none.\\
FXOUT\>YES\>\begin{minipage}[t]{100mm}
Flag for putting up on the ARGS the crosses from any input list.
Choices are YES,NO.
\end{minipage}\\
NUMINP\>NO\>Flag for adding the number of the cross of the input list.\\
OUTPUT\>\>The output file containing the positions.\\
TITLE\>XYCURA output\>The Title to be included in the output file.\\
IDENTITY\>\>If inputting star names, this is the name of the star.\\
\\
(normally defaulted) :\\
\\
ILEVEL\>2\>\begin{minipage}[t]{100mm}
Flag for outputting on the terminal the positions from the cursor as you go
along.
The default (=2) makes it happen automatically.
Setting it to 1 on running the program supresses it.
\end{minipage}\\
MARKER\>CROSS\>\begin{minipage}[t]{100mm}
The type of mark made at the positions of the input XYlist.
Choices are crosses; ticks at (-2,0) and (2,0) alternately; spots.
Choices are CROSS, ATICK, SPOT
\end{minipage}\\
COLOUR\>B\>\begin{minipage}[t]{100mm}
The colour of the cross put on the input list positions.
Choices are: White, Red, Blue, Yellow, Cyan, Magenta, Green, (use only first
letter).
\end{minipage}\\
MAXENTRY\>2000\>\begin{minipage}[t]{100mm}
The maximum number of positions that can be input by the cursor --- default
2000.
You can change this when running the program.
\end{minipage}\\
IDENTIFY\>FALSE\>\begin{minipage}[t]{100mm}
Flag for choosing whether to put in via keyboard the name of the star at the
position chosen via the cursor.
To select this option, put IDENTIFY=TRUE when running XYCURA.
If not chosen, the stars are labelled as sequential number (\#n).
\end{minipage}
\end{tabbing}
\item [NOTES]:
First an image must be on the ARGS (by ICDISP, ADISP, etc.\ (qv)).
You can clear any existing crosses displayed on the image, and then choose
whether to have an input XYlist.
If you do, then you choose whether to display the input positions as blue
crosses and whether to display their numbers in the list.

Then input the new positions by means of the cursor.
The positions are then marked with red crosses.
If you have chosen to add identifiers you enter them on the keyboard after each
use of the cursor.
(The choice is made by setting IDENTIFY=TRUE on runnung the program.)
To exit, input a position at or beyond the bottom left hand corner of the image.
The positions are then output to the XYlist.

The trackerball buttons are used as follows:
\begin{quote}
\begin{tabbing}
XXXXXX\=XXXX\=\kill
GREEN\>1\>Reverts centre to image centre and magnification to 1.\\
WHITE\>2\>Decreases magnification by times 2.\\
WHITE\>3\>Increases magnification by times 2.\\
RED\>4\>\begin{minipage}[t]{100mm}
Inputs present position of cursor as star position.
If this is the bottom lh of image blh corner, the program finishes.
\end{minipage}
\end{tabbing}
\end{quote}
Perform all initialisation to allow use of bit plane `n' for overlays.
Overlays will be in colour `col' (see `args\_decol').
\end{description}

\goodbreak
\rule{\textwidth}{0.3mm}
{\Large {\bf XYCURB} \hfill {\bf XYCURB}}
\begin{description}
\item [FUNCTION]:
This program displays a section of an XY list of positions on the ARGS,
numbered in their order in the list.
By means of the cursor you can amend the positions of the stars in that section.
You can then choose whether to store just the amended section or the whole input
list with the amended section.
The identifiers of the objects are preserved.
Optionally,  positions which lie outside the displayed area can not be put up,
with their positions left unchanged.
You can choose whether to keep points which lie outside the displayed area.
A position can be removed from the list by putting the refined position to the
top l.h. corner of the image.
\item [COMMAND]:
\begin{quote}
{\tt XYCURB  [ILEVEL=]}
\end{quote}
\item [PARAMETERS] :
\begin{tabbing}
XXXXXXXX\=XXXXXXXX\=\kill
INPUT\>\>The input XY list.\\
LIMITS\>whole\>The section of the list to be amended.\\
OVERCL\>N\>\begin{minipage}[t]{100mm}
Flag for clearing the crosses left on the ARGS from any previous program.
Choices are N,Y.
\end{minipage}\\
NUMINP\>NO\>Flag for adding the number of the cross of the input list.\\
OUTPUT\>\>The Output file containing the positions.\\
TITLE\>\>The Title to be included in the Output file.\\
ALLINP\>YES\>\begin{minipage}[t]{100mm}
The output list to be the input list with the amended positions, or not, when
only the amended positions are stored.
\end{minipage}\\
OPTION\>INSIDE\>\begin{minipage}[t]{100mm}
Choice to display only those inside the area on the ARGS or all.
Choices are INSIDE,ALL.
\end{minipage}\\
SAVEOUT\>NO\>\begin{minipage}[t]{100mm}
Choice whether to save points outside displayed area.
Choices are YES,NO.
\end{minipage}\\
\\
(normally defaulted) :\\
\\
ILEVEL\>\>\begin{minipage}[t]{100mm}
Flag for outputting on the terminal the positions from the cursor as you go
along.
The default (=2) makes it happen automatically.
Setting it to 1 on running the program supresses it.
\end{minipage}
\end{tabbing}
\item [NOTES]:
When the section is put up, the cursor is set to the 1st object.
You move the cursor to an amended position and press the red button.
The position is stored and the cursor moves to the next object, until the
section is finished.
A position to the top l.h. of the image means that that position is to be
removed from the list.
To exit from amending positions, place the cursor at the bottom l.h. corner of
the image or below and to the left of that corner and press the button.
The remaining positions will still be copied over.

The trackerball buttons are used as follows:
\begin{quote}
\begin{tabbing}
XXXXXX\=XXXX\=\kill
GREEN\>1\>Reverts centre to image centre and magnification to 1.\\
WHITE\>2\>Decreases magnification by times 2.\\
WHITE\>3\>Increases magnification by times 2.\\
RED\>4\>\begin{minipage}[t]{100mm}
Inputs present position of cursor as star position.
If this is to bottom lh of image blh corner, the program finishes.
\end{minipage}
\end{tabbing}
\end{quote}
Perform all initialisation to allow use of bit plane `n' for overlays.
Overlays will be in colour `col' (see `args\_decol').
\end{description}

\goodbreak
\rule{\textwidth}{0.3mm}
{\Large {\bf XYDRAWA} \hfill {\bf XYDRAWA}}
\begin{description}
\item [FUNCTION]:
Draws out curves from XYlists.
Takes two paramters from one or two XY lists and plots out a smooth curve
joining the points.
The interpolation is by cubic splines.
The numbers are taken from file A, parameter no A1 and file B, parameter no B1,
where A, A1, B, B1 are input by the user.
More than one line may be drawn on a graph.
The output can be put out on the ARGS, Tek, Versatec, Calcomp, GOC, or CC81.
\item [COMMAND]:
\begin{quote}
{\tt XYDRAWA}
\end{quote}
\item [PARAMETERS] :
\begin{tabbing}
XXXXXXXX\=XXXXXXXX\=\kill
INPUTA\>\>The 1st input XY list.\\
PARNOA\>\>The parameter number in the first file to take the X values from.\\
INPUTB\>\>The 2nd input file.\\
PARNOB\>\>The parameter number in the second file to take the Y values from.\\
DEVICE\>ARGS\>\begin{minipage}[t]{100mm}
The display device for the graph.
Choices are ARGS, GOC, TEKTRONIX, VERSATEC, CC81, CALCOMP.
\end{minipage}\\
DEVSIZE\>Various\>For graphs,the size of the picture.\\
DEVLIMX\>Min,Max\>For graphs, the range of Xs to be plotted out.\\
DEVLIMY\>Min,Max\>For graphs, the range of Ys to be plotted out.\\
TEXTX\>\>Text to write on X axis.\\
TEXTY\>Like TEXTX\>Text to write on Y axis.\\
TEXT\>\>For hard-copy output, this is some text to be put below the diagram.\\
ANOTHER\>No\>Flag to add another line. Choices are YES/NO.
\end{tabbing}
\end{description}

\goodbreak
\rule{\textwidth}{0.3mm}
{\Large {\bf XYEDIT} \hfill {\bf XYEDIT}}
\begin{description}
\item [FUNCTION]:
Used to enable EDRS XY lists to be edited easily.
\item [COMMAND]:
\begin{quote}
{\tt XYEDIT}
\end{quote}
\item [PARAMETERS] :
\begin{tabbing}
XXXXXXXX\=XXXXXXXX\=\kill
OPTION\>TOASCII\>\begin{minipage}[t]{100mm}
Flag for which way to go --- choices are TOASCII, FROMASCII.
\end{minipage}\\
INPUTA\>ASCII.LIS\>\begin{minipage}[t]{100mm}
The name of the ASCII input file.
It must be a .LIS file, but the user does not type the .LIS
\end{minipage}\\
OUTPUTX\>\>The name of the XY list output file.\\
INPUTX\>\>The name of the XY list input file.\\
OUTPUTA\>ASCII.LIS\>\begin{minipage}[t]{100mm}
The name of the ASCII output file.
It will be made a .LIS file, but again do not type the .LIS.
\end{minipage}
\end{tabbing}
\item [NOTES]:
By choosing the option, an XY list can be turned into an ASCII file, which can
then be edited normally.
By choosing the other option, such an appropriately structured ASCII file can be
turned (back perhaps) into an XY list.

By choosing TOASCII, the XY list is turned into a file of format:
\begin{tabbing}
XXXX\=XXXX\=\kill
\>One header line with Title, No of parameters, No of entries as:\\
\>\>7X,A30,8X,I5,8X,I10\\
\>A series of entries with the header descriptors as:\\
\>\>1X,'HEADER ',A30,' END'\\
\>A series of the entries with identifier and the parameter values as:\\
\>\>1X,20A1,5F12.4\\
\>\>5F12.4\\
\>\>\ldots
\end{tabbing}
By choosing option FROMASCII, such an ASCII file is turned into an XY list.
\end{description}

\goodbreak
\rule{\textwidth}{0.3mm}
{\Large {\bf XYLIST} \hfill {\bf XYLIST}}
\begin{description}
\item [FUNCTION]:
Reads any XY file in the EDRS format and types out the identifiers, XY coords
and the first of the any parameters, and also makes a .LIS file containing all
the data.
The .LIS file can then be output on any printing device.
\item [COMMAND]:
\begin{quote}
{\tt XYLISTA}
\end{quote}
\item [PARAMETERS] :
\begin{tabbing}
XXXXXXX\=XXXXXXXXXX\=\kill
INPUT\>\>The name of the .BDF file which you want to list.\\
OUTPUT\>XYLISTA.LIS\>\begin{minipage}[t]{100mm}
The name of the formatted file to which the results are written.
The user supplies the name only; the program adds .LIS.
XYLISTA is the default name.
\end{minipage}
\end{tabbing}
\item [NOTES]:
It is complementary to XYLIST and XYLISP, which are parts of the EDRS package.
It differs from them in printing up to 200 numbers per record.
\end{description}
\end{document}
