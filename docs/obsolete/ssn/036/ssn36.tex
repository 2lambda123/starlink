\documentclass[11pt]{article}
\pagestyle{myheadings}

% -----------------------------------------------------------------------------
\newcommand{\stardoccategory}  {Starlink System Note}
\newcommand{\stardocinitials}  {SSN}
\newcommand{\stardocnumber}    {36.1}
\newcommand{\stardocsource}    {ssn\stardocnumber}
\newcommand{\stardocauthors}   {K P Duffey}
\newcommand{\stardocdate}      {7th February 1996}
\newcommand{\stardoctitle}     {XpressWare Installation User guide}
% -----------------------------------------------------------------------------

\newcommand{\stardocname}{\stardocinitials /\stardocnumber}
\markright{\stardocname}
\setlength{\textwidth}{160mm}
\setlength{\textheight}{230mm}
\setlength{\topmargin}{-2mm}
\setlength{\oddsidemargin}{0mm}
\setlength{\evensidemargin}{0mm}
\setlength{\parindent}{0mm}
\setlength{\parskip}{\medskipamount}
\setlength{\unitlength}{1mm}

% -----------------------------------------------------------------------------
% Hypertext definitions.
% These are used by the LaTeX2HTML translator in conjuction with star2html.

% Comment.sty: version 2.0, 19 June 1992
% Selectively in/exclude pieces of text.
%
% Author
%    Victor Eijkhout                                      <eijkhout@cs.utk.edu>
%    Department of Computer Science
%    University Tennessee at Knoxville
%    104 Ayres Hall
%    Knoxville, TN 37996
%    USA

%  Do not remove the %begin{latexonly} and %end{latexonly} lines (used by
%  star2html to signify raw TeX that latex2html cannot process).
%begin{latexonly}
\makeatletter
\def\makeinnocent#1{\catcode`#1=12 }
\def\csarg#1#2{\expandafter#1\csname#2\endcsname}

\def\ThrowAwayComment#1{\begingroup
    \def\CurrentComment{#1}%
    \let\do\makeinnocent \dospecials
    \makeinnocent\^^L% and whatever other special cases
    \endlinechar`\^^M \catcode`\^^M=12 \xComment}
{\catcode`\^^M=12 \endlinechar=-1 %
 \gdef\xComment#1^^M{\def\test{#1}
      \csarg\ifx{PlainEnd\CurrentComment Test}\test
          \let\html@next\endgroup
      \else \csarg\ifx{LaLaEnd\CurrentComment Test}\test
            \edef\html@next{\endgroup\noexpand\end{\CurrentComment}}
      \else \let\html@next\xComment
      \fi \fi \html@next}
}
\makeatother

\def\includecomment
 #1{\expandafter\def\csname#1\endcsname{}%
    \expandafter\def\csname end#1\endcsname{}}
\def\excludecomment
 #1{\expandafter\def\csname#1\endcsname{\ThrowAwayComment{#1}}%
    {\escapechar=-1\relax
     \csarg\xdef{PlainEnd#1Test}{\string\\end#1}%
     \csarg\xdef{LaLaEnd#1Test}{\string\\end\string\{#1\string\}}%
    }}

%  Define environments that ignore their contents.
\excludecomment{comment}
\excludecomment{rawhtml}
\excludecomment{htmlonly}

%  Hypertext commands etc. This is a condensed version of the html.sty
%  file supplied with LaTeX2HTML by: Nikos Drakos <nikos@cbl.leeds.ac.uk> &
%  Jelle van Zeijl <jvzeijl@isou17.estec.esa.nl>. The LaTeX2HTML documentation
%  should be consulted about all commands (and the environments defined above)
%  except \xref and \xlabel which are Starlink specific.

\newcommand{\htmladdnormallinkfoot}[2]{#1\footnote{#2}}
\newcommand{\htmladdnormallink}[2]{#1}
\newcommand{\htmladdimg}[1]{}
\newenvironment{latexonly}{}{}
\newcommand{\hyperref}[4]{#2\ref{#4}#3}
\newcommand{\htmlref}[2]{#1}
\newcommand{\htmlimage}[1]{}
\newcommand{\htmladdtonavigation}[1]{}

% Define commands for HTML-only or LaTeX-only text.
\newcommand{\html}[1]{}
\newcommand{\latex}[1]{#1}

% Use latex2html 98.2.
\newcommand{\latexhtml}[2]{#1}

% Starlink cross-references and labels.
\newcommand{\xref}[3]{#1}
\newcommand{\xlabel}[1]{}

%  LaTeX2HTML symbol.
\newcommand{\latextohtml}{{\bf LaTeX}{2}{\tt{HTML}}}

%  Define command to recentre underscore for Latex and leave as normal
%  for HTML (severe problems with \_ in tabbing environments and \_\_
%  generally otherwise).
\newcommand{\setunderscore}{\renewcommand{\_}{{\tt\symbol{95}}}}
\latex{\setunderscore}

% -----------------------------------------------------------------------------
%  Debugging.
%  =========
%  Un-comment the following to debug links in the HTML version using Latex.

% \newcommand{\hotlink}[2]{\fbox{\begin{tabular}[t]{@{}c@{}}#1\\\hline{\footnotesize #2}\end{tabular}}}
% \renewcommand{\htmladdnormallinkfoot}[2]{\hotlink{#1}{#2}}
% \renewcommand{\htmladdnormallink}[2]{\hotlink{#1}{#2}}
% \renewcommand{\hyperref}[4]{\hotlink{#1}{\S\ref{#4}}}
% \renewcommand{\htmlref}[2]{\hotlink{#1}{\S\ref{#2}}}
% \renewcommand{\xref}[3]{\hotlink{#1}{#2 -- #3}}
%end{latexonly}
% -----------------------------------------------------------------------------
% Add any document-specific \newcommand or \newenvironment commands here

% -----------------------------------------------------------------------------
%  Title Page.
%  ===========
\begin{document}
\thispagestyle{empty}

%  Latex document header.
\begin{latexonly}
   CCLRC / {\sc Rutherford Appleton Laboratory} \hfill {\bf \stardocname}\\
   {\large Particle Physics \& Astronomy Research Council}\\
   {\large Starlink Project\\}
   {\large \stardoccategory\ \stardocnumber}
   \begin{flushright}
   \stardocauthors\\
   \stardocdate
   \end{flushright}
   \vspace{-4mm}
   \rule{\textwidth}{0.5mm}
   \vspace{5mm}
   \begin{center}
   {\Large\bf \stardoctitle}
   \end{center}
   \vspace{5mm}

%  Add heading for abstract if used.
%   \vspace{10mm}
%   \begin{center}
%      {\Large\bf Description}
%   \end{center}
\end{latexonly}

%  HTML documentation header.
\begin{htmlonly}
   \xlabel{}
   \begin{rawhtml} <H1> \end{rawhtml}
      \stardoctitle
   \begin{rawhtml} </H1> \end{rawhtml}

%  Add picture here if required.

   \begin{rawhtml} <P> <I> \end{rawhtml}
   \stardoccategory\ \stardocnumber \\
   \stardocauthors \\
   \stardocdate
   \begin{rawhtml} </I> </P> <H3> \end{rawhtml}
      \htmladdnormallink{CCLRC}{http://www.cclrc.ac.uk} /
      \htmladdnormallink{Rutherford Appleton Laboratory}
                        {http://www.cclrc.ac.uk/ral} \\
      Particle Physics \& Astronomy Research Council \\
   \begin{rawhtml} </H3> <H2> \end{rawhtml}
      \htmladdnormallink{Starlink Project}{http://www.starlink.ac.uk/}
   \begin{rawhtml} </H2> \end{rawhtml}
   \htmladdnormallink{\htmladdimg{source.gif} Retrieve hardcopy}
      {http://www.starlink.ac.uk/cgi-bin/hcserver?\stardocsource}\\

% HTML document table of contents (if used).
% ==========================================
% Add table of contents header and a navigation button to return
% to this point in the document (this should always go before the
% abstract \section). This places the table of contents on the title
% page. Do not use this if you want the normal behaviour.
   \label{stardoccontents}
   \begin{rawhtml}
     <HR>
     <H2>Contents</H2>
   \end{rawhtml}
   \htmladdtonavigation{\htmlref{\htmladdimg{contents_motif.gif}}
                                            {stardoccontents}}

%  Start new section for abstract if used.
%  \section{\xlabel{abstract}Abstract}

\end{htmlonly}

% -----------------------------------------------------------------------------
%  Document Abstract. (if used)
%  ==================
% -----------------------------------------------------------------------------
%  Latex document Table of Contents. (if used)
%  ===========================================
%  Replace the \latexonlytoc command with \tableofcontents if you're
%  not only having a contents list on the title page.
 \begin{latexonly}
    \setlength{\parskip}{0mm}
    \tableofcontents
    \setlength{\parskip}{\medskipamount}
    \markright{\stardocname}
 \end{latexonly}
% -----------------------------------------------------------------------------


\clearpage

\section {Introduction}

XpressWare is a set of X terminal software, released by Tektronix Inc, that
accommodates the X Window system on a range of host computers. The software
comprises boot files (the X server image), configuration files, fonts, and
font tools to support the X terminal. The files can be installed on one host
or distributed across multiple hosts

The purpose of this guide is to present the system or network administrator
with a step-by-step account of how to install XpressWare, and how subsequently
to configure the X terminals appropriately for the environment in which they
operate.

It is assumed that the administrator is familiar with the operating system of
the intended host computer, and conversant with the network protocols and
concepts related to the local-area network (LAN). A working knowledge of the
X environment, including window managers and display managers is also assumed.

The contents of the guide include:

\begin {enumerate}

\item Extracting the INSTALL script from the distribution medium.

\item Using the INSTALL script to extract the required XpressWare files.

\item X terminal installation.

\item Booting the first X terminal.

\item Starting a Telnet session.

\item Configuring the X terminal environment.

\item A sample INSTALL session.

\end {enumerate}


\section {Getting Started}


\subsection {....Before You Begin!}

Each Version of XpressWare is accompanied by a set of Release notes, which
contain information that is likely to influence decisions made during the
installation. It is highly recommended that the Release notes be read thoroughly
before commencing the installation of the relevant Version for the first time.
Though it should be apparent that the essence of the Release notes is contained
within this user guide, with emphasis on installation for Sun5 (Solaris 2.X)
and DEC (OSF/1) systems.


\subsection {Software Distribution}

The XpressWare Software is (re-)released periodically as a new Version, which
generally incorporates a number of bug-fixes for the previous version(s), and
perhaps one or more additional features. The current version at the time of
writing is XpressWare Version 8.0.

Under a maintenance agreement with Tektronix UK, the Starlink Project receive
the XpressWare updates on CD-ROM, with a ``right to copy'' for each of the
Starlink Sites covered by the agreement.


\section {Installation from CD-ROM}

{\bf Note:} You must be logged in as {\tt root} to perform an installation.
Before starting, verify that your root umask is set correctly to preserve file
permissions. Your umask should be set to 022, so that {\tt root} has read,
write and execute privileges, but others only have read and execute. To set your
umask type:

{\bf \# umask 022}


\subsection {Installation Considerations}

\paragraph {NFS Boot Security}

To boot via NFS, the installation directory must be exported so the X terminal
can access the boot files. Select an installation directory tree that does not
contain secured or proprietary information. For example, you might install
files under the directory {\tt /usr/tekxp} instead of {\tt /tekxp}. Thereby
exporting the {\tt /usr/tekxp} partition instead of the root ({\tt /}) or
{\tt /usr} partition.

\paragraph {Secure tftp}

There are considerations if the boot or font hosts use {\tt secure tftp}. Secure
tftp does not follow symbolic links to files outside the secure directory, so
all boot and configuration files must be installed under the common secure
directory (for example, {\tt /usr/tekxp}). The secure directory is either
specified in the {\tt inetd.conf} file or in the {\tt /etc/passwd} file,
depending on your host.

Although you cannot link the files outside the secure directory, it is possible
to symbolically link the secure directory to another partition if disk space is
limited in the secure directory. For example, {\tt /tftpboot/tekxp} could be
linked to {\tt /usr/tftpboot/tekxp}. Refer to your host documentation for
specific information regarding {\tt secure tftp}.


\subsection {Extracting the INSTALL Script}

It is only necessary to extract the INSTALL script from the distribution medium
in preparation for the first installation of a given Release of XpressWare.
The same script can be re-used to extract additional files or to re-install the
software.

\begin {enumerate}

\item If you have your own distribution CD-ROM, then insert the CD-ROM into the
drive.

\item Use {\bf cd} to change to the parent directory on the local host, where
the installation software is to be off-loaded. The directory must have world
read and execute privilege. This is where the {\tt INSTALL} script will build
the {\tt tekxp} directory tree. A typical installation requires about 80 MByte
of free space. Online Xpress, the online documentation set, requires another
100 MByte or so.

{\bf Note} that we are using the {\tt root} partition by way of example,
throughout:

{\bf \# cd /}

{\bf Note}, moreover, the comments regarding security, given at section 3.1
above.

\item Enter the command to mount the CD-ROM to the {\tt /cdrom} directory (if
you are running the volume manager, this step is not necessary).

Sun5: {\bf \# mount -F hsfs -r /dev/sdNc /cdrom}

The {\bf -F} indicates the file system type ({\tt hsfs}), the {\bf -r} indicates
the contents are {\tt read-only}, and {\tt /dev/sd}{\bf N}{\tt c} is the device
name where {\bf N} is the logical unit number of the CD-ROM.

Digital Unix: {\bf \# mount -t cdfs -o noversion /dev/rzNc /cdrom}

The {\bf -t} indicates the file system type ({\tt cdfs}), {\bf -o noversion}
strips version numbers and does not convert file names to uppercase, and
{\tt /dev/rz}{\bf N}{\tt c} is the device name where {\bf N} is the logical
unit number of the CD-ROM.

Where the distribution software is to be copied from RAL, it is recommended
that the CD-ROM be mounted remotely. For example, the following command would
be appropriate for a network transfer of XpressWare from a CD-ROM set up on the
host {\tt rlssp2}, running Solaris 2.X.

{\bf \# mount -F nfs rlssp2.bnsc.rl.ac.uk:/cdrom/cdrom0 /mnt}

\item Extract the INSTALL script.

Sun5: {\bf \# tar -xvpf /cdrom/cdrom0/sun/install.tar}

or

{\bf \# tar -xvpf /mnt/sun/install.tar} if the CD-ROM has been mounted
remotely with a mount-point of {\bf /mnt}.

Digital Unix: {\bf \# tar -xvpf /cdrom/cdrom0/ultrix/install.tar}

or

{\bf \# tar -xvpf /mnt/ultrix/install.tar} if the CD-ROM has been mounted
remotely with a mount-point of {\bf /mnt}.

\end {enumerate}


\subsection {Using the INSTALL Script}

When you run {\tt INSTALL}, it creates a log file
({\tt /tekxp/INSTALL/install.log}) to keep track of installation activities.
If you have multiple log files, the installation date and time are appended to
the file. If, for example, you perform a partial installation initially, and
subsequently include further X Terminal models in the System, the additional
XpressWare files can be installed as required.

If you have a current installation, your configuration files are saved for you.
Be sure to check the new configuration files for any new commands, and add any
applicable commands to your saved configuration files.

\begin {itemize}

\item If there is a previous installation in the current directory, the
configuration files ({\tt .cnf} and {\tt .tbl} files) are automatically
preserved. The new configuration files unloaded from the media are stored in
the file {\tt /tekxp/config\_date} where {\tt date} is the date and time.

\item If you have an installation in another directory, you can invoke the
{\tt INSTALL} script with the {\bf -oldq} and {\bf -oldu} switches to preserve
existing configuration files. New configuration files unloaded from the media
are saved as {\tt file.date,} where {\tt date} is the date and time.

\item There are {\bf -host} and {\bf -user} switches to install files from a
remote host.

\end {itemize}

To run the installation script:

\begin {enumerate}

\item Use {\bf cd} to change to the {\tt INSTALL} directory,
{\tt tekxp/INSTALL}.

{\bf \# cd tekxp/INSTALL}

\item Make sure that the media is still in the drive, then run the script:

{\bf \# ./INSTALL -f} {\tt file} [{\bf -oldq} {\tt dir}] [{\bf -oldu} {\tt dir}]
[{\bf -host} {\tt host}] [{\bf -user} {\tt user}]

{\bf -f} {\tt file} specifies the name of the tar file. {\tt file} is one of the
following:

Sun5 (with volume manager), Digital Unix: {\tt /cdrom/cdrom0/common/tekxp.tar}

or {\tt /mnt/common/tekxp.tar} if the CD-ROM has been mounted remotely with a
mount-point of {\bf /mnt}.

{\bf -oldq} and {\bf -oldu} save the configuration files from a previous
installation using the {\tt TekXpress Quick Install} and {\tt TekXpress Utility}
tapes.

{\tt dir} specifies the location of those configuration files. For
example, the default path for the {\tt xp.cnf} file would be:

{\bf -oldq}{\tt /tftpboot/XP}

and for the {\tt .tbl} files:

{\bf -oldu}{\tt /usr/lib/X11/XP/site}.

{\bf -host} and {\bf -user} options are used to install the files from a remote
{\tt host}. You must specify a valid user name ({\tt user}) for the remote host.

\item The CD-ROM supports multiple hosts. Unless you want to install all of the
host files, answer {\tt no} ({\bf n}) to the question:
{\tt Do you wish to do a full install}. The script prompts you through the
available host choices, and then through the file choices. Each file group
displays the approximate size. Answer {\tt yes} ({\bf y}) to install a group,
{\tt no} ({\bf n}) to skip the group, or, when listed, {\tt partial} ({\bf p})
to selectively install a file group (such as font groups).

{\bf Note:} An example INSTALL session is summarised as an appendix to this
guide.

\item When you have answered all the questions, the groups you have selected
are displayed. Enter {\bf y} to accept the groups, of {\bf n} to abort.

\item When complete, unmount and remove the disc from the drive:

{\bf \# umount /cdrom}

or

{\bf \# eject}\ \ \ \ \ \ \ if using the volume manager on a local system

\end {enumerate}


\subsection {Media Contents}

The directory hierarchy is created relative to the current directory. For
example, if you are at the {\bf root} partition ({\bf /}), the directory
{\tt /tekxp} is created., if you are at {\tt /usr}, the directory is
{\tt /usr/tekxp}.  Instructions in this guide assume that the initial directory
is {\tt /tekxp}. The {\tt tekxp} directories are as follows:

\begin {itemize}

\item {\tt /tekxp/INSTALL} contains the installation scripts and log files.

\item {\tt /tekxp/bin/<host>} contains X terminal-specific utilities.

\item {\tt /tekxp/boot} contains boot files, fonts and clients. Files that
differ between models use a {\tt .model} suffix.

\item {\tt /tekxp/boot/config} contains the X terminal configuration files.

\item {\tt /tekxp/boot/fonts} contains the supplied non-resident fonts.

\item {\tt /tekxp/boot/<language\_directory>/app-defaults} contains translated
text for X terminal clients.

\item {\tt /tekxp/doc/bin} contains binaries to view Online Xpress.

\item {\tt /tekxp/doc/online} contains Online Xpress, the online documents.

\item {\tt /tekxp/examples} contains examples such as {\tt Xsession} and
{\tt bootptab}.

\item {\tt /tekxp/man} contains man pages for many of the included utilities.

\item {\tt /tekxp/mgmt} contains a sample SNMP MIB file.

\item {\tt /tekxp/obsolete} contains files that support earlier X terminals.

\item {\tt /tekxp/src} contains source files to support generic UNIX hosts.

\end {itemize}


\subsection {Conserving Disk Space}

Depending on the manufacturer of the intended host, model type of the X
terminals, and whether or not the software options are to be used or the Online
Xpress documentation is to be installed, options are available to minimise the
disk space used for the XpressWare installation. Ways to conserve space
include:

\begin {itemize}

\item Using the fonts already available on the host instead of those supplied
with XpressWare software.

For Sun5 (Solaris 2.X) hosts, native fonts are typically available in

{\tt /usr/openwin/lib/X11/fonts}.

For DEC (Digital Unix) hosts, native fonts are typically available in

{\tt /usr/lib/X11/fonts}.

{\tt TekXpress} X terminals can use these fonts directly. If you choose not to
off-load the fonts from the media, the INSTALL script will automatically update
the {\tt fonts.tbl} file with the default path for the native fonts. If the
fonts on your host are located elsewhere, change the path in the {\tt fonts.tbl}
file to the correct directory.

As a second step to access the default fonts, add to the {\tt xp.cnf} file an
nfs\_table entry that mounts the X terminal to the {\tt /usr} directory of the
host. Remember to export {\tt /usr} to allow the X terminal access. Both the
{\tt xp.cnf} and {\tt fonts.tbl} files are located at {\tt /tekxp/boot/config}
following the installation. For more information on either file, consult the
{\tt TekXpress} X terminal documentation. Any unresolved issues may be addressed
to {\tt Kevin Duffey} (see section 6).

\item Performing a partial install supporting just those X terminal models that
are present in the required system.

If your X terminal system includes only one or two model types, such as XP350
and XP200 Series, you can elect to perform a partial installation. You have the
opportunity to specify a partial installation shortly after invoking the INSTALL
script. When the script asks the question: {\tt Do you wish to do a full install}
respond with {\bf n}. The INSTALL script responds with questions about specific
model types in your system. Requesting only specific model support can save you
a substantial amount of disk space.

\item Removing certain files from the installation based on purchased options.
If you decide not to use the Online Xpress documents, refuse the
{\tt doc\_viewer} and {\tt doc\_online} bundles while running the INSTALL
script. If you want to use the online documents but do not need the
multimedia functions, remove the
{\tt /sights} and {\tt /sounds} subdirectories from each document subdirectory
located at {\tt /tekxp/doc/online} after the installation. You
may also remove Frame's Online Manuals directory at

{\tt /tekxp/doc/bin/frame4/fminit/usenglish}.

This is the only directory under {\tt /tekxp/doc/bin/frame4}
that you may remove.

Another group of files that you might remove following the installation are the
optional local clients, located at {\tt /tekxp/boot}. Local clients that you
might consider removing are:

\begin {description}

\item [-\ \ ] {\tt mwm.*} and {\tt mwmv1.*} (local Motif window managers)

\item [-\ \ ] {\tt olwm.*} (local OPEN LOOK window manager)

\item [-\ \ ] {\tt tek3179g.*} and {\tt tek3270.*} (local IBM terminal emulators)

\item [-\ \ ] {\tt tek340.*} (local DEC terminal emulator)

\item [-\ \ ] {\tt vplay.350} and {\tt aplay.350} (local digital video and
audio players)

\item [-\ \ ] {\tt windd.*} (local Windows NT access client)

\item [-\ \ ] {\tt xie.*, xie\_dis.*,} and {\tt xieview.*} (local XIE imaging
extensions and player)

\item [-\ \ ] {\tt xpwm.*} (local {\tt TekXpress} window manager)

\end {description}

If in the future you decide to take advantage of some or all of these software
options, simply restore these files from the media using the INSTALL script.

\end {itemize}


\section {X Terminal Installation}

This section describes adding the first X terminal to your system. It contains
step-by-step procedures for performing a basic X terminal installation. By
following these instructions, you become familiar with the X terminal's network
configuration parameters, {\tt Boot Monitor, HostMenu} and {\tt Client Launcher}
utilities. In addition to learning about the X terminal, the procedures
presented in this section lead you through some of the basic host configuration
procedures for supporting X terminals.

These steps are performed during the basic X terminal installation:

\begin {itemize}

\item Collect information about your environment which is needed to boot the
X terminal.

\item Configure host files to support the X terminal.

\item Use the {\tt Boot Monitor} to enter X terminal communication parameters
and establish a host connection.

\item Log in through a Telnet session from {\tt HostMenu} or
{\tt Client launcher}

\end {itemize}

After the basic X terminal installation is complete, the X terminal can:

\begin {itemize}

\item Locate the host on the network.

\item Download the operating system, configuration, and font files.

\item Apply configuration files residing on the host.

\end {itemize}


\subsection {X Environment Parameters}

In a distributed computer environment, host computers perform a variety of
functions. The {\tt boot\_host} sends the {\tt boot\_file} to the terminal,
allowing it to function as an X display. The {\tt font\_host} contains the
files defining the appearance of the various character fonts that can be
displayed by the X terminal. The X terminal needs access to this host
frequently in the course of a user session. The {\tt login\_host} is the host
where the user's login account is found. The login host supplies a login window
to the X terminal, using some utility that provides login services. These
various host functions can be performed by a single host, or distributed across
multiple hosts.

Here is an example showing information about a sample environment:

{\bf Boot Host Name:}\ \ \ \ \ rlxxp9\ \ \ \ \ \ {\bf IP Address:} 130.235.35.37

{\bf X Terminal Model:}\ \ \ XP115M

{\bf X Terminal Name:}\ \ \ xtkd\ \ \ \ \ \ \ \ {\bf IP Address:} 130.235.35.38

{\bf Netmask:}\ \ \ \ \ \ \ \ \ \ \ \ \ \ \ \ \ 255.255.252.0

{\bf Gateway Address:}\ \ \ \ 130.235.33.31

{\bf Broadcast Address:}\ \ 130.235.33.255

{\bf Boot Method:}\ \ \ \ \ \ \ \ \ \ nfs

{\bf Font Host Name:}\ \ \ \ \ \ rlxxp9\ \ \ \ \ \ {\bf IP Address:} 130.235.35.37

{\bf File Access Method:}\ nfs


\subsection {Configuring Host Files}

\paragraph {Enabling NFS} By way of example, the primary method for booting and
file access is {\tt NFS}.

\begin {enumerate}

\item Set up the {\tt boot\_directory (/tekxp)} for export from the host. To
enable use of the host fonts, treat likewise the {\tt font\_directory (/usr)}.
These directories might be exported as {\tt read only}, but consideration
should be given to local security policy; it may not be the intention to
permit world read access to sensitive or proprietary software or other data!

Sun5: Edit the {\tt /etc/dfs/dfstab} file to export the directories

\ \ \ \ {\bf share -F nfs -o ro -d ``X Terminal files'' /tekxp}

\ \ \ \ {\bf share -F nfs -o ro -d ``X Terminal files'' /usr}

Digital Unix: Edit the {\tt /etc/exports} file to export the directories

\ \ \ \ {\bf /tekxp -ro}

\ \ \ \ {\bf /usr -ro}

\item Edit the remote configuration file {\tt /tekxp/boot/config/xp.cnf} to add
the host name and address and the local X terminal mount points and read sizes.
Search for the {\tt ip\_host\_table} and {\tt nfs\_table} commands. Be sure to
remove the comment character ({\bf \#}).

\ \ \ \ ip\_host\_table\ \ ``130.235.35.37'' \ ``rlxxp9''

\ \ \ \ nfs\_table\ \ \ \ \ \ rlxxp9:/tekxp\ \ \ /tekxp\ \ \ 8192

\ \ \ \ nfs\_table\ \ \ \ \ \ rlxxp9:/usr\ \ \ \ \ /usr\ \ \ \ \ 8192

\item Edit the {\tt /etc/hosts} file, or the equivalent Network Information
Service (NIS) file, to add the X terminal name and address.

\ \ \ \ 130.235.35.38\ \ \ xtkd

\item To enable the mount point immediately without booting:

Sun5: {\bf \# shareall}

\item To verify the export, type

Digital Unix: {\bf \# showmount -e}

\item Use the {\bf ps} command to see if {\bf nfsd} is running. For System
V-type OS (e.g. SunOS) use {\bf edf} for {\tt options}, for Berkeley-type OS
(e.g. DEC Unix), use {\bf aux} for {\tt options}.

{\bf \# ps} {\tt -options} {\bf $|$ grep nfsd $|$ sed /grep/d}

root 62 1\ \ 0\ \ Jan 26\ \ ?\ \ 0:01 nfsd

If {\bf nfsd} is not running, type:

{\bf \# /usr/etc/nfsd 8 \&}

or for Sun5: {\bf \# /usr/lib/nfs/nfsd 8 \&}

If {\bf nfsd} is running, use the {\bf kill} command with the PID to restart
the daemon (the PID is 62 in the preceding {\bf ps} example):

{\bf \# kill -1} {\tt PID}

\item Sun5: Use the {\bf ps} command to see if {\bf mountd} is running:

{\bf \# ps -edf $|$ grep mountd $|$ sed /grep/d}

root 62 1\ \ 0\ \ Jan 26\ \ ?\ \ 0:01 /usr/lib/nfs/mountd

If {\bf mountd} is not running, type:

{\bf \# /usr/lib/nfs/mountd \&}

\end {enumerate}


\subsection {Powering on the X Terminal}

Verify the physical installation of the X terminal's cables, power cords,
keyboard, mouse and network connection with the pictorial installation sheet
included in the packing box. Perform any electrical equipment testing as
specified at your site.

Turn on the X terminal's power switch. On the first power-up, you must specify
the keyboard you are using. By default, the North American 101/102 or VT200 is
selected (depending on the connected keyboard). Press {\tt Enter} or
{\tt Return} to accept this keyboard. If using a different keyboard or
nationality, press the {\tt Spacebar} to scroll through the list of available
keyboards and press {\tt Enter} or {\tt Return} to select the appropriate
keyboard.

After specifying the keyboard, press the {\tt Enter} or {\tt Return} key again
to display the {\bf BOOT$>$} prompt.

\subsection {Booting the X Terminal}

The {\bf BOOT$>$} prompt indicates that you are in the {\tt Boot Monitor}. The
{\tt Boot Monitor} is a simple command-line utility that provides an easy way
to input boot commands. These boot commands set the parameters which describe
the X terminal in your network environment.

The {\tt scoreboard} is an area in the upper right-hand corner of the boot
screen. At this time, the scoreboard shows only default values. Use the
scoreboard to verify the entries you make in the steps that follow.

To enter a boot command, type the command plus its associated parameter after
the {\bf BOOT$>$} prompt. To complete an entry, press {\tt Enter}. To see a list
of the {\tt Boot Monitor} commands, type {\bf help} and press {\tt Enter}. Check
the scoreboard to verify your entries. If there is an error, re-enter the
command.

\begin {enumerate}

\item Use the {\bf iaddr} command to enter the terminal IP address.

{\bf BOOT$>$} {\bf iaddr} {\tt ip\_address}

\item Use the {\bf bpath} command to enter the boot path.

{\bf BOOT$>$} {\bf bpath} {\tt /<boot\_directory>/boot/os.<model>}

\item Use the {\bf imask} command to enter the subnet mask.

{\bf BOOT$>$} {\bf imask} {\tt ip\_subnet\_mask}

\item Use the {\bf ihost} command to enter the boot host IP address.

{\bf BOOT$>$} {\bf ihost} {\tt ip\_address}

\item Use the {\bf igate} command to enter the IP address for a gateway host if
the X terminal is booting through a gateway.

{\bf BOOT$>$} {\bf igate} {\tt ip\_address}

\item Use the {\bf bmethod} command to specify NFS as the boot method. The 8192
parameter in the example at section 4.2 represents an NFS read size.

{\bf BOOT$>$} {\bf bmethod nfs 8192}

\item Use the {\bf nsave} command to save the values in non-volatile memory.

{\bf BOOT$>$} {\bf nsave}

\item Use the {\bf boot} command to initiate the boot process.

{\bf BOOT$>$} {\bf boot}

\end {enumerate}

If the X terminal locates the host and boot files, a bar appears on the boot
screen showing the percent of download complete. Once the terminal downloads
all of the files required, a grey screen with an X-shaped cursor appears. If
the boot process fails without error messages, you probably entered an
incorrect parameter. Enter the appropriate command and parameter to correct
the error. After correcting the error, execute the {\bf nsave} and {\bf boot}
commands.


\subsection {Starting a Session}

Once the X terminal is booted, the {\tt HostMenu} client appears by default.
You can either start a session from {\tt HostMenu} or from the
{\tt Client Launcher}.

{\tt HostMenu} is a local client that displays a list of host names and
network addresses. The buttons across the top select the login list to be
displayed. The default host connection is via the X Display Manager {\bf xdm}.
The X terminal broadcasts XDMCP requests and displays a list of hosts that
respond. Instructions for enabling {\bf xdm} are included in the {\tt TekXpress}
X terminal documentation.

Hosts can be accessed via other protocols, such as {\tt Telnet} and {\tt Cterm}.
To see the alternate lists, click on the appropriate button. For most lists, you
can enter your host name or address in the Host name field, then press
{\tt Enter}.


\subsection {Client launcher}

The {\tt Client Launcher} provides a menu of local clients and host connections
available to the X terminal. {\tt Client Launcher} is accessed on most keyboards
by pressing the {\tt Setup} key. For alternate keyboards, use these sequences:

101/102 ({\tt Pause} key):\ \ \ {\tt Shift-Setup}

UNIX:\ \ \ \ \ \ \ \ \ \ \ \ \ \ \ \ \ \ \ \ \ \ {\tt AltGraph-Setup}

Sun5:\ \ \ \ \ \ \ \ \ \ \ \ \ \ \ \ \ \ \ \ \ \ \ {\tt AltGraph-Help}

LK401:\ \ \ \ \ \ \ \ \ \ \ \ \ \ \ \ \ \ \ \ \ {\tt Compose-Backspace}

You can customise the Launcher client for other users through the
{\tt system.launcher} file. Refer to the {\tt TekXpress} X terminal
documentation.

\clearpage

\subsection {Opening a Telnet Session}

Telnet provides a direct connection from the X terminal to a host computer.
When Telnet is running, the X terminal acts as a standard VT102 terminal. To
open a Telnet session from {\tt HostMenu}:

\begin {enumerate}

\item Click on the {\bf TELNET} list button.

\item Click on your host, or enter your host name or address in the Host Name
field.

\end {enumerate}

To open a Telnet session from the {\tt Client Launcher}:

\begin {enumerate}

\item Position the pointer on the Host Connections option to display the
submenu. Select {\tt Telnet} to open a Telnet window.

\item Use the mouse to position the pointer in the Telnet window. You can type
{\bf H} and press {\tt Enter} to see a list of Telnet commands.

\item At the {\bf Telnet$>$} prompt, type:

{\bf Telnet$>$} {\bf open} {\tt hostname}

where {\tt hostname} is the name or network address of the host.

\end {enumerate}

Once a connection is made you can login to your host.

\begin {enumerate}

\item At the {\bf login:} prompt, login to the host computer by entering your
user name:

{\bf login:} {\tt user\_name}

\item At the {\bf password} prompt, enter your password:

{\bf password:} {\tt password}

\item Now that you are connected to the host, type the command plus its
associated parameter after the {\bf \#} prompt. To terminate an entry, press
{\tt Enter}. Set the display environment variable for the X terminal by entering
the following using the X terminal IP address:

{\bf \# setenv DISPLAY} {\tt ip\_address}

\item Enter the command to start your window manager.

If the Motif Window Manager is installed, and setup in your path, type:

{\bf \# mwm \&}

If the OPENLOOK Window Manager is installed, and setup in your path, type:

{\bf \# olwm \&}

If you do not have a host-based window manager, you can start a local window
manager via the {\tt Client launcher}. Local window managers can also be started
through the {\bf xpsh} command discussed in the online documentation.

\end {enumerate}


\subsection {Closing a Telnet Session}

When you are ready to conclude the Telnet session, logout of your host:

{\bf \# exit}

To close the Telnet window from the {\bf Telnet$>$} prompt:

{\bf Telnet$>$ quit}


\section {Configuring Your Environment}

After installing the software and starting a session, use the X terminal to
finish configuring your environment. The following is a list of recommended
tasks for the configuration process:

\begin {itemize}

\item Start {\tt Online Xpress}, if selected during the INSTALL phase, to view
the online documentation. (Instructions for starting {\tt Online Xpress} are
located in the {\tt XpressWare} Release Notes.) Otherwise refer to the ``paper
documentation'', which has been made available to the Starlink Site Managers.

\item From the main menu, select {\tt Configuring the X Terminal Environment}.
As you are working your way through the strategies section, you can edit the
necessary files by invoking your text editor from within your Telnet window.

\item If using {\tt Online Xpress}, then for additional ease of use, you can
also open the {\tt Accessing X Terminal Reference Information} from the main
menu. By using the included index and the Previous Jump button, you can quickly
look up command details while keeping your Strategy window open.

\item To save time, gather in advance the information about each X terminal
that you are going to install in your environment.

\item Physically connect each additional X terminal to the network. Once the
configuration is complete, one should simply be able to power-on the X terminal
and use the environment that has been defined.

\end {itemize}


\section {Further Help}

If you have any comments concerning this User Guide and information contained
therein, or require additional help with XpressWare installation, please
contact {\tt Kevin Duffey}:

c/o Space Science Department

\ \ \ \ \ Rutherford Appleton Laboratory

\ \ \ \ \ Chilton, Didcot, OXON, OX11 0QX

Tel: 01235 44 6362

Fax: 01235 44 5848

E-mail: kpd@star.rl.ac.uk

\clearpage

\appendix

\section {Sample INSTALL Session}

Given below is a summary of an INSTALL session for XpressWare Version 8.0.

On invoking the INSTALL procedure, you will be automatically advised that the
size of the entire installation is 285324 Kilobytes. The entire installation
caters for all applicable X terminal types, includes all available fonts, and
the online documentation and viewing software.

On starting-up, the procedure will ask if a full installation is required. For
a case where a full installation has {\bf not} been requested, the dialogue
might procede along the following lines ({\bf Note} that certain text has been
re-phrased or omitted here for brevity):

{\tt Size of entire installation is 285324 KB}

{\tt Full install (y/n)} n

{\tt Load hp files (y/n)} n

{\tt Load sun files (y/n)} y

{\tt Load dec files (y/n)} n

{\tt Load sgi files (y/n)} n

{\tt Load ibm files (y/n)} n

{\tt Size of the boot group is 2256 KB}

{\tt Install `boot' group (y/n)} y

{\tt Size of xpbinaries is 34848 KB (for XP10, XP100, XP330, XP350, pex)}

{\tt Install xpbinaries (y/n/p)} p

{\tt Size of xpbinaries.xp10 group is 10512 KB; install `xp10' (y/n)} n

{\tt Size of xpbinaries.xp330 group is 13712 KB; install `xp330' (y/n)} n

{\tt Size of xpbinaries.xp350 group is 16268 KB; install `xp350' (y/n)} y

{\tt Size of xpbinaries.pex group is 3552 KB; install `pex' (y/n)} n

{\tt Size of fonts group is 12880 KB}

{\tt Install `fonts' group (y/n/p)} p

{\tt Size of fonts.misc group is 2496 KB; install `misc' (y/n)} y

{\tt Size of fonts.100dpi group is 2396 KB; install `100dpi' (y/n)} y

{\tt Size of fonts.75dpi group is 2092 KB; install `75dpi' (y/n)} y

{\tt Size of fonts.japanese group is 2532 KB; install `japanese' (y/n)} n

{\tt Size of fonts.oldx11 group is 1056 KB; install `oldx11' (y/n)} y

{\tt Size of fonts.openlook group is 104 KB; install `openlook' (y/n)} y

{\tt Size of fonts.tek100dpi group is 564 KB; install `tek100dpi' (y/n)} y

{\tt Size of fonts.Speedo group is 568 KB; install `Speedo' (y/n)} n

{\tt Size of fonts.Type1 group is 1072 KB; install `Type1' (y/n)} n

{\tt Size of man group is 512 KB}

{\tt Install `man' group (y/n)} y

{\tt Size of mgmt group is 76 KB}

{\tt Install `mgmt' group (y/n)} y

{\tt Size of obsolete group is 468 KB}

{\tt Install `obsolete' group (y/n)} y

{\tt Size of examples group is 212 KB}

{\tt Install `examples' group (y/n)} y

{\tt Size of Xext group is 108 KB}

{\tt Install `Xext' group (y/n)} y

{\tt Size of bin group is 7016 KB}

{\tt Install `bin' group (y/n)} y

{\tt Size of src group is 1100 KB}

{\tt Install `src' group (y/n)} n

{\tt Size of audioIntercept group is 756 KB}

{\tt Install `audioIntercept' group (y/n)} n

{\tt Size of Acrobat group is 8448 KB}

{\tt Install `Acrobat' group (y/n)} n

{\tt Size of DpsNx group is 24412 KB}

{\tt Install `DpsNx' group (y/n)} n

{\tt Size of doc\_viewer group is 41296 KB}

{\tt Install `doc\_viewer' group (y/n)} n

{\tt Size of doc\_online group is 61120 KB}

{\tt Install `doc\_online' group (y/n)} n


{\tt Total size of install is 35624 KB}


{\tt Do you wish to do the installation (y/n)}


If you go ahead with an installation without the online documentation and
viewer, these can be brought in with a subsequent installation at a cost of an
additional 102416 KB

\end{document}
