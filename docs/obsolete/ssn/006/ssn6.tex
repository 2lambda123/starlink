\documentstyle{article}
\pagestyle{myheadings}

%------------------------------------------------------------------------------
\newcommand{\stardoccategory}  {Starlink System Note}
\newcommand{\stardocinitials}  {SSN}
\newcommand{\stardocnumber}    {6.1}
\newcommand{\stardocauthors}   {C A Clayton}
\newcommand{\stardocdate}      {9 October 1990}
\newcommand{\stardoctitle}     {The Cipher M990 Tape drive}
%------------------------------------------------------------------------------

\newcommand{\stardocname}{\stardocinitials /\stardocnumber}
\markright{\stardocname}
\setlength{\textwidth}{160mm}
\setlength{\textheight}{240mm}
\setlength{\topmargin}{-5mm}
\setlength{\oddsidemargin}{0mm}
\setlength{\evensidemargin}{0mm}
\setlength{\parindent}{0mm}
\setlength{\parskip}{\medskipamount}
\setlength{\unitlength}{1mm}

\begin{document}
\thispagestyle{empty}
SCIENCE \& ENGINEERING RESEARCH COUNCIL \hfill \stardocname\\
RUTHERFORD APPLETON LABORATORY\\
{\large\bf Starlink Project\\}
{\large\bf \stardoccategory\ \stardocnumber}
\begin{flushright}
\stardocauthors\\
\stardocdate
\end{flushright}
\vspace{-4mm}
\rule{\textwidth}{0.5mm}
\vspace{5mm}
\begin{center}
{\Large\bf \stardoctitle}
\end{center}
\vspace{5mm}

\section{Introduction}

Starlink uses a variety of 1600/6250 bpi tape drives at its various sites.
These include the SI 9621, the CDC Keystone 3, the Cipher M990 and one or two
other drives.
Starlink currently has Cipher M990 tape drives at Manchester, Leicester, ROE,
Cambridge, Birmingham, Armagh, Belfast, Oxford, UCL, Southampton, St. Andrews,
Sussex, Cardiff and QMW. The majority of these are controlled by Dilog DQ153
and DQ3153 controllers. Managers as
sites which do not have one of these controllers and who
are unhappy with the functionality of their present controller should contact
the author.

This document is aimed at Site Managers and describes how to make the most of
their Cipher M990s. Some of these devices have not been optimally set-up by
engineers and some simple tuning may improve performance dramatically.
Any corrections, improvements or suggestions for this document will be most
gratefully received by the author. This document does not attempt to explain
to users how to use the drive since this is almost impossible to do this
without complex diagrams and it is far more effective for a Site Manager to
actually demonstrate the procedures to new users.


\section{Loading and Unloading tapes}

The following tips should be passed on to new users when you show them
how to use the Cipher.

\subsection{Automatically loading a tape}

To load a tape you simply insert the tape into the drive and onto the hub,
close the door and press the load button and then the on-line button.  However,
it is worth noting that you can increase the chance of the drive successfully
loading your tape first time and more quickly by leaving about 30cm of tape
hanging out of the drive when you close the door. This is known  as the
Dragon's tongue method of loading a tape, since as the tape starts to  rotate
the ``tongue'' of tape flicks around as it is pulled into the  interior of the
drive. Tapes sometimes fail to load when the end of the tape adheres to the
main roll of tape and the tape drive is unable to  blow the end along the tape
path and wind it onto the take-up hub. The dragon's tongue method of loading
keeps the end of the tape free and allows the drive to blow it easily along the
tape path.

\subsection {Manually loading a tape}

In cases of dire emergency, it is also possible to load a tape manually but
this is generally not recommended since there is potential to damage both the
user and the drive. In addition, it is very time consuming. The first time the
author tried this, it took him over 10 minutes to succeed in loading a tape
manually. The difficult parts are getting the feet to lock onto the tape
reel properly and
ensuring that the correct amount of tape is wound onto the rear hub.

\begin{enumerate}

\item Switch off unit.

\item Extend front support leg of rack (if any) and slide drive out of rack.

\item Lift up top cover and switch on unit.

\item Place reel of tape on the front hub.

\item Depress and hold the white manual unlock button, located behind the
front-panel door on the bottom left-hand side of the tape reel opening,
while simultaneously rotating the front hub clockwise until the front hub is locked
in place.

\item Manually thread the tape leader around the tape guides. {\bf Take care}
when moving the spring-loaded tachometer (the arm with the rubber wheel which
rests against the rear hub) towards the rear hub. {\bf Do not} let the
tachometer strike sharply against the hub.

\item Move the tachometer away from the rear hub and thread the tape
leader around the hub in a clockwise direction. Gently rest the tachometer on
the rear hub.
{\bf Never} to try to operate the
drive with the tachometer arm  pulled away from the take up hub. The tachometer
measures the speed at which the tape is going past it and adjusts the rotation
speed of the hub to a constant number of inches per second past the read/write
head. If you take the tachometer arm away (because you have got some tape
wrapped around it perhaps) then the thing takes off at warp 9 and shreds your
tape. Very spectacular, but not recommended.

\item Manually rotate the rear hub about three turns.

\item Activate test 133 to override the cover interlocks (see Section 5).

\item Press the Load switch while pressing the Density select switch.

\item Release both switches, stand well back and wait for the tape to load.
There is some degree of danger at this stage since there
is nothing to protect you against the tape end as it whips around.


\end{enumerate}

\subsection{Unloading a tape}

To unload a tape, simply take the drive off-line and press the unload  button.
Occasionally, users leave the drive in a state where the three ``feet'' that hold
the tape onto the hub get stuck in the engaged position and it is not possible
to take the tape out. {\bf Do not} try using brute force to  remove the tape. You
will damage the feet and render the drive useless. The correct procedure to
release the tape is to open the door, press and hold in the white button below
and left of the tape and then slowly rotate the tape anti-clockwise. The feet will retract into
the body of the hub and you will be able to remove the tape without damaging
the drive. The last part of the rotation before the feet retract requires a
{\bf slight} amount of additional force.


\section{Configuring your Cipher M990 for efficient operation}

To optimise performance of the Cipher M990 unit, certain operating
parameters for blocksize, transfer rate and various other options can be
adjusted using diagnostic test 142 (Edit NOVRAM).

\subsection{Using test 142}

Take the drive off-line and unload tape (if any).

The front panel of the drive appears as follows

\begin{verbatim}
       load       unload      online      wrt test    density
      -------     -------     -------     -------     -------
         1           2           3           4           5
      -------     -------     -------     -------     -------
\end{verbatim}

Key in 451425 and the display will display the current value of the serial
port baud rate parameter (Parameter 1). The values of the configuration
parameters can be displayed in the same order as the following list:

\begin{center}\begin{tabular}{l@{\hspace{1cm}}l@{\hspace{1cm}}l}

Number 	& Description				& Recommended value \\

&&\\

1	& Serial Port Baud Rate                   & INACTIVE \\
2	& Host Supplied Parity                    & NO \\
3	& Echo Read Strobes on Write              & YES \\
4	& EOT Mode                                & NORMAL \\
5	& Forward Hitch on Reverse                & YES \\
6	& Echo 3200 bpi ID Burst                  & NO \\
7	& Reserved                                & INACTIVE \\
8	& Abort Active Writes on Overwrites       & NO \\
9	& Interface Transfer Rate (kHz)           & 90.4  \\
10	& Default Density on Power-up             & 6250 \\
11	& Maximum Block Size (kbytes)            & 9  \\
12	& Interface Ramp Delay                    & 3  \\
13	& File Mark Write Sync                    & SINGLE  FMK \\
14	& Read Error Retries                      & 3  \\
15	& Write Error Retries                     & 8  \\
16	& Error Correction on                     & YES \\
17	& Unit                                    & 0 \\
18	& Lock Out 3200 bpi Writes                & YES \\
19	& Remote Density Select enabled           & YES \\
20	& Report Corrected Errors                  & YES \\
21	& Allow One Track Down on Writes          & NO  \\
22	& 6250 Write Current                      & --- \\
	& (read as XX.XX  ma) & \\
23	& 1600 Write Current                      & --- \\
24	& 3200 Write Current                      & --- \\
25	& 6250 RAW Threshold                      & --- \\
	& (read as .XXX  mv) & \\
26	& 1600 RAW Threshold                      & --- \\
27	& 3200 RAW Threshold                      & --- \\
28	& 6250/3200 select BPI as 6250            & NO \\
29	& Display ft                              & YES \\
30	& Write error override                    & NO \\
31	& Load/online                            &  NO

\end{tabular}\end{center}

\begin{enumerate}

\item Press the Load switch to display the next parameter. If you press and hold the
Load switch, the display will increment through the parameter list.

\item Press the Unload switch to display the previous parameter. Press and hold the
Unload switch to decrement through the parameter list.

\item Press the on-line switch to re-display the current parameter.

\item To exit the test without changing any of the parameters, press the Wrt En/Test
switch.

\item If you want to change a parameter value, select the appropriate number, using
the Load and Unload switch, and enter the edit mode by pressing the Density
Select switch. Note that parameters 22 through 27 are not alterable by test 142
and should only be adjusted by trained engineers.

\item Press the Load switch to get a yes/no parameter value to yes, or  increment a
value parameter to its next higher value.

\item Press the Unload switch to set a yes/no parameter to no, or decrement a value
parameter to its next lower value.

\item To scroll the parameter name through the display, press the on-Line switch. The
parameter value will remain in the display.

\item Press the Density Select switch to return to the test execution mode.

\item To save the new parameter values and exit the test, press the Wrt En/Test and
Density Select switches together. These values will be stores in the NOVRAM
when the unit is powered down. Hence, to implement changes immediately you  must
power cycle the drive.

\item Please note that the recommended values may not apply to you machine. Make sure
that you record your old values before attempting to tune your device or else
you may end up worse off than you started.

\end{enumerate}

\subsection{Explanation of selected parameters}

{\bf Parameter 3. Echo Read Strobe on Writes} - echoes a read strobe on write
operations for systems that require an echo to know that data are being written.

{\bf Parameter 4. EOT mode} - causes the last record to be written at, or
before, the EOT marker. This function supports systems that do not accept
records written beyond the EOT marker. This option should only be
used where necessary, because performance degradation near EOT will occur.

{\bf Parameter 5. FWD Hitch enabled} - causes the drive to jog the tape
forward prior to any reverse operations in order to overcome static friction.

{\bf Parameter 9.  Interface Transfer Rate (ITR)} - sets the burst rate at
which data are transferred between the Cipher and the VAX. The
optimum setting is VAX-type dependent and
can be determined as follows:

\begin{enumerate}

\item Select a data transfer rate compatible with your system using test 142.
Start with 90.4 kHz.

\item Run an actual tape program to establish basic compatibility; e.g. measure
time to BACKUP 20,000 blocks on an unloaded system as a reference.
%Calculate the throughput using the following formula:

%Total number of bytes written /
%		1000 * Total seconds required to complete write = ITR (kHz)

\item Select the next highest transfer rate using test 142.

\item Repeat steps 2 and 3 until the data rate of the Cipher exceeds
the data rate capability of the VAX, as evidenced by a substantial increase
in re-positioning activity in the Cipher (caused by write retries
due to incomplete data transfers).

\item Select the next lowest transfer rate using test 142 to complete
the configuration.

\end{enumerate}

{\bf Parameter 11.  Maximum block size.} The accepted sizes are
9K, 16K, 32K and 64K. This is usually set to 9K by default.
Increasing it will improve performance. However, beware of setting it to 64K
since it will use the whole cache for even a 1K transfer. The optimal setting
can be found by starting with a low maximum block size and increasing it
step-by-step using test 142 until the performance degrades.


{\bf Parameter 12. Interface Ramp Delay} -  is a command
response time which, in a start/stop drive, would be the ramp time (delay while
the drive spins up to speed).  If you set the ramp time low then the CPU is not
waiting for the tape to get up speed and so it can get on with the transfer of
data. On a microVAX, the CPU is not able to keep up with the Cipher so it never
streams at 6250bpi.  With BACKUP on VMS 5.3, a Cipher 990 on a microVAX II will go
just about flat out at 1600bpi. It tends to get ahead of the processor about
every 5 to 10 seconds and have to do a re-position cycle (on write), but on a
verify pass it goes flat out all the time.

{\bf Parameter 14.  Read error retries.} The default is 3. One might  expect that this should be
set high to give yourself the maximum chance of recovering  data. However, this
parameter is the number of times the controller tries to read each time
VMS requests a retry (up to 15 times, I believe) so the default of 3 actually
results in 45 read retries, which should be sufficient.

{\bf Parameter 15.  Write error retries.}  This is set at 15 by default. The author believes that
it should be set lower (e.g. 8). If the drive requires excessive tries at
writing to  a tape, then the user should be warned since it may mean that the media
is bad which in turn may lead to problems reading the data back at a later
date.

{\bf Parameter 18. Lock out 3200-bpi Write} - prevents the selection of the
3200-bpi density. However, the drive can still {\bf read} a 3200-bpi tape.

{\bf Parameter 21.  Allow One track Down on Writes.} These drives are nine-track; eight for data
and one for parity. This parameter should always be set to NO. If a track goes down,
it is important that it be fixed rather than continue running with one
track down without anyone realising. Once the fact that the head is
defective has been discovered, this option could be enabled to allow
usage of the drive for {\bf read only} until an engineer arrives to fix it. This is
possible since the error correction system will allow recovery of the missing data track
by  using the information in the parity track.

{\bf Parameter 29. Display ft.} Displays number of feet to end of tape. This is very useful
even if just to tell users that  the tape is going around.
The display sometimes shows a negative number of feet remaining.
This is normal since the method used to judge the number of remaining feet is
very approximate. The relative rotation rate of the tachometer wheel compared
with that of the rear hub indicates the thickness,
and hence length of tape wrapped around the rear hub.

{\bf Parameter 31. Load/on-line.} Determines if the device automatically goes on-line when
loaded. I recommend leaving this disabled in order to be consistent with other
tape drives. Even with this disabled, one can press the on-line button
before the drive has finished loading the tape so there is no need to
wait by the drive while it load before you press the on-line button
(unlike some drives).


\section{Checking your Cipher's read/write performance}

If your users complain that your Cipher seems to be
reading/writing very slowly, then it may either be due to
very heavy use of system or the drive failing and retrying to read/write
many times (perhaps due to a defective head). Using test 255 will
determine which.

During a BACKUP operation, CTRL/Y and then take the Cipher off-line.
The front panel of the drive appears as follows

\begin{verbatim}
       load       unload      online      wrt test    density
      -------     -------     -------     -------     -------
         1           2           3           4           5
      -------     -------     -------     -------     -------
\end{verbatim}

Key in 452555 and the display will come up STATUS.

The buttons then function as follows:

\begin{itemize}

\item Button 1 \qquad	Shows Hard errors and Correctable errors

\item Button 2 \qquad	Shows Write Retries and Read retries

\item Button 3 \qquad	Shows Track error History

\item Button 4 \qquad	Gives the option to clear the above (yes/no = button 1/2)

\item Button 5 \qquad	Exits the test.

\end{itemize}

There should be no (or virtually no) errors. If you find numerous errors
on one or more tracks, contact your service engineer.

Press on-line and type

\begin{quote}

\$ CONTINUE

\end{quote}

in order to resume the BACKUP.

Note that if you dismount a tape, the error history is cleared, and
hence the need to CTRL/Y out of a BACKUP.

Also if you wish to monitor the streaming progress of a tape,
the code 452335 disables the door lock. DO NOT rewind a tape with
the door or lid open; it can be dangerous.

The Cipher tape drive heads come ready calibrated from the factory. You should
have a sticker on the side of the head (left side in the case of the Cambridge
drive) with the optimum read and write currents written on it as determined by
the manufacturer. Do not attempt changing the currents yourself.

The author once had trouble reading some recently written tapes.
An engineer informed him that they were written with the write current set
10\% too high and hence the tape was saturated. Our current was set to the
factory recommendation i.e. these recommendations are not always correct.
The optimum value can be determined to some extent by varying the current
and trying test 255. Only usage over several weeks
can tell you if you have the currents set correctly.

\section{Other useful tests}

The Cipher M990 has three separate types of built-in diagnostic tests:

\begin{itemize}

\item Power-up self-tests. If your drive fails these, call an engineer.

\item Series 100 tests - Diagnostic tests {\bf without} a tape loaded

\item Series 200 tests - Diagnostic tests {\bf with} a tape loaded

\end {itemize}

I have already described examples of both Series 100 and 200 tests
(tests 142 and 255) but there are a number of other tests that might prove useful
to you:

\begin{itemize}

\item Test 125 (PROM Revision) - displays the part number of the PROMs
installed in the unit.

\item Test 133 (Door open) - deactivates the front panel and top cover lock
solenoid so that the covers may be opened during the next load attempt.
Beware of the end of the tape which may harm you.

\item Test 212 (Read/Write Data) - writes test data to tape, without error
correction.

\item Test 233 (Door open) - disengages the front panel and top cover lock.

\item Test 242 (Edit NOVRAM) - Same as Test 142 except that the tape is loaded
on the drive during the test.

\end{itemize}

\subsection{Starting the Diagnostic Tests}

Check to see that the on-line indicator is off. If the indicator is on, press
the on-line
switch to switch the unit to an off-line mode.

Steps 2 through 4 must be accomplished within 3 seconds between keystrokes. If
you are too slow, the M990 will automatically return to the normal
operating mode.

\begin{enumerate}

\item Press the Wrt En/Test switch number 4

\item Press the Density Select switch number 5.

\item Press the switch numbers that represent the test to be run.

\item Press the Density Select switch number 5 to START the diagnostic test.

\item Press the Wrt En/Test switch number 4 to STOP the test from running.

\end{enumerate}

When the diagnostic test is first entered, specific front panel indicators
are lit to indicate the options available at this point. During the test, the
alphanumeric display will display information about the drive's status.
See the Cipher maintenance manual for detailed information about each test.

The manual recommends that you remove any tape prior to running series 100
tests, since they may result in damage to the tape. However, it is necessary
to use test 133 while trying to manually load a tape (see Section 2.2).


\section{Tape caching}

The area of tape caching is quite confusing, partly because caches may
exist on both the drive itself and the controller. Also, it is not
obvious under what conditions the cache is of use.

In a system where the I/O rate of the CPU/disk combination is greater than
that of the tape drive, the system will always be limited by the tape drive I/O
rate. This situation does not occur in Starlink systems.

In a system where the I/O rate of the CPU/disk combination is less than
that of the tape drive, the system will always be limited by the CPU/disk
combination I/O rate. This is the normal situation at Starlink sites
where we use relatively slow Q-buses and microVAXes (compared with big
VAXes, such as the VAX 6000).

On a lightly loaded microVAX, the realised tape I/O rate will be slightly below that
of the CPU/disk combination. As the load on the VAX increases (i.e. more
processes are making demands on the CPU), the realised tape I/O rate on a non-cached
system will fall. However, on a cached system, the cache is able to keep
data flowing to the drive during the short periods while the CPU is servicing
other processes and avoids the drive running out of data and re-positioning,
leading to a drop in performance.
Hence, the cache is of most use on a heavily loaded system.

It is important to note that re-positioning does not necessarily mean that
your drive is not running at full capacity. There is no loss of performance if
the drive re-positions while the cache is filling up with data from the CPU
since this indicates that the CPU has been unable to supply data fast
enough and the re-positioning is not preventing writing taking place. If
the cache is full while the drive is re-positioning, then the realised I/O
rate will fall and the system will be limited by the drive re-positioning
speed. Matching of the drive transfer rate with the data rate capability
of the host VAX can be achieved using the procedure described in Section 3.2.

The Cipher M990 has a built-in 64K cache. As I understand it, this is in use
all of the time. The DQ153/3153 controllers also contain a 64K cache. By
default, the controller cache is disabled. To enable write caching for the tape
controller you must mount your tape with the qualifier

\begin{quote}

\$ MOUNT MUxx: /CACHE=TAPE\_CACHE

\end{quote}

%The DQ153/3153 controllers emulate TU81s (MU devices) which support tape
%caching. The advantage of tape caching is that it should allow the drive to
%stream and cope with hiccups in the flow of data to the drive (likely on a
%multi-user machine).

However, timing tests have shown no performance improvement when mounting tapes
with the /CACHE qualifier and running BACKUP.

\section{Remote density selection}

Most Starlink Cipher M990 tape drives are controlled by either a DQ153
(microVAX II) or DQ3153 (microVAX 3xxx) controller. Both controllers are
exactly the same except for minor physical differences that reflect the
different cabinets in which they are installed. Your tape drive should be set up
for remote density select; i.e. if you initialize a tape with the /DENSITY=XXXX,
the controller will put the tape drive into this  density mode.

Manual (or local) density selection is done by a button
on the tape drive. Density selection, both manual and remote is
usually done only at BOT for write-type operations and is then stored in the
drive until changed. For read operations, density is automatically selected at
BOT by testing the ``ID burst'' on the tape.
In any case, remote density selection can be
done only if the tape drive is in the REMOTE DENSITY  select mode (see Section
3.1). On
DQ153/3153 controllers of revision level lower than G remote density
selection is not possible on a tri-density drive such as the M990. All Starlink
controllers should be at revision level G or later. If not, then they should be
upgraded free of charge under maintenance.

It is worth noting that with these controllers, the density of the drive as
shown with

\begin{quote}

\$ SHOW DEV/FULL MUxx:

\end{quote}

is correct. This is not the case for multi-density drives emulating a TS11
drive (MSxx:), which was a single density drive. In this case, the density shown
with the above command will always show 1600 bpi, regardless of what density
the drive is actually working at.  Drives on a controller emulating a TS11 can
only have the density set on the front panel.

\end{document}
