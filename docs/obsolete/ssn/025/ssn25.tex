\documentstyle[11pt]{article}
\pagestyle{myheadings}

% -----------------------------------------------------------------------------
% ? Document identification
\newcommand{\stardoccategory}  {Starlink System Note}
\newcommand{\stardocinitials}  {SSN}
\newcommand{\stardocsource}    {ssn\stardocnumber}
\newcommand{\stardocnumber}    {25.2}
\newcommand{\stardocauthors}   {V S Dhillon\\G J Privett}
\newcommand{\stardocdate}      {10th March 1997}
\newcommand{\stardoctitle}     {PERIOD --- Installation Notes}
% ? End of document identification
% -----------------------------------------------------------------------------

\newcommand{\stardocname}{\stardocinitials /\stardocnumber}
\markright{\stardocname}
\setlength{\textwidth}{160mm}
\setlength{\textheight}{230mm}
\setlength{\topmargin}{-2mm}
\setlength{\oddsidemargin}{0mm}
\setlength{\evensidemargin}{0mm}
\setlength{\parindent}{0mm}
\setlength{\parskip}{\medskipamount}
\setlength{\unitlength}{1mm}

% -----------------------------------------------------------------------------
%  Hypertext definitions.
%  ======================
%  These are used by the LaTeX2HTML translator in conjunction with star2html.

%  Comment.sty: version 2.0, 19 June 1992
%  Selectively in/exclude pieces of text.
%
%  Author
%    Victor Eijkhout                                      <eijkhout@cs.utk.edu>
%    Department of Computer Science
%    University Tennessee at Knoxville
%    104 Ayres Hall
%    Knoxville, TN 37996
%    USA

%  Do not remove the %\begin{rawtex} and %\end{rawtex} lines (used by 
%  star2html to signify raw TeX that latex2html cannot process).
%\begin{rawtex}
\makeatletter
\def\makeinnocent#1{\catcode`#1=12 }
\def\csarg#1#2{\expandafter#1\csname#2\endcsname}

\def\ThrowAwayComment#1{\begingroup
    \def\CurrentComment{#1}%
    \let\do\makeinnocent \dospecials
    \makeinnocent\^^L% and whatever other special cases
    \endlinechar`\^^M \catcode`\^^M=12 \xComment}
{\catcode`\^^M=12 \endlinechar=-1 %
 \gdef\xComment#1^^M{\def\test{#1}
      \csarg\ifx{PlainEnd\CurrentComment Test}\test
          \let\html@next\endgroup
      \else \csarg\ifx{LaLaEnd\CurrentComment Test}\test
            \edef\html@next{\endgroup\noexpand\end{\CurrentComment}}
      \else \let\html@next\xComment
      \fi \fi \html@next}
}
\makeatother

\def\includecomment
 #1{\expandafter\def\csname#1\endcsname{}%
    \expandafter\def\csname end#1\endcsname{}}
\def\excludecomment
 #1{\expandafter\def\csname#1\endcsname{\ThrowAwayComment{#1}}%
    {\escapechar=-1\relax
     \csarg\xdef{PlainEnd#1Test}{\string\\end#1}%
     \csarg\xdef{LaLaEnd#1Test}{\string\\end\string\{#1\string\}}%
    }}

%  Define environments that ignore their contents.
\excludecomment{comment}
\excludecomment{rawhtml}
\excludecomment{htmlonly}
%\end{rawtex}

%  Hypertext commands etc. This is a condensed version of the html.sty
%  file supplied with LaTeX2HTML by: Nikos Drakos <nikos@cbl.leeds.ac.uk> &
%  Jelle van Zeijl <jvzeijl@isou17.estec.esa.nl>. The LaTeX2HTML documentation
%  should be consulted about all commands (and the environments defined above)
%  except \xref and \xlabel which are Starlink specific.

\newcommand{\htmladdnormallinkfoot}[2]{#1\footnote{#2}}
\newcommand{\htmladdnormallink}[2]{#1}
\newcommand{\htmladdimg}[1]{}
\newenvironment{latexonly}{}{}
\newcommand{\hyperref}[4]{#2\ref{#4}#3}
\newcommand{\htmlref}[2]{#1}
\newcommand{\htmlimage}[1]{}
\newcommand{\htmladdtonavigation}[1]{}

%  Starlink cross-references and labels.
\newcommand{\xref}[3]{#1}
\newcommand{\xlabel}[1]{}

%  LaTeX2HTML symbol.
\newcommand{\latextohtml}{{\bf LaTeX}{2}{\tt{HTML}}}

%  Define command to re-centre underscore for Latex and leave as normal
%  for HTML (severe problems with \_ in tabbing environments and \_\_
%  generally otherwise).
\newcommand{\latex}[1]{#1}
\newcommand{\setunderscore}{\renewcommand{\_}{{\tt\symbol{95}}}}
\latex{\setunderscore}

%  Redefine the \tableofcontents command. This procrastination is necessary 
%  to stop the automatic creation of a second table of contents page
%  by latex2html.
\newcommand{\latexonlytoc}[0]{\tableofcontents}

% -----------------------------------------------------------------------------
%  Debugging.
%  =========
%  Remove % on  the following to debug links in the HTML version using Latex.

% \newcommand{\hotlink}[2]{\fbox{\begin{tabular}[t]{@{}c@{}}#1\\\hline{\footnotesize #2}\end{tabular}}}
% \renewcommand{\htmladdnormallinkfoot}[2]{\hotlink{#1}{#2}}
% \renewcommand{\htmladdnormallink}[2]{\hotlink{#1}{#2}}
% \renewcommand{\hyperref}[4]{\hotlink{#1}{\S\ref{#4}}}
% \renewcommand{\htmlref}[2]{\hotlink{#1}{\S\ref{#2}}}
% \renewcommand{\xref}[3]{\hotlink{#1}{#2 -- #3}}
% -----------------------------------------------------------------------------
% ? Document specific \newcommand or \newenvironment commands.
% Environment for indenting and using a small font.
\newenvironment{myquote}{\begin{quote}\begin{small}}{\end{small}\end{quote}}
% in-line verbatims
\newcommand{\myverb}[1]{{\small \verb+#1+}}
% ? End of document specific commands
% -----------------------------------------------------------------------------
%  Title Page.
%  ===========
\renewcommand{\thepage}{\roman{page}}
\begin{document}
\thispagestyle{empty}

%  Latex document header.
%  ======================
\begin{latexonly}
   CCLRC / {\sc Rutherford Appleton Laboratory} \hfill {\bf \stardocname}\\
   {\large Particle Physics \& Astronomy Research Council}\\
   {\large Starlink Project\\}
   {\large \stardoccategory\ \stardocnumber}
   \begin{flushright}
   \stardocauthors\\
   \stardocdate
   \end{flushright}
   \vspace{-4mm}
   \rule{\textwidth}{0.5mm}
   \vspace{5mm}
   \begin{center}
   {\Large\bf \stardoctitle}
   \end{center}
   \vspace{5mm}

% ? Heading for abstract if used.
%  \vspace{10mm}
%  \begin{center}
%     {\Large\bf Abstract}
%  \end{center}
% ? End of heading for abstract.
\end{latexonly}

%  HTML documentation header.
%  ==========================
\begin{htmlonly}
   \xlabel{}
   \begin{rawhtml} <H1> \end{rawhtml}
      \stardoctitle
   \begin{rawhtml} </H1> \end{rawhtml}

% ? Add picture here if required.
% ? End of picture

   \begin{rawhtml} <P> <I> \end{rawhtml}
   \stardoccategory \stardocnumber \\
   \stardocauthors \\
   \stardocdate
   \begin{rawhtml} </I> </P> <H3> \end{rawhtml}
      \htmladdnormallink{CCLRC}{http://www.cclrc.ac.uk} /
      \htmladdnormallink{Rutherford Appleton Laboratory}
                        {http://www.cclrc.ac.uk/ral} \\
      \htmladdnormallink{Particle Physics \& Astronomy Research Council}
                        {http://www.pparc.ac.uk} \\
   \begin{rawhtml} </H3> <H2> \end{rawhtml}
      \htmladdnormallink{Starlink Project}{http://www.starlink.ac.uk/}
   \begin{rawhtml} </H2> \end{rawhtml}
   \htmladdnormallink{\htmladdimg{source.gif} Retrieve hardcopy}
      {http://www.starlink.ac.uk/cgi-bin/hcserver?\stardocsource}\\

%  HTML document table of contents. 
%  ================================
%  Add table of contents header and a navigation button to return to this 
%  point in the document (this should always go before the abstract \section). 
  \label{stardoccontents}
  \begin{rawhtml} 
    <HR>
    <H2>Contents</H2>
  \end{rawhtml}
  \renewcommand{\latexonlytoc}[0]{}
  \htmladdtonavigation{\htmlref{\htmladdimg{contents_motif.gif}}
        {stardoccontents}}

% ? New section for abstract if used.
% \section{\xlabel{abstract}Abstract}
% ? End of new section for abstract

\end{htmlonly}

% -----------------------------------------------------------------------------
% ? Document Abstract. (if used)
%  ==================
% ? End of document abstract
% -----------------------------------------------------------------------------
% ? Latex document Table of Contents (if used).
%  ===========================================
% \newpage
 \begin{latexonly}
   \setlength{\parskip}{0mm}
   \latexonlytoc
   \setlength{\parskip}{\medskipamount}
   \markright{\stardocname}
 \end{latexonly}
% ? End of Latex document table of contents
% -----------------------------------------------------------------------------
\newpage
\renewcommand{\thepage}{\arabic{page}}
% \setcounter{page}{1}

\section{\xlabel{introduction}Introduction}
 
This document describes how to install {\tt PERIOD} version 4.2, a
software package designed to search for periodicities in data.  A
detailed description of how to use {\tt PERIOD} can be found in
\xref{SUN/167}{sun167}{} and the program itself contains extensive
on-line help. 

This release of {\tt PERIOD} makes use of {\tt FITSIO} routines
to read {\tt OGIP FITS} files so that must be installed if {\tt
PERIOD} is to be built.  {\tt FITSIO} is normally installed as part of
the Starlink Software Connection, which is also required to build the
package.

Version 4.2 represents the first Starlink port of this package to
to encompass LINUX. It is also  the first version of {\tt PERIOD} to use 
PGPLOT routines rather than the XANADU {\tt PLT} routine.  

\section{\xlabel{installation}Installation}

\subsection{Miscellaneous}

The {\tt PERIOD} package consists of a number of FORTRAN subroutines
which are compiled and linked to form the executable {\tt period\_main}
on UNIX based systems.  The software provides largely the same functionality as
the VMS based version 4.0.

For the source only distribution, the compressed tar file supplied
contains the source archive, the makefile, its driver {\tt mk} script,
a news file and the three documents about {\tt PERIOD}.  These are:
\xref{SUN/167}{sun167}{} --- the User's Guide, SSN/25 --- the
Installation Notes (this document), and the program history file.  In
addition, there are two
archives containing hypertext versions of the User's Guide and
Installation Notes.

For the ready-to-run distributions, the compressed tar file supplied
contains in addition to the source files listed above, the complete
{\tt PERIOD} system built for an appropriate system.

\subsection{Building PERIOD from source}

To build {\tt PERIOD} from source, you need the source distribution, the
PGPLOT system, and the Starlink infrastructure software.

{\tt PERIOD} is supplied as a compressed tar file: {\tt period.tar.Z}.

To build {\tt PERIOD} the contents of the compressed tar file must be
extracted to a suitable directory.  On Starlink systems, this
would be {\tt /star/period}.

\begin{myquote}
\begin{verbatim}
% mkdir /star/period ; cd /star/period
% zcat /tmp/period.tar.Z | tar xvf -
\end{verbatim}
\end{myquote}
 
You must then set an environment variable to identify the type of
operating system you are using.  The example given below shows the
value of SYSTEM needed to create the Solaris 2 version of the software.
The values required for other Starlink supported systems may be found
in the files {\tt mk} and {\tt makefile}.

\begin{myquote} 
\begin{verbatim} 
% setenv SYSTEM sun4_Solaris
\end{verbatim} 
\end{myquote}

The UNIX `make' facility may then be employed to generate {\tt PERIOD}
in the current directory, and clean up after itself.  The `make' facility
should be driven the the {\tt mk} script provided which sets certain 
system dependent make macros before invoking the makefile:

\begin{myquote} 
\begin{verbatim} 
% ./mk build 
% ./mk clean 
\end{verbatim} 
\end{myquote}

\subsection{Ready-to-run distribution}

The ready-to-run distribution is supplied in a compressed tar archive
with a name which represents the system for which is is built.

If you have a ready-to-run distribution, you should unpack the
compressed tar file to the source directory, as detailed above for
building.  However, you will not need to build {\tt PERIOD} or clean up
the distribution.

You will also need the Starlink HTX utility (see
\xref{SUN/188}{sun188}{}) to allow the installation to procedure to
manipulate the hypertext versions of the documentation.

The ready-to-run distribution only requires installation, as follows.

\subsection{Installing PERIOD}

The makefile will install the binary and help files in the standard
locations on Starlink systems:  the binary in {\tt INSTALL\_BIN}, the 
documentation in {\tt INSTALL\_DOCS} and the help files in 
{\tt INSTALL\_HELP}.  

For {\tt PERIOD}, {\tt INSTALL\_BIN} is {\tt \$INSTALL/bin/period},
{\tt INSTALL\_DOCS} is {\tt \$INSTALL/docs}, and
{\tt INSTALL\_HELP} is {\tt \$INSTALL/help/period}.  These
directories may be overridden by specifying root directory using the
{\tt INSTALL} environment variable.

You do not need to create the installation directories, the makefile will
do that for you, provided you have write access to the directory defined
as {\tt INSTALL}.

On a Starlink system, {\tt INSTALL} is set to {\tt /star}, thus:

\begin{myquote}
\begin{verbatim}
% setenv INSTALL /star
\end{verbatim}
\end{myquote}

To install {\tt PERIOD}, invoke `make' thus:

\begin{myquote}
\begin{verbatim}
% ./mk install
\end{verbatim}
\end{myquote}

Once the software is installed, you should then define an alias for 
{\tt PERIOD} in your {\tt .cshrc} file:

\begin{myquote}
\begin{verbatim}
alias period /star/bin/period/period_main
\end{verbatim}
\end{myquote}

In addition, define the environment variable {\tt PERIOD\_HELP} in your
{\tt .login} file to point to the help files directory:

\begin{myquote}
\begin{verbatim}
setenv PERIOD_HELP /your-INSTALL-directory/help/period
\end{verbatim}
\end{myquote}

Once this has been done, any new user will be able to start up {\tt PERIOD}
by simply typing {\tt period} at the command line.

If you are installing {\tt PERIOD} on a Starlink system, in the default
location, the standard Starlink {\tt login} and {\tt cshrc} files will
detect its presence and define both the alias and environment variable
automatically on login.

\section{\xlabel{version4.2}Version 4.2}
 
Version 4.2 is the second release of {\tt PERIOD} under the UNIX operating
system but the first to encompass the LINUX operating system.
 
The XANADU {\tt PLT} plotting system (not available for LINUX) employed by 
previous versions of {\tt PERIOD} has been replaced by the more widely 
available PGPLOT system. This has involved the loss of some of {\tt PLT}
interactive plotting functions. However, these were rarely employed by users
and it is felt that the small loss of functionality is greatly outweighed by
the ability to run the software on a much larger number of machines.
 
\section{\xlabel{version4.1}Version 4.1}

Version 4.1 is the first release of {\tt PERIOD} under the UNIX operating
system.  The functionality of the system is essentially the same as
that of version 4.0 but with the ability to read some types of 
{\tt OGIP FITS} files.  One minor difference is that the directory
structure restrictions present within the VMS version have been
removed.

\section{\xlabel{version4.0}Version 4.0}

Version 4.0 of {\tt PERIOD} is the fourth general release of the
package, and the second release on Starlink. 

The major change over version 3.0 is a number of bug fixes (most
importantly to {\tt PERIOD\_CLEAN}), the removal of the {\tt CHAOS} and
{\tt MEM} options, and the inclusion of a new option ({\tt SIG}) which
calculates reliable false alarm probabilities for all periodograms.

\section{\xlabel{Bugs}Contacts and Bug Reports}

Please report any bugs, problems or suggestions for improvements to the
author (who may be contacted by E-mail; SPAN: 29146::VSD, Internet: {\tt
vsd@lpve.ing.iac.es}), who will be be very happy to receive them.

Bug reports from Starlink systems should be sent to the Starlink
Software Librarian, email {\tt ussc@star.rl.ac.uk}

Note that any changes made to {\tt PERIOD} are documented in the
history file {\tt period\_history.tex} (located in the source
directory), which should be consulted for further information on
version changes, bug fixes, {\em etc}.

\end{document}
