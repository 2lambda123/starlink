\documentstyle{article}
\pagestyle{myheadings}

%------------------------------------------------------------------------------
\newcommand{\stardoccategory}  {Starlink System Note}
\newcommand{\stardocinitials}  {SSN}
\newcommand{\stardocnumber}    {7.2}
\newcommand{\stardocauthors}   {C A Clayton}
\newcommand{\stardocdate}      {5 March 1991}
\newcommand{\stardoctitle}     {Administering Starlink Sun systems}
%------------------------------------------------------------------------------

\newcommand{\stardocname}{\stardocinitials /\stardocnumber}
\renewcommand{\_}{{\tt\char'137}}     % re-centres the underscore
\markright{\stardocname}
\setlength{\textwidth}{160mm}
\setlength{\textheight}{240mm}
\setlength{\topmargin}{-5mm}
\setlength{\oddsidemargin}{0mm}
\setlength{\evensidemargin}{0mm}
\setlength{\parindent}{0mm}
\setlength{\parskip}{\medskipamount}
\setlength{\unitlength}{1mm}

%------------------------------------------------------------------------------
% Add any \newcommand or \newenvironment commands here
%------------------------------------------------------------------------------

\begin{document}
\thispagestyle{empty}
SCIENCE \& ENGINEERING RESEARCH COUNCIL \hfill \stardocname\\
RUTHERFORD APPLETON LABORATORY\\
{\large\bf Starlink Project\\}
{\large\bf \stardoccategory\ \stardocnumber}
\begin{flushright}
\stardocauthors\\
\stardocdate
\end{flushright}
\vspace{-4mm}
\rule{\textwidth}{0.5mm}
\vspace{5mm}
\begin{center}
{\Large\bf \stardoctitle}
\end{center}
\vspace{5mm}

\setlength{\parskip}{0mm}
\tableofcontents
\setlength{\parskip}{\medskipamount}
\markright{\stardocname}

\newpage

\section {Introduction}

The purpose of this document is to ensure that all of the Starlink Sun systems
are set up in a secure, logical and consistent fashion and
that they do not begin to diverge from one another. One of the strengths of
Starlink is that 90\% of a site manager's technical problems can be
solved either simply by looking at how another Starlink system is set up or by
asking other site managers. If all systems are set up nearly identically,
then all managers will encounter the same problems but only one manager need
actually solve each problem. Solutions can be posted in the MAVAD::UNIX\_MANAGEMENT
and RLVAD::SUN\_WORKSTATIONS VAXnotes conferences. The former conference
has restricted access and hence sensitive matters such as security
should be discussed there.

It is assumed that the reader has a basic understanding of Sun system
management and no attempt is made either to clarify Sun \&
UNIX\footnote{UNIX is a registered trademark of AT\&T Bell Laboratories in the
United States and other countries} terminology
or to explain system management functions which are clearly described in
the Sun documentation. The topics covered are those that apply directly to
the Starlink systems and the particular way in which we need to install
and use them; these topics are not covered in the Sun manuals.

It is intended that this document will grow as we gain more knowledge
of Sun systems and eventually split up into a number of smaller documents.
Any corrections, comments or suggestions will be gratefully received by the
author. This document is based on SunOS 4.1.

%This guide should be considered as a supplement to SGP/25 and SGP/37.

\section{Initial installation}

\subsection{Installing SunOS}

If you have a single Sun, you should configure it as a standalone
system. Turning it into a server at a later stage is simple.
If you have two or more Suns, you should set one up as a homogeneous server
and the others as dataless clients. You should also specify the
server to be an NIS master and the other machines to be NIS clients.
If you have spare local disk space (100Mb or so), you could keep
{\tt /usr} locally and this would improve performance since you would
not have to load these programs across the ethernet. However, this
type of system, while more efficient, would require considerably more
maintenance.

Systems purchased with an internal disk currently come with the latest
version of SunOS installed on that disk. However, managers should still
go through the customized installation rather than the quick installation.
Quick Install only offers a choice of standard installations whereas
the custom install allows you to customize each phase of the installation --
from setting up the file systems to selecting the software categories to
load. Furthermore, Quick Install cannot be used if you wish to set up a
server, although upgrading to a server can be done manually.

\subsubsection{Disk partitions}

One of the difficulties of configuring a new UNIX system is getting the
filestore layout correct. If the layout is wrong then there may be little room
for expansion, or the system may run slowly. Once the disk has been partitioned
its geometry is fixed and it is time consuming to change.

Colour workstations should have a swap partition size of {\bf at least}
24Mb of swap
space, which should be at least 2Mb larger than the physical memory.
A swap space of twice the physical memory should be used for systems
with greater than 12Mb of RAM. Applications that process large arrays
require enormous amounts of swap space so you should be generous
when allocating disk space for this purpose.
If your
swap space is less than your physical memory, the extra memory will not
be used (unlike under VMS) and hence it is vital that you correctly size
your swap space both during installation and after memory upgrades.

You should {\bf never} use the first cylinder on a disk in a swap partition
(i.e. cylinder 0). The disk labelling information is held in the first sector
or so. Regular file systems are smart enough not to use those sectors; swap
files scribble everywhere.

Be generous when allocating space to {\tt /usr}. Running out of space in {\tt
/usr} later is a major inconvenience. Furthermore, you will need some free
space in {\tt /usr} to allow you to build customized kernels (see Section 2.4). As a
rule of thumb, add 10Mb to {\tt /usr} after allocating space for software options.

A partition called {\tt /home} should be created on one of the disks for user login
directories. If you have multiple disks, you should keep {\tt /home} on a different
disk to {\tt /} and {\tt /usr}. If all partitions are on a single disk, it is customary
to have {\tt /}, {\tt /usr} \& {\tt /home} in partitions a, g \& h  respectively.

After you have finished installing your machine, take a hardcopy of the output
from the command {\tt dkinfo} for each disk and keep them in a safe place.
{\tt dkinfo} gives information about a disk's partitioning. If you have a head crash,
you will need this in order to re-create the original disk label that will
be needed to allow you to recover lost information from dumps. Failure
to re-create the label correctly can (and already has within Starlink) lead
to data corruption.

\subsubsection{Optional software}

During the SUNinstall procedure, you are asked what optional software you
wish to install. The options listed in Table 1 should be installed.

\begin{table}[htb]
\begin{center}
\begin{tabular}{||l|l|l||}
\hline
Category 		& Install? & Description\\
\hline
root			& Yes & Contents of {\tt /}, including SunOS kernel\\
usr			& Yes & Required portions of {\tt /usr}\\
Kvm			& Yes & Kernel-architecture-dependant programs\\
Install			& Yes & Installation software\\
Networking		& Yes & Basic networking software\\
Debugging		& Yes & Debugging tools\\
RFS                     & No  & Remote File System, NFS alternative\\
Sys                     & Yes & Software for building kernels\\
System\_V                & Yes & System-V files\\
TLI                     & No & Communications protocol used by RFS\\
SunView\_Users           & Yes & Sun Proprietary Windowing system\\
Demo                    & No & Demonstration programs\\
Games                   & No & Computer games\\
Graphics                & Yes & SunView graphics\\
Manual                  & Yes & Manual pages\\
Security                & No &  C2-level security\\
Shlib\_Custom            & Yes & Software to build shared libraries\\
SunView\_Demo            & Yes & SunView demonstration software\\
SunView\_Programmers     & Yes & SunView programmers' software\\
Text                    & Yes & Text processing software\\
User\_Diag               & Yes & Diagnostic software\\
uucp    		& No & Files to support uucp\\
Versatec                & No & Spooling support for Versatec\\
\hline
\end {tabular}
\caption{Software to be selected with SUNinstall}
\end{center}
\end {table}

The manual pages take up 8Mb and managers should consider only installing
them on one machine (usually the server) at each site.

Starlink strongly encourages the use of X-Windows rather than SunView for programming
since SunView applications are non-portable. However, it is necessary to
install all the SunView programming options during SUNinstall so that
non-Starlink astronomical applications that use SunView can be built on your
system.

Managers are reminded that computer games are not allowed on Starlink.


\subsection {Upgrading SunOS}

It is vital that all sites run the same version of SunOS. Software prepared
at RAL for release on one version may not function on older versions.
Sites should aim to upgrade to new versions of SunOS as soon as
possible after RAL.

Some SunOS upgrades are much more painful than VMS ones. One is often required
to  effectively install a new system each time and hence it is vital that  you
keep a copy of files that have been modified locally and put these back on the
system after the upgrade. Such files include {\tt /etc/hosts}, {\tt
/etc/passwd} and {\tt /etc/rc.local}. The SunOS documentation on installing
SunOS gives a list of files that should be saved.

It is clear that one must be careful where one puts local files. If
possible, keep them out of the {\tt /} and {\tt /usr} partitions. Do not liberally scatter
your files around these partitions. {\tt /} and {\tt /etc} may be equivalent to
SYS\$MANAGER in VMS but you should aim to keep all of your local system
management utilities and files elsewhere. If you must put them in {\tt /usr}, put
them in {\tt /usr/local} and save this directory tree at upgrade time.

Modifications to the startup sequence should go in {\tt /etc/rc.local}. Such
modifications might include starting the screenblank process and the etherd
daemon.


\subsection{Unbundled software}

Unbundled software is SUNspeak for layered products.
Some unbundled software packages by default will spray files all over
{\tt /usr}.
However, many of these packages do give you the option to put the files
elsewhere. Try to do this if possible, even though this may require you to
modify your library, command, manual page and other path names and will make
the initial installation take slightly longer. Unbundled software installed in
{\tt /usr} will have to be reinstalled each time the operating system is upgraded and
hence installing it elsewhere may save effort in the long run.
Some pieces of software may cause problems if not
installed in the default location. In the case of languages, it is trivial to
re-install them after an upgrade and so managers should install these products
in the default directories.

The following piece of software has been bought for every Starlink
Sun and should be installed in the following directories:

\begin{itemize}

\item	Sun Fortran compiler ({\tt /usr/lang})

\end{itemize}

The following pieces of software have been obtained for some Suns and should be
installed in the following directories:

\begin{itemize}

\item	GNU C compiler ({\tt /usr/local})
\item	SUNlink DNI (DECnet end-node emulating software) ({\tt /usr/sunlink/dni})
\item	SUNlink TE100  (VT100 emulator for SunView) ({\tt /usr/sunlink/te100})
\item	XTech   X-Windows server ({\tt /usr/local/XTech} or {\tt /home/XTech})
\item	XTech   X-Windows Motif toolkit ({\tt /usr/local/XTech} or {\tt /home/XTech})

\end{itemize}

The following piece of software has been bought for all sites with Suns and
should be installed on your VAXcluster.

\begin{itemize}

\item   VMS/ULTRIX connection software

\end{itemize}

The organization of the Starlink Software on UNIX systems  and the distribution
mechanism for this software is discussed  in detail in SSN/66.

\subsection{Kernel Configuration}

After you install your system, you should trim your kernel in order to increase
the amount of free physical memory available. The procedure for doing this is
described in the Sun System \& Network Administration manual. However, be sure
to include the following lines in your kernel description to ensure that you
have support for the {\tt traffic} ethernet monitor (requires the etherd daemon
to be running), accounting, disk quotas and the CD-ROM
reader. The SCSI target number may be different at your site and should
be checked. It can be found on the back of the CD-ROM reader.

\begin{verbatim}

options QUOTA                   # disk quotas
options SYSACCT                 # process accounting
options HSFS                    # High Sierra (ISO 9660) CD-ROM file system

pseudo-device snit              # Lines required for etherd
pseudo-device pf
pseudo-device nbuf
pseudo-device clone

disk sr0 at scsibus0 target 6 lun 0     # CD-ROM device

\end{verbatim}

\subsection{Networking}

\subsubsection{VMS/ULTRIX Connection software (UCX)}

All communications between VAXes and Suns should be done using the VMS/ULTRIX
connection software (UCX) installed on the VAXcluster. This software allows the
VAXes to use TCP/IP communications (Transmission Control Protocol / Internet
Protocol), the native UNIX networking system.  In particular, the software
offers FTP (file transfer), Telnet (remote login)  and NFS (Network File
System). NFS allows UNIX workstations to access data that reside on VAX disks
in a transparent manner, as well as data on other Suns. The alternative would
be to use DECnet for communicating between the two types of system. Sun
produce an unbundled product called SUNlink DNI that allows a Sun to emulate a
DECnet end-node. However, the DECnet emulation is simplistic and likely to
cause problems when we migrate to DECnet Phase V and hence it is recommended
that you persuade users to use TCP/IP for all Sun/VAX communications.
DNI is only installed at Cambridge and Durham.

Each machine using TCP/IP requires an IP address (equivalent to a DECnet node
number). IP addresses should be obtained from your local computing centre
(or the equivalent), even if your Suns are isolated from the rest of your
campus. At some point in the future, you are likely to be connected to
a larger network and
selecting unique numbers at this stage will avoid problems in the future.


\subsubsection{Access to Suns}

Sun systems do not support LAT so it is not possible to log directly on to
a Sun from a terminal on a LAT-based terminal server such as a DECserver.
There are a number of solutions to this problem. Different solutions
have been adopted at different sites.

\begin{itemize}

\item Log into a VAX which has UCX loaded and start a Telnet session
from the VAX to the Sun. This is clearly an inefficient use of a VAX
login slot but
will work at all sites.

\item VT1000s support TCP/IP so it is possible to directly log into a Sun
with a Telnet session.

\item Emulex terminal servers can be upgraded so that they support both LAT
and TCP/IP.

\item It is possible to buy terminal servers that not only provide dual
protocol support (LAT \& TCP/IP) but which can also act as transparent
protocol converter and allow you to log into a Sun ``directly'' from
a LAT terminal server.

\end{itemize}

\subsubsection{Network File System (NFS)}

The way you set up your NFS will depend entirely on your configuration.
However, here are some guidelines.

\begin{enumerate}

\item Manual pages should be installed on only one machine and served to the
others.

\item {\tt /home} and partitions with applications software should be served to
all other machines. These should be mounted with the following options:

\begin{itemize}

\item bg -- if first mount attempt fails, it will retry in the background.

\item hard -- it will continue to request until the server responds.
This is necessary on a read/write file system but has the unfortunate
side effect that if you try to access a NFS disk that is not currently
available, your process will hang. This is a feature of SunOS
rather than a bug, but a required feature. Hard is the default.
Mounting something soft,ro is sensible for non-vital file structures
such as manual pages.

\item intr -- allows keyboard interrupts on hard mounts.

\item rw -- read/write (default).

\item noquota -- during the login sequence, the system runs the {\tt quota}
program for all mounted file systems. Remotely mounted file systems which are
mounted with the {\tt noquota} option in {\tt /etc/fstab} are ignored. If this
option is not specified for a hard mounted filestore on a machine that is
down, the login will wait for that machine to come back up i.e. the logging
in process will hang and users will not be able to log in.

\end{itemize}

\item The local disk on workstations used for {\tt /} and swap space should
not be served to other machines.

\item The local disks on workstations used for scratch space should
not be served to other machines. The local scratch space is for use
by users of that machine only. Data should be copied to the local
scratch disk from the main Sun server disks, NFS served disks on a VAX or
from tape and not left on the local scratch space indefinitely. You may wish to
mount these
disks read-only when you NFS mount them to allow users to copy data from the
scratch disk of a machine on which they were previously working to a new
location.

\end{enumerate}

Managers must be aware of the difference between the two sorts of file
systems on VMS disks that can be accessed from UNIX machines via the
VMS/ULTRIX connection software. Firstly there is the VMS file system that
resides on a Files-11 On-Disk Structure (ODS-2) disk. An example of this
would be your usual VAX user files which can be accessed from the UNIX
machine (e.g. you can access your VAX LOGIN.COM from the Sun). The other sort
of file system is a UNIX file system which consists of a special
file called a container file and a collection of subdirectories. This container
file, which is an RMS data file, contains the file system parameters and
directory structure for a UNIX file system. Individual UNIX files in this UNIX
file system cannot be accessed from the VAX. There are a number of other
differences between the two file systems that are described in length in the
VMS/ULTRIX connection  documentation.

Sites should utilize both VMS and UNIX filestore systems with their Suns.
The VMS filestore is particularly useful for sharing source code and
allowing the development of UNIX code using VMS programming tools
(e.g. EVE, VAXset etc.). The choice as to which file system to use for
a particular set of files will depend on a number of considerations.

You may wish to use a Connection UNIX file system for the following reasons:

\begin{itemize}

\item Applications may require a UNIX name space (file names are case
sensitive).

\item Applications may require symbolic links.

\item Applications may require hard links.

\end{itemize}

You may wish to use a Connection VMS file system for the following reasons:

\begin{itemize}

\item The situation requires extensive sharing of data between the host VAX and
your Suns (e.g. source code).

\item The situation requires multiple versions of the files (the UNIX file system
does not support version numbers).

\end{itemize}

SunOS offers a utility called the automounter. The mount command is restricted
to the superuser. If we were to allow normal users to use mount, they could
over mount directories such as {\tt /usr} and {\tt /usr/bin}. This would lead to an
insecure and uncontrolled environment. The automounter can be used to supply
NFS partitions to users on demand. It allows users to mount remote file
systems themselves in a secure and transparent manner. System administrators
should determine if the automounter may be of use at their site. The
automounter utility is described in detail in the Sun System and Network
Administration manual.

\subsubsection{Network Information Service (NIS)}

The Network Information Service (NIS), formerly known as Yellow Pages, is a
distributed database lookup network service. NIS allows single copies of
frequently updated files to be made available to many machines and avoids
having to update the relevant files on every machine whenever there is a
modification. Files affected by the NIS include {\tt /etc/passwd} and
{\tt /etc/hosts}.
NIS is described in detail in the Sun System \& Network Administration manual.

Sites with a single Sun need not implement NIS.
Sites with 2 or more Suns should use NIS and have at least one NIS server.
At sites where there is a SparcServer and a fleet of SparcStations, the
server should be the NIS master server. The addition of an NIS slave server
would provide some degree of redundancy if the master server
should fail, but the extra management required to set-up and
maintain this configuration is not considered worth the investment until
managers have acquired more Sun expertise. New Sun
managers have enough to do without worrying about the subtleties of NIS.

\subsection{Mail}

For the present, we are expecting users to continue to use the VAXcluster as
their mailbox and hence there are only a few basic recommendations on
configuring the mail system as yet.

If you are running an NIS cluster, then you should set up your system so that
mail sent to any host in the cluster can be read from any other host in that
cluster (as with VAXclusters). The easiest way to make this possible is to
mount the file server's mail directory on each of the clients. You will
need to include an appropriate line in your {\tt /etc/exports} file

\begin{verbatim}

/var/spool/mail -rw=client1:client2,secure

\end{verbatim}

You will also need a corresponding line in {\tt /etc/fstab} on each of the
clients

\begin{verbatim}

server:/var/spool/mail /var/spool/mail nfs rw,bg,intr,hard,secure 0 0

\end{verbatim}

\section{Security}

Unix is infamous for having poor security. It would appear that, by
default, Suns are relatively insecure compared with VAXes but it is possible to
take a few simple steps to plug up the loopholes and make your system
considerably more secure. At this time C2 level security is not considered
worth implementing. The overheads in terms of both disk space (up to 5MB per
machine per day) and the CPU overhead of the auditing processes do not justify
the gains in security. Secure NFS is also not considered necessary at present.

\subsection{Basic security}

ALL accounts MUST have passwords. There should be NO exceptions.
For maximum security, all passwords should be at least 6 characters long
(8 is the maximum number of characters that are used).
New accounts should immediately be given passwords by the system
manager using:

\begin{verbatim}

# passwd username

\end{verbatim}

{\tt /etc/passwd} is an ASCII file with a one-line entry for
each user. The different fields in the entry are described in the
Sun documentation. By default, this file is world readable. This is not
a serious security risk since only the encrypted passwords are stored. However,
it does mean that details of accounts without a password are available to all
users and hence it is vital that all accounts have an entry in the encrypted
password field in {\tt /etc/passwd}.

Accounts currently not in use should be disabled by putting a * in the
encrypted password field in {\tt /etc/passwd}. The following example
{\tt /etc/passwd}
illustrates these points:

\begin{verbatim}

root:GGpUaycc3at1Y:0:1:Operator:/:/bin/csh
nobody:*:65534:65534::/:
daemon:*:1:1::/:
sys:*:2:2::/:/bin/csh
bin:*:3:3::/bin:
uucp:*:4:8::/var/spool/uucppublic:
news:*:6:6::/var/spool/news:/bin/csh
ingres:*:7:7::/usr/ingres:/bin/csh
audit:*:9:9::/etc/security/audit:/bin/csh
sync::1:1::/:/bin/sync
sysdiag:*:0:1:Old System Diagnostic:/usr/diag/sysdiag:/usr/diag/sysdiag/sysdiag
sundiag:*:0:1:System Diagnostic:/usr/diag/sundiag:/usr/diag/sundiag/sundiag
djr:eJd/NTxy7qMn2:24500:24500:David Rawlinson:/home/djr:/bin/csh
cac::24501:24501:Chris Clayton:/home/cac:/bin/csh
dlt:*:24502:24502:Dave Terrett:/home/dlt:/bin/csh
+::0:0:::

\end{verbatim}

Note that, in the above bogus example, username cac has an empty encrypted
password field and hence no password. This is forbidden. Account dlt is
currently disabled.

You must be superuser to modify this file. You should always use the
{\tt /etc/vipw} command to modify {\tt /etc/passwd} since this locks the file and
ensures that no other modifications can be made (by another privileged
user) until you are finished.

Managers should exploit the -x and -e qualifiers for {\tt passwd} to enable
password aging.

The security of a Sun workstation set up with the defaults can be compromised by any
user who has access to the keyboard. This involves halting the  machine and
booting up single user. This requires seven keystrokes and can quite literally
be done with your eyes closed (the author tried it as a technical exercise).

To prevent users gaining root access to a machine in this way without
giving the correct authorization, you must modify the file {\tt /etc/ttytab} as
follows:

\begin{verbatim}

#
# @(#)ttytab 1.6 89/12/18 SMI
#
# name	getty				type		status	comments
#
console	"/usr/etc/getty std.9600"	sun		on local
ttya	"/usr/etc/getty std.9600"	unknown		off local secure
ttyb	"/usr/etc/getty std.9600"	unknown		off local secure
tty00	"/usr/etc/getty std.9600"	unknown		off local secure
.
.
.

\end{verbatim}

The only change is that the ``secure'' has been removed from the console
entry. Secure in this instance refers to the notion that the terminal is
``secured'' away from straying hands i.e. under lock and key. This is clearly
not the case with a common-user workstation. The effect of this
modification is:

\begin{enumerate}

\item When the machine is booted single-user, the root password must be known in
   order to gain root privileges. Make sure that root does indeed have
   a password before removing the secure comment, otherwise when you
   boot single-user, the system will still prompt for a password which
   you will be unable to give and hence you will not be able to
   complete the single-user boot.

\item In multi-user mode root cannot log into the console. Instead it is
   necessary to log into a non-root account (e.g. your usual username) and
   use the su command to gain root privileges. This might seem a nuisance
   but does ensure that the usernames used to access the root account
   are recorded. Make sure that you have  added a non-root account before
   removing the secure comment on the console entry else you will not be able to
   get access to the root account  in multi-user mode.

\end{enumerate}

It is also possible to prevent access to the root account over the network
by removing the local comment on the console entry.

If you make any changes to {\tt /etc/ttytab}, you must notify the init process with

\begin{verbatim}

# kill -HUP 1

\end{verbatim}

to implement the changes on the running system.

It is also possible to control which users can use the su command
by modifying the file {\tt /etc/groups}. To do this, add the following line
at the beginning of {\tt /etc/groups}.


\begin{verbatim}

wheel:*:0:djr,cac,dlt

\end{verbatim}


Now IF the secure comment is removed from the console entry in
{\tt /etc/ttytab},
THEN only djr,cac and dlt can use the su command to become the superuser.

\subsection{Network security}

The method of network communication between UNIX machines depends whether
the connection is to a trusted or non-trusted host. The commands to
communicate with a trusted host may differ from those used to communicate with
a non-trusted host. Trusted host commands are more analogous to commands
used on a local machine (e.g. {\tt cp} \& {\tt rcp}) and simplify access by bypassing the
password security check otherwise required. The VMS analogy to the trusted
host is the proxy login.
One can define which remote hosts and which users on those hosts are trusted.

The file {\tt /etc/hosts.equiv} specifies the names of machines from which users
are allowed to login ({\tt rlogin}), remote copy ({\tt rcp}) or
perform remote shell ({\tt rsh})
functions. {\tt /etc/hosts.equiv} requires only machine hostnames as entries and
allows a user on a machine registered in the file to login to the local machine
only if an entry for that specific user is found in {\tt /etc/passwd} on the
local machine. If there is an entry for the specific user in {\tt /etc/passwd} but no
entry for the machine of that user in {\tt /etc/hosts.equiv}, then the user will
not be allowed to use the above commands.
Note that the symbol + is a wildcard i.e. anyone can rlogin without a password
if there is an account with the same username on that machine.


The file $\sim${\tt USERNAME/.rhosts} may exist in the home directory of each user. Its
purpose is to allow a specific user to access a particular machine, but not
just from any machine. Again, the user must also have an entry in the
{\tt /etc/passwd} file of the machine used. Beware of having  a {\tt /.rhosts} file on your
systems since users placed in this file will automatically have access to
your system as root.



Unfortunately, if is necessary to use these mechanisms to access peripherals
despite the potential security risk. For example,
it is necessary
to use {\tt /.rhosts} in order to allow a system manager to dump partitions
on a tapeless SparcStation disk to an Exabyte on a SparcServer (see
Section 4.2).
In this case, the author  recommends that you create
{\tt /.rhosts} on the server and enter the SparcStation name in it just before doing
the dump and remove the file immediately after.
Similarly, at Cambridge,
{\tt /etc/hosts.equiv} on SUNDOG contains a + to allow all other machines
to print transparently on its laser printer. However, in this case it is
possible to grant printing rights alone to a given machine by using the
{\tt /etc/hosts.lpd} file instead of {\tt /etc/hosts.equiv}. If you do not trust a
machine's users but are willing to give them printing rights, use the
{\tt hosts.lpd}
file.

\section{Operating standards}

\subsection{Adding users}

At present there is no separate `APPLICATION FOR THE USE OF STARLINK' form required
for users who wish to use the Starlink Sun systems, but only registered users
of Starlink are permitted Sun accounts.

Usernames should follow normal Starlink conventions and preferably be identical
to users' VMS usernames.
Shared usernames are not permitted.

Before you start to add users to your system you should check with your local
Ethernet manager if there is a restriction on which UID/GID numbers you should
use for your users, in order to ensure that they are unique on the Ethernet.
Non-unique UID numbers are a major security risk when using NFS.

Each user should have the login directory {\tt /home/USERNAME}
(or {\tt /home2/USERNAME}).
New user accounts should be given template {\tt .login} and {\tt .cshrc} files. These
should contain a call to system wide versions of these files which set up
the environment variables and aliases necessary to allow the use of unbundled
and other additional software packages. The system wide versions of these
files should be {\tt /usr/local/.login} and {\tt /usr/local/.cshrc}. These should be
called from {\tt USERNAME/.login} and {\tt USERNAME/.cshrc} respectively using the source command.

It is probably worth spelling out the difference between {\tt .login} and
{\tt.cshrc}.
{\tt .login} is read once by the system during the log in procedure. This file is
normally used to set up any environment variables which will be available
to all subsequent processes. Environment variables are independent of the
shell being used. Paths are usually set up in {\tt .login}.
{\tt .cshrc} is read by the system {\it each time}\, a `C-shell' is created.
Shell variables and aliases are set up in {\tt .cshrc}.

The program {\tt /usr/etc/pwck} (available with the System V software option) should
be used occasionally to ensure that the password file does not contain errors.

\subsection{Dumping the file structure}

Site managers should establish a dump (BACKUP in VMSspeak) schedule
to protect against accidental loss of file by user error and hardware failure.
On Sun systems with a large amount of local disk space, it will be necessary
to establish a comprehensive dumping timetable. The following recommendations
should be followed, although the details of how you implement them are
up to local AMCs. A variety of dump schemes are described in the
Sun System \& Network Administration manual.

\begin{itemize}

\item Level 0 dumps should be taken every 4 to 6 weeks. This should be done
in single-user mode, if possible, and preferably out of prime time.

\item Incremental dumps should be made on partitions containing user files
at least once per week.

\end{itemize}

At sites where there are only one or two Suns, with small local disks, it is
recommended that most user files be stored on VAXcluster disks and accessed via
NFS. All users should still have a login directory {\tt /home/USERNAME} on a
local disk to allow normal use of the Sun (i.e. access to his or her
{\tt .login}, {\tt .cshrc}
and some local disk space) when the VAXcluster is down. Symbolic links
can be used to make VAX directories appear as subdirectories of
{\tt /home/USERNAME}.
This will ensure that user files are backed up as part of the VAX backup
procedure. 0 level dumps only need then be made of the system files when
necessary.

One can only dump disk partitions and not NFS served file structures. Hence,
to dump a partition on a local disk on a tapeless workstation, one must use
a remote tape device. An entry in {\tt .rhosts} of the remote machine is required
to allow use of its tape drive (see Section 3.2).

The following are template dump commands that have been tried out by the
author and which can be used for 0 level dumps on the present Starlink Sun
configurations. The various qualifiers are described  in the Sun documentation.

\begin{enumerate}

\item To dump a partition (xd0a) on a local disk to a local Exabyte drive.

\begin{verbatim}

/usr/etc/dump 0ucbsdf 56 51900 4100000 /dev/nrst1 /dev/rxd0a

\end{verbatim}

\item To dump a partition (sd0a) on a local disk to a remote Exabyte drive on
host CAST0. An entry will have to be made in {\tt /.rhosts} on CAST0 to allow this.

\begin{verbatim}

/usr/etc/dump 0ucbsdf 56 51900 4100000 cast0:/dev/nrst1 /dev/rsd0a

\end{verbatim}

 \item To dump a partition (sd3a) to a local 150Mb 1/4$"$ tape (with a DC6150
cartridge).

\begin{verbatim}

/usr/etc/dump 0ucstf 620 18 /dev/nrst1 /dev/rsd3a

\end{verbatim}

\end{enumerate}

The UNIX {\tt dump} command will copy the whole or part of a partition to tape
dependent on the level of backup chosen. A level 0 backup is a complete
copy of a single partition. Incremental backups may be at levels 1 through
to 9. The {\tt dump} command will copy all files which have been altered since
the last backup of the current or higher levels. For instance a level 3
backup copies all files that have been altered since the last backup at level
3, including any files already backed up at levels 4 through to 9. This system
allows a backup sequence to be devised which provides for many incremental
dumps between full dumps, but allows the incremental tapes to be
``merged'', which reduces the number of tapes needed to backup and restore the
filestore. As time goes on, the lowest incremental dump tapes (usually level 1)
will gradually become more and more full.

Incremental dumping is only really necessary if you either have a large amount
of disk space or are using a low capacity medium, such as tape cartridge.
If you are using a Exabyte drive and have only a few disks, it is probably
better to do full dumps all of the time. No firm recommendations are made here
for a dump schedule since, as indicated above, the frequency of dumps is
dictated by your hardware and usage of your disks.

\subsection{Accounting}

By default, full accounting is switched off under SunOS. Connect time
accounting is enabled by default and the accounting information is stored in the
{\tt /var/adm/wtmp}. The presence of this file determines if this information
is recorded or not.

Process-time accounting is disabled by default. This accounting information is
stored in {\tt /var/adm/acct}. To enable process accounting on your system you
must modify {\tt /etc/rc} and {\tt /var/spool/cron/crontabs/root} as follows:

***** {\tt /etc/rc} *****

\begin{verbatim}

#      accounting is off by default.
#
/usr/lib/acct/startup

\end{verbatim}

***** {\tt /var/spool/cron/crontabs/root} *****

\begin{verbatim}

0 * * * * /usr/lib/acct/ckpacct
0 1 1 * * /usr/lib/acct/dodisk
0 2 1 * * /usr/lib/acct/runacct 2> /usr/adm/acct/nite/fd2log
15 5 1 * * /usr/lib/acct/monacct

\end{verbatim}


The function of each of these accounting programs is described in
the SunOS System and Network Administration manual.

The fields in the root crontab file are Minutes, Hours, Day of Month,
Month of Year, Day of Week and command to be executed.

It is worth mentioning at this point the value of the crontab utility.
It can be used to execute commands automatically at pre-defined times. The
crontab utility gives you the functionality of a system jobs batch queue
on a VAX, but with a more advanced scheduling system. All system managers
should familiarize themselves with the crontab utility, which is described
in detail in the Sun documentation.

Each month, a file {\tt /var/adm/acct/fiscal/fiscrptNN} (e.g. fiscrpt11)
is created which contains the monthly usage report for the previous month
containing the following information:

\begin{itemize}

\item UID: The user ID.

\item LOGIN NAME: The login name of the user.

\item CPU (MINS): The amount of time the user's processes used the CPU. This
is broken down into PRIME and NPRIME (non-prime) usage. The definition of
prime time is kept in {\tt /etc/acct/holidays}.

\item KCORE-MINS: This is a cumulative measure of the amount of memory a process
uses while running. the amount shown reflects kilobyte segments of memory used
per minute.

\item CONNECT (MINS): The amount of real time that the user is connected for.

\item DISK BLOCKS: The average number of disk blocks (512 bytes) allocated to
the user.

\item \# OF PROCS. Number of processes invoked by the user.

\item \# OF SESSIONS: The number of times a user logged into the system.

\item DISK SAMPLES: This indicates how many times the disk accounting program
was run to obtain the average number of DISK BLOCKS listed earlier.

\item FEE: The FEE accounting field is not currently used by Starlink.

\end{itemize}

The file also contains a breakdown of usage of the system by command and a
list of last login dates.

Managers should be aware that the accounting system used by Starlink differs
slightly from that described in the Sun manuals. We only require a summary
of usage for each month rather than for each day. Furthermore, the accounting
procedures that summarize usage each day destroy the original files which contain the
information needed to track down, for example, unauthorized usage of the
system. Hence, we simply run the ``daily'' accounting procedure once per
month on the records for the whole month, followed by the monthly accounting
procedure.

During the month, the accounting information can be reviewed using the commands
{\tt last} (last login), {\tt ac} (login accounting) \& {\tt sa} (system accounting).
A full description of these can be found in the manual pages. However, it is
worth noting the following options:

\begin{itemize}

\item {\tt last cac} (All logins by cac this month)

\item {\tt ac -d} (Daily total connect time)

\item {\tt ac -d -p} (Daily connect time by user)

\item {\tt sa - m} (Total CPU usage by user)

\end{itemize}

\subsubsection{Printer accounting}

To enable printer accounting, you must amend the spooling control file
{\tt /etc/printcap}, and ensure that the accounting file exists and has the correct
owner and group.

Add the following to the printer's printcap entry, using a different
file name for each printer:

\begin{verbatim}

:af=/var/adm/lp.acct:if=/usr/lib/lpf:

\end{verbatim}

lp.acct can be any name you choose.

\begin{verbatim}

# touch /var/adm/lp.acct
# chown daemon.daemon /var/adm/lp.acct

\end{verbatim}

The printer accounting information is logged in the file specified above.
The information will be logged for any printer where such an entry exists.
The program used to analyse the data is {\tt pac}, which will give the total
number of jobs, pages printed and associated cost per user.

\subsection{Quotas}

The UNIX disk quota system should be used, where necessary, to control the
amount of disk space used by each user in a given disk partition. The system
works on a hard partition basis, for both local and NFS partitions. The
administrator must set up limits for each user on each partition on which
quotas are to be applied.

When configuring your kernels (see Section 2.4), make sure that you specify
OPTIONS QUOTA on each machine where quotas are to run, or where hard partitions
exist to which quotas are to be applied i.e. even if they are to be mounted
over NFS and will not use quotas locally.

You should check the {\tt /etc/rc} files for the presence of the {\tt quotaon} and
{\tt quotacheck} commands, and ensure they are not commented out, and
edit the file {\tt /etc/fstab} to add the quota option to any file system mounts
for which you want quotas applied. You need  only do this for hard partitions,
i.e. not on an NFS client.

The quotas file must be present in the top level of every partition for which
quotas are to be used. The file holds details for this partition about all the
users and their quotas. An empty quotas file can be created at the top level
of each file system with the touch command.

Quotas can be controlled using the {\tt edquota} command, as described in the Sun
documentation. Users use the {\tt quota} and {\tt du} commands to get information
about disk usage and limits.

Please note that the use of the quota system has a high performance overhead.

\subsection{Monitoring system performance}

The memory management system of UNIX is very similar to that of VMS. The main
differences that a VAX/VMS system manager will notice are that:

\begin{itemize}

\item There are fewer parameters that can be tuned. In future, there will be
even fewer as these parameters become dynamic and are modified by the system
itself.

\item The page and swap files are combined into a single swap space. Additional
swap partitions and files may be added by the system administrator with the
{\tt swapon} command.

\item One cannot limit the working set of individual processes or those owned
by a given user. Users themselves can limit resource consumptions of their
own processes with the {\tt limit} built--in function of {\tt csh}.

\end {itemize}

System administrators should monitor the performance of their systems and
become familiar with {\tt vmstat} (report virtual memory statistics -
equivalent to \$ MONITOR),
{\tt pstat} (print system facts) and {\tt ps} (process status - equivalent to
\$ SHOW SYSTEM).

You should use the {\tt ps} command to identify processes that are using system
resources unnecessarily. For example, workstations with only one ethernet
do not need to do dynamic routing. Instead, they can route statically by
commenting out the lines in {\tt /etc/rc.local} that startup the {\tt
in.routed} daemon. Similarly, at present, the {\tt sendmail} daemon is not
required if you have followed the recommendations in Section 2.6.

System administrators should re-configure their kernels (see Section 2.4)
to free up unnecessarily allocated memory. The kernel is not swapped or paged
and hence memory assigned to the kernel is fixed and unavailable for other
applications to use.

Fine tuning of the system can be done by editing the {\tt param.c} file that resides
in the directory that is created when you run {\tt /etc/config} on your new kernel
configuration. System administrators should not need to modify this file, but
it is drawn to your attention for interest. {\tt param.c} is equivalent to the
parameter files such as PARAMS.DAT under VMS.

As file systems fill up, performance will degrade owing to the difficulty of
finding contiguous blocks for files. The messages produced by {\tt fsck} at
boot time should be noted. If the fragmentation becomes greater than 2.5\%,
it may be worthwhile doing a logical copy using {\tt tar} of the file system
in single user mode, remaking the file system using {\tt newfs} and then
restoring the files.


\subsection{Cleaning the file system}

Your file system will gradually fill up. SunOS uses a number of files for
logging purposes. Some of these are automatically truncated, other continue to
grow. Furthermore, temporary files can often be left when no longer required.
If your file system begins to fill up, try the following:

\begin{itemize}

\item Rebuilding your kernel creates many files that can be deleted. Look in
{\tt /usr/share/sys/sun4c} for directories containing old kernel build
files. This can save a lot of space in {\tt /usr}.

\item Application core dumps produce vast files called {\tt core}. It is
well worth the occasional search of the entire file system for these rogue
files and deleting them. The following command will do this for you

\begin{verbatim}

	# find / -name core -exec rm -f {} \; - print

\end{verbatim}

System administrators are urged to become familiar with the {\tt find} utility.
It is particularly useful for periodic maintenance.

\item Look out for ``odd'' devices in {\tt /dev}. These may be created if users
specify the wrong device name for tape, for example. They can become
very large.

\end{itemize}

\end{document}
