\documentstyle{article} 
\pagestyle{myheadings}
\markright{SSN/48.1}
\setlength{\textwidth}{160mm}
\setlength{\textheight}{240mm}
\setlength{\topmargin}{-5mm}
\setlength{\oddsidemargin}{0mm}
\setlength{\evensidemargin}{0mm}
\setlength{\parindent}{0mm}
\setlength{\parskip}{\medskipamount}
\setlength{\unitlength}{1mm}

\begin{document}
\thispagestyle{empty}
SCIENCE \& ENGINEERING RESEARCH COUNCIL \hfill SSN/48.1\\
RUTHERFORD APPLETON LABORATORY\\
{\large\bf Starlink Project\\}
{\large\bf Starlink System Note 48.1}
\begin{flushright}
P M Allan\\
6th October 1987
\end{flushright}
\vspace{-4mm}
\rule{\textwidth}{0.5mm}
\vspace{5mm}
\begin{center}
{\Large\bf BACKUP --- Recovering from incremental tape backups}
\end{center}
\vspace{5mm}

The importance of making regular backups of your disks in well known, but
it can be very frustrating to have a head crash and then discover that you
cannot recover from your backups because the manuals do not tell you what
to do.
The DEC documentation covers most possibilities, but not the case where a
removable pack is copied to a new pack, the new pack is then used as the
current pack, and incremental backups to tape are then made.
This situation is a fairly common one when you have two or more removable
disk drives.
This note covers backup and recovery in general terms, and details
exactly what to do in the above mentioned case.
                        
When you do a full backup of a disk there are two possibilities: you can
do a disk copy or backup the contents of the disk to a saveset.
In the second case, the saveset can be on disk or tape, the only difference is
speed.
For example, to do a disk copy from DRA0: to DRA1: you must type:
\begin{verbatim}
    $ MOUNT/FOREIGN DRA1: 
    $ BACKUP/IMAGE/RECORD/verify/trunc DRA0: DRA1:
    $ DISMOUNT DRA1:
\end{verbatim} 
for a disk to disk copy, or:
\begin{verbatim}
    $ BACKUP/IMAGE/RECORD/verify DRA0: DRA1:saveset_name/SAVE_SET
\end{verbatim} 
to make a saveset on disk. 

The qualifiers in lower case are optional, but are recommended.
You must not write protect the old disk (DRA0:), since the date of this
backup must be recorded on it.

To make incremental backups to tape, one then does:
\begin{verbatim}
    $ MOUNT/FOREIGN MTA0: 
    $ BACKUP/RECORD/SINCE=BACKUP/JOURNAL=journal_file/VERIFY DRA0: -
       MTA0:incremental_saveset_name
    $ DISMOUNT MTA0:
\end{verbatim} 
This much is well known.
The important part is recovering from a disk crash.
If you are recovering from a full saveset plus incremental savesets, then
the VMS manual tells you what to do.
It stresses that when recovering from the full saveset, you must use /RECORD
so that the date that this backup was done is written to the new disk.
If you are recovering from an exact copy of a disk then you must NOT use
/RECORD as it will write today's date in the backup date field, and the 
recovery from the incremental backups will then not proceed correctly.
It is this point that is vital and is very well hidden in the manuals.
The sequence of operations to recover from a crash is:
\begin{enumerate}
\item Copy the old disk pack to a new one (after all you might get another
crash while you are recovering).
\item Bring the new disk up to date from the incremental backup tapes.
The DEC manuals suggest that it is quicker to go through the tapes in reverse
\end{enumerate}
Let us consider a concrete example.

A full backup copy of the disk USER1 is made on 2 March (first Monday in the
month). 
\begin{verbatim}
    $ BACKUP/IMAGE/VERIFY/RECORD/TRUNC DRA0: DRA1:
\end{verbatim}
I do this with stand alone backup, so there is no mounting of disks.

I then do daily incremental backups, cycling through the tapes once per week:
\begin{verbatim}
    $ MOUNT/FOREIGN MT: TAPE
    $ BACKUP/RECORD/REWIND/FAST/JOURNAL=3MAR87/SINCE=BACKUP -
      DISK$USER1:[*...] -
      TAPE:3MAR87.BCK/PROTECTION=(S:REWD,O:REWD,G,W)
\end{verbatim}
and weekly incremental backups:
\begin{verbatim}
    $ MOUNT/FOREIGN MT: TAPE
    $ BACKUP/RECORD/REWIND/FAST/JOURNAL=9MAR87/SINCE=2-MAR-1987:09:13:23 -
      DISK$USER1:[*...] -
      TAPE:9MAR87.BCK/PROTECTION=(S:REWD,O:REWD,G,W)
\end{verbatim}
The dates, command qualifiers etc.\ are looked after by a command procedure
so that I do not have to remember what goes where.
This is added as an appendix to this document for those who are interested.

When the dreaded crash happens (on 18 March, for example), the recovery
procedure is as follows:
\begin{verbatim}
    $ BACKUP/VERIFY/IMAGE DRA0: DRA1:

    $ MOUNT/FOREIGN MT: TAPE
    $ BACKUP/INCREMENT TAPE:18MAR87.BCK DRA1:
    $ DISMOUNT TAPE
    $ MOUNT/FOREIGN MT: TAPE
    $ BACKUP/INCREMENT TAPE:17MAR87.BCK DRA1:
    $ DISMOUNT TAPE
    $ MOUNT/FOREIGN MT: TAPE
    $ BACKUP/INCREMENT TAPE:16MAR87.BCK DRA1:
    $ DISMOUNT TAPE
    $ MOUNT/FOREIGN MT: TAPE
    $ BACKUP/INCREMENT TAPE:9MAR87.BCK DRA1:
    $ DISMOUNT TAPE
\end{verbatim}
The disk is now restored.
\appendix
\newpage
\section{INCREMENT --- A command procedure to backup disks}
\begin{verbatim}
$! Do an incremental backup of a disk.		[OPER.BACKUP]INCREMENT.COM
$!                                              31.03.87
$!
$! Prevent any interference so that this can be left running safely.
$!
$ ON CONTROL_Y THEN LOGOUT
$ ON ERROR THEN CONTINUE
$!
$! Set up logical names
$!
$ ASSIGN 'F$TRNLNM("SYS$DISK","LNM$SYSTEM_TABLE")[OPER.BACKUP] BACKUPDIR
$!
$! Get the time, date and day of the week.
$!
$ TIME = F$TIME()
$ DATE = F$EXTRACT(0,2,TIME) + F$EXTRACT(3,3,TIME) + F$EXTRACT(9,2,TIME)
$ IF F$EXTRACT(0,1,DATE) .EQS. " " THEN DATE = F$EXTRACT(1,6,DATE)
$ DATE_OF_MONTH = F$EXTRACT(0,2,TIME)
$ DAY_OF_WEEK = F$CVTIME(TIME,,"WEEKDAY")
$!
$! Ask which area is to be backed up
$!
$10:
$ INQUIRE AREA -
     "Is the backup of USER1 [U1] or the system disk [S0] ?"
$ IF AREA .EQS. "U1" THEN GOTO U1
$ IF AREA .EQS. "S0" THEN GOTO S0
$ GOTO 10
$!
$! Set up for backing up USER1
$!
$U1:
$ FROM = "DISK$USER1:[*...]"
$ TO = "TAPE:''DATE'.BCK"
$ OPTIONS = "/RECORD/REWIND/FAST"
$ PROT = "/PROTECTION=(S:REWD,O:REWD,G,W)"
$!
$! Ask for the type of incremental backup.
$!
$U10:
$ INQUIRE TYPE "Is this a weekly [W] or daily [D] backup?"
$ IF F$EXTRACT(0,1,TYPE) .NES. "W" THEN GOTO U11
$!
$! Weekly backup
$!
$ ROOT = "BACKUPDIR:USER1_WEEKLY"
$ IF DATE_OF_MONTH .LE. 14 THEN -
    RENAME 'ROOT'.BJL 'ROOT'_OLD.BJL
$ OPEN/READ SAVE_DATE 'ROOT'.DATE
$ READ SAVE_DATE SINCE
$ CLOSE SAVE_DATE
$ GOTO PROCESS_TAPE
$!
$U11:
$ IF F$EXTRACT(0,1,TYPE) .NES. "D" THEN GOTO U12
$!
$!  Daily backup
$!
$ ROOT = "BACKUPDIR:USER1_''DAY_OF_WEEK'"
$ RENAME 'ROOT'.BJL 'ROOT'_OLD.BJL
$ SINCE="BACKUP"
$ GOTO PROCESS_TAPE
$!
$U12:
$ GOTO U10
$!
$! Set up for backing up SYSTEM
$!
$S0:
$ FROM = "SYS$SYSDEVICE:[*...]/EXCLUDE=([SYS0.001001...],[SYS0.001002...],"+-
    "[SYS0.001054...])"
$ OPTIONS = "/RECORD/REWIND/FAST"
$ PROT = "/PROTECTION=(S:REWD,O:REWD,G,W)"
$!
$! Ask for the type of incremental backup.
$!
$S10:
$ INQUIRE TYPE "Is this a weekly [W] or daily [D] backup?"
$ IF F$EXTRACT(0,1,TYPE) .NES. "W" THEN GOTO S11
$!
$!  Weekly backup
$!
$ TO = "TAPE:''DATE'.BCK"
$ ROOT = "BACKUPDIR:SYSTEM_WEEKLY"
$ IF DATE_OF_MONTH .LE. 14 THEN -
   RENAME 'ROOT'.BJL 'ROOT'_OLD.BJL
$ OPEN/READ SAVE_DATE 'ROOT'.DATE
$ READ SAVE_DATE SINCE
$ CLOSE SAVE_DATE
$ GOTO PROCESS_TAPE
$!
$S11:
$ IF F$EXTRACT(0,1,TYPE) .NES. "D" THEN GOTO S12
$!
$!  Daily backup
$!
$ TO = "DISK$USER3:[BACKUP.SYSTEM]''DATE'.BCK"
$ ROOT = "BACKUPDIR:SYSTEM_''DAY_OF_WEEK'"
$ SINCE="BACKUP"
$ GOTO PROCESS_TAPE
$!
$S12:
$ GOTO S10
$!
$! Allocate the tape, mount it and do the backup.
$!
$PROCESS_TAPE:
$ SET VERIFY
$ DRIVE:=MT:
$ IF P1 .NES. "" THEN DRIVE=P1
$ ALL 'DRIVE' TAPE
$ MOU/FOR TAPE
$ SET PROC/PRIV=READALL
$ BACKUP'OPTIONS'/SINCE='SINCE'/JOURNAL='ROOT' 'FROM' 'TO''PROT'
$ SET PROC/PRIV=NOREADALL
$ DISMOUNT TAPE
$ DEAL TAPE
$ SET NOVERIFY
$!
$! Save the date and time if this is a weekly backup
$!
$ IF TYPE .NES. "W" THEN GOTO TIDY1
$ BACKUP_TIME = TIME
$ BACKUP_TIME[11,1]:=:
$ OPEN/WRITE SAVE_DATE 'ROOT'.DATE
$ WRITE SAVE_DATE BACKUP_TIME
$ CLOSE SAVE_DATE
$ PURGE 'ROOT'.DATE
$ GOTO END
$!
$TIDY1:
$ PURGE 'ROOT'_OLD.BJL
$!
$END:
$ LOGOUT
\end{verbatim}
\end{document}
