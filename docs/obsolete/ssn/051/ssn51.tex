\documentstyle{article}
\pagestyle{myheadings}
\markright{SSN/51.1}
\setlength{\textwidth}{160mm}
\setlength{\textheight}{240mm}
\setlength{\topmargin}{-5mm}
\setlength{\oddsidemargin}{0mm}
\setlength{\evensidemargin}{0mm}
\setlength{\parindent}{0mm}
\setlength{\parskip}{\medskipamount}
\setlength{\unitlength}{1mm}

\begin{document}
\thispagestyle{empty}
SCIENCE \& ENGINEERING RESEARCH COUNCIL \hfill SSN/51.1\\
RUTHERFORD APPLETON LABORATORY\\
{\large\bf Starlink Project\\}
{\large\bf Starlink System Note 51.1}
\begin{flushright}
D L Terrett\\
8 February 1988
\end{flushright}
\vspace{-4mm}
\rule{\textwidth}{0.5mm}
\vspace{5mm}
\begin{center}
{\Large\bf VAXnotes}
\end{center}
\vspace{5mm}

VAXnotes is a computer conferencing system that lets people conduct on-line
conferences or meetings. Each conference has a subject and is hosted by
a particular node on DECnet and has a `moderator' who has the job of keeping
the discussion in order.

VAXnotes has been purchased to facilitate the exchange of information between
the Starlink project and its users, and between Starlink users. For this to
happen it must be possible to find out easily what conferences exist, and
conference topics must not overlap excessively. To ensure the orderly growth of
the use of VAXnotes, the following rules will be adhered to, at least until the
project has more experience with the product.

\begin{itemize}

\item All conferences will reside in the standard directory {\tt
NOTES\$LIBRARY:}.

\item Conferences will be created by the site manager, although often at
the instigation of a user. The task of moderating a conference can be
delegated.

\item Before agreeing to create a conference, the site manager must
satisfy him or herself that
\begin{itemize}
\item The proposed topic is not already covered by an existing conference
elsewhere on the network.
\item The topic is compatible with the aims of the Starlink project; i.e.\
has some connection with either astronomy or the running of Starlink.
\end{itemize}

\item The conference name and title  will be chosen to be as sensible and
meaningful as possible.

\item The first topic in the conference will be a description of the
subject matter and aims of the conference. Any restrictions on the
participants in the conference must also be explained, and the name and
network address of the moderator included.

\item A copy of this topic will be inserted by the site manager in a
conference on {\tt RLVAD} called {\tt CONFERENCES} and linked with the new
conference with the {\tt SET NOTE/CONFERENCE} command. This applies to all
conferences even where the conference is of only local interest or has
restricted participation.  `Replies' can also be added to other topics in the
{\tt CONFERENCES} conference to act as cross references to other conferences on
related topics. To avoid confusion, no other conference on the network will be
called {\tt CONFERENCES}.

\item Each node should have a conference called {\tt LOCAL\_CONFERENCES}
containing the topics of conferences of particular interest to the local
users.

\item Site managers should periodically review the conferences on their
system and close any that have become inactive.

\item The project management has the right to order the closure of any
conference if it is considered unseemly or excessively boring.

\end{itemize}

The {\tt CONFERENCES} conference which lists all conferences on the network
will be moderated by the Starlink Software Librarian and all site managers will
be active participants (i.e.\ able to contribute topics and replies); all
Starlink users will be passive participants (i.e. able to read but not
contribute).

\end{document}
