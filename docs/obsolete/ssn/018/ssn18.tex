\documentstyle[11pt]{article}
\pagestyle{myheadings}

%------------------------------------------------------------------------------
\newcommand{\stardoccategory}  {Starlink System Note}
\newcommand{\stardocinitials}  {SSN}
\newcommand{\stardocnumber}    {18.1}
\newcommand{\stardocauthors}   {C A Clayton}
\newcommand{\stardocdate}      {14 January 1994}
\newcommand{\stardoctitle}     {Administering Solaris Sun systems}
%------------------------------------------------------------------------------

\newcommand{\stardocname}{\stardocinitials /\stardocnumber}
\renewcommand{\_}{{\tt\char'137}}     % re-centres the underscore
\markright{\stardocname}
\setlength{\textwidth}{160mm}
\setlength{\textheight}{230mm}
\setlength{\topmargin}{-2mm}
\setlength{\oddsidemargin}{0mm}
\setlength{\evensidemargin}{0mm}
\setlength{\parindent}{0mm}
\setlength{\parskip}{\medskipamount}
\setlength{\unitlength}{1mm}

%------------------------------------------------------------------------------
% Add any \newcommand or \newenvironment commands here
%------------------------------------------------------------------------------

\begin{document}
\thispagestyle{empty}
SCIENCE \& ENGINEERING RESEARCH COUNCIL \hfill \stardocname\\
RUTHERFORD APPLETON LABORATORY\\
{\large\bf Starlink Project\\}
{\large\bf \stardoccategory\ \stardocnumber}
\begin{flushright}
\stardocauthors\\
\stardocdate
\end{flushright}
\vspace{-4mm}
\rule{\textwidth}{0.5mm}
\vspace{5mm}
\begin{center}
{\Large\bf \stardoctitle}
\end{center}
\vspace{5mm}

\setlength{\parskip}{0mm}
\tableofcontents
\setlength{\parskip}{\medskipamount}
\markright{\stardocname}

\newpage

\section {Introduction}

This document is intended for Sun System administrators
as a follow up to SSN/7 which described how to set up
Starlink Sun systems under SunOS 4.X. The present document refers to the set up
of Sun systems under Solaris 2.X (a.k.a. SunOS 5.X). Throughout this document,
the term Solaris is understood to mean Solaris 2.X., Sun's new Unix Operating
environment based on SVR4. SSN/7 will remain in circulation
until all Starlink Suns have moved over to Solaris. Areas of system
administration which have not changed  between SunOS 4.X and Solaris and which
were discussed in SSN/7 are not covered again in this document. Instead the
reader is referred to SSN/7 when appropriate. Some sections however have been
significantly updated and thus are reproduced here in full.  This document can
be considered as an  extension to SSN/7 in which changes between the two
systems are highlighted. Eventually, there will be a single document containing
the information in both documents, but relating just to Solaris.

The purpose of SSN/7 and this document is to ensure that all of the Starlink
Sun systems are set up in a secure, logical and consistent fashion and that
they do not begin to diverge from one another. One of the strengths of Starlink
is that 90\% of a site manager's technical problems can be solved either by
simply looking at how another Starlink system is set up or by asking other site
managers. If all systems are set up nearly identically, then all managers will
encounter the same problems but only one manager need actually solve each
problem. Solutions can be posted in the UNIX\_MANAGEMENT VAXnotes
conference. This conference has restricted access and hence sensitive matters
such as security can be discussed there. New system software and patches are
also announced in this conference so it is vital that system managers read
entries regularly.

This is not a tutorial document. It is assumed that the reader is capable of
reading man pages and manuals.  It is assumed that the reader has a basic
understanding of Sun system management and no attempt is made either to clarify
Sun \& UNIX terminology or to explain system
management functions which are clearly described in the Sun documentation. The
topics covered are those that apply directly to the Starlink systems and the
particular way in which we need to install and use them; these topics may not
be covered in the Sun manuals.

This document is based on Solaris 2.3. Solaris is scheduled to be upgraded
every 6 months. However, it is unlikely that this document will be updated more
than once a year partly due to pressure of work and partly because the changes
between successive versions of the OS will probably not warrant a re-write of
the document. As it is, this first edition is not as comprehensive as the
author would like. It was felt that a limited document on Solaris would be of
more use than no document at all. The sections in need of work are are obvious
and site managers who gain expertise in these areas are invited to contribute
to this document.

It is intended that this document will grow as we gain more knowledge
of Solaris. Any corrections, comments or suggestions will be
gratefully received by the author.


\section{Initial installation}

\subsection{Installing Solaris}

If you have a single Sun, you should configure it as a server system in
anticipation of getting more Suns in the future. If you have two
or more Suns, the choice is less clear. You should set one up as a
server and the others up either as diskless or as dataless clients.
Dataless clients have their own copy of / and a swap area on a local disk.
Diskless clients have all of their files (including the swap area) on a
remote server. The dataless client setup is more efficient in terms of
ethernet traffic but the diskless client system {\bf may} be easier to manage,
particularly when it comes to upgrades. It is also possible to set up
``diskless'' clients with swap areas on local disks. This set up helps
reduce ethernet traffic but still retains many of the management
advantages of the true diskless system.

Different sites will find a given set up best for their site.
At RAL we run our Suns as a mixture of standalone and server/dataless client
machines; we only have three CPUs and hence there is little advantage in
a server/diskless client setup. Also, having only 3 CPUs makes it difficult to
compare the management overheads of the different schemes and that is where
your experiences would be of interest. We need your feedback on the
relative advantages of the two approaches.

We asked a Sun technical expert about the relative merits of a
server/diskless client scenario versus a server/dataless client scenario
from a system management point of view. Somewhat surprisingly, he
recommended the dataless approach. Sun intend to provide a full suite
of tools under Solaris to allow the administrator to run a network of
Suns from a single machine and hence a diskless setup will not have a
significant management advantage. Solaris 2.3 includes a ``distributed
administration framework'' on which these tools will be based.  For
example, Solaris 2.3 offered the option to upgrade a machine from
Solaris 2.2 by only replacing the files that had changed. This upgrade was
very easy and required very little intervention by the Site Manager.
In a future
release, this process is intended to be automated so that you can stick
a CD-ROM on the server and the dataless clients will upgrade
themselves.

If you wish to have some personal advice on this complex topic, fell free to
contact the author.

\subsubsection{Disk partitions}

One of the difficulties of configuring a new UNIX system is getting the
file store layout correct. If the layout is wrong then there may be little room
for expansion, or the system may run slowly. Once the disk has been partitioned
its geometry is fixed and it a time consuming to change.


Sun recommend at least 24Mb for the swap partition, 32Mb being the default.
Fortunately, it is not critical that you estimate the size of swap partition
required correctly since it is trivial to add additional (albeit less efficient)
swap areas later as required, even while the system is running, using the
\verb+mkfile+ and \verb+swap+ commands. Alternatively, a partition
from a local data disk could be used as a second swap partition.
You can monitor swap space usage using the command \verb+swap -l+.

Be generous when allocating space to {\tt / \& /usr}. Running out of
space in either of these partitions later is a major inconvenience.
Note that the root partition has to be much larger under Solaris than
under SunOS 4.X.  This is because the new System V packages system
keeps all of its administration files in {\tt /var/sadm} and these can
grow very large.  Do NOT delete these files. They keep a track of what
is installed on your system and are needed by the system software
installation and upgrade procedures. Furthermore, when patches are applied
to the system, by default files which are replaced in the patch are copied
to {\tt /var/sadm} so that one can ``back out'' of the patch at a later
stage. Thus, once several patches have been applied, {\tt /var/sadm}
will grow quite large. Hence, there should either be a separate partition
for {\tt /var} or the root partition should be configured with plenty
of free space. Note that there is an option with patch installations which
prevents the old files from being saved in {\tt /var}.
The author allows 30+Mb for his
root partitions under Solaris, of which 15Mb is currently in use.
Alternatively, {\tt /var} can be a separate partition altogether. This
avoids the root partition filling up completely and hanging the machine.

\subsubsection{Software options}

During the SUNinstall procedure, you are asked what optional software you
wish to install. Adding and removing software packages is trivial
under Solaris with the introduction of the SVR4 packages system. Hence,
I would advise that you initially install the entire distribution and
delete unnecessary packages as necessary when you require additional
disk space. This option requires minimal work for system managers and also
reduces the risk of not installing an apparently unnecessary piece of software
which is actually quite vital to the running of the system.

If you are short of disk space, you may wish to consider not installing the
packages listed in Table 1. Make sure that the options which are hardware specific
are not needed by your system (e.g. Leo Device Drivers).

\begin{table}[htb]
\begin{center}
\begin{tabular}{||l|l|r||}
\hline

Package Title & Name & Size (Mb)\\
\hline
GT Runtime Support Software	& SUNWgtu	& 1.20  \\
GT device drivers 		& SUNWgt	& 1.74  \\
Leo Device Drivers		& SUNWleo	& 4.44  \\
Leo Runtime Support		& SUNWleou	& 1.70  \\
Leo Shapes Support		& SUNWleoow	& 0.17  \\
Online Diagnostic Tool		& SUNWdiag	& 13.48 \\
OpenWindows Demo Images		& SUNWowdim	& 5.04  \\
OpenWindows Demo Programs	& SUNWowdem	& 1.08  \\
OpenWindows Online Handbooks	& SUNWowbk	& 2.25 	\\
OpenWindows Sample Source	& SUNWowsrc	& 6.70  \\
OpenWindows Static Libraries	& SUNWowslb	& 15.91 \\
Solaris PEX Runtime		& SUNWpexcl	& 1.92  \\
Solaris PEX Runtime		& SUNWpexsv	& 0.34  \\
XGL English localization	& SUNWxgler	& 2.34  \\
XGL Generic Loadable Libraries	& SUNWxgldr	& 0.03  \\
XGL Runtime Environment		& SUNWxglrt	& 2.07  \\
\hline
\end {tabular}
\caption{Non-critical software options}
\end{center}
\end {table}

The manual pages take up over 15Mb and managers should consider only installing
them on one machine at each site and NFS serving them to all the other
machines.

There are some packages which you might be tempted to de-install but which
you must not:

\begin{itemize}

\item OpenWindows optional fonts (SUNWowpft - 9.58Mb).
These are needed by the answerbook software.

\item Basic networking software (Networking UUCP Utilities).
Although we do not use UUCP networking, the useful \verb+uudecode+ program
is part of this option.

\item The binary compatibility options (SUNWscbcp, SUNWowbcp \& SUNWbcp). These
are needed if you want to run old SunOS executables under Solaris.

\end{itemize}

\subsection {Upgrading Solaris}

It is vital that all sites run the same version of Solaris. Software prepared
at RAL for release on one version may not function on older versions
since Solaris is evolving so rapidly.
Sites should aim to upgrade to new versions of Solaris as soon as
possible after RAL.

SunOS upgrades used to be much more painful than VMS ones.
Many amounted to installing a new system  and hence it was vital that you kept
a copy of files that had  been modified locally and put these back on the
system after the upgrade. Such files included {\tt /etc/hosts}, {\tt
/etc/group} and {\tt /etc/resolv.conf}.

Sun offered an upgrade option from Solaris 2.2 to Solaris 2.3
This upgrade was trivial to perform  and required almost no
post-upgrade work by the site manager to get everything working again.

Alternatively, you can re-install the OS each time but unless you have
kept very careful notes on all changes that you have made to the system,
this is likely to be more time consuming. This is {\bf not} recommended.

\subsection{Unbundled software}

Unbundled software is SUNspeak for layered products.
Under Solaris, all of these products should be installed under \verb+/opt+.
This should mean that they are mostly unaffected by OS upgrades,
although in practice there have been some minor problems.

The following pieces of software have been bought for every Starlink
Sun site and should be installed in the following directories:

\begin{itemize}

\item	Sun Fortran compiler ({\tt /opt/SUNWspro})
\item	Sun C compiler ({\tt /opt/SUNWspro})
\item   Sun SPARCworks software tool set ({\tt /opt/SUNWspro})

\end{itemize}

Other pieces of software have been obtained from the public domain
or from third-party suppliers. These are flagged in the
UNIX\_MANAGEMENT VAXnotes conference and should be installed when announced.
At the time of writing these include:

\begin{itemize}

\item GNU C compiler

\item MIT X11R5 windowing system and additional Motif window manager and
libraries.

\item nu/TPU (TPU emulator)

\item pine mail utility

\item tcsh shell (extended C-shell)

\item jed (EDT-like editor)

\item emacs editor

\end{itemize}

Details of how to obtain and install the above software is given in
the UNIX\_MANAGEMENT VAXnotes conference.

\subsection{Kernel Configuration}

Unlike SunOS 4.x, Ultrix or OSF/1, it is not necessary (or indeed even
possible) to rebuild the kernel on a Solaris machine. All Solaris 2
drivers are packaged as loadable modules and hence it is not necessary to
configure options into the kernel any longer. However, if you do add {\bf any}
new hardware to your system you will need to boot with the \verb+-r+ option
so that the new device will be configured into the appropriate databases.

Administrators can adjust kernel
integer variables by resetting a field in a system file rather than
rebuilding the kernel. Kernel configuration is achieved using the
\verb+nm+ and \verb+sysdef+ commands and the \verb+/etc/system+ file.
Further details can be found in the Sun documentation.

You should be aware that there is no longer an absolute link between
SCSI ID and tape drive number. The first drive seen is always 0, the next
is 1, etc. This can cause some confusion if you are not aware of the
numbering process.

\subsection{Networking}

\subsubsection{Network access to Suns}

Sun systems do not support LAT so it is not possible to log directly on to
a Sun from a terminal on a LAT-based terminal server such as a DECserver 200.
There are a number of solutions to this problem. Different solutions
have been adopted at different sites.

\begin{itemize}

\item Log into a VAX which has UCX loaded and start a telnet session
from the VAX to the Sun. This is clearly an inefficient use of a VAX
login slot but will work at all sites which still have a VAX.

\item X-terminals support TCP/IP so it is possible to directly log into a Sun
with a telnet session.

\item Emulex terminal servers can be upgraded so that they support both LAT
and TCP/IP.

\item It is possible to buy terminal servers that not only provide dual
protocol support (LAT \& TCP/IP) but which can also act as transparent
protocol converter and allow you to log into a Sun ``directly'' from
a LAT terminal server.

\item Set-up a DEC Unix workstation as a LAT to TCP/IP gateway. This is what
has been done at Cambridge to allow there many LAT-only DECserver 200s
access to their TCP/IP only Suns.

\end{itemize}

\subsubsection{Network File System (NFS)}

There have been few changes in NFS between SunOS 4.x and
Solaris. The reader is referred to SSN/7 on general advice on
configuring NFS but it is worth pointing out here that the commands
and files associated with NFS administration have changed (see Table 2).
NFS is also supposed to be more efficient under Solaris.

\begin{table}[htb]
\begin{center}
\begin{tabular}{||l|l||}
\hline

SunOS 4.x & Solaris 2.X\\
\hline
mount -a 		&       mountall     	\\
umount -a		&	umountall	\\
showmount -d		&	dfsmounts	\\
showmount -e		&	dfshares	\\
exportfs		&	share		\\
exportfs -a		&	shareall	\\
exportfs -u		&	unshare		\\
\hline
{\tt /etc/exports}	& {\tt /etc/dfs/dfstab} \\
{\tt /etc/fstab}	& {\tt /etc/vfstab}	\\
{\tt /etc/xtab}		& {\tt /etc/dfs/sharetab} \\
{\tt /etc/mtab}		& {\tt /etc/mnttab}	\\
\hline
\end {tabular}
\caption{Changes in NFS administration commands and files}
\end{center}
\end {table}


Solaris (as did SunOS) offers a utility called the automounter. The mount command is restricted
to the superuser. If we were to allow normal users to use mount, they could
over-mount directories such as {\tt /usr} and {\tt /usr/bin}. This would lead to an
insecure and uncontrolled environment. The automounter can be used to supply
NFS partitions to users on demand. It allows users to mount remote file
systems themselves in a secure and transparent manner. System administrators
will probably want to use the automounter rather than the mount command
wherever practical.

The automounter utility is described in detail in the Sun documentation but
here is an example of simple use of the automounter under Solaris.

\verb+/etc/auto_master+

\begin{verbatim}

# Master map for automounter
#
+auto_master
/-              /etc/auto.misc
/net            -hosts          -nosuid

\end{verbatim}

\verb+/etc/auto.misc+

\begin{verbatim}

/user1          -rw,hard,intr                   rlstar:/user1
/var/mail       -rw,hard,intr                   rlssp1:/var/spool/mail
/rlssp1         -rw,hard,intr                   rlssp1:/rlssp1
/ds_scratch     -rw,hard,intr                   adam2:/ds_scratch

\end{verbatim}

With the above files, the automounter will mount up the file systems
in \verb+/etc/auto.misc+ on demand. The {\tt /net} entry in \verb+/etc/auto_master+
means that any user can access any file system exported globally by any
other machine on the local network! Access is via the path

\verb+/net/<host>/<filesystem>+

For example, at RAL we can access all globally exported files systems
on non-Starlink machine \verb+hardy+ (without any intervention by
managers of either \verb+hardy+ or the Starlink machines) via
the path \verb+/net/hardy+.

Note that the above information can be distributed ``cluster-wide'' via an
NIS map. Thus, changes made to the master file will be propagated to all
NIS clients.

Use of indirect maps is also recommended since they more readily allow for
the movement of disks between machines. More details can be found in the
Solaris manual ``Administering NFS and RFS''.


\subsubsection{Network Information Service (NIS) and (NIS+)}

The Network Information Service (NIS), formerly known as Yellow Pages, is a
distributed database lookup network service. NIS allows single copies of
frequently updated files to be made available to many machines and avoids
having to update the relevant files on every machine whenever there is a
modification. Files affected by the NIS include {\tt /etc/group} and
{\tt /etc/hosts} and also password file information.

With the introduction of Solaris, Sun have replaced NIS with a more
comprehensive (and complex) system called NIS+. Unfortunately, none
of the other Unix operating systems support NIS+ (yet). However, a
Solaris NIS+ server will support NIS clients.

Under Solaris, your can either configure your clients as NIS or NIS+
clients but servers can only be NIS+ servers, i.e. a Sun under Solaris
cannot be a NIS server. Hence, you have two choices:

\begin{enumerate}

\item Set up a Sun running Solaris as a NIS+ server. Solaris clients can be
NIS+ (or NIS) clients and SunOS 4.x (Ultrix, OSF/1, HP-UX etc) clients can be
NIS clients since a NIS+ server will also support NIS clients.

\item Set up a non-Solaris machine as a NIS server. All clients (including
Solaris machines) then become NIS clients.

\end{enumerate}

Option 2 is attractive partly because NIS is simpler than NIS+ but also
because there are less chances of inter-operability problems between different
versions of Unix. However, sites with all Sun or mostly Sun kit will in due
course want to have a NIS+ server as all the machines move to Solaris. Sites
should move over to using NIS+ at a pace that suits their needs.
At RAL, we have a NIS+ server but are still using NIS since the majority of
our machines are non-SUN and hence don't support NIS+ (e.g. Ultrix, OSF/1,
HP-UX, SunOS 4.x).


Sites with mostly DEC Alpha workstations will probably stick with NIS
until DEC adopt NIS+.

\subsubsection{Mail}

Encourage users to use the pine mailer (SUN/169) rather than the terse
\verb+mailx+ program that comes as part of the operating system.

If you are running a NIS/NIS+ cluster, (i.e. you have multiple machines)
then you should set up your system so that
mail sent to any host in the cluster can be read from any other host in that
cluster (as with VAXclusters).

For the purposes of this explanation, it is {\bf not} assumed that all machines
are running Solaris 2.x, even though it complicates the explanation.
In practice this is likely to be the case since not
all of your machines will necessarily have been upgraded to Solaris 2.x yet.
This is certainly the case at RAL where the machine on which the mail is stored
is still running SunOS 4.x. Not all possible options are covered below but
enough examples are given for you to figure out what to do at your site.

To serve the file server's mail directory to each of the clients, you will
need to include an appropriate line in the file controlling NFS file
sharing.


On a SunOS 4.x machine edit {\tt /etc/exports}
\begin{verbatim}

/var/spool/mail -rw=client1:client2

\end{verbatim}

On a Solaris machine edit {\tt /etc/dfs/dfstab}
\begin{verbatim}

share -F nfs -o -rw=client1:client2 /var/mail

\end{verbatim}

You will also need a corresponding line in the relevant file
on each of the clients so that the mail hosts mail directory
will be over-mounted on top of the local mail directory.

On a Solaris 2.X client with a SunOS 4.x mail host edit {\tt /etc/auto.misc} (or
equivalent)
\begin{verbatim}

/var/mail       -rw,hard,intr                   othermachine:/var/spool/mail

\end{verbatim}

On a Solaris 2.X client with a Solaris 2.X mail host edit {\tt /etc/auto.misc} (or
equivalent)
\begin{verbatim}

/var/mail       -rw,hard,intr                   othermachine:/var/mail

\end{verbatim}


The details of setting your Suns up so that they can send mail off site
is site-specific and you should consult your local computing service
(or equivalent) for instructions. In the case of RAL, this procedure
simply involved editing \verb+/etc/mail/sendmail.cf+ and inserting
the RAL mail relay name into the section marked \verb+major relay host+
and making a couple of other minor alterations.
At your site it may more complex. If you have difficulty understanding
the basic requirements of what you have to do, call Dave Terrett
or Chris Clayton for advice.


\subsubsection{VMS/ULTRIX Connection software (UCX)}

This software has been fully described in SSN/7 and that information is not
repeated here.


\section{Security}

Unix is infamous for having poor security. It would appear that, by
default, Suns are relatively insecure compared with VAXes but it is possible to
take a few simple steps to plug up the loopholes and make your system
considerably more secure. At this time C2 level security is not considered
worth implementing. The overheads in terms of both disk space (up to 5MB per
machine per day) and the CPU overhead of the auditing processes do not justify
the gains in security. Secure NFS is also not considered necessary at present.

SSN/7 contains a large section on security. This information is not repeated
here but changes between SunOS 4.x and Solaris are noted below. This section
should be read in conjunction with SSN/7.

The whole of the section on network security in SSN/7 still applies.
Be sure to read this, especially the parts relating to trusted hosts.



\begin{itemize}

\item Encrypted passwords are no longer kept in the world-readable
file {\tt /etc/passwd}. Instead they are stored in the protected
file {\tt /etc/shadow}. This prevents intruders from obtaining
encrypted passwords and attempting to decode them (which is quite
easy to do). You can convert your old password file to the new system
using the {\tt pwconv} command. {\tt pwconv} should also
be used to update password information rather than {\tt vipw} since it
ensures that both the {\tt shadow} and {\tt passwd} files remain consistent.

\item The root password is required when the machine is brought to single-user
mode by default. There is {\bf no way} to disable this security feature.

\item Solaris provides better control on password aging then SunOS 4.x.
New aging controls include:

\begin{itemize}

\item Days since password last changed

\item Maximum number of days the password is valid

\item an expiration date

\end{itemize}

\item Solaris gives an exponential delay on login attempts if a password is
incorrectly entered. Unsuccessful logins are stored in {\tt /var/adm/loginlog}.

\item Solaris includes an Automatic Security Enhancement Tool (ASET)
which system managers can use for automatic security administration.
Further details can be found in the Sun documentation.

\item The file {\tt /etc/ttytab} which was previously used to controls
root access to a machine, no longer exists under Solaris. It has been replaced
by the Service Access Facility (SAF).

\end{itemize}


\section{Operating standards}

\subsection{Adding users}

There are no changes to this section in SSN/7. However, managers should
be aware that Starlink distributes template startup files for new users
which you can modify as appropriate to your site. For more details see
the VAXnotes conference UNIX\_MANAGEMENT.

\subsection{Dumping the file structure}

Site managers should establish a dump (BACKUP in VMSspeak) schedule
to protect against accidental loss of file by user error and hardware failure.
On Sun systems with a large amount of local disk space, it will be necessary
to establish a comprehensive dumping timetable. The following recommendations
should be followed, although the details of how you implement them are
up to your or your local management committee(s). A variety of dump schemes are
described in the Sun documentation.

\begin{itemize}

\item Level 0 dumps should be taken every 4 to 6 weeks. This should be done
in single-user mode, if possible, and preferably out of prime time. However,
taking the system down to single-user mode may not be practical at your site.

\item Incremental dumps should be made on partitions containing user files
several times per week, preferably daily.

\end{itemize}

One can only dump disk partitions and not NFS served file structures. Hence,
to dump a partition on a local disk on a tapeless workstation, one must use
a remote tape device. An entry in {\tt .rhosts} of the remote machine is required
to allow use of its tape drive.

The following are template dump commands that have been tried out by the
author and which can be used for 0 level dumps on the present Starlink Sun
configurations. The various qualifiers are described  in the Sun documentation.
The device names will probably be different on your system, depending how they
have been configured on the SCSI bus.

Note that under Solaris, the dump command has been renamed \verb+ufsdump+.
Unfortunately, they have also included a new command also called \verb+dump+
which dumps selected parts of an object file so beware!. The {\tt restore}
command has also been renamed to \verb+ufsrestore+.

\begin{enumerate}

\item To dump a partition (/dev/rdsk/c0t3d0s6) on a local disk to a local
Exabyte drive.

\begin{verbatim}
/usr/sbin/ufsdump 0ucbsdf 56 51900 4100000 /dev/rmt/0h /dev/rdsk/c0t3d0s6
\end{verbatim}

\item To dump a partition (/dev/rdsk/c0t3d0s6) on a local disk to a remote
Exabyte drive on host rlssp1 (running SunOS 4.x). An entry will have to be
made in {\tt /.rhosts} on rlssp1 to allow this.

\begin{small}
\begin{verbatim}
/usr/sbin/ufsdump 0ucbsdf 56 51900 4100000 rlssp1:/dev/rst1 /dev/rdsk/c0t3d0s6
\end{verbatim}
\end{small}

\item To dump a partition (/dev/rdsk/c0t3d0s6) to a local DAT drive.

\begin{verbatim}
/usr/sbin/ufsdump 0ucbsdf 126 5000 61000  /dev/rmt/1h /dev/rdsk/c0t3d0s
\end{verbatim}

\end{enumerate}

The Solaris {\tt ufsdump} command will copy the whole or part of a partition to
tape depending on the level of backup chosen. A level 0 backup is a complete
copy of a single partition. Incremental backups may be at levels 1 through to
9. The {\tt ufsdump} command will copy all files which have been altered since
the last backup of the current or lower levels. For instance a level 3 backup
copies all files that have been altered since the last backup at level 3,
including any files already backed up at levels 4 through to 9. This system
allows a backup sequence to be devised which provides for many incremental
dumps between full dumps, but allows the incremental tapes to be ``merged'',
which reduces the number of tapes needed to backup and restore the file store.
As time goes on, the lowest incremental dump tapes (usually level 1) will
gradually become more and more full.

\subsection{Accounting}

Accounting under Solaris 2 appears to be similar to that under SunOS 4.x
with the exception of some file name changes.
For further details see SSN/7. Note that at this time, implementing
accounting on your systems is entirely optional.

\subsection{Quotas}

The UNIX disk quota system should be used, where necessary, to control the
amount of disk space used by each user in a given disk partition. The system
works on a hard partition basis, for both local and NFS partitions. The
administrator must set up limits for each user on each partition on which
quotas are to be applied.

Quotas can be controlled using the {\tt edquota} command, as described in the
Sun documentation. Users use the {\tt quota} and {\tt du} commands to get
information about disk usage and limits.

\subsection{Monitoring system performance}

The memory management system of UNIX is very similar to that of VMS. The main
differences that a VAX/VMS system manager will notice are that:

\begin{itemize}

\item There are few parameters that can be tuned under SunOS 4.x
and even less under Solaris.

\item The page and swap files are combined into a single swap space. Additional
swap partitions and files may be added by the system administrator with the
{\tt swap} command ({\tt swap} under Solaris replaces {\tt swapon} under
SunOs 4.x.)

\item One cannot limit the working set of individual processes or those owned
by a given user. Users themselves can limit resource consumptions of their
own processes with the {\tt limit} built--in function of {\tt csh}. System
managers should be aware of this problem since since it is possible for
a single job to grab virtually all of the physical memory on a system and
apparently bring the machine to a halt (as seen by everyone else!).

\end {itemize}

System administrators should monitor the performance of their systems and
become familiar with {\tt vmstat} (report virtual memory statistics -
equivalent to \$ MONITOR),  {\tt iostat}(I/O statistics)
{\tt sar} (system activity reporter) and {\tt ps}
(process status - equivalent to  \$ SHOW SYSTEM). Note that the useful
{\tt pstat} command found on SunOS 4.x systems is no longer available.
Also the {\tt ps} qualifiers under Solaris are different.

\end{document}
