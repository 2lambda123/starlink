\documentstyle{article}
\pagestyle{myheadings}

%------------------------------------------------------------------------------
\newcommand{\stardoccategory}  {Starlink System Note}
\newcommand{\stardocinitials}  {SSN}
\newcommand{\stardocnumber}    {1.2}
\newcommand{\stardocauthors}   {P M Allan}
\newcommand{\stardocdate}      {26 February 1991}
\newcommand{\stardoctitle}     {NOCBS --- Use of CBS mail on satellites}
%------------------------------------------------------------------------------

\newcommand{\stardocname}{\stardocinitials /\stardocnumber}
\markright{\stardocname}
\setlength{\textwidth}{160mm}
\setlength{\textheight}{240mm}
\setlength{\topmargin}{-5mm}
\setlength{\oddsidemargin}{0mm}
\setlength{\evensidemargin}{0mm}
\setlength{\parindent}{0mm}
\setlength{\parskip}{\medskipamount}
\setlength{\unitlength}{1mm}

%------------------------------------------------------------------------------
% Add any \newcommand or \newenvironment commands here

% Make $ a character instead of the maths mode delimiter
\catcode`\$=12

% Use the real underscore character instead of a box
\renewcommand{\_}{{\tt\char'137}}
%------------------------------------------------------------------------------

\begin{document}
\thispagestyle{empty}
SCIENCE \& ENGINEERING RESEARCH COUNCIL \hfill \stardocname\\
RUTHERFORD APPLETON LABORATORY\\
{\large\bf Starlink Project\\}
{\large\bf \stardoccategory\ \stardocnumber}
\begin{flushright}
\stardocauthors\\
\stardocdate
\end{flushright}
\vspace{-4mm}
\rule{\textwidth}{0.5mm}
\vspace{5mm}
\begin{center}
{\Large\bf \stardoctitle}
\end{center}
\vspace{5mm}

\section{Introduction}

Since the introduction of version 5 of VMS and the Coloured Books software
(CBS), it has been possible to send CBS mail from within VMS mail by using an
address of the form {\tt CBS\%nodename::username}. This only works
on cluster nodes that have CBS installed. The usual situation at Starlink sites
that have VAXclusters is that CBS is installed on only one machine in the
cluster. Throughout this document this node will be referred to as `the boot
node', although there is no requirement that this is the case and indeed some
clusters have more than one boot node. Machines that do not have CBS installed
will be referred to as satellites. Again this need not be the case.

There are two problems with the use of CBS mail as it stands. Firstly it can
only be used on the systems which have it installed; this can be very annoying
when trying to reply to mail. Secondly, if an attempt is inadvertently made to
use CBS mail on a satellite, the error message is not very informative. This
node describes how to set up a cluster so that CBS mail can be used indirectly
from satellites and a utility called NOCBS that produces a more helpful error
message if a user attempts to use CBS mail directly on a satellite. The two
items are completely separate. They just happen to address different aspects of
the same problem.

\section{Use of CBS mail on a satellite}
\label{cbs}

While it is currently not possible to send mail to {\tt CBS\%nodename::username}
from a satellite, it is possible to use {\tt CBS\%} indirectly. To enable the
use of CBS mail on a satellite, you must do the following:

\begin{itemize}
\item Find out which account on your cluster handles incoming mail. This is
probably either DECNET or MAIL$SERVER. You can do this by typing
\begin{quote}
$ MC NCP SHOW OBJECT MAIL CHAR
\end{quote}
If the object has a user id associated with it, then that is the username that
will be used when fielding incoming mail messages. If there is not a user id,
then type
\begin{quote}
$ MC NCP SHOW EXEC CHAR
\end{quote}
and look for the non privileged user id. If neither of these are present, then
mail will not work.

\item Set the system logical name {\tt MAIL$SYSTEM\_FLAGS} to an even number on
all of the satellites. This enables intra-cluster DECnet for mail messages. It
can be set to an even number on the boot node as well, although this is neither
required nor causes problems if not done (but see section~\ref{reply}). The
effect of {\tt MAIL$SYSTEM\_FLAGS} is described in the VMS Mail Utility Manual
(pages 14--15 for the VMS 5.0 manual).

\item Make sure that the username that handles mail has a MAIL.MAI file on the
boot node. If there is not one, send a mail message to
{\tt <boot node>::<username>} to create one.

\item Ensure that the user that handles incoming DECnet mail (probably
MAIL\$SERVER or DECNET) is authorized by NRS to send CBS mail to all JANET
sites. Since this actually allows anyone to send mail to all JANET sites, the
best way of doing this is to authorize the default user to send mail to all
JANET sites. To do this, type
\begin{quote}{\tt
$ RUN NET$DIR:NETAUTH\\
ADD USER [] UK.AC.*
}
\end{quote}

\end{itemize}

You should now be able to send CBS mail from a satellite by sending mail to
{\tt <boot node>::CBS\%<NRS node name>::<username>}, e.g.\ to send a mail
message from a RAL satellite to Manchester, one can send to
{\tt RLSTAR::CBS\%MAN.AST.STAR::DSB}.

\subsection{Problems with outgoing mail}

Sending mail to {\tt RLSTAR::CBS\%MAN.AST.STAR::DSB} (for example) will cause a
network login on RLSTAR and the process will be owned by the username that
handles incoming mail. The problem is that it is this process that needs to
forward the mail using CBS. If this account does not have authorization to send
the CBS mail to the site in question, then the mail cannot be sent. Even worse,
the user on the satellite is not told that there is a problem, but the process
just hangs.
It is quite all right for every user to have access to NRS site names
beginning with UK.AC as these are on JANET and it costs nothing to access these
sites. This is set up as described in the section above. The real problem is
with mailing directly to overseas sites. Authorizing a particular user to
access a particular site will not allow them to do so from a satellite. To do
so, it is necessary to authorize the account that handles incoming mail to send
CBS mail to the remote site. This would give access to anyone, so should not be
done. The user will have to remember to use the boot node in such cases.
However, if a user sends mail explicitly through one of the gateways in the UK,
then there will be no problem.

\subsection{Replying to mail sent from satellites}
\label{reply}

All things considered, it is better to send a new mail message rather than try
replying to messages sent from satellites, however, if you want to do this,
read on.

If user PMA sends a CBS mail message from a satellite of RLSTAR, then the mail
will appear to come from {\tt
CBS\%UK.AC.\-RUTHERFORD.\-STARLINK::"RLVAD::PMA"}. This mail message can be
replied to provided that the originating site has defined the logical name {\tt
FTP$MAIL\_DECNET} on the boot node. This logical name gives the DECnet nodes
that CBS will forward mail on to. This should definitely be set to the cluster
alias and for good measure, it is worth including the individual node names of
your cluster as well.\\E.g.
\begin{quote}{\tt
$ DEFINE/SYSTEM FTP$MAIL\_DECNET RLVAD,RLSTAR,RLSSD1,RLSVS1,RLSVS2
}
\end{quote}

If this is not done then although the mail can be replied to, it will be
bounced back to the user with a message indicating that the DECnet node was not
accessible from CBS mail. Also, since the mail message has come from {\tt
CBS\%...} then the reply must be done on the boot node.

When the reply is received by the originating author, it will appear to come
from somewhere like {\tt RLVAD::CBS\%\-MANCHESTER.\-ASTRONOMY.\-STARLINK::DSB}.
You will be able to reply to this message if you are on the boot node and have
set {\tt MAIL$SYSTEM\_FLAGS} to an odd number on the boot node. You may or may
not be able to reply to this if you are on a satellite or have set {\tt
MAIL$SYSTEM\_FLAGS} to an even number on the boot node. What is even worse is
that in this situation, you may be able to reply sometimes and not
others. This is because this second reply is being sent to the cluster alias.
If several machines in the cluster field mail then the message may or may not
get to the boot node.

\section{NOCBS}
\label{nocbs}

Even though it is now possible to send CBS mail from a satellite, it is still
not possible to send it directly to {\tt CBS\%<nodename>::<username>}.

If an attempt is made to do this, an error message of the form

\begin{verbatim}
%MAIL-E-ERRACTRNS, error activating transport CBS
%LIB-E-ACTIMAGE, error activating image  $1$DUA0:[SYS0.SYSCOMMON.][SYSLIB]CBS_MAILSHR.EXE;
-RMS-R-FNF, file not found
\end{verbatim}

is produced. This is not very informative and causes much confusion. The purpose
of NOCBS is to produce a meaningful error message. Since it is now possible to
indirectly send CBS mail from a satellite, the error message produced by NOCBS
tells you how to do this rather than suggesting that you log in to the boot
node as it did in an earlier version of the software.

\subsection{The files}

The system consists of 5 files in the directory pointed to by the logical name
{\tt NOCBS\_DIR}.

\begin{tabbing}
\hspace{5mm}\=BUILD.COM\hspace{35mm}\=A procedure to build the system\\
\>LOGICALS.COM         \>Defines the logical names used by the system\\
\>NOCBS\_MAILSHR.EXE   \>The shareable image containing the error routine\\
\>NOCBS\_MAILSHR.MAR   \>The macro subroutine called from VMS MAIL\\
\>NOCBS\_MESSAGES.MSG  \>
The source file for the error message\\
\end{tabbing}

The file NOCBS\_MESSAGES.MSG contains the error text that is printed.
The default error message is

\begin{verbatim}
     To use CBS mail on this system, send mail to
     <boot node>::CBS%<NRS node name>::<username>
\end{verbatim}

If you wish to tailor this message to be more specific to your site, then edit
the message source file NOCBS\_MESSAGES.MSG and run BUILD.COM to rebuild the
shareable image NOCBS\_MAILSHR.EXE.

The file LOGICALS.COM defines the logical names used by this program. They
should be defined at boot time on the machines on which this is run, i.e.\ those
machines that {\em do not} have CBS available. The command procedure
LOGICALS.COM can be executed on all machines in the cluster as it checks if CBS
is available by testing whether or not the logical name NET$DIR is defined. This
check will of course not work if you define NET$DIR on machines that do not have
CBS available. Furthermore, you should of course run LOGICALS.COM after you have
started CBS on the boot node and not before.

\subsection{How to install the system}

The only things that are required for a functioning system are the shareable
image NOCBS\_MAILSHR.EXE and logical names of the form xxx\_MAILSHR, where xxx
is the transport being used. As released, only the logical name CBS\_MAILSHR is
defined. If you run Philip Taylor's JANET\_MAILSHR you will need to define
other logical names such as JANET\_MAILSHR and EARN\_MAILSHR.

\section{Acknowledgments}

The NOCBS software was developed with the assistance of Philip Taylor of Royal
Holloway and Bedford New College.

\end{document}
