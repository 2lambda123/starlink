\documentstyle{article} 
\pagestyle{myheadings}

%------------------------------------------------------------------------------
\newcommand{\stardoccategory}  {Starlink System Note}
\newcommand{\stardocinitials}  {SSN}
\newcommand{\stardocnumber}    {53.1}
\newcommand{\stardocauthors}   {A P Lotts}
\newcommand{\stardocdate}      {10 June 1989}
\newcommand{\stardoctitle}     {UNIBUS and Q-BUS Configuration}
%------------------------------------------------------------------------------

\newcommand{\stardocname}{\stardocinitials /\stardocnumber}
\renewcommand{\_}{{\tt\char'137}}
\markright{\stardocname}
\setlength{\textwidth}{160mm}
\setlength{\textheight}{240mm}
\setlength{\topmargin}{-5mm}
\setlength{\oddsidemargin}{0mm}
\setlength{\evensidemargin}{0mm}
\setlength{\parindent}{0mm}
\setlength{\parskip}{\medskipamount}
\setlength{\unitlength}{1mm}

%------------------------------------------------------------------------------
% Add any \newcommand or \newenvironment commands here
%------------------------------------------------------------------------------

\begin{document}
\thispagestyle{empty}
SCIENCE \& ENGINEERING RESEARCH COUNCIL \hfill \stardocname\\
RUTHERFORD APPLETON LABORATORY\\
{\large\bf Starlink Project\\}
{\large\bf \stardoccategory\ \stardocnumber}
\begin{flushright}
\stardocauthors\\
\stardocdate
\end{flushright}
\vspace{-4mm}
\rule{\textwidth}{0.5mm}
\vspace{5mm}
\begin{center}
{\Large\bf \stardoctitle}
\end{center}
\vspace{5mm}

%------------------------------------------------------------------------------
%  Add this part if you want a table of contents
\setlength{\parskip}{0mm}
\tableofcontents
\setlength{\parskip}{\medskipamount}
\markright{\stardocname}
%------------------------------------------------------------------------------

\section{INTRODUCTION}

This note is intended to assist managers with Q-BUS and UNIBUS  configuration.
It is not a complete description.
Please send any corrections or suggested additions to the Starlink Librarian.

{\bf WARNING} Make sure that the current  configuration 
both in terms of addresses and vectors and physical placement
 of all boards are carefully noted before any changes are made.
Also note carefully any changes that are made so that there is a reasonable
chance of getting back to the starting position without mistakes.
 Read all the relevant sections of this and any other information
 sources well before starting any configuration changes.

Also note that permission should be obtained from whoever is maintaining
the system (and any boards if maintained by third parties) before attempting
configuration changes.
Failure to do this may invalidate warranty or maintenance agreements.


\section{PHYSICAL PLACEMENT}
This section will only deal with bus integrity and cabling
 problems and {\bf not} with placement to avoid bus interaction problems and
 efficiency.

	
\subsection{UNIBUS}
\subsubsection{Cabling}
Problems can arise with some boards when connecting and routing the
 cables from the board out of the UNIBUS enclosure to the user device.
 Beware especially of boards that require top entry.
These can conflict with  enclosure itself and with cables already in place.
Take a good look at the cable routing of any slots to be changed.

\subsubsection{Backplane Integrity}
This section is mainly concerned with the bus grant cards and the
 problems of NPG.
The Non Processor Grant is a signal used for DMA transfers.
DMA stands for Direct Memory Access and means that the device communicate
directly with memory without CPU intervention.
Common DMA devices are disc and tape drives, Image display systems and some
others.

A UNIBUS expects to see all of its backplane lines connected all
 the way to the terminator.
All the address and data lines are `shared' by every board and are no problem.
The problems arise with the bus grant lines.
These are the BR and NPG lines and control the bus arbitration -- i.e. which
 device gets to talk to the CPU next.
Because of the way the grant lines are
 used they all enter each board on one backplane connector and leave by another.
 Thus a `board' is needed in every UNIBUS backplane slot.

The backplane is divided up into 6 areas per slot.
The area which is
 third from the bottom has the BR grants while the next one up has the NPG.
 A `new' UNIBUS slot has a wire wrap to jumper the NPG and if no board is
 inserted will only need a simple `bus grant' card (e.g.\ G727A).
If the slot
 is used by a NON DMA device the wire wrap jumper is normally left in place.
 However if a DMA board is to be used the wire wrap must be removed.
If the device is later removed and either the slot is to be left empty or
a non DMA board is to be inserted, problems can arise.
If left empty, the simple G727A is not enough and either the wire wrap jumper
must be replaced or a `double grant' card such as the G7273 will be needed.
The G7273 is much  preferred because it can be removed or inserted
without removing surrounding boards and also does not entail wire wraps.
Try to get at least one from your DEC engineer whenever you get the chance!

It is believed some of the modern non DMA boards will pass the NPG but
some will not.

UNIBUS boards come in two sizes, quad and hex.
Quad boards are placed in the bottom 4 areas while a hex fills the slot.
No jumpers or changes are required in the top two areas nor in the bottom two.

\subsubsection{Choosing a slot}

Clearly if removing one board and inserting another the
first  choice will be a simple replacement.
However, ignoring any other considerations, the boards must be checked to 
see if both are DMA or not.
If both are DMA there is no problem.
If both not DMA you will also be alright if both  pass or do not pass the NPG.
 Problems may arise if mixed DMA and non DMA or both non DMA but mixed NPG 
jumpering.
Suppliers should be able to help with this information.
Failing that you could (with care) simply see what happens when you access
the bus.
All devices will configure properly  but DMA devices may not work if the 
NPG is not contiguous in the correct way.
Be especially  careful if disc drives are involved.
It is always a good idea to set {\bf \it all}  drives to write protected when
you are not sure about a configuration.
Note that VMS will run in a limited way with even the system disc write
protected.
Also remember to check all devices on the bus after configuration changes.

Another possible solution will be use use a different spare slot 
or in some other way re-arrange things -- remembering that the idea is to 
get contiguous grant lines.
Major changes are {\bf \it not} recommended without an engineer since
machines should really only be re-configured by DEC (or whoever is maintaining
the system)


\subsection{Q-BUS}

\subsubsection{Cabling}

Other than the cramped working space it is believed that there are few
 significant problems with cables and cable routing with Q-BUS machines.
On 3000 series machines however, problems may arise because of the 
requirement for the baffles on the front of the boards.
These can interfere with cables unless the correct baffles are available.

\subsubsection{Backplane Integrity}

The Q-BUS is much better than a UNIBUS from this point of view.
 The only restriction is that there should be no empty areas between the
 CPU and the last board or grant board.
There is no need for grant boards after the last board (although note that
some Q-BUSes are serpentine and the way the signals are routed is not
obvious).

The only complications arise because the backplane is divided into areas
AB and CD for each slot.
Each normal slot will take either 1 quad size board or two `dual' height
boards.
The word normal has been used because in some Q-BUSes the first slots 
(on the right ?) are used for the CPU.
In fact the first slot normally {\bf is} the CPU, 2 and possibly 3 are
normally memory.
 DEC in fact recommend that slot 3 is reserved for memory and although 
 it is possible to use the AB position for a normal board it should, if
 not used for memory, have a standard M9047 grant card in the AB position.
 
 After these first 3 slots, the remaining are sometimes connected
 in a serpentine fashion.
 This in practice means that unless you can work 
 out the pattern (see the CPU hardware manual), the safest thing is to
make sure that the last quad slot is
 completely filled by including a grant card if necessary.

\subsubsection{Choosing a slot}

The only rule that seems important (other than those detailed above)
 is to have an Ethernet interface (Deqna or Delqa) as close to the CPU
as possible.
This seems to be over-ridden by a DZQ11 in at least one DEC configuration
guideline.

There will be other rules but they seem to change as new devices are used 
and will not be dealt with in this section.

\section{ADDRESSING}

Addressing can be a very confusing on both UNIBUS and Q-BUS.
As far as is known, these notes will apply to both, although the actual values 
of some addresses and vectors differ from the UNIBUS tables supplied for 
VMS when using a Q-BUS -- despite the appearance of Q-BUS devices in the 
tables.

Please note that the address of a device is totally independent of its
physical placement within any given Q-BUS or UNIBUS.

A list of terms used in addressing appears in Appendix B

\subsection{Background}

Each time VMS boots, it needs to locate the devices available on the system,
load the appropriate device driver, and then associate (CONNECT) the device
with the driver.
This is done at boot time using the SYSGEN utility.
Note that the boot device is made available very early in the boot.

The rest of the devices are made available by either an automatic method
or by commands entered into startup files by the system manager.

It may be useful to describe what happens, when during the boot it occurs
and how the manager can control the actions.

\begin{sloppypar}
The intended place for a manager to control the device configuration is the 
file SYS\$MANAGER:SYCONFIG.COM.
Here a manager can configure devices explicitly and, as may be useful for
debugging configurations, switch off the normal automatic configuration
of devices by the use of the following command.
\end{sloppypar}
\begin{quote}
\begin{verbatim}
$ STARTUP$AUTOCONFIGURE_ALL == 0
\end{verbatim}
\end{quote}

\begin{sloppypar}
However, it is sometimes necessary to place device configuration commands
in either SYS\$SPECIFIC:[SYSEXE]SATELLITE\_PAGE.COM or SYPAGSWPFILES.COM.
These are called before SYCONFIG.COM and so if page and swap files are to
be installed on discs that are not yet available, configuration and mount
commands will be needed in these files.
\end{sloppypar}

Note that discs that are served {\bf\it to} this machine are available without
configuration but would need to be mounted.

The automatic configuration that occurs after any SYCONFIG.COM has been 
executed is achieved by the command SYSGEN AUTOCONFIGURE ALL.
This causes the SYSGEN programme to search through the bus address space,
looking for addresses that respond.
If an address responds, assumptions are made depending on the actual address
involved and an attempt is made to load the driver and connect the device.

The assumptions are controlled by tables used by SYSGEN.
DEC documents these tables in the SYSGEN UTILITY manual and sometimes in
VMS Release Notes.

Please also note that every documented table in recent years had errors
--  the only table that really counts is the one used by the current version
of the operating system!

Full details of how is is believed SYSGEN AUTOCONFIGURE actually works
  are contained in Appendix A -- hopefully it will not need to be consulted.

\subsection{Addressing details}

Before you start to change anything, you need to obtain your
 current configuration. The basic command for this is:-
\begin{quote}
\begin{verbatim}
$ RUN SYS$SYSTEM:SYSGEN
SHOW /CONFIG
\end{verbatim}
\end{quote}

This should be done when your system is fully configured -- i.e.
when ready for normal use (or even during normal use).

{\bf WARNING} -- all the SYSGEN SHOW commands are safe during normal
 operations EXCEPT for SHOW/UNIBUS (is also used for Q-BUS) which should
 never be used on a system during normal processing.
While in general even this does no damage, the possibility is quite real.

Two variations of SHOW /CONFIG could be useful -- a /OUTPUT=file-name
 will put the output into a file for later reference and /ADAP=UB0 or UB1
 will show just 1 UNIBUS.

Below is an example of the output for a 750.
\begin{quote}
\begin{verbatim}

	System CSR and Vectors on 17-JUL-1987 10:47:28.95
 Name: DRA  Units: 1  Nexus:4  (MBA) 
 Name: IDA  Units: 1  Nexus:8  (UBA) CSR: 764064  Vector1: 274  Vector2: 000
 Name: MMA  Units: 1  Nexus:8  (UBA) CSR: 772440  Vector1: 224  Vector2: 000
 Name: MTA  Units: 1  Nexus:5  (MBA) 
 Name: DRC  Units: 4  Nexus:6  (MBA) 
 Name: LPA  Units: 1  Nexus:8  (UBA) CSR: 777514  Vector1: 200  Vector2: 000
 Name: TTA  Units: 8  Nexus:8  (UBA) CSR: 760100  Vector1: 300  Vector2: 304
 Name: TTB  Units: 8  Nexus:8  (UBA) CSR: 760110  Vector1: 310  Vector2: 314
 Name: XGA  Units: 1  Nexus:8  (UBA) CSR: 760404  Vector1: 320  Vector2: 324
 Name: TXA  Units: 8  Nexus:8  (UBA) CSR: 760414  Vector1: 340  Vector2: 344

\end{verbatim}
\end{quote}

It is necessary to identify which are DEC devices and which are 3rd party.
Your  site management guide should help (the configuration section).
Once it is known what devices there are and what there will be, SYSGEN can 
be used give the expected configuration.


If there are problems associating the physical devices with the addresses
in use, look at the SHOW /CONFIG output for the current
 system and find the name: field (without the trailing A B C) under the
 DEVICE NAME column in a SYSGEN DEVICE table.
 This should provide the full  device name in the Controller name column 
 (e.g.\ an LPA --$>$ LP --$>$ LP11).
These controller names are the ones that will be needed.

\subsection{New Configuration}

Having found all the names of DEC devices on the new configuration, it may be 
necessary to consult the SYSGEN  CONFIGURE documentation in the VMS
documentation of the utility.
This shows how to enter multiple devices on the same controller and
how to handle multiple UNIBUSes.

To investigate new configurations, run SYSGEN, enter CONFIGURE and 
then enter the {\bf current} set to see if it agrees with the current
configuration.
Then start again with a new CONFIGURE and enter the devices in the new
configuration.
Floating vectors or addresses are indicated with an *.

If this is a pure DEC system, it should have been completely described
and the rest of this section can be ignored.

If this is not a pure DEC or DEC compatible system you need to 
check that there are no conflicts between your new DEC addresses and the 
non DEC ones.
This may not be quite as simple as it sounds.
Check all the devices in the sysgen tables for the number of vectors and
number of registers since you need to leave 4 bytes of space for each
vector and probably 4 bytes for each register with an extra 8 bytes at
the end of each device.
The worst case is probably a DMF 32 which has 8 vectors and 16
registers.
Also remember that you are working in {\bf OCTAL} !
If there are conflicts then  something will have to be moved -- probably
the non DEC things are safest.
Note that the vectors and addresses are not tied
together and for fixed ones  can be changed without problems. 

Be {\bf very} careful about non DEC devices that are said to be 
DEC compatible.
For example the UNIBUS MBS DR-11W equivalent would be
expected  to have addresses of a DR-11W -- however, when the actual
addresses it uses are those of a DR-11B.
The Q-BUS version actually does use the addresses of a DEC DR-11W !


\subsection{Preparing to Set addresses and vectors}

If addresses on boards need to be changed, the manuals must be found for them. 
It is probably wise to find all such manuals in any case just in
 case they are needed. Make sure that all the tools that may be needed
are available -- e.g.\  anti static wrist strap, wire wrap tools, soldering
iron etc. 

\subsection{Check software}

The drivers for DEC or DEC compatible devices that say SUPPORTED YES
 in the sysgen tables are loaded automatically when the machine autoconfigures.
 If the driver is absent from SYS\$SYSTEM (VMS 4) or SYS\$LOADABLE\_IMAGES 
(VMS 5) there will be an error reported.
 If SUPPORTED  NO then
 the address and vector space is allocated but no driver loaded and no error
 reported.

 Ensure that all the correct drivers in the correct places
 and that it is known which files are used to configure any that are not loaded
 by sysgen. Inspect to SYS\$MANAGER:SYCONFIG.COM and your startup files -- not
 forgetting things like PSIxxxLOAD that also may have addresses.

 Remember that all the logical names above are normally search lists.

\subsection{Re-configuring}

Should now be ready to re-configure the machine.
The notes in SSN 47 about memory replacement give a guide to how to handle
boards using  grounding wrist straps etc..
Also note that if using a currently  empty UNIBUS slot there will be a grant
card in it. It will need to be removed which may well entail removing other
boards to get enough working space. 

Always allow plenty of time for these operations and do not rush !

Once any addresses have been reset and the boards inserted,
try to boot the machine again but you want to write protect disks and 
generally do a `no users' type of boot.
Remember that even the normal autoconfigure can be prevented either as 
described above or by invoking a minimum startup -- that is setting 
STARTUP\_P1 to ``MIN".
Useful things  to do in these conditions include using
SYSGEN SHOW /UNIBUS to see what  bus addresses respond.
Note the output shows all the addresses that respond not just device CSR
addresses.
Try autoconfiguring  by hand using SYSGEN AUTOCONFIG  with perhaps ALL or for
selected devices.
See  the SYSGEN manual.
All of this will just check that the addresses are correct.
It will NOT check that you have all the grants properly installed.
The  only way to fully check is to actually use all the
devices.
Note that  incorrect jumpers can affect devices other than the one
or slot in error. 

\appendix
\section{Sysgen Autoconfigure}

Or how SYSGEN AUTOCONFIGURE is believed to work.

\begin{itemize}

\item ignores some addresses completely - the so called NON-Digital or USER
addresses.

\item It works purely on the Table order given in the SYSGEN manual
but often updated by VMS release notes but remember that the only table
that really counts is the one inside SYSGEN. Also the printed tables are
sometimes in error.

\item When it finds a device at an address, looks for the
corresponding driver in SYS\$SYSTEM (VMS 4) or SYS\$LOADABLE\_IMAGES (VMS 5)
-- both logical names are usually search lists.
If driver is not there and
and device is {\bf supported}, it gives an error.
If device is {\bf\it not} supported, allocates CSR and vector space and
gives no error.

\item Each device has a known first try address - even the `floating'
ones. The ones that float will have changing addresses. If there
is a preceding `floating' device seen, any floating device will
use its second address as first try unless the previous floating
address defined is too close when it will move on to the next etc.

\item The floating addresses start at 760010 (a DJ11) and continue to
about 761000 (multiple AAV11D s !)

\item Floating vectors are assigned in order in which floating vectors
are seen starting at 300. Note that some devices with fixed csr can
have floating vectors.

\item Some devices have multiple entries in the SYSGEN tables with
the first line with fixed CSR and a float entry (s) afterwards.
As one would guess, the first device of this type is at the fixed
address but a second would take a corresponding float address.
\end{itemize}

\subsection{Notes}

\begin{itemize}
\item Some devices appear {\bf\it twice} in the table at widely spaced
intervals - e.g.\ UDA occurs early as a fixed device and then
a couple of pages on as a float. Presumably this means that
the first is fixed but a second is FLOAT and is separated because
of floating address ordering. Moral - make sure you start looking
for a device {\it starting at the beginning of the sysgen tables} !

Also note that `first' and `second' mean, for these purposes, the order
in which SYSGEN locates devices as it searches the address space in its
pre-defined fashion.
It has nothing to do with physical placement and not necessarily anything
to do with order in address space.
Suppose it is possible for some of the very clever controllers whose address
and vectors are loaded dynamically may break this rule but it is thought 
that the identity problem this may produce is resolved by straps on the board 
which define the device order.

\item Some devices (for example UDA) actually work in an even more
complicated way.
Their addresses and vectors can be set by AUTOGEN itself!
These would usually have no physical way to set the address and vector 
although they may have straps to define if it is the first or second
similar device.
Under some circumstances, it may appear that the device has not actually 
configured but it will do later.
(I believe this can occur when autoconfigure has been suppressed and a
SHOW/UNIBUS is performed but this is open to correction).

\item The UNIBUS MBS DR11-Ws for the Digisolve IKON in fact have addresses
that correspond to a DR11-B !

\item Full details can be found in the Guide to Device drivers starting
at section 14.3 (VMS 4) or in Device Support Manual section 15.4 (VMS 5).
Other related sections may also be useful.
    
\end{itemize}

\section{Terms in addressing}

\begin{itemize}
\item{CSR} is the device Control and Status Register and is what is
 normally called the device address.
 Note however that multifunction devices like the DMF-32 can actually appear
 to have more than one CSR.
 In the example above, XGA and TXA are actually two of the DMF-32
 devices.
 The actual base address of the DMF-32 in the case above is 760400
 which does not appear in the list !

\item{FIXED address} This is an address that does not change because of
 other devices currently configured in the machine.
They are addresses outside the ranges described below.

\item{FLOATING address} This is an address that depends on what other
 floating address devices have been configured when SYSGEN first sees them.
 Floating addresses start at 760010 and continue until approximately
 761000.

\item{USER address} These are addresses that SYSGEN totally ignores when it
 configures the machine automatically.
They are approximately 764100 through to 767776.
These are often used by third party devices and, provided the device has
fixed address and vector and these are well seperated from other 
addresses and vectors, they should take no part in autoconfiguration.

\item{FIXED vector} These are vectors that do not depend on other devices.

\item{FLOATING vector} These are vectors that depend on what other floating
vectors have already been seen.
\end{itemize}
\end{document}

