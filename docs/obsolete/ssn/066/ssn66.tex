\documentstyle{article}
\pagestyle{myheadings}

%------------------------------------------------------------------------------
\newcommand{\stardoccategory}  {Starlink System Note}
\newcommand{\stardocinitials}  {SSN}
\newcommand{\stardocnumber}    {66.2}
\newcommand{\stardocauthors}   {D L Terrett}
\newcommand{\stardocdate}      {27 June 1992}
\newcommand{\stardoctitle}     {Starlink Software organization on UNIX}
%------------------------------------------------------------------------------

\newcommand{\stardocname}{\stardocinitials /\stardocnumber}
\renewcommand{\_}{{\tt\char'137}}     % re-centres the underscore
\markright{\stardocname}
\setlength{\textwidth}{160mm}
\setlength{\textheight}{240mm}
\setlength{\topmargin}{-5mm}
\setlength{\oddsidemargin}{0mm}
\setlength{\evensidemargin}{0mm}
\setlength{\parindent}{0mm}
\setlength{\parskip}{\medskipamount}
\setlength{\unitlength}{1mm}

%------------------------------------------------------------------------------
% Add any \newcommand or \newenvironment commands here
%------------------------------------------------------------------------------

\begin{document}
\thispagestyle{empty}
SCIENCE \& ENGINEERING RESEARCH COUNCIL \hfill \stardocname\\
RUTHERFORD APPLETON LABORATORY\\
{\large\bf Starlink Project\\}
{\large\bf \stardoccategory\ \stardocnumber}
\begin{flushright}
\stardocauthors\\
\stardocdate
\end{flushright}
\vspace{-4mm}
\rule{\textwidth}{0.5mm}
\vspace{5mm}
\begin{center}
{\Large\bf \stardoctitle}
\end{center}
\vspace{5mm}

%------------------------------------------------------------------------------
%  Add this part if you want a table of contents
%  \setlength{\parskip}{0mm}
%  \tableofcontents
%  \setlength{\parskip}{\medskipamount}
%  \markright{\stardocname}
%------------------------------------------------------------------------------

This note describes the organization of the Unix Starlink Software
Collection (USSC). The reader is assumed to be familiar with the
directory structure used for the Starlink Software Collection (SSC)
on VAX/VMS systems and with the UNIX file
system. This is not intended to be a user document.

\section{Directory Structure}
The directory structure should:
\begin{enumerate}
\item Enable easy installation of the software.
\item Enable easy installation of updates.
\item Make the software easy to use.
\item Make development of the software easy.
\item Be as least confusing as possible for users familiar with the SSC on
VAX/VMS.
\end{enumerate}

The last of these criteria is best satisfied by a structure that is identical
to that on VAX/VMS. However, there is no facility equivalent to VMS logical
names on UNIX so having programs and other files that users regularly access
spread throughout the file system is not compatible with 3 and~4. The VAX/VMS
structure has been strongly influenced by 1 and~2 and the same considerations
apply equally well to a UNIX system.

\subsection{The users' view}
In general the location on a UNIX system of most files in the SSC can be
obtained by taking the VMS file specification, changing {\tt STARDISK:} to {\tt
/star} and converting the directory specification to the UNIX equivalent in
lower case. For example:
\begin{quote}\tt
STARDISK:[STARLINK.LIB.GNS]GKSDEVICES.TXT
\end{quote}
becomes
\begin{quote}\tt
/star/starlink/lib/gns/gksdevices.txt
\end{quote}

However, there is not an exact one--to--one correspondence between files on VMS
and files on UNIX, and in any case file names may be different in
order to conform to normal UNIX practice. For example, Fortran source files
should have extension {\tt .f} instead of {\tt .for} and the object code
library {\tt GNS.OLB} would be
called {\tt libgns.a}. Furthermore, files which are normally referred to by a
logical name will have a name that matches the logical name (converted to lower
case) rather than the VMS file name. For example, the VMS file
{\tt [STARLINK.LIB.GNS]GNS\_ERR.FOR} which has the logical name {\tt GNS\_ERR}
becomes {\tt /star/starlink/lib/gns/gns\_err}.

However, for files accessed by users, such as executable programs
and Fortran include files, this is not acceptable because of the long and hard
to remember path names involved and because of the disruption caused by any
re-organisation of the directory structure. Therefore, these files will be
either copied or moved to the following directories:

{\renewcommand{\arraystretch}{1.5}
\[\begin{tabular}{ll}
\tt/star/bin & Executable programs and shell scripts\\
\tt/star/lib & Subroutine libraries\\
\tt/star/etc & Miscellaneous data and configuration files\\
\tt/star/include & Fortran include files and C header files\\
\tt/star/man & Manual pages\\
\tt/star/bin/examples & Example programs
\end{tabular}\]}


There is also a directory {\tt /star/local} which contains similar
subdirectories to those directories listed above
(e.g. {\tt /star/local/bin,
/star/local/lib}, etc). This directory tree is provided as an empty
tree structure into which sites can install local software.
It will not be used by
the Starlink Software Librarian.

On UNIX, programs and shell scripts can be run simply by typing the
appropriate filename
provided that the file can be found in a directory in
the current path.  Most executables in
the SSC will be moved to {\tt /star/bin} so that they are available to a user
by simply adding {\tt /star/bin} to the current path. Some packages with a
large number of commands may have their executable files in a subdirectory of
{\tt/star/bin} along with a package initialization shell script.
This initialization script must be `sourced' to define aliases for
individual packages via which programs can be run (see SUN/118 for more
details).

Some packages may have subdirectories of others of these directories
if the number of files warrants this; for example, the 23 PGPLOT example programs
are found in {\tt/star/bin/examples/pgplot}. This avoids directories containing
so many files that it becomes hard to locate files when you are unsure of the
exact name. However, the number of subdirectories has to be kept under control
in order to avoid the opposite problem where you know the name of the file
but are unable to find it because the directory structure is too complicated.

Object module libraries will be moved to {\tt /star/lib} so that programs that
use Starlink libraries can be linked on Suns with {\tt -l{\em xxx}} where
the name of the library is {\tt lib{\em xxx}.a}, once the Starlink login
script ({\tt /star/etc/login}) has been run. On DECstations it is necessary
to give the name of the directory with the qualifier {\tt -L/star/lib}.

\subsection{The system managers' view}
The use of symbolic links allows the actual location of files to be different
from their apparent location as seen by the users; this gives the system
manager considerable freedom when deciding where to place files. However, some
rules have to be followed so that software installations and updates will
work.

Since the root partition gets replaced by operating system upgrades
{\tt /star} should be
either the mount point of a disk partition or nfs file system, or a symbolic
link to a directory somewhere else. Directories in {\tt /star} can be any of
the above or real directories (or a mixture) but everything below these
``package'' directories must be real directory trees so that a complete package
can be easily restored from a tar file.

On a server which serves several machines of different types, each machine type
requires a different set of binary files. This is achieved by making {\tt
/star/bin}, {\tt /star/lib}, {\tt /star/etc} and {\tt /star/bin/examples}
either symbolic links to or the mount points  of nfs file systems which
contain the binary files for the appropriate machine architecture.

\section{Software Distribution}
The actual distribution of files is done in much the same way as on VMS; {\tt
tar} can do the job that {\tt backup} does on VMS. However, UNIX systems are
not binary compatible in the way that VAXs are -- even when dealing with
different models from the same manufacturer. Therefore the distribution
mechanism must allow binary files to be built from the source files on the
target machine. Fortunately the {\bf make} utility is ideal for this purpose.

This means that every piece of software must have a well defined and consistent
mechanism for re-creating the binary files. The mechanism will be to {\bf cd}
to the package top level directory and execute {\bf make}. Every package
directory will therefore contain a make file called {\tt makefile}.

This make file must contain four standard targets (unless not applicable)
that do the following:
\begin{description}

\item[build] Builds the package from its source files but does not install it.
The {\bf build} option target should be the first target in the makefile.
This is useful for developing software and testing it in a local directory.

\item[install] Installs the files created with the {\bf build} target
into {\tt /star/bin}, etc.
The action of the {\tt install} target will normally be to build any binary
files (eg. executable images and subroutine libraries) in the package directory
and then move them to {\tt /star/bin} etc. Files which are part of the package
source but are referred to by users (eg. include files and example programs) are
copied to {\tt /star/include} etc.

\item[clean] Deletes intermediate files created during the {\bf build}
operation, once the package has been successfully installed.

\item[deinstall] Deletes files placed in the runtime trees (e.f. {\tt
/star/bin}, etc.) by the {\bf install} target.


\end{description}


There may also be a {\tt test} target provided to test the installation of the
software (see SGP/11 for more details).


The package level make file will often do no more than {\tt cd} to a series
of sub-directories and execute make in each of them.
Example make files for part of the {\tt /star/starlink} package
directory is presented in appendix A with a commentary for those unfamiliar
with make.

In a few cases it may be necessary to distribute machine type specific
directories (for example, when we are not permitted to distribute source code).
Where this occurs, the directory names will incorporate the machine type so
that the correct version can be selected by defining the environment variable
{\tt SYSTEM} to be the machine type.
For example the gks object libraries for the
mips processor (DECstation) and the SUN SPARCSystem will be stored in:
{\tt /star\-/starlink\-/lib\-/gks\-/mips} and {\tt
/star\-/starlink\-/lib\-/gks\-/sun4} respectively which can be referred to
in make files as {\tt /star\-/starlink\-/lib\-/gks\-/\$(SYSTEM)}.

\appendix
\section{Example make files}

The following example makefiles are cut--down versions of those currently
released. However, they do illustrate all of the basic requirements of
their class of makefile. More complex examples can be found on--line in
the current release of the USSC.


\subsection{\tt /star/starlink/makefile}
\begin{verbatim}
# /star/starlink/makefile
#
#   Installation make file for the "core" starlink package.
\end{verbatim}
Lines beginning {\tt\#} are comments.
\begin{verbatim}
install:
        if test -d /star/bin; then echo; else mkdir /star/bin; fi
        if test -d /star/lib; then echo; else mkdir /star/lib; fi
        if test -d /star/etc; then echo; else mkdir /star/etc; fi
        if test -d /star/include; then echo; else mkdir /star/include; fi
        if test -d /star/man; then echo; else mkdir /star/man; fi
        if test -d /star/bin/examples; \
                                then echo; else mkdir /star/bin/examples; fi

        if test -d /star/local; then echo; else mkdir /star/local; fi
        if test -d /star/local/bin; then echo; else mkdir /star/local/bin; fi
        if test -d /star/local/lib; then echo; else mkdir /star/local/lib; fi
        if test -d /star/local/etc; then echo; else mkdir /star/local/etc; fi
        if test -d /star/local/include; \
                                then echo; else mkdir /star/local/include; fi
        if test -d /star/local/man; then echo; else mkdir /star/local/man; fi

        cp /star/starlink/login /star/etc/login

        cd lib; $(MAKE) install; cd ..
        cd utility; $(MAKE) install; cd ..
        cd system; $(MAKE) install; cd ..

\end{verbatim}
The ``starlink'' package must always be the first package to be installed so
the first job is to ensure that the {\tt /star/bin} etc. directories exist.
Each action line is passed to the Bourne shell; the {\tt test} command
returns {\em true} if the specified file exists and is a directory and returns
{\em false} if it is not. The {\tt echo} command with no arguments is
essentially a null command which is only there because the syntax of the {\tt
if} command requires a command in the ``then'' clause.

The {\tt install} target then just changes directory to {\tt lib} (the only
one that exists at present) and executes {\tt make install}.

\begin{verbatim}
deinstall:
        -rm /star/etc/login
        cd lib; $(MAKE) deinstall; cd ..
        cd utility; $(MAKE) deinstall; cd ..
        cd system; $(MAKE) deinstall; cd ..

clean:
        cd lib; $(MAKE) clean; cd ..
        cd utility; $(MAKE) clean; cd ..
        cd system; $(MAKE) clean; cd ..
\end{verbatim}
The {\tt deinstall} and {\tt clean} targets operate in a similar way.

This make file does not have a {\tt build} target because some parts of the
``starlink'' package cannot be built until other parts of the package have
been installed.
\subsection{\tt /star/starlink/lib/makefile}
\begin{verbatim}
#  Makefile to build and install the "lib" component of the Starlink software
#  collection.
#
install:
        cp sae_par /star/include
        cp sae_par.h /star/include
\end{verbatim}
The first step is to install the include files {\tt sae\_par} and
{\tt sae\_par.h} in {\tt/star/include}.
\begin{verbatim}
        cd chr; $(MAKE) rebuild; $(MAKE) install; cd ..
        cd ems; $(MAKE) rebuild; $(MAKE) install; cd ..
        cd gks; $(MAKE) rebuild; $(MAKE) install; cd ..
        cd gns; $(MAKE) rebuild; $(MAKE) install; cd ..
        cd sgs; $(MAKE) rebuild; $(MAKE) install; cd ..
        cd pgplot; $(MAKE) build; $(MAKE) install; cd ..
        cd ncar; $(MAKE) build; $(MAKE) install; cd ..
        case "$(SYSTEM)" in \
                sun4) cd pgplot/examples; \
                      $(MAKE) build; $(MAKE) install; cd ../..;; \
                      cd ncar/examples; \
                      $(MAKE) build; $(MAKE) install; cd ../..;; \
                *) ;; \
        esac
\end{verbatim}
The installation process is completed by changing directory to each of the
sub-directories of {\tt /star\-/starlink\-/lib} in turn and using make to first
build the binary files from the source code and then installing the necessary
files in {/star/bin} etc. In this example, example programs are for pgplot and
ncar only built if the environment variable SYSTEM is set to sun4, in order to
save disk space.
\begin{verbatim}
clean:
        cd chr; $(MAKE) clean; cd..
        cd ems; $(MAKE) clean; cd..
        cd gks; $(MAKE) clean; cd..
        cd gns; $(MAKE) clean; cd..
        cd sgs; $(MAKE) clean; cd..
        cd pgplot; $(MAKE) clean; cd ..
        cd ncar; $(MAKE) clean; cd ..
        cd pgplot/examples; $(MAKE) clean; cd ../..
        cd ncar/examples; $(MAKE) clean; cd ../..

deinstall:
        -rm /star/include/sae_par
        -rm /star/include/sae_par.h
        cd chr; $(MAKE) deinstall; cd ..
        cd ems; $(MAKE) deinstall; cd ..
        cd gks; $(MAKE) deinstall; cd ..
        cd gns; $(MAKE) deinstall; cd ..
        cd sgs; $(MAKE) deinstall; cd ..
        cd pgplot; $(MAKE) deinstall; cd ..
        cd ncar; $(MAKE) deinstall; cd ..
        cd pgplot/examples; $(MAKE) deinstall; cd ../..
        cd ncar/examples; $(MAKE) deinstall; cd ../..

 \end{verbatim}
The {\tt clean} and {\tt deinstall} targets just execute  {\tt make clean}
and {\tt make deinstall} respectively in each of the sub
directories.

Again, there is no {\tt build} target because for example, the programs in
the gns package have to be linked with the ems and chr libraries.
\subsection{\tt /star/starlink/lib/chr/makefile}
\begin{verbatim}
#   Installation make file for the CHR package
#
#        Prerequisites: /star/include/sae_par
#
\end{verbatim}
The comments at the start of the file
should indicate which other Starlink software items need to be installed before
the package can be built. In this case all that is required is that the file
{\tt sae\_par} is in {\tt /star/include} but for most packages there will be a
list of Starlink software items rather than the names of individual files.
\begin{verbatim}
#   Shareable library version
#
VERS = 1.1
\end{verbatim}
This sets the macro {\tt VERS} to the version number of the shareable
library that will be built on systems that support shareable libraries.
\begin{verbatim}
#   Object files
#
OBJECT_FILES = chr_appnd.o chr_clean.o chr_copy.o chr_ctoc.o chr_ctod.o \
               chr_ctoi.o chr_ctol.o chr_ctor.o chr_dcwrd.o chr_delim.o \
               chr_dtoc.o chr_equal.o chr_fandl.o chr_fill.o chr_fiwe.o \
               chr_fiws.o chr_htoi.o chr_index.o chr_inset.o chr_isalf.o \
               chr_isalm.o chr_isdig.o chr_isnam.o chr_itoc.o chr_lcase.o \
               chr_ldblk.o chr_len.o chr_lower.o chr_ltoc.o chr_move.o \
               chr_otoi.o chr_putc.o chr_putd.o chr_puti.o chr_putl.o \
               chr_putr.o chr_rmblk.o chr_rtoan.o chr_rtoc.o chr_simlr.o \
               chr_size.o chr_swap.o chr_term.o chr_trunc.o chr_ucase.o \
               chr_upper.o
\end{verbatim}
The macro {\tt OBJECT\_FILES} lists all the object files that make up the
library. The \verb+\+ character continues a statement onto the next line.
\begin{verbatim}
#   Source files
#
SOURCE_FILES = chr_appnd.f chr_clean.f chr_copy.f chr_ctoc.f chr_ctod.f \
               chr_ctoi.f chr_ctol.f chr_ctor.f chr_dcwrd.f chr_delim.f \
               chr_dtoc.f chr_equal.f chr_fandl.f chr_fill.f chr_fiwe.f \
               chr_fiws.f chr_htoi.f chr_index.f chr_inset.f chr_isalf.f \
               chr_isalm.f chr_isdig.f chr_isnam.f chr_itoc.f chr_lcase.f \
               chr_ldblk.f chr_len.f chr_lower.f chr_ltoc.f chr_move.f \
               chr_otoi.f chr_putc.f chr_putd.f chr_puti.f chr_putl.f \
               chr_putr.f chr_rmblk.f chr_rtoan.f chr_rtoc.f chr_simlr.f \
               chr_size.f chr_swap.f chr_term.f chr_trunc.f chr_ucase.f \
               chr_upper.f
\end{verbatim}
The macro {\tt SOURCE\_FILES} lists all the source files that make up the
library.
\begin{verbatim}
#   target for building the package
#
build : libchr.a chr_err
\end{verbatim}
The {\tt build} target depends on the files {\tt libchr.a} and {\tt chr\_err};
these being the two files that are needed to make use of the library once it has
been built. Although nothing has to be done to make {\tt chr\_err} it is
included in the list of files so that the make will fail if it is missing.
\begin{verbatim}
install:
        case "$(SYSTEM)" in \
                sun4) mv libchr.so.$(VERS) /star/lib;; \
                *) rm libchr.so.$(VERS);; \
        esac
        mv libchr.a /star/lib
        ranlib /star/lib/libchr.a
        cp chr_err /star/include
        cp chr\_link /star/bin
        cp chr\_link\_adam /star/bin
\end{verbatim}
The {\tt install} target installs the necessary files in the Starlink
directories. The action taken depends on the type of system that the
installation is being done on; on SUN 4 systems the shareable library ({\tt
libchr.a.1.1}) is moved to {\tt /star/lib} as well as the
object module library ({\tt
libchr.a}). On other systems (such as a DECstation running ULTRIX
which does not support sharable libraries) it is just the object module library
that is moved and the dummy shareable library is deleted.

The libraries can be ``moved'' since they can always be rebuilt
if required, but {\tt chr\_err} and {\tt chr\_link} must by copied so that {\tt
/star/starlink/lib/gns} always contains the complete source of the package.

The object module library must be ``randomized'' after it is copied otherwise
the library's symbol table will appear to the loader to be out of date.
\begin{verbatim}
#   target for de-installing from Starlink directories
#
deinstall:
        -rm /star/lib/libchr.a
        -rm /star/lib/libchr.so.$(VERS)
        -rm /star/include/chr_err
        -rm /star/bin/chr_link

\end{verbatim}
The {\tt deinstall} target simply removes files installed by the {\tt install}
target.
\begin{verbatim}
#   target for cleaning up the build directory
#
clean :
        -rm *.o *.f
\end{verbatim}
The {\tt clean} target just deletes the intermediate files created during
the build process.
\begin{verbatim}
#   dependencies for building the library
#
$(SOURCE_FILES) : chr.a
        ar x chr.a $@
\end{verbatim}
This rule states that the source files are created by extracting the file from
the archive file {\tt chr.a}. {\tt \$@} is a pre-defined macro that expands to
the name of the target file that is out-of-date.
\begin{verbatim}
libchr.a : $(OBJECT_FILES)
        case "$(SYSTEM)" in \
            sun4) touch libchr.a;; \
            *)    ar rv libchr.a $? \
                  ranlib libchr.a;; \
        esac
\end{verbatim}
This rule states that on systems other than the SUN 4 the object library is
created inserting the object files
into the library and building the library index with {\tt ranlib}. \$? is a
pre-defined macro that expands to the name of the source file that is newer
than the target.

On the SUN 4 the object library is not needed so an empty file with the same
name is simply created with {\tt touch}.
\begin{verbatim}
libchr.so.$(VERS) : $(OBJECT_FILES)
        case "$(SYSTEM)" in \
            sun4) ld -o libchr.so.$(VERS) -assert pure-text $?;; \
            *)    touch libchr.so.$(VERS);; \
\end{verbatim}
This rule builds the shareable library on SUN 4 systems and creates and empty
file on the rest.

The make file does not have to contain a rule for creating object files from
Fortran  files because make has a built in rule for doing this.
\end{document}
