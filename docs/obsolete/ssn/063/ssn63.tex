\documentstyle{article}
\pagestyle{myheadings}
\markright{SSN/63.1}
\setlength{\textwidth}{160mm}
\setlength{\textheight}{240mm}
\setlength{\topmargin}{-5mm}
\setlength{\oddsidemargin}{0mm}
\setlength{\evensidemargin}{0mm}
\setlength{\parindent}{0mm}
\setlength{\parskip}{\medskipamount}
\setlength{\unitlength}{1mm}
%------------------------------------------------------------------------------
% Add any \newcommand or \newenvironment commands here

\renewcommand{\_}{{\tt\char'137}}     % recentres the underscore
%------------------------------------------------------------------------------
\begin{document}
\thispagestyle{empty}
SCIENCE \& ENGINEERING RESEARCH COUNCIL \hfill SSN/63.1\\
RUTHERFORD APPLETON LABORATORY\\
{\large\bf Starlink Project\\}
{\large\bf Starlink System Note 63.1}
\begin{flushright}
C A Clayton\\
1 September 1989
\end{flushright}
\vspace{-4mm}
\rule{\textwidth}{0.5mm}
\vspace{5mm}
\begin{center}
{\Large\bf PAD\_AUDIT --- PAD Auditing Package}
\end{center}
\vspace{5mm}

The PAD (Packet Assembler Disassembler) utility is the part of the VAX/VMS
Coloured Book Software (CBS) which allows a user to log onto remote
computers from a local VAX. Unfortunately, logging into a computer via
either the  Packet SwitchStream (PSS) or the International Packet
SwitchStream (IPSS) costs real money. Some users either do not appreciate
this or do not care and have been known to clock up rather large quarterly
bills. This software package allows a system manager to determine {\bf who} has
used PAD to call {\bf where} and (most importantly) {\bf how much} it has cost.
The system manager can then take appropriate action - either charging the
individuals, warning them to use the facility with more care or even denying
access to a greedy user to one or more sites.

\section {Different approaches to PAD auditing}

One reason that there is not already a software package around Starlink to
deal with PAD auditing is that it is rather a messy problem. The following
technique of PAD auditing was been arrived at through trial and error. It
may seem cumbersome in places but the author has found that a more
streamlined approach is not possible due to the lack of {\bf accurate}
 information
{\bf available} to system managers. This package has been in use at Cambridge
for many months now and has proven invaluable in determining from which grants
our depressingly large PSS bill should be paid.

There are two main ways to tackle the problem of auditing PAD calls.

\begin{enumerate}

\item One can ask for a breakdown of your PSS bill. For example, at
Cambridge we receive a breakdown of the bill (on request) from the Network
Executive at RAL (JNET@UK.AC.RUTHERFORD.GEC-B). This breakdown currently
gives details of the individual calls and includes the date and time of the
call, the DTE number of the destination, the volume of traffic (in
segments), the duration of the call (in seconds) and last, but by no means
least, the cost of the call (in pounds). If you happen to know  which of
your users call given sites, then it is relatively easy determine how much
of the bill each user is responsible for. The program NET\_\/EXEC in this
package reduces the call-by-call breakdown and produces a list of cost of
calls to each site. Unfortunately, this breakdown does {\bf not} indicate which of
your users has made which calls. {\it This is the root of most of the problems
associated with PAD auditing.} If the username were included, then a modified
version of NET\_\/EXEC.EXE would be sufficient to allow accurate PSS billing to
individual users.

\item The second method is to monitor the calls on-site and produce your own
breakdown of the calls. This has a number of advantages:

\begin{itemize}

\item This breakdown will include the username of the caller.

\item You are not relying on a third party to send you a breakdown.

\item You can get a breakdown whenever you like rather than
waiting for the Network Executive breakdown every quarter.

\item Auditing of JANET calls is also possible.

\end{itemize}

However, the on-site method has one major disadvantage -- the costs as
calculated from on-site accounting do not tally with those charged by the
Network Executive (i.e. with the costs that really count). The reason for
this is that the volume (V) and duration (D) of the calls as recorded
on-site and those recorded at the gateway often differ:

\begin{enumerate}

\item
V \& D can differ since there is a minimum call charge and hence for short
calls, the gateway figures will appear greater.  One can accommodate for this
in software but the situation is further complicated by the fact that the minimum
charge is only implemented if the call is successful. If the call fails,
then BT only charges for the resources actually used. The information
required to determine if the call was successful or not is {\bf not} recorded by
the on-site call monitor.

\item
To {\bf some} sites there is a systematic error between the volume of traffic V
recorded locally and at the gateway. For example, in the sample data supplied
with this package one can see that the volume of the calls to IAC
that is recorded locally is 30\% to 45\% greater than that recorded at the
gateway. For AAOEPP and SDC the equivalent figures are 20\% to 35\% and up to 10\%
respectively. The cause of this differential has yet to be determined but is
under investigation. However, this problem can be worked around
by normalising the local accounts w.r.t. the Network Executive accounts
(see Section 5).

\end{enumerate}

There are also further complications with on-site auditing. The Network
Executive breakdown includes the cost of file transfers (including mail) as
well as that of remote logins. If you also wish to charge for this type of
network use then you should use Peter Allan's CBS\_\/ACCOUNTS package for
on-site logging of this facility. Also some calls are discounted between
certain times and on certain days but this could be allowed for in software.

\end{enumerate}

In summary, on-site accounting can only give you an approximate idea
of how much each user is costing you. However, when used in conjunction
with the Network Executive breakdown, it is possible (with a little effort
and common sense) to charge users very accurately for their calls.


\section {Summary of the package files}

All of the files initially supplied in this package (except SSN63.TEX) should
reside in the directory PADAUDIT\_\/DIR. The sample files are all for June 1989
for the  Cambridge Starlink node.

All symbols and logical names required for this package are set up in
the files SSC:STARTUP.COM and SSC:LOGIN.COM.

\newlength{\numlen}
\settowidth{\numlen}{0000000000000000000000}
\settowidth{\labelsep}{000}
\begin{list}{}{\setlength{\labelwidth}{\numlen}
\setlength{\leftmargin}{\numlen}
\addtolength{\leftmargin}{\labelsep}}

\item [SSN63.TEX] This document.

\item [NET\_\/EXEC.IN]

Breakdown of PSS calls from RAL
with header and trailing information
removed. Input for NET\_\/EXEC.EXE.
A sample file is included in this
package.

\item [NET\_\/EXEC.EXE]

Takes the Network Executive breakdown
NET\_\/EXEC.IN as input and produces a file
called NET\_\/EXEC.OUT which lists
the total costs of calls to each of
the sites that are
listed in TARGET\_\/SITES.LIS.

\item [NET\_\/EXEC.OUT]

Sample output from NET\_\/EXEC.EXE showing
total cost of calls to each site.

\item [TARGET\_\/SITES.LIS]

A list of DTEs (and associated site names)
for which NET\_\/EXEC.EXE should calculate
the total cost of calls made to that DTE.
The template file supplied contains a few
astronomical sites to get you started.

\item [PAD\_\/ACCOUNTS.COM]

Extracts the PAD accounting information
from your on-site accounts and runs
PAD\_\/ACCOUNTS.EXE to produce PAD.DAT
ready for auditing with PAD\_\/AUDIT.EXE
(OPAD\_\/AUDIT.EXE for older accounts).
This command procedure also deletes the
temporary files it creates when they are
no longer needed.

\item [PAD\_\/ACCOUNTS.EXE]

Takes the local PAD accounting information
and produces a list (PAD.DAT) containing
the username, volume, duration
and destination associated with each call.
This program should be run via
PAD\_\/ACCOUNTS.COM.

\item [PAD.DAT]

Sample output from PAD\_\/ACCOUNTS with which
to test out PAD\_\/AUDIT.

\item[PAD\_\/AUDIT.EXE]

An interactive program which allows one to
interrogate PAD.DAT (the output from
PAD\_\/ACCOUNTS.EXE).

\item [OPAD\_\/AUDIT.EXE]

An interactive program which allows one to
interrogate PAD.DAT (the output from
PAD\_\/ACCOUNTS.EXE). This old version is
for accounts from before June 1989.

\item [PAD\_\/AUDIT.LOG]

Log file to which PAD\_\/AUDIT.EXE
(and OPAD\_\/AUDIT.EXE) output is
directed. A sample file is provided.

\item [CHARGES\_\/JUN89.DAT]

A file containing details of the current
charging rates for calls to PSS and IPSS.
This file is used by PAD\_\/AUDIT.EXE to calculate
the call cost from the duration, volume of
traffic and destination of the call.

\item [CHARGES.DAT]

A file containing details of the pre-June 1989
charging rates for calls to PSS and IPSS.
This file is used by OPAD\_\/AUDIT.EXE to calculate
the call cost from the duration, volume of
traffic and destination of the call.

\end {list}

\section {Implementing PAD auditing}

\begin{enumerate}

\item {\bf Activate on-site PAD accounting on the node which runs your
CBS software.}

This should be done as soon as possible. On-site PAD accounting cannot take
place until you have done this. To do this you simply run
NET\$DIR:PADCONFIG.EXE. This is fully explained in section 4.2 of the CBS
V5.0 installation and management guide. In brief, the first question asked
is

\begin{quote} \tt
Accounting is OFF

Do you want to change this ?
\end{quote}


Answer ``YES'' to this question and take the defaults on all the others. The
defaults in PADCONFIG.EXE are the current settings.

\item {\bf Arrange for the Network Executive Breakdown to be sent to you
each quarter}.

Mail JNET@UK.AC.RUTHERFORD.GEC-B and ask them to send you a breakdown of
your PSS bill every quarter. You may need to remind them a couple of weeks
after the invoice arrives on your administrator's desk. If they sound
confused, ask them to send you a breakdown like the one Bill Jenkins sends
out for account CAAS (Cambridge astronomy).

\item {\bf Reduce Network Executive breakdown to cost per site.}

When the breakdown arrives, edit out the header information and any trailing
information and write out to a file called NET\_\/EXEC.IN. An example file is
included in this package.

Review the sites in TARGET\_\/SITES.LIS and make any necessary additions
applicable for your site. You can use NET\$DIR:NETAUTH to compile a list of
sites which can be called from your site via your local PSS gateway.

Run NET\_\/EXEC.EXE via the global symbol NET\_\/EXEC. The total cost per site (and
the total cost of all calls) will be written to NET\_\/EXEC.OUT. The total cost at
the bottom should tally with the cost on the invoice that your administrator
receives. If your total is less, then calls are being made to sites which you
have not included in TARGET\_\/SITES.LIS. Take a closer look through your
NETAUTH\_\/DATABASE.ATH using NETAUTH.EXE.

Remember that these totals also include the cost of file transfers and that
these usages of the network are of course NOT recorded in the local PAD
accounts.


\item {\bf Extract local accounting information from VMS accounts.}

At the beginning of each month you are required to run an accounting
package and send usage graphs for the month to RAL. After this package is
run on each node in your cluster, you will be left with files of the form
mmmyy\_\/ACCOUNTS.DAT (e.g. APR89\_\/ACCOUNTS.DAT) which contains accounting
information for the entire month. You must now extract the PAD accounting
information from the accounting file mmmyy\_\/ACCOUNTS.DAT for the node on
which CBS is running. This can be done with the command

\begin{quote} \tt
\$ ACC/BINARY/TYPE=(USER)/OUTPUT=FOR001.DAT mmmyy\_\/ACCOUNTS.DAT
\end{quote}

Note that there is a bug in the V5.0 version of ACC.EXE such that the
/TYPE=(USER) does not function. The old V4 version of ACC.EXE should be
substituted in SYS\$COMMON:[SYSEXE]. As of VMS 5.1, this problem still
remains.

The important information (the username, volume, duration and destination
associated with each call) should then be extracted and output into a
formatted file called PAD.DAT (sorted in order of username) using the
program PAD\_\/ACCOUNTS.EXE. This is a slightly modified version of Peter
Allan's PAD\_\/READ program. It has been modified to work with CBS V5 and also
to sort the accounts by username, as required by the PAD\_\/AUDIT program.

Simply typing PAD\_\/ACCOUNTS will perform all of these stages
for you and tidy up the temporary files created, leaving you just with
PAD.DAT ready to use with PAD\_\/AUDIT.EXE. It is recommended that you maintain
a collection of the PAD.DAT files by copying PAD.DAT to mmmyy.PAD in case
you need to return at some point in the future for another auditing session
(i.e. once the bill comes in and you want to discover who clocked up half
your budget logging into AAOCBN).

\item {\bf Run PAD\_\/AUDIT interrogation program}

Run PAD\_\/AUDIT via the global symbol PAD\_\/AUDIT. This is a self explanatory,
interactive program which allows one to interrogate PAD.DAT. Once the breakdown
of the calls has arrived from JNET@UK.AC.RUTHERFORD.GEC-B and you have reduced
it to the level of cost of calls to each site using NET\_\/EXEC.EXE, then you
will need to determine what fraction of the total is attributable to each user.
If possible, you should append together the three PAD.DAT files for the three
months covered by the Network Executive breakdown and

\begin{quote} \tt
\$ SORT PAD.DAT PAD.DAT
\end{quote}

Each PAD.DAT is already sorted in order of username (as expected by
PAD\_\/AUDIT.EXE) by PAD\_\/ACCOUNTS.EXE but the combined file must be resorted.

PAD\_\/AUDIT first prompts in the top window for a site name and then a
username. In general you should take the default wildcard for username to
ensures that all users who have called this site during the month will be
brought to your attention. It will list in the middle window each user who
has called the site along with the total number of calls, call duration,
traffic and how much they cost you. Hitting any key moves you on to the next
user. After all users have been audited, total number of calls, call
duration, traffic and cost of calls to the site are displayed in the lower
window. Also, you are then prompted in the top window for the next site. If
you do not wish to audit a second site then hit return to exit the program.
The PAD auditing session is logged in a file called PAD\_\/AUDIT.LOG. You
should audit all of the sites which are listed in NET\_\/EXEC.OUT as having
cost you money. A couple of minutes spent comparing NET\_\/EXEC.OUT and
PAD\_\/AUDIT.LOG is usually sufficient to allow PSS billing to individual users
to take place. At this stage you should bear in mind that NET\_EXEC.OUT
includes file transfer costs in addition to remote logins whereas
PAD\_AUDIT.LOG include only the latter. Hence the costs to a site
as given in NET\_EXEC.OUT may be greater than those given in PAD\_AUDIT.LOG.

It should also be noted that PAD\_\/AUDIT can also be used to investigate JANET
calls since these calls are all recorded in the on-site accounts. The
procedure of use is the same but it now becomes necessary to check
abbreviations of JANET addresses as well. For example, if I wish to audit
calls to the PHOENIX mainframe in Cambridge I must audit sites PHX, CAM.PHX
and PHOENIX (plus strictly all the other allowable combinations up to a
maximum of 11 characters).

Finally, this package has already allowed to the author to catch a would-be
hacker. A user was found to have called a large number of JANET sites
with sequential DTE numbers for short durations. The user has now been
cured of this habit.

\end{enumerate}

\section {Usage summary}


\begin{figure}[h]
\begin{center}
\begin{picture}(100,130)(15,0)
\thicklines
%\put (0,0){\framebox(100,160){ }}
\put (0,90){\framebox(40,10){NET\_\/EXEC.IN}}
\put (0,50){\framebox(40,10){NET\_\/EXEC.OUT}}
\put (60,120){\framebox(40,10){VMS accounts}}
\put (60,90){\framebox(40,10){PAD.DAT}}
\put (60,50){\framebox(40,10){PAD\_\/AUDIT.LOG}}
\put (80,90){\vector (0,-1){30}}
\put (80,120){\vector (0,-1){20}}
\put (20,90){\vector (0,-1){30}}
\put (20,50){\vector (1,-4){10}}
\put (80,50){\vector (-1,-4){10}}
\put (30,65){\vector (0,-1){5}}
\put (90,65){\vector (0,-1){5}}
\put (30,0){\framebox(40,10){Billing of users}}
\put (22,80) {\bf NET\_\/EXEC.EXE}
\put (82,80) {\bf PAD\_\/AUDIT.EXE}
\put (82,110) {\bf PAD\_\/ACCOUNTS.COM}
\put (32,25) {\bf Comparison `by eye'}
\put (82,65){\framebox(40,10){CHARGES.DAT}}
\put (22,65){\framebox(40,10){TARGET\_\/SITES.DAT}}
\end {picture}
\end {center}
\end {figure}

\section {Known bugs}

As mentioned in Section 1, there is currently a discrepancy between
the volume of traffic associated with a call to some sites recorded by the
Network Executive and that recorded in the local accounts.
The reason for this differential is under investigation and a
modified release of this package will be released when a robust
workaround is found. In the meantime, it is necessary to
multiply the cost to a site attributed to user X (as given in
PAD\_AUDIT.LOG) by the ratio of the total cost to that site
given in NET\_EXEC.OUT and that given in PAD\_AUDIT.LOG. This
`fix' assumes that there is no non--PAD traffic to that site
which, of course, may not be true. However, even in the worst cases,
reasonably accurate PAD billing is still possible.

Any bug reports, suggestions or queries concerning this package should be
sent to CAVAD::CAC.

\newpage

\appendix

\section{PSS and IPSS Charges BEFORE 1st June 1989}

This appendix gives the charging  rates  for  making  calls  to  the Packet
SwitchStream  (PSS) and the International Packet SwitchStream (IPSS) up to
1st June 1989. The charges are calculated on the basis of  both  time  and
the amount  of  information sent, in Kilosegments.  One segment is a packet
of information that can hold up to 64 bytes of data.  Thus if you were to
send 65 bytes of data you would use 2 segments.

\begin{table}[h]
\begin{center}
\begin{tabular} {||l|l|l||}
\hline
Destination  &  Duration charge   & Volume charge \\
\hline
U.K. Full rate      &    0.0167p /  2 secs   &    0.030p / 1 segment \\
U.K. Discount rate  &    0.0111p /  2 secs   &    0.020p / 1 segment   \\
Europe              &    1.5000p / 30 secs   &    0.900p / 5 segments  \\
N. America          &    3.7500p / 30 secs   &    2.250p / 5 segments  \\
Rest of world       &    4.5000p / 30 secs   &    2.700p / 5 segments  \\
\hline
\end{tabular}
\end{center}
\end{table}

  A PSS/IPSS accounting unit is one segment (up to 64  characters)  or
  that duration of call which would cost the same.

  This duration is 3.6 seconds (Volume charge / Duration  charge)  for
  all destinations.

\begin{table}[h]
\begin{center}
\begin{tabular} {||l|c||}
\hline
Destination   &  Cost(pence)/Unit \\
\hline
U.K. Full rate       &    0.030 \\
U.K. Disc rate       &    0.020 \\
Europe               &    0.180 \\
N. America           &    0.450 \\
Rest of world        &    0.540 \\
\hline
\end{tabular}
\end{center}
\end{table}

  The cost of a call is (Duration(seconds)/3.6 + segments)  multiplied
  by the cost per unit.  This does not include V.A.T.

  British Telecom have introduced a  minimum  charge  for  all  call
  attempts except  those  which  fail  because of their network.  This
  will be the cost of transmitting 5 segments of data for  30  seconds
  for all  IPSS  calls  and  the  cost of 50 segments plus appropriate
  duration charge for U.K.  calls.

  All U.K.  calls between 1800 and 0800 plus  Sundays,  Christmas  and
  New Years day are charged at the discount rate.


\newpage

\section {PSS and IPSS Charges AFTER 1st June 1989}

This appendix gives the charging  rates  for  making  calls  to  the Packet
SwitchStream  (PSS) and the International Packet SwitchStream (IPSS) AFTER
1st June 1898.

British Telecom  introduced  a  different  pricing  structure  for  all
international calls  from  the 1st June 1989, whereby all international
calls will be charged in 1 segment steps (previously 5) with a  minimum
charge of  10 segments.  The charge for international call duration has
been removed,  however,  thus   benefitting   those   users   who   are
transferring small amounts of data over relatively long periods.

There is a minimum charge for all call attempts except those which fail
because of  public  network  faults.   This  will  be   the   cost   of
transmitting 10  segments of data for all IPSS calls and the cost of 50
segments plus appropriate duration charge for U.K.  calls.

All U.K.  calls between 1800 and 0800 plus Sundays, Christmas  day  and
New Years  day  are  charged  at  the  discount rate.  Please note that
Saturday charges are not discounted.

\begin{table}[h]
\begin{center}
\begin{tabular} {||l|c|c||}
\hline
Destination  &  Duration charge   & Volume charge \\
\hline
U.K. Full rate       &   0.0167p /  2 secs   &    0.030p / 1 segment \\
U.K. Discount rate   &   0.0111p /  2 secs   &    0.020p / 1 segment \\
Europe               &   No Charge           &    0.28 p / 1 segment  \\
N. America           &   No Charge           &    0.70 p / 1 segment  \\
Rest of world        &   No Charge           &    0.82 p / 1 segment   \\
\hline
\end{tabular}
\end{center}
\end{table}

 An accounting unit on the JANET PSS Gateways is one segment (up  to  64
 characters) or  that  duration of call which would cost the same.  i.e.
 3.6 secs.  (for UK calls only from 1st June 1989.)

\begin{table}[h]
\begin{center}
\begin{tabular} {||l|c||}
\hline
Destination    & Cost(pence)/Unit \\
\hline
U.K. Full rate    &       0.030 \\
U.K. Disc rate    &       0.020 \\
Europe            &       0.280 \\
N. America        &       0.700 \\
Rest of World     &       0.820 \\
\hline
\end{tabular}
\end{center}
\end{table}

The cost of a UK call is
(Duration in seconds/3.6 + segments) multiplied by cost per unit.  This
does not include V.A.T.

The cost of an international call is segments multiplied  by  cost  per
unit.
All of these prices  exclude VAT but the output from PAD\_\/AUDIT.EXE adds
VAT at 15\%.

\newpage

\section {Format of CHARGES.DAT }
\begin{table}[htb]
\begin{center}
\begin{tabular} {||l|l||}
\hline
Column   & Usage \\
\hline
 1$\rightarrow$10           &       Site Name                   \\
11$\rightarrow$20           &       Area identifier (not used)  \\
21$\rightarrow$30           &       Volume cost                 \\
31$\rightarrow$40           &       Duration cost               \\
41$\rightarrow$80           &       Comments (not used)         \\
\hline
\end{tabular}
\end{center}
\end{table}

Note: The IPSS volume charge is in pence per 5 segments.
      The IPSS duration charge is in pence per 30 seconds.


\section{Format of CHARGES\_\/JUN89.DAT }
\begin{table}[htb]
\begin{center}
\begin{tabular} {||l|l||}
\hline
Column   & Usage   \\
\hline
 1$\rightarrow$10        &          Site Name                     \\
11$\rightarrow$20        &          Area identifier (not used)    \\
21$\rightarrow$30        &          Volume cost                   \\
31$\rightarrow$40        &          Duration cost (no longer used)\\
41$\rightarrow$80        &          Comments (not used)           \\
\hline
\end{tabular}
\end{center}
\end{table}

\noindent Note: The IPSS volume charge is in pence per segment.
      The IPSS duration charge no longer exists.


\end{document}
