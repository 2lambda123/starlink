\documentstyle{article} 
\pagestyle{myheadings}

%------------------------------------------------------------------------------
\newcommand{\stardoccategory}  {Starlink System Note}
\newcommand{\stardocinitials}  {SSN}
\newcommand{\stardocnumber}    {33.6}
\newcommand{\stardocauthors}   {D L Terrett}
\newcommand{\stardocdate}      {26 February 1990}
\newcommand{\stardoctitle}     {Configuring Queues for Starlink Graphics}
%------------------------------------------------------------------------------

\newcommand{\stardocname}{\stardocinitials /\stardocnumber}
\markright{\stardocname}
\setlength{\textwidth}{160mm}
\setlength{\textheight}{240mm}
\setlength{\topmargin}{-5mm}
\setlength{\oddsidemargin}{0mm}
\setlength{\evensidemargin}{0mm}
\setlength{\parindent}{0mm}
\setlength{\parskip}{\medskipamount}
\setlength{\unitlength}{1mm}

\begin{document}
\thispagestyle{empty}
SCIENCE \& ENGINEERING RESEARCH COUNCIL \hfill \stardocname\\
RUTHERFORD APPLETON LABORATORY\\
{\large\bf Starlink Project\\}
{\large\bf \stardoccategory\ \stardocnumber}
\begin{flushright}
\stardocauthors\\
\stardocdate
\end{flushright}
\vspace{-4mm}
\rule{\textwidth}{0.5mm}
\vspace{5mm}
\begin{center}
{\Large\bf \stardoctitle}
\end{center}
\vspace{5mm}

The Starlink Software Collection contains graphics software for plotting on
Canon Laser and Printronix P300 printer/plotters, and this requires that the
printer queues for these devices are set up according to the rules given in
this document.
NB: This note applies to VMS Version 5.2 onwards.

\section{Printronix P300}

To enable graphics files to be printed on a Printronix you must:
\begin{enumerate}
\item Set the printer /PRINTALL
\item Set the printer width to 132 characters
\item Ensure that printer logic board A, location 8K, has jumper W5 installed.
\end{enumerate}
Software which queues graphics files to the printer automatically
requires that {\bf SYS\_PRINTRONIX} be the name of a queue, or a logical name
for a queue, which spools to a Printronix printer.
Graphics files must be printed with the /PASSALL qualifier.

\section{Canon LBP-8 Laser Printer}

Starlink software which queues graphics files to a Canon Laser Printer
automatically requires that {\bf SYS\_LASER} be the name of a queue, or a
logical name for a queue, which spools to a laser printer.
Graphics files must be printed with the /PASSALL qualifier.

\appendix

\section{Queue startup procedure used on RAL Starlink VAX}

\begin{verbatim}
    $!++
    $!  SYS$MANAGER:STARTQ.COM
    $!
    $!  Start all queues on RLSSD1
    $!--
    $ SET NOON
    $!
    $!  Start terminal server queues
    $!
    $!@SYS$MANAGER:REMOTE_PRINT
    $!
    $!  Printer queues
    $!
    $!SET TERMINAL/PERMANENT/DEVICE_TYPE=UNKNOWN/NOWRAP/PAGE=63 -
    $!		EIGHTBIT/NOTYPEHEAD/NOBROADCAST/FORM/SPEED=9600	TXA6:
    $!SET DEVICE/SPOOLED=(SYS_LASER,SYS$SYSDEVICE) TXA6:
    $!START/QUEUE SYS_LASER
    $!
    $!  Batch queues
    $!
    $ START/QUEUE RLSSD1_BATCH
    $ START/QUEUE RLSSD1_SHORT
    $!
    $ EXIT
\end{verbatim}
\end{document}
