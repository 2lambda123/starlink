\documentstyle[11pt]{article}
\pagestyle{myheadings}

%------------------------------------------------------------------------------
\newcommand{\stardoccategory}  {Starlink System Note}
\newcommand{\stardocinitials}  {SSN}
\newcommand{\stardocnumber}    {41.7}
\newcommand{\stardocauthors}   {M D Lawden \& M J Bly}
\newcommand{\stardocdate}      {1 February 1993}
\newcommand{\stardoctitle}     {Starlink Software Changes}
%------------------------------------------------------------------------------

\newcommand{\stardocname}{\stardocinitials /\stardocnumber}
\markright{\stardocname}
\setlength{\textwidth}{160mm}
\setlength{\textheight}{230mm}
\setlength{\topmargin}{-2mm}
\setlength{\oddsidemargin}{0mm}
\setlength{\evensidemargin}{0mm}
\setlength{\parindent}{0mm}
\setlength{\parskip}{\medskipamount}
\setlength{\unitlength}{1mm}

\begin{document}
\thispagestyle{empty}
SCIENCE \& ENGINEERING RESEARCH COUNCIL \hfill \stardocname\\
RUTHERFORD APPLETON LABORATORY\\
{\large\bf Starlink Project\\}
{\large\bf \stardoccategory\ \stardocnumber}
\begin{flushright}
\stardocauthors\\
\stardocdate
\end{flushright}
\vspace{-4mm}
\rule{\textwidth}{0.5mm}
\vspace{5mm}
\begin{center}
{\Large\bf \stardoctitle}
\end{center}
\vspace{5mm}

%  table of contents
\setlength{\parskip}{0mm}
\tableofcontents
\setlength{\parskip}{\medskipamount}
\markright{\stardocname}

\newpage
\section {Introduction}

Starlink Software Changes (SSC's) are the mechanism by which the Starlink
Software Collection is maintained and updated.
They are prepared and distributed by the Starlink Software Librarian who is
responsible for the management of the Collection.
This paper gives a detailed description of the preparation, implementation and
distribution of SSC's.

The Collection is implemented on every Starlink computer, all of which have
access to the JANET network.
It is also implemented on many non-Starlink computers, some of which have access
to JANET.

There are two methods of updating the Collection.
The first consists of obtaining and implementing a sequence of SSC's issued by
the Librarian at RAL.
This method is appropriate for sites with access to JANET or the Starlink
DECnet.
SSC's are implemented at RAL in the same way as at every other site which uses
this method.
This means that every SSC is tested before being distributed and that every site
is treated the same way.
It also ensures that the Collection is identical at every site (unless
unauthorized changes or omissions are made).
The second method consists of a complete reissue on magnetic tape of the
standard software, plus whatever optional software or data is required.
Because of the effort, expense, and number of sites involved, this can only be
done infrequently, but it is the only method available to sites without access
to JANET.
In this paper, only the first method is considered.

Each SSC is prepared at RAL in its own directory on the Starlink VAX cluster.
When it is ready, it is implemented at RAL.
If any problems are detected at this stage, these are resolved and the SSC is
re-implemented.
When an SSC has been successfully implemented at RAL, an SSC notice is sent
by electronic mail to all the sites which use the network updating method.
A compressed BACKUP save set of the directory is made to make it easier to copy
over a network.
Each Site Manager then copies this save set and recreates the directory.
He or she then implements the SSC by following the instructions in the SSC
notice.
Normally, this consists simply of executing a supplied command procedure.

The sections that follow describe the network method in detail.
The steps fall naturally into three phases:
\begin{description}
\begin{description}
\item [PREPARATION]
\item [RELEASE]
\item [CLEAN-UP]
\end{description}
\end{description}
The control and maintenance of the Starlink Software Collection is a big and
challenging job.
It needs technical knowledge, insight, imagination and initiative.
Above all it requires care, constant checking, attention to detail, and a
systematic working method.
This all takes time, but in the long run it is the quickest way of doing it
properly.
The steps described below should be regarded as recommendations based on years
of experience in managing the Collection.
They need to be applied intelligently rather than dogmatically.
For instance, the exact order in which steps are carried out is usually not
critical and not every contingency can be anticipated.
The important thing is that the Collection be properly controlled and
documented and that required actions are not overlooked.
Improvements in the method are to be welcomed, however it is important that it
remains simple and robust.
Avoid complicated, clever (and fragile) command procedures as these sometimes
go wrong with disastrous results (to the surprise and amusement of the
perpetrator).
Avoid adding extra steps that seem attractive but are not worth the continual
effort involved; remember, over the years each step will be done hundreds of
times.

A release may be associated with a single item, or several  items may be
released together, and when a new version of VMS is released many items may
need changing simultaneously.

\section {Preparation}

All work on the Collection should be carried out under the Librarian's username
{\tt STAR}.
The process starts when the Librarian learns about a change which is needed to
an existing item in the Collection, or about the advent of a new item which
should be added to the Collection.
One way or another, the required information about how to implement the change
or addition must be supplied to the Librarian.
The formal way to do this is for the software supporter to submit a completed
Starlink Software Item Submission Form (SSISF).
This can be found in the file {\tt RLVAD::LDOCSDIR:SSISF.TEX}.
However, the information may be supplied by a mail message or some other
method of communication.
The advantage of the SSISF is that it asks for information that may be omitted
from a less formal submission.
It doesn't really matter how the information is supplied; the important thing
is that the Librarian must acquire the information needed to prepare the
release.

It may be worth mentioning that many people who submit software assume that the
Librarian has the same detailed understanding of it that they have and often
leave out vital instructions on the grounds that these are `obvious'.
These naive people need educating as you have to continually pump them for
information, which wastes time.

Software submission is dealt with further in SGP/19.

\begin{description}

\item [A1 : HOMELESS FILES] ---
Some files in the Collection are not associated with a specific software item.
These contain such things as username lists, general documentation and
management information.
When new versions of these become available, store them in
{\tt [STAR.TEMP.PENDING]} until a new release is prepared, then include them
in that release.

\item [A2 : HOMEWORK] ---
When a software item is submitted for release, study the information supplied
on the proposed new issue or update.
If necessary, discuss the release with the originator to resolve any problems
which arise.
Make sure the item is suitable for general release ({\em e.g.} has it been
coordinated with general Starlink software development?).
Decide where to put the files; this may involve the creation of new directories.
If a new item is large ($>$2000 blocks) or of limited interest, it should be
released as an Optional Starlink Software Item (see SGP/20) since the ideal
of an integrated software collection identical at all sites may break down if
the standard collection gets too large for convenient storage.

\item [A3 : SUN's] ---
Every item should have a Starlink User Note (SUN) associated with it.
If one has not been provided you should produce one yourself.
This can be either a reformatted version of an existing piece of documentation,
or a brief introduction which points to internal documentation.
Check the spelling with SPELL (LUN/26 (RAL)).

\item [A4 : CREATE DIRECTORY] ---
Create directory {\tt [STAR.TEMP.SSCn]} to contain a complete update package:
\begin{verbatim}
    $ CREATE/DIR/OWNER=STAR [STAR.TEMP.SSCn]
\end{verbatim}
{\tt n} stands for the number of the Starlink Software Change being prepared.
SSC's should be issued and implemented in numerical order, although several
can be in preparation at any one time.

If the item is an Optional item, the submitted files should be stored in a
separate directory called {\tt [STAR.TEMP.SSCnX]} (and {\tt [STAR.TEMP.SSCnY]}
if needed).
This is because a Site Manager may choose not to implement this item and may
therefore not want to copy it over a network.
However, directory {\tt [STAR.TEMP.SSCn]} will be produced for every release
because you need to add some files to those submitted and these are stored here.
In some cases it is easier to implement the release if the submitted files are
stored in subdirectories while the ones you add are stored in
{\tt [STAR.TEMP.SSCn]}.

Site Managers should copy and implement {\em every}\, {\tt SSCn.COM} released;
even if they do not want the item itself.

\item [A5 : SELECT FILES] ---
Decide what files to put in {\tt [STAR.TEMP.SSCn]} and copy them across.
Do not forget to update the revision dates of any files you modify.
The following categories of files are eligible:
\begin{description}
\begin{description}
\item [Submitted files]:
These should be specified by the person submitting the software.
What is appropriate depends on the nature of the software, however it will
include things like source code, libraries, executable code, command procedures,
test data and documentation.
\item [Homeless files]:
Usually, these are documentation files for storage in {\tt DOCSDIR}.
They may have been stored temporarily in {\tt [STAR.TEMP.PENDING]}.
\item [Information files]:
Copy the ones which need updating and prepare new versions.
\begin{verbatim}
    DOCSDIR:DOCS.LIS            ADMINDIR:SIG.LIS
            SUBJECT.LIS                  SSI.LIS
            NEWS.LIS                     USERNAMES.LIS
            BUGS.LIS                     WHOSWHO.LIS
            ANALYSIS.LIS                 HAG.LIS
            ADAMDOCS.LIS                 SUPPORT.LIS
            MUD.LIS
            DOCBUGS.LIS
\end{verbatim}
\item [Environment files]:
These set up an environment for Starlink software.
Copy the ones which need updating and prepare new versions.
\begin{verbatim}
    SSC:LOGIN.COM         (Global symbol definitions)
        STARTUP.COM       (Logical name definitions)
        ADAM_PACKAGES.ICL (ADAM packages ICL startup file)
\end{verbatim}
For items stored outside {\tt [STARLINK]} you may need to edit:
\begin{verbatim}
    LSSC:LOGIN.COM         (Local Global symbol definitions)
         STARTUP.COM       (Local Logical name definitions)
         ADAM_PACKAGES.ICL (Local ADAM packages ICL startup file)
\end{verbatim}

\item [SSC files]:
The SSC notice and implementation command procedure (see {\bf A7}, {\bf A8}):
\begin{verbatim}
    SSCn.COM          (SSC implementation command procedure)
    SSCn.LIS          (SSC notice)
\end{verbatim}
\end{description}
\end{description}

\item [A6 : SAVE SPACE] ---
Save as much disk space as possible.
Space economy is very important for Starlink software, not just to save disk
space but to minimise the network traffic involved in software distribution.
Make sure the directories you release are purged.

If the release contains many source code files, these should be stored in a text
library in reduced format.
Thus, suppose the submitted software contains a number of {\tt *.FOR} files
containing Fortran source code.
First create a text library:
\begin{verbatim}
    $ LIB/TEXT/CREATE item
\end{verbatim}
where {\tt item} is the software item name.
Add the source files to the library as modules:
\begin{verbatim}
    $ LIB item.TLB *.FOR
\end{verbatim}
Then store it in reduced form which should save about a third of the original
space:
\begin{verbatim}
    $ LIB/DATA=REDUCE item.TLB
\end{verbatim}
You can check that the modules are in the library by:
\begin{verbatim}
    $ LIB/LIST item.TLB
\end{verbatim}
Also, a single source file can be restored by:
\begin{verbatim}
    $ LIB/EXTRACT=name/OUTPUT=name.FOR item.TLB
\end{verbatim}
Compress all libraries that are submitted for release
(We have known people submit libraries containing 90\% waste space):
\begin{verbatim}
    $ LIB/COMPRESS name.typ
\end{verbatim}
Make sure the space allocated to files is as low as possible.
Find out what the situation is by:
\begin{verbatim}
    $ DIR/SIZE=ALL
\end{verbatim}
If there is wasted space, truncate the files:
\begin{verbatim}
    $ SET FILE/TRUNCATE *.*
\end{verbatim}
The saving depends on the original cluster size.
The Librarian allocates 100 blocks for even a small library, so these should be
truncated after being set up.
Truncation resets the file date.

Use STRIP (LUN/47 (RAL)) to strip the blanks off the end of lines in text files.

\item [A7 : PREPARE SSCn.COM] ---
This is the SSC implementation command procedure.
Look at previous examples for the correct format and suitable techniques; one
example is shown in Appendix \ref{se:ex.proc}.
A skeleton command procedure is stored in {\tt [STAR.TEMP]SSC.COM}.
Copy this into {\tt [STAR.TEMP.SSCn]\-SSCn.\-COM} and edit it.
\begin{verbatim}
    $ COPY [-]SSC.COM SSCn.COM
    $ EVE SSCn.COM
\end{verbatim}
Change {\tt ***} to the number of the release.
When using the EVE editor, this is most easily done as follows:
\begin{verbatim}
    <DO> or GOLD-KP7
       Replace *** XXX   (where XXX is the number of the release)
       A[ll]
\end{verbatim}
Delete any lines which are not required, {\em e.g.} information summaries that
are not changed, and add the commands required to implement the release.

\item [A8 : PREPARE SSCn.LIS] ---
This is the SSC notice.
This should tell Site Managers how to install the release and give them
information about it.
Enter the EVE editor using the special symbol,
\begin{verbatim}
    $ SSCEVE SSCn.LIS

    <DO> or GOLD-KP7
       Replace *** XXX
       A[ll]
\end{verbatim}
from within {\tt [STAR.TEMP.SSCn]}.
(If you use the EDT editor you can copy the skeleton file from [-]SSC.EVE.)
This will create a partially complete file which can be edited to provide the
final version.
Try to maintain a format and style similar to previous releases; one example is
shown in Appendix \ref{se:ex.note}.
Try to include the following information in the `Notes' section:
\begin{itemize}
\item The purpose of the release ({\em i.e.} what's new) or a pointer to this
information.
\item A test that can be carried out to confirm that the software has been
installed correctly (see {\bf B3}).
\end{itemize}
If users are likely to be affected noticeably by the release, ask the Site
Managers to announce the release in their NEWS file (see {\bf C1}).
If a news item is required, prepare a file called `NEWS\_name.TXT' with the
required information and ask the Site Manager to post it into his News file.
The required format is described in SUN/51.
Put in a deletion date of about a month hence.

Every software item should have a version number, and this should change with
every release that affects it.
Every item should have a file called `VERSION.NUM' in its top level directory.
This should contain a single line of text like:
\begin{verbatim}
    This is version n.n-n of XXXXXXX, released in SSCnnn.
\end{verbatim}

\item [A9 : FILE ATTRIBUTES] ---
Check the protection code and owner for all files.
The protection code should be {\tt (RWE,RWE,RE,RE)} for directories and
{\tt (RE,RWED,RE,RE)} for other files.
The owner should be UIC = {\tt [STAR]}.
However, there are files in the directories {\tt STARDISK:[STARSEC...]} that
require special privileges set by the Project.
Occasionally, some files need Write privilege for users ({\em e.g.}\ ARGS
data base files).
The lack of this privilege is sometimes the cause of mysterious errors.

There should be no ACL information on any released files, except some of those
in {\tt STARDISK:\-[STARSEC...]}.

You can look at the current file protection and ownership by:
\begin{verbatim}
    $ DIR/PROT/OWN
    $ DIR/SEC          ! checks the ACL protection on files [STARSEC...]
\end{verbatim}
You can set the protection by:
\begin{verbatim}
    $ SET PROT *.*
\end{verbatim}
For directory trees try:
\begin{verbatim}
    $ SET PROT [...]*.*
    $ SET PROT=(S:RWE,O:RWE) [...]*.DIR
\end{verbatim}
See Vol. 3 System Management for details of the ACL editor.
You can set the ownership (if you have {\tt SYSPRV} and {\tt BYPASS}
privileges) by:
\begin{verbatim}
    $ SET FILE/OWN *.*
\end{verbatim}

\item [A10 : FINAL CHECK] ---
Purge {\tt [STAR.TEMP.SSCn\ldots]} and check the {\tt SSCn.LIS/COM} files
carefully. Make sure no files are missing or superfluous.
Note down the size of all directories in the release and note these in
{\tt SSCn.LIS}.
Make sure the size of the directory is described correctly in {\tt SSCn.LIS}
as this will give a remote manager confidence that he has copied the software
correctly or warn him that some files are missing.
If errors are detected after release, issue corrections in new releases called
{\tt SSCnA}, {\tt SSCnB}, {\em etc}.
\end{description}

\section {Release}

We implement software updates while the computer is in normal operation.
Perfectionists will want to do updates during special `system development'
sessions during which users are excluded from the computer.
In practice this is highly inconvenient for users, so we think the current
trade-off between safety and speed is acceptable.
In any case, warnings can be given (see {\bf B1}).

\begin{description}

\item [B1 : WARN USERS] ---
If the update will affect users presently on the system it is wise to inform
them (you need {\tt OPER} privilege):
\begin{verbatim}
    $ REPLY/USER/BELL "<message>"
\end{verbatim}
To display a message to any user logging into the system (you need {\tt SYSNAM}
privilege):
\begin{verbatim}
    $ ASSIGN/SYSTEM "<message>" SYS$WELCOME
\end{verbatim}
To cancel the message:
\begin{verbatim}
    $ DEASSIGN/SYSTEM SYS$WELCOME
\end{verbatim}
Only use the above for information of immediate relevance to users.
Use the NEWS system for routine announcements of software releases (see {\bf
C1}).

\item [B2 : IMPLEMENT] ---
Implement the update at RAL.
Make {\tt [STAR.TEMP]} your default directory (command {\tt TEMP} does this).
Define logical name {\tt SSCTEMP}:
\begin{verbatim}
    $ DEF SSCTEMP DISK$SOFTDEV:[STAR.TEMP.SSCn]
\end{verbatim}
Check that you have done this correctly by:
\begin{verbatim}
    $ DIR SSCTEMP
\end{verbatim}
Then, implement the release:
\begin{verbatim}
    $ @SSCTEMP:SSCn
\end{verbatim}
If you have any problems you may need to take corrective action and revise the
{\tt SSCn.LIS} and/or {\tt SSCn.COM} files.
After the command procedure has finished successfully, do any manual operations
that are required to complete the installation.
These should have been documented in {\tt SSCn.LIS}.

\item [B3 : TEST] ---
If possible, test the software you have installed to make sure that it appears
to work.
It is the responsibility of the originator to give you an appropriate test.
If there are any problems, correct the release before sending the SSC notice to
other sites (see {\bf B7}).
Ideally, every software release should be exhaustively tested before general
distribution.
In reality, this is just not possible for Starlink because of limited manpower,
remember {\em `The Best is the enemy of the Good'}.
Our aim is simply to ensure that the software submitted has been installed
as intended and works as it did for the person who submitted it.
This does not imply that this person got it to work properly in the first place.

\item [B4 : SSC FILES] ---
Copy {\tt SSCn.LIS} into {\tt [STAR.TEMP]} and print out {\tt SSCn.COM} as
soon as you are satisfied that the SSC is in a satisfactory state for general
release.

\item [B5 : SAVE SET] ---
Produce the save set {\tt SSCn.BCK} which is to be copied by other sites.
Make {\tt [STAR.TEMP]} your default directory.
To create the save set containing the release and compress, type the following:
\begin{verbatim}
    $ COMPRESS SSCn   ! this executes a command procedure COMPRESS.COM
\end{verbatim}
or explicitly
\begin{verbatim}
    $ BACKUP [STAR.TEMP.SSCn...]*.* SSCn.BCK/SAVE
    $ LZCMP -V SSCn.BCK SSCn.BCK
\end{verbatim}
Do the same for {\tt [STAR.TEMP.SSCnX]} if it exists.
Note the sizes of the save-sets in standard and compressed form, and add these
to the copy of {\tt SSCn.LIS} in {\tt [STAR.TEMP]}.

\item [B6 : SAVE SET PROTECTION] ---
Save sets need to be protected from copying between the time of creation
(after the release has been implemented at RLVAD) and the time of release
(e-mailing).
This protection should be extended to directories and the release note.

Save sets will be created by the COMPRESS command with the protection set as:
\begin{verbatim}
    (S:RE,O:RWE,G:RE,W)
\end{verbatim}
After completing the release note in [STAR.TEMP] by adding the save set sizes,
set the protection to World No Access:
\begin{verbatim}
    $ SET PROT=W SSCnnn.LIS
\end{verbatim}
Also, set the same protection on the top release directories:
\begin{verbatim}
    $ SET PROT=W SSCnnn*.DIR
\end{verbatim}
Just before release, reset World protection to W:RE:
\begin{verbatim}
    $ SET PROT=W:RE SSCnnn*.*
\end{verbatim}

\item [B7 : RELEASE] ---
Send the SSC notice to the sites on the distribution lists and to the person
who submitted the software for his/her information.
This constitutes the actual {\em software release} for remote sites and should
be carried out on the VAX cluster node that runs the Coloured Book Software
(usually the BOOT NODE).

The message should be sent to all sites on {\bf DECNET} by using the
batch mailing procedure.

\begin{verbatim}
    $ SET DEF [STAR.TEMP]
    $ BM SSCn.LIS
    Version 6.1 -- 18 April 1985
    To: @SSC_DECNET
    Subject: SSCn
    CC:
\end{verbatim}
The symbol {\tt BM} submits a batch procedure {\tt [STAR.BMAIL]BMAIL.COM}
that mails all sites in the distribution list {\tt [STAR.TEMP]SSC\_DECNET.DIS}.
This file MUST be kept up-to-date.
Any failed attempts to mail DECNET sites will be requeued for later.
The procedure reports its progress and eventually lists the sites it was
unable to contact over a 24 hour period.

Because of a bug in the Batch Mailer, it does not handle mail to {\bf JANET}
sites correctly. Use {\em either} of the following methods:

\begin{enumerate}
\item Mail the release note to all the sites in {\tt [STAR.TEMP]SSC\_JANET.DIS}
at one shot using the VMS MAIL command:
\begin{verbatim}
    $ @CBSMAILER
      File to be mailed: SSCnnn.LIS
      Distribution File (including extension): SSC_JANET.DIS
      Subject: SSCnnn
\end{verbatim}
\item Mail the release note to each site in {\tt [STAR.TEMP]SSC\_JANET.DIS}
individually using VMS MAIL.
This is a bit time consuming, but if a site is not reached, the returned
message will always indicate which site was not reached.
We use this method.
\end{enumerate}

JANET mail that cannot be sent will eventually be returned as `undelivered
mail' via the ftp\_manager.

Messages from the Batch Mailer and ftp\_manager should be stored on-line in a
mail `folder' called SSCn until the Batch Mailer has successfully reached all
DECNET sites. Note sites which cannot be contacted and try them individually
later. Read and print the {\tt SSCn.LIS} from your own MAIL.
Trim {\tt SSCn.LIS} and {\tt SSCn.COM} to A4 size and file in the
`STARLINK -- Operations' folder. When you are satisfied that all sites have been
reached, delete the contents of MAIL `folder' {\tt SSCn}.

\item [B8 : SSC\_RELEASE CONFERENCE] --- A VAXNotes conference has been set up
on RLSTAR. This conference should contain the {\tt SSCn.LIS} release notes for
each release as a separate topic.
Any associated messages should be entered as replies to the relevant topic.

\begin{verbatim}
    $ NOTES
    Notes> OPEN SSC_RELEASE
    Notes> WRITE
    (in the editor, include the file SSCn.LIS by pressing the `Do' button
    and telling the system to include SSCnnn.  Then exit.)
    ^Z
    Enter a title for your topic: SSCnnn
    Your topic is complete.  Proceed to enter it? (Y?N) <Y> Y
    Notes> EXIT
\end{verbatim}

\end{description}

\section {Clean-up}

A number of things need to be done after an SSC has been implemented.
Some can be delayed until the end of a batch of releases; use your discretion.
When printing files use the /NOFEED qualifier to save paper (or use the
{\tt PNF} command defined for {\tt STAR}), trim the output to A4 size and file.

If you need to print a complete set of information files, the command
procedure {\tt [STAR.TEMP]\-PRINT\_SSC\_DOC.COM} which will print them in three
batches.

\begin{description}

\item [C1 : NEWS] ---
If users will be noticeably affected by the release ({\em e.g.}\, they may
need to relink programs) add a message to the RAL user news system as follows
(you need {\tt SYSPRV} privilege):
\begin{verbatim}
    $ ADDNEWS
\end{verbatim}
This invokes the NEWSMAINT utility and can be used to ADD, MODIFY and DELETE
news items. A menu will be presented and you should select the required
option.
When you select [A]dd or [M]od you will find yourself in the EDT editor.
Enter the required text in the form of a HELP file; {\em e.g. }
\begin{verbatim}
      1 <keyword>
                                           <expiry date>
        <text associated with the keyword>
      2 <sub-keyword>
        <text associated with the sub-keyword>
\end{verbatim}
The numbers in front of the keywords are the HELP system level numbers.
The keywords must not include blanks.
When a user logs in he will be bleeped if there are any news messages that he
has not read.
To read the news, type
\begin{verbatim}
    $ NEWS <item>
\end{verbatim}
Delete the message after the expiry date by running the NEWSMAINT utility again.

\item [C2 : STORE NOTES] ---
File any notes associated with released items in their folders.

\item [C3 : DELETE DIRECTORY] ---
Delete directory {\tt [STAR.TEMP.SSCn]} (and any optional directories) to save
space.
You will need to set the protection of the DIR file before deletion:
\begin{verbatim}
    $ SET PROT [.SSCn...]*.DIR
    $ DELETE [.SSCn...]*.*.*
    $ SET PROT SSCn.DIR
    $ DELETE SSCn.DIR.*
\end{verbatim}
The information is still available in {\tt SSCn.BCK} and can be restored if
required by:
\begin{verbatim}
    $ BACKUP SSCn.BCK/SAVE [*...]
\end{verbatim}
The release directory tree is usually retained for a few days after release
in case of problems.

\item [C4 : DELETE SSC NOTICE] ---
Delete {\tt [STAR.TEMP]SSCn.LIS} when every site on the distribution list has
received it. Make sure you have a printout of it in the operations file.

\item [C5 : INFORMATION FILES] ---
Print any new information summaries.

\item [C6 : ENVIRONMENT FILES] ---
Print a new summary of the startup and login files if {\tt SSC:\-LO\-GIN.\-COM}
or {\tt SSC:STARTUP.COM} have been changed:
\begin{verbatim}
    $ @LADMINDIR:CPSHORT
\end{verbatim}

\item [C7 : DIRECTORY MAPS] ---
Update and print {\tt LADMINDIR:DIRMAP.LIS} illustrating the directory
structure of {\tt [STARLINK\ldots]} if new directories were created or old
directories deleted:
\begin{verbatim}
    $ @LADMINDIR:DIRMAP
    $ PNF LADMINDIR:DIRMAP
\end{verbatim}

\item [C8 : SIZE ANALYSIS] ---
An analysis of the software sizes is kept in {\tt [STAR.ADMIN.SIZES]\-ALL.LIS}.
This is an `unloaded' REXEC file. Update the file, and then load the new data:
\begin{verbatim}
      $ REXEC
      REXEC> COPY STAR_ALL STAR_ALL (0)
      REXEC> LOAD ALL.LIS STAR_ALL
      REXEC> EXIT
      $ @SIZES
\end{verbatim}
The command procedure (final step above) will print out sorted lists of the
software sizes in various categories.

\item [C9 : DOCUMENT ANALYSIS] ---
Update and print {\tt LDOCSDIR:ANALYSIS.LIS} containing the documentation
analysis if a new document number or item has been released.

\item [C10 : DOCUMENT MASTERS] ---
If new documentation ({\em e.g.}\,SUN) has been released and masters have not
been supplied, prepare them yourself.
For RUNOFF files the best current method of producing a master is to use the
Sanders printer in the lineprinter room in `letter' mode (although some people
prefer the laser printer, see LUN/27 (RAL); use the same positioning as for the
Sanders).
Masters of \LaTeX\ files should be produced on the Laser printer.

\begin{description}

\item [\LaTeX] :
These files should be suitable for processing by \LaTeX\ (this is the Starlink
recommendation).
The standard method of producing a master is:
\begin{verbatim}
    $ LATEX file
    $ DVIPSA4 file
    $ PRCN file.DVI-PS
\end{verbatim}
Output is on a laser printer. You may need to \LaTeX\ the master several times
to get the internal references correct.

\item [RUNOFF] :
Produce a suitable file for the Sanders from the RUNOFF file and print it by:
\begin{verbatim}
    $ RUNOFF/RIGHT=4 <filename>
    $ PSAN <filename>.MEM
\end{verbatim}
The value of the RIGHT parameter may need to be changed if people have fiddled
with the printer.

If the Sanders needs preparing for action, put A4 photocopying paper in its
hopper, press the bar down and press `START'.
It may need resetting by pressing the `RESET' button at the back, or by
switching it off and on.
The ribbon may need changing at any time.
\end{description}

Add diagrams or special notation if required.
Make sure the masters are clear and have good contrast as faint copies of
documents are very irritating to users.

To get copies, complete a pink form N127a for each document to be copied
specifying `Double-sided', `Stapled' and `Top left hand corner'.
The number of copies to specify varies, but at the time of writing was 70 for
SUN, SGP, SSN, SG, and 40 for LUN, LGP, LSN, MUD.
Enclose the master in a plastic folder, put in form N127a, and send them to
`Repro R3' for copying.
When they return, distribute them to everyone on the `/ALL' distribution list
(see {\tt LADMINDIR:PEOPLE.ADR}) and file the left-overs in the drawers holding
Starlink documents.
Update the document set in the terminal room.
We find it helpful to note on the blackboard when documents are sent to and
returned from Repro, otherwise we forget about them.

\item [C11 : TAPES] ---
Some items are too large (say $>$10000 blocks -- $>$1000 for the AAO) to send
over the network, or a site may request the release to be sent on tape
(usually AAO and Hawaii).
Write and post any such tapes required.
Use the sticky address labels which should be available (see LSN/28).
Packets of tapes can be left in the OUT tray.
Jiffy bags are normally adequate.


% \item [C13 : BACKUP] ---
% Backup DISK\$STAR periodically to exabyte tape.
% This can be done by typing:
% \begin{verbatim}
%    $ BUP
%    $ @STAR_IMAGE
%    select MUB0: on RLSTAR
% \end{verbatim}
% Backup any optional items which have changed.
% The tapes are kept in R68/1.02; please keep them in order and with the
% contents clearly visible --- it's a bore having to shuffle through a rack of
% tapes to find the right one.

\item [C12 : DELETE SAVE SET] ---
Copy SSC save-sets to {\tt DISK\$RLSSD1:\-[STAR]} for storage until a batch of
10 SSCs is reached.
Delete {\tt [STAR.TEMP]SSCn.BCK} and any optional save set one month after
its initial release to save disk space.
All sites on the distribution list should be aware that this will be done
without notice.
Any site that hasn't copied the save sets within a month is delinquent and
should be made to wait for recovery.

\item [C13 : BACKUP] ---
For safety, the SSC backup save-sets are archived to exabyte tape in batches
of ten SSCs. The procedure {\tt [STAR.TEMP]SSCBACK.COM} does this.
Copy the appropriate batch of save sets to {\tt DISK\$RLSSD1:[STAR]}.
Edit the procedure to change the tape save-set name and batch of save-sets
appropriately.
Load the backup tape in the exabyte drive and run the command procedure as a
batch job or interactively.

\item [C14 : NON-STARLINK SITES] ---
Prepare and send out update tapes to active sites which cannot (or will not)
use the network update method.
Do this about once a year (it's a terrible job but you might be able to delegate
it).
The non-Starlink sites are listed in {\tt LADMINDIR:EXTERNAL.LIS} which should
be kept up-to-date. Refer to the `External Distribution' folder.

\end{description}

\section {References}

\begin{description}
\begin{description}
\item [SGP/19]: Starlink software submission
\item [SGP/20]: Starlink software management
\item [LUN/26 (RAL)]: SPELL --- Spelling checker
\item [LUN/27 (RAL)]: R1LASER  --- Printing files on the R1 laser printer
\item [LUN/47 (RAL)]: STRIP --- Strip blanks from the end of line
\item [LSN/28 (RAL)]: STICKY --- Produce address labels from an address list
\end{description}
\end{description}

\appendix

\newpage
\section {Example SSC Command Procedure}
\label{se:ex.proc}
\begin{quote}
\begin{verbatim}
$!	SSC550.COM
$!
$!	2.19 GKS 7.2 v1.0-1
$!	2.41 PSX v1.0-2
$!
$!	M J Bly 07.01.92
$!
$	SET PROTECTION=(S:RE,O:RWED,G:RE,W:RE)/DEFAULT
$	SAVE_DIR = F$ENVIRONMENT("DEFAULT")
$	SAVE_VERIFY = F$VERIFY()
$	SET VERIFY
$	ON WARNING THEN GOTO EXIT
$!
$! Implement update
$!
$! GKS 7.2 v1.0-1
$	SET DEFAULT SSCTEMP
$	COPY [.GKS]*.* GKS_DIR:*.*
$	PURGE GKS_DIR
$!
$! PSX v1.0-2
$	SET DEFAULT SSCTEMP
$	COPY [.PSX]*.* PSX_DIR:*.*
$	SET DEFAULT PSX_DIR
$	DELETE PSX_ERR.H;*
$	PURGE
$!
$! Copy other files ready for release:
$!
$	COPY SSCTEMP:STARTUP.COM SSC:*.*
$	PURGE SSC
$!
$! Update information summaries
$!
$	COPY SSCTEMP:NEWS.LIS      DOCSDIR:NEWS.LIS
$	COPY SSCTEMP:SSI.LIS       ADMINDIR:SSI.LIS
$!
$	LIB  ADMINDIR:SSCLIS.TLB   SSCTEMP:SSC550.LIS
$	LIB  ADMINDIR:SSCCOM.TLB   SSCTEMP:SSC550.COM
$!
$	PURGE DOCSDIR:
$	PURGE ADMINDIR:
$!
$! Record this update in ADMINDIR:STATUS.LIS
$!
$	OPEN/APPEND STATUS         ADMINDIR:STATUS.LIS
$	SSCTIME=F$TIME()
$	SSCUSER=F$USER()
$	WRITE STATUS "SSC550 implemented ",SSCTIME, " by ",SSCUSER
$	CLOSE STATUS
$!
$ EXIT:
$	SET DEF 'SAVE_DIR'
$	IF .NOT. SAVE_VERIFY THEN SET NOVERIFY
$	EXIT
\end{verbatim}
\end{quote}

\newpage
\section {Example SSC Notice}
\label{se:ex.note}
\begin{verbatim}
SSC550 - STARLINK SOFTWARE CHANGE 550

The following software items have been modified at RAL.
Please install at your own site.

ITEM:   2.19 GKS 7.2 v1.0-1                             (*** Cock up fix ***)
        Submitted and supported by Dave Terrett

ITEM:   2.41 PSX v1.0-2                                     (*** Re-link ***)
        Submitted and supported by Peter Allan

SSC550  Directory size: 809 blocks (3 dirs, 26 files)
        Environment:VMS 5.4
        Save set: blocks normal, blocks compressed

Installation instructions:

1.      Install a copy of directory DISK$SOFTDEV:[STAR.TEMP.SSC550] and give
        it the logical name SSCTEMP (include the device name if this is
        different from the device holding the Starlink software).

2.      Execute the command procedure SSCTEMP:SSC550.COM.

3.      The logical name GKS_IMAGE_ADAM has been added to SSC:STARTUP.COM.
        Define it on all nodes on your cluster. It requires the .EXE suffix as
        shown.

                $ DEFINE/SYSTEM GKS_IMAGE_ADAM GKS_DIR:GKS_IMAGE_ADAM.EXE

4.      If you have the other GKS shareable images installed as known images,
        install GKS_IMAGE_ADAM.EXE and replace GKS_IMAGE.EXE on all nodes in
        your cluster:

                $ MCR SYSMAN
                SYSMAN> SET ENV/CLUSTER
                SYSMAN> SET PROFILE/PRIV=(SYSPRV,CMKRNL)
                SYSMAN> DO INSTALL ADD GKS_DIR:GKS_IMAGE_ADAM.EXE /SHARED
                SYSMAN> DO INSTALL REPLACE GKS_DIR:GKS_IMAGE.EXE
                SYSMAN> EXIT
                $

         Add an entry to your LSSC:STARTUP.COM to ensure the GKS images are
         installed with the rest of the installed images:

                ADD GKS_DIR:GKS_IMAGE_ADAM.EXE /SHARED
                ADD GKS_DIR:GKS_IMAGE.EXE /SHARED
                ADD GKS_DIR:GKS_WS_IMAGE.EXE /SHARED

        Remove the entry for GKS_DIR:GKSSHARE.EXE in the same section.

        Please NOTE: The modification to LSSC:STARTUP.COM regarding image
        installation for GKS_IMAGE.EXE and GKS_WS_IMAGE.EXE should have
        occurred at SSC546. The omission of this change has not effected GKS.

5.      Rebuild the non-ADAM shareable image library:

                $ SET DEFAULT LIB_COMMON_DIR:
                $ @STAR_LINK

Documentation:

        No Change

Notes:

1.      PSX v1.0-2 is a re-link to tidy up the shareable images. The PSX images
        PSX_IMAGE and *_ADAM) are now linked with EMG_IMAGE(_ADAM) as
        appropriate instead of ADAMSHARE and the EMS object Library.

        This is a complete release of PSX, but the only changes that affect
        users are that PSX_CALLOC and PSX_MALLOC now report how many bytes were
        being requested if they fail.

        There is no change to the documentation.

        A VERSION.NUM file has been added to PSX at this release.

2.      GKS 7.2 v1.0-1 corrects a DLT `cock up' (his  phrase, not mine!). There
        is no effect on users.


Other files released in this SSC:

        SSC:STARTUP.COM

Information summaries updated:

        DOCSDIR:  NEWS
        ADMINDIR: SSI

Martin Bly 07.01.92
\end{verbatim}

\newpage

\section {Installing Software at Remote Sites}

The directory containing the files for release SSCn is stored at RAL as a
compressed save set in {\tt DISK\$SOFTDEV:\-[STAR.\-TEMP]\-SSCn.\-BCK}.
Copy this to your local site and install the release.
There may be an optional save set called {\tt SSCnX.BCK} in the same directory.
This need not be copied if you do not want to install the software released
(if this is an optional item).
However, you should {\em always} copy {\tt SSCn.BCK} and run the command
procedure {\tt SSCn.COM} that it contains.
Also, you should install updates in their numerical order since information
files are updated in every release.

\begin{description}

\item [COPY]: Copy {\tt SSCn.BCK} and any optional save sets from RAL.
If on DECNET, store the command:
\begin{verbatim}
    $ COPY RLVAD::DISK$SOFTDEV:[STAR.TEMP]SSCn.BCK disk:[STAR.TEMP]*.*
\end{verbatim}
in file CPYSSC.COM, then submit it to a batch queue:
\begin{verbatim}
    $ SUBMIT CPYSSC
\end{verbatim}
If not on DECNET, use the TRANSFER command:
\begin{verbatim}
    $ TRANSFER/CODE=FAST/USERNAME=NETUSR,NETUSR -
         UK.AC.RL.STAR::DISK$SOFTDEV:[STAR.TEMP]SSCn.BCK -
         disk:[STAR.TEMP]SSCn.BCK
\end{verbatim}

\item [RECREATE]: Recreate the {\tt [STAR.TEMP.SSCn]} directory:
\begin{verbatim}
    $ SET DEF [STAR.TEMP]
    $ LZDCM  SSCn.BCK SSCn.BCK
    $ PURGE
    $ BACKUP SSCn.BCK/SAVE [*...]
\end{verbatim}
Use [name...] instead of [*...] to put the directory structure in the directory
of your choice.
Notice that you must de-compress the copied version of the save-set using
LZDCM before using BACKUP to recreate the directory.
People of a nervous disposition may use the qualifiers /LOG and /VERIFY in the
BACKUP command if this makes them feel happier.

\item [IMPLEMENT]: Follow the implementation instructions in {\tt SSCn.LIS}
(which you should already have available) to install the release.

\item [CLEAN-UP]: Carry out any actions of the CLEAN-UP phase that you consider
necessary.
\end{description}

BACKUP can be a tricky command to use.
Here are some versions we have found useful:

To copy a directory tree to a specific location:
\begin{verbatim}
   $ BACKUP [...] [name...]
\end{verbatim}
To copy a subdirectory and its lower branches:
\begin{verbatim}
   $ BACKUP/LOG name.BCK/SAVE/SEL=[...DIR...] [...]
\end{verbatim}

\rule{\textwidth}{0.3mm}
\begin{quote}\large
N.B.\ You must copy the SSCn.BCK file from RAL within a month of notification
of a release as the file will disappear without notice after this time.
The file may not be backed-up, so you may have to wait a long time to be
rescued if you have not copied it.
\end{quote}
\rule{\textwidth}{0.3mm}

\end{document}
