\documentstyle[11pt]{article} 
\pagestyle{myheadings}

%------------------------------------------------------------------------------
\newcommand{\stardoccategory}  {Starlink System Note}
\newcommand{\stardocinitials}  {SSN}
\newcommand{\stardocnumber}    {15.8}
\newcommand{\stardocauthors}   {M.\ J.\ Bly, M.\ D.\ Lawden, D.\ L.\ Terrett}
\newcommand{\stardocdate}      {20 November 1992}
\newcommand{\stardoctitle}     {Starlink Software Installation --- VMS}
%------------------------------------------------------------------------------

\newcommand{\stardocname}{\stardocinitials /\stardocnumber}
\markright{\stardocname}
\setlength{\textwidth}{160mm}
\setlength{\textheight}{230mm}
\setlength{\topmargin}{-2mm}
\setlength{\oddsidemargin}{0mm}
\setlength{\evensidemargin}{0mm}
\setlength{\parindent}{0mm}
\setlength{\parskip}{\medskipamount}
\setlength{\unitlength}{1mm}

\begin{document}
\thispagestyle{empty}
SCIENCE \& ENGINEERING RESEARCH COUNCIL \hfill \stardocname\\
RUTHERFORD APPLETON LABORATORY\\
{\large\bf Starlink Project\\}
{\large\bf \stardoccategory\ \stardocnumber}
\begin{flushright}
\stardocauthors\\
\stardocdate
\end{flushright}
\vspace{-4mm}
\rule{\textwidth}{0.5mm}
\vspace{5mm}
\begin{center}
{\LARGE\bf \stardoctitle}
\end{center}
\vspace{5mm}

%------------------------------------------------------------------------------
%  Add this part if you want a table of contents
\setlength{\parskip}{0mm}
\begin{small}
\tableofcontents
\end{small}
\setlength{\parskip}{\medskipamount}
\markright{\stardocname}
%------------------------------------------------------------------------------

\newpage

\section {Introduction}
\label{se:intro}

This paper describes, in general terms, the characteristics of the Starlink
Software Collection and how to install it on a VAX computer. It is meant for
the System Manager of that VAX since some integration with and adaptation to
the local systems software will be necessary.

\vfill

The Starlink Software Collection is a set of astronomical and general purpose
software put together and maintained by the Starlink project. The definition
and management of the Collection is described in SGP/20. It is not a tightly
integrated system but a set of software items, some of which are autonomous.
The Starlink Software Librarian manages it, but does not have extensive
knowledge about individual items. Each item is maintained by an individual
named in the Starlink Software Index ({\tt RLVAD::LADMINDIR:SUPPORT.LIS}); some
items may become unsupported because of staff changes. The general level of
support for each item is indicated in {\tt ADMINDIR:SUPPORT.LIS}.

\vfill

The non-proprietary part of the Collection may be distributed to non-Starlink
sites at the discretion of the Starlink Software Librarian, however you should
not redistribute Starlink software without his prior approval. The Collection
contains some proprietary software which cannot be distributed outside Starlink
without the consent of the proprietor. The distribution policy is explained in
more detail in SGP/21. Some items in the non-proprietary part depend on items
in the proprietary part.

\vfill

The Collection is not a commercial package, therefore only very limited support
and assistance can be provided by Project staff to non-Starlink users. It is
undergoing continual development; updates are available on request, subject to
the approval of the Project Manager and the availability of the effort required
to do the distribution.

\vfill

The Collection was created specifically for the VAXs at Starlink sites and is
adapted to their specific hardware and software environments. No guarantee can
be given that any particular item will work on a non-Starlink VAX, although the
Collection has been installed successfully on many such VAXs and microVAXs. The
effort and management controls necessary to ensure that it is portable,
exhaustively documented and tested are not available. Ultimately, it is up to
the recipient to solve whatever problems arise when installing it on a
non-Starlink VAX.

\vfill

Starlink is always anxious to improve the correctness  and portability of its
software. If you experience any installation difficulties or operational
problems, please send clear and specific details to:

\vfill

\begin{center}
\parbox{80mm}
{Starlink Software Librarian,\\
Building R68,\\
Rutherford Appleton Laboratory,\\
Chilton, DIDCOT, Oxon, OX11 0QX,\\
UNITED KINGDOM}

\vspace*{5mm}

\parbox{80mm}
{Tel: 0235 445363\\
Fax: 0235 445848 \\
Telex: 83159 RUTHLB G \\
Email: STAR@UK.AC.RL.STAR \quad (JANET) \\
Email: star@star.rl.ac.uk \quad (Internet) }
\end{center}

\newpage

\section {Configuration requirements}
\label{se:config}

Specifications of the hardware and software requirements of each item are not
available. However, the following is a summary of the prerequisites for
successful installation and full use of the Collection as a whole.

\begin{description}

\item [\bf Processor]: Any computer with the DEC VAX architecture should be
suitable.

\item [\bf Operating System]: VAX VMS (currently version 5.5), and 
DECwindows-Motif. You may experience difficulties if your version of VMS
differs from that at RAL.

\item [\bf Disk-space]: The Collection has seven separate parts. Their size is
continually changing; at the time of writing (revision level 598) they were:

\begin{itemize}
\item Standard Software -- (151 Mbyte)
\item Secure Software -- (0.3 Mbyte)
\item Optional Software --  (280 Mbyte)
\item Restricted Software -- (132 Mbyte)
\item Devolved Software -- (\verb+>+230 Mbyte)
\item Local Software (RAL) -- (35 Mbyte)
\item Data (Star catalogues {\em etc}) -- (569 Mbyte)
\end{itemize}

Thus, the total disk space required is approximately 1400 Mbyte. This can be
reduced by deleting unwanted items or files (see section \ref{ss:reduce}). 

New non-Starlink sites are normally sent only the Standard software, including
GKS, plus any optional items specifically requested, so 200 Mbyte is often
enough space. 

\item [\bf Magnetic Tape]: Normally, software will be supplied as VMS Backup
save-sets on Exabyte cassette. By request, DEC TK50 and TK70 cartridges can be
provided, as can industry standard magnetic tape at densities of 6250 bpi or
1600 bpi.

\item [\bf Line Printer]: For example the Printronix P300.

\item [\bf Graphics terminals]: Pericom MG100, Pericom Graphpac; Tektronix
T4010 and T4014 are also suitable. See also the section on GKS workstations in
SUN/83.

\item [\bf Graphics hardcopy]: Starlink has a considerable variety of hardcopy
devices. At RAL there is a Printronix P300 lineprinter, Canon LBP-8III Laser
printer (operating in PostScript mode), a Zeta (NICOLET) plotter, and Honeywell
Color Graphics Recorder.

\item [\bf Image Display]: Digisolve Ikon and Xwindows display devices. These
can display a colour picture and have a number of advanced features such as
zoom, pan and colour lookup table. The maximum format of the Ikon is 1024x780.

\item [\bf GKS]: GKS (version 7.2) is commercial software. It can only be
distributed to people who use other Starlink software for non-commercial
astronomical research. Prospective users must complete a form which is
obtainable from the Starlink Software Librarian at RAL, or by copying
Appendix~C of SGP/21.

\end{description}

\section {Structure}
\label{se:struct}

The Starlink Software Collection comprises seven parts:
\begin{description}
\begin{description}
\item [Standard Software]
\item [Secure Software]
\item [Optional Software]
\item [Restricted Software]
\item [Devolved Software]
\item [Local Software]
\item [Data]
\end{description}
\end{description}

\paragraph{Standard Software:} comprises several directory trees:

\begin{itemize}

\item {\tt [STARLINK...]} --- general purpose software that can be distributed at
the discretion of the Software Librarian, and administrative information such
as user lists and document indexes. This is also referred to as the {\bf Core
Software}.

\item {\tt [ADAM...]} --- the Starlink software environment for data reduction
and image display applications. This software is needed to run many of the
applications packages in the {\bf Option set}.

\item {\tt [KAPPA...]} --- a comprehensive applications package for image
processing and display.

\item {\tt [CONVERT]} --- a set of data format conversion routines.

\item {\tt [ADAMAPP...]} --- This suite is currently being split up into
separate items in the Standard and Option sets, and now only holds SST.

\item {\tt [FIGARO...]} --- a large set of spectroscopy and related
applications. 

\item {\tt [FIGPACK...]} --- some additional applications associated with
FIGARO, and the applications in the main FIGARO tree that have had bugs fixed,
or have been enhanced or altered for UK astronomical use. 

\end{itemize}

\paragraph{Secure Software:} a small set of systems software distributed
only to Starlink sites in the UK, concerned generally with Starlink Networking.
The Secure set is contained in:

\begin{itemize}
\item {\tt [STARSEC...]}
\end{itemize}

\paragraph{Restricted Software:} is the set of mainly commercial applications
software for which Starlink licences its own sites in the UK. In some cases the
software is only installed on one central machine at RAL. This software is not
distributed outside Starlink. Potential users should contact the vendors for
more information (see SGP/21). The restricted set currently comprises:

\begin{itemize}
\item {\tt [CLUSTAN]} --- specialised multivariate cluster analysis
\item {\tt [FORCHECK]} --- Fortran standards verifier
\item {\tt [GENSTAT...]} --- GENSTAT general statistical analysis
\item {\tt [IDL...]} --- Image display and manipulation package
\item {\tt [MAPLE...]} --- mathematical manipulation language
\item {\tt [PRISM\_314]} --- Fortran `spaghetti unscrambler'
\item {\tt [JM.USSP...]} --- for access to the IUE low-dispersion archive
\end{itemize}

\paragraph{Optional Software:} comprises the majority of the applications in
the  Software Collection. They are generally large, and are not distributed
unless specifically requested. Some require ADAM and/or FIGARO to be installed
in order to run. Commercial items are indicated thus: {\tt *}.

\begin{itemize}
\item {\tt [APIG...]} --- APIG
\item {\tt [ASPIC...]} --- ASPIC/DSCL
\item {\tt [ASTERIX88...]} --- ASTERIX
\item {\tt [CATPAC]} --- CATPAC
\item {\tt [CCDPACK]} --- CCDPACK
\item {\tt [CHI...]} --- CHI
\item {\tt [DAOPHOT2]} --- DAOPHOT
\item {\tt [DIPSO...]} --- DIPSO
\item {\tt [ECHWIND]} --- ECHWIND
\item {\tt [FITSIO]} --- FITSIO
\item {\tt [HRTS...]} --- HRTS
\item {\tt [HXIS...]} --- HXIS
\item {\tt [IRAS...]} --- IRAS
\item {\tt [IRCAM]} --- IRCAM
\item {\tt [IUEDR...]} --- IUEDR
\item {\tt [JCMTDR...]} --- JCMTDR
\item {\tt [JPL]} --- JPL
\item {\tt [LTEX...]} --- LTEX
\item {\tt [MEMSYS...]} --- MEMSYS*
\item {\tt [MONGO]} --- MONGO*
\item {\tt [NAG...]} --- NAG*
\item {\tt [NOD2]} --- NOD2
\item {\tt [PHOTOM]} --- PHOTOM
\item {\tt [PISA]} --- PISA
\item {\tt [QDP...]} --- QDP
\item {\tt [REXEC]} --- REXEC
\item {\tt [RGASP]} --- RGASP
\item {\tt [SAM...]} --- SAM
\item {\tt [SCAR53...]} --- SCAR
\item {\tt [SCP...]} --- SCP
\item {\tt [SPECDRE...]} --- SPECDRE 
\item {\tt [SPECX...]} --- SPECX
\item {\tt [STARMAN...]} --- STARMAN
\item {\tt [TEX...]} --- TEX
\item {\tt [TOOLPACK...]} --- TOOLPACK*
\item {\tt [TPOINT]} --- TPOINT
\item {\tt [TSP]} --- TSP
\item {\tt [ROSAT...]} --- WFCSORT
\end{itemize}


\paragraph{Devolved Software:} is the classification of items which are not
distributed by Starlink, but are included in the index files for reference
because the software is installed at many Starlink sites.

\begin{itemize}
\item {\tt [AIPS]} --- AIPS
\item {\tt [IRAF]} --- IRAF
\item {\tt [MIDAS]} --- MIDAS
\end{itemize}

\paragraph{Local Software:} comprises two directories:

\begin{itemize}

\item {\tt [STARLOCAL]} --- local modifications to the Collection and local
administration files. Its structure should be similar to {\tt [STARLINK]}. Its
contents is decided locally, but RAL can supply their version as a starting
point. It is not well controlled or documented. References to Local Software in
this paper ({\em e.g.}\ {\tt LDOCSDIR:ONLINE.LIS}) refer to the RAL version.

\item {\tt [STARLHOLD]} --- local proprietary software which is not distributed.
\end{itemize}

\paragraph{Data:} currently comprises the following directories:

\begin{itemize}
\item {\tt [STARCATS]} --- Star catalogues for CHART.
\item {\tt [SCARDATA]} --- SCAR online database.
\item {\tt [CDS]} --- SCAR offline database.
\end{itemize}

Starlink directories are usually referred to by logical names defined in
{\tt SSC:STARTUP.COM} and {\tt LSSC:\-STARTUP.\-COM}. In particular:

\begin{description}
\begin{description}
\item [{\tt SSC}] is the logical name for \verb+<disk>:[STARLINK]+
\item [{\tt LSSC}] is the logical name for \verb+<disk>:[STARLOCAL]+
\end{description}
\end{description}

\begin{quote}\em
There is a constant turnover of software items and the information in this
paper rapidly gets out-of-date. The latest information on the current items is
stored in {\tt ADMINDIR:SSI.LIS}
\end{quote}

\section {Activation}
\label{se:active}

Once one or more parts of the Collection have been installed and adapted to the
local system (see sections 5 and 6), it is activated in two phases:

\begin{itemize}
\item Startup Phase
\item Login Phase
\end{itemize}

\subsection {Startup phase}
\label{ss:startup}

This occurs when the VAX is booted. The System Manager is responsible for what
happens during this phase. It is concerned with the general configuring of VMS
on a specific VAX. The following command procedures are particularly
significant:

\begin{enumerate}
\item {\tt SYS\$STARTUP:SYSTARTUP\_V5.COM}
\item {\tt SSC:STARTUP.COM}
\item {\tt LSSC:STARTUP.COM}
\end{enumerate}

Procedures 2 and 3 are called from 1. As far as the Collection is concerned,
their main functions are to:

\begin{itemize}
\item Define system logical names
\item Install known images
\end{itemize}

\subsection {Login phase}
\label{ss:login}

This is controlled by the User Authorization File (UAF) which specifies the
command procedure to be executed each time a particular user logs in and
establishes his quotas and privileges. At Starlink sites, a normal user will
have the following procedures executed at login time:

\begin{enumerate}
\item {\tt SYS\$MANAGER:SYLOGIN.COM}
\item {\tt SYS\$LOGIN:LOGIN.COM}
\end{enumerate}

If a user wishes to use items in the Collection, he must call the following
procedure from his own {\tt LOGIN.COM} file:

\begin{itemize}
\item {\tt SSC:LOGIN.COM}
\end{itemize}

This executes the local Starlink Login procedure, if it exists:

\begin{itemize}
\item {\tt LSSC:LOGIN.COM}
\end{itemize}

The main function of the Starlink Login procedures is to define Global Symbols.
The quotas and privileges needed by some items are discussed in Section
\ref{se:quotas} .

\section {Installation}
\label{se:install}

The precise procedure for installing the Collection is clearly at the
discretion of the Site Manager who will be aware of particular constraints
affecting his system. However, the general procedure is as follows:

\begin{enumerate}
\item Study Documentation.
\item Store Starlink Directories on Disk.
\item Install Items.
\item Activate Collection.
\item List Key documentation.
\item Test Collection.
\item Reduce Size of Collection.
\end{enumerate}

The process by which the tapes containing the Collection were produced is
described in Appendix \ref{se:tapes}. A summary of the actions required in
steps 2, 3 and 4 is given in Appendix \ref{se:actions}. It is suggested that
you read this paper fully before attempting to install the Collection and then
use Appendix \ref{se:actions} to guide your actions during the actual
installation.

\subsection {Study documentation}
\label{ss:studydocs}

The tapes supplied should be accompanied by the following documentation:

\begin{description} 
\begin{description} 
\item [\bf RELEASE]: Release note (describes what is on the tape) 
\item [\bf ACKNOWLEDGEMENT]: Acknowledgement note (return to Software Librarian)
\item [\bf GKS]: Copy of the GKS form (return to Software Librarian)
\item [\bf SSN/15]: `Starlink Software Installation' (this document) 
\item [\bf SGP/25]: `Starlink Site Manager's Guide --- major nodes' 
\item [\bf SSN/39]: `GKS --- Installation and Modification'
\item [\bf SUG]: `Starlink User Guide' 
\item [\bf ONLINE]: On-line documentation index 
\item [\bf WHOSWHO]: Starlink Who's Who
\item [\bf STARTUP]: System startup procedure ({\tt
SYS\$STARTUP:SYSTARTUP\_V5.COM}) 
\end{description} 
\end{description}

A large amount of documentation is stored in the Collection. This is indexed in
{\tt LDOCSDIR:\-ONLINE.\-LIS}, {\tt DOCSDIR:\-DOCS.\-LIS}, {\tt
DOCSDIR:\-MUD.\-LIS} and {\tt LDOCSDIR:\-DOCS.\-LIS}. Selections of non-\LaTeX\
documents may be printed once {\tt [STARLINK.\-DOCS]} has been stored. \LaTeX\
documents should normally be studied in the form of the paper documents
supplied; however, you may be able to produce masters by running \LaTeX\ on the
appropriate `{\tt .TEX}' file. The contents of the files should be obvious from
their names, thus {\tt DOCSDIR:\-SUN33.\-TEX} holds the \LaTeX\ source for
SUN/33. Of particular interest at the planning stage are: 

\begin{description} 
\begin{description}
\item [\bf SUN/1]: `Starlink Software -- An Introduction' 
\item [\bf SSN/33]: `Configuring Queues for Starlink Graphics' 
\item [\tt ADMINDIR:SSCLIS.TLB]: Modules contain Starlink Software
Change notices describing the updates to the Collection (large). 
\item [\tt ADMINDIR:SSCCOM.TLB]: Modules contain command procedures associated
with the updates (large). 
\end{description} 
\end{description}

\subsection {Store Starlink Directories on Disk}
\label{ss:store}

The tapes you receive will probably contain a save set {\tt STAR.BCK} holding
{\tt [STARLINK]} and possibly a save set {\tt STARLOCAL.BCK} containing RAL's
version of {\tt [STARLOCAL]}. They may also contain one or more other save sets
containing Optional Software and Data directories. These directories may be
stored on different disks. Before carrying out the installation, delete any
existing copies of the supplied directories.

\begin{enumerate}

\item Allocate tape deck:

\begin{verbatim}
      $ ALLOC Mxxx TAPE
\end{verbatim}

\item Load first tape onto allocated deck and copy {\tt [STARLINK]} from tape 
to disk:

\begin{verbatim}
      $ BACKUP/REWIND/NOASSIST TAPE:STAR.BCK <device>:[*...]
\end{verbatim}

By default, the files will be written with the UIC set to the users's current
default. At RAL, all Starlink files have a UIC of {\tt [STAR]}. If you want to
preserve this, qualify the output file with `{\tt /OWNER\_UIC=ORIGINAL}'. If
you want to specify a UIC different from the original and your default, qualify
it with `{\tt /OWNER\_UIC=[uic]}'. You may wish to add the qualifier `{\tt
/VERIFY}' which will cause the created directories to be compared with those on
tape, or the qualifier `{\tt/LOG}' which will cause the name of each file
copied to be displayed on {\tt SYS\$OUTPUT} (warning: there are thousands of
files).

\item Copy the remaining Standard software items from tape to disk, as for {\tt
[STARLINK]}, substituting the relevant save set name (see the {\bf RELEASE}
note), and omitting the {\tt /REWIND} qualifier if you do not need to rewind
the tape to find the file.

\item If you have been sent any Optional Software or Data, deal with these as
above.

\end{enumerate}

\subsection {Install items}
\label{ss:install}

After Starlink directories have been stored on disk, it is still necessary to
carry out further actions to integrate them with the local environment. The
specific actions required are described in section \ref{se:specific} for each
item which needs attention. Items are referred to by the acronym used to
identify them in the Starlink Software Index ({\tt ADMINDIR:SSI.LIS}). One part
of the Core is fundamental in that it creates an environment upon which most
of the other items depend:

\begin{quote}
{\bf Startup, Shutdown and Login files}
\end{quote}

Other items are essentially systems software which may need to be integrated
with the local system; at the time of writing, they are:

\begin{quote} 
{\bf PSSMB} --- Print symbiont for non-DEC PostScript printer
\end{quote}

Other items may need special consideration. These are mentioned in Section
\ref{ss:applications}.

If you have problems with an item, there are two sources of information that
might help you:

\begin{itemize}
\item Starlink Software Change documentation ({\tt ADMINDIR:SSCLIS.TLB},
{\tt SSCCOM.TLB})
\item Starlink Notes and Papers (SUN, SGP, SSN)
\end{itemize}

The relevant documents are specified in the entries in the Starlink Software
Index and in {\tt DOCSDIR:ANALYSIS.LIS}. Sometimes, users have problems with an
item because of a lack of quotas or privileges; read Section \ref{se:quotas} if
you encounter such problems. The standard protection for Starlink files is {\tt
(RE,RWED,RE,RE)}. An exception to this are the files in {\tt [ASPIC.ARGSDB]}
which should have the protection {\tt (RWE,RWED,RWE,RWE)}.

\subsection {Activate Collection}
\label{ss:activate}

\subsubsection {Startup phase}
\label{subss:startup}

Activate the Collection by executing the commands you have added to
{\tt SYS\$STARTUP:\-SYSTART\-UP\_V5.\-COM}.

This can be done by either rebooting, or explicitly typing them in.

\subsubsection {Login phase}
\label{subss:login}

Execute the login commands:
\begin{verbatim}
      $ @SSC:LOGIN
\end{verbatim}

Add this command to your login command file so it will be executed automatically
in future.

\subsection {List Key Documentation}
\label{ss:listdocs}

Starlink documentation is described in the Starlink User Guide ({\tt 
DOCSDIR:SUG.TEX}). It is stored in directories {\tt DOCSDIR:} and {\tt
LDOCSDIR:} and indexed by files {\tt LDOCSDIR:ONLINE}, {\tt DOCSDIR:DOCS}, {\tt
DOCSDIR:MUD}, {\tt LDOCSDIR:DOCS} and {\tt DOCSDIR:SUBJECT}. Other
documentation is stored elsewhere in the Collection and is referenced by
Starlink documents; print out what you need (non-\LaTeX\ documents only) or
refer to the supplied documentation. \LaTeX\ documents will normally be
supplied on paper as you may not be able to process the {\tt .TEX} files with
\LaTeX\ or have a suitable output device such as a laser printer.

\subsection {Test Collection}
\label{ss:testit}

A comprehensive test package for the Collection does not exist.
Some items do have tests provided; these are usually described in their
associated Starlink Software Change notices (see {\tt ADMINDIR:SSCLIS.TLB} and
{\tt ADMINDIR:SSI.LIS}) or SUN's.
In other cases you will simply have to try to run the program and see if you
get sensible results.

\subsection {Reduce size of Collection}
\label{ss:reduce}

The Collection is large enough to cause on-line storage problems at some sites.
Three methods are recommended for reducing its size:

\begin{itemize}

\item Delete files corresponding to unwanted software items. These can be
identified by examining the Starlink Software Index. You can delete the
definitions of logical names and symbols in the Starlink Startup and Login
files for uninstalled items.

\item Delete files not needed by users; these are mainly those containing
source code.

\item Delete the {\tt .LIS} forms of files in DOCSDIR. Generate readable
versions of these files when required. 

\end{itemize}
We recommend that whenever the Collection is truncated, details of what has been
omitted are recorded in file {\tt LADMINDIR:SURGERY.LIS}.

\section {Specific items}
\label{se:specific}

If you have trouble with any particular software item, look in {\tt
DOCSDIR:\-ANALYSIS.\-LIS} and identify the documents associated with the item.
In particular, look for any release notes or installation instructions (see
\ref{ss:applications}).

\subsection{Fundamental items}
\label{ss:fundamental}

\begin{description}

\item [Startup, Shutdown, Login files] --- We assume you want the Starlink
logical names to be defined as system logical names at startup time and that
you have command procedure {\tt SYS\$STARTUP:SYSTARTUP\_V5.COM} called from 
{\tt SYS\$\-SYSTEM:\-STARTUP.\-COM}.

Study {\tt SYS\$STARTUP:SYSTARTUP\_V5.COM} in the listing provided; this shows
what happens at RAL during a system boot.

Note 2 in Appendix \ref{se:actions} shows what extra statements will probably
need to be added to your version of this file.

\begin{enumerate}
\item Define {\tt STARDISK} as the disk holding {\tt [STARLINK]},
and define {\tt LSTARDISK} as the disk holding {\tt [STARLOCAL]} if provided.

These disks can be different, but if so they must both be on-line during a
system boot.

\item Study {\tt SSC:STARTUP.COM} and make it compatible with your local system.
You may wish to delete definitions which are associated with items or
directories you will not use. 

You will need to add logical name definitions for optional items to {\tt
SSC:STARTUP.COM} if {\tt LSSC:STARTUP.COM} is not provided. Many Option set
packages have {\tt package\_LSSC\-\_STARTUP.COM} files which act as templates.

\item If provided, study {\tt LSSC:STARTUP.COM} and make it compatible with
your local system.

Notice that near the end of the procedure several Starlink executable images
are installed. Delete these statements if the images do not exist or you do not
want them installed.

\item Study {\tt SSC:\-LOGIN.\-COM} and (if provided) {\tt LSSC:\-LOGIN.\-COM}.

Option set package startup commands are set to produce a `package not installed
message' by default in {\tt SSC:LOGIN.COM}, and are overridden in 
{\tt LSSC:LOGIN.COM}. If you are not provided with this you will need to add
the startup commands to  {\tt SSC:LOGIN.COM} --- many packages now have 
a {\tt package\_LSSC\-\_LOGIN.COM} file which should be used as a template.

No modifications should be necessary, although you may wish to delete
definitions you think will not be used.

\end{enumerate}

Any user who wishes to use the Collection should include the command:
\begin{verbatim}
      $ @SSC:LOGIN			
\end{verbatim}
in his own login command procedure; please inform them.

{\tt SSC:SHUTDOWN.COM} doesn't do anything significant and can be deleted.

\item [Graphics] --- GKS {\em etc}

The Starlink software graphics systems use GKS, included in the Standard set.
GKS and its associated systems need to be customised to your site specific
requirements, though the supplied version (customised for use on the Starlink
Project VAXcluster at RAL) will run at most sites.

\begin{enumerate} \item Customise GKS for your system according to the
instructions in SSN/39. \item Customise your GNS Graphics names service data
files to contain references to your chosen devices, both for GKS and IDI. The
files  {\tt GNS\_DIR:GKSNAMES.TEMPLATE} and {\tt GNS\_DIR:IDINAMES.TEMPLATE}
provide templates for the working versions {\tt GNS\_DIR:\-GKSNAMES.\-DAT} and
{\tt GNS\_DIR:\-IDINAMES.DAT}.

\end{enumerate}

\end{description}

\subsection {Systems Software}
\label{ss:systems}

\begin{description}

\item [PSSMB] --- Postscript printer symbiont.

PSSMB is the printer symbiont used by Starlink to drive non-DEC printers for
PostScript printing. If you have a PostScript printer and are not already using
a suitable symbiont, PSSMB may be used.

In its supplied from the symbiont is suitable for queues to printers connected
via terminal servers and LAT. If you are using a directly connected printer,
the symbiont must FIRST be rebuilt according to the details in  {\tt
STARDISK:\-[STARLINK.\-SYSTEM.\-PSSMB.\-SOURCE]\-README\-.TXT}.

\begin{enumerate}

\item Copy the executable image {\tt PSSMB\_DIR:PSSMB.EXE} to 
{\tt SYS\$COMMON:[SYSEXE]}. 

\item Edit your queue definition and startup command procedures so that the
queue for the  PostScript printer  uses the PSSMB symbiont. The file {\tt
PSSMB\_DIR:\-DEFINE\_QUEUES.\-TEMPLATE} may be used as a template. 

\item Redefine the queue using the new symbiont.

\end{enumerate}

\end{description}

\subsection {Applications Software}
\label{ss:applications}

There are a number of Starlink System Notes which deal with the installation
of specific applications software items; among these are:
\begin{description}
\begin{description}
\item [SSN/39]: GKS
\item [SSN/40]: FIGARO
\item [SSN/44]: ADAM
\end{description}
\end{description}


The following sections deal with further items whose installation needs comment.

\begin{description}
\begin{description}
\item [ASPIC] --- Collection of application programs.

If you move the {\tt [ASPIC]} directory, change the definitions of logical
names {\tt EDRS} and {\tt PER} in {\tt ASPDIR:ASPPAK.COM}.

\item [CHART] --- Finding chart and stellar data system.

CHART uses star catalogues in a directory with logical name {\tt CATALOGDIR}
and a plotting directory with logical name {\tt PLOTDIR}. These are defined in
{\tt LSSC:STARTUP.COM} which you must modify if you install the catalogues.
These are normally not distributed with the Collection because of their large
size (65203 blocks) and must be requested separately from the Librarian.

\end{description}
\end{description}

\section {Quotas, Privileges and Relationships}
\label{se:quotas}

\subsection {Quotas}

Starlink software is concerned with manipulation large quantities of data in
the form of images or data sets. In order to use the Starlink Software, a user
must have process quotas adequate for the amount of data, and the machine must
also have adequate quotas. 

A typical user at RAL has process quotas set thus:
\begin{quote}\small
\begin{verbatim}
 CPU limit:                      Infinite  Direct I/O limit:       100
 Buffered I/O byte count quota:     65535  Buffered I/O limit:     100
 Timer queue entry quota:              40  Open file quota:        100
 Paging file quota:                 60000  Subprocess quota:        10
 Default page fault cluster:           16  AST quota:               98
 Enqueue quota:                       199  Shared file limit:        0
 Max detached processes:                0  Max active jobs:          0
\end{verbatim}
\end{quote}

We recommend that in addition, the users {\tt JTquota} be at least 3072.

The system parameters that have been found to cause problems are:

\begin{itemize}
\item {\tt VIRTUALPAGECNT} --- use 100000 or more
\item {\tt GBLPAGFIL} --- use 1400 more that AUTOGEN thinks necessary
\item {\tt PROCSETCNT} --- use 100 or more
\end{itemize}

\subsection {Privileges}

\begin{description} 
\begin{description} 

\item [Logical Names] --- Starlink recommends that the Starlink logical names
are defined at {\tt /SYSTEM} level as in {\tt SSC:STARTUP.COM}.

However, excepting the GWM logical names (see below), the Starlink software
will run with the logical names set at {\tt /JOB} level. Some software will run
with Starlink logical names at {\tt /PROCESS} level, but ADAM applications
will not work when run from ICL. 

For the GKS system to be able to use X-windows without the displayed image
disappearing each time an executable image (application) runs down, a Graphics
Window Manager (GWM) has been developed. The logical names associated with GWM
({\tt GWM\_DIR} and the image names) {\bf must} be defined at {\tt /GROUP} or
{\tt /SYSTEM} level, otherwise the detached window manager process will not
be able to see the software and will fail to generate a window. 

\item [Installed Images] --- The standard AZUSS image for the ADAM system needs
to be INSTALLed as a known image in order that ICL works. A dummy version of
the AZUSS image is also available that need not be installed. Swapping between
the two is archived by pointing the {\tt AZUSS} logical name appropriately.
AZUSS is part of the real-time data acquisition capabilities of ADAM and is not
needed for data-reduction. See SSN/44 for more details. 

\end{description}
\end{description}

\subsection {Dependencies}

In general, the relationships and dependencies between Starlink software items
have not been documented, although some installation notes mention them.
Any item which features graphical output will need {\tt GKS}.

The dependencies of applications packages on other applications packages where
known are given in {\tt ADMINDIR\-:SSI.LIS}.

\section {References}
\label{se:references}

\begin{description}
\begin{description}
\item[\bf SGP/20]: Starlink Software Management
\item[\bf SGP/21]: Starlink Software Distribution Policy
\item[\bf SGP/25]: Starlink Site Manager's Guide -- major nodes
\item[\bf SSN/33]: Configuring Queues for Starlink Graphics
\item[\bf SSN/44]: ADAM -- Installation Guide
\item[\bf SUG]: Starlink User Guide
\item[\bf SUN/1]: Starlink Software -- An Introduction
\item[\bf SUN/83]: GKS -- Version 7.2
\end{description}
\end{description}

\newpage

\appendix
\section{How the tapes were written}
\label{se:tapes}

Because making tapes is tedious, the tapes are generally written using site
specific command procedures, submitted as batch jobs on the RAL Starlink
VAXcluster. The log of the batch job will be included with the tape(s) sent to
sites. A typical batch job log is shown below:

\begin{quote}\small
\begin{verbatim}
$ verify = f$verify(0)
$!+
$! BAMBERG_6250.COM
$!
$! Software for BAMBERG
$!
$! M J Bly 24-OCT-92
$!-
$       set verify
$       on error then goto exit
$       old_name = f$process()
$       set proc/name="Bamberg"
$       allocate muc0
%DCL-I-ALLOC, _RLSTAR$MUC0: allocated
$! core
$       backup/noassist/rewind/ignore=label/label="STAR" - 
        /density=6250/block_size=64000 -         
        disk$star:[starlink...]*.*;* muc0:star.bck/save
%MOUNT-I-OPRQST, Please mount volume STAR   in device _RLSTAR$MUC0:
BACKUP requests: Saveset STAR.BCK, Volume number 01, write ENABLED
%MOUNT-I-MOUNTED, STAR mounted on _RLSTAR$MUC0:
%MOUNT-I-RQSTDON, operator request canceled - mount completed successfully
$!
$! adam
$       backup/noassist/ignore=label/density=6250/block_size=64000 -         
        disk$star:[adam.release...]*.*;* muc0:adam.bck/save
$!
$! adamapp
$       backup/noassist/ignore=label/density=6250/block_size=64000 -         
        disk$star:[adamapp...]*.*;* muc0:adamapp.bck/save
$!
$! kappa
$       backup/noassist/ignore=label/density=6250/block_size=64000 -         
        disk$star:[kappa...]*.*;* muc0:kappa.bck/save
$!
$! figaro
$       backup/noassist/ignore=label/density=6250/block_size=64000 -         
        disk$star:[figaro...]*.*;* muc0:figaro.bck/save
$!
$! figpack
$       backup/noassist/ignore=label/density=6250/block_size=64000 -         
        disk$star:[figpack...]*.*;* muc0:figpack.bck/save
$!
$! dipso
$       backup/noassist/ignore=label/density=6250/block_size=64000 -         
        disk$star:[dipso...]*.*;* muc0:dipso.bck/save
$!
$! iuedr
$       backup/noassist/ignore=label/density=6250/block_size=64000 -         
        disk$star:[iuedr...]*.*;* muc0:iuedr.bck/save
$!
$ exit:
$       dism muc0
$       deal muc0
$       set proc/name="BATCH_714"
$       exit
  STAR         job terminated at 24-OCT-1992 15:11:04.54
 
   Accounting information:
  Buffered I/O count:           15162         Peak working set size:    1989
  Direct I/O count:             19467         Peak page file size:      4259
  Page faults:                   9384         Mounted volumes:             1
  Charged CPU time:           0 00:07:28.02   Elapsed time:     0 00:35:30.06
\end{verbatim}
\end{quote}

In this case, the tape was an industry standard 2400' half inch tape written at
a density of 6250bpi. The software provided was:
\begin{enumerate}
\item The Starlink Core
\item The mini release of ADAM
\item ADAMAPP suite
\item KAPPA 
\item FIGARO
\item FIGPACK
\item DIPSO --- optional item
\item IUEDR --- optional item
\end{enumerate}

\newpage

\section{Action list}
\label{se:actions}

This is a summary of actions required to install the Collection; if in doubt,
consult the appropriate section of this paper. The numbers in parentheses -
{\em e.g.}\ (5.2) - refer to sections of this paper. Actions which may not be
relevant are indicated as follows - (?).

\begin{enumerate}

\item Store the Starlink directories on disk (\ref{ss:store}).

\begin{verbatim}
    $ ALLOC Mxxx TAPE
    $ BACKUP/REWIND/NOASSIST TAPE:STAR.BCK - 
            <device>:[*...]/OWNER_UIC=ORIGINAL
    $ BACKUP/NOASSIST TAPE:STARLOCAL.BCK - 
            <device>:[*...]/OWNER_UIC=ORIGINAL
\end{verbatim}

(?) Directories containing Optional Software and Data may also need to be
stored.

\item Edit file {\tt SYS\$STARTUP:SYSTARTUP\_V5.COM} (\ref{ss:fundamental})\\
Add these statements and insert variables ({\em e.g.}\ \verb+<device>+):

\begin{quote}\small
\begin{verbatim}
$!******************************************************
$!
$! EXTRA COMMANDS ADDED FOR STARLINK SOFTWARE COLLECTION
$!
$! Define initial Starlink logical names
$!
$	ASSIGN/SYSTEM <device> STARDISK
$	ASSIGN/SYSTEM <device> LSTARDISK
$!
$! Call the Starlink startup files
$!
$	@STARDISK:[STARLINK]STARTUP
$	@LSTARDISK:[STARLOCAL]STARTUP
$!
$!******************************************************
\end{verbatim}
\end{quote}

\item Edit files {\tt [STARLINK]:STARTUP.COM} and {\tt [STARLOCAL]:STARTUP.COM}
(\ref{ss:fundamental}).

Check that the installed images are acceptable. Check the relevance of the
definitions of the logical names.

\item Activate the Collection (\ref{ss:activate}).

Reboot or explicitly enter commands added to SYS\$STARTUP:SYSTARTUP\_V5.COM.

\item Customise the Graphics system (\ref{ss:fundamental}).

\item (?) Install PSSMB (\ref{ss:systems}). 

\item Files {\tt SSC:LOGIN.COM} and {\tt LSSC:LOGIN.COM}.

Execute procedure
{\tt SSC:LOGIN.COM} and add the command `{\tt \$ @SSC:LOGIN.COM}' to your login
command file.

\end{enumerate}

\end{document}
