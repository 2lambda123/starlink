
\documentstyle{article}
\pagestyle{myheadings}
\markright{SSN/65.1}
\setlength{\textwidth}{160mm}
\setlength{\textheight}{240mm}
\setlength{\topmargin}{-5mm}
\setlength{\oddsidemargin}{0mm}
\setlength{\evensidemargin}{0mm}
\setlength{\parindent}{0mm}
\setlength{\parskip}{\medskipamount}
\setlength{\unitlength}{1mm}

\begin{document}
\thispagestyle{empty}
SCIENCE \& ENGINEERING RESEARCH COUNCIL \hfill SSN/65.1\\
RUTHERFORD APPLETON LABORATORY\\
{\large\bf Starlink Project\\}
{\large\bf Starlink System Note 65.1}
\begin{flushright}
H.E. Huckle\\
C.A. Clayton\\
25 October 1989
\end{flushright}
\vspace{-4mm}
\rule{\textwidth}{0.5mm}
\vspace{5mm}
\begin{center}
{\Large\bf VSHC -- VAXstation VWS hardcopy}
\end{center}
\vspace{5mm}

\section{How it works}

A detached process is run at boot time which runs a .EXE file that creates a
permanent mailbox and redefines UIS\$PRINT\_DESTINATION to that mailbox. The
program then goes into an infinite loop which includes a read to that mailbox.
When a hardcopy is initiated, sixel graphics commands are sent to
UIS\$PRINT\_DESTINATION and thus go to the mailbox. The program then reads those
graphics commands from the mailbox and interprets them into equivalent Canon
commands, using a `State Machine' technique to determine how far it's got, i.e.
is it a start of a plot, end of plot, middle of plot, next plot etc. It spools
the file of Canon graphics commands thus created (in VSHC\_SCRATCH:), to a queue
pointed at by the logical name VSHC\_QUEUE. UIS\$PRINT\_DESTINATION can be
mysteriously reset to its default value of CSA0: and so every few minutes an
AST timeout occurs to reset UIS\$PRINT\_DESTINATION.

VSHC does not support the `eight colours' option in the `colour conversion
method' menu (a sub-menu of the printer set-up menu).

\section{Usage}

\begin{enumerate}

\item Use the mouse to select ``Print (portion of) screen'' from the
the Workstation Options menu. The pointer changes shape to resemble an
arrow and it points to the upper left-hand corner of the screen.

\item Move the pointer to a corner of the rectangular area you want to
print.

\item Click and {\it hold down} the SELECT button. The arrow now points to the
lower right-hand corner of the screen.

\item Move the pointer to create a box around the area you want to print.

\item When the box surrounds the area you want to print, release the  SELECT
button and the pointer again changes shape, so that it resembles an hourglass
shape. When all data has been sent to the mailbox, the pointer returns to its
arrow shape. The workstation display is frozen while the hourglass is visible.

\end{enumerate}

To cancel the operation before you begin printing, position the pointer at
the starting point and release the SELECT button.

The plot will automatically be sent to the Canon laser queue pointed at by
the logical name VSHC\_QUEUE.

The VSHC code is supported by H.E. Huckle on a `best efforts' basis. The
interface to Starlink is supported by C.A. Clayton.

\end{document}
