\documentstyle[11pt]{article}
\pagestyle{myheadings}

%------------------------------------------------------------------------------
\newcommand{\stardoccategory}  {Starlink System Note}
\newcommand{\stardocinitials}  {SSN}
\newcommand{\stardocnumber}    {10.1}
\newcommand{\stardocauthors}   {Martin Bly}
\newcommand{\stardocdate}      {14 April 1992}
\newcommand{\stardoctitle}     {GENSTAT~5 --- Installation Instructions}
%------------------------------------------------------------------------------

\newcommand{\stardocname}{\stardocinitials /\stardocnumber}
\renewcommand{\_}{{\tt\char'137}}     % re-centres the underscore
\markright{\stardocname}
\setlength{\textwidth}{160mm}
\setlength{\textheight}{230mm}
\setlength{\topmargin}{-2mm}
\setlength{\oddsidemargin}{0mm}
\setlength{\evensidemargin}{0mm}
\setlength{\parindent}{0mm}
\setlength{\parskip}{\medskipamount}
\setlength{\unitlength}{1mm}

%------------------------------------------------------------------------------
% Add any \newcommand or \newenvironment commands here
%------------------------------------------------------------------------------

\begin{document}
\thispagestyle{empty}
SCIENCE \& ENGINEERING RESEARCH COUNCIL \hfill \stardocname\\
RUTHERFORD APPLETON LABORATORY\\
{\large\bf Starlink Project\\}
{\large\bf \stardoccategory\ \stardocnumber}
\begin{flushright}
\stardocauthors\\
\stardocdate
\end{flushright}
\vspace{-4mm}
\rule{\textwidth}{0.5mm}
\vspace{5mm}
\begin{center}
{\LARGE \bf \stardoctitle}
\end{center}

%------------------------------------------------------------------------------
%  Add this part if you want a table of contents
%  \setlength{\parskip}{0mm}
%  \tableofcontents
%  \setlength{\parskip}{\medskipamount}
%  \markright{\stardocname}
%------------------------------------------------------------------------------

\section{Introduction}

The GENSTAT~5 package is a general statistical analysis package. It is
commercial software for which Starlink has a licence agreement with Numerical
Algorithms Group (NAG) Ltd. Starlink is currently licenced to install it on any
five Starlink VAX clusters within the Starlink network, in the UK only. The
agreement with NAG allows the installations to be `mobile', that is, for it to
be de-installed at a site that no-longer requires it, and installed at one that
does. Provided the number of cluster installations does not exceed five,
Starlink is within the agreement.

If your site wants to install GENSTAT~5, you should contact the Starlink
Software Librarian (RLVAD::STAR), who will provide a copy of the standard
installation. Your site will be assigned one of the `mobile' licences, and you
may proceed to install it.

For users who require only infrequent access, one of the `mobile' licences is
used for the RLVAD cluster which includes the STADAT machine. Users  may use
their sites' username to access it on STADAT.

This document provides the installation (and de-installation) instructions for
GENSTAT~5 on Starlink VAX machines. It is intended for Starlink Site Mangers
who need to install or de-install it. Various tasks within these instructions
require OPER or SYSPRV privileges.

Starlink does not provide GENSTAT~5 on UNIX machines.

\section{Installation}

This section tells you how to install GENSTAT~5 from the save set which will be
provided by the Starlink Software Librarian.

\subsection{Copying the Software}

You should first copy the save set to your own node, and decompress it if need
be. GENSTAT~5 takes up about 23600 blocks (version 5.2-2), the save set is
about 26650 blocks. You can request that the save set be compressed before you
copy it.

The save set will be protected on RLVAD by ACLs, to prevent unauthorised
copying. Access to the GENSTAT save set will be granted by ACL only to the
username at RLVAD of the site manager authorised to copy the save set.

This means you will need to use COPY or TRANSFER with your RLVAD username
and password to copy the save set. For example:
\begin{verbatim}
      $ COPY RLVAD"rlvad_username rlvad_password"::location:GENSTAT.BCK *.*
\end{verbatim}
where {\tt location} is the location of the save set at RLVAD.

\subsection{Installing the software}

GENSTAT~5 needs just over 23600 blocks of diskspace. Select a disk with
sufficient free space to hold it, and create an empty top-level
directory called {\tt [GENSTAT]} to hold it:
\begin{verbatim}
      $ SET DEFAULT <disk$name>:[000000]
      $ CREATE/DIRECTORY/OWN=STAR/PROTECTION=W:RE [GENSTAT]
\end{verbatim}
Ownership should be assigned to the Starlink Software manager's username at
your node (usually STAR), and the protection set so that all users have READ
and EXECUTE access to the {\tt [GENSTAT]} directory.

Next, you need to unpack the software from the save set:
\begin{verbatim}
      $	SET DEFAULT <disk$name>:[GENSTAT]
      $ BACKUP <where-ever>:GENSTAT.BCK/SAVE/SEL=[GENSTAT...]*.* [...]*.*
\end{verbatim}

You should now have a directory tree containing the GENSTAT~5 software.

\subsection{Local STARTUP and LOGIN}

You now need to modify your {\tt LSSC:STARTUP.COM} file to contain an entry to
set up  the necessary logical names for GENSTAT~5, and to modify your {\tt
LSSC:LOGIN.COM} file to contain the symbol definitions.

\subsubsection{Logical names}

The {\tt [GENSTAT]} directory contains several files for setting up GENSTAT~5.
The first of these is {\tt GENSTAT\_LSSC\_STARTUP.COM}.
\begin{enumerate}
\item Edit {\tt GENSTAT\_LSSC\_STARTUP.COM} to alter \verb+<disk>+ to
be the name of the disk on which you installed GENSTAT~5.
\item Run {\tt GENSTAT\_LSSC\_STARTUP.COM} on all nodes in your cluster.
\item Add the section shown below to your {\tt LSSC:STARTUP.COM} file, ensuring
you have the correct disk specified.
\begin{small}
\begin{verbatim}
      $! GENSTAT
      $       DEFINE/SYSTEM   GENSTAT_DISK    <disk>
      $       DEFINE/SYSTEM   GENSTAT_DIR     GENSTAT_DISK:[GENSTAT]
      $       @GENSTAT_DIR:GENSTAT_STARTUP
\end{verbatim}
\end{small}
The command procedure {\tt GENSTAT\_STARTUP.COM} will define the many system
logical names necessary for GENSTAT~5 when your node is booted.
\end{enumerate}

\subsubsection{Symbols}

Another file in {\tt [GENSTAT]} is {\tt GENSTAT\_LSSC\_LOGIN.COM}. This
contains an insert for your {\tt LSSC:LOGIN.COM} file. The two symbols defined
override the standard Starlink definition which causes the ``Package Not
Installed'' message to be displayed if GENSTAT~5 is not installed.  The section
shown below should be added to your {\tt LSSC:LOGIN.COM}.
\begin{small}
\begin{verbatim}
      $! GENSTAT
      $       GENSTAT     == "$G22SUP:GENSTPHR"
      $       GENSTATHELP == "HELP/LIBRARY=GENSTAT_DIR:GENSTAT/NOINST/NOLIBL GENSTAT"
\end{verbatim}
\end{small}

\subsection{Ready to go}

If you have got this far, you are now ready to run GENSTAT~5. You need to logon
to your node, and execute the {\tt @SSC:LOGIN} procedure to make GENSTAT~5
available.

To run GENSTAT~5, merely type:
\begin{verbatim}
      $ GENSTAT
\end{verbatim}
There is a comprehensive help facility available within GENSTAT~5, and also a
help facility available from DCL, without needing to invoke GENSTAT first:
\begin{verbatim}
      $ GENSTATHELP
\end{verbatim}

\subsection{Documentation}

There are various guides available for GENSTAT~5, which should be obtainable
from Site Managers. If not, contact the Starlink Documentation Librarian
(RLVAD::MDL).
\begin{itemize}
\item SUN/54 --- {\em GENSTAT --- Statistical Data Analysis}.
\item {\em GENSTAT~5 --- An Introduction} (book).
\item {\em GENSTAT~5 --- A Second Course}.
\item {\em GENSTAT~5 --- Reference Manual} (book).
\item {\em GENSTAT~5 --- Release 2 reference Manual Supplement}
\item {\em GENSTAT~5 --- Procedure Library Manual, Release 2.2}.
\end{itemize}

\section{De-installation}

The de-installation of GENSTAT~5 is rather easier. The process involves removing
all the GENSTAT logical names from all nodes in your cluster, deleting the
software and the {\tt [GENSTAT]} directory tree, and amending the
Local Startup and Login files to remove the GENSTAT~5 entries.

\begin{enumerate}
\item Remove most of the logical name definitions. A command procedure is
provided to do this, and should be run on all nodes in the cluster:
\begin{verbatim}
      $ MCR SYSMAN
      SYSMAN> SET ENV/CLUSTER
      SYSMAN> DO @GENSTAT_DIR:GENSTAT_REMOVE
      SYSMAN> EXIT
      $
\end{verbatim}
\item Delete the software and directory tree:
\begin{verbatim}
      $ SET DEFAULT GENSTAT_DIR
      $ DELETE/EXCLUDE=*.DIR;* [...]*.*;*
      $ SET PROTECTION=(O:RWED) *.DIR
      $ DELETE *.DIR;*
      $ SET DEFAULT [-]
      $ SET PROTECTION=(O:RWED) GENSTAT.DIR
      $ DELETE GENSTAT.DIR;*
\end{verbatim}
You can leave the top level {\tt [GENSTAT]} directory itself, if you wish, for
future re-installation of GENSTAT.

\item Remove the remaining GENSTAT~5 logical names on all nodes in your
cluster:
\begin{verbatim}
      $ MCR SYSMAN
      SYSMAN> SET ENV/CLUSTER
      SYSMAN> DO DEASSIGN/SYS GENSTAT_DIR
      SYSMAN> DO DEASSIGN/SYS GENSTAT_DISK
      SYSMAN> EXIT
      $
\end{verbatim}
\item Remove the entries for GENSTAT~5 in your {\tt LSSC:STARTUP.COM} and
{\tt LSSC:LOGIN.COM} files.
\end{enumerate}

You should now have no trace left of GENSTAT~5. Inform the Starlink Software
Librarian (RLVAD::STAR) that you no-longer have GENSTAT~5 installed. Your
`mobile' licence will then be free to assign to another node.
\end{document}
