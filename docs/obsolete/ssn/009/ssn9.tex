\documentstyle[11pt]{article}
\pagestyle{myheadings}

% -----------------------------------------------------------------------------
% ? Document identification
\newcommand{\stardoccategory}  {Starlink System Note}
\newcommand{\stardocinitials}  {SSN}
\newcommand{\stardocsource}    {ssn9.5}
\newcommand{\stardocnumber}    {9.5}
\newcommand{\stardocauthors}   {M.\, J.\, Bly}
\newcommand{\stardocdate}      {15 November 1995}
\newcommand{\stardoctitle}     {Installing the Unix Starlink Software}
% ? End of document identification
% -----------------------------------------------------------------------------

\newcommand{\stardocname}{\stardocinitials /\stardocnumber}
\markright{\stardocname}
\setlength{\textwidth}{160mm}
\setlength{\textheight}{230mm}
\setlength{\topmargin}{-2mm}
\setlength{\oddsidemargin}{0mm}
\setlength{\evensidemargin}{0mm}
\setlength{\parindent}{0mm}
\setlength{\parskip}{\medskipamount}
\setlength{\unitlength}{1mm}

% -----------------------------------------------------------------------------
%  Hypertext definitions.
%  ======================
%  These are used by the LaTeX2HTML translator in conjunction with star2html.

%  Comment.sty: version 2.0, 19 June 1992
%  Selectively in/exclude pieces of text.
%
%  Author
%    Victor Eijkhout                                      <eijkhout@cs.utk.edu>
%    Department of Computer Science
%    University Tennessee at Knoxville
%    104 Ayres Hall
%    Knoxville, TN 37996
%    USA

%  Do not remove the %\begin{rawtex} and %\end{rawtex} lines (used by 
%  star2html to signify raw TeX that latex2html cannot process).
%\begin{rawtex}
\makeatletter
\def\makeinnocent#1{\catcode`#1=12 }
\def\csarg#1#2{\expandafter#1\csname#2\endcsname}

\def\ThrowAwayComment#1{\begingroup
    \def\CurrentComment{#1}%
    \let\do\makeinnocent \dospecials
    \makeinnocent\^^L% and whatever other special cases
    \endlinechar`\^^M \catcode`\^^M=12 \xComment}
{\catcode`\^^M=12 \endlinechar=-1 %
 \gdef\xComment#1^^M{\def\test{#1}
      \csarg\ifx{PlainEnd\CurrentComment Test}\test
          \let\html@next\endgroup
      \else \csarg\ifx{LaLaEnd\CurrentComment Test}\test
            \edef\html@next{\endgroup\noexpand\end{\CurrentComment}}
      \else \let\html@next\xComment
      \fi \fi \html@next}
}
\makeatother

\def\includecomment
 #1{\expandafter\def\csname#1\endcsname{}%
    \expandafter\def\csname end#1\endcsname{}}
\def\excludecomment
 #1{\expandafter\def\csname#1\endcsname{\ThrowAwayComment{#1}}%
    {\escapechar=-1\relax
     \csarg\xdef{PlainEnd#1Test}{\string\\end#1}%
     \csarg\xdef{LaLaEnd#1Test}{\string\\end\string\{#1\string\}}%
    }}

%  Define environments that ignore their contents.
\excludecomment{comment}
\excludecomment{rawhtml}
\excludecomment{htmlonly}
%\end{rawtex}

%  Hypertext commands etc. This is a condensed version of the html.sty
%  file supplied with LaTeX2HTML by: Nikos Drakos <nikos@cbl.leeds.ac.uk> &
%  Jelle van Zeijl <jvzeijl@isou17.estec.esa.nl>. The LaTeX2HTML documentation
%  should be consulted about all commands (and the environments defined above)
%  except \xref and \xlabel which are Starlink specific.

\newcommand{\htmladdnormallinkfoot}[2]{#1\footnote{#2}}
\newcommand{\htmladdnormallink}[2]{#1}
\newcommand{\htmladdimg}[1]{}
\newenvironment{latexonly}{}{}
\newcommand{\hyperref}[4]{#2\ref{#4}#3}
\newcommand{\htmlref}[2]{#1}
\newcommand{\htmlimage}[1]{}
\newcommand{\htmladdtonavigation}[1]{}

%  Starlink cross-references and labels.
\newcommand{\xref}[3]{#1}
\newcommand{\xlabel}[1]{}

%  LaTeX2HTML symbol.
\newcommand{\latextohtml}{{\bf LaTeX}{2}{\tt{HTML}}}

%  Define command to re-centre underscore for Latex and leave as normal
%  for HTML (severe problems with \_ in tabbing environments and \_\_
%  generally otherwise).
\newcommand{\latex}[1]{#1}
\newcommand{\setunderscore}{\renewcommand{\_}{{\tt\symbol{95}}}}
\latex{\setunderscore}

%  Redefine the \tableofcontents command. This procrastination is necessary 
%  to stop the automatic creation of a second table of contents page
%  by latex2html.
\newcommand{\latexonlytoc}[0]{\tableofcontents}

% -----------------------------------------------------------------------------
%  Debugging.
%  =========
%  Remove % on  the following to debug links in the HTML version using Latex.

% \newcommand{\hotlink}[2]{\fbox{\begin{tabular}[t]{@{}c@{}}#1\\\hline{\footnotesize #2}\end{tabular}}}
% \renewcommand{\htmladdnormallinkfoot}[2]{\hotlink{#1}{#2}}
% \renewcommand{\htmladdnormallink}[2]{\hotlink{#1}{#2}}
% \renewcommand{\hyperref}[4]{\hotlink{#1}{\S\ref{#4}}}
% \renewcommand{\htmlref}[2]{\hotlink{#1}{\S\ref{#2}}}
% \renewcommand{\xref}[3]{\hotlink{#1}{#2 -- #3}}
% -----------------------------------------------------------------------------
% ? Document specific \newcommand or \newenvironment commands.
% ? End of document specific commands
% -----------------------------------------------------------------------------
%  Title Page.
%  ===========
\renewcommand{\thepage}{\roman{page}}
\begin{document}
\thispagestyle{empty}

%  Latex document header.
%  ======================
\begin{latexonly}
   CCLRC / {\sc Rutherford Appleton Laboratory} \hfill {\bf \stardocname}\\
   {\large Particle Physics \& Astronomy Research Council}\\
   {\large Starlink Project\\}
   {\large \stardoccategory\ \stardocnumber}
   \begin{flushright}
   \stardocauthors\\
   \stardocdate
   \end{flushright}
   \vspace{-4mm}
   \rule{\textwidth}{0.5mm}
   \vspace{5mm}
   \begin{center}
   {\Large\bf \stardoctitle}
   \end{center}
   \vspace{5mm}

% ? Heading for abstract if used.
  \vspace{10mm}
  \begin{center}
     {\Large\bf Abstract}
  \end{center}
% ? End of heading for abstract.
\end{latexonly}

%  HTML documentation header.
%  ==========================
\begin{htmlonly}
   \xlabel{}
   \begin{rawhtml} <H1> \end{rawhtml}
      \stardoctitle
   \begin{rawhtml} </H1> \end{rawhtml}

% ? Add picture here if required.
% ? End of picture

   \begin{rawhtml} <P> <I> \end{rawhtml}
   \stardoccategory \stardocnumber \\
   \stardocauthors \\
   \stardocdate
   \begin{rawhtml} </I> </P> <H3> \end{rawhtml}
      \htmladdnormallink{CCLRC}{http://www.cclrc.ac.uk} /
      \htmladdnormallink{Rutherford Appleton Laboratory}
                        {http://www.cclrc.ac.uk/ral} \\
      \htmladdnormallink{Particle Physics \& Astronomy Research Council}
                        {http://www.pparc.ac.uk} \\
   \begin{rawhtml} </H3> <H2> \end{rawhtml}
      \htmladdnormallink{Starlink Project}{http://star-www.rl.ac.uk/}
   \begin{rawhtml} </H2> \end{rawhtml}
   \htmladdnormallink{\htmladdimg{source.gif} Retrieve hardcopy}
      {http://star-www.rl.ac.uk/cgi-bin/hcserver?\stardocsource}\\

%  HTML document table of contents. 
%  ================================
%  Add table of contents header and a navigation button to return to this 
%  point in the document (this should always go before the abstract \section). 
  \label{stardoccontents}
  \begin{rawhtml} 
    <HR>
    <H2>Contents</H2>
  \end{rawhtml}
  \renewcommand{\latexonlytoc}[0]{}
  \htmladdtonavigation{\htmlref{\htmladdimg{contents_motif.gif}}
        {stardoccontents}}

% ? New section for abstract if used.
  \section{\xlabel{abstract}Abstract}
% ? End of new section for abstract

\end{htmlonly}

% -----------------------------------------------------------------------------
% ? Document Abstract. (if used)
%  ==================

This note is the release note and installation instructions for the
DEC Alpha AXP / Digital UNIX, Sun Sparc / Solaris v2.x, and Sun Sparc / 
SunOS 4.1.x versions of the Starlink Software Collection (USSC).

You will be supplied with pre-built (and installed) versions on tape
and will just need to copy the tape to disk to have a working version.

The tapes (where appropriate) will contain in addition, copies of the
NAG and MEMSYS libraries, and Tcl, Tk, Expect, Mosaic, TeX, Pine, Perl,
Jed, Ispell, Ghostscript, LaXeX2html and Ftnchek for the relevant system.

The Sun Sparc SunOS 4.1.x version of the USSC was frozen at USSC111 and
no further updates are available.  The instructions for installing the
main section of the USSC may continue to be used for installing Sun
Sparc SunOS 4.1.x version.

 
% ? End of document abstract
% -----------------------------------------------------------------------------
% ? Latex document Table of Contents (if used).
%  ===========================================
 \newpage
 \begin{latexonly}
   \setlength{\parskip}{0mm}
   \latexonlytoc
   \setlength{\parskip}{\medskipamount}
   \markright{\stardocname}
 \end{latexonly}
% ? End of Latex document table of contents
% -----------------------------------------------------------------------------
\newpage
\renewcommand{\thepage}{\arabic{page}}
 \setcounter{page}{1}

\section{General Introduction} 
\label{s:intro}

\subsection{Organisation}
\label{s:intro:organ}

The Starlink Software is distributed in several parts to enable easy 
installation.  The main part is the entire USSC proper, source included. 

There are also two items of commercial software (NAG and MEMSYS), which 
will not be sent to sites not covered by the Starlink Licences for those
products.  

In addition, there is a Base Set of mostly Public Domain utilities
which Starlink recommends that its sites install.  Some of these are
required by some parts of the Starlink Software, others are provided as
base level utilities for Starlink users to give a consistent `feel' to
the UK Starlink sites.  The installation instructions indicate which of
these are necessary for the USSC.

\subsection{Installing the parts}
\label{s:intro:insta}

Each part of the Software Collection is installed according to the details
given in the sections that follow.  In practice, you need only install
those parts you wish.  Table \ref{t:compo} shows the components.

\begin{table}[ht]
\begin{center}
\begin{tabular}{l|l}
{\em Component} & {\em Reference} \\ \hline \hline
Main USSC   & Section \ref{s:imain} \\
NAG         & Section \ref{s:inag} --- Starlink Sites only \\
MEMSYS      & Section \ref{s:imemsys} --- Starlink Sites only \\
Local       & Section \ref{s:ilocal} --- local software \\
Base Set    & Section \ref{s:ibase} --- optional, some parts required by USSC \\
Setting up  & Section \ref{s:setup} \\
Information & Section \ref{s:info} \\
\end{tabular}
\caption{Components of USSC}
\label{t:compo}
\end{center}
\end{table}

\subsection{Software Sizes}
\label{s:intro:sizes}

Table \ref{t:sizes} shows the Software sizes at USSC135.

\begin{table}[ht]
\begin{center}
\begin{tabular}{|l|l|l|l|l|} \hline
                       & {\tt /star} & NAG   & Memsys & \TeX\ \\ \hline \hline
Alpha AXP Digital Unix & 669 Mb      & 25 Mb & 1.8 Mb & 37 Mb \\
Sun Sparc Solaris 2    & 648 Mb      & 22 Mb & 1.9 Mb & 32 Mb \\
Sun Sparc SunOS 4      & 554 Mb      & 36 Mb & 1.8 Mb & 22 Mb (USSC111)\\
\hline
\end{tabular}
\caption{Software Sizes}
\label{t:sizes}
\end{center}
\end{table}

\subsection{Technical Issues}
\label{s:intro:techi}

User {\tt {\tt `star'}} is assumed throughout. {\tt `star'} may be an
actual userid, but could be any convenient owner account.  You must be
logged in as {\tt `star'} for the entirety of this installation, though
{\tt `root'} access is required at times.

This note assumes that you are familiar with some system management concepts,
or that you have access to someone who does, and who can do the sections 
requiring {\tt `root'} access.

On some systems, you may need to skip over the EOF marks used to separate
tar files on tapes.  If you issue the tar command to read the tape and
you receive an error message {\tt `tar:blocksize=0'}, you should issue
the tar command again.

\subsection{System requirements}
\label{s:intro:reqs}

The USSC is supported on the system configurations shown in Table
\ref{t:reqs}. The compilers marked {\tt (*)} are required at run
time\footnote{N.B. The USSC will not work with Sun Fortran
v2.0.1 or v3.0.1 under SunOS v4.1.x.}.

\begin{table}[h]
\begin{small}
\begin{center}
\begin{tabular}{|l||l||l||l|} \hline
{\em Hardware} & DEC Alpha         & Sun Sparc & Sun Sparc \\ \hline
{\em System}   & Digital Unix v3.2 & Solaris 2.4 & SunOS 4.1.2 \\ \hline
{\em Fortran}  & DEC Fortran v3.7  & Sun Fortran v3.0.1{\tt (*)} &
                 Sun Fortran v1.4.1{\tt (*)} \\ \hline
{\em C}        & DEC C             & Sun C v3.0.1 & GNU C v2.2.2 \\ \hline
{\em Other}    & X11 & Openwindows X11 & Openwindows or X11 \\ \hline
\end{tabular}
\caption{System configurations and requirements for the USSC}
\label{t:reqs}
\end{center}
\end{small}
\end{table}

\section{Installing the USSC main part}
\label{s:imain}

This section covers installing the main  part of the USSC, and once done,
you will have a working system that just need a few initialisation tasks
completed before use (see Section \ref{s:setup}).

\begin{enumerate}

\item Create a home for the USSC.  This must be a directory to which
user {\tt `star'} has write access,  and should be a directory to which a
soft link {\tt /star} points.  You will need {\tt `root'} access to
create the soft link {\tt /star}.

Note: {\tt /star} should NOT be, or point to, the home directory for user
{\tt `star'}, since files created in the {\tt `star'} directory might interfere
with the operation of the Starlink Software.

\begin{verbatim}
      <gain root access>
      % mkdir /wherever/star
      % chown star /wherever/star
      % ln -s /wherever/star /star
      <resume `star' userid>
\end{verbatim}

\item Install the software from the tape:

\begin{verbatim}
      % cd /star
      % tar xvf /tape_device_name
\end{verbatim}

where \verb+/tape_device_name+ is the name of the tape drive in use. It is
VITAL that this should be of the no-rewind variety.

\end{enumerate}

\section{Installing NAG}
\label{s:inag}

This section details the installation of the NAG libraries, and applies to
Starlink sites only.

If you are not a Starlink site, please proceed to the Base Set installation
(see Section \ref{s:ibase}).

\begin{enumerate}

\item Create a home for the NAG software. 

As provided, the USSC contains a top level soft link {\tt /star/nag} to the
home of the NAG software as installed at RAL.  You should first remove
this soft link:

\begin{verbatim}
      % rm /star/nag
\end{verbatim}

There are two possibilities for a home directory for NAG:

You may wish to have NAG installed away from the {\tt /star} tree, in which
case you should create a home directory for NAG and then add a soft
link {\tt /star/nag} to point to this location:

\begin{verbatim}
      % mkdir /wherever/nag
      % chown star /wherever/nag
      % ln -s /wherever/nag /star/nag
\end{verbatim}

If you want to add NAG into the {\tt /star} tree create a {\tt
/star/nag} directory:

\begin{verbatim}
      % mkdir /star/nag
\end{verbatim}

If you put the NAG system in the {\tt /star} tree, you must not make copies
of {\tt /star} that include {\tt /star/nag} for distribution to sites
not having a NAG licences.

\item Install NAG:

\begin{verbatim}
      % cd /star/nag
      % tar xvf /tape_device_name
\end{verbatim}

where \verb+/tape_device_name+ is defined as before.

\end{enumerate}

\section{Installing MEMSYS}
\label{s:imemsys}

This section details the installation of the MEMSYS libraries, and
applies to Starlink sites and UK observatories only.

If you are not one of the above sites, please proceed to the Base Set
installation (Section \ref{s:ibase}).

This section assumes some understanding of Unix system management and
how access rights are defined.  You will need to get your system manager
to set up the required group as detailed below.  The software installation
is simple.

\begin{enumerate}

\item Create a GID for the memsys group and call it memsys. Add user
{\tt `star'} to the memsys group, and any users at your site that have
signed Licences to use the memsys libraries. If your are running
net-wide groups, the groups must be recognised and enforced on all CPUs
to which the memsys software is exported. You will need {\tt `root'}
access to do this part.

The protection mechanism depends on there being no {\tt `other'} access
to the memsys libraries, and the access methods for authorised users
depends upon the GID memsys.  User {\tt `star'} should be the owner of the
memsys files.

\item As provided, the USSC has a top level soft link {\tt /star/memsys} to
the home of the MEMSYS software, as installed at RAL.  This must be
removed:

\begin{verbatim}
      % rm /star/memsys
\end{verbatim}

\item You must have MEMSYS installed away from the {\tt /star} tree.
Create a home directory for MEMSYS and then add a new soft link {\tt
/star/memsys} to point to this location:

\begin{verbatim}
      % mkdir /wherever/memsys
      % chown star /wherever/memsys
      % ln -s /wherever/memsys /star/memsys
\end{verbatim}

Note: The home directory does not need to be protected, just the
contents.  The directory may have {\tt `other'} access.

\item Install MEMSYS:

\begin{verbatim}
      % cd /star/memsys
      % tar xvf /tape_device_name
\end{verbatim}

where \verb+/tape_device_name+ is defined as before.

\item Protect the memsys software:

\begin{verbatim}
      % cd /star/memsys
      % ./protect
\end{verbatim}

This step will fail if you have not got the memsys GID setup and being
recognised by your system.

\end{enumerate}

\section{Local software}
\label{s:ilocal}

The USSC has provision for adding local software to the collection, in a 
{\tt /star/local} directory.  

The USSC provided contains a top level soft link {\tt /star/local} 
to the home of the locxal software as installed at RAL.

The {\tt /star/local} soft link should be removed, and either a new
one substituted to point to a suitable location for your local
software, or, a {\tt /star/local} directory created.

\section{Installing the Base Set}
\label{s:ibase}

The Base Set consists of the following software:

\begin{itemize}
\item \TeX\   --- required for document processing
\item Mosaic  --- WWW browser
\item Perl  --- required by news scripts
\item Jed  --- recommended EDT-like editor
\item Pine --- recommended mail interface
\item Tcl --- tool command language --- required for XADAM GUI
\item Tk --- tool kit --- required for XADAM GUI
\item Expect --- Interactive program control --- required for XADAM GUI
\item Ghostscript --- required by Star2html, LaTeX2html and xdvi
\item LaTeX2html --- required by Star2html
\item Ftnchek --- Fortran source checker
\item Emacs --- GNU editor
\item Ispell --- Spelling checker
\end{itemize}

If you already have your own versions of these software items, you may not
need to use the Starlink supplied versions, though some are required to be
provided at or above the version levels of the Starlink provided versions
(see Tcl, Tk, Expect, Mosaic: Section \ref{s:ibase:tcl}).

\begin{quote}
NOTE for Sun SunOS 4: Emacs is included with the main USSC section and is 
an old version. 
\end{quote}

If you intend to use the XADAM GUI, you must have Tcl, Tk, Expect and Mosaic
installed.  

You will need a version of \TeX/\LaTeX\ to process the Starlink documentation,
but it need not be the version provided by Starlink.

\subsection{Retrieving the base set from tape}
\label{s:ibase:tape}

Create a home for the Base Set software and copy it from the tape:

\begin{verbatim}
      % mkdir /wherever/base
      % chown star /wherever/base
      % cd /wherever/base
      % tar xvf /tape_device_name
\end{verbatim}

where \verb+/tape_device_name+ is defined as before.

You will now have a directory {\tt /wherever/base} containing compressed tar 
files for \TeX , Tcl, Tk, Expect, Mosaic, Pine, Jed, Emacs, Perl, Ispell, 
and Ftnchek.  

\subsection{Installing \TeX}
\label{s:ibase:tex}

Optional, required at UK Sites, and for document processing.

TeX is provided as a compressed tar file for the system you require.
The {\tt tex\_readme} file contains the installation instructions.

\subsection{Installing Tcl, Tk, Expect, Mosaic}
\label{s:ibase:tcl}

These items are required for the XADAM GUI.

If you want to run the XADAM GUI, you must install Tcl, Tk and Expect,
and to use the hyper-text help system, you should also install Mosaic.
Versions suitable for use with the USSC are included in
{\tt /wherever/base}.

You need to have versions of these items at or above the levels of the
Starlink provided versions for the XADAM system to run.  Compressed tar
files and associated {\tt `readme'} files giving installation instructions
are provided in {\tt /wherever/base}.

The USSC now has its own built in versions of Tcl and Tk for use with
the Starlink extnsion set to Tcl and Tk.  These are components of the
main USSC part and will be used where appropriate by Starlink
applications.  XADAM still requires the additional set of Tcl, Tk and
expect mentioned above.

\subsection{Ispell, Pine, Jed, Emacs, Ftnchek, Perl, Ghostscript, LaTeX2html}
\label{s:ibase:rest}

These items are optional, but are required at UK sites.

Install Pine, Emacs, Perl, Jed, Ispell, Ghostscript, LateX2html and
Ftnchek. This bit is optional for non-Starlink sites, though you may
wish to take advantage of the versions provided on the tape.

The directory {\tt /wherever/base} contains the compressed tar files and
associated {\tt `readme'} files necessary for installation.  Of these,
only Perl is required, for the Starlink {\tt `news'} utility.  In
addition, if you wish to use the LaTeX2html and Star2html systems for
converting \LaTeX\ documentation to hypertext, you will need
Ghostscript.

\section{Setting Up}
\label{s:setup}

This section contains the general `setting up' instructions, which you
should complete before the software can be used.

\subsection{Unloading the tape}
\label{s:setup:unload}

To unload the tape once you have finished, type the following:

\begin{verbatim}
      % mt -f /tape_device_name rew
\end{verbatim}

Please remember to return the tape to RAL at the address below.

\subsection{The GKS Licence}
\label{s:setup:gks}

The GKS system is commercial software for which a Licence is required.
GKS is FREE to non-profit making institutions.  Complete and return the
GKS licence to Starlink.  It should be signed on behalf of your
Institution by an Officer of the Institution.

\subsection{Tape devices}
\label{s:setup:tapedev}

If you have tape devices on your machine, and wish to use them for reading
FITS tapes, you must ensure the USSC recognises them.

As provided, the software will recognise the first 3 devices numerically 
from zero, {\em e.g.}, on an Digital Unix machine: {\tt /dev/nrmt0h}, 
{\tt /dev/nrmt1h} and {\tt /dev/nrmt2h}.

\begin{quote}
NOTE that the software does not recognise {\it `rewind'} type device
names as they will not allow the tape manipulation software to function
correctly.
\end{quote}

If you have other devices, they must be added, and the tape device database
rebuilt, as follows:

\begin{enumerate}

\item The file that controls the tape device database is {\tt
create\_tape\_dev} in {\tt /star/bin}.  Either edit this version or
copy it and edit your own version.  You should modify it to contain
references to any tape drives you have which are not already listed.

\item Create a \verb+~/adam+ directory and execute the procedure:

\begin{verbatim}
      % mkdir ~/adam
      % ./create_tape_dev
\end{verbatim}

This will create a file {\tt devdataset.sdf} in the current dircetory.
Copy the {\tt devdataset.sdf} file to {\tt /star/etc}.  The \verb+~/adam+
directory is required by the {\tt tapecreate} program run by {\tt
create\_tape\_dev}.

\end{enumerate}

\subsection{Using the Software}
\label{s:setup:using}

In order to use the software, users need to be running the C-shell or
an equivalent such as {\tt tcsh}.

Each user should have in their {\tt \$\{HOME\}/.login} file, the command:

\begin{verbatim}
      source /star/etc/login
\end{verbatim}

and in their {\tt \$\{HOME\}/.cshrc} file, the command:

\begin{verbatim}
      source /star/etc/cshrc
\end{verbatim}

Between them, these will set up the necessary environment variables,
paths and aliases for the software.

Applications packages can then be accessed by issuing the package
startup command, {\em e.g.}:

For Dipso:

\begin{verbatim}
      % dipsosetup      # sets up Dipso system
      % dipso           # starts dipso
\end{verbatim}

For Kappa:

\begin{verbatim}
      % kappa           # set all Kappa command aliases
\end{verbatim}


\section{Information}
\label{s:info}

\subsection{Documentation}
\label{s:info:docs}

The complete documentation set in included in {\tt /star/docs}.  The
documentation is in \LaTeX\ form, and may be processed by the Starlink
provided version of \TeX/\LaTeX.

In addition, an increasing number of Starlink documents have hypertext 
versions available.

\subsection{Information files}
\label{s:info:files}

All admin and documentation information files are included in {\tt
/star/admin} or {\tt /star/docs} respectively.

\subsection{Updates}
\label{s:info:updates}

Updates to the Starlink Software Collection are frequent, and may be
picked up by anonymous ftp from the Starlink FTP server at
{\tt starlink-ftp.rl.ac.uk}.  If you would like to receive update
notifications by email, please contact Starlink at the address below.

If your local Netnews server carries the {\tt uk.*} heirarchy, you can 
see the update notices posted on the newsgroup {\tt uk.org.starlink.announce}.
This moderated group also carries astronomy related job adverts and
general Starlink announcements and news items.

\subsection{Support}
\label{s:info:help}

If you have any problems with the installation, or with running the
USSC, please contact Starlink at the address given below.

\begin{verbatim}
  ---------------------------------------------------------------------------
        Martin Bly,                     | Internet: ussc@star.rl.ac.uk
        Starlink Software Librarian     | Tel:      +44 (0)1235 445363
        Starlink, RAL, Chilton, DIDCOT  | Fax:      +44 (0)1235 445848
        OXON, OX11 0QX, United Kingdom. | URL:  http://star-www.rl.ac.uk/
  ---------------------------------------------------------------------------
\end{verbatim}

\end{document}
