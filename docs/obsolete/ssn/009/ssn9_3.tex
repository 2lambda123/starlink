\documentstyle[11pt]{article} 
\pagestyle{myheadings}

%------------------------------------------------------------------------------
\newcommand{\stardoccategory} {Starlink System Note}
\newcommand{\stardocinitials} {SSN}
\newcommand{\stardocnumber}   {9.3}
\newcommand{\stardocauthors}  {M J Bly}
\newcommand{\stardocdate}     {26 January 1994}
\newcommand{\stardoctitle}    {Installing the Unix Starlink Software Collection}
%------------------------------------------------------------------------------

\newcommand{\stardocname}{\stardocinitials /\stardocnumber}
\renewcommand{\_}{{\tt\char'137}}     % re-centres the underscore
\markright{\stardocname}
\setlength{\textwidth}{160mm}
\setlength{\textheight}{230mm}
\setlength{\topmargin}{-2mm}
\setlength{\oddsidemargin}{0mm}
\setlength{\evensidemargin}{0mm}
\setlength{\parindent}{0mm}
\setlength{\parskip}{\medskipamount}
\setlength{\unitlength}{1mm}

%------------------------------------------------------------------------------
% Add any \newcommand or \newenvironment commands here
%------------------------------------------------------------------------------

\begin{document}
\thispagestyle{empty}
SCIENCE \& ENGINEERING RESEARCH COUNCIL \hfill \stardocname\\
RUTHERFORD APPLETON LABORATORY\\
{\large\bf Starlink Project\\}
{\large\bf \stardoccategory\ \stardocnumber}
\begin{flushright}
\stardocauthors\\
\stardocdate
\end{flushright}
\vspace{-4mm}
\rule{\textwidth}{0.5mm}
\vspace{5mm}
\begin{center}
{\Large\bf \stardoctitle}
\end{center}
\vspace{5mm}

\section{Introduction}

This document is intended for people who have to install the Unix
Starlink Software Collection (USSC).

\section{Requirements}

The USSC is supported on the system configurations shown in Table
\ref{tab:cons}. The compilers marked {\tt (*)} are required at run
time\footnote{N.B. The USSC will not work with Sun Fortran
v.2.0.1 under SunOS v4.1.x.}.

\begin{table}[h]
\begin{small}
\begin{center}
\begin{tabular}{|l|l|l|l|} \hline
{\em Hardware} & {\em Unix} & {\em Compilers} & {\em Other} \\ \hline \hline
DEC Alpha & OSF/1 v1.3  & DEC Fortran, DEC C  & X11/Motif \\ \hline
Sun Sparc & Solaris 2.2 & Sun Fortran v2.0.1{\tt (*)}, Sun C v2.0.1 & X11/Motif 
or Openwindows \\ \hline
DEC Mips  & Ultrix v4.2 & DEC Fortran, DEC C c89 & X11/Motif \\ \hline
Sun Sparc & SunOS 4.1.2 & Sun Fortran v1.4.1{\tt (*)}, GNU C v2.2.2 & X11/Motif 
or Openwindows \\ \hline \hline
\end{tabular}
\caption{System configurations and requirements for the USSC}
\label{tab:cons}
\end{center}
\end{small}
\end{table}

In addition, the USSC requires that a user be running the C-shell (or
an extended variant such as {\tt tcsh}) as the login shell.

The USSC is only supported on the Hardware/Software configurations listed,
though it may be possible to build it for other combinations.

The USSC is managed by the Starlink Software Librarian. Please report any
problems you may have with installing the USSC, and any bugs you may
find, to the Software Librarian who can be contacted by email at {\tt
ussc@star.rl.ac.uk}.

Updates are made to the USSC regularly, and are available as compressed
tar files via anonymous ftp. Please contact the Software Librarian if 
you wish to receive update notes by email.

\newpage

\section{Installation}

You will receive a tape of the USSC, usually an exabyte, containing 
the entire collection (binaries and source), in a version built for 
the system requested.

\begin{enumerate}

\item Create a home for the software. For this, you need root access.
There are three options:

\begin{enumerate}

\item A dedicated local disk partition, which should be mounted onto the
mount point {\tt /star}, owned by `{\tt star}'.

\item An NFS mounted partition, ``pasted'' into the file system of your
local machine at mount point {\tt /star}, owned by `{\tt star}'.

\item A local but non-dedicated disk area, owned by `{\tt star}', and
soft linked into the file system as {\tt /star}. For example, at RAL the
Starlink software is stored in:

\begin{verbatim}
      /soft1/star-sun4
\end{verbatim}

Hence, we have created a soft link from {\tt /star} to {\tt /soft1/star-sun4}

\begin{verbatim}
      # ln -s /soft1/star-sun4 /star
\end{verbatim}

\end{enumerate}

\item Install the USSC from the tape. Login as the owner of the {\tt /star}
partition or directory. This is normally `{\tt star}'. Then move to the {\tt 
/star} directory and using an appropriate no-rewind tape device, copy the 
USSC to disk. For example:

\begin{verbatim}
      % cd /star
      % tar xf /dev/nrst0
\end{verbatim}

\item Install NAG (Starlink sites only)\footnote{Some of the USSC
applications use NAG Fortran routines. NAG Ltd allow Starlink to 
distribute applications binaries linked with NAG objects to academic
sites. If you want to build the USSC, you will need a NAG licence.}.
Create a home for the NAG libraries. This can be in {\tt /star/nag} 
or a softlink to somewhere else to which user `{\tt star}' has 
write access. 

\begin{verbatim}
      % cd /star/nag
      % tar xf /dev/nrst0
\end{verbatim}

\item Users wanting to use the USSC should add the command:
\begin{verbatim}
      source /star/etc/login
\end{verbatim}
to their \verb+~/.login+ file and:
\begin{verbatim}
      source /star/etc/cshrc
\end{verbatim}
to their \verb+~/.cshrc+ file.

\item Complete and return to the address shown, the RAL-GKS licence form.
RAL-GKS is free to academic/non-profit organisations.

\end{enumerate}

Since the USSC is large, you may save space by removing the source code 
and still have a running system. You can also remove applications you 
do not need. Most of the USSC can be treated this way. For advice on 
what may or may not be removed safely, please contact the Starlink 
Software Librarian.

\end{document}
