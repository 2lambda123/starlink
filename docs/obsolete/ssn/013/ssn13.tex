\documentstyle[11pt]{article}
\pagestyle{myheadings}

%------------------------------------------------------------------------------
\newcommand{\stardoccategory}  {Starlink System Note}
\newcommand{\stardocinitials}  {SSN}
\newcommand{\stardocnumber}    {13.1}
\newcommand{\stardocauthors}   {D J Rawlinson}
\newcommand{\stardocdate}      {11 August 1992}
\newcommand{\stardoctitle}     {COLDMON --- Cold File Analysis Package}
%------------------------------------------------------------------------------

\newcommand{\stardocname}{\stardocinitials /\stardocnumber}
\renewcommand{\_}{{\tt\char'137}}     % re-centres the underscore
\markright{\stardocname}
\setlength{\textwidth}{160mm}
\setlength{\textheight}{230mm}
\setlength{\topmargin}{-2mm}
\setlength{\oddsidemargin}{0mm}
\setlength{\evensidemargin}{0mm}
\setlength{\parindent}{0mm}
\setlength{\parskip}{\medskipamount}
\setlength{\unitlength}{1mm}

%------------------------------------------------------------------------------
% Add any \newcommand or \newenvironment commands here
%------------------------------------------------------------------------------

\begin{document}
\thispagestyle{empty}
SCIENCE \& ENGINEERING RESEARCH COUNCIL \hfill \stardocname\\
RUTHERFORD APPLETON LABORATORY\\
{\large\bf Starlink Project\\}
{\large\bf \stardoccategory\ \stardocnumber}
\begin{flushright}
\stardocauthors\\
\stardocdate
\end{flushright}
\vspace{-4mm}
\rule{\textwidth}{0.5mm}
\vspace{5mm}
\begin{center}
{\Large\bf \stardoctitle}
\end{center}
\vspace{5mm}

%------------------------------------------------------------------------------
%  Add this part if you want a table of contents
%  \setlength{\parskip}{0mm}
%  \tableofcontents
%  \setlength{\parskip}{\medskipamount}
%  \markright{\stardocname}
%------------------------------------------------------------------------------

\section{Introduction}

The COLDMON package has been written to allow system managers to identify those
items of software that are not used (or used infrequently) on their systems. It
consists of a few command procedures and a Fortran program to analyze the
results. It makes use of the AUDIT facility and security ACLs in VMS.

\section{Installation}

The package consists of the following files:

\begin{description}
\item [ANAL\_AUDIT.COM]  -- Command procedure to analyze the audit file
\item [ANAL\_BUILD.COM]  -- Command procedure to build executable
\item [ANAL\_DATES.DAT]  -- File to contain the start/stop dates
\item [ANAL\_FILES.DAT] -- List of files to analyze
\item [ANAL\_RESULTS.DAT]  -- Results file written by ANAL\_AUDIT.COM
\item [ANAL\_SETUP.COM]  -- Setup script to set ACLs and enable Auditing
\item [ANAL\_STATS.DAT]  -- Final statistics file written by $\ldots$
\item [ANAL\_STATS.EXE]  -- Program to search the results file and produce
statistics.
\item [ANAL\_STATS.FOR]  -- Source
\end{description}

Simply install these files in a directory.

\section{Starting the analysis}

A file {\bf ANAL\_FILES.DAT} should be created to contain the full pathname of
the files to be tested, one per line, {\em e.g.}

\begin{verbatim}
      DISK$SOFTDEV:[DJR.COLD.RELEASE]TEST1.COM
      DISK$SOFTDEV:[DJR.COLD.RELEASE]TEST2.COM
\end{verbatim}

The analysis can be carried out using executable files, command procedures or
object libraries.

The next step is to enable ACL security auditing and set ACLs on the files
specified. This requires {\bf OPER} \& {\bf SECURITY} privileges. The command
file {\bf ANAL\_SETUP.COM} will do this all for you. The procedure will first
check whether you currently have ACL auditing switched on and will ask you if
you would like to turn it on. An example of the output is given below:

\begin{verbatim}
          You do not appear to have ACL auditing switched on.
          Do you want to turn it on now (Y/N) : y


          Setting up the ACLs on the files specified in ANAL_FILES.DAT

          Please wait.....

      DISK$SOFTDEV:[DJR.COLD.RELEASE]TEST1.COM........acl done
      DISK$SOFTDEV:[DJR.COLD.RELEASE]TEST2.COM........acl done
      DISK$SOFTDEV:[DJR.COLD.RELEASE]TEST3.COM........acl done
          Enabling ACL Auditing...

          Finished.
\end{verbatim}

The date/time that analysis was started will be written to the file {\bf
ANAL\_DATES.DAT}.

\subsection{IMPORTANT}

From now on, everytime any of the files specified are accessed, an entry will
be placed in the audit journal file.  You should also note that the console
will log each access and so this software should only be used to gather
statistics of files believed to be infrequently accessed otherwise the security
audit file will get very large and you will waste a lot of paper.

Care should also be taken with system backups. If the number of files involved
in the analysis is large, then it may be wise to turn off security auditing
during backups using the command:

\begin{verbatim}
       SET AUDIT/ALARM/DISABLE=ACL
\end{verbatim}
and on again afterwards with:
\begin{verbatim}
       SET AUDIT/ALARM/ENABLE=ACL
\end{verbatim}

\section{Extracting results from the audit file}

After the required period of analysis, the command procedure {\bf
ANAL\_AUDIT.COM} will extract the ACL accesses from the audit file and produce
a raw results file. Please edit this command procedure  and change the name of
the security audit file if it is different from the default:

\begin{verbatim}
      SYS$MANAGER:SECURITY_AUDIT.AUDIT$JOURNAL
\end{verbatim}

The command procedure will ignore accesses by the {\bf BACKUP} and {\bf
DIRECTORY} commands. The date/time that analysis was stopped will be appended
to the file {\bf ANAL\_DATES.DAT}.

This command procedure will also stop ACL security auditing and then remove the
ACLs from the files.

\section{Producing file usage statistics}

The programme {\bf ANAL\_STATS} will read the results file to produce a file
{\bf ANAL\_STATS\-.DAT} containing the start/stop times and a usage statistic
for each file in {\bf ANAL\_FILES\-.DAT}. The programme will ignore accesses by
users {\bf SYSTEM}, {\bf OPER} and {\bf STAR}. You may wish to alter this by
editing the source and rebuilding using {\bf ANAL\_BUILD.COM}.

An example output is:
\begin{verbatim}
      ANAL_STATS.DAT   Cold file access statistics
      --------------------------------------------
      31-MAR-1992 09:53:42.32 -- Date/time analysis started
      31-MAR-1992 09:53:50.95 -- Date/time analysis stopped
      TEST1.COM            Number of accesses=            3
      TEST2.COM            Number of accesses=            2
      TEST3.COM            Number of accesses=            1
\end{verbatim}

\end{document}
