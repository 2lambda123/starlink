\documentclass[11pt]{article}
\pagestyle{myheadings}

% -----------------------------------------------------------------------------
% ? Document identification
\newcommand{\stardoccategory}  {Starlink System Note}
\newcommand{\stardocinitials}  {SSN}
\newcommand{\stardocnumber}    {31.1}
\newcommand{\stardocsource}    {ssn\stardocnumber}
\newcommand{\stardocauthors}   {K.\, P.\, Duffey}
\newcommand{\stardocdate}      {17th November 1995}
\newcommand{\stardoctitle}     {A Starlink Guide to Cartridge Tape Subsystems}
% ? End of document identification
% -----------------------------------------------------------------------------

\newcommand{\stardocname}{\stardocinitials /\stardocnumber}
\markright{\stardocname}
\setlength{\textwidth}{160mm}
\setlength{\textheight}{230mm}
\setlength{\topmargin}{-2mm}
\setlength{\oddsidemargin}{0mm}
\setlength{\evensidemargin}{0mm}
\setlength{\parindent}{0mm}
\setlength{\parskip}{\medskipamount}
\setlength{\unitlength}{1mm}

% -----------------------------------------------------------------------------
%  Hypertext definitions.
%  ======================
%  These are used by the LaTeX2HTML translator in conjunction with star2html.

%  Comment.sty: version 2.0, 19 June 1992
%  Selectively in/exclude pieces of text.
%
%  Author
%    Victor Eijkhout                                      <eijkhout@cs.utk.edu>
%    Department of Computer Science
%    University Tennessee at Knoxville
%    104 Ayres Hall
%    Knoxville, TN 37996
%    USA

%  Do not remove the %begin{latexonly} and %end{latexonly} lines (used by
%  star2html to signify raw TeX that latex2html cannot process).
%begin{latexonly}
\makeatletter
\def\makeinnocent#1{\catcode`#1=12 }
\def\csarg#1#2{\expandafter#1\csname#2\endcsname}

\def\ThrowAwayComment#1{\begingroup
    \def\CurrentComment{#1}%
    \let\do\makeinnocent \dospecials
    \makeinnocent\^^L% and whatever other special cases
    \endlinechar`\^^M \catcode`\^^M=12 \xComment}
{\catcode`\^^M=12 \endlinechar=-1 %
 \gdef\xComment#1^^M{\def\test{#1}
      \csarg\ifx{PlainEnd\CurrentComment Test}\test
          \let\html@next\endgroup
      \else \csarg\ifx{LaLaEnd\CurrentComment Test}\test
            \edef\html@next{\endgroup\noexpand\end{\CurrentComment}}
      \else \let\html@next\xComment
      \fi \fi \html@next}
}
\makeatother

\def\includecomment
 #1{\expandafter\def\csname#1\endcsname{}%
    \expandafter\def\csname end#1\endcsname{}}
\def\excludecomment
 #1{\expandafter\def\csname#1\endcsname{\ThrowAwayComment{#1}}%
    {\escapechar=-1\relax
     \csarg\xdef{PlainEnd#1Test}{\string\\end#1}%
     \csarg\xdef{LaLaEnd#1Test}{\string\\end\string\{#1\string\}}%
    }}

%  Define environments that ignore their contents.
\excludecomment{comment}
\excludecomment{rawhtml}
\excludecomment{htmlonly}

%  Hypertext commands etc. This is a condensed version of the html.sty
%  file supplied with LaTeX2HTML by: Nikos Drakos <nikos@cbl.leeds.ac.uk> &
%  Jelle van Zeijl <jvzeijl@isou17.estec.esa.nl>. The LaTeX2HTML documentation
%  should be consulted about all commands (and the environments defined above)
%  except \xref and \xlabel which are Starlink specific.

\newcommand{\htmladdnormallinkfoot}[2]{#1\footnote{#2}}
\newcommand{\htmladdnormallink}[2]{#1}
\newcommand{\htmladdimg}[1]{}
\newenvironment{latexonly}{}{}
\newcommand{\hyperref}[4]{#2\ref{#4}#3}
\newcommand{\htmlref}[2]{#1}
\newcommand{\htmlimage}[1]{}
\newcommand{\htmladdtonavigation}[1]{}

% Define commands for HTML-only or LaTeX-only text.
\newcommand{\html}[1]{}
\newcommand{\latex}[1]{#1}

% Use latex2html 98.2.
\newcommand{\latexhtml}[2]{#1}

%  Starlink cross-references and labels.
\newcommand{\xref}[3]{#1}
\newcommand{\xlabel}[1]{}

%  LaTeX2HTML symbol.
\newcommand{\latextohtml}{{\bf LaTeX}{2}{\tt{HTML}}}

%  Define command to re-centre underscore for Latex and leave as normal
%  for HTML (severe problems with \_ in tabbing environments and \_\_
%  generally otherwise).
\newcommand{\setunderscore}{\renewcommand{\_}{{\tt\symbol{95}}}}
\latex{\setunderscore}

% -----------------------------------------------------------------------------
%  Debugging.
%  =========
%  Remove % from the following to debug links in the HTML version using Latex.

% \newcommand{\hotlink}[2]{\fbox{\begin{tabular}[t]{@{}c@{}}#1\\\hline{\footnotesize #2}\end{tabular}}}
% \renewcommand{\htmladdnormallinkfoot}[2]{\hotlink{#1}{#2}}
% \renewcommand{\htmladdnormallink}[2]{\hotlink{#1}{#2}}
% \renewcommand{\hyperref}[4]{\hotlink{#1}{\S\ref{#4}}}
% \renewcommand{\htmlref}[2]{\hotlink{#1}{\S\ref{#2}}}
% \renewcommand{\xref}[3]{\hotlink{#1}{#2 -- #3}}
%end{latexonly}
% -----------------------------------------------------------------------------
% ? Document-specific \newcommand or \newenvironment commands.
% ? End of document-specific commands
% -----------------------------------------------------------------------------
%  Title Page.
%  ===========
\renewcommand{\thepage}{\roman{page}}
\begin{document}
\thispagestyle{empty}

%  Latex document header.
%  ======================
\begin{latexonly}
   CCLRC / {\sc Rutherford Appleton Laboratory} \hfill {\bf \stardocname}\\
   {\large Particle Physics \& Astronomy Research Council}\\
   {\large Starlink Project\\}
   {\large \stardoccategory\ \stardocnumber}
   \begin{flushright}
   \stardocauthors\\
   \stardocdate
   \end{flushright}
   \vspace{-4mm}
   \rule{\textwidth}{0.5mm}
   \vspace{5mm}
   \begin{center}
   {\Large\bf \stardoctitle}
   \end{center}
   \vspace{5mm}

% ? Heading for abstract if used.
%  \vspace{10mm}
%  \begin{center}
%     {\Large\bf Abstract}
%  \end{center}
% ? End of heading for abstract.
\end{latexonly}

%  HTML documentation header.
%  ==========================
\begin{htmlonly}
   \xlabel{}
   \begin{rawhtml} <H1> \end{rawhtml}
      \stardoctitle
   \begin{rawhtml} </H1> \end{rawhtml}

% ? Add picture here if required.
% ? End of picture

   \begin{rawhtml} <P> <I> \end{rawhtml}
   \stardoccategory\ \stardocnumber \\
   \stardocauthors \\
   \stardocdate
   \begin{rawhtml} </I> </P> <H3> \end{rawhtml}
      \htmladdnormallink{CCLRC}{http://www.cclrc.ac.uk} /
      \htmladdnormallink{Rutherford Appleton Laboratory}
                        {http://www.cclrc.ac.uk/ral} \\
      \htmladdnormallink{Particle Physics \& Astronomy Research Council}
                        {http://www.pparc.ac.uk} \\
   \begin{rawhtml} </H3> <H2> \end{rawhtml}
      \htmladdnormallink{Starlink Project}{http://www.starlink.ac.uk/}
   \begin{rawhtml} </H2> \end{rawhtml}
   \htmladdnormallink{\htmladdimg{source.gif} Retrieve hardcopy}
      {http://www.starlink.ac.uk/cgi-bin/hcserver?\stardocsource}\\

%  HTML document table of contents.
%  ================================
%  Add table of contents header and a navigation button to return to this
%  point in the document (this should always go before the abstract \section).
  \label{stardoccontents}
  \begin{rawhtml}
    <HR>
    <H2>Contents</H2>
  \end{rawhtml}
  \htmladdtonavigation{\htmlref{\htmladdimg{contents_motif.gif}}
        {stardoccontents}}

% ? New section for abstract if used.
% \section{\xlabel{abstract}Abstract}
% ? End of new section for abstract

\end{htmlonly}

% -----------------------------------------------------------------------------
% ? Document Abstract. (if used)
%  ==================
% ? End of document abstract
% -----------------------------------------------------------------------------
% ? Latex document Table of Contents (if used).
%  ===========================================
% \newpage
\begin{latexonly}
   \setlength{\parskip}{0mm}
   \tableofcontents
   \setlength{\parskip}{\medskipamount}
   \markright{\stardocname}
\end{latexonly}
% ? End of Latex document table of contents
% -----------------------------------------------------------------------------
\newpage
\renewcommand{\thepage}{\arabic{page}}
\setcounter{page}{1}

\section {Introduction}

Various data storage solutions have been adopted by the Starlink Project,
including SCSI (Small Computer Systems Interface) Winchester Disks and
different types of magnetic tape storage.

SCSI Winchester Disk devices are representative of Primary Data Storage,
where data is maintained online, and may be accessed randomly. Secondary
Data Storage is removable data storage; data is only accessible when the
media has been inserted in the appropriate device as, for example with
floppy disk drives. Backup Data Storage generally works sequentially, and is
often represented by Magnetic Tape Subsystems.

This guide is concerned with Backup Data Storage, and sets out to highlight
the similarities and differences between the three tape options most
commonly selected for Starlink applications: DAT (Digital Audio Tape),
Exabyte and DLT (Digital Linear Tape). The purpose of this guide is to
provide a helpful starting-point for anyone seeking to know more about one
or other of these options.

The contents of the guide include:

\begin {itemize}

\item Information on the different technologies employed by DAT, Exabyte and
DLT.

\item Textual examples of available devices together with performance data and
a brief description of notable features.

\item A quick-lookup table embodying the principal features of all three tape
options covered.

\item Useful hints for operation and maintenance of media and drives.

\item Technical details that relate to DAT and Exabyte (included as
appendices).

\end {itemize}

\section {DAT, EXABYTE OR DLT?}

\subsection {Selection Criteria}

What do DAT, Exabyte and DLT have to offer, and at what price?

We have to accept that prices are continually on the move, driven by market
conditions and advances in manufacturing technologies. Let us consider some
current (November 1995) Price-Performance figures. At the lower end of the
price-range for these options, one can purchase a 4 GByte native capacity
DAT subsystem that incorporates switchable data compression, for around
\pounds 1000. A similarly disposed Exabyte subsystem with a native capacity
of 3 GByte can be purchased for a little over \pounds 1000; a 7 GByte native
version would cost about 50\% more. Then, for around \pounds 3000, DLT
subsystems are available with a native capacity of 10 GByte; a 20 GByte
native version would, again, cost about 50\% more. A 15 GByte native DLT
subsystem has recently appeared that currently costs little more than
its 10 GByte predecessor. In each case, data compression generally permits
up to a twofold increase in capacity.

If `Cost per GigaByte' is paramount then, faced with these example costings,
we might look no further than the DLT. But what other factors could possibly
influence our selection? We need to consider how much money we can outlay,
and how we can spend this most effectively. Our spending power will, for
example, determine how far up the product ranges we can go.

Reliability is an important issue (see Section 2.2), compatibility with
existing systems is another. Familiarity with a particular product can be
the deciding factor; one may prefer to stay with the technology which one
understands best, or which has a well-proven track-record, but one must be
careful not to overlook new technologies, or perhaps radical developments
of existing ones.

\subsection {Points for Consideration}

In seeking to make a well-balanced appraisal, we need to understand the
differing technologies behind the currently available tape storage options.
With a greater appreciation of the weaknesses and strengths of each, we are
less likely to make an inappropriate choice.

Despite generally slow access times (relative to those for SCSI Winchester
Disk subsystems, for example), Cartridge Tape Subsystems (CTS's) are excellent
for backup purposes because they are relatively inexpensive. The main
distinguishing features are storage capacity, data transfer rate, physical
format and media format.

The following points regarding storage capacities and transfer rates are worth
noting:

\begin {description}

\item [\ \ \ 1.] Net capacities can only be given for preformatted data. The
storage capacity depends on the available usable length of the tape as well as
its density (bits per inch). The recording format itself reduces the available
data storage capacity through the incorporation of Error Correction Codes
(ECC), Control Codes, and filemarks.

\item [\ \ \ 2.] Some drives afford data compression. The compression factor
will depend on the data compression algorithm as well as the number of redundant
data sequences. This is why the capacity indicated by manufacturers usually
refers to the total storage capacity for the appropriate media.

\item [\ \ \ 3.] If the data transfer rate for a given host exceeds the
recording rate of the tape device, then the tape will be written-to
near-continuously, and thus be utilised to its full potential; this action is
termed `Streaming'. The system's transfer rate is dependent upon the access time
of the media being backed up, as well as the transfer rates of the controller
and the bus. The backup software's logic and the size of the file being backed
up are additional influences.

\end {description}

If a new backup system is being considered, and overall data capacity is not
paramount, then a brief comparison of the major features of DAT and Exabyte
won't go amiss. The following points are therefore worth noting:

\begin {description}

\item [\ \ \ 4.] Under high usage, DAT devices are reckoned to prove themselves
to be longer lasting and more reliable than Exabyte devices.

\item [\ \ \ 5.] Thanks to faster tape transport and a different filemark
system, files can be found on a DAT drive up to five times faster than on an
Exabyte drive.

\item [\ \ \ 6.] The data transfer rates of these two types of drive are almost
identical. In fact, the computer and the controller, which send the data to the
tape drive, have more of an effect on the performance than the tape drive
itself. In most cases, the access time for the primary storage device and the
CPU performance of the computer on which the data backup is based are the
limiting factors, and not the transfer rate of the backup tape device.

\item [\ \ \ 7.] A particular technical advantage of DAT over Exabyte is the
tape's minimal tape/drum wrap angle - 90 degrees, compared to 221 degrees for
the Video8-based technology. This feature results in lower tape tension as well
as a simpler drive construction.

\end {description}

Points that particularly concern DLT are mentioned separately, as the
underlying technology differs radically from that for DAT and Exabyte:

\begin {description}

\item [\ \ \ 8.] Utilising a less complicated recording mechanism, DLT drives
should be inherently more reliable than their DAT and Exabyte counterparts.

\item [\ \ \ 9.] DLT drives currently have a data transfer rate of up to three
times that of DAT or Exabyte.

\item [\ 10.] Although Exabyte drives are currently available that will always
store more data on a cartridge than any available DAT drive, DLT drives offer up
to three times the capacity of the Exabytes.

\end {description}

\subsection {Recommendations}

\begin {itemize}

\item[{\Large\bf +}] Starlink recommends DAT drives for use by users for the
loading and backing up of their own personal data sets.

\item[{\Large\bf +}] DLT drives are recommended for use by system managers for
system backups, except at the smaller sites where a DAT drive will suffice.

\item[{\huge\bf -}] At the time of writing, Starlink is {\bf not} recommending
the purchase of Exabyte drives.

\end {itemize}
\section {Magnetic Tape Cartridge Media}

As far back as 1990/1991, accelerated life testing by the NML (National
Media Lab, USA) revealed problems with the dimensional stability of tape
substrates, and problems with the permanence of the magnetic characteristics
of the `pigment' used. The problem with the substrate is that most 4mm and
8mm tapes for DAT and Exabyte applications are highly tensilised (stretched)
in the longitudinal direction. This adds strength and stiffness to the tape.
Unfortunately, in an environment with high temperature and/or humidity, the
substrate tries to revert to its original shape. This causes distortion of
the tape geometry, which poses significant problems especially for helical
recording systems, as employed by DAT and Exabyte drives. Note that QIC
(Quarter Inch Cartridge) drives and DLT drives (which utilise half-inch
tape cartridges) do not use tensilised tape, and moreover do not depend on
helical recording.

\subsection {Magnetic Tape Composition}

The tapes used in DAT and 8mm (Exabyte, for example) helical scan technologies
are based on MP (Metal Particle) formulations designed originally for audio and
video applications. When exposed to temperature and humidity extremes, it
has been shown that oxidation can occur which changes the magnetic state of the
medium, and can alter or destroy data; this can occur in spite of the
encapsulation of the metal particles in a ceramic material, designed to counter
such effects. Apparently, QIC tapes use an older magnetic layer, which provides
for greater stability.

DAT's main problem for the future is packing the higher capacities on to
such a small thin tape, as DAT has the lowest area of tape available of all
the technologies.

\subsection {Caring for your Tapes}

Only data media especially designed for long term archival storage should be
used. Media bearing the DDS (Digital Data Storage) seal should be used with DAT
drives, since this media has been specifically developed for data storage;
audio cartridges should {\bf not} be used.

Original Exabyte cartridges are particularly advised for Exabyte drives.
Exabyte and Sony guarantee a minimum archive longevity of 10 years for their
own manufactured media.

In a report by NML on ``Proper Archival of Data Storage on 8mm Metal-Particle
Media'' (6th December 1990) they claim that ten year archival is entirely
feasible under the following conditions:

\begin {itemize}

\item Archival tapes should not be brand new. Rather they should have
undergone somewhere between four and twenty passes through a drive before
the data are recorded on them. This is because new tapes often contain
manufacturing debris that gets removed in the first few passes.

\item The operational and storage environments should be maintained at a
constant temperature and humidity to reduce media stress. The preferred
environment is between 62 and 75 degrees Fahrenheit at 40\% humidity.

\item Always return the tapes to their protective enclosures and store them
on edge rather than stacked flat.

\item Do not drop them, and do not put heavy weights on top of them.

\item Keep the operational and storage areas clean.

\item Clean your Exabyte drives regularly, using only the Exabyte cleaning
tapes. In order to assure that things are going well, sample data sets
should be read back periodically and the error rates should be carefully
monitored.

\end {itemize}

{\bf Note} that it would seem appropriate to apply these handling conditions to
all Magnetic Cartridge Tape media.

The Exabyte manufacturers themselves had concluded, after their own extensive
tests, that the only point considered crucial was that a tape be allowed to
acclimatise to the environment for 24 hours before reading from or writing
to it. This implies that one should remove the plastic wrapping film from a
brand new tape at least a day before intended use.

\section {Operational Maintenance}

With the degree to which Cartridge Tape Subsystems have been modularised,
the remedy for a faulty unit might involve the replacement of a module, or
the replacement of an entire unit - under a Maintenance Contract of course!

The only regular attention required of the `operator' is the cleaning of the
read/write heads, using the appropriate recommended cleaning kit. If cleaning
agents other than those prescribed by the drive manufacturer are applied,
then the read/write heads can be damaged. Drive manufacturers invariably
recommend cleaning after a maximum data transfer of 30 GByte, or at least
once a month.

\section {Example Cartridge Tape Subsystems - past and present}

Whilst technology marches ahead at its own pace, manufacturers and suppliers
are keen to maintain their own edge on the market by shipping out `the next
generation' of a product ahead of the competition. When one is faced with a
large choice from a range with not dissimilar specifications and price-tags,
we are quite likeley to be moved toward a given product for its packaging.
This is not unreasonable when selecting an external (i.e. desktop) Cartridge
Tape Subsystem. For a given technology, DAT for example, there may be a
number of different suppliers who are utilising the same basic drive, but
have incorporated this into their own custom-designed enclosure. Features
specific to the enclosure might include LED (Light-Emitting Diode) diagnostics
display lamps, DIP Switches (Dual-Inline-Pin hardware switches) (for
controlling data compression, selection of recording mode, host selection,
etc.), SCSI-select switches and read-out displays (for error rates or
tape-remaining, etc.).

Brief descriptions of some example Cartridge Tape Subsystems for DAT,
Exabyte and DLT are given in this Section. Current models that are in place at
some or all Starlink Sites are represented, as are some earlier models which
may be present at some sites. Also included, for comparison, are some that have
never been adopted by Starlink.

For a given technology one will recognise identical performance figures
for certain models; this will often be due to the same tape drive unit being
incorporated into these models.

But please {\bf note}:
\begin {itemize}

\item[{\LARGE\bf $\star$}] The Starlink Project view all Manufacturers' capacity
and throughput performance figures as being only ``Guaranteed not to exceed''.

\end {itemize}
It is likely that some models have already been superseded by others, and
that some newer products have not been included. Nonetheless, the intention
herein is to provide the basis for making a comparison between the three
tape options in question, and for appreciating what is basically available.

\subsection {DAT Subsystems with HP Drives}

\paragraph {HP35480A}

The HP35480A (3.5 inch Half Height form factor) incorporates a DDS drive, and
can store up to 2 GByte native on a 90 metre tape cartridge, with a quoted data
transfer rate of 183 KByte/sec. Up to 4 GByte can be stored on the same
cartridge using DDS-DC data compression, with a quoted data transfer rate of
0.4 MByte/sec. The HP35480A features a 1 MByte data buffer.

Data compression is generally controlled through SCSI commands and the setting
of DIP Switches (Dual-Inline-Pin hardware switches). This should allow for a
range of operation from transparent data compression, where the HP35480A is
connected to systems with no changes to existing host drivers, to full control
and optimisation of the data compression features.

Fast-Search provides an average access time of 30 seconds for any part of a
90 metre tape.

\paragraph {HP35480A DAT drive under SUN Solaris 2.X}

For HP35480A DAT drives on Solaris 2.X platforms, users are unable to flip
between compression and non-compression modes by simply referring to the
device by different device names; though the drives in theory should support
this.

Some drivers are `smart' in that they can enable/disable compression
according to device file minor numbers (i.e. the name by which the device
is referred to). Sun's Solaris driver is different in that it cannot control
DC-ON/OFF in the DAT drive based on the device file name (even though it can
do this for other tape devices). The problem arises due to multiple methods
in the SCSI specification for switching DC {\tt on} and {\tt off}. HP uses
one method, Exabyte uses another, etc.

A small custom program has been produced that modifies the system kernel
data structures to support DC-ON and DC-OFF, according to the value of a
user-supplied parameter. For this fix to function, the HP35480A {\bf must}
be installed with Version 11 (or later) of the DAT Firmware.

The program can be found on
{\tt http://www.starlink.ac.uk/}\~{ }{\tt cac/sunspot/dat.sun}; {\bf note}
that it must generally be run as {\tt root}, but could conceivably be installed
with the {\tt setuid} bit set, if users are to be allowed to switch DC modes
themselves.

The recommended DIP Switch settings on the DAT drive are as follows:

\ \ \ \ Switch 1 {\tt off}\ \ *\ \ Switches 2-5 {\tt on}\ \ *\ \ Switches
6-8 {\tt off}

The setting of Switch 1 is not important if DC control is via this program.
Usually, if Switch 1 is {\tt on}, then data compression is {\tt on}, and vice
versa.

\paragraph {HP35480A DAT drive under DEC OSF/1}

The above-mentioned driver problem associated with SUN Solaris 2.X does not
affect operation of the HP35480A under DEC OSF/1; hence, for DEC Alphas,
there is no need to run the custom program as specified for switching DC
{\tt on} and {\tt off} under SUN Solaris 2.X. However, the requirement for
Version 11 (or later) of the DAT Firmware {\bf does} apply.

The recommended DIP Switch settings on the DAT drive are as follows:

\ \ \ \ Switches 1-5 {\tt on}\ \ *\ \ Switches 6-8 {\tt off}

\paragraph {HPC1533A}

The HPC1533A (3.5 inch Half Height form factor) supports the newer DDS-2
standard, which is backwards-compatible with the DDS standard used by the
HP35480A. Up to 4 GByte can be stored on a 120 metre cartridge on this drive
in native mode, with a quoted data transfer rate of 510 KByte/sec. The max.
capacity increases to 8 GByte with DDS-DC data compression, with a quoted data
transfer rate of 1 MByte/sec.

Because they are longer and thinner, the 120 metre tapes intended for use
with these drives should {\bf not} be used in other DAT drives.

Media can be exchanged with other drives (with the notable exception of the
120 metre tape cartridges), as long as each supports the same format, and
compatible backup software is used. If data compression is in play, the
drives must use the same compression technique (DDS-DC).

\subsection {DAT Subsystems from SUN}

SUN produce a DAT drive (3.5 inch Half Height form factor), which they
document as a ``5GB 4mm Backup Tape Drive''. It is, in reality, a 2 GByte
native capacity subsystem, for a 90 metre tape. Incorporating a DDS/DDS-DC
tape drive unit, it features a 1 MByte data buffer.

Fast-Search and tape rewind are performed at up to 100 times the nominal
tape speed, and the average file access time for a 90 metre tape is under
30 seconds.

The native data transfer rate quoted is 366 KByte/sec.  Built-in data
compression is said to increase native capacity and transfer rate by between
two and four times; the amount of data compression realised will vary with
the actual data being compressed; 5 GByte and 915 KByte/sec assume a typical
compression ratio of 2.5:1.

Originally, under Solaris 2.1, the drive automatically defaulted to using
data compression mode only. Beyond Solaris 2.1, the drive should be able to
operate optionally in either compressed or non-compressed mode via a
software {\tt mode-select} command.

\subsection {DAT Subsystems from TTi}

\paragraph {CTS-4000}

The CTS-4000 (3.5 inch Half Height form factor) incorporates a DDS drive that
can store up to 2 GByte on a 90 metre tape, with a quoted data transfer rate
of 183 KByte/sec.

The CTS-4000 features a 1 MByte data buffer, and a Fast-Search capability that
functions at approximately 200 times the nominal read/write speed.

\paragraph {CTS-4410}

The CTS-4410 (3.5 inch Half Height form factor) incorporates a DDS-2/DDS-DC
drive. Using a 120 metre tape it has a native capacity of up to 4 GByte,
with a quoted data transfer rate of 366 KByte/sec. The max. capacity increases
to 8 GByte with DCLZ data compression, with a quoted data transfer rate of
732 KByte/sec.

The CTS-4410 features a 1 MByte data buffer, and a Fast-Search capability that
functions at approximately 200 times the nominal read/write speed.

The front-panel LED display on the device enclosure indicates the remaining
native tape capacity in Megabytes. The display shows, also, ECC error
correction code usage to help monitor tape quality and wear.

Additional front-panel LEDs relate to the drive's current mode of operation
(Read, Write, WFM (Writing-File-Mark), Rewind, Search), `Write Protection' on
the data cartridge, `Tape in Use' and `Use Cleaning Tape'; the latter is
triggered when the `Percentage Rewrites/ECC' exceeds a fixed, pre-determined
threshold.

Data compression on the CTS-4410 is controlled by a pushbutton switch on the
front panel. Host control over compression can be set using a switch located
on the rear of the unit.

\subsection {DAT Subsystems from Exabyte}

\begin {itemize}

\item[{\huge\bf -}] The Exabyte 4200 Series subsystems are unlikely to be found
in any Starlink Site's hardware base. They are included here for comparison with
the HP, SUN and TTi subsystems.

\end {itemize}

\paragraph {EXB-4200c}

The EXB-4200c (3.5 inch Half Height form factor) incorporates a DDS-DC drive
and features a 1 MByte data buffer. The EXB-4200c can store up to 2 GByte
native on a standard 90 metre tape cartridge, with a quoted data transfer
rate of 233 KByte/sec. The max. capacity increases to 4 GByte with data
compression, with a quoted data transfer rate of 466 KByte/sec.

\subsection {Exabyte 8mm Subsystems}

\paragraph {EXB-8200T}

The EXB-8200T tabletop Cartridge Tape Subsystem (5.25 inch Half Height form
factor) is an earlier model in the Exabyte range, offering up to 2.5 GByte
of data storage capacity on a 112 metre tape, with a quoted data transfer
rate of 246 KByte/sec; data compression is not an option. The EXB-8200T
incorporates a 256 KByte `Speed-Matching' buffer.

Fast-Search is performed at 10 times nominal tape speed, and maximum tape
rewind speed is 75 times nominal.

Various operational states are indicated via the firmware-controlled activity
on a pair of LEDs on the front-panel. A combination of {\tt fast, medium, slow}
and {\tt random} flash, {\tt on}, and {\tt off} for the amber and green
LEDs reveal:\ \ power-on, self-test, ready and tape-loaded statuses, SCSI bus
activity, normal tape motion, Fast-Search/Rewind, and certain fail-conditions.

\paragraph {EXB-8500T}

The EXB-8500T tabletop Cartridge Tape Subsystem (5.25 inch Half Height form
factor) is a more recent product in use at some Starlink Sites, offering up
to 5 GByte of data storage capacity and a quoted data transfer rate of 0.5
MByte/sec in its EXB-8500 mode; or up to 2.5 GByte and 246 KByte/sec in
EXB-8200 mode. The EXB-8500T features a 1 MByte `Speed-Matching' buffer. The
EXB-8500T can also read or write cartridges in the EXB-8200 format.

Fast-Search is performed at up to 37.5 MByte/sec, and maximum tape rewind
speed is 75 times nominal.

Various operational states are indicated via the firmware-controlled activity
on a pair of LEDs on the front-panel. A combination of {\tt fast, medium, slow}
and {\tt random} flash, {\tt on}, and {\tt off} for the amber and green
LEDs reveal:\ \ power-on, self-test, ready and tape-loaded statuses, SCSI bus
activity, normal tape motion, Fast-Search/Rewind, and certain fail-conditions.

More up-to-date Exabyte models (EXB-8205XL and EXB-8505XL, for example)
exist that utilise specifically designed cartridges to achieve greater
storage capacities.

\paragraph {EXB-8205XL}

The EXB-8205XL (5.25 inch Half Height form factor) can store up to 3.5 MByte
native on a 160 metre tape, with a quoted data transfer rate of 263
KByte/sec. The max. capacity increases to 7 GByte with IDRC data compression,
with a quoted data transfer rate of 0.5 MByte/sec.

\paragraph {EXB-8505XL}

The EXB-8505XL (5.25 inch Half Height form factor) can store up to 7 MByte
native on a 160 metre tape, with a quoted data transfer rate of 0.5
MByte/sec. The max. capacity increases to 14 GByte with IDRC data compression,
with a quoted data transfer rate of 1 MByte/sec.

The EXB-8505XL is read/write compatible with the EXB-8205XL, but it has two
read and two write heads and so can record with twice the density.

The EXB-8505XL automatically adjusts to the format of the inserted cartridge.
Other recording formats can be used through the selection of other device
files via a set of user-accessible mode switches. Blank cartridges are
automatically initialised in the standard format of the drive.

{\bf Note} that these new 160 metre cartridges {\bf cannot} be used in older
drives. The tape used in this new medium is considerably thinner than that
in standard cartridges, and is likely to be torn if used in older Exabyte
drives.

\paragraph {Exabyte Mammoth}

The Exabyte Mammoth, supposedly available from the end of 1995, should be
capable of storing up to 20 GByte native, and up to 40 GByte with data
compression. This is three times the capacity of currently available Exabyte
drives. The suggested data transfer rate of around 3 MByte/sec is presumably
for compressed data.

\subsection {8mm Subsystems from TTi}

\begin {itemize}

\item[{\huge\bf -}] The following TTi 8mm subsystem is unlikely to be found in
any Starlink Site's hardware base, but has been included for comparison with its
Exabyte counterparts.

\end {itemize}

\paragraph {CTS-8000H}

The CTS-8000H (5.25 inch Half Height form factor) is a single-drive tabletop
unit that can store up to 5 GByte native or up to 10 GByte compressed on a
112 metre tape; the quoted data transfer rates are 0.5 MByte/sec and 1
MByte/sec, respectively. The CTS-8000H features a 1 MByte `Speed-Matching'
buffer.

Fast-Search allows the drive to fast-forward to any tape location, with an
average access time of 85 seconds for a tape that contains up to 10 GByte of
data. The rewind speed is 75 times the nominal tape speed.

The CTS-8000H permits full data interchange with existing EXB-8200 and
EXB-8500 based 8mm drives. Tapes written on any 2.5 GByte or 5 GByte drive
can be read by Series 8000H drives; tapes can also be written that can be
read by 2.5 GByte or 5 GByte drives.

The front-panel LED display on the device enclosure indicates the remaining
native tape capacity in Megabytes. The display shows, also, ECC error
correction code usage to help monitor tape quality and wear.

The required recording format can be used through the selection of the
appropriate device file via a set of user-accessible mode switches. The
available emulations are Exabyte EXB-8200/8500, DEC TK50Z, IBM RS/6000 and
HP-3645.

\subsection {DLT Subsystems from TTi}

\paragraph {CTS-2110}

The CTS-2110 (5.25 inch Full Height form factor) has a native capacity of up
to 10 GByte on a 1200ft tape, with a quoted data transfer rate of 1.25
MByte/sec. The max. capacity increases to 20 GByte with data compression, with a
quoted data transfer rate of 2.5 MByte/sec.

The front-panel LED display on the device enclosure indicates the remaining
native tape capacity in Megabytes. The display shows, also, ECC error
correction code usage to help monitor tape quality and wear.

Additional front-panel LEDs relate to the drive's current mode of operation
(Read, Write, WFM (Writing-File-Mark), Rewind, Search), `Write Protection' on
the data cartridge, `Tape in Use' and `Use Cleaning Tape'; the latter is
triggered when the `Percentage Rewrites/ECC' exceeds a fixed, pre-determined
threshold.

Data compression is facilitated via a `Density Select' pushbutton switch on the
front-panel.

The CTS-2110 was subject of a Starlink Project evaluation exercise early
this year (January 1995). Details are presented later in this paper (see
Section 8.6).

\paragraph {CTS-2110XT}

The CTS-2110XT (5.25 inch Full Height form factor) has a native capacity of
up to 15 GByte on an 1800ft tape, with a suggested data transfer rate of about
1.38 MByte/sec. The max. capacity increases to 30 GByte with data compression,
with a suggested data transfer rate of about 2.75 MByte/sec.

The CTS-2110XT may also operate with a 1200ft tape, when the capacities and
data transfer rates are as given for the CTS-2110.

The front-panel LED display on the device enclosure indicates the remaining
native tape capacity in Megabytes. The display shows, also, ECC error
correction code usage to help monitor tape quality and wear.

Additional front-panel LEDs relate to the drive's current mode of operation
(Read, Write, WFM (Writing-File-Mark), Rewind, Search), `Write Protection' on
the data cartridge, `Tape in Use' and `Use Cleaning Tape'; the latter is
triggered when the `Percentage Rewrites/ECC' exceeds a fixed, pre-determined
threshold.

Data compression is facilitated via a `Density Select' pushbutton switch on the
front-panel.

\paragraph {CTS-2210}

The CTS-2210 (5.25 inch Full Height form factor) has a native capacity of up
to 20 GByte on an 1800ft tape, with a quoted data transfer rate of 1.5
MByte/sec. The max. capacity increases to 40 GByte with DLZ (Digital Lempel-Ziv)
high-efficiency data compression, with a quoted data transfer rate of 3
MByte/sec. A dual-channel read/write head and a tape mark directory maximise
data throughput and minimise data access time.

The front-panel LED display on the device enclosure indicates the remaining
native tape capacity in Megabytes. The display shows, also, ECC error
correction code usage to help monitor tape quality and wear.

Additional front-panel LEDs relate to the drive's current mode of operation
(Read, Write, WFM (Writing-File-Mark), Rewind, Search), `Write Protection' on
the data cartridge, `Tape in Use' and `Use Cleaning Tape'; the latter is
triggered when the `Percentage Rewrites/ECC' exceeds a fixed, pre-determined
threshold.

Data compression is facilitated via a `Density Select' pushbutton switch on the
front-panel.

The CTS-2210 features, also, a configurable Personality Module to emulate
Exabyte EXB-8200/8500, DEC TK50Z and IBM RS/6000. A set of switches on the rear
panel of the enclosure work with custom firmware in the subsystem. The required
emulation is selected via the front panel or switch settings at the rear.

\subsection {DLT Subsystems with Quantum Drives}

\begin {itemize}

\item[{\Large\bf +}] The DLT200 and DLT4000 subsystems are DEC-compatible, and
are approved by the Starlink Project for operating with DEC host systems.

\item[{\huge\bf -}] The DLT2000 and DLT4000 subsystems, however, do {\bf not}
feature a configurable Personality Module, and hence are {\bf not} recommended
by the Starlink Project for operating with SUN host systems.

\end {itemize}

\paragraph {DLT2000}

The DLT2000 (5.25 inch Full Height form factor) can store up to 20 GByte of
data on a half-inch tape cartridge using DLZ (Digital Lempel-Ziv) data
compression, with a quoted data transfer rate of 2.5 MByte/sec.

Front-panel LED information includes `Write Protection' on the data cartridge,
`Tape in Use' and `Use Cleaning Tape'.

\paragraph {DLT4000}

By using a half-inch tape cartridge with a higher density and DLZ data
compression, the DLT4000 (5.25 inch Full Height form factor) can store up to
40 GByte with a quoted data transfer rate of 3 MByte/sec. This drive is
read-compatible with DLT2000 tapes as well as with tapes created on older
generation DLT drives.

Front-panel LED information includes `Write Protection' on the data cartridge,
`Tape in Use' and `Use Cleaning Tape'.

\subsection {A Summary of DAT, Exabyte and DLT Performance - Table 1}

\begin{table}[h]
\begin{scriptsize}
\begin {center}
\begin{tabular}{||c|c|c|c|c|c|c|c|c||} \hline
& & \multicolumn{2}{c|}{{\em (Maximum)}} &
\multicolumn{2}{c|}{{\em (Maximum)}} & & & \\
{\em TAPE} & {\em DRIVE} & \multicolumn{2}{c|}{{\em DATA CAPACITY}}
& \multicolumn{2}{c|}{{\em TRANSFER RATE}}
& {\em TAPE} & {\em FORM} & {\em DRIVE} \\ \cline{3-6}
{\em SYSTEM} & {\em MODEL} & {\em Native} & {\em Compr.}
& {\em Native} & {\em Compr.} & {\em SIZE} & {\em F'CT'R}
& {\em FORMAT} \\ \hline
{\em DAT} & {\bf HP35480A} & 2GB & 4GB & 183KB/s & 0.4MB/s
& 90m & 3.5HH & DDS/DDS-DC \\
{\em (4mm)} & {\bf HPC1533A} & 4GB & 8GB & 510KB/s & 1MB/s
& 120m & 3.5HH & DDS/DDS-2 \\
& & & & & & & & /DDS-DC \\
& {\bf SUN 5GB} & 2GB & 5GB & 366KB/s & 915KB/s
& 90m & 3.5HH & DDS/DDS-DC \\
& {\bf CTS-4000} & 2GB & n/a & 183KB/s & n/a
& 90m & 3.5HH & DDS \\
& {\bf CTS-4410} & 4GB & 8GB & 366KB/s & 732KB/s
& 120m & 3.5HH & DDS/DDS-DC \\
& {\bf EXB-4200c} & 2GB & 4GB & 233KB/s & 466KB/s
& 90m & 3.5HH & DDS/DDS-DC \\ \hline \hline
{\em EXABYTE} & {\bf EXB-8200T} & 2.5GB & n/a & 246KB/s & n/a
& 112m & 5.25HH & \\
{\em (8mm)} & {\bf EXB-8500T} & 2.5GB & 5GB & 246KB/s & 0.5MB/s
& 112m & 5.25HH & \\
& {\bf EXB-8205XL} & 3.5GB & 7GB & 263KB/s & 0.5MB/s
& 160m & 5.25HH & \\
& {\bf EXB-8505XL} & 7GB & 14GB & 0.5MB/s & 1MB/s
& 160m & 5.25HH & \\
& {\bf CTS-8000H} & 5GB & 10GB & 0.5MB/s & 1MB/s
& 112m & 5.25HH & \\ \hline \hline
{\em DLT} & {\bf CTS-2110} & 10GB & 20GB & 1.25MB/s & 2.5MB/s
& 1200ft & 5.25FH & \\
{\em (0.5in)} & {\bf CTS-2110XT} & 15GB & 30GB & 1.38MB/s & 2.75MB/s
& 1800ft & 5.25FH & \\
& {\bf CTS-2210} & 20GB & 40GB & 1.5MB/s & 3MB/s
& 1800ft & 5.25FH & \\
& {\bf DLT2000} & n/a & 20GB & n/a & 2.5MB/s
& 1200ft & 5.25FH & \\
& {\bf DLT4000} & n/a & 40GB & n/a & 3MB/s
& 1800ft & 5.25FH & \\ \hline
\end{tabular}
\caption {Nominal Performance Specifications for DAT, Exabyte and DLT}
\end{center}
\end{scriptsize}
\end{table}

\section {DAT SUBSYSTEMS}

Sony entered the market in 1988 with 4mm DAT (Digital Audio Tape) drives based
on their commercial audio DAT drives, but it was Hewlett Packard who developed
the UK market for DAT.

\subsection {DAT Technology}

Certain computer tape drives and audio mechanisms, such as QIC (Quarter
Inch Cartridge) and DLT (Digital Linear Tape, see Section 8), record on
parallel tracks along the length of the tape. The data storage capacities
achieved via this method, though, can be limited by mechanical tolerances
and magnetic crosstalk (interaction between signals from neighbouring tracks).

These problems are largely overcome within DAT devices through helical scan
technology, in which high recording density and data storage capacity are
attained by recording tracks diagonally across the tape.

Two read/write heads are mounted on a rotating drum whose axis is off-set
6 degrees from the vertical. With a 90 degree tape/drum wrap angle, the tape
moves slowly (8mm/sec) across the drum, which rotates in the same direction
at 2000rpm, while the diametrically-opposed heads describe segments of a
helix on the tape. Each head writes a track of data on the tape from bottom
to top.

The heads are wider than the tracks, so the tracks overlap with no wasted
space between them. Crosstalk between the tracks is minimised by each head
writing its data in angled strips along the track; this angle is called the
azimuth angle. Each head is set with a different azimuth angle, so alternate
tracks on the tape have their data written at different angles. This
technique is known as `Alternate-Azimuth Recording'.

In the design of the DDS-format DAT drives used by Starlink, two more heads
are located within the rotating drum at 90 degrees to the others. These
enable the drive to read data immediately after it has been written. If a
write error is detected, the drive can rewrite the erroneous frame repeatedly
until it can be read back successfully.

A disadvantage of the DAT technology is that, in order to provide the required
amount of contact between the drum and the tape, the tape needs to be drawn
out of the cartridge housing and wrapped around a sector of the drum. To
achieve this level of tape manipulation, a number of expensive servo motors
are required.

Another drawback with the mechanism design is that as the drum spins at a
high rate, the heads accordingly tend to wear-out at a higher rate than for
fixed-head design.

\subsection {DAT Recording Formats}

The DDS (Digital Data Storage) format was designed specifically for computer
data storage, and was originally implemented for DAT subsystems, yielding an
intended capacity of 1.3 GByte on a 60 metre tape. This soon progressed to
2 GByte with the introduction of 90 metre tapes.

The DDS format has been extended to include data compression. This newer
format is called DDS-DC (Digital Data Storage - Data Compression), allowing
up to 8 GByte of compressed data to be stored on a tape. The DDS-DC format
uses the same error correction techniques as implemented in the DDS format.

Data capacities have been further enhanced with the introduction of DDS-2.
This advancement owes much to improvements in manufacturing techniques that
permit the production of longer, thinner tapes. Up to 4 GByte native can be
stored on a 120 metre tape. The DDS-2 standard is backward compatible with
the DDS standard. Some Starlink Sites have DDS-2 drives.

Another recording format for DAT, DataDAT, enables random access to the
stored data, as opposed to the sequential access provided for by DDS.
However, it requires that the tape first be formatted, which can take of the
order of 60 minutes for a 90 metre tape. {\bf Note} that this format is
{\bf not} recommended by Starlink.

\subsection {Fast-Search for DAT}

DDS, DDS-2 and DDS-DC format DAT drives can perform a Fast-Search at up to
200 times the nominal read/write speed.

Fast-Search is invoked by an internal algorithm making decisions about what
is in the buffer, and depends on whether the command is a {\tt space} forward
or reverse of a number of blocks or filemarks, for example.

Fast-Search for DAT is described in more detail in Appendix A.

\subsection {Error Handling for DAT}

Audio DAT has two levels of Error Correction Code (ECC) - C1 and C2. DDS
uses the same levels as the audio DAT format but adds extra error-correction
techniques:

\begin {enumerate}

\item C3 ECC.

\item Read-After-Write (RAW).

\item N-Group Writing.

\item Data Randomiser.

\item Checksums.

\end {enumerate}

Error-correction techniques employed by DDS are described in detail in
Appendix A.

\subsection {Data Compression for DAT}

Data compression reduces the amount of physical activity that is needed to
get the data onto tape. Consequently, the compression decreases the error
rate in real terms. For example, if the error rate without compression is
1 in 10 to the 15th bits, and the compression ratio is 2:1, the error rate
with compression becomes 1 in 2 times 10 to the 15th bits - twice the
performance. If a hard error does occur however, more data will be lost,
because more data is written in the same area of tape.

The data compression algorithm is itself lossless; that is, it guarantees
that what was compressed can be decompressed without error. Equally, this
means that if an error exists in the data before it is compressed, the error
will still be there after decompression.

Most current generation DAT tape drives have data compression capability
built in. This affords a twofold to fourfold increase in both the native
capacity and the data transfer rate specifications of the drive, depending
on the nature of the data being compressed. The data compression algorithm
employed by DAT is reckoned to be one of the most robust available, usually
of the LZ (Lempel-Ziv) type, making the data transfer rate performance of
DAT drives comparable to that of 8mm tape drives utilising data compression.

In the DDS-DC format, data blocks are compressed into entities which consist
of a header and one or more compressed data blocks. The header, which
contains control information about the compressed data blocks, remains
uncompressed. The DDS-DC format thus allows compressed data to be stored in
a way that maintains the full functionality of the DDS format, whilst
ensuring backward compatibility with existing DDS-format drives.

In general, the existence of entities is completely transparent to the host
system. However, in order for interchange to be possible between devices
which support compression and those that do not, some host systems may
implement software decompression. In this case it is necessary for the host
system to have a full knowledge of the structure of an entity.

\subsection {DAT Cartridge Handling Subsystems}

Cartridge handling subsystems are designed for facilitating the backing up
of large data sets. Of these subsystems, known variously as stackers,
autoloaders, automatic-changers or jukeboxes, models are currently available
for DAT that can handle from 6 to 50 cartridges.

The simpler models allow only sequential manipulation of the cartridges,
whereby the next tape in physical sequence is automatically loaded following
an {\tt unload} command. One disadvantage of this method is that the autoloader
may have to search through all of the tapes before finding the correct
cartridge. Random access models are generally more expensive, and usually
require special firmware to load specific tapes in the magazine.

\subsection {DAT Form Factor}

DAT has the advantage of small media with a large capacity, at a relatively
low cost. It is the only technology in this mid-range tape market with a 3.5
inch Half Height form factor as standard. This allows for very flexible
packaging options, fitting the same space as a 3.5 inch disk or floppy drive,
for example, within a desktop system package. Additionally, DAT's small form
factor has fuelled the development of very small tape autoloaders that fit
standard 5.25 inch form factor packaging. With such small autoloaders, DAT
can provide low cost unattended backup of around 20 to 30 GByte capacity, and
fit 5.25 inch disk drive packaging found in many data centre file servers.

\subsection {DAT Performance}

DAT drives have very fast file access times due to features originally
developed for audio applications. Individual files can be located within
30 seconds on average, twice as fast as for 8mm devices. Moreover, the
reliability of 4mm drives generally is deemed to be significantly higher
than for 8mm technology, reflecting more robust head drum motors and
assemblies. This is almost certainly a consequence of the audio origins of
4mm technology, being designed to handle a high degree of start/stop
activity.

Manufacturers' nominal data transfer rates for 4mm DAT drives can be as high
as 0.5 MByte/sec native, and 1 MByte/sec compressed; these figures are
near-identical to those for comparative 8mm subsystems.

But please {\bf note}:
\begin {itemize}

\item[{\LARGE\bf $\star$}] The Starlink Project view all Manufacturers' capacity
and throughput performance figures as being only ``Guaranteed not to exceed''.

\end {itemize}
\subsection {Applications for DAT}

DAT is reckoned to be particularly suitable for archive and retrieval, data
interchange, software distribution, besides standard backup applications.
For backup functions, where on-going, unattended operation is required, DAT
drives are expected to experience fewer failures and provide more consistent
operation than their 8mm counterparts.

DAT drives can be considered a cost-effective option over the 2 to 4 GByte
range, where they are used typically for backups. However, with the cost of
alternative technologies falling, DAT may well be outpaced by DLT, for
example, where capacities are more positively on the increase.

\section {EXABYTE SUBSYSTEMS}

In 1985 the founders of Boulder, Colorado-based Exabyte Corp. took their cue
from the 8mm tapes used in portable camcorders, and set about adapting the
technology for computer data storage peripherals. Prompted into action by
tape technology that had failed to keep up with the ever-increasing capacity
of Winchester disk drives, Exabyte introduced its first 8mm tape drive for
computer data storage in 1987.

Although Exabyte competes well with DAT (Digital Audio Tape, see Section 6)
in the mid-market range, it is already being surpassed by DLT (Digital Linear
Tape, see Section 8), and is likely to fall further behind.

\subsection {Exabyte Technology}

Exabyte drives originated out of Sony ``Video8'' camcorder technology, using
double the width of DAT tape to allow even greater data storage capacities
to be obtained. 8mm has gone through a series of changes but still gives
full backwards compatibility with the earlier tapes.

In common with DAT, Exabyte also uses helical scan technology to attain high
recording density, with opportunity for correspondingly high data storage
capacity. The tape/drum wrap angle, however, is 221 degrees, compared with
90 degrees for DAT; the consequences are higher tape tension and a more
complex drive construction.

A refinement of helical scan technology termed `Alternate-Azimuth Recording',
utilises two read heads and two write heads to permit recording at twice the
density.

\subsection {Exabyte Recording Formats}

\begin {itemize}

\item[{\huge\bf -}] Two Exabyte recording formats, selected by way of example,
relate to the Exabyte EXB-8200 and EXB-8500 subsystems. These particular
subsystems are {\tt not} in the {\tt preferred} Starlink set; though the
formats are available to other drives, in some instances via a Personality
Module.

\end {itemize}

The Manufacturer's quoted capacity for the Exabyte EXB-8200 drive operating
with a 112 metre Exatape cartridge is 2.5 GByte, utilising the EXB-8200
format. The corresponding figure  for the EXB-8500 drive is 5 GByte with the
same tape but utilising the EXB-8500 format. The EXB-8500 can also read or
write cartridges in the EXB-8200 format.

But please {\bf note}:
\begin {itemize}

\item[{\LARGE\bf $\star$}] The Starlink Project view all Manufacturers' capacity
and throughput performance figures as being only ``Guaranteed not to exceed''.

\end {itemize}
More up-to-date Exabyte models exist that use specifically designed 160 metre
cartridges to achieve, in the case of the EXB-8505XL, a native capacity of
up to 7 GByte, or up to 14 GByte with data compression.

\paragraph {Exabyte Tape Capacity}

A few words of warning are offered on the writing of data to Exabyte. In a
simple test in which many single-block files were `copied' to a 2.3 GByte
capacity cartridge tape, only 316 files were accommodated. This is only 161
KByte on a full cartridge, a factor of 15,000 less than advertised! The
reason for this is that the tape marks that separate files are several
inches long, and take up as much space as about 2 MByte (4,000 blocks) of
data. Also, on a labelled tape, each file is preceded by a header file (plus
tape mark) and a trailer file (plus tape mark). The upshot of this is that
unless files are more than 30,000 blocks in size then more than 50\% of the
cartridge will be wasted.

This simple test, which wasted 99.993\% of the capacity of the cartridge,
reveals how easy it is to use the Exabyte inefficiently.

It is recommended that the `copying' of files should be avoided unless
files are {\bf much} bigger than 30,000 blocks. {\tt tar} or {\tt dump}
should be utilised instead; these are more efficient as no matter how many
files are being saved, these applications write only a single saveset (plus
header and trailer) on the cartridge. Writing multiple savesets is fine, but
again they should be bigger than 30,000 blocks to use the cartridge really
effectively.

\paragraph {Exabyte Tape Layout}

In both EXB-8500 format and EXB-8200 format, each physical track contains
eight physical blocks. A physical block can contain user data or other
information. In both formats, a physical block containing user data includes
the following information:

\begin {itemize}

\item 2 bytes of Cyclic Redundancy Check (CRC) data.

\item 14 bytes of header information

\item 400 bytes of Error Correction Code (ECC) data.

\item 1,024 bytes of user data

\end {itemize}

{\bf Note} that this information is arranged somewhat differently between
the two formats.

Since each physical track contains eight physical blocks, each track can
contain a maximum of 8,192 bytes of user data. The header, ECC data and CRC
data do not affect the capacity of the tape.

\paragraph {End of Data (EOD)}

When writing tapes in EXB-8500 format, the EXB-8500 writes an End of Data
(EOD) mark after the last data written to tape. This mark is used by the
{\tt space} command to perform a Fast-Search to the last data written to tape.
The EOD mark is overwritten when additional data is written to tape.

\subsection {Fast-Search for Exabyte EXB-8500}

A track written in EXB-8500 format also contains search fields that enable
the EXB-8500 to perform Fast-Search at up to 75 times the nominal tape speed.
Fast-Search occurs when the initiator issues a {\tt locate} or a {\tt space}
command. The search fields are the only areas of the tape that are readable
during a Fast-Search. They consist of small data areas interspersed with clock
synchronisation areas. The search field data contains information for
locating files and blocks.

{\bf Note} that tapes written in EXB-8200 format do not contain search fields
and do not support EXB-8500 Fast-Search.

\subsection {Streaming and Start/Stop Modes for Exabyte EXB-8500}

The EXB-8500 features a 1 MByte data buffer that enables it to operate as
either a Streaming tape device or as a start/stop tape device. The mode of
operation depends on the rate at which data can be transferred between the
initiator and the EXB-8500. If the initiator can sustain a transfer of 0.5
MByte/sec, the EXB-8500 operates in Streaming mode. If the initiator cannot
sustain this transfer rate, starting and stopping of the tape occur
automatically.

Streaming and Start/Stop Modes are described in more detail in Appendix B.

\subsection {Tape Loading for Exabyte EXB-8500}

To load a data cartridge into the EXB-8500, follow these steps:

\begin {enumerate}

\item Ensure that the write-protect switch on the data cartridge has been
set correctly for the desired operation.

\item If the EXB-8500 has just been powered-on, be sure that the green LED
on the front panel is off, indicating that the EXB-8500 is ready to load the
data cartridge.

\item If necessary, press the unload button to open the door on the EXB-8500.

\item Insert the data cartridge into the EXB-8500 with the label-side up and
the write-protect switch facing you.

\item Gently close the door. The EXB-8500 automatically loads the data
cartridge and presents ready status (green LED on).

\end {enumerate}

{\bf Note} that, if autoload has been disabled with a {\tt mode-select}
command, the EXB-8500 will not go to the ready state until a {\tt load} command
has been executed.

\paragraph {Load Time}

The time required to load the data cartridge and position the tape to LBOT
(Logical Beginning of Tape) after the door is closed is approximately 30
seconds for a rewound cartridge.  When loading a tape, the EXB-8500 spaces
forward, from PBOT (Physical Beginning of Tape) and determines the tape
format (blank, EXB-8500 format or EXB-8200 format).

\subsection {Error Handling for Exabyte}

The EXB-8200 format and EXB-8500 format both feature ECC (Error Correction
Code), CRC (Cyclic Redundancy Check) and RAW (Read-After-Write)
error-correction techniques.

\subsection {Data Compression for Exabyte}

Data compression, when available, uses the IDRC (Improved Data Recording
Capability) compression algorithm.

\subsection {Exabyte Cartridge Handling Subsystems}

Cartridge handling subsystems are available for Exabyte, with models that
can handle in excess of 100 cartridges. The comments on access-mode (i.e.
serial versus sequential) given for DAT (see Section 6.6) apply also to Exabyte.

Tape array subsystems, with their multiple-drive configuration are another
option engineered for Exabyte, which have a number of advantages over
single-drive systems. In common with stackers, `automatic drive switching'
enables continuous writing on several cartridges without requirement for
operator intervention. Tape arrays generally have a higher storage capacity.

These systems also allow `tape mirroring', i.e. writing data to several
different cartridges at the same time. Tape mirroring is an advantage when
critical data must be stored in several locations, when writing on at least
two storage media is required, for example.

`Striping' may be offered as a special feature. This allows data to be
distributed evenly across several cartridges during writing, effectively
increasing the data transfer rate by up to 1 MByte/sec for each additional
drive utilised. Striping is particularly advantageous when very large data
sets are required to be backed-up in a short space of time. However, in
order to make use of this feature, the host system must also support the
appropriate high transfer rate.

\subsection {Exabyte Form Factor}

Exabyte drives are packaged with a 5.25 inch Half Height form factor. This
renders them slightly bulkier than the DAT drives, which have a standard
form factor of 3.5 inch.

\subsection {Exabyte Performance}

Although Exabyte tends to be slower than DAT for `fast file access', there
is little to choose between the two when comparing the data transfer rates
for the higher-performance models.

Manufacturers quote typical file transfer rates, for Exabyte systems, of 0.5
MByte/sec and 1 MByte/sec with uncompressed data and compressed data
respectively.

But please {\bf note}:
\begin {itemize}

\item[{\LARGE\bf $\star$}] The Starlink Project view all Manufacturers' capacity
and throughput performance figures as being only ``Guaranteed not to exceed''.

\end {itemize}
\subsection {Applications for Exabyte}

Any applications for which DAT is considered ideal (see Section 6.9) are
equally suitable for Exabyte. However, DAT has certain recognised advantages
over Exabyte, generally having the edge on speed and reliability. Dat is
therefore the {\bf preferred} solution.

\section {DLT SUBSYSTEMS}

DLT (Digital Linear Tape) is the development of the TK-series of DEC half-inch
cartridge drives. This development was initially a joint venture between
Digital and Cipher.

\subsection {DLT Technology}

DLT technology uses essentially the same recording procedure as the QIC
(Quarter Inch Cartridge) technology. As the data cartridge is inserted into the
drive, a non-spinning head lowers onto the tape. Whilst this action alone would
not allow a very high recording density to be achieved, the head is stepped
vertically up the tape a small way when the end-of-tape is reached, allowing
further data to be recorded back along the tape. This process is repeated,
allowing data to be recorded as a series of thin linear strips.

Utilising stationary magnetic heads and an uncomplicated tape transport
mechanism, the tension placed on heads and tapes is less than that for the
comparative technologies for DAT (Digital Audio Tape, see Section 6) and
Exabyte (see Section 7).

TTi (Transitional Technology inc) who are among the market leaders for DLT
subsystems, claim an unequalled hard error rate specification for their
Series 2200 drives of 1 error in 10 to the 17th bits - supposedly the best
data integrity specification in a 5.25 inch format tape subsystem. This
impressive statistic is largely attributed to a custom Reed Solomon ECC (Error
Correction Code), a custom 64-bit CRC (Cyclic Redundancy Check) on each 4 KByte
of data on the media, end-to-end 16-bit CRC on each record overlapped with
parity from the SCSI (Small Computer Systems Interface) bus, and internal
checking on the cache buffer.

The comparative error rate specification for DAT subsystems is typically
1 error in 10 to the 15th bits.

\subsection {DLT Recording Formats}

In common with QIC, DLT uses a linear serpentine track format, which is the
result of the tape and incremental head movements described in Section 8.1.

This recording format does have one particular drawback in that this
serpentine track must also be traversed when reading-back the data.

DLT records on half-inch tapes in compact cartridges. Compared to all other
currently available technologies, DLT drives offer the greatest storage
capacities. At the time of writing, DLT drives exist that can supposedly
record up to 20 GByte native on an 1800ft tape, or up to 40 GByte with data
compression.

\subsection {Data Compression for DLT}

By way of example, TTi's Series 2200 drives offer DLZ (Digital Lempel-Ziv)
high-efficiency data compression.

\subsection {DLT Cartridge Handling Subsystems}

Autoloaders are currently available that handle five or seven cartridges.
The latter can store up to 140 GByte native using 20 GByte cartridges.

TTi provide a dual-drive subsystem for Series 2200 that permits Cascade Mode
selection, whereby data will be automatically written to the second drive
when the first tape is full; Mirroring Mode is similarly accommodated, in
which data may be written to both drives simultaneously.

\subsection {DLT Form Factor}

All currently known DLT drives are produced with a 5.25 inch Full Height
form factor.

\subsection {DLT Performance}

The manufacturers' quoted transfer rates for all DLT subsystems covered in
Section 5 exceed those for the DAT and Exabyte subsystems.

By way of example, TTi's Series 2200 single drive subsystems can supposedly
store up to 20 GByte native and up to 40 GByte compressed, with quoted data
transfer rates of up to 1.5 MByte/sec native and up to 3 Mbyte/sec
compressed. A dual-channel read/write head and a tape mark directory
maximise data throughput and minimise data access time.

But please {\bf note}:
\begin {itemize}

\item[{\LARGE\bf $\star$}] The Starlink Project view all Manufacturers' capacity
and throughput performance figures as being only ``Guaranteed not to exceed''.

\end {itemize}

\paragraph {Starlink Project evaluation of the CTS-2110 DLT Subsystem}

TTi's CTS-2110 variant of the DLT range (tested as the TTi2000) has been
evaluated by the Starlink Project. The supplied technical literature claims
a capacity of 10 GByte native and 20 GByte compressed, with quoted
respective maximum throughput of 1.25 MByte/sec and 2.5 MByte/sec.

In the Project's evaluation of the CTS-2110 DLT Subsystem, a maximum of
around 14 GByte data storage with data compression was achieved, when
performing multiple dumps of a {\tt /usr} partition from a DEC Alpha
3000/300. This underlines the point that the extent to which data compresses
depends very much on the nature of the data being compressed.

As part of the same evaluation, on the DEC Alpha 3000/300, the maximum
throughput achieved, for compressed data, was around 0.7 MByte/sec. A
congested SCSI bus was suspected, and so the CTS-2110 was moved to the
external SCSI on a DEC Alpha 3000/400. The internal SCSI had an {\bf rz26}
and an {\bf rz28} on it and the only device on the external SCSI was the
CTS-2110. A maximum throughput of 1.5 MByte/sec was observed when using
{\tt dump/dd/tar/cpio} in compressed mode (still a good way off the 2.5
MByte/sec mark!).

The OSF tape exerciser utility, {\tt tapex}, was then used to test the
throughput of the DLT drive; 2.8 MByte/sec compressed and around 1.25
MByte/sec native were achieved. {\tt tapex} appears to write directly to the
drive.

The limiting factor is the interaction between the filesystem/device and
the drive. If data can be supplied to the drive sufficiently quickly, then
one has a chance of achieving the advertised throughput. However, any real
application ({\tt dump, tar}, etc.) is limited by the filesystem.

\subsection {Applications for DLT}

DLT drives have a data transfer rate of up to three times that of DAT or
Exabyte drives, and up to five times the data storage capacity. Taking these
figures into consideration, along with design features that should render
them effectively more reliable than their DAT and Exabyte counterparts, DLT
drives can be {\bf highly recommended} for networks requiring unattended
backups of very large data sets.

\clearpage

\appendix

\section {Technical Information Relating to DAT}

This appendix is intended to supplement the technical information included
earlier for DAT devices.

\subsection {The DDS Recording Format}

In addition to the error-correction and data integrity features provided by
DAT, the DDS format has the following features:

\begin {itemize}

\item A Fast-Search capability at up to 200 times the nominal read/write
speed.

\item The option of formatting the tape into a One-Partition or Two-Partition
structure.

\item A third level of error-correction (C3 ECC) which can recover errors
that are are too severe for the basic DAT format techniques (C1 and C2 ECC)
to correct.

\item A Read-After-Write (RAW) facility which checks the data for errors
immediately after it is written, and rewrites it if necessary.

\item The option of N-Group Writing (Multiple Group Writing), where each
group of data is repeated a specified number of times before the next group
is written.

\end {itemize}

The DDS format is structured to overlay the basic audio format. It does this
by organising the frames (pair of tracks) of the audio format into a
sequence of data groups on the tape, each group with a fixed data capacity.
Each group consists of 22 frames-worth of data, where a frame is a pair of
tracks at a skew angle across the tape.

As with the audio format, 60\% of the length of a track is user data. The
rest consists of:

\begin {enumerate}

\item Margin Areas, which are used as guard bands.

\item Sub-Code areas, which are used to facilitate Fast-Search at up to 200
times the nominal read/write speed. This enables the drive to access any
file on the tape rapidly, within 30 seconds on average. The timing of
Fast-Search will depend on where the desired block is located on tape.
When the drive is at `Beginning of Partition', the search time for blocks
near the beginning of tape which require Fast-Search will be approximately
7 seconds. For blocks near the end of the tape, the search time will be
approximately 50 seconds.

\item ATF (Automatic Track Finding) areas, which are used to centre the head
on the track.

\end {enumerate}

A host computer sends data and separator marks to a tape drive which
supports the DDS format. The separator marks identify where logical
collections of data (files and sets of files, for example) begin and end.
The tape drive organises the information into groups and writes it to tape.
An index in each group identifies and locates the data blocks and separator
marks contained in the group, and each group can be followed by an optional
error-correction frame.

The method of indexing allows for fixed and variable length blocks and for
separator marks to be encoded onto the tape without significantly affecting
the amount of data that can be stored on a particular cartridge.

\subsection {DAT Tape Layout}

The DDS format supports the formatting of tapes into a One-Partition or a
Two-Partition structure. The host decides whether a blank tape is to be
formatted into one or two partitions before writing any data. If no format
command is sent by the host before writing to a blank tape, the tape will
default to a One-Partition structure.

If two partitions are created, then the partition closest to the end of the
tape is known as Partition 0. The size of Partition 1, in Megabytes, is
determined by the host during formatting.

Creating two partitions allows a single tape to contain two separate areas,
which can be written to and read independently. The size of each partition
can be defined to suit application needs, and redefined later if necessary.
In each of the two partitions, new data may be appended to existing data at
any time. However, when data is written to a partition, any existing data
which is beyond the current writing point in that partition will be lost.

A possible use of the Two-Partition structure is for Partition 1 to contain
a directory of the data files held in Partition 0. The directory could then
be accessed very rapidly by the drive to get information about the location
of data files. Such a directory could be written to Partition 1 after the
files have been written to Partition 0.

Note that in all circumstances, write operations can only be performed if
the tape is write-enabled. A write-protect tab is located along one corner
edge of the tape cartridge; if the red tab is not visible, then the tape is
write-enabled.

\subsubsection {One-Partition Tape}

The overall tape layout consists of six areas:

\begin {enumerate}

\item Beginning of Medium

\item Device Area.

\item Reference and System Area.

\item Data Area.

\item End of Data (EOD) Area.

\item Early Warning and End of Partition/Medium

\end {enumerate}

\paragraph {Beginning of Medium}

Beginning of Medium (BOM) is the point where the magnetic tape is physically
joined to a transparent leader tape.

\paragraph {Device Area}

The Device area has three sections:

\begin {latexonly}

\begin {enumerate}
\renewcommand{\theenumi}{\alph{enumi}}

\item The load section, which is the part of the tape
that is wrapped around the drum when the tape is first loaded.

\item A test section, where read and write tests of the
drive's electronics and servo are performed.

\item A guard section, which provides a safety zone
between the test section and the start of recorded frames.

\end {enumerate}

\end {latexonly}

\begin{htmlonly}
\begin{rawhtml}
<UL>
<LI> The load section, which is the part of the tape that is wrapped around
the drum when the tape is first loaded.
<LI> A test section, where read and write tests of the drive's electronics
and servo are performed.
<LI> A guard section, which provides a safety zone between the test section
and the start of recorded frames.
</UL>
\end{rawhtml}
\end{htmlonly}

\paragraph {Reference and System Area}

The Reference area defines the Beginning of Partition (BOP), and facilitates
efficient positioning when updating the System area. The tape logs of usage
and soft error occurrences are stored in the System area. This logged
information is lost whenever a tape is (re-)formatted.

\paragraph {Data Area}

The Data area is written as a sequence of groups, starting with a special
Vendor Group which is written automatically by the drive. The Vendor Group
holds details of the drive which created the partition, and the date. It is
followed by data groups, which have a fixed capacity, and are used to store
blocks, filemarks and setmarks written by the host. Fixed-length or
variable-length data blocks may be written. These, together with tape marks,
are mapped into the fixed-capacity groups by a method known as Indexing,
which uses a minimal amount of the data capacity of the tape.

A block may be written within a single group or span several groups,
depending upon its size. The mapping process is invisible to the host system.

DDS provides for two types of separator marks - filemarks and setmarks -
which are each represented by 4 bytes in the index of a group. It is the
responsibility of the application developer to define the logical
significance of these marks.

The meaning of filemarks is defined by the host.

Setmarks provide an additional method of data segmentation. This new type
of mark gives the host the freedom to include any number of blocks and
filemarks within a set, and the ability to search to the end of it in one
motion. Searching to setmarks automatically invokes the drive's Fast-Search
facility, unless the setmark is in the buffer. The host does not need to
know the number of blocks and filemarks contained in a set in order to
position past it, before appending more data.

The use of the setmark is not restricted to marking the end of data. The
host is free to assign any meaning to this mark that is desired.

\paragraph {End of Data (EOD) Area}

The End of Data (EOD) area specifies the point on the tape where the host
stopped writing the data. The host does not specifically command the writing
of this section of the tape. It is the drive's responsibility to detect
conditions which indicate that the host has stopped writing data, and generate
the EOD area at this point.

Marking the end of the recorded data serves two purposes:

\begin {itemize}

\item It limits Fast-Search to the area of the tape containing recorded
data, and no time is wasted searching unrecorded areas.

\item It shows which data is valid. When data is written to the tape it may
either be appended to the tape's existing contents, starting at the current
EOD, or it may be written over existing data. If no overwriting is done, the
Data area grows and the EOD area is written nearer the end of the tape. If
existing data is overwritten, and there is less new data than was originally
on the tape, the EOD area is written closer to the beginning of the tape. In
this case, any recorded data beyond the new EOD area is no longer valid.

\end {itemize}

\paragraph {Early Warning and End of Partition/Medium}

At the end of the tape are the Early Warning (EW) and End of
Partition/Medium (EOP/M) points.

EW is a fixed distance (205 frames - at least 500mm) from EOP/M, and is
generated automatically by the drive. When the drive detects the EW point while
writing, it indicates to the host that it should stop writing data to the tape.

End of Medium (EOM) is the point where the magnetic tape is physically joined
to a transparent trailer tape.

\subsubsection {Two-Partition Tape}

A Two-Partition tape has a single Device area, the same as for a
One-Partition structure. Each partition has its own Reference and System
Area, Data area and EOD area. In addition, an EW mark and EOP are defined for
Partition 1 and BOP is defined for Partition 0.

Each partition has its own System area, as it is likely that the two
partitions will experience different degrees and types of use, and so show
different error occurrences; a single log might mask significant differences
between the two partitions.

For both partitions, the System area is preceded by a Reference area.

Separate Vendor Groups are written automatically by the drive(s) employed to
write each partition. They provide information about the drive, the time the
partition was initialised, and the type of interface.

\subsection {Tape Loading}

The tape loading algorithm varies depending on whether the tape is blank,
formatted as a single partition, or formatted into two partitions.

\subsubsection {Loading Algorithm for Blank or One-Partition Tapes}

For tapes that are blank or formatted as One-Partition, the following
algorithm generally applies:

\begin {enumerate}

\item A data cassette is inserted in the front slot.

\item The drive detects the insertion of the data cassette and draws it
fully in.

\item The tape is threaded onto the drum.

\item The tape is rewound to BOM.

\item The drive searches for BOP and checks the tape to see if there is a
Reference area, whether the tape is DDS format, and if the tape has one or
two partitions.

\item The drive's internal data area is updated to reflect the current
status. The information is then mapped to the appropriate interface status.

\item If the tape is DDS format, the System area logs are read from the tape
into the drive's internal log area.

\item The drive positions the tape at BOM and issues a Normal report to the
interface.

\end {enumerate}

{\bf Note} that if a tape is already loaded, the drive will perform steps
6 through 8.

\subsubsection {Loading Algorithm for Two-Partition Tapes}

\begin{htmlonly}
\begin{rawhtml}
<P>
For tapes that formatted as Two-Partition, the following algorithm generally
applies:
<OL>
<LI> A data cassette is inserted in the front slot.
<LI> The drive detects the insertion of the data cassette and draws it
fully in.
<LI> The tape is threaded onto the drum.
<LI> The tape is rewound to BOM.
<LI> The drive searches for BOP and checks the tape to see if there is a
Reference area, whether the tape is DDS format, and if the tape has one or
two partitions.
<LI> The drive searches for the second Reference Area and checks the
validity of that area.
<LI> The System area logs of the second partition (Partition 0) are read
into the drive's internal log area.
<LI> The drive rewinds the tape to BOM, and the System area logs of
Partition 1 are read into the drive's internal log area.
<LI> The drive's internal data area is updated to reflect the current
status. The information is then mapped to the appropriate interface status.
<LI> The drive positions the tape at BOM and issues a Normal report to the
interface.
</OL>
\end{rawhtml}
\end{htmlonly}

\begin {latexonly}

For steps 1 through 5, the drive performs the same operations as those
described above.

\begin {enumerate}

\setcounter {enumi}{5}

\item The drive searches for the second Reference Area and checks the
validity of that area.

\item The System area logs of the second partition (Partition 0) are read
into the drive's internal log area.

\item The drive rewinds the tape to BOM, and the System area logs of
Partition 1 are read into the drive's internal log area.

\item The drive's internal data area is updated to reflect the current
status. The information is then mapped to the appropriate interface status.

\item The drive positions the tape at BOM and issues a Normal report to the
interface.

\end {enumerate}

\end {latexonly}

The conclusions are as follows:

\begin {itemize}

\item Loading a Two-Partition tape can take significantly longer than loading
a blank or One-Partition tape.

\item When a Two-Partition tape is reformatted into one partition, the
loading time is reduced accordingly (to typically 25 seconds).

\end {itemize}

\subsection {Performing Fast-Search}

The drive uses information encoded in the Sub-Code areas of each track to
perform Fast-Search at up to 200 times the nominal read/write speed.

The information in the Sub-Code areas includes a group count, setmark count,
filemark count and block count. Fast-Search is carried-out in response to a
{\tt space} command from the host to find a specified setmark, filemark or
block.

When performing Fast-Search, the tape head follows a different path to that
in the normal read mode, because both the head rotation rate and tape speed
are changed.

To allow the drive's electronics to lock on and acquire the positional
information required for Fast-Search, the information in the Sub-Code areas
is repeated for each frame in a group.

Once the largest group has been found, the drive uses the information in the
group index to find out where specific setmarks, filemarks or blocks are
located within the group.

Fast-Search is invoked by an internal algorithm making decisions about what
is in the buffer, and depends on whether the command is a {\tt space} forward
or reverse of a number of blocks, filemarks or setmarks.

\subsection {Error Handling}

Audio DAT has two levels of Error Correction Code (ECC) - C1 and C2. DDS
uses the same levels as the audio DAT format but adds extra error-correction
techniques:

\begin {enumerate}

\item C3 ECC.

\item Read-After-Write (RAW).

\item N-Group Writing.

\item Data Randomiser.

\item Checksums.

\end {enumerate}

\subsubsection {Audio Format C1 and C2 ECC}

C1 ECC is (32,28) Reed Solomon code and can detect and correct errors in any
two symbols, or it can correct four symbols where the error location is
known. A symbol is basically equivalent to one byte. C2 ECC is (32,26) Reed
Solomon code and can correct errors up to three symbols long, or six symbols
when the error location is known.

To minimise the probability of faulty detection of an error, the C2 code
decoder typically corrects two symbols, and six symbols where the error
location is known, using C1 error condition flags.

C1/C2 code can correct up to 0.3mm width horizontal error pattern. If the
error exceeds this limit, all the tracks become uncorrectable, and so no
type of C3 ECC can help.

C1/C2 code can correct up to 2.6mm width vertical error pattern. Outside
this limit, about 190 tracks become uncorrectable simultaneously, assuming
circular defects. Correcting an error like this using a track-based
correction would require a C3 capable of correcting more than 190 tracks.

\subsubsection {C3 ECC}

C3 ECC is (46,44) Reed Solomon code, and introduces an extra level of
error-correction in addition to those offered by the audio DAT format. This
feature has the ability to correct any two tracks which are bad in a group.
The error-correction bits are stored in an additional (ECC) frame at the end
of each group.

C3 is well suited to correct helical errors (soft errors caused by `head
clog'). One way of removing the error is through `retry', but this affects
the data transfer rate. The chance of a head clog occurring depends very
much on the condition in which the drive is maintained.

\subsubsection {Read-After-Write (RAW)}

The DDS format supports a Read-After-Write technique to ensure that any data
written to the tape can be read back without error. Each frame is examined
after it has been written to check that it has been recorded correctly.

When a frame is identified as bad, it is rewritten later down the tape
after two other dummy frames have been written. Each frame, and the two in
sequence after it, can be rewritten multiple times, thus having the effect
of skipping over bad areas on the tape. The number of instances of a
repeated sequence is set at a default value in the drive's firmware. When
this number is exceeded, the writing of that group is aborted and the drive
reports a hard error.

Also, in order to limit the physical group size, the maximum number of RAW
frame rewrites that are allowed is set in the drive's firmware.

When reading a group, any rewritten frames need to be identified. If a
frame has been rewritten, there will be more than one frame with the same
Logical Frame Number. To recover the data in the group, all that is required
is to make sure that at least one of each logical frame is read in the
correct order. Any duplicates are ignored. If any frame is unreadable, the
drive reads ahead up to six frames to see if the frame has been rewritten.
If it has, the rewritten frame is used and the drive continues reading from
that point.

To perform Read-After-Write, two additional heads are located on the drum,
making a total of four at 90 degrees to each other.

\subsubsection {N-Group Writing}

N-Group Writing is a technique that writes every group more than once. Each
group is repeated on the tape a number of times before the next group is
written. Each set of rewritten groups is contiguous and contains instances
of only one group.

The data and indexes of the rewritten groups are identical.

The maximum number of times a group can be repeated is seven; that is up to
eight instances may be written.

\subsubsection {Data Randomiser}

The error rates between random data and worst-pattern data differ by a
factor of 10. By using a data randomiser, it is possible to reduce the
worst-case error rate. The randomiser also produces data with a spread of
transitions which have a more consistent RF envelope, allowing accurate RF
level detection as a Read-After-Write criterion.

\subsubsection {Checksums}

If a head clog occurs during writing, there is a chance that the previously
recorded data is not overwritten but remains intact. This is called
`drop-in'. It is important to check for this type of error occurring during
both writes and reads, because it will not be detected by the track-based C1
and C2 codes.

The structure of the C3 ECC is such that it will correct any two tracks if
the location of the erroneous tracks is known, but only one track if the
track location is unknown. By recording the track checksum in four subcode
areas, the location of a drop-in is sure to be detected using the track
checksum. This allows a full two tracks-worth of ECC to be provided with the
C3 algorithm.

\pagebreak

\section {Technical Information Relating to Exabyte}

This appendix is intended to supplement the technical information included
earlier for Exabyte devices.

\subsection {EXB-8500 and EXB-8200 Formats}

\paragraph {Alternate-Azimuth Recording}

When reading data written in EXB-8500 format, the EXB-8500 uses its two read
heads (R1 and R2) to read the two partially-overlapping physical tracks. The
single servo head (SVO) reads the servo data that was written on the tape by
the second write head (W2).

Of the two tracks in the pair, track 1 has a +20 degree azimuth. This track
is written by head W1 and read by head R1. Track 2 has a -10 degree azimuth.
This track is written by head W2 and read by heads R2 and SVO.

\subsection {Streaming and Start/Stop Write Operations}

The EXB-8500 features a 1 MByte data buffer that enables it to operate as
either a Streaming tape device or as a start/stop tape device. The mode of
operation depends on the rate at which data can be transferred between the
initiator and the EXB-8500. If the initiator can sustain a transfer of 0.5
MByte/sec, the EXB-8500 operates in Streaming mode. If the initiator cannot
sustain this transfer rate, starting and stopping of the tape occur
automatically.

In a start/stop write operation for the EXB-8500, the initiator-to-buffer
transfer speed is slower than the buffer-to-tape transfer speed (that is,
data transfers from the initiator occur at less than 500 KBytes/sec). In
this mode of operation, the motion threshold value represents the minimum
amount of data (in 4 KByte increments) that must be in the EXB-8500's
1 MByte buffer before tape motion will start and data will be written to
tape.

When the motion threshold value is exceeded, tape motion starts. The
write-to-tape operation continues until the buffer is empty and the tape
motion stops. Tape motion does not restart until the amount of data in the
buffer once again exceeds the motion threshold value or until the buffer is
flushed for some other reason (such as a {\tt reverse-tape-motion} command).

When the reconnect threshold value is exceeded, the EXB-8500 reconnects to
the initiator and data transfer resumes. The data transfer to the initiator
continues until the buffer is empty. Then, the EXB-8500 disconnects from the
initiator but continues to transfer data from the tape to the buffer.

\subsubsection {Motion Threshold}

In start/stop mode, the `motion threshold' can be used to fine-tune the
starting and stopping of tape motion by controlling data transfers between
the buffer and the tape.

The motion threshold is measured in 4 KByte increments. The default value
for motion threshold is 512 KBytes; this value can be changed with a
{\tt mode-select} command.

\subsubsection {Reconnect Threshold}

In Streaming mode, the `reconnect threshold' can be used to fine-tune the
rate of disconnects and reconnects between the EXB-8500 and the initiator.

The reconnect threshold is measured in 4 KByte increments. The default
value for reconnect threshold is 512 KBytes; this value can be changed with
a {\tt mode-select} command.

\subsubsection {Guidelines for Motion and Reconnect Thresholds}

The motion threshold and the reconnect threshold can be used to fine-tune
the rate of data transfer as follows:

\begin {itemize}

\item If the initiator transfer rate is greater than 500 KBytes/sec, lower
the motion threshold and raise the reconnect threshold an equal amount.

\item If the initiator transfer rate is less than 500 KBytes/sec, raise the
motion threshold and lower the reconnect threshold an equal amount.

\end {itemize}

\end{document}
