\documentstyle[11pt]{article} 
\pagestyle{myheadings}

%------------------------------------------------------------------------------
\newcommand{\stardoccategory}  {Starlink System Note}
\newcommand{\stardocinitials}  {SSN}
\newcommand{\stardocnumber}    {45.17}
\newcommand{\stardocauthors}   {A J Chipperfield}
\newcommand{\stardocdate}      {1 Feb 1994}
\newcommand{\stardoctitle}     {ADAM --- Release 2.1}
%------------------------------------------------------------------------------

\newcommand{\stardocname}{\stardocinitials /\stardocnumber}
\markright{\stardocname}
\setlength{\textwidth}{160mm}
\setlength{\textheight}{230mm}
\setlength{\topmargin}{-2mm}
\setlength{\oddsidemargin}{0mm}
\setlength{\evensidemargin}{0mm}
\setlength{\parindent}{0mm}
\setlength{\parskip}{\medskipamount}
\setlength{\unitlength}{1mm}

%------------------------------------------------------------------------------
% Add any \newcommand or \newenvironment commands here
%------------------------------------------------------------------------------

\font\tt=cmtt10 scaled 1095
\renewcommand{\_}{{\tt\char'137}}

\begin{document}
\thispagestyle{empty}
SCIENCE \& ENGINEERING RESEARCH COUNCIL \hfill \stardocname\\
RUTHERFORD APPLETON LABORATORY\\
{\large\bf Starlink Project\\}
{\large\bf \stardoccategory\ \stardocnumber}
\begin{flushright}
\stardocauthors\\
\stardocdate
\end{flushright}
\vspace{-4mm}
\rule{\textwidth}{0.5mm}
\vspace{5mm}
\begin{center}
{\Large\bf \stardoctitle}
\end{center}
\vspace{20mm}
\begin{center}
{\Large\bf Summary}
\end{center}
This is a `complete' release of ADAM -- its main features are:
\begin{itemize}
\item Enhanced Parameter System.
\begin{itemize}
\item A MIN/MAX system for parameters.
\item Ability to set the null parameter (!) value on the command line.
\item Command line syntax to set individual parameters into the ACCEPT state.
\item PAR as a separate Starlink Software Item.
\end{itemize}
\item A new structure available for monoliths.
\item Revised shareable image organization.
\item Bugs fixed in the parameter system and ICL.
\end{itemize}

\newpage
%------------------------------------------------------------------------------
%  Add this part if you want a table of contents
  \setlength{\parskip}{0mm}
  \tableofcontents
  \setlength{\parskip}{\medskipamount}
  \markright{\stardocname}
%------------------------------------------------------------------------------

\section{INSTALLATION}
Full installation instructions are given in SSN/44 and the Starlink Software 
Change Notice.

{\em Note that there has been a change in the significance of the version 
numbering system. The absence of a hyphen no longer implies a complete 
release -- it implies functional changes. 
The presence of a hyphen implies a release consisting only of bug fixes or 
other minor changes.}

The full system requires about 37000 blocks of disk storage and includes a
mini-system which can be extracted and put up separately. The mini-system
requires about 1200 blocks and allows tasks to be run and  developed.


\section{NEW FEATURES OF THE PARAMETER SYSTEM}

\subsection{Augmented PAR Library}
PAR will in future be a separate Starlink Software Item described in 
SUN/114.
In preparation for this, ADAM tasks will link with a separate 
PAR\_IMAGE\_ADAM shareable library.

In addition to new routines supporting the MIN/MAX system described below,
a number of other routines, mainly based on the undocumented AIF library
which has been in use by KAPPA and other applications for some time, will
be included in the new PAR library.

As a temporary measure, A version of the PAR library (including a
PAR\_IMAGE\_ADAM.EXE) supporting the MIN/MAX system but not the other PAR 
enhancements is retained in this release --
it will be removed at the next release. 

See also APPLOG.COM, DIR.COM and LOGICAL.COM changes described in 
Section \ref{procs}.

\subsection{The MIN/MAX System}
The parameter system now has a MIN/MAX system which allows programs to define
dynamic minimum and/or maximum values for parameters. MIN or MAX may be
specified as a parameter `value' in which case the appropriate value will be
used.

Tasks will use the system via the PAR interface -- for details of the
new routines, see SUN/114.

The following points relate to the way the underlying parameter 
system, SUBPAR, implements the system:
\begin{itemize}
\item Minimum/maximum values are set by the routines
\begin{quote} \begin{verbatim}
   SUBPAR_MINt( NAMECODE, VALUE, STATUS ) 
or SUBPAR_MAXt( NAMECODE, VALUE, STATUS ) 
\end{verbatim} \end{quote}
where: \verb%t% is R, D, I or C,\\
\hspace*{12mm} \verb%NAMECODE% is the index to the parameter,\\
and \hspace{4mm} \verb%VALUE% has the type corresponding with \verb%t%.

The PAR library contains corresponding PAR routines which use the 
parameter name rather than the namecode.
\item MIN and MAX values must be convertible to the type of the parameter and
must be within any RANGE set by the interface file.
\item The SUBPAR\_GET0x routines will check any obtained parameter values 
against limits defined by any specified MIN/MAX values and/or any specified 
lower or upper RANGE values. 
The ASCII collating sequence is used in comparing CHARACTER values.
\item If a minimum value is greater than the maximum value, the parameter must
NOT lie between them.
\item MIN or MAX (case independent) may be given as the parameter value on the
command line (either positional or keyword=), in response to a prompt, or in
a `SEND task SET parameter value' command.
If a dynamic MIN/MAX value has been set when the parameter value is  `got' it
will be used. 
If there is no dynamic MIN/MAX value, any specified RANGE values will be used 
instead. 
If no minimum/maximum value has been specified, an error is reported and a 
prompt issued for another value.
\item MIN and MAX values will be cancelled at the end of each task invocation.
If required, they may also be cancelled explicitly during a task invocation 
by calling a new subroutine, SUBPAR\_UNSET, which can also unset dynamic 
defaults. Again there is a PAR version of this routine.
\end{itemize}
(New: SUBPAR\_MAXC, \_MAX.GEN, \_MINC, \_MIN.GEN, \_UNSET, \_RANGEC/D/I/R,
\_MNMX)\\
Modified: SUBPAR\_CMDLINE, \_CMDPAR, \_DEACT, \_HDSIN, \_FETCHC/D/I/R, 
\_LIMITC/D/I/R, \_LOADIFC, \_LDIFC0, \_LDIFC1, SUBPAR\_CMN)

\subsection{Parameter States}
Parameter states are not part of the parameter system interface but for 
special purposes the PAR\_STATE routine can access them. 
There have been a number of changes to possible states which may affect any 
tasks using PAR\_STATE.
Additional parameter states are defined to implement the MIN/MAX system and
to properly differentiate between SUBPAR\_\_ACCEPT and SUBPAR\_\_ACCPR.

For more details, see Appendix \ref{states}.\\
(New: SUBPAR\_ACCPT1,\_ACCPT)\\
(Retired: SUBPAR\_ACCPR))\\
(Modified: SUBPAR\_PAR, SUBPAR\_CMDLINE, \_FINDHDS, \_GETNAME, \_HDSIN, \_INPUT,
\_RESET, \_FRPOMPT, \_STATE)

\subsection{Command Line Syntax}
\subsubsection{The ACCEPT Keyword}
The ACCEPT command-line keyword will now only affect parameters 
which would otherwise be prompted for. 
Previously it would completely override the VPATH --
that behaviour is now obtained by combining ACCEPT and PROMPT.

It is now also possible to put an individual parameter into an `accept' state 
by specifying  `\verb%keyword=\%' (where \verb%keyword% is the keyword of the
parameter) on the command line. The presence or absence
of any additional RESET or PROMPT keywords will determine the precise state 
entered.\\
(New: SUBPAR\_ACCPT1,\_ACCPT)\\
(Retired: SUBPAR\_ACCPR))\\
(Modified: SUBPAR\_PAR, SUBPAR\_CMDLINE, \_FINDHDS, \_GETNAME, \_HDSIN, 
\_INPUT, \_RESET, \_FRPOMPT, \_STATE)

\subsubsection{Null (!) on the Command Line}
The null value will now be accepted on the command line (either positional or 
as keyword=!) and and the `value' in an ICL `SEND task SET parameter value' 
command.\\
(SUBPAR\_CMDLINE, \_CMDPAR)

{\em Note that this is not available when running tasks direct from DCL as DCL
uses ! to introduce end of command line comments.}

\subsubsection{Abbreviation of Special Keywords}
The undocumented abbreviations ACC, RES and PR for the special keywords 
ACCEPT, RESET and PROMPT are no longer permitted as they have caused
confusion and no-one seems to use them. A proper scheme for abbreviating 
keywords is in future plans.\\
(SUBPAR\_CMDLINE)
\subsection{Listing Parameters}
A new subroutine, SUBPAR\_INDEX, has been provided to enable a list of 
parameters names for the current A-task to be obtained. 
The routine accepts a namecode identifying one of the parameters and returns 
a namecode identifying the next parameter. A given namecode of 0 will start 
the sequence and namecode 0 will be returned when there are no more 
parameters for the action.
The name of the parameter identified by a namecode may by obtained by calling
subroutine SUBPAR\_PARNAME, which has been updated to report errors.\\
{\em N.B. This is not a public routine and should only be used in special
circumstances.}\\
(New: SUBPAR\_INDEX; Modified: SUBPAR\_PARNAME)

\subsection{Portability and Dependencies}
A number of changes have been made to improve the portability of the system
and to avoid problems due to dependencies between libraries which are
hierarchically incorrect and lead to problems when building the complete
ADAM system from scratch. 
\begin{itemize}
\item An additional include file, SUBPAR\_PARERR, has been created
to define, for SUBPAR internal use only, those PAR errors such as PAR\_\_NULL 
and PAR\_\_ABORT which SUBPAR must set.\\
(New: SUBPAR\_PARERR,  Modified: LOGICAL.COM, numerous subroutines)
\item An additional include file SUBPAR\_SYS, has been created to define some
of the SUBPAR symbolic constants which were previously defined in SUBPAR\_CMN.
This allows them (in particular SUBPAR\_\_NAMELEN) to be used in other,
higher-level packages without including the whole of SUBPAR\_CMN.
SUBPAR\_SYS is included by SUBPAR\_CMN.\\
(New: SUBPAR\_SYS.FOR; Modified: SUBPAR\_CMN.FOR)
\item Several new error codes are defined in SUBPAR\_ERR to replace other
packages codes which were being set by SUBPAR. They are:
\begin{itemize} 
\item SUBPAR\_\_ERROR instead of PAR\_\_ERROR.\\
(SUBPAR\_CURLOC,\_CURNAME,\_CURVAL,\_HDSDEF,\_HDSDYN,\_VALASS)
\item SUBPAR\_\_ICACM instead of PAR\_\_ICACM.\\
(SUBPAR\_ASSOC,\_CREAT,\_EXIST)
\item SUBPAR\_\_NAMIN instead of DAT\_\_NAMIN.\\
(SUBPAR\_\_SPLIT)
\end{itemize}
\item INCLUDE 'DAT\_PAR' has been added to numerous routines as it is not
included by including SAE\_PAR on Unix.
\item CHR\_LEN is used instead of STR\$TRIM.\\
(SUBPAR\_PROMPTCL)
\item Avoid the use of the backslash character in code.\\
(SUBPAR\_CMDLINE,\_INPUT)
\item MESSYS\_PAR is used as an alternative  name for MESDEFNS and MESSYS\_ERR
for MESERRS.\\
(SUBPAR\_\_SYNC,\_PROMPTCL,\_WRITE)
\item The output routine used with the help system has been changed to use
SUBPAR\_WRMSG for consistency with the standard output system. 
This also makes the routine common to VMS and Unix systems.\\
(SUBPAR\_OPUT)
\item Symbolic constants, notably DAT\_\_SZLOC are used in defining the SUBPAR
common blocks.\\
(SUBPAR\_CMN)
\end{itemize}

\subsection{Code Simplification}
Some code which is common to a number of routines has been split off into
separate routines.
\begin{itemize}
\item SUBPAR\_ARRAY Parses an array.\\
(SUBPAR\_CMDLINE,\_CMDPAR,\_HDSIN)
\item SUBPAR\_DIRFX handles the business of replacing \verb%[]% with \verb%<>%
 in directory specifications.\\
(SUBPAR\_CMDLINE,\_CMDPAR,\_HDSIN)
\item SUBPAR\_CTYPE returns the parameter type as a string.\\
(SUBPAR\_CMDLINE,\_MNMX)
\end{itemize}

\subsection{Task Control}
SUBPAR\_ACTDCL has been modified to return the name of the task to
DTASK\_DCLTASK when the task is being run from the shell. 
(DTASK\_DCLTASK is changed accordingly.)
This will allow a new style of monolith to be run from the Unix shell
For more details, see Section \ref{monoliths}.

\subsection{Re-prompting for Parameters}
The SUBPAR\_GET routines will now re-prompt in most error situations. The
exceptions are PAR\_\_NULL, PAR\_\_ABORT and PAR\_\_NOUSR.\\
(SUBPAR\_GETnx)


\section{NEW-STYLE MONOLITHS}
\label{monoliths}
It is now possible to have the same code for a monolith to support running
from ICL (on either VMS or Unix) or from the Unix shell.
The new-style monoliths are linked using ALINK (alink on Unix). MLINK is still
available to link old-style monoliths on VMS but will not be available on Unix.

ICL shareable monoliths must remain in the old form.

The top-level routine will be of the form:
\begin{small}
\begin{verbatim}
          SUBROUTINE TEST( STATUS )
          INCLUDE 'SAE_PAR'
          INCLUDE 'PAR_PAR'

          INTEGER STATUS

          CHARACTER*(PAR__SZNAM) NAME

          IF (STATUS.NE.SAI__OK) RETURN

*       Get the action name
          CALL TASK_GET_NAME( NAME, STATUS )

*       Call the appropriate action routine
          IF (NAME.EQ.'TEST1') THEN
            CALL TEST1(STATUS)
          ELSE IF (NAME.EQ.'TEST2') THEN
            CALL TEST2(STATUS)
          ELSE IF (NAME.EQ.'TEST3') THEN
            CALL TEST3(STATUS)
          END IF
          END
\end{verbatim}
\end{small}

To run such a monolith from a Unix shell, link the required action name to the
monolith, then execute the linkname (possibly via an alias). For example:

\begin{quote} \begin{verbatim}
% ln -s $KAPPA_DIR/kappa add
% add
\end{verbatim} \end{quote}

Separate interface files are required for each action.

\section{REVISED SHAREABLE IMAGE STRUCTURE}
\label{shares}
To enable a proper hierarchy of libraries and to allow for PAR and ERR changes
to be released separately from ADAM, the old ADAMSHARE shareable image has been
split into:
\begin{description}
\item[PAR\_IMAGE\_ADAM] Containing the PAR library. This will normally be part
of a separate Starlink Software Item.
\item[DATPAR\_IMAGE\_ADAM] Containing the DATPAR library.
\item[ERR\_IMAGE\_ADAM] Containing the ERR and MSG routines for ADAM.
This will normally be part of a separate Starlink Software Item.
\item[SUBPAR\_IMAGE\_ADAM] Containing SUBPAR, PARSECON, LEX, STRING and the ICL
terminal handling routines. There are no transfer vectors for the LEX or
PARSECON routines\footnote{With the exception of PARSECON\_FINDACT which is
called by DTASK\_REMOVE\_ACTION}.
\item[ADAM\_IMAGE\_ADAM] Containing ADAM, MESSYS and UTIL.
\end{description}
Procedures and transfer vectors for building the images are kept in 
[ADAM.LIB.SHARE] and the images themselves in [ADAM.RELEASE.EXE].

The shareable image library, ADAMSHRLIB, which is searched by all the ADAM
task linking procedures (ALINK, ILINK {\em etc.}), has been altered to include 
the DATPAR, ADAM, SUBPAR and ADAMGRAPH7 shareable images but not ADAMSHARE. 
ADAMSHARE now contains only dummy transfer vectors which allow tasks already 
linked with it to continue working but will not allow anything to be 
linked with it from now on.
Any shareable images which were linked explicitly with ADAMSHARE/SHARE should
in future be linked with ADAMSHRLIB/LIB or the individual shareable images
required.

The PAR shareable image is now picked up via the STAR\-\_LINK\-\_ADAM link 
options file which is now included in {\em all} ADAM task link procedures.
If a non-standard version of this is used, it must be altered appropriately.

The increased number of shareable images will cause a slight degradation in
load times -- users may wish to think about linking non-shareably.

\section{NEW FEATURES OF OTHER SUBSYSTEMS}

\subsection{DTASK}
\label{dtask}
A number of changes have been made in the interests of allowing, as far as
possible, the same code to be used on both Unix and VMS systems. This requires
stricter rules about package hierarchy.
\begin{itemize}
\item The name of the task, as determined by SUBPAR\_GTCMD,  is now returned 
from SUBPAR\_ACTDCL to DTASK\_DCLTASK.
The name, rather than \verb%'RUN'%, is then passed on to DTASK\_OBEY.
On Unix, this allows new-style monolith to be run from the Unix shell.
(For more details, see Section \ref{monoliths}.)
\item DTASK\_GET now calls SUBPAR\_GET0C instead of PAR\_GET0C
\item Routines too numerous to list have been modified to:
\begin{itemize}
\item Use SUBPAR\_\_NAMELEN instead of PAR\_\_SZNAM.
\item Use MESSYS\_\_VAL\_LEN instead of MSG\_VAL\_LEN.
\item Remove unnecessary INCLUDE statements.
\item Remove unused variable declarations.
\end{itemize}
\end{itemize}
\subsection{TASK}
Routines too numerous to list have been modified to:
\begin{itemize}
\item Use SUBPAR instead of PAR parameters and INCLUDE files.
\item Remove unused variable declarations.
\end{itemize}

\subsection{MESSYS}
\begin{itemize}
\item MESSY\_PAR.FOR now defines MESSYS\_\_VAL\_LEN which should be used in
preference to MSG\_VAL\_LEN where the maximum length of the value field of a
message is required.
\item MESSYS\_PAR.FOR is made available in the ADAM\_INC directory for
application development
\end{itemize}

\subsection{FIO/RIO}
As forewarned at release V2.0-2, the FIO/RIO libraries have now been removed 
from the ADAM release.
For details of how to use FIO/RIO now, see SUN/143.

\subsection{Procedures {\em etc.}}
\label{procs}
\begin{description}
\item[Link Procedures] All link procedures, including ILINK, DLINK and CDLINK,
will now use the STAR\_LINK\_ADAM link options file to pick up the PAR library.

{\em If you are using a non-standard version of STAR\_LINK\_ADAM, you must 
ensure that it will in future pick up the PAR library.}

For more information, see Section \ref{shares} and SUN/44 Section 6.
\item[SYSLOGNAM.COM (SYSLOGNAM.MINI)] The procedures for setting up SYSTEM
logical names {\em etc.} for ADAM at VMS system startup have been changed as
follows:
\begin{itemize}
\item Add SYS\$DISK: to [] in the definition of ADAM\_HELP.
\item Define shareable images SUBPAR\_IMAGE\_ADAM, ADAM\_IMAGE\_ADAM and
DATPAR\_IMAGE\_ADAM
\end{itemize}
{\em Note that if you have modified versions of these files, they should be
updated in line with the new version.}
\item[ADAMSTART.COM] The ADAM startup procedure is changed as follows:
\begin{itemize}
\item The version number is set to `2.1'.
\item A symbol, BETA, is defined and will be added to the ADAM version number 
in displays and in setting the logical name ADAM\$\_VERSION.
This will be used to label Beta releases as Version 2.1\_Beta for example.
\end{itemize}

\item[ADAMSHARE] See Section \ref{shares}
\item[ADAMSHRLIB7] See Section \ref{shares}
\item[APPLOG.COM]
The procedure  for setting up process logical names for application 
development have been changed as follows:
\begin{itemize}
\item MESSYS\_PAR is now defined. 
\item PAR\_DEV is obeyed if it is defined; if not, PAR include files are
picked up from ADAM\_INC.
\end{itemize}
\item[DIR.COM] The procedure for defining ADAM system directories has been 
changed to use a SYSTEM logical name PAR\_DIR if it is defined; if not,
[ADAM.LIB.PAR] is used.
\item[LOGICAL.COM] The procedure for defining logical names required for
ADAM system development has been changed to use PAR\_DEV if it is defined;
if not, PAR include files are picked up from PAR\_DIR.
\end{description}


\subsection{Documentation}
\label{docs}
\begin{itemize}
\item SSN/45 and ARN/2.1 describe ADAM release 2.1.
\item The following applicable Starlink document has been released since
 the last ADAM release:
\begin{itemize}
\item SUN/114 PAR - Interface to the ADAM Parameter System, Programmer's Guide.
\end{itemize}
\item The following applicable Starlink documents have been updated since the
last ADAM release.
\begin{itemize}
\item SSN/44 ADAM -- Installation Guide
\item SUN/115 ADAM -- Interface Module Reference Manual
\end{itemize}
\item The following document from the old ADAM series has been withdrawn:
\begin{itemize}
\item APN/6 -- The PAR Library
\end{itemize}
\item The ADAM\_CHANGES help library has been updated 
with the information contained in this document. The information for earlier
releases is
retained in the library.
\item The summaries, 0CONTENTS.LIS, FULLDOCS.LIS and NEWDOCS.LIS in
ADAM\-\_DOCS, have been updated. 
\end{itemize}

\section{BUGS FIXED}

\subsection{SUBPAR}
The following bugs have been corrected:
\begin{itemize}
\item Suggested values which were arrays of type \_DOUBLE were displayed 
incorrectly.\\
(SUBPAR\_CURVAL, \_VALASS)
\item `=' was not accepted in a string given in response to a prompt.\\
(SUBPAR\_HDSIN
\item Names beginning with a number could get corrupted on Sun systems due to
a peculiarity of Sun Fortran.
For example 2DF became 2EF.
(SUBPAR\_CMDLINE, \_CMDPAR, \_HDSIN, \_ARRAY, LEX\_CMDSET).
\item Suggested values for \_REAL or \_DOUBLE arrays were rounded to integers.\\
(SUBPAR\_CURVAL)
\end{itemize}

\subsection{PARSECON}
A bug which prevented task names in INTERFACE or MONOLITH statements in the
interface file being of the form `D' or `E' possibly followed by
an integer, has been corrected. The mod also fixes a problem on Suns whereby a 
single-letter name could not be specified.\\
(PARSECON\_TOKTYP)

\subsection{UFACE}
A modification to only send the used part of a message buffer, got lost at
ADAM V2.0-3. This has been restored in the ICL, SMS and ADAMCL versions of
UFACE. 
ICL and SMSICL have been re-linked to include the change.\\(UFACE\_SEND)

\subsection{ICL}
ICL V1.6 is released containing the following bug fixes.
\begin{itemize}
\item CHECKTASK will now work correctly with tasks on ADAMNET.\\
(ICLADAM)
\item PUTNBS will now correctly write an ICL variable into a \_REAL NBS
component.\\
(ICLNBS)
\item SYS\$INPUT is now correctly re-assigned after a \$ (DCL) command.
This allows the editor to be run in the DCL subprocess after a
command procedure has been run.\\
(ICLIO.PAS)
\end{itemize}
The following bugs, which were fixed in ADAM V2.0-3, are documented here as
there was no release note for V2.0-3.
\begin{itemize}
\item A bug which caused AST quota decrementing for each command sent to
the DCL subprocess (ICLIO).
\item The code to abort a running DCL subprocess command by typing cntrl/Y
to ICL now works correctly (ICLIO).
\item ICL crash when compiling one type of incorrect syntax (ICLMAIN).
\item Failure of some get and put NBS commands (ICLPARSE).
\item Error messages are now annulled if the error is handled by an ICL
exception handler (ICLPROC, ICLMAIN).
\item The ICL prompt may now be changed in a startup file (ICLMAIN).
\item PRIMDAT error codes returned from tasks are now translated 
(LINK\_ICL.COM).
\item The Portable Help library is now correctly linked (LINK\_ICL.COM).
\end{itemize}

\subsection{ADAMSHARE}
The following bug, which was fixed in ADAM V2.0-4, is documented here as
there was no release note for V2.0-4.

There was a bug in the dummy transfer vector scheme used to direct
references to the ADAM\-SHARE transfer vector through to routines in other
shareable images. Problems arose if the calling sequence of referenced routines
changed.

\section{PROPOSED DEVELOPMENTS}
\subsection{Parameter System}
It is proposed to introduce a scheme allowing abbreviation of keywords given 
on the command line.
\subsection{STRING}
The STRING library is being considered for removal. It seems appropriate that
its general purpose functions should be transferred to portable code in the
CHR library and other more specific functions be incorporated into other
ADAM system packages.

\newpage
\appendix
\section{Parameter States}
\label{states}
The new list of possible states is given below, changes are starred.

\begin{tabular}{cllp{3.5in}}
& SUBPAR\_\_GROUND &= 0  &State after task invocation.\\
& SUBPAR\_\_ACTIVE &= 1  &Parameter has a value.\\
& SUBPAR\_\_CANCEL &= 2  &Parameter value has been cancelled:
do not update associated GLOBAL values,
ignore VPATH - prompt if new value required.\\
& SUBPAR\_\_NULL   &= 3  &Status PAR\_\_NULL will be returned to requesting
routine.\\
& SUBPAR\_\_EOL    &= 4  &Unused.\\
{*} &SUBPAR\_\_RESET  &= 5  &Ignore CURRENT on VPATH or PPATH.\\
{*} &SUBPAR\_\_ACCEPT &= 6  &Take suggested value if a prompt would otherwise 
be issued.\\
& SUBPAR\_\_RESACC &= 7  &Combine RESET and ACCEPT.\\
& SUBPAR\_\_FPROMPT &= 8 &Ignore VPATH and issue a prompt.\\
& SUBPAR\_\_RESPROM &= 9 & Combine RESET and FPROMPT.\\
{*} &SUBPAR\_\_MAX    &= 10 &Use MAX value for this parameter but value not 
yet requested.\\
{*} &SUBPAR\_\_MIN    &= 11  & Use MIN value for this parameter but value
not yet requested.\\
{*} &SUBPAR\_\_ACCPR  &= 12  &Combine FPROMPT and ACCEPT.\\
{*} &SUBPAR\_\_RESACCPR &= 13  &Combine FPROMPT, RESET and ACCEPT.\\
\end{tabular}

Note that the value of SUBPAR\_\_ACCPR has changed.


\subsection{State Classification}
For convenience in documentation {\em etc.} the states may be grouped. 
Note that a given state may be in more than one group.

\begin{description}
\item[The `Ground' States]
SUBPAR\_\_GROUND, SUBPAR\_\_ACCEPT, SUBPAR\_\_RESET, \\
SUBPAR\_\_FPROMPT, SUBPAR\_\_ACCPR, SUBPAR\_\_RESPROM, \\
SUBPAR\_\_RESACCPR

Following task initialisation the system has not been told which value to take
but may have had its VPATH/PPATH search modified.

\item[The `Set' States]
SUBPAR\_\_ACTIVE, SUBPAR\_\_MIN, SUBPAR\_\_MAX, \\
SUBPAR\_\_NULL

The parameter `value' to be used has been specified. In the case of MIN and
MAX, the actual value is obtained when the PAR\_GETx routine is called -- 
the parameter will then move into SUBPAR\_\_ACTIVE state.

\item[The `Accept' States]
SUBPAR\_\_ACCEPT, SUBPAR\_\_ACCPR, SUBPAR\_\_RESACC, \\
SUBPAR\_\_RESACCPR

If a prompt would otherwise occur and there is a suggested value, it will be 
used; if there is no suggested value, the prompt will be issued.

\item[The `Reset' States]
SUBPAR\_\_RESET, SUBPAR\_\_RESACC, SUBPAR\_\_RESPROM, \\
SUBPAR\_\_RESACCPR

CURRENT values will be ignored in VPATH or PPATH searches

\item[The `Prompt' States]
SUBPAR\_\_FPROMPT, SUBPAR\_\_RESPROM, SUBPAR\_\_ACCPR, SUBPAR\_\_RESACCPR

The VPATH is ignored and a prompt issued (or suggested value taken).


\item[The `Cancelled' State]
SUBPAR\_\_CANCEL

This state is entered if pkg\_CANCEL is called for the parameter. 
When a parameter is in the cancelled state:
\begin{itemize}
\item A further request for a parameter value will result in a prompt,
regardless of anything (including NOPROMPT) on the VPATH.
\item If the action ends, any GLOBAL associations will not be written.
\end{itemize}
\end{description}

\subsection{Setting Accept States}
Existing Methods:
\begin{description}
\item[ACCEPT or $\backslash$ on the command line] - May be used alone, or in 
combination with RE\-SET and/or PROMPT, to set all unset parameters into one 
of the ACCEPT states.
\item[$\backslash$ in response to a prompt] - Takes the suggested value for 
this parameter and sets all remaining unset parameters into one of the ACCEPT 
states (depends on whether RESET and/or PROMPT were set on the command line).
\end{description}
New Method:
\begin{description}
\item[keyword=$\backslash$ or keyword=ACCEPT on  the command line] - May be 
used alone, or in combination with RESET and/or PROMPT, to set the specified 
parameter into one of the ACCEPT states.
(SUBPAR\_CMDLINE)
\end{description}

\subsection{Setting the NULL State}
Existing Methods:
\begin{itemize}
\item Typing ! in response to a prompt.
\item Failure to find a value using the VPATH.
\item Using a default of !.
\end{itemize}
New Methods:
\begin{itemize}
\item Specify ! as the parameter `value' on the command line (either positional 
or as keyword=!).\\
{\em Note that this is not available when running tasks direct from DCL as DCL
uses ! to introduce end of command line comments.}
\item Specify ! as the `value' in an ICL `SEND task SET parameter value' command.
\end{itemize}
\end{document}
