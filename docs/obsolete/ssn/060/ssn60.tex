\documentstyle{article} 
\pagestyle{myheadings}

%------------------------------------------------------------------------------
\newcommand{\stardoccategory}  {Starlink System Note}
\newcommand{\stardocinitials}  {SSN}
\newcommand{\stardocnumber}    {60.1}
\newcommand{\stardocauthors}   {D S Berry}
\newcommand{\stardocdate}      {25 November 1988}
\newcommand{\stardoctitle}     {ASPIC --- Release notes for EDRS and EDRSX}
%------------------------------------------------------------------------------

\newcommand{\stardocname}{\stardocinitials /\stardocnumber}
\markright{\stardocname}
\setlength{\textwidth}{160mm}
\setlength{\textheight}{240mm}
\setlength{\topmargin}{-5mm}
\setlength{\oddsidemargin}{0mm}
\setlength{\evensidemargin}{0mm}
\setlength{\parindent}{0mm}
\setlength{\parskip}{\medskipamount}
\setlength{\unitlength}{1mm}

\begin{document}
\thispagestyle{empty}
SCIENCE \& ENGINEERING RESEARCH COUNCIL \hfill \stardocname\\
RUTHERFORD APPLETON LABORATORY\\
{\large\bf Starlink Project\\}
{\large\bf \stardoccategory\ \stardocnumber}
\begin{flushright}
\stardocauthors\\
\stardocdate
\end{flushright}
\vspace{-4mm}
\rule{\textwidth}{0.5mm}
\vspace{5mm}
\begin{center}
{\Large\bf \stardoctitle}
\end{center}
\vspace{5mm}

\section{EDRS}

This section describes an update to EDRS to support GKS graphics and image
handling on Digisolve IKON displays, submitted for release on 12 September 1988.
An update to this was submitted on 26 October; this included an extra program
called IRESET for clearing the ARGS.

The documentation for this update is mainly in the on-line help library
[ASPIC.EDRS]HELPLIB.HLB, but some new information on EDRS has been included
in new versions of SUN/23 and SUN/24.

\subsection{Changes}
The programs in EDRS which produce graphics or handle images have been modified
so that all GKS graphics devices can be used, and so that equivalent image
handling operations can be performed on an IKON.
Their default behaviour is as it was before (with the exception of plotting
style).
If the user wants to use any of the new features, parameter assignments must be
made on the command line.

The following programs have been modified and the corresponding help modules
updated:
ARGPIC, CONT, ITFPLOT, NORMALIZE, PROFILE, STARFIT, TRIM, XYCUR.

The following non-graphics programs have been modified:
CUT, MATHS, XYKEY, XYJOIN.

The following program has been added:
IRESET.

The following routines from EDRSLIB have been modified:
ARGPIC, CONT, CUTIT, IMCONT, ITFPLO, MATHS, NMPLOT, NORMAL, PLTCUT, PROFIL,
RADPRF, SPARAM, STARFI, TABPLT, TRIM, XYCUR, XYINCR, XYINKY, XYJOIN.

The following new routines have been added to EDRSLIB:
AGCHAX, AGCHIL, ARGSCL, ARGSOP, DREBAR, GETDEV, GTFONT, GTINAM, I2TOI4, IKONCL,
IKONOP, IKON\_CLS, IKON\_OVCL, IKON\_OVOP, LENSTR, NCRBCK, NCRCOL, NCROPN,
OVCLOS, OVCROS, OVOPEN, OVPOLY, UPPERC, WRUSER, XYIKON, XYSCRN.

The following files declaring common blocks have been added to EDRSLIB.TLB:
ARGSCOM, GRCOM, NCCOM.

\subsection{General points}
\begin{description}

\item [AGI]:

Programs ARGPIC, XYCUR, PROFILE, TRIM and CONT use either the ARGS database,
or the AGI database (produced by Nick Eaton at Durham), depending on whether
the ARGS or IKON is being used.
AGI was installed in SSC373.
The logical name AGI\_USER must be defined before using any of the new programs.
It should point to the directory holding the AGI database (AGI\_DAT.SDF).

\item [Pan and zoom]:

At the moment, none of the programs which provide a panning and zooming facility
on the ARGS (e.g.\ XYCUR, PROFILE, TRIM) do so on the IKON.
A normal cross-hair cursor which moves over the displayed image (in fact a GKS
locator) is used instead.

\item [Overlays on IKON]:

The new version of EDRSLIB contains three routines IKON\_OVOP, IKON\_CLS and
IKON\_OVCL which handle overlays on the IKON in the same way that ARGS\_OVOP,
ARGS\_CLS and ARGS\_OVCL do for the ARGS.
The IKON overlay device should be opened using SGS before any of the above
IKON routines are used.
Polylines can then be drawn using GKS routine GPL.
The routines give 8 overlay planes which can each have a different colour, and
can be written to and cleared independently.
N.B.\ IKON\_OVOP does not clear the opened plane as ARGS\_OVOP does.
The plane can be cleared with IKON\_CLS if required.

\item [EXE file sizes]:

The new .EXE files for the modified programs are considerably bigger than the
old ones.
This is due mainly to NCAR.

\item [Linking]:

The DCL command procedure EDRSLINK.COM has been modified to include the
required extra libraries, etc.
The AGI software is in three libraries which should be in a directory pointed
to by the logical name AGI\_DIR.
AGI requires the ADAM development system to be started up.
This is done automatically by EDRSLINK if it has not already been done.
The current version of AGI produces linker warning messages because routine
TRN\_ANNUL is referenced but not satisfied.
However this routine is never actually called and so the messages do not
matter.
There also several ``conflicting attributes for psect xxxx" messages from the
linker, which again don't matter.

\item [Graphics]:

The NCAR graphics package is used to produce most of the graphics, although
some SGS and GKS calls are also made.

\item [Default graphics device]:

For programs which previously were ARGS specific (e.g.\ ARGPIC,  XYCUR), and
for other line graphics programs which had a connection file default of ARGS,
there is a new way of determining the run-time default device.
The connection files for these programs now contain a null value so that the
run time defaults are automatically accepted.
The default device is chosen  as follows.
First, an attempt is made to translate the logical name EDRS\_DEVICE.
If this produces a valid SGS device name, then that device is used.
If it doesn't, then the local SGS workstation table is searched for an ARGS 
which if found is used as the default device.
If no ARGS is found, a search is made for an IKON and is used as the default if
found.
Otherwise, the value INFO is used as default.
This means that sites which have an ARGS will continue to use the ARGS as
default, but sites which only have an IKON will use the IKON by default and so
avoid the need for specifying DEVICE=IKON1 on every command line.
It also allows for sites with both ARGS and IKON to set up the IKON as default
by assigning the device name to the logical name EDRS\_DEVICE.

\end{description}

\subsection{New and modified parameters}
Parameter changes have been made to the following programs:
\begin{description}

\item [ARGPIC]:

\begin{itemize}
\item New parameters (all with defaults in ARGPIC.CON):
\begin{description}
\item [DEVICE](see general note on default device)\\
 The GKS device on which to display the image.
 This can be given the name of any SGS workstation identifier which is valid at
 the users node.
 In addition the value INFO can be given which produces site-specific
 information on the available devices (such as plotfile names, etc). 
 Not all devices are capable of displaying images.
 A check is made to ensure that the device has a raster display and the program
 quits with an error if it isn't.
\item [BOX](F)\\
 A logical parameter determining whether to draw a coloured box round the image
 in an overlay plane.
 This can be useful if the image background is low and the edge of the image is
 thus difficult to see.
\item [COLOUR](RED)\\
 The colour in which to draw an outline box if BOX is true.
 Can be red, blue, green, cyan, magenta, yellow, black or white.
\item [URANGE]()\\
 This is an output parameter written to by ARGPIC.
 It holds the value of parameter DRANGE actually used by the program.
 It is useful when several images need to be displayed with the same grey scale,
 because ARGPIC\_URANGE can be given as the value for DRANGE instead of having
 to copy the numbers from the previous run by eye.
\item [BRIGHT](null)\\
 This parameter specifies the brightness corresponding to the upper DRANGE data
 value.
 By default, the program uses the maximum brightness available on the selected
 device (on an ARGS for instance it is 255).
 Reducing the brightness to less than maximum can be useful for increasing the
 contrast between the image and overlaying graphics, especially when the overlay
 contains contours.
\end{description}
\end{itemize}

\item [CONT]:

\begin{itemize}
\item New parameters (all with default in ARGPIC.CON):
\begin{description}
\item [COLOUR](RED)\\
 The colour in which to draw the contours when plotting in an overlay
 (currently ARGS or IKON).
 On other devices the contours are always drawn with the default GKS pen.
 Can be red, green, blue, magenta, cyan, yellow, black or white.
\item [OVERLAY](3)\\
 The overlay plane to use when plotting on an overlay device.
 Currently can be 1 to 8.
\end{description}
\item Modified parameters:
\begin{description}
\item [DEVICE](see general note on default device)\\
 The GKS device on which to display the contours.
 This can be given as the name of any SGS workstation identifier which is valid
 at the users node.
 In addition, the value INFO can be given which produces site-specific
 information on the available devices (such as plotfile names etc).
 The old OVERLAY value has been done away with.
 Instead the SGS names for the ARGS or Ikon overlays can be used (161, 3201,
 ARGSOV1, IKONOV1, etc) in which case the contours are correctly registered
 with the previously displayed image.
\end{description}
\end{itemize}

\item [CUT]:

A bug has been fixed.
If the input image had no INVAL descriptor, any invalid pixels in the output
were assigned the value -100000 which caused an integer overflow due to the
output being stored in I*2 format.

\item [ITFPLOT]:

\begin{itemize}
\item Modified parameters:
\begin{description}
\item [DEVICE](see general note on default device)\\
 The GKS device on which to display the plot.
 This can be given as the name of any SGS workstation identifier which is valid
 at the users node.
 In addition the value INFO can be given which produces site-specific
 information on the available devices (such as plotfile names etc). 
\end{description}
\end{itemize}

\item [MATHS]:

An extra parameter called VALUE has been added.
This is an output parameter and holds the value of pixel (1,1) of the output
image.
The idea is that MATHS can then be used for doing arithmetic between real
parameter values, producing the answer as another parameter which can then be
used as input for another program.
If the formula given by the user does not reference any input images, and if
the size of the output image is given as 1 by 1, then the user is allowed to
enter a null for the output image, in which case no output image is produced.
The value of the single pixel calculated is still written to the new parameter
however.

\item [NORMALIZE]:

\begin{itemize}
\item New parameters:
\begin{description}
\item [HEADING](see description)\\
 A heading for the plot.
 Run time default is `NORMALIZATION PLOT'
\item [XHEAD](see description)\\
 A heading for the X axis.
 Default is of the form ``INTENSITY IN IMAGE A ($<$file name$>$)" where
 $<$file name$>$ is the file name associated with AIMAGE.
\item [YHEAD](see description)\\
 A heading for the Y axis.
 Default is of the form ``INTENSITY IN IMAGE B ($<$file name$>$)" where 
 $<$file name$>$ is the file name associated with BIMAGE.
\end{description}
\item Modified parameters:
\begin{description}
\item [DEVICE](NONE)\\
 The GKS device on which to display the plot.
 This can be given as the name of any SGS workstation identifier which is valid
 at the users node.
 In addition, the value INFO can be given which produces site-specific
 information on the available devices (such as plotfile names etc).
 The value NONE still suppresses graphical output as it used to.
\end{description}
\end{itemize}

\item [PROFILE]:

\begin{itemize}
\item New parameters (all with default in ARGPIC.CON):
\begin{description}
\item [FONT](101)\\
 The GKS font number to use for the titles.
 Default is Roman, medium, sans serif.
\end{description}
\item Modified parameters:
\begin{description}
\item [XYS](PARAMETER)\\
 The coordinates of the profile end points can now also come from the IKON.
 XYS can take values PARAMETER, ARGS or IKON.
\item [DEVICE](see general note on default device)\\
 The GKS device on which to display the trace.
 This can be given as the name of any SGS workstation identifier which is valid
 at the users node.
 In addition the value INFO can be given which produces site-specific
 information on the available devices (such as plotfile names etc). 
\end{description}
\end{itemize}

\item [STARFIT]:

\begin{itemize}
\item Modified parameters:
\begin{description}
\item [DEVICE](NONE)\\
 The GKS device on which to display the plot.
 This can be given as the name of any SGS workstation identifier which is valid
 at the users node.
 In addition the value INFO can be given which produces site-specific
 information on the available devices (such as plotfile names etc).
 The value NONE still suppresses graphical output as it used to.
\end{description}
\end{itemize}

\item [TRIM]:

\begin{itemize}
\item Modified parameters:
\begin{description}
\item [XYSOURCE](see general note on default device)\\
 The coordinates of the corners of the trimmed image can now also come from the
 Ikon.
 XYSOURCE can take values PARAMETER, ARGS or IKON.
\end{description}
\end{itemize}

\item [XYCUR]:

\begin{itemize}
\item New parameters (all with default in ARGPIC.CON):
\begin{description}
\item [DEVICE](see general note on default device)\\
 The device on which the image is displayed from which coordinates are required.
 The device must be specified as an SGS workstation identifier.
 At the moment only IKONs and ARGSs are supported, any other device causes the
 program to quit with an error.
\end{description}
\end{itemize}

\item [XYJOIN]:

The defaults for APREFIX and BPREFIX have been changed from both being null,
to APREFIX being ``A" and BPREFIX being ``B".

\item [XYKEY]:

A bug has been fixed.
If an input xy list was given to XYKEY, any new values entered at the keyboard
overwrote the input values.

\end{description}

\section{EDRSX}

This section describes an update to EDRSX submitted for release on 13 September
1988.

The documentation for this update is contained in the on-line help
library [ASPIC.EDRSX]EDSRXHELP.HLB.
A new version of SUN/40 on EDRSX was also submitted, but this has been
incorporated into revised versions of SUN/23 and SUN/24.

\subsection{Changes}
The update contains new programs BDFGEN and IRASBACK, and enhancements
to programs CRDDTRACE, DATARANGE, DRAWSCAN, HISTOGRAM, IRASDSCR,
IRASSTACK, SIMCRDD and SKYSUB.

\end{document}
