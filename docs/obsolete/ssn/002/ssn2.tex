\documentstyle{article} 
\pagestyle{myheadings}
%------------------------------------------------------------------------------
\newcommand{\stardoccategory}  {Starlink System Note}
\newcommand{\stardocinitials}  {SSN}
\newcommand{\stardocnumber}    {2.1}
\newcommand{\stardocauthors}   {Jeremy Bailey\footnote{Science and Engineering
Research Council, Joint Astronomy Centre Hawaii}}
\newcommand{\stardocdate}      {6 December 1989}
\newcommand{\stardoctitle}     {MSP --- Message System Primitive Routines\\
Version 2.1}
%------------------------------------------------------------------------------

\newcommand{\stardocname}{\stardocinitials /\stardocnumber}
\markright{\stardocname}
\setlength{\textwidth}{160mm}
\setlength{\textheight}{240mm}
\setlength{\topmargin}{-5mm}
\setlength{\oddsidemargin}{0mm}
\setlength{\evensidemargin}{0mm}
\setlength{\parindent}{0mm}
\setlength{\parskip}{\medskipamount}
\setlength{\unitlength}{1mm}

%------------------------------------------------------------------------------
%+                              M A N . S T Y
%
%  Module name:
%    MAN.STY
%
%  Function:
%    Default definitions for \LaTeX\ macros used in MAN output
%
%  Description:
%    As much as possible of the output from the MAN automatic manual generator
%    uses calls to user-alterable macros rather than direct calls to built-in
%    \LaTeX\ macros. This file contains the default definitions for these
%    macros.
%
%  Language:
%    \LaTeX
%
%  Support:
%    William Lupton, {AAO}
%-
%  History:
%    16-Nov-88 - WFL - Add definitions to permit hyphenation to work on
%		 words containing special characters and in teletype fonts.

\typeout{Default MAN macros. Released 8th November 1988}

% permit hyphenation when in teletype font (support 9,10,11,12 point only -
% could extend), define lccodes for special characters so that the hyphen-
% ation algorithm is not switched off. Define underscore character to be
% explicit underscore rather than lots of kerns etc.

%\hyphenchar\nintt=`-\hyphenchar\tentt=`-\hyphenchar\elvtt=`-\hyphenchar\twltt=`-

\lccode`_=`_\lccode`$=`$

\renewcommand{\_}{{\tt\char'137}}

%+                      M A N _ I N T R O
%
%  Section name:
%    MAN_INTRO
%                                 
%  Function:
%    Macros used in the .TEX_INTRO file
%
%  Description:
%    There are no such special macros.
%-

%+                      M A N _ S U M M A R Y
%
%  Section name:
%    MAN_SUMMARY
%
%  Function:
%    Macros used in the .TEX_SUMMARY file
%
%  Description:
%    There is a command to introduce a new section (mansection) and a list-like
%    environment (mansectionroutines) that handles the list of routines in the
%    current section. In addition a mansectionitem command can be used instead
%    of the item command to introduce a new routine in the current section.
%-

\newcommand {\mansection}[2]{\subsection{#1 --- #2}}

\newenvironment {mansectionroutines}{\begin{description}\begin{description}}%
{\end{description}\end{description}}

\newcommand {\mansectionitem}[1]{\item [#1:] \mbox{}}

%+                      M A N _ D E S C R
%
%  Section name:
%    MAN_DESCR
%
%  Function:
%    Macros used in the .TEX_DESCR file
%
%  Description:
%    There is a command to introduce a new routine (manroutine) and a list-like
%    environment (manroutinedescription) that handles the list of paragraphs
%    describing the current routine. In addition a manroutineitem command can
%    be used instead of the item command to introduce a new paragraph for the
%    current routine.
%
%    Two-column tables (the ones that can occur anywhere and which are
%    triggered by "=>" as the second token on a line) are bracketed by a
%    new environment (mantwocolumntable). Other sorts of table are introduced
%    by relevant  environments (manparametertable, manfunctiontable and
%    manvaluetable). The definitions of these environments call various other
%    user-alterable commands, thus allowing considerable user control over such
%    tables... (to be filled in when the commands have been written)
%-

\newcommand {\manrule}{\rule{\textwidth}{0.5mm}}

\newcommand {\manroutine}[2]{\subsection{#1 --- #2}}

\newenvironment {manroutinedescription}{\begin{description}}{\end{description}%
\manrule}

\newenvironment {mansubparameterdescription}{\begin{description}}%
{\end{description}}

\newcommand {\manroutineitem}[2]{\item [#1:] #2\mbox{}}

% two column tables

\newcommand {\mantwocolumncols}{||l|p{80mm}||}

\newcommand {\mantwocolumntop}{\hline}

\newcommand {\mantwocolumnblank}{\mantwocolumnzss\mantwocolumnzhl%
\gdef\mantwocolumnzhl{}\gdef\mantwocolumnzss{}}

\newcommand {\mantwocolumnbottom}{\mantwocolumnzss\mantwocolumnzhl}

\newenvironment {mantwocolumntable}{\gdef\mantwocolumnzss{}%
\gdef\mantwocolumnzhl{}\hspace*{\fill}\vspace*{-\partopsep}\begin{center}%
\begin{tabular}{\mantwocolumncols}\mantwocolumntop}{\mantwocolumnbottom%
\end{tabular}\end{center}}

\newcommand{\mantwocolumnentry}[1]{\mantwocolumnzss\gdef\mantwocolumnzss{\\}%
\gdef\mantwocolumnzhl{\hline}#1 & }

% parameter tables

\newcommand {\manparametercols}{lllp{80mm}}

\newcommand {\manparameterorder}[3]{#1 & #2 & #3 & }

\newcommand {\manparametertop}{}

\newcommand {\manparameterblank}{\gdef\manparameterzhl{}\gdef\manparameterzss{}}

\newcommand {\manparameterbottom}{}

\newenvironment {manparametertable}{\gdef\manparameterzss{}%
\gdef\manparameterzhl{}\hspace*{\fill}\vspace*{-\partopsep}\begin{trivlist}%
\item[]\begin{tabular}{\manparametercols}\manparametertop}{\manparameterbottom%
\end{tabular}\end{trivlist}}

\newcommand {\manparameterentry}[3]{\manparameterzss\gdef\manparameterzss{\\}%
\gdef\manparameterzhl{\hline}\manparameterorder{#1}{#2}{#3}}

% return tables

\newcommand {\manreturncols}{lllp{80mm}}

\newcommand {\manreturnorder}[3]{#1 & #2 & #3 & }

\newcommand {\manreturntop}{}

\newcommand {\manreturnblank}{\gdef\manreturnzhl{}\gdef\manreturnzss{}}

\newcommand {\manreturnbottom}{}

\newenvironment {manreturntable}{\gdef\manreturnzss{}%
\gdef\manreturnzhl{}\hspace*{\fill}\vspace*{-\partopsep}\begin{trivlist}%
\item[]\begin{tabular}{\manreturncols}\manreturntop}{\manreturnbottom%
\end{tabular}\end{trivlist}}

\newcommand {\manreturnentry}[3]{\manreturnzss\gdef\manreturnzss{\\}%
\gdef\manreturnzhl{\hline}\manreturnorder{#1}{#2}{#3}}

% function tables

\newcommand {\manfunctioncols}{||l|l|p{80mm}||}

\newcommand {\manfunctionorder}[2]{#1 & #2 & }

\newcommand {\manfunctiontop}{\hline}

\newcommand {\manfunctionblank}{\manfunctionzss\manfunctionzhl%
\gdef\manfunctionzss{}\gdef\manfunctionzhl{}}

\newcommand {\manfunctionbottom}{\manfunctionzss\manfunctionzhl}

\newenvironment {manfunctiontable}{\gdef\manfunctionzss{}\gdef\manfunctionzhl{}%
\hspace*{\fill}\vspace*{-\partopsep}\begin{center}\begin{tabular}%
{\manfunctioncols}\manfunctiontop}{\manfunctionbottom\end{tabular}\end{center}}

\newcommand {\manfunctionentry}[2]{\manfunctionzss\gdef\manfunctionzss{\\}%
\gdef\manfunctionzhl{\hline}\manfunctionorder{#1}{#2}}

% value tables

\newcommand {\manvaluecols}{||l|l|l|p{80mm}||}

\newcommand {\manvalueorder}[3]{#1 & #2 & #3 & }

\newcommand {\manvaluetop}{\hline}

\newcommand {\manvalueblank}{\manvaluezss\manvaluezhl\gdef\manvaluezss{}%
\gdef\manvaluezhl{}}

\newcommand {\manvaluebottom}{\manvaluezss\manvaluezhl}

\newenvironment {manvaluetable}{\gdef\manvaluezss{}\gdef\manvaluezhl{}%
\hspace*{\fill}\vspace*{-\partopsep}\begin{center}\begin{tabular}%
{\manvaluecols}\manvaluetop}{\manvaluebottom\end{tabular}\end{center}}

\newcommand {\manvalueentry}[3]{\manvaluezss\gdef\manvaluezss{\\}%
\gdef\manvaluezhl{\hline}\manvalueorder{#1}{#2}{#3}}

% list environments

\newenvironment {manenumerate}{\begin{enumerate}}{\end{enumerate}}

\newcommand {\manenumerateitem}[1]{\item [#1]}

\newenvironment {manitemize}{\begin{itemize}}{\end{itemize}}

\newcommand {\manitemizeitem}{\item}

\newenvironment {mandescription}{\begin{description}\begin{description}}%
{\end{description}\end{description}}

\newcommand {\mandescriptionitem}[1]{\item [#1]}

\newcommand {\mantt}{\tt}

% "semi-verbatim" environment (modified from LaTeX source)

\def\zmansemiverbatim{\trivlist\item[]\ifzminipage\else\vskip\parskip\fi%
\leftskip\ztotalleftmargin\rightskip\zz%
\parindent\zz\parfillskip\zflushglue\parskip\zz%
\ztempswafalse\def\par{\ifztempswa\hbox{}\fi\ztempswatrue\zzpar}%
\obeylines}

\def\mansemiverbatim{\zmansemiverbatim\frenchspacing\zvobeyspaces}

\let\endmansemiverbatim=\endtrivlist
%------------------------------------------------------------------------------

\begin{document}
\thispagestyle{empty}
SCIENCE \& ENGINEERING RESEARCH COUNCIL \hfill \stardocname\\
RUTHERFORD APPLETON LABORATORY\\
{\large\bf Starlink Project\\}
{\large\bf \stardoccategory\ \stardocnumber}
\begin{flushright}
\stardocauthors\\
\stardocdate
\end{flushright}
\vspace{-4mm}
\rule{\textwidth}{0.5mm}
\vspace{5mm}
\begin{center}
{\Large\bf \stardoctitle}
\end{center}
\vspace{5mm}

%------------------------------------------------------------------------------
%  Add this part if you want a table of contents
\setlength{\parskip}{0mm}
\tableofcontents
\setlength{\parskip}{\medskipamount}
\markright{\stardocname}
%------------------------------------------------------------------------------
\newenvironment{cozy}[1]%
{\begin{list}{}{%
\settowidth{\labelwidth}{\large\tt #1}%
\setlength{\labelsep}{5mm}%
\setlength{\leftmargin}{\labelwidth}\addtolength{\leftmargin}{\labelsep}%
\setlength{\parsep}{\medskipamount}%
}}{\end{list}}
\newpage
\section{Introduction}

This document describes the message system primitive (MSP) routines.
These routines were specified by the workshop on data acquisition environments
held at AAO in October 1985. The routines provide a fast, general means of
synchronous communication between VMS processes. They are used by version
2.0 of the ADAM system to provide the low level message system facilities.
However, the routines form an independent package which does not require
the existence of any other parts of the ADAM system. Thus they could be used 
for any application which requires fast interprocess communication.

\section{General Description}

\subsection{Queues}

The system is built around the concept of message queues. A queue is
simply a first-in first-out buffer for messages. Any number of messages may
be placed on a queue by one or more sending processes, and read off the
queue by a receiving process. Each queue in the system is owned by a
specific process. Only the owner of a queue may receive a message on
the queue. However, any process may send a message to a queue. A process enters
itself into the message system by calling the routine MSP\_ENTER\_TASK. The
process provides a task name, which other processes can use to obtain
access to its queues. The call to MSP\_ENTER\_TASK creates a queue owned
by the process, which is referred to as the command queue. A process may
also create and delete additional queues as it needs them by calling
the routines MSP\_CREATE\_QUEUE, and MSP\_DELETE\_QUEUE.

\subsection{Queue identifiers}

Each queue is referred to by means of a queue identifier. The
queue identifier for the command queue of a task can be obtained from
the task name by means of the routine MSP\_GET\_TASK\_QUEUE. It is also
possible to get the name of the owner task from a queue identifier by using
the routine MSP\_GET\_TASK\_NAME. How can a task find the queue identifier
of another task's queue which is not the command queue? The answer is that it 
can be sent the queue identifier in a message. Each message can carry a queue
identifier of a queue (referred to as the reply queue) as part of the message.
(NB The terms `command queue' and `reply queue' describe the intended usage
of these queues in the ADAM system, but in fact the facilities provided are
completely general and are not restricted to being used in this way. In 
particular the `reply queue' is not restricted to being a queue owned by the
sending process, so this facility could be used to pass a queue identifier
down a chain of processes.) 

Queue identifiers are generated by the system and are represented as
integer variables. A valid queue identifier never has the value zero,
so zero may be used as a null value in certain circumstances.

\subsection{Sending and Receiving messages}

The routine MSP\_SEND\_MESSAGE is used to send a message. A call to
this routine specifies the queue to which the message is to be sent, the
reply queue to be passed with the message, and the message itself. The
message is in the form of a byte array of a specified length.

Messages are received by calls to MSP\_RECEIVE\_MESSAGE. This routine
specifies a list of queues on which messages are to be received. The order of
queues on the list specifies the order of precedence. The message returned is
the one on the first queue in the list for which a message is present.

If no messages are present on any of the queues the action of the routine
depends on the parameter WAIT. If WAIT is false it returns immediately. If
WAIT is true the routine does not return until a message arrives on
one of the specified queues. As well as the message itself, the queue identifiers
of the queue on which the message arrived, and of the reply queue are returned.

\subsection{AST Delivery in Response to a Message}

Instead of waiting for a message to arrive, it is possible to specify that
an AST will be delivered when a message arrives. This enables a task to carry
on with other processing while it waits for a message to arrive. This is
achieved using the MSP\_ENABLE\_AST routine which specifies the address of an
AST routine to be triggered when a message arrives on one of a list of queues.
The queue identifier of the queue on which the message arrived is passed to
the AST routine as its AST parameter.
                                                    
\section{Scope of Message System}

MSP communicates by setting up queues in a global section shared by each
communicating process. The name of the global section is, by default, derived 
from the username of the user and has the form MSP\_username. Thus all
processes under the same username (including multiple login sessions) can
communicate with each other.

Because global section names are subject to logical name translation it
is possible to override this default behaviour and provide wider or
more restrictive coverage. To change the name of the section define the
logical name GBL\$MSP\_username to be the required section name. In this way
a special message system section can be used to communicate between
different users in a group, or between a specific subset of processes for
an individual user. Any one VMS process can only use one message system
section.

To send a message to a user with a different UIC, but within the same group,
GROUP privilege is required. It is not possible with the current version
of MSP to communicate between users in different groups (this is because
the global section is a group global section).

\section{System Parameters}

The message system code includes several system parameters which are
currently set up as follows:

\begin{tabular}{|l|l|}  \hline
  Max length of a task name     &          20 characters   \\
  Max length of a message       &          512 bytes      \\
  Max number of messages at one time   &   200    \\
  Max number of task entries           &   50     \\
  max number of queues at one time     &   200     \\ \hline
\end{tabular}

With these parameters the message system uses 230 global pages for each
section in use (normally each simultaneous user of MSP will use one
global section). The SYSGEN parameter GLBPAGFIL must be sufficient to
allow for this.

\section{Installation}

The message system files should be copied into a directory with logical name
MSP\_DIR. The logical name MSP\_SHR must also be defined as
MSP\_DIR:MSP\_SHR.EXE. The message system makes use of the AZUSS system service
package. To install the AZUSS system services run the command procedure
MSP\_DIR:INSTAZUSS.COM. CMKRNL and SYSPRV privileges are required for this.
With MSP version 2.0 it is no longer necessary to install the MSP shareable
image. Nor is it necessary (or possible) to run the MSINIT program.

\section{Using the MSP System}

Fortran programs using the message system should use the include
file MSPDEF.FOR, to obtain declarations of the functions as integers,
and definitions of the status code identifiers. 

To link a program with the message system use the command:
\begin{verbatim}

    LINK program,MSP_DIR:MSP/OPT

\end{verbatim}
This will link with the shareable image version of MSP. To link with a non
shareable version use the following command:
\begin{verbatim}

    LINK program,MSP_DIR:MSPNOSHR/OPT
                    
\end{verbatim}
The program MSMON may be found useful in debugging programs which
use the message system. It outputs on the terminal a list of the tasks
currently entered into the message system, and the queues owned by
these tasks.

\section{Contents of MSP\_DIR}

The message system directory contains the following files:

\begin{tabular}{|l|l|}    \hline
AZUSS.EXE      &   AZUSS system services shareable image    \\
AZUSS.MAR      &   Macro source of AZUSS system services package   \\
INSTAZUSS.COM   &  Command procedure to install AZUSS package  \\
LINK.COM        &  Command procedure to link a program with the message system\\
LNKAZUSS.COM    &  Command procedure to link the AZUSS shareable image \\
MSMON.EXE       &  Program to monitor current use of message system \\
MSMON.PAS      &   Source of above program  \\
MSP.TEX        &   Message system documentation (this document) \\
MSP.OBJ        &   Message system routines object code \\
MSP.OPT       &    Linker options file to link with message system  \\       
MSPNOSHR.OPT  &    Linker options file to link with nonshareable MSP \\
MSP.PAS       &    Pascal source of message system routines  \\
MSP.PEN       &    Pascal environment from above \\
MSPDEF.FOR    &    Fortran include file of message system definitions  \\
MSPERR.MSG    &    Message system status codes --- source file \\
MSPERR.OBJ    &    Message system status codes --- object file \\
MSPERR.FOR    &    Message system status codes --- Fortran version \\
MSPERR.PIN    &    Message system status codes --- Pascal version \\
MSPFOR.FOR    &    Fortran source of `wraparound' routines \\
MSPFOR.OBJ    &    Object code of above  \\
MSPLINK.COM   &    Command procedure to link message system shareable image \\
MSP\_SHR.EXE    &    Message system shareable image \\
MSPVEC.MAR    &    Message system transfer vectors --- source file \\
MSPVEC.OBJ    &    Message system transfer vectors --- object file \\
TEST*.*       &    Assorted test programs   \\    \hline
\end{tabular}


\newpage

\section{Message System Routines}

\subsection{Deviation from the specification}

There is one difference between this implementation and the
specification given in the workshop report. The specification
for MSP\_ENABLE\_AST said that the queue identifier of the queue
on which the message arrived should be passed by reference to the
AST routine using the AST parameter. To do this the value would have
to be stored in shared memory, and the routine delivering the AST would
have to have some means of finding the address of the shared memory in the
virtual address space of the process to which the AST was being delivered.

In this implementation the AST parameter is the value of the
queue identifier.

\subsection{Status Values}

The message system routines use the Starlink status convention.
The parameter STATUS must have the value MSP\_\_NORMAL (=0) on entry to
the routine, otherwise the routines will simply return with no action.
If an error occurs during execution of one of the routines this is indicated
by an error status code. The error STATUS values conform to the VAX VMS
condition code convention (the success status value does not!). A message
file is linked into the shareable image, so that LIB\$SIGNAL and LIB\$STOP
will generate meaningful error messages fromn the status codes. The function
return value gives the same value as the STATUS parameter on exit.   
              
In addition to the error codes listed in the routine descriptions, the
first call to an MSP routine by any process may return an error status from
the \$CRMPSC system service. Also it could return the error status
MSP\_\_TIMEOUT indicating a timeout in accessing the message system section.

\subsection{Additional Routines}

MSP\_EXIT can be used to terminate a task's association with
the message system. It is not normally needed because the same action
is performed by an exit handler which is established by MSP\_ENTER\_TASK.
However it is in principle possible for a program to enter and exit the
message system several times and in this case the MSP\_EXIT routine would
be required.

Two further routines (not described here) were provided to meet the needs of
the MSMON program.
              
\newpage
\begin{appendix}

\section{Detailed Routine descriptions}

\manroutine {{\mantt{MSP\_CREATE\_QUEUE}}}{      Create a new queue associated %
with this task.}
\begin{manroutinedescription}
\manroutineitem {Function}{}
      Create a new queue associated with this task.

\manroutineitem {Description}{}
      A new queue is created, and its queue identifier returned
      in {\mantt{QUEUE\_ID}}.

\manroutineitem {Language}{}
      Written in Pascal. Suitable for calling from any language.

\manroutineitem {Call}{}
      {\mantt{FSTATUS}} {\mantt{=}} {\mantt{MSP\_CREATE\_QUEUE}} ( {\mantt{%
QUEUE\_ID}}, {\mantt{STATUS}} )

\manroutineitem {Parameters}{}
\begin{manparametertable}
\manparameterentry {{\mantt{<}}}{{\mantt{QUEUE\_ID}}}{integer, ref}
                          The queue identifier of the created queue.
\manparameterentry {{\mantt{!}}}{{\mantt{STATUS}}}{integer, ref}
      {\mantt{STATUS}} Values
\end{manparametertable}
\begin{mantwocolumntable}
\mantwocolumnentry {{\mantt{MSP\_\_NORMAL}}}  Routine completed succesfully.
\mantwocolumnentry {{\mantt{MSP\_\_NOTINITED}}}  {\mantt{MSP}} Not initialized.
\mantwocolumnentry {{\mantt{MSP\_\_TASKNOTENT}}}  The task has not entered %
itself
                         into the message system.
\mantwocolumnentry {{\mantt{MSP\_\_TOOMANYQUEUES}}}  All available queue entries
                         have been used.
\end{mantwocolumntable}
\manroutineitem {External Variables used}{ None}
\manroutineitem {External routines used}{ None}
\manroutineitem {Author}{ J. Bailey}
\manroutineitem {Date}{ 13th Jan 1986}
\end{manroutinedescription}
\manroutine {{\mantt{MSP\_DELETE\_QUEUE}}}{      Delete a queue.}
\begin{manroutinedescription}
\manroutineitem {Function}{}
      Delete a queue.

\manroutineitem {Description}{}
      Delete the indicated queue. The queue must be owned by
      this task. A queue will not be deleted if it currently
      has a read with wait or Enable\_ast operation uncompleted.

\manroutineitem {Language}{}
      Written in Pascal. Suitable for calling from any language.

\manroutineitem {Call}{}
      {\mantt{FSTATUS}} {\mantt{=}} {\mantt{MSP\_DELETE\_QUEUE}} ( {\mantt{%
QUEUE\_ID}}, {\mantt{STATUS}} )

\manroutineitem {Parameters}{}
\begin{manparametertable}
\manparameterentry {{\mantt{>}}}{{\mantt{QUEUE\_ID}}}{integer, ref}
                          The queue identifier of the queue to be
                          deleted.
\manparameterentry {{\mantt{!}}}{{\mantt{STATUS}}}{integer, ref}
      {\mantt{STATUS}} values
\end{manparametertable}
\begin{mantwocolumntable}
\mantwocolumnentry {{\mantt{MSP\_\_NORMAL}}}  Routine completed succesfully.
\mantwocolumnentry {{\mantt{MSP\_\_NOTINITED}}}  {\mantt{MSP}} Not initialized.
\mantwocolumnentry {{\mantt{MSP\_\_INVQUEUEID}}}  The queue\_id is not valid.
\mantwocolumnentry {{\mantt{MSP\_\_NOTOWNED}}}  The queue specified is not owned
                           by this process.
\mantwocolumnentry {{\mantt{MSP\_\_READINPROG}}}  A read with wait, or enable {%
\mantt{AST}} operation
                             is incomplete on the specified queue.
\mantwocolumnentry {{\mantt{MSP\_\_TIMEOUT}}}  Timeout accessing critical code.
\end{mantwocolumntable}
\manroutineitem {External Variables used}{ None}
\manroutineitem {External routines used}{ None}
\manroutineitem {Author}{ J. Bailey}
\manroutineitem {Date}{ 13th Jan 1986}
\end{manroutinedescription}
\manroutine {{\mantt{MSP\_ENABLE\_AST}}}{      Specify an {\mantt{AST}} %
routine to be triggered in response to a message.}
\begin{manroutinedescription}
\manroutineitem {Function}{}
      Specify an {\mantt{AST}} routine to be triggered in response to a message.

\manroutineitem {Description}{}
      Specify an {\mantt{AST}} routine to be triggered when a message
      is added to any one of the specified list of queues.
      If there is already a message on any of the queues
      (order of queues within the list specifies precedence),
      the {\mantt{AST}} is delivered immediately. When the {\mantt{AST}} is %
delivered,
      the queue identification is passed by value to the {\mantt{AST}}
      routine in its parameter.

\manroutineitem {Language}{}
      Written in Pascal. Suitable for calling from any language.

\manroutineitem {Call}{}
      {\mantt{FSTATUS}} {\mantt{=}} {\mantt{MSP\_ENABLE\_AST}} ( {\mantt{%
QUEUES}}, {\mantt{NUMBER\_QUEUES}}, {\mantt{AST\_ROUTINE}},
                         {\mantt{STATUS}} )

\manroutineitem {Parameters}{}
\begin{manparametertable}
\manparameterentry {{\mantt{>}}}{{\mantt{QUEUES}}}{integer, ref} Array of size %
{\mantt{NUMBER\_QUEUES}}
                         of queue identifiers of the queues on which
                         messages will cause {\mantt{AST}} triggering.
\manparameterentry {{\mantt{>}}}{{\mantt{NUMBER\_QUEUES}}}{integer, ref}
                         The number of queues.
\manparameterentry {{\mantt{>}}}{{\mantt{AST\_ROUTINE}}}{address, value}
                         Address of the {\mantt{AST}} routine.
\manparameterentry {{\mantt{!}}}{{\mantt{STATUS}}}{integer, ref}
      {\mantt{STATUS}} values
\end{manparametertable}
\begin{mantwocolumntable}
\mantwocolumnentry {{\mantt{MSP\_\_NORMAL}}}  Completed succesfully.
\mantwocolumnentry {{\mantt{MSP\_\_NOTINITED}}}  {\mantt{MSP}} Not initialized.
\mantwocolumnentry {{\mantt{MSP\_\_INVQEUEID}}}  One or more of the specified %
queue\_ids
                             is invalid.
\mantwocolumnentry {{\mantt{MSP\_\_NOTOWNED}}} One or more of the specified %
queues is
                             not owned by this task.
\mantwocolumnentry {{\mantt{MSP\_\_READINPROG}}} A read with wait or {\mantt{%
ENABLE\_AST}} operation
                             is already in progress on one or more
                             of the specified queues.
\mantwocolumnentry {{\mantt{MSP\_\_TIMEOUT}}} Timeout accessing critical code.
\end{mantwocolumntable}
\manroutineitem {External Variables used}{ None}
\manroutineitem {External routines used}{ None}
\manroutineitem {Author}{ J. Bailey}
\manroutineitem {Date}{ 14th Jan 1986}
\end{manroutinedescription}
\manroutine {{\mantt{MSP\_ENTER\_TASK}}}{      Make this task known to the %
message system.}
\begin{manroutinedescription}
\manroutineitem {Function}{}
      Make this task known to the message system.

\manroutineitem {Description}{}
      Make this task known to the message system by {\mantt{TASK\_NAME}}, and
      create the associated command queue.

\manroutineitem {Language}{}
      Written in Pascal. Suitable for calling from any language.
      ( A Fortran wraparound routine is used to provide the fixed
      descriptor passing of the task name )

\manroutineitem {Call}{}
      {\mantt{FSTATUS}} {\mantt{=}} {\mantt{MSP\_ENTER\_TASK}} ( {\mantt{TASK\_%
NAME}}, {\mantt{STATUS}} )

\manroutineitem {Parameters}{}
\begin{manparametertable}
\manparameterentry {{\mantt{>}}}{{\mantt{TASK\_NAME}}}{fixed string, descr}
                       name by which the calling task wishes to
                       be known. This name is used purely by the
                       message system and does not necessarily
                       bear any relation to the {\mantt{VMS}} process name.
\manparameterentry {{\mantt{!}}}{{\mantt{STATUS}}}{integer, ref}
      {\mantt{STATUS}} values
\end{manparametertable}
\begin{mantwocolumntable}
\mantwocolumnentry {{\mantt{MSP\_\_NORMAL}}}        The routine completed %
succesfully
\mantwocolumnentry {{\mantt{MSP\_\_TOOMANYTASKS}}}  All available task entries %
have been
                                used.
\mantwocolumnentry {{\mantt{MSP\_\_TOOMANYQUEUES}}}  All available queue %
entries have been
                                used.
\mantwocolumnentry {{\mantt{MSP\_\_TIMEOUT}}}   Timeout initializing the %
global section.
\mantwocolumnentry {{\mantt{MSP\_\_NAMETOOLONG}}}   The task name is too long.
\mantwocolumnentry {{\mantt{SS\$\_*}}}   Error status returned by \$GETJPI.
\end{mantwocolumntable}
\manroutineitem {External variables used}{ None}
\manroutineitem {External routines used}{ None}
\manroutineitem {Author}{  J. Bailey}
\manroutineitem {Date}{ 13th Jan 1986}
\end{manroutinedescription}
\manroutine {{\mantt{MSP\_GET\_TASK\_NAME}}}{      Get the name of the task %
which owns the indicated queue.}
\begin{manroutinedescription}
\manroutineitem {Function}{}
      Get the name of the task which owns the indicated queue.

\manroutineitem {Language}{}
      Written in Pascal. Suitable for calling from any language.

\manroutineitem {Call}{}
      {\mantt{FSTATUS}} {\mantt{=}} {\mantt{MSP\_GET\_TASK\_NAME}} ( {\mantt{%
QUEUE\_ID}}, {\mantt{TASK\_NAME}}, {\mantt{STATUS}} )

\manroutineitem {Parameters}{}
\begin{manparametertable}
\manparameterentry {{\mantt{>}}}{{\mantt{QUEUE\_ID}}}{integer, ref}
                          The queue identifier of the queue.
\manparameterentry {{\mantt{<}}}{{\mantt{TASK\_NAME}}}{fixed string, descr}
                          The name of the task which owns the
                          specified queue.
\manparameterentry {{\mantt{!}}}{{\mantt{STATUS}}}{integer, ref}
      {\mantt{STATUS}} Values
\end{manparametertable}
\begin{mantwocolumntable}
\mantwocolumnentry {{\mantt{MSP\_\_NORMAL}}}  Routine completed succesfully.
\mantwocolumnentry {{\mantt{MSP\_\_NOTINITED}}}  {\mantt{MSP}} Not initialized.
\mantwocolumnentry {{\mantt{MSP\_\_NAMETOOLONG}}} Name is too long for buffer %
supplied.
\mantwocolumnentry {{\mantt{MSP\_\_INVQUEUEID}}}  The Queue\_id specified was %
invalid.
\end{mantwocolumntable}
\manroutineitem {External Variables used}{ None}
\manroutineitem {External routines used}{ None}
\manroutineitem {Author}{ J. Bailey}
\manroutineitem {Date}{ 13th Jan 1986}
\end{manroutinedescription}
\manroutine {{\mantt{MSP\_GET\_TASK\_QUEUE}}}{      Get the identifier of the %
command queue of a named task}
\begin{manroutinedescription}
\manroutineitem {Function}{}
      Get the identifier of the command queue of a named task

\manroutineitem {Language}{}
      Written in Pascal. Suitable for calling from any language.
      ( A Fortran wraparound routine is used to provide the fixed
      descriptor passing of the task name )

\manroutineitem {Call}{}
      {\mantt{FSTATUS}} {\mantt{=}} {\mantt{MSP\_GET\_TASK\_QUEUE}} ( {\mantt{%
TASK\_NAME}}, {\mantt{QUEUE\_ID}}, {\mantt{STATUS}} )

\manroutineitem {Parameters}{}
\begin{manparametertable}
\manparameterentry {{\mantt{>}}}{{\mantt{TASK\_NAME}}}{fixed string, descr}
                          The name of the task for which the
                          queue identifier is required.
\manparameterentry {{\mantt{<}}}{{\mantt{QUEUE\_ID}}}{integer, ref}
                          The queue identifier of the command queue
                          of the task
      ({\mantt{!}}) {\mantt{STATUS}}
      {\mantt{STATUS}} values
\end{manparametertable}
\begin{mantwocolumntable}
\mantwocolumnentry {{\mantt{MSP\_\_NORMAL}}}  Routine completed succesfully.
\mantwocolumnentry {{\mantt{MSP\_\_NOTINITED}}}    {\mantt{MSP}} not %
initialized.
\mantwocolumnentry {{\mantt{MSP\_\_NAMETOOLONG}}}  Task name is too long.
\mantwocolumnentry {{\mantt{MSP\_\_NOSUCHTASK}}}  No task of the specified %
name had
                              entered itself into the message
                              system.
\end{mantwocolumntable}
\manroutineitem {External Variables used}{ None}
\manroutineitem {External routines used}{ None}
\manroutineitem {Author}{ J. Bailey}
\manroutineitem {Date}{ 13th Jan 1986}
\end{manroutinedescription}
\manroutine {{\mantt{MSP\_RECEIVE\_MESSAGE}}}{      Receive a message on one %
of a list of queues.}
\begin{manroutinedescription}
\manroutineitem {Function}{}
      Receive a message on one of a list of queues.

\manroutineitem {Description}{}
      Attempt to read a message from any one of the given list of queues.
      If messages are present, then
         the order of the list of {\mantt{QUEUES}} specifies
         precedence. Return the message and queue information.
      else if {\mantt{WAIT}} is false then
         return with zero queue identifiers
         and {\mantt{ACTUAL\_LENGTH}} of zero.
      else
         wait for the first message to arrive on any of the listed {\mantt{%
QUEUES}}
         and return the message and queue information.
      endif

\manroutineitem {Language}{}
     Written in Pascal. Suitable for calling from any language.

\manroutineitem {Call}{}
      {\mantt{FSTATUS}} {\mantt{=}} {\mantt{MSP\_RECEIVE\_MESSAGE}} ( {\mantt{%
QUEUES}}, {\mantt{NUMBER\_QUEUES}}, {\mantt{WAIT}},
            {\mantt{MAXLEN}}, {\mantt{MESSAGE}}, {\mantt{ACTUAL\_LENGTH}}, {%
\mantt{QUEUE\_ID}}, {\mantt{REPLY\_QUEUE}},
            {\mantt{STATUS}} )

\manroutineitem {Parameters}{}
\begin{manparametertable}
\manparameterentry {{\mantt{>}}}{{\mantt{QUEUES}}}{integer, ref} Array of size %
{\mantt{NUMBER\_QUEUES}}
                         of queue identifiers of the queues on which
                         messages may be received.
\manparameterentry {{\mantt{>}}}{{\mantt{NUMBER\_QUEUES}}}{integer, ref}
                         The number of queues.
\manparameterentry {{\mantt{>}}}{{\mantt{WAIT}}}{logical, ref}
                         True if the routine is required to wait
                         for a message to arrive, when no message
                         is on any of the specified queues.
\manparameterentry {{\mantt{>}}}{{\mantt{MAXLEN}}}{integer, ref}
                         Maximum length of message to be received.
\manparameterentry {{\mantt{<}}}{{\mantt{MESSAGE}}}{byte array of size {\mantt{%
MAXLEN}}, ref}
                         The message received.
\manparameterentry {{\mantt{<}}}{{\mantt{ACTUAL\_LENGTH}}}{integer, ref}
                         The actual length of the message received.
\manparameterentry {{\mantt{<}}}{{\mantt{QUEUE\_ID}}}{integer, ref}
                         The queue identifier of the queue on which
                         the message arrived.
\manparameterentry {{\mantt{<}}}{{\mantt{REPLY\_QUEUE}}}{integer, ref}
                         The queue identifier of the reply queue
                         specified by the calling process.
\manparameterentry {{\mantt{!}}}{{\mantt{STATUS}}}{integer, ref}
      {\mantt{STATUS}} values
\end{manparametertable}
\begin{mantwocolumntable}
\mantwocolumnentry {{\mantt{MSP\_\_NORMAL}}}  Routine completed succesfully
\mantwocolumnentry {{\mantt{MSP\_\_NOTINITED}}}  {\mantt{MSP}} Not initialized.
\mantwocolumnentry {{\mantt{MSP\_\_INVQEUEID}}}  One or more of the specified %
queue\_ids
                             is invalid.
\mantwocolumnentry {{\mantt{MSP\_\_NOTOWNED}}}  One or more of the specified %
queues is
                             not owned by this task.
\mantwocolumnentry {{\mantt{MSP\_\_READINPROG}}}  A read with wait or {\mantt{%
ENABLE\_AST}} operation
                             is already in progress on one or more
                             of the specified queues.
\mantwocolumnentry {{\mantt{MSP\_\_NOMESSAGE}}} No message on any of the %
specified
                             queues, and {\mantt{WAIT}} was not specified.
\mantwocolumnentry {{\mantt{MSP\_\_TRUNCATED}}} Message was truncated to fit %
buffer supplied.
\mantwocolumnentry {{\mantt{MSP\_\_TIMEOUT}}} Timeout accessing critical code.
\end{mantwocolumntable}
\manroutineitem {External Variables used}{ None}
\manroutineitem {External routines used}{ None}
\manroutineitem {Author}{ J. Bailey}
\manroutineitem {Date}{ 14th Jan 1986}
\end{manroutinedescription}
\manroutine {{\mantt{MSP\_SEND\_MESSAGE}}}{      Send a message to a specified %
queue.}
\begin{manroutinedescription}
\manroutineitem {Function}{}
      Send a message to a specified queue.

\manroutineitem {Description}{}
      Put a message on the indicated queue, and specify a queue
      to which replies should be sent. If the message requires
      no reply, then  {\mantt{REPLY\_QUEUE}} may be given a the value zero.

\manroutineitem {Language}{}
      Written in Pascal. Suitable for calling from any language.

\manroutineitem {Call}{}
      {\mantt{FSTATUS}} {\mantt{=}} {\mantt{MSP\_SEND\_MESSAGE}} ( {\mantt{%
MESSAGE}}, {\mantt{MESSAGE\_LENGTH}}, {\mantt{QUEUE\_ID}},
           {\mantt{REPLY\_QUEUE}}, {\mantt{STATUS}} )

\manroutineitem {Parameters}{}
\begin{manparametertable}
\manparameterentry {{\mantt{>}}}{{\mantt{MESSAGE}}}{byte array of size {\mantt{%
LENGTH}}, ref}
                          The message to be sent.
\manparameterentry {{\mantt{>}}}{{\mantt{MESSAGE\_LENGTH}}}{integer, ref}
                          The length of the message to be sent.
\manparameterentry {{\mantt{>}}}{{\mantt{QUEUE\_ID}}}{integer, ref}
                          The queue identifier of the queue to
                          which the message will be sent.
\manparameterentry {{\mantt{>}}}{{\mantt{REPLY\_QUEUE}}}{integer, ref}
                          The queue identifier of the queue to
                          which replies should be sent.
\manparameterentry {{\mantt{!}}}{{\mantt{STATUS}}}{integer,ref} The status %
value.
      {\mantt{STATUS}} values
\end{manparametertable}
\begin{mantwocolumntable}
\mantwocolumnentry {{\mantt{MSP\_\_NORMAL}}}  Routine completed succesfully.
\mantwocolumnentry {{\mantt{MSP\_\_NOTINITED}}}  {\mantt{MSP}} Not initialized.
\mantwocolumnentry {{\mantt{MSP\_\_INVQUEUID}}}  Either {\mantt{QUEUE\_ID}} or %
{\mantt{REPLY\_QUEUE}} is
                          not a valid queue\_id.
\mantwocolumnentry {{\mantt{MSP\_\_NOSLOTS}}} No more message slots are %
available.
\mantwocolumnentry {{\mantt{MSP\_\_DELETED}}}  Queue was deleted during send %
operation.
\mantwocolumnentry {{\mantt{MSP\_\_TIMEOUT}}}  Timeout accessing critical code.
\end{mantwocolumntable}
          Any status from \$WAKE or {\mantt{AZ\$SNDAST}}, e.g.
\begin{mantwocolumntable}
\mantwocolumnentry {{\mantt{SS\$\_NOPRIV}}} No privilege to send message to
                          receiving process (see below)
\end{mantwocolumntable}
\manroutineitem {External variables used}{ None}
\manroutineitem {External Routines called}{}
      Calls the user written system service {\mantt{AZ\$SNDAST}} written by
      Lew Waller, and part of the {\mantt{AZUSS}} package.

\manroutineitem {Restrictions}{}
      The sending process will require {\mantt{GROUP}} privilege if the
      receiving process is in the same group but with a different {\mantt{UIC}}.
      No privilege is required if the receiving process has the same {\mantt{%
UIC}}.

\manroutineitem {Author}{ J. Bailey}
\manroutineitem {Date}{ 14th Jan 1986}
\end{manroutinedescription}

%\input{MSP.TEX_DESCR}


                 
\end{appendix}
\end{document}
