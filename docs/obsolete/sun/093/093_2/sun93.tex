% An appendix contained in the files sun93-app1.tex and sun93-app2.tex
% belongs with this document.

\input sun93lay
\font\smaller=cmr7
\def\LaTeX {{L\kern -.36em\raise .3ex\hbox {\smaller A}\kern -.15em \TeX}}
\def\verbatim{\begingroup\setupverbatim\doverbatim}
\def\doverbatim#1{\def\next##1#1{##1\endgroup}\next}

\nopagenumbers
\parindent=0in
\null
\vskip 15truemm
\headline={\ifnum\pageno=1
            \else\it SUN/93.2\hss \rm\folio\fi}
\line{SCIENCE~AND~ENGINEERING~RESEARCH~COUNCIL \hfil SUN/93.2}
\line{RUTHERFORD~APPLETON~LABORATORY\hfil}
\line{{\twelvebf Starlink~Project} \hfil}
\line{\twelvebf Starlink~User~Note~93.2\hfil}
\line{\hfil Jo Murray}
\line{\hfil 19th July 1988}

\medskip
\hrule depth0.5truemm
\vskip 2cm

\centerline{\bf\Big{\TeX} --- A superior document preparation system.}

\vskip 1.5cm

\TeX\ is an extremely flexible computer type-setting system designed by 
Donald Knuth.
The full capabilities of {\TeX} are described by its creator in
`The \TeX book' with the help of which the enthusiastic {\TeX} 
user can produce
elaborate documents of almost limitless complexity.
However, unadventurous users can produce ordinary documents with a minimum of
effort; simple {\it Runoff}{}
files for example can be edited into {\TeX} format very easily.

In addition to plain {\TeX}, a special version, {\LaTeX}, is available.
This package was
devised by Leslie Lamport and  is fully documented by him in
`{\LaTeX --- Users Guide and Reference manual}'.
Short descriptions  and instructions for use on Starlink machines are 
given in SUNs 9 and~13.
\LaTeX\ is {\it not} simply a collection of \TeX\ macros; for example, although
many plain {\TeX} commands can be used within {\LaTeX},  others are
incompatible with, or are made obsolete by \LaTeX.

An introduction to the use of \TeX\  on Starlink machines is give below:
%\footnote{$^{\bf1}$}{The {\TeX} source for this document is contained in
%DOCSDIR:SUN93.TEX and can be processed as described in section 2.}

\bigskip

{\bf \big Contents}

\medskip
 \line{{\bf\big~~{1.} Producing simple {\TeX} files}\hfil 2}
 \line{~~~~~{\bf{1.1}} Control Sequences\hfil 2}
 \line{~~~~~{\bf 1.2} Grouping\hfil 2}
 \line{~~~~~{\bf 1.3} Simple layouts\hfil 3}
 \line{~~~~~{\bf 1.4} Miscellaneous points\hfil 3}
\medskip
\line{~~{\bf\big{2.} Producing output from a {\TeX} file}\hfil 4}
\line{~~~~~{\bf 2.1} \TeX ing the file\hfil 4}
\line{~~~~~{\bf 2.2} Processing with an appropriate device driver\hfil 4}
\line{~~~~~{\bf 2.3} Printing\hfil 5}
\medskip     
\line{~~{\bf\big {3.} More advanced \TeX} \hfil 5}
\line{~~~~~{\bf 3.1} Maths font\hfil 5}
\line{~~~~~{\bf 3.2} Defining control sequences\hfil 5}
\medskip
\line{{\bf\big~~{4.} Summary of useful control sequences}\hfil 6}
\medskip
\line{{\bf\big~~{Appendix.} Font selection}\hfil 7}



\vfil\eject
\sect { Producing simple {\TeX} files}
 
A file containing only alphabetic text can be made suitable for \TeX ing by 
the inclusion of a final line:
\medskip
\line{{\tt~~~~~~$\backslash$bye\hss}}
%\verbatim.    \bye.
On being processed (see section 2), such a file would result in
justified text in the default size roman font which fits on an A4 page.
New paragraphs are produced by inserting blank lines, and non breakable blanks
(such as occur in the word x~ray) are produced via the character 
`\th$\sim$\th'. 
Thus a {\it Runoff}{} file might be edited by replacing all the occurrences of
`.paragraph' with a blank line, and all the {\it Runoff} non breakable blank
characters `\#' with `\th$\sim$\th'.

%\filbreak

\subsect { Control Sequences}

The string `$\backslash$bye' appended to the end of the above {\TeX} file 
is an example 
of a control sequence, this particular one meaning `{\it fill the rest of
the last page and eject the document'}.
Such control sequences are prefixed by the {\TeX} control character 
`$\backslash$'
 which gives 
notice that what follows is not part of the document being produced but is 
rather some instruction to \TeX. 
For example, the control sequence $\backslash$it tells {\TeX} to 
switch to italic style 
and $\backslash$' puts an acute accent over the following character.

Two important points arise when dealing with control sequences.
Firstly upper and lower case are not equivalent, thus $\backslash$pi and
$\backslash$Pi are two different control sequences.
Secondly, {\TeX} recognizes two distinct types of control sequence, viz 
those which comprise a control word (that is, one or more letters
such as $\backslash$it or $\backslash$bye) and single control symbols 
(such as $\backslash$').
Control words must be terminated with a non-letter such as a blank,
whereas {\TeX} assumes control symbols are only one character in length.

%\filbreak

\subsect { Grouping}

As described above, the control sequence $\backslash$it changes the 
character style used by {\TeX} to italic style. 
The previous font style can be restored by specifically selecting it;
however the more usual practice is to define a group using curly brackets
$\{$$\}$ to which the effect of a particular control sequence is limited.
For example, to highlight a single word in a text the simplest
way is to enclose the control sequence and word in a group thus:
\medskip
\line{{\tt~~~~~~this is a $\{\backslash$bf bold$\}$ word and 
this is in $\{\backslash$it italics$\}$.\hss}}

Some control sequences require one or more arguments. For example a 
piece of text can be centred on a line using the control sequence 
$\backslash$centerline. 
This control sequence will adopt the next `object' it encounters as 
its argument and will centre that object  by itself on a line. 
(In this context an `object' means an indivisible item, perhaps an individual 
letter, a control sequence or a group.)
Thus to centre the word `Introduction' on a line it is necessary to
type;
\medskip
\line{{\tt~~~~~~$\backslash$centerline$\{$Introduction$\}$}\hss}

The omission of the curly brackets would result in only the first object,
the `I' being centred by itself on the line, with the remaining letters on
a subsequent line.

%\filbreak

\subsect { Simple Layouts}

Should a layout other than single spaced A4 in the default
font size be required,
this can be achieved by specifying the necessary instructions either
at the beginning of the {\TeX} file, or in a separate file, (which has the
file extension `.TEX').
In the latter case the layout file is accessed by inserting the line below
at the beginning of the {\TeX} file:
\medskip
\line{{\tt~~~~~~$\backslash$input layout\hss}}
This is analogous to the use of `.include A4.RNO' in {\it Runoff}, where
A4.rno is a file containing {\it Runoff} layout instructions.

The first part of the layout file\footnote{$\bf^1$}{This file is
contained in DOCSDIR:SUN93LAY.TEX} used to generate this document
contains the lines:
\medskip
\tt
\line{~~~~~~$\backslash$hsize=153truemm   \hfil       \% page width}
\line{~~~~~~$\backslash$vsize=220 truemm  \hfil      \% page height}
%\line{$\backslash$magnification=$\backslash$magstep1 \hfil \%   1.2 magnification}
\line{~~~~~~$\backslash$parskip=14 truept plus 5 truept minus 2 
truept\hfil\% paragraph skip}
%\line{~~~~~~$\backslash$special$\{$right=500$\}$  \hfil \% move body of text right 0.5}
\line{~~~~~~$\backslash$hfuzz=1pt \hfil \% tolerance}
%\line{$\backslash$font$\backslash$bi=cmbi10\hfil\% font choice}
\line{~~~~~~$\backslash$baselineskip=13truept plus .2 truept 
\hfil \% inter-line spacing$\simeq$2.5lines/cm}
%\line{$\backslash$lineskip=2 truept  \hfil \% minimum interline clearance}
%\line{$\backslash$parskip=18 truept plus 5 truept minus 2 truept 
%\hfil \% paragraph skip one line}
%\line{$\backslash$mathsurround=0pt   \hfil \% }

%\line{$\backslash$outer$\backslash$def$\backslash$newchap\#1$\backslash$par 
%{ $\backslash$message{\#1} $\backslash$centerline{$\backslash$bf \#1 } 
%\backslash$bigskip }}
%\line{$\backslash$outer$\backslash$def$\backslash$newsect\#1$\backslash$par 
%{$\backslash$vskip 0pt plus .25$\backslash$vsize $\backslash$penalty -250 
%    $\backslash$vskip 0pt plus -.25$\backslash$vsize 
%$\backslash$smallskip $\backslash$vskip $\backslash$parskip $\backslash$message{\#1}
%    $\backslash$leftline {$\backslash$bf \#1} $\backslash$nobreak }
%$\backslash$outer$\backslash$def$\backslash$subsect\#1$\backslash$par 
%{ $\backslash$smallskip $\backslash$vskip$\backslash$parskip $\backslash$message{\#1}
%    $\backslash$leftline { \#1} $\backslash$nobreak }    

\rm
A complete explanation of this file is beyond the scope of this
user note.
Suffice to say here that this file could be edited to produce a different
page size by altering the arguments of the control sequences in the first two
lines, larger characters could be specified by changing from $\backslash$magstep1 to
$\backslash$magstep2, and wider (or narrower) vertical line spacing achieved 
by increasing
(or decreasing) the numerical value of the argument in the
$\backslash$baselineskip control sequence. 
The \% sign is a signal to {\TeX} to ignore the remainder of the line, 
thus enabling the inclusion of comments.
Should you wish to produce the \% sign in a document it is necessary to use the
control sequence $\backslash$\%.

Also defined in this file are the control sequences $\backslash$sect,
$\backslash$subsect, and $\backslash$subsubsect; these can be used to 
replace the {\it Runoff}\ 
`header--level' commands .hl~1, .hl~2, .hl~3.
$\backslash$sect$\{$TITLE$\}$  has 
the effect of centering the
title (as given in brackets), putting it in bold font and taking a larger than
usual skip before the next line.
$\backslash$subsect$\{$TITLE$\}$ and $\backslash$subsubsect$\{$TITLE$\}$ 
have progressively less dramatic effects.
These macros will number sections, subsections {\it etc.} automatically --- 
a facility not present in plain \TeX.
%\filbreak

\subsect { Miscellaneous Points}

\item{(1)} After a control sequence comprising the {\TeX} control character $\backslash$\ 
and a control word, the sequence must be terminated by
a non-letter as mentioned in section 1.1.
If a blank is used {\TeX} will assume this is functioning as a
terminator and will not include a blank in the text.
Should you want a blank the control sequence [$\backslash$\ ] should
be used, (that is, the {\TeX} control character followed by a blank).
Alternatively the control word can be enclosed within curly brackets $\{ \}$.

\item{(2)} {\TeX} possesses complicated algorithms for deciding how to break
lines into paragraphs and pages.
However to spotlight places where a break between paragraphs or
sections is more than usually desirable, the  control sequence $\backslash$goodbreak
can be used.
Similarly the control sequence $\backslash$nobreak will prevent {\TeX} starting a new
page at the point of insertion.

\item{(3)} Hyphenation algorithms provided within {\TeX} usually break words in
a suitable way.
However in the event of unsatisfactory hyphenation for particular words
the user can specify the required hyphenation at the start of the {\TeX} 
file thus:
\medskip
\line{{\tt~~~~~~$\backslash$hyphenation$\{$ inter-stellar mag-nesium$\}$\hss}}



\vfill\eject

\sect { Producing output from a {\TeX} file}

Subsequent to creating a {\TeX} file of type `filename.tex' 
the output can displayed on a suitable VDU or printed on a Canon 
laser printer or
on a Versatec by following the three steps outlined below.

\subsect { `\TeX ing' the file to produce a device independent file}

This is done by typing:
\medskip
\line{{\tt~~~~~~\$ TEX filename\hss}}
{\TeX} will process the file, indicating its progress by
listing section titles and page numbers on the screen. 
If {\TeX} encounters any difficulties the user is prompted with a ?
and the problem line is displayed.
Various options are available in such an event, the simplest being to
respond with a $<$cr$>$ which causes {\TeX} to continue, improvising what it
considers the optimum solution to its dilemma. 
(A HELP facility and the opportunity to
correct the {\TeX} file within the {\TeX} process do exist at this point.)
 
Two new files
are produced, namely a device independent file --- `filename.dvi' which
contains all the information on layout etc. but makes no assumptions 
regarding the nature of the eventual output device, and
also a file, `filename.lis' which contains a variety of information on 
the {\TeX} processing.
This latter file includes details of the nature and location of errors;
therefore in the event of error reports during `\TeX ing' it should be
examined and the original {\TeX} file suitably corrected and re-\TeX ed.

%\filbreak

\subsect { Processing the file in the way appropriate for the selected device}

The device-independent file must be processed by an appropriate device
driver in order to generate a file suitable for printing.
Suitable graphics VDUs can be used to display the output for a `quick look'
prior to printing.
For example to preview a file on a Pericom terminal type:
\medskip
\line{{\tt~~~~~~\$ DVITOVDU filename} \hss }

At this point the VDU screen will be divided into a `dialogue region' and 
a `window region'.
The former comprises  several lines at the top of the screen which are
reserved for information on the current status of the DVItoVDU program and 
contain the `command' prompt.
The remainder of the screen, the window region is used for the representation of
the page.
Details are available in the dvitovdu help file which
can be accessed by typing ? in response to the {\bf{\tt command}} 
prompt within the DVItoVDU program, and a sample run is shown below;
\medskip
\line{{~~~~~~\bf\tt command:} 1\hfil Look at page 1}
\line{{~~~~~~\bf\tt command:} F\hfil A {\bf F}aithful (but much slower) 
representation of the page}
\line{{~~~~~~\bf\tt command:} T\hfil Back to {\bf T}erse mode}
\line{{~~~~~~\bf\tt command:} v 20cm\hfil Set page length displayed on screen to 20cm} 
\line{{~~~~~~\bf\tt command:} H 15cm\hfil Set page width displayed on screen to 15cm} 
\line{{~~~~~~\bf\tt command:} ?\hfil Read the DVItoVDU help file}
\line{{~~~~~~\bf\tt command:} Q\hfil Quit}

A full description of this facility is given in the 
{\it `DVItoVDU User's Guide'} which is included in the MUD series 
(Miscellaneous User Documents) under \LaTeX.


Hardcopy can be produced via the Canon laser printer.
The appropriate device driver is activated by typing:
\medskip
\line{~~~~~~{\tt\$ DVICAN filename} \hfil --- this generates 
a file `filename.DVI-CAN'}

Various qualifiers can be specified at this stage.
For example you can decide to print only selected pages, or to 
shift the output to one side of a page (see SUN34).

\subsect { Printing the file on the selected device}

The file is submitted for printing by typing:
\medskip
\line{~~~~~~{\tt\$ PRINT/PASSALL/QUEUE=$\{$laser$\}$ filename.DVI-CAN}\hfil}

where the queue names in brackets should be that appropriate for the local
site, (SYS\_LASER on Starlink sites).


\sect { More advanced \TeX}

\nobreak
\subsect { Maths Font}
\nobreak

{\TeX} provides a variety of fonts in a large range of sizes.
Bold and italic fonts have already been mentioned and others listed
in section~4 are activated in an analogous way.
The exception is the {\TeX} `maths' font which provides the facilities to
produce subscripts, superscripts, Greek letters, complicated mathematical 
formulae {\it etc}.
Maths font is both selected and terminated by the \$ character.
Within maths mode, subscripts and superscripts are achieved using 
the underscore [\_]
and up-arrow [$\uparrow$] characters respectively.
In each case the following character or following group is shifted.
For example the formulae for carbon dioxide and $x$ to the power of 23 could
be typed:
\medskip
\line{{\tt~~~~~~CO\$\_2\$       and    \$x$\uparrow\{23\}$\$ \hss}}
In the second example, the 23 was defined as a group as otherwise only
the 2 would have been superscripted, the 3 reverting to normal position.
 
Greek letters are selected within maths font simply by using their names 
as control sequences. 
Therefore the Greek letter $\mu$ is produced using the control 
sequence $\backslash$mu. 
Capital Greek letters result when the first letter in the control sequence
is capitalized.
 
{\TeX} has a special mode for displaying mathematical formulae which
is both initiated and terminated with the characters \$\$.
This results in the enclosed formula being centred on a new line, with
various symbols in a larger size than in text mode.
Equations can be numbered using the control sequence $\backslash$eqno
followed by an appropriate number.
This produces a right justified equation number. 
For example:
\medskip
\line{{\tt~~~~~~\$\$ $\backslash$alpha(x)=x$\uparrow$2 $\backslash$eqno(3.1)\$\$
\hss}}

Note that as this is in maths mode it is unnecessary to surround the
parts specifying the Greek letter and superscript with further \$ characters.

%\filbreak

\subsect { Defining simple control sequences}

In addition to the many control sequences defined within \TeX, there exists
the facility to define others for personal use.
For example in a document which frequently uses the abbreviation for
micrometre, as an alternative to always typing the string:
\medskip
\line{\tt~~~~~~\$$\backslash$mu\$m\hfil}
the control sequence $\backslash$um can be defined to replace it thus:
\medskip
\line{\tt~~~~~~$\backslash$def$\backslash$um$\{\backslash$\$mu\$m$\}$\hfil}

\rm
A convenient practice is to define all such control sequences in a file
called `macros.tex' and in any {\TeX} file which requires access to these
include the following line:
\medskip
\line{{\tt~~~~~~$\backslash$input macros
\hss}}

%\filbreak
\vfill\eject

\sect { Summary of useful Control Sequences}

\nobreak

Sample arguments are given with those control sequences which
require them.

\nobreak

\subsect { Layout control sequences}

    
\line{~~~~~~$\backslash$rm                 \hfil \% select roman font}
\line{~~~~~~$\backslash$bf                 \hfil \% select {\bf bold} font}
\line{~~~~~~$\backslash$it                 \hfil \% select {\it italic} font}
\line{~~~~~~$\backslash$sl                 \hfil \% select {\sl slanted roman} font}
\line{~~~~~~$\backslash$tt                 \hfil \% select {\tt typewriter-like} typeface}
\line{~~~~~~$\backslash$hsize=5 truein     \hfil \% defines page width}
\line{~~~~~~$\backslash$vsize=10 truein    \hfil \% defines page length}
\line{~~~~~~$\backslash$smallskip          \hfil \% causes a small vertical skip}
\line{~~~~~~$\backslash$medskip            \hfil \% causes a medium vertical skip}
\line{~~~~~~$\backslash$bigskip            \hfil \% causes a big vertical skip}
\line{~~~~~~$\backslash$centerline$\{$ A title$\}$ \hfil \% centres following
 group}


{\bf  Maths Font control sequences}
\medskip

%{\begindoublecolumns{

 \line{~~~~~~$\backslash$hbar                     \hfil \%  $\hbar$~~~~~~}
 \line{~~~~~~$\backslash$partial                  \hfil \%  $\partial$~~~~~~}
 \line{~~~~~~$\backslash$infty                    \hfil \%  $\infty$~~~~~~}
 \line{~~~~~~$\backslash$prime                    \hfil \%  $\prime$~~~~~~}
 \line{~~~~~~$\backslash$surd                     \hfil \%  $\surd$~~~~~~}
 \line{~~~~~~$\backslash$sum                      \hfil \%  $\sum$~~~~~~}
 \line{~~~~~~$\backslash$int                      \hfil \%  $\int$~~~~~~}
 \line{~~~~~~$\backslash$odot                     \hfil \%  $\odot$~~~~~~}
 \line{~~~~~~$\backslash$AA                       \hfil \%  \AA~~~~~~}
%}}

In addition to the symbols $=,<,>$ (which should be in maths mode)
the following are available. Most can be negated by prefixing
with the control sequence $\backslash$not, {\it e.g.} $\backslash$not=.

%{\begindoublecolumns

\line{~~~~~~$\backslash$simeq       \hfil \%  $\simeq$~~~~~~}
\line{~~~~~~$\backslash$sim         \hfil \%  $\sim$~~~~~~}
\line{~~~~~~$\backslash$leq         \hfil \%  $\leq$~~~~~~}
\line{~~~~~~$\backslash$geq         \hfil \%  $\geq$~~~~~~}
\line{~~~~~~$\backslash$ll          \hfil \%  $\ll$~~~~~~}
\line{~~~~~~$\backslash$gg          \hfil \%  $\gg$~~~~~~}
\line{~~~~~~$\backslash$equiv       \hfil \%  $\equiv$~~~~~~}
\line{~~~~~~$\backslash$propto      \hfil \%  $\propto$~~~~~~}
     
%\enddoublecolumns}

A variety of delimiters are available; these can be produced directly if
a keyboard includes them, otherwise a control sequence must be used.
The more commonly used are given below.
Only the left-side delimiter is listed, the right differing
only by the replacement of the initial l with r. In each case successively bigger
symbols are achieved by prefixing the appropriate delimiter control
sequence with the control sequences (in ascending order of size)
$\backslash$bigl,$\backslash$Bigl,$\backslash$biggl,$\backslash$Biggl. As before r replaces l when dealing with
right-side delimiters.


\line{~~~~~~$\backslash$lbrack                  \hfil\%  $[$~~~~~~}
\line{~~~~~~$\backslash$lbrace                  \hfil\%  $\{$~~~~~~}
\line{~~~~~~$\backslash$langle                  \hfil\%  $<$~~~~~~}
\bye



