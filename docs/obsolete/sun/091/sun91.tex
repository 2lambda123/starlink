\documentclass[nolof,noabs,11pt]{starlink}


%------------------------------------------------------------------------------
\stardoccategory    {Starlink User Note}
\stardocinitials    {SUN}
\stardocnumber      {91.2}
\stardocauthors     {D C Parsons, P J Richards, D L Giaretta}
\stardocdate        {12 July 1989}
\stardoctitle     {IRAS --- Calibrated Raw Detector Data Analysis
\linebreak \linebreak User Guide}
%------------------------------------------------------------------------------

% -----------------------------------------------------------------------------
% ? Document specific \providecommand or \newenvironment commands.
% ? End of document specific commands
% -----------------------------------------------------------------------------

\begin{document}
\scfrontmatter

\section {Introduction}

This guide describes the IRAS Calibrated Raw Detector Data, how to access the
data, and the programs available to process it and examine the results.

The results from the IRAS mission have been processed in several ways to give
for example the point source catalogue (PSS) and the sky flux plates. In
general, the processing used to derive these products was somewhat conservative
in order to achieve a high reliability for the PSS sources and a uniform
resolution among the bands for sky flux plates. However, if a lower reliability
is accepted fainter sources can be extracted from the data, and a relaxation of
the resolution uniformity requirement allows more spatial information to be
preserved in the construction of images.
Complete details of the reliability, completeness, and resolution of the
processed data can be found in the Explanatory Supplement.

The Calibrated Raw Detector Data (CRDD) has been made available so that
astronomers can process it themselves to obtain data of whatever standard
they feel is acceptable. The Explanatory Supplement gives the necessary details
of the sampling procedure and the inherent reliability considerations.

However, processing CRDD is not a task to be undertaken lightly. The data is
voluminous and processing it takes a lot of both the users and the computers'
time. It is always worthwhile looking at the processed data available to see
whether it is suitable before embarking on CRDD processing.

\section{CRDD data}

The following provides only a brief introduction to CRDD data, notes on the raw
data processing can be found in the Explanatory Supplement.

The IRAS mission observations were scheduled in 12 hour periods called
Satellite Operations Plans (SOP's). A SOP consisted of a number of survey scans
and Additional Observations (AOs). The CRDD data archive consists of the
results of the survey scans and is the subject of this SUN.

The survey scans covered any particular position on the sky up to 6 times. The
first time an object was scanned several half overlapping observations would be
made on different parts of the same SOP, or consecutive SOPs, this group of
observations is termed an HCON (for Hours Confirmed). Similar sets of
observations would be made between several weeks to 6 months later.

The data in each scan consists of the samples taken from each detector at each
waveband. The detectors in a single waveband partly overlap, the processing
programs enable you to combine overlapping samples correctly.

The CRDD data archive is held as files called plates. Each plate contains all
the sections of sample scans which cross a region of sky corresponding to the
region covered by a Palomar Sky Plate. Although the plates overlap by about a
degree at each edge, some of the big regions of interest will fall over the
boundaries of a plate. Data can be supplied from two plates but normal
processing may lead to a different background being subtracted from each half.

In estimating whether there is a source at a given position you should bear in
mind that each point on the sky is crossed typically by 12 detectors in each
band. Individual traces may not show any obvious signal but the expected
improvement arising from COADDing the individual detector data may give a
significant signal.

\section {CRDD data availability}

CRDD data archive is held on 6250 bpi tapes at RAL. IPMAF staff at RAL will
extract this data for you, either to tape, or for a small quantity of data to
disk for you to copy. A mail message should be sent to IRASMAIL at RLVAD
stating the coordinates of the center of the source, its extent and scan
length, and the wavebands required (see details of the I\_SNIP\_CRDD program
for the explanation of the data required). At the moment data will be supplied
as the output from the I\_SNIP\_CRDD program, but users requiring large areas
may be sent copies of the relevant plates that they can process themselves.
Please note that supplying CRDD data is a VERY time consuming
process, an area 2 degrees square will take a morning to process, 5 small
sources will take a day, and obviously these times will be much increased if
there is a backlog of work to be done. So please use the system with
consideration.

\section {Processing CRDD data}

There are two types of processing you may wish to use on CRDD data, either to
prepare an image, or to see if there is a significant source at a particular
position, and measure its flux.

You should be careful to process each waveband separately.

\subsection{CRDD Images}

The processing of CRDD data to make images falls into three stages:-
\begin{itemize}
\item The extraction of data for the region of interest from the data archive
tapes.
\item Forming an image from the data.
\item Examining the image.
\end{itemize}

\subsubsection{Extracting an image}

Data is extracted from the archive tapes using the I\_SNIP\_CRDD program.
The output from the program consists of a series of files each of which
contains the sample data for one scan in one waveband only.

\subsubsection{Forming an image}

Forming an image from the data takes place in three stages:-
\begin{itemize}
\item Each detector has its own differences in calibration, which means that two
detectors each looking a the same piece of sky would give a different background
level. By Destriping you can use a 'flat-fielding' technique  to adjust the
detector levels to reduce this effect substantially. An option in the
destriping process also allows you to remove a constant or a linearly sloping
background.
\item The samples from the half overlapping detectors are next distributed over
the area of sky seen by that detector, and added together to give a strip
like image whose pixels are aligned along the scan direction.
\item The subimages from each scan required to form the total image, are then
added together on a new matrix of pixels which are parallel to the RA and Dec
axes (Rebinning).
\end{itemize}

There are five programs to carry out this processing

I\_DESTRIPE carries out the first stage of the process. The program has to be
run for each scan file needing destriping.

I\_CRDIMAGE carries out the second stage of the process. The program has to be
run for each scan file to be made into a subimage.

I\_COMBINE carries out the third stage of the process. The program has to be
run for each image you want to prepare.

I\_EASYIMAGE is an easier way of carrying out the first two stages. You supply
it with a list of scan files you want destriped and made into a subimage, and
the program prepares a complete set of subimage files which you then have to
combine into the required images using I\_COMBINE.

I\_CRDD\_COMBINE provides a more refined combination of the
second and third stages of the processing thus reducing the number of rebinning
stages from two to one. This gives a better resolution than using I\_CRDIMAGE
and I\_COMBINE to produce an image.

\subsubsection{Examining an image}

There are two ways of examining an image.

I\_CONTOUR gives you a contour plot, and enables you to measure flux levels
within boxes and at points. You should not use this to obtain point source flux
measurements because the program does not take into account the effect of edge
detection etc.\ (see Photometry below).

The KAPPA suite of programs gives you the capability of manipulating images,
but, as yet, it has no facilities to enable you to determine the coordinates of
a pixel in RA and Dec. It will enable you to display images colored according
to the intensity of the flux, but it will not enable you to add a 60 micron
image in red to a 12 micron image in blue as in skyflux plates.

\subsection  {Photometry}

The CRDD data values on tape have to be multiplied by a constant factor to
reproduce the published IRAS calibration, the appropriate factor for each band
is incorporated in the I\_CRDIMAGE, I\_CRDD\_COMBINE and  I\_COADD programs.
Total flux densities for sources bigger than 5 arcmins taken from images
produced by I\_CRDIMAGE (e.g.\ by using the flux integration facilities in
I\_CONTOUR) have uncertainties of approx 30\%. For sources smaller than 5
arcmins, fluxes estimated by integration are likely to be in error by as much
as 50\%, due to undersampling in the cross-scan direction, and the effects of
missing or noisy detectors. Point source fluxes should be obtained either from
the point source catalogue, or by using I\_COADD.

\subsection{Evaluating point source fluxes}

There is a program (I\_COADD) developed to determine the fluxes at an assumed
point source, and to evaluate its statistical significance. The data must
first be extracted using I\_SNIP\_CRDD.

\section {Examples}

\subsection {Image of Cometary Globule CG22}

Figure 1 in the appendix shows 60 micron maps of cometary globule CG22, derived
from a SKY Plate image. The corresponding one from COMBINE'd CRDD data is shown
in Figure 2. The latter clearly shows superior spatial resolution and the
effect of removing a local, sloping background.

\subsection {Beware of Spurious Features}

Users should always examine each HCON individually before COMBINEing them
together. This procedure will help avoid wrongly interpreting moving sources,
glitches and/or detector artifacts as infrared structure. The set of 12 micron
maps in figures 3 to 6 show how the asymmetric point spread function apparent
on brighter sources, the change in scan direction in HCON3 and the odd moving
or variable source can combine into what looks like ``interesting structure" to
the uninitiated.

\section {EDRSX Package}

There is also a package, EDRSX, which can be applied to CRDD data. This provides
alternative programs to perform, for example, destriping and image
manipulation. The data produced by the programs documented here must be
converted to EDRS format before using EDRSX programs. Please refer to \xref{SUN/23}{sun23}{} for
brief descriptions.

\section {Detailed descriptions of programs}


Detailed descriptions of all the programs except KAPPA follow. The section on
KAPPA merely describes the type of processing you could do with KAPPA, and how
to obtain a file which will interface with KAPPA, the full documentation of
KAPPA is in \xref{SUN/95}{sun95}{}.

All programs run in the Starlink Interim environment. Plans are in hand to
migrate to the ADAM environment.

\begin{center}
\large{PLEASE MAKE SURE YOU INCLUDE
\\
\$ IRASTART
\\
IN YOUR LOGIN.COM FILE SO THAT THE CORRECT LOGICAL NAMES ARE DEFINED.}
\end{center}
The symbol IRASTART is defined in the Starlink login file SSC:LOGIN.

\pagebreak
\section {I\_SNIP\_CRDD --- Data extraction}

\subsection {Introduction}

I\_SNIP\_CRDD is a program which retrieves data from IRAS Calibrated Raw
Detector Data tapes, and produces .BDF files which may be used by other IRAS
programs (e.g.\ EASYIMAGE and I\_COMBINE, I\_DESTRIPE and I\_CRDD\_COMBINE or
I\_COADD) to produce images, or estimated fluxes of IRAS sources.


\subsection {The information you will need to run I\_SNIP\_CRDD}

\begin{itemize}

\item A suitable source name for each source --- although you can use up to 16
characters it is advisable to use only 4 as these are used as the root of your
.BDF filenames.

\item The Right Ascension and Declination (1950) of the center of each source
region.

\item The extent and scan length required (in arcmins) for the area required.
The direction of the survey scans is skewed with respect to the RA and Dec
axes, and with respect to each other. I\_SNIP\_CRDD refers to the coordinate
system related to the detector array in terms of scan length and extent.
Unfortunately there is no easy relationship between extent and scan length, and
R.A. and Dec. Extent and scan length are measured across or along the
direction of scan of the detectors and this direction may be at an angle to
parallels of R.A. and Dec. The best you can do is to estimate extent as size
of region in R.A. times a suitable factor (say 1.2). Scan length will have to
be larger than the extent (usually a minimum of 120 arcmin or 2 x extent,
whichever is the greater) to find some background level for destriping. A map
is produced by I\_SNIP\_CRDD so that you can check that the area you require is
covered.

\item Select the bands that you want. You may select any combination of bands
1,2,3,4. They are entered separated by commas.


\item You will need to know what kind of data tape you will be using. If you
are running the program at RAL or Queen Mary College you will probably be using
the full archive data set. If you are running it elsewhere you will probably be
using a tape of a plate or plates extracted from the archive set. Then you will
need to answer ``No" to the question ``Is INPUT on a RAL I\_SNIP\_CRDD archive
tape", and to know the position of the plate on the tape if the tape has
several plates on it.

\item When running the program you will need to decide which crossings to use.
You may choose to allow the program to extract all crossings, but this takes a
long time if there are a lot of crossings. Alternatively you can opt to choose
the crossings you require. The program will print a multicolumn list of
crossings (with more than one page if necessary) and invite you to choose those
you require. Crossings from the same HCON are identified by having the same or
consecutive SOP numbers. You may choose any selection of crossings as long as
you obey the format rules: ','  between individual choices, '-'  between a
range of choices.

\end{itemize}

\subsection {An example of using I\_SNIP\_CRDD:}
\begin{terminalv}
    $ I_SNIP_CRDD
    Input source name - up to 16 characters
    NAME:=LMC1
    Input source RA in hrsminsecs (eg HHMMSS.S)
    RA:=5000.0
    Input source DEC in degminsecs (eg DDMMSS.S)
    DEC:=-732200.0
    Input source extent in arcmins (<=300.0)
    EXTENT:=30.0
    Input scan length in arcmins (120.0-300.0)
    SCANLEN:=120.0
    Input Bands that you want
    BANDS:=2,3

        Name              RA             DEC         Extent    Scan Len.
                     hr  min  sec   deg  min   sec   A/Mins     A/Mins    Bands
    LMC1              0   50  0.0   -73   22   0.0    30.0      120.0      2,3

    is this correct (y/n)?
    ANS:=Y
    Do you wish to choose which crossings to use? (y/n)
    ANS:=Y
    Is INPUT on RAL I_SNIP_CRDD archive tape? (y/n):
    ANS:=N
\end{terminalv}

\vspace {3mm}
If the answer is N the computer will then ask

\begin{terminalv}
    Which position is plate on tape (eg 1= first plate on tape)
    1
    Which tape drive [XXXX]?
    MUA0

    Please load tape < tape number > on drive < drive name >
    Please press < RETURN > when tape is loaded on tape deck:
\end{terminalv}

The user must now mount the correct tape (without a write ring), put
the device online, and press $<$~RETURN~$>$

\begin{terminalv}
    %MOUNT-I-MOUNTED, mounted on _ <drive name>
\end{terminalv}
If you have selected the choose crossings option they will all be listed
on the screen, and written to a file XXXXCROSS.LIS (where XXXX is the source
name).  There may be more than one screen full, in which case you have
the option of stopping the program and looking at the list of crossings.

\begin{terminalv}
    There are more crossings - do you want to stop and study
    the crossings or continue and make a choice.
    CONTINUE? (Y/N):
    ANS:= Y
\end{terminalv}
N will stop the program at this point.

The computer now lists other pages of crossings

\begin{terminalv}
    Enter your choices (default none) for example 1,4,6-9
    2-7

    Your choices were 2-7
    Are these correct? (Y/N)
    ANS:=Y
\end{terminalv}
In both choose crossings and non-choose crossings modes the program will then
proceed to create the .BDF files for each crossing, generating the following
output as it processes the data.

\pagebreak
\begin{terminalv}
               I_SNIP_CRDD Version 1.2 released  1-APR-1989
             run on  6-APR-1989    11:00:11.73

      Source:       LMC1
      RA:            0 50  0.0
      DEC:         -73 22  0.0
      Extent:        30.0
      Scan length:  120.0
      CRDD files generated:

        LMC1_         X SCAN POS      UTCS TIME    SOP NO     LENGTH
\end{terminalv}
\hrule
\begin{terminalv}
       B2S2  B3S2        16.2      89299711.857     557        258.0
       B2S3  B3S3        21.8      89349168.647     558        258.0
       B2S4  B3S4        10.0      89355351.011     559        258.0
       B2S5  B3S5        -1.8      89361534.357     559        258.0
         .     .           .              .           .           .
         .     .           .              .           .           .
        etc   etc         etc            etc         etc         etc
\end{terminalv}

N.B.\ Any crossing marked with \texttt{S} between SOP NO and LENGTH is one that
is shorter than the scan length.

\subsection {Files produced}

\begin{tabular}{lp{1.8in}p{2.9in}}
filename & explanation & contents \\ \hline

\texttt{XXXXXX.LIS} & where \texttt{XXXXXX} is the source name & A log of the files
produced, and source details \\

\texttt{XXXX\_BnSm.BDF} & where \texttt{XXXX} are the first 4 chars of source name,
n is the band no, and m is the crossing number & BDF file containing
CRDD data for a crossing. \\

\multicolumn{3}{p{\textwidth}}{The number of bands and crossings you chose will determine the number of BDF
files  produced by I\_SNIP\_CRDD. The filenames correspond to the crossings
chosen and are produced automatically.} \\

\texttt{XXXXCROSS.LIS} & where \texttt{XXXX} are the first 4 chars of the
source name & Crossing details. \\

\texttt{XXXXMAP.LIS} & where \texttt{XXXX} are the first 4 chars of the source name & A crude
map of the plate area.
Shows asterixes for each 5 secs of data found and places them on the plate
area.\\

\texttt{DUMMYn.BDF} & where n is the band no of any bands NOT chosen by you & Empty;
needed for certain INTERIM routines, not needed by the user and may be
deleted when the program ends. \\
\end{tabular}

\pagebreak
\section {I\_DESTRIPE --- Data destriping}

\subsection {Introduction}

Each detector has its own differences in calibration, which means that two
detectors each looking a the same piece of sky would give a different background
level. By Destriping you can use a 'flat-fielding' technique  to adjust the
detector levels to reduce this effect substantially. An option in the
destriping process also allows you to remove a constant or a linearly sloping
background.

\subsection {The destriping process}

The destriping removes the effects of the detector calibration by calculating,
for each detector, a suitable constant to be removed from all its data samples.
The program calculates the median value for each detector scan, and the mean of
these medians, and thence the constants required for each detector to bring all
detectors to the same median value. For this to work correctly it is necessary
that you give the program a scan in which there is at least as much background
as source.

The destriping program also gives you the option of removing either a constant
or a constant plus a sloping background. The constant is calculated as the
median value of all the data in the scan. If you require the removal of a
sloping background the program will calculate the median for each eighth of the
scan and calculate the slope and constant from that.

\subsection {The information you will need to run I\_DESTRIPE}

\begin {itemize}

\item The name of the file containing the data. This will usually have been
extracted with I\_SNIP\_CRDD and will be of the form XXXX\_BnSm.BDF
(where XXXX is the first 4 characters of the source name, n is the band
number, and m is the crossing number).

\item You will need to decide whether you want just a constant background
removed or both constant and slope.

\item A suitable file name for the output data file --- usually this is chosen
as the input name with the characters \_DS added to it (indicating
destriped data).
\end {itemize}

\subsection {An example of using I\_DESTRIPE}

\begin {terminalv}
$ I_DESTRIPE
CRDD file to be destriped
IN:= LMC1_B2S3
Remove background (y/n)?
REMBAK/N/:= Y
Constant(n), Sloping(y)
SLOPE/N/:= Y
Destriped file
OUT:= LMC1_B2S3_DS
\end{terminalv}
\pagebreak

\section{I\_CRDIMAGE --- Converting scan samples to scan images}

\subsection{Introduction}

I\_CRDIMAGE  converts the raw detector data for an individual scan
onto an image grid with square pixels. The data does not need to be destriped
but we strongly recommend that you should destripe it.

\subsection{The conversion process}

The raw detector data is presented to the program as a series of detector flux
samples. I\_CRDIMAGE takes into account how the detectors
overlap and, by assuming that flux was gathered evenly over the detector area,
distributes the flux from each sample over the area of smaller pixels
corresponding to the detector aperture. Thus I\_CRDIMAGE forms an image by
adding together the contributions from each sample of each detector, on a pixel
grid which is still oriented along the direction of the scan.

\subsection{The information you will need to run I\_CRDIMAGE}

\begin{itemize}

\item The name of the destriped CRDD file from which the image is to be made.
These will usually have been extracted with I\_SNIP\_CRDD, and destriped with
I\_DESTRIPE, and will be of the form XXXX\_BnSm\_DS.BDF (where XXXX
is the first 4 characters of the source name, n is the band number, and m is
the crossing number, and \_DS indicates a destriped file).

\item Select the size of the pixels you require depending
on the wavelength band you are processing. The recommended minimum pixel sizes
are :- 12 micron  0.25 arcmin, 25 micron  0.25 arcmin, 60 micron   0.5 arcmin,
100 micron  1.0 arcmin. These are selected as default.

\item Determine whether you wish to process the whole scan length, to reduce
the scan length will reduce the computing time considerably and the extra
length may be unnecessary after destriping. The shortened scan length will be
centered on the target position input to I\_SNIP\_CRDD.

\item Decide a suitable filename for the output image. A suitable suffix eg \_IM
could be used to indicate an image produced by I\_CRDIMAGE.
\end {itemize}

\subsection{An example of using I\_CRDIMAGE}

\begin {terminalv}
$ I_CRDIMAGE
Name of file containing CRDD data
CRDDF:=     GCAS_B1S1_DS
Size of pixels in output image(arc mins)
PIXSIZE/0.2500000/:= <RETURN>
Pixel size(arc mins):  0.25 x 0.25

Length of scan to be processed(arcmin)

SCANLEN/120.00000/:=   120
Name of file to contain output image
IMAGE:=     GCAS_B1S1_IM
\end{terminalv}
\pagebreak

\section {I\_COMBINE --- Image combination}

\subsection {Introduction}

I\_COMBINE is a program used to combine the sub images from single scans into a
complete image. The direction of the survey scans is skewed with respect to the
RA and Dec axes, and with respect to each other. I\_COMBINE takes images whose
pixels are not oriented along the RA and Dec parallels and converts them to a
tangential plane centered at the center of the image with Dec on the y axis,
and combines them. Near the equator this gives almost RA and Dec oriented
pixels.

CRDD data is normally first destriped and turned into image format. The result
of this processing, on a single scan, will be an image oriented along the scan
direction which is about half a degree wide and a few degrees long. For a
typical region of sky there will be 2 or 3 sets of scans (HCONSs). Each HCON
will consist of several half overlapping scans. Usually  I\_COMBINE will be run
to combine the scans for each HCON into an HCON image.
These can be compared to assess the reality
of a feature. Finally I\_COMBINE may be used to assemble a complete image from
these sub images.To obtain the correct weights of the images in producing the
overall image, all the desired separate scans should be combined in a single
run of the program.


\subsection {The information you will need to run I\_COMBINE}

\begin{itemize}
\item The names of the files containing the data. These will usually have been
extracted with I\_SNIP\_CRDD, destriped with I\_DESTRIPE, and made into an
image with I\_CRDIMAGE and will be of the form XXXX\_BnSm\_IM.BDF (where XXXX
is the first 4 characters of the source name, n is the band number, and m is
the crossing number).

\item You will need to decide whether you want the image to be the maximum size
available from the size of scans chosen, or whether you want the option of
choosing the size of the image and its centre.

\item You will need to decide the size of the pixels you require depending
on the wavelength band you are processing. The recommended pixel sizes are :-
12 micron  0.25 arcmin, 25 micron  0.25 arcmin, 60 micron   0.5 arcmin,
100 micron  1.0 arcmin.

\item A suitable file name for the output data file --- for example
XXXX\_Bn\_Hp\_IM1.BDF (where XXXX is the first 4 characters of the source
name, n is the band number, and p is the HCON number).

\end {itemize}

\subsection {An example of using I\_COMBINE}

\begin {terminalv}
$ I_COMBINE

Version 3.0 (3_AUG_1987)

IMAGE:=GCAS_B1S1_IM
IMAGE:=GCAS_B1S2_IM
IMAGE:=GCAS_B1S3_IM
IMAGE:=  etc. etc.
IMAGE:= <RETURN>

Create the whole image or some other area?

WHOLE/T/:= <RETURN>

Select output pixel size in arcmin,(X,Y)

SIZE/0.25,0.25/:= <RETURN>

Processing image 1.......        ~o)(o~                    :

OUTPUT:= N6543_B1_HC1_IM
\end{terminalv}

\subsection {Processing time}

I\_COMBINE is CPU intensive and for more than 4 or 5 images should be run
during off-peak hours. Timing tests showed that combining 4 scan images
took 5mins and 100 CPU secs on a MicroVAX 3500.

\subsection {Problems}

You cannot combine images from different wavebands, If you try to you will get
problems. Normally these manifest themselves as an error message :-
\begin{terminalv}
%FOR-F-OUTSTAOVE, output statement overflows record
\end{terminalv}
\pagebreak

\section{I\_EASYIMAGE --- Destriping and image forming}

\subsection {Introduction}

I\_EASYIMAGE is a program which is used to destripe and to form images from one
or more .BDF data files, containing the data for a scan in one wavelength band,
but it does not combine these images.

\subsection {The I\_EASYIMAGE process}

A list of files to be processed is passed to I\_EASYIMAGE each of which are
processed separately.

The destriping part of the program enables you to remove differences in
calibration, and registration of background intensities between detectors,
by subtracting a uniform or sloping background. The detector background value
is calculated as the median value of the data over the whole scan. For this to
work correctly it is necessary that you give the program a scan in which there
is at least as much background as source. If you require the removal of both
background and slope, the program will calculate the median for each eighth of
the scan and calculate a slope and constant from that.

The imaging part of the program takes into account how the detectors
overlap and, by assuming that flux was gathered evenly over the detector area,
distributes the flux from each sample over the area of smaller pixels that
would have been seen by the detector. Thus it forms an image by adding
together the contributions from each sample of each detector, on a pixel matrix
which is still oriented along the direction of the scan.

\subsection {The information you will need to run I\_EASYIMAGE}

\begin {itemize}

\item The names of the files containing the data. These will usually have been
extracted with I\_SNIP\_CRDD and will be of the form XXXX\_BnSm.BDF
(where XXXX is the first 4 characters of the source name, n is the band
number, and m is the crossing number).

\item A file called PREP.DAT should be set up containing a list of filenames to
be processed

**** \hfill ONLY FILES CONTAINING DATA FOR ONE WAVELENGTH BAND \hfill****
\linebreak
**** \hfill SHOULD BE PROCESSED TOGETHER OR THE WRONG SIZE OF PIXEL \hfill****
\linebreak
**** \hfill FOR THE WAVEBAND MAY BE USED \hfill****

A suitable format for the file is :-

\begin{terminalv}
	GCAS_B1S1
	GCAS_B1S2
	GCAS_B1S3
	GCAS_B1S4
	GCAS_B1S5
	GCAS_B1S6
\end{terminalv}

\item You will need to decide the size of the pixels you require depending
on the wavelength band you are processing. The recommended pixel sizes are :-
12 micron  0.25 arcmin, 25 micron  0.25 arcmin, 60 micron   0.5 arcmin,
100 micron  1.0 arcmin.

\item You will need to decide whether you want just a constant background
removed or both constant and slope.

\end {itemize}

\pagebreak

\subsection{An example of using I\_EASYIMAGE}

\begin {terminalv}
$ I_EASYIMAGE

Remove background (Y/N)? :=
Y
Remove slope in background (Y/N) ? :=
Y
Size of square pixel in output image (arcmin)? :=
0.25
Length of scan (arcmin) ? :-
120
PROCESSING IMAGE  1  :  GCAS_B1S1
DESTRIPING AND GRIDDING COMPLETED FOR IMAGE 1
PROCESSING IMAGE  2  :  GCAS_B1S2
DESTRIPING AND GRIDDING COMPLETED FOR IMAGE 2
   .       .      .        .      .    .   .
   .       .      .        .      .    .   .
   .       .      .        .      .    .   .
   .       .      .        .      .    .   .

*********************************
**ALL FILES DESTRIPED & GRIDDED**
*********************************
\end{terminalv}
N.B.\ This procedure is CPU intensive.
If more than 4 or 5 scans are to be processed, the user is advised to run the
job during off-peak hours, ie lunch times or after 6.00 pm.
\pagebreak

\section{I\_CRDD\_COMBINE --- Combined image forming}

\subsection {Introduction}

I\_CRDD\_COMBINE takes a set of destriped CRDD scan data files and forms a
combined image from them. It combines the features of I\_CRDIMAGE and I\_COMBINE
in a single rebinning step thereby improving the resolution.

\subsection {The I\_CRDD\_COMBINE process}

    I\_CRDD\_COMBINE combines the functions of I\_CRDIMAGE and I\_COMBINE and
thus minimizes the smoothing on the output image due to rebinning by reducing
the number of stages from two to one. It provides the best
resolution available using rebinning techniques. To obtain
better resolution other techniques have to be applied to the data. The
improvement in resolution over the I\_CRDIMAGE/I\_COMBINE method is achieved at
the cost of increased computation time, so this program should only be used when
optimum resolution is important for your data analysis. You should not use
this program to produce images of extended structure covering an area of
sky of several square degrees, since I\_CRDIMAGE/I\_COMBINE will give a similar
result faster especially for the $12\mu m$ and $25\mu m$ bands.
   Before describing how to run the program, an explanation of the purpose of
some of the parameters is necessary. The parameter SFACT determines the size of
sectors into which each detector sample is split for rebinning onto the image
grid. The size of a sector is given by PIXSIZE/SFACT and thus the larger the
value of SFACT, the smaller the sector. Small values of SFACT (i.e.\ large
sector sizes) can lead to smoothing and loss of resolution in the final image,
whereas large values of SFACT can result in large computation times with no
improvement in resolution over images produced with smaller values of SFACT.
To allow some flexibility, but to avoid inordinate computation time, SFACT is
restricted to a range 2.0 to 4.0 (the default is 2.0). Note also that the size
of a sector is limited to less than the half-width of a detector which is
0.38, 0.38, 0.755 and 1.515 arc mins. at 12, 25, 60 and 100 microns
respectively.


    There is also an the option to weight the sectors within a data
sample using a Gaussian weighting function. Basically the further the sector is
from the centre of a sample, the smaller the weight assigned to it. The use
of Gaussian weighting has the benefit of producing smoother responses for
point sources and minimizes the effects of 'edge' detections (see pV-19 of
the IRAS Explanatory Supplement). The parameters AYEDGE and AZEDGE determine
the shape of the Gaussian weighting function. They represent the amplitudes
relative to the centre at the in-scan and cross-scan edges of the sample
respectively. The definitions of the terms in-scan and cross-scan are given in
chapter V of the IRAS explanatory supplement. This method of defining the shape
of the Gaussian was chosen rather than specifying a width because it indicates
directly the degree of weighting at the edges.

\subsection{The information you will need to run I\_CRDD\_COMBINE}

\begin{itemize}

\item The names of the destriped CRDD files from which the image is to be
assembled. Usually I\_CRDD\_COMBINE will be run to combine the scans for each
HCON into an HCON image. These can be compared to assess the reality of a
feature. Finally I\_CRDD\_COMBINE may be used to assemble a complete image from
these sub images.To obtain the correct weights of the images in producing the
overall image, all the desired separate scans should be combined in a single
run of the program.

\item You will need to decide whether you want the image to be the maximum size
available from the size of scans chosen, or whether you want the option of
choosing the size of the image and its centre.

\item Select the size of the pixels you require depending
on the wavelength band you are processing. The recommended minimum pixel sizes
are :- 12 micron  0.25 arcmin, 25 micron  0.25 arcmin, 60 micron   0.5 arcmin,
100 micron  1.0 arcmin. These are selected as default.

\item You will need to decide the value for the Sector factor. The sector factor
determines the number of sectors into which a detector sample is split for
rebinning onto the image grid (see note above). The value that can be selected
is restricted to between 2.0 and 4.0.

\item Determine whether you wish to process the whole scan length, to reduce
the scan length will reduce the computing time considerably and the extra
length may be unnecessary after destriping. The shortened scan length will be
centered on the target position input to I\_SNIP\_CRDD.

\item Decide whether you want the detector data to be weighted, and if so what
ratio of center to edge weighting you require in both the in-scan and the
cross-scan directions. These can take any value between 0.01 and 0.999, the
suggested default value is 0.5.

\item Decide a suitable filename for the output image. A suitable suffix eg \_CC
could be used to indicate an image produced by I\_CRDD\_COMBINE.

\end{itemize}

\subsection{An example of using I\_CRDD\_COMBINE}

\begin{terminalv}
$ I_CRDD_COMBINE

CRDDF:= GCAS_B1S1_DS
CRDDF:= GCAS_B1S2_DS
CRDDF:= GCAS_B1S3_DS
CRDDF:=  etc. etc.
CRDDF:=  <RETURN>

Whole image to be produced (T/F)?
WHOLE/T/:= <RETURN>
Size of pixels in output image (arc mins)
PIXSIZE/0.25000x0.25000/:= <RETURN>
Pixel size (arc mins): 0.25000x0.25000
Sector factor
SFACT/2.000000/:= <RETURN>
Length of scan to be processed (arc mins)
SCANLEN/120.0000/:= <RETURN>
Weighting of detector samples required (T/F)?
WEIGHT/F/:= T
AYEDGE/0.5000000/:= <RETURN>
Amplitude at in-scan edge: 0.50
AZEDGE/0.5000000/:= <RETURN>
Amplitude at X-scan edge: 0.50
Processing image 1...      =(o-o\                                         |

OUTPUT:= GCAS_B1_H1_CC
\end{terminalv}

\subsection{Examples of output from I\_CRDD\_COMBINE}

   Some examples of the output from this program with and without weighting
are given in the Appendix at the end of this document.

Comparison of images produced by I\_CRDIMAGE and I\_CRDD\_COMBINE:

Figures 7 to 10 show various 60 micron maps of the spiral galaxy M81.
Figure 7 shows the I\_CRDIMAGE / I\_COMBINE combined image from four survey
scans comprising a single HCON.
Figure 8 is the I\_CRDD\_COMBINE image produced from the same data without
weighting.
Figure 9 shows the image obtained by applying Gaussian weighting with an
amplitude of 0.1 at both the in-scan and cross-scan edges.
This compares favourably with the map from a pointed observation of M81 shown in
Figure 10.

\pagebreak

\section{I\_CONTOUR --- Flux contour mapping}

\subsection {Introduction}

I\_CONTOUR draws contour maps of IRAS images and allows source positions and
fluxes to be measured.

CRDD data is normally prepared by extracting it with I\_SNIP\_CRDD, destriping
it, forming images of the scans, and combining the scan images.


There are many optional facilities in I\_CONTOUR which are described in detail
below.

\subsection{The type of processing you can do with I\_CONTOUR}

\begin{itemize}

\item Plot a map of a whole image with an RA and Dec grid, lines showing the
boundary of the image, and with contours calibrated in absolute units.

\item Plot a map of a subsection of an image.

\item Plot a map with different contour levels.

\item Measure the flux at a point or within a box, and determine the position of
maximum flux within a box.

\item Carry the positions of points and boxes over to other images to enable
you to compare the fluxes at different wavelengths.

\item Display the output on many different devices including sending it to a
file suitable for hardcopy output.
\end{itemize}

\subsection{An example of using I\_CONTOUR}

Here is a walkthrough of a typical examination of an image formed from CRDD
data.

\begin{itemize}
\item [Part 1:] Prepare a contour map of the whole image
\begin{terminalv}
$ I_CONTOUR
Version 3.0  (3-AUG-1987)
IMAGE:=GCAS_B4_HC3_IM3.BDF
Processing image  1...                                                   :
Special plot or use defaults?
SPECIAL/F/:= <RETURN>
Min and Max values are:-      -24.09       93.90
Enter up to 20 contour levels,
OR just the base level,
OR the base level and further qualifiers,
OR just press return for default levels
LEVELS:= <RETURN>
Base contour is 5.00 and subsequent ones *1.50
Enter a title:-
TITLE//:=GCAS_B4_HC3_IM3
\end{terminalv}
\pagebreak
\begin{terminalv}
  name        description

  7800        Pericom 7800
  ARGS1
  ARGSOV1
  BBC         BBC micro
  CANON       Canon Laser printer - bit map file is CANON.DAT
  CANON_L     Canon Laser (Landscape) - bit map file is CANON.DAT
  CANON_P     Canon Laser (Portrait) - bit map file is CANON.DAT
  CIFER_2634  Cifer 2634G
  CIFER_T5    Cifer T5
  GRAPHPACK   Pericom Graph Pack
  IKON        Digisolve Ikon
  IKONOV      Ikon overlay
  MG100       Pericom Graph Pack
  PRINTRONIX  Printronix printer - bit map file is PRINTRONIX.BIT
  TEK         Tektronix 4010
  ZETA        Zeta 8 pen plotter - plot file is ZETA.DAT

Which do you want to use?
DEVICE/7800/:= <RETURN>
\end{terminalv}
Press the $<$GRAPH$>$ key on the Pericom to get graphical display.

The computer then draws a contour map of the source region on a grid of RA and
Dec coordinates. The boundary of the region is marked with a dotted line. The
map is annotated with the coordinates of the origin, the title and a key to
the contour levels (which are given in absolute flux levels).

In the small graphics type face it prints at the top of the screen
\begin{terminalv}
Do you want to measure fluxes and positions?
FLUXES/F/:=
\end{terminalv}
Press $<$shift$>$$<$VDU$>$ to obtain the subsequent output in standard typeface
\item [Part 2:] To determine a suitable size of area for a subimage.

To map a sub image you need to know both the coordinates of the centre of the
image, and the size of the required area in x (i.e.\ RA) and y (i.e.\ Dec)
coordinates, this can be done either by eye using the grid, or using the flux
measuring facility. The x and y size are in arc min.


The procedure for doing the latter is as follows:-

Enter T for true to the question
\begin{terminalv}
Do you want to measure fluxes and positions?
FLUXES/F/:= T
A record will be kept in FLUXES.LIS
Manual,file or quit
MODE/M/:= <RETURN>
\end{terminalv}
Press graph only key to get a graphic screen

A cross wire cursor should be present on the screen if not press $<$RETURN$>$ to
obtain one. The cross wire can be moved to the centre of the region of interest
by using $<$SHIFT$>$ and arrow keys for coarse movement and arrow keys alone for
fine movement. When you are satisfied with the position push any key twice, and
the position and the flux will be typed in very small graphics type. It will
also type:-
\begin{terminalv}
Manual,file or quit
MODE/M/:= <RETURN>
\end{terminalv}
Press the VDU key to type in standard typeface and type \texttt{Q}

\pagebreak
\item [Part 3:] To draw a contour map of a subimage, and to plot with a lower
base contour level.

After your preceding runs the program will type:-

\begin{terminalv}
Do you want to plot this image again?
Enter F - No,
     T - yes, change all parameters,
     D - yes, change Device only,
     L - yes, change contour Levels,
AGAIN/F/:=T
Special plot or use defaults?
SPECIAL/F/:=T
Plot the whole image or some other area?
WHOLE/T/:=F
Current centre:  RA =   0 54 59.3
               DEC =  60 38 10
   size is  125.0 x  121.0 arcmins
Enter centre RA and DEC and size of the area you want to use as:-
raH,raM,raS, decD,decM,decS, Xsize,Ysize
AREA:=0,55,30,60,46,00,20,20

Label the contours?
LABELS/T/:= <RETURN>
Mark the edge of the area covered?
EDGE/T/:= <RETURN>
RA and DEC in lines or ticks?
LINES/T/:= <RETURN>
Enter RA and DEC line spacings in secs,arcmins
SPACINGS/15.00000, 5.000000/:= <RETURN>
Plot particular objects on the map?
OBJECTS/F/:= <RETURN>
Min and Max values are:-       -5.25       93.90
Enter up to 20 contour levels,
OR just the base level,
OR the base level and further qualifiers,
OR just press return for default levels
LEVELS:=0.5,2.0
Base contour is 0.500 and subsequent ones *2.00
Enter a title:-
TITLE/GCAS_B4_HC3_IM3/:= <RETURN>
\end{terminalv}

Press the $<$GRAPH$>$ key on the Pericom to get graphical display.

The computer then draws a contour map of the subimage.

In the small graphics type face it prints at the top of the screen
\begin{terminalv}
Do you want to measure fluxes and positions?
FLUXES/F/:=
\end{terminalv}
Press $<$shift$>$$<$VDU$>$ to obtain the subsequent output in standard typeface

\pagebreak
\item [Part 4:] To find the flux within a box on the subimage and several point
fluxes.


The procedure for drawing a box is :-

Enter T for true to the question

\begin{terminalv}
Do you want to measure fluxes and positions?
FLUXES/F/:= T
A record will be kept in FLUXES.LIS
Manual,file or quit
MODE/M/:= <RETURN>
\end{terminalv}

Press graph only key to get a graphic screen

A cross wire cursor should be present on the screen if not press $<$RETURN$>$ to
obtain one. The cross wire is first moved to the bottom left hand side of the
region of interest by using $<$SHIFT$>$ and arrow keys for coarse movement and
arrow keys alone for fine movement. When you are satisfied with the position
push any key once (this needs a sharp tap as the keys are repeating). Using the
up and down arrows move to the top left and press any key once. Move back to
the bottom left (part of the mark and the crosswire disappear) and then using
sideways arrows move to the bottom right, and press any key once. Finally move
up to just above the left top mark and press any key once or twice. The program
will display the flux in graphics text, but it is easier to read from the
FLUXES.LIS file. It will also mark the position of the maximum flux within the
box, and give the value of the flux at this point.

It will also type:-
\begin{terminalv}
Save marked area?
SAVE/T/:= <RETURN>
FILE:= GCAS_B4_HC3_CENTERBOX.LIS

Manual,file or quit
MODE/M/:=
\end{terminalv}

To continue with any other boxes, or to obtain fluxes at a point press
$<$RETURN$>$.

To obtain the fluxes at any point follow the method outlined in finding the
center of the subregion above.

To quit from the cycle of finding fluxes type Q in reply to the mode prompt.

After finishing the run you can examine the results of the flux measurements
by printing or typing the file FLUXES.LIS eg.

\begin{terminalv}
type fluxes.lis
  gcas_b4_hc3_im3

  Peak  RA   0 55 32.7      Dec  60 48 12
  Peak flux  =  9.3896E+07  Jy/Ster
  Total flux =   204.6      Jy
  (using   100  points)
  Total - background =   139.1      Jy
  Average FWHM =  4.10 mins

  RA   0 54 25.8       Dec  60 53  9
  Flux  =  6.3332E+07  Jy/Ster

  RA   0 57  3.1       Dec  60 50 10
  Flux  =  2.6011E+07  Jy/Ster

  RA   0 56 18.9       Dec  60 38 21
  Flux  =  6.9975E+07  Jy/Ster
\end{terminalv}

\pagebreak
\item [Part 5:] To prepare a file containing the contour map for printing on the
laser printer.

Note that the position of any boxes will not appear on the map.

\begin{terminalv}
Do you want to plot this image again?
Enter F - No,
     T - yes, change all parameters,
     D - yes, change Device only,
     L - yes, change contour Levels,
AGAIN/F/:=D
\end{terminalv}

The program will then print out a list of available devices

\begin{terminalv}
Which do you want to use?
DEVICE/7800/:=CANON
Enter plot scale in mm/arcmin
SCALE/7.813919/:= <RETURN>
When the program finishes to get your plot type
PRINT/QUEUE=SYS_LASER/PASSALL CANON.DAT
Finished plotting
Do you want to plot this image again?
Enter F - No,
     T - yes, change all parameters,
     D - yes, change Device only,
     L - yes, change contour Levels,
AGAIN/F/:= <RETURN>
$
\end{terminalv}

\item [Part 6:] To plot the points and boxes used on this image on the image of
the same source at another wavelength, and determine corresponding flux
measurements.

The replotting of boxes and measurements of the flux within them can be done
automatically. The position of points can be entered automatically but the
flux must be measured by repositioning the cross wire cursor over the object
on the map.

To replot the position of points you must first prepare a file containing
containing their coordinates, the easiest way to do this is to edit the
FLUXES.LIS file to look like this:-

\begin{terminalv}
$ type gcas_b4_hc3_positions.lis

0 54 25.8       60 53  9

0 57  3.1       60 50 10

0 56 18.9       60 38 21
\end{terminalv}

Then you can run I\_CONTOUR using the data file for the other waveband as
follows:-

\begin{terminalv}
$ I_CONTOUR
Version 3.0  (3-AUG-1987)
IMAGE:=GCAS_B2_HC3_IM3.BDF
Special plot or use defaults?
SPECIAL/F/:=T
Plot the whole image or some other area?
WHOLE/T/:= <RETURN>
Label the contours?
LABELS/T/:= <RETURN>
Mark the edge of the area covered?
EDGE/T/:= <RETURN>
RA and DEC in lines or ticks?
LINES/T/:= <RETURN>
\end{terminalv}
\pagebreak
\begin{terminalv}
Enter RA and DEC line spacings in secs,arcmins
SPACINGS/120.0000, 15.00000/:= <RETURN>
Plot particular objects on the map?
OBJECTS/F/:=T
Which file contains their positions?
FILE/TT/:= gcas_b4_hc3_positions.lis
\end{terminalv}
Continue answering the prompts as in part 3 above. The contour map will be
drawn with numbered stars at the points. Because of graphics overtyping it is
easiest first to display any boxes. You can do this by entering:-
\begin{terminalv}
Do you want to measure fluxes and positions?
FLUXES/F/:= T

A record will be kept in FLUXES.LIS

Manual,file or quit
MODE/M/:= F
FILE:= GCAS_B4_HC3_CENTERBOX.LIS
\end{terminalv}

To measure the fluxes at points you need to follow the procedure outlined
in part 2 positioning the cross wires over each point in turn. The fluxes both
in boxes and at the points will be found in the next version number of
FLUXES.LIS when you exit the program. A list of the objects plotted will also
be found in a file OBJECTS.LIS.
\end{itemize}

\vspace{ .1in}
\begin{center}
{\large *****\hfill  NOTE TYPING $<$CTRL$>$ C AT ANY TIME \hfill*****
\\ *****\hfill TAKES YOU BACK TO THE ``AGAIN" PROMPT \hfill*****}
\end{center}


\subsection{Options offered by I\_CONTOUR}

\begin{description}
\item[\mbox{}]\item{}
\begin{description}
\item[IMAGE]\hspace{-1.5mm}:

This can be any .BDF file containing an IRAS image.

\item[LABELS]\hspace{-1.5mm}:

TRUE here results in labelled contours, FALSE does not.

\item[EDGE]\hspace{-1.5mm}:

TRUE will mark on the contour map the edge of the area actually containing data.

\item[LINES]\hspace{-1.5mm}:

TRUE here means the RA and Dec will be plotted as continuous lines across the
map.
FALSE just puts tick marks at the edges (except when the pole is included in the
map).

\item[SPACINGS]\hspace{-1.5mm}:

The spacings of the RA and Dec lines or ticks can be changed.

\item[OBJECTS]\hspace{-1.5mm}:

The positions of particular objects can be marked on the map.
If you respond TRUE here, you will also be asked for the name of a file
containing the list of positions (in free format, 6 numbers to a line
representing the RA and Dec in Hours, Minutes, Seconds, Degrees, Minutes,
Seconds).
The first line of the file can, optionally, consist of a single number
representing the epoch.
The actual value is ignored, but this makes the file compatible with that
accepted by the CHART program.
A file name of TT represents your terminal.

\item[LEVELS]\hspace{-1.5mm}:

Here you specify the contour levels to be plotted.

If you press $<$RETURN$>$ default levels are used, a single value is taken to
represent the lowest contour level and subsequent ones formed by multiplying the
previous one by a constant.

If you enter two values, the first is taken as the lowest contour level and the
second as the multiplication factor for subsequent levels.

If you enter two or three values and the second one is negative, the absolute
magnitude of the second
value is then taken as the difference between subsequent levels and the third
value, if present, is the total number of contour levels to draw (up to a
maximum of 40).

\item[TITLE]\hspace{-1.5mm}:

The title is also written to files FLUXES.LIS and OBJECTS.LIS which keep a
record of any fluxes measured and any particular objects plotted on the contour
map (if the relevant options were selected).

\item[FLUXES]\hspace{-1.5mm}:

Details of how to obtain the fluxes are described in the examples above.
There are some caveats however on the measurement of fluxes.
Some attempt is made to remove the local background flux by subtracting the
values in the strip around the edge of the area (for integrated fluxes only,
not peak fluxes).
If this strip is at all contaminated by a neighbouring source, clearly the
calculation will be grossly inaccurate.
It is then up you to measure a small piece of the map containing just
background flux and subtract it from the total flux after allowing for the
number of pixels used in the integration.

The size (FWHM) of the source is a rough measurement only and should be
disregarded unless the area contains only one, fairly circular, source with just
background flux around the edge of the area.
\end{description}
\end{description}
\pagebreak

\section{The use of KAPPA with CRDD data}

\subsection{Introduction}

This short note is not intended as a guide to KAPPA which is fully documented
in \xref{SUN/95}{sun95}{}. It is intended to give you some idea of the type of processing one
could do with KAPPA and the preprocessing needed to get CRDD data which is in
INTERIM .BDF format into .HDS Image Data format.

\subsection{The type of processing you can (and can't) do with KAPPA}

\begin{itemize}
\item Perhaps the most obvious thing you can do with KAPPA is to look at the
IRAS image as a picture, it gives more of a feel both for the intensity of the
various regions and of the resolution available. Colour images based on the
intensity of an image can be made. Contour plots can be overlaid on grey or
colored images. Pictures can be panned and zoomed for inspection.
\item Hardcopy greylevel plots can be produced.
\item Images can be manipulated in ways that are not possible in the CRDD
system, for instance the log of the image data can be calculated and displayed
as an image. The images in different bands can be algebraically manipulated. For
example the image in band 3 can be divided by that in band 4 to emphasise the
temperature difference between clouds and embedded stars.
\item Histograms and thresholding can be used to help with background removal.
The peaks of a histogram will help to distinguish between source radiation
levels and background radiation levels. Thresholding can be used to form an
image mask for the true source region which can be used to multiply the data to
obtain the source region alone.
\item Unfortunately KAPPA does not provide any good facilities for measuring
position or for performing operations based on position. For instance you
cannot inquire the RA and Dec of a particular pixel, you cannot cut out a
subimage determined by RA and Dec coordinates.
\item Theoretically it should be possible to take a .BDF image from the CRDD
system, change it into a .HDF file for KAPPA while copying the descriptors,
perform algebraic manipulation in KAPPA, change the KAPPA output into a .BDF
file again copying the descriptors, and process the resulting image with
I\_CONTOUR to obtain accurate positions and fluxes.
\end{itemize}

\subsection{Changing a CRDD data file into a file for use with KAPPA}

A program STARIN is available to do the necessary conversion. It is fully
documented in \xref{SUN/96}{sun96}{}. The options we require for CRDD image file conversion
that the HDS data type should be \_real, you can also specify no descriptors to
be copied unless you wish to copy them back (using the descriptors portion of
STAROUT which is also described in that SUN).

You should read the introductory parts of \xref{SUN/95}{sun95}{} before using KAPPA especially
that relating to the definition of work space.

\pagebreak

\section{I\_COADD --- Estimate point source fluxes from CRDD data}

\subsection{Introduction}

I\_COADD is a program which estimates the fluxes seen by IRAS at the position
of a known point source (e.g.\ single star (not binary), distant galaxy, QSO
etc.).

Details of all the bright reliable point sources found in the IRAS survey can
be found in the Point Source Catalogue. However you may have an optical source
for example for which there is no corresponding entry in the Point Source
Catalogue, and wish to ask what IRAS saw at this point.


Each point on the sky is crossed typically by 12 detectors in each band,
individual traces may not show any obvious signal but the expected improvement
arising from COADDing the individual detector data may give a significant
signal. Point source flux amplitudes are estimated by correlating the source
signal with that expected from an ideal point source, the latter has been
derived from IRAS crossings of asteroids and the calibration source NGC 6543.

\subsection{The I\_COADD process}

The main processing of the I\_COADD program consists of the following stages
\begin{itemize}
\item The first stage is to determine the detectors to be used and to extract
the detector sample data from the files of scan data extracted with
I\_SNIP\_CRDD.

It uses the RA and Dec coordinates given to I\_SNIP\_CRDD to find each detector
which could see the region. The normal detector scan width is 4 arcmin and if
the source position falls within that field of view the detector data is used by
the program, but you have the option of modifying the criterion by selecting a
non zero detector width extension DZ. This means that if the source falls
within DZ/2 of the cross-scan edge of the detector field of view that detector
is used.It is recommended that if you want to alter this you set it to a
value between $0.0$ and $-1.0$ as this will mean that detectors which see the
source at the edge of their field of view will not be selected .

For each detector a sequence of 50 samples is extracted centered on the
expected source position, these are sufficient for source signal and noise
estimation.

\item The next stage in the process is to calculate a composite value for each
sample from the samples for each detector.

First the median of the samples is calculated for each detector and subtracted
from the associated samples.

The detector samples are then weighted and corresponding samples added.
The default weights the samples so that the most noisy detectors have the
lowest weighting (actually the detector weight is the reciprocal of its R.M.S.
noise measured in the checkout phase of the IRAS mission). However you have the
option of adding the detector samples with equal weighting.

\item The program then determines the median of the composite data and
subtracts it from the coadded samples.

\item The program calculates the noise level of the composite data. The data
are convolved with an eight pixel wide, zero sum, square wave filter.
This is equivalent to performing a four pixel wide integration
with the local background level and slope removed. It scans the resultant
values for signals above a given signal to noise threshold.
The default threshold value is 1.0, but you may change it if you wish for a more
stringent or more relaxed criterion. The program will report on and process a
region surrounding any sample meeting this criterion.

\item For each region satisfying the threshold criterion the program
calculates the correlation between the samples and the detection profile of an
ideal point source. It uses
this to estimate source flux and background, the raison d'etre of the program.
It reports if the correlation found exceeds a chosen level. The default
level is 0.7 but you can change it if you wish.

\item The program has an option for removing fractional sample offsets, this
means that if the detectors do not have samples which correspond to identical
fields of view, the data can be adjusted to simulate identical fields of view.
A cubic spline interpolation is used. However the telescope pointing probably
is not known well enough to justify this procedure.

\item There are also options for printing out the raw data values and a
histogram of the noise.
\end{itemize}

\subsection{I\_COADD output}

The output is in the form of a file which can be printed. It consists of the
following sections
\begin{enumerate}
\item A list of the option values used in the processing
\item A list of the detectors crossing the coordinates of the assumed source.

For each scan the listing gives the source name, coordinates, band, SOP number,
crossing time for the first sample selected, the difference in  crossing time
between the first sample selected and the sample corresponding to the source
coordinates, the position of the source across the scan, and the
identifier numbers of the detectors which should see the source.

\item A list of the detector sample data.

For each detector the listing gives the SOP number, the detector number, the
position of the source across the detector the noise weighting value, and a set
of 20 sample values centered about the source position, the sample values being
those calculated after removal of the detector background level
(i.e.\ subtraction of the median value).

It also provides the weighted sum data values for each sample.

\item A summary of statistics for the expected source position.

The listing gives the sample position of the expected peak, an estimated noise
level, and for each of 6 samples either side of the expected peak it gives, the
summed weighted sample values after removal of the background, the fitted slope
value, and the fitted point source values. Note that the fitted slope value is
a useful indicator of whether the source is of the type expected, a binary
source with an abnormally wide profile will give an anomalous slope value.

It also gives the following statistics, the band number, the signal in
Jansky's, the signal in Watts per meter squared (times E-15), the signal to
noise ratio, the baseline signal in Watts per meter squared (times E-15),
the correlation coefficient between the signal and the ideal detection profile,
and the conversion factor between Jansky's and Watts.

\item An optional noise histogram can also be printed

\end{enumerate}
{\large Examples of the output are to be found in the Appendix }

\subsection{An example of using I\_COADD}

\begin{terminalv}
$ I_COADD

FILTER SNR threshold,            SNR  = 1.0
FILTER correlation threshold     CORL = 0.7
Detector width extension(arcmin) DZ   = 0.0
Detector samples noise weighted
Detector samples Printed
Detector samples Not interpolated to
remove fractional offsets

Are defaults to be changed(y/n)?
ANS:= N
\end{terminalv}
If you decide to alter the defaults and answer \texttt{Y} to the question above
you will get the following prompts:-

\begin{terminalv}
ANS:= Y
SNR/1.0/:= <RETURN>
CORL/0.7/:= <RETURN>
DZ/0.0/:=   <RETURN>
Is data interpolation required(y/n)?
ANS:= N
Is noise weighting required(y/n)?
ANS:= Y
Are raw data values to be printed(y/n)?
ANS:= Y
Is a noise histogram required(y/n)?
ANS:= N
\end{terminalv}
Whether you accept or change the default values the following prompts will be
printed
\begin{terminalv}
Name of CRDD file?
CRDDF:= N6543_B1S1
N6543       BAND 1 SOP: 546 UTC1ST: 12345678.123

Is another CRDD file to be input(y/n)?
ANS:= N
\end{terminalv}
N terminates requests for CRDD files for this source.

\begin{terminalv}
Is another source to be processed(y/n)?
ANS:= N
\end{terminalv}
N  terminates  program  run. Y causes requests for CRDD files for another source.

\begin{terminalv}
Output is in file COADOP.LIS
\end{terminalv}

\pagebreak

\section{Appendix}

\setlength{\unitlength}{1mm}
\begin{picture}(170,230)
\put (0,200){Figure 1}
\put (0,93){Figure 2}
\end{picture}

\begin{picture}(170,250)
\put (70,250){Figure 3}
\put (70,160){Fig 4}
\put (70,70){Figure 5}
\put (120,80){Figure 6}
\end{picture}

\begin{center}
Figure 7
\end{center}

\newpage

\begin{center}
Figure 8
\end{center}

\newpage

\begin{center}
Figure 9
\end{center}

\newpage

\begin{center}
Figure 10
\end{center}

\newpage

\begin{center}
Figure 11 (part 1)
\end{center}

\newpage

\begin{center}
Figure 11 (part 2)
\end{center}

\end {document}
