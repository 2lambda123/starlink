\documentstyle{article} 
\pagestyle{myheadings}

%------------------------------------------------------------------------------
\newcommand{\stardoccategory}  {Starlink User Note}
\newcommand{\stardocinitials}  {SUN}
\newcommand{\stardocnumber}    {122.1}
\newcommand{\stardocauthors}   {C A Clayton}
\newcommand{\stardocdate}      {11 March 1991}
\newcommand{\stardoctitle}     {Colour hard-copy from workstation screens}
%------------------------------------------------------------------------------

\newcommand{\stardocname}{\stardocinitials /\stardocnumber}
\markright{\stardocname}
\setlength{\textwidth}{160mm}
\setlength{\textheight}{240mm}
\setlength{\topmargin}{-5mm}
\setlength{\oddsidemargin}{0mm}
\setlength{\evensidemargin}{0mm}
\setlength{\parindent}{0mm}
\setlength{\parskip}{\medskipamount}
\setlength{\unitlength}{1mm}

\begin{document}
\thispagestyle{empty}
SCIENCE \& ENGINEERING RESEARCH COUNCIL \hfill \stardocname\\
RUTHERFORD APPLETON LABORATORY\\
{\large\bf Starlink Project\\}
{\large\bf \stardoccategory\ \stardocnumber}
\begin{flushright}
\stardocauthors\\
\stardocdate
\end{flushright}
\vspace{-4mm}
\rule{\textwidth}{0.5mm}
\vspace{5mm}
\begin{center}
{\Large\bf \stardoctitle}
\end{center}
\vspace{5mm}

\section{Introduction}

It is possible to produce a colour print on the DEC LJ250 inkjet printer of
either the entire screen or a portion of the screen from VAXstations,
DECstations, SUN workstations and the IKON image display.  This document
describes how to achieve this with each of the above workstations. The {\tt
IKONPAINT} software which is used to produce colour hard-copy from the IKON
screen on the inkjet printer is fully documented in SUN/71 and is not described
here.


\section{VAXstation running DECwindows (VMS 5.4 and after)}


DECwindows's Print Screen feature allows you to take a snapshot of all or part
of your screen and print the file containing the snapshot immediately or later.
However, first you must set up your personal Print Screen settings.

\begin{itemize}

\item Select the {\tt Print Screen...} option from the Session Manager's
{\tt Customize} menu.

\item Select {\tt Sixel} as the {\tt Output Format} and {\tt Color} as the 
{\tt Output Color}. Note that colour PostScript output is not currently 
supported.

\item Select other options as required, e.g. {\tt Output File name, Prompt for
File Name}, {\it etc.}

\item Click on {\tt Apply} and {\tt OK} to complete selection of print settings.

\item Select {\tt Save Current Settings} from the Session Manager's 
{\tt Customize }
menu.

\end{itemize}

In order to print part of your screen:

\begin{itemize}

\item Select {\tt Capture Portion of Screen} from the Session Manager's
{\tt Print Screen} menu. This option is recommended over the other three options
because:

\begin{enumerate} 

\item The screen is frozen while the screen is being captured to file. Hence it
is quicker to capture just that part of the screen that you require.

\item Capturing the image to a file rather than sending it directly to the 
inkjet printer has the advantage that you can print further copies later
if you are happy with the dump.

\end{enumerate}

\item Select file name for the captured Sixel image and click on OK. This step
will not occur if you have de-selected the {\tt Prompt for File Name} option on
the Customize Print Screen dialog box.

\item When the pointer changes to a `+' cursor, move the cursor to the upper
left corner of the area that you wish to capture. 

\item Press and hold down the left-hand mouse button. A box will appear. Drag
the cursor until the box surrounds the area that you want to capture and 
then release the mouse button.

\end{itemize}

The captured Sixel image can be printed on your DEC LJ250 Inkjet printer with 
a normal {\tt PRINT} command. For example:

\begin{quote}

{\tt \$ PRINT/QUE=inkjet-queue-name filename}

\end{quote}

You should first check that the inkjet printer is in DEC mode, i.e. the orange
light on the control panel should be lit. Your system manager may have set your
inkjet print queue up so this occurs automatically when you print a job. The
printer can be placed in DEC mode by pressing the DEC/PCL button until the
orange light appears. Consult your Site Manager if you are unsure which 
queue to send your print job to.

\section{DECstation running DECwindows (UWS 4.0 and after)}

The procedure for screen printing from a DECstation is almost exactly the same
as outlined above for the VAXstation. However, your will have ask your Site
Manager how to get your Sixel file printed on the LJ250. He/she may have
set up a queue on the DECstation pointing to the LJ250 or you may have
first to copy the file containing the captured image to the VAX disks either
via NFS or using FTP and print it from the VAXcluster as outlined above.

\section{SUN workstation running OpenWindows (Version 2.0 or later)}

Dumping the screen from the SUN to the DEC inkjet printer is rather more
awkward but the  quality of the output is very high. The procedure is as
follows:

\begin{itemize}

\item Start up the {\tt snapshot} X-windows-based tool, either via 
the default OpenWindows
program menu or via the command line:

\begin{quote}

{\tt \# \$OPENWINHOME/bin/xview/snapshot \&}

\end{quote}

{\tt Snapshot} has several options including snapping the whole screen, a
single window or a user defined region. {\tt Snapshot} also has an option to view a snapshot image file
after capture. The {\tt snapshot} tool is self-explanatory but a full
description of it can be found in the OpenWindows documentation (DeskSet
Environment Reference Guide).
    
\item {\tt Snapshot} produces a SUN
rasterfile. Convert the SUN rasterfile into a PostScript file using the 
undocumented command {\tt ras2ps}:
    
\begin{quote}

    {\tt \# \$OPENWINHOME/bin/xview/ras2ps -C input-file output-file}

\end{quote}
    
the {\tt -C} option donates colour. There are a number of other 
command line qualifiers available. 
    
\begin{quote}
{\tt 

-x\# -y\#   origin in inches from lower left corner

-X\# -Y\#   scale of image

-w\# -h\#   width/height of image, in inches

-r\#       rotate image

-i        invert image

-C        produce Colour PostScript

-l        landscape mode

-n        don't include showpage

-v        verbose mode

-q        query - gives list of optional qualifiers for ras2ps

}
\end{quote}

\item Copy the PostScript file to a VAX disk either via NFS  or using FTP. You
can give {\tt ras2ps} an NFS served VMS file system filename as the output file
name so that the output goes directly to a VAX disk. In this case you can skip
this copying stage. Your Site Manager will know the best method at your site 
of transferring the file to the VAX disks.
    
\item Log onto the VAXcluster and print the PostScript file on your inkjet
printer with the command

\begin{quote}
    
    {\tt \$ PSPRINT/QUE=inkjet-queue-name filename}
    
\end{quote}

Note that you must use the command {\tt PSPRINT} and not {\tt PRINT} because
you are printing a PostScript file. Consult your Site Manager if you are unsure 
which queue to send the print job to.


\end{itemize}

This procedure may seem complicated but the quality of the final print is very
good -- far better than that of the VAXstation Sixel dumps.
The image capture and {\tt ras2ps} stages 
are quick and the whole process is dominated by the time taken for {\tt
PSPRINT} to convert the PostScript file to Sixel format. Temporary disk space
usage can be high, up to 5000 blocks for single large window and 13000 blocks 
if you capture the whole screen. 
    
\end{document}
