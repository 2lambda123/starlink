\documentstyle[11pt]{article}
\pagestyle{myheadings}

%------------------------------------------------------------------------------
\newcommand{\stardoccategory}  {Starlink User Note}
\newcommand{\stardocinitials}  {SUN}
\newcommand{\stardocnumber}    {147.1}
\newcommand{\stardocauthors}   {M J  Bly}
\newcommand{\stardocdate}      {5 May 1992}
\newcommand{\stardoctitle}     {TAUCAL --- TAURUS data reduction}
%------------------------------------------------------------------------------

\newcommand{\stardocname}{\stardocinitials /\stardocnumber}
\renewcommand{\_}{{\tt\char'137}}     % re-centres the underscore
\markright{\stardocname}
\setlength{\textwidth}{160mm}
\setlength{\textheight}{230mm}
\setlength{\topmargin}{-2mm}
\setlength{\oddsidemargin}{0mm}
\setlength{\evensidemargin}{0mm}
\setlength{\parindent}{0mm}
\setlength{\parskip}{\medskipamount}
\setlength{\unitlength}{1mm}

%------------------------------------------------------------------------------
% Add any \newcommand or \newenvironment commands here
%------------------------------------------------------------------------------

\begin{document}
\thispagestyle{empty}
SCIENCE \& ENGINEERING RESEARCH COUNCIL \hfill \stardocname\\
RUTHERFORD APPLETON LABORATORY\\
{\large\bf Starlink Project\\}
{\large\bf \stardoccategory\ \stardocnumber}
\begin{flushright}
\stardocauthors\\
\stardocdate
\end{flushright}
\vspace{-4mm}
\rule{\textwidth}{0.5mm}
\vspace{5mm}
\begin{center}
{\Large\bf \stardoctitle}
\end{center}
\vspace{5mm}

%------------------------------------------------------------------------------
%  Add this part if you want a table of contents
%  \setlength{\parskip}{0mm}
%  \tableofcontents
%  \setlength{\parskip}{\medskipamount}
%  \markright{\stardocname}
%------------------------------------------------------------------------------

\section{Introduction}

TAUCAL is an add-on set of applications to the FIGARO package, which are
designed to be used for TAURUS data reduction. TAURUS is the instrument on the
William Herschel Telescope at Roque de los Muchachos Observatory on La Palma.
TAUCAL was written by Jim Lewis (RGO, Cambridge).

\section{Getting Started}

TAUCAL is an add-on to FIGARO so all you need to do to make the TAUCAL
applications available is start FIGARO:

\begin{verbatim}
      $ ADAMSTART
      $ FIGARO
\end{verbatim}

\section{More information}

This note is just a basic introduction to TAUCAL to tell you how to get the
applications going. For more information on the theory and use of the TAUCAL
package, refer to the TAUCAL user guide --- {\em ``TAURUS Data and How to
Reduce it''}\/,\ by Jim Lewis and Steve Unger. Copies are available from Site
Managers. The \LaTeX\ source in on-line in the directory {\tt
FIGPACK\_DISK:[FIGPACK.TAURUS.DEV.DOCS]} as file {\tt TAURUS\_RED.TEX}.

On-line help is also available as an additional part of the FIGARO HELP system.
\end{document}
