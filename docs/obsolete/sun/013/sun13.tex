\documentstyle[11pt]{article} 
\pagestyle{myheadings}

%------------------------------------------------------------------------------
\newcommand{\stardoccategory}  {Starlink User Note}
\newcommand{\stardocinitials}  {SUN}
\newcommand{\stardocnumber}    {13.2}
\newcommand{\stardocauthors}   {P.M.\ Allan}
\newcommand{\stardocdate}      {8 April 1992}
\newcommand{\stardoctitle}     {\Large\bf ASURV --- Astronomical Survival 
Statistics}
%------------------------------------------------------------------------------

\newcommand{\stardocname}{\stardocinitials /\stardocnumber}
\renewcommand{\_}{{\tt\char'137}}     % re-centres the underscore
\markright{\stardocname}
\setlength{\textwidth}{160mm}
\setlength{\textheight}{230mm}
\setlength{\topmargin}{-2mm}
\setlength{\oddsidemargin}{0mm}
\setlength{\evensidemargin}{0mm}
\setlength{\parindent}{0mm}
\setlength{\parskip}{\medskipamount}
\setlength{\unitlength}{1mm}

%------------------------------------------------------------------------------
% Add any \newcommand or \newenvironment commands here
%------------------------------------------------------------------------------

\begin{document}
\thispagestyle{empty}
SCIENCE \& ENGINEERING RESEARCH COUNCIL \hfill \stardocname\\
RUTHERFORD APPLETON LABORATORY\\
{\large\bf Starlink Project\\}
{\large\bf \stardoccategory\ \stardocnumber}
\begin{flushright}
\stardocauthors\\
\stardocdate
\end{flushright}
\vspace{-4mm}
\rule{\textwidth}{0.5mm}
\vspace{5mm}
\begin{center}
{\Large\bf \stardoctitle}
\end{center}
\vspace{5mm}

%------------------------------------------------------------------------------
%  Add this part if you want a table of contents
%  \setlength{\parskip}{0mm}
%  \tableofcontents
%  \setlength{\parskip}{\medskipamount}
%  \markright{\stardocname}
%------------------------------------------------------------------------------


\section{Introduction}

Astronomers often have to contend with data that is incomplete in the sense
that it contains upper or lower limits. This is known as censored data. The
term `survival statistics' comes from their use in testing objects to find
their average lifetime without actually testing all the objects to destruction.
ASURV is a program that performs statistical calculations on such data. The
program has been written by Takashi Isobe, Michael LaValley and Eric Feigelson
from the Department of Astronomy, Pennsylvania State University, and is based
on methods outlined in two papers in the Astrophysical Journal
\cite{isobe-1,isobe-2}. Anyone wishing to use the program should read the
document describing the program and consult these two papers for the
description of the statistical tests and how they should be applied to
astronomical data. If you use ASURV, please cite the papers
\cite{isobe-1,isobe-2} in your publications.

The program is being released exactly as it was received from the authors. The
user interface is not particularly attractive in that there is a limit of nine
characters on the length of file names, the amount of introductory information
is rather verbose and all output is in upper case. Also censored data are
specified by a flag of -1 or +1 rather than the more obvious $<$ and $>$. 

In the future, there may be an ADAM program that provides statistical analysis
of data with upper limits, has a friendly user interface, and reads data from
SCAR or CHI files as well as simple text files. Although the user interface of
an ADAM program that handles censored data will be radically different from
ASURV, there are no plans to alter the statistical tests at all. 

Any suggestions for other modifications to the user interface should be send to
the person maintaining the software for Starlink. Suggestion for other
modifications, such as providing addition statistical tests, will have to be
passed on to the authors, or will have to be done yourself. The program is
currently supported by Peter Allan at RAL. Bug reports should be reported to
the Starlink Software Librarian (RLVAD::STAR), possibly via your Site Manager. 

\subsection{Other documentation}

The ASURV program is described more thoroughly in the document `ASURV:
Astronomical SURVival Analysis - A Software Package for Statistical Analysis of
Astronomical Data containing Nondectections' \cite{mud-1}. A conference paper
`Censored Data in Astronomy' \cite{mud-2} also contains useful comments on the
use of survival statistics in astronomy. These are both Starlink miscellaneous
user documents (MUDs) and should be available at your site.

\subsection{Changes from version 0.0}

The current version of ASURV is version 1.1. The previous version of ASURV that
was released on Starlink was ASURV version 0.0. Some of the statistical tests
available in version 1.1 are significantly different from version 0.0. For
details, see reference \cite{mud-1}.

If you have a copy of ASURV version 1.0, then it should be replaced with
version 1.1 as version 1.0 contains known bugs. It was never distributed via
Starlink.

\section {Using the program} 

To run the program, type:                                         

\begin{verbatim} 
      $ asurv
\end{verbatim} 

The command must be in lower case on Unix machines. You will be prompted for
the types of tests that you wish to perform on the data and for the name of the
file that contains the data. Currently, file names are restricted to nine
characters.

\subsection {Prompts and Responses}

When you start the program, it will print out two screens of introductory
messages (type carriage return to get the next screen each time) and it will
then ask you whether you have univariate or bivariate data. You should reply 1
for univariate data, 2 for bivariate data or 3 to exit the program. You will
then be offered several statistical calculation, depending on the type of data
that you have indicated.

\subsection {Statistical Tests}

The program will perform the following tests.
\begin{small}
\begin{verbatim}
      Univariate data:

         Distribution             Kaplan-Meier estimator

         Two-sample tests         Gehan test
                                  Logrank test
                                  Peto and Peto test
                                  Peto and Prentice test
      Bivariate data:

         Correlation              Cox proportional hazard model
                                  Generalized Kendall's Tau (BHK method)
                                  Generalized Spearman Rho

         Linear regression        EM algorithm with normal distribution
                                  Buckley-James method
                                  Schmitt method for dual censored data
\end{verbatim}
\end{small}

For further information on these statistical methods, see the references in the
papers listed at the end of this document.

\section{Examples}

There are three examples of using the ASURV program. These are stored in the
directory ASURV\_DIR on VMS and in /star\-/bin\-/examples\-/asurv on Unix. To
run them, you will need to copy the input files to your own directory due to
the nine character file name limitation. The input commands are stored in the
files {\bf testN.dat}, when N = 1,2,3 (i.e.\ ASURV\_DIR:\-TESTn.DAT on VMS;
/star\-/bin\-/examples\-/asurv\-/testN.dat on Unix). The data files are called
{\bf galN.dat} (ASURV\_DIR:\-GALn.DAT on VMS;
/star\-/bin\-/examples\-/asurv\-/galN.dat on Unix), and the output from the
program should be the same as the contents of the files {\bf testN.lis}
(ASURV\_DIR:\-TESTn.LIS on VMS; /star\-/bin\-/examples\-/asurv\-/testN.lis on
Unix). Test~1 is an example of calculating the Kaplan-Meier estimator for
univariate data, test~2 is an example of the two sample test for univariate
data and test~3 is an example of correlation and regression for bivariate data.
See \cite{mud-1} for further details of these tests.

\subsection{Test 1}

To run test~1, type the following:

\begin{verbatim}
         On VMS                                On Unix

$ COPY ASURV_DIR:TEST1.DAT *.*        % cp /star/bin/examples/asurv/test1.dat .
$ COPY ASURV_DIR:GAL1.DAT *.*         % cp /star/bin/examples/asurv/gal1.dat .
$ ASURV                               % asurv
<carriage return>                     <carriage return>
<carriage return>                     <carriage return>
1                                     1
1                                     1
Y                                     y
TEST1.DAT                             test1.dat
N                                     n
\end{verbatim}

This will create an output file called {\bf gal1.out} which should be identical
with the file ASURV\-\_DIR:\-TEST1.LIS on VMS and
/star\-/bin\-/examples\-/asurv\-/test1.lis on Unix.

\subsection{Test 2}

To run test~2, type the following.

\begin{verbatim}
         On VMS                                On Unix

$ COPY ASURV_DIR:TEST2.DAT *.*        % cp /star/bin/examples/asurv/test2.dat .
$ COPY ASURV_DIR:GAL2.DAT *.*         % cp /star/bin/examples/asurv/gal2.dat .
$ ASURV                               % asurv
<carriage return>                     <carriage control>
<carriage return>                     <carriage control>
1                                     1
2                                     2
Y                                     y
TEST2.DAT                             test2.dat
N                                     n
\end{verbatim}

This will create an output file called {\bf gal2.out} which should be identical
with the file ASURV\-\_DIR:\-TEST2.LIS on VMS and
/star\-/bin\-/examples\-/asurv\-/test2.lis on Unix.

\subsection{Test 3}

To run test~3, type the following.

\begin{verbatim}
         On VMS                                On Unix

$ COPY ASURV_DIR:TEST3.DAT *.*        % cp /star/bin/examples/asurv/test3.dat .
$ COPY ASURV_DIR:GAL3.DAT *.*         % cp /star/bin/examples/asurv/gal3.dat .
$ ASURV                               % asurv
<carriage return>                     <carriage control>
<carriage return>                     <carriage control>
2                                     2
Y                                     y
TEST3.DAT                             test3.dat
N                                     n
\end{verbatim}

This will create an output file called {\bf gal3.out} which should be identical
with the file ASURV\-\_DIR:\-TEST3.LIS on VMS and
/star\-/bin\-/examples\-/asurv\-/test3.lis on Unix.

\begin{thebibliography}{9}
\bibitem{isobe-1} Isobe, T.\ \& Feigelson, E., 1985.\ Astrophys.J.,{\bf 293},
192.
\bibitem{isobe-2} Isobe, T.\ \& Feigelson, E., 1986.\ Astrophys.J.,{\bf 306},
490.
\bibitem{mud-1} Isobe, T.,\ LaValley, M.\ \& Feigelson, E., `ASURV:
Astronomical SURVival Analysis - A Software Package for Statistical Analysis of
Astronomical Data Containing Nondetections'. This document is available as a
Starlink Miscellaneous User Document. It supercedes the following document.
\bibitem{mud-1-old} Isobe, T.\ \& Feigelson, E., `ASURV: A
Software Package for Statistical Analysis of Astronomical Data with Upper
Limits'. This document is available as a Starlink Miscellaneous User Document,
or directly from the Department of Astronomy at Pennsylvania State University.
\bibitem{mud-2} `Censored Data in Astronomy' -- a Starlink Miscellaneous User
Document.
\end{thebibliography}

\end{document}
