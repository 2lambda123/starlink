\documentstyle[11pt]{article} 
\pagestyle{myheadings}

%------------------------------------------------------------------------------
\newcommand{\stardoccategory}  {Starlink User Note}
\newcommand{\stardocinitials}  {SUN}
\newcommand{\stardocnumber}    {155.1}
\newcommand{\stardocauthors}   {V. Laspias}
\newcommand{\stardocdate}      {20 October 1992}
\newcommand{\stardoctitle}     {TEXLSE --- An LSE Environment for \LaTeX }
%------------------------------------------------------------------------------

\newcommand{\stardocname}{\stardocinitials /\stardocnumber}
\renewcommand{\_}{{\tt\char'137}}     % re-centres the underscore
\markright{\stardocname}
\setlength{\textwidth}{160mm}
\setlength{\textheight}{230mm}
\setlength{\topmargin}{-2mm}
\setlength{\oddsidemargin}{0mm}
\setlength{\evensidemargin}{0mm}
\setlength{\parindent}{0mm}
\setlength{\parskip}{\medskipamount}
\setlength{\unitlength}{1mm}

%------------------------------------------------------------------------------
% Add any \newcommand or \newenvironment commands here

\newcommand{\BibTeX}{{\rm B\kern-0.05em{\sc i\kern-0.025em b}\kern-0.08em
    T\kern-0.1667em\lower0.7ex\hbox{E}\kern-0.125emX}}

\newcommand{\SLiTeX}{{\rm S\kern-0.06em{\sc l\kern-0.035emi}\kern-0.06em
    T\kern-0.1667em\lower0.7ex\hbox{E}\kern-0.125emX}}

\hfuzz=5pt
\newcommand\bs{\char '134 }  % A backslash character for \tt font
\newcommand{\prompt}{{\tt \$~}}
\newcommand{\gold}{\fbox{\sc gold} }
\newcommand{\help}{\fbox{\sc help} }
\newcommand{\return}{\fbox{\sc return} }
\newcommand{\ctrl}[1]{\fbox{{\sc ctrl/#1}} }
\newcommand{\kp}[1]{\fbox{{\sc kp} #1} }
\newcommand{\us}{\char '137 } % An underscore character for \tt font
\newcommand\lb{\char '173 }  % A left brace character for \tt font
\newcommand\rb{\char '175 }  % A right brace character for \tt font
\newcommand{\vbar}{\char '174 } % A vertical bar character for \tt font
\newcommand{\tlde}{\char '176 } % A tilde for \tt font
%
\def\descriptionlabel#1{\tt #1\ \hfil}
\makeatletter
\settowidth{\labelwidth}{\tt another\_bibitem}
\addtolength{\labelwidth}{\labelsep}
\def\description{\list{}{\leftmargin\labelwidth \advance\leftmargin\labelsep 
  \itemindent\z@ \let\makelabel\descriptionlabel}}
\makeatother

%------------------------------------------------------------------------------

\begin{document}
\thispagestyle{empty}
SCIENCE \& ENGINEERING RESEARCH COUNCIL \hfill \stardocname\\
RUTHERFORD APPLETON LABORATORY\\
{\large\bf Starlink Project\\}
{\large\bf \stardoccategory\ \stardocnumber}
\begin{flushright}
\stardocauthors\\
\stardocdate
\end{flushright}
\vspace{-4mm}
\rule{\textwidth}{0.5mm}
\vspace{5mm}
\begin{center}
{\Large\bf \stardoctitle}
\end{center}
\vspace{5mm}

{\bf NOTE:}~{\em This guide provides a brief summary of the commands and
functions available with {\sl LSEDIT\/} and the \LaTeX\ environment on
Starlink. TEXLSE is part of the LTEX software item which an optional addition
to Starlink \TeX . If LTEX is not installed at your site, see your Site
Manager.}

%------------------------------------------------------------------------------
%  Add this part if you want a table of contents
\setlength{\parskip}{0mm}
\tableofcontents
\setlength{\parskip}{\medskipamount}
\markright{\stardocname}
%------------------------------------------------------------------------------

\newpage

\section*{Scope and Intent of this Guide}

This guide presents information about the Language-Sensitive Editor ({\sl
LSEDIT\/}) and the environment defined for \LaTeX.  It is a supplement to the
{\em ``VAX Language-Sensitive Editor User's Guide''\/} \cite{lse}, the 
{\em ``\/\LaTeX\, A Document Preparation System, User's Guide \& Reference
Manual''\/} \cite{latex} and 
{\em ``\/\LaTeX\ ---A Document Preparation System'' \/} \cite{sun9}.

\bigskip

This guide acts as a summary and memory refresher for the commands and
functions covered in the manuals mentioned above. It is {\em not\/} intended to
replace either of these manuals.  It should supplement these documents and
provide information that is specific to this environment. 

\section*{Conventions}

Various symbols and syntax conventions are used throughout this guide.

\begin{center}
\begin{tabular}{l@{\hspace{.5in}}p{2.5in}}
\multicolumn{1}{c}{\bf Symbol} & \multicolumn{1}{c}{\bf Meaning} \\ \hline
\\
\prompt & System prompt \\ [5pt]
\return & Return key \\ [5pt]
\ctrl{x} & Control key + main keyboard key \\ [5pt]
\gold & PF1 key \\ [5pt]
\help & PF2 key \\ [5pt]
\kp{0} \ldots & Numeric keypad key \\
\end{tabular}
\end{center}

\newpage
\section{Introduction}

The Language-Sensitive Editor, hereafter referred to as {\sl LSEDIT\/}, is an
interactive text editor designed specifically for software development. It
assists the user in the quick and accurate development of syntactically correct
``source'' code in a variety of languages.

In this case, \LaTeX\ or \SLiTeX{} are the ``languages'' being used.

\section{{\sl LSEDIT\/} Definitions}

{\bf Placeholders} are optional or required constructs which are usually
inserted into the file you are editing as a part of a template. Placeholders
represent places in the source code where the user must provide additional
information. In some cases, a placeholder may be expanded to provide a template
for this additional text.

{\bf Tokens} are keywords that may be typed anywhere into the file you are
editing and expanded to provide a template for a corresponding construct.

For a complete list of the placeholders and tokens that are defined in the
\LaTeX\ environment, see Appendices~\ref{place} and~\ref{token},
respectively. 

\section{{\sl LSEDIT\/} Commands}

The following key commands are the most commonly used. They include:

\begin{center}
\begin{tabular}{l@{\hspace{.5in}}l}
\multicolumn{1}{c}{\bf Command} & \multicolumn{1}{c}{\bf Meaning} \\
\hline
\ & \  \\ 
\ctrl{e} & Expand the current placeholder or token \\ [5pt]
\gold + \ctrl{e} & Un-expand the current placeholder or token \\ [5pt]
\ctrl{k} & Delete the current placeholder \\ [5pt]
\gold + \ctrl{k} & Un-delete the current placeholder \\ [5pt]
\ctrl{n} & Go to the next placeholder \\ [5pt]
\ctrl{p} & Go to the previous placeholder \\ [5pt]
\gold + \help & Get help for the current placeholder or token \\
\end{tabular}
\end{center}

\section{Conventions within the \LaTeX\ Environment}

There are two types of placeholders; {\sl required\/}, and {\sl
optional\/}.  They are denoted within the \LaTeX\ environment as follows:

\begin{center}
\begin{tabular}{l@{\hspace{1in}}l}
\multicolumn{1}{c}{\bf Symbol(s)} & \multicolumn{1}{c}{\bf Meaning} \\
\hline
\ & \   \\
\verb"< >" & Single required placeholder \\
\verb"< >..." & List of required placeholders \\
\verb"<< >>" & Single optional placeholder \\
\verb"<< >>..." & List of optional placeholders \\
\end{tabular}
\end{center}

\section{Invoking {\sl LSEDIT\/}}

\subsection{Accessing the \LaTeX\ Environment} 
 
You can invoke LSEDIT and access the \LaTeX\ environment in one of two ways:

\begin{enumerate}
\item You can use the {\tt /ENVIRONMENT} qualifier on the command line. For
example,
\begin{quote}
\tt
\prompt{} LSEDIT/ENVIRONMENT=LTEX\_LSE:LATEX.ENV {\it filename\/}.TEX \hfil
\end{quote}
where {\it filename\/} is the name of your input file.  The file type {\bf 
must} be {\tt .TEX} or {\tt .STY}.

\item You can define a logical name {\tt LSE\$ENVIRONMENT} in your LOGIN.COM:
\begin{quote}
\tt
\prompt{} DEFINE LSE\$ENVIRONMENT LTEX\_LSE:LATEX.ENV \hfil
\end{quote}
Then you may omit the {\tt /ENVIRONMENT} qualifier from the command line when
you invoke LSEDIT. This logical may be a search list.
\end{enumerate}

Either method will accomplish the same result, although the second method will
reduce the number of keystrokes and will assure that you don't forget the
qualifier when you enter LSEDIT.

\subsection{Accessing the LSEKEYS Section File}

In the \verb|LTEX_LSE| directory, there is a file called \verb|LSEKEYS.GBL|. 
This is an LSEDIT section file. It contains a number a very useful key
definitions that will make the task of editing files much simpler.

Like the \LaTeX\ environment, this section file can be accessed in one of two
ways:
\begin{enumerate}
\item You can use the {\tt /SECTION} qualifier on the command line. For
example,
\begin{quote}
\tt
\prompt{} LSEDIT/SECTION=LTEX\_LSE:LSEKEYS.GBL {\it filename\/}.TEX \hfil
\end{quote}

\item You can define a logical name {\tt LSE\$SECTION} in your LOGIN.COM:
\begin{quote}
\tt
\prompt{} DEFINE LSE\$SECTION LTEX\_LSE:LSEKEYS.GBL \hfil
\end{quote}
Then you may omit the {\tt /SECTION} qualifier from the command line when
you invoke LSEDIT. 
\end{enumerate}

The two examples above assume that the logical name \verb|LSE$ENVIRONMENT| has
been defined.  

\subsection{Easier Access to the \LaTeX\ Environment}

To invoke {\sl LSEDIT\/} for use with \LaTeX\ source files, use the 
following command:
\begin{quote}
\tt
\prompt{} EDLATEX {\it filename\/}.TEX \hfill
\end{quote}

where {\it file\/} is the name of your input file.  The file type {\bf must} be
{\tt .TEX} or {\tt .STY}. The command defines the correct environment and
editing keys for operation with \LaTeX\ source files. 
The {\tt EDLATEX} command is a {\sl DCL\/} symbol defined as

\begin{quote}
\tt
LSEDIT/ENVIRONMENT=LTEX\_LSE:LATEX.ENV/SECTION=LTEX\_LSE:LSEKEYS.GBL \hfill
\end{quote}

The editing keys for use with {\sl LSEDIT\/} are defined in the
section file {\tt LTEX\_LSE:LSEKEYS.GBL}. 

\section{Supported \LaTeX\ Document Styles}

When you invoke {\sl LSEDIT\/} to create a new \LaTeX\ file, the screen will
have the intial placeholder \verb"<Latex>" on the first line of the buffer.
When you expand this placeholder by pressing \ctrl{e}, a menu will be displayed
listing the different \LaTeX\ styles that are supported by the environment. 
The document styles supported are: 

\begin{itemize}
\item {\tt article} -- standard style 
\item {\tt report} -- standard style
\item {\tt letter} -- standard style
\item {\tt memo} -- Inter-Office Correspondence
\item {\tt slides} -- \SLiTeX{}
\item {\tt milstd} -- MIL-STD-490 compatible style
\item {\tt bookform} -- bookform style
\item {\tt laa} -- Astronomy and Astrophysics style
\item {\tt laamt} -- Astronomy and Astrophysics style with Monotype Times fonts
\item {\tt thesis} -- Thesis document style
\end{itemize}

The style is chosen by using the arrow keys to select the desired style and
pressing \mbox{ \return.} See Appendix~\ref{templates} for the style templates 
created when the desired style is selected. Further styles may be added in the
future.
 
\section{Running \LaTeX\ from within {\sl LSEDIT\/}}

Your \LaTeX\ source file can be processed through \LaTeX\ {\em without\/}
leaving {\sl LSEDIT\/} When you have finished editing, press \gold {\tt C}. (If
you are not using the {\tt LTEX\_LSE:LSEKEYS.GBL} section  file, press \gold +
\kp{7} In response to the \verb"LSE Command>" prompt, enter ``{\tt COMPILE}'').
{\sl LSEDIT\/} will write out your buffer and spawn a subprocess to run \LaTeX.
You needn't wait for that subprocess to complete. You may continue editing the
file if there is more to do, or you may choose to bring in another file and
edit it while you wait. You will receive a completion message when the
subprocess finishes. 


If \LaTeX\ detects errors, the subprocess will tell you it completed with
errors. To view those errors, you can press \gold {\tt I}  and respond ``{\tt
.LIS}'' to the {\tt \us Input file:} prompt. If the main buffer file is called
{\tt FILE.TEX}, then the file {\tt FILENAME.LIS} will be read into a buffer
called FILENAME.LIS (If you are not using the {\tt LTEX\_LSE:LSEKEYS.GBL}
section file, press \gold + \kp{7} In response to the  \verb"LSE Command>"
prompt, enter {\tt GOTO FILE/READONLY FILENAME.LIS}).   This file can then be
examined to determine the cause of the error. To return to the main buffer,
press \gold {\tt M}  (or press \gold + \kp{7} and enter {\tt GOTO BUFFER
FILENAME.TEX}). Correct the problem, and repeat the process. 

If no errors occur, the subprocess will report that the ``compilation'' is
complete. At this point, you can exit {\sl LSEDIT\/} and process the {\tt
FILE.DVI} file produced by \LaTeX.

\section{Accessing Help for Placeholders and Tokens}

On-line help is available for most of the placeholders and tokens defined
within the \LaTeX\ environment. 

For placeholders, position the cursor at the next placeholder by pressing 
\ctrl{n}\nolinebreak. Then press \gold + \help. If there is no help for this
placeholder, a message will be given stating that fact.

For tokens, type the token and press \gold + \help.

If the placeholder or token is a menu, you can access help for items in the
menu by pointing at the menu item with the arrow keys and then pressing \gold +
\help. 

\begin{thebibliography}{99}
\addcontentsline{toc}{section}{References}

\bibitem{lse} Digital Equipment Corporation, 
{\em VAX Languase-Sensitive Editor and VAX Source Code Analyser}
Order \#AA-PAJLA-TK, December 1989.

\bibitem{latex} Leslie Lamport {\em \LaTeX\ A Document Preparation System, 
User's Guide \& Reference Manual \/} Addison-Wesley Publishing Company,
ISBN~0-202-15790-X

\bibitem{sun9} V. Laspias \& M. Bly {\em \LaTeX\ --- A Document Preparation
System} \\
Starlink User Note 9, Starlink Project.

\end{thebibliography}

\newpage
\appendix
\section{\LaTeX\ Placeholders}\label{place}

The following placeholders are defined within the \LaTeX\ environment. A
placeholder is delimited by a single ``\verb"< >"'' pair (required) or a double
``\verb"<< >>"'' (optional) pair. 
\begin{small}
\begin{description}
\item[\bf Placeholder] {\bf Description}
\item[*\lb n\rb \lb \rb ]       Tabular environment repeat command
\item[0\vbar 1\vbar 2\vbar 3\vbar 4]    0 or 1 or 2 or 3 or 4
\item[11pt]         Typeset the document in 11 point type
\item[12pt]         Typeset the document in 12 point type
\item[@\lb \rb ]          @-expression for tabular preamble
\item[a]	    System Specification
\item[accents]      Accents: Non-math and math mode
\item[address]      Letter recipient address
\item[addressee]     Name of the letter recipient
\item[alph]          Alpha counter definitions, capital and lower case
\item[another\us bibitem]  \mbox{} \\
thebibliography \verb|\bibitem| command
\item[another\us item]  Environment \verb|\item| command
\item[another\us slide] More \SLiTeX{} slide templates
\item[array\us entry]   Array entry
\item[array\us pos]     Vertical positioning, center is default
\item[arrow\us symbols] Arrow symbols, see Table 3.6 of \LaTeX\ manual
\item[article]       \LaTeX\ article document style
\item[article\vbar report\vbar memo\vbar letter\vbar slides\vbar %
milstd\vbar bookform\vbar laa\vbar laamt\vbar thesis] \mbox{} \\ 
  Valid arguments for the \verb|\documentstyle| command
\item[b2]            Critical Item Development Spec
\item[b5]            Computer Program Development Spec
\item[bib\us files]     First names of the bibliography files, file type .BBL
\item[binary\us ops]    Binary operations symbols, see Table 3.4 of \LaTeX\ manual
\item[bookform]      Bookform document style
\item[box\us name]      \verb|\newsavebox|, \verb|\savebox|, and \verb|\usebox| box name
\item[c]             Center
\item[c5]            Computer Program Product Spec
\item[caption]       Figure and Table caption command
\item[cc]            Letter copies
\item[cite\us key]      String used to associate \verb|\bibitem| with \verb|\cite|
\item[colors]        \SLiTeX{} colors definition
\item[color\us or\us bw]   Select color slides or black and white slides
\item[column]        Tabular preamble column specifier
\item[column\us entry]  Tabular or Array column entry
\item[col\us pos]       Menu of valid arguments in the tabular preamble
\item[col\us range]     Column range specifier for the \verb|\cline| command
\item[commands]      \LaTeX\ command categories
\item[command\us definition]  \mbox{} \\ 
  \verb|\newcommand| and \verb|\renewcommand| new command definition
\item[copies\us list\us one\us column] \mbox{} \\ 
                    Convert memo `Copies' list to one column
\item[counter\us name]  Used by \verb|\setcounter|, \verb|\addtocounter|, and 
  \verb|\newcounter| commands
\item[dash\us size]     Size of the dashes in the \verb|\dashbox| command
\item[dedication]    Produce a dedication string in thesis document style
\item[degree]    Produce a degree string in thesis document style
\item[delimiters]    Math formula delimiters, see table 3.10 of \LaTeX\ manual
\item[delta\us x]       Delta x value used by the \verb|\multiput| picture command
\item[delta\us y]       Delta y value used by the \verb|\multiput| picture command
\item[denominator]   Fraction denominator
\item[department]    Department name or portion thereof
\item[deptlist]      Memo author's department or section
\item[diameter]      Value entry (no units) for picture commands
\item[dimension]     Length entry
\item[document\us type] MIL-STD-490 specification type
\item[draft]         Mark overfull hboxes
\item[encl]          Letter enclosures
\item[environments]  \LaTeX\ paragraph-making structures
\item[env\us name]      User-supplied environment name
\item[env\us star]      Star form of environments
\item[ext\us above\us height]  \mbox{} \\
The amount the text extends above the bottom of the line
\item[ext\us below\us height]  \mbox{} \\
The amount the text extends below the bottom of the line
\item[faculty]       Produce a faculty string in thesis document style
\item[file]          Auxiliary file type, {\tt .TOC}, {\tt .LOF}, or {\tt .LOT}
\item[fleqn]         Displayed math environments flush left
\item[fn\us number]     \verb|\footnote| and \verb|\footnotetext| optional footnote number argument
\item[font\us name]     Font name, e.g. \verb|cmr10 scaled\magstep2|
\item[formula]       Math formula
\item[framebox\us p\us placement] \mbox{} \\
                    Text placement within a \verb|\framebox| in a picture
\item[from\us list\us two\us columns] \mbox{} \\
                    Convert memo `From' list to two columns
\item[greek\us letters] Greek letter commands, math mode ONLY
\item[height]     \verb"\rule" required height parameter
\item[here\vbar top\vbar bottom\vbar float\us page] \mbox{} \\
                     Figure/table location: here, top, bottom, float page
\item[hline]         Tabular horizontal rule.
\item[institution]   Produce an institution string in thesis document style.
\item[item\us arg]      \verb|\item| macro with optional argument for list-making environments
\item[item\us label]    Optional label for \verb|\item| command
\item[key\us list]      List of citation keys of \verb|\bibitem| commands
\item[l]             Left justify
\item[label]         Cross referencing command
\item[latex]         \LaTeX\ grammar
\item[length]        Length value
\item[length\us name]   Length name identifier
\item[leqno]         Put equation numbers on the left
\item[letter]        \LaTeX\ letter document style
\item[linebreak]     Double backslash command
\item[lined]         Vertical line character for tabular environment
\item[location]      Figure and Table placement location
\item[lof]           List of Figures file
\item[loglike\us functions]  \mbox{} \\
   Log-like functions, see Table 3.9 of \LaTeX\ manual
\item[lot]           List of Tables file
\item[makebox\us p\us placement]  \mbox{} \\
\verb|\makebox| quadrant specifier for the picture environment
\item[math\us accents]  \mbox{} \\
Math mode accents, see table 3.11 of \LaTeX\ manual
\item[math\us misc]     Math mode miscellaneous
\item[memo]           \LaTeX\ memo document style
\item[memo\us signature] Memo signature options
\item[mil-date]      Military date
\item[mil-std-number] MIL-STD document number
\item[milstd]        MIL-STD-490 document style
\item[minipage\us pos]  Valid minipage positioning options
\item[misc\us symbols]  Miscellaneous symbols, see Table 3.7 of \LaTeX\ manual
\item[multicolumn]   Tabular command to span multiple columns
\item[name]          Memo individual name within the name list
\item[namelist]      Memo list of names
\item[newtheorem]    Create a new theorem environment
\item[non\us math\us accents] \mbox{} \\
  Non-math mode accents, see table 3.1 of \LaTeX\ manual
\item[no\us args]       \verb|\newcommand| and \verb|\renewcommand| optional arguments argument
\item[no\us col]        Number of columns to be spanned by a \verb|\multicolumn| command
\item[no\us obj]        Integer entry in the \verb|\multiput| command
\item[nth\us root]      Nth root
\item[number]        Integer entry
\item[numbered\us like] Name of a pre-defined theorem-like environment
\item[numerator]     Fraction numerator
\item[object\us p]      Picture environment commands
\item[offset\us height] Picture environment optional offset width
\item[offset\us width]  Picture environment optional offset width
\item[onlyslides]    \SLiTeX{} command to produce a subset of your slides
\item[optional]      Optional parameter for sectioning commands
\item[optional\us width] \verb|\makebox|, \verb|\framebox|, and 
  \verb|\savebox| optional width argument
\item[optionlist]    Optional \LaTeX\ document substyles
\item[options]       Valid substyle options, 10pt is the default for all styles
\item[origin\us offset] Optional picture environment offset
\item[oval\us height]   Height of an oval in the picture environment
\item[oval\us section]  Oval quadrant specifier for the picture environment
\item[oval\us width]    Width of an oval in the picture environment
\item[pagestyle]     Document page style commands
\item[paragraph\us text] Text followed by a \verb"\\"
\item[parbox\us pos]    Valid \verb|\parbox| positioning options
\item[percent\_sign]    Produce \% symbol
\item[pgno\us options]  Valid options for the \verb|\pagenumbering| command
\item[picture\us height]Picture environment width
\item[picture\us width] Picture environment width
\item[portrait]      \SLiTeX{} portrait optional argument
\item[position]      \verb|\makebox|, \verb|\framebox|, \verb|\shortstack|
   optional position argument
\item[pre\us commands]  Pre \verb|\begin{document}| commands
\item[pre\us letter\us cmds]Letter style commands
\item[priority]      \verb|\[no]linebreak| and \verb|\[no]pagebreak| priority
\item[prompted\us slide\us file] \mbox{} \\
 \verb|\typein| command for variable slide file input
\item[ps]	Letter postscript
\item[put\vbar multiput] Picture environment placement commands
\item[p\lb \rb ]          p-expression for tabular environment only
\item[quadrant]      Picture \verb|\makebox| and \verb|\oval| quadrant selection
\item[r]             Right justify
\item[raise\us height]  Height a \verb|\rule| will be raised above the line
\item[recipient]     Name and address of letter recipient
\item[referee]		Typeset document for referee in Astronomy \& 
Astrophysics style
\item[relation\us symbols]  \mbox{} \\
Relation symbols, see Table 3.5 of \LaTeX\ manual
\item[report]        \LaTeX\ report document style
\item[reset\us counter] Optional reset for the \verb|\newcounter| command
\item[right\vbar center\vbar left] \mbox{} \\
		     Right or center or left
\item[roman]         Roman counter definitions, capital and lower case
\item[root\us file]     \SLiTeX{} root file
\item[sections]      Valid sectioning commands
\item[sec\us unit]      Controls the format of the entry
\item[slidefile]     The name of your \SLiTeX{} slide file
\item[slides]        \SLiTeX
\item[slide\us colors]  \SLiTeX{} layer colors
\item[slide\us file]    \SLiTeX{} slides template
\item[slide\us fill]    Push the text of the slide to the top of the page
\item[slide\us number]  Slide number(s) entry for the \verb|\onlyslides| command
\item[slide\us title]   Slide title options
\item[star]          *-form of \LaTeX\ commands
\item[styles]        Predefined \verb|\pagestyle| and \verb|\thispagestyle| arguments
\item[subject]       Subject of an memo
\item[symbols]       Symbols: see table 3.2 in the \LaTeX\ manual
\item[tabbing\us column\us entry]  \mbox{} \\
                    Text for tabbing environment columns
\item[tabbing\us entry] Tabbing entry
\item[table\us entry]   Tabular entry
\item[tabular\us width]  \mbox{} \\
Width of a table made with {\tt tabular*} 
\item[tab\us set]       Set the tabs for the {\tt tabbing} environment
\item[text]          General text entry
\item[thesis]        Typeset document in thesis style
\item[toc]           Table of Contents file
\item[top\vbar bottom\vbar right\vbar left] \mbox{} \\
                    Top or bottom or right or left
\item[top\vbar center\vbar bottom]   \mbox{} \\
  Top or center or bottom
\item[to\us list\us two\us columns]  \mbox{} \\
   Convert memo `To' list to two columns
\item[twocolumn]     Typeset the document with two columns per page
\item[twoside]       Define twosided substyle
\item[type\us sizes]    Valid \LaTeX\ type size declaration
\item[type\us styles]   Valid \LaTeX\ type styles
\item[underlined]    Underlined slide title line
\item[user\us defined]  User defined slide title
\item[varsize\us symbols] Variable-sized symbols, see Table 3.8 of \LaTeX\ manual
\item[widest\us label]  Widest text for \verb|\bibitem| label
\item[width]         \verb|\parbox|, \verb|\rule|, \verb|\minipage| required 
  width parameter
\item[within]        Optional counter name, e.g. chapter, section
\item[x\us coord]       X coordinate for picture command
\item[y\us coord]       Y coordinate for a picture command
\item[\bs\bs\us or\us \bs kill]   Tabbing line terminator
\end{description}
\end{small}

\newpage
\section{\LaTeX\ Tokens}\label{token}

The following tokens are defined within the \LaTeX\ environment. Any token can
be typed on a blank line within the buffer and expanded with 
\ctrl{e}\nolinebreak. You need not type in the complete token name. For
example, you can type {\tt \bs set}, press \ctrl{e} and a menu of all tokens
that begin with {\tt \bs set} will be displayed. Simply select the proper
command using the arrow keys and press \return.  

\begin{small}
\begin{description}
\item[\bf Token] {\bf Description}
\item[accents]         Accents: Non-math and math mode
\item[array]           Multi-column array environment
\item[arrow\us symbols]   Arrow symbols, see Table 3.6 of \LaTeX\ manual
\item[article]         \LaTeX\ article document style
\item[begin]           Any other environment
\item[binary\us ops]      Binary operations symbols, see Table 3.4 of \LaTeX\ manual
\item[bookform]        Bookform document style
\item[center]          Paragraph centering environment
\item[copies\us list\us one\us column]  \mbox{} \\
                       Convert memo `Copies' list to one column
\item[delimiters]      Math formula delimiters, see table 3.10 of \LaTeX\ manual
\item[description]     Description list environment
\item[documentlist]    Document list environment
\item[enumerate]       Numbered list environment
\item[environments]    \LaTeX\ paragraph-making structures
\item[eqnarray]        Multi-line formula environment
\item[equation]        Mathematical equation environment
\item[example]         Example paragraph environment
\item[figure]          \LaTeX\ figure environment, NOT available in \SLiTeX
\item[flushleft]       Paragraph flush to the left margin environment
\item[flushright]      Paragraph flush to the right margin environment
\item[from\us list\us two\us columns] \mbox{} \\
                       Convert memo `From' list to two columns
\item[greek\us letters]   Greek letter commands, math mode ONLY
\item[itemize]         Bulleted list environment
\item[laa]      Astronomy \& Astrophysics style
\item[laamt]    Astronomy \& Astrophysics style with Monotype Times fonts
\item[letter]          \LaTeX\ letter document style, specific
\item[list]            Custom list making environment
\item[loglike\us functions] \mbox{} \\
   log-like functions, see Table 3.9 of \LaTeX\ manual
\item[math\us accents]    Math mode accents, see table 3.11 of \LaTeX\ manual
\item[math\us misc]       Math mode miscellaneous
\item[memo]             \LaTeX\ memo document style
\item[memo\us signature]   memo signature options
\item[milstd]          MIL-STD-490 document style
\item[minipage]        Minipage environment (box containing paragraph environments)
\item[misc\us symbols]    Miscellaneous symbols, see Table 3.7 of \LaTeX\ manual
\item[non\us math\us accents] \mbox{} \\
Non-math mode accents, see table 3.1 of \LaTeX\ manual
\item[note]            \SLiTeX{} note environment
\item[one\us signature]   Single signature for memo
\item[overlay]         \SLiTeX{} overlay environment
\item[picture]         Picture environment
\item[quotation]       Quotation with paragraphs indented environment
\item[quote]           Quotation without paragraphs indented environment
\item[relation\us symbols]  \mbox{} \\
Relation symbols, see Table 3.5 of \LaTeX\ manual
\item[report]          \LaTeX\ report document style
\item[root\us file]       \SLiTeX{} root file
\item[sections]        Valid sectioning commands
\item[slides]          \SLiTeX
\item[slide\us file]      \SLiTeX{} slides template
\item[symbols]         Symbols: see table 3.2 in the \LaTeX\ manual
\item[tabbing]         Tabbing environment
\item[table]           Floating table environment, NOT available in \SLiTeX
\item[tabular]         Table making environment
\item[thebibliography] Produce a bibliography list
\item[theorem]         Mathematical theorem environment
\item[thesis]          Thesis document style
\item[titlepage]       Titlepage environment
\item[to\us list\us two\us columns] \mbox{} \\ 
   Convert memo `To' list to two columns
\item[two\us signatures]  Double signature for memo
\item[type\us sizes]      Valid \LaTeX\ type size declaration
\item[type\us styles]     Valid \LaTeX\ type styles
\item[underlined]      Underlined slide title line
\item[varsize\us symbols] Variable-sized symbols, see Table 3.8 of \LaTeX\ manual
\item[verbatim]        Verbatim paragraph environment
\item[verse]           Verse environment (Think poetry!)
\item[\bs addcontentsline] \mbox{} \\
Adds an entry to the specified list or table
\item[\bs address]        Return address. If null, business letter is made.
\item[\bs addtocontents]  Adds text directly to the file that generates the
table of contents and lists of figures and tables. 
\item[\bs addtocounter]   Add a specified amount to an existing counter
\item[\bs addtolength]    Add a specified amount to an existing length command
\item[\bs addvspace]      Add vertical space between paragraphs
\item[\bs alph]           Error on /PLACEHOLDER in definition
\item[\bs alpha]          Greek letter alpha, lowercase
\item[\bs appendix]       Change the way sectional units are numbered
\item[\bs arabic]         Arabic counter definitions
\item[\bs author]         Author used by the \verb|\maketitle| command
\item[\bs begin]          Any other environment
\item[\bs beta]           Greek letter beta, lowercase
\item[\bs bf]             Bold
\item[\bs bibitem]        thebibliography \verb|\bibitem| command
\item[\bs bibliography]   Used with BiBTeX to create a bibliography
\item[\bs bigskip]        Vertical space of \verb|\bigskipamount|
\item[\bs blackandwhite]  \SLiTeX{} \verb|\blackandwhite| command
\item[\bs caption]        Figure and Table caption command
\item[\bs cc]             Letter copies
\item[\bs cdots]          Horizontal  ellipsis,  center of line, math mode ONLY
\item[\bs centering]      Declaration equivalent to the center environment
\item[\bs chapter]        \LaTeX\ sectioning command, report style ONLY
\item[\bs chi]            Greek letter chi, lowercase
\item[\bs circle\us p]       Picture environment circle making command
\item[\bs cite]           Generate an in-text citation
\item[\bs cleardoublepage] \mbox{} \\
Break the current page here and output any floats
\item[\bs clearpage]      Break the current page here and output any floats
\item[\bs cline]          Draw a horizontal line over the specified column range
\item[\bs closing]        Letter closing, e.g. `Sincerely,'
\item[\bs colors]         \SLiTeX{} colors definition
\item[\bs colorslides]    \verb|\colorslides| command
\item[\bs copyright]      Copyright symbol
\item[\bs dashbox\us p]      Picture environment \verb|\dashbox| command
\item[\bs date]           Declare document date, used by \verb|\maketitle|
   command
\item[\bs ddots]          Diagonal ellipsis, math mode ONLY
\item[\bs delta\us l]        Greek letter delta, lowercase
\item[\bs delta\us u]        Greek letter delta, uppercase
\item[\bs documentstyle]  \LaTeX\ document style selection command
\item[\bs dotfill]         Horizontal fill with dots
\item[\bs em]             Emphasis: toggle between roman and italics
\item[\bs encl]           Letter enclosures
\item[\bs epsilon]        Greek letter epsilon, lowercase
\item[\bs eta]            Greek letter eta, lowercase
\item[\bs fbox]           Short form of the \verb|\framebox| command
\item[\bs flushbottom]    Make all text pages the same height
\item[\bs fnsymbol]       Produce footnote symbols, math mode, counter $\leq$ 9
\item[\bs footnote]       Footnote generating command
\item[\bs footnotemark]   Mark a footnote, used in conjunction with 
  \verb|\footnotetext|
\item[\bs footnotesize]   2 sizes smaller than \verb|\normalsize|
\item[\bs footnotetext]   Footnote generating command, used with 
  \verb|\footnotemark|
\item[\bs frac]           Generate a fraction, math mode ONLY
\item[\bs framebox]       Frame a box as specified containing the text specified
\item[\bs framebox\us p]     Picture environment \verb|\framebox| command
\item[\bs frame\us p]        Picture environment framing command, similar to 
  \verb|\fbox|
\item[\bs gamma\us l]        Greek letter gamma, lowercase
\item[\bs gamma\us u]        Greek letter gamma, uppercase
\item[\bs hfill]          Add infinitely stretchable horizontal space
\item[\bs hline]          Tabular horizontal rule
\item[\bs hrulefill]      Horizontal fill with a line
\item[\bs hspace]         Insert horizontal space
\item[\bs huge\us 1]         4 sizes larger than \verb|\normalsize|
\item[\bs huge\us 2]         5 sizes larger than \verb|\normalsize|
\item[\bs hyphenation]    Hyphenation command
\item[\bs include]        Read the specified file
\item[\bs includeonly]    Define the files to be read by the \verb|\include| command
\item[\bs indent]         Add horizontal space in the amount of normal indentation
\item[\bs input]          Read the specified file into your document
\item[\bs invisible]      Produce `invisible' text, \SLiTeX{} ONLY
\item[\bs iota]           Greek letter iota, lowercase
\item[\bs it]             Italics
\item[\bs item]           Environment \verb|\item| command
\item[\bs kappa]          Greek letter kappa, lowercase
\item[\bs label]          Cross referencing command
\item[\bs lambda\us l]       Greek letter lambda, lowercase
\item[\bs lambda\us u]       Greek letter lambda, uppercase
\item[\bs large\us 1]        1 size larger than \verb|\normalsize|
\item[\bs large\us 2]        2 sizes larger than \verb|\normalsize|
\item[\bs large\us 3]        3 sizes larger than \verb|\normalsize|
\item[\bs latex]          Produce \LaTeX\ logo
\item[\bs ldots]          Horizontal ellipsis, bottom of line
\item[\bs left]           Left delimiter command for math formula
\item[\bs linebreak]      Break the current line
\item[\bs linethickness]  Set the thickness of lines drawn in the picture environment
\item[\bs line\us p]         Picture environment \verb|\line| command
\item[\bs listoffigures]  Generate a list of figures
\item[\bs listoftables]   Generate a list of tables
\item[\bs location]       Letter location (e.g. M/S 121)
\item[\bs makebox]        Generate a box as specified containing the text specified
\item[\bs makebox\us p]      Picture environment \verb|\makebox| command
\item[\bs maketitle]      Generate a title page
\item[\bs marginpar]      Produce a marginal note
\item[\bs markboth]       Define the headings for the headings and myheadings page style
\item[\bs markright]      Define the headings for the headings and myheadings page style
\item[\bs mbox]           Short form of the \verb|\makebox| command
\item[\bs medskip]        Vertical space of \verb|\medskipamount|
\item[\bs mu]             Greek letter mu, lowercase
\item[\bs multicolumn]    Tabular command to span multiple columns
\item[\bs multiput\us p]     Object placement command for the picture environment
\item[\bs newcommand]     Define a new \LaTeX\ command
\item[\bs newcounter]     Create a new counter
\item[\bs newenvironment] \mbox{} \\
			    Define a new environment
\item[\bs newfont]        Define a new font
\item[\bs newlength]      Generate a  new length command
\item[\bs newline]        Break the current line here
\item[\bs newpage]        Start a new page
\item[\bs newsavebox]     Create a new box for future use
\item[\bs newtheorem]     Create a new theorem environment
\item[\bs newtheorem\us 1]   New theorem numbered within a sectional unit
\item[\bs newtheorem\us 2]   New theorem numbered like another theorem
\item[\bs nocite]         Produce no text, but write the key list
\item[\bs nofiles]        Turn off auxiliary file output
\item[\bs noindent]       Suppress paragraph indentation
\item[\bs nolinebreak]    Do not break the current line
\item[\bs nopagebreak]    Do not break the current page
\item[\bs normalsize]     Default font size
\item[\bs nu]             Greek letter nu, lowercase
\item[\bs omega\us l]        Greek letter omega, lowercase
\item[\bs omega\us u]        Greek letter omega, uppercase
\item[\bs onecolumn]      Start one column text
\item[\bs onlynotes]      \SLiTeX{} command to produce a subset of your notes
\item[\bs onlyslides]     \SLiTeX{} command to produce a subset of your slides
\item[\bs opening]        Letter opening, e.g. `Dear Sir:'
\item[\bs oval\us p]         Picture environment oval making command
\item[\bs overbrace]      Generate formula with brace above it, math mode ONLY
\item[\bs overline]       Generate formula with line above it, math mode ONLY
\item[\bs pagebreak]      Break the current page
\item[\bs pagenumbering]  Set the style of the page numbers
\item[\bs pageref]        Reference the page number of a label created with a 
  \verb|\label| command
\item[\bs pagestyle]      Set the page style for the entire document
\item[\bs paragraph]      \LaTeX\ sectioning command
\item[\bs parbox]         Generate a box in paragraph mode
\item[\bs part]           \LaTeX\ sectioning command
\item[\bs phi\us l]          Greek letter phi, lowercase
\item[\bs phi\us u]          Greek letter phi, uppercase
\item[\bs pi\us l]           Greek letter pi, lowercase
\item[\bs pi\us u]           Greek letter pi, uppercase
\item[\bs ps]                Latter postscript
\item[\bs psi\us l]          Greek letter psi, lowercase
\item[\bs psi\us u]          Greek letter psi, uppercase
\item[\bs put\us p]          Object placement command for the picture environment
\item[\bs raggedbottom]   Let height of text vary from page to page
\item[\bs raggedleft]     Declaration equivalent to the flushright environment
\item[\bs raggedright]    Declaration equivalent to the flushleft environment
\item[\bs raisebox]       Raise the text as specified
\item[\bs ref]            Reference a label created with the \verb|\label| command
\item[\bs renewcommand]   Redefine an existing \LaTeX\ command
\item[\bs renewenvironment]  \mbox{} \\
Re-define an existing environment
\item[\bs rho]            Greek letter rho, lowercase
\item[\bs right]          Right delimiter command for math formula
\item[\bs rm]             Roman
\item[\bs roman]          Roman counter definitions, capital and lower case
\item[\bs rule]           Draw a horizontal line
\item[\bs savebox]        Define the contents of the named box
\item[\bs sc]             Small caps
\item[\bs scriptsize]     3 sizes smaller than \verb|\normalsize|
\item[\bs section]        \LaTeX\ sectioning command
\item[\bs setcounter]     Set an existing counter to a specified value
\item[\bs setlength]      Set an existing length to a specified value
\item[\bs settowidth]     Set an existing length command to the width of the specified text
\item[\bs sf]             Sans serif
\item[\bs shortstack\us p]   Picture environment \verb|\shortstack| command
\item[\bs sigma\us l]        Greek letter sigma, lowercase
\item[\bs sigma\us u]        Greek letter sigma, uppercase
\item[\bs signature]      Letter signature
\item[\bs sl]             Slanted
\item[\bs small]          1 sizes smaller than \verb|\normalsize|
\item[\bs smallskip]      Vertical space of \verb|\smallskipamount|
\item[\bs sqrt]           Generate the nth root of the argument, math mode ONLY
\item[\bs subparagraph]   \LaTeX\ sectioning command
\item[\bs subsection]     \LaTeX\ sectioning command
\item[\bs subsubparagraph] \mbox{} \\
    \LaTeX\ sectioning command, milstd \& bookform styles ONLY
\item[\bs subsubsection]  \LaTeX\ sectioning command
\item[\bs subsubsubparagraph] \mbox{} \\
     \LaTeX\ sectioning command, milstd \& bookform styles ONLY
\item[\bs tableofcontents] \mbox{} \\
Generate a table of contents
\item[\bs tau]            Greek letter tau, lowercase
\item[\bs telephone]      Letter telephone, only used if \verb|\address| is null
\item[\bs TeX]            Produce \TeX\ logo
\item[\bs thanks]         Produce a footnote to the title made by 
  \verb|\maketitle|
\item[\bs theta\us l]        Greek letter theta, lowercase
\item[\bs theta\us u]        Greek letter theta, uppercase
\item[\bs thicklines]     Thicker lines for picture environment
\item[\bs thinlines]      Standard thickness of lines in picture environment
\item[\bs thispagestyle]  Set the page style for this page only
\item[\bs tiny]           4 sizes smaller than \verb|\normalsize|
\item[\bs title]          Text used by the \verb|\maketitle| command
\item[\bs today]          Today's date
\item[\bs tt]             Typewriter
\item[\bs twocolumn]      Start two column text
\item[\bs typein]         Interactive input command
\item[\bs typeout]        Print a message on your terminal
\item[\bs underbrace]     Underbrace a formula or text
\item[\bs underline]      Underline a formula or text
\item[\bs upsilon\us l]      Greek letter upsilon, lowercase
\item[\bs upsilon\us u]      Greek letter upsilon, uppercase
\item[\bs usebox]         `Execute' the saved box
\item[\bs usecounter]     Enable counter for use in numbered list
\item[\bs value]          Produce the value of a counter
\item[\bs varepsilon]     Greek letter varepsilon, lowercase
\item[\bs varphi]         Greek letter varphi, lowercase
\item[\bs varpi]          Greek letter varpi, lowercase
\item[\bs varrho]         Greek letter varrho, lowercase
\item[\bs varsigma]       Greek letter varsigma, lowercase
\item[\bs vartheta]       Greek letter vartheta, lowercase
\item[\bs vdots]          Vertical ellipsis, math mode ONLY
\item[\bs vector\us p]       Picture environment \verb|\vector| command
\item[\bs verb]           Produce a literal string
\item[\bs vfill]          Add infinitely stretchable vertical space
\item[\bs vline]          Tabular vertical rule
\item[\bs vspace]         Insert vertical space
\item[\bs xi\us l]           Greek letter xi, lowercase
\item[\bs xi\us u]           Greek letter xi, uppercase
\item[\bs zeta]           Greek letter zeta, lowercase
\item[\bs \bs]              Start a new line
\end{description}
\end{small}

\newpage
\section{Pre-defined Keys} \label{keydefs}

The following keys are defined within the {\tt LSEKEYS.GBL}
section file. 
\def\descriptionlabel#1{\rm #1\ \hfil}
\renewcommand{\thefootnote}[0]{\fnsymbol{footnote}}
\begin{small}
\begin{description}
\item[\bf Key] {\bf Description}
\item[\gold {\tt [}] Set the screen width to 132 characters
\item[\gold {\tt ]}] Set the screen width to 80 characters
\item[\gold {\tt \bs}] Clear the message buffer
\item[\gold {\tt `}] Set a mark called ``spot''
\item[\gold {\tt \tlde}] Go to a mark called ``spot''
\item[\gold {\tt C}] Compile the current buffer
\item[\ctrl{d}]   Compile the current buffer for debug \footnotemark[1]
\item[\gold + \ctrl{d}]  \mbox{} \\
Compile and review the current buffer for
debug\footnotemark[1]
\footnotetext[1]{Do NOT use these with \LaTeX ; our system does not understand
{\tt LATEX/DEBUG}. } 
\item[\gold {\tt B}] Go to the specified buffer
\item[\gold {\tt E}] End review mode
\item[\gold {\tt F}] Do a GOTO FILE command
\item[\gold {\tt H}] Display the HELP.KEYS buffer
\item[\gold {\tt Q}] Leave the editor with the QUIT command (delete new file)
\item[\gold {\tt R}] Set the specified buffer to READ ONLY status
\item[\gold + \ctrl{r}]  \mbox{} \\ Set the specified buffer to WRITE status
\item[\gold {\tt S}] Do a SUBSTITUTE command
\item[\gold {\tt X}] Leave the editor with the EXIT command (save new file)
\item[\gold {\tt BKSP}] Toggle the last two characters 
\item[\gold {\tt M}] Go to LSE main buffer
\item[\gold {\tt I}] Read in a temporary file
\item[\gold {\tt P}] Find the next occurence of a paragraph
\item[\gold {\tt <}] Shrink the current window (2 window mode)
\item[\gold {\tt >}] Enlarge the current window (2 window mode)
\item[\gold {\tt \lb}] Balance and highlight braces `{\tt \lb \rb}', brackets
`{\tt []}', and parentheses `{\tt ()}'
\item[\gold {\tt \rb}] Turn off highlight generated by \gold {\tt \lb}
\item[\gold {\tt K}] Define your own key (emulates EDT \ctrl{k} feature)
\end{description}
\end{small}

\newpage
\section{\LaTeX\ Style Templates} 
\label{templates}

\begin{itemize}

\item article

\begin{small}
\begin{verbatim}
\documentstyle<<optionlist>>{article}
<<pagestyle>>
<<pre_commands>>...
\begin{document}

<<commands>>...

\end{document}
\end{verbatim}
\end{small}

\item report

\begin{small}
\begin{verbatim}
\documentstyle<<optionlist>>{report}
<<pagestyle>>
<<pre_commands>>...
\begin{document}

<<commands>>...

\end{document}
\end{verbatim}
\end{small}

\item memo

\begin{small}
\begin{verbatim}
\documentstyle<<optionlist>>{memo}
<<pre_commands>>...
<<To_list_two_columns>>
<<From_list_two_columns>>
<<Copies_list_one_column>>
\begin{document}
\memohdr To:<<namelist>>...
From:<<namelist>>...
Copies:<<namelist>>...
Subject:{<<subject>>}

<<commands>>...

<memo_signature>

\end{document}
\end{verbatim}
\end{small}

\newpage

\item letter

\begin{small}
\begin{verbatim}
\documentstyle<<optionlist>>{letter}
<<pre_letter_cmds>>...
\begin{document}
\begin{letter}{<recipient>}
\opening{<paragraph_text>...}

<<environments>>...

\closing{<paragraph_text>...}
<<cc>>
<<encl>>
<<ps>>
\end{letter}
\end{document}
\end{verbatim}
\end{small}

\item slides---slide file

\begin{small}
\begin{verbatim}
<<pagestyle>>
%
\begin{slide}{<<slide_colors>>...}
<<slide_title>>
 
<<environments>>...
<<slide_fill>>
\end{slide}
% form feed
<<another_slide>>
\end{verbatim}
\end{small}

\item slides---root file

\begin{small}
\begin{verbatim}
\documentstyle{slides}
%
% If you only have one slide file, and you would like SLiTeX to prompt
% you for the name of the file when you run SLiTeX, then expand the
% next placeholder. Otherwise enter the name of your slide file as
% the argument to the \colorslides or \blackandwhite command.
%
<<prompted_slide_file>>
%
\begin{document}
<<colors>>
<<onlyslides>>
<color_or_bw>...
\end{document}
\end{verbatim}
\end{small}

\newpage

\item bookform

\begin{small}
\begin{verbatim}
\documentstyle<<optionlist>>{bookform}
\documentnumber{<text>}
\documentdate{<text>}
\title{<paragraph_text>...
\docnumber \\         % DO NOT DELETE \docnumber or
\docdate \\           % \docdate.
Contract XXXXXX \\    % The contract or PO number must be filled in.
Data Item No. XXX  \\ % The data item number must be filled in.
}

\begin{document}

\begin{titlepage}
\maketitle
\end{titlepage}

\cleardoublepage
%
% Page 1 reserved for revision history page
%
\setcounter{page}{2}
\tableofcontents
\newpage

% Since all bookform documents don't necessarily follow the same
% section structure, the author must layout the document as desired.

\end{document}
\end{verbatim}
\end{small}

\item laa---Astronomy and Astrophysics

\begin{small}
\begin{verbatim}
\documentstyle<<optionlist>>{laa}
<<pre_commands>>...
\begin{document}
\thesaurus{<thesaurus_code>
\title{<title>}
\subtitle{<subtitle>}
\{author{<author_name>\inst{<institute_number>\thanks{<text>}} 
\and <author_name>\inst{<institute_number>}}
\offprints{<name>}
\institute{<institute> 
\and <institute>...}
\date{Received date: <date>; Accepted date: <date>}
\maketitle

\begin{abstract}
<text>
\keywords{<keyword>}
\end{abstract}

\section{<text>}
<text>
\subsection{<text>}
<text>
\subsubsection{<text>}
<text>

\begin{thebibliography}{}	% DO NOT delete {}
\bibitem[<year>]{label:name}
<text>
\end{thebibliography}

\end{document}
\end{verbatim}
\end{small}

\item laamt---Astronomy and Astophysics with Monotype Times fonts 

\begin{small}
\begin{verbatim}
\documentstyle<<optionlist>>{laamt}
<<pre_commands>>...
\begin{document}
\thesaurus{<thesaurus_code>
\title{<title>}
\subtitle{<subtitle>}
\{author{<author_name>\inst{<institute_number>\thanks{<text>}} 
\and <author_name>\inst{<institute_number>}}
\offprints{<name>}
\institute{<institute> 
\and <institute>}
\date{Received date: <date>; Accepted date: <date>}
\maketitle

\begin{abstract}
<text>
\keywords{<keyword>}
\end{abstract}

\section{<text>}
<text>
\subsection{<text>}
<text>
\subsubsection{<text>}
<text>

\begin{thebibliography}{}	% DO NOT delete {}
\bibitem[<year>]{label:name}
<text>
\end{thebibliography}

\end{document}
\end{verbatim}
\end{small}

\item thesis

\begin{small}
\begin{verbatim}
\documentstyle[<<optionlist>>,thesis]{report} 
  
\setlength{\textwidth}{150mm} 
\setlength{\textheight}{223mm} 
\setlength{\oddsidemargin}{15mm} 
\setlength{\evensidemargin}{15mm} 
\setlength{\topmargin}{0mm} 
\setlength{\headheight}{0mm} 
\setcounter{secnumdepth}{10} 
\setcounter{tocdepth}{10} 
 
<<pre_commands>>... 
  
\title{<text>} 
\author{<name>} 
\department{<department>} 
\submissiondate{<date>} 
\degree{<degree>} 
\faculty{<faculty>} 
\institution{<institution>} 
\thesis{<thesis>} 
\ded{<dedication>} 
  
\begin{document} 

\maketitle{y}{y}{\input{abstract}}  
% 
%You must have the abstract in a file called ABSTRACT.TEX 
% 
\chapter{<text>} 
<text> 
\section{<text>} 
<text> 
\subsection{<text>} 
<text> 
\subsubsection{<text>} 
<text> 
\end{document} 
\end{verbatim}
\end{small}

\item milstd---System Specification

\begin{small}
\begin{verbatim}
\documentstyle<<optionlist>>{milstd}
\A
\documentnumber{<mil-std-number>}
\documentdate{<mil-date>}
\title{System Specification\\
<paragraph_text>...}

\begin{document}

\begin{titlepage}
\maketitle
\end{titlepage}

\pagenumbering{roman}

\tableofcontents
\newpage
\listoffigures
\listoftables

\cleardoublepage
\pagenumbering{arabic}

\section{SCOPE}

\section{APPLICABLE DOCUMENTS}
.
. \subsection's, \subsubsection's, etc. 
. have been omitted for brevity
.  

\section{REQUIREMENTS}

\subsection{System definition.}

\subsubsection{General description.}

\subsubsection{Missions.}

\subsubsection{Threat.}

\subsubsection{System diagrams.}

\subsubsection{Interface definition.}

\subsubsection{Government furnished property list.}

\subsubsection{Operational and organizational concepts.}

\subsection{Characteristics.}

\subsubsection{Performance characteristics.}

\subsubsection{Physical characteristics.}

\subsubsection{Reliability.}

\subsubsection{Maintainability.}

\subsubsection{Availability.}

\subsubsection{System effectiveness models.}

\subsubsection{Environmental conditions.}

\subsubsection{Nuclear control requirements.}

\subsubsection{Transportability.}

\subsection{Design and construction.}

\subsubsection{Materials, processes, and parts.}

\subsubsection{Electromagnetic radiation.}

\subsubsection{Nameplates and product marking.}

\subsubsection{Workmanship.}

\subsubsection{Interchangeability.}

\subsubsection{Safety.}

\subsubsection{Human performance/human engineering.}

\subsection{Documentation.}

\subsection{Logistics.}

\subsubsection{Maintenance.}

\subsubsection{Supply.}

\subsubsection{Facilities and facility equipment.}

\subsection{Personnel and training.}

\subsubsection{Personnel.}

\subsubsection{Training.}

\subsection{Functional area characteristics.}

\subsection{Precedence.}

\section{QUALITY ASSURANCE PROVISIONS}

\subsection{General.}

\subsubsection{Responsibility for tests.}

\subsubsection{Special tests and examinations.}

\subsection{Quality conformance inspections.}

\section{PREPARATION FOR DELIVERY}

\section{NOTES}

\appendix
%
% \section will now generate appendices starting with section 10, 20, etc.
%
\section{<text>} % Section 10, Appendix I

\section{<text>} % Section 20, Appendix II

\end{document}
\end{verbatim}
\end{small}

\item milstd---Critical Item Development Specification 

\begin{small}
\begin{verbatim}
\documentstyle<<optionlist>>{milstd}
\Btwo
\documentnumber{<mil-std-number>}
\documentdate{<mil-date>}
\title{Critical Item Development Specification\\
<paragraph_text>...}

\begin{document}

\begin{titlepage}
\maketitle
\end{titlepage}

\pagenumbering{roman}
\tableofcontents
\newpage
\listoffigures
\listoftables

\cleardoublepage
\pagenumbering{arabic}

\section{SCOPE}

\section{APPLICABLE DOCUMENTS}
.
. \subsection's, \subsubsection's, etc. 
. have been omitted for brevity
.  

\section{REQUIREMENTS}

\subsection{Item definition.}

\subsection{Characteristics.}

\subsubsection{Performance.}

\subsubsection{Physical characteristics.}

\subsubsection{Reliability.}

\subsubsection{Maintainability.}

\subsubsection{Environmental conditions.}

\subsubsection{Transportability.}

\subsection{Design and construction.}

\subsubsection{Materials, processes, and parts.}

\subsubsection{Electromagnetic radiation.}

\subsubsection{Nameplates and product marking.}

\subsubsection{Workmanship.}

\subsubsection{Interchangeability.}

\subsubsection{Safety.}

\subsubsection{Human performance/human engineering.}

\subsection{Documentation.}

\subsection{Logistics.}

\subsubsection{Maintenance.}

\subsubsection{Supply.}

\subsection{Precedence.}

\section{QUALITY ASSURANCE PROVISIONS}

\subsection{General.}

\subsubsection{Responsibility for test.}

\subsubsection{Special tests and examinations.}

\subsection{Quality conformance inspections.}

\section{PREPARATION FOR DELIVERY}

\section{NOTES}

\appendix
%
% \section will now generate appendices starting with section 10, 20, etc.
%
\section{<text>} % Section 10, Appendix I

\section{<text>} % Section 20, Appendix II

\end{document}
\end{verbatim}
\end{small}

\item milstd---Computer Program Development Specification 

\begin{small}
\begin{verbatim}
\documentstyle<<optionlist>>{milstd}
\Bfive
\documentnumber{<mil-std-number>}
\documentdate{<mil-date>}
\title{Computer Program Development Specification\\
<paragraph_text>...}

\begin{document}

\begin{titlepage}
\maketitle
\end{titlepage}

\pagenumbering{roman}
\tableofcontents
\newpage
\listoffigures
\listoftables

\cleardoublepage
\pagenumbering{arabic}

\section{SCOPE}

\subsection{Identification.}

\subsection{Functional summary.}

\section{APPLICABLE DOCUMENTS}
.
. \subsection's, \subsubsection's, etc. 
. have been omitted for brevity
.  

\section{REQUIREMENTS}

\subsection{Program definition.}

\subsection{Detailed functional requirements.}

\subsubsection{Inputs.}

\subsubsection{Processing.}

\subsubsection{Outputs.}

\subsubsection{Special requirements.}

\subsection{Adaptation.}

\subsubsection{General environment.}

\subsubsection{System parameters.}

\subsubsection{System capacities.}

\section{QUALITY ASSURANCE PROVISIONS}

\subsection{Introduction.}

\subsection{Test requirements.}

\subsection{Acceptance test requirements.}

\section{PREPARATION FOR DELIVERY}

\section{NOTES}

\appendix
%
% \section will now generate appendices starting with section 10, 20, etc.
%
\section{<text>} % Section 10, Appendix I

\section{<text>} % Section 20, Appendix II

\end{document}
\end{verbatim}
\end{small}

\item milstd---Computer Program Product Specification 

\begin{small}
\begin{verbatim}
\documentstyle<<optionlist>>{milstd}
\Cfive
\documentnumber{<mil-std-number>}
\documentdate{<mil-date>}
\title{Computer Program Product Specification\\
<paragraph_text>...}

\begin{document}

\begin{titlepage}
\maketitle
\end{titlepage}

\pagenumbering{roman}
\tableofcontents
\newpage
\listoffigures
\listoftables

\cleardoublepage
\pagenumbering{arabic}

\section{SCOPE}

\section{APPLICABLE DOCUMENTS}
.
. \subsection's, \subsubsection's, etc. 
. have been omitted for brevity
.  

\section{REQUIREMENTS}

\subsection{Functional allocation description.}

\subsection{Functional description.}

\subsection{Storage allocation.}

\subsection{Computer program functional flow diagram.}

\subsubsection{Program interrupts.}

\subsubsection{Logic of subprogram.}

\subsubsection{Special control features.}

\section{QUALITY ASSURANCE PROVISIONS}

\section{PREPARATION FOR DELIVERY}

\section{NOTES}

\appendix
%
% \section will now generate appendices starting with section 10, 20, etc.
%
\section{<text>} % Section 10, Appendix I

\section{<text>} % Section 20, Appendix II

\end{document}
\end{verbatim}
\end{small}

\item milstd---System Specification

\begin{small}
\begin{verbatim}
\documentstyle<<optionlist>>{milstd}
\A
\documentnumber{<mil-std-number>}
\documentdate{<mil-date>}
\title{System Specification\\
<paragraph_text>...}

\begin{document}

\begin{titlepage}
\maketitle
\end{titlepage}

\pagenumbering{roman}

\tableofcontents
\newpage
\listoffigures
\listoftables

\cleardoublepage
\pagenumbering{arabic}

\section{SCOPE}

\section{APPLICABLE DOCUMENTS}
.
. \subsection's, \subsubsection's, etc. 
. have been omitted for brevity
.  

\section{REQUIREMENTS}

\subsection{System definition.}

\subsubsection{General description.}

\subsubsection{Missions.}

\subsubsection{Threat.}

\subsubsection{System diagrams.}

\subsubsection{Interface definition.}

\subsubsection{Government furnished property list.}

\subsubsection{Operational and organizational concepts.}

\subsection{Characteristics.}

\subsubsection{Performance characteristics.}

\subsubsection{Physical characteristics.}

\subsubsection{Reliability.}

\subsubsection{Maintainability.}

\subsubsection{Availability.}

\subsubsection{System effectiveness models.}

\subsubsection{Environmental conditions.}

\subsubsection{Nuclear control requirements.}

\subsubsection{Transportability.}

\subsection{Design and construction.}

\subsubsection{Materials, processes, and parts.}

\subsubsection{Electromagnetic radiation.}

\subsubsection{Nameplates and product marking.}

\subsubsection{Workmanship.}

\subsubsection{Interchangeability.}

\subsubsection{Safety.}

\subsubsection{Human performance/human engineering.}

\subsection{Documentation.}

\subsection{Logistics.}

\subsubsection{Maintenance.}

\subsubsection{Supply.}

\subsubsection{Facilities and facility equipment.}

\subsection{Personnel and training.}

\subsubsection{Personnel.}

\subsubsection{Training.}

\subsection{Functional area characteristics.}

\subsection{Precedence.}

\section{QUALITY ASSURANCE PROVISIONS}

\subsection{General.}

\subsubsection{Responsibility for tests.}

\subsubsection{Special tests and examinations.}

\subsection{Quality conformance inspections.}

\section{PREPARATION FOR DELIVERY}

\section{NOTES}

\appendix
%
% \section will now generate appendices starting with section 10, 20, etc.
%
\section{<text>} % Section 10, Appendix I

\section{<text>} % Section 20, Appendix II

\end{document}
\end{verbatim}
\end{small}

\item milstd---Critical Item Development Specification 

\begin{small}
\begin{verbatim}
\documentstyle<<optionlist>>{milstd}
\Btwo
\documentnumber{<mil-std-number>}
\documentdate{<mil-date>}
\title{Critical Item Development Specification\\
<paragraph_text>...}

\begin{document}

\begin{titlepage}
\maketitle
\end{titlepage}

\pagenumbering{roman}
\tableofcontents
\newpage
\listoffigures
\listoftables

\cleardoublepage
\pagenumbering{arabic}

\section{SCOPE}

\section{APPLICABLE DOCUMENTS}
.
. \subsection's, \subsubsection's, etc. 
. have been omitted for brevity
.  

\section{REQUIREMENTS}

\subsection{Item definition.}

\subsection{Characteristics.}

\subsubsection{Performance.}

\subsubsection{Physical characteristics.}

\subsubsection{Reliability.}

\subsubsection{Maintainability.}

\subsubsection{Environmental conditions.}

\subsubsection{Transportability.}

\subsection{Design and construction.}

\subsubsection{Materials, processes, and parts.}

\subsubsection{Electromagnetic radiation.}

\subsubsection{Nameplates and product marking.}

\subsubsection{Workmanship.}

\subsubsection{Interchangeability.}

\subsubsection{Safety.}

\subsubsection{Human performance/human engineering.}

\subsection{Documentation.}

\subsection{Logistics.}

\subsubsection{Maintenance.}

\subsubsection{Supply.}

\subsection{Precedence.}

\section{QUALITY ASSURANCE PROVISIONS}

\subsection{General.}

\subsubsection{Responsibility for test.}

\subsubsection{Special tests and examinations.}

\subsection{Quality conformance inspections.}

\section{PREPARATION FOR DELIVERY}

\section{NOTES}

\appendix
%
% \section will now generate appendices starting with section 10, 20, etc.
%
\section{<text>} % Section 10, Appendix I

\section{<text>} % Section 20, Appendix II

\end{document}
\end{verbatim}
\end{small}

\item milstd---Computer Program Development Specification 

\begin{small}
\begin{verbatim}
\documentstyle<<optionlist>>{milstd}
\Bfive
\documentnumber{<mil-std-number>}
\documentdate{<mil-date>}
\title{Computer Program Development Specification\\
<paragraph_text>...}

\begin{document}

\begin{titlepage}
\maketitle
\end{titlepage}

\pagenumbering{roman}
\tableofcontents
\newpage
\listoffigures
\listoftables

\cleardoublepage
\pagenumbering{arabic}

\section{SCOPE}

\subsection{Identification.}

\subsection{Functional summary.}

\section{APPLICABLE DOCUMENTS}
.
. \subsection's, \subsubsection's, etc. 
. have been omitted for brevity
.  

\section{REQUIREMENTS}

\subsection{Program definition.}

\subsection{Detailed functional requirements.}

\subsubsection{Inputs.}

\subsubsection{Processing.}

\subsubsection{Outputs.}

\subsubsection{Special requirements.}

\subsection{Adaptation.}

\subsubsection{General environment.}

\subsubsection{System parameters.}

\subsubsection{System capacities.}

\section{QUALITY ASSURANCE PROVISIONS}

\subsection{Introduction.}

\subsection{Test requirements.}

\subsection{Acceptance test requirements.}

\section{PREPARATION FOR DELIVERY}

\section{NOTES}

\appendix
%
% \section will now generate appendices starting with section 10, 20, etc.
%
\section{<text>} % Section 10, Appendix I

\section{<text>} % Section 20, Appendix II

\end{document}
\end{verbatim}
\end{small}

\item milstd---Computer Program Product Specification 

\begin{small}
\begin{verbatim}
\documentstyle<<optionlist>>{milstd}
\Cfive
\documentnumber{<mil-std-number>}
\documentdate{<mil-date>}
\title{Computer Program Product Specification\\
<paragraph_text>...}

\begin{document}

\begin{titlepage}
\maketitle
\end{titlepage}

\pagenumbering{roman}
\tableofcontents
\newpage
\listoffigures
\listoftables

\cleardoublepage
\pagenumbering{arabic}

\section{SCOPE}

\section{APPLICABLE DOCUMENTS}
.
. \subsection's, \subsubsection's, etc. 
. have been omitted for brevity
.  

\section{REQUIREMENTS}

\subsection{Functional allocation description.}

\subsection{Functional description.}

\subsection{Storage allocation.}

\subsection{Computer program functional flow diagram.}

\subsubsection{Program interrupts.}

\subsubsection{Logic of subprogram.}

\subsubsection{Special control features.}

\section{QUALITY ASSURANCE PROVISIONS}

\section{PREPARATION FOR DELIVERY}

\section{NOTES}

\appendix
%
% \section will now generate appendices starting with section 10, 20, etc.
%
\section{<text>} % Section 10, Appendix I

\section{<text>} % Section 20, Appendix II

\end{document}
\end{verbatim}
\end{small}

\end{itemize}

\end{document}
