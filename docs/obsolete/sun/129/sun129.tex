\documentstyle[11pt]{article} 
\pagestyle{myheadings}

%------------------------------------------------------------------------------
\newcommand{\stardoccategory}  {Starlink User Note}
\newcommand{\stardocinitials}  {SUN}
\newcommand{\stardocnumber}    {129.3}
\newcommand{\stardocauthors}   {P M Allan}
\newcommand{\stardocdate}      {4 October 1993}
\newcommand{\stardoctitle}     {TPAU --- The Peter Allan Utilities}
%------------------------------------------------------------------------------

\newcommand{\stardocname}{\stardocinitials /\stardocnumber}
\renewcommand{\_}{{\tt\char'137}}     % re-centres the underscore
\markright{\stardocname}
\setlength{\textwidth}{160mm}
\setlength{\textheight}{240mm}
\setlength{\topmargin}{-5mm}
\setlength{\oddsidemargin}{0mm}
\setlength{\evensidemargin}{0mm}
\setlength{\parindent}{0mm}
\setlength{\parskip}{\medskipamount}
\setlength{\unitlength}{1mm}

%------------------------------------------------------------------------------
% Add any \newcommand or \newenvironment commands here
%------------------------------------------------------------------------------

\begin{document}
\thispagestyle{empty}
SCIENCE \& ENGINEERING RESEARCH COUNCIL \hfill \stardocname\\
RUTHERFORD APPLETON LABORATORY\\
{\large\bf Starlink Project\\}
{\large\bf \stardoccategory\ \stardocnumber}
\begin{flushright}
\stardocauthors\\
\stardocdate
\end{flushright}
\vspace{-4mm}
\rule{\textwidth}{0.5mm}
\vspace{5mm}
\begin{center}
{\Large\bf \stardoctitle}
\end{center}
\vspace{5mm}

%------------------------------------------------------------------------------
%  Add this part if you want a table of contents
%  \setlength{\parskip}{0mm}
%  \tableofcontents
%  \setlength{\parskip}{\medskipamount}
%  \markright{\stardocname}
%------------------------------------------------------------------------------

\section{Introduction}

TPAU\footnote{The name of this set of utilities was suggested by the Starlink
software librarian. If you think it is too pretentious, why not submit some
utilities of your own so we have to change the name?} is a mixed bag of
Starlink utilities. They have been grouped together simply to reduce the number
of small software items. The utilities currently consist of:

\begin{tabular}{lcl}
\underline{Name} & \underline{Version} & \underline{Description} \\
LOG & 1.2 & A logical name utility \\
XDISPLAY & 2.0 & Easy use of remote X windows \\
\end{tabular}

%\begin{description}
%\item[LOG] A logical name utility
%\item[XDISPLAY] Easy use of remote X windows
%\end{description}

\section{LOG - A logical name utility}

This utility provides the ability to add or remove an item in a VMS logical
name search list.

\subsection{Description}

Occasionally it is necessary to add an item to a logical name search list.
Surprisingly, there is no simple way of doing this from DCL. The command
procedure {\tt ADD\_TO\_SEARCH\_LIST} has been written to perform this task. It
is invoked with the command ATSL. The procedure has four parameters; two
mandatory and two optional. These are as follows:

\begin{itemize}
\item The first parameter is the logical name to be created or modified. 
\item The second parameter is the item to be added to, or removed from, the
search list. 
\item The third (optional) parameter indicates in which logical name table the
search list should be created.
\item The fourth (optional) parameter indicates whether the new item should be
added at the beginning or the end of the current list, or whether it should be
removed from the list. The default is to add the item at the end of the list.
If the fourth parameter is BEGIN then the new item will be added at the
beginning of the list. If the fourth parameter is COPY, then the search list
will be copied from the current logical name table to the one specified by the
third parameter, without change. In this case the second parameter is ignored.
It is also possible to delete an item by specifying DELETE as the value of the
fourth parameter. If a logical name exists more than once in the search list,
the DELETE option will remove all occurrences of the name.
\end{itemize}

If the logical name does not already exist, then the command procedure will
create it.

Here is an example of building up a search list.

\begin{verbatim}
      $ ATSL C$INCLUDE EMS_DIR
      $ ATSL C$INCLUDE CNF_DIR
      $ ATSL C$INCLUDE CHR_DIR
\end{verbatim}

Typing

\begin{quote}{\tt 
\$ SHOW LOGICAL C\$INCLUDE
}
\end{quote}

will now give 

\begin{verbatim}
      "C$INCLUDE" = "EMS_DIR" (LNM$PROCESS_TABLE)
           = "CNF_DIR"
           = "CHR_DIR"

    1  "EMS_DIR" = "STARDISK:[STARLINK.LIB.EMS.STANDALONE]" (LNM$SYSTEM_TABLE)
    1  "CNF_DIR" = "STARDISK:[STARLINK.LIB.CNF]" (LNM$SYSTEM_TABLE)
    1  "CHR_DIR" = "STARDISK:[STARLINK.LIB.CHR]" (LNM$SYSTEM_TABLE)
\end{verbatim}

Note that since the items in the search list are themselves logical names, the
fully recursive translation of those items is also given. The above example is
not realistic in the sense that you could define the search list much more
simply with a single DEFINE command. However, if the individual additions to
the list are spread around several command procedures, then such a sequence is
the easiest way to create the search list.

\subsection{Different Logical Name Tables}

The third parameter of the command procedure can be given as PROCESS, JOB,
GROUP, SYSTEM or as the name of another logical name table. If this parameter
is specified, then a new logical name will be created in the table specified.
If it is not specified, then the logical name is created in the table that it
is already in, if it exists, and in the process table if it does not. This
means that if you wish to add an item to a system search list, but have the
result in your process table, then you must specify PROCESS for the third
parameter.

There is one exception to the rule that the new item is added to the current
logical name table or to the one specified by the third parameter. If the user
does not have the required privilege to create or modify the logical name, then
the search list is created in the process logical name table. For example, if a
logical name exists in the system table and you try to modify it without SYSNAM
privilege, then the logical name search list will be copied to your process
logical name table where it will have the new item added.

The third parameter does not affect the logical name tables that get searched
for the translation of the current logical name, if it exists.

\subsection{Changes from version 1.0}

In version 1.0 of this utility, the function of the fourth parameter was
slightly different. If it was not specified, then the new item was added at the
end of the search list (this behaviour has not changed), but if is was anything
else, then the new item was added to the beginning of the search list. Also,
you could not copy a search list from one logical name table to another
unchanged.

The behaviour is now such that if the fourth parameter not specified, or is
anything other than BEGIN, COPY or DELETE, then the new item is added at the
end of the search list.


\section{XDISPLAY - Easy use of remote X windows}

This section describes how to use the XDISPLAY utility on Starlink systems that
are running X~windows. It makes no attempt to describe the X~windows system. It
assumes that you have some familiarity with using X~windows, although not to
any great depth.

\subsection{The problem}

If you are using X~windows purely on a workstation, then there is no problem.
Graphical output created by X will appear automatically on your
screen. The problem arises if you are logged on to an X~client other than
the machine that you are sat in front of (the X~server). Suppose
that you are sat in front of a VAXstation called RLSVS2, but that you are
remotely logged into RLSTAR, a microVAX. To get X output from a program run
on RLSTAR to be displayed on the screen of RLSVS2, you need to type

\begin{quote}{\tt
\$ SET DISPLAY/CREATE/NODE=RLSVS2
}
\end{quote}

If you are running programs that execute in a subprocess, such as ADAM programs
run from ICL, then you also need to type

\begin{quote}{\tt
\$ DEFINE/JOB DECW\$DISPLAY 'F\$TRNLNM( "DECW\$DISPLAY", "LNM\$PROCESS" )
}
\end{quote}

This defines the logical name DECW\$DISPLAY to be a job logical name.

If you are sat in front of a Sun, but logged into a VAX then the SET DISPLAY
command become something like

\begin{quote}{\tt
\$ SET DISPLAY/CREATE/NODE="rlssp1.bnsc.rl.ac.uk"/TRANSPORT=TCPIP
}
\end{quote}

This is becoming far too easy to forget and far too much to type every time you
do a remote login. If you log into a Sun from a remote X~server the command
that you type is different, so that is even more to remember.

\subsection{The solution}

To wrap up what you need to type to get X output from the machine that you are
logged into back to your X~server, procedures called XDISPLAY are available on
all STARLINK machines. Of course, on the Unix systems, the
command must be typed in lower case. To get X output drawn on your X~server
from a remote client, just type {\tt xdisplay} before running the program. If
you type {\tt xdisplay} when you are not logged into a remote system, then
nothing will happen and the output will still appear on your screen.

What XDISPLAY does is to see where you logged in from and which protocol you
came in with (DECnet, LAT or TCP/IP) and executes the appropriate command to
point the X output back to you. On a VAX, it also executes the DEFINE command
mentioned above.

\subsection{What can go wrong?}

There are a few things that might stop XDISPLAY from working as you intended.
These are problems with security, multi-hop logins and inappropriate use.

For example, if you are logged into the microVAX RLSTAR from the VAXstation
RLSVS2 and you type {\tt xdisplay}, then it is possible that you will get a
message such as 

\begin{quote}{\tt
Xlib:  Client is not authorized to access server
}
\end{quote}

To fix this, you need to make sure that the client is authorized to send X
output to the server. If you do not know how to do this, consult your site
manager.

A problem that is not so easily overcome, but is probably less likely to occur,
is that of multi-hop logins. If you are using VAXstation RLSVS2 and log into
RLSTAR and then to STADAT (to give a concrete example), and then type {\tt
xdisplay} on STADAT, then XDISPLAY thinks that you are on RLSTAR and so tries
to point the display back to RLSTAR. Since RLSTAR is not a workstation, this
will fail. A somewhat more embarrassing possibility is that you have
logged in from one workstation (A) to a second (B), and thence to a third
machine (C). If you then type {\tt xdisplay} on C it will successfully point
the X output back to B if the security setting allows it. The trouble is that
you are sat in front of A wondering where the window has gone and the user sat
in front of B is wondering why a random window has suddenly appeared! Making
XDISPLAY automatically cope with multi-hop logins verges on the impossible, but
there is a simple manual override available. If you type

\begin{quote}{\tt
xdisplay <nodename>
}
\end{quote}

where {\tt <nodename>} is the name of the computer or X-terminal that you want
the display to appear on, then the display will appear on that X-server,
regardless of the route that you used to log in.

Note that on Unix systems, you can only use the xdisplay command interactively.
You cannot put it in a shell script as {\tt xdisplay} is actually an alias that
uses the shell's command line recall features. If you need to use xdisplay in a
file there are two possibilities. If you want to have xdisplay work out where
you logged in from, put:

\begin{quote}{\tt
\% source /star/etc/xdisplay
}
\end{quote}

in your file. This will generally be your {\tt .login} file. If you want to
explicitly set the node that output will appear on, then simply set the
environment variable DISPLAY, e.g.

\begin{quote}{\tt
\% setenv DISPLAY adam4.bnsc.rl.ac.uk:0
}
\end{quote}

which is all that XDISPLAY does anyway.

If you give XDISPLAY the explicit node name, it is also possible to specify the
transport mechanism to use. On Unix systems, you can select the transport to be
used by the case of the node name. If it is purely in upper case, then DECnet
will be used. Anything else will cause TCP/IP to be used, which is normally
what is required. On VMS, the transport can be selected by a second parameter
to XDISPLAY, which can take one of the values DECNET, TCPIP, LAT or LOCAL. If
it is omitted, DECNET is the default. If you select the TCP/IP protocol while
logged into a VAX, you may need to put the node name in double quotes to
prevent DCL from converting what you type to upper case.

For example:

\begin{quote}{\tt
\$ xdisplay "rlssp1.bnsc.rl.ac.uk" tcpip
}
\end{quote}

Finally, if you try to use XDISPLAY in an inappropriate way then you probably
will not get what you expected. For example, if you log into a computer from a
VT type terminal connected to a TCP/IP terminal server, then XDISPLAY will quite
happily point your X output to the terminal server. It is only when you come to
run an X application that you will see the problem.

\end{document}
