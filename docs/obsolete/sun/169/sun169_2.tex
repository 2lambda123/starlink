\documentstyle[11pt]{article} 
\pagestyle{myheadings}

%------------------------------------------------------------------------------
\newcommand{\stardoccategory}  {Starlink User Note}
\newcommand{\stardocinitials}  {SUN}
\newcommand{\stardocnumber}    {169.2}
\newcommand{\stardocauthors}   {Martin Bly}
\newcommand{\stardocdate}      {15 October 1993}
\newcommand{\stardoctitle}     {PINE --- Electronic mail interface}
%------------------------------------------------------------------------------

\newcommand{\stardocname}{\stardocinitials /\stardocnumber}
\renewcommand{\_}{{\tt\char'137}}     % re-centres the underscore
\markright{\stardocname}
\setlength{\textwidth}{160mm}
\setlength{\textheight}{230mm}
\setlength{\topmargin}{-2mm}
\setlength{\oddsidemargin}{0mm}
\setlength{\evensidemargin}{0mm}
\setlength{\parindent}{0mm}
\setlength{\parskip}{\medskipamount}
\setlength{\unitlength}{1mm}

%------------------------------------------------------------------------------
% Add any \newcommand or \newenvironment commands here
%------------------------------------------------------------------------------

\begin{document}
\thispagestyle{empty}
SCIENCE \& ENGINEERING RESEARCH COUNCIL \hfill \stardocname\\
RUTHERFORD APPLETON LABORATORY\\
{\large\bf Starlink Project\\}
{\large\bf \stardoccategory\ \stardocnumber}
\begin{flushright}
\stardocauthors\\
\stardocdate
\end{flushright}
\vspace{-4mm}
\rule{\textwidth}{0.5mm}
\vspace{5mm}
\begin{center}
{\Large\bf \stardoctitle}
\end{center}
\vspace{5mm}

%------------------------------------------------------------------------------
%  Add this part if you want a table of contents
%  \setlength{\parskip}{0mm}
%  \tableofcontents
%  \setlength{\parskip}{\medskipamount}
%  \markright{\stardocname}
%------------------------------------------------------------------------------

\section{Introduction}

The PINE mailer is an easy-to-use electronic mail interface for Unix
platforms. It originates from the University of Washington Office of
Computing and Communications, and was written by Mike Siebel, Mark
Crispin and Laurence Lundblade.

Starlink has provided PINE `as-is' to be used as part of the 
Base Set of applications for new users, and for users moving between
sites. It should be installed at all Starlink sites.

\section{Features of PINE}

PINE has many features, including:

\begin{itemize}

\item A mail index with message summary including the message status,
sender, size, date and subject.

\item Processing and viewing of mail using {\tt forward}, {\tt reply},
{\tt save}, {\tt print}, {\tt export}, {\tt delete}, address capture,
and search.

\item Address book for storing complex addresses, personal nicknames
and distribution lists.

\item Multiple folders and folder management for filing messages.

\item Message composer with editor (PICO) and spelling checker.

\item Comprehensive online screen and context sensitive help.

\end{itemize}

PINE is available for a large range of machines. The version provided by
Starlink has been built on Sun/SunOS 4.1.2, Sun/Solaris 2.2, Dec Alpha OSF/1
1.2 and DEC/Ultrix v4.2. The current Pine version is Pine 3.87.

\section{Using PINE}

PINE is started by typing:

\begin{verbatim}
      % pine
\end{verbatim}

to your login shell. If you are invoking PINE for the first time you
will see some administrative messages, and then the main PINE menu. The
main menu looks like the display shown below.

You may find PINE objects to the type of terminal it thinks you are
using. If your terminal type is `unknown' or `network', reset it to 
{\tt vt100} or something else in the {\tt /etc/termcap} list. This is
achieved in the C--Shell thus: {\tt \% setenv TERM vt100}.

\begin{small}
\begin{center}
\begin{verbatim}
  PINE 3.87   MAIN MENU                             Folder: INBOX  0 Messages


       ?     HELP               -  Get help using Pine

       C     COMPOSE MESSAGE    -  Compose and send a message

       I     FOLDER INDEX       -  View messages in current folder

       L     FOLDER LIST        -  Select a folder to view

       A     ADDRESS BOOK       -  Update address book

       S     SETUP              -  Configure or update Pine

       Q     QUIT               -  Exit the Pine program




   Copyright 1989-1993.  PINE is a trademark of the University of Washington.
                    [Folder "INBOX" opened with 0 messages]
? Help                     P PrevCmd                  R RelNotes
O OTHER CMDS L [ListFldrs] N NextCmd                  K KBLock
\end{verbatim}

Figure 1. The PINE 3.87 main menu.

\end{center}
\end{small}

This is a quote direct from the PINE brochure: {\em ``It is intended that
PINE can be learned by exploration rather than reading manuals.''}

PINE {\em is} a very user-friendly application, and I was able to start
emailing colleagues after a very short time. I encourage you to do just
that and explore the world of Unix email. Happy emailing!

More detailed information about how PINE works can be found in the
document: ``{\em PINE Technical Notes version 3.85, September 1993}\/''. This
has been distributed as MUD/141 in the Starlink Miscellaneous User
Documents series.

\end{document}
