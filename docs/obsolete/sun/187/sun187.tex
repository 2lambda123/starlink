\documentstyle[11pt]{article} 
\pagestyle{myheadings}

%------------------------------------------------------------------------------
\newcommand{\stardoccategory}  {Starlink User Note}
\newcommand{\stardocinitials}  {SUN}
\newcommand{\stardocnumber}    {187.1}
\newcommand{\stardocauthors}   {M. J. Bly}
\newcommand{\stardocdate}      {16 January 1995}
\newcommand{\stardoctitle}     {SKYCAL --- Interactive Almanac and Calculator}
%------------------------------------------------------------------------------

\newcommand{\stardocname}{\stardocinitials /\stardocnumber}
\renewcommand{\_}{{\tt\char'137}}     % re-centres the underscore
\markright{\stardocname}
\setlength{\textwidth}{160mm}
\setlength{\textheight}{230mm}
\setlength{\topmargin}{-2mm}
\setlength{\oddsidemargin}{0mm}
\setlength{\evensidemargin}{0mm}
\setlength{\parindent}{0mm}
\setlength{\parskip}{\medskipamount}
\setlength{\unitlength}{1mm}

%------------------------------------------------------------------------------
% Add any \newcommand or \newenvironment commands here
%------------------------------------------------------------------------------

\begin{document}
\thispagestyle{empty}
DRAL / {\sc Rutherford Appleton Laboratory} \hfill {\bf \stardocname}\\
{\large Particle Physics \& Astronomy Research Council}\\
{\large Starlink Project\\}
{\large \stardoccategory\ \stardocnumber}
\begin{flushright}
\stardocauthors\\
\stardocdate
\end{flushright}
\vspace{-4mm}
\rule{\textwidth}{0.5mm}
\vspace{5mm}
\begin{center}
{\Large\bf \stardoctitle}
\end{center}
\vspace{5mm}

%------------------------------------------------------------------------------
%  Add this part if you want a table of contents
%  \setlength{\parskip}{0mm}
%  \tableofcontents
%  \setlength{\parskip}{\medskipamount}
%  \markright{\stardocname}
%------------------------------------------------------------------------------

\section{Introduction}

SKYCAL consists of two programs, {\tt skycalc} and {\tt skycalendar}. 

The {\tt skycalc} program is an ``Interactive Almanac''. It is intended
for use by astronomers planning and executing observing runs, and
allows easy calculation of airmasses, twilight, lunar interference,
coordinate transformations, and such --- nearly everything except the
weather.

The {\tt skycalendar} program is a ``Nighttime Astronomical Calendar'',
for calculating and printing astronomical calendars.  Both have a
number of preset Observatory locations to choose from, or you can load
your own.

\section{Using Skycalc and Skycalendar}

Both programs have been installed in the USSC, so you just type their
names at the Unix shell prompt, thus:

\begin{quote}
{\tt \% skycalc} \\
or \\
{\tt \% skycalendar}
\end{quote}

where {\tt \%} is the Unix shell prompt.

More information about both {\tt skycalc} and {\tt skycalendar} can be
found in the ``{\sc Skycalc User's Manual}'', copies of which are
available from Site Managers.

\section{Acknowledgement}

The {\tt skycalc} and {\tt skycalendar} programs were written by John 
Thorstensen, who has kindly given permission for their distribution as part
of the Starlink Software Collection.  

John Thorstensen may be contacted at The Department of Physics and
Astronomy, Dartmouth College, Hanover, NH 03755, USA; and by email at
{\tt john.thorstensen@dartmouth.edu}.


\end{document}
