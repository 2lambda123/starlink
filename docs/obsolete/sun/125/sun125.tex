\documentstyle{article}    % uses 10pt to fit it on 2 sides of A4
\pagestyle{myheadings}

%------------------------------------------------------------------------------
\newcommand{\stardoccategory}  {Starlink User Note}
\newcommand{\stardocinitials}  {SUN}
\newcommand{\stardocnumber}    {125.1}
\newcommand{\stardocauthors}   {Adrian Fish}
\newcommand{\stardocdate}      {7 November 1991}
\newcommand{\stardoctitle}     {STEVE --- a thinking person's screen editor}
%------------------------------------------------------------------------------

\newcommand{\stardocname}{\stardocinitials /\stardocnumber}
\renewcommand{\_}{{\tt\char'137}}     % re-centres the underscore
\markright{\stardocname}
\setlength{\textwidth}{160mm}
\setlength{\textheight}{230mm}
\setlength{\topmargin}{-2mm}
\setlength{\oddsidemargin}{0mm}
\setlength{\evensidemargin}{0mm}
\setlength{\parindent}{0mm}
\setlength{\parskip}{\medskipamount}
\setlength{\unitlength}{1mm}

\newcommand{\gold}{\mbox{\fbox{\scriptsize GOLD}}}
\newcommand{\keyname}[1]{\mbox{\fbox{\scriptsize #1}}}
\newcommand{\STEve}{\mbox{STEVE}}

\begin{document}
\thispagestyle{empty}
SCIENCE \& ENGINEERING RESEARCH COUNCIL \hfill \stardocname\\
RUTHERFORD APPLETON LABORATORY\\
{\large\bf Starlink Project\\}
{\large\bf \stardoccategory\ \stardocnumber}
\begin{flushright}
\stardocauthors\\
\stardocdate
\end{flushright}
\vspace{-4mm}
\rule{\textwidth}{0.5mm}
\vspace{5mm}
\begin{center}
{\Large\bf \stardoctitle}
\end{center}
\vspace{5mm}

%------------------------------------------------------------------------------
%  Add this part if you want a table of contents
%  \setlength{\parskip}{0mm}
%  \tableofcontents
%  \setlength{\parskip}{\medskipamount}
%  \markright{\stardocname}
%------------------------------------------------------------------------------


\section*{What is \STEve\ ?}

\STEve\ is an acronym for STarlink EVE and is an extended EDT-style EVE
editor for use at Starlink nodes. The facility provides extra commands which
are not part of standard EVE, and improves on one or two of the standard EVE
commands. Help on all topics and keys is available from within the editor. The
extensions and modifications present in \STEve\ are particularly useful to
Starlink users. Below is a brief listing of the extra features provided by
\STEve\ (in addition to all the standard EVE features).

\begin{itemize}

\item Default enhanced EDT keypad emulation.
\item A constantly visible list of numbered buffers is maintained at the
       bottom of the screen.
\item Move between up to nine buffers in the buffer list with a single
      keystroke combination.
\item Box-mode select, cut and paste with insert or overstrike.
\item Compile FORTRAN or C code from within the editor (will operate on
      the entire buffer or on a selected range e.g. a subroutine).
\item Run \LaTeX\ on the contents of the current buffer. Jumps to
      line where first error occurs.
\item Run the Starlink Spell Checker from within the editor.
\item Automatically insert matching closing brackets, braces, dollars etc.
      (v. useful in \LaTeX).
\item Trim trailing spaces from buffers or automatically on exit.
\item Insert the current date and time into your buffer.
\item The Fill Paragraph command recognizes \LaTeX\ paragraphs.
\end{itemize}

\section*{Absolute Beginners}

This short document is intended for the Starlink beginner and those users who
up until now have been using EDT or some other editor. It aims to introduce
\STEve\ and demonstrate how to use some commonly used features. For a more
detailed description of \STEve\ (and indeed EVE), a more comprehensive tutorial
and useful hints and tips, the user should refer to the \STEve\ User Guide and
Reference Manual (SUN/126).

A brief tutorial follows on the next page.

\newpage

\subsection*{Starting up \ldots}
First of all you will need a terminal. To fully utilise the facilities \STEve\
offers, a VT200 is ideal but a VT100 will do. Ensure your terminal has been set
to a VT200 (or VT100) terminal-type with an appropriate DCL command \footnote{
Generally {\tt SET TERM/DEV=VT200} or {\tt SET TERM/DEV=VT100} is fine.}. Then
all you need to do is type:

\begin{verbatim}
      $ STEVE <filename>
\end{verbatim}

Once you are in you will see a buffer status line near the bottom of the screen
with appropriate buffer information superimposed. The EDT keypad is available
immediately. Further EVE and \STEve\ commands are entered from the keyboard
after pressing the \keyname{Do} key (or \gold\ \keyname{KP7} on VT100
terminals); commands are entered at the {\tt Command:} prompt below the buffer
status line. Note however that although all EVE and \STEve\ commands can be
entered on the command line, many commonly used commands have been assigned to
various quick and easy keystroke combinations. When entering  commands at the
command prompt, they may be abbreviated to the smallest unique initial string.
If an abbreviation is not unique, the editor will display the possible choices
and prompt for the command name again.

If you are working from a VT200 terminal, press the \keyname{HELP} key and a
keypad diagram will appear on the screen. This is one of several ways to access
help information in \STEve, including simply typing {\tt HELP} at the command
prompt (see the \STEve\ User Guide and Reference Manual (SUN/126) for
further details).

\subsection*{A few of the many useful features}
Now you can try out a couple of the more commonly used commands. The buffer map
list and two window capability are really invaluable. You will notice below the
buffer status line an area in which a number is placed next to a buffer name
(probably your current buffer). To demonstrate the buffer map feature, fetch
another file by pressing \gold\ \keyname{G} (you will be prompted for
another file). When this new file has been loaded into the editor you will
notice that there now appears an extra entry on the buffer list line, namely
your new buffer name with a number next to it. You can move between these two
buffers simply by typing \gold\ \keyname{1} and \gold\ \keyname{2} alternately.
You can have up to nine buffers at once in the buffer list, and you can move
rapidly between them by typing \gold\ followed by the buffer number as it
appears in the buffer list.

Box mode select, cut and paste are provided and can be extremely useful for
editing columns of data. With \STEve, the \keyname{F17} key toggles between box
select mode and the default select mode (you will see the legend {\tt Box}
appear on the buffer status line when in box select mode).

\keyname{CTRL}\ \keyname{L} will \LaTeX\ your current buffer, and \gold\
\keyname{S} will Spell your current buffer.

\subsection*{Getting out \ldots}
To exit from the editor, simply press key \keyname{F10}. Alternatively press
the \keyname{Do} key followed by {\tt EXIT} at the {\tt Command:} prompt. Both
these actions will write your file and exit from the editor. To quit the editor
without writing your file, type \gold\ \keyname{Q} or \keyname{Do} followed
by {\tt QUIT}.

\subsection*{Getting more from \STEve}
Well, that's a brief look at \STEve. For those of you still not entirely
familiar with all the available commands and their keystrokes, do stick at it.
The \STEve\ User Guide and Reference Manual (SUN/126) is worth having handy,
even if only for the complete list of EVE and \STEve\ commands in the
appendices. EDT users will find \STEve\ extremely easy to use and the large
range of additional features available for editing and manipulating files will
soon have you wondering how on earth you managed before.

\end{document}
