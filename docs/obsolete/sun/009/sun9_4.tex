\documentstyle[11pt]{article} 
\pagestyle{myheadings}

%------------------------------------------------------------------------------
\newcommand{\stardoccategory}  {Starlink User Note}
\newcommand{\stardocinitials}  {SUN}
\newcommand{\stardocnumber}    {9.4}
\newcommand{\stardocauthors}   {Vassilis Laspias}
\newcommand{\stardocdate}      {29 November 1993}
\newcommand{\stardoctitle}     {\LaTeX\ --- A Document Preparation System}
%------------------------------------------------------------------------------

\newcommand{\stardocname}{\stardocinitials /\stardocnumber}
\renewcommand{\_}{{\tt\char'137}}     % re-centres the underscore
\markright{\stardocname}
\setlength{\textwidth}{160mm}
\setlength{\textheight}{230mm}
\setlength{\topmargin}{-2mm}
\setlength{\oddsidemargin}{0mm}
\setlength{\evensidemargin}{0mm}
\setlength{\parindent}{0mm}
\setlength{\parskip}{\medskipamount}
\setlength{\unitlength}{1mm}

%------------------------------------------------------------------------------
% Add any \newcommand or \newenvironment commands here

\newcommand{\BibTeX}{{\rm B\kern-0.05em{\sc i\kern-0.025em b}\kern-0.08em
    T\kern-0.1667em\lower0.7ex\hbox{E}\kern-0.125emX}}
\newcommand\bs{\char '134 }  % A backslash character for \tt font
\newcommand{\GloTeX}{{\rm G\kern-0.05em{\sc l\kern-0.025em o}\kern-0.08em
    T\kern-0.1667em\lower0.7ex\hbox{E}\kern-0.125emX}}
\newcommand{\IdxTeX}{{\rm I\kern-0.05em{\sc d\kern-0.025em x}\kern-0.08em
    T\kern-0.1667em\lower0.7ex\hbox{E}\kern-0.125emX}}
\newcommand{\TeXPS}{{\TeX\kern-.1667em\raise-.5ex\hbox{P}\kern-.125emS}}
\def\PS{{\sc Post\-Script}}
\font\tenlogo=logo10
\def\MF{{\tenlogo META}\-{\tenlogo FONT}}
\def\indexindent{\par\hangindent 50pt\hspace*{40pt}}
\def\tilde{\char126\relax}
\def\endverb{\par\endgroup}
\def\begverb#1{\begingroup\def\par{\leavevmode\endgraf}
\catcode`\\=12\catcode`\{=12
\catcode`\}=12\catcode`\$=12\catcode`\&=12
\catcode`\#=12\catcode`\%=12\catcode`\~=12
\catcode`\_=12\catcode`\^=12\obeyspaces\obeylines\tt
\parindent=0pt\catcode#1=0}
%------------------------------------------------------------------------------

\begin{document}
\thispagestyle{empty}
SCIENCE \& ENGINEERING RESEARCH COUNCIL \hfill \stardocname\\
RUTHERFORD APPLETON LABORATORY\\
{\large\bf Starlink Project\\}
{\large\bf \stardoccategory\ \stardocnumber}
\begin{flushright}
\stardocauthors\\
\stardocdate
\end{flushright}
\vspace{-4mm}
\rule{\textwidth}{0.5mm}
\vspace{5mm}
\begin{center}
{\Large\bf \stardoctitle}
\end{center}
\vspace{5mm}

\begin{center}
{\Large Description}
\end{center}
 
This is primarily a guide to the \LaTeX\ facilities available on
Starlink, NOT the use of \LaTeX\ commands and environments within
documents.

%------------------------------------------------------------------------------
%  Add this part if you want a table of contents

\setlength{\parskip}{2mm}
\tableofcontents
\setlength{\parskip}{\medskipamount}
\markright{\stardocname}

%------------------------------------------------------------------------------

\begin{center}
{\Large Conventions}
\end{center}

Throughout this document, the following conventions are used:

\begin{itemize}

\item VAX/VMS and UNIX are treated identically, except where one of them
is explicitly stated.

\item {\tt \$} is the prompt for VAX/VMS.

\item {\tt \#} is the prompt for UNIX.

\item Examples are given only for VAX/VMS, but apply to both operating
systems.

\item Examples are given for UNIX only if the command is different than
that for VAX/VMS.

\item {\tt csh}, or equivalent, is assumed for the UNIX shell.

\end{itemize}

%------------------------------------------------------------
% main part of doc starts here
%------------------------------------------------------------

\newpage

\section{Introduction}
\label{se:intro}

\LaTeX\ is a sophisticated system for laying out documents of many types 
using powerful `macro' commands. The style of the printed page may be 
controlled using the commands within the text of the document --- these commands
being interpreted by a powerful `typesetting' program to lay out the page.
Further processing then allows the page to be printed or displayed on a large
variety of devices. 

\LaTeX\ was developed by Leslie Lamport. Its commands and structures are based
upon the computer typesetting program called \TeX\, developed by Donald Knuth.
For further details of using the commands and structures offered by \TeX\ and 
\LaTeX , refer to the following books and documents:

\begin{itemize}
\item ``{\em \LaTeX\ A Document Preparation System --- User's Guide and 
 Reference Manual\/}''~by Leslie Lamport. (Addison-Wesley)
\item ``{\em The \TeX book\/}''~by Donald Knuth. (Addison-Wesley)
\item SUN/12 : ``{\em \LaTeX\ Cook-Book''\/}~by Mike Lawden.
\item SUN/93 : ``{\em \TeX\ --- A Superior document preparation system\/}''~
 by Jo Murray.
\item ETR 7/91 : ``{\em The \LaTeX\ Cookbook\/}''~by F Teagle.
\end{itemize}

All of the above should be available at your site.
For using \LaTeX\, the most useful are SUN/12
and the books by Teagle book and Lamport.

If you intend to use \LaTeX\ to write Starlink documentation (such as SUNs), 
please read the following paper:
\begin{itemize}
\item SGP/28 : ``{\em Starlink Documentation Production\/}''~by Mike Lawden.
\end{itemize}

This note is concerned with the options available for processing \LaTeX\
(and \TeX) source files, not with the guts of \LaTeX .


The only problem in using \LaTeX\  involves the files that
\LaTeX\ reads.  The file whose name you type with \mbox{\tt LATEX}
command is called the {\em root file}.  In addition to reading the root
file, \LaTeX\ also reads the files specified by \hbox{\verb|\input|}
and \hbox{\verb|\include|} commands.  
\LaTeX\ must know not only the names of these files but also in what
directories they are situated.  It will have no problem finding the correct
files if you follow two simple rules:
\begin{enumerate}
 \item Run \LaTeX\ from the directory containing the root file.
 \item Keep all files specified by \hbox{\verb|\input|} and 
      \hbox{\verb|\include|} commands in the same directory as the root
       file.
\end{enumerate}
If you follow these rules, you will never have to type a  directory
specification when using \LaTeX.

You should never break the first rule, otherwise \LaTeX\ will have
trouble finding auxiliary files.  (To run \LaTeX\ on someone else's
file, copy the file to your directory.) You can break the second
rule, by specifying a file from another directory in an
\hbox{\verb|\input|} or \hbox{\verb|\include|} command.

For example, 
\begin{enumerate}
\item On VAX/VMS, to include the file \mbox{\tt hisfile.tex} 
from XYZ's directory \hbox{\verb|[xyz.foo.bar]|} on VAX/VMS, you can type
\begin{verbatim}
      \input{[xyz.foo.bar]hisfile}
\end{verbatim}
but you cannot \hbox{\verb|\include|} a file from someone else's directory
because \LaTeX\ will try to write an auxiliary file into that directory.

\item On UNIX, to include the file \mbox{\tt hisfile.tex} from 
\hbox{\verb|\home\xyz\foo\bar|} you have to type
\begin{verbatim}
      \input{\home\xyz\foo\bar\hisfile}
\end{verbatim}
 
\end{enumerate}

For people who do not like to obey rules, here is exactly how \LaTeX\ finds its
files.  The root file is found by the operating system according to its usual
rules. The root auxiliary file is read and written in the directory from which
\LaTeX\ is run, all other auxiliary files are written in the directory in which
the corresponding input file was found. All file names specified in the \LaTeX\
input, including the names of document-style (\mbox{\tt .sty}) files specified
by the \hbox{\verb|\documentstyle|} command, are interpreted relative to the
directory from which \LaTeX\ is run.  If \LaTeX\ does not find a file starting
in this directory, it looks in the system directory defined to contain them. 
You can change the directories in which \LaTeX\ looks for its input files by
re-defining the appropriate logical names on VAX/VMS, and environment 
variables on UNIX. 

For example, on VAX/VMS putting the command
\begin{verbatim} 
      $ DEFINE TEX_INPUTS [ME.TEX],F$LOGICAL("TEX_INPUTS")
\end{verbatim}
in your \mbox{\tt LOGIN.COM} file causes \LaTeX\ to look for files
first in the current directory, then in the directory \mbox{\tt [ME.tex]},
and then in the system directory.  You might want to do this if
you have your own personal document-style files in \mbox{\tt [ME.tex]}.

Similarly, on Unix putting the following line in your {\tt .cshrc} or {\tt
.login}
\begin{verbatim}
 setenv TEXINPUTS /home/vl/tex:${TEXINPUTS}
\end{verbatim}
will have the same effect.



\section{\LaTeX\ Processing --- How it works}
\label{se:works}

The processing of \LaTeX\ source is a two stage process. Firstly, the
typesetting program is used to convert a source file to a DeVice Independent
(DVI) form. Then a translator is used to convert the DVI form to a format that
allows a particular display or print device to display or print the result.

Note that a page file quota (PGFLQUO) of greater than 4000 pages 
is required in order to run the \LaTeX\ processor, on VAX/VMS.

The process may be demonstrated by following an example.

\subsection{A VAX/VMS Example}
\label{se:example}
A good example of \LaTeX\ source comes with the installation.
Copy this to your current directory:
\begin{verbatim}
      $ copy tex_docs:sample.tex *.*
\end{verbatim}
the \LaTeX\ processor to convert the source 
\mbox{\tt .tex} file to the device independent \mbox{\tt .dvi} file:
\begin{verbatim}
      $ latex sample
\end{verbatim}
Note that the file extension is omitted. This is assumed to be \mbox{\tt .tex} 
unless you give the extension explicitly. 

Output will appear on your terminal telling you what is going on. When this is
finished a file \mbox{\tt sample.dvi} will be present in your directory.

The file DVI file needs to be processed for printing. Most sites have a Canon
Laser printer operating in native Canon mode. Some sites have a Canon printer
operating in \PS\ mode. The processing of a DVI file is different
depending on which mode your printer uses.


To process the \mbox{\tt .dvi} file for printing on a Canon laser printer in
native mode, the DVICAN processor must be used:
\begin{verbatim}
      $ dvican sample
\end{verbatim}
For \PS\ mode, the DVIPS processor is used:
\begin{verbatim}
      $ dvips sample
\end{verbatim}
Again, note that the file extension is omitted. The translator will assume an
extension of \mbox{\tt .dvi} unless you give it explicitly. More output will
appear on your terminal, telling you what is happening. 

When the processing is complete, either a file \mbox{\tt sample.dvi-can} will 
be present in your directory if you used DVICAN, or a file 
\mbox{\tt sample.dvi-ps} will be present if you used dvips.

All that is needed now is to print the file:
\begin{verbatim}
      $ print/passall/nofeed/notify/queue=sys_laser/delete sample.dvi-can
\end{verbatim}
for a \mbox{\tt .dvi-can} file, or:
\begin{verbatim}
      $ print/notify/queue=sys_ps/delete/form=post sample.dvi-ps
\end{verbatim}
for a \mbox{\tt .dvi-ps} file.

On VAX/VMS note the {\tt /delete} qualifier. This will delete the \mbox{\tt
.dvi-can} or  \mbox{\tt .dvi-ps} file after printing. This is a useful  trick
since print files from the DVI translators are often very large.

After the document has been printed, you will find that there are several files
in your directory with the name \mbox{\tt sample.nnn}. These are as follows:
\begin{itemize}
\item {\tt sample.tex} - The \LaTeX\ source file.
\item {\tt sample.dvi} - The device independent file.
\item {\tt sample.lis} - A log (transcript) of the \LaTeX\ run.
\item {\tt sample.aux} - Cross reference information.
\end{itemize}
All of these but the \mbox{\tt .tex} file may be deleted, as they are generated 
each time the \LaTeX\ processor is run. On VAX/VMS, regular use of the DCL
command \mbox{\tt PURGE} is recommended if you are making many \LaTeX\ runs.

Files with other extensions may also be generated, depending upon what
instructions are contained in the source file. These may also be deleted, but 
keeping them often makes future \LaTeX\ runs easier, since the information they
contain is used by the \LaTeX\ processor. For example, the \mbox{\tt .toc} 
file is a Table Of Contents. Several passes through the \LaTeX\ processor may 
be required to get the page numbering and references correct. Preserving the 
{\tt .toc} file gets around the problem.

\subsection{A Summary of \LaTeX\ Environments}
\label{se:environments}

\LaTeX\ has a construct called an {\em environment}, which is typed
\begin{quote}
{\tt $\backslash$begin\{{\sl name}\}} \ldots {\tt $\backslash$end\{{\sl name}\}}
\end{quote}
where {\em name} denotes the name of the environment.  Placing text within an
environment directs \LaTeX\ to change the way it formats that text.  For
example, text between \hbox{\verb|\begin{center}|} and
\hbox{\verb|\end{center}|} is centered.   Environments include:
\begin{itemize}
\item
{\tt document} ~for the main body of the document;
\item
{\tt itemize, enumerate, description} ~for producing lists of items;
\item
{\tt array, tabbing, tabular} ~for creating tables of numbers and text;
\item
{\tt math} ~for producing mathematical and special symbols;
\item
{\tt eqnarray, equation} ~for producing annotated equations;
\item
{\tt abstract, titlepage} ~for making abstracts and do-it-yourself 
title pages;
\item
{\tt quotation, quote, verse, verbatim} ~for producing displayed
paragraphs as quotations, as verses or as typed;
\item
{\tt picture} for making simple pictures;
\item
{\tt table, figure} ~for creating annotated table and figure captions,
situating them;
\item
{\tt bibliography} ~for making bibliography and reference lists.
\end{itemize}
You may define your own environments or redefine existing ones.\medskip\\
There are facilities for producing:y


\begin{itemize}
\item
different fonts (American for `founts'), accents;
\item
labelled and annotated chapters, sections, subsections, paragraphs;
\item
cross-references to sections, tables, figures, references, page numbers;
\item
an index, a glossary;
\item
tables of contents, figures and tables;
\item
footnotes, marginal notes and running headings;
\item
your own commands and environments;
\item
changes of type size;
\end{itemize}
and many others. 

The use of these environments is described in detail in 
Lamport's book. SUN/12 and Teagle's book 
give numerous useful examples of the results that can be obtained from the 
various \LaTeX\ environment commands. 

\subsection{\LaTeX\ Processing Options on VAX/VMS}
\label{se:latexopts}

The \LaTeX\ processor can be used in a simple manner, just as it comes. There
are however several command--line options that may be specified, to make the
\LaTeX\ processor do interesting optional things.

Note in passing that the \LaTeX\ processor is just the \TeX\
processor, invoked by a different command, causing the \LaTeX\ specific
commands to be recognised as well as \TeX\ commands.

The following options are provided:
\begin{itemize}

\item  {\bf   /BATCH :} Run \TeX\ in batch mode sending no output to the
terminal and ending with a fatal error if input is necessary. The default 
is /NOBATCH. This option suppresses all but the initial ``{\em This is \TeX\,
Version 3.141 on VAX/VMS ...}\/'' message.

\item  {\bf   /CONTINUE :} Indicates that \TeX\ is to continue execution after
the editor is invoked with an `E' response at an error prompt. The default is 
/NOCONTINUE.

\item  {\bf   /DIAGNOSTICS :} Indicates that an LSE Diagnostics file be
written. A file name can be specified using /DIAGNOSTICS=fn. The default is 
/NODIAGNOSTICS.

\item  {\bf   /DVI\_FILE :} Indicate the name of the DVI file to write.
The default is to use the name of the \TeX\ job for the DVI file name. 
This qualifier is negatable ({\em i.e.} /NODVI\_FILE). Negating this qualifier 
is useful for processing large documents where tables-of-contents and 
reference labels mean that the file must be processed several times. Only at 
the last pass need a DVI file be produced.

\item  {\bf   /EDITOR :} Indicate the name of the editor to be used at the `E'
response. The options are:
\begin{itemize}
\item Callable\_EDT
\item Callable\_TPU
\item Callable\_LSE
\item Callable\_TECO
\item The name of a command to be run in a subprocess which will 
take three arguments:
\begin{enumerate}
\item The name of the file to edit.
\item The line number with the error.
\item The column number of the error.
\end{enumerate}
\end{itemize}
If the value given with /EDITOR ends in a colon, \TeX\ will assume that it's 
a logical name and attempt to translate it. The default is /EDITOR=TEX\_EDIT:. 
This qualifier is negatable.

\item  {\bf   /EIGHT\_BIT :} Indicates that characters with values in the 
numerical range 160..254 may be output as those values, rather than conversion 
to the form \verb+^^a0..^^fe+. If {\bf all} terminals at a site support such 
characters, then it is safe to build a pre-loaded format file with this 
qualifier on the TEX/INIT command. Since the character codes are stored in the 
format file, this qualifier has no effect on production versions of \TeX.

\item  {\bf   /FORMAT :} Indicate the name of a format to pre-load when
running. The default varies depending on the specific verb used. This is how
\LaTeX\ processing is invoked when the \TeX\ processor starts --- the verb
{\tt LATEX} is defined as \mbox{\tt TEX/FORMAT=LPLAIN}.
This qualifier is negatable.

\item  {\bf   /INIT  :} Run Ini\TeX\ rather than \TeX. The default is /NOINIT. 
The INITEX symbol is set to TEX/INIT/NOFORMAT.

\item  {\bf   /JOBNAME\_SYMBOL :} Indicate the name of a DCL symbol to which
the \TeX\ jobname is to be written. The default is 
/JOBNAME\_SYMBOL=TEX\_JOBNAME. This qualifier is negatable. Negation causes 
the symbol to not be written.

\item  {\bf   /LOG\_FILE :} Indicate the name of the log file (transcript of 
the run) to write. The default is to use the name of the \TeX\ job for the 
log file name, and the extension .lis. If you specify just a name, the
extension defaults to .lis. This qualifier is negatable.

\item  {\bf   /TEXFONTS,   /TEXFORMATS,     /TEXINPUTS :}  
These qualifiers are not intended to be used by the end-user; they specify the 
names of the format and pool files, and input files respectively. They are 
provided to allow sites to customize these values without recompiling \TeX.

\end{itemize}

\subsection{\LaTeX\ Errors}
\label{se:latexerrs}

If \LaTeX\ encounters an error it is reported, the problem line is displayed,
and you are prompted with a `{\tt ?}'.  Various options are available in such
an event, the simplest being to respond with a {\tt $<$cr$>$} which causes
\LaTeX\ to continue, improvising what it considers the optimum solution to its
dilemma.  (A HELP facility within the \LaTeX\ process does exist at this
point.) You can edit your file to correct the error by typing \mbox{\tt E}.
This will  stop temporarily the execution of \LaTeX\ and bring you into your
favourite  editor at the point where the error was reported, but only if you
have defined the logical name TEX\_EDIT to be your favourite editor (see
Section \ref{se:latexopts}). Note that when you leave the editor a new file has
been created, but \LaTeX\ still uses the initial version of your file where the
error has not been corrected. 

The opportunity to correct the \LaTeX\ file within the \LaTeX\
process does exist at this point. If the logical name TEX\_EDIT or
the parameter /EDITOR have not been defined, \LaTeX\ will respond with a message
indicating the filename and the line where the error has occurred and stop. In
this case you can correct the error by editing the file. In all cases, after
the error has been corrected, it is necessary to re-activate \LaTeX\ to process
your file.

If you are fed up of answering repeated prompts, perhaps because of some
serious error, type \mbox{\tt Q}, and this will put \LaTeX\ into
{\em batch mode.} Then examination of the \mbox{\tt .lis} file will give the 
unexpurgated details.

However, don't panic---a long series of errors can be generated by just
one mistake. Often all you have done is to omit a `{\tt \$}' or a brace, or
not matched a \hbox{\verb|\begin{}|} with an \hbox{\verb|\end{}|}, or tried
to use a special character as an ordinary printing symbol, or \TeX\
itself has been unable to find a suitable break point for a line 
(``Underfull hbox \ldots'' or ``Overfull hbox \ldots'') or a page 
(``Underfull vbox \ldots'' or ``Overfull vbox \ldots'').  

The \LaTeX\ manual gives information on
interpreting error and warning messages, and how to fix or circumvent them.
Do not correct errors one at a time---go through the whole file, or as much 
as possible in the case of catastrophic errors; only then modify your
source file and try again.

If you want to stop \LaTeX\ in the middle of its execution, perhaps
because it is printing a seemingly unending string of uninformative
error messages, type {\em Control-C\/} (press $C$ while holding down
the key labeled {\em CTRL\/}).  This will make \LaTeX\ stop as if it
had encountered an ordinary error, and you can return to DCL command
level by typing \mbox{\tt X}, as described in the manual.  If typing {\em
Control-C\/} doesn't work, typing {\em Control-Y\/} will get you
immediately to DCL command level.

\subsection{Spelling}
\subsubsection{Spelling on VAX/VMS}
\label{se:vaxspell}

To use the \mbox{\tt spell} utility for finding spelling errors in a
\LaTeX\ input file named \mbox{\tt myfile.tex}, type the following
command:
\begin{verbatim}
      $ spell myfile
\end{verbatim}

For details on how to use the \mbox{\tt spell} utility, see LUN/26 (PRO), or 
consult your local documentation index {\tt LDOCSDIR:ANALYSIS.LIS} for local 
documentation.

\subsubsection{Spelling on UNIX}
\label{se:unixspell}
To use the spell utility for finding spelling errors in a
\LaTeX\ input file named \mbox{\tt myfile.tex}, type the following
command:
\begin{verbatim}
      # ispell myfile.tex
\end{verbatim}

For details on how to use the \mbox{\tt ispell} utility, read the manual pages.
(i.e. {\tt sman ispell}). 

\section{DVI File Translators}
\label{se:dvi}

\LaTeX\ produces files which are DeVice Independent --- the DVI files. To get a
viewable or printable version, you must use a DVI translator program, sometimes
called a previewer, which can interpret the DVI code for the device you
require. 

There are a variety of DVI translators available:
\begin{itemize}
\item XDVI --- previewer for Xwindows terminals, and workstations running X11
or DECwindows.
%\item DVIDIS --- previewer for Vaxstations running VWS.
\item DVI2VDU --- previewer for Pericom, Graphon and some VT terminals.
\item DVICAN --- translator for Canon LBP-8II and LBP-8III Laser printers.
\item DVICA2 --- translator for Canon LBP-8A2 laser printer (not available on
Solaris 2.x and OSF/1).
\item DVIPS --- translator for \PS\ devices.
\item DVIPSA4 --- translator for \PS\ devices (available only on VAX/VMS).
\item DVIPRI --- translator for line printer devices (available only on
VAX/VMS).
\end{itemize}

Laser printer paper is expensive, and using one of the DVI translators for a
laser printer while laying out a document is a waste. If you have access to
Xwindows terminals, Vaxstations, SUNs, DECstations or Pericom and Graphon
graphics terminal, you should try using the DVI previewers. Much time and
effort can be saved, as well as the odd forest.

\subsection{XDVI}
\label{se:dvixdvi}

The \verb+xdvi+ translator will work on Vaxstations, DECstations and Alpha
workstations running DECwindows, Motif or X11, VT1x00 series Xwindows terminals
and SUNs running X11.  

On the VT1x00 series terminals, you can use \verb+xdvi+ provided that \verb+xdvi+ knows where
to send the Xwindows output. A DECwindows session will know this, but a normal
terminal session must be told. The DCL command 
\begin{verbatim}
                               tcpip
$ set display/create/transport=decnet/node=<nodename>
                               lat 
\end{verbatim}
is used for this, 
but you have to know various things about the specific X-terminal to use it,
(i.e. the nodename for the type of transport you use).

On UNIX, the
equivalent command will be 
\begin{verbatim} 
# setenv DISPLAY <workstation_name>:0 
\end{verbatim}


The procedure XDISPLAY will do the necessary {\tt set/display} automatically
by interrogating your terminal and access method. {\em You must use a terminal
session accessing the VAX, SUN or DECstation host via LAT, TCP/IP or DECnet,
not a session through the  VT1x00 host port.}

To use XDISPLAY, type:
\begin{verbatim}
      $ xdisplay 
\end{verbatim}
and you will get a message similar to:
\begin{verbatim}
      The X client is MAVAD
      The X server is a remote X terminal connected via LAT.
\end{verbatim}
You only need to set a DISPLAY once per session.

For {\tt xdisplay} to work properly you need to authorize the remote
workstation to open a window on the local workstation. To do this:
\begin{itemize}
\item If the local workstation is a VAX/VMS workstation, pull down the option
menu from the session manager and select security. Add the remote workstation's
name (or TCP/IP number), the preferred transport and the remote username (or an
asterisk) to the list of authorized workstations.
\item if the local workstation is a UNIX workstation then type
\begin{verbatim}
	# xhost +<workstation_name>
\end{verbatim}
where \verb+<workstation_name>+ is the full name or valid alias or TCP/IP
address of the remote workstation. The following commands are equivalent:
\begin{verbatim}
	# xhost +axp1.ast.man.ac.uk
	# xhost +axp1
	# xhost +130.88.9.41
\end{verbatim}
\end{itemize}

\subsection{Command Syntax}
The command:
\begin{verbatim}
      $ xdvi <filename>
\end{verbatim}
will create a window on the default X-windows display and display the first 
page of {\tt <filename>.dvi}.

This program has the capability of showing the file shrunken by various
(integer) factors, and also has a ``magnifying glass'' which allows one
to see a small part of the unshrunk image momentarily.

Before displaying any page or part thereof, \verb+xdvi+ checks to see if the
DVI file  has changed since the last time it was displayed. If this is the
case, \verb+xdvi+  will re-initialize itself for the new DVI file. For this
reason, exposing parts  of the \verb+xdvi+ window while \TeX\ is running should
be avoided. This feature  allows you to preview many versions of the same file
while running \verb+xdvi+ only  once.

\subsubsection{Options}
\label{se:dvixdvioptions}
In addition to specifying the DVI file (with or without the {\tt .dvi}),
\verb+xdvi+ supports the following command line options.
If the option begins with a ``{\tt+}'' instead of a ``{\tt-}'',
the option is restored to its default value.  By default, these options can
be set via the resource names given in parentheses in the description of
each option
\begin{list}%
{}%
{\settowidth{\labelwidth}{\tt-icongeometry {\em geometry}}
\settowidth{\labelsep}{aaaa}
\settowidth{\rightmargin}{aaa}
\addtolength{\labelwidth}{\labelsep}
\setlength{\leftmargin}{\labelwidth}}
%
\item[\tt+ \em page]
Specifies the first page to show.  If {\tt+} is given without a
number, the last page is assumed; the first page is the default.

\item[\tt -altfont \em font]
({\tt .altFont})
Declares a default font to use when the font in the DVI file cannot be found.
This is useful, for example, with \PS\ fonts.

\item[\tt-background \em colour]
Same as {\tt-bg}.

\item[\tt-bd \em colour]
({\tt .borderColor})
Determines the colour of the window border.

\item[\tt-bg \em colour]
({\tt .background})
Determines the colour of the background.

\item[\tt-bordercolor \em colour]
Same as {\tt-bd}.

\item[\tt-borderwidth \em width]
Same as {\tt-bw}.

\item[\tt-bw \em width]
({\tt .borderWidth})
Specifies the width of the border of the window.

\item[\tt-copy]
({\tt .copy})
Always use the {\tt copy} operation when writing characters to the display.
This option may be necessary for correct operation on a colour display, but
overstrike characters will be incorrect.

\item[\tt-cr \em colour]
({\tt .cursorColor})
Determines the colour of the cursor.  The default is the colour of the page
border.

\item[\tt-density \em density]
({\tt .densityPercent})
Determines the density to be used when shrinking bitmaps for fonts.
A higher value produces a lighter font.  The default value is 40.

\item[\tt-display \em host::display.screen]
Specifies the host,display and screen to be used for displaying the DVI file.
The display must be specified in the form {\tt host::display.screen}.
The default is normally obtained on VAX/VMS from the logical name 
``{\tt DECW\$DISPLAY}'', or on UNIX by the environment variable \verb+DISPLAY+.


\item[\tt-fg \em colour]
({\tt .foreground})
Determines the colour of the text (foreground).

\item[\tt-foreground \em colour]
Same as {\tt-fg}.

\item[\tt-gamma \em gamma] ({\tt .gamma}) Controls the interpolation of colours
in the greyscale anti-aliasing colour palette. The default value is 1.0. For
0$<$ 1, the fonts will be lighter (more like the background), and for {\tt
gamma} $>$ 1, the fonts will be darker (more like the foreground). Negative
values behave the same way, but use a slightly different algorithm.

\item[\tt-geometry \em geometry]
({\tt *geometry})
Specifies the initial geometry of the window.

\item[\tt-hl \em colour]
({\tt .highlight})
Determines the colour of the page border. The default is the foreground colour.

\item[\tt-hush]
({\tt .Hush})
Causes \verb+xdvi+ to suppress all suppressible warnings.

\item[\tt-hushchars]
({\tt .hushLostChars})
Causes \verb+xdvi+ to suppress warnings about references to characters which are not
defined in the font.

\item[\tt-hushspecials]
({\tt .hushSpecials})
Causes \verb+xdvi+ to suppress warnings about  \hbox{\verb|\special|} strings which
cannot process.

\item[\tt-icongeometry \em geometry]
({\tt .iconGeometry})
Specifies the initial position for the icon.

\item[\tt-iconic]
({\tt .iconic})
Specifies the \verb+xdvi+ window to start into the iconic state. The default is to
start with the window open.

\item[\tt-keep]
({\tt .keepPosition)}
Sets a flag to indicate that \verb+xdvi+ should not move to the home position when
moving to a new page. (See also {\tt `k'} in section \ref{se:dvixdvikeys}.)

\item[\tt-l]
({\tt .listFonts})
Causes the names of the fonts used to be listed.

\item[\tt-margins \em dimensions]
({\tt .Margin})
Specifies the size of both the top and side margins.  This
should be a decimal number optionally followed by ``cm'' (e.g. 1.5 or 3cm)
giving the measurements in inches or centimetres, respectively. \verb+xdvi+ determines
the ``home'' position of the page within the window as follows.  If the entire
page fits in the window, then the margin settings are ignored.  If, even
after removing the margins from the left, right, top, and bottom, the page
still cannot fit in the window, then the page is put in the window such that
the top and left margins are hidden, and presumably the upper left-hand corner
of the text on the page will be in the upper left-hand corner of the window.
Otherwise, the text is centered in the window.  See also `{\tt M}' under
section \ref{se:dvixdvikeys}.


\item[\tt-mgs{[{\em n}] \em size}]
({\tt .magnifierSize[{\em n}]})
Where $n=1,2,3,4,5$.
Specifies the size of the window to be used for the ``magnifying glass''
for Button {\em n}.  See section \ref{se:dvixdvimouse}.  Defaults are 200,
350, 600, 900, and 1200.

\item[\tt-mgs \em size]
Same as {\tt -mgs1}.

\item[\tt-nogrey]
({\tt .grey})
Turns off the use of greyscale anti-aliasing when printing shrunken bitmaps.
(In this case the logic of the corresponding resource is the reverse: {\tt
-nogrey} corresponds to grey:off, {\tt +nogrey} to grey:on). See also under
{\tt `G'} in section \ref{se:dvixdvikeys}.

\item[\tt-offsets \em dimensions]
({\tt .Offset})
Specifies the size of both horizontal and vertical offsets of the output on the
page. This should be a decimal number optionally followed by ``cm'' (e.g. 1.5
or 3cm) giving the measurements in inches or centimetres, respectively. The
default page orientation is always one inch over and down from the top-left
page corner.

\item[\tt-p \em pixels]
({\tt .pixelsPerInch})
Defines the size of the fonts to use, in pixels per inch.  The
default value is 300.

\item[\tt-paper \em papertype]
({\tt .paper})
Specifies the size of the printed page.  This may be of the form
$W\times H$ (or $W\times H${\tt cm}), where $W$ is the width in
inches (or cm) and $H$ is the height in inches (or cm), respectively.
There are also synonyms which may be used:  us ($8.5\times 11$), usr ($11\times
8.5$),
legal ($8.5\times 14$), foolscap ($13.5\times 17$), as well as the ISO sizes 
A1-A7,
B1-B7, C1-C7, A1R-A7R (A1-A7 rotated), etc.  The default size is
$21\times 29.7$cm (A4 size).

\item[\tt-rv]
({\tt .reverseVideo})
Causes the page to be displayed with white characters on a
black background, instead of vice versa.

\item[\tt-s \em shrink]
({\tt .shrinkFactor})
Defines the initial shrink factor.  The default value is 3.

\item[\tt-sidemargin \em dimensions]
({\tt .sideMargin})
Specifies the side margin (see {\tt -margin}).

\item[\tt-thorough]
({\tt .thorough})
\verb+xdvi+ will usually try to ensure that overstrike characters (e.g.
\verb+\notin+) are printed correctly.  On monochrome displays, this is always
possible with one logical operation, either {\tt AND} or {\tt OR}.  On
colour displays, however, this may take two operations, one to set the
appropriate bits and one to clear other bits.  If this is the case, then
by default \verb+xdvi+ will instead use the {\tt copy} operation,
which does not handle overstriking correctly.  The ``{\tt thorough}'' option
chooses the slower but more correct choice.  See also {\tt-copy}.

\item[\tt-topmargin \em dimension]
({\tt .topMargin})
Specifies the top and bottom margins (see {\tt-margins}).

\item[\tt-version]
Pint information on the version of \verb+xdvi+.

\item[\tt-xoffset \em dimensions]
({\tt .xOffset})
Specifies the size of the horizontal offset of the output on the page (see
{\tt-offsets}.

\item[\tt-yoffset \em dimensions]
({\tt .yOffset})
Specifies the size of the vertical   offset of the output on the page (see
{\tt-offsets}).

\end{list}


\subsubsection{Keystrokes}
\label{se:dvixdvikeys}

\verb+xdvi+ recognizes the following keystrokes when typed in its window.
Each may optionally be preceded by a (positive or negative) number, whose
interpretation will depend on the particular keystroke. Note that the
keystrokes are case sensitive.


\begin{list}%
{}%
{\settowidth{\labelwidth}{\em Control \tt L}
\settowidth{\labelsep}{aaaa}
\settowidth{\rightmargin}{aaa}
\addtolength{\labelwidth}{\labelsep}
\setlength{\leftmargin}{\labelwidth}}
%
\item[\tt q]
Quits the program.  {\em Control \tt C}, {\em Control \tt D} and  {\em Control
\tt Z} will do this, 
too.

\item[\tt n]
Moves to the next page (or to the {\em n}th next page if a number is given).
Synonyms are `{\tt f}', Space, Return, Line Feed and $<$Next Screen$>$.

\item[\tt p]
Moves to the previous page (or back n pages).  Synonyms are
`{\tt b}', {\em Control \tt H}, Delete and $<$Prev Page$>$.

\item[\tt g]
Moves to the page with the given number.  Initially, the first page is assumed
to be page number 1, but this can be changed with the `{\tt P}' keystroke. 
If no page number is given, then it goes to the last page.

\item[\tt P]
``This is page number n.''  This can be used to make the `{\tt g}'
keystroke refer to actual page numbers instead of absolute page numbers.

\item[\tt {\em Control} L]
Re-displays the current page.

\item[\tt \symbol{94}]
Move to the ``home'' position of the page.  This is normally the upper
left-hand corner of the page, depending on the margins as described in
the \-margins option, above.

\item[\tt u]
Moves up two thirds of a window-full. The $<$Up Arrow$>$ key is a synonym for
this keystroke.                       

\item[\tt d] Moves down two thirds of a window-full. The $<$Down Arrow$>$ key
is a synonym for this keystroke.                       

\item[\tt l] Moves left two thirds of a window-full. The $<$Left Arrow$>$ key
is a synonym for this keystroke.                       

\item[\tt r] Moves right two thirds of a window-full. The $<$Right Arrow$>$ key
is a synonym for this keystroke.                       

\item[\tt c]
Moves the page so that the point currently beneath the cursor is moved to
the middle of the window.  It also warps the cursor to the same place.

\item[\tt M]
Sets the margins so that the point currently under the cursor is the upper
left-hand corner of the text in the page. Not that this command itself does not
move the image at all. for details on how the margins are used see the
{\tt-margins} option in section \ref{se:dvixdvioptions}.

\item[\tt s]
Changes the shrink factor to the given number.  If no number is given, the
smallest factor that makes the entire page fit in the window will be used.
(Margins are ignored in this computation.)

\item[\tt S]
Sets the density factor to be used when shrinking bitmaps.  This should
be a number between 0 and 100; higher numbers produce lighter characters.

\item[\tt R]
Forces the dvi file to be reread.  This allows you to preview many versions
of the same file while running \verb+xdvi+ only once.

\item[\tt k]
Normally when \verb+xdvi+ switches pages it moves to the home position as well.
The `{\tt k}' keystroke toggles a `keep-position' flag which, when set, will
keep the same position when moving between pages.  Also `{\tt 0k}' and `{\tt
1k}' clear and set this flag, respectively. See also the {\tt-keep} option in
section \ref{se:dvixdvioptions}.

\item[\tt G]
The key toggles the use of greyscale anti-aliasing for displaying shrunken
bitmaps. In addition the key sequence `{\tt 0G}' and `{\tt 1G}' clear and set
this flag respectively. See also the {\tt-nogrey} option in section
\ref{se:dvixdvioptions}. 
\end{list}


\subsubsection{Mouse Actions}
\label{se:dvixdvimouse}
If the shrink factor is set to any number other than one, then clicking
any mouse button will pop up a ``magnifying glass'' which shows the unshrunk
image in the vicinity of the mouse click.  This sub-window disappears when
the mouse button is released.  Different mouse buttons produce different sized
windows, as indicated by the {\tt -mgs} option, in section
\ref{se:dvixdvioptions}.  Moving the cursor
while holding the button down will move the magnifying glass.

Also, the scroll-bars (if present) behave in the Xwindows standard way:  pushing
Button 2 in a scroll-bar moves the top or left edge of the scroll-bar to that
point and optionally drags it; pushing Button 1 moves the image up or right by
an amount equal to the distance from the button press to the upper left-hand
corner of the window; pushing Button 3 moves the image down or left by the same
amount. Note that this is different than the way DECwindows normally defines
the actions of the mouse buttons in scroll-bars.


\subsubsection{Logical Name}
\label{se:dvixdvilognames}

Some logical names can be defined to override the values defined when
\verb+xdvi+ was compiled.

\begin{itemize}



\item Unless the {\tt-display} option is used on the command line, \verb+xdvi+
uses the logical name \\ ``{\tt DECW\$DISPLAY}'' on VAX/VMS, or the environment
variable \verb+DISPLAY+ on UNIX, to specify which bit map display terminal to
use.  This logical name may be defined with the {\tt SET DISPLAY} command.

\item The logical name ``\verb+XDVIFONTS+'' determines the directory path(s)
searched for fonts in the following manner.  For compatibility with some
versions of \TeX, you may also use the logical name ``\verb+TEXFONTS+'' in
place of ``\verb+XDVIFONTS+''. \verb+xdvi+ also recognizes the
``\verb+PKFONTS+'' logical name, which is checked after ``\verb+XDVIFONTS+''
but before ``\verb+TEXFONTS+''.


\item The logical name ``\verb+XDVISIZES+'' must be set to indicate which
sizes of fonts are available.  It should consists of a list of numbers
separated by slashes.  If the list begins with a slash, the system
default sizes are used, as well.  Sizes are expressed in dots per
inch; decimals may be used for ``pxl'' files: for example, a 300 dots
per inch file magnified by half a step comes out to 1643 dots per five
inches, which should be encoded as 328.6.  The current default set of
sizes is 300,328.6,360,432,518.4,622,746.4.  \verb+xdvi+ will also try the actual
size of the font before trying any of the given sizes.


\item Virtual fonts are supported, although \verb+xdvi+ does not have any built-in
fonts to which they can refer.  The search path for .VF files can be specified
with the ``\verb+XDVIVFS+'' logical name in a similar manner to that for the
``\verb+XDVIFONTS+'' logical name.  \verb+xdvi+ will also check the ``\verb+VFFONTS+'' logical name
if the ``\verb+XDVIFONTS+'' logical name is not defined.  Virtual fonts are searched
for immediately after looking for the font as a normal font in the exact size
specified.

\end{itemize}

\subsubsection{Resource Names}

   All of the command line options may be set via the resource names given in
   the descriptions of the options.  This may be used to define a
   specific set of options as the default each time you run \verb+xdvi+.  To make use
   of this feature \begin{itemize}
\item on VAX/VMS, create a file named {\tt DECW\$XDEFAULTS.DAT} in the same
directory as the rest of your {\tt DECW*.DAT} files
\item on UNIX, create a file {\tt .Xdefaults} in your top level directory.
\end{itemize}
  Include in this file the resource names
   and arguments of each of the options you wish to specify.  For example:
\begin{verbatim}
   XDvi.copy: off
   XDvi.thorough: on
   XDvi.shrinkFactor: 2
   XDvi.Margin: 0.95
   XDvi*geometry: 1015x750+3+25
\end{verbatim}
   When \verb+xdvi+ is invoked, it would behave as if it had been invoked with the
   following command:
\begin{verbatim}
   $ xdvi +copy -thorough -s 2 -margins 0.95 -geometry 1015x750+3+25 dvifile
\end{verbatim}
   Specifying options on the command line will override any options specified
   via resource names in the  file.








\subsection{DVI2VDU}
\label{se:dvi2vdu}


{\tt dvi2vdu} is an interactive program; you can enter commands to
select a particular page for display, look at the overall format of the entire
page and then request a smaller region for closer examination.
The manner in which the page is displayed can be varied from a full,
accurate representation to a terse, fast display for when fine details are
unimportant.

To look at \hbox{\verb|filename.dvi|}, just type
\begin{verbatim}
        $ dvi2vdu/vdu=vt200          filename on a VT220 terminal
        $ dvi2vdu/vdu=mg100          filename on a PERICOM MG100 terminal
        $ dvi2vdu/vdu=graphpack      filename on a PERICOM GRAPHPAC terminal
\end{verbatim}

Some command options may be necessary if {\tt dvi2vdu} is to work properly.
In particular, the {\tt-v} option must correctly describe the type of VDU
you are using.

\subsubsection{Command options.}

{\tt dvi2vdu} command has a number of options, each of which
must be followed by a value.
If the same option appears more than once then the last value will be used.
Each option is followed by a value of a certain type:

\begin{list}%
{}%
{\settowidth{\labelwidth}{\em string}
\settowidth{\labelsep}{aaaa}
\settowidth{\rightmargin}{aaa}
\addtolength{\labelwidth}{\labelsep}
\setlength{\leftmargin}{\labelwidth}}

\item[\em string]
 is a string of alphanumeric characters,

\item[\em number]
 is a positive integer, 

\item[\em file]
 is a file or directory, and
\item[\em dimension]
is a positive integer or real number followed by a
two-letter unit: {\tt in}, {\tt cm}, {\tt mm},  {\tt pc}, {\tt pt} or {\tt px}.
(Most of these should be familiar from \TeX.  {\tt dvi2vdu} provides an
additional unit, {\tt px}, which stands for ``paper pixels''.
These two-letter sequences can later be used as commands to
change the units of dimensions.)
\end{list}

The following options are recognized:
\begin{list}
{}
{\settowidth{\labelwidth}{\tt-v {\em string}}
\settowidth{\labelsep}{aaaa}
\settowidth{\rightmargin}{aaa}
\addtolength{\labelwidth}{\labelsep}
\setlength{\leftmargin}{\labelwidth}}

\item[\tt -v \em string]
This option explicitly tells {\tt dvi2vdu} what type of VDU you are using. The value
of the {\tt TERM} environment variable is used if {\tt-v} is not given. For
example, if you're using a Tektronix 4010 terminal and {\tt TERM} does not
equal {\tt tek4010} then you'll need to type 
\begin{verbatim}
        $ dvi2vdu -v tek4010 filename
\end{verbatim}
{\tt dvi2vdu} will accept the  {\em string\/}
values in table \ref{table:dvi2vduvalues} in either upper or lower case.
\begin{table}[h]
\begin{center}
\begin{tabular}{ll}

AED483&    (AED 512 with 512 by 483 screen)\\
AED512&    (AED 512 with 512 by 512 screen)\\
ANSI&      (any ANSI compatible VDU; synonym = VT100)\\
REGIS&     (any ReGIS compatible VDU; synonyms = GIGI, VK100,
                                                   VT125, VT240)\\
VIS240&    (VISUAL 240; synonym = VIS241)\\
VIS500&    (VISUAL 500)\\
VIS550&    (VISUAL 550)\\
VT100132&  (any VT100 compatible VDU in 132 column mode)\\
VT220&     (VT220 using down-loaded chunky graphics)\\
VT640&     (VT100 with Retro-Graphics)\\
MG100&     (Pericom MG100)\\
MG200&     (Pericom MG200)\\
GRAPHPACK&     (Pericom GRAPHPACK)\\
\end{tabular}
\caption{Values accepted by {\tt dvi2vdu}}
\label{table:dvi2vduvalues}
\end{center}
\end{table}
It is assumed your terminal has been set up to obey XON/XOFF flow control.
This is the normal setting for terminals connected to a VAX so don't worry
about it unless problems occur while {\tt dvi2vdu} is displaying a page.

Some VDUs require special settings for {\tt dvi2vdu} to work properly:

\begin{itemize}

\item The VIS500 and VIS550 terminals have function keys which control how
information on the screen is to be displayed.  {\tt dvi2vdu}
assumes graphic images and alphanumeric text can be seen at the
same time.  
Both terminal types should also have the Scale Factor set at 3:4.

\item The REGIS VDUs use ANSI escape sequences to update the dialogue region,
and the VIS550 VDU is assumed to be emulating a VT100.
Change the appropriate SET-UP value if your screen becomes full of junk.

\item If you want to select VT100132 you have to switch to
132 column mode yourself.
The VT220 VDU will, however, automatically switch to 132 columns
(and then to 80 columns when you quit).
\end{itemize}


\item[\tt-r \em number]

{\tt dvi2vdu} treats the imaginary sheet of paper on which a DVI page will appear as
a two-dimensional array of tiny dots called paper pixels. {\em number} is a
positive integer that defines the number of paper pixels per inch (horizontally
{\em and} vertically).  This value should match the resolution of the device
that will be used to print your document.

\item[\tt -x \em dimension]
\item[\tt -y \em dimension]

These two options define the width and height of the paper upon which your
document will eventually be printed.
Note that their default values specify A4 paper
(8.3in wide and 11.7in high).
Every time you select a page, {\tt dvi2vdu} uses these values
to check that all the printed material will fit on the paper.


\item[\tt -m \em number]

This option allows you to replace the magnification used
in the DVI file with some other value;
{\em number} is a positive integer 1000 $\times$ the desired magnification.
The given value should be chosen carefully so that the new font sizes
still correspond to existing \verb+PK+ files.

Unless you know exactly what you are doing you should avoid changing
the DVI magnification,
especially if your \TeX\ source file uses  \hbox{\verb|\magnification|}
 {\em and} true
dimensions.
You should only supply a replacement magnification if you intend to
print the DVI file with the same override.
             

\item[\tt -f \em file]

{\tt dvi2vdu} gets all its font information from \verb+PK+ files.
These files should be kept within the directory specified by this option.

\item[\tt -d \em file]
\TeX\ gets all its font information from \verb+TFM+ files.  For each \verb+TFM+ file there
are usually a number of corresponding \verb+PK+ files, each of which
contains the character shapes for a font at a particular size.
Although \TeX\ allows you to scale a font to virtually any size,
it is obviously impossible to provide an infinite number of \verb+PK+ files.
That is why it is best to
stick to the pre-defined \hbox{\verb|\magstep|}
values when scaling fonts---you're
much more likely to stay within the range of existing \verb+PK+ files.

{\tt dvi2vdu} will warn you if your document attempts to use a
non-existent \verb+PK+ file.
Rather than abort, it will load the \verb+PK+ file specified after {\tt-d} and continue
so you can look for more errors.
Paragraphs using this dummy font are likely to have ragged right margins.

The quickest way to check your DVI file for missing fonts is to type
the {\tt-S} command.  The resulting display will indicate any \verb+PK+ files that do
not exist.  You should not attempt to print a DVI file using such fonts.

\item[\tt-h \em file]

The {\tt ?} command reads the help file specified by this option.
The default help file contains a brief summary of all the commands.
\end{list}


\subsubsection{The Dialogue Region.}

If your command line is correct and if the {\tt -v} value matches the type of
terminal you're actually using, then {\tt dvi2vdu} will clear the screen
and display
\begin{verbatim}
Total pages=n   DVI page=0   TeX page=[0]   Next=>   Terse
Window at (h,v) wwd by wht
      Page at (minh,minv) pwd by pht   IN

Command:
\end{verbatim}
These lines represent the ``dialogue region''; the rest of the screen is
called the ``window region'' and should be blank at this stage.


The top two lines show status information.  The first status line shows:
\begin{itemize}
\item {\tt n}, the total number of pages in the DVI file.

\item The current DVI page and its corresponding \TeX\ page counters.
   Particular pages can be selected by their position in the DVI
   file (1 to n) or by the value of their \TeX\ counters.

\item The direction in which the {\tt N} command will select consecutive pages;
initially {\tt $>$}.  The {\tt $>$} and {\tt $<$} commands allow you to move
forwards or backwards through the DVI file.

\item The current display mode; initially {\tt Terse}.
   The {\tt T}, {\tt B} and {\tt F} commands allow you to switch between 
{\tt Terse}, {\tt Box} and
   {\tt Full} display modes.
\end{itemize}

The second status line shows:

\begin{itemize} \item The current location and size of the window. {\tt h} and
{\tt v} are horizontal and vertical coordinates that define the current paper
position of the top left corner of the window region. The following section has
details on the coordinate scheme used by {\tt dvi2vdu}.  The {\tt W} command allows
you to move the window to an absolute position; the {\tt U}, {\tt D}, {\tt L}
and {\tt R} commands allow relative positioning. {\tt wwd} and {\tt wht}
represent the current width and height of the window region; their values are
changed by the {\tt H} and {\tt V} commands. The initial window size depends on
the {\tt -r} value and the size and resolution of your VDU screen.

\item The location and size of the ``page rectangle''. {\tt minh}, {\tt minv},
{\tt maxh} and {\tt maxv} define the smallest rectangle containing all the
rules and characters in the current page. ({\tt minh,minv}) is the top left
corner of this rectangle; it is {\tt pwd} units wide and {\tt pht} units high.

\item The current units; initially inches.
   All numbers shown in this line are dimensions in terms of these units.
   The two letters shown at the end of the line correspond to the commands
   that allow you to switch between inches (IN),
   centimetres (CM), millimetres (MM), points (PT), picas (PC)
   and pixels (PX).
   Pixel values are shown as integral numbers and represent exact dimensions;
   all other values are shown as real numbers rounded to one decimal place.

\end{itemize}
Until a page is selected, most of the status values are
meaningless and set to zero.

The third line is initially blank.  {\tt dvi2vdu} displays messages of various
kinds in this line.  Some of these messages appear only briefly but may convey
helpful information.
Others are more important and indicate some sort of problem, such as
an invalid command or a page that won't fit on the paper;
in these cases {\tt dvi2vdu} will prompt you to hit the $<$RETURN$>$ key
before continuing.

The last line in the dialogue region is for entering commands.  The first
thing you normally want to do is choose a particular page for display.
For example, typing `{\tt 1}' will select the first page in the DVI file.
Many commands can be entered in the one command line.  Hit $<$RETURN$>$ to
execute the command(s).

\subsubsection{The Window Region.}

The number of paper pixels per inch is given by the {\tt -r} option.
A pixel can be either black (corresponding to a tiny blob of ink) or white
(no ink).
A typical DVI page contains characters from one or more fonts, and
perhaps a few rules.
A rule is simply a rectangular region of black pixels, usually in the shape of
a thin horizontal or vertical line.
A character is usually a more complicated pattern of black and white pixels.
Every character and rule has a
paper position ---or ``reference point''--- defined
by a pair of pixel values ({\tt h,v})
where {\tt h} is the horizontal coordinate and $v$ is the vertical coordinate.
{\tt dvi2vdu} uses a coordinate scheme in which the
position ({\tt 0,0}) is a pixel one inch in from the top and left edges of
the paper.  Vertical coordinates increase down the paper,
horizontal coordinates increase to the right. 

The window region is used to view the current DVI page.
{\tt dvi2vdu} treats this region of the VDU screen as a two-dimensional array
of dots called ``screen pixels'' (to distinguish them
from the paper pixels described above).
A screen pixel is the smallest possible area on a VDU screen that
can be drawn or erased; the greater the number of screen pixels in a given area,
the higher the resolution of the VDU.

The initial size of the window region depends on the VDU; the width and height
are set so that each paper pixel corresponds to exactly one screen pixel.
The most accurate representation of a page will occur at these values of Table
\ref{table:windowsize} since
neither horizontal nor vertical scaling is necessary:
\begin{table}[htb]
\begin{center}
\begin{tabular}{lc}
VDU&initial window size \\
\hline
AED483&             512 by 442\\
AED512&             512 by 471\\
ANSI&                80 by 20\\
REGIS&              768 by 400\\
VIS240&             800 by 500\\
VIS500, VIS550&    1024 by 688\\
VT100132&           132 by 20\\
VT220&              132 by 100\\
VT640&             1024 by 650\\
MG100&             1024 by 650\\
MG200&             1024 by 650\\
GRAPHPACK&         1024 by 650\\
\end{tabular}
\caption{Initial window sizes}
\label{table:windowsize}
\end{center}
\end{table}

Use the {\tt H} command to change the width (the Horizontal size), and the
and {\tt V} command to change the height (the Vertical size).
Try to avoid weird aspect ratios: `{\tt H0.1 V100}' is a perfectly legal
command string, but produces a very distorted display!
The higher the resolution of the VDU,
the greater the accuracy of such scaled displays.
The command string `{\tt HV}' will restore the window size to the above unscaled
dimensions.

Note the very low resolution of the ANSI and VT100132 VDUs.
Since the individual dots making up their screens cannot be turned on and off,
{\tt dvi2vdu} has to define a screen pixel to be an entire character position.
An ANSI screen typically consists of 24 lines of 80 characters,
therefore the initial window region is 80 pixels wide and 20 pixels high
(the top 4 lines are used for the dialogue region).
At more useful window sizes the resulting displays will be extremely crude.
Nevertheless, a variety of formatting errors can still be detected on such
terminals---just don't try proofreading your document!

The size and location of the window region are automatically set
every time a page is selected.
{\tt dvi2vdu} tries to show as much of the
paper (and presumably the page) as possible, but without too much distortion.
After comparing the shape of the paper with the shape of your
VDU's {\em initial} window region, and depending on the location of the
page, {\tt dvi2vdu} may show the entire paper,
or the top or bottom half, or the left or right half.
If any part of the page is off the paper then the entire paper {\em and} the
entire page will be shown.
Since most paper sizes have portrait dimensions (width $<$ height), and most
VDU screens have landscape dimensions (width $>$ height), {\tt dvi2vdu} will
normally show the top half of a sheet of paper containing the selected page.


\subsubsection{The Commands.}

In response to the `{\tt Command:}' prompt you can enter one or more of the
following commands in upper or lower case.  Multiple commands are
processed in the order given but the window region is only updated,
if necessary, at the end.  For example,
if you type `{\tt NFD}' then {\tt dvi2vdu} will get the Next page, switch to
Full display mode, move the window Down, and only then display the page.
If an illegal command is detected, any further commands are ignored.
Most commands consist of only one or two characters; some can be followed by
parameters.  Spaces before and after commands and parameters are optional.







\paragraph{Miscellaneous}
\begin{list}%
{}%
{\settowidth{\labelwidth}{\tt S}
\settowidth{\labelsep}{aaaa}
\settowidth{\rightmargin}{aaa}
\addtolength{\labelwidth}{\labelsep}
\setlength{\leftmargin}{\labelwidth}}


\item[\tt ?]Display brief help on the available commands.
\item[\tt S]
Show the option values being used, as well as statistics about the number of
fonts, characters and rules used on the current page.
You'll also be warned about any PXL files that don't exist.
\item[\tt Q]
Quit from {{\tt dvi2vdu}}\null.
({\em Control-C} will return you to the `Command:' level and 
{\em Control-Z} will suspend {\tt dvi2vdu}.)
\end{list}

\paragraph{Selecting a Page}

\begin{list}%
{}%
{\settowidth{\labelwidth}{\tt $i_0$.$i_1$. $\cdots$ .$i_9$}
\settowidth{\labelsep}{aaaa}
\settowidth{\rightmargin}{aaa}
\addtolength{\labelwidth}{\labelsep}
\setlength{\leftmargin}{\labelwidth}}

\item[$i$]Select the $i$th DVI page.  $i$ must be a positive integer from
1 to $n$ where $n$ is the total number of pages in the DVI file.

\item[$i_0$.$i_1$. $\cdots$ .$i_9$]
Select the DVI page whose \TeX\ page counters match the given specification.
$i_0$ to $i_9$ are integers separated by periods.  Any number of these
integers may be absent and trailing periods may be omitted.
An absent integer will match any value in the corresponding counter.
If more than one DVI page matches, the lowest will be chosen.
E.g., [ ] is equivalent to [.........] and will select the first DVI page,
even though the request matches every possible page.

\TeX\ stores the values of \hbox{\verb|\count0|},\hbox{\verb|\count1|},...,
\hbox{\verb|\count9|} in each DVI page. Plain \TeX\ only uses
\hbox{\verb|\count0|} to control page numbering; the remaining counters are set
to zero. Some macro packages may use the other counters for section or chapter
numbering; e.g., [2.5] might mean ``select page~2 from chapter~5''. When
showing the current \TeX\ page in the top status line, {{\tt dvi2vdu}} always shows
the value of \hbox{\verb|\count0|}, but trailing counters with zero values are
not shown. Note that plain \TeX\ uses negative values in  \hbox{\verb|\count0|}
to indicate page numbers in roman numerals (such as in a preface). In most
documents, the $i$th DVI page will be the same as \TeX\ page [$i$].


\item[\tt N]
Select the Next DVI page according to the current direction ($>$ or $<$).
Before any page has been requested,
N will select the first DVI page if the current direction is $>$, or
the last DVI page if the current direction is $<$.

\item[$>$]
Future {\tt N} commands will now select DVI pages in ascending order.
If $i$ is the current DVI page,
an {\tt N} command will get page $i+1$ (unless $i$ is the last DVI page).

\item[$<$]
Future N commands will now select DVI pages in descending order.
If $i$ is the current DVI page,
an N command will get page $i-1$ (unless $i$ is 1).
\end{list}

\paragraph{Changing The Way a Page is Displayed}

The \verb+T+, \verb+B+ and \verb+F+ commands provide three different ways of displaying a
page.  The choice of display mode depends on the capabilities of your VDU
and the level of detail you wish to see.
Whatever the display mode, the window region
is always updated in the following manner:
Visible paper edges are drawn first, followed by visible rules.
Visible characters are then shown on a font by font basis; those fonts
with the least number of characters on the page are drawn first.

While the window region is being updated, {{\tt dvi2vdu}} will check to see if
you've typed something at the keyboard.
You can hit the RETURN key to abort the display,
or you can change the display mode by hitting the \verb+T+, \verb+B+ or \verb+F+ keys
(without hitting RETURN).  There might be a slight delay between the time
you hit a key and the time something actually happens.

\begin{list}%
{}%
{\settowidth{\labelwidth}{\tt T}
\settowidth{\labelsep}{aaaa}
\settowidth{\rightmargin}{aaa}
\addtolength{\labelwidth}{\labelsep}
\setlength{\leftmargin}{\labelwidth}}

\item[\tt T]
Display a Terse representation of characters.
On high resolution VDUs the \TeX\ text fonts should be quite readable;
characters will be in
approximately the right position and may even be about the right size.
The display on low resolution VDUs can be a bit confusing; characters get
overwritten if they are too close together.  Don't worry too much
if you suddenly see many more spelling mistakes than usual!
Note that the text fonts will all look alike.
The only way to distinguish between different fonts is to note the
order in which characters are displayed (the less used fonts are drawn first).
Most VDUs assume all characters come from a \TeX\ text font and then attempt
to map them into similar-looking ASCII characters.
Characters from non-text fonts, such as math symbols, will usually appear
incorrect.

\item[\tt B]
Display Box outlines of the smallest rectangles containing all
black pixels in characters.
The reference point of a \TeX\ character is usually located near
the bottom left corner of one of these boxes.
Box mode is intermediate in speed between Terse and Full modes;
it is a quick and accurate way of checking
the alignment of entries in a table, for example.

\item[\tt F]
Display a Full representation of all pixels in characters.
This display is the most accurate but may take some time;
hit RETURN or switch to Terse or Box mode if you get bored.
The fact that the display mode can be changed {\em while the window is
being updated\/} can be very useful.
A good compromise between speed and accuracy is
to start off in Full mode so that math symbols and any other special
characters are displayed correctly, and to switch to Terse mode (by hitting T)
when the bulk of the text begins.
\end{list}

\paragraph{Changing the Units of Dimensions}

All the numbers in the second line of the dialogue region are dimensions in
terms of the units shown at the end of the line.
The parameters following some commands are also dimensions in terms of
these units.
Unlike the dimensions in \TeX, you don't explicitly type the units when you
need to specify a dimension after a {\tt dvi2vdu} command---simply
enter an integer value or real value.
This value is rounded up internally to the nearest paper pixel based on the
current units and the conversion factors shown below.
For example, `CM H5 V3.4' will change the units to centimetres and set the
window size to be 5cm wide and 3.4cm high.

\begin{itemize}
\item{\tt IN}
Dimensions are assumed to be in terms of inches
({\tt -r} defines the number of paper pixels per inch).

\item{\tt CM}
Dimensions are assumed to be in terms of centimetres
(2.54cm $=$ 1in).

\item{\tt MM}
Dimensions are assumed to be in terms of millimetres
(10mm $=$ 1cm).

\item{\tt PC}
Dimensions are assumed to be in terms of picas
(1pc $=$ 12pt).

\item{\tt PT}
Dimensions are assumed to be in terms of points
(72.27pt $=$ 1in).

\item{\tt PX}
Dimensions are assumed to be in terms of paper pixels.
\end{itemize}

\paragraph{Moving the Window}


The window region can be moved to any position over the current page.
You will be told if the window moves entirely outside the page rectangle
defined by $minh$, $minv$, $maxh$ and $maxv$.  If this does happen, movement
is restricted to {\em just outside\/} the edges to make it easier to get back
over the page using only the \verb+U+, \verb+D+, \verb+L+ and \verb+R+ commands.
Note that the location of the window is automatically set every time a
page is selected.  This position will normally be
the top left corner of the paper; i.e., $(-1,-1)$ in inches.
The parameters $h$ and $v$ are dimensions ranging from $-480$ inches to
$+480$ inches (for those of you \TeX ing billboards).
\begin{list}%
{}%
{\settowidth{\labelwidth}{\tt W $h$,$v$}
\settowidth{\labelsep}{aaaa}
\settowidth{\rightmargin}{aaa}
\addtolength{\labelwidth}{\labelsep}
\setlength{\leftmargin}{\labelwidth}}

\item[\tt W $h$,$v$]
Move the Window region's top left corner to the given paper position.
$h$ is the horizontal coordinate, $v$ is the vertical coordinate.
If $h$ and $v$ are absent then the window is moved to $(minh{,}minv)$,
the top left corner of the page rectangle.

\item[\tt U $v$]
Move the window Up $v$ units.
If $v$ is absent then move up an amount equal to the window's current height.

\item[\tt D $v$]
Move the window Down $v$ units.
If $v$ is absent then move down an amount equal to the window's current height.

\item[\tt L $h$]
Move the window Left $h$ units.
If $h$ is absent then move left an amount equal to the window's current width.

\item[\tt R $h$]
Move the window Right $h$ units.
If $h$ is absent then move right an amount equal to the window's current width.
\end{list}

\paragraph{Changing the Size of the Window}


The \verb+H+ and \verb+V+ commands are used to change the
width and height of the window region.
It is up to you to maintain a sensible aspect ratio.
The location of the window will not change unless the page becomes invisible.
Note that the width and height of the window are automatically set every time a
page is selected; the values chosen will depend on the paper dimensions.
The parameters $wd$ and $ht$ are dimensions ranging from 1 pixel to 480 inches.

\begin{list}%
{}%
{\settowidth{\labelwidth}{\tt H $wd$}
\settowidth{\labelsep}{aaaa}
\settowidth{\rightmargin}{aaa}
\addtolength{\labelwidth}{\labelsep}
\setlength{\leftmargin}{\labelwidth}}

\item [\tt H $wd$]
Set the Horizontal size of the window to the given width.
If $wd$ is absent then set window width to its initial, unscaled value.

\item[\tt V $ht$]
Set the Vertical size of the window to the given height.
If $ht$ is absent then set window height to its initial, unscaled value.
\end{list}


\subsection{DVICAN \& DVICA2 }
\label{se:dvidvican}

Because there are two types of Canon printers in use within Starlink, two 
Canon Printer translators are provided. There is the additional complication
that some Canon printers may be operating in \PS\ mode. 

If your site has one printer operating in \PS\ mode, the \verb+dvican+ and
\verb+dvica2+
translators will be disabled. If you type:
\begin{verbatim}
      $ dvican <filename>
\end{verbatim}
or
\begin{verbatim}
      $ dvica2 <filename>
\end{verbatim}
you will get a message similar to:
\begin{verbatim}
      dvican has been withdrawn. Please use dvips 
\end{verbatim}

For sites with a Canon printer operating in Native mode, the correct \verb+dvican+
translator will be enabled as \verb+dvican+, whichever model is in use. 

If your site has both types of Canon printer, two types of translators will be
available. \verb+dvican+ will be for later versions of the Canon  printers,
\verb+dvica2+ will be for the early version. Both take the same arguments. {\bf
NOTE:} {\em \verb+dvica2+ is not available on Solaris 2.x and OSF/1.}

\subsubsection{Options}
\label{se:dvidvicanoptions}

The \verb+dvican+ and \verb+dvica2+ commands take a number of qualifiers consisting 
of a hyphen followed by a single letter optionally followed by a parameter. 
For example
\begin{verbatim}
      $ dvican <filename> -o3
\end{verbatim}
or 
\begin{verbatim}
      $ dvica2 <filename> -o3
\end{verbatim}
will process page 3 of the document only.
Note that you must not put any spaces between the qualifier and its
parameters.
The syntax of qualifiers means that it is impossible to process a file whose
name begins with a hyphen.

The following qualifiers are recognized:

\begin{list}{}{\settowidth{\labelwidth}{\tt-f\em filename}
\setlength{\leftmargin}{\labelwidth}
\addtolength{\labelwidth}{\labelsep}}

\item[\tt-b] Process pages in reverse order; this can be useful on the A2 model
printers which stack their pages face up and therefore in the reverse of
the order in which they were printed.

\item[\tt-c\#]  Print {\tt\#} copies of each page; a photocopier is cheaper so
this qualifier shouldn't really be needed.

\item[\tt-f\it filename] Specify an alternative font substitution file.

\item[\tt-h] Select landscape mode.

\item[\tt-l] If any errors occur the messages are written to a file with
the extension {\tt .dvi-err} as well as to the terminal. {\tt-l} suppresses
the writing of this log file.

\item[\tt-m\#] Reset magnification to {\tt\#}. For values greater than 25
{\tt\#} is taken to be a \TeX\ magnification; the default is 1500 (not 1000
because the Canon is a 300dpi printer) and must correspond to one of the
standard \TeX\ magnifications. Values less than 25 are equivalent to the
corresponding \TeX\ \verb+\magstep+ magnification ($-1$ equals
\verb+\magstepahalf+).

\item[\tt-o\#] Process page {\tt\#} only.

\item[\tt-o\#$_1$:\#$_2$] Process pages {\tt\#$_1$} to {\tt\#$_2$}. {\tt-o}
can be used more than once on the same command to process discontiguous
ranges of pages.

\item[\tt-o\#$_1$:\#$_2$:\#$_3$] Print every {\tt\#$_3$}th page from 
{\tt\#$_1$} to
{\tt\#$_2$}. This is most commonly used with {\tt\#$_3$} equal to 2 when
the output is going to be used for double sided photocopying, once to produce
the odd numbered pages and once to produce the even numbered pages.

\item[\tt-q] Suppress the progress messages normally printed on the terminal.

\item[\tt-r]  Select full paint mode. This is required
when including graphics
files (see below) or for very complicated pages. A page that is too complicated
will cause the printer to halt with error 21 (24 on the A2). On the A2 model, 
using full
paint mode leaves very little space for down-loading fonts and it is advisable
only to process the pages that need it in full paint mode and process the
rest of the document in normal paint mode.


\item[\tt-t]  Select manual paper feed (for printing 
transparencies). Consult
an expert before feeding anything but ordinary paper into the printer to
be sure that you are using suitable material.


\item[\tt-x\#\it unit] Set the left margin to {\tt\#} (which can be both
a real number and negative); {\it unit} is one of:

\begin{tabular}{ll}
bp &big point (1in = 72bp)\\
cc &cicero (1cc = 12dd)\\
cm &centimeter (1in = 2.54cm)\\
dd &didot point (1157dd = 1238pt)\\
in &inch\\
mm &millimeter\\
pc &pica (1pc = 12pt)\\
pt &point (72.27pt = 1in)\\
sp &scaled point (65536sp = 1pt)\\
\end{tabular}

The default margin is 1in.

\item[\tt-y\#\it unit] Set top margin. The default is 1in.
\end{list}

\subsubsection{Font Substitution}


If a font cannot be found at the required magnification, \verb+dvican+ and \verb+dvica2+ 
will by default
attempt to use the closest magnification that it can find. However it is also
possible to substitute another font or magnification under the control of a font
substitution file.  (See section \ref{se:fontsub}.)

\subsubsection{Temporary Files}

The Canon laser printer requires that all the characters in a font are down-%
loaded at one time; characters cannot be added to a font later. Until the
entire DVI file has been processed \verb+dvican+ does not know which characters are
needed in each font so it writes the type-setting information to a
temporary file which is appended to the font down-loading instructions once the
end of the DVI file has been reached. The temporary file is called {\tt
CAnnnnnnn.}, where {\tt nnnnnnn} is derived from the process id. and is normally
deleted when \verb+dvican+ exits. If however \verb+dvican+ does not exit normally the
temporary file must be deleted manually. 

\subsection{DVIPS}
\label{se:dvidvips}


The {\tt dvips} program has a number of features that set it apart from other
\PS\ drivers for \TeX.  This rather long section describes the advantages
of using {\tt dvips}, and may be skipped if you are just interested in learning
how to use the program.

The {\tt dvips} driver generates excellent, standard \PS, that can be
included in other documents as figures or printed through a variety of
spoolers.  The generated \PS\ requires very little printer memory,
so very complex documents with a lot of fonts can easily be printed even
on \PS\ printers without much memory, such as the original Apple
LaserWriter.  The \PS\ output is also compact, requiring less disk
space to store and making it feasible as a transfer format.

Even those documents that are too complex to print in their entirety
on a particular printer
can be printed, since {\tt dvips} will automatically split such documents
into pieces, reclaiming the printer memory between each piece.

The {\tt dvips} program supports graphics in a natural way, allowing \PS\
graphics to be included and automatically scaled and positioned in a variety
of ways.

Printers with resolutions other than 300 dpi are also supported, even if they
have a different resolution in the horizontal and vertical directions.
High resolution output is supported for typesetters, including an option
that compresses the bitmapped fonts so that typesetter virtual memory is
not exhausted.  This option also significantly reduces the size of the
\PS\ file and decoding in the printer is very fast.

Missing fonts can be automatically generated if \MF\ exists on the system,
or fonts can be converted from {\tt gf} to {\tt pk} format on demand.
If a font cannot be generated, a scaled version of the same font at a
different size can be used instead, although {\tt dvips} will complain
loudly about the poor \ae sthetics of the resulting output.

One of the most important features is the support of virtual fonts, which
add an entirely new level of flexibility to \TeX.  Virtual fonts are used to
give {\tt dvips} its excellent \PS\ font support, handling all the font
remapping in a natural, portable, elegant, and extensible way.  The {\tt dvips}
driver even comes with its own {\tt afm2tfm} program that creates the necessary
virtual fonts and \TeX\ font metric files automatically from the Adobe
font metric files.

For more information on \verb+dvips+ and \PS\ read SUN/176.


\subsubsection{Command Line Options} 
\label{dvipsoptions}

The {\tt dvips} driver has a plethora of command line options.  Reading
through this section will give a good idea of the capabilities of the
driver.

Many of the parameterless options listed here can be turned off by immediately
suffixing the option with a zero (0); for instance, to turn off page reversal
if it is turned on by default, use {\tt -r0}.  The options that can be turned
off in this way are {\tt a}, {\tt f}, {\tt k}, {\tt i}, {\tt m}, {\tt q}, {\tt
r},
{\tt s}, {\tt E}, {\tt F}, {\tt K}, {\tt M}, {\tt N}, {\tt U}, and {\tt Z}.

This is a handy summary of the options; it is printed out when you run
{\tt dvips} with no arguments.
{\vskip0pt\parskip=0pt\begverb{`\\}
     Usage: dvips [options] filename[.dvi]
 a*  Conserve memory, not time      y # Multiply by dvi magnification
 b # Page copies, for posters e.g.  A   Print only odd (TeX) pages
 c # Uncollated copies              B   Print only even (TeX) pages
 d # Debugging                      C # Collated copies
 e # Maxdrift value                 D # Resolution
 f*  Run as filter                  E*  Try to create EPSF
 h f Add header file                F*  Send control-D at end
 i*  Separate file per section      K*  Pull comments from inclusions
 k*  Print crop marks               M*  Don't make fonts
 l # Last page                      N*  No structured comments
 m*  Manual feed                    O c Set/change paper offset
 n # Maximum number of pages        P s Load config.$s
 o f Output file                    R   Run securely
 p # First page                     S # Max section size in pages
 q*  Run quietly                    T c Specify desired page size
 r*  Reverse order of pages         U*  Disable string param trick
 s*  Enclose output in save/restore X # Horizontal resolution
 t s Paper format                   Y # Vertical resolution
 x # Override dvi magnification     Z*  Compress bitmap fonts
     # = number   f = file   s = string  * = suffix, `0' to turn off
     c = comma-separated dimension pair (e.g., 3.2in,-32.1cm)
\endverb}

\begin{list}%
{}%
{\settowidth{\labelwidth}{\tt -P \em printername}
\settowidth{\labelsep}{aaaa}
\settowidth{\rightmargin}{aaa}
\addtolength{\labelwidth}{\labelsep}
\setlength{\leftmargin}{\labelwidth}}
%

\item [\tt -a]  Conserve memory by making three passes over the {\tt dvi} file
instead of two and only loading those characters actually used.
Generally only useful on machines with a very limited amount of
memory.

\item [\tt -b \em num] Generate {\it num} copies of each page, but duplicating the
page body rather than using the {\tt\#numcopies} option.  This can be
useful in conjunction with a header file setting {\tt\char92bop-hook}
to do colour separations or other neat tricks.

\item [\tt -c \em num] Generate {\it num} copies of every page, by using
\PS 's {\tt \#copies} feature.
Default is 1. (For collated copies, see the {\tt -C} option below.)

\item [\tt -d \em num]  Set the debug flags.  This is intended only for emergencies
or for unusual fact-finding expeditions; it will work only if
{\tt dvips} has been compiled with the {\tt DEBUG} option.
For what the values of {\it num} can be see section \ref{diagnostics}.
Use a value of $-1$ for maximum output.

\item [\tt -e \em num]
Make sure that each character is placed at most this many pixels from its
`true' resolution-independent
position on the page. The default value of this parameter
is resolution dependent (it is the number
of entries in the list [100, 200, 300, 400, 500, 600, 800, 1000, 1200, 1600,
2000, 2400, 2800, 3200, $\ldots\,$] that are less than or equal to the
resolution in dots per inch). Allowing individual
characters to `drift' from their correctly rounded positions
by a few pixels, while regaining the true position at the beginning of
each new word, improves the spacing of letters in words.

\item [\tt -f] Run as a filter.
Read the {\tt dvi} file from standard input and write the \PS\ to
standard output.  The standard input must be seekable, so it cannot
be a pipe.  If you must use a pipe, write a shell script that copies
the pipe output to a temporary file and then points {\tt dvips} at this file.
This option also disables the automatic reading of the {\tt PRINTER}
environment variable, and turns off the automatic sending of control D
if it was turned on with the {\tt -F} option or in the configuration file;
use {\tt -F} after this option if you want both.

\item [\tt -h \em name]
Prepend file {\it name}
as an additional header file. (However, if the name is simply `{\tt -}',
suppress all header files from the output.)  This header file
gets added to the \PS\ {\tt userdict}.

\item [\tt -i] Make each section be a separate file.  Under certain circumstances,
{\tt dvips} will split the document up into `sections' to be processed
independently; this is most often done for memory reasons.  Using this
option tells {\tt dvips} to place each section into a separate file; the
new file names are created replacing the suffix of the supplied output
file name by a three-digit sequence number.
This option is most often used in
conjunction with the {\tt -S} option which sets the maximum section length
in pages.  For instance, some phototypesetters cannot print more than
ten or so consecutive pages before running out of steam; these options
can be used to automatically split a book into ten-page sections, each
to its own file.

\item [\tt -k]  Print crop marks.  This option increases the paper size
(which should be specified, either with a paper size special or
with the {\tt -T} option) by a half inch in each dimension.  It
translates each page by a quarter inch and draws cross-style
crop marks.  It is mostly useful with typesetters that can set
the page size automatically.

\item [\tt -l \em num]
The last page printed will be the first one numbered {\it num.}
Default is the last page in the document.  If the {\it num} is
prefixed by an equals sign, then it (and any argument to the
{\tt -p} option) is treated as a sequence number, rather than
a value to compare with {\tt \char92 count0} values.  Thus,
using {\tt -l =9} will end with the ninth page of the document,
no matter what the pages are actually numbered.

\item [\tt -m]  Specify manual feed for printer.

\item [\tt -n \em num]
At most {\it num} pages will be printed. Default is 100000.

\item [\tt -o \em name]  The output will be sent to file {\it name.}
If no file name is given, the default name is {\tt {\it file}.dvi-ps} where
the {\tt dvi} file was called {\tt {\it file}.dvi};
if this option isn't given, any default in the configuration file is used.
If the first character of the supplied output file name is an
exclamation mark, then
the remainder will be used as an argument to {\tt popen}; thus, specifying
{\tt !lpr} as the output file will automatically queue the file for printing.
This option also disables the automatic reading of the {\tt PRINTER}
environment variable, and turns off the automatic sending of control D
if it was turned on with the {\tt -F} option or in the configuration file;
use {\tt -F} after this option if you want both.

\item [\tt -p \em num]
The first page printed will be the first one numbered {\it num.}
Default is the first page in the document.  If the {\it num} is
prefixed by an equals sign, then it (and any argument to the {\tt -l}
option) is treated as a sequence number, rather than a value to
compare with {\tt \char92 count0} values.  Thus, using {\tt -p =3}
will start with the third page of the document, no matter what the
pages are actually numbered.  Another form of page selection is
available by using {\tt -pp} followed by a comma-separated list of
pages or page-ranges, where the page ranges are colon-separated pairs
of numbers.  Thus, you can print pages 3--10, 21, and 73--92 with the
option {\tt -pp 3:10,21,73:92}.

\item [\tt -q] Run in quiet mode.
Don't chatter about pages converted, etc,  report nothing but errors to
standard error.

\item [\tt -r]
Stack pages in reverse order.  Normally, page 1 will be printed first.

\item [\tt -s]
Causes the entire global output to be enclosed in a save/restore pair.
This causes the file to not be truly conformant, and is thus not recommended,
but is useful if you are driving the printer directly and don't care too
much about the portability of the output.

\item [\tt -t \em papertype] 
This sets the paper type to {\it papertype}.  The {\it papertype} should
be defined in one of the configuration files, along with the appropriate
code to select it.  See the documentation for {\tt @} in the configuration
file option descriptions.  You can also specify
{\tt -t landscape}, which rotates a document by 90 degrees.
To rotate a document whose size is not letter, you can use the 
{\tt -t} option twice, once for the page size, and once for {\tt landscape}.
The upper left corner of each page in
the {\tt dvi} file is placed one inch from the left and one inch from the top.
Use of this option is highly dependent on the configuration file.
Note that executing the {\tt letter} or {\tt a4} or other \PS\
operators cause the document to be nonconforming and can cause it not
to print on certain printers, so the default paper size should not
execute such an operator if at all possible.

\item [\tt -x \em num]
Set the magnification ratio to {\it num}/1000. Overrides the magnification
specified in the {\tt dvi} file.  Must be between 10 and 100000.  It is
recommended that you use standard magstep values (1095, 1200, 1440, 1728,
2074, 2488, 2986, and so on) to help reduce the total number of {\tt PK}
files generated.

\item [\tt -A]
This option prints only the odd pages.  This option uses the \TeX\
page numbering rather than the sequence page numbers.

\item [\tt -B]
This option prints only the even pages.  This option uses the \TeX\
page numbering rather than the sequence page numbers.

\item [\tt -C \em num]
Create {\it num}
copies, but collated (by replicating the data in the \PS\ file).
Slower than the {\tt -c} option, but easier on the hands, and faster than
resubmitting the same \PS\ file multiple times.

\item [\tt -D \em num]
Set the resolution in dpi (dots per inch) to {\it num.}
This affects the choice of bitmap fonts that are loaded and also the positioning
of letters in resident \PS\ fonts. Must be between 10 and 10000.
This affects both the horizontal and vertical resolution.  If a high resolution
(something greater than 400 dpi, say) is selected, the {\tt -Z} flag should
probably also be used.

\item [\tt -E]
Makes {\tt dvips} attempt to generate an EPSF file with a tight bounding
box.  This only works on one-page files, and it only looks at marks made
by characters and rules, not by any included graphics.  In addition, it
gets the glyph metrics from the {\tt tfm} file, so characters that
lie outside their enclosing {\tt tfm} box may confuse it.  In addition,
the bounding box might be a bit too loose if the character glyph has
significant left or right side bearings.  Nonetheless, this option works
well for creating small EPSF files for equations or tables or the like.
(Note, of course, that {\tt dvips} output is resolution dependent and
thus does not make very good EPSF files, especially if the images are
to be scaled; use these EPSF files with a great deal of care.)

\item [\tt -F]
Causes Control-D (ASCII code 4) to be appended as the very last character
of the \PS\ file.  This is useful when {\tt dvips}
is driving the printer directly instead of working through a spooler,
as is common on extremely small systems.  Otherwise, it is not recommended.

\item [\tt -K]
This option causes comments in included \PS\ graphics, font files,
and headers to be removed.  This is sometimes necessary to get around bugs
in spoolers or \PS\ post-processing programs.  Specifically, the
{\tt \%\%Page} comments, when left in, often cause difficulties.
Use of this flag can cause some included graphics to fail, since the
\PS\ header macros from some software packages read portions of
the input stream line by line, searching for a particular comment.
This option has been turned on by default because \PS\ previewers
and spoolers still have problems with the structuring conventions.

\item [\tt -M]
Turns off the automatic font generation facility.  If any fonts are
missing, commands to generate the fonts are appended to the file
{\tt missfont.log} in the current directory; this file can then be
executed and deleted to create the missing fonts.

\item [\tt -N]
Turns off structured comments; this might be necessary on some systems
that try to interpret \PS\ comments in weird ways, or on some
\PS\ printers.  Old versions of TranScript in particular cannot
handle modern Encapsulated \PS .

\item [\tt -O \em offset]
Move the origin by a certain amount.  The {\it offset} is a comma-separated
pair of dimensions, such as {\tt .1in,-.3cm} (in the same syntax used in
the {\tt papersize} special).   The origin of the page is shifted from the
default position (of one inch down, one inch to the right from the upper
left corner of the paper) by this amount.

\item [\tt -P \em printername]
Sets up the output for the appropriate printer.  This is implemented
by reading in {\tt config.{\it printername}}, which can then set the output pipe
(as in, {\tt o !lpr -Pprintername}) as well as the font paths and any other
defaults for that printer only.  It is recommended that all standard
defaults go in the one master {\tt config.ps}
file and only things that vary
printer to printer go in the {\tt config.{\it printername}}
files.  Note that {\tt config.ps}
is read before {\tt config.{\it printername}}.
In addition, another file called {\tt \tilde/.dvipsrc}
is searched for immediately after {\tt config.ps};
this file is intended for user defaults.  If no {\tt -P} command is
given, the environment variable {\tt PRINTER} is checked.  If that
variable exists, and a corresponding configuration
file exists, that configuration file is read in.

\item [\tt -S \em num]
Set the maximum number of pages in each `section'.  This option is most
commonly used with the {\tt -i} option; see that documentation above for more
information.

\item [\tt -T \em offset]
Set the paper size to the given pair of dimensions.  This option takes
its arguments in the same style as {\tt -O}.  It overrides any paper
size special in the {\tt dvi} file.

\item [\tt -U]
Disable a \PS\ virtual memory saving optimization that stores the
character metric information in the same string that is used to store
the bitmap information.  This is only necessary when driving the Xerox
4045 \PS\ interpreter.  It is caused by a bug in that interpreter
that results in `garbage' on the bottom of each character.  Not
recommended unless you must drive this printer.

\item [\tt -X \em num]
Set the horizontal resolution in dots per inch to {\it num.}

\item [\tt -Y \em num]
Set the vertical resolution in dots per inch to {\it num.}

\item [\tt -Z]
Causes bitmapped fonts to be compressed before they are downloaded,
thereby reducing the size of the \PS\ font-downloading information.
Especially useful at high resolutions or when very large fonts are
used.  Will slow down printing somewhat, especially on early 68000-based
\PS\ printers.
\end{list}




\subsubsection{environment Variables} 
\label{envvar}


The {\tt dvips} program reads a certain set of environment variables to
configure its operation.  The path variables are read after all
configuration files are read, so they override values in the configuration
files.  (The {\tt TEXCONFIG} variable, of course, is read before the
configuration files.)  The rest are read as needed.

\begin{list}%
{}%
{\settowidth{\labelwidth}{\tt DVIPSHEADERS}
\settowidth{\labelsep}{aaaa}
\settowidth{\rightmargin}{aaa}
\addtolength{\labelwidth}{\labelsep}
\setlength{\leftmargin}{\labelwidth}}
%

\item [\tt HOME]{\rm no default}
  This environment variable is automatically set by the shell and is
used to replace any occurrences of {\tt \tilde} in a path.

\item [\tt MAKETEXPK]{MakeTeXPK \%n \%d \%b \%m}
This environment variable sets the command to be executed to create
a missing font.  A \%n is replaced by the base name of the font to
be created (such as {\tt cmr10}); a \%d is replaced by the resultant
horizontal resolution of the font; a \%b is replaced by the
horizontal resolution at which {\tt dvips} is currently generating
output, and any \%m is replaced by a string that \MF\ can use as
the right hand side of an assignment to {\tt mag} to create the
desired font at the proper resolution.  If a mode for \MF\ is set in
a configuration file, that is automatically appended to the command
before execution.  Note that these substitutions are different than
the ones performed on PK paths.

\item [\tt DVIPSHEADERS]{/star/TeX-3.1415/tex/ps:.}
  This environment variable determines where to search for header
files such as {\tt tex.pro}, font files, arguments to the
{\tt -h} option, and such files.

\item [\tt PRINTER]{\rm no default}
  This environment variable is read to determine which default printer
configuration file to read in.  Note that it is the responsibility of
the configuration file to send output to the proper print queue, if such
functionality is desired.

\item [\tt TEXFONTS]{/star/TeX-3.1415/tex/fonts/tfm:/star/local/tex/fonts/tfm/:.}
  This is where {\tt tfm} files are searched for.  A {\tt tfm} file only
needs to be loaded if the font is a resident (\PS ) font or if
for some reason no {\tt pk} file could be found.

\item [\tt TEXPKS]{/star/TeX-3.1415/tex/fonts/pk/pk300:/star/local/tex/fonts/pk}
This environment variable is a path on which to search for {\tt pk} fonts.
Certain substitutions are performed if a percent sign is found anywhere
in the path.  See the description of the {\tt P} configuration file
option for more information.

\item [\tt TEXINPUTS]{/star/TeX-3.1415/tex/inputs:.}
  This environment variable is used to find \PS\ figures when they
are included.

\item [\tt TEXCONFIG]{/star/TeX-3.1415/tex/ps:.}
  This environment variable sets the directories to search for configuration
files, including the system-wide one.  Using this single environment variable
and the appropriate configuration files, it is possible to set up the program
for any environment.  (The other path environment variables can thus be
redundant.)

\item [\tt  VFFONTS]{/star/TeX-3.1415/tex/fonts/vf}
  This environment variable sets where {\tt dvips} looks for virtual fonts.
A correct virtual font path is essential if \PS\ fonts are to be
used.
\end{list}


\subsection{DVIPRI}
\label{se:dvipri}

The output from DVIPRI isn't nice, but you will be
amazed how relatively good it is --- it is actually usable. \verb+dvipri+ can be very
useful in the early stages of debugging \TeX\ input, because you do not need
any special equipment to run it and lineprinter output is less expensive than
laser printer output. The output file can also be inspected with any editor,
but for this you need a 132-column wide terminal. Also because VDUs cannot show
overstruck characters, these appear incorrectly on the next line. The main
fault with the output is that it does not fit on A4 size paper, and it is very
extravagant in vertical space.  {\bf NOTE:}{\em \verb+dvipri+ is only available
on VAX/VMS.}

The \verb+dvipri+ command takes a number of qualifiers:
\begin{list}%
{}%
{\settowidth{\labelwidth}{\tt-f\em filename}
\settowidth{\labelsep}{aaaa}
\settowidth{\rightmargin}{aaa}
\addtolength{\labelwidth}{\labelsep}
\setlength{\leftmargin}{\labelwidth}}


\item[\tt /b] Generates output more suitable for viewing on a terminal.
Essentially, this means that overstruck characters get omitted. This is not use
unless your terminal supports 132 characters wide.

\item[\tt /c] Stop after this many pages. The default is $\pm 10^6$ and so
normally \verb+dvipri+ will print all pages.

\item[\tt  /d filename]   Take font from this directory, instead of the default
which is {\tt TEX\_FONTS:}.

\item[\tt /f] Start printing at the first page whose {\tt count0} parameter is
greater than or equal to the specified number.

\item[\tt /i number] Generates output directly on a terminal. The number is
optional. If is specified, say {\tt N}, then \verb+dvipri+ will pause every {\tt N}
lines and type the message `print return to continue'. If {\tt N=0}, \verb+dvipri+
does not pause. If {\tt N} is not given, it defaults to 20.

\item[\tt /m number] Magnify by number/100. Some magnification is needed
because most \TeX\ characters are narrower than lineprinter characters. If it
is not specified it defaults to 100\%.                                  

\item[\tt /p] Send output to the named file instead of the default. The default
output filename is [current directory][DVI file name]\verb+.dvi-pri+. But note that
when this qualifier is used VMS sometimes alters the name ({\em e.g.}~{\tt /p
horse} generates the output file {\tt horse.dat} instead of {\tt horse}.

\item[\tt /q] Suppress information messages.

\item[\tt /r] Suppress form feeds; instead \verb+dvipri+ prints a string like
\begin{quote}\tt ----- PAGE n ----- \end{quote} 

\item[\tt /s] Suppress blank line in the printed output. Normally, output is
double--spaced to allow space for superscripts and subscripts.

\item[\tt /x number] Horizontal magnification. If it
is not specified it defaults to 100\%.                                   

\item[\tt /y number] Vertical magnification. If it
is not specified it defaults to 100\%.                                   

\end{list}

\section{Printing}
\label{se:printing}

If you have used a DVI translator that produces printable output, a file will
have been created that can be printed on a suitable printer. Note that in the
following sections, the queue names are the default queue names that Starlink
assigns for the devices. If your site has more than one suitable printer,
other queue names may need to be substituted.

\subsection{Printing Files on Canon Laser Printers}

These are produced by \verb+dvican+ and are intended for Canon Laser Printers. 
They are printed thus:
\begin{verbatim}
      $ print/passall/nofeed/notify/queue=sys_laser/delete <filename>.dvi-can
\end{verbatim}
Output will be sent to the \mbox{\tt SYS\_LASER} queue.                  
                            

\subsection{Printing Files on \PS\ Printers}

These are produced by dvips and are intended for printers operating 
in \PS\ Mode. They are printed thus:
\begin{verbatim}
      $ print/notify/queue=sys_ps/form=post/delete <filename>.dvi-ps
\end{verbatim}
Output will be sent to the \mbox{\tt SYS\_PS} queue.   
                                                


\subsection{Printing Files on Line Printers}

These are produced by \verb+dvipri+ for line printers, and are printed thus:
\begin{verbatim} 
      $ print/notify/delete <filename>.dvi-pri 
\end{verbatim}
Output will be sent to the system default print queue. 
                                               


\subsection{Short Cut}

{\bf NOTE :}~This facility may have been locally over-ridden and will not
therefore behave as described here. See the later {\bf NOTE}~in this section. 

To save typing, there is a short cut for printing the \mbox{\tt .dvi-nnn} files
resulting from DVI processing. The system-defined symbol \mbox{\tt prcn} 
executes a command procedure which will print the appropriate file:
\begin{verbatim}
      $ prcn sample
\end{verbatim}

\mbox{\tt prcn} will check first if a file \mbox{\tt sample.dvi-can} exists in 
your current directory and submit it to the \mbox{\tt SYS\_LASER} queue. 
{\em Once printed, \mbox{\tt sample.dvi-can} will be deleted.}

If a file \mbox{\tt sample.dvi-can} cannot be found, \mbox{\tt prcn} will 
prompt you for another extension. Here you can reply .lis in which case 
\mbox{\tt prcn} will submit the \mbox{\tt sample.lis} file to the  \mbox{\tt
SYS\_PRINTRONIX} queue. This is useful when you want hard copy  details of the
nature and location of errors, and warning messages, statistics and  names of
files accessed by \TeX\ and \LaTeX\ when processing the document.  This
procedure will save you from the usual mistake of printing the  \mbox{\tt .dvi}
file. {\em The file you name will NOT be deleted.}

When the document has been printed, you should delete
\mbox{\tt sample.dvi} by typing:
\begin{verbatim}
      $ DELETE sample.dvi;*
\end{verbatim}

{\bf NOTE:}~Some sites have more than one laser printer, or use \PS\
printers. In either case locally different definitions of \mbox{\tt prcn} may
exist to direct print output at certain printers. For example, at RLVAD, the 
\mbox{\tt prcn} symbol is defined as:
\begin{verbatim}
      prcn == "print/queue=sys_ps/form=post/delete"
\end{verbatim}
in order to send the print file to the \PS\ printer. This bypasses the
\mbox{\tt prcn} command procedure.

You should check the definition of the \mbox{\tt prcn} symbol  before you use 
it. You may also want to redefine \mbox{\tt prcn} yourself.

\section{Graphics in \LaTeX\ Documents}
\label{se:graphics}

One major deficiency of the \TeX\ system is that it does not contain any
facilities for combining text and graphics in the same document. \LaTeX\ has
its `picture' environment which can draw simple diagrams but is limited
in scope and tedious to use for anything but the simplest picture. However,
the \TeX\ {\tt\verb+\special+} command allows device dependent commands
to be inserted in the DVI file where they can be interpreted by the program
that converts the DVI file to the commands that drive a specific printer.

\verb+dvican+ and dvips understand a set of commands that allow files
containing Canon graphics (\verb+dvican+) or \PS\ (dvips) to be inserted 
into a \TeX\ or \LaTeX\
document. Suitable graphics files can be generated by GKS.

Although similar possibilities for inserting graphics exist for other printer 
devices the exact mechanism is likely to be different so this technique should 
be used with discretion.  

For example, it would be quite inappropriate to use such techniques in a 
document that is to be sent to a non-Starlink site, but for a one-off such as 
a scientific paper that will never be revised once it is published it is more 
convenient and flexible than pasting up the document with scissors and glue. 

If a document containing a diagram is to revised in the future the program and 
data that produced the graphics (not the graphics file) must be regarded as a 
part of the document source and stored with the same care as the \TeX\ file 
itself. Since generating a copy of the document becomes much more complicated 
than merely running \TeX\ or \LaTeX\ followed by \verb+dvican+ or dvips it is wise to 
include complete instructions on how to generate a copy in comments in the 
\TeX\ source.

\subsection{The \LaTeX\ Source}

The drivers expect the \hbox{\verb|\special|} command to look like one of the
following: 
\begin{verbatim} 
\special{overlay <filename>}      %absolute positioning 
\special{include <filename>}      %relative positioning
\special{insert <filename>}       %relative positioning 
\end{verbatim} 
In each case \hbox{\verb|<filename>|} refers to a file containing the picture
or \PS\ code. The {\tt overlay} case will map the file onto the page starting
at the top left corner of the printable region, or precisely the coordinates
which the \PS\ code specifies. when the {\tt include} and {\tt insert} cases,
which are synonymous, are used the driver places the upper-left corner of the
bounding box at the current point. Files generated by GKS have the graphics
origin at the top left corner of the picture.

The {\tt\verb+\special+} command is not interpreted by \TeX\ in any way, but
is simply copied into the DVI file. So it is up to
you to ensure that an appropriate sized blank space appears in the document. 

When using \LaTeX, the correct space and positioning is most easily achieved
by making the {\tt\verb+\special+} command the argument of a {\tt\verb+\put+} 
command in a {\tt picture} environment of the same size as the picture. For
example, a picture 10cm by 7cm stored in a file called {\tt mypic.dat} could
be inserted with the following commands:
\begin{verbatim}
      \setlength{\unitlength}{1cm}
      \begin{picture}(10,7)
      \put(0,7){\special{include mypic.dat}}
      \end{picture}
\end{verbatim} 

Such a picture would often be part of a {\tt figure} environment so that
it can float to a convenient place in the document.

To achieve the same result with plain \TeX\, refer to page~228 of the
{\TeX}book. the command
           
\begin{verbatim}
      \special{overlay <filename>}
\end{verbatim}
inserts the picture with the graphics origin at the top left corner of the
page and \TeX\ does not influence the position of the picture (other than
on which page it appears).


The {\tt include} works best with files which contain bit map graphics data.
The files may include other printer commands, as long as they do not disturb
the environment which the driver assumes; for example, the files should 
{\bf not}
contain commands to eject pages, alter downloaded fonts, change margins or
other such global matters unless the purpose is to deliberately interfere
with driver assumptions. You must determine bounding boxes using software which
originally produced the {\tt include} file, and insert spacing in the \TeX\
input to take this into account. For example:
\begin{verbatim}
 \makebox[1.90in][l]{\vrule width 0in height 0.74in depth 0in}
 \raisebox{0.74in}{\rule{0.003in}{0.003in}}
 \special{include sign.dig}
\end{verbatim}
This example inserts a file ``{\tt sign.dig}'' (perhaps a digitized handwritten
signature) at the top left corner of a box 1.90 inches wide and 0.74 inches
high, which will appear to \LaTeX\ as an \verb+\hbox+ of those dimensions. The
\verb+\rule+ is mandatory; this tiny dot (which will be indistinguishable on
the printer) tricks \TeX\ into emitting DVI output to move the printer to the
current output position.

A useful feature under UNIX is the ability to specify any shell command instead
of a file name. In this case the driver, instead of copying the contents of a
named file, copies the standard output of the shell command to the output. To
activate this feature, you simply begin the shell command with an exclamation
mark ({\tt !}). For example,
\begin{verbatim}
 \special{include !uncompress < sign.dig.Z}
\end{verbatim}
This example says that the driver is to fork a shell, which runs the {\tt
uncompress} utility with the file {\tt sign.dig.Z} as standard input. The
driver uses the standard output of the command as the file contents to insert in
the output.




DVI file pages which contain \verb+\special+ commands must be processed with
the {\tt -r} qualifier when processed with \verb+dvican+ because the printer must be
in  full paint mode to process vector commands.

\subsection{The Graphics files}

The graphics files for the Canon are created by using the GKS workstation  type
2610 (landscape) or 2611 (portrait); and for \PS\ with 2706 (portrait) or 
2607 (landscape). Apart from the format of the files, these workstations differ
from the normal Canon and \PS\ workstations in that  (a) each new frame
generates a separate file and (b) the files cannot be plotted directly on the
printer; only by processing them with \verb+dvican+ or dvips. 

It is the responsibility of the program that generates the picture to ensure
that it matches the gap left for it in the text.
GKS sets the graphics origin to the top left corner of the workstation viewport
so the most obvious way to do this is to set the workstation viewport to
the required size; the position of the viewport on the display surface is
irrelevant. 
However, with some high level graphics packages, setting the
workstation viewport may interfere with the working of the package and the
following alternatives are suggested:
\begin{description}
\item[SGS ---] Use {\tt SGS\_ZSIZE} with a position specification of {\tt TL}.
\item[AUTOGRAPH ---] Use {\tt SGS\_ZSIZE} followed by {\tt SNX\_AGWV}.
\item[PGPLOT ---] Use PGPAPER.
\end{description}


\section{Carrying On --- More About \LaTeX}
\label{se:carryon}



\subsection{Document Styles}
\label{se:dstyle}
\LaTeX\ provides a number of standard document styles:
{\tt article, book, letter, report} and \mbox{\tt slides}. 
(The \mbox{\tt article} style is for short reports, and is the standard style
for Starlink documents.)  
Additionally, there are qualifying options, {\em e.g.} size of the type, 
whether one or two-column formatting of the text is required.
There are Starlink styles for producing the standard documentation, {\em e.g.} 
SUN's, and a user guide, something like that produced for MIDAS. These are
based on the \mbox{\tt article} style and shell documents are stored in
\verb+DOCSDIR+ as \verb+sun.tex+, \verb+sgp.tex+, \verb+ssn.tex+ and \verb+sg.tex+.

There are four document-style options available  that are not described in the
manual: the \mbox{\tt proc} style option for making camera-ready copy for
conference proceedings, the \mbox{\tt bezier} option for drawing curves, the
\mbox{\tt ifthen} option for implementing {\bf if-then-else} and {\bf while-do}
control structures, and the \mbox{\tt ralhead} option for making headed
letters.  They are described below. 

\subsubsection{The \mbox{\tt proc} Style Option}
\label{se:procsty}

The \mbox{\tt proc} option is used with the \mbox{\tt article} document
style.  It produces two-column output for conference
proceedings.  The command \hbox{\verb|\copyrightspace|} makes the blank
space at the bottom of the first column of the first page, where the
proceedings editor will insert a copyright notice.  This command works
by producing a blank footnote, so it is placed in the text of the first
column.  It must go after any \hbox{\verb|\footnote|} command that
generates a footnote in that column.

\LaTeX\ automatically numbers the output pages.  It is a good idea 
to identify the paper on each page of output.  Placing the command
\begin{verbatim}
      \markright{Jones---Foo}
\end{verbatim}
in the preamble (before the \hbox{\verb|\begin{document}|} command)
prints ``Jones---Foo'' at the bottom of each page.  

\subsubsection{The \mbox{\tt bezier} Style Option}
\label{se:beziersty}

This option defines a single command, \hbox{\verb|\bezier|}, that draws
a curved line in a \mbox{\tt picture} environment.  Let $P_{i}$ be the point
with coordinates $(x_{i},y_{i})$, for $i=1$, 2, and 3.  The command
\begin{itemize} \tt
\item[]
\verb|\bezier{|$n$\verb|}(|$x_{1}$,$y_{1}$)($x_{2}$,$y_{2}$)($x_{3}$,$y_{3}$)
\end{itemize}
draws $n$ points on the quadratic Bezier spline determined by the three
points $P_{1}$, $P_{2}$, and $P_{3}$.  The locus of points on this
spline is a parabolic arc from $P_{1}$ to $P_{3}$ having the line
$P_{1}P_{2}$ tangent to it at $P_{1}$ and the line $P_{2}P_{3}$ tangent
to it at $P_{3}$.  Note that $P_{2}$ is {\em not\/} on this arc unless
$P_{1}$, $P_{2}$, and $P_{3}$ are colinear, in which case the arc is a
straight line.  Bezier splines are useful because it's easy to join two
of them together smoothly by giving them the same tangent line where
they meet.

It takes roughly 75 points per inch to form a solid line, depending
upon the line thickness.  See Section~C.13.3 of the manual for commands
to specify line thickness in a \mbox{\tt picture} environment.  This command
is {\em very\/} slow, and \TeX\ has enough memory to hold only about
1000 points plus a page of text.  (Remember that \TeX\ keeps the
current page plus all as yet unprinted figures in memory.) So, the
\verb|bezier| command should be used for only a small number of small
curves.

\subsubsection{The \mbox{\tt ifthen} Style Option}
\label{se:ifthensty}

This option provides two programming language features that are useful
only for people who already know how to program.  It defines the
two commands
\begin{itemize} 
\item[]
\verb|\ifthenelse{|{\em test\/}\verb|}{|{\em then clause\/}\verb|}{|%
{\em else clause\/}\verb|}|\\
\verb|\whiledo{|{\em test\/}\verb|}{|{\em do clause\/}\verb|}|
\end{itemize}
that implement the following two Pascal language structures
\begin{itemize}
\item[]
\begin{tabbing}
{\bf if} {\em test\/} \= {\bf then} \= {\em then clause\/} \\
 \> {\bf else} \> {\em else clause\/} \\[2pt]
{\bf while} {\em test\/} {\bf do} {\em do clause\/}
\end{tabbing}
\end{itemize}
The {\em then\/}, {\em else\/}, and {\em do\/} clauses
are ordinary \LaTeX\ input; {\em test\/} is one of the following:
\begin{itemize}
 \item A relationship between two numbers formed with \mbox{\tt <}, \mbox{\tt
>}, or \mbox{\tt =}; for example, \hbox{\verb|\value{page}>3|}. 
\item \verb|\equal{|{\em string1\/}\verb|}{|%
{\em string2\/}\verb|}|, which evaluates to {\em true\/} if {\em
string1\/} and {\em string2\/} are the same strings of characters after
all commands have been replaced by their definitions.  (Upper- and
lowercase letters are unequal.)
 \item A logical combination of the above two kinds of tests
    using the operators \hbox{\verb|\or|}, \hbox{\verb|\and|},
    and \hbox{\verb|\not|} and the parentheses \hbox{\verb|\(|}
    and \hbox{\verb|\)|}---for example:
\begin{verbatim}
      \not \( \value{section} = 1  \and  \equal{Jones}{\myname} \)
\end{verbatim}
\end{itemize}
These commands, together with \hbox{\verb|\renewcommand|} and the
commands of Section~C.7.4 of Lamport's book for manipulating counters, open 
up a whole new world of \LaTeX ing.

\subsubsection{Letters} \label{se:letters}

The \mbox{\tt letters} document style, described in the manual, should be used
for generating personal letters.  For generating letters on simulated RAL headed
notepaper, use the \mbox{\tt ralhead} style option. This option defines two
extra commands: \hbox{\verb|\extension|} for adding your telephone extension to
the RAL switchboard number, and the \hbox{\verb|\directline|} command for
adding a direct-line phone number. 


\subsection{\BibTeX}
\label{se:bibtex}
Often you will require citations or references to other sources of
information. You can produce the list of references directly with
\LaTeX.  Alternatively,
a separate programme called \BibTeX\ can produce the bibliography from
databases of sources. 
The databases have extension \mbox{\tt .bib}. Unfortunately, we
currently have no special tool to speed the creation of a database, but it
could be provided if demand warranted.
There is a bibliography style to give you a choice, {\em e.g.} whether
to order the citations alphabetically or in the order they appear in the 
main body of the text. Only the the standard styles are available.

There are three stages in using the bibliography databases.
\begin{enumerate}
\item
First run \LaTeX\ on your input text file as described in
Section~\ref{se:example}.
\item
Then run \BibTeX\ by typing
\begin{verbatim}
      $ bibtex filename
\end{verbatim}
where \mbox{\tt filename} is the name of your \LaTeX\ input
file.
\BibTeX\ reads the auxiliary file produced by \LaTeX\ 
(\mbox{\tt filename.aux}) from \mbox{\tt filename.tex}, in addition to the
databases.
\BibTeX\ creates a file called \mbox{\tt filename.bbl} containing \LaTeX\
commands to produce the list of references.
\item
Finally re-run \LaTeX\ as stage 1. \LaTeX\ reads \mbox{\tt filename.bbl}
to actually produce the list references at the end of your document.
\end{enumerate}

\BibTeX\ should be run from the directory containing
\mbox{\tt filename.tex} (which should be the same directory from which
\LaTeX\ was run on that file).

If the \mbox{\tt .bib} file is not in the same directory as the \LaTeX\ input
file---for example, if you are using someone else's \mbox{\tt .bib}
file---then you must include a directory as part of the file name specified
by the \hbox{\verb|\bibliography|} command. For example, the \LaTeX\ command
\begin{verbatim}
      \bibliography{[xyz.bibfiles]photom
\end{verbatim}
specifies the file \mbox{\tt photometry.bib} kept by XYZ in his 
\mbox{\tt [.BIBFILES]} directory.

There is no formal provision for sharing bibliographic-database information.
Suggestions for forming one or more common \mbox{\tt .bib} files
are welcome.

See Appendix B of the \LaTeX\ manual for full details on \BibTeX\ and
how to format a database.

\subsection{\IdxTeX}
\label{se:idxtex}

The \IdxTeX\ program is used to automate the generation of an Index in a
\LaTeX\ document.  

\subsubsection{Index Processing}

To make an Index using \LaTeX, the writer includes the 
\verb~\makeindex~ command
in the preamble of the document.  This causes an auxiliary 
file (with file type {\tt .idx}) to be generated.  For every 
\verb~\index~ command in the document, a
\verb~\indexentry~ command is written to 
the auxiliary file, containing the text supplied in the \verb~\index~ command
and the page number of the page on which the \verb~\index~ command occurred. 


While this procedure is capable of generating an effective, high-quality Index,
a substantial amount of work is required to convert the auxiliary {\tt .idx}
file into the appropriate set of commands.  

\IdxTeX\ uses the {\tt .idx} auxiliary file to generate a new file with file
type {\tt .ind} which contains all of the necessary commands to create an
attractive, two-column index.  This {\tt .ind} file may then be included in the
document at the appropriate place to generate the Index. 


\subsubsection{Building an Index}
                             

\begin{itemize}



\item Include the \verb~\makeindex~ command in the preamble area of the
document source file (i.e., between the \verb~\documentstyle~ and
\verb~\begin{document}~ commands).  Also, include the appropriate \verb~\index~
commands (as described below) in the source file.

\item Use \LaTeX\ to generate an initial form of the document.  \LaTeX\ will
generate an auxiliary file which has the same filename as the main document
file and a filetype of .idx.


\item Use \IdxTeX\ to generate the formatted index file.

\item Insert the command \verb+\input{file.ind}+ into your document source file
just before the \verb~\end{document}~ command. 

\item Finally, run the document through \LaTeX\ twice (to make sure that the
Table of Contents includes the page reference for the Index).  Your document is
now complete, including a formatted Index. 

\end{itemize}

\subsubsection{Command Syntax}

\begin{verbatim}

        $ idxtex   file  [/toc={ ARTICLE | REPORT }]

\end{verbatim}

where ``file'' is the filename of the .idx 
file.  \IdxTeX\ generates a file with the same filename as the .idx file and a
filetype of .ind. 

The \verb+/toc+ qualifier is optional.  If you
specify it, you must specify either the value \verb+article+ or the value
\verb+report+.  The \verb+/toc+ qualifier is used to add an entry into the Table of
Contents for the Index.  The standard document styles ({\em article\/} and
{\em report\/}) do not automatically make a Table of Contents entry for the 
Index.  If you want the Index to be listed in the Table of Contents, you 
{\bf must} use the \verb+/toc+ qualifier.  The value \verb+article+ should be used
with documents which use the {\em article\/} document style or one of its 
derivatives.  The value \verb+report+ should be used with documents which use
the {\em report\/} or {\em book\/} document styles. 

A number of document styles have been developed which automatically include
a Table of Contents entry for the Index\footnote{Examples of these styles 
include the Monsanto {\em pamphlet}, {\em manual}, and {\em memo\/} styles.}.
You {\bf should not} use the \verb+/toc+ qualifier with one of these document styles.
If you do, the Index will be listed in the Table of Contents twice.


\subsubsection{Special Features}

The \IdxTeX\ program attempts to perform all of the operations needed to 
generate an Index.  Of course, the quality of the
Index is a function of the amount of effort you put into your writing --- the
program can only generate index entries for items you tell it to index for you.
But if you do a good job of generating index entries, \IdxTeX\ should do a good
job formatting them for you.


\paragraph{Index  Generation} 
The .ind file generated by \IdxTeX\ contains all of the commands
necessary to generate the index.  

The file contains a 
\verb~\begin{theindex}~ command.  This
sets up the Index environment according to your current \verb~\documentstyle~.
For example, in the {\tt article\/} document style, the index is an
unnumbered subsection.  In the {\tt report\/} document style, the index is
an unnumbered chapter.

The last line in the file is a \verb~\end{theindex}~ command.  This command
ends the Index environment.  Between these lines are found all of the commands
needed to generate the Index.

\paragraph{Alphabetical Index}  The index is generated in an alphabetized,
two-column format by major index entry.  Special characters such as
``\verb~\~'' don't participate in the alphabetization, so a term like
\verb~\begin~ is indexed under ``B'', not under ``\verb~\~''.

For clarity, the group of index entries which begin with one letter of the
alphabet are set off from the index entries beginning with the next letter of
the alphabet with vertical spacing and a heading.  

\paragraph{Index Levels} 
Items are inserted into the Index by using the \verb~\index~ command.  For 
example, to index the term ``entry'', enter \verb~\index{Entry}~
at the appropriate place in the text of your document.  The information entered
as the argument to \verb~\index~ will appear as 
specified (special visual effects are discussed below).  The page number on
which the \verb~\index~ command appeared will be displayed. 

This approach generates a simple Index.  That is, all terms are on the same
footing.  It is often convenient, however, to have terms appear as subindices
under an index term.  For example, if you were discussing ``Commands'', you
might wish to have separate index entries for ``Add Command'' and ``Change
Command''.  However, in addition, you should create an index entry for 
``Commands'' and place ``Add'' and ``Change'' as subindices under it.  This 
technique generates an index which is more helpful for the person looking for
information.

\IdxTeX\ actually supports three such levels of indexing --- a main index item,
a subindex item, and a subsubindex item.  In analogy to the indexing 
capabilities of the RUNOFF program, the 
``\verb~>~'' symbol is used to separate
the index term from the subindex term and the subindex term from the subsubindex
term in the \verb~\index~ command.  

For example,
\begin{verbatim}

        \index{Top Level Item>SubItem>SubSubItem}

\end{verbatim}
generates such three-level deep
index entry in the index.  
There are two important points to note here.

\begin{itemize}
\item First, each \verb~\index~ command generates a single page number reference
in the index.  The page number is associated with the lowest level of the index
specified.  Therefore, in this example, the page number is associated with the
``SubSubItem'' entry, not any of the higher entries.  If you want page numbers
at all levels, you must use several \verb~\index~ commands.

All subitems are displayed in alphabetical order below their top-level item.
All subsubitems are displayed in alphabetical order below their second-level
item.

\item Second, the ``\verb~>~'' symbol used to separate items of different
levels is normally a math symbol.  As such, it is possible that you will want
to index a math expression which contains it.  This is okay, because \IdxTeX\
is sensitive to the use of {\tt \$} as a command to enter or leave math mode
and will not consider ``\verb~>~'' a level separator unless it is {\bf not\/}
in math mode.

\end{itemize}

While \IdxTeX\ allows you to be pretty wordy in your index terms (there is a
practical limit of about 110 characters for the text contained in an
\verb+\index+ command), your index will be most effective if you select short
and concise index terms.   

\paragraph{Item   highlighting} 
Sometimes it is desirable to make a word or phrase (or even an entire index
entry) stand out in the Index.  This is referred to as Item Highlighting, and
is an effective technique to make the Index more useful, if not overdone.

Use the normal \LaTeX\ commands to generate the desired highlighting.  For 
example, if the following commands may appear in the document.
\begin{verbatim}

        \index{Highlighting>A {\bf boldface} entry}
        \index{Highlighting>An {\em italic\/} entry}

\end{verbatim}
The result will be 
\begin{itemize}
\item{Highlighting}
\subitem{A {\bf boldface} entry, \mbox{\rm 6}}
\subitem{An {\em italic\/} entry, \mbox{\rm 6}}
\end{itemize}
From the standpoint of \IdxTeX, the item, subitem, and subsubitem are processed
as different text elements.  Therefore, if you wish to highlight more than one
level of the index, you must handle each one independently.  That is,
\begin{verbatim}
\index{{\bf One}>{\it Two}>{\tt Three}}
\end{verbatim}
will produce
\begin{itemize}
\item{{\bf One}, \mbox{\rm 6}} 
\subitem{{\it Two}, \mbox{\rm 6}} 
\subitem{{\tt Three}, \mbox{\rm 6}} 
\end{itemize}

\paragraph{Page Number highlighting}  Sometimes, when you are indexing the same
term in many places in a document, you may wish to help the reader decide which
is the primary  reference and which are secondary references.  One way of doing
this is to provide visual highlighting of important page numbers.

\IdxTeX\ provides a mechanism to cause a page number of a reference to be
printed in {\bf boldface}, {\em italics}, or \underline{underlined}.  There
are three special characters recognized by \IdxTeX\  when they appear as the
first character of an \verb~\index~ command argument.

\begin{itemize}
\item The \verb~^~ command\index{\verb~\index~ Command>Use of \verb~^~} is 
used to make the page number reference appear in {\bf boldface}. 
\item The \verb!~! command\index{\verb~\index~ Command>Use of \verb!~!} is 
used to make the page number reference appear in {\em italics}.
\item The {\tt\_} command\index{\verb~\index~ Command>Use of {\tt\_}} is used to
make the page number reference appear \underline{underlined}. 
\end{itemize}

For example, if the following \verb~index~ commands appear,
\begin{verbatim}
        \index{^page Numbers>boldface}
        \index{~Page Numbers>italics}
        \index{_Page Numbers>underlined}
\end{verbatim}
the result will be
\begin{itemize}
\item{Page Numbers}
\subitem{boldface, \mbox{\bf 6}}
\subitem{italics, \mbox{\em 6}}
\subitem{underlined, \mbox{\rm\underline{6}}}
\end{itemize}

These special characters must appear as the first character of an \verb~\index~
command argument.  If they appear after a ``\verb~>~'' (i.e., as the first
character of a subindex entry or a subsubindex entry), they will not work. In
fact, they will cause a \LaTeX\ error!

\paragraph{Special   \LaTeX   Commands}
Occasionally, it is desirable to include \LaTeX\
commands in the index entry to create special effects or because
the command itself is being referenced.  

The trick is that \IdxTeX\ tries to alphabetize your entries correctly.  This
means that it must figure out how to alphabetize \LaTeX\ commands, whose 
displayed images are different than the character strings which generate them.

The following rules attempt to indicate how \IdxTeX\ treats \LaTeX\ commands
when it finds them.

\begin{itemize}
\item Accents are ignored when figuring out how to spell a term.  This means,
for example, that ``\verb+se\~{n}or+'' is treated as if it were spelled 
``\verb+senor+''.

\item Text contained in ``\verb~\verb~'' or ``\verb~\verb*~'' commands is
spelled as if the command were not present.  For example, the index entry
``\verb~\verb+\Pi+~'' is treated as if it was spelled ``\verb~\Pi~''.

\item Commands which affect font size or type style are ignored when figuring
out how something is spelled.  For example, ``\verb~\sc term~'' is treated as if
it were spelled ``\verb~term~''.  Note, however, that this applies only to
spelling. The term will be displayed as you want it.

\item Case is ignored in spelling considerations.  Terms like ``large'' and
``Large'' are treated as identical.  Only one reference (with two page numbers)
will appear --- the first form seen will be used.

\item \IdxTeX\ ignores the grouping commands ``\{'', ``\}'', and ``\$'' when
placing a term in alphabetical order.  On the other hand, the sequences which
generate explicit characters (i.e., ``\verb+\{+'', ``\verb+\}+'', and 
``\verb+\$+'') are handled as if they were the characters themselves --- the
``\verb+\+'' is ignored for spelling. For example, the term ``\verb~{\bf The \{
Command}~'' will be alphabetized as if it was spelled ``\verb~The { Command~''.

\item All ``\verb~\~'' characters are ignored.  The term ``\verb~\begin~'' will
be located in the Index as if it began with the ``b'', not the ``\verb~\~''. 

\end{itemize}


\subsubsection{Page Ranges}

\IdxTeX\ supports two types of page ranges --- implicit and
explicit. 

An implicit page range is one which is discovered during index processing.  For
example, if it turns out that a particular item is indexed on three consecutive
pages, then this is an implicit range. 

An explicit page range is one in which you specifically indicate that an item
being referenced is discussed on a particular page and succeeding pages. 

\paragraph{Implicit Page Ranges}

Suppose \index{#Page Ranges>Implicit} that you index the term ``Command'' a
number of times within your document and it turns out (when you use \LaTeX\ to
format it) that the references occur on pages 10, 15, 16, 17, 20, and 21.
\IdxTeX\ will recognize that there are two implicit page ranges here and will
display the reference in the index as
\begin{quote}
\qquad Command, 10, 15--17, 20--21
\end{quote}
This processing occurs automatically --- you need not do anything special to
get it to occur (except, of course, insert the \verb+\index+ commands where
they are needed).

In some document styles, chapter-oriented page numbering is used.  From the
standpoint of \IdxTeX, a chapter oriented page number is any page number which
consists of a string (which is usually numeric) followed by one or more dashes,
followed by a page number.  For example, \verb+2--6+ and \verb+Ref--12+ are
typical chapter oriented page numbers.

Since the \IdxTeX\ program parses and recognizes page numbers in this format
as well as simple numeric page numbers, it is able to correctly build ranges
for chapter oriented documents.  For example, if references to ``Command''
occurred on pages 2--6, 3--7, 3--8, 3--9, and 4--10, then \IdxTeX\ would
build an index entry of the form
\begin{quote}
\qquad Command, 2--6, 3--7 to 3--9, 4--10
\end{quote}

 If
one or more of the page number references in a range has a highlight, the range
is broken up into sub-ranges, so that each sub-range has the same visual
highlight.  For example, if references to ``Noise'' occurred on pages 20 through
26, with the references on pages 22 and 23 flagged as underlined and the 
references on pages 24 and 25 flagged as boldface, then \IdxTeX\ would generate
an index entry of the form
\begin{quote}
\qquad Noise, 20--21, \underline{22--23}, {\bf 24--25}, 26
\end{quote}

\paragraph{Explicit Page Ranges}

Another form of page range occurs when a single
reference indicates that a particular topic is covered on a particular page
and following pages.  In this kind of reference, only one index item is
needed --- for the first page referenced.  By convention, the letters ``{\sf
ff}''
follow that page reference to mean ``and following''.

To specify this type of reference, use the {\tt \#}
symbol as the first
character in the index entry (just as if it were a special type of page 
highlight).  For example, the reference 
\begin{verbatim}
      \index{#Page Ranges>Explicit}+
\end{verbatim}
will produce 
\begin{itemize}
\item{Page Ranges, \mbox{\rm 8\sf ff}}
\subitem{Explicit, \mbox{\rm 9\sf ff}}
\end{itemize}

There are two special features of this notation.
\begin{itemize}
\item Explicit page ranges take precedence over implicit page ranges.  That is,
if an explicit range entry occurs within an implicit range, then the whole
implicit range is condensed into a single explicit reference.  For example,
if the reference to ``Command'' was made on pages 20 through 25 and the 
reference on page 23 was explicit, then the output would have the form
\begin{quote}
\qquad Command, 20{\sf ff}
\end{quote}
rather than
\begin{quote}
\qquad Command, 20--22, 23{\sf ff}, 24--25
\end{quote}
\item Explicit page ranges take precedence over all page highlighting.  If
\begin{verbatim}
    \index{~Command}\index{#Command}
\end{verbatim}
occurs, for example, then the italic reference is ignored.
\end{itemize}

\subsubsection{Cross References}

It\index{#Cross References} is not uncommon in an index to want to reference a
term in a variety of different ways, so that people will be easily able to find
the reference they need.  This is often a lot of work, however, because you
don't normally want to repeat a big list of \verb+\index+ commands for every 
place where a term is referenced.  Instead, a great deal of effort can be saved
if you can index the term thoroughly once, and then {\em cross reference\/} it
in all its synonyms.

For example, suppose you are describing the syntax of a command.  You index
the various forms of the command under the term ``Syntax''.  However, you also
want to place an alternate entry in the index under ``Command Syntax'' in case
people look there first.  Obviously, you'd rather not repeat all of your
index commands twice to get a complete set of references.  Instead, you'd like
to {\bf refer} the reader who looks up ``Command Syntax'' to try ``Syntax''
for the information.

The special symbol ``\verb+&+''is used to indicate the beginning of a cross
reference, which is used  instead of the page reference in the entry.  In our
example, entering \verb+\index{Command Syntax&Syntax}+ produces 
\begin{itemize} 
\item{Command Syntax, {\em see} }
\indexindent{\ $\bullet$ {\rm Syntax}} 
\end{itemize} 
You can combine cross references with page references. 
For example, 
\begin{verbatim} 
\index{Combining References} 
\index{Combining References&Cross References} 
\end{verbatim} 
contains a page reference and a
cross reference, as follows 
\begin{itemize} 
\item{Combining References, \mbox{\rm 10}; {\em see also} }
\indexindent{\ $\bullet$ {\rm Cross References}}
\end{itemize} 
For convenience, the notation \verb+\index{aaa&bbb>ccc>ddd}+
generates a cross reference to \verb+bbb+, \verb+ccc+, \verb+ddd+, so you can
use the same syntax in your cross references as you do in your index terms
themselves.  Note that the text following the ``\verb+&+'' must obey all of the
rules associated with index entries in general.

You can include as many cross references as you like.  For example, the
sequence
\begin{verbatim}
    \index{aaa&bbb} 
    \index{aaa&ccc}  
     \index{aaa&ddd}
\end{verbatim}
is perfectly acceptable, and it will generate the following in the index 
\begin{quote}
\hspace*{10pt}aaa, {\em see} \\
\hspace*{40pt}$\bullet$ bbb \\
\hspace*{40pt}$\bullet$ ccc \\
\hspace*{40pt}$\bullet$ ddd
\end{quote}

Also note that a cross-reference entry does not, in itself, contain any
page number information.  Therefore, the index entries containing cross
referencing may appear anywhere.  For example, you could build a basic index
in your document, then add all the appropriate cross reference entries later,
all in one place in your text.  Note, however, that cross references in index
entries are treated like simple text --- \IdxTeX\ does not check to make sure
that a cross reference entry actually exists.  So be careful when creating them,
to avoid misleading your reader.

\subsubsection{Master Index Processing}

A Master Index 
is a document in a set of documents which contains a complete
index of all terms indexed in {\bf any} of the other volumes of the document
set.  \IdxTeX\ provides support for the automatic generation of a Master
Index.


\paragraph{Prepare the   Individual Indices}
The first step is to run all of the other documents in the document set 
through \LaTeX, so that all of the {\tt .idx} files are generated.  These
files are used by \IdxTeX\ to format the Master Index.

\paragraph{Build the {\tt .mdx} File}
\IdxTeX\ must be made aware of which {\tt .idx} files to use to build the
Master Index.  In addition, it is important to define a {\em label\/} which
is associated with each volume in the Master Index, so that the reader can
figure out which volume to look to to find the reference.

Both of these requirements are handled by a new auxiliary file which has the
extension {\tt \verb+.mdx+}.
The format of this file is quite simple.  For each volume which you wish to
be included in the Master Index, insert a line in the {\tt \verb+.mdx+} file of the
form
\begin{verbatim}
    \usefile{label}{idx-file}
\end{verbatim}
Here, ``{\em label\/}'' is the label which you wish to be associated with
the volume in the Master Index and ``{\em idx-file\/}'' is the file 
specification of the associated {\tt .idx} file which contains the relevant
information.

For example, suppose you have a document set containing four documents ---
a User's Guide, a Reference Manual, an Operations Manual, and an Installation
Guide.  You wish to build an Introduction document which contains a Master
Index of the document set.  In this case, you might build an {\tt \verb+.mdx+} file
named {\tt intro.mdx} which contains the following lines
\begin{verbatim}
    \usefile{User}{user_guide}
    \usefile{Ref}{ref_manual}
    \usefile{Operator}{op_manual}
    \usefile{Install}{install}
\end{verbatim}
This example assumes that the {\tt .idx} file for the User's Guide is
{\tt user\_guide.idx}, etc.  

\paragraph{Run \IdxTeX}
  Next, process the {\tt \verb+.mdx+} file with
\IdxTeX.
\begin{verbatim}
    $ idxtex intro /master
\end{verbatim}
The \verb+/master+  qualifier indicates that \IdxTeX\ is to prepare a Master Index.
In this case, the filename parameter of the
\IdxTeX\ command is the specification of the {\tt .mdx} file.  

\IdxTeX\ will read the specified {\tt .mdx} file and will read and process
each of the {\tt .idx} files specified in it.  It will generate a master
index output file which has the same name as the {\tt .mdx} file, but with
the {\tt .mnd} filetype.  In this example, \IdxTeX\ will generate the file
{\tt intro.mnd}.

\paragraph{Build the   Master Index} 
 The final step in building the Master Index is
to include a \verb+\input+ command to incorporate the {\tt .MND} file into the
final volume and to run \LaTeX\ on it, just as in the normal index case.

In our example, include the command
\begin{verbatim}
    \input{intro.mnd}
\end{verbatim}
in the file {\tt intro.tex} and process this file using \LaTeX\ as usual.

The result will be a Master Index, which contains all of the index information
contained in the volume set, formatted so that users will easily be able to
find it.  In particular, if a term is referenced in more than one volume of
the volume set, its references for each volume are clearly separated from each
other to make it clearer which volumes are associated with which page 
references.





\subsection{\GloTeX}
\label{se:glotex}

The \GloTeX\ program is used to automate the generation of a Glossary in a
\LaTeX\ document. 
It also helps you maintain a database of definitions for inclusion in 
glossaries.  It supports the use of multiple definition files in a single 
document, so you can keep definitions for different terms in different places.
For example, if you are writing about an application program, you might want
to have one glossary of computer terms and another which describes the
vocabulary appropriate to that application.

\subsubsection{Command Syntax}

The \GloTeX\ program is run as a foreign DCL command.

The \GloTeX\ program is run using a command line of the form

\begin{verbatim}
    $ glotex  file [/style={article | report | special}] 
                                          [/glossary=(def1[,def2[,...]])]
\end{verbatim}

The ``{\tt file}'' parameter 
is required.  It is the name of the {\tt .glo} file
which is \GloTeX 's input file.  If you don't specify an input file, \GloTeX\
will prompt you for one.  You need not specify the {\tt .glo} file type.  The
default file specification for the input file is file type {\tt .glo} in your
current default directory.

\GloTeX\ prints a list of all those labels which are present in your input file,
but for which it can't find any definition.  This report is output both on your
terminal and to a Glossary Log File ({\tt .glg} file).

You can use this feature to make a list of terms needing definition.

\begin{itemize}
\item The {\tt /style} qualifier 
is optional.  It describes how you wish \GloTeX\ to
format the glossary it builds.  There are three possibilities: 

\begin{itemize}

\item The {\tt article} keyword is used if your document uses the 
standard \LaTeX\ {\tt article} document style or any other section oriented
style.  If you specify this keyword, then the glossary is built as a regular,
numbered section in your document, and it will appear in your Table of
Contents, {\tt .toc} file, automatically.

\item  The {\tt report} keyword is used if your document uses the 
standard \LaTeX\ {\tt report} or {\tt book} document styles or any other
chapter oriented style.  If you specify this keyword, then the glossary is built
as an unnumbered chapter in your document, and it will appear in your Table of
Contents automatically.

\item  The {\tt special} keyword is used if your document uses any of
the Monsanto CR\&DS document styles ({\tt pamphlet}, {\tt manual}, or 
{\tt memo}).   The {\tt special} keyword tells \GloTeX\ not to include
commands in its output file to redefine the commands that format the glossary
appropriately, since these document styles have defines these commands.

In section-oriented Monsanto CR\&DS document styles, the glossary is formatted
as a numbered section.  In the chapter-oriented {\tt manual} document style, the
glossary is an unnumbered chapter, with page numbers of the form 
``\mbox{Glossary--nn}''.

\end{itemize}
If you do not include the {\tt /style} qualifier, {\tt special} is assumed.


\item The {\tt /glossary} qualifier is used to specify the list of Glossary
Definition Files which \GloTeX\ will search to find the definitions you wish to
include. You may include as many files as you wish --- \GloTeX\ will search
them in the order you specify.  For each file, \GloTeX\ will look for a file
with the file type {\tt .gdf} in your current default directory, unless you
override the defaults. 

The {\tt /glossary} qualifier is optional.  However, you must use it unless
you are using a Monsanto CR\&DS document style, because there is no other way
for \GloTeX\ to know where your definitions are to be found.\footnote{If you use
a Monsanto CR\&DS document style, special commands are available to allow you to
build the glossary definition file specifications into your document. They are
described later.} 
\end{itemize}    

\subsubsection{Building a Glossary --- Step by Step}

To automate the generation of a Glossary, here is what you do: 

\begin{itemize}

\item \LaTeX\ creates a list of those terms to be included in the Glossary in a
file named (in this example) {\tt myfile.glo}. This file is not created
automatically.  If you want one, you must include the \verb+\makeglossary+
command in your document. 

The \verb+\makeglossary+ command must be placed in the preamble of your source 
file --- that is, between the \verb+\documentstyle+ command and the
\verb+\begin{document}+ command.

 

\item When you come across a word or phrase in your document which you would
like to define in the Glossary, insert a \verb+\glossary+ command in your
source file. The format of this command is  \begin{verbatim} \glossary{label}
\end{verbatim} where ``label'' is an identifier you assign to this word or
phrase.  This identifier is used to locate the definition in a Glossary
Definition File which you wish to be included.  That is, if ``label'' matches
the identifier belonging to a definition in the database, then that definition
will be included in the Glossary.


\item If the definitions you wish to include are not already in a Glossary
Definition File, create the appropriate file.  See section \ref{glodeffile}
for more information.

\item Run \LaTeX\ on your document.  This
will generate a file {\tt myfile.glo} which contains the collected information
from the \verb+\glossary+ commands which you placed in your document. 


                                                                           

\item Run \GloTeX\ on the {\tt .glo} file, specifying the name of one or more
Glossary Definition Files from which definitions are to be drawn.  \GloTeX\ will
produce a file which has the same filename as you \LaTeX\ file, but with the
file type {\tt .gls}. and contains the glossary information. 
In our example, we will make several assumptions.  
\begin{enumerate}
\item Lets assume that your document {\tt myfile.tex} uses the standard \LaTeX\
{\tt article} document style. 
\item Lets also assume that you are using definitions from three sources ---
a local glossary of application specific terms, a public glossary of 
biological terms, and a public glossary of computer terms.  The biological
glossary is assumed to be located in the {\tt genlocdoc:} area, while the
computer glossary is located in the {\tt crl\_documentation:} area.
\end{enumerate}

Then, the command line
\begin{verbatim}
     $ glotex myfile /style=article -
     $_    /glossary=(myglos,genlocdoc:biodefs,-
     $_              crl_documentation:edpdefs)
\end{verbatim}
will process the {\tt myfile.glo} file, using the three specified Glossary
Definition Files, and produce the file {\tt myfile.gls}, formatted appropriately
for the standard \LaTeX\ {\tt article} document style.

\item Add the command
\begin{verbatim}
     \input{myfile.gls}
\end{verbatim}
to include the {\tt .gls} file in your document at the appropriate place.  

\item Now you are ready to generate the final document.  To do this, run
\LaTeX\ on the document twice.  The first time you run \LaTeX, the Glossary
will be included in your document and a reference will be made to include a
Table of Contents entry for it.  The second time you run \LaTeX, the Table of
Contents will be updated to include an entry for the Glossary. 
\end{itemize}



\subsubsection{Glossary Definition Files}
\label{glodeffile}

A Glossary Definition File is a text file which contains one or more
definitions of words or phrases (an ``entry'') in a special, defined format. 
The default file type for these files is {\tt .gdf}. 

An entry in the file begins with ``{\tt @entry}'' in lower case as the first
characters on a new line, and has three parts (two of which are arguments to
the {tt @entry} command.

The format of an entry in a {\tt .gdf} file is
\begin{verbatim}
     @entry{label, item} definition
     definition (continued...)
\end{verbatim}

The {\tt @entry} command takes two arguments the {\tt label} and the {\tt
item}
strings.

\begin{itemize}
\item The {\tt label\/} is an identifier used to associate a particular
definition with a particular \verb+\glossary+ command from the document.  

The label is a {\bf case sensitive} string.  If the label of an entry in the
{\tt .gdf} file matches a label specified in a \verb+\glossary+ command exactly
(character for character, including any whitespace characters), then the
term will be included in the Glossary.

\item The {\tt item\/} is the word or phrase being defined by the entry.
Visually, the item appears on the left-hand portion of the page, separated from
the definition.

  The item string is optional --- if it is
not specified, then the label string will be used as the item string as well.
\end{itemize}

Glossary entries are formatted in an {\tt infomap} environment,  a special
variation of the {\tt description} environment.\footnote{If you specify
\verb+/style=article+ or \verb+/style=report+ in your command line, \GloTeX\ will
automatically include in its output file the commands needed to define this new
environment.}  In an {\tt infomap} environment, the item is placed in a
\verb+\parbox+.  If the item is too long for the space allotted it, it will be
typeset in multiple lines.  You can often help this process along by careful
inclusion of \verb+\\+ commands in the item to control where the lines are
split. 

The default formatting in an item is {\bf boldface}.  Other possibilities are
available, however.  In this document, for example, the item ``glo File'' was
entered as ``\verb+{\tt glo} File+''.  Notice that the term appears in the
correct alphabetical order, in spite of the presence of the special formatting
characters.

The {\tt definition} is the body of the glossary entry.  It 
consists of as much normal \LaTeX\ source as you wish --- multiple paragraphs,
lists, etc., are all supported.  




The {\tt definition} may begin on the {\tt @entry} line, although that is not
required.  All text is considered as {\tt definition}, beginning with the first
character after the closing brace of the {\tt @entry} command up to the
next {\tt @entry} command or the end of the file.

\subsubsection{\LaTeX\ Extensions Which Support Glossaries}

Standard \LaTeX\ document styles define the \verb+\makeglossary+ and
\verb+\glossary+ commands described above.  

If you use one of the CR\&DS document styles ({\tt pamphlet}, {\tt manual}, or
{\tt memo}), then additional commands are defined which make glossary generation
even nicer.


The ``{\tt theglossary}'' environment is
the context for a glossary.  That is, a glossary begins with the
\verb+\begin{theglossary}+ command and ends with the \verb+\end{theglossary}+
command.  Each CR\&DS document style defines this environment so that the
formatting of the Glossary is appropriate to the document.  When you specify
{\tt /style=article} or {\tt /style=report}, \GloTeX\ includes in the 
{\tt .gls} file the necessary commands to define this environment for you.


The \verb+\glossaryfile+ command is used
to specify one or more {\tt .gdf} files to be used to locate definitions.
The format of the command is
\begin{verbatim}
     \glossaryfile{filelist}
\end{verbatim}
where {\tt filelist} is either a single file specification or a list of file
specifications, separated by commas.

You may specify more than one \verb+\glossaryfile+ command if you wish.  The
{\tt .gdf} files you specify are searched in the following order:
\begin{enumerate}
\item Any {\tt .gdf} files specified on the command line using the {\tt 
/glossary} qualifier are searched in the order specified.
\item Any {\tt .gdf} files specified in the first \verb+\glossaryfile+ command
in the source document are searched in the order specified.
\item Addition {\tt .gdf} files are searched in order as specified by additional
\verb+\glossaryfile+ commands.
\end{enumerate}

The \verb+\insertglossary+ command is
used to place a glossary at a particular place in the document.  It is
equivalent to issuing the \verb+\input{myfile.gls}+ command, where {\tt myfile}
is the name of the source document.



\subsection{\protect\SLiTeX}
\label{se:slitex}
\SLiTeX\ is a version of \LaTeX\ used to produce monochrome or colour
slides, or transparencies. Colour is obtained by producing a set of
black-and-white outputs, one for each colour required. These
{\em colour layers\/} are copied onto a special sheet that produces
a transparency of the appropriate colour, and overlaying the
sheets produces the coloured effect.

The input into \SLiTeX\ consists of a {\em root file\/} and a separate
{\em slide file\/} both with extension \mbox{\tt .tex}. 
The root file contains a \hbox{\verb|\documentstyle{slides}|}
command, instructions about the colours required, the name of
the slide file, and some commentary.
(There are no other local styles or options available so far.)
The slide file contains the text to appear in the slides within
a series of \mbox{\tt slide} environments.  The \mbox{\tt slide} environment
specifies the colour of the text.

To run \SLiTeX\ type:
\begin{verbatim}
      $ slitex root_file
\end{verbatim}
Then proceed with stage 2 of {\em Running the Sample File\/}
to obtain the printed slides.
The output produced by \SLiTeX\ is larger than produced by the
corresponding \LaTeX\ fonts, and it is in a sans-serif typeface.

See Appendix A of the \LaTeX\ manual for full details.

Refer to Section~\ref{se:intro} if you want slide
files or \hbox{\verb|\input|} files to be in a different directory from
your root file.

\subsection{Fonts}
\label{se:fonts}
Almost all the symbols available in our fonts can be generated by
ordinary \LaTeX\ commands.  However, there are type sizes not
obtainable by \LaTeX's size-changing commands with the ordinary
document styles.  Consult a local \TeX\ expert to find the
\TeX\ name for such a font.

Tables~\ref{tab:styles} and \ref{tab:fonts} allow you
to determine if the font for a type style at a particular
size is pre-loaded, loaded on demand, or unavailable.
\begin{table}[htb]
\centering
\begin{tabular}{l|r|r|r|}
\multicolumn{1}{l}{size} & 
\multicolumn{1}{c}{default (10pt)} &
        \multicolumn{1}{c}{11pt option}  &
        \multicolumn{1}{c}{12pt option}\\
\cline{2-4}
\verb|\tiny|       & 5pt  & 6pt & 6pt\\
\cline{2-4}
\verb|\scriptsize| & 7pt  & 8pt & 8pt\\
\cline{2-4}
\verb|\footnotesize| & 8pt & 9pt & 10pt\\
\cline{2-4}
\verb|\small|      & 9pt  & 10pt & 11pt\\
\cline{2-4}
\verb|\normalsize| & 10pt & 11pt & 12pt \\
\cline{2-4}
\verb|\large|      & 12pt & 12pt & 14pt \\
\cline{2-4}
\verb|\Large|      & 14pt & 14pt & 17pt \\
\cline{2-4}
\verb|\LARGE|      & 17pt & 17pt & 20pt\\
\cline{2-4}
\verb|\huge|       & 20pt & 20pt & 25pt\\
\cline{2-4}
\verb|\Huge|       & 25pt & 25pt & 25pt\\
\cline{2-4}
\end{tabular}
\caption{Type sizes for \LaTeX\ size-changing commands.}\label{tab:styles}
\end{table}

\begin{table}[htb]
\centering
\begin{tabular}{l|c|c|c|c|c|c|}
\multicolumn{1}{l}{}& 
\multicolumn{1}{c}{\tt \bs it} &
\multicolumn{1}{c}{\tt \bs bf} &
\multicolumn{1}{c}{\tt \bs sl} &
\multicolumn{1}{c}{\tt \bs sf} &
\multicolumn{1}{c}{\tt \bs sc} &
\multicolumn{1}{c}{\tt \bs tt} \\
\cline{2-7}
5pt  & D & D & X & X & X & X \\
\cline{2-7}
6pt  & X & D & X & X & X & X \\
\cline{2-7}
7pt  & P & D & X & X & X & X \\
\cline{2-7}
8pt  & P & D & D & D & D & D \\
\cline{2-7}
9pt  & P & P & D & D & D & P \\
\cline{2-7}
10pt & P & P & P & P & D & P \\
\cline{2-7}
11pt & P & P & P & P & D & P \\
\cline{2-7}
12pt & P & P & P & P & D & P \\
\cline{2-7}
14pt & D & P & D & D & D & D \\
\cline{2-7}
17pt & D & P & D & D & D & D \\
\cline{2-7}
20pt & D & D & D & D & D & D \\
\cline{2-7}
25pt & X & D & X & X & X & X \\
\cline{2-7}
\end{tabular}
\caption{Font classes: P = pre-loaded, D = loaded on demand, 
         X = unavailable.}\label{tab:fonts}
\end{table}
Table~\ref{tab:styles} tells you what size of type is used for each
\LaTeX\ type-size command in the various document-style options.  For
example, with the \mbox{\tt 12pt} option, the \hbox{\verb|\large|}
declaration causes \LaTeX\ to use 14-pt type.  Table~\ref{tab:fonts}
tells, for every type size, to which class of fonts each type style
belongs.  For example, in 14-pt type, \verb|\bf| uses a pre-loaded
font and the other five type-style commands use load-on-demand fonts.
Roman (\verb|\rm|) and math italic (\verb|\mit|) fonts are all
pre-loaded; the \hbox{\verb|\em|} declaration uses either italic
(\verb|\it|) or roman.



\subsubsection{Font Substitution}
\label{se:fontsub}

If a font cannot be found at the required magnification, the DVI translators will by default
attempt to use the closest magnification that they can find.
However it is also possible to substitute another font or magnification under 
the control of a font substitution file. 

If the current DVI file is {\tt foo.dvi}, the files {\tt foo.sub},
{\tt texfonts.sub}, and {\tt TEX\_INPUTS:texfonts.sub} will be tried in that
order.
The first two will be found on the current directory, and the last is the
system default.
This gives the option of document-specific, user-specific, and system-specific
substitutions.

Font substitution lines have the form:
\begin{verbatim}
      % comment
      oldname.oldmag  ->      subname.submag  % comment
      oldname oldmag  ->      subname submag  % comment
      oldname         ->      subname         % comment
\end{verbatim}
Examples are:
\begin{verbatim}
      % These provide replacements for some LaTeX invisible fonts:
      icmr10.1500     ->      cmr10.1500      % comment
      icmr10 1500     ->      cmr10 1500      % comment
      icmssb8         ->      cmssb8          % comment
\end{verbatim}
The first two forms request substitution of a particular font and
magnification.
The third form substitutes an entire font family; the closest available
magnification to the required one will be used.
Any dots in the non-comment portion will be converted to spaces, and
therefore, cannot be part of a name field.

The first matching substitution will be selected, so magnification-specific
substitutions should be given first, before family substitutions.

Comments are introduced by percent and continue to end-of-line,
just as for \TeX. One whitespace character is equivalent to any amount of 
whitespace. Whitespace and comments are optional.
                                     

\subsection{Special Versions}
\label{se:special}

Several foreign-language and other special versions of \LaTeX\
are available. Your system manager should be able to 
tell you if any foreign language version of \LaTeX\ is available on 
your system. In the case where you need a foreign language or special version 
of \LaTeX\ you can contact V. Laspias (MAVAD::VL) who should be able to tell 
you if a version can be created, provided there is sufficient demand.

\subsection{Running \mbox{\tt lablst.tex} and \mbox{\tt idx.tex}}
\label{se:lablstidx}
You can obtain a list of the cross-referencing keys, the pages and sections
that appear in your document. To do this type:
\begin{verbatim}
      $ latex tex_inputs:lablst
\end{verbatim}
and you will be prompted for the name of your input file, which should be
typed without an extension, and for the name of the main document style
({\em e.g.} \mbox{\tt article}), used by that file, excluding options. 
If you have your own document style you will need a 10-pt version, though
it could be just a copy of your 12pt. \mbox{\tt lablst} first creates a file
of \LaTeX\ commands to produce the list of keys within a file called
\mbox{\tt lablst.lab} in your current directory.  Then it uses this file to
produce a \mbox{\tt lablst.dvi} which can be processed as in Section
\ref{se:example} to produce a typeset listing. 

The index entries on an \mbox{\tt .idx} file are printed by running \LaTeX\
on the file \mbox{\tt idx.tex}, which is done by typing
\begin{verbatim}
      $ latex tex_inputs.idx
\end{verbatim}
\LaTeX\ will ask for the name of the \mbox{\tt .idx} file, which is typed
without an extension.


\subsection{Miscellaneous}
\label{se:misc}

\section{Acknowledgements}

Three old SUNs have been withdrawn and their contents merged with this
document. They are:
\begin{itemize}
\item SUN/22 ``Hindlegs --- Lineprinter output from \TeX\ files''
\item SUN/34 ``\TeX\ DVI file translators''
\item SUN/77 ``DVIview --- DVI file previewers''
\end{itemize}

Thanks are due to Dave~Terrett, Alan~Lotts, Mike~Lawden, Martin Bly and 
Malcolm Curry authors of the above documents, from which sections of this
User Note are formed.

\section{Trademarks}
\begin{itemize}
\item \TeX\ and \MF\ are Trademarks of the American Mathematical Society,
\item \PS\ and TranScript are trademarks of Adobe Systems, Inc.,
\item UNIX is a trademark of Bell Laboratories,
\item SUN is a trademark of Sun Microsystems Ltd,
\item VAX/VMS, DECstation and VAXstation are trademarks of Digital Equipment 
Corporation.
\end{itemize}


\begin{thebibliography}{99}
\addcontentsline{toc}{section}{References}

\bibitem{latexbook} Leslie Lamport {\em \LaTeX\ A Document Preparation System
--- User's Guide and  Reference Manual\/} Addison-Wesley Publishing Company,
ISBN~0-202-15790-X 
\bibitem{texbook} Donald Knuth {\em The \TeX book\/} Addison-Wesley Publishing
Company, ISBN~0-201-13448-9
\bibitem{sun12} M. Lawden {\em \LaTeX\ Cook-Book''\/} Starlink User Note 12,
Starlink Project. 
\bibitem{sun93} J. Murray {\em \TeX\ --- A Superior document preparation 
system\/}, Starlink User Note 93, Starling Project. 
\bibitem{sun176} V. Laspias {\em DVIPS: A \TeX\ driver\/}, Starlink User Note
176, Starlink Project.
\bibitem{etr791} F Teagle {\em The \LaTeX\ Cookbook\/} 
\bibitem{sgp28} M. Lawden  {\em Starlink Documentation Production\/} Starlink
General Paper 28, Starlink Project.
\end{thebibliography}

\end{document}


XXXXXXXXXXXXXXXXXXXXXXXXXXXXXXXXXxx


\subsection{DVIDIS --- Vaxstations with VWS}
\label{se:dvidis}

The command:
\begin{verbatim}
      $ DVIDIS <filename>
\end{verbatim}
will create a special, nearly full screen window on the VAXstation, read the
beginning of {\tt <filename>.dvi} and display the top of the first page in 
the window. The program then waits for further commands to be entered via the
keyboard of the VAXstation.

\subsubsection{Commands}

While executing, DVIDIS will accept commands from both the keypad and the
normal keyboard.
Most of the keypad commands are simple key hits while
the keyboard commands are single character followed by a {\tt<RETURN>}.
The Find key causes the command window to appear, overlaying part of the
special window displaying the output.
This window also notes all commands
used and other information such as the physical page selected as
different parts of the file are displayed.

The full command set and key identities can be viewed by hitting the VAXstation
{\tt<HELP>} key but the following will give a guide to the commands available.

\subsubsection{Keypad Commands}

\begin{list}%
{}%
{\settowidth{\labelwidth}{\bf Select Prev Screen x}
\settowidth{\labelsep}{aaaa}
\settowidth{\rightmargin}{aaaaaaaaaa}
\addtolength{\labelwidth}{\labelsep}
\addtolength{\labelwidth}{\labelsep}
\setlength{\leftmargin}{\labelwidth}}
%
\item [{\bf Help}] Open command window and show Help.
This is the VAXstation help key, not the editing keypad help key
\item [{\bf Find}] Show command window
\item [{\bf Remove}] Hide command window
\item [{\bf Next Screen}] move on to next screen ({\em not} next \TeX\ page)
\item [{\bf Prev Screen}] move back to previous screen ({\em not} previous 
 \TeX\ page)
\item [{\bf Select}] followed by {\tt\verb@{+-}@} {\it nnn}{\tt<RETURN>}
will jump forward or back {\it nnn} pages.
\item [{\bf Select Next Screen}] (i.e. Select key then Next Screen key) jump
 to next \TeX\ page
\item [{\bf Select Prev Screen}] jump to previous \TeX\ page
\item [others] Various keypad keys are defined to enable look Left, Right, Up,
Down, Paper Top and Paper Bottom
\end{list}

\subsubsection{Keyboard Commands}

Although grouped together here (and in the help display), all these are
single character commands which should be followed by {\tt<RETURN>}

\begin{list}%
{}%
{\settowidth{\labelwidth}{\tt eq {\em control} Z}
\settowidth{\labelsep}{aaaa}
\addtolength{\labelwidth}{\labelsep}
\setlength{\leftmargin}{\labelwidth}}
%
\item[{\tt h}]  displays help information
\item[{\tt lrup LRUP}] look left, right, up down -- upper case for larger steps
\item[{\tt v\symbol{94}}] page bottom or top (at current horizontal position)
\item[{\tt np}]  next or previous page (not screen)
\item[{\tt bct}]  page bottom left, centre or top left
\item[{\tt io}]  Zoom in or out by one zoom step
\item[{\tt 1Ww}]  Set to no zoom, zoom out three steps, zoom in two steps
\item[{\tt eq {\em control} Z}] exit from DVIDIS
\end{list}

\subsubsection{Options}

Various qualifiers are available as DVIDIS is started. They cannot be
changed while the programme is executing. 
All options are of the form {\tt-l}
where {\tt l} is any single letter, with the single exception of 
{\tt-verbose} (for historical reasons).
Some are followed -- after a space -- by one or a pair of
numeric arguments; the {\tt-s} command must be followed by a paper size
specification.

For example:
\begin{verbatim}
      $ DVIDIS -w -p 2 4 <filename> 
\end{verbatim}
means {\sl display pages 2 to 4 of }{\tt <filename>.dvi}{\sl\/ in 
{\bf wide} mode.}

The following qualifiers are recognized:

\begin{list}%
{}%
{\settowidth{\labelwidth}{\tt-verbose a}
\settowidth{\labelsep}{aaaa}
\settowidth{\rightmargin}{aaa}
\addtolength{\labelwidth}{\labelsep}
\setlength{\leftmargin}{\labelwidth}}
%
\item[\tt-c]  The fonts that DVIDIS uses are meant for a 300 dpi printer.
A VAXstation display has about 75 dpi.
As a result, if you display the glyphs from these fonts at a 1-1 pixel-to-pixel
mapping, they are about 4 times as large as they would be on the page, and only
about 1/9 of the page is visible at a time.
Normally, DVIDIS compresses all fonts by a 2-to-1 factor.
The fonts are still readable, but you can now get most of the width of a page --
all with any reasonable margins and over half the height on the screen. 
Specifying {\tt-c} turns off this compression; it's useful mainly for
looking at the glyphs, not the text. 
\item[\tt-w] Specifies a 4-to-1 glyph compression factor.
This allows you to see how a whole page looks, but unfortunately it's rather
hard to read the resulting text.
\item[\tt-m]
Tells DVIDIS to try to use smaller fonts, rather than compressing
larger ones.
This only works if you actually {\it have} the small fonts on your system.
When it works, it improves the appearance of individual
characters, although it often makes lines come out with slightly
different lengths than they should.
{\tt-m} is especially useful in conjunction with {\tt-w}.
{\tt-m} in conjunction with {\tt-c} is meaningless.
\item[\tt-s ss] Set the paper size.
This determines what size paper the program assumes you are printing on.
The default is {\bf letter}, which is 8.5 x 11 inches -- this is not quite A4.
The argument is the name of a paper size; type DVIDIS {\tt-s ?} for a list
of possible values.

The paper size controls where DVIDIS places a visible gray border,
and when it tells you that characters have run off the page. 
It always displays the characters, however.

Paper size {\bf huge} is meant for special purposes; using it will make
many of the commands you can type to DVIDIS while it is running work
in strange ways.
You should generally avoid using size huge.

\item[\tt-verbose]
Makes the program talkative.
This is mainly useful for debugging, but since it displays all the font files
it opens, successfully or not, it can be used to determine which files are
needed.
The needed files can be created if they don't already exist.
They can also be put on your workstation's local disk to improve performance.

\end{list}

The rest of the options are rarely used:

\begin{list}%
{}%
{\settowidth{\labelwidth}{\tt-verbose a}
\settowidth{\labelsep}{aaaa}
\settowidth{\rightmargin}{aaa}
\addtolength{\labelwidth}{\labelsep}
\setlength{\leftmargin}{\labelwidth}}
%

\item[{\tt-v \#}] vertical margin.
The argument {\tt\#} is the margin, in milli-inches.
Default is 1000 milli-inches, which is the same default used by many common
DVI printing programs.
If you find a lot of {\sl off-screen glyph}\/ errors, and you have the paper
size set correctly, try changing these parameters.
(Besides 1000, the second most common value expected is 0 for both margins.)

Note that the default corresponds to the current TeX standard.
If you find you have to use some other value to make your output look
right, some other piece of software you are running is obsolete.

\item[\tt-h \#] Set the horizontal margin.
The default is 1000 milli-inches.


\item[\tt-p \#$_1$ \#$_2$] Limits the display to pages {\tt\#$_1$} to
 {\tt\#$_2$} inclusive.
You can only specify one page range at a time, but you can specify multiple
{\tt-p} options, each specifying a page range.
When you specify page ranges, DVIDIS will only {\sl see}\/ pages within the
ranges given.
Nothing you can do while the program is running will let you display any page
not within any page range -- DVIDIS will simply skip past the pages.

\item[\tt-d] Debug.  Generates a lot of output that's useful only to developers.
\end{list}

\subsubsection{Restrictions}

If the original \TeX\ run specified a magnification other than 1000, the
spacing will be incorrect.
The only current work--around is to re-run \TeX\ without the magnification.

Graphics cannot be included using \verb+\special+.

For corrections to be viewed, a new .dvi file must be generated.

\subsubsection{Hints}

DVIDIS constructs a complete page whenever it enters a \TeX\/ page.
This makes moves about a page fast but does involve a delay on page start.
There may be no response for several seconds after the display window has
been completed.
It also stores font characters in memory as it uses them and so becomes
faster as more of the document is examined.

It is normally very useful to have an editing window available while previewing.
This is most easily achieved by creating a second window before starting
DVIDIS.
The VAXstation CYCLE key can then be used to move between the editing
window and the DVIDIS command window.
However, to get back to the normal DVIDIS display, leave the mouse cursor in
an area of the DVIDIS special window which is not overlayed by the editing
window and simply use the mouse button to pop it to the front.

When leaving the editing window, remember that you need to CYCLE back to the
DVIDIS command window before entering any DVIDIS commands -- simply popping
the DVIDIS display window is not sufficient !


