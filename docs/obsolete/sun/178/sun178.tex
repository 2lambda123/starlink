\documentstyle[11pt]{article}
\pagestyle{myheadings}

%------------------------------------------------------------------------------
\newcommand{\stardoccategory}  {Starlink User Note}
\newcommand{\stardocinitials}  {SUN}
\newcommand{\stardocnumber}    {178.1}
\newcommand{\stardocauthors}   {D L Terrett}
\newcommand{\stardocdate}      {15 March 1994}
\newcommand{\stardoctitle}     {XAdam --- a GUI for running ADAM applications}
%------------------------------------------------------------------------------

\newcommand{\stardocname}{\stardocinitials /\stardocnumber}
\renewcommand{\_}{{\tt\char'137}}     % re-centres the underscore
\markright{\stardocname}
\setlength{\textwidth}{160mm}
\setlength{\textheight}{230mm}
\setlength{\topmargin}{-2mm}
\setlength{\oddsidemargin}{0mm}
\setlength{\evensidemargin}{0mm}
\setlength{\parindent}{0mm}
\setlength{\parskip}{\medskipamount}
\setlength{\unitlength}{1mm}

%------------------------------------------------------------------------------
% Add any \newcommand or \newenvironment commands here
%------------------------------------------------------------------------------

\begin{document}
\thispagestyle{empty}
SCIENCE \& ENGINEERING RESEARCH COUNCIL \hfill \stardocname\\
RUTHERFORD APPLETON LABORATORY\\
{\large\bf Starlink Project\\}
{\large\bf \stardoccategory\ \stardocnumber}
\begin{flushright}
\stardocauthors\\
\stardocdate
\end{flushright}
\vspace{-4mm}
\rule{\textwidth}{0.5mm}
\vspace{5mm}
\begin{center}
{\Large\bf \stardoctitle}
\end{center}
\vspace{5mm}

%------------------------------------------------------------------------------
%  Add this part if you want a table of contents
%  \setlength{\parskip}{0mm}
%  \tableofcontents
%  \setlength{\parskip}{\medskipamount}
%  \markright{\stardocname}
%------------------------------------------------------------------------------

\section{Introduction}
Xadam is an X windows Graphical User Interface (GUI) for running ADAM
applications. The following applications packages can be run from xadam:

\setlength{\tabcolsep}{0.3in}
\[\begin{tabular}{llll}
CCDpack &Figaro &Iras90 &KAPPA \\
Photom &PISA &Specdre
\end{tabular}\]

Further ADAM packages will be supported in the future as they become available
on UNIX and a future release of Xadam will allow users own programs to be
run as well.

This note does not attempt to give a complete description of Xadam; instead
it present a ``guided tour'' which should be read while running Xadam. To do
this you will need a UNIX system with the Starlink Software installed (Xadam
is {\em not} available on VMS) and a workstation or X windows terminal,
preferably with a colour screen.

Xadam is designed to be of most use in two situations:

\begin{enumerate}

\item When finding out what a package or application can do. All the commands
in a package are displayed and all of an applications parameters and their
default values can be viewed before attempting to run it.

\item When exploring new data before the details of the reduction process
have been decided.

\end{enumerate}

It is not intended to replace typing ADAM commands on a terminal. For
performing pre-defined sequences of commands, typing at a terminal or writing
scripts will continue to be the most efficient way to use ADAM.

\newpage
\section{The Guided Tour}
\newcounter{step}
\begin{list}{
\arabic{step}}{\usecounter{step}\setlength{\rightmargin}{\leftmargin}}
\item To start the tour, change your current directory to a suitable
place (you will create several files as
part of the tour) and type:

\begin{quote}
\tt xadam \&
\end{quote}

Two windows should appear on your screen; One is a box telling you what
version you are running; click on the button labelled ``OK'' with the
left hand mouse button (from now on all mouse operations will be with
the left hand button unless stated otherwise) to close the box.

\item Now look at the other window; for the moment, ignore everything
except the menu bar at the top.  Click on the button labelled {\em
Packages} and a menu listing all the supported ADAM packages will
appear.  Click on {\em KAPPA} and another window will appear with a
column of buttons labelled with application categories on the left and
alphabetic list of kappa commands on the right.

This is where you select the application you want to run. If you know
the name of the application you can select it by clicking on the
command name in the alphabetic list; the list can be scrolled by
dragging the scrollbar on the right hand side.

If you don't know the name, clicking on one of the category buttons
will pull down a menu listing the commands in that category. Selecting
one of the commands will normally make the menu disappear, but if you
use the middle mouse button instead of the left had button to click on
the category name the menu can be dragged across the screen by holding
down the mouse button and moving the mouse. The menu will now remain on
the screen.

All the menus in the system behave in this way; to remove a menu from the
screen, click on the bar at the top of the menu.

\item Now Click on {\em block} in the alphabetical list (be careful to
click the mouse button only once; a double click will cause something
different to happen) and another window will appear with the names of
all block's parameters and boxes for entering and displaying their
values.  Some of the parameter boxes may have values in them (it
depends on how you have used KAPPA in the past); if so, select {\em
Clear All} from the {\em QuickSelect} menu so that all the boxes are
empty.  Type the name of an existing NDF ({\em e.g.}\/ {\tt
/star/bin/kappa/comwest}) into the box and click on the button labelled
``Run/Cont''.

The message:
\begin{quote}\tt
Smoothing box size (box)
\end{quote}

will be displayed. In addition a suggested value may appear in the box
below, this depends on whether or not you have run {\tt block} before
and whether therefore the parameter ``box'' has a current value or
not). If a value did appear then just click on ``Run/Cont'' again, if
not type in something suitable first ({\em e.g.}\/ {\tt 3,3}).  A
value for ``out'' will then be requested in a similar way; this time
enter a name in the box labelled ``out'' instead of the box underneath
the message. Click on {\em Run/Cont} again and the message will change
to ``Running...'' until block has finished when it will change to
``Finished...'' (on a fast machine you may not see the ``Running''
message at all).

\item Run block again. When it request a value for ``box'', click on
{\em Help} and help for ``box'' will be displayed. Click on {\em Store}
and value displayed in the box will be copied into the box labelled
``box''.  Click on {\em Run/Cont} again and ``block'' will run. This
time there was no request for ``box'' because a value for ``box'' has
already been supplied.

\item Now look at the panels in the window labelled ``Xadam''; the
panel on the right lists all the {\tt .sdf} and {\tt .dst} files in the
current directory (there should be at least the one you have just
created with {\tt block}).  Click on one of the names and it will be
highlighted.  Click on the button labelled ``Clear'' next to the box
for ``in'' in the parameter entry box and then click on the box itself
with mouse button 2. The full name of the file you selected will be
copied into the box.

You can move around the directory structure by clicking on the
directory names in the left hand panel of the ``Xadam'' window or by
typing a directory name followed by Return into the box at the very
bottom of the window.

\item Pull down the menu labelled ``Parameters'' at the top of the
parameter entry window; a menu will appear with a list of block's
parameters.  This menu is used to select which parameters are displayed
in the window. Whether or not a parameter is displayed effects the way
the Xadam responds to prompts from the application; If the parameter
being prompted for is not visible you will only get a request to enter
a value as a last resort; {\em i.e.}\/ if there is no value entered in
the parameter's box and there is no default suggested by the
application.  If the parameter is visible, you get the opportunity to
edit or replace any suggested default before proceeding.

The other menu has commands for showing, hiding and clearing all the
parameters at once and a command for reseting them all back to the
state that they were in when the application was first selected.

If an application has too many parameters to display at one time they
are divided up into pages which can be selected with the {\em Prev
page} and {\em Next page} buttons. You can change the number of
parameters displayed on each page by selecting {\em Page Size...} from
the {\em QuickSelect} menu.

\item Try running the ``stats'' program on the file you have just
created.  The output is written into the window underneath the
parameters; you can use the scroll bars to view any text that has
scrolled off the top of the window.

You can use any of the numbers in the output window as input to another
command by cutting and pasting in the same way as you did with the file
names in the list of ndfs.

\item Run a few more programs and then pull down the {\em Options} menu
on the directory browser window and select {\em History}. Another
window will appear with a list of all the applications you have used so
far. To re-run an application you can select it, by clicking on it
once, from this list instead of hunting for on the lists of all
applications. The list can be edited in various ways by selecting items
from the {\em Edit} menu; for example you can sort it into order of
frequency of use so that the most frequently used are at the top.

The list can also be saved to disk and restored next time Xadam is used
and the list can be frozen ({\em i.e.}\/ the addition of new commands to the
list disabled); see the {\em File} menu.

\end{list}

To exit from Xadam, select {\em Exit} from the {\em File} menu on the
ndf browser window and click on the {\em OK} button in the box that appears
asking you to confirm that you want to exit.


\subsection{Other Features}

The {\em Options} menu in the ``Xadam'' window has the commands for:

\begin{list}{}{\setlength{\rightmargin}{\leftmargin}}
\item[\bf xmake...] Creating a gwm window.

\item[\bf Auto Display...] Setting the name of a program that displays
the new ndf every time an application generates one.

\item[\bf Browser filter...] Changing the filter that selects the files
to displayed in the browser window.

\item[\bf Change directory] cd'ing to the directory displayed in the
``ndf browser''. Applications are normally run in the directory that
you were in when you started xadam regardless of which directory is
displayed by the browser. This option allows you to change the default
directory for applications.

\end{list}

The {\em File} menu has commands for:

\begin{list}{}{\setlength{\rightmargin}{\leftmargin}}

\item[\bf Logging...] Logging the interaction with ADAM applications to
either a file and/or the terminal window running Xadam. This can be
used to record a sequence of data processing operations or to find out
exactly how an application is being run if it isn't doing what you
expect.

\end{list}

The {\em Help} menu will starts the Mosaic hypertext browser (SUN/175)
where there is more detailed information on how to use Xadam.

\subsection{Using the Keyboard}

As well as using the mouse to move around the interface you can use
keyboard ``short cuts''. Select {\em Auto display} from the {\em
Options} menu and a dialog box will appear; move the mouse over it and
the button labeled ``none'' will appear highlighted - this means that
this widget has the keyboard focus. Press the down arrow and the
highlighting will move to ``Kappa display''; press the space bar to
select it. Press Tab and the focus moves to the list of window names;
press it again and it moves to the {\em OK} button. The left and right
arrow keys then move between the {\em OK} and {\em Cancel} buttons.
Pressing return always activates the default button ({\em OK} in this
case) which is indicated by the sunken border.

Widgets for entering text indicate that they have focus by displaying
a vertical bar at the insertion point and have a number of short cuts
for editing the text:

\begin{itemize}
\item The left arrow key and Control-d move the insertion point left by
  one character.

\item The right arrow key and Control-f move the insertion point right
  by one character.

\item Control-e moves the insertion point to the end of the line.

\item Control-k moves the insertion point to the beginning of the line.

\item Control-u deletes everything from the beginning of the line to the
  insertion point.
\end{itemize}
\end{document}
