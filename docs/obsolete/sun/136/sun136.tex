\documentstyle[11pt]{article} 
\pagestyle{myheadings}

%------------------------------------------------------------------------------
\newcommand{\stardoccategory}  {Starlink User Note}
\newcommand{\stardocinitials}  {SUN}
\newcommand{\stardocnumber}    {136.4}
\newcommand{\stardocauthors}   {M J Bly}
\newcommand{\stardocdate}      {9 March 1994}
\newcommand{\stardoctitle}     {FITSIO --- Disk FITS 
				Input/Output Subroutines\\[2.0ex]
				Version 3.4}
%------------------------------------------------------------------------------

\newcommand{\stardocname}{\stardocinitials /\stardocnumber}
\renewcommand{\_}{{\tt\char'137}}     % re-centres the underscore
\markright{\stardocname}
\setlength{\textwidth}{160mm}
\setlength{\textheight}{230mm}
\setlength{\topmargin}{-2mm}
\setlength{\oddsidemargin}{0mm}
\setlength{\evensidemargin}{0mm}
\setlength{\parindent}{0mm}
\setlength{\parskip}{\medskipamount}
\setlength{\unitlength}{1mm}

%------------------------------------------------------------------------------
% Add any \newcommand or \newenvironment commands here
%
%  Remove this for final version
%\special{!userdict begin /bop-hook{gsave 200 30 translate 65 rotate
%/Times-Roman findfont 216 scalefont setfont 0 0 moveto 0.93 setgray
%(DRAFT) show grestore} def end}
%------------------------------------------------------------------------------


\begin{document}
\thispagestyle{empty}
SCIENCE \& ENGINEERING RESEARCH COUNCIL \hfill \stardocname\\
RUTHERFORD APPLETON LABORATORY\\
{\large\bf Starlink Project\\}
{\large\bf \stardoccategory\ \stardocnumber}
\begin{flushright}
\stardocauthors\\
\stardocdate
\end{flushright}
\vspace{-4mm}
\rule{\textwidth}{0.5mm}
\vspace{5mm}
\begin{center}
{\Large\bf \stardoctitle}
\end{center}

%------------------------------------------------------------------------------
%  Add this part if you want a table of contents
%  \setlength{\parskip}{0mm}
%  \tableofcontents
%  \setlength{\parskip}{\medskipamount}
%  \markright{\stardocname}
%------------------------------------------------------------------------------

\section{Introduction}

The FITSIO package is a series of subroutines for easy Fortran access
to FITS files on disk. It was written by William D. Pence of HEASARC
(High Energy Astrophysics Science Archive Research Center) at the
Goddard Space Flight Center, USA.  FITSIO was submitted for release on
Starlink by Julian Osborne at the University of Leicester.  Starlink is
grateful to William Pence for giving his permission for FITSIO to be
released on Starlink.

\section {Access to FITSIO routines}

FITSIO is a series of subroutines, built into a object library.

\subsection {Unix}

To link a program with the FITSIO library on Unix, include the command
{\tt `fitsio\_link` } in your list of libraries. For example:

\begin{verbatim}
      % f77 myprog -o myprog -L/star/lib `fitsio_link`
\end{verbatim}

\subsection {VMS}

To link a program with the FITSIO library, you should use a LINK
command such as:

\begin{verbatim}
      $ LINK myprog,FITSIO_DIR:FITSIO/LIB,...
\end{verbatim}

adding any other libraries your program requires as necessary.

{\bf NOTE:}~{\em On VMS, FITSIO is an Optional Item in the Starlink
Software Collection. If it is not installed at your site, please see
your Site Manager.}

\section{More Information}

This Starlink User Note is a brief introduction to the FITSIO package.
For more information, please read the document {\em ``FITSIO --- A
Subroutine Interface to FITS format Files''\/} by William D Pence. The
document has been issued as a Starlink Miscellaneous User Document
(MUD/16). The source is also available as the file:

\begin{verse}
{\tt /star/starlink/lib/fitsio/fitsio.tex} on Unix systems, \\
{\tt FITSIO\_DIR:FITSIO.TEX} on VMS systems.
\end{verse}

\section {Release Information}

FITSIO carries a file containing the latest information about the
release and bug fixes, changes and enhancements. This file is:

\begin{verse}
{\tt /star/starlink/lib/fitsio/fitsio.news} on Unix systems, \\
{\tt FITSIO\_DIR:RELEASE.DOC} on VMS systems.
\end{verse}

Users are advised to read the release information file at each new
release.

There are versions of FITSIO for VAX/VMS, SUN workstations (SunOS and
Solaris), DEC Alpha OSF/1 and Open VMS, DEC Ultrix, IBM mainframe, IBM
PC and compatibles, and various other machine types.

The Starlink Unix release carries only the DEC Alpha OSF/1, Sun Solaris
and SunOS, and DEC Ultrix version of FITSIO in a form buildable by the
Starlink makefile system. The sources for other versions can be
obtained as detailed in MUD/16 (see above).

The Starlink VMS release carries the code for most versions in the text
library {\tt FITSIO\_DIR:\-FITSIO.TLB}.

\end{document}
