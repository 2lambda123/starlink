\documentclass[11pt,nolof,noabs]{starlink}

%------------------------------------------------------------------------------
\stardoccategory    {Starlink User Note}
\stardocinitials    {SUN}
\stardocnumber      {34.1}
\stardocauthors     {G R Mellor}
\stardocdate        {7 October 1991}
\stardoctitle       {GNU EMACS --- Display Editor}
%------------------------------------------------------------------------------

%------------------------------------------------------------------------------
% Add any \providecommand or \newenvironment commands here
%------------------------------------------------------------------------------

\begin{document}
\scfrontmatter

\section{Introduction}

This document describes the GNU EMACS editor as distributed by Starlink
for UNIX systems. It is intended to provide an introduction for the
new user. For a comprehensive description of the editor, refer to
the \textit{GNU EMACS Manual}.


The GNU EMACS editor is a powerful editor with many advanced customizable
features.
However, it is relatively simple to provide
an EDT style emulation
for those users already familiar with VAX/VMS editors and the windowing
facilities similar to the EVE editor are also available.

\section{Getting Started}

An Emacs tutorial is provided with the editor and can be accessed
by first running Emacs by typing \texttt{emacs} and then typing \texttt{C-h t}
where \texttt{C-h} means pressing the Control and h keys simultaneously.
This teaches the raw commands which are all key combinations commencing
with either the Control Key (\texttt{C}) or the Meta/Escape Key (\texttt{M}).
A more
user friendly system can be obtained by customizing the editor.
The customized definitions reside in an initialisation file \texttt{.emacs}
in the home directory. A template \texttt{.emacs} file providing
a basic EDT keypad emulation is available in \texttt{/star/emacs}.
Some further basic commands that might also be required are listed
in the GNU Emacs Reference Card which is appended to this document.

\section{Exiting Emacs}

There are two different ways of exiting Emacs: \textit{suspending} and
\textit{killing} Emacs. \textit{killing} means destroying the Emacs job,
however \textit{suspending} Emacs allows you to return to the session later.
The best way to finish a session is to save the file and then suspend
Emacs. Emacs can be resumed later with a \texttt{\%emacs} command if you are
using the C shell.

\end{document}
