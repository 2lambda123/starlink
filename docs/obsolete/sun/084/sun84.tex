\documentstyle{article}
\pagestyle{myheadings}

%------------------------------------------------------------------------------
\newcommand{\stardoccategory}  {Starlink User Note}
\newcommand{\stardocinitials}  {SUN}
\newcommand{\stardocnumber}    {84.1}
\newcommand{\stardocauthors}   {P.~T.~Wallace \& A.~P.~Lotts}
\newcommand{\stardocdate}      {18 August 1989}
\newcommand{\stardoctitle}     {MAILCLUB --- Several users sharing
                                one MAIL address}
%------------------------------------------------------------------------------

\newcommand{\stardocname}{\stardocinitials /\stardocnumber}
\markright{\stardocname}
\setlength{\textwidth}{160mm}
\setlength{\textheight}{240mm}
\setlength{\topmargin}{-5mm}
\setlength{\oddsidemargin}{0mm}
\setlength{\evensidemargin}{0mm}
\setlength{\parindent}{0mm}
\setlength{\parskip}{\medskipamount}
\setlength{\unitlength}{1mm}

%------------------------------------------------------------------------------
% Add any \newcommand or \newenvironment commands here
%------------------------------------------------------------------------------

\begin{document}
\thispagestyle{empty}
SCIENCE \& ENGINEERING RESEARCH COUNCIL \hfill \stardocname\\
RUTHERFORD APPLETON LABORATORY\\
{\large\bf Starlink Project\\}
{\large\bf \stardoccategory\ \stardocnumber}
\begin{flushright}
\stardocauthors\\
\stardocdate
\end{flushright}
\vspace{-4mm}
\rule{\textwidth}{0.5mm}
\vspace{5mm}
\begin{center}
{\Large\bf \stardoctitle}
\end{center}
\vspace{5mm}

%------------------------------------------------------------------------------
%  Add this part if you want a table of contents
%  \setlength{\parskip}{0mm}
%  \tableofcontents
%  \setlength{\parskip}{\medskipamount}
%  \markright{\stardocname}
%------------------------------------------------------------------------------

\section{INTRODUCTION}

Collaborating groups\footnote{Throughout this
document the word
{\it group} has its normal, non-VMS-specific, meaning.}
of users sometimes want to share a common
MAIL address.  This enables anyone from outside the group
to send MAIL messages to the group in general,
and know that someone from the group will check the MAIL even
if some of its members are temporarily away.  In the past, two
solutions, of beguiling simplicity, have been employed unofficially:
\begin{enumerate}
\item All the users share a single account.
\item All the users know each others' passwords and login
as each other from time to time.
\end{enumerate}
Neither of these is acceptable, from two standpoints:
\begin{itemize}
\item In both schemes there is sharing of passwords, which is
not allowed under Starlink's rules on security.  Bitter
experience has shown that anything other than strictly personal
passwords tends to lead to cases of unauthorised access; and
unauthorised access has, in several instances, led to loss of
files and interruption of service.
\item In the second scheme, especially, the members of the group
have sacrificed their individual privacy.  Even if they don't
themselves mind this, it can
be embarrassing for others to discover that confidential messages
sent to one person have, in fact, been read by someone else,
especially a close colleague (possibly the subject of the
message).
\end{itemize}

A better solution is a {\bf MAILclub} account.  This is based on
the first of the two
na\"{\i}ve schemes described above, but with extra security
provided by running as a {\it captive account}.  A MAILclub account
essentially provides access to the MAIL utility alone;  any member
of the club can login to check for incoming messages, which can then
be printed, forwarded, {\it etc.\ }, but cannot use the account for
other types of work.  The club members should keep the password
to themselves;  however, if unauthorised access does occur, the
captive nature of the account greatly reduces the amount of mischief which
can be perpetrated.

Please note that MAILclubs are {\bf not} to be used for setting up
improvised ``bulletin boards'', with a wide circle of users logging
in to read the latest gossip.  These needs are addressed by the
VAXnotes utility -- see SUN/44.  Improper use of a
MAILclub account will lead to its closure!

\section{HOW TO ASK FOR A MAILCLUB}
A group of users who feel they would like to operate a MAILclub
should contact the manager of the Starlink node on which the account
is to be operated, giving the following information:
\begin{itemize}
\item The purpose of the proposed MAILclub.
\item The preferred username.
\item A complete list of all those who will be entitled
      to login.  All must be Starlink users in their own right.
\end{itemize}
The node manager
must then do the following:
\begin{enumerate}
\item Obtain the approval of
      the Area Management Committee Chairman and the Starlink
      Project Scientist.
\item Make an entry in
      the first section of the file LADMINDIR:USERNAMES.LIS, consisting
      of the
      account name, a description, and the names of {\bf all} the
      members.
\end{enumerate}
\section{OPERATING A MAILCLUB}
A member of the club logs in to the account in the normal way, using
a password known only to the club's authorised users.  An
announcement appears, followed by a request for input;
a reply of $<$CR$>$ causes the MAIL utility to be
entered, while anything else causes a menu to be displayed.  The
available commands are as follows:
\begin{itemize}
\item MAIL
\item MAIL/EDIT
\item MAIL/EDIT=(SEND,REPLY,FORWARD)
\item PASSWORD
\item DIRECTORY
\item HELP
\item FILE\_DELETE
\item EXIT
\item SETPAD
\end{itemize}
The menu items are numbered, and the required command is
invoked by typing its number.

The {\bf MAIL} options provide access to all the usual MAIL
facilities.  The most common function will be to forward
incoming messages to the appropriate group member(s), and
to delete messages that have been dealt with.

The {\bf PASSWORD} command invokes the DCL command SET PASSWORD.
On completion, the user is warned that the other members of the
MAILclub will need to be informed of the change.

The {\bf DIRECTORY} command lists the names of all the files in
the account's top-level (and only) directory, with the exclusion
of any .MAI files.

The {\bf HELP} command displays a summary of the main features of
MAILclubs and the available commands.  It is not related to the DCL
command of the same name.

The {\bf FILE\_DELETE} command allows deletion of files in the top
directory. Files used by MAIL are excluded and /CONFIRM is used.

The {\bf EXIT} command causes LOGOUT.  So does a CTRL/C or
CTRL/Y at any point.

The {\bf SETPAD} command is only available where login occurred from
an X.25 connection.  It offers two options for setting the
terminal characteristics: {\it message} mode and {\it native}
mode.  In {\bf message} mode,
complete lines are echoed locally and only sent to the remote
computer when $<$CR$>$ is pressed.  In {\bf native}
mode, character by
character echoing by the remote computer is selected.  Only in
native mode can the screen editors ({\it e.g.\ } EDT) be used;  however,
for long-distance connections, message mode should be used to
improve speed and minimise costs.


\end{document}
