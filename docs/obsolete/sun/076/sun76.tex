\documentstyle{article}
\pagestyle{myheadings}

%------------------------------------------------------------------------------
% STAR STAR STAR
\newcommand{\stardoccategory}  {Starlink User Note}
\newcommand{\stardocinitials}  {SUN}
\newcommand{\stardocnumber}    {76.5}
\newcommand{\stardocauthors}   {James Tappin\footnote{School of Physics and
Space Research, University of Birmingham}}
% STAR STAR STAR
\newcommand{\stardocdate}      {18 December 1992}
\newcommand{\stardoctitle}     {HXIS --- Hard X-ray Imaging Spectrometer}
\newcommand{\hxdispvers}{8.0}
%------------------------------------------------------------------------------

% STAR STAR STAR
\newcommand{\stardocname}{\stardocinitials /\stardocnumber}
% STAR STAR STAR
\markright{\stardocname}
\setlength{\textwidth}{160mm}
\setlength{\textheight}{240mm}
\setlength{\topmargin}{-5mm}
\setlength{\oddsidemargin}{0mm}
\setlength{\evensidemargin}{0mm}
\setlength{\parindent}{0mm}
\setlength{\parskip}{\medskipamount}
\setlength{\unitlength}{1mm}

\begin{document}
\thispagestyle{empty}
% STAR STAR STAR
SCIENCE \& ENGINEERING RESEARCH COUNCIL \hfill \stardocname\\
RUTHERFORD APPLETON LABORATORY\\ {\large\bf Starlink Project\\}
{\large\bf \stardoccategory\ \stardocnumber}
% STAR STAR STAR
\begin{flushright}
\stardocauthors\\
\stardocdate
\end{flushright}
\vspace{-4mm}
\rule{\textwidth}{0.5mm}
\vspace{5mm}
\begin{center}
{\Large\bf \stardoctitle}
\end{center}
\begin{center}
{\Large\em Portable Version}
\end{center}

\vspace{2mm}

\setlength{\parskip}{0mm}
\tableofcontents
\setlength{\parskip}{\medskipamount}
\markright{\stardocname}

\newpage

\section{Introduction}

The programs which make up the HXIS software package allow the display
of images produced by the Hard X-ray Imaging Spectrometer (HXIS)
carried on the Solar Maximum Mission (SMM). There are also facilities
for displaying time series and for editing bad data. Programs are also
provided for accessing and archiving the data files to and from tapes.

\subsection{Programs}

The following programs make up the software package.

\begin{description}
\item[\verb!hxdisp!: ] (Page \pageref{HXDISP}) This is the primary
image display and manipulation program. Also contains options for
time-series plots and some rudimentary analysis of spectral data.

\item[\verb!rdhim!: ]  (Page \pageref{RDHIM}) Reads data from a HIMSEL
tape and stores data on disk.

\item[\verb!haxim!: ] (Page \pageref{HAXIM}) Converts Production Data
Tapes to HIMSEL image tapes or to disk images.

\item[\verb!hxlib!: ] (Page \pageref{HXLIB}) Allows access to the HIMSEL tape
catalogue which is stored on disk. If it is desired to update the
database then it must be run from an account with write access to the
database.

\item[\verb!wrhim!: ] (Page \pageref{WRHIM}) Writes disk images to a
HIMSEL tape.

\item[\verb!hxjoin!: ] (Page \pageref{HXJOIN}) Concatenates two disk
image files.
\end{description}

More complete descriptions of the various programs are given in the
remainder of this document.

\subsection{Accessing HXIS Software}

\subsubsection{Commands}

To use the software you will need to add the line \verb!source
${HXIS_DIR}/HXset! to your \verb!.login! file (UNIX) or
\verb!@HXIS_DIR:HXSET! to your \verb!LOGIN.COM! file (VMS) (or run it
explicitly each time you want to use the software).

This will set up the necessary PATH additions and environment variables
so that the commands can be run by typing their names (e.g.
\verb!hxdisp! or
\verb!rdhim!). A suitable place to put it would probably be straight after
the \verb!source /star/etc/login! (\verb!@SSCLOGIN!) line.

\subsubsection{Image files}

In all the programs if you do not give a file type {\tt .dta} is
assumed.  An image file with no extension is not accessible (one ending
in a dot is accessible).

The significance of the individual locations in each image is described
in Appendix~\ref{DATA} on page~\pageref{DATA}.

\subsubsection{Tapes}

In those commands which always use tapes (or tape-like devices, e.g. an
EXABYTE cartridge) the command sequence will prompt the user for the
desired drive(s).

On VMS you must allocate and mount the drive before running the program
and dismount and deallocate afterwards. On Unix systems you do not
mount drives, you just put the tape on or in the drive and use it.

\subsubsection{File Transport}

Files can be moved between different systems, but care needs to be taken
with the file attributes when moving from Unix to VMS.

A utility \verb!FIXATR! obtained from the National Center for
Supercomputing Applications, at the University of Illinois has been
provided for making the necessary conversions on a VMS system.
The standard setup procedure initializes this and provides two sets of
options one for the \verb!ipd*! files (\verb!fixipd!) and one for the
\verb!.dta! files (\verb!fixdta!).

The following points should also be observed:
\begin{itemize}
\item Files moved using \verb!ftp! should always be moved in binary (image)
mode (and then \verb!FIXATR!ed for Unix to VMS).
\item Files read from VMS backup tapes using \verb!vmsbackup! should
always be retrieved using the binary \verb!-b! option.
\item IPD files cannot safely be moved from VMS to Unix with the
\verb!vms2tar! utility as this will mangle the records {\bf
(IRRETRIEVABLY)}, \verb!tar2vms!
followed by \verb!FIXATR! should be O.K. for moving files to VMS.
\end{itemize}


\section{HXDISP}
\label{HXDISP}

\subsection{General}

This program allows the display and editing of images and time-series
from HXIS, and also a certain amount of elementary spectral analysis.

To make the images available to the program they should be read from
tape using either \verb!rdhim!~(p.~\pageref{RDHIM}) (HIMSEL tapes) or
\verb!haxim!~(p.~\pageref{HAXIM}) (production data tapes).

To execute this program just type \verb!hxdisp!, and then reply to the
prompts.

Firstly the filename of the desired data is requested, the name given
must include the directory where the file is stored (unless it is in
the current directory).

Note that there is a limit of 69
\label{longname} characters for the length of the filename (excluding any
implicit {\tt .dta} filetype, but including any explicit filetype).  If
the filename is longer than this you must define a symbolic link (or a
logical name on VMS) so that a name of less than 69 characters can be
given: e.g. if the file is (UNIX)
\begin{verbatim}
/work1/solar/data_directory/images/hxis-smm/new_versions/24_september_1980.dta
\end{verbatim}
then you could use:
\begin{verbatim}
ln -s /work1/solar/data_directory/images/hxis-smm/new_versions ./imdir
\end{verbatim}
and then give the filename as \verb!imdir/24_september_1980! or for VMS:
\begin{verbatim}
DISK$WORK1:[SOLAR.DATA_DIRECTORY.IMAGES.HXIS-SMM.NEW_VERSIONS]24_SEPTEMBER_1980.DTA
\end{verbatim}
then you could use:
\begin{verbatim}
DEFINE IMDIR DISK$WORK1:[SOLAR.DATA_DIRECTORY.IMAGES.HXIS-SMM.NEW_VERSIONS]
\end{verbatim}
and then give the filename as \verb!IMDIR:24_SEPTEMBER_1980!

There is a limit of 2200 images per file, but as it is not possible to
create a disk image file of more than 2200 images by the standard
methods, there should not be any difficulties at this stage.

If the number type of the file is different from that of the machine
there will be a delay while the file is translated and re-written.

When the image file has been opened the program informs the user of the
number of images in the file and waits for commands with the main
command level prompt:
\begin{verbatim}
Command :_
\end{verbatim}

A list of commands and a brief description of their function is given
below, with the commands grouped by the type of operation performed. A
more detailed description of the command is given on the page indicated
by each command.  The interaction of the various setting commands with
the display commands is summarized in Table~\ref{interact}.

\begin{table}
\caption{\label{interact} The interaction of the setting and device control
commands with the data display commands in {\tt hxdisp}.} {\small
\begin{center}
\begin{tabular}{|l|*{7}{c}|*{4}{c}|*{2}{c}|} \hline
& DUM & VIE & CON & DEC & ITC & ALL & MOV & TSE & SIX & RAT & PXT & SPE
& SAV \\
\hline
AVERA & & & & & & & & Y & Y & Y & Y & & \\
BACKG & & & & P & P & & & Y & Y & Y & Y & Y & \\
BANDS & O & Y & O & O & O & & Y & Y & & O & Y & & Y \\
COARS & & Y & Y & Y & Y & Y & Y & O & O & O & O & O & \\
DEADT & & & & & & & & Y & Y & & Y & Y & \\
FINE & & Y & Y & Y & Y & Y & Y & O & O & O & O & O & \\
IMAGE & Y & Y & Y & Y & Y & Y & Y & Y & Y & Y & Y & Y & Y \\
MASTE & Y & Y & Y & Y & Y & Y & Y & Y & Y & Y & Y & Y & Y \\
SETPI & & & & & & & & Y & Y & Y & & Y & \\
SLAVE & Y & Y & Y & Y & Y & Y & Y & Y & Y & Y & Y & Y & Y \\
\hline
ERRBA & & & & & & & & Y & Y & Y & Y & & \\
FSD & & & & & & & & & & & Y & & \\
LOGAR & & & & & & & & Y & Y & Y & Y & & \\
MINRA & & & & & & & & Y & Y & Y & Y & & \\
SCALE & & Y & & & & Y & Y & & & & & & \\
SETLE & Y & & & & & & & & & & & & \\
THRES & & & & Y & Y & & & & & & & & \\
\hline
AUTOP & Y & Y & Y & Y & Y & Y & Y & Y & Y & Y & Y & & \\
COLOU & & Y & & & O & Y & Y & & & & & & \\
DELAY & & & & & & & Y & & & & & & \\
DEVIC & Y & Y & Y & Y & Y & Y & Y & Y & Y & Y & Y & & \\
DXDY & & Y & & & & & Y & & & & & & \\
FONT & Y & Y & Y & Y & Y & Y & Y & Y & Y & Y & P & & \\
INTER & Y & & & & & & & Y & & Y & & & \\
LAYOU & & & & & & & & & Y & & & & \\
PAPER & Y & Y & Y & Y & Y & Y & Y & Y & Y & Y & Y & & \\
SIZE & & Y & Y & Y & Y & & Y & Y & & Y & & & \\
WRPIX & & & & & & & & Y & Y & Y & & & \\
\hline
\end{tabular}
\end{center}
Notes:
\begin{enumerate}
\item For items marked Y the option affects the results of the display command
in the normal way.
\item For items marked O the effect is dependent on the setting of other
options (e.g. SETPI overrides the FINE/COARSE setting for TSER), or may
be overridden by options within the command.
\item For items marked P the option has only a partial effect:
\begin{itemize}
\item For PXTSER, FONT affects the header only.
\item For DECON and ITCON, the levels set by BACK affect the lowest contour
levels and the default base level for greyscales, irrespective of the
switch setting.
\end{itemize}
\item The values for ITCON refer to contoured, greyscale and coloured output,
the behaviour of dump output is as for DUMP.
\item FTSER, FRATIO and FSIXTSER are similar to TSER etc. except that the only
``plot'' command with any effect is ERRBAR.
\end{enumerate}}
\end{table}

The first five letters of each command are significant, but any unique
abbreviation of two or more characters may be used, the essential part
of each command is underlined in the main descriptions of the command.
For example for the DECONVOLVE command:
\begin{quote}
{\tt DEC, DECON} and {\tt DECONXXXXXXX} are all acceptable, but\\ {\tt
DE} and {\tt DECXXXX} are not.
\end{quote}

Many commands prompt the user for necessary information if it has not
been supplied in an argument to the command. The following features
apply to most prompts.
\begin{enumerate}
\item If a real number is requested, an integer is acceptable.
\item Any improper input will result in
the request being re-issued.
\item Where a range is requested but that range could
be of length 1 (e.g. a single image or pixel) then only a single number
need be entered.
\item If no values are entered at a prompt (i.e. carriage return)
then the existing values are retained.
\item All character responses apart from file-names
are case-insensitive.
\item Entering control-D will normally return you to the \verb!hxdisp!
command level. There are a few exceptions where it will return you to
the beginning of the prompt sequence (in which case a second control-D
returns you to the command level).
\end{enumerate}

When arguments are used to pass information to commands {\tt YES} is
always a synonym for {\tt ON} and {\tt NO} for {\tt OFF}. Any other
synonyms will be listed with the appropriate commands.

\subsection{Display Commands}

The commands listed in this section are those which actually output
graphical display of image data, or write data to disk files.

\begin{tabbing}
{\bf SHDETRUNCATIONXXX} \= p.~9999 \= Show truncation corrections.\kill
{\bf ALLBANDS} \> p.~\pageref{al} \> Display images for all bands
together.\\
{\bf CONTOUR} \> p.~\pageref{con} \> Contour raw image.\\
{\bf DECONVOLVE} \> p.~\pageref{dec} \> Contour deconvolved image
(simple default options only).\\
{\bf DUMP} \> p.~\pageref{du} \> Print Image.\\
{\bf ITCON} \> p.~\pageref{it} \> Display deconvolved image (many options).\\
{\bf FRATIO} \> p.~\pageref{fr} \> Ratio time series to file.\\
{\bf FSIXTSER} \> p.~\pageref{fsi} \> 6 band time series to file.\\
{\bf FTSER} \> p.~\pageref{ft} \> Write time series to file.\\
{\bf MOVIE} \> p.~\pageref{mo} \> ``Movie'' of images.\\
{\bf PXTSER} \> p.~\pageref{px} \> Time series for separate pixels.\\
{\bf RATIO} \> p.~\pageref{ra} \> Plot time series of band ratios.\\
{\bf SAVE} \> p.~\pageref{sa} \> Save image to disk.\\
{\bf SIXTSER} \> p.~\pageref{six} \> Time series of all six bands.\\
{\bf SPECTRA} \> p.~\pageref{sp} \> Simple spectral analysis.\\
{\bf TSER} \> p.~\pageref{ts} \> Plot time series.\\
{\bf VIEW} \> p.~\pageref{vi} \> Display image on screen.\\
\end{tabbing}

\subsection{Selection Commands}

These commands select which part of the data will be displayed by the
display commands.

\begin{tabbing}
{\bf SHDETRUNCATIONXXX} \= p.~9999 \= Show truncation corrections.\kill
{\bf AVERAGE} \> p.~\pageref{av} \> Set averaging time.\\
{\bf BACKGROUND} \> p.~\pageref{bac} \> Set background subtraction.\\
{\bf BANDS} \> p.~\pageref{ban} \> Select band range.\\
{\bf COARSE} \> p.~\pageref{coa} \> Select coarse field.\\
{\bf DEADTIME} \> p.~\pageref{dea} \> Set deadtime correction.\\
{\bf FINE} \> p.~\pageref{fin} \> Select fine field.\\
{\bf IMAGE} \> p.~\pageref{im} \> Select image range.\\
{\bf MASTER} \> p.~\pageref{mas} \> Select ``Master'' images.\\
{\bf SELECT} \> p.~\pageref{sel} \> Select pixels above level.\\
{\bf SETLEVEL} \> p.~\pageref{setl} \> Select selection level.\\
{\bf SETPIXEL} \> p.~\pageref{setp} \> Select pixels.\\
{\bf SLAVE} \> p.~\pageref{sl} \> Select ``Slave'' images.\\
{\bf TIME} \> p.~\pageref{ti} \> Select image range by times.\\
\end{tabbing}

\subsection{Data Control Commands}

These commands control how the data will be displayed (e.g. logarithms,
error bars, scaling etc.).

\begin{tabbing}
{\bf SHDETRUNCATIONXXX} \= p.~9999 \= Show truncation corrections.\kill
{\bf ERRBAR} \> p.~\pageref{er} \> Select error bars.\\
{\bf FSD} \> p.~\pageref{fsd} \> Set pixel plot scaling.\\
{\bf INTER} \> p.~\pageref{int} \> Select interactive graphics.\\
{\bf LOGARITHMIC} \> p.~\pageref{lo} \> Select logarithmic plots.\\
{\bf MINRATE} \> p.~\pageref{mi} \> Lower cut-off in logarithmic
plots.\\
{\bf SCALE} \> p.~\pageref{sc} \> Set image saturation level.\\
{\bf THRESHOLD} \> p.~\pageref{th} \> Contouring threshold.\\
\end{tabbing}

\subsection{Device Control Commands}

These commands control how the data will be displayed on the device
(e.g. size, position etc.).

\begin{tabbing}
{\bf SHDETRUNCATIONXXX} \= p.~9999 \= Show truncation corrections.\kill
{\bf AUTOPLOT} \> p.~\pageref{au} \> Select automatic plot spooling\\
{\bf COLOUR} \> p.~\pageref{col} \> Select colour table.\\
{\bf DELAY} \> p.~\pageref{del} \> Set delay time.\\
{\bf DEVICE} \> p.~\pageref{dev} \> Select graphics device.\\
{\bf DXDY} \> p.~\pageref{dx} \> Set plot offset.\\
{\bf FONT} \> p.~\pageref{fo} \> Set graphics font.\\
{\bf LAYOUT} \> p.~\pageref{la} \> Arrangement of 6-band plots.\\
{\bf PAPER} \> p.~\pageref{pa} \> Paper size for plots.\\
{\bf SIZE} \> p.~\pageref{siz} \> Set plot size.\\
{\bf WRPIXEL} \> p.~\pageref{wr} \> Set time series output file.\\
\end{tabbing}

\subsection{Data Editing Commands}

These commands actually alter the image file in some way, or else
display how it has been altered. All have a permanent effect on the
disc file with the image, however not all are irreversible.

\begin{tabbing}
{\bf SHDETRUNCATIONXXX} \= p.~9999 \= Show truncation corrections.\kill
{\bf FLAG} \> p.~\pageref{fl} \> Flag bad images.\\
{\bf LFLAG} \> p.~\pageref{lf} \> List flagged images.\\
{\bf PIXEL} \> p.~\pageref{pi} \> Set pixel values.\\
{\bf SUBTRACT} \> p.~\pageref{sub} \> Subtract background image.\\
{\bf UNFLAG} \> p.~\pageref{un} \> Restore flagged images.\\
\end{tabbing}

\subsection{Program Control Commands}

Exiting, changing input file etc.

\begin{tabbing}
{\bf SHDETRUNCATIONXXX} \= p.~9999 \= Show truncation corrections.\kill
{\bf CTRL-D} \> p.~\pageref{eof} \> Return to command level (UNIX).\\
{\bf EXIT} \> p.~\pageref{ex} \> Exit from program.\\
{\bf FILE} \> p.~\pageref{fil} \> New image file.\\
{\bf CTRL-Z} \> p.~\pageref{eof} \> Return to command level (VMS). \\
\> p.~\pageref{dcl} \>  Suspend \verb!hxdisp! and start a
new shell (UNIX).\\
{\bf \$} \> p.~\pageref{dcl} \> Execute a system command.\\
\end{tabbing}

\subsection{Miscellaneous Commands}

Give information about the image etc.

\begin{tabbing}
{\bf SHDETRUNCATIONXXX} \= p.~9999 \= Show truncation corrections.\kill
{\bf CHECK} \> p.~\pageref{ch} \> Check for bad images.\\
{\bf DOY} \> p.~\pageref{do} \> Convert date format.\\
{\bf INFO} \> p.~\pageref{inf} \> Image and option setting
information.\\
{\bf HELP} \> p.~\pageref{he} \> HXDISP help.\\
{\bf HMS} \> p.~\pageref{hm} \> Convert time format.\\
{\bf LPIXELS} \> p.~\pageref{lp} \> List selected pixels.\\
{\bf MAXIMUM} \> p.~\pageref{max} \> Find maximum in images.\\
{\bf POINTING} \> p.~\pageref{po} \> List pointing information.\\
{\bf SURVEY} \> p.~\pageref{sur} \> Image times and durations.\\
{\bf VERIFY} \> p.~\pageref{ve} \> Check image numbers.\\
\end{tabbing}

\subsection{Output of Hardcopy}

Plot files for hardcopies are stored in the current directory. When the
plot file is closed the filename will be printed on your terminal.  If
autoplot mode is selected you will then be prompted for a command for
the disposal of the file, this command will then be spawned.  If you do
not spool the plot and delete the file then on a VMS system, the next
file will have the same name with a version number one higher; on a
Unix system the name will be suffixed with a $.<n>$ (e.g.
\verb!GKS_72.PS.2!).

When the \TeX\ devices are used the filename will be the same and the
file should be renamed and included into a \TeX\ or \LaTeX\ file with a
\verb!\special! command.

The commands {\tt VIEW, ALLBANDS, MOVIE} and the {\tt F} option of {\tt
ITCON} produce colour plots which require that the device has a minimum
of 32 available colour indices. If one of these commands is used when
the selected device does not have sufficient indices, then after
opening the device; \verb!hxdisp! will close it, delete any plot file
produced, print an error message and return to the command level.

\subsection{Batch Mode}

Although \verb!hxdisp! was designed with interactive operation in mind,
the current version may be operated reasonably safely with data read
from a file.

To run \verb!hxdisp! in batch mode create a file containing the
commands for
\verb!hxdisp! e.g. (UNIX version)

\begin{verbatim}
/work2/d234_1341
COARSE
IMAGE 30,70
AVERAGE 30
MASTER ON
DEVICE POSTSCRIPT_L
AUTOPLOT OFF
PXTSER
BAND 4,6
IMAGE ALL
SETPIXEL
23,25
42,44
-1
ERRBAR ON
DEVICE POSTSCRIPT_P
TSER
EXIT
\end{verbatim}

If this file is \verb!hxdisp.inp! then giving the command
\begin{verbatim}
hxdisp < hxdisp.inp &
\end{verbatim}
will run the program in the background.

For VMS add the line \verb!$ HXDISP! at the top of the file and use
\verb!submit!.

The following commands are not available in batch mode:
\begin{quote} \tt
VIEW, MOVIE, ALLBANDS, COLOUR, INTERACTIVE, VERIFY {\rm and} DELAY
\end{quote}
If they are specified they will be ignored.

It is recommended that in batch mode arguments should be used to pass
information to commands where possible this reduces the risk of chaos
resulting from a missing value (this will usually take the form of a
series of invalid value messages ending with an exit on end of file).
For the same reason it is a good idea to use the {\tt DEVICE} command
to select plot devices in batch mode.

On VMS systems the \verb!$! command will be confused with end-of-file
in batch mode (unless you use the \verb!DECK! and \verb!EOD! commands).

\subsection{Command Descriptions}

The following section contains the full descriptions of the
\verb!hxdisp! commands, listed in alphabetical order. The underlined
portion of each command name is the minimum abbreviation allowed.

\begin{description}
\item[\underline{AL}LBANDS: ] \label{al}
Places an image of the current field in each of the 6 energy bands on
the screen.  The images are accumulated over the selected images. The
output of this command is affected by the {\tt SCALE} command.

This command needs 255 available colour indices to work correctly.
Available in interactive mode only.

\item[\underline{AU}TOPLOT: ] \label{au}
Select whether to spool plot files to the appropriate queue
automatically.

The default state is ON. When automatic plotting is turned off then a
message is output, including the name of the plot file.

Setting may be specified by the arguments {\tt ON} and {\tt OFF}.

\item[\underline{AV}ERAGE: ] \label{av}
Set time for averaging in time series. The averaging time is an elapsed
time, which will be greater than or equal to the accumulation time.

Setting may be specified by the arguments {\tt ON} and {\tt OFF} which
switch the option but do not change the averaging time (if {\tt ON} is
given without a value having been set a warning will be given and no
averaging will occur). Specifying a real number as argument switches
averaging on and sets the averaging time to that value.

Hard, default off.

\item[\underline{BAC}KGROUND: ] \label{bac}
Background removal selection. If selected the user is shown the default
levels, and asked whether to use these or to set new levels.  If levels
have already been set then these are shown as well as the defaults and
the user may choose to reset, set new values or keep the current ones.

Setting may be specified by the arguments {\tt ON} and {\tt OFF} which
switch the option without changing the levels. If {\tt NEW} or {\tt
SET} is specified then the option is switched on and new levels are
requested.  If {\tt RESET} is specified then the option is switched on
and the default levels are restored.

Hard, default off.

\item[\underline{BAN}DS: ] \label{ban}
Specify the range of energy bands over which images should be
accumulated. For example, specifying 1,3 will cause all images to be
summed over bands 1 to 3. There are a total of 6 energy bands.

The band(s) may be given as an argument to the command as 1 or 2
integers.

This is a hard option, default band 1.

\item[\underline{CH}ECK: ] \label{ch}
This command is used to search for bad images which you may want to
flag. First use the {\tt IMAGE} command to set the desired range of
images. You should also set the desired band range with the {\tt BANDS}
command.

The {\tt CHECK} command requests whether to send its output to the line
printer or the terminal (the latter is not recommended for large
numbers of images). It then produces the following outputs on the
selected device :
\begin{quote}\tt
(IMAGE NO., MAXIMUM COUNTS, PIXEL LOCATION)
\end{quote}
By scanning the output one can easily check for any data glitches since
this would produce a maximum count in a pixel location which differed
greatly from the pixel location of the true maximum. The use of this
command in conjunction with {\tt TSER} in interactive mode provides a
relatively easy way of finding and eliminating bad images.

The user is prompted for the destination of the output: either the
terminal or a printer.

The device may be specified as an argument to the command:
\begin{quote}
{\tt T} or {\tt TERM} for the terminal\\ {\tt P} or {\tt PRINT} for the
line printer.
\end{quote}

\item[\underline{COA}RSE: ] \label{coa}
Selects the coarse field of view image data.

If the current field is fine and the pixel selection is for the entire
fine field, then the pixel selection will be changed to the coarse
field, if however any other pixel selections have been made with {\tt
SETPIXEL} then no changes will be made.

\item[\underline{COL}OUR: ] \label{col}
Specify desired colour table. Enter a number between 1 and 10.

At present only tables 1 through 5 exist. If a non-existent table is
specified then the default of 1 is used. If the program fails to read
any colour tables then a standard ``greyscale'' table is used.

The table may be specified as an integer argument to the command.

Available in interactive mode only.

\item[\underline{CON}TOUR: ] \label{con}
Make a contour map of the selected field.

The contour levels may be selected either as a percentage of the image
peak (the initial default) or by counts/pixel.  If a file called {\tt
CONTOUR.DCN} exists then this will be read and offered as a default set
of contour levels. If such a file is not found or is rejected then an
internal set of defaults is offered.  If these too are rejected then
the program will prompt for levels, up to a maximum of 50, these will
be saved as {\tt CONTOUR.DCN}.

Header information on the plot gives the image range, start time and
integration time for the image. The energy bands and range in keV are
given along with the scaling of the image. Contour levels used are
listed in counts and as a percentage of the peak.

The routine will plot either the currently selected bands or produce
separate plots of each of the six bands.  This may be selected by an
argument to the command with:
\begin{quote}
{\tt ALL} for all bands (6 separate plots) or\\ {\tt C} for accumulated
over the current bands
\end{quote}
Otherwise the program will prompt for a choice with current as the
default.

\item[\underline{DEA}DTIME: ] \label{dea}
Correct for instrument dead-time on the basis of the DFM fields. These
fields monitor a radioactive source so the count rate should be
independent of levels of emission on the Sun.

The command requests a range of images over which to accumulate the
DFM's to create the base level, this should be a time when the count
rates were low.  This applies to {\tt TSER} and {\tt SPECTRA}.

Setting may be specified by the arguments {\tt ON} and {\tt OFF}.  If
{\tt ON} is specified this implies that the image range for forming the
base level is unchanged from the last time the option was set. If 1 or
2 integers are given as an argument then the base range is reset to the
range given and the option is switched on.

Hard, default off.

\item[\underline{DEC}ONVOLVE: ] \label{dec}
This command produces a contour plot of the data deconvolved with the
point spread function of HXIS using a basic set of default options. If
more advanced options are needed use {\tt ITCON} (see p.~\pageref{it}).

The details of the deconvolution algorithm are described in
Appendix~\ref{ALGO} on p.~\pageref{ALGO}.

The deconvolved image is then contoured with a standard set of contour
levels.  The levels are:
\begin{quote}
100, 99, 98, 90, 80, 70, 55, 40, 26, 14, 7 \& 3\% of the image peak
\end{quote}

The only input needed from the user is the selection of a plot device,
if no default has been selected.

During the deconvolution process the program will type a summary of
each step of the deconvolution at the users terminal.

Header information on the plot states that the image has been
deconvolved and gives the image range, start time and integration time
for the image. The energy bands and range in keV are given along with
the scaling of the image. Contour levels used are listed in counts and
as a percentage of the peak.

The routine will plot either the currently selected bands or produce
separate plots of each of the six bands.  This may be selected by an
argument to the command with:
\begin{quote}
{\tt ALL} for all bands (6 separate plots) or\\ {\tt C} for accumulated
over the current bands
\end{quote}
Otherwise the program will prompt for a choice with current as the
default.

\item[\underline{DEL}AY: ] \label{del}
Specifies how long, in seconds, the program is to wait between
displaying successive images in {\tt MOVIE}.

The delay may be specified as a real argument to the command or by {\tt
OFF} which will set the delay to zero.

Available in interactive mode only.

Hard, default is 5.0.

\item[\underline{DEV}ICE: ] \label{dev}
Select the device for PGPLOT output. This may be any valid PGPLOT
device given either in mnemonic or numeric form.

When no device has been selected by this command then all graphical
commands prompt for a device with a default of the last device used. If
a device is set with {\tt DEVICE} then that device is used and no
prompt is issued.

To return to prompting by individual commands reply {\tt NONE} to the
prompt for device.

To check that the device selected is valid this routine opens the
device and then closes it again (for hardcopy devices the file so
produced is deleted).

The device may be specified as an argument to the command. {\tt OFF} or
{\tt NONE} will return to prompting by plotting commands and {\tt ON}
will select the current default device for automatic use.

If ``{\tt ?}'' is specified then a list of available devices will be
produced, the prompt will then be repeated.

\item[\underline{DO}Y: ] \label{do}
This command will either return a day of the year when given a date, or
a date given a day of the year (default).

Unlike the equivalent command in the database management program
(\verb!hxlib!)  this command uses dates in the British format (i.e.
day/month/year).

If the year is omitted 1980 is assumed (all HXIS data were from 1980).

If 2 integers are given as an argument to the command then they are
assumed to be a day of year and year. If 3 are given then they are
taken as a date (d/m/y order). Therefore, if the values are passed as
an argument the year must be given.

\item[\underline{DU}MP: ] \label{du}
This produces a plot of the accumulated counts in all the HXIS pixels.
It is possible to produce dumps accumulated over the currently selected
energy bands or to output separate plots plots for the first two bands,
or for all six bands.

Pixels selected by {\tt SETLEVEL} are highlighted by ringing. If the
appropriate options have been set by {\tt INTER} and the output is
directed to an interactive device then pixels may be selected
interactively.

The bands to be plotted may be selected by an argument to the command
with:
\begin{quote}
{\tt ALL} for all bands (6 separate plots),\\ {\tt C} for accumulated
over the current bands or\\ a number from 1 to 6 for bands 1--$<${\tt
N}$>$ (separate plots for each band).
\end{quote}

\item[\underline{DX}DY: ] \label{dx}
Specify the $x$ and $y$ screen coordinates in millimetres at which to
place the image. The coordinates correspond to the upper left hand
corner of the image. The upper left corner of the portion of the screen
remaining after titling is $(0., 0.)$ and distances are measured down
and to the right.

Unless the size of the image has been reduced by the {\tt SIZE} command
specifying non-zero values for dx and/or dy may result in part of the
image being clipped at the edge of the screen.

The offsets may be specified as arguments to the command (1 or 2
reals), or by {\tt OFF} or {\tt NONE}.

\item[\underline{ER}RBAR: ] \label{er}
Select error bars on time series plots.

Setting may be specified by the arguments {\tt ON} and {\tt OFF}.

Hard, default off.

\item[\underline{EX}IT: ] \label{ex}
Exits \verb!hxdisp! program.  If End of File is encountered on two
consecutive attempts to read a command the program will exit. This is a
safety feature to prevent the possibility of an infinite loop in batch
mode.

\item[\underline{FIL}E: ] \label{fil}
This command allows you to read in another data file. The new file is
written into the same data array so the previous data are lost.

The filename may be given as an argument to the command.

When a new file is selected ``soft'' options are reset to default
values, these are: {\tt IMAGE}, {\tt SETPIXEL} and {\tt DEADTIME}
(accumulation range). All other options are hard and their settings are
retained.

Note that the filename may not exceed 69 characters in length, the way
to access a long filename is given on page~\pageref{longname}.

\item[\underline{FIN}E: ] \label{fin}
Selects the fine field of view image data.  This is the default state,
but the setting is hard (retained on change of data file).

If the current field is coarse and the pixel selection is for the
entire coarse field, then the pixel selection will be changed to the
fine field, if however any other pixel selections have been made with
{\tt SETPIXEL} then no changes will be made.

\item[\underline{FL}AG: ] \label{fl}
Specify the image number you wish to flag as a bad image.  Once an
image has been flagged other commands will not act on that image. This
is a permanent change; i.e.  when you terminate execution and return
some other time all previously flagged images will still be flagged.

Up to 20 images may be listed for flagging as an argument to the
command (however the total length of the command may not exceed 72
characters).

\item[\underline{FO}NT: ] \label{fo}
Select font for graphics. The choice is the four PGPLOT fonts: N for
normal, R for roman, I for italic and S for script.  Normal font is
much quicker than the other three.

If given as an argument then the font may be specified as a single
letter, the full name or the font number.

\item[\underline{FR}ATIO: ] \label{fr}
As {\tt RATIO} (p.~\pageref{ra}) but output to a file only.  For
details of the filename see the description of {\tt WRPIXEL} on
page~\pageref{wr}.

\item[\underline{FSD}: ] \label{fsd}
Set scaling for the plots produced by {\tt PXTSER}.  If 0 is used then
all plots are individually scaled to the maximum rate in that pixel, if
a value is given than that overrides the automatic scaling. The value
given is in counts per second.  (The peak count rate can be obtained by
using the {\tt MAXIMUM} command.)

If specified as an argument the scaling is given as a real number or
{\tt PEAK} or {\tt OFF} which both reset the scaling to 0.0.

\item[\underline{FSI}XTSER: ] \label{fsi}
As {\tt SIXTSER} (p.~\pageref{six}) but no plot is produced, only a
disk file, irrespective of the state of {\tt WRPIXEL}. For details of
the filename see the description of {\tt WRPIXEL} on page~\pageref{wr}.

\item[\underline{FT}SER: ] \label{ft}
As {\tt TSER} (p.~\pageref{ts}) but no plot is produced, only a disk
file, irrespective of the state of {\tt WRPIXEL}. For details of the
filename see the description of {\tt WRPIXEL} on page~\pageref{wr}.

\item[\underline{HE}LP: ] \label{he}
Invoke the HELP facility. {\tt HELP} on its own produces a list of
topics.  {\tt HELP} $<${\em topic name}$>$ gives help on the requested
topic.  After the information has been given HELP prompts for subtopics
or for further topics, to return to the \verb!hxdisp! command level
enter one or more returns or control-D.

\item[\underline{HM}S: ] \label{hm}
This command will convert a time expressed in hours to its
corresponding hours, minutes, and seconds. You will be prompted for the
hour to be converted.

The hour may be given as a real argument to the command.

\item[\underline{IM}AGE: ] \label{im}
Set desired first and last image numbers. This restricts access to
those images whose image numbers lie within this range. Almost all
other commands use the image range set by {\tt IMAGE}.

If either the first or last image selected has been flagged, then the
program will ask for that value to be re-entered ({\tt TSER} and
possibly other commands can become confused about times when the first
or last image is flagged).

The image values may be set as an argument to the command. If 2 values
are given then this is the range, if only 1 is specified then the range
is the single image and if `{\tt ALL}' is specified then all images in
the file are used (excluding initial and terminal flagged images).

This is a soft option (i.e. the value is reset when a new file is
selected).  The default is the entire file (less any flagged images at
the ends).

\item[\underline{INF}O: ] \label{inf}
Gives information on the current file (name, image range etc) and on
the major option settings (Dead time correction, field etc).

\item[\underline{INT}ER: ] \label{int}
Select interactive graphics. If a suitable device is selected for
output, then the user may select pixels interactively in {\tt DUMP} and
image ranges in {\tt TSER}.

Setting may be specified by the arguments {\tt ON} and {\tt OFF}.
Available in interactive mode only.

Hard, default off.

\item[\underline{IT}CON: ] \label{it}
Display a deconvolved image.  This command allows a much wider range of
options than {\tt DECON} (p.~\pageref{dec}).  The deconvolution
algorithm is the same as for {\tt DECON,} see Appendix~\ref{ALGO} on
p.\pageref{ALGO} for details.

As with {\tt CONTOUR} and {\tt DECON} this command can produce either a
plot of the current bands or a set of plots of the six separate bands.
This may be selected by argument as:
\begin{quote}
{\tt ALL} for all bands (6 separate plots) or\\ {\tt C} for accumulated
over the current bands
\end{quote}
Otherwise the program will prompt for a choice with current as the
default.

The routine will then prompt for the following information:
\begin{enumerate}
\item Contoured output (reply {\tt C}), greyscales (reply {\tt G}), colour
plots (reply {\tt F} [for Fancy]) or none of these (reply {\tt N}). The
initial default is contours, but the previous selection is subsequently
remembered.
\item Whether to produce DUMP format output (see the {\tt DUMP} command for a
description (p.~\pageref{du})). If none was selected at the previous
level then the default is to produce DUMP output, else the default is no
DUMP output.
\item Number of iterations for deconvolution.
\begin{itemize}
\item Zero (the default) means continue until $\chi^2 < $ degrees of freedom.
\item A positive value means continue for that number of iterations unless the
convergence criterion above is satisfied first.
\item A negative value means continue for {\em exactly} the specified number of
iterations.
\end{itemize}
\end{enumerate}

After the image has been deconvolved then the levels for the display
will be requested. For contours you will be asked:
\begin{enumerate}
\item Define the levels by counts or percent.
\item If a previous run has defined a set of contour levels then these will
be displayed and you will be asked whether to use these.
\item If there is no previous set of levels (or if they were rejected) then
the default levels are displayed and you have the choice of using these
or defining your own set. If you define a new set of levels they will
be saved in a file named {\tt LEVELS.DCN} on the selected output
directory, superceding any previous sets.
\end{enumerate}
For greyscales or colour plots you will be asked:
\begin{enumerate}
\item Define the levels by counts or percent.
\item Map the greyscale linearly or logarithmically.
\item ``saturated'' level (Default peak of the data).
\item ``empty'' level (Default: Background level $ +~T\sigma $, where $T$ is
the level selected by the {\tt THRESHOLD} command).
\end{enumerate}

If no plot device has been selected you will also be prompted for it.

Note that not all devices produce satisfactory greyscales.
\begin{itemize}
\item Devices with 255 available colour indices produce good greyscales by
using 128 of the colour indices as different shades of grey.
``Saturated'' is white.  These are the \underline{only} devices able to
produce colour plots.

\item The Canon and PostScript laser printers produce good greyscales by
creating half-tone images with dots, but this is very slow (it used
about $1^{\rm m}20^{\rm s}$ of cpu on a microVAX 3400 for 1 image and
about 1000--1500 blocks to the laser printer). ``Saturated'' is black.

\item The Printronix uses the same
algorithm as the laser printer to produce greyscales but the larger
pixel size means that these are hardly worth the paper they are printed
on.  ``Saturated'' is black.

\item Line graphics terminals (e.g. CIFER T5) will make a stab at
greyscales with a quasi-random array of dots, but the results aren't
terribly impressive and it is inordinately slow.

\end{itemize}

The plot header contains extensive information about the image and
deconvolution. For contours there is also material on the number of
counts within each level.  For greyscales and colour plots this is
replaced by a colour bar.

\item[\underline{LA}YOUT: ] \label{la}
Control the layout of plots produced by {\tt SIXTSER}.

The numbers of columns and rows may be given as arguments, or they will
be prompted for. Both values must be given, any pair of integers with a
product of 6 is acceptable. If the argument ``{\tt DEF}'' is given the
default state of 3~columns and 2~rows is restored.

\item[\underline{LF}LAG: ] \label{lf}
List the images which have been flagged in the current dataset.

\item[\underline{LO}GARITHMIC: ] \label{lo}
Select or deselect logarithmic plotting of time-series by {\tt TSER}
etc.

The setting may be specified by the arguments {\tt ON} or {\tt OFF}.

Hard, default off.

\item[\underline{LP}IXELS: ] \label{lp}
List the currently selected pixels and the fields in which each range
starts.

\item[\underline{MAS}TER: ] \label{mas}
Select images from the master microprocessor only. This command is
applicable only at times when both microprocessors were working.

This command and {\tt SLAVE} are mutually exclusive. In addition if the
command {\tt MASTER~OFF} is issued when {\tt SLAVE} is selected, then
normal operation is resumed, this is also the case if {\tt MASTER} with
no argument is followed by carriage return.

Setting may be specified by the arguments {\tt ON} and {\tt OFF}.

Hard, default off.

\item[\underline{MAX}IMUM: ] \label{max}
Locates the maximum pixel value of an image. The search will be made
among the images set by the {\tt IMAGE} command, which have been
accumulated for a specified time in seconds over the band range
determined by the {\tt BANDS} command.  The count rate in counts per
second is also given.

The accumulation may be given as a real argument to the command.  If no
argument is given the command will prompt you for an accumulation time
in seconds.

This is most useful if you wish to manually scale the images to the
maximum value of the data set in {\tt VIEW} (etc.) or in {\tt PXTSER}.

\item[\underline{MI}NRATE: ] \label{mi}
Set lower cut-off for logarithmic plots. When a value is set then this
is treated as the lower limit of the data, thus cutting off very low
level material. The setting has no effect on linear plots. If the peak
rate on a plot is below the cut-off then the plot will not be generated
(in the same way as when there are no non-zero data in a plot).

The cut-off may be switched on and off by the arguments {\tt ON} and
{\tt OFF}, which do not change the level (if no level has been selected
then {\tt ON} will be faulted). Specifying a value will set the cut-off
to that value and switch it on.

\item[\underline{MO}VIE: ] \label{mo}
This command is used if you wish to display images on the screen in
movie form.

The output from this command is controlled by the following commands:
\begin{quote}
{\tt IMAGE}: To set first and last image.\\ {\tt FINE/COARSE}: To
choose fine or coarse image.\\ {\tt BANDS}: To set low and high band.\\
{\tt SIZE}: To set the size of the plot.\\ {\tt DELAY}: To specify wait
time between display of successive images.\\ {\tt SCALE}: To set the
saturation level on the plots.\\ {\tt COLOUR}: To choose the colour
table.
\end{quote}

The only input the command requests is an accumulation time in seconds.
This may be given as a real argument to the command.  Note that the
time is an elapsed time, therefore if there was only one processor
running, then the accumulation time appearing on the plot will be
approximately half this.  If zero is specified then each image is
displayed separately.

This command needs 255 available colour indices to work correctly.
Available in interactive mode only.

\item[\underline{PA}PER: ] \label{pa}
Set the size of the paper to be used in {\tt TSER, PXTSER} etc.  The
sizes are the X and Y dimensions of the plotting page in millimetres.
This should only affect the output to hard copy devices (a feature of
the plotting library not of
\verb!hxdisp!). The default behaviour is to fill the entire available sheet of the
selected device.

The main purpose of this command is to allow the selection of a
suitable size of plot to be included in a \TeX\ or \LaTeX\ file, as in
these cases the standard paper is usually too big.

The sizes may be specified as arguments to the command (1 or 2 reals, 1
gives a square page), or by {\tt OFF} which resets the default values
(i.e. full page).

This command is entirely independent of the {\tt SIZE} command.

\item[\underline{PI}XEL: ] \label{pi}
This command allows the user to change the value of individual pixels
in a HXIS image. The user is prompted for the following information:
\begin{quote}\tt
ENTER IMAGE NUMBER: {\rm Hitting RETURN will bring you back to the
command level of \verb!hxdisp!.}\\ LOW,HIGH BAND: {\rm The band or
bands in which you want to change pixel values.}\\ PIXEL NO., NEW
VALUE: {\rm Hit RETURN when you have finished inputting new values. See
HXIS image map for proper pixel numbers.}
\end{quote}

The argument to the command may be used in several ways:
\begin{itemize}
\item If a single integer is given as an argument then that is used as the image
number, and the remaining values are requested, no further image
numbers will be requested.
\item If 2 or 3 values are given then the bands are also set from the argument
and only the pixel numbers and values are requested.
\item If 5 values are given then all values are taken from the argument and no
prompts are issued.
\item Specifying 4 values is an error.
\item Anything after the fifth value will be ignored.
\end{itemize}

\item[\underline{PO}INTING: ] \label{po}
Make a list of the HXIS pointing information contained in the image
file, for the selected image range. The information is given as Pitch
in arcsec South of Sun centre, Yaw in arcsec East of Sun centre, and
Roll in degrees West of North.

The user is prompted for the destination of the output: either the
terminal or a printer.

The device may be specified as an argument to the command:
\begin{quote}
{\tt T} or {\tt TERM} for the terminal\\ {\tt P} or {\tt PRINT} for the
line printer.
\end{quote}

\item[\underline{PX}TSER: ] \label{px}
Plot a {\em montage} of time-series for individual pixels. If {\tt
COARSE} has been selected then the entire coarse field is plotted. If
{\tt FINE} has been selected then the user is prompted to choose a
quadrant (the choices are {\tt FC} (default), {\tt TL, TR, BL {\rm or}
BR}, for field centre, top left etc. which refer to the position in the
field as conventionally plotted --- i.e. with North to the right).

Most of the commands which set options for {\tt TSER} apply here but
{\tt SIZE} and {\tt INTER} are ignored and {\tt FONT} only affects the
title of the plot.

By default the plots are scaled individually and the peak count rate
(in counts/second) is given in the top left corner of each plot. The
time axes are identical and the scale is given in the first plot of the
last row. If {\tt FSD} has been called then all the plots are scaled as
if that were their peak value and the scaling is given on the $y$-axis
of the first plot of the last row.

The field to be plotted may be selected by an argument to the command,
the acceptable values are {\tt FC, TL, TR, BL, BR} or {\tt CO} (for
coarse).  Any other value will be ignored, a prompt will be given for a
quadrant if the currently selected field is fine.

\item[\underline{RA}TIO: ] \label{ra}
Produce a plot of the ratio of 2 bands. If the selected bands are
different then these are offered as a default, if not or if that is
rejected, then the user is prompted for the bands to form the ratio.
(Note that the first band entered [the ``low'' band] will be the
denominator.)

This command is affected by the same options as {\tt TSER} except that
{\tt DEADTIME} is ignored since that correction is the same for all
bands and will therefore cancel in a ratio.

\item[\underline{SA}VE: ] \label{sa}
This command writes an image data array onto disk in a separate file.
The data in this file are then available for use with other
user-written programs.

The saved data may be written with formatted i/o in the form of 768
integers written one to a line in I7 format, or a a single record with
unformatted i/o.  As some of the values in the header part contain
REAL*4 values, the data must be read into an INTEGER*2 array of 768
elements with suitable equivalence statements to be used.

Details of the header information are the same as for the image format
described in Appendix~\ref{DATA} on page~\pageref{DATA}.  In addition
the spare fields 163--169 are used as follows:
\begin{tabbing}
888--888XX \= ITIMXX \= \kill 163 \> IBL \> Low energy band. \\ 164 \>
IBH \> High energy band. \\ 165 \> MSTR \> Accumulated over: All (0),
Master only (1), slave only (-1).\\ 166-169 \> ITIM \> Start time UT
Day, hour, minute, second.
\end{tabbing}
Elements 193 to 768 then contain the accumulated image over the current
bands and image range.

The saved files have the following naming convention:
\begin{quote}\tt
dddhhmmss.Flh {\rm for fine field of view.}\\ dddhhmmss.Clh {\rm for
coarse field of view.}
\end{quote}
where {\tt dddhhmmss} = start time of image, day of year and U.T.\\
{\tt F}/{\tt C} = indicates whether fine or coarse field of view is
currently selected. (Data written are identical.)\\ {\tt lh} =
indicates the image is accumulated from band {\tt l} to {\tt h}\\ e.g.
{\tt 176121833.C13} means the saved image is of the coarse field of
view accumulated over bands 1 to 3 and the image start time is day 176
at 12:18:33~UT.

The arguments ``{\tt F}'' or ``{\tt FORM}'' select formatted writing,
and ``{\tt U}'' or ``{\tt UNFO}'' select unformatted. If no argument is
given then a prompt is issued, with an initial default of formatted.

\item[\underline{SC}ALE: ] \label{sc}
Specify maximum number of counts per pixel for saturation in {\tt VIEW,
MOVIE} etc. By entering a number greater than zero (default is 0.0) for
the maximum counts, all displayed images will be scaled to that maximum
value. If zero is specified then the scaling is done to the actual
maximum.

The scaling may be specified as a real number argument to the command,
or by specifying {\tt OFF} or {\tt PEAK} which resets the scaling to
zero.

\item[\underline{SEL}ECT:] \label{sel}
Selects pixels in the fine and coarse fields according to the level
specified by {\tt SETLEVEL}. This selection is the same as that in {\tt
DUMP} when used on the current energy bands.

\item[\underline{SETL}EVEL: ] \label{setl}
This command is used to set a threshold value, which is input as a
percentage, which is then used by the pixel selection facility of the
{\tt DUMP} or {\tt SELECT} command.  The pixels selected are those
whose counts, expressed as a percentage of the maximum counts recorded
in a pixel, are greater than or equal to the set percentage. The
calculation is performed for both the coarse and fine fields of view,
and the pixels selected are indicated by having a box placed around
them.

The percentage level is firstly requested (a real number), enter 0.0 to
cancel pixel selection in {\tt DUMP} and {\tt SELECT}.  If a number
other than 0.0 has been selected, the program then asks if, when
outputting more than one energy range, the pixels selected in the
lowest energy band (band 1) should be used throughout, or the pixels
re-selected in each band. Finally, the program asks if the pixels
selected are to be saved in a file for later use.

If the pixels are to be saved then when {\tt DUMP} or {\tt SELECT} is
run the selected pixels will be written to a file called {\tt
Pdddhhffffeeeelh.PXL} where: {\tt ddd} is the day of the year, {\tt hh}
is the hour of the day at the first image, {\tt ffff} and {\tt eeee}
are the first and last images and {\tt l} and {\tt h} are the low and
high energy bands (or both 7 for band 1 pixels to be used for all
bands).

If pixels are to be selected in this manner, the {\tt SETLEVEL} command
must be issued before the {\tt DUMP} or {\tt SELECT} command.

The percentage level may be given as a real number argument to the
command.  The writing of the selected pixels to disk is controllable by
the arguments `{\tt DISK}' and `{\tt NODISK}'. The pixel selection for
multiple band plots may be controlled by the arguments `{\tt ALL}'
(same pixels for all bands) and `{\tt RESE}' (reselect for each band).
If several settings are made then the percentage must be the first item
in the argument.

\item[\underline{SETP}IXEL: ] \label{setp}
Select the pixels to be used for {\tt TSER} etc.

If there is a file of pixels corresponding to the current image and
band selections (see SETLEVEL (p.\pageref{setl}) for details of naming
etc.) or if a file name has been given as a argument (see below) then
the user is given the option of reading and editing it.

If not then the pixels selected in the last run of {\tt DUMP} or of
{\tt SETPIXEL} are chosen. If there are no previously selected pixels
then the current field (as selected by {\tt FINE/COARSE}) are the
default. The user is given the choice of using the current set of
pixels or entering a new set as a series of ranges.

The pixel numbers are entered as pairs of numbers representing start
and stop pixels. If a single number is given then it is assumed that
start and stop are the same. Input is terminated by a blank line or a
negative value.

A file other than the default may be read if the file name is given as
an argument to the command, enclosed in double quotes. e.g. to read
{\tt ALL.PXL} use
\begin{verbatim}
SETPI "ALL"
\end{verbatim}

If the argument {\tt NODISK} is given then no search will be made for
an input pixel file. {\tt DISK} causes a search to be made firstly for
any specified filename, and then for the default, if the file is found
then it is automatically read.

If the argument {\tt SAVE} is given then the pixels selected will be
saved to either the specified file, or to the default file. {\tt
NOSAVE} directs that the pixels will not be saved, this is the default
action. If {\tt SAVEONLY} is specified then the currently selected
pixels will be saved.

The pixel selection setting is soft.

\item[\underline{SIX}TSER: ] \label{six}
This command displays a time series plot for each energy band from the
selected pixels (see {\tt SETPIXEL} (p.~\pageref{setp})).

The time-series is displayed in counts/second. The following commands
control the output:
\begin{quote}
{\tt IMAGE}: To set first and last image.\\ {\tt SETPIXEL:} Select
pixels to plot. (May also be set using {\tt DUMP}).\\ {\tt
FINE/COARSE}: To choose fine or coarse image, if pixels have not been
explicitly selected.\\ {\tt BACKGROUND:} Make corrections of instrument
background noise.\\ {\tt DEADTIME:} Make corrections for instrument
dead-time.\\ {\tt AVERAGE:} Average the plot over a specified time
(real, not accumulation).\\ {\tt ERRBARS:} Put error bars on the
plot.\\ {\tt WRPIXEL:} Write the time-series to a disk file.\\ {\tt
LOGARITHMIC:} Plot time-series on logarithmic scale.\\ {\tt MINRATE:}
Lower cut-off for logarithmic plots.\\ {\tt LAYOUT:} Arrangement of the
plots on the page.
\end{quote}

If no pixels have been selected by {\tt SETPIXEL} or by {\tt
SETLEVEL/DUMP} then the entire current field is used.

\item[\underline{SIZ}E: ] \label{siz}
Allows adjustment of the size of the image displayed by all display
commands apart from {\tt ALLBANDS} and {\tt PXTSER}.  The size is
determined in terms of the largest scale which will fit, without
distortion of the scales, into the selected device after the title has
been written.  The user specifies this as two real numbers less than or
equal to 1.0, X then Y. If only one number is given then it is assumed
that X and Y scalings are identical.

The size may be set by specifying 1 or 2 real numbers as an argument to
the command. If {\tt MAX} or {\tt FULL} is specified that is equivalent
to setting both X and Y sizes to 1.0.

Hard, default is 1.0.

\item[\underline{SL}AVE: ] \label{sl}
Select images from the slave microprocessor only. This command is
applicable only at times when both microprocessors were working.

This command and {\tt MASTER} are mutually exclusive.  In addition if
the command {\tt SLAVE~OFF} is issued when {\tt MASTER} is selected,
then normal operation is resumed, this is also the case if {\tt SLAVE}
with no argument is followed by carriage return.

Setting may be specified by the arguments {\tt ON} and {\tt OFF}.

Hard, default off.

\item[\underline{SP}ECTRA: ] \label{sp}
Performs a basic spectral analysis on the selected images and pixels.
The raw or corrected counts for each band are given along with the
ratios between all the possible combinations of bands. The counts are
converted to count rates (in counts sec$^{-1}$ keV$^{-1}$ cm$^{-2}$)
for a plasma temperature of 20 million K. Temperatures and emission
measures are computed from adjacent pairs of bands and these are
printed (this will give an indication as to whether the count
conversions are realistic!).

N.B. It is the responsibility of the user to ensure that there are
enough counts in the bands before making any use of the temperatures
and emission measures.

The command takes an argument to specify the destination of the output.
The recognized values are:
\begin{quote}
{\tt T} or {\tt TERM} for the terminal (or batch job log),\\ {\tt P} or
{\tt PRINT} for the line printer.
\end{quote}
If no argument is given the command will prompt for the destination.

\item[\underline{SUB}TRACT: ] \label{sub}
This command allows the user to form a background image and a series of
accumulated images from which the background can be subtracted. The
user is prompted for the following information:
\begin{quote}\tt
ENTER LOW AND HIGH BAND\\ FORM A BACKGROUND IMAGE?\\ FIRST AND LAST
IMAGE TO ACCUMULATE FOR BACKGROUND\\ FIRST AND LAST IMAGE TO ACCUMULATE
{\rm - This prompt will repeat until you enter 0 or CR.}
\end{quote}

The new images will be stored back in the original data set and the
user will be told exactly where they are located. Any other commands
can then be used on these images.

\item[\underline{SUR}VEY: ] \label{sur}
Give a summary of the properties of each image in the currently
selected range (Start time, duration, duty cycle). If an image has been
flagged this will be indicated.

The user is prompted for the destination of the output: either the
terminal or a printer.

The device may be specified as an argument to the command:
\begin{quote}
{\tt T} or {\tt TERM} for the terminal\\ {\tt P} or {\tt PRINT} for the
line printer.
\end{quote}

\item[\underline{TH}RESHOLD: ] \label{th}
Set threshold for lowest contour (and minimum level in greyscales and
colour plots) for deconvolved images. The level is set as a number of
standard deviations above the background level (as set by {\tt BACK}),
thus if the threshold has been set as $T\sigma$ and the background
level is $C$~counts/pixel then the lowest contour that can be plotted
is: $ C + T \sqrt C $.  The default is 3.0$\sigma$.

The value may be given as a real argument to the command or ``{\tt
DEF}'' which resets the default level of 3.0. If no argument is given
then a prompt is issued for a level.

\item[\underline{TI}ME: ] \label{ti}
Set image range by time. This command selects a subrange of the current
image range, starting at a given U.T. and continuing for a specified
period.

The selected start image will be the first image {\em ending after} the
specified U.T. the end image will be the last image {\em starting
before} that U.T. plus the requested duration.

Details of the first and last images will be displayed, along with the
actual duration (in minutes) from the start of the first image to the
end of the last.

An argument may be given specifying the initial time as 4 integers, and
optionally the accumulation as a real value. If the accumulation is
specified then all 4 integers (day, hour, minute and second)
\underline{must} be specified, otherwise zero for the seconds (and
minutes if appropriate) may be omitted.

\item[\underline{TS}ER: ] \label{ts}
This command displays a time series plot of data from the selected
pixels (see {\tt SETPIXEL} (p.~\pageref{setp})).

The time-series is displayed in counts/second. The following commands
control the output:
\begin{quote}
{\tt IMAGE}: To set first and last image.\\ {\tt BANDS}: To set low and
high band.\\ {\tt SIZE}: To set the size of the plot.\\ {\tt SETPIXEL:}
Select pixels to plot. (May also be set using {\tt DUMP}).\\ {\tt
FINE/COARSE}: To choose fine or coarse image, if pixels have not been
explicitly selected.\\ {\tt INTER:} Allow interactive selection of a
range of pixels.\\ {\tt BACKGROUND:} Make corrections for instrument
background noise.\\ {\tt DEADTIME:} Make corrections for instrument
dead-time.\\ {\tt AVERAGE:} Average the plot over a specified time
(real, not accumulation).\\ {\tt ERRBARS:} Put error bars on the
plot.\\ {\tt WRPIXEL:} Write the time-series to a disk file.\\ {\tt
LOGARITHMIC:} Plot time-series on logarithmic scale.\\ {\tt MINRATE:}
Lower cut-off for logarithmic plots.
\end{quote}

If no pixels have been selected by {\tt SETPIXEL} or by {\tt
SETLEVEL/DUMP} then the entire current field is used.

\item[\underline{UN}FLAG: ] \label{un}
Specify the image number which was incorrectly flagged. This will
restore the image to the active dataset. If $-1$ is specified then all
flagged images are restored.

Up to 20 images may be listed for unflagging as an argument to the
command (however the total length of the command may not exceed 72
characters).  If $-1$ or {\tt ALL} is specified then all images are
unflagged.

\item[\underline{VE}RIFY: ] \label{ve}
Checks that the image numbers in the selected range are correct.  If an
incorrect number is found the user is asked whether to flag the image
or to leave it.

This is most likely to be of use if a set of images has been read from
tape with several parity errors.

Available in interactive mode only.

\item[\underline{VI}EW: ] \label{vi}
Places an accumulated image on the screen.  Accumulates from first to
last image over specified range of bands (as specified by {\tt IMAGE}
and {\tt BANDS}).

This command will only work correctly when the device has 255 available
colour indices.  Available in interactive mode only.

\item[\underline{WR}PIXEL: ]  \label{wr}
Control whether time-series generated by {\tt TSER}, {\tt RATIO} and
{\tt SIXTSER} are written to disk as well as being plotted.

The file name will be:
\begin{quote}
{\tt Ndddhhffffeeeelh.NUM} for a count rate file from {\tt TSER},\\
{\tt Sdddhhffffeeee.NUM} for a six-band count rate file from {\tt
SIXTSER} and \\ {\tt Rdddhhffffeeeelh.RAT} for a ratio file from {\tt
RATIO},\\ where\\ {\tt ddd}= start day of data (I3)\\ {\tt hh}= start
hour of data (I2)\\ {\tt ffff, eeee} = first and last images saved
(both I4)\\ {\tt l, h} = settings of low and high energy bands (both
I1)
\end{quote}

The first line of the (formatted) file will contain the start day and
hour, and the number of records (N).  The next N lines will hold the N
values of image start time (in minutes) and the corresponding
counts/second (6 values for a file from {\tt SIXTSER} otherwise 1
value).  The next line will hold the number of lines of pixel
information (J).  The next J lines will have that information as START,
STOP.  If error bars have been requested with the {\tt ERRBAR} command
then there will be a further N lines in the file with error limits
(differences for {\tt TSER} or {\tt SIXTSER} and ratios for {\tt
RATIO}).  Setting may be specified by the arguments {\tt ON} and {\tt
OFF}.


\item[CTRL-D: CTRL-Z:] \label{eof}

For VMS systems read Z for D in all of what follows.

At any prompt for input depressing the CTRL and D keys simultaneously
will either return you to the \verb!hxdisp! command level or to the
beginning of the sequence of prompts (in this case a second CTRL-D will
return you to the command level).

If End of File is encountered on two consecutive attempts to read at
the command level the program will exit. This is a safety feature to
prevent the possibility of an infinite loop in batch mode. (Thus a
maximum of four successive CTRL-D's will exit from \verb!hxdisp!
anywhere.)

\item[\$: CTRL-Z:] \label{dcl}

If you enter \verb!$! on its own at the command level prompt an
interactive session will be started as a subprocess. For VMS this is
straightforward. For Unix the session will be started up as a subshell
running the C-shell (not \verb!tcsh!), but because of the way in which
it is run your login aliases etc. are not set. On VMS type
\verb!LOGOUT! to return, on Unix type \verb!exit!.

If you enter \verb!$ <command>! then a subprocess will be spawned which
runs the specified command and then returns. On VMS again this is
straightforward. On Unix owing to the vagaries of the way that the
system works the command will be run under the Bourne shell (so beware).

If at any time on a Unix system supporting process suspension (i.e.
C-shell and \verb!tcsh!) you enter CTRL-Z, then the process running
\verb!hxdisp!  will be suspended, and a new process spawned. This
allows you to run normal Unix commands. To return to \verb!hxdisp! use
the \verb!fg!  command to recall the \verb!hxdisp! process to the
foreground.
\end{description}

\section{Data access programs}

\subsection{HXLIB}
\label{HXLIB}

This program is used to update, search, and display information from
the HXIS flare database file which lists all the events currently
stored on the HIMSEL tapes. To execute simply type \verb!hxlib!. The
following commands are available:

\subsubsection{Search and display commands}

\begin{description}
\item[LIST: ]    To list at the user's terminal the HIMSEL tapes and file
numbers for one day, (and optionally hour). If the hour is omitted a
list of all the data for that day will be produced, if neither day nor
hour is given that the entire database will be listed.

\item[PRINT: ]    Output the entire file on the line printer.

\item[SURVEY: ]   To survey a HIMSEL tape. Output is at the user's terminal.
Gives file number, start time, stop time and the total number of
images.

\item[DAY: ] Convert a day of the year to a date or {\em vice versa}.

\item[HELP: ] Get help on \verb!hxlib! commands.

\item[EXIT: ]     Terminate execution of program. (Two consecutive EOF's at
the command level have the same effect, as a safety valve for an
omitted EXIT in a batch job).
\end{description}

\subsubsection{Database maintenance commands}

These commands are used to modify the database and require that the
user has write access to it.  If you do not have write access a message
will be given on entry to \verb!hxlib! and also if you attempt to use
any of these commands.

\begin{description}
\item[ADD: ]  To update the file if a new HIMSEL tape has been created.
Obtains HIMSEL tape number to be added from the header if it is
present, otherwise this is prompted for.

\item[DELETE: ] Deletes all records for a given HIMSEL tape from the file.
Prompts for HIMSEL tape number to be deleted.

\item[SORT: ]  To sort the entire file in order of increasing day of
year. Note: This is very slow when the file contains many tapes,
therefore if possible it should be run as a batch job.
\end{description}

To access a database other than the standard one use the command
sequence:
\begin{verbatim}
setenv HXFLARE <filename>
hxlib
\end{verbatim}
For VMS use \verb!DEFINE! instead of \verb!setenv!.

WARNING: {\em The file numbers given in the database are those
contained within the files} (rather than a count taken from the tape
during the preparation of the database) these should correspond to the
values which should be given to RDHIM but in a few cases the numbers
are known to be incorrect. If an attempt to read a file from a HIMSEL
tape gives the wrong data then use the SURVEY option to list the tape
and then count the files.

NOTE: If your database has been transferred from a different type of
machine, use
\begin{verbatim}
hxtr <f> <t> <oldfile> <newfile>
\end{verbatim}
to convert it,
where \verb!<f>! and \verb!<t>! are the from and to machine types and
are each one of [\verb!v s d!] for VMS, Sun and DECstation respectively.

\subsection{RDHIM}
\label{RDHIM}

This program reads HXIS image files from the HIMSEL tapes and stores
the images on disk. To execute just type RDHIM.

The user will be prompted for the following information:
\begin{quote} \tt
Tape drive with HIMSEL tape? :\_\\
Re-read on read error? (Y/N) :\_
\end{quote}
and then for each file to be read
\begin{quote} \tt
File number? (EOF to end) :\_\\
First and last image :\_ \\
Output filename :\_
\end{quote}

File numbers and total number of images are given in the HIMSEL tape
library (accessed by the \verb!hxlib! command). The program currently
allows a maximum of 2200 images to be stored on disk, (the largest
number of images in the light-time of a single orbit was 2196).

If parity errors are encountered on the tape and re-reading has been
requested then a message will be written and the tape will be back
skipped and a maximum of three further attempts will be made to read
the record. If the record cannot be read then the image is flagged (see
\verb!hxdisp! for details of flagged images, p.~\pageref{fl}) and a
further message is written. If the damaged record corresponds to the
part of the image containing the image number then the number will be
forced to be 1 more than that of the previous image. If re-reading was
not requested then the program will abort the reading of a file on
parity errors.

When several files are requested, the file numbers need not be in
sequence as the program is able to skip backwards on a tape (but
repeated long moves on a tape will take time).

\subsection{HAXIM}
\label{HAXIM}

Converts the data on Production Data Tapes to image files on HIMSEL
image tapes or to images on disk.  Can also read production data tape
files from disk.

Note that to ensure compatibility between production files read onto a
VMS system using BACKUP and then transferred to a UNIX machine via NFS
or with the binary mode of \verb!ftp! and files read using
\verb!vmsbackup! it is necessary that the \verb!-b! option be used when
reading the backup tape. Also if transferring production files via
\verb!ftp!, you must use \verb!binary! or \verb!image! mode.

In HAXIM the user is prompted with the following:

\begin{description}
\item[{\tt INPUT file (TT or $<$CR$>$ for terminal) }]{\tt :\_}\\
Give the name of a disc file which contains a list of answers to the
prompts that follow (the name given must be preceded by the name of the
disc on which the command file is held, unless it is on the disk
specified as the scratch disk on entering HAXIM) or hit $<$CR$>$ for
interactive input.

\item[{\tt Tape drive with HIMSEL tape or D for disk }]{\tt :\_}\\
Answer tape drive name (e.g. \verb!/dev/rmt0h! or \verb!mub0! for
HIMSEL tape output, or {\tt d} for output to a disk file.

\item[{\tt Drive holding PD tape? (or D for disk) }]{\tt :\_}\\
Answer name of drive with the data tape (e.g. \verb!/dev/rmt1h! or {\tt
D} if the raw data files are on disk (as when read from {\tt BACKUP}
format tapes).

\item[{\tt Skip on read errors? (Y/N) }]{\tt :\_}\\
If response is {\tt Y} then if HAXIM fails to read a block from tape it
will continue to the next block and write a warning message to the
terminal or job log. Otherwise the program will abort. (This request is
omitted if input is from disk.)

\item[{\tt New output tape (Y/N) :\_}] (Only if output is to tape)
\begin{description}
\item Answer ``{\tt N}'' if the HIMSEL output tape  already has image files written
on it. The program will then attempt to find the HIMSEL tape number
from the HIMSEL header. The program will then skip forward through the
tape until it finds a double tape mark. At this point it will print the
number of files it has found (excluding the header) and request if this
is the right place.
\begin{description}
\item If it is then answer ``{\tt Y}'' then the program will continues.
\item If this seems to be the wrong place answer ``{\tt N}''. You will then be
given the option of looking for another double tape mark (answer ``{\tt
F}'') or quitting (answer ``{\tt Q}'').
\end{description}
{\em N.B. the verification of location is always at the terminal.}
\item Answer {\tt Y} if this is a new HIMSEL tape. The program will then
request the number to be assigned to the new HIMSEL tape and create a
new HIMSEL header.
\end{description}
These requests are omitted if output is to disk.

\item[{\tt Input file numbers }]{\tt :\_}\\
Give the file numbers on the PDT which you want converted to image
files (maximum of 25).  The list entered is sorted into ascending order
so that the tape is not skipped backwards. This is omitted if input is
from disk.
\end{description}

Then for each file:
\begin{description}
\item[{\tt Name of input file? }]{\tt :\_}\\
(Only if input is from disk.) The name of the input disk file,
including disk and directory if not the same as the scratch directory.

\item[{\tt Start time (H,M,S (Integers)) }]{\tt :\_}\\
This prompt will appear when the file is located, this message is
preceded by a message indicating the file number (or name for disk
files) and the start and end times of the file (as obtained from the
file header). The response is the U.T. of the start of the data you
wish to read. Note if the time is not found in the specified file then
no images will be read.

\item[{\tt Duration (Minutes (I3)) }]{\tt :\_}\\
The number of minutes of data which will be read. (Note: The data will
be read from the last image starting before the given time up to the
first one ending after it).  If end of file is encountered before the
requested amount of data has been read, then the read will stop (to
read a set of images bridging files read from the two files and then
use \verb!HXJOIN! (p~\pageref{HXJOIN}) to concatenate the resulting disk
files.
\end{description}

If output is to disk then following the reading of the tape file the
user is asked:
\begin{description}
\item[{\tt Output file name }]{\tt :\_}\\
This is the name of the ``permanent'' image file.
\end{description}

\subsection{WRHIM}
\label{WRHIM}

This program will write one or more image files from disk onto a HIMSEL
tape.

The program prompts for the drive with the HIMSEL tape to be written.
When the tape has been successfully mounted and set to beginning of
tape the program the requests whether this is a new tape or not.
\begin{description}
\item If the tape is new answer ``{\tt Y}''. The program will request the
number to be assigned to the HIMSEL tape. This will be written to the
header and writing will commence immediately after.
\item Otherwise answer ``{\tt N}''. The program will then try to find the  tape
number from the header. It will then skip through the tape until it
finds a double tape mark. When this is found it will print the number
of data files that it has skipped and request whether that is the right
place to begin writing.
\begin{description}
\item If it is then answer ``{\tt Y}''. The program will then proceed to
request files to write.
\item If the wrong number of files appears to have been skipped answer ``{\tt
N}''.  Then you will be given the option of searching for another
double tape mark (answer ``{\tt F}'') or quitting (answer ``{\tt Q}'')
(according to whether you think that there is an empty file on the
tape, or you have mounted the wrong tape).
\end{description}
\end{description}

After this the program simply prompts for file names until the user
enters an end-of-file (ctrl-D or ctrl-Z). When all the files are
written then the tape is rewound.

If the physical end of tape is reached while writing a file, then the
program will abandon that file, notify the user, write a logical
end-of-tape after the previous file, rewind and exit.

\subsection{HXJOIN}
\label{HXJOIN}

This program concatenates two (or more) image files on disk.

The files are requested by the program. For the first file the image
numbers found are reported. For subsequent files the image numbers are
altered to be consecutive in the resulting file and the new numbers are
reported. To terminate the list of input files enter ``{\tt **}''. End
of File (control-D or control-Z) will terminate execution and blank lines are
ignored. If more than 2200 images are read (in total over all the
files) the program will output a warning and request the output file,
to which it will write the first 2200 images read.

After the input files have been read the program then requests the name
of the output file. At the end of execution the total number of images
transferred is reported.

\subsection{FIXATR}
\label{FIXATR}

\verb!FIXATR! is a utility obtained from NCSA\footnote{National Center
for Supercomputing studies; University of Illinois, Champaign, Il, USA}
which modifies the attributes of a VMS file without altering the
contents of the file itself.

When the HXIS software has been initialized, then typing:
\begin{verbatim}
HELP @FIXATR FIXATR
\end{verbatim}
will give more information.

The following aliases have been created in the HXIS initialization:
\begin{verbatim}
fixipd :== fixatr /rfm=var:6660 /rat=none
fixdta :== fixatr /rfm=fix:7680 /rat=none
\end{verbatim}
these will fix the processed data files (input to \verb!HAXIM!) and the
HXIS image files (input to \verb!HXDISP! etc) respectively.

{\large \bf NOTE: processed data files transferred from VMS to UNIX in
ascii mode or written to a tar file using {\tt vms2tar} will be irreparably
damaged!}

\appendix
\section{Deconvolution algorithm}
\label{ALGO}
The deconvolution routine makes a guess at the sky distribution of
X-ray flux, computes the image, compares that with the observations,
computes $\chi^2$ and either accepts the image (if $\chi^2$ is less
than the degrees of freedom or the iteration limit has been reached) or
uses the discrepancies between the images to make a new guess at the
sky.

\subsection{Initial Guess}

The first guess is a uniform distribution
with the same mean count rate as the observed image.

\subsection{Computation of model image from model sky}

The model image is computed by taking the model sky and summing the
contribution of each point in the model sky to the current pixel
according to the HXIS pixel response function.

\subsection{Computation of $\chi^2$}

The basic computation of $\chi^2$ is:
$$ \chi^2 = \sum {{\rm (Observed - Model)}^2\over {\rm Model}} $$
over all the pixels for which there exist observations.

But it is recommended (e.g. Seigel --- Non-parametric statistics
for the behavioural
sciences) that when computing $\chi^2$ no more than 20\% of the categories
should have expected counts $< 5$. If each pixel is treated independently
then this condition is unlikely to be met for most normal HXIS images. Also
if any category has zero expectation and finite observation $\chi^2$ will
diverge. Therefore all pixels with expectation 1 or 0
are treated as a single category (if the expectation is still zero
it is treated as 1).

If the maximum pixel in a model image is less than
5 then the image is treated as a series of strips, with summing being
carried out in both the Y and Z directions.

\subsection{Model sky regeneration}

At each image point the ratio between the observed and model images
is computed. Then
at each point in the sky grid the ratios for the four surrounding
image points are formed into an average weighted by the pixel responses at that
location.
If there was no observation at one or more of those points then the
sum of the weights is adjusted accordingly (i.e. we assume that the value
there is unknown rather than zero) this allows proper handling
of edges and data holes.

If all four surrounding data are missing then the sky is {\em assumed}
to be zero. This is justified on the grounds that the telemetry system
preferentially omitted low regions of the image when in the fast
sampling modes.

If a region of the model sky is a true zero in one iteration then ratios are
replaced by differences in that region to prevent the model becoming ``frozen''
in that region.

The values of peak flux, $\chi^2$, degrees of freedom etc are reported
for each iteration, e.g.:
\begin{verbatim}
FIRST: 129  LAST: 141  TIME:323:15: 5: 5  ACCU TIME:  59.904 SEC
Iteration #   0  Peak rate =     40  CHI**2 = 112077.7      DF = 287  Strips = F
Iteration #   1  Peak rate =    735  CHI**2 = 642.5782      DF = 187  Strips = F
Iteration #   2  Peak rate =    897  CHI**2 = 192.1034      DF = 172  Strips = F
Iteration #   3  Peak rate =    972  CHI**2 = 117.4386      DF = 183  Strips = F
Number of iterations =            3
\end{verbatim}

If the difference between the value of $\chi^2$ and the degrees of freedom is
reduced by less than 0.2\% for two consecutive iterations then a warning is
output and the iteration is stopped.
If this happens then unless you are very sure that everything is in order it is
probably a good idea to go back and take a close look at your data for
defective images which are inconsistent with any physically realizable sky.


\section{Files written and/or read by {\tt hxdisp}}
\label{FILES}

The following files may be written or used by \verb!hxdisp!.
\begin{description}
\item[\tt CONTOUR.DCN] Contour levels for CONTOUR command.
\item[\tt LEVELS.DCN] Contour levels for ITCON command.
\item[\tt dddhhmmss.Flh] Saved image for fine field of view.
\item[\tt dddhhmmdd.Clh] Saved image for coarse field of view.
\item[\tt Pdddhhffffeeeelh.PXL] Pixel file (default name).
\item[\tt Ndddhhffffeeeelh.NUM] Time-series file for a single band.
\item[\tt Sdddhhffffeeee.NUM] Time-series file for six bands.
\item[\tt Rdddhhffffeeeelh.RAT] Time-series file for a ratio of bands.
\item[\tt CANON.DAT] Graphical output to go a CANON laser printer, or
for inclusion in \TeX\ files.
\item[\tt GKS\_72.PS] Graphical output to go to a PostScript printer.
\item[\tt ZETA.DAT] Graphical output for the pen plotter.
\item[\tt PRINTRONIX.BIT] Graphical output to go to the line printer.
\end{description}
In all the above uppercase letters represent themselves and lower case
represent values as:
\begin{description}
\item[\tt ddd] Day of year at start of accumulation. (I3.3)
\item[\tt hh] U.T. Hour at start (I2.2)
\item[\tt mm] minute at start (I2.2)
\item[\tt ss] second at start (I2.2)
\item[\tt ffff] first image number (I4.4)
\item[\tt eeee] final image number (I4.4)
\item[\tt l] low energy band (I1)
\item[\tt h] high energy band (I1)
\end{description}

\begin{table}
\begin{center}
\caption{\label{data} The format of the data in an individual HXIS image.}
{\small
\begin{tabular}{|llp{75mm}|} \hline
Element(s) & Contents &  Notes \\
\hline
1 & Platform\hfill(H/A) & A ``magic number'' to indicate the machine
type, 0 for VAX, 257 for SUN and 514 for DECstation. \\
2 & ICAL\# \hfill(H) & Version of image calibration (always 0)\\
3 & IPOS\# \hfill(H) & Version of position calibration (always 0)\\
4 & FILE\# \hfill(H) & Position on HIMSEL tape.\\
5 & Record count \hfill(H) & Cumulative image number on HIMSEL tape. Only correct on the
original GSFC HIMSEL tapes.\\
6 & Image count \hfill(H) & Self-explanatory. \\
7--11 & Creation date \hfill(H) & (HRX2) CHARACTER*10, date read from PDT by HAXIM.\\
12--14 & End time of this image. \hfill(S) & I*2,R*4 - Hour of year, Sec of hour\\
15 & Bad image flag. \hfill(A) & 0 if image is O.K., 1 if bad.\\
16 & Nminor \hfill(S/H) & Number of telemetry minor frames in image.\\
17 & Spare & Reserved for encoding detruncation of telemetry.\\
18 & Data quality \hfill(H) & 100 $\times$ quality as a percentage. (usually
= 10000)\\
19--20 & Image index \hfill(S) & R*4, sequence\#, separate series for each
microprocessor.
Increases by 2 each image for a given processor. Even values for master,
odd for slave.\\
21 & Version\# \hfill(S) & 0 means ROM. (Always 0). \\
22 & Duration of image \hfill(S) & in units of 0.128s.\\
23--24 & HV1, HV2 \hfill(S) & High voltages, about 800 means supply on.\\
25 & STAT11 \hfill (S) & Flare start bit. \\
26 & STAT10 \hfill (S) & Flare present bit. \\
27 & STAT09 \hfill (S) & Calibration mode bit. \\
28 & STAT08 \hfill (S) & Internal belt bit. \\
29 & STAT07 \hfill (S) & HV/off bit during belt.\\
30 & Flare mode \hfill(S) & 0, 1 or 2.\\
31 & coarse+fine+slits counts \hfill(S) & Modes 1 and 2 only.\\
32--71 & Housekeeping data \hfill(S) & I*2 values.\\
72--77 & EBS words. \hfill(S/H) & Indicate energy ranges of the bands, each word
(2 bytes) has 2 1-byte values which refer to the table which follows
with the first element treated as number zero.\\
78--91 & KeV table. \hfill(H) & 100 $\times$ energies in KeV.\\
92 & NO \hfill(S) & Number of overflows.\\
93--156 & Overflows \hfill(S) & RAM addresses? \\
157--158 & Pitch \hfill(S) & R*4 degrees.\\
159--160 & Yaw \hfill(S) & R*4 degrees.\\
161--162 & Roll \hfill(S) & R*4 degrees.\\
163--192 & Spare & \\
193--768 & Band 1 \hfill(S) & \\
769--1344 & Band 2 \hfill(S) & \\
1345--1920 & Band 3 \hfill(S) & \\
1921--2496 & Band 4 \hfill(S) & \\
2497--3072 & Band 5 \hfill(S) & \\
3073--3648 & Band 6 \hfill(S) & \\
3649--3662 & Rates \hfill(S) & R*4 values.\\
3663--3678 & Monitors \hfill(S) & I*2 values.\\
3679--3806 & Data transfer pulses \hfill(S) & I*2 values, Number of pulses
for each 0.128s interval of image.\\
3807--3840 & Spare & \\
\hline
\end{tabular}}
\end{center}
\end{table}

\section{Image file data format}
\label{DATA}

The data array is composed of images of 3840 INTEGER*2 values which
contain the information as listed in Table~\ref{data}.

The source of the values is indicated at the end of the contents field as:
(S) --- Value set by spacecraft, (H) --- Value set by HAXIM, (A) --- Value set
by analysis software. There are two cases given as (S/H)
these
are: the number of minor frames, which is a count generated by HAXIM from the
spacecraft data, and the EBS words, which have default values set by HAXIM and
the spacecraft setting substituted if they are found.

The ``platform'' flag indicates the format of the numbers contained in
the remainder of the image, the values are chosen to be numbers
symmetrical under byte reversal. The value is set when the image is
generated or translated. (A file does not need to have all images of
the same type but whenever \verb!hxdisp! issued to view a file it
will be converted to native type.)

Each band contains 572 pixels which are arranged as listed in Table~\ref{pxls}.
\begin{table}
\begin{center}
\caption{\label{pxls} The locations of the different HXIS fields within the
bands.}
\begin{tabular}{|ll|}
\hline
Pixel Number & Field \\ \hline
1--128 & Coarse field \\
129--432 & Fine field \\
433--472 & Y-slits \\
473--512 & Z-slits \\
513--528 & High energy monitor \\
529--544 & Drift field monitor \\
545--560 & Amplitude and alignment monitor \\
561--576 & Unused. \\ \hline
\end{tabular}
\end{center}
\end{table}

The spatial arrangement of the pixels is illustrated in Figure~1.

\end{document}

