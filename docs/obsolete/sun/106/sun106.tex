
\documentstyle{article}
\pagestyle{myheadings}

%------------------------------------------------------------------------------
\newcommand{\stardoccategory}  {Starlink User Note}
\newcommand{\stardocinitials}  {SUN}
\newcommand{\stardocnumber}    {106.1}
\newcommand{\stardocauthors}   {A R Wood}
\newcommand{\stardocdate}      {19 October 1990}
\newcommand{\stardoctitle}     {SCAR --- Beginner's Guide}
%------------------------------------------------------------------------------

\newcommand{\stardocname}{\stardocinitials /\stardocnumber}
\markright{\stardocname}
\setlength{\textwidth}{160mm}
\setlength{\textheight}{240mm}
\setlength{\topmargin}{-5mm}
\setlength{\oddsidemargin}{0mm}
\setlength{\evensidemargin}{0mm}
\setlength{\parindent}{0mm}
\setlength{\parskip}{\medskipamount}
\setlength{\unitlength}{1mm}

\begin{document}
\thispagestyle{empty}
SCIENCE \& ENGINEERING RESEARCH COUNCIL \hfill \stardocname\\
RUTHERFORD APPLETON LABORATORY\\
{\large\bf Starlink Project\\}
{\large\bf \stardoccategory\ \stardocnumber}
\begin{flushright}
\stardocauthors\\
\stardocdate
\end{flushright}
\vspace{-4mm}
\rule{\textwidth}{0.5mm}
\vspace{5mm}
\begin{center}
{\Large\bf \stardoctitle}
\end{center}
\vspace{5mm}

\section {Absolute Beginner}

This document should be read by first time users of the SCAR system. Further
, more comprehensive, information can be found in the Starlink document
SUN70 and in SCAR\_DOC\_DIR.

New versions of SCAR will appear from time to time.
All software changes are documented in SCARDOC1.TEX in SCAR\_DOC\_DIR.

\section {Introduction}

The Starlink Catalogue Access and Reporting system (SCAR) is a relational
database management system for astronomical catalogues.
Whilst it is primarily designed for the processing of astronomical catalogue
data, such as the IRAS catalogue, it can be used for processing any category of
relational data.
SCAR has the capability to extract data from the requested catalogue using input
criteria, to manipulate it using various statistical and plotting routines,
to output data from the catalogue, and to assimilate new catalogues.


\section {Getting Started}

First time users of SCAR start here!

SCAR allows you to put a catalogue into the database, to SEARCH
catalogues in that database, and generate REPORTs on what you have found.
The catalogues available in SCAR may be found using CAT\_HELP.
You can employ the SEARCH and REPORT facilities to get the information you need
much more quickly from an online catalogue than going to a library and browsing
through a book or microfiche.
Other database facilities allow the SORTing, MERGing, JOINing and DIFFerencing
of catalogues.
Facilities are also provided to plot sources in a gnomonic (tangent plane) or
Aitoff (equal area) projection, to analyse the fields of a catalogue by
scatterplot and histogram, and to calculate new fields.

It is recommended that users have a page file quota of about 20,000 in order
to run the larger SCAR programs for plotting and analysis of the data.
When SCAR and ADAM are installed at your site you should login to your own
username and type:
\begin{verbatim}
    $ ADAMSTART
    $ SCARSTART
\end{verbatim}
If you get an error message then either SCAR  and/or ADAM has not been
installed or you may not have the necessary quotas; ask your site manager.

You can run SCAR from either ICL or DCL.
For simple, quick jobs DCL is the more appropriate choice.
The whole of SCAR is loaded into the system with ICL, so that it is more
efficient when you are using many different CAR routines; this takes a
noticeable amount of time when the first CAR routine is called, so
be patient, and remember that subsequent tasks will start almost immediately.
\begin{verbatim}
    $ ICL
    ICL> SCAR
    ICL>
\end{verbatim}
If the query you are running is getting you into more and more trouble
you should exit from ICL and start again.


\section {Getting Help}
There are three main areas of help information: CAR\_HELP for information
about the CAR routines; CAT\_HELP for information about the catalogues and
SCAR\_HELP for general information.
\begin{verbatim}
    ICL> CAR_HELP
    SCAR COMMANDS Subtopic? CAR_SEARCH
\end{verbatim}
You would find information about the CAR\_SEARCH procedure.
\begin{verbatim}
    ICL> CAT_HELP
    Topic? $Summary
\end{verbatim}
The catalogues are summarised in this help library.
The acronym for the IRAS point source catalogue is IRPS.
To find out what the right ascension and declination fields are called in IRPS,
type:
\begin{verbatim}
    ICL> CAT_HELP
    Topic? IRPS
    IRPS Subtopic? FIELDS
\end{verbatim}
You would see that RA and DEC are the names of fields in the catalogue for right
ascension and declination.
In the IRAS catalogue, fluxes are given at four infrared wavelengths, so FLUX is
stored as an array.
In the CAR system, such quantities are called GENERICs and you can find out
about all the GENERICs in the IRPS by typing:
\begin{verbatim}
    Topic? IRPS
    IRPS Subtopic? GENERICS
\end{verbatim}
You will see that FLUX is one of them.
To pick out the 60 micron and 100 micron fluxes, refer to them as FLUX(3) and
FLUX(4).

In addition to these help facilities responding with a '?' to any parameter
prompt will give help information about that parameter.

If you have a problem running any of the CAR routines you should refer to the
section describing that routine in the CAR Routines section of SUN70. The
routines are alphabetically ordered and provide complete information on all
CAR routines.

\section {Some Examples}

So let us start with a demonstration.

In the directory SCAR\_DOC\_DIR there is a file SCAR\_TEST.ICL.
This is the SCAR demonstration script which contains several detailed
examples of the use of SCAR programs. You may find it useful to read the
rest of this document in conjunction with running the demonstration.
Print the SCAR\_TEST.ICL file and then run it.
Note that, by default, SCAR creates new catalogues in the current directory so
run SCAR\_TEST.ICL from an empty directory so that you can easily identify the
files produced by the system.

\begin{verbatim}
    ICL> LOAD SCAR_DOC_DIR:SCAR_SCRIPT
\end{verbatim}

The same result could have been achieved by allowing the system to prompt for
the parameter values.

\begin{verbatim}
    ICL> CAR_SEARCH
    INPUT - name of intput catalogue /'gcvs'/ > IRPS
    OUTPUT - name of output catalogue /'g1'/ > TEST1
    TITLE - Title for file or plot /'test'/ > TEST1S
    RANGE Key criterion - finish with .END.> DEC.GE.+133000.AND.DEC.LE.+143000.END.
    QUERY - Criterion - finish with .END.> RA.GE.030000.AND.RA.LE.050000.END.
    SATISFIED - Are you satisfied with the whole query, OK to continue (Y/N)? > YES

    ICL> CAR_BINARY
     etc.

\end{verbatim}

The SCAR procedures are of three types:
\begin{enumerate}
\item Procedures that perform database management functions, {\em eg.} extract
sources from the catalogue.
\item Procedures that manipulate and display catalogue data.
\item Procedures that create and convert description files, {\em eg.} to change
coordinate systems.
\end{enumerate}
Procedures can be identified on the help menu by the topic names in UPPERCASE in
the CAR\_HELP library.
You run the program by prefixing the program name with the package name,
eg.\ CAR\_SEARCH.
Detailed specifications for all these programs are given below.
\subsubsection {Database Management Functions}
You can do the following to a data catalogue (omitting the CAR\_):
\begin{description}
\begin{description}
\item [SEARCH] it to select subsets.
\item [SORT] it to reorder the catalogue.
\item [JOIN] it with another catalogue to find objects in common.
\item [DIFFERence] it with another catalogue to find objects not in common.
\item [MERGE] it with another catalogue to add them together.
\item [SPLIT] it into two catalogues.
\item [WITHIN] a polygon, select objects from it.
\item [CONVERT] it to another format, make indexes, new fields etc.
\item [EDIT] it to add and delete records and correct values.
\end{description}
\end{description}
\subsubsection {Display Catalogue Data}
In order to study the contents of a catalogue you can:
\begin{description}
\begin{description}
\item [REPORT] it to list all or some of its contents.
\item [LISTOUT] the values of a given field.
\item [PRINT] it to a default file.
\item [CALCulate] new fields for it.
\item [RECALCulate] to update existing fields.
\item [CONVERT] it to another format.
\item [WRAP] producing a printable version of a catalogue with records longer
than 132 characters.
\item [CHART] objects in it by plotting on a finding chart.
\item [AITOFF] plot all the objects on an all-sky plot.
\item [IMAGEPLOT] objects in it for COSMOS style processing.
\end{description}
\end{description}
\subsubsection {Perform Statistics}
In order to generate statistics on the contents of a catalogue you can use:
\begin{description}
\begin{description}
\item [LITTLEBIG] to find either the largest or smallest numbers in the
  catalogue.
\item [SAMPLE] to select every Nth object from the catalogue.
\item [HISTOGRAM] to plot a histogram of a given field.
\item [SCATTER] to plot two fields, and perform regression analysis.
\item [CORRELATE] to correlate (non-parametrically) two fields.
\item [LINCOR] to compute the Pearson product-moment linear correlation
  coefficient.
\end{description}
\end{description}
\subsubsection {Create and Process Description Files}
\begin{description}
\begin{description}
\item [EXTAPE] -- Examines the contents of an ASCII tape and dumps it to disk.
\item [LISTIN] -- Reads a simple VMS file into SCAR.
\item [FORM1] -- Creates a description file in screen mode.
\item [FORM2] -- Creates a description file in prompt mode.
\item [ASPIC] -- Creates a description file for an ASPIC/IAM catalogue.
\item [ASCII] -- Converts a description file for BINARY data to one for ASCII
 data.
\item [BINARY] -- Converts a description file for ASCII data to one for BINARY
 data.
\item [POLYGON] -- Creates a description file of polygon vertices.
\end{description}
\end{description}
\subsubsection {Small Tools}
\begin{description}
\begin{description}
\item [SETUP] -- Defines default values for certain options.
\item [CATSIZE] -- Finds the number of objects in a catalogue.
\item [COUNTREC] -- Counts the number of objects in a sequential file.
\item [COSMAGCAL] -- Calculates magnitudes from COSMOS magnitudes.
\item [HARDCOPY] -- Produces hardcopy of output produced by graphics programs.
\item [MOUNT] -- Allocates a deck, mounts a foreign tape and assigns it a
logical name.
\item [DISMOUNT] -- Dismounts a tape, deallocates a deck and deassigns the
logical name.
\item [DSCFHELP] -- Inserts a description file into a help library.
\item [DEBUG] -- Switches on the VMS debugger (for Programmers only).
\item [GETPAR] -- Get a parameter value when in the ICL environment.
\end{description}
\end{description}

See SUN70 for using these procedures.

\section{Creating your own Catalogue}

So far the examples have shown how to look at and manipulate existing cataloges
but what if we want to create our own catalogue. For example we may have the
NAME, RA, DEC, VALUE1 and VALUE2 of several stars in a file TEST.DAT

\begin{verbatim}
STAR1  12 30 00   +44 30 00  1223.78000  1348.76000
STAR2  12 35 00   +43 45 00  1472.16000  1302.45000
STAR3  12 00 00   +45 00 00  1524.63000  1234.34000
\end{verbatim}

SCAR requires more information before it can interpret this file correctly.
The user has to supply this information in the form of a DSCF or description
file. The description file DSCFTEST.DAT for the above catalogue was produced
by CAR\_FORM1 and looks like. (\$ is used as a continuation character.)


\begin{verbatim}
 P TITLE                                               A17    TEST      $
        Title for test                                     X
 P FILENAME                                            A17    TEST      $
                                                           X
 P MEDIUM                                              A17    DISK      $
                                                           X
 P ACCESSMODE                                          A17    SEQUENTIAL$
                                                           X
 P RECORDSIZE                               BYTE       I5        51     $
                                                           X
 P NRECORDS                                 BYTE       I         3      $
                                                           X
 F NAME                  1     5 A5       0            A5               $
        Name of star                                       X
 F RA                    8     8 A8       0 TIME       A8               $
        HH MM SS!                                          X
 F DEC                  19     9 A9       0 ANGLE      A9               $
        SDD MM SS!                                         X
 F VALUE1               30    10 F10.5    0            F10.5     0.00000$
                                                           X
 F VALUE2               42    10 F10.5    0            F10.5     0.00000$
                                                           X
 E                                                                      $
                                                           X
 C No catnotes                                                          $
                                                           X
 A No ADC notes                                                         $
                                                           X
\end{verbatim}

A copy of this description file and the associated catalogue can be found in
SCAR\_DOC\_DIR.

It is a useful exercise for users to try to produce this description file using
CAR\_FORM1.

Some explantion of this descriptor file is required. Consider the definition
of a field in the above example, denoted by an 'F' as the first character then
we have :-

\begin{verbatim}

  FIELDNAME  (or NAME)    The name of the quantity
  START                   The position of the first character
  LENGTH                  The width of the field
  FORMAT     (or FORMAT1) The FORTRAN format of the data in the field
  EXPONENT   (or SCALE)   The scale factor for the units (eg kilo, micro)
  UNIT                    The unit of the quantity
  NULLFORMAT (or FORMAT2) The FORTRAN format for externally representing
                          the field
  NULLVALUE  (or VALUE)   The value that appears in the field when there
                          is no data
  COMMENT                 The description of the quantity (optional)
  EXPRESSION              The functional definition of the quantity
                          (optional)
\end{verbatim}

You can print a more detailed explanation of how to interpret a description
file by typing:-

\begin {verbatim}

     PRINT SCAR_DAT_PATH:DSCFDSCF.DAT

\end{verbatim}

This is the system description file.
In SCAR a description
file describes how to interpret the catalogue data in the same way the
system description file describes how to interpret the description file.

In this example RA is
expressed as a character string, the units TIME and the comment HH MM SS tell
SCAR how to interpret the field. DEC is dealt with in a similar way. SUN70 gives
further details about allowed units and the possible formats for RA and DEC.

\section{Efficiency Considerations}

It is inefficient to hold large catalogues as ASCII files so SCAR has the
facility to convert such catalogues to a binary form. The IRPS is an example
of this. We can convert our test catalogue to binary using CAR\_CONVERT.


\begin{verbatim}
    ICL> CAR_CONVERT
    INPUT - name of intput catalogue /'gcvs'/ > TEST
    OUTPUT - name of output catalogue /'g1'/ > TESTB
    OPTIONS - Do you want to specify options? /'yes'/ > YES
    ASCII - YES for ascii or NO for binary output /'no'/ > NO
    INDEX - Output an Index (Y) or a Master (N) catalogue? /'no'/ > NO
    TITLE - Title for file or plot /'test'/ > TESTB

\end{verbatim}

Two new files DSCFTESTB.DAT and TESTB.DAT are produced. If we look at
DSCFTESTB.DAT the record size has been reduced from 51 to 28. Note also the
other differences between this and DSCFTEST.DAT.

Another way that SCAR optimises disc space is by using indexes. In general
SCAR will take care of the indexing for you, but you should be aware of indexes
and more advanced user will exploit the advantages of indexing.
Instead of making a copy of the complete record
of a catalogue only a pointer or index to that record is stored. Consider
sorting the catalogue TESTB


\begin{verbatim}
    ICL> CAR_SORT SIMPLE=N
    INPUT - name of intput catalogue /'gcvs'/ > TESTB
    OUTPUT - name of output catalogue /'g1'/ > TESTBSS
    KEYS - Key fields e.g: DEC or [DEC,RA] /'[dec,ra]'/ > [DEC,RA]
    ASCEND - Sort values in ascending order? e.g: Y or [Y,Y] /'[Y,Y]'/ > [N,Y]
    TAG - Yes or No (NO is usually better) /'no'/ > NO
    OPTIONS - Do you want to specify options? /'yes'/ > YES
    ASCII - YES for ascii or NO for binary output /'no'/ > NO
    INDEX - Output an Index (Y) or a Master (N) catalogue? /'no'/ > YES
    TITLE - Title for file or plot /'test'/ > TESTBS

\end{verbatim}

We have sorted on two fields descending DEC values and ascending RA values.
CAR\_REPORTing TESTBS shows the sorted catalogue. If we look at the description
file DSCFTESTBS.DAT we see that the record size has been reduced to 21 by
indexing, greater
savings are made on catalogues with more fields. Note the differences between
this description file and those already seen. SCAR is able to deal with global
indexing where records in more than one catalogue are indexed.
We can convert from one type of catalogue
to another using CAR\_CONVERT.

In general catalogues are sorted on descending DEC and ascending RA in binary
form. IRPS is a good example. Sorting catalogues in this way allows SCAR
procedures like CAR\_SEARCH and CAR\_JOIN to operate efficiently. It is a good
idea to adopt this form for your own catalogues unless you have a special
reason to do otherwise.


\section{Further Reading}

For the general SCAR user SUN70 is an essential document. This can be found in
SCAR\_DOC\_DIR which contains all documentation relating to SCAR. See the
README file in this directory for uptodate information.

\end {document}
