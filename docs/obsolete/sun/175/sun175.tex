\documentstyle[11pt]{article} 
\pagestyle{myheadings}

%------------------------------------------------------------------------------
\newcommand{\stardoccategory}  {Starlink User Note}
\newcommand{\stardocinitials}  {SUN}
\newcommand{\stardocnumber}    {175.1}
\newcommand{\stardocauthors}   {P.\,M.\,Allan}
\newcommand{\stardocdate}      {1 December 1993}
\newcommand{\stardoctitle}     {MOSAIC --- A hypertext help system}
%------------------------------------------------------------------------------

\newcommand{\stardocname}{\stardocinitials /\stardocnumber}
\renewcommand{\_}{{\tt\char'137}}     % re-centres the underscore
\markright{\stardocname}
\setlength{\textwidth}{160mm}
\setlength{\textheight}{230mm}
\setlength{\topmargin}{-2mm}
\setlength{\oddsidemargin}{0mm}
\setlength{\evensidemargin}{0mm}
\setlength{\parindent}{0mm}
\setlength{\parskip}{\medskipamount}
\setlength{\unitlength}{1mm}

%------------------------------------------------------------------------------
% Add any \newcommand or \newenvironment commands here
%------------------------------------------------------------------------------

\begin{document}
\thispagestyle{empty}
SCIENCE \& ENGINEERING RESEARCH COUNCIL \hfill \stardocname\\
RUTHERFORD APPLETON LABORATORY\\
{\large\bf Starlink Project\\}
{\large\bf \stardoccategory\ \stardocnumber}
\begin{flushright}
\stardocauthors\\
\stardocdate
\end{flushright}
\vspace{-4mm}
\rule{\textwidth}{0.5mm}
\vspace{5mm}
\begin{center}
{\Large\bf \stardoctitle}
\end{center}
\vspace{5mm}

%------------------------------------------------------------------------------
%  Add this part if you want a table of contents
%  \setlength{\parskip}{0mm}
%  \tableofcontents
%  \setlength{\parskip}{\medskipamount}
%  \markright{\stardocname}
%------------------------------------------------------------------------------

\section{Introduction}

This document provides an introduction to the use of a hypertext help
system. It only gives a brief overview since the tools are described more
fully in the documents listed in section~\ref{muds}. The tools have not been
written by Starlink, but are freely available over the Internet.

\section{What {\em is} hypertext?}

Hypertext is a system for displaying plain text, graphics, images and (where
the hardware allows it) sound. A fundamental part of hypertext is the fact that
each page has links to other pages. You read the information by following these
links. DEC's bookreader, Sun's AnswerBook and Microsoft Windows' Help are
examples of hypertext systems. It is difficult to adequately describe in words
what is essentially a graphical system. The simplest thing to do it to try it
out. It really is easy to use. 

\section{The Mosaic program}

The program that you use to read Starlink hypertext information is called
Mosaic. This is was written by Marc Andreessen at NCSA\footnote{National Center
for Supercomputing Applications}. To use it, you simply type\footnote{after
ensuring that {\tt /usr/local/bin} is in your path}: 

\begin{quote}{\tt
\% Mosaic
}
\end{quote}

This is an X~windows application, so you must run the program from a
workstation or an X~terminal. When you start Mosaic, it will display a window
on your screen and connect to an information server. This will normally be the
Starlink one, although section~\ref{service} describes how to chose a different
server. The first page that is displayed is called the home page. If you get
warning messages reported on the window where you typed {\tt Mosaic}, then see
appendix~\ref{warnings}.

Mosaic is currently not available for VMS. However, it is quite sensible to
run Mosaic on a Unix workstation and display the output on a VAXstation.

\section{How to read a hypertext document}

You browse hypertext information by reading a page and then clicking the mouse
on a highlighted link to go to the next page. The Mosaic program records links
that you have taken, so that you know when you have visited a page before. It
marks the links by underlining them with a broken line (and changing their
colour on a colour workstation), but does not prevent you from following the
same link again. It is possible to reset the list of links that you have
followed by selecting the {\bf Clear Global History} option from the {\bf
Options} pull down menu. Hypertext information does not have to be organised
hierarchically or sequentially, although it is possible to do both. A page can
contain several links to other pages and a page can be pointed to by any number
of other pages. 


\section{The World Wide Web}

You can use Mosaic simply to read files on your own computer. However, this is
only a small part of the functionality of the program. Mosaic is what is know
as a World Wide Web client. The World Wide Web (WWW) is a network of
information servers located at many places around the world. The system was
developed at CERN, which is a good place to look for more information about it.
Mosaic can connect to any WWW server to read the information that it provides.
An up to date list of registered servers is kept at CERN and can be accessed
through the Starlink page containing introductory information. The World Wide
Web contains information on many subjects other than astronomy. When you have
found a useful source of information, you can store its location by selecting
{\bf Hotlist...} from the {\bf Navigate} pull down menu. 

\section{Selecting an information service}
\label{service}

The Mosaic program is written so that, by default, it connects you to the
information service at NCSA. This is a good place to learn about Mosaic, but it
is clearly not the best place from which to get information about Starlink
software. The place that Mosaic looks for its initial source of information
(known as the `home page') can be selected by means of the environment variable
{\tt WWW\_HOME}. The standard Starlink login file sets this to point to the
Starlink home page at {\tt www.starlink.ac.uk} unless you have already set {\tt
WWW\_HOME} to something else. If you wish to explore the information at NCSA
you can go there by selecting the {\bf Internet Starting Points} option from
the {\bf Navigate} pull down menu. Alternatively there is a link to NCSA from
within the Starlink page containing introductory information. 

It is possible to set your default home page to be anywhere in the world that
provides a world wide web server. For example, you can set your home page to be
the one at NRAO by typing:

\begin{quote}{\tt
\% setenv WWW\_HOME http://info.aoc.nrao.edu/
}
\end{quote}

\section{Other Mosaic facilities}

Mosaic has abilities other than just reading hypertext documents. From the {\bf
Navigate} pull down menu, you can select {\bf Internet Starting Points} for a
list of resources available on the Internet. One of these is a list of FTP
servers. By selecting this page and then choosing a site, you can browse
through the software that is available on many FTP archives. You can copy files
to your local computer by selecting the name of the file.

Mosaic also provides an interface to Gopher servers and WAIS (Wide Area
Information System). These are systems for searching the Internet for
information. For further information on these topics, look at {\bf Internet
Starting Points}.

\section{Customizing Mosaic}

Since Mosaic is an X windows program, you can customize its appearance by
setting the values of resources in your {\bf .Xdefaults} file. For example,
when you click on a hypertext link, Mosaic changes the colour of the text to
remind you that you have already been to that page. Personally, I find the
colour of the selected links not to be sufficiently different to the colours of
the unselected links, so I add the following line to my {\bf .Xdefaults} file
to show selected links in red:

\begin{quote}{\tt
Mosaic*visitedAnchorColor:   red
}
\end{quote}

A list of all the resources used by Mosaic and their default values are
available in the file {\bf /usr/local/lib/Mosaic/app-defaults.color}.

If you have heavily customized the colours on your workstation or X terminal,
you may need to use custom values of resources for Mosaic to achieve an
acceptable result.

\subsection{Example files}

There are a set of example files for customizing the appearance of Mosaic in
the directory {\bf /usr/local/lib/Mosaic}. These are as follows:

\begin{description}
\item[app-defaults.color] All of the standard settings used by Mosaic on a
colour workstation
\item[app-defaults.mono] All of the settings needed by Mosaic when used on a
monochrome workstation
\item[Xdefaults-decstation] The resources that need to be set when using a
DECstation as an X server to prevent warning messages being generated.
\item[Xdefaults-openwin] The resources that need to be set when using a Sun
running OpenWindows as an X server to prevent warning messages being generated.
\item[Xdefaults-vt1000] The resources that need to be set when using a
VT1000 as an X server to prevent warning messages being generated.
\end{description}


\section{Other documents}
\label{muds}

There are several documents that describe the Mosaic program and associated
facilities in greater detail. These are available in the Starlink
`miscellaneous user document' series.

\begin{description}
\item[MUD/147] Getting Started with NCSA Mosaic
\item[MUD/148] NCSA Mosaic Technical Summary
\item[MUD/149] A Beginner's Guide to HTML
\item[MUD/150] A Beginner's Guide to URLs
\item[MUD/151] Hypertext Markup Language (HTML)
\item[MUD/152] A LaTeX to HTML Translator
\end{description}

\appendix

\section{HTML}

This section is only relevant to users who want to write their own hypertext
documents.

For a document to be readable by Mosaic, it must be in a format known as
Hypertext Markup Language (HTML). This is a format not dissimilar to \LaTeX,
so if you know that, you should have little trouble with HTML. Documents
describing HTML are available as MUDs (see section~\ref{muds}). There is also a
program for translating \LaTeX\ documents to HTML format. The program is
written in perl and uses several Unix utilities, not all of which may be
installed at your site. See MUD/152 for more details. Contact your site manager
and the author of this document if you wish to run this program, but do not
have all of the tools.

\section{What to do if you get error messages}
\label{warnings}

When you run Mosaic, there are two types of messages that you are likely
to encounter. The first sort are warning messages that may be printed on the
terminal window where you typed the {\tt Mosaic} command. The second sort are
windows that pop up saying that Mosaic could not connect to a server.

The `cannot connect to information server' messages mean exactly what they say.
When you clicked on a link, the program tried to connect to a server at some
remote location and failed to do so. If you really want that information, there
is nothing that you can do but wait for the connection to the remote site to
come back. In the mean time, you can click on the {\tt BACK} button and
continue browsing. If you fail to connect to many sites, then it is likely that
{\em your} connection to the outside world is broken.

The warning messages that are printed on your terminal are usually to do with
Mosaic not being able to find the appropriate fonts or not being able to map
keys. Neither of these are fatal errors, but they are clearly untidy. The
warning messages about fonts are known to occur when using DECstations running
Ultrix or Suns running OpenWindows as the X server. 
They are due to the fact that different machines have different sets of fonts.
If you are happy with the way that Mosaic works, but you are annoyed with the
warning messages, you can pipe them to {\tt /dev/null}. If you want to do a
better job, read the rest of this section.

{\em N.B. It is shortcomings in the X server that gives rise to the warning
messages, not the machine on which you are running the Mosaic program.}

\subsection{Missing fonts}

Warnings about fonts look something like this:

\begin{verbatim}
Warning: Cannot convert string "-*-lucidatypewriter-medium-r-normal-*-14-*-iso8
859-1" to type FontList
\end{verbatim}

This means that Mosaic is trying to use a particular font that is 14 pixels
high and it cannot find it on your X server. The simplest solution is to just
ignore the warning messages. Mosaic will chose a suitable replacement font. If
you want to stop Mosaic complaining, you should put the appropriate lines in
your {\bf .Xdefaults} file. The first thing that you need is a suitable
replacement font. The command {\tt xlsfonts} will list all the fonts known to
your X server. There will be a lot of them and a suitable replacement font is
likely to exist. The best choice is a font where the first part of the name is
the same as the one you are replacing, but the pixel size is slightly
different. Alternatively, chose a font from a different family, but with the
same weight, slant and set width (the {\tt medium-r-normal} part).  Having
found the name of the font, you next need to know which X resource needs this.
The default values that Mosaic uses for resources are in the file {\bf
/usr/local/lib/Mosaic/app-defaults.color}. Search this file to see which
resource was using the font that could not be found. Then in your {\bf
.Xdefaults} file, set that resource. For example, given the warning message
shown above, a suitable replacement font on a DECstation is:

\begin{verbatim}
-dec-terminal-medium-r-normal--14-140-75-75-c-80-iso8859-1
\end{verbatim}

A search through the file {\bf Mosaic.color} reveals that the resources that
use the missing lucida font are:

\begin{verbatim}
Mosaic*XmText.fontList
Mosaic*XmTextField.fontList
\end{verbatim}

Consequently, what you would enter in your {\bf .Xdefaults} file would be:

\begin{verbatim}
Mosaic*XmText.fontList: -dec-terminal-medium-r-normal--14-*-iso8859-1
Mosaic*XmTextField.fontList: -dec-terminal-medium-r-normal--14-*-iso8859-1
\end{verbatim}

Example {\bf .Xdefaults} record for a DECstation, Sun (running OpenWindows) and
VT1000 are available in {\bf /usr/local/lib/Mosaic}. Remember that the one you
need depends on your X~server, not the machine on which you are running Mosaic.

\subsection{Unmapped keys}

Warnings about unmapped keys look something like this:

\begin{verbatim}
Warning: Cannot convert string "<Key>DRemove    " to type VirtualBinding
Warning: translation table syntax error: Unknown keysym name: osfActivate
Warning: ... found while parsing '<Key>osfActivate:     ManagerGadgetSelect() '
\end{verbatim}

If you get one, you are likely to get a lot more. Sun workstations appear to be
the worst offenders in generating these warnings. As with missing fonts, you
can ignore the warnings, but it is tidier if you get rid of them. The best
solution is to copy a file called {\bf XKeysymDB} into the directory {\bf
/usr/lib/X11}. This requires root privilege, so see your site manager if you
need to do it. A copy of {\bf XKeysymDB} is stored in the directory {\bf
/usr/local/lib/Mosaic}.

\section{Acknowledgments}

Much of the hypertext help information on Starlink software was prepared by
Dave Mills.

\end{document}
