\documentstyle[11pt]{article}
\pagestyle{myheadings}

%------------------------------------------------------------------------------
\newcommand{\stardoccategory}  {Starlink User Note}
\newcommand{\stardocinitials}  {SUN}
\newcommand{\stardocnumber}    {149.1}
\newcommand{\stardocauthors}   {J.\ R.\ Lewis}
\newcommand{\stardocdate}      {11 May 1992}
\newcommand{\stardoctitle}     {SAM --- A Spectral Extraction Package}
%------------------------------------------------------------------------------

\newcommand{\stardocname}{\stardocinitials /\stardocnumber}
\renewcommand{\_}{{\tt\char'137}}     % re-centres the underscore
\markright{\stardocname}
\setlength{\textwidth}{160mm}
\setlength{\textheight}{230mm}
\setlength{\topmargin}{-2mm}
\setlength{\oddsidemargin}{0mm}
\setlength{\evensidemargin}{0mm}
\setlength{\parindent}{0mm}
\setlength{\parskip}{\medskipamount}
\setlength{\unitlength}{1mm}

%------------------------------------------------------------------------------
% Add any \newcommand or \newenvironment commands here
%------------------------------------------------------------------------------

\begin{document}
\thispagestyle{empty}
SCIENCE \& ENGINEERING RESEARCH COUNCIL \hfill \stardocname\\
RUTHERFORD APPLETON LABORATORY\\
{\large\bf Starlink Project\\}
{\large\bf \stardoccategory\ \stardocnumber}
\begin{flushright}
\stardocauthors\\
\stardocdate
\end{flushright}
\vspace{-4mm}
\rule{\textwidth}{0.5mm}
\vspace{5mm}
\begin{center}
{\Large\bf \stardoctitle}
\end{center}
\vspace{5mm}

\begin{large}
\begin{em}
SAM is an optional item within the Starlink Software Collection. If it is not
installed at your site, please see your Site Manager.
\end{em}
\end{large}

%------------------------------------------------------------------------------
%  Add this part if you want a table of contents
\setlength{\parskip}{0mm}
\tableofcontents
\setlength{\parskip}{\medskipamount}
\markright{\stardocname}
%------------------------------------------------------------------------------

\newpage
\section{Introduction}

In this note a description is given of SAM, a package written at RGO for the
extraction of spectra from two dimensional data frames.  The need to extract
spectra from two dimensional frames in an optimal manner (e.g. one in which the
signal to noise ratio was maximised) was the primary reason for the writing of
the package. The programs were originally written with FOS, ISIS and IDS in
mind, but contain nothing which  is instrument specific and hence should be
applicable to any two dimensional spectral data frame.

The programs have been written in accordance with the methods and conventions
of FIGARO (SUN/86) and the Starlink Programming Standard (SGP/16). SAM requires
FIGARO to be installed in order to run. The applications will accept any two
dimensional FIGARO image or NDF.  The output files will be recognised by any
FIGARO or ADAM application.

\section{How to run the programs}

SAM is accessed through its own startup command:

\begin{verbatim}
      $ SAM
\end{verbatim}

You will also need to start FIGARO to use the graphics facilities:

\begin{verbatim}
      $ FIGARO
\end{verbatim}

All of the relevant logical and symbolic definitions should then be done for
you.  To run a particular program, just issue the name of the program, much the
same as if it were a standard FIGARO command. In addition, a help file will
be set up for you. Help on an individual program may be obtained by
entering:
\begin{verbatim}
      $ HELP prog
\end{verbatim}

This will give a list of the program's input parameters as well as an option to
look at the source header comments for the main routine.

\section{Features supported}

SAM can deal with data arrays of any type, but will convert any non-real data
to real before any processing is done.  Output files will be of the same type
as the input file with the exceptions of (1) files which may be created which
contain the optimal extraction profiles and (2) the final output one
dimensional spectra.  These will both be real.  Both input and output files
may contain flagged values.  Quality arrays are dealt with by converting to
flagged values.

SAM expects the input file to contain error (or variance) arrays if spectra
are to be ``optimally extracted''.  If a file does not contain such an error
array, then it won't be rejected and a simple tramline extraction will be
performed by default.  Error arrays must be generated and propagated properly
during the initial processing of the data frames (e.g. flat fielding).
Packages such as CCDPACK and SCP will do this and their use is recommended.

\section{Package overview}

As will hopefully become clear, SAM can be used in two ways.  The first method
involves running separate applications for each step of the reduction process.
This has the advantage that it allows the user to examine the data after each
step in order to see what exactly is happening.  The disadvantages are that it
(1) is more time consuming and (2) creates extra output files which may be of
no interest to the user and could take up a lot of valuable space on disk.
The second method for running SAM involves using just two routines---one
which  sets up all of the input parameters and another which combines all the
repetitive applications into one. Such a system can therefore be run fairly
automatically or even in batch.

One thing to remember about SAM is that it assumes that the spectral axis of
the data is parallel to the $y$ axis of the frame.  There are several
instances where SAM will ask which axis the spectra are parallel to.  The
answer should be either `1' for the $x$ axis or `2' for the $y$ axis.  Thus if
the answer is `1' then SAM will flip the data array before doing any
processing.  If a two dimensional array is output, its orientation will always
be that of the input array.

Extracting spectra with SAM falls into two stages.  The first is to define a
DEKKER file and a TRACE file.  The DEKKER file defines the area of interest
along the spatial axis (in order that only the data region of interest is
carried around for the sake of efficiency). The TRACE file defines the spatial
distortion of the spectra (e.g. a FOS spectrum).  The second stage involves
the actual location, sky subtraction, profile fit and extraction of the of the
spectra.  Generally for a given run, stage one only needs to be done once
while stage two is repeated for each spectrum.  The basic data reduction
procedure is summarised in figure \ref{fig:flow}.  Variations on this
procedure will hopefully become apparent later.

\begin{figure}[tbp]
\setlength{\unitlength}{1in}
\begin{picture}(5.15,7.5)(0,2.5)
\put(1.0,9.4){\framebox(1.5,0.3){\shortstack{Frames have \\ been pre-processed?}}}
\put(2.5,9.55){\vector(1,0){1.0}}
\put(2.625,9.65){\makebox(0,0){No}}
\put(3.5,9.4){\framebox(1.5,0.3){\shortstack{{\bf SCP} or \\ {\bf CCDPACK}}}}
\put(4.25,9.4){\line(0,-1){0.15}}
\put(4.25,9.25){\vector(-1,0){2.0}}
\put(1.75,9.4){\vector(0,-1){0.3}}
\put(1.95,9.25){\makebox(0,0){Yes}}
\put(1.0,8.8){\framebox(1.5,0.3){\shortstack{Distorted \\ spectra?}}}
\put(1.75,8.8){\vector(0,-1){0.3}}
\put(1.95,8.65){\makebox(0,0){Yes}}
\put(1.0,8.2){\framebox(1.5,0.3){\bf SAM\_TRACE}}
\put(1.75,8.2){\vector(0,-1){0.3}}
\put(2.625,9.05){\makebox(0,0){No}}
\put(2.5,8.95){\line(1,0){0.25}}
\put(2.75,8.95){\line(0,-1){0.9}}
\put(2.75,8.05){\vector(-1,0){1.0}}
\put(1.0,7.6){\framebox(1.5,0.3){\bf SAM\_DEKKER}}
\put(0.5,7.3){\dashbox{0.25}(2.5,2.7){}}
\put(0,8.65){\makebox(0,0){Stage 1}}
\put(1.75,7.6){\vector(0,-1){0.6}}
\put(1.0,6.7){\framebox(1.5,0.3){\shortstack{Use individual \\ applications?}}}
\put(1.75,6.7){\vector(0,-1){0.3}}
\put(1.95,6.6){\makebox(0,0){Yes}}
\put(1.0,6.1){\framebox(1.5,0.3){\bf SAM\_FIND}}
\put(1.75,6.1){\vector(0,-1){0.3}}
\put(1.0,5.5){\framebox(1.5,0.3){\bf SAM\_SKY}}
\put(1.75,5.5){\vector(0,-1){0.3}}
\put(1.0,4.9){\framebox(1.5,0.3){\shortstack{Optimal \\ extraction?}}}
\put(1.75,4.9){\vector(0,-1){0.3}}
\put(1.95,4.75){\makebox(0,0){Yes}}
\put(1.0,4.3){\framebox(1.5,0.3){\shortstack{{\bf SAM\_PROFILE} \\ Marsh or
Horne}}}
\put(1.75,4.3){\vector(0,-1){0.3}}
\put(1.0,3.7){\framebox(1.5,0.3){\bf SAM\_EXTRACT}}
\put(1.75,3.7){\vector(0,-1){0.3}}
\put(1.0,3.1){\framebox(1.5,0.3){More?}}
\put(1.75,3.1){\line(0,-1){0.15}}
\put(1.95,3.0){\makebox(0,0){Yes}}
\put(1.75,2.95){\line(-1,0){1.0}}
\put(0.75,2.95){\line(0,1){3.60}}
\put(0.75,6.55){\vector(1,0){0.75}}
\put(2.5,3.25){\vector(1,0){0.5}}
\put(2.625,3.35){\makebox(0,0){No}}
\put(3,3.1){\framebox(1.5,0.3){Done}}
\put(2.5,5.05){\line(1,0){0.15}}
\put(2.625,5.15){\makebox(0,0){No}}
\put(2.65,5.05){\line(0,-1){0.9}}
\put(2.65,4.15){\vector(-1,0){0.75}}
\put(2.5,6.85){\line(1,0){1.25}}
\put(2.625,6.95){\makebox(0,0){No}}
\put(3.75,6.85){\vector(0,-1){0.3}}
\put(3.0,6.25){\framebox(1.5,0.3){\bf SAM\_SETUP}}
\put(3.75,6.25){\vector(0,-1){0.3}}
\put(3.0,5.65){\framebox(1.5,0.3){\bf SAM\_PROC}}
\put(3.75,5.65){\vector(0,-1){0.3}}
\put(3.0,5.05){\framebox(1.5,0.3){More?}}
\put(3.75,5.05){\line(0,-1){0.15}}
\put(3.95,4.95){\makebox(0,0){Yes}}
\put(3.75,4.90){\line(-1,0){1.0}}
\put(2.75,4.90){\line(0,1){1.2}}
\put(2.75,6.10){\vector(1,0){0.75}}
\put(4.5,5.20){\line(1,0){0.15}}
\put(4.625,5.30){\makebox(0,0){No}}
\put(4.65,5.2){\line(0,-1){1.95}}
\put(4.65,3.25){\vector(-1,0){0.15}}
\put(0.5,2.5){\dashbox{0.25}(4.65,4.7){}}
\put(0,4.85){\makebox(0,0){Stage 2}}
\end{picture}
\caption{A flow diagram for spectral extraction using SAM \label{fig:flow}}
\end{figure}
\setlength{\unitlength}{1mm}

\subsection{Stage 1 - setting up}

Here a brief description of stage one of the process is given. Note that this
is not an attempt to explain the  individual routines comprehensively.  The
user should refer to section \ref{sec:indiv} for this.

From here onwards the assumption is made that the input data frames have been
processed (e.g. flat-fielded, bias subtracted, etc.) and contain the necessary
error arrays.

The first question to ask is: are the spectra distorted?  The answer
obviously depends on the instrument which was used in the observations.   But
it also depends on what one means by ``distorted''.  SAM offers two  optimal
extraction algorithms and a tramline extraction algorithm.  One of  the
optimal extraction routines (based on Horne 1986) deals with spectra  which
are either straight or mildly distorted.  If the maximum distortion in the
spatial direction is of the order of the seeing disc width, then for SAM this
is a `straight' spectrum.  The second optimal extraction algorithm (based on
Marsh 1989) deals with more severly distorted spectra.  To do  this it needs
the help of a TRACE file.  If there is any doubt, then a TRACE file should be
used. The tramline algorithm works both  with or without a TRACE file.

A TRACE file is created with SAM\_TRACE.  With the help of the  user, this
routine locates and centroids the spectrum.  It then does a  polynomial fit to
the centroid positions as a function of wavelength.  In  general the shape of
the distortion will not change substantially over  the course of a night, so
it should only be necessary to do this step once for a given night's data.

The second step in this stage is to define the DEKKER file.  SAM doesn't  like
to carry around any more data than it has to (especially as CCD frames get
bigger and bigger).  Generally a spectrum is only likely to fall on a
particular part of a chip and the rest is pretty much irrelevant.  A  DEKKER
file is used to define the region of the chip where the data (both  spectrum
and sky) are going to fall. SAM\_DEKKER displays a given frame and  asks the
user to indicate with the cursor where the limits of the data  along the
spatial axes are.  Care should be taken here to ensure that a  large enough
area is indicated in order to include some pixels for sky  subtraction.

\subsection{Stage 2 - extraction}

As was previously mentioned this stage involves the actual extraction of  the
spectra and can be done in two different ways.  A description of each method
is given here.

\subsubsection{Using individual applications} \label{subsec:useind}

It is quite often the case that the user will want to examine the data  after
each step of the reduction process in order to get an idea of what's  actually
happening to the data.  For that reason in this section the  reduction process
is broken down into bite-size chunks.

{\bf Locating the spectra}

The first thing which is required to is to get a proper location for the
spectrum, a definition of the range of spatial pixels to include as part of
the spectrum (i.e. this is the `width' of the seeing disk) and some sky bands
on either side of the spectrum.  This is accomplished with SAM\_FIND.
SAM\_FIND either allows you to define all these things yourself or will do it
automatically.  In both methods, the first thing which SAM\_FIND does is to
use the window defined by the DEKKER file and the curvature defined by  the
TRACE file (if one exists) in order to sum up the data in the spectral
direction.  What is left is a one dimensional array of total flux as a
function of `comoving' position on the dekker.  What is done with this  array
makes the difference between the two methods for defining the  quantities
listed above.

In the interactive version this array is plotted.  The user is asked to
indicate on the plot the location of the spectrum (i.e. the peak of the  flux
in general) and the limits of the spectrum in the spatial direction  (i.e.
where the spectral flux has fallen off such that it isn't really much above
sky).  SAM\_FIND then goes away and finds the median location of the
spectrum.  The resulting median is indicated on the plot by an arrow. The user
is then asked to indicated two points on either side of the spectrum which
define the range of sky pixels.  Care should be taken here that  none of the
indicated sky pixels overlaps with the spectral pixels which have  just been
defined.

In the automatic version SAM\_FIND tries to define these quantities by
looking for pixels whose flux is a certain threshold over what it has decided
is sky.  These pixels are then defined as part of the object.  The user is
asked to supply this threshold and a nominal width for the spectrum (used as a
first guess). A polynomial is fit to the sky pixels in order to define the sky
values and sigma. The order of this polynomial should  not in general be
greater than 1 as this is not a proper sky subtraction. The user is also asked
for a maximum distance for the sky pixels to be from the last object pixel.
This is to guard against large scale variations in  the sky value which may be
a result of bad flat fielding or vignetting by  the dekker.  This routine is
designed to cope with multiple  objects on the slit.  Thus the user must
specify how many objects are on the slit and (if there are more than one) the
number of the spectrum which  is to be `found'.  This is assuming that for
spectra which are parallel to the $x$ axis, counting is done from the bottom
upwards and for spectra which are parallel to the $y$ axis, counting from left
to right.

SAM\_FIND writes the results of all this into the input file header.  The
information can be found by doing:

\begin{verbatim}
      $ EXAM file.MORE.FIGARO.SAM...
\end{verbatim}

{\bf Sky subtraction}

Sky subtraction is done by SAM\_SKY and is very straight forward.  Most of
the information which SAM\_SKY requires is stored in the header of the  input
file.  If these can't be found there, then the user will be prompted  for
them. See section \ref{sec:indiv} for more details.  Basically what is  done
is that the program fits a polynomial to the sky rows defined in the  previous
section at each wavelength.  The polynomial is then interpolated  over the
pixels which contain the object spectrum.  It is thus important to be  sure
that the order of the polynomial is kept low (generally 2 or less) as the
first derivative of the polynomial may vary wildly in the object  pixels.  In
addition the fits can be done iteratively. The user is  asked to give a limit
to the number of pixels to reject in the fitting  procedure.  After each
iteration the pixel with the largest residual is  rejected.  The output file
is a copy of the input frame with the sky  subtracted in the region defined by
the spectrum and in the sky bands.

{\bf Optimal extraction profile fitting}

If the user has decided on optimal extraction, then this is the next step. The
routine to use is SAM\_PROFILE.  As was previously mentioned, this  routine
has two optimal extraction profile algorithms on offer.  No attempt will be
made to explain optimal extraction in any detail as this is far  better
explained elsewhere (Horne 1986, Marsh 1989 and Mukai 1990) and  the
interested  reader is strongly advised to consult these sources for more
information.

Optimal extraction attempts to maximise the signal to noise ratio of the
output spectrum by giving lower weights to pixels which have little or no
signal from the object.  To do this the seeing disk is modelled with a series
of polynomials, each fitting the relative fraction of the object flux in  a
spatial pixel as a function of wavelength. In the case of Horne's method the
polynomials  run along actual pixels in the wavelength direction. In the case
of Marsh's method the polynomials run along `pseudo-pixels' which are parallel
to the distortion defined in the TRACE file.   The output is a frame with
weights for the extraction routine to use.  SAM\_PROFILE will know which
optimal extraction algorithm is wanted as the input header will say whether a
TRACE file is being used or not.

The fits are done iteratively hence the user specifies a threshold (in
sigma---usually 4 or 5 suffices) above which a pixel will be deleted from the
fits.  The user is also asked for the order of the polynomials to be fit.
High order polynomials are not usually required as, if the distortion is bad,
then the TRACE file takes out most of that anyway. Since Marsh's method
involves using rows and columns which have been defined by the TRACE file
(e.g. not the real things) then it is necessary to interpolate between rows
and columns to construct the optimal extraction profile.  Hence the user will
be requested an incremental distance between the polynomials.  This must be
less than 1 given the linear interpolation scheme used and ideally should be
less than 0.5.  Finally, the user can block in the spectral direction if it
looks as though the signal to noise of the original is just too low.  This is
not recommended though for any other case.

{\bf Spectral extraction}

Finally the extraction routine SAM\_EXTRACT is needed.  This is the routine to
use whether or not an optimal extraction is being done.  This routine has a
hidden parameter which tells it whether or not to do an optimal extraction.
Hence even though optimal extraction profiles have been calculated, the user
can still opt for an ordinary tramline extraction.  This is explained in more
detail in section \ref{sec:indiv}.  If an optimal extraction is being done,
then the user will be asked for a rejection threshold.  SAM\_EXTRACT will
extract the spectrum iteratively, rejecting pixels with residual values over
this threshold one at a time until no more have been rejected.  The value of
this is that things such as cosmic ray hits are rejected from the final
spectrum.

{\bf Other useful programs}

There are a few more programs which may be of use along the way.  First of all
SAM\_SHOW will do a two dimensional plot of a frame and will draw tramlines
to indicate the location of the spectrum, the limits of the spectrum in the
spatial direction and the limits of the two sky bands.

SAM\_STRT takes a TRACE file and a frame with a distorted spectrum and
straightens the latter.  This involves linearly interpolating as the shifts
are not in general integral pixels.  This then violates the independence of
the error estimates which is crucial for optimal extraction.  Hence if this
routine is used, it is not advisable to go on to do an optimal extraction of
the resulting frame.

SAM\_ARC was written specifically with arc frames in mind, but can in theory
be used for any frame.  If a spectrum is optimally extracted, then very likely
the user will want the same optimal extraction weights to be applied to the
corresponding arc exposures.  As most of the information in SAM is passed
around in the input file headers, SAM\_ARC will copy the necessary header
information from one frame (the `basis' frame) to another (the `output'
frame).  For example, the user can optimally extract a star. Then SAM\_ARC can
be run specifying the input star frame as the basis and the arc frame as the
output.  Finally the arc frame can be run through SAM\_EXTRACT and the same
optimal extraction weights will be used on the arc as were used on the star.

Finally there is SAM\_WHOLE.  This program sets up the tramline extraction
parameters in the header assuming that the spectrum covers the entire defined
dekker.  This is useful for things such as arcs which don't need to be
located, sky subtracted etc.  This only needs a DEKKER file and a TRACE file
if you are using one. Note that once SAM\_WHOLE has been run, SAM\_EXTRACT
still is needed to extract the spectrum.

\subsubsection{Using combined routines}

Once the user has established how the data should be reduced, then using the
procedure outlined in section \ref{subsec:useind} can become a downright bore.
In this section a method which combines all the repetitive elements of stage 2
is discussed.

The following method involves the use of two routines.  The first one,
SAM\_SETUP, sets up the parameters.  These are basically all of the parameters
which one would enter for SAM\_FIND, SAM\_SKY, SAM\_PROFILE and SAM\_EXTRACT.
The second program, SAM\_PROC, does all of the processing of the data and only
asks for input and output file names.

SAM\_SETUP works by defining all the parameters necessary to process the data
in a file called SAM\_VARS.DST.  This is very similar to the FIGARO parameter
file, VARS.DST.  The file will appear in either the user's home directory or
in the current default directory.  SAM\_SETUP may not prompt for every
veriable, but makes sure that a sensible value is placed in SAM\_VARS.DST just
in case.

When it was stated that SAM\_PROC only has two parameters I wasn't telling the
whole truth.  SAM\_PROC will accept {\em any} of the parameters which is
accepted in SAM\_SETUP, but only if they are specified on the command line.
This allows the user to tweak the reduction procedure without having to go
through the exercise of re-running SAM\_SETUP.  Thus if a comparison was to be
made of a spectrum extracted optimally and non-optimally, then SAM\_SETUP only
needs to be run once, say with the optimal extraction flag set.  Then
SAM\_PROC is run once with nothing else on the command line and then a second
time with the optimal extraction parameter unset on the command line.

\section{Individual program descriptions} \label{sec:indiv}

In this section a description of the individual routines and the parameters
they require is given.  In some routines information is read directly out of
the input file header.  If for some reason this information doesn't exist then
the user will be prompted for them.  Such parameters will be flagged with an
asterisk.

\subsection{SAM\_ARC}

SAM\_ARC copies information from the header of one file (`basis') to the
header of another (`output').  This was written so that two or more spectra
can be extracted using exactly the same extraction weights.  The parameters
are  as follows

\begin{description}

\item[BASIS] This is the name of the file whose header is to act as the basis.
It is important that this file has been run completely through the extraction
process.  Otherwise some of the necessary information may be missing.

\item[OUTPUT]  This is the name of the file to whose header the information
will be written.

\item[AXIS*] This is the axis to which the spectra are parallel.  Answer `1'
for the x axis and `2' for the y axis.

\item[TRACK*] This is TRUE if the a TRACE file is going to be used.

\item[TRACE*] This is the name of the TRACE file as written by SAM\_TRACE.

\item[SPEC\_RNG*] The user can restrict the spectral range of pixels to be
extracted. These are the lower and upper limits in pixel numbers.  Two values
are required.

\item[SPEC\_LIMITS*] This is the first and last row/column in the spatial
direction which are to be considered part of the spectrum.  Note that these
limits are found by SAM\_FIND. Two values are required.

\item[SPEC\_POS*] This is the median location of the spectrum in the spatial
direction as found by SAM\_FIND.

\item[DEKKER*] This is the name of the DEKKER file as written by SAM\_DEKKER.

\item[SKY1*] These are the limits to the first sky band.  Note that these
limits are found by SAM\_FIND.  Two values are required.

\item[SKY2*] These are the limits to the second sky band.  Note that these
limits are found by SAM\_FIND.  Two values are required.

\item[PROFILE*] This is the name of the file containing the optimal extraction
profile weights and obviously only required if an optimal extraction is being
performed.

\end{description}

\subsection{SAM\_DEKKER}

SAM\_DEKKER sets up the DEKKER file.  Some sort of two dimensional display is
required as the user indicates the limits of the dekker with a cursor.  The
parameters are as follows:

\begin{description}

\item[INPUT] This is the name of the data frame which is to be displayed.

\item[AXIS] This is the axis to which the spectra are parallel.  Answer `1' for
the x axis and `2' for the y axis.

\item[TABLE] This is the colour lookup table for the plot

\item[LOW, HIGH] These are the low and high data values for the greyscale.

\item[TRACK] This is TRUE if the a TRACE file is going to be used.

\item[TRACE] This is the name of the TRACE file as written by SAM\_TRACE.

\item[DEKKER] This is the output file for the dekker coordinates.

\end{description}

\subsection{SAM\_EXTRACT}

This program does the actual extraction of the spectra.  It offers two optimal
extraction algorithms as well as tramline extraction.  If the user wants
optimal extraction, then the choice of the extraction method to be used will
be made automatically based on whether a TRACE file has been used in the
previous reduction procedures.   The parameters are as follows:

\begin{description}

\item[INPUT]  This is the name of the input sky-subtracted frame.

\item[EXT\_THR] This is a threshold over which pixels will be rejected from
the optimal extraction.  This is a way of getting rid of cosmic ray hits and
other defects.  This is only prompted for when optimal extraction is being
used.  The units are in number of sigma of the individual pixels.

\item[PROFILE*] This is the name of the file containing the optimal extraction
profile weights and obviously only required if an optimal extraction is being
performed.

\item[AXIS*] This is the axis to which the spectra are parallel.  Answer `1'
for the x axis and `2' for the y axis.

\item[TRACK*] This is TRUE if the a TRACE file is going to be used.

\item[TRACE*] This is the name of the TRACE file as written by SAM\_TRACE.

\item[SPEC\_RNG*] The user can restrict the spectral range of pixels to be
extracted. These are the lower and upper limits in pixel numbers.  Two values
are required.

\item[SPEC\_LIMITS*] This is the first and last row/column in the spatial
direction which are to be considered part of the spectrum.  Note that these
limits are found by SAM\_FIND. Two values are required.

\item[OUTPUT] This is the name of the output file.

\item[OPTIMAL]  This is TRUE if the user wants optimal extraction.  Note that
it is not prompted for and it defaults to TRUE.  If tramline extraction is
wanted, then NOOPTIMAL must be specified on the command line.

\end{description}

\subsection{SAM\_FIND}

SAM\_FIND locates the spectrum, defines the width of the spectrum (e.g. the
width of the seeing disc), and defines the bands to be used for sky
subtraction.  It can be used in either an interactive mode (where the user
indicates all these things with a cursor) or an automatic mode (where the
program locates the spectrum itself with the help of a few parameters). If
interactive mode is used, then a device for line plotting will be needed.  The
parameters are as follows:

\begin{description}

\item[INPUT]  This is the name of the input frame.

\item[AXIS] This is the axis to which the spectra are parallel.  Answer `1' for
the x axis and `2' for the y axis.

\item[NITER] SAM\_FIND attempts to locate the centroid of the spectral profile
by locating the median position.  This is the maximum number iterations allowed
for this algorithm to converge.

\item[TOL] This is the tolerance in pixels for the convergence of the median
finding routine.

\item[NOM\_WIDTH] This is the nominal width of the spectrum in the spatial
domain.  This is used as first guess in automatic mode.

\item[SPEC\_THR] In automatic mode the rows/columns which contain the spectrum
are defined as begin a certain threshold in sigma over the sky value.  The sigma
is the dispersion from the sky fit.

\item[QSKY\_ORD] In automatic mode a quickie sky value is estimated from a
polynomial fit.  This is the order of that polynomial.  This should be kept
very low.  Note that this is {\em not} the sky fit which will be used for the
sky subtraction!

\item[NQREJ] In doing the quickie sky fit in automatic mode the routine will
allow for some pixels to be rejected.  This is the maximum number of pixels
which the user wants to reject.  Again this should be kept low, otherwise the
sky value may be badly estimated and this will result in a poor estimate of the
width of the spectrum.

\item[MAX\_SEP]  In automatic mode when the sky pixels are being chosen, the
routine needs to know how far away from the spectral pixels the sky pixels are
allowed to be.  Having sky pixels which are reasonably far from the spectral
pixels could affect the sky subtraction because of large scale variations in
the sky data.  This is the maximum distance away from the edge of the defined
object region which sky pixels are allowed to be.  For example, if MAX\_SEP = 5
then the most distant sky pixel will be at most 5 pixels away from the nearest
object pixel on that side.

\item[TRACK] This is TRUE if the a TRACE file is going to be used.

\item[TRACE] This is the name of the TRACE file as written by SAM\_TRACE.

\item[DEKKER] This is the name of the DEKKER file as written by SAM\_DEKKER.

\item[SPEC\_RNG] The user can restrict the spectral range of pixels to be extracted.
These are the lower and upper limits in pixel numbers.  Two values are
required.

\item[NOBJ] This is the number of objects on the slit.

\item[IOBJ] This is the object which is to be `found' -- counting from bottom
to top for spectra parallel to the x axis and left to right for spectra
parallel to the y axis.  Note that is is only used if $\mbox{NOBJ} > 1$.

\item[AUTOFIND] This is TRUE if the user wants the spectra and sky bands
located automatically.

\end{description}

\subsection{SAM\_PROC}

SAM\_PROC is a routine which combines all the functions of routines SAM\_FIND,
SAM\_SKY,  SAM\_PROFILE and SAM\_EXTRACT into one routine.  It needs SAM\_SETUP
to have been run first.  If it is to be run in batch, then SAM\_SETUP must also
be run in batch.  It has many parameters, but only two of them are unhidden.
This means that a parameter which has been set by SAM\_SETUP can be overridden
for one particular run of SAM\_PROC (by specifying it on the command line),
but will still remain the default for future runs until it is reset by
SAM\_SETUP.  In addition to the list below, one should consult section
\ref{sec:samsetup} for a list of the hidden parameters which can be set.

\begin{description}

\item[INPUT] This is the name of the input frame.

\item[OUTPUT] This the output file name.  The output will be in the form of a
one dimensional spectrum.

\end{description}

It is worth noting that in the case where a series of spectra are to be
extracted all in the same way, then one could place the names of all of the
input frames in a list file and all the output names in another list file. Then
when INPUT and OUTPUT are requested specifying the filenames with an `@' before
them (e.g. @filename) will mean that SAM\_PROC will be done repeatedly until it
runs out of names.  In this event, care should be taken that both list files
have the same number of names!

\subsection{SAM\_PROFILE}

This is the routine which generates the optimal extraction profile.  If optimal
extraction isn't wanted, then this routine should be skipped.  There are two
optimal extraction algorithms on offer.  The choice is made automatically by
looking in the input file header to see if a TRACE file is being used.  The
parameters are:

\begin{description}

\item[INPUT] This is the name of the input sky-subtracted frame.

\item[AXIS*] This is the axis to which the spectra are parallel.  Answer `1'
for the x axis and `2' for the y axis.

\item[TRACK*] This is TRUE if the a TRACE file is going to be used.

\item[TRACE*] This is the name of the TRACE file as written by SAM\_TRACE.

\item[SPEC\_RNG*] The user can restrict the spectral range of pixels to be
extracted. These are the lower and upper limits in pixel numbers. Two values
are required.

\item[SPEC\_LIMITS*] This is the first and last row/column in the spatial
direction which are to be considered part of the spectrum.  Note that these
limits are found by SAM\_FIND. Two values are required.

\item[BLOCK] The user may block the spectra in wavelength in order to increase
the signal to noise. The value tells how many rows/columns will be added
together to make one block. It is recommended that this value be 1 except in
cases of very bad signal/noise.

\item[POL\_DIST]  If a TRACE file is being used, then the algorithm used will
be that of Marsh.  In this case there needs to be defined a distance in pixels
between the polynomials in order for the interpolation scheme to work.  The
value must be less than 1 and preferably would be less than 0.5.

\item[PORD] This is the order of the polynomials to be fit for the optimal
extraction profile.

\item[PRO\_THR]  The profile fitting is done iteratively where pixels whose
residuals are this number of sigma above the predicted profile are rejected.
The sigma here is the error for the individual pixel.

\item[OUTPUT] This is the name for the output file which will contain the
optimal extraction profile weights.

\end{description}

\subsection{SAM\_SETUP} \label{sec:samsetup}

This is the routine which does all the setting up of the parameters for
SAM\_PROC.  It first checks that entries for each of the parameters exist in
the parameter file (SAM\_VARS.DST). If they don't then they are created and
values inserted.  The user is then prompted for a number of parameters.  The
number prompted for depends upon how the data is to be reduced.  For example,
if the spectra are going to be located interactively, then there is no point
asking for parameters which involve the automatic spectrum finding algorithm.
For this reason, SAM\_SETUP tries to make sure that there is in the parameter
file either a reasonable value for each parameter or a value which will flag
an error if it hasn't been set.  The exception is that a prompt will be given
for optimal extraction parameters even if an optimal extraction isn't to be
done.  This is because it is sometimes very useful to be able to switch
between optimal and non-optimal extraction without having to re-run SAM\_SETUP
and it is essential to have the parameters available. The parameters in the
following list are divided into categories according to the procedure which
needs them.

{\bf Parameters from SAM\_FIND}

\begin{description}

\item[NITER] SAM\_FIND attempts to locate the centroid of the spectral profile
by locating the median position.  This is the maximum number iterations
allowed for this algorithm to converge.

\item[TOL] This is the tolerance in pixels for the convergence of the median
finding routine.

\item[NOM\_WIDTH] This is the nominal width of the spectrum in the spatial
domain.  This is used as first guess in automatic mode.

\item[SPEC\_THR] In automatic mode the rows/columns which contain the spectrum
are defined as a certain threshold in sigma over the sky value.  The sigma is
the dispersion from the sky fit.

\item[QSKY\_ORD] In automatic mode a quickie sky value is estimated from a
polynomial fit.  This is the order of that polynomial.  This should be kept
very low.  Note that this is {\em not} the sky fit which will be used for the
sky subtraction!

\item[NQREJ] In doing the quickie sky fit in automatic mode the routine will
allow for some pixels to be rejected.  This is the maximum number of pixels
which the user wants to reject.  Again this should be kept low, otherwise the
sky value may be badly estimated and this will result in a poor estimate of
the width of the spectrum.

\item[MAX\_SEP]  In automatic mode when the sky pixels are being chosen, the
routine needs to know how far away from the spectral pixels the sky pixels are
allowed to be.  Having sky pixels which are reasonably far from the spectral
pixels could affect the sky subtraction because of large scale variations in
the sky data.  This is the maximum distance away from the edge of the defined
object region which sky pixels are allowed to be. For example, if MAX\_SEP = 5
then the most distant sky pixel will be at most 5 pixels away from the nearest
object pixel on that side.

\item[NOBJ] This is the number of objects on the slit.

\item[IOBJ] This is the object which is to be `found' -- counting from bottom
to top for spectra parallel to the x axis and left to right for spectra
parallel to the y axis.  Note that is is only used if $\mbox{NOBJ} > 1$.

\item[AUTOFIND] This is TRUE if the user wants the spectra and sky bands
located automatically.

\end{description}

{\bf Parameters from SAM\_SKY}

\begin{description}

\item[SKY\_ORD] This is the order of the polynomials to be fit to the sky
pixels.  One should be extremely careful about using orders higher than about
1 since the interpolation in the object pixels may involve areas where the
gradient of the polynomials becomes very large.

\item[NREJECT]  The sky polynomial fits are done iteratively with the worst
fitting pixel from each fit being rejected in each iteration.  This is the
number of pixels which will be rejected.  Again one should be careful to not
reject too many pixels as it may become impossible to get a good fit.

\end{description}

{\bf Parameters from SAM\_PROFILE}

\begin{description}

\item[BLOCK] The user may block the spectra in wavelength in order to increase
the signal to noise.  The value tells how many rows/columns will be added
together to make one block. It is recommended that this value be 1 except in
cases of very bad signal/noise.

\item[POL\_DIST]  If a TRACE file is being used, then the algorithm used will
be that of Marsh.  In this case there needs to be defined a distance in pixels
between the polynomials in order for the interpolation scheme to work.  The
value must be less than 1 and preferably would be less than 0.5.

\item[PORD] This is the order of the polynomials to be fit for the optimal
extraction profile.

\item[PRO\_THR]  The profile fitting is done iteratively where pixels whose
residuals are this number of sigma above the predicted profile are rejected.
The sigma here is the error for the individual pixel.

\end{description}

{\bf Parameters from SAM\_EXTRACT}

\begin{description}

\item[EXT\_THR] This is a threshold over which pixels will be rejected from the
optimal extraction.  This is a way of getting rid of cosmic ray hits and other
defects.  This is only prompted for when optimal extraction is being used.  The
units are in number of sigma of the individual pixels.

\item[OPTIMAL]  This is TRUE if the user wants optimal extraction.

\end{description}

{\bf Other parameters}

\begin{description}

\item[AXIS] This is the axis to which the spectra are parallel.  Answer `1' for
the x axis and `2' for the y axis.

\item[TRACK] This is TRUE if the a TRACE file is going to be used.

\item[TRACE] This is the name of the TRACE file as written by SAM\_TRACE.

\item[SPEC\_RNG] The user can restrict the spectral range of pixels to be
extracted. These are the lower and upper limits in pixel numbers.  Two values
are required.

\item[DEKKER] This is the name of the DEKKER file as written by SAM\_DEKKER.

\item[USE\_WH] This is TRUE if the spectrum is going to cover the whole of the
dekker, e.g. an arc spectrum, where sky subtraction etc doesn't matter.  This
is the same as using SAM\_WHOLE.

\end{description}

\subsection{SAM\_SHOW}

This routine displays a frame as a two-dimensional colour plot. superimposed
tramlines show the position of the spectrum, the limits of the spectral
rows/columns and the positions of the two sky bands.  The parameters are:

\begin{description}

\item[INPUT] This is the name of the input frame to be displayed.

\item[AXIS*] This is the axis to which the spectra are parallel.  Answer `1'
for the x axis and `2' for the y axis.

\item[TRACK*] This is TRUE if the a TRACE file is going to be used.

\item[TRACE*] This is the name of the TRACE file as written by SAM\_TRACE.

\item[DEKKER*] This is the name of the DEKKER file as written by SAM\_DEKKER.

\item[SPEC\_LIMITS*] This is the first and last row/column in the spatial
direction which are to be considered part of the spectrum.  Note that these
limits are found by SAM\_FIND. Two values are required.

\item[SKY1*] These are the limits to the first sky band.  Note that these
limits are found by SAM\_FIND.  Two values are required.

\item[SKY2*] These are the limits to the second sky band.  Note that these
limits are found by SAM\_FIND.  Two values are required.

\item[TABLE] This is the colour lookup table for the plot

\item[LOW, HIGH] These are the low and high data values for the greyscale.

\end{description}

\subsection{SAM\_SKY}

SAM\_SKY does the sky subtraction after the sky regions and the object region
has been defined.  The parameters are:

\begin{description}

\item[INPUT] This is the name of the input frame to be displayed.

\item[AXIS*] This is the axis to which the spectra are parallel.  Answer `1'
for the x axis and `2' for the y axis.

\item[TRACK*] This is TRUE if the a TRACE file is going to be used.

\item[TRACE*] This is the name of the TRACE file as written by SAM\_TRACE.

\item[DEKKER*] This is the name of the DEKKER file as written by SAM\_DEKKER.

\item[SKY1*] These are the limits to the first sky band.  Note that these
limits are found by SAM\_FIND.  Two values are required.

\item[SKY2*] These are the limits to the second sky band.  Note that these
limits are found by SAM\_FIND.  Two values are required.

\item[SPEC\_RNG*] The user can restrict the spectral range of pixels to be
extracted. These are the lower and upper limits in pixel numbers.  Two values
are required.

\item[SKY\_ORD] This is the order of the polynomials to be fit to the sky
pixels.  One should be extremely careful about using orders higher than about
1 since the interpolation in the object pixels may involve areas where the
gradient of the polynomials becomes very large.

\item[NREJECT]  The sky polynomial fits are done iteratively with the worst
fitting pixel from each fit being rejected in each iteration.  This is the
number of pixels which will be rejected.  Again one should be careful to not
reject too many pixels as it may be impossible to get a good fit.

\item[OUTPUT] This the output file for the sky subtracted spectral frame.

\end{description}

\subsection{SAM\_STRT}

This routine will take a frame with a distorted spectrum and straighten it. To
do this it needs the help of TRACE file.  Note that the interpolation required
will affect the variance estimates such that they are no longer independent of
each other.  Thus an optimal extraction should not be  performed on a frame
which has been straightened with this routine.  Note also that only the data
within the limits defined by the DEKKER file and by the parameter SPEC\_RNG
will be processed.  The rest of the input data is discarded. The parameters
are:

\begin{description}

\item[INPUT] This is the name of the input frame with the spectrum to be
straightened.

\item[AXIS] This is the axis to which the spectra are parallel.  Answer `1' for
the x axis and `2' for the y axis.

\item[TRACE] This is the name of the TRACE file as written by SAM\_TRACE.

\item[DEKKER] This is the name of the DEKKER file as written by SAM\_DEKKER.

\item[SPEC\_RNG] The user can restrict the spectral range of pixels to be extracted.
These are the lower and upper limits in pixel numbers.  Two values are
required.

\item[OUTPUT] This is the name of the output file for the straightened
spectrum.

\end{description}

\subsection{SAM\_TRACE}

This is the routine which creates a TRACE file.  It generally only needs to be
run once since the distortion should not vary much from night to night.  For
best results it should be run on a spectrum with reasonably high  signal/noise
(e.g. a standard star).  The output file not only contains the polynomial
coefficients, but also information used in later programs to scale the
coordinates.  The parameters are:

\begin{description}

\item[INPUT] This is the name of the input frame with the spectrum to be fit.

\item[AXIS] This is the axis to which the spectra are parallel.  Answer `1' for
the x axis and `2' for the y axis.

\item[TRACE] This is the name of the TRACE file as written by SAM\_TRACE.

\item[TABLE] This is the colour lookup table for the plot

\item[LOW, HIGH] These are the low and high data values for the greyscale.

\item[BLOCK] The user may block the spectra in wavelength in order to increase
the signal to noise.  The value tells how many rows/columns will be added
together to make one block. It is recommended that this value be 1 except in
cases of very bad signal/noise.

\item[NITER] SAM\_TRACE attempts to locate the centroid of the spectral profile
by locating the median position.  This is the maximum number iterations allowed
for this algorithm to converge.

\item[TOL] This is the tolerance in pixels for the convergence of the median
finding routine.

\item[CLIP] This is the number of sigma for the clipping threshold.

\item[NCYCLE] This is the number of iterations in the actual polynomial
fitting.  Once cycle consists of centering, fitting a polynomial to the centres
and clipping bad points.

\item[SPEC\_RNG] The user can restrict the spectral range of pixels to be extracted.
These are the lower and upper limits in pixel numbers.  Two values are
required.

\item[NORD] This is the order of the polynomial to be fit to the centroids.

\end{description}

\subsection{SAM\_WHOLE}

SAM\_WHOLE is a routine to be used if the spectrum covers the whole of the
dekker and thus things such as sky-subtraction and profile fitting are
irrelevant (e.g. arc or flat field spectra).  All this routine does is to set
up information in the input file header.  SAM\_EXTRACT does the actual
extraction.  The parameters are:

\begin{description}

\item[INPUT] This is the name of the input frame.

\item[AXIS] This is the axis to which the spectra are parallel.  Answer `1' for
the x axis and `2' for the y axis.

\item[TRACK] This is TRUE if the a TRACE file is going to be used.

\item[TRACE] This is the name of the TRACE file as written by SAM\_TRACE.

\item[DEKKER] This is the name of the DEKKER file as written by SAM\_DEKKER.

\item[SPEC\_RNG] The user can restrict the spectral range of pixels to be
extracted. These are the lower and upper limits in pixel numbers.  Two values
are required.

\end{description}

\section {Acknowledgments and references}

Several of the routines in SAM are based on algorithms stolen from routines
written by T.R. Marsh.  The author wishes to thank Dr. Marsh publicly for his
generous permission to ransack his software.

\begin{tabbing}
\=\kill
\>Horne, K, 1986, {\it P.A.S.P.}, {\bf 98}, 609. \\
\>Marsh, T.R., 1989, {\it P.A.S.P.}, {\bf 101}, 1032. \\
\>Mukai, K, 1990, {\it P.A.S.P.}, {\bf 102}, 183. \\
\end{tabbing}

\end{document}
