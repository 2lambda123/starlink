\documentstyle{article} 
\pagestyle{myheadings}

%------------------------------------------------------------------------------
\newcommand{\stardoccategory}  {Starlink User Note}
\newcommand{\stardocinitials}  {SUN}
\newcommand{\stardocnumber}    {112.1}
\newcommand{\stardocauthors}   {P M Allan}
\newcommand{\stardocdate}      {14 February 1991}
\newcommand{\stardoctitle}     {JTMP -- Using the Job Transfer Manipulation Protocol}
%------------------------------------------------------------------------------

\newcommand{\stardocname}{\stardocinitials /\stardocnumber}
\renewcommand{\_}{{\tt\char'137}}     % re-centres the underscore
\markright{\stardocname}
\setlength{\textwidth}{160mm}
\setlength{\textheight}{240mm}     
\setlength{\topmargin}{-5mm}
\setlength{\oddsidemargin}{0mm}
\setlength{\evensidemargin}{0mm}
\setlength{\parindent}{0mm}
\setlength{\parskip}{\medskipamount}
\setlength{\unitlength}{1mm}

%------------------------------------------------------------------------------
% Add any \newcommand or \newenvironment commands here
%------------------------------------------------------------------------------

\begin{document}
\thispagestyle{empty}
SCIENCE \& ENGINEERING RESEARCH COUNCIL \hfill \stardocname\\
RUTHERFORD APPLETON LABORATORY\\
{\large\bf Starlink Project\\}
{\large\bf \stardoccategory\ \stardocnumber}
\begin{flushright}
\stardocauthors\\
\stardocdate
\end{flushright}
\vspace{-4mm}
\rule{\textwidth}{0.5mm}
\vspace{5mm}
\begin{center}
{\Large\bf \stardoctitle}
\end{center}
\vspace{5mm}

%------------------------------------------------------------------------------
%  Add this part if you want a table of contents
%  \setlength{\parskip}{0mm}
%  \tableofcontents
%  \setlength{\parskip}{\medskipamount}
%  \markright{\stardocname}
%------------------------------------------------------------------------------

\section{Introduction}

The JTMP software is a component of the Coloured Books software in addition to
PAD, CBS-mail and TRANSFER. It is also  known as Red Book software. It allows
you to submit batch jobs to run on remote computers (not necessarily VAXs) and
to copy files between computers. It is very flexible and is very powerful, but
as with most such things, it can be difficult to get over the first bump of the
learning curve and make it do anything at all.

The software is documented in the DEC document `JTMP User Guide', however,
experience has shown that this manual is fairly incomprehensible and some of
the examples are simply wrong. The purpose of this document is to give a brief
description of what JTMP can do and to provide some examples that have been
proven to work. For further information, despite the caveat above, you are
referred to the DEC documentation.

\section{Submitting remote batch jobs}

Let us begin by considering how to submit a batch job to run on a remote
computer system. This first example performs a fairly simple operation, but
serves to illustrate what can be achieved. The task is simple in the JTMP sense
in that it submits a batch job to run on a remote computer with only a single
input file and a single output file (the log file from the batch job). The
contents of the batch job input file could range from the most trivial (just a
{\tt SHOW SYSTEM} command) to a complex set of VMS commands, but that is not
particularly important here. 

JTMP tasks are controlled by a JTMP input file (with default file type of {\tt
.JIN}), of which the following is an example.

\vspace{5mm}

\verb#%TO/SITE=UK.AC.LEICESTER.STARLINK/NODEFAULT                                         # \fbox{1}\\
\verb#%SINK/NAME=-JOBMILL                                                                 # \fbox{2}\\
\verb#%URGENCY/VALUE=LOW                                                                  # \fbox{3}\\
\verb#%AT/SITE=UK.AC.LEICESTER.STARLINK/USERNAME=/PASSWORD=                               # \fbox{4}\\
\verb#%REPORT/SINK=-MAIL                                                                  # \fbox{5}\\
\verb#%FILE/NAME=/LOG                                                                     # \fbox{6}\\
\verb#%BEGIN/TARGET=UK.AC.MANCHESTER.ASTRONOMY.STARLINK/SPAWN=SUCCESS/LOCAL_SINK=-FILE    # \fbox{7}\\
\verb#%FILE/OUTPUT/FORMAT=FORMATTED/SINK=LEICESTER_OUTPUT.LIS                             # \fbox{8}\\
\verb#%END                                                                                # \fbox{9}\\

This particular file is called called {\tt LEICESTER\_BATCH.JIN}. To begin a
JTMP task, you type

\begin{verbatim}
$ SEND filename
\end{verbatim}

where filename is the name of the JTMP input file, so the above task would be
started by typing

\begin{verbatim}
$ SEND LEICESTER_BATCH
\end{verbatim}

{\tt Notes:}

\begin{enumerate}

\item The {\tt \%TO} command defines which site the job is being sent to. In
order to provide examples that have been thoroughly tested, the local site will
be UK.AC.MANCHESTER.ASTRONOMY.STARLINK throughout this document. The remote site
will be either UK.AC.LEICESTER.STARLINK when the remote machine is a VAX or 
UK.AC.MANCHESTER--COMPUTING--CENTRE.CMS for non-VAX examples. The {\tt
NODEFAULT} qualifier says that the task should not get default input from a
skeleton file (see section~\ref{skeletons}).

\item The {\tt \%SINK} command defines the device that is to receive your task.
It might be  a file, a printer or, as in this case, a batch queue.

\item The {\tt \%URGENCY} command defines the priority to be assigned to your
task at the remote site.

\item The {\tt \%AT} command lets you define the username and password to be
used at the remote site. While it is possible to give these in the file, it is
insecure to do so as someone else might read the file. The method used here of
giving a null string for your username and password will cause the system to
prompt you for them.

\item The {\tt \%REPORT} command specifies the type of reports that you will
receive on the progress of the task. In this case you will be sent mail
messages.

\item This {\tt \%FILE} command defines the input file for the task. This is
another example of the use of a null string and again the system will prompt
you for the name of the file. This lets you use the same {\tt .JIN} file to
send many different tasks, although they would have to be to the same remote
site. In this case there is only one input file that will be run as a batch job
on the remote machine. A trivial example would be for this file to just contain
the line 

\begin{verbatim}
$ SHOW SYSTEM
\end{verbatim}

\item This is the beginning of what JTMP calls a proforma. It specifies how to
handle the output from a task. The {\tt \%BEGIN} command specifies where to
send the output of the task with the TARGET qualifier. In principle it could go
to a site other than the one that originated the task. It says that this
proforma should be spawned (which means generated, rather than a DCL spawn) if
the task is successful and the output should be returned to a local file. The
name of the file is specified on the next line.

\item This {\tt \%FILE} command gives the name of the file that will receive the
output of the task. In this case it is the log file of the batch job.

\item The {\tt \%END} command indicated the end of the definition of the
proforma, not the end of the task. It is a closing bracket to go with the {\tt
\%BEGIN} command.

\end{enumerate}

When you start a task, you will receive a message like {\tt Task 12345 has been
queued}. As the task progresses you will receive several mail messages from the
JTMP\_Manager saying that the task has arrived at its destination, has been
accepted for processing, etc. In this example the log file from the batch job
will be returned as {\tt LEICESTER\_OUTPUT.LIS} in your top level directory.

\subsection{Submitting batch jobs to non-VAX systems}

JTMP is not a VMS specific protocol so any system that has the software
installed can be accessed with it. Here is an example of submitting a batch job
to run a simple Fortran program on the Amdahl VP1100 vector processor at the
Manchester Computing Centre. For those more familiar with Crays, the syntax of
the input file may not be completely transparent, but it should be fairly
obvious what is happening. The JTMP input file is called {\tt VP.JIN} and
contains the following:

\begin{verbatim}
%TO/SITE=UK.AC.MCC.CMS/NODEFAULT
%SINK/NAME=-JOBMILL(VP)
%AT/SITE=UK.AC.MCC.CMS/USER=VMNETWRK
%REPORT/SINK=-MAIL/SELECT=(NORMAL_TERMINATION,ABNORMAL_TERMINATION)
%FILE/NAME=VP.JOB/LOG
%BEGIN/TARGET=UK.AC.MANCHESTER.ASTRONOMY.STARLINK/LOCAL_SINK=-FILE()/SPAWN=SUCCESS
%FILE/SINK=VP.OUT/OUTPUT/FORMAT=FORMATTED
%END
\end{verbatim}

The differences (other than the sites names) between this job input file and
the previous one are due to the fact that this example is taken directly from a
standard command procedure on the MCC VAXcluster.

The file that is sent to the VP is called {\tt VP.JOB} and contains the
following:

\begin{verbatim}
//jobname JOB account,personal_id,CLASS=A,USER=username,PASSWORD=password
//*VP1
//STEP1   EXEC  FMSCLG
//C.SYSIN DD    *
      DO 1 I=1,20
      PRINT *,I
    1 CONTINUE
      END
/*
\end{verbatim}

The lines beginning with // are JCL commands and cause the Fortran program to be
compiled, linked and run. The output from the batch job is returned in the file
{\tt VP.OUT}. In this case the job input file does contain an explicit
username, but this is only because of the way that the MCC system works. The
security is provided by the need to specify a username and password on the {\tt
//~JOB} command in the second of the above files.

\section{Sending files to remote sites}

To send a single file to a remote computer, you can use the {\tt TRANSFER}
command. Unfortunately this does not accept wild cards so multiple files
cannot be sent in one transfer. An example of sending a single file using 
JTMP follows.
The job input file {\tt SEND\_FILE.JIN} contains the following JTMP commands:

\begin{verbatim}
%TO/SITE=UK.AC.LEICESTER.STARLINK/NODEFAULT
%SINK/NAME=-FILE
%URGENCY/VALUE=LOW
%AT/SITE=UK.AC.LEICESTER.STARLINK/USERNAME=/PASSWORD=
%REPORT/SINK=-MAIL
%FILE/NAME=ARTICLE.TEX/SINK=COPY_OF_ARTICLE.TEX
\end{verbatim}

The task is executed with the command

\begin{verbatim}
$ SEND/TASK=DOCUMENT SEND_FILE
\end{verbatim}

and you will be prompted for your username and password at the remote site. This
will copy the file {\tt ARTICLE.TEX} from your current directory to a file
called {\tt COPY\_OF\_ARTICLE.TEX} in your default directory at Leicester. Note
the {\tt /TASK} qualifier. The task will fail if this is not included as this
is a document disposition task and the default is to perform an execution task
(i.e. to submit a batch job). Of course, it would be much simpler in this case
to use the {\tt TRANSFER} command, however, if you wanted to transfer several
files, you can do it as follows:

\begin{verbatim}
%TO/SITE=UK.AC.LEICESTER.STARLINK/NODEF
%SINK/NAME=-FILE
%URGENCY/VALUE=LOW
%AT/SITE=UK.AC.LEICESTER.STARLINK/USERNAME=/PASSWORD=
%REPORT/SINK=-MAIL
%FILE/NAME=*.COM/SINK=*.TST
\end{verbatim}

This will copy all of your {\tt .COM} files to Leicester. The resultant
filenames at Leicester will be the same as at your local site, but they will
have an extension of {\tt .TST} instead of {\tt .COM}.

\section{Fetching file from remote sites}

Here is an example of copying a file from a remote site to your local site. The
remote file being copied is described in the lines 7 -- 9 of the job input
file.The file name at the remote site is {\tt DOCUMENT.TWO} and the name of the
file that will be created at the local site is {\tt LEICESTER.LIS}. The file
will appear in your top level directory by default. The odd thing about this
file is line 6. We appear to be unnecessarily copying a file to the remote
site. This line is needed because it is invalid for a JTMP task to have an
output proforma without an input document. This seems pretty silly in this
context, but is the way that the system works.

\begin{verbatim}
%TO/SITE=UK.AC.LEICESTER.STARLINK/NODEFAULT
%SINK/NAME=-FILE
%URGENCY/VALUE=LOW
%AT/SITE=UK.AC.LEICESTER.STARLINK/USERNAME=/PASSWORD=
%REPORT/SINK=-MAIL
%FILE/NAME=LOGIN.COM/SINK=RUBBISH.TMP
%BEGIN/TARGET=UK.AC.MANCHESTER.ASTRONOMY.STARLINK/LOCAL_SINK=-FILE
%FILE/NAME=DOCUMENT.TWO/SINK=LEICESTER.LIS/LOG
%END
\end{verbatim}

The task is executed with the command

\begin{verbatim}
$ SEND/TASK=DOCUMENT FETCH_FILE
\end{verbatim}

The effect of this task is to copy the file {\tt LOGIN.COM} from the current
directory to the remote site and to copy the file {\tt DOCUMENT.TWO} to the
local file {\tt LEICESTER.LIS}. An example of the mail messages that are
received when performing this task is given in appendix~\ref{mail}.

A single file can always be fetched from a remote computer, using the  {\tt
TRANSFER} command. Unfortunately this does not accept wild cards so multiple
files cannot be fetched in one transfer. Multiple {\tt TRANSFER} commands may
be typed, but sometimes you do not know all the names of the files to be
fetched. In such a case you must use JTMP to fetch the files. Here is an
example of a {\tt .JIN} file that will fetch all of the {\tt .OUT} files from
the top level directory of your account on the remote computer.

\begin{verbatim}
%TO/SITE=UK.AC.LEICESTER.STARLINK/NODEFAULT
%SINK/NAME=-FILE
%URGENCY/VALUE=LOW
%AT/SITE=UK.AC.LEICESTER.STARLINK/USERNAME=/PASSWORD=
%REPORT/SINK=-MAIL
%FILE/NAME=SYS$LOGIN:LOGIN.COM/SINK=DUMMY.JTMP
%BEGIN/TARGET=UK.AC.MANCHESTER.ASTRONOMY.STARLINK/LOCAL_SINK=-FILE
%FILE/NAME=*.OUT/SINK=*.OUT
%END
\end{verbatim}

This is something that simply could not in general be achieved without using
JTMP, although if the remote machine was accessible via DECnet, then it would
be possible to use the COPY command.

\section{Skeleton files}
\label{skeletons}

If you find that you are using the same JIN file many times, but with minor
alterations between each use, then it may be appropriate to set up a skeleton
file. This is pointed to by the logical name JTP\$CUPBOARD (computer humour). A
skeleton file is a text library that contains modules which in turn contain the
commonly used JTMP commands to contact a certain sites. Essentially the modules
contain copies of the sort of job input files use in the examples with the
important exception that on the {\tt \%TO} command, the {\tt /NODEFAULT}
qualifier should be changed to {\tt /DEFAULT}. The module name in the skeleton
file are made up from the NRS name of the site being called and are something
like UK.AC.\-LEICESTER.\-STARLINK\_DEFAULT.

\section{Common problems}

There are a few problem that the author has found can easily catch out the
unwary. 

It is natural to want to use the short form for the NRS names of the sites that
need to be specified in the JIN file. Unfortunately the target of the output
proforma (specified on the BEGIN command in the first example) must be given by
the long form of the NRS name when using the current version (5.2) of the
software. It is easy to forget this, and if you do remember that a long name is
needed somewhere, it is even easier to forget exactly where it is. Consequently 
it is recommended that you always use the long form of names in JIN files.

A related problem with skeleton files is that module names in the skeleton file
should be made up from long NRS names. Short ones generally will not work as
the name gets converted to the long form before the skeleton file is opened.

Another problem is one specific to VAXclusters. If you send a batch job to a
site using JTMP, then the default batch queue for it to run on is {\tt
SYS\$BATCH}. If the remote node is part of a VAXcluster, then it is common for
{\tt SYS\$BATCH} to be a generic queue that points to several execution queues.
However, if the batch job is not run on the CPU that has JTMP installed, then
there can be problems in returning the output to the person who originated the
job. Sometimes it works and sometimes it does not. The exact reason for this is
not clear. The JTMP user guide says that the default execution queue can be
defined to be a queue other than {\tt SYS\$BATCH}, however, it is the
experience of the author that this does not work. It is better to specify an
explicit execution queue when installing the software and when using it.


\appendix
\section{Example mail messages}
\label{mail}

These are some examples of the sort of mail that is received from the JTMP
manager. This example concerned the fetching of a file from Leicester to
Manchester.

\begin{verbatim}

From:   JTMP_Manager 29-AUG-1990 11:53:22.53
To:     OPER        
CC:
Subj:   JTMP Report for Task : 0000018403

Task reference: 0000018403

Report from UK.AC.LEICESTER.STARLINK
Task type : DOCUMENT DISPOSITION
Task has been accepted.

Report message : 
JTMP task analysed and accepted

\end{verbatim}
\rule{\textwidth}{0.5mm}
\begin{verbatim}

From:   JTMP_Manager 29-AUG-1990 11:53:29.31
To:     OPER        
CC:
Subj:   JTMP Report for Task : 0000018403

Task reference: 0000018403

Report from UK.AC.LEICESTER.STARLINK
Task type : DOCUMENT DISPOSITION
Task successfully created.
Target site : UK.AC.MANCHESTER.ASTRONOMY.STARLINK

Report message : 
Task name "UNNAMED" spawned

Report from UK.AC.LEICESTER.STARLINK
Task type : DOCUMENT DISPOSITION
Task terminated normally.
Number of descriptors spawned : 1

Report message : 
normal successful completion

\end{verbatim}
\rule{\textwidth}{0.5mm}
\begin{verbatim}

From:   JTMP_Manager 29-AUG-1990 11:53:44.56
To:     OPER        
CC:
Subj:   JTMP Report for Task : 0000018403

Task reference: 0000018403
Report from local host

Arrival report message:
 JTMP task analysed and accepted

\end{verbatim}
\rule{\textwidth}{0.5mm}

\end{document}
