\documentstyle[11pt]{article}
\pagestyle{myheadings}

%------------------------------------------------------------------------------
\newcommand{\stardoccategory}  {Starlink User Note}
\newcommand{\stardocinitials}  {SUN}
\newcommand{\stardocnumber}    {116.2}
\newcommand{\stardocauthors}   {D.S. Berry}
\newcommand{\stardocdate}      {15th April 1992}
\newcommand{\stardoctitle}   {MEMCRDD --- Maximum Entropy Imaging of IRAS CRDD}
%------------------------------------------------------------------------------

\newcommand{\stardocname}{\stardocinitials /\stardocnumber}
\renewcommand{\_}{{\tt\char'137}}     % re-centres the underscore
\markright{\stardocname}
\setlength{\textwidth}{160mm}
\setlength{\textheight}{230mm}
\setlength{\topmargin}{-2mm}
\setlength{\oddsidemargin}{0mm}
\setlength{\evensidemargin}{0mm}
\setlength{\parindent}{0mm}
\setlength{\parskip}{\medskipamount}
\setlength{\unitlength}{1mm}

%------------------------------------------------------------------------------
% Add any \newcommand or \newenvironment commands here
\newcommand{\micron}{$\mu m$}
\newcommand{\chisq}{$\chi^2$}
%------------------------------------------------------------------------------

\begin{document}
\thispagestyle{empty}
SCIENCE \& ENGINEERING RESEARCH COUNCIL \hfill \stardocname\\
RUTHERFORD APPLETON LABORATORY\\
{\large\bf Starlink Project\\}
{\large\bf \stardoccategory\ \stardocnumber}
\begin{flushright}
\stardocauthors\\
\stardocdate
\end{flushright}
\vspace{-4mm}
\rule{\textwidth}{0.5mm}
\vspace{5mm}
\begin{center}
{\Large\bf \stardoctitle}
\end{center}
\vspace{5mm}

%------------------------------------------------------------------------------
%  Add this part if you want a table of contents
\setlength{\parskip}{0mm}
\tableofcontents
\setlength{\parskip}{\medskipamount}
\markright{\stardocname}
%------------------------------------------------------------------------------

\newpage

\section {What MEMCRDD does}

The MEMCRDD program maps destriped IRAS CRDD (Calibrated Reconstructed Detector
Data) into a ``high resolution'' two dimensional image of the sky using a
Maximum Entropy Method (MEM) algorithm. The program is based on the MEMSYS3
subroutine package produced by S.Gull and J.Skilling of Maximum Entropy Data
Consultants Ltd. at Cambridge. For more information on MEMSYS3, see SUN 117 and
the MEMSYS3 users manual. For more information on the IRAS CRDD, see the IRAS
Catalogs and Atlases Explanatory Supplement, which describes how the data was
acquired and potential defects to be aware of. For a historical review of
Maximum Entropy in  Astrophysics, see Narayan and Nityanada (Ann. Rev. A\&A,
24, 1986).

The resolution enhancement over images produced by the simple co-addition of
data samples is bought at the cost of {\em much} greater CPU requirements. A
typical run may take between 5 and 20 hours of CPU on a microVAX 3400,
although this obviously depends on the amount of CRDD being mapped and the
pixel size in the final image. It is usually a good idea to create a low
resolution image by the normal co-addition technique before going on to use
MEMCRDD. This can help to identify any likely problems that may crop up in the
use of MEMCRDD, and give something with which to compare the MEMCRDD  results.


\section {MEMSYS3}

This section contains a extremely simplified outline of the MEMSYS3 algorithm.
For a more complete description see the MEMSYS3 users manual.

The MEMSYS3 algorithm is fully Bayesian and aims to produce {\em the most
probable} image, given the data. It uses the concept of a {\em prior
probability} $P(f)$ for any image $f$. $P(f)$ gives the initial probability
that image $f$ represents the ``true'' sky {\em before any data relating to
that part of the sky is taken into account}. If no data is available to
estimate the image $f$, then there is no evidence to assign any area of the
image a different value to any other area. Thus, the prior probability of an
image decreases with the  amount of structure present in the image. In other
words, {\em simple images are  ``more believable'' than complex images}.
MEMSYS3 uses an entropy function for  this prior probability, in which zero
entropy is defined by an image which can  be supplied by the user (the
``entropy model''), but which is usually taken to  be a flat surface equal to
the mean of the data. Any image which deviates from  this entropy model has a
negative entropy value, the greater the deviation the  more negative the
entropy.

The other key quantity in the MEMSYS3 algorithm is the {\em likelihood}. This
is  the probability that the observed data $D$ could have been generated from a
given image $f$. The probability of $D$ given $f$  is written as $P( D \mid f
)$ where the ``$\mid$'' character means ``given''. The likelihood is based on a
model of the noise statistics in the data, and on a model of the response
function of the experimental system. Given an image, the response model can be
used to generate the data which {\em would} have been observed (in the absence
of noise) if the true sky looked exactly like the given image. The noise model
can then be used to find the probability that all the deviations of  this
simulated data from the real data could be explained as noise. As an  example,
if $F$ is the simulated data from some image $f$, and $D$ is the  corresponding
real data, subject to Gaussian noise of standard deviation  $\sigma$, then

\begin{equation}
P(D \mid f) = (2\pi\sigma^{2})^{-0.5}.e^{-(D-F)^{2}/2\sigma^{2}}
\end{equation}

What is needed to do the reconstruction is not $P( D \mid f)$ but
$P( f \mid D)$ i.e. the probability of the image $f$ given the data $D$.
Bayes theorem can be used to calculate $P( f \mid D )$ as follows:

\begin{equation}
P(f \mid D) = P(f)*P(D \mid f)/P(D) \label{EQ:BAYES}
\end{equation}

$P(f)$ is the prior probability of image $f$, $P( D \mid f )$ is the likelihood
of  the data given $f$, and $P(D)$ (known as the data ``evidence'') is a
constant  which normalises $P( f \mid D )$ to a sum of unity, and gives the
probability of  the data.

MEMSYS3 is an iterative algorithm, each iteration producing a new version of
the reconstruction image $f$. It starts with $f$ equal to the zero entropy
image.  This gives a high prior probability, but will in general give a {\em
very} low likelihood, thus the probability of the reconstruction $P(f \mid D)$
will  also be low. Each successive iteration introduces more structure into the
reconstruction, moving it away from the entropy model and towards the data. As
this happens the prior probability goes down but the likelihood goes up. At
some  point a peak occurs in the product of the two, and the corresponding
reconstruction is thus the most probable image. If the algorithm were to be
continued beyond this peak the extra complexity introduced into the image would
causes the  prior probability to drop faster than the likelihood is rising and
so the overall  probability would go down. In practice, going beyond the peak
results in noise  being interpreted as real structure.

A crucial point to understand when interpreting MEMSYS3 results is that the
image returned is only the most probable. There could be other images only
{\em slightly} less probable than the one returned. In fact MEMSYS3 does, in
effect, keep track of the probability of {\em all} images. If inferences about
the reconstruction are to made (such as the integrated flux in some region)
then all images should really be taken into account, weighting the
contributions to the final answer by the probabilities $P( f \mid D)$. MEMSYS3
provides a facility to do just this. A mask image can be given which is
multiplied by the reconstruction. The best estimate of the total data sum in
the product is returned together with the standard deviation on the total data
sum. This uncertainty represents the spread in values between the different
plausible images.

\section {The MEMCRDD program}
This section describes the way in which MEMCRDD uses the MEMSYS3 package.

\subsection {The noise model}
\label {SEC:NOISE}
The MEMSYS3 Gaussian noise option is used (the other option is Poisson noise).
MEMCRDD calculates an estimate of the standard deviation of each CRDD sample
separately, to allow variations in the noise level across the image. These
noise levels can be calculated in two ways:
\begin{enumerate}

\item The noise on each sample is made up from two components; the ``field''
noise which is constant for all samples from a given detector data stream, and
the ``source'' noise which is related to the flux level in each sample. The
variances of these two components are added to get the variance of the total
noise.

The field noise used for each detector data stream is the RMS value of
the residuals between the data stream and a smoothed form of the data stream.
Point sources are not included in this process.

The source noise takes account of the uncertainties in the detectors PSFs, and
is given a variance which is a constant times the square of the sample value.
This constant can be set by the user (see parameter K) but is defaulted to a
value close to 5\% (the exact value depends on the IRAS wave band of the data
and is based on the uncertainty quoted by IPAC for the detectors effective solid
angles).

\item The noise on each sample is made up of 4 components.
\begin {itemize}
\item The field noise, calculated as above.
\item ``Correlated noise''. i.e. structure on a scale larger than a single
sample which is {\em not} present in every coverage of a given region. The
presence of correlated noise can be seen by forming low resolution maps of each
HCON (or even each scan within an HCON) and comparing the overlapping regions.
Low level background structure can often be seen in one coverage which is not
there in another. The origin of such structure is not clear. Some will be due to
solar system objects which can move between scans, and some will be due to
residual striping. The low-pass electronic filter on board IRAS could also
account for some of the correlated noise.

The correlated noise is estimated as follows:
\begin{itemize}

\item A low resolution image is formed from all the input scans using a simple
co-addition algorithm. Features which are not present in all scans will be
attenuated in this image due to the averaging process.

\item A set of low resolution images are formed in which each image contains
data from only one scan.

\item A set of images containing the squared residuals between each single-scan
image and the all-scan image are formed. These residuals will be high for
features which are present in the single-scan image but which are attenuated in
the all-scan image.

\item The OPUS routine is applied to each residual image to create something
which looks like a simulated data set, but in which each sample is actually
an estimate of the mean variance between the single-scan image and the
all-scan image over the area of the corresponding data sample.

\end{itemize}

\item The uncertainty caused by errors in the estimates of the detector Point
Spread Functions.
\item The uncertainty caused by pointing errors.
\end{itemize}
The last two of these can only be calculated if an estimate of the high
resolution sky is already available. This leads to an ``iterative'' use of
MEMCRDD, in which the result from one run of MEMCRDD is used to refine the noise
model in the next run (see parameter SKYIN).

\end{enumerate}

If a null value is given for parameter SKYIN, then the first type of noise
model is  used, otherwise the image given for the SKYIN parameter is used as
the basis for  calculating the second type of noise model. The second, more
complex, noise  model was provided because the simple model tended to produce
spurious structure  in the reconstruction. MEMSYS3 assumes that the noise on
each data sample is  uncorrelated with that of its neighbours. This is not true
with IRAS data.  That correlated noise exists can be seen by producing images
from several HCONs separately. They often contain structure which is present in
one HCON but not  another. Left to itself MEMSYS3 tends to interpret this
correlated noise as  real structure. For this reason a correlated noise
component is included in the  complex model, and the MEMSYS3 automatic noise
scaling feature is switched  off (see parameter METHOD).

\subsection {The entropy model}
The image used to define the entropy model can be given by the user in the form
of an EDRS image (see parameter MODEL). If no such image is provided a flat
model equal to the mean of the data is used.

\subsection {The detector response functions}
The generation of simulated IRAS data is done by convolving a given image with
a set of mean detector PSFs. The individual detector PSFs used are those
produced by Mehrdad Moshir at IPAC. These PSFs are contained in the 3
dimensional image stacks IPAC1.BDF, IPAC2.BDF, IPAC3.BDF and IPAC4.BDF
(for the 12, 25, 60 and 100 $\mu$m bands respectively) contained in the EDRSX
directory.
The way in which the detectors are grouped
to produce the mean PSFs can be controlled by the parameter DGROUPS. The
convolutions are done using Fourier transform techniques.

\subsection {The Intrinsic Correlation Function}
\label {SEC:ICFA}

The use of a Maximum Entropy prior probability function is only valid if the
image has no pixel-to-pixel correlations. This is usually not the case with
most images; there is usually considerable correlation between pixel values.
That is to say, if a high resolution, noise free image could be obtained of
some area of the sky, there would still be a tendency for neighbouring pixels
to have similar values, just because of the extended structure in the image. If
this correlation is not taken into account, MEMSYS3 will tend to produce a
``spikey'' image in which noise is interpreted as real structure. Gull and
Skilling have proposed using an ``Intrinsic Correlation Function'' (ICF) as a
means of including prior knowledge about spatial correlations in the
reconstruction process. This works by defining a ``hidden'' image. This is the
image which gives the required image when convolved with the ICF. It is assumed
that the hidden image has no spatial correlation. Instead of maximising the
entropy of the required reconstruction, MEMSYS3 maximising the entropy of the
hidden image. The result is that the final reconstruction tends to be smooth on
the scale of the ICF. MEMCRDD provides an option to use a Gaussian ICF. The
user specifies the full width at half maximum of this Gaussian (see parameter
FILTER).


\section {How to run MEMCRDD}
\label {SEC:RUN}
Before using MEMCRDD, suitable input data must be obtained. The
CRDD should
first be extracted from the tape archives using program I\_SNIP\_CRDD (see SUN
91), or a request should be sent to RLVAD::IRASMAIL for the required
data.
Next the data should be destriped. There are several ways of doing this, the
simplest being to put each scan through the I\_DESTRIPE program (see SUN 91).
However, since poor destriping can seriously degrade the quality of the final
high resolution image, it is usually advisable to use a more sophisticated
destriping technique such as the iterative destriping technique based on the
EDRSX program CRDDSAMPLE. Use of CRDDSAMPLE is described in the EDRSX Cookbook,
contained in the on-line EDRSX help library.

MEMCRDD is currently implemented to run under the Starlink INTERIM environment,
and thus can be run either from DSCL or by using the RUNSTAR command. In view
of the fact that MEMCRDD is usually run in batch (due to its large CPU time
requirements), and also in view of the large number of parameters associated
with the program, it is usually more convenient to use the RUNSTAR command. A
convenient way of running MEMCRDD is as follows

\begin{enumerate}

\item Create a copy of the text file EDRSX:MEMCRDD.CON in your own directory
calling it by any name which identifies the particular run (eg JOB1.CON,
M51\_BIG.CON or whatever). Note, the name should have 8 or less characters.

\item Edit the file, assigning values to the required parameters. Each line
of the connection file starts with a parameter name followed by the parameter
type (recognised types are VALUE, FRAME(R), FRAME(W) or ERROR). The ``/''
character is used to separate these fields, so a typical line in the
connection file could start with:

\begin{verbatim}
   CRDDF1/FRAME(R)/
\end{verbatim}

If the line consisted of nothing more than this, the user would receive a prompt
for parameter CRDDF1 when the program was run. However, it is possible to put
the value to be assigned to the parameter into the connection file, and so
remove the need for the user to be prompted. This is very convenient for
programs run in batch mode.
To give a parameter a particular value, the
value should be added to the end of the parameter type and terminated with
``/''. For example,

\begin{verbatim}
   CRDDF1/FRAME(R)/ARP220_B3S2_DS/
\end{verbatim}

would give the frame type parameter CRDDF1 the value ARP220\_B3S2\_DS. A null
value can be specified for any parameter, e.g.

\begin{verbatim}
   CRDDF1/FRAME(R)//
\end{verbatim}

The null value is recognised by the program, which then decides what to do
about  it. Often, a null parameter value causes a ``run-time default'' value
calculated  by the program to be used for the parameter. Connection files can
use an  exclamation mark ``!'' as a comment delimiter. Any text occurring
between an exclamation mark and the end of the line is ignored when processing
the file.

For further information on specifying parameter values
for all INTERIM programs see SUN4 section 5.

\item Create a command file to contain the RUNSTAR command. This file will usually
look something like:

\begin{verbatim}
   $ SET DEFAULT DISK$USER1:[FRED.MYOBJECT.BAND1]
   $ DEFINE JOB1 EDRSX:MEMCRDD
   $ RUNSTAR JOB1
\end{verbatim}

The first line puts the batch job into the correct directory (i.e. the one
containing JOB1.CON, the users copy of MEMCRDD.CON). The second line specifies
the name of the connection file to use (in this case JOB1.CON). The third line
runs MEMCRDD (via the logical name defined on the second line). An  alternative
way of specifying parameter values is to place them on the ``command  line''.
For example, to assign the value 0.8 to the parameter RATE, and the  value
SKY\_IMAGE to parameter SKYIN, the command line would be:

\begin{verbatim}
   $ RUNSTAR JOB1/RATE=0.5/SKYIN=SKY_IMAGE
\end{verbatim}
Parameter values assigned in this way take priority over values stored in the
connection file.

The above file containing the RUNSTAR command is submitted to batch using the
DCL SUBMIT command, e.g. if the file was called JOB1.COM, the command may
look like

\begin{verbatim}
   $ SUBMIT/NOTIFY/QUEUE=LONG JOB1.COM
\end{verbatim}

This would produce a log file in the users login directory called JOB1.LOG
which would contain any numerical diagnostics the user had requested by
selection of appropriate MEMCRDD parameter values.

\end{enumerate}

MEMCRDD has many program parameters, which provide many facilities. Most of
them  have default values which can be used initially. To start with, the user
need  only assign values to the following parameters:

\begin {description}

\item [ANALOUT] - A file to hold information needed to calculate integrated
fluxes  and uncertainties. It also holds information which can be used to
restart a job  which is interrupted for any reason. Be warned, the file could
be several  thousand blocks in extent.

\item [BOXSIZE] - The size of the square area to be mapped, in arc-minutes. The
final image will be slightly bigger than the requested box size, since a blank
border must be added to the image to avoid wrap-around problems when smoothing
the image.

\item  [CRDDF1 - CRDDF20] - Up to 20 survey CRDD files to be mapped. These must
have been previously destriped. Poor destriping is a major cause of poor
reconstructions. The limit of 20 is imposed to limit the amount of virtual
memory needed to run MEMCRDD. The figure of 20 could be increased in future
if available virtual memory quotas increase. If MEMCRDD crashes with a
mysterious "quota exceeded" message, it will probably mean that you need a
higher page file quota. Ask your local manager if your quota can be raised.

\item [DEC\_DEG] - The degrees field of the declination at the centre of the
mapped square.

\item [DEC\_MIN] - The arc-minutes field of the declination at the centre of
the  mapped square.

\item [DEC\_SEC] - The arc-seconds field of the declination at the centre of
the  mapped square.

\item [HIRES] - The name of the output image.

\item [RA\_HRS] - The hours field of the right ascension at the centre of the

\item [RA\_MINS] - The minutes field of the right ascension at the centre of the
mapped square.

\item [RA\_SECS] - The seconds field of the right ascension at the centre of
the  mapped square.

\end {description}

With these parameter values, MEMCRDD will use MEMSYS3 in ``Classic'' mode with
automatic scaling of the initial noise values (calculated by the simple method
described in section \ref {SEC:NOISE} ). For information about the meaning of
all MEMCRDD parameters and suitable choices of value for each, see appendix
\ref {APP:PARS}.

A DCL command procedure (called MEMJOB) exists for running MEMCRDD iteratively
in order to use the more sophisticated noise model described in paragraph 2 of
section \ref {SEC:NOISE}. It is recommended that people wishing to use MEMJOB
should examine the file EDRSX:MEMJOB.COM to see the details of procedure.
Basically, the procedure first produces a low resolution image (using the LORES
option of parameter FUNCTION), and then runs MEMCRDD in HISTORIC mode, using
the low resolution image to define the noise model (see parameter SKYIN). The
map created by this run of MEMCRDD is then used to define the noise model in a
further run. MEMCRDD is run repeatedly in this fashion for a maximum of six
times. If the noise model converges before this, then the loop is left before
six iterations have been completed. Finally, another run is performed in
CLASSIC (non-automatic) mode to produce the final map.

To run MEMJOB, a local copy of MEMCRDD.CON should be obtained and values for
the parameters BOXSIZE, CRDDF1-CRDDFn, DEC\_DEG, DEC\_MIN, DEC\_SEC, RA\_HRS,
RA\_MINS, RA\_SECS should be edited into the local copy. MEMJOB is then
initiated (usually from batch) by the DCL command

\begin{verbatim}
   $ @EDRSX:MEMJOB <name>
\end{verbatim}

where \verb+<name>+ is the name of the local copy of MEMCRDD.CON (without the
file type). The following files are produced:

\begin{description}
\item [\verb+<name>+\_LO.BDF] --- The low resololution image.
\item [\verb+<name>+\_CO.BDF] --- The coverage image (see parameter COVER).
\item [\verb+<name>+\_HIi.BDF (i=1-6)] --- The maps from each HISTORIC run of
MEMCRDD.
\item [\verb+<name>+\_NON.BDF] --- The final CLASSIC (non\_automatic) map.
\item [\verb+<name>+\_AN.BDF] --- The analysis file from the final CLASSIC run.
Used by the ``inference facility'' (see section \ref {SEC:INFER}).
\item [\verb+<name>+\_VAR.BDF] --- The variances used in the final CLASSIC run.
See parameter VAROUT.
\end{description}

\section {Interpreting the results}
\subsection {Using the MEMSYS3 inference facility}
\label {SEC:INFER}

Care should be taken when interpreting structure in MEMCRDD images. The
returned  image is one sample from the total ``probability bubble'' of all
images, and  will generally contain a certain amount of ``spurious'' structure.
Before making  conclusions based on structure visible in the image, the
significance of the  structure should be checked by using the MEMSYS3 inference
facility. An  interface to this is provided by the ANALYSE function in MEMCRDD,
selected by the parameter FUNCTION. In ANALYSE mode, the user supplies an
``analysis  file'' generated by a previous CLASSIC or NONAUTO run of MEMCRDD
(see parameters ANALIN, ANALOUT and METHOD), and one of the following

\begin{itemize}
\item An EDRS image containing a ``mask''. If $f_{j}$ is the $j$th pixel of
the hi-res image generated by MEMCRDD, and $M_{j}$ is the $j$th pixel of the
supplied mask image, then an analysis consists of calculating $\rho$ and
$\delta\rho$ where:

\begin{equation}
\rho=\sum_{j=1}^{Npixels} f_{j}*m_{j}
\end{equation}

and $\delta\rho$ is the standard deviation on $\rho$. $\rho$ is converted to  a
flux measurement in units of Janskys by multiplying it by the solid angle of a
pixel. The total data sum in the supplied mask is also displayed, allowing the
user to calculate the effective solid angle in the mask and so convert the
integrated flux to a mean surface brightness.

\end{itemize}

or

\begin{itemize}
\item An EDRS XY list describing a polygonal sub-region of the hi-res image. A
mask  image is generated from this XY list by setting all pixels inside the
polygon to  a value of unity, and all other pixels to a value of zero. This
mask image is  then used in the same way as a mask image supplied by the user.
XY lists can be  generated by programs XYCUR, XYKEY etc.
\end{itemize}

A common use of a mask image is to calculate the total flux in some source
region after subtracting off a background surface brightness. An analysis
returns the total source flux $\rho$ with an error estimate $\delta\rho$.
Comparison of these two quantities can determine if the source is ``real'' or
if  it is an artifact of MEM. A mask to do this has the value 1.0 in the source
area, a value of $-\Delta$ in area used to define the background surface
brightness, and a value of zero everywhere else. $\Delta$ should be equal to
the  ratio of the number of pixels in the source area to the number of pixels
in the  background area. This ensures that the mask has a total data sum of
zero.

A DSCL command procedure has been written to create such a mask. It is called
MEMASK and is run by typing

\begin{verbatim}
   $ DSCL GO EDRSX
   EDRSX> MEMASK
\end{verbatim}

The procedure gives the following prompts:

\begin {description}
\item [MEM image] - Give the name of the hi-res image generated by MEMCRDD.
\item [Output mask image] - The name of file to receive the output mask image.
\end {description}

MEMASK then displays the hi-res image. The user is asked to use the cursor to
define a polygonal area to use as the source region, and a polygonal area to use
as the background region. It is allowable for this background region to
intersect or contain the source region. When the mask has been created, the
total data sum within the mask is displayed. This should be close to zero.

Having created a mask image using MEMASK. MEMCRDD should be run again, using
the same connection file that was used to create the hi-res image and analysis
file. This run of MEMCRDD will take about the same time as a single MEMSYS3
iteration. As this can become considerable towards the end of a reconstruction,
it is recommended that it is done in batch. If the original .COM file used to
run MEMCRDD was:

\begin{verbatim}
   $ SET DEFAULT DISK$USER1:[FRED.MYOBJECT.BAND1]
   $ DEFINE JOB1 EDRSX:MEMCRDD
   $ RUNSTAR JOB1
\end{verbatim}

then a new .COM file should be created containing the lines

\begin{verbatim}
   $ SET DEFAULT DISK$USER1:[FRED.MYOBJECT.BAND1]
   $ DEFINE JOB1 EDRSX:MEMCRDD
   $ RUNSTAR JOB1/FUNCTION=AN/ANALIN=<analysis file>/MASK=<mask>
\end{verbatim}

where \verb+<analysis file>+ is the name of the analysis file generated by
MEMCRDD and \verb+<mask>+ is the name of the mask generated by MEMASK. This
.COM file should then  be submitted to batch as before.

\subsection {MEMCRDD diagnostics}
\label {SEC:DIAG}
If the ILEVEL parameter (see appendix \ref {APP:PARS} for information on ILEVEL
and all other MEMCRDD parameters) is given a value greater than or equal to 2,
then the following diagnostic values are displayed after each MEMSYS3 iteration:

\begin {description}
\item [OMEGA] - This is the ``rescaled termination criterion''. The
reconstruction is considered complete when OMEGA rises to close to 1.0.
\item [ENTROPY] - The entropy of the current reconstruction. This will always
be negative, thus the maximum entropy image has the smallest
absolute value for ENTROPY.

\item [TEST] - This indicates how far from the maximum entropy trajectory the
algorithm has strayed. The smaller the value of TEST the more accurately the
trajectory is being followed. Values approaching 1.0 are dangerous, and may
well  cause MEMSYS3 to crash with a ``floating overflow'' message. If this
happens the  RATE parameter should be lowered and MEMCRDD re-run. RATE controls
the size of  the step each iteration can produce. Large steps can allow the
algorithm to  wander from the trajectory.

\item [CHI-SQ] - This is only displayed if METHOD is HISTORIC or MCM. It is the
normalised $\chi^{2}$ statistic for the current reconstruction. It should fall
to close to 1.0 at termination. NB, reconstructions using the MCM method
rarely reach a $\chi^{2}$ of 1.0.
\item [Log(probability)] - This is only produced if METHOD is CLASSIC or
NONAUTO. It gives the data ``evidence'', that is, the log of the overall
probability of the
input data. The correct Bayesian reconstruction is the one which gives the
highest value for the evidence. The best values to use for MEMCRDD parameters
can in principle be found by re-running MEMCRDD many times with different
parameter values. The best values are the ones which maximise the evidence.
This usually requires an impractically large amount of CPU time.

\item [No. of transforms] - This is the number of times the transform routines
(OPUS and TROPUS) have been called. Each call to OPUS or TROPUS requires
several  Fourier transform operations to be performed. The number of calls to
OPUS and  TROPUS required to perform a single iteration varies with iteration
number.  The earlier iterations generally required fewer transforms than the
later ones.

\item [RATE] - If the RATE parameter is given a negative value by the user,
then  the value actually used for RATE starts at the absolute value, but is
allowed to  change between iterations. An attempt is made to change RATE to
keep TEST  between 0.5 and 0.05. If the RATE parameter is given a positive
value, the value used for RATE is fixed at that value.

\end {description}

When and if the reconstruction is completed, the user is told by what factor
MEMSYS3 chose to scale the initial noise values.

\subsection {Structured background noise in the reconstruction}
\label {SEC:ICFB}

MEMCRDD reconstructions often exhibit high levels of ``spurious'' structure in
background regions. This is partly due to the fact that a single sample has
been taken from the probability bubble of all images, rather than integrating
over the entire bubble. The reconstruction image should not be seen as the
``result'' of the MEM analysis. The real results are obtained by using the
inference facility as described in section \ref {SEC:INFER}. When this is done,
the results are produced by integrating over the entire probability bubble.

However, there are various methods that can be used to produce images which
show  less structured noise in the background. None of them are likely to be
completely effective, but they can often reduce the level of the spurious
structure to some extent.

\begin {itemize}
\item Use an ``Intrinsic Correlation Function'' (ICF). See section \ref
{SEC:ICFA}. One of the axioms of
MEM is that there are no spatial correlations in the true sky. This is usually
not true, and ignoring such correlations can produce structured  noise in the
final reconstruction. Prior knowledge about spatial correlations can be
incorporated into MEM by use of an ICF.

Instead of maximising the entropy of the required reconstruction, the entropy
of a ``hidden image'' is maximised. This hidden image is defined so that the
hidden image convolved with the ICF gives the required reconstruction. The
hidden image can then be assumed to have no spatial correlations and so MEM can
be applied to it. In effect, what this does is to pretend that the point spread
function of each detector was more extended than it actually was. The final
estimate of the hidden image is smoothed with the ICF to obtain the required
reconstruction, thus encouraging smoothness in the final image on the scale of
the ICF.

MEMCRDD provides an option to use a Gaussian ICF. The user specifies the full
width at half maximum of the ICF as the value for parameter FILTER. The best
value for FILTER can be found by running MEMCRDD several times with different
values and selecting the value which maximises the data evidence (see section
\ref {SEC:DIAG}).

\item The presence of a large dynamic range in the input data seems to
exacerbate the background noise. Therefore removing bright sources from the
input  data may help. Point sources can be removed by setting the DEGLIT
parameter to  1.0. This causes MEMCRDD to treat point sources like glitches.
This approach is  not always useful, since the region of interest is often
associated with bright  sources.

\item Background noise can be over-fitted due to limitations in the simple
noise model described in section \ref {SEC:NOISE}. MEMCRDD provides a facility
for using the second, more sophisticated noise model described in section \ref
{SEC:NOISE}. To do this, MEMCRDD must be run originally with the simple noise
model, and then re-run specifying the output image from the first run as the
value for parameter SKYIN. The calculation of pointing and PSF noise is based
on  the output from the previous run. MEMCRDD should ideally be run several
times,  giving the output from the previous run as the SKYIN value for the next
run,  until the noise model converges (as shown by the noise model diagnostics
displayed before starting iteration 1 of the reconstruction process).

\item It can happen that doing a ``noise loop'' as described above, can cause
the data evidence to drop rather than fall. This seems to be a manifestation of
the presence of correlated noise in the data, MEMSYS3 interprets this noise as
real structure and modifies the noise scaling factor to remove it from the
noise estimates. For this reason it is suggested that the automatic noise
scaling option be switched off by giving the value NONAUTO for the parameter
METHOD.

\item Another method which seems to decrease background noise, is to re-run
MEMCRDD several times, modifying the default model to incorporate structure
revealed in the previous run. The default model is normally a flat image, but
the user can override this by specifying an image for the MODEL parameter. This
image is then used as the default model.  In this way prior knowledge obtained
about an image by running MEMCRDD, can be used in a further run of MEMCRDD.

This can be thought of as follows; the first run of MEMCRDD, using a flat
model, reconstructs the image assuming nothing about what structure it will
find in the image. This process reveals that (for instance) there is a bright
source in the middle of the image. If MEMCRDD had known {\em a priori} that
there was a source in the middle of the image, it would have been able to make
a rather better job of the reconstruction. What is needed is some means of
including this prior information in a second run of MEMCRDD.

One could for instance find the integrated flux of the source from the original
reconstruction, and assume that the source was a {\em point} source with this
flux. A model image could be created containing such a point source and MEMCRDD
re-run using this model rather than the flat model. The source in the resulting
image would be as close to a point source as possible within the constraints
of the data.

Another method which seems to be successful to a certain extent, is to run
MEMCRDD with a flat model and then create a new model in which each pixel is a
linear combination of the old model, and the reconstruction produced using the
old model. The proportion of one to the other is varied on the basis of the
entropy of each pixel in the reconstruction. Model pixels which correspond to
reconstruction pixels with a large entropy, are moved closer to the
reconstruction than those corresponding to pixels with small entropy. Such a
modified model can be produced at the end of a MEMCRDD run by specifying a
value for the MODELOUT parameter. The new model is written to a file with the
given name, and this image can be specified for parameter MODEL on a further
run of MEMCRDD.

\item Reconstructions which have a zero background level show less background
noise than those with a large background level. Crudely, this can be thought of
as being due to the fact that MEMSYS3 constrains the output pixel values to be
positive. This means that negative going noise spikes cannot be produced on a
zero background, and consequently neither can the positive going spikes needed
to balance out negative going spikes. By default a flat background is
subtracted from the input data which results in the minimum data value being
5\% of the maximum data value after background subtraction. This can be
overridden by specifying an image for parameter BACKGRND. A background image
could for instance be calculated on the basis of the result from a previous run
of MEMCRDD, by using the SKYSUB procedure in EDRSX, there are also many other
possible ways. What ever background is used, it is added back onto the
reconstruction at the end of the run. The used background is also subtracted
from the default model at the start of the run.

Subtracting a background can produce negative input data values. MEMSYS3 can
cope  with this so long as the negative values are comparable with the noise.
If there  are many large negative values, MEMSYS3 can fail to converge. For
this reason  negative data values large than 1.0 sigma are not included in the
reconstruction.

\end {itemize}

Different combinations of these methods may work best for different fields.

\section {``Floating overflow'' crashes}

Due to the way the MEMSYS3 subroutine package is coded, it is possible for the
MEMCRDD program to crash with a ``floating overflow''. This is usually caused
by  the parameter RATE being too high for the given field. This can be verified
by  checking that the diagnostic TEST went to a value greater than or about
equal to  1.0. A remedy for this problem could be to re-run MEMCRDD giving RATE
a value a  factor of 10 lower.

\appendix
\section {MEMCRDD parameters}
\label {APP:PARS}

This appendix contains a list of all MEMCRDD parameters. For each parameter the
following information is given:
\begin{itemize}
\item A symbol signifying how important it is that the user should be aware
of the parameter. It varies from ``!!!'' meaning that the parameter is {\em
very} important, to ``!'' meaning that it is unlikely that the supplied default
will need to be changed.
\item USE - What the parameter value is used for within MEMCRDD
\item DEFAULT - The default value contained in the file EDRSX:MEMCRDD.CON.
This value will be used unless the user explicitly specifies another value,
either on the RUNSTAR command line, or by producing an edited copy of
MEMCRDD.CON.
\item RESTRICTIONS - Any restrictions on the values which can be given
\item ADVICE - Advice on selecting suitable values for the parameter.
\end{itemize}

\rule{\textwidth}{0.3mm}
{\Large {\bf AIM } (!)}
\begin{description}
\item [USE]:
AIM controls how closely the data must be fitted before the result is
considered acceptable (see the MEMSYS3 manual).
A value of 1.0 corresponds to the ``most
probable'' Classic result, or a normalised \chisq of 1.0 in Historic mode.
Giving smaller values will cause the data to be fitted more closely.
\item [DEFAULT]:
1.0
\item [RESTRICTIONS]:
AIM must be greater than or equal to zero.
\item [ADVICE]:
AIM should usually be left at the default of 1.0.
\end {description}


\rule{\textwidth}{0.3mm}
{\Large {\bf ANALIN} (!!)}
\begin{description}
\item [USE]:
ANALIN specifies a BDF file generated on a previous run of MEMCRDD
(see parameter ANALOUT) which holds internal data used within MEMCRDD.
This data can be used for two reasons: firstly,  to allow the user to analyse
the results of a previous MEMCRDD run, and secondly, to continue a MEMCRDD
run which was interrupted for any reason (for instance, if the batch
queue CPU limit is reached). See parameter FUNCTION for further details.
\item [DEFAULT]:
There is no default value. The user will be prompted for ANALIN if MEMCRDD
is being run interactively.
\item [RESTRICTIONS]:
The BDF file specified for ANALIN must be an ANALOUT file generated on
a previous run of MEMCRDD (see parameter ANALOUT).
\item [ADVICE]:
A value for ANALIN is required only if the parameter FUNCTION is given the value
ANALYSIS or CONTINUE. For other functions, no value need be specified.
\end {description}


\rule{\textwidth}{0.3mm}
{\Large {\bf ANALOUT} (!!!)}
\begin{description}
\item [USE]:
ANALOUT specifies a BDF file to hold internal data used by MEMCRDD. These
``analysis'' files can be used to analyse the results of a MEMCRDD run or
to continue a MEMCRDD run which was interrupted for any reason (see parameter
ANALIN). An analysis file is produced at the end of each MEMSYS3 iteration,
overwriting the file produced at the end of the previous iteration.
\item [DEFAULT]:
None.
\item [RESTRICTIONS]:
A null value for ANALOUT will result in no analysis file being created,
otherwise a BDF file is created with the given name.
\item [ADVICE]:
Analysis files can be quite big, up to a maximum of about 12500 blocks.
The size of an analysis file is hard to predict before running MEMCRDD because
it depends on how the data samples are grouped. However, if the ILEVEL
parameter is
set to 2 or more, then the percentage of internal memory used is displayed
fairly early on in the initialisation (prior to the first MEMSYS3 iteration).
The analysis size will be approximately this percentage of 12500 blocks.
The actual size depends on how much CRDD is being mapped and how many
pixels there are in the final image. The user should be careful to ensure that
he has sufficient disk quota to create the analysis file. It is a very good
idea to create an analysis file on every run of MEMCRDD.
\end {description}

\rule{\textwidth}{0.3mm}
{\Large {\bf BACKGRND} (!!)}
\begin{description}
\item [USE]:
BACKGRND specifies a two dimensional image which is used to define a background
level for the CRDD. Simulated CRDD is formed from this image and subtracted from
the real CRDD before performing the reconstruction. Once the reconstruction is
complete the background image specified by BACKGRND is added onto the final
image before writing it to disk. The background image is also subtracted from
the entropy model image before doing the reconstruction. Any CRDD samples which
have a negative value bigger than the samples noise value after
subtraction of the background, are excluded from the mapping process.
\item [DEFAULT]:
The default for BACKGRND is a null value. If this default is not replaced,
a flat image is generated and used. The value of this flat image is chosen to
ensure that the minimum CRDD value being mapped ({\em after} background
subtraction), is 5\% of the maximum CRDD value. This means that all CRDD values
will be strictly positive.
\item [RESTRICTIONS]:
The image specified for BACKGRND must be in EDRS format, and must correspond
exactly to the same area of sky as the image being produced by the current
MEMCRDD run. This means that such background images must be derived
from the results of previous runs of MEMCRDD.
\item [ADVICE]:
Usually, the null default will be OK. However, background noise in the final
image can sometimes be reduced by supplying an image which fits the lower
envelope of the data rather more closely than the 5\% value used by default.
For instance, this could be done using the EDRSX:SKYSUB procedure, or any of
the other EDRS/X facilities.
\end {description}

\rule{\textwidth}{0.3mm}
{\Large {\bf BAND1PSF} (!)}
\begin{description}
\item [USE]:
BAND1PSF specifies a three dimensional BDF containing the Point Spread Function
images for all the band 1 (12 \micron) detectors.
\item [DEFAULT]:
The default value is EDRSX:IPAC1. This uses the file IPAC1.BDF in the main
EDRSX directory which contains the PSFs produced by Mehrdad Moshir at IPAC.
\item [RESTRICTIONS]:
If the user wishes to use any other PSFs, they must be stacked in conformance
with the format of EDRSX:IPAC1.BDF. The author can provided details of this.
\item [ADVICE]:
The most common reason for over-riding the default is to allow several people
to use MEMCRDD simultaneously. The EDRSX directory is protected in such a way as
to prevent more than one process accessing the PSF stacks at any one time. To
get round this the PSF stack can be copied to the users own disk space, and this
copy specified for BAND1PSF instead of the default version.
\end {description}


\rule{\textwidth}{0.3mm}
{\Large {\bf BAND2PSF} (!)}
\begin{description}
\item [USE]:
As BAND1PSF except that it hold the PSFs for band 2 (25 \micron).
\item [DEFAULT]:
EDRSX:IPAC2
\item [RESTRICTIONS]:
As for BAND1PSF.
\item [ADVICE]:
As for BAND1PSF.
\end {description}

\rule{\textwidth}{0.3mm}
{\Large {\bf BAND3PSF} (!)}
\begin{description}
\item [USE]:
As BAND1PSF except that it hold the PSFs for band 3 (60 \micron).
\item [DEFAULT]:
EDRSX:IPAC3
\item [RESTRICTIONS]:
As for BAND1PSF.
\item [ADVICE]:
As for BAND1PSF.
\end {description}

\rule{\textwidth}{0.3mm}
{\Large {\bf BAND4PSF} (!)}
\begin{description}
\item [USE]:
As BAND1PSF except that it hold the PSFs for band 4 (100 \micron).
\item [DEFAULT]:
EDRSX:IPAC4
\item [RESTRICTIONS]:
As for BAND1PSF.
\item [ADVICE]:
As for BAND1PSF.
\end {description}


\rule{\textwidth}{0.3mm}
{\Large {\bf BOXSIZE} (!!!)}
\begin{description}
\item [USE]:
This specifies the size of the final square image in arcmins. The box size given
will be extended to produce a blank border round the final image to facilitate
the convolutions which occur within MEMCRDD. This border has a width of 0.75 of
the maximum diagonal size of all detectors being used (usually about 3 to 4
arc-minutes).
\item [DEFAULT]:
The default for BOXSIZE is a null value. If a null value is given, then the
final image is rectangular and is the smallest size which will cover {\em all}
the data in the input CRDD files.
\item [RESTRICTIONS]:
BOXSIZE must be at least four times the pixel size in the final image (see
parameter PIXSIZE).
\item [ADVICE]:
BOXSIZE and PIXSIZE should be selected to keep the total number of pixels in the
image as small as possible, to reduce CPU requirements.
\end {description}

\rule{\textwidth}{0.3mm}
{\Large {\bf CORFACT} (!)}
\begin{description}
\item [USE]:
A factor by which to multiply the calculated correlated noise variances before
use.
\item [DEFAULT]:
1.0
\item [RESTRICTIONS]:
None.
\item [ADVICE]:
There will usually be no need to change this value.
\end {description}

\rule{\textwidth}{0.3mm}
{\Large {\bf COVER} (!!)}
\begin{description}
\item [USE]:
COVER is only used in ``LORES'' mode (see parameter FUNCTION). It is used to
specify a BDF which is to receive the coverage image. This image gives the
effective solid angle (in steradians) with which each pixel was observed. Thus,
if many data samples cover a particular pixel then that pixel will have a high
coverage value.
\item [DEFAULT]:
None.
\item [RESTRICTIONS]:
A value {\em must}  be given for this parameter if LORES mode is being used.
The output is an EDRS image covering the same area as the Low-Res image (see
parameter LORES).
\item [ADVICE]:
None.
\end {description}


\rule{\textwidth}{0.3mm}
{\Large {\bf CRDDF1 - 20} (!!!)}
\begin{description}
\item [USE]:
The parameters CRDDF1 to CRDDF20 are used to specify the input CRDD files
containing the data to be reconstructed. If less than 20 scans are to be
processed, the first null value marks the end of the list.
\item [DEFAULT]:
None.
\item [RESTRICTIONS]:
The files can be either standard SURVEY CRDD files (as produced by programs
I\_SNIP\_CRDD, I\_DESTRIPE, etc), or AO CRDD files read from IPAC footprints tapes.
\item [ADVICE]:
None.
\end {description}

\rule{\textwidth}{0.3mm}
{\Large {\bf CRDOUT1 - CRDOUT20} (!!)}
\begin{description}
\item [USE]:
The parameters CRDOUT1 to CRDOUT20 are used to specify output CRDD files. They
may be used in two situations, firstly if simulated CRDD is being generated
(see parameter FUNCTION), secondly, if a dump of internal data files is
required (see parameter OPTION). The data which is written to parameter
CRDOUTi corresponds to the data read from parameter CRDDFi (i.e. the pointing
information for each sample in the two CRDD files is the same).
\item [DEFAULT]:
None.
\item [RESTRICTIONS]:
None.
\item [ADVICE]:
None.
\end {description}

\rule{\textwidth}{0.3mm}
{\Large {\bf CROTA} (!!)}
\begin{description}
\item [USE]:
CROTA specified the orientation of the final image. It gives the angle (in
degrees) between north and the negative direction of the second pixel
coordinate (i.e. downwards if the image is displayed as normal). It is
measured positive in the direction north through east. Thus a value of 180
will put north upwards and east to the left.
\item [DEFAULT]:
The default is a null value, which causes MEMCRDD to calculate an orientation
which minimises the size of the final image. This is done to reduce run time.
Correct RA and DEC values can still be obtained from such images as usual, by
using the RADEC program in EDRSX.
\item [RESTRICTIONS]:
The value must be in the range 0 - 360.
\item [ADVICE]:
Unless it is important that the final image has north upwards (for instance),
then the default null value should be left unchanged.
\end {description}


\rule{\textwidth}{0.3mm}
{\Large {\bf DATATYPE} (!)}
\begin{description}
\item [USE]:
Specifies if the input CRDD files contain SURVEY or AO data. AO data must be in
``footprints'' format.
\item [DEFAULT]:
SURVEY
\item [RESTRICTIONS]:
Must be either SURVEY or AO.
\item [ADVICE]:
At the moment there is no official support for the distribution AO CRDD
footprints data in the UK, but see appendix \ref{APP:AO} for an interim
work-around.
\end {description}

\rule{\textwidth}{0.3mm}
{\Large {\bf DEC\_DEG} (!!!)}
\begin{description}
\item [USE]:
The degrees field of the Declination of the required field centre. If this
is given a non-integer value then it is assumed that it gives the entire DEC
value.
\item [DEFAULT]:
If the default value for parameter BOXSIZE is accepted, then the field centre
is taken as the centre of {\em all} the available data. If the default for
BOXSIZE is overriden, then there is no default for DEC\_DEG.
\item [RESTRICTIONS]:
Must be in the range +90 to -90.
\item [ADVICE]:
None.
\end {description}

\rule{\textwidth}{0.3mm}
{\Large {\bf DEC\_MIN} (!!!)}
\begin{description}
\item [USE]:
The minutes field of declination of the required field centre. This is not
prompted for if DEC\_DEG was given a non-integer value.
\item [DEFAULT]:
As for DEC\_DEG.
\item [RESTRICTIONS]:
Must be the range 0 - 60.
\item [ADVICE]:
None.
\end {description}

\rule{\textwidth}{0.3mm}
{\Large {\bf DEC\_SEC} (!!!)}
\begin{description}
\item [USE]:
The seconds field of declination of the required field centre. This is not
prompted for if DEC\_MIN was given a non-integer value.
\item [DEFAULT]:
As for DEC\_DEG.
\item [RESTRICTIONS]:
Must be the range 0 - 60.
\item [ADVICE]:
None.
\end {description}

\rule{\textwidth}{0.3mm}
{\Large {\bf DEGLIT } (!)}
\begin{description}
\item [USE]:
This value controls the effectiveness of the deglitching applied to the data
before reconstruction commences. Glitches are defined as spikes in the
data which exceed the local noise level, and which are of duration less
than the value of DEGLIT times the size of a point source. A value of zero will
suppress all deglitching, whereas a value of 1.0 will give very strong
deglitching which will probably reject point sources as well. If the ILEVEL
parameter is set to 3 or more, then the number of samples rejected as glitches
form each input CRDD file is displayed.

The deglitching is based on an iterative scheme in which the input detector data
stream is smoothed with a box filter twice the width of a point source. An
estimate of the RMS residual between input and smoothed data is formed, and
input data corresponding to residuals of greater than three times the RMS
residual are rejected. This is done eight times, resulting in all features
smaller than two point source widths being rejected. The final RMS residual
value is used as the default value for the field noise in the corresponding data
stream. Glitches are then identified by looking for blocks of adjacent rejected
output pixels. Any such blocks with sizes less than or equal to DEGLIT times a
point source width are rejected from the original input data. All other input
data is retained in its original form.


\item [DEFAULT]:
0.3
\item [RESTRICTIONS]:
Must be in the range 0.0 to 1.0.
\item [ADVICE]:
Setting DEGLIT to 1.0 can be useful if a map is required which does not
include point sources. This can often reduce the noise in the background
regions.
\end {description}

\rule{\textwidth}{0.3mm}
{\Large {\bf DGROUPS} (!)}
\begin{description}
\item [USE]:
This parameter specifies which detectors are to be used. If the ILEVEL
parameter is set to 3 or more, then a list of the detector numbers
in the input CRDD files is displayed just before prompting for DGROUPS. A list
of integer values should be given for DGROUPS containing one value per detector
in the input CRDD files. If the value corresponding to a given detector is zero,
then {\em all} data from that detector is excluded from the reconstruction. If
the same (non-zero) value is given for several detectors then a ``detector
group'' is defined and the same PSF is used for all detectors in the group.
The PSF used for the group is the mean of the PSFs of all detectors in the
group. For instance, if the input CRDD files contained 16 detectors (as is
usual), and if DGROUPS was given the value
{\tt 0,1,2,3,4,5,6,7,8,9,10,11,12,13,1,0} then the first and last detectors
would not be used, the second and fifteenth detectors would be processed using
a common PSF (the mean of the PSFs of the two detectors), and all other
detectors would be used, and processed using their own PSF.
\item [DEFAULT]:
The default is a null value which causes all detectors to be used, and grouped
as follows (detectors identified by detector number, given in the IRAS
Catalogs and Atlases Explanatory Supplement Fig. II.C.6):
\begin{verbatim}

  100 um: group 1 contains #55, #62
            "   2    "     #4
            "   3    "     all others

   60 um: group 1 contains #31, #11
            "   2    "     #38, #12
            "   3    "     all others

   25 um: group 1 contains #39, #46
            "   2    "     #19, #22
            "   3    "     all others

   12 um: group 1 contains #47, #26
            "   2    "     #27, #54
            "   3    "     all others

\end{verbatim}
The first group in each band contain the small edge detectors, the second group
contain the medium sized edge detectors, and the third group contain the
full sized detectors.
\item [RESTRICTIONS]:
The list must contain the correct number of values, one per detector present.
\item [ADVICE]:
None.
\end {description}

\rule{\textwidth}{0.3mm}
{\Large {\bf ELEMENT} (!)}
\begin{description}
\item [USE]:
ELEMENT is used in diagnostic mode. It allows the user to specify an element
number within an internal image or data set.
\item [DEFAULT]:
None.
\item [RESTRICTIONS]:
Must be an integer larger than zero and less than the maximum size of the image
or data set.
\item [ADVICE]:
See parameters OPTION and UDIAG for more information on diagnostic mode.
\end {description}

\rule{\textwidth}{0.3mm}
{\Large {\bf FIELD} (!!)}
\begin{description}
\item [USE]:
This parameter allows the user to specify the standard deviation of the
``field'' noise in the input CRDD data in Janskys. This is the random noise
which is independent of signal strength. The value specified is the mean of the
field noise in each detector. The actual field noise values used for each
detector are obtained by scaling the supplied value by a factor dependant on the
measured ``Noise Equivalent Flux Density'' for each detector (see the
Explanatory Supplement Fig. IV.A.1).
\item [DEFAULT]:
The default is zero which causes MEMCRDD to calculate a separate value
for each detector data stream by applying a one dimensional high pass filter
to each data stream. If the parameter ILEVEL is given a value of 4 or more, then
the values calculated are displayed by MEMCRDD.
\item [RESTRICTIONS]:
The value must be greater than or equal to zero.
\item [ADVICE]:
The default value is usually OK . However, if the final image is noisy, it may
be necessary to give a larger FIELD value. If parameter FUNCTION has value
SIMULATE then the default value of zero causes no noise to be added to the
simulated data.
\end {description}

\rule{\textwidth}{0.3mm}
{\Large {\bf FILE} (!)}
\begin{description}
\item [USE]:
Used in diagnostic mode, to specify an internal file.
\item [DEFAULT]:
None.
\item [RESTRICTIONS]:
The files are identified either by an internal file {\em number}, or a string
describing its contents. The legal values are 1, 2, 20, 21 to 28, X, Y, SOL,
BACK or PSF.
\item [ADVICE]:
See parameters OPTION and UDIAG for more information about diagnostic mode.
\end {description}

\rule{\textwidth}{0.3mm}
{\Large {\bf FILTER} (!!)}
\begin{description}
\item [USE]:
FILTER specifies the full width at half maximum of a Gaussian Intrinsic
Correlation Function (see sections \ref {SEC:ICFA} and \ref {SEC:ICFB}), in
units of arc-minutes. A value of zero causes no ICF to be used.
\item [DEFAULT]:
0.5 arcmins
\item [RESTRICTIONS]:
None
\item [ADVICE]:
Using an ICF encourages smoothness in the final image on the scale of the ICF.
The best value to use for FILTER could be found by running MEMCRDD several times
using different values, and selecting the value which gives the greatest
evidence value ( log(probability) ).
\end {description}

\rule{\textwidth}{0.3mm}
{\Large {\bf FUNCTION} (!!)}
\begin{description}
\item [USE]:
Specifies what MEMCRDD is to do. There are several options:
\begin{description}
\item [HIRES] - Produce a high resolution image from scratch, using the
input CRDD files specified by parameters CRDDF1 to CRDDF20.
The method used to produce the hi-res image is specified by parameter METHOD.
\item [LORES] - Produce a low resolution image by sample averaging. The detector
PSFs are used to weight each sample. This is {\em much} faster than HIRES mode
but slower than using I\_CRDD\_COMBINE.
\item [SIMULATE] - Produce simulated CRDD. The user gives an image holding a
trial sky, and a set of CRDD files. The sky is convolved with the detector
PSFs and sampled at the position of each of the samples in the input CRDD files.
The sampled values are written to the output CRDD files (CRDOUT1 - 20). Noise
is added to the output CRDD as determined by parameters FIELD, XSCAN, INSCAN,
and STRIPES.
\item [CONTINUE] - Continues a HIRES run which was interrupted for any reason
(e.g. the batch queue CPU limit was reached, or the VAX went down). It is
possible to restart the run using different values for the following parameters:
AIM, HIRES, ILEVEL, LEVEL, LOGFILE, NITER, RATE, STOPFILE, TOL. It is
only possible to restart a run if the previous run generated an analysis file
(see parameters ANALOUT and ANALIN). If an analysis file is available it must be
specified by parameter ANALIN.
\item [ANALYSIS] - Allows the user to analyse the results from a non-HISTORIC run
of MEMCRDD (see parameter METHOD). An analysis file must be specified using
parameter ANALIN. The user can specify areas of the image, and the integrated
flux within that area is displayed together with the standard deviation on the
total flux. Each area takes a significant amount of CPU to process (roughly
equivalent to a single classic iteration). The area may be specified either as
an EDRS XY list (see parameter MASKPOLY) created using EDRS program XYCUR
for instance, or as an image (see parameter MASK). Care must be taken when
interpreting the results from this form of analysis, as the standard deviations
calculated depend on the form of the entropy model used (see the MEMSYS3 users
manual). In general, if errors on integrated fluxes are required, MEMCRDD should
be run using a flat model, such as is generated if the parameter MODEL is given
a null value.

A second use of the ANALYSIS function is to allow an analysis file to be
examined internally. This is referred to as ``diagnostic mode'', and allows
various operations to be performed on the data stored in the analysis file
(see parameter OPTION).
Diagnostic mode may be used on both HISTORIC and CLASSIC results.
\end{description}

\item [DEFAULT]:
The default is HIRES.
\item [RESTRICTIONS]:
The given value must be one of the above values, or an unambiguous abbreviation.
\item [ADVICE]:
None.
\end {description}

\rule{\textwidth}{0.3mm}
{\Large {\bf GROUP} (!)}
\begin{description}
\item [USE]:
Used in diagnostic mode, to specify a sample group number.
\item [DEFAULT]:
None.
\item [RESTRICTIONS]:
Must be greater than zero and less than or equal to the maximum group number.
\item [ADVICE]:
See parameters OPTION and UDIAG for more information about diagnostic mode.
\end {description}

\rule{\textwidth}{0.3mm}
{\Large {\bf HIRES} (!!!)}
\begin{description}
\item [USE]:
This parameter is used to specify a BDF to receive the output high resolution
image. The image is in EDRS format.
\item [DEFAULT]:
None.
\item [RESTRICTIONS]:
It must be a valid BDF name, a null value is not allowed.
\item [ADVICE]:
None.
\end {description}

\rule{\textwidth}{0.3mm}
{\Large {\bf ILEVEL} (!)}
\begin{description}
\item [USE]:
This is an integer value controlling how much information is displayed on the
users screen or written to the batch job log file. A value of 1 results in no
information being displayed, value of 5 results in all information being
displayed. Information displayed at a lower value of ILEVEL is more likely to
be of interest to the user.
\item [DEFAULT]:
2
\item [RESTRICTIONS]:
Must be in the range 1 to 5.
\item [ADVICE]:
Use a larger value of ILEVEL if you are not happy with the initial results, in
order to see what is happening inside MEMCRDD.
\end {description}

\rule{\textwidth}{0.3mm}
{\Large {\bf INSCAN} (!)}
\begin{description}
\item [USE]:
This specifies the in-scan uncertainty in the pointing data, in arcminutes.
It is used to generate pointing noise estimates if a value for the SKYIN
parameter is given when FUNCTION=HIRES. It is also used to
add pointing errors into simulated CRDD when FUNCTION=SIMULATE.
\item [DEFAULT]:
The default is a null value which causes MEMCRDD to use a value dependant on the
IRAS band. The bands 1 to 4 have values of 0.05, 0.05, 0.1 and 0.1 respectively.
\item [RESTRICTIONS]:
The value must be positive or zero.
\item [ADVICE]:
The default value should be OK.
\end {description}

\rule{\textwidth}{0.3mm}
{\Large {\bf K } (!)}
\begin{description}
\item [USE]:
This is the fractional uncertainty to assume in the effective solid angle values
for each detector. It is taken into account when calculating the uncertainties
in simulated data values.
\item [DEFAULT]:
The default is a null value, which causes the following default values to be
used for IRAS bands 1 to 4: 0.01, 0.01, 0.03, 0.04. These value are the band
averaged relative errors in the detector effective solid angles quoted by IPAC.
\item [RESTRICTIONS]:
The value must be positive.
\item [ADVICE]:
None.
\end {description}

\rule{\textwidth}{0.3mm}
{\Large {\bf LEVEL} (!)}
\begin{description}
\item [USE]:
This controls the amount of diagnostic information which the MEMSYS3 package
generated. See the MEMSYS3 users manual for information on the legal values for
LEVEL. NB, LEVEL and ILEVEL function differently. LEVEL controls the MEMSYS3
diagnostics, ILEVEL controls the amount of information displayed by the rest
of MEMCRDD.
\item [DEFAULT]:
The default is a null value which suppresses all MEMSYS3 diagnostic information.
\item [RESTRICTIONS]:
See MEMSYS3 users manual.
\item [ADVICE]:
None.
\end {description}

\rule{\textwidth}{0.3mm}
{\Large {\bf LOGFILE} (!)}
\begin{description}
\item [USE]:
This parameter specifies a text file to receive any MEMSYS3 diagnostic
information.
\item [DEFAULT]:
SYS\$OUTPUT. This causes information to go to the terminal screen if the user
is interactive, or to the log file if in batch mode.
\item [RESTRICTIONS]:
None.
\item [ADVICE]:
None.
\end {description}

\rule{\textwidth}{0.3mm}
{\Large {\bf LOOP} (!)}
\begin{description}
\item [USE]:
Specifies if the user is to be repeatedly prompted for new MASKS when running in
ANALYSIS mode (see parameter FUNCTION). If not, then the user specifies a single
mask, and when complete, MEMCRDD terminates.
\item [DEFAULT]:
NO
\item [RESTRICTIONS]:
YES,NO,TRUE or FALSE, or any abbreviation.
\item [ADVICE]:
Looping should only be used when running MEMCRDD interactively. Since it
takes quite a long time to analyse a single mask, this won't usually be the case.
\end {description}

\rule{\textwidth}{0.3mm}
{\Large {\bf LORES} (!!)}
\begin{description}
\item [USE]:
Specifies the output BDF to receive the low resolution image when running in
LORES mode (see parameter FUNCTION). The image is held in EDRS format.
\item [DEFAULT]:
None.
\item [RESTRICTIONS]:
A null value is not allowed.
\item [ADVICE]:
None.
\end {description}

\rule{\textwidth}{0.3mm}
{\Large {\bf MASK } (!!)}
\begin{description}
\item [USE]:
This is used when running in ANALYSIS mode (see parameter FUNCTION). It
determines the area of the high-res image to be integrated. Specifically, the
high-res image is multiplied pixel-for-pixel with the mask image and the total
data sum of the product image is calculated (converted to Janskys), together
with the standard deviation in the sum. The total data sum in the supplied mask
is also displayed, allowing the user to calculate the effective solid angle in
the mask and so convert the integrated flux to a mean surface brightness.

\item [DEFAULT]:
None.
\item [RESTRICTIONS]:
The image must be held in EDRS format, and must correspond pixel-for-pixel with
the high-res image.
\item [ADVICE]:
A mask will typically be created by taking the high-res image, and setting
certain pixels to the value 1.0 and all other pixels to the value 0.0. This can
be done in various ways using EDRS. It is also possible to calculate a weighted
mean over a given area by lowering the mask value in the regions which are to
receive lower weight.
\end {description}

\rule{\textwidth}{0.3mm}
{\Large {\bf MASKPOLY } (!)}
\begin{description}
\item [USE]:
This is used when running in ANALYSIS mode (see parameter FUNCTION). It
determines the area of the high-res image to be integrated.
It specifies a BDF holding an XY list defining a polygonal area of a high-res
image. The total data sum within the polygon (converted to Janskys) is found
from the high-res image, together with the standard deviation of the sum.
\item [DEFAULT]:
None.
\item [RESTRICTIONS]:
The BDF must be a standard EDRS XY list, as produced by XYCUR or XYKEY (for
instance).
\item [ADVICE]:
The XY list would usually be created by displaying the high-res image on an
image display device and then using XYCUR to select the polygonal area to be
summed over.
\end {description}

\rule{\textwidth}{0.3mm}
{\Large {\bf METHOD} (!!!)}
\begin{description}
\item [USE]:
Specifies the MEMSYS3 method to use when performing the reconstruction. See the
MEMSYS3 manual for more details. It can take one of the following values:
\begin {description}
\item [HISTORIC] - Runs MEMSYS3 in ``Historic'' mode. This is the algorithm used
by previous versions of the MEMSYS package, but which has been superceded by
the ``Classic'' algorithm in MEMSYS3. The final result is
an image which has an overall normalised \chisq value of 1.0 (if AIM has the default
value of 1.0). This mode is faster than the others, but is non-Bayesian and
is not considered to be rigorously justifiable by Gull and Skilling.
The results usually contain less fine detail than the classic results. It also
suffers from the fact that the more accurate samples are fitted more closely
than \chisq=1.0 and the less accurate samples are fitted less closely than
\chisq=1.0. Since there is a strong signal dependance in the
noise, this means that the background regions are over-fitted at the expense of
the source regions. It is not possible to ``analyse'' historic results (see
parameter FUNCTION).
\item [CLASSIC] - Runs MEMSYS3 in ``classic automatic'' mode. The final result
is the ``most probable'' image (see MEMSYS3 manual). This method incorporates
automatic noise scaling. That is, MEMSYS3 determines a single scaling factor to
apply to all the sample noise estimates which maximises the likelihood that the
given data could have been generated by observing the same sky area. Note that
this is just a single scaling factor applied to {\em all} samples, the
sample-to-sample variation in noise level is thus not altered. CLASSIC results
take much longer to produce and contain more fine detail than HISTORIC results.
Errors estimates can be produced from CLASSIC results by using  ``analyse''
mode (see parameter FUNCTION).
\item [NONAUTO] - Runs MEMSYS3 in ``classic non-automatic'' mode. This is the
same as CLASSIC except that there is no scaling of the input noise values. The
value of the input noise scaling is in doubt since MEMSYS3 assumes that the
noise is uncorrelated, whereas in fact the noise in IRAS data can be strongly
correlated. Automatic mode seems to interpret this correlated noise as real
structure (see section \ref {SEC:NOISE}).
\item [MCM] - {\bf This is NOT a Maximum Entropy Method!!} It is a simple
implementation of the ``Maximum Correlation Method'' for reconstructing images
from IRAS data
described by Aumann et al (Astronomical Journal  Vol.99, No.5, P.1674). It is
{\em much} faster than any MEMSYS3 method, but produces lower resolution images
with poorer fits to the data. The final images are much smoother than MEM
images. The results cannot be ``analysed''. It must be emphasised that this method
makes no use what-so-ever of the MEMSYS3 package.
\end {description}
\item [DEFAULT]:
CLASSIC
\item [RESTRICTIONS]:
Must be one of the above values, or an abbreviation.
\item [ADVICE]:
If using a simple noise model (i.e if a null value was specified for parameter
SKYIN) then CLASSIC is probably the best method to use since the noise values
will be uncertain. If the more sophisticated noise model option is used (based
on running  MEMCRDD several times, giving the output image as the SKYIN parameter
for the next run) then NONAUTO mode is best since the noise scaling provided by
CLASSIC will tend to lower the noise values too much.
\end {description}

\rule{\textwidth}{0.3mm}
{\Large {\bf MODEL } (!)}
\begin{description}
\item [USE]:
Specifies an image which is to be used as the default model in the
reconstruction. The background image is subtracted from it before being used.
\item [DEFAULT]:
The default is a null value, which causes a flat model to be used with value
equal to the mean of the input data (converted from flux to surface brightness).
\item [RESTRICTIONS]:
The image must be greater than the background at every point.
\item [ADVICE]:
It is usually a good idea to use the default flat image, especially if a
non-zero value was specified for FILTER or if an analysis is to be performed on
the reconstruction. However, it seems that background structure can sometimes
be reduced by specifying an image which is closer to the final reconstruction
(see parameter MODELOUT).
\end {description}

\rule{\textwidth}{0.3mm}
{\Large {\bf MODELOUT } (!)}
\begin{description}
\item [USE]:
MODELOUT specifies an image to receive a modified version of the default model
specified for parameter MODEL. Pixel $i$ of the new model is given by

\begin {equation}
m'_{i}=m_{i}+gain*(f_{i}-m_{i})*s_{i}/s_{max}
\end {equation}

where $m_{i}$, $f_{i}$ and $s_{i}$ are pixel $i$ of the old model, the
reconstruction, and the entropy map. $s_{max}$ is the maximum entropy value in
the entropy map, and $gain$ is a factor used to reduce the change in
the model when the entropy of all pixels is similar.
\item [DEFAULT]:
None.
\item [RESTRICTIONS]:
None.
\item [ADVICE]:
The new model can be used in a further run of MEMCRDD, with the provisos
mentioned under parameter MODEL. If this process is repeated the model will
eventually become sufficiently in agreement with the data, that MEMSYS3 sees no
evidence for any change from the model.
\end {description}

\rule{\textwidth}{0.3mm}
{\Large {\bf NITER} (!!)}
\begin{description}
\item [USE]:
The maximum number of iterations of the MEMSYS3 algorithm to perform. If the
reconstruction is not complete when NITER iterations have been performed. The
current reconstruction is returned using parameter HIRES.
\item [DEFAULT]:
20
\item [RESTRICTIONS]:
None
\item [ADVICE]:
If the maximum number of iterations is exceeded, all is not lost! If an analysis
file was created (see parameter ANALOUT), then the process can be restarted from
where it left off by re-running MEMCRDD with parameter METHOD set to CONTINUE,
and specifying a higher value for NITER.
\end {description}

\rule{\textwidth}{0.3mm}
{\Large {\bf NOISE\_OK} (!)}
\begin{description}
\item [USE]:
An output parameter which is set to YES if the noise values used for the
reconstruction differed little from the noise values specified by the VARIN
parameter. When this happens, the noise model has converged.
\item [DEFAULT]:
None
\item [RESTRICTIONS]:
None
\item [ADVICE]:
This parameter is intended for use in command procedures which re-run MEMCRDD
several times to produce a refined noise model. The command procedure can look
at the value of NOISE\_OK after each run to see if another run is needed. E.g.
\begin{verbatim}
$!
$! Run MEMCRDD basing the noise model on a previous MEMCRDD result, IMAGE_A.
$!
$ DEF JOB1 EDRSX:MEMCRDD/HIRES=IMAGE_B/SKYIN=IMAGE_A
$ RUNSTAR JOB1
$!
$! Check to see if the noise model has converged.
$!
$ IF F$TRNLNM("JOB1_NOISE_OK").NES."YES"
$    THEN
$!
$! If not, re-run MEMCRDD basing the noise model on the previous high res image.
$!
$       RUNSTAR JOB1/HIRES=IMAGE_C/SKYIN=IMAGE_B

.. (etc) ..

$    ENDIF

.. (etc) ..

\end{verbatim}
See section \ref {SEC:RUN} for a description of the use of ``command line''
parameter assignments.
\end {description}

\rule{\textwidth}{0.3mm}
{\Large {\bf NTRIAL} (!)}
\begin{description}
\item [USE]:
Specifies the maximum number of data simulations to be performed when
calculating the contribution to the noise due to pointing errors. For each input
sample, the image specified by parameter SKYIN is sampled up to NTRIAL times with
random Gaussian variations introduced into the position of the sample.The
spread in sample values is taken as the noise due to pointing. The FWHM of the
Gaussian can be specified separately for the in-scan and cross-scan directions
(see parameters INSCAN and XSCAN).
\item [DEFAULT]:
1000
\item [RESTRICTIONS]:
None
\item [ADVICE]:
A low value will speed up the calculation of the pointing noise, but will
increase the uncertainty in the noise level.
\end {description}

\rule{\textwidth}{0.3mm}
{\Large {\bf OPTION} (!)}
\begin{description}
\item [USE]:
Used in diagnostic mode, to specify an option from the menu of diagnostic tools.
The menu is displayed if parameter UDIAG is set to a true value during an
analysis, and also on entry to every MEMSYS3 subroutine if the least significant
digit in the value of parameter LEVEL is 3.
\item [DEFAULT]:
None
\item [RESTRICTIONS]:
Must be an integer in the range 1 to 18.
\item [ADVICE]:
None.
\end {description}

\rule{\textwidth}{0.3mm}
{\Large {\bf OUTPUT } (!)}
\begin{description}
\item [USE]:
Used in diagnostic mode to specify a file to which is written an internal file.
\item [DEFAULT]:
None.
\item [RESTRICTIONS]:
None.
\item [ADVICE]:
None.\end {description}

\rule{\textwidth}{0.3mm}
{\Large {\bf PIXSIZE } (!)}
\begin{description}
\item [USE]:
PIXSIZE gives the size of the square pixels in the output image, in arc-minutes.
\item [DEFAULT]:
The default is a null value which causes the values 0.25, 0.25, 0.5, 1.0 to be
used for data in the 12, 25, 60 and 100 $\mu$m wavebands.
\item [RESTRICTIONS]:
None.
\item [ADVICE]:
The smaller the pixel size, the longer MEMCRDD will take to run.
\end {description}

\rule{\textwidth}{0.3mm}
{\Large {\bf PSFERR} (!)}
\begin{description}
\item [USE]:
Specifies the fractional uncertainty to assume for the values in the PSF images.
\item [DEFAULT]:
The default is 0.1, the quoted error on the IPAC PSFs.
\item [RESTRICTIONS]:
Must lie between 0.0 and 1.0.
\item [ADVICE]:
Increasing PSFERR will raise the noise estimates for the input data if a value
was specified for parameter SKYIN.
\end {description}

\rule{\textwidth}{0.3mm}
{\Large {\bf RA\_HRS} (!!!)}
\begin{description}
\item [USE]:
The hours field of the Right Ascension of the required field centre. If this
is given a non-integer value then it is assumed that it gives the entire RA
value.
\item [DEFAULT]:
If the default value for parameter BOXSIZE is accepted, then the field centre
is taken as the centre of {\em all} the available data. If the default for
BOXSIZE is overriden, then there is no default.
\item [RESTRICTIONS]:
Must be in the range 0.0 to 24.0
\item [ADVICE]:
None.
\end {description}

\rule{\textwidth}{0.3mm}
{\Large {\bf RA\_MINS} (!!!)}
\begin{description}
\item [USE]:
The minutes field of right ascension of the required field centre. This is not
prompted for if RA\_HRS was given a non-integer value.
\item [DEFAULT]:
As for RA\_HRS.
\item [RESTRICTIONS]:
Must be the range 0 - 60.
\item [ADVICE]:
None.
\end {description}

\rule{\textwidth}{0.3mm}
{\Large {\bf RA\_SECS} (!!!)}
\begin{description}
\item [USE]:
The seconds field of right ascension of the required field centre. This is not
prompted for if RA\_MINS was given a non-integer value.
\item [DEFAULT]:
As for RA\_HRS.
\item [RESTRICTIONS]:
Must be the range 0 - 60.
\item [ADVICE]:
None.
\end {description}


\rule{\textwidth}{0.3mm}
{\Large {\bf RATE} (!!)}
\begin{description}
\item [USE]:
RATE controls the maximum size of the step that can be taken towards the final
reconstruction by a single MEMSYS3 iteration. It should be of order 1. If a
negative value is given then the absolute value is used initially, but RATE is
then changed between iterations in an attempt to keep the TEST diagnostic between the
limits of 0.5 and 0.05 (see section \ref {SEC:DIAG}). If a positive value is
given for RATE, then the given value is used for all MEMSYS3 iterations without
change.
\item [DEFAULT]:
-0.5.
\item [RESTRICTIONS]:
Must not be exactly zero.
\item [ADVICE]:
Values of RATE which are too high can allow the algorithm to wander from its
correct trajectory. This can sometimes cause MEMCRDD to crash with a ``floating
overflow'' message. If this happens try reducing RATE by a factor of ten.
\end {description}

\rule{\textwidth}{0.3mm}
{\Large {\bf SCALAR} (!)}
\begin{description}
\item [USE]:
Used in diagnostic mode, to specify a scalar value.
\item [DEFAULT]:
None.
\item [RESTRICTIONS]:
None.
\item [ADVICE]:
See parameters OPTION and UDIAG for more information on diagnostic mode.
\end {description}

\rule{\textwidth}{0.3mm}
{\Large {\bf SKYIN} (!!)}
\begin{description}
\item [USE]:
This parameter specifies an EDRS image holding an estimate of the sky to be
used to define the noise in the input CRDD caused by pointing errors and
 errors in the detector PSFs.
\item [DEFAULT]:
The default is a null value which causes a simple noise model to be used which
does not contain any terms due to pointing or PSF errors, or any ``correlated''
noise.
\item [RESTRICTIONS]:
The image must cover exactly the same area of the sky as the output image.
\item [ADVICE]:
This image is usually the output from a previous run of MEMCRDD. Doing this sort
of ``noise loop'' can sometimes reduce the level of spurious background
structure.
\end {description}

\rule{\textwidth}{0.3mm}
{\Large {\bf SPEED } (!)}
\begin{description}
\item [USE]:
Controls the rate at which noise estimates change between
successive runs of MEMCRDD when performing a ``noise loop'' (see section
\ref {SEC:ICFB}). In fact the used variances $v_{k}$ are given by

\begin {equation}
v_{k}=speed*t_{k}+(1.0-speed)*l_{k}
\end {equation}

where $t_{k}$ are the total variances calculated from the input CRDD and the
input SKYIN image (i.e. field noise + correlated noise + pointing noise + PSF
noise), and $l_{k}$ are last times variances. Thus if speed were set to zero,
the used variances would always be last times variances.
\item [DEFAULT]:
0.3
\item [RESTRICTIONS]:
Must be between 0.0 and 1.0.
\item [ADVICE]:
Reducing the speed increase the stability of the noise loop process. Too high a
value for speed can cause the process to oscillate.
\end {description}


\rule{\textwidth}{0.3mm}
{\Large {\bf SRCRATIO} (!)}
\begin{description}
\item [USE]:
The correlated noise calculation is invalid near small sources. The values used
in such areas are therefore extrapolated from the surrounding regions. SRCRATIO
is used to define source areas for this purpose. Samples for which the
pointing noise exceeds SRCRATIO times the field noise are excluded from the
correlated noise calculation. Since pointing noise is required to do this,
correlated noise estimates can only be produced if a value is given for
parameter SKYIN.
\item [DEFAULT]:
2.0
\item [RESTRICTIONS]:
Must be positive.
\item [ADVICE]:
The default value should usually be OK.
\end {description}

\rule{\textwidth}{0.3mm}
{\Large {\bf STOP } (!)}
\begin{description}
\item [USE]:
If MEMCRDD is run interactively, the user can press CONTROL-C to halt the
program temporarily. The user is then put into diagnostic mode to examine the
contents of internal files, dump them to disk, etc. After the user leaves
diagnostic mode, STOP is prompted for. If a true value is given, then MEMCRDD
terminates in the same manner as when the iteration limit is reached (see
parameter NITER). If a false value is given for STOP, then MEMCRDD continues
with the next MEMSYS3 iteration.
\item [DEFAULT]:
None.
\item [RESTRICTIONS]:
TRUE, FALSE, YES, NO or any abbreviation.
\item [ADVICE]:
MEMCRDD is rarely run interactively.
\end {description}

\rule{\textwidth}{0.3mm}
{\Large {\bf STOPFILE} (!!)}
\begin{description}
\item [USE]:
It is possible that the user may want to stop a MEMCRDD job running in batch
without losing everything. After each iteration, MEMCRDD checks to see if a file
exists with the name specified for the STOPFILE parameter. If such a file does
exist then the program is terminated tidily in the same manner as if the
iteration limit had been reached. So if the user specifies a value of FRED.LIS
(say) for STOPFILE when the job is run, he can at any time create a file called
FRED.LIS which will cause MEMCRDD to terminate after the end of the current
iteration.
\item [DEFAULT]:
The default is a null value which causes no file checks to be made between
iterations.
\item [RESTRICTIONS]:
Must be a legal VMS file name.
\item [ADVICE]:
Note, the file is NOT deleted by MEMCRDD. Thus if the same job is re-submitted,
it would find the stop file on the first iteration and terminate immediately.
Always make sure you delete stop files after they have been finished with.
\end {description}

\rule{\textwidth}{0.3mm}
{\Large {\bf STRIPE} (!)}
\begin{description}
\item [USE]:
The RMS spread in detector offsets to use when simulating data (see parameter
FUNCTION). This will introduce typical IRAS stripes into the simulated data.
\item [DEFAULT]:
0.0
\item [RESTRICTIONS]:
None.
\item [ADVICE]:
None.
\end {description}

\rule{\textwidth}{0.3mm}
{\Large {\bf TOL} (!)}
\begin{description}
\item [USE]:
This specifies the fractional tolerance to which MEMSYS3 works.
\item [DEFAULT]:
0.1
\item [RESTRICTIONS]:
Must be between 0.0 and 1.0
\item [ADVICE]:
Smaller values for TOL cause MEMCRDD to take much more CPU time.
\end {description}

\rule{\textwidth}{0.3mm}
{\Large {\bf XSCAN} (!!)}
\begin{description}
\item [USE]:
This specified the cross-scan uncertainty in the pointing data, in arcminutes.
It is used to generate pointing noise estimates if a value for the SKYIN
parameter is given when FUNCTION=HIRES. It is also used to
add pointing errors into simulated CRDD when FUNCTION=SIMULATE.
\item [DEFAULT]:
The default is a null value which causes MEMCRDD to use a value dependant on the
IRAS band. The bands 1 to 4 have values of 0.2, 0.2, 0.25 and 0.25 respectively.
(see the IRAS Catalogs and Atlases Explanatory Supplement pages VII-6 and VII-7).
\item [RESTRICTIONS]:
The value must be positive or zero.
\item [ADVICE]:
The default value should be OK.
\end {description}

\rule{\textwidth}{0.3mm}
{\Large {\bf UDIAG } (!)}
\begin{description}
\item [USE]:
UDIAG is only prompted for if parameter FUNCTION has the value ANALYSE.
If UDIAG is given a true value then diagnostic mode is entered before performing
the analysis. This allows the user to examine the internal files, dump them to
disk, etc.
\item [DEFAULT]:
NO.
\item [RESTRICTIONS]:
YES, NO, TRUE, FALSE or any abbreviation.
\item [ADVICE]:
In diagnostic mode the user is given a list of self explanatory options which
are selected using the parameter OPTION. The contents of internal files are
described in the MEMSYS3 users manual.
\end {description}

\rule{\textwidth}{0.3mm}
{\Large {\bf VARIN} (!!)}
\begin{description}
\item [USE]:
If parameter SKYIN has been given a value in order to continue a ``noise loop''
(see section \ref {SEC:ICFB}), then VARIN should be given the name of the file
produced from the previous run of MEMCRDD, through the VAROUT parameter. This
file holds the variances used in the previous iteration, which are used to lag
the new variances calculated on the basis of the SKYIN image (see parameter
SPEED). If no file is given then the new variances are not lagged, and the
parameter SPEED is forced to take a value of 1.0.
\item [DEFAULT]:
A null value, causing no lagging of the variance values.
\item [RESTRICTIONS]:
Must be a file produced by a previous run of MEMCRDD (see parameter VAROUT).
\item [ADVICE]:
None.
\end {description}

\rule{\textwidth}{0.3mm}
{\Large {\bf VAROUT} (!!)}
\begin{description}
\item [USE]:
Specifies a BDF name to which will be written the variances used in the
current run of MEMCRDD. A null value causes no file to be produced.
\item [DEFAULT]:
A null value.
\item [RESTRICTIONS]:
None.
\item [ADVICE]:
If performing a ``noise loop'', the created file should be specified as the
value for parameter VARIN on the next run of MEMCRDD.
\end {description}

\section{Using Pointed Observation CRDD rather than Survey CRDD}
\label {APP:AO}

AO CRDD is not currently distributed within the UK, but visitors to IPAC
can obtain such data in the form of ``footprints'' tapes. It is expected
that such data will eventually be available within the UK, and the ADAM
package ``IRAS90'' has been designed in such a way as to make it easy
to add support for this data. Until then, a rather circuitous route must
be followed in order to use footprints obtained at IPAC. This involves
using the FITSIN application within KAPPA (see SUN/95) to read the
footprint tapes into a set of data files in NDF format, and then using
AOCRDD (within EDRSX) to convert these NDFs into BDFs which can be used
by MEMCRDD. A further complication is introduced by the fact that the
FITS GROUP format used at IPAC is non-standard. This means that a version
of FITSIN specifically modified to handle IPAC footprints tapes must be used,
rather than the KAPPA version of FITSIN. IPAC have made a decision to move
to standard group format, but as yet no data has been seen in the UK which
is in the standard format.

All this has been packaged into a DCL command procedure, READFOOT.COM. To
use this procedure issue the DCL command

\begin{verbatim}
   $ @EDRSX:READFOOT
\end{verbatim}

The user is asked for a prefix for the file names, and the name of a tape
drive. The user should then mount the IPAC footprints tape on the drive. The
data is then read from the tape into many .SDF files (one per detector per leg
of the AO). These SDF files are converted into BDF files (one per band per leg
of the AO), and these files are destriped by subtracting the median value from
each detector data stream. The final destriped files are called
\verb+<prefix>+\_BiLj\_DS.BDF  where \verb+<prefix>+ is the string supplied by
the user in response to the prompt issued at the start, i is the band number (1
to 4), and j is the leg number in the range 1 to the maximum number of legs.
The leg number is equivalent to the scan number for a survey CRDD file.

To use these destriped AO CRDD files within MEMCRDD, the parameter DATATYPE
should be given the value AO, and the CRDD file names should be assigned
to CRDDF1-CRDDFn in the normal way.

\section{Changes since version 1 of this document}
\begin {enumerate}
\item The job name is now restricted to 8 characters (see section \ref
{SEC:RUN}).
\item The command procedure MEMJOB has been released (see section \ref
{SEC:RUN}). This makes it easier to use the more sophisticated noise model
described in section \ref {SEC:NOISE}.
\item An interim method of using Pointed Observation footprints data (available
from IPAC) has been included (see appendix \ref {APP:AO}).
\end{enumerate}

\end{document}
