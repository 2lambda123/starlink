\documentstyle[11pt]{article}
\pagestyle{myheadings}

%------------------------------------------------------------------------------
\newcommand{\stardoccategory}  {Starlink User Note}
\newcommand{\stardocinitials}  {SUN}
\newcommand{\stardocnumber}    {138.1}
\newcommand{\stardocauthors}   {M McSherry}
\newcommand{\stardocdate}      {9th December 1991}
\newcommand{\stardoctitle}     {HRTS --- NRL High Resolution Telescope and 
Spectrograph Package}
%------------------------------------------------------------------------------

\newcommand{\stardocname}{\stardocinitials /\stardocnumber}
\renewcommand{\_}{{\tt\char'137}}     % re-centres the underscore
\markright{\stardocname}
\setlength{\textwidth}{160mm}
\setlength{\textheight}{230mm}
\setlength{\topmargin}{-2mm}
\setlength{\oddsidemargin}{0mm}
\setlength{\evensidemargin}{0mm}
\setlength{\parindent}{0mm}
\setlength{\parskip}{\medskipamount}
\setlength{\unitlength}{1mm}

\begin{document}
\thispagestyle{empty}
SCIENCE \& ENGINEERING RESEARCH COUNCIL \hfill \stardocname\\
RUTHERFORD APPLETON LABORATORY\\
{\large\bf Starlink Project\\}
{\large\bf \stardoccategory\ \stardocnumber}
\begin{flushright}
\stardocauthors\\
\stardocdate
\end{flushright}
\vspace{-4mm}
\rule{\textwidth}{0.5mm}
\vspace{5mm}
\begin{center}
{\Large\bf \stardoctitle}
\end{center}
\vspace{5mm}

\setlength{\parskip}{0mm}
\tableofcontents
\setlength{\parskip}{\medskipamount}
\markright{\stardocname}

\newpage

\LARGE
\begin{center}
{\bf Preface}
\end{center}
\normalsize

The Naval Research Laboratory (NRL) High Resolution Telescope and Spectrograph
(HRTS) recorded high quality ultraviolet spectra of the Sun during 6 rocket 
flights (21 July 1975, 13 February 1978, 1 March 1979, 11 December 1987 
and 20 November 1988) and during extended operations on the Space 
Shuttle Spacelab 2 between 29 July to 6 August 1985. The spectral region 
covered by the spectrograph was $1170 \ - \ 1700\AA$. This region 
contains many important chromospheric (C I, O I, Fe II), transition 
region (Si III, Si IV, O IV, C IV) and coronal (Fe XII, Fe XXI) lines. 
The spectrograph has a spectral resolution better than $0.1\AA$ and a 
spatial resolution between 1 and 2 arc sec. Solar features observed 
during the various missions were (i) sunspot and the quiet solar limb 
(21 July 1975 \& 13 February 1978), (ii) sequential spatial rasters of 
an extended solar region roughly centred on C IV (1 March  1979), 
sunspot and the solar limb inside a polar coronal hole (Spacelab), (iii) 
a coordinated HRTS and AS\&E X-ray programme --- HRTS performed a large 
area raster near $1550\AA$ (1 December 1987) and (iv) solar wind studies 
(20 November 1988). For further information about HRTS and observational 
data, interested users should contact Dr K. Dere (Naval Research Lab., 
Washington --- e-mail {\tt 11323::DERE}).

 The guide which follows is intended to be used in conjunction with the 
 HRTS package, which was originally developed at N.R.L. by {\bf John Ewing}
 \& {\bf Ken Dere} but has now been modified for general Starlink use.
 The guide details each program
 in the package, which has been divided up according to the categories:
 data analysis, data reduction and miscellaneous. These sections are also
 further divided according to language, either DCL$+$Fortran or IDL.


\newpage

\section{Introduction}

This manual describes the various programs available for HRTS data-reduction
and data-anal\-ysis.  Some programs are to be run from DCL level, while others
are to be run from IDL. A short description of HRTS data files and how to
examine them is given in the following pages.

HRTS-related data files consist of both data and header information. The header
is usually contained in the first record of the data file, and contains
information describing the data. These files are created in a manner which
strives to conserve disk storage, and cannot be edited with EDT or any other
text editor.

The use of the IDL environment is recommended for examining this data primarily
for its flexibility and ease of graphics generation. A common practice is to
enter IDL (by typing IDL) and then to load some data into memory by using the 
GETDATA  procedure. If one has a two-dimensional array of data in memory one
may then wish to display it on a colour monitor (by using the TVSTART and TVSCL
commands), or plot a cross section of the data on a graphics terminal (by using
the intrinsic PLOT procedure), plot a histogram of the data (by using the
intrinsic HISTOGRAM and PLOT procedures), print out numerical values, or try
other options. 

To find out more details about each file one may examine that file's header
information. There are programs for both the DCL and IDL environments which
allow one to do this (both programs are named  HDR). Most of the programs used
for data-reduction are written in FORTRAN since that language is more efficient
than IDL for that purpose. The data-reduction programs will not be of much
interest for most users.

Notes concerning the names of arguments for IDL procedures and functions as
they appear in this guide:
\begin{enumerate}   
\item Argument names shown in bold type are required arguments.
\item An undefined variable must not be passed in as an argument whose name is 
in italics.
\end{enumerate}
\begin{description}
\item [Example: ] {\tt HRTS\_IDL>}{\bf GETMMT,ARR}{\it FILENAME}
\end{description}

\newpage

\section{Example Section for New Users}


This section gives details regarding how to examine different types of HRTS
data files.  The files that are mentioned below are permanently stored within
the DEMO directory, \\(HRTS\_DISK:[HRTS.DATA.DEMO] ) expressly for the purpose
of allowing this demonstration.  Commands which are to be typed in by the user
are shown after the `\verb+IDL>+' prompt.

\begin{verbatim}

            <Enter the HRTS Software Package>
            $ HRTS
        HRTS$>                       
            <Activate the image display system>
        HRTS$> idl                                    ! Enter IDL environment


            <1. Examine a .CIS file via the image display>

        HRTS_IDL> getdata,a,`DEMO1'                   ! DEMO1 = E03013548.CIS
        HRTS_IDL> tvstart
        HRTS_IDL> tvscl,a(*,0:1023)


            <2. Examine a .FSS file via the image display>

        HRTS_IDL> getdata,a,`DEMO2'                   ! DEMO2 = E03013548.FSS
        HRTS_IDL> tvstart
        HRTS_IDL> tvscl,a(*,0:1023)


            <3. Examine a .MMT file via the image display>

        HRTS_IDL> getmmt,a,`DEMO3'                    !DEMO3 = E03013548.MMT
        HRTS_IDL> plot,a


            <4. Examine a .IMG file via the image display>

        HRTS_IDL> tvimg,`                   
        HRTS_IDL> Inappropriate or non-existing file: .IMG
        HRTS_IDL> Re-enter the file-name(`q'-quit): DEMO4
        HRTS_IDL> Please enter image display device:
                                                      !DEMO4 = R43S15I2C4OE.IMG
        HRTS_IDL> exit
        HRTS$>

\end{verbatim}

\newpage

\section{Description of HRTS Data Files} 

\begin{center}
\begin{tabular}{||l l l||}
\hline
{\bf IMAGE} & {\bf FILE-TYPE} & {\bf DESCRIPTION}\\
\hline
{\em 1} & {\em arbitrary} & Any data file which does not fall into one of \\
        &                 & the categories listed below.\\
        &                 &                             \\

{\em 2} & {\em .RDS} & {\bf Rectilinear Densitometer Scan ---} A two dimensional 
\\
        &            &      array of density values throughout a region of a \\
        &            &      plate.\\
        &            &            \\
  
{\em 3} & {\em .CDF} & {\bf Curvilinear Densitometer Scan ---} A two dimensional 
\\
        &            &      array of density values adjusted for curvature,\\
        &            &      warping and magnification.\\
        &            &                                \\

{\em 4} & {\em .CIS} & {\bf Curvilinear Intensity Scan ---} A two dimensional \\
        &            &      array of intensity values adjusted for curvature,\\
        &            &      warping and magnification.\\
        &            &                                \\

{\em 5} & {\em .FSS} & {\bf Flat Stigmatic Spectra ---} A two dimensional array \\
        &            & array of intensity values in which the dispersion is \\
        &   & linear and $\lambda$ is a function of column number only,\\
  &  &    {\it i.e.} the $\lambda$ correction has been applied to the data. \\
  &    &  \\

{\em 6} & {\em .SSS} & {\bf Summed Stigmatic Spectra ---} A composite spectral 
line \\    
  &   &  formed by summing the intensities from various \\
  &   &    actual spectral lines together.\\
  &   &  \\

{\em 7} & {\em .MMT} & {\bf Moment Data ---} A series of one dimensional arrays of
 \\

        &            & intensity, velocity and/or line width as a function \\
        &            & of position for various ions. \\
        &            &                               \\     

{\em 8} & {\em .IMG} & {\bf Moment Image ---} a three dimensional array containing 
\\
        &            &  various moments as a function of position in an x-y \\
        &            &  plane. \\
        &            & \\
\hline
\end{tabular}
\end{center}

\newpage

\section{Description of Header Variables}

    These variables are listed in what may roughly be construed as the order
    of importance.  Knowledge of these variables is not essential for most
    applications.  Not all variables are defined for each image.  Variables
    with a `*' to the left are arrays.
\begin{description}
\begin{description}
\item[NDIM] -- {\em number of dimensions of the data array}
\item[* ISIZ] -- {\em size of each dimension of the data array}
\item[DTYPE] -- {\em data type (following the IDL `SIZE' convention)}
\item[IMAGE] -- {\em image type}
\item[NINT] -- {\em number of integer variables in the file header}
\item[NFLT] -- {\em number of floating point variables in the file header}
\item[NDATSCL] -- {\em type of scaling performed on the data}
\item[IVARS] -- {\em number of files summed together to make composite image}
\item[TEXT/HDRTXT] -- {\em title of the scan}
\item[* ID] -- {\em names of each dimension of the data array}
\item[* UNIT] -- {\em units of each respective dimension}
\item[* SCUNIT] -- {\em scaled units of each respective dimension}
\item[IDDAT] -- {\em the type of quantity which the data represents}
\item[UNITDAT] -- {\em units of the data}
\item[HRTS] -- {\em name of the particular HRTS instrument}
\item[PLATE] -- {\em name of the plate which was digitized}
\item[SEQ] --  {\em sequence number}
\item[INST] -- {\em institution at which the plate was digitized}
\item[DENSIT] -- {\em name of the densitometer employed to scan the plate}
\item[SCDATE] -- {\em date of the scan}
\item[IYEAR] -- {\em year of the measurement}
\item[MONTH] -- {\em month of the measurement}
\item[IHOUR] -- {\em hour of the measurement}
\item[MINUT] -- {\em minute of the measurement}
\item[ISEC] -- {\em second of the measurement}
\item[MILSEC] -- {\em millisecond of the measurement}
\item[* ISPOFST] -- {\em spatial series offset for each dimension}
\item[* ITSOFST] -- {\em time series offset for each dimension}
\item[* MVOFST] -- {\em movie series offset for each dimension}
\item[NMMTWVL] -- {\em number of moment lines}
\item[* IDMMT] -- {\em identification of each line}
\item[* NMMT] -- {\em number of moments of each line}
\item[* NELMMT] -- {\em number of elements of each line}
\item[* IPLATE] -- {\em plate number}
\item[IOFST] -- {\em offset of each respective plate}
\item[IDIMG] -- {\em identification of line}
\item[NMMTIMG] -- {\em number of moments of line}
\item[NELIMG] -- {\em number of elements of line}
\item[* W0] -- {\em initial value for each respective dimension}
\item[* WINC] -- {\em step size for each respective dimension}
\item[SLITW] -- {\em densitometer slit width}
\item[SLITH] -- {\em densitometer slit height}
\item[CPD] -- {\em clear plate density}
\item[DSS] -- {\em densitometer scan speed (mm/s)}
\item[BACKLASH] -- {\em densitometer backlash}
\item[WVLO] -- {\em initial wavelength}
\item[EXPT] -- {\em plate exposure time (s)}
\item[FMIN] -- {\em minimum usable data value}
\item[FMAX] -- {\em maximum usable data value}
\item[WVLSTP] -- {\em wavelength at which dim\#2 step size is applied}
\item[WVLCFW] -- {\em wavelength at which CFW position is measured}
\item[XCFW] -- {\em position of CFW along dim\#1 in pixels}
\item[YCFW] -- {\em position of CFW along dim\#2 in pixels}
\item[WVLRE] -- {\em reference wavelength (Angstroms)}
\item[XREF] -- {\em reference position along dim\#1}
\item[YREF] -- {\em reference position along dim\#2}
\item[* DATSCL] -- {\em scaling factor for each respective dimension}
\item[* NSCL] -- {\em number of scaling factors for each dimension}
\item[* SCL] -- {\em scaling factors}
\item[NCRV] -- {\em number of parameters user for curvature correction}
\item[* CRV] -- {\em curvature correction parameters}
\item[* NSFF] -- {\em number of spectral filtering factors for each dimension}
\item[* SFF] -- {\em spectral filtering factors}
\item[NDNCAL] -- {\em number of density values used for calibration}
\item[NWVLCAL] -- {\em number of wavelength values used for calibration}
\item[* DNCAL] -- {\em density values used for calibration}
\item[* WVLCAL] -- {\em wavelength values used for calibration}
\item[* CAL] -- {\em calibration coefficients}
\item[NSUMWVL] -- {\em number of summed lines}
\item[* SUMWVL] -- {\em wavelength of each summed line}
\item[* SUMNRM] -- {\em normalization value of each line}
\item[* SUMMED-FILES] -- {\em names of files summed together to make composite 
image}
\end{description}
\end{description}

\newpage

\section{Quick Program Reference}

\subsection{HRTS data-analysis programs --- DCL level}

\begin{description}
\begin{description}
\item[HDATA] --      Examine details concerning processed data.
\item[HDIR] --       Make a directory listing showing data array sizes.
\item[HDR] --        Examine and/or change the header information of data files.
\item[INTPLT] --     Plot intensity/density of a spectral file for a given scan 
line.
\item[MMTPLT3] --   Plot intensity, velocity, and line width for a selected ion.
\item[MMTTYP] --     Type out values of moments on the terminal.
\item[PLIST] --      Generate a list of processed HRTS data.
\item[PROFILE] --    Plot intensity vs. ($\lambda$-$\lambda$rest) for various 
line profiles.
\item[PRTHEADR] --   Print out all of the header information for a given data 
file.
\item[WVLPLT] --     Plot intensity vs.$\lambda$ for one scan line of a spectral 
file
\end{description}
\end{description}

\subsection{HRTS data-analysis programs --- IDL level}

\begin{description}
\begin{description}
\item[GETDATA] --    Reads data from a file into an IDL array.
\item[GETDATAWVL] -- Read data within a specified wavelength range from a file.
\item[GETMMT] --     Read moment data from an image type \#7 file (.MMT, .MOD, 
etc).
\item[HDR] --        Examine and/or change the header information of data files.
\item[MMTCALC] --    Calculate the first three moments for a single profile.
\item[MMTPLT] --     Plot intensity, velocity, and line width for a specified 
ion.
\item[PLOTPROFILE] --Plot specified profiles vs. ($\lambda$-$\lambda$rest) from 
a spectral file.
\item[READCAL] --    Obtain density and log intensity values from a .CAL file
\item[PUTDATA] --    Write data from an IDL array into a file.
\item[TVEXTRACT] --  Extract sections of data from a spectral file.
\item[TVIMG] --      Visually and numerically examine .IMG data.
\item[TVIMGPROFILE] -- Plot profiles by selecting positions within a .IMG file.
\item[VARVELCT] --   Obtain a modified colour table for velocity images.
\item[VELCT] --      Obtain the standard colour table used for velocity images.
\item[WAVEL] --      Obtain the wavelength scale for data read in by GETDATA.
\end{description}
\end{description}

\subsection{HRTS data-reduction programs --- DCL level}

\begin{description}
\begin{description}
\item[C4DFAC] --     Make either .CDF or .CIS data from input .RDS data.
\item[DEN\_INT] --    Create a .CIS file from a .CDF file, or a .FSS from an 
.FSD.
\item[FLIPX] --      Flip a 2-D HRTS image in the X-direction.
\item[FLIPY] --      Flip a 2-D HRTS image in the Y-direction.
\item[FSS] --        Create a .FSS file from a .CIS file.
\item[GETCIS] --     Invoke C4DFAC to make .CDF's or .CIS's from .RDS files.
\item[MMTIMG] --  Create a .IMG file from a series of moment (.MMT or .MOD) 
files.
\item[MMTMOD] --  Create a modified moment file (.MOD) from a moment file 
(.MMT).
\item[MOMENT] --     Create a moment file (.MMT) from a spectral file (.CIS).
\item[RESTORECIS] -- Restore the header of a .CDF or .CIS to its original state.
\item[STNDRD] --     Modify a spectral file's header to improve $\lambda$ 
associations.
\item[SUM] --        Create a .SSS-type file from any spectral file.
\item[SUMCDF\_IG] --  Make a composite intensity image by combining .CDF or .FSD 
data.
\end{description}
\end{description}

\subsection{HRTS data-reduction programs --- IDL level}

\begin{description}
\begin{description}
\item[ATODATA] --    Read ato data from a .ATO file into an IDL array.
\item[SCALIM] --     Generate a .SMG file from a .IMG file.
\item[SETCRV] --     Adjust a spectral file's header to improve $\lambda$ 
associations.
\item[STNDRD] --     Estimate a correction that will improve $\lambda$ 
associations.
\item[SUMCDF\_IG] --  Set up a batch job to run SUMCDF\_IG.FOR.
\end{description}
\end{description}

\subsection{H\&D curve development programs --- DCL level}

\begin{description}
\begin{description}
\item[AMOEBA] --  Produce H\&D curve coefficients, given at least two density 
histograms.
\item[MAKEHIST] --   Generate histogram of density values for an image.
\item[TESTHND] -- Evaluate the appropriateness of a set of H\&D curve 
coefficients.
\end{description}
\end{description}

\subsection{H\&D curve development programs --- IDL level}

\begin{description}
\begin{description}
\item[GRHIST] --     Plot a histogram of density values.
\item[GRHND] --      Plot an H\&D curve.
\end{description}
\end{description}

\subsection{Miscellaneous programs --- DCL level}

\begin{description}
\begin{description}
\item[CALC] --       Perform a calculator simulation.
\item[IONEQ] --      Get ion equilibria from the tables of Arnaud et al.
\item[TMAX] --       Calculate temperature maximum values.
\end{description}
\end{description}

\subsection{Miscellaneous programs --- IDL level}

\begin{description}
\begin{description}

\item[COLSUM] --     Sum the columns of a 2-D array.
\item[DISPLAY] --    Display an image on a graphics terminal.
\item[INT] --        Display an image on a graphics screen with 9 brightness 
levels.
\item[POLYFIT] --    Fit a polynomial to a weighted set of (x,y) observation 
points.
\item[POS] --        Read values off a graph plotted on a graphics terminal
\item[ROWSUM] --     Sum the rows of a 2-D array.
\item[TVALIGN] --    Use the image display to align two images.
\item[TVCOMPARE] --  See the sum and difference of two images.
\item[TVCUT] --      Read data from an arbitrary area of the image screen.
\item[TVPOS] --      Print out the current coordinates of the TV cursor.
\item[TVRESTORE] --  Put an image onto the screen that was saved via TVSAVE.
\item[TVSAVE] --     Save an image from the screen into a file.
\item[TVSTART] --    Prepare the image display unit to display images.
\end{description}
\end{description}

\clearpage

{\small
\setlength{\unitlength}{1.0pt}

\begin{figure}
\begin{center}
\begin{picture}(150,300)
\thicklines
\put (-100,450){\Large \bf  5.9 Flow Chart for Data Reduction Programs}
\put (-20,370){\framebox(200,45){\Huge GETCIS}}
\put (-200,270){\framebox(60,25){\Large \it FSS}}
\put (280,270){\framebox(60,25){\Large \it FSS}}
\put (40,290){\framebox(90,20){\large \it MOMENT}}
\put (15,200){\framebox(60,20){\it MMTMOD}}
\put (40,110){\framebox(90,20){\large \it MMTIMG}}
\put (240,170){\framebox(60,25){\large \it DEN\_INT}}
\put (55,25){\framebox(60,20){\large \it SCALIM}}
\put (-25,-10){\framebox(60,25){\Large \it SUM}}
\put (128,-10){\framebox(60,25){\Large \it SUM}}
\put (195,70){\framebox(70,20){\it SUMCDF\_IG}}
\put (50,370){\vector(0,-1){59}}
\put (120,370){\vector(0,-1){59}}
\put (29,345){\circle{25}}
\put (21,342){CIS}
\put (141,345){\circle{25}}
\put (129,342){CDF}
\put (51,345){\line(-1,0){9}}
\put (130,345){\line(-1,0){9}}
\put (8,370){\vector(0,-1){353}}
\put (160,370){\vector(0,-1){353}}
\put (-15,155){\circle{25}}
\put (-23,152){CIS}
\put (7,155){\line(-1,0){10}}
\put (183,155){\circle{25}}
\put (171,152){CDF}
\put (171,155){\line(-1,0){10}}
\put (110,290){\vector(0,-1){159}}
\put (134,220){\circle{30}}
\put (121,217){MMT}
\put (119,220){\line(-1,0){8}}
\put (70,289){\vector(-1,-2){33}}
\put (35,265){\circle{30}}
\put (22,262){MMT}
\put (57,265){\line(-1,0){6}}
\put (35,198){\vector(2,-3){43}}
\put (35,160){\circle{30}}
\put (23,157){MOD}
\put (59,160){\line(-1,0){9}}
\put (84,110){\vector(0,-1){63}}
\put (110,79){\circle{30}
\put (-10,-2){IMG}}
\put (94,79){\line(-1,0){9}}
\put (84,25){\vector(0,-1){100}}
\put (8,-10){\vector(0,-1){100}}
\put (160,-10){\vector(0,-1){100}}
\put (110,-25){\circle{30}}
\put (-18,-50){\circle{39}}
\put (187,-50){\circle{39}}
\put (99,-29){SMG}
\put (-33,-53){SSS\_c4}
\put (171,-53){SSS\_c4}
\put (94,-25){\line(-1,0){9}}
\put (168,-50){\line(-1,0){7}}
\put (7,-50){\line(-1,0){8}}
\put (-12,370){\line(0,-1){40}}
\put (173,370){\line(0,-1){39}}
\put (-12,330){\line(-1,0){160}}
\put (173,330){\line(1,0){140}} 
\put (225,330){\vector(0,-1){238}}
\put (-172,330){\vector(0,-1){33}}
\put (270,330){\vector(0,-1){133}} 
\put (312,330){\vector(0,-1){33}}
\put (325,269){\vector(-1,-2){36}}
\put (-90,350){\circle{25}}
\put (245,350){\circle{25}} 
\put (-99,347){CIS}
\put (233,347){CDF}                    !
\put (245,338){\line(0,-1){7}}
\put (-94,338){\line(0,-1){7}}
\put (290,245){\circle{25}}
\put (280,242){FSD}
\put (303,245){\line(1,0){7}}
\put (250,170){\vector(0,-1){50}}
\put (290,170){\vector(0,-1){200}}
\put (225,70){\vector(0,-1){150}}
\put (255,0){\circle{48}}
\put (308,90){\circle{25}}
\put (235,-3){cis\_sum}
\put (300,86){CIS}
\put (235,0){\line(-1,0){9}}
\put (295,88){\line(-1,0){5}}
\put (270,150){\circle{25}}
\put (260,147){FSS}
\put (250,149){\line(1,0){8}}
\put (-172,270){\vector(0,-1){350}}
\put (-195,95){\circle{30}}
\put (-204,91){FSS}
\put (-172,93){\line(-1,0){8}}
\put (-100,-130){\em Square boxes refer to command routines; the circles are the 
resultant files.}
\end {picture}
\end{center}
\end{figure}
\clearpage
\setlength{\unitlength}{1mm}}

\section{Data Analysis Programs}

\subsection{DCL level}

\subsubsection{HDATA}
  
{\em To run this program type:}
\begin{verbatim}
     HRTS$> HDATA
\end{verbatim}
   {\em HDATA} may be used to display information about processed HRTS data.
    When the program is invoked, it starts off by showing a screen which
    displays identification names of processed HRTS data.  These names have
    been organized into `pages', of which one page may be shown on the
    screen at any time.  The user may inspect other pages by hitting the
    `N' key (this and all other available commands are listed on the bottom
    of the screen).
    While a page of processed data is displayed on the screen, a user
    may examine the header data or processing details for the various data
    files associated with any identification name displayed on the page.
    An example is shown below:
\begin{center}
\begin{tabular}{|l|l|l|l|l|}     
\hline
\multicolumn{4}{|l}{\bf Processed Data List} & \multicolumn{1}{l|}{\bf Page 1 of 
4} 
\\ \hline     
       A00101216 &  A00201640 &  E02802302 &  E18415548 &  E18516548  \\
       A00101230 &  A00201656 &  E02802335 &  E18416548 &  E18517548  \\
       A00101302 &  A00401216 &  E02802355 &  E18417548 &  E18518548  \\
\hline \hline
\multicolumn{2}{||l}{\bf Processing Details} & 
\multicolumn{3}{l||}{E18404548.CIS;1} 
\\  
\hline \hline
\multicolumn{2}{||l}{Program Version} & \multicolumn{3}{l||}{\em 
C4DFAC\_2.0(Oct.86)} \\              
\multicolumn{2}{||l}{Program File Name} & \multicolumn{3}{l||}{\em 
C4DFAC.FOR;86} \\                   
\multicolumn{2}{||l}{Date of Execution} & \multicolumn{3}{l||}{\em 12-JAN-87}    
 \\                   
\multicolumn{2}{||l}{ Calibration Data} & \multicolumn{3}{l||}{\em 
R067F550.CAL;1} \\                   
\multicolumn{2}{||l}{Curvature Data} & \multicolumn{3}{l||}{\em CURV.DAT;1} \\   
\multicolumn{2}{||l}{Magnification Data} & \multicolumn{3}{l||}{\em 
MAGN.DAT;1}\\                      
\multicolumn{2}{||l}{Dispersion Data} & \multicolumn{3}{l||}{\em DISP.DAT;1} \\                   
\multicolumn{2}{||l}{Synth. Spectra Data} & \multicolumn{3}{l||}{\em ATO.548;1} 
\\                       
\multicolumn{2}{||l}{Carbon IV Treatment} & \multicolumn{3}{l||}{\em YES}  \\                            
\hline \hline
       A00201555 &  E02801555 &  E18413540 &  E18514548 &  E18604548 \\
       A00201613 &  E02802216 &  E18414548 &  E18515548 &  E18605548 \\
\hline                
\end{tabular}
\end{center}                
    In the example, the user used the arrow keys and the return key to
    select the identification name `E18404548' and then hit the `P' key in
    order to request to see processing information.  The user then
    manipulated the arrow and return keys again to select the specific file
    `E18404548.CIS;1' which then resulted in the display of processed data
    as shown.  At this point the program instructs the user to hit any key
    in order to continue, after which the processing information will      
    removed from the screen and other commands may be entered.
    To examine header information, the user would employ the same
    procedure as illustrated above, only he would hit `H' instead of `P'.
    The user may exit the program by hitting the `E' key.

\subsubsection{HDIR}                

{\em To run this program type:}    
\begin{verbatim}
     HRTS$> HDIR <file_specification>   
\end{verbatim}
   {\em HDIR} is used to make directory listings which show file names and
    data array sizes.  The program begins by asking for a directory and or file
    (e.g. *.CIS). It then lists all files in the designated directory.
    A sample usage of HDIR is shown below:
         
\begin{verbatim}

           HRTS$> hdir 
                                            
              Directory  
                
              E03013548.CIS;1        447  1856
              E03013548.FSS;2        414  1856
              E03013548.MMT;2       1856     9
              E03013548.SMT;4         92     3
              E03013548.STNPLT;1  <not HRTS data>
              E03013548.STNPRT;4  <not HRTS data>
              R43S15I2C4OE.IMG;3    1879    10     2
              R43S15I2C4OE.SMG;2    1879    10     2
                
              Total of 8 files.

             HRTS$>              

\end{verbatim}

\subsubsection{HDR}

{\em To run this program type:}
\begin{verbatim}
     HRTS$> HDR                  
\end{verbatim}
  This program allows the user to examine the header information of
  HRTS data files.  When the program is run, it will ask the user to type
  in a file name.  A menu of variable names is then displayed (shown as
  follows) in which one may select a variable name by using the arrow
  keys and the return key.
\begin{center}
\begin{tabular}{||l l l l l ||}
\hline


           HRTS   &     Inst  &     WvlRef  &    DType  &     DatScl \\    
           Plate  &    Densit &     HdrTxt  &    IVars  &     NScl   \\   
           Seq    &    SlitH  &      Image  &      Id   &     Scl    \\   
           ExpT   &    SlitW  &      NDim   &     Unit  &     NCrv   \\   
           IYear  &  BackLash &      ISiz   &    ScUnit &     Crv    \\   
           Month  &     XRef  &       W0    &     FMin  &     NSff   \\   
           IDay   &     YRef  &      Winc   &     FMax  &     Sff    \\   
           CPD    &     XCFW  &      IdDat  &    NDnCal &     Cal    \\   
           ScDate &      YCFW &    UnitDat  &     DnCal &    HndInt  \\   
\hline
\end{tabular}                                        
\end{center}
   After making a selection, the value of the selected variable is then
   displayed on the screen.  The user may hit `H' to obtain help
   information.  The `N' key may be hit if the user wishes to examine the
   header of another file after he is done examining the current header
  (the program will then ask for the name of the next file).  The user
   also has the ability to change values of header variables within the
   permanent file by hitting `C' instead of carriage return when selecting
   a variable, and then typing in the new value for that variable.  To
   exit the program, the user may select the exit box, or else simply hit
   the `E' key.

\subsubsection{INTPLT}

{\em To run this program type:}
\begin{verbatim}
     HRTS$> INTPLT                       
\end{verbatim}
  {\em INTPLT} plots intensity/density of a spectral data file vs. position
   for a given scan-line number.  The program first asks the user to type
   in the input file name.  The program then asks the user a question
   to which the user may simply hit $<$return$>$ to use default value.  The
   question asks if the user wants the data plotted in the reverse
   direction. The program then displays the number of scan lines of data, and 
the number
   of points per scan line.  The program will next display the following
   request, `TYPE ISCAN,IPT1,IPT2',  and then wait for the user to type in
   three numbers; separated by commas. The first must be the number of the 
desired scan line
  (if the user gives a negative number for this then the program will
   exit) and the second and third numbers will be interpreted as
   indicating the range of points along the scan line which will be
   displayed.  A graph will now be drawn on the graphics terminal, after
   which the user may type return to get the `TYPE ISCAN,IPT1,IPT2'
   message again and so either exit the program (by typing a negative
   number) or else plot other scan lines.

\subsubsection{MMTPLT3}

{\em To run this program type:}   
\begin{verbatim}
     HRTS$> MMTPLT3                      
\end{verbatim}
  {\em MMTPLT3} will make plots of moment data on the terminal graphics
   screen.  When run, this program will first ask for the complete name of
   the moment file to be examined.  After that is typed in, the program
   will then display information about the moments in a manner similar to
   that shown below:
\begin{center}
\begin{tabular}{||l l l l l ||}
\hline 
\multicolumn{5}{||l||}{HRTS, SL-2, AUG 02, 1985, REV 56, SEQ 61, EXPT 5.0S,
C SCANS} \\
          {\em 1  C IV}  &      &      1548.202  &   1.000  &  3 \\             
          {\em 2  C IV}  &      &      1550.774  &   1.000  &  3 \\             
          {\em 3  Si I}  &      &      1552.209  &   1.000  &  2 \\             
          {\em 4  CONTINUUM} &  &      0.000     &   1.000  &  1 \\             
\multicolumn{2}{||l}{NO OF SCANS} & \multicolumn{3}{l||}{2002} \\    
\multicolumn{5}{||l||}{TYPE ILINE,MMMT,ISCAN1,ISCAN2, NEGATIVE ILINE TO QUIT} \\  
\hline
\end{tabular}
\end{center}                    
   The first line shows the descriptive 80 character string associated
   with the moment file.  In the above example moment data is available
   for three spectral lines.  The number in the rightmost column shows
   what moment data is available for each line.  A `1' means just
   intensity data, a `2' means intensity and velocity data, and a `3'
   means intensity, velocity, and line width data.  The number of scans is
   shown and a request is made for the user to type in four numbers which
   will identify:  which spectral line is of interest (ILINE may be
   1,2,3,or 4 in the above example), what moment data will be shown (MMMT,
   if given a value of 3 for example, will result in plotting all three
   moments, if all three are available), and finally what should be the
   starting and ending scan numbers. When the user enters all four numbers
  (all on one line, separated by commas), the program will then plot the
   data.  When the user is finished examining the plot he may hit 
   $<$return$>$,
   after which point the screen will be cleared and he will be given
   another opportunity to plot more data or else to exit the program.

\subsubsection{MMTTYP}     

{\em To run this program type:}
\begin{verbatim}
     HRTS$> MMTTYP                       
\end{verbatim}
  {\em MMTTYP} will display the numerical data stored in a .MMT file (i.e.
   values of intensity, velocity and line width).  When run, the program
   will first ask for the name of the moment file.  It will next print out
   a list of emission lines for which moment data is available, and the
   number of scan lines.  The program will then print `TYPE
   ILINE,ISCAN1,ISCAN2' and then wait for the user to type in three
   integer values.  The first value designates which emission line is of
   interest and the second and third values designate the beginning and
   ending scan lines.  The program will then print out columns of
   intensity, velocity and line width data (if all three are available)
   within the specified range of scan lines.  After this the user may type
   $<$return$>$ to get the `TYPE ILINE,ISCAN1,ISCAN2' message again and then
   either exit the program (by typing a negative number) or else print out
   additional data.

\subsubsection{PLIST}           

{\em To run this program type:}
\begin{verbatim}                
     HRTS$> PLIST                        
\end{verbatim}
{\em PLIST} will generate a list of processed HRTS data and store the list
   in a file named {\em PLIST.DAT.}  The user will probably like to print out
   that file when the program is finished.  A section of a generated list
   is shown below:
\begin{center}
\begin{tabular}{||l l l l l l||}
\hline
\multicolumn{5}{||l}{\bf Summary of HRTS data processing as of 26-AUG-87}  &
\multicolumn{1}{l||}{\bf Page 2 of 8} \\ 
\hline                                                                    
{\em File no} & {\em .RDS} & {\em .CIS(g)} & {\em .CIS(s)} & {\em .FSS} & {\em 
.MMT} \\  
      E18418548    &     *    &     *   &        &           &              \\
      E18419548    &     *    &     *   &        &           &              \\
      E18420548    &     *    &     *   &        &           &              \\
      E18503548    &     *    &     *   &      * &         * &         *    \\
      E18604548    &     *    &     *   &      * &         * &         *    \\
      E18605548    &     *    &     *   &      * &         * &         *    \\
      E18606548    &     *    &     *   &      * &         * &         *    \\
      E18607548    &     *    &     *   &      * &         * &         *    \\
      E18643548    &     *    &     *   &      * &         * &         *    \\
\hline                
\end{tabular}
\end{center}
  An `*' indicates that the specific file has been generated.
 (Note:{\em `.CIS(g)'} refers to a .CIS file generated by GETCIS and
{\em `.CIS(s)'} refers to one generated by Fortran STNDRD).

\subsubsection{PROFILE}             

{\em To run this program type:}
\begin{verbatim}                
     HRTS$> PROFILE               
\end{verbatim}
  This program will make intensity vs.wavelength plots for the
  spectral lines specified in the  PROFILE.DAT file; (an example of 
  this file is contained in the DEMO directory-- to look at the file type 
  HRTS\_DEMO followed by TYPE profile.dat)
  using data from one row of a spectral data file (e.g. .CIS).  When
  run, the program will ask the user for the name of the input spectral
  data file.  It will next ask for a scan line to use.  If the user
  inputs a valid scan line number, then the appropriate plot will be
  generated.  When the user hits any key after this, the program will
  then ask for another scan line value.  The user may enter another
  another valid value to get another plot, or he may enter a negative
  number to exit.

\subsubsection{PRTHDR}

{\em To run this program type:}
\begin{verbatim}                
     HRTS$> PRTHEADR                       
\end{verbatim}
  The user is asked for the name of the file whose header is to be
  displayed.  A file named {\em PRTHEADR.PRT} will be generated in the user's
  current directory which may be edited or typed out to reveal a listing
  of variable names and their respective values.

\subsubsection{WVLPLT}            

{\em To run this program type:}
\begin{verbatim}                
     HRTS$> WVLPLT                       
\end{verbatim}
{\em WVLPLT} produces a graph of intensity vs.wavelength.  This program
  is very similar to the {\em INTPLT} program, except that the graphs which are
  produced by {\em INTPLT} have the abscissa expressed in terms of position
  instead of wavelength.  When {\em WVLPLT} is run, the program first asks the
  user for the name of the input file (which will typically be a .CIS
  type file).   The program will then ask for a scan number after
  printing the maximum allowable scan number.  If the user types in a
  negative number here instead of a valid scan number, then the program
  will exit.  The program will next ask the user for beginning and ending
  wavelength values, after printing the largest range of wavelength
  values available from the data.  If the user just hits return at this
  point, then the program will use the default wavelength range (which is
  the largest possible range).  The intensity vs. wavelength graph will
  then be plotted on the graphics terminal.  The user may now type 
  $<$return$>$ to get the prompt that asks for another
  scan number, at which point he may either exit the program (by typing a
  negative number) or else plot other scan lines.

\subsection{IDL level}    

\subsubsection{GETDATA}

{\em To run this program type:}
\begin{quote}
{\tt HRTS\_IDL> }{\bf GETDATA,ARR,}{\it FILENAME,ROW1,ROW2,COL1,COL2,NSUM}  
\end{quote}
{\em GETDATA} reads data from a file named{\em FILENAME} into an array named       
{\em ARR.}  What the user should give for the first parameter, {\em ARR,} is any
 arbitrary variable name by which he wishes to refer to the data.  After
 successful completion of this procedure, {\em ARR} will have either one, two,
 or three dimensions and a type of either byte, integer, long integer or
 floating point depending on the file from which the data was read in.
{\em FILENAME} must be either a literal file name enclosed in quotes (such as
 the following: 'e02801548.rds'), or a variable
 whose value is a literal file name. {\em ROW1} and {\em ROW2} are optional, and 
if
 given, they should both have integer values which will be
 interpreted as the starting and ending rows of the stored image from
 which data will be read. {\em COL1} and{\em COL2} are also optional, and if 
given
 as integer values, they will define the range of columns from
 which data will be read. {\em NSUM} may be assigned a value in order to sum
 a number of rows of the data together.  If the number of parameters
 given is either 3 or 5 (i.e. a value for {\em ROW1} is given without a value
 for {\em ROW2,} or a value for {\em COL1} is given without a value of {\em
 COL2),} then
 the value of last parameter will be ignored, but the program will
 operate in a `silent' mode (i.e. there will be no message printed out
 at the end of the program that tells the size of the generated array).
 This feature allows {\em GETDATA} to be called by other programs without
 generating unwanted screen output.
 This procedure reads spectral (.RDS, .CIS, .FSS), moment (.MMT) and
 image (.IMG) files.  There is an automatic linear scaling of the data
 from .CIS and .FSS files based on scaling parameters stored in the
 file's header.  If scaling is performed then the procedure will
 explicitly say so, else the data is always unscaled.
 This procedure may take a while to run for large data files.  As the
 procedure ends, it will inform the user what is the size of the
 generated array.  When the procedure has ended, the user may type in
`COMMON WVL,W', if so desired, by which he can access an array named
`W' which contains wavelength position data associated with the data
 stored in the array{\em ARR.} (Note: If the array size is particularly large 
 IDL may be unable to allocate enough core memory to set up such arrays, in
 order to reduce array size sections of the data would have to be selected, by
 specifying a range using the row and column arguments.)

\subsubsection{GETMMT}               

{\em To run this program type:}
\begin{quote}
{\tt HRTS\_IDL> }{\bf GETMMT,ARR,}{\it FILENAME}          
\end{quote}
{\em GETMMT.PRO} is used to read in an array of data from a moment file.
  The parameters{\em ARR} and{\em FILENAME} are the same as those described in 
 {\em GETDATA.PRO,} only here{\em FILENAME} must specify a file which is of
  image type \#7 (i.e. .MMT, .MOD, .GMT, etc.)  As the program runs, it
  shows the user what moment data is available in the file and asks him
  to use the arrow and carriage return keys to select a moment.
  For most moment data files (where the maximum value of NMMT is less
  than or equal to 3), the following illustration will appear after this
  program is invoked:
\begin{verbatim}

                
        Use arrow keys + <cr> to select a moment 
\end{verbatim}
\begin{center}
\begin{tabular}{|l l l l l|}
\hline
                &        & {\em Int.} & {\em Vel.} & {\em L.W} \\    
\hline
\multicolumn{2}{|l|}{\em C IV  1548.202} & \multicolumn{3}{l|}{\em CURS} \\
\cline{3-5}   
\multicolumn{2}{|l|}{\em C IV  1550.774} & \multicolumn{3}{l|}{} \\                       
\multicolumn{2}{|l|}{\em Si I  1552.209} & \multicolumn{2}{l}{} &
\multicolumn{1}{l|}{n.a.} \\
\multicolumn{2}{|l|}{\em Continuum --} & \multicolumn{1}{l}{} &
\multicolumn{1}{l}{n.a.} & \multicolumn{1}{l|}{n.a.} \\ \cline{3-5}
     &  &  &  &  \\ \hline    
\end{tabular}           
\end{center}    
  In the above diagram the symbol `CURS' represents an extended cursor
  which the user can move (by using the four arrow keys) to various
  places within the smaller box to select a moment.  The symbol `n.a.'
  designates moments which are not available.  After the user hits a
  carriage return, the program will terminate, returning a 
  dimensional array through the ARR parameter.
  In the case in which the maximum value of NMMT is equal to 6 (which
  is the case for .GMT-type files), the program will instead display the
  following type of figure:
\begin{verbatim}                

          Use arrow keys + <cr> to make a selection: 
          (hit A to select all data for a given ion)
\end{verbatim}                
\begin{center}          
\begin{tabular}{|l l l l l l l l|}
\hline
   &       & {\em Amp.} & {\em D.L.} & {\em FWHM} & {\em Amp.} & {\em D.L.} &
{\em FWHM} \\
\hline
\multicolumn{2}{|l|}{\em C IV  1548.209} & \multicolumn{6}{l|}{\em CURS} \\
\cline{3-8}                                      
\multicolumn{2}{|l|}{\em C IV  1548.774} &  &  &  &  &  
\multicolumn{2}{l|}{} \\                                        
\multicolumn{2}{|l|}{\em CONTINUUM --} & &  n.a. &  n.a. &  n.a.&
\multicolumn{2}{l|}{n.a.} \\ \cline{3-8}
  & & & & & & &   \\ \hline
\end{tabular}                
\end{center}
   If one hits return, then a 1-D array is returned.  If instead the
  `A' key is pressed, then a 2-D array is returned which contains all the
   data available for the selected ion.

\subsubsection{GETADATAWVL}

{\em To run this program type:}
\begin{quote}
{\tt HRTS\_IDL> }{\bf GETDATAWVL,ARR,}{\em FILENAME,ROW1,ROW2,WVL1,WVL2,NSUM}  
\end{quote}
 This routine is equivalent to {\em GETDATA,} except that wavelength values
 in Angstroms are to be supplied as the 5th and 6th parameters instead
 of column numbers.
  
\subsubsection{HDR}

{\em To run this program type:}
\begin{quote}
{\tt HRTS\_IDL> }{\bf HDR,}{\em FILENAME}                   
\end{quote}
{\em HDR.PRO} allows the user to examine file header information in IDL.
   FILENAME may be the name of any spectral data file. {\em HDR.PRO} operates
   in a menu-driven environment in which the user can perform the
   following operations:
\begin{itemize}                
\item   obtain a listing of the most important header
       information all at once
\item   list the names of all the header variables
\item   get a description of what each variable represents
\item   examine the values of individual variables by typing
       in variable names
\item   change the stored header information within the file

\end{itemize}                
   Additional documentation regarding this program is available on-line
   as the program is running.

\subsubsection{MMTCALC}                

{\em To run this program type:}                                
\begin{quote}
{\tt HRTS\_IDL>}{\bf MMTCALC,WV,IN,WVLREF,}{\it CONTFLAG,WVLWIDTH}                
\end{quote}
{\em MMTCALC} will calculate the first three moments
for a single spectral profile.  The parameter WV must be assigned a one
dimensional array of wavelengths (acquired via {\em WAVEL}), IN must be assigned
a one dimensional array of profile data (acquired via {\em GETDATA}), WVLREF
must be assigned a real number which denotes the rest wavelength in Angstroms
for the spectral line to be processed (e.g. 1552.209).  CONTFLAG is optional
and is to be assigned a value of 0 if the user wants to use a constant value
for the continuum, or a value of 1 if he wants the continuum level to be
defined by a straight line connecting two specified points. The default value
is 0.  WVLWIDTH is also optional, and may be used to specify the desired width
of the wavelength scale (x-axis) to be plotted.  The supplied value will be
interpreted as being in Angstroms. The default value is .6. Regardless of the
value of CONTFLAG, two points must be selected to determine the continuum and
the maximum limits of integration.  A plot will be made on the graphics screen
that shows the spectral line, and the user will then manipulate the graphics
cursor to mark where he thinks the spectral line begins and ends.  The program
will then perform various calculations and print out the first three moments as
well as the full width at half maximum.                                        

\subsubsection{MMTPLT}                                                            

{\em To run this program type:}                                               
\begin{quote}
{\tt HRTS\_IDL>}{\bf MMTPLT,}{\it FILENAME}                                        
\end{quote}
{\em MMTPLT} is an
IDL version of the DCL level{\em  MMTPLT3} program.  It plots out intensity,
velocity and line width data from .MMT files.  The name of a .MMT file must be
given as an argument.  After being called, the procedure will ask the user if
he wants to enable the graphics cursor in order to read off numerical data from
the moment plots at user- defined points.  The user must respond to this
question by typing either `Y' or `N'.  The procedure will then display a list
of available ions, the wavelength appropriate for each ion, and the number of
moments available for each ion.  The user will then be asked to input four
integer numbers which will identify what plots should be made. The first
quantity, ILINE, selects the ion, the second, MMMT, selects the number of
moments to be plotted, and the last two, ISCAN1 and ISCAN2, set the scan number
range. ILINE is the only required quantity. If the user just enters a value for
ILINE, then the total number of moments available for that ion will be plotted,
and the complete range of scan numbers will be used.  The procedure will then
ask the user to input a value for SMIN (minimum value of intensity).  The user
may just hit the $<$return$>$ key here to use the default value.  The requested 
plots
will then be made on the graphics screen.  If the user had requested to use the
graphics cursor, then he may use the arrow keys to position the cursor at a
given place and then hit return in order to have the numerical values of the
moments for the selected scan number be plotted out on the text screen.  He may
then hit `C' to re-use the cursor to obtain other numerical values, or else he
can hit any other key at this point to return to the state in which the
procedure asks for another value of ILINE.  If the user did not request to use
the graphics cursor, then when he is through looking at the plots he should hit
any key in order to return to the text screen.  At this point the user may
enter in one to four numbers to generate another plot, or may exit by entering
`E'.                                                                           

\subsubsection{PLOTPROFILE}

{\em To run this program type:}
\begin{quote}
{\tt HRTS\_IDL>}{\bf PLOTPROFILE,}{\it FILENAME}             
\end{quote}
   This procedure is analogous to the DCL level {\em PROFILE} program.  The
   information about what profiles are to be plotted is contained within
   the file PROFILE.LST. (an example of this file is contained in 
   the DEMO directory)
   The one required argument, FILENAME, must be the complete name of a
   spectral data file (usually of type .CIS).  When run, the program will
   ask the user to enter a scan line number.  A profile plot for that scan
   line is then made on the graphics screen.  When the user hits any key,
   the program will revert to the text window and ask for another scan
   line.  The user may exit the program at this point by entering `-1'.

\subsubsection{PUTDATA}             

{\em To run this program type:}
\begin{quote}
{\tt HRTS\_IDL>}{\bf PUTDATA,}{\it ARR,FILENAME,TEXT}
\end{quote}
{\em PUTDATA} performs the inverse function of {\em GETDATA} in that it takes
   data from an IDL array and writes it into a file. The output file will 
   contain a one-line image-type-1 header, followed by unformatted data on
   subsequent lines.  Two parameters for this procedure are required.  ARR
   must reference a one, two, or three dimensional array with a type of
   either byte, integer, long integer, or floating point.  FILENAME must
   denote a character string denoting the name of the file to which the
   data is to be written.  The program will not operate if the output file
   is of type .RDS, .CIS, or any of the other special file types, since
   those types of files require a specific kind of header.  If a character
   string enclosed by quotes is given for the variable TEXT, then that
   string will be placed in the file's header.  If no value was given for
   TEXT, then the program will ask the user to type in a one line textual
   description of the data.  This string, whether given via the variable
   TEXT or typed in as the program runs, must have a length that is less
   than or equal to 80 characters.

\subsubsection{READCAL}

{\em To run this program type:}
\begin{quote}
{\tt HRTS\_IDL>}{\bf READCAL,LGI,DEN,}{\it FILENAME}          
\end{quote}
{\em READCAL} will read log intensity and density values from a .CAL file
   into two one-dimensional arrays.  LGI and DEN represent the output
   arrays, and FILENAME must be assigned the full name of the desired .CAL
   file.

\subsubsection{TVEXTRACT}                

{\em To run this program type:} (NOTE: This routine must be run in WIDL.)
\begin{quote}
{\tt HRTS\_IDL>}{\bf TVEXTRACT,}{\it FILENAME,}WAVELN,INTEN       
\end{quote}
{\em TVEXTRACT} may be used to make cross-sectional plots of spectral data
   contained in the file named FILENAME (usually of type .CIS), and/or to
   extract a section of that data into two output arrays, in which one
   (WAVELN) contains wavelength data and the other (INTEN) contains
   intensity data.

      This program will display a reduced resolution image of the input
   data on the image display. The user will be asked first to specify 
   the  type of image display and type 0 for the IKON or 1 for a device 
   operating X-windows. (e.g. Workstations) followed by RETURN. 
   A menu of options will then be displayed; if the image being 
   displayed uses X-windows, then the commands are made using the 
   mouse by pressing one of the mouse buttons over the required 
   command on the screen. If the display is an IKON, the commands are 
   entered via the keyboard.
   The user will then be asked to use the mouse to select
   an area of interest for which full resolution data will be read in and
   displayed.  Hitting the first mouse button causes the program to enter a
   mode in which mouse movements result in moving a box on the graphics
   screen.  Hitting the second mouse button causes a mode to be entered in
   which mouse movements change the shape of the box.  When manipulations
   have been performed so that the box encloses the desired area, then the
   user should hit the third mouse button.

      After the selected area has been displayed , the program
   will ask the user to enter a  command.  The program will have
   previously displayed the following list to define the 
   commands:
\begin{verbatim}   

                           Keyboard Commands 

              F : display full image again                       
              S : select an area for the output arrays           
              X : make a cross-sectional plot in the X-direction 
              E : exit                                           
\end{verbatim}          
   
      The S and X commands require some explanation.  If the S command is
   entered, then a tvroam state will be entered.  The user must hit the
   first mouse button to zoom in and the third mouse button to exit.  The
   user is then given an opportunity to use a box on the screen
   to select an area.  The selection method is equivalent to that
   described above.  Upon exiting the program, the WAVELN and INTEN arrays
   will contain data for the last area that was selected in this fashion.
   If the X command is entered, then a TVROAM state will be entered once
   again.  When this state is exited via the pressing of the third mouse
   button, the program will ask the user to press the first mouse button
   to mark the left-most point of the cross-sectional plot.  Then the
   program will ask that the right-most point be selected in the same
   manner.  The cross-sectional plot will then be made on the 
   screen.  If the user is operating with the IKON display they may then hit 
   any key to go back into the keyboard command mode. 

\subsubsection{TVIMG}

{\em To run this program type:} (NOTE: This routine must be run in WIDL.)
\begin{quote}
{\tt HRTS\_IDL>}{\bf TVIMG,}{\it FILENAME,VMAX}                 
\end{quote}
{\em TVIMG} displays intensity, velocity, and line width images (the
   contents of a .IMG file) on the image display and allows the user to also 
look at
   the numerical data.  FILENAME should be a .IMG file. VMAX is an optional 
parameter
   and, if assigned a value, then that value will be interpreted as the
   largest acceptable wavelength shift (in Angstroms).  If no value is
   given for VMAX, then a default value of 0.13 will be used.  When the
   program runs, it will first ask for the type of image display  
   were upon the user should type 0 for IKON or 1 for X-WINDOWS. 
   A menu will then be displayed on the screen and 
   the user may select whether he wishes to see intensity, velocity, or
   line width images.  The user should use the arrow keys to move the
   cursor to one of the options and then press return in
   order to select that option if they are using the IKON image 
   display, if they are using X-windows, selections are made using the mouse by 
   pressing any mouse button when the cursor is over the required 
   option on the command board. 
   When the user selects one of the options, then the
   current screen is cleared and a second screen is shown which displays
   sections of what would otherwise be a very long, thin image which
   represents values of the selected attribute over a two-dimensional area
   of the sun. The user may then Zoom\_In on a selected part of the 
   image by using the command `Zoom\_in' and selecting a location from a section 
of the image
   (using the mouse), then a third screen will be displayed which shows all 
three attributes
   (intensity, velocity, and line width, if all three are available) for
   an area centered around the selected point.  The user may now select
   different points of any of the displayed images, and the numerical
   values of the available attributes of that point will be displayed on
   the screen.  In order to examine additional data, the program then returns
   to the initial menu enabling further selections to be made.

\subsubsection{TVIMGPROFILE}

{\em To run this program type:} (NOTE: This routine must be run in WIDL.)
\begin{quote}
{\tt HRTS\_IDL>}{\bf TVIMGPROFILE,}{\it IMG\_FILE, FILENAME}              
\end{quote}
      This program allows one to plot spectral line profiles by selecting
   points within .IMG images.  It is assumed that the  associated .IMG\_ALN
   files, are all stored in the user's current directory.
   As for {\em PLOTPROFILE.PRO}, there exists a PROFILE.LST file which specifies
   the spectral lines to be used for profile plots.  The  required
   arguments to {\em TVIMGPROFILE}, are IMG\_file,which must be the generic name 
of the .IMG
   file, for example `REV82CIV' and FILENAME which is the literal name of a
   .CIS file.  When run, the program behaves exactly
   like TVIMG, only when points are selected on the zoomed-in data, a
   profile plot will be generated.

\subsubsection{VARVELCT}

{\em To run this program type:}
\begin{quote}
{\tt HRTS\_IDL>}{\bf VARVELCT,}{\it MAX,} R,G,B                
\end{quote}   
      This routine will return colour table arrays to be used to display
   velocity images, as does VELCT, only a parameter MAX is here input
   which sets a peak intensity for the colour table.  This method is used
   to produce velocity images which will be slightly dim so as not to
   dominate an output image.  MAX must be an integer value between 0 and
   255.

\subsubsection{VELCT}   

{\em To run this program type:}
\begin{quote}
{\tt HRTS\_IDL>}{\bf VELCT,R,G,B}                   
\end{quote}   
      This procedure will return three one-dimensional arrays of size 256
   which constitute the colour table generally used to display velocity
   images.

\subsubsection{WAVEL}

{\em To run this program type:}
\begin{quote}
{\tt HRTS\_IDL>}{\bf WAVEL,ARR,}{\it ROW}               
\end{quote}
{\em WAVEL} will return a one dimensional array which contains a
   wavelength scale that is valid for a specified row of the spectral
   image which has been most recently read in by GETDATA within the
   current IDL session.  This procedure has two required parameters.  ARR
   must be an arbitrary variable name which will, upon completion of the
   procedure, reference the wavelength scale data.  ROW must be a integer
   number which denotes that row of the data read in by GETDATA at which
   the wavelength scale will be valid.  A value of ROW equal to 0 refers
   to the first row of data as stored in the data file.
\newpage

\section{Data Reduction Programs}

\subsection{DCL level}

\subsubsection{C4DFAC}

{\em To run this program type:}
\begin{verbatim}
     HRTS$> C4DFAC             
\end{verbatim}   
{\em C4DFAC} performs dewarping of a density image, smooths the image by
   removing noise, calculates the autocorrelation with a synthetic
   spectrum, converts density to intensity, and extends the area for C IV
   (if it is appropriate to do so).  The input for this program is a .RDS
   file, and the output consists of either a .CDF or .CIS file, plus a
   .ATO file.  Since this program takes a long time to run, it is best to
   submit it as a batch job.  The program {\em GETCIS.EXE} has been designed to
   create and submit .RUN files that execute {\em C4DFAC.}  Because of the
{\em GETCIS} program, most users will not have to deal with the .RUN files
   directly if they wish to run this program, nevertheless the structure
   of a typical .RUN file for {\em C4DFAC} is presented below for the sake of
   completeness:
\begin{center}   
\begin{tabular}{||l l l l l||}
\hline
\multicolumn{5}{||l||}{\tt assign userdisk1:[dir]imsl.err for002} \\
{\tt \$ c4dfac} & & & & \\                                                      
{\tt Scratchdisk1:[filename]} &  &  &  \multicolumn{2}{l||}{\em Input Directory} 
 \\
{\tt Scratchdisk1:[filename]} &  &  &  \multicolumn{2}{l||}{\em Output 
Directory} \\
\hline
{\tt E42101548.RDS} &  &  &  &  \\                                         
{\tt R082L550.CAL;1} &  &  &  \multicolumn{2}{l||}{\em Calibration Data} \\ 
{\tt CURV.DAT;1} &  &  &  \multicolumn{2}{l||}{\em Curvature Data} \\ 
{\tt MAGN.DAT;1} &  &  &  \multicolumn{2}{l||}{\em Magnification Data} \\ 
{\tt DISP.DAT;1} &  &  &  \multicolumn{2}{l||}{\em Dispersion Data} \\ 
{\tt ATO.548;1}  &  &  &  \multicolumn{2}{l||}{\em Synthetic Spectra Data} \\  
{\tt .CIS} &  &  &  \multicolumn{2}{l||}{\em Output File Type} \\ 
{\tt YES}  &  &  &  \multicolumn{2}{l||}{\em Carbon IV Treatment} \\ 
{\tt YES}  &  &  &  \multicolumn{2}{l||}{\em Fourier Filtering} \\ 
\hline
{\tt E} &  &  &  &  \\
\hline                                                                      
\end{tabular}
\end{center}
   
      The first line directs any IMSL error or warning messages (which are
   sometimes generated as {\em C4DFAC} runs) into a directory where there should
   be enough space to store them.  The second line starts executing {\em 
C4DFAC.}
   The third line specifies the input directory which contains the .RDS
   file(s) that will be used as input, and the fourth line specifies the
   directory where the newly-created .CIS files will be stored.

      Now for each .RDS file to be processed (one may process several at a
   time), there is a group of 9 consecutive lines of information bounded
   above and below by one line of hyphens (for the sake of readability).
   The first line contains the name of a .RDS file (as shown above), the second
   line contains the complete name of the appropriate density to intensity
   calibration file, the third line the curvature data file name, the fourth
   line has the magnification data file name, the fifth line has the
   dispersion data file name, the sixth line has the standard lines data
   file name, the seventh line shows the file type of the output files (either
  `.CIS' or `.CDF'), the eighth line contains either `YES' or `NO', which
   tells if the Carbon IV treatment should be employed for this data, and
   finally the ninth line contains either `YES' or `NO' to indicate whether
   or not Fourier filtering should be performed.  A line that consists
   simply of an `E' designates that there are no more .RDS files to be
   processed. (examples of the .CAL and .DAT files can be found in the 
   directory HRTS\_DISK:[HRTS.DATA] )

\subsubsection{DEN\_INT}

{\em To run this program type:}
\begin{verbatim}
     HRTS$> DEN_INT                        
\end{verbatim}   
{\em DEN\_INT} will create a .CIS file from a .CDF file, or a .FSS file
   from a .FSD file.  When run, the program will ask for the full name of
   the input density file (including the `.CDF' or `.FSD').  After the
   user types that in, then the program will ask for the full name of the
   output intensity file.  The program will then generate the output file,
   and next ask the user for the name of another input file to be
   processed.  The program may be exited if one enters `E' at this point.

\subsubsection{FLIPX}

{\em To run this program type:}
\begin{verbatim}
     HRTS$> FLIPX
\end{verbatim}
This program will flip two-dimensional HRTS data in the x-direction.
It is occasionally run to correct a .RDS file which was created from a plate
scanned in the wrong direction. When run, it will ask the user for the input
and output file names.

\subsubsection{FLIPY}

{\em To run this program type:}
\begin{verbatim}
     HRTS$> FLIPY
\end{verbatim}
{\em FLIPY} is similar to {\em FLIPX} except that the data is flipped in the 
y-direction.

\subsubsection{FSS}

{\em To run this program type:}
\begin{verbatim}   
     HRTS$> FSS             
\end{verbatim}
{\em FSS} will produce a .FSS file, using a .CIS file as input.  The .FSS
   file will still contain a spectral image, but now the spatial and
   spectral coordinates will be rectangular and the dispersion will be
   constant.  This program is generally run by submitting the following
   type of file to a batch queue:
\begin{verbatim}   
                    
                      HRTS$> FSS                    
                      DIRECTORY:[FILENAME] 
                      DIRECTORY:[FILENAME]  
                      E02801400.CIS            
                      E                        
                    
\end{verbatim}   

      The first line of the file calls the FSS program.  The second
   line specifies the input directory and the third line specifies the
   output directory.  Beginning on the fourth line is a list of file names
   to be processed, with one file name per line.  An `E' denotes the end
   of the list.

\subsubsection{GETCIS}

{\em To run this program type:}
\begin{verbatim}
     HRTS$> GETCIS                  
\end{verbatim}
{\em GETCIS} is used to create .CIS or .CDF files from .RDS files by
   submitting a batch job which invokes C4DFAC.  The GETCIS program serves
   as a user interface to the C4DFAC program, so the user need not deal
   with the complicated .RUN files used to submit C4DFAC.  When a user
   starts to run GETCIS, he will see a screen that shows default
   attributes concerning the processing to be performed.
   The user can use the arrow keys to move the cursor to various parts
   of the screen, and then press the $<$return$>$ key to change attributes.
   For instance one can change the default input directory, and then
   select/highlight the names of various .RDS files from that directory to
   be used as input files.  The user should select the Submit option when
   he is ready to submit the job.
      The user then sees another screen on which he may examine low-level
   details about the processing.
   At this point the user may examine and/or change the names of files
   that will be used to supply accessory information as each input file is
   processed.  In most instances, the user will simply select Submit as
   soon as this screen appears.
   Unless the user selected Abort, the GETCIS program will generate and
   submit a file names GETCIS.RUN which will invoke the C4DFAC program.

\subsubsection{MMTIMG}

{\em To run this program type:}
\begin{verbatim}
     HRTS$> MMTIMG               
\end{verbatim}
      This program will create a .IMG file from a series of .MMT or .MOD
   files.  In order to run this program, the user must first prepare (via
   the editor) a .IMG\_ALN file which contains the names of the input files
   and their offsets.  An example of a .IMG\_ALN file is shown below:
\begin{verbatim}      
   

            'hrts sl 2   Rev 43   Seq 54   2.0 sec expos.  1555' 
            1    SPATIAL SERIES                                  
            0   1  'e00312555.mod                                
            6   1  'e00401555.mod                                
            4   1  'e00402555.mod                                
            -15 1  'e00501555.mod                                
            -14 1  'e00502555.mod                                
            -32 1  'e00601555.mod                                
             3  1  'e00602555.mod                                
            -32 1  'e00701555.mod                                
            -29 1  'e00702555.mod                                
            -43 1  'e00801555.mod                                
            -36 1  'e00802555.mod                                

\end{verbatim}   
      The first line of the file contains character data that will be
   stored in the output file's header.  The second line should contain
   either the number 1 or 2.  A `1' results in the output file being of
   type .IMG, and a `2' results in a .TSI type output file.  The remainder
   of MMTIMG.DAT should consist of one line of data for each input file.
   There must be three items per line.  The first item is an offset
   number.  
   The second item is either the number 0 or 1.  This number is
   referenced when performing histogram equalizations or velocity
   corrections to see whether or not the data from a particular file
   should be included in some aspect of the calculations.  A `0' should be
   assigned to data that has some problem associated with it, or a `1'
   otherwise.  The third item on each line should be the name of the input
   moment data file.

      When the user executes the {\em MMTIMG} program (by tying in `MMTIMG' at
   the `HRTS\$' prompt), he will first be asked to enter the generic name of
   the .IMG\_ALN file.  He will then be asked for the number of moments
   that he wishes the output file to contain.  This number should be
   either 1, 2, or 3.  The intensity data is always output.  The
   velocity data will be output if the user enters 2 or 3.  The line-
   width data will be output if the user enters 3.
      The program then asks the user to enter a value of GMIN for each
   moment.  This is interpreted by the program as being the minimum valid
   input value for that moment.
      The program then displays a list of available ions on the screen.
   The ions in the list are contained within each of the .MMT/.MOD files
   listed in the .IMG\_ALN file.  The user should use the arrow keys to
   select the ion of interest.
      The program will then ask the user various questions concerning
   whether or not he wishes to perform a histogram equalization of the
   intensity image or to calculate and/or apply velocity correction
   factors.

      The program will then generate the output file (which will have the
   same generic name as the .IMG\_ALN file) and then exit.

\subsubsection{MMTMOD}

{\em To run this program type:}
\begin{verbatim}   
     HRTS$> MMTMOD          
\end{verbatim}
{\em MMTMOD} is used to create .MOD files from .MMT files.  This program
   is usually run by submitting the following type of file to a batch queue:
\begin{center}   
\begin{tabular}{||l||}
\hline
                 
                      {\tt HRTS\$ mmtmod}  \\                  
                      {\tt DIRECTORY:[FILENAME]} \\    
                      {\tt DIRECTORY:[FILENAME]} \\    
                      {\tt E03013548}  \\                 
                      {\tt 100.}    \\                    
                      {\tt E03014548} \\                   
                      {\tt 100.} \\                       
                      {\tt E}  \\                         
\hline
\end{tabular}
\end{center}
   
      The first line invokes the {\em MMTMOD} program.  The second and third
   lines define the input and output directories.  Then there are two
   lines of information for each .MMT file to be processed.  The first
   line contains the nine-character generic name of the .MMT file and the
   second contains a value for CIVMIN (`100.' is often used).
   An `E' marks the end of the list of .MMT files to be processed.
      There is a file named MMTMOD.LST  that lists
   the ions that should be combined together into a composite form.  A
   example of the contents of this file is shown below:
\begin{center}   
\begin{verbatim}


                   20.                                
                   'C IV OPT THIN'  3 2 1548.202  1.  
                                        1550.774  0.5 

\end{verbatim}   
\end{center}
   The number on the first line represents an SMIN value. For this
   example there will be one modified ion entry inserted into the output
   .MOD file.  The `C IV OPT THIN' (IDMOD) is a descriptive character
   string, the `3' (NMOD) species that intensity, Doppler shift, and line
   width values will be made available, the `2' (NELMOD) indicates that
   data from two wavelengths will be combined together, the `1548.202' and
   the `1550.774' (WVLMOD) are those wavelengths, and the `1' and `0.5'
   (RMOD) are weights.

\subsubsection{MOMENT} 

{\em To run this program type:}
\begin{verbatim}   
     HRTS$> MOMENT                  
\end{verbatim}
      {\em MOMENT} will create .MMT files by using spectral files as input.
   This program is generally run by submitting the following type of 
   file to a batch queue:
\begin{center}   
\begin{verbatim} 

                  HRTS$> Moment                              
                  E66901555.CIS                      
                  E                                   

\end{verbatim}  
\end{center}
      The first line runs the {\em MOMENT} program, and starting on the second
   line is a list of names of spectral files (with the file type
   specified) which will be used as input to create .MMT files to be
   stored in the same directory.  The list is terminated by an `E'.

\subsubsection{RECTORECIS}

{\em To run this program type:}
\begin{verbatim}
     HRTS$> RESTORECIS                      
\end{verbatim}
{\em RESTORECIS} restores the header of a .CDF or .CIS file to the state
   in which it was in prior to running IDL STNDRD and Fortran STNDRD on
   it.  Specifically, this program modifies the existing header
   of a .CIS file so that NCRV=0, NSCL(1)=4, and SCL has the same values
   it had when the .CIS file was originally produced via GETCIS.  This
   program is commonly run when the results of running the two STNDRD
   programs has proved unsatisfactory and the operator wishes to restart
   the reduction process back at the IDL STNDRD stage.  When this program
   is run, the user must answer three questions regarding the processing.
   Those questions, along with a sample series of responses, are shown
   below:
\begin{verbatim}
   
              Enter full name of .CIS file (or `E' to Exit)
              E18503548.CIS
              Enter Curvature Data FileName
              Curv.Dat
              Enter Dispersion Data FileName
              Disp.Dat
              Enter full name of .CIS file (or `E' to Exit)
              E
              End RESTORECIS.

\end{verbatim}   

\subsubsection{STNDRD}

{\em To run this program type:}
\begin{verbatim}   
     HRTS$> STNDRD      
\end{verbatim}
      The goal in running this program is to enable proper wavelength
   values to be assigned to each pixel of a .CDF or .CIS file.  This
   program modifies a file's header (specifically the values of the SCL,
   NCRV, and CRV variables) in such a way as to (hopefully) improve the
   assignment of wavelength value as a function of pixel coordinates.
   This program frequently has to be rerun with varying input conditions
   in order better to attain its desired goal.
      The input file for this program is assumed to have been previously
   operated upon by the IDL SETCRV program (after having run IDL ATODATA
   and IDL STNDRD), so that there is already some guess as to what is the
   curvature associated with that image (specifically, NCRV should have a
   value of 1, CRV(1) should be assigned a value, and SCL(1,1) should have
   a new value which was improved over the original value assigned when
   GETCIS was run).
      This program is generally run by submitting the following type of
    file to a batch queue:
\begin{center}      
\begin{tabular}{||l||}   
\hline   

                   {\tt HRTS\$ Stndrd} \\                            
                   {\tt E03013548.CIS}\\                        
                   {\tt E} \\
\hline
\end{tabular}
\end{center}   

      As the program begins to run, it reads values of various parameters
   from two small data files stored in the directory.
   The names of the files are STNDRD.DAT and STNDRD.LST ( examples of 
   these file may be found in the DEMO directory ) 
   The parameters
   stored in STNDRD.DAT frequently have to modified for each different
   .CDF/.CIS file.  An example of possible contents of STNDRD.DAT are
   shown below:
\begin{center}
\begin{tabular}{||l l l l l l ||}   
\hline
           ISCAN1  &   ISCAN2 &   NSUM  &     &        &  \\            
           1       &   2100   &   20    &     &        &  \\           
           FACT    &   NDTERMS &  NCTERMS &   &        &  \\           
           .1      &   4       &  2       &   &        &  \\           
           AVAR & (0.=NO VARIATION) &     &   &        &  \\           
           1.   &     0.  &      0. &       0.  &      1.   &  1 \\
           SMIN &      NIT &      IWRITE &  NSIG  &      &  \\         
           .5   &      4   &      1    &    2.0   &      &  \\         
           IPREL  &    IPLOTF &   IPOSUM  &       &      &  \\         
           0    &      20    &    20   &          &      &  \\         
\hline
\end{tabular}   
\end{center}
      ISCAN1 and ISCAN2 give the range of rows of the .CDF/.CIS data array
   which will be included in the calculations.  NSUM tells how many rows
   of data will be averaged together within the program to reduce noise.
   NDTERMS and NCTERMS give the number of dispersion terms and the number
   of curvature terms.  The values of these variables should probably not
   be changed.  AVAR is an array of six elements which specify which
   header variables will be subject to modification as the program runs.
   AVAR(1) corresponds to SCL(1,1), AVAR(2) to SCL(2,1), AVAR(3) to
   SCL(3,1), AVAR(4) to SCL(4,1), AVAR(5) to CRV(1), and AVAR(6) to
   CRV(2). The sequence {1,0,0,0,1,1} is most commonly used. SMIN gives a
   low end cutoff for calculated moments which will be used. SMIN
   frequently has to be set to different values for different .CDF/.CIS
   files. NIT sets the number of iterations to be used for polynomial
   fitting calculations.  IWRITE tells whether or not the header of the
   input data file will be modified (IWRITE=1 : modify file, 0 : don't
   modify file).  NSIG is such that any points that will be used for a
   polynomial fit which have a standard deviation associated with them
   that is greater than NSIG's value, will be assigned a weight of 0.
      The file STNDRD.LST must contain details regarding the wavelengths
   which will be used to determine the image curvature.  This file should
   look something like the one shown below:
\begin{center}
\begin{tabular}{||l l l l l l l||}      
\hline   
    {\em `Si I'}      &  2 1 & 1548.715 &  1548.62 &  1548.81 &   1.&   1. \\ 
    {\em `Fe II'}     &  2 1 & 1550.260 &  1550.16 &  1550.36 &   1.&   1. \\ 
    {\em `Si I'}      &  2 1 & 1551.240 &  1551.14 &  1551.34 &   1.&   1. \\ 
    {\em `Si I'}      &  2 1 & 1552.209 &  1552.08 &  1552.36 &   1.&   1. \\ 
    {\em `CONTINUUM'} &  1 1 &    0.    &  1548.9  &  1549.3  &   1.&   0. \\ 
    {\em `CONTINUUM'} &  1 1 &    0.    &  1549.6  &  1549.8  &   1.&   0. \\ 
    {\em `CONTINUUM'} &  1 1 &    0.    &  1552.4  &  1552.6  &   1.&   0. \\ 
    {\em `CONTINUUM'} &  1 1 &    0.    &  1553.6  &  1554.4  &   1.&   0. \\ 
\hline
\end{tabular}
\end{center}
      This program is usually run for each .CDF/.CIS file a minimum of two
   times.

\subsubsection{SUM}

{\em To run this program type:}
\begin{verbatim}
     HRTS$> SUM               
\end{verbatim}
      This program makes summed stigmatic spectra. If a .CIS file is used
   as input then the output will be a .SSS\_C4 file, else if the input is a
   .CDF file, then the output will be a .SSD\_C4 file.
      When the program is run it asks for the complete name of the input
   file. The output file will be in the same directory that the input file
   is in. The program will continue to ask for more input files until an
   `E' is encountered.
      Below is an example of a batch job that will run this program:   
\begin{center}
\begin{tabular}{||l||}
\hline

                    {\tt HRTS\$  sum} \\                         
                    {\tt e38501555.cdf} \\                  
                    {\tt e38502555.cdf} \\                  
                    {\tt e38503555.cdf} \\                  
                    {\tt E.} \\                              
\hline
\end{tabular}   
\end{center}   

\subsubsection{SUMCDF\_IG}

{\em To run this program type:}
\begin{verbatim}
     HRTS$> SUMCDF_IG              
\end{verbatim}
{\em SUMCDF\_IG.FOR} will output a .CIS\_SUM file from a group of up to nine
   input .CDF or .FSD files.  The input images are aligned together and
   intensities are generated such as to eliminate problems associated with
   overexposure.  A detailed description of the algorithm is given in
   comments at the end of the program's code.
      There is an IDL program, named {\it SUMCDF\_IG.PRO}, that is commonly used
   to set up a SUMCDF\_IG.RUN file which is to be used to submit this
   program as a batch job.  Please refer to the descriptive entry in this
   manual for that IDL program.
   {\em SUMCDF\_IG.FOR} will first ask for the number of input files, and then
   for a list of the complete names of all the input files, with one name
   listed per line, and next it will ask for the output generic file name.
   Following this information there should be a list of pairs of {x,y}
   offsets.  There should be one offset pair per input file, and each pair
   should be listed on a separate line.  These offsets give an initial
   guess as to how the files should be aligned relative to the first input
   file.  This program should be run in batch since it takes a while to
   run.  Shown below is a sample version of {\em SUMCDF\_IG.RUN:}
\begin{center}   
\begin{tabular}{||l l||}   
\hline   

                  {\tt HRTS\$ sumcdf\_ig} & \\                        
                  {\tt 4}   &  \\                               
                  {\tt a00101400.cdf} & \\                      
                  {\tt a00102400.cdf} & \\                      
                  {\tt a00201400.cdf} & \\                      
                  {\tt a00202400.cdf} & \\                      
                  {\tt a400sum}       & \\                            
                    {\tt 0}  &    {\tt 0}  \\                         
                  {\tt -42}  &   {\tt -6}  \\                         
                  {\tt -31}  &   {\tt -9}  \\                         
                   {\tt 18} &    {\tt -40} \\                        
\hline                 
\end{tabular}   
\end{center}
\newpage

\subsection{IDL level}

\subsubsection{ATODATA}

{\em To run this program type:}

\begin{quote}   
      {\tt HRTS\_IDL>} {\bf ATODATA,ARR,}{\it FILENAME,NAVE}        
\end{quote}   
{\em ATODATA.PRO}, {\em STNDRD.PRO}, and {\em SETCRV.PRO} are to be run one 
after the
   other in order to apply a curvature correction to .CIS or .CDF files.
   The purpose of {\em ATODATA.PRO} is to read data from a .ATO file into IDL
   memory so that that data may be used as the input to the {\em STNDRD.PRO}
   procedure.  (.ATO files are created whenever GETCIS is run).  The first
   argument for ATODATA must be an arbitrary name which will used to
   reference the two-dimensional output array that will be generated.  The
   second argument is the name of the input (.ATO) file, without the
   `.ATO'characters (e.g. one may start the program by typing:
   atodata,a,`e02901400'.) The third argument is
   optional and, if specified, should be an integer number (such as 2 or
   4) which will result in averaging the .ATO data over the specified
   number of lines.  (This averaging is performed in order to make the
   central spectral line stand out more when {\em STNDRD.PRO} is run).
   to hit `R' if he wants to remove the most-recently selected point, or
   else to hit `E' if he wants to exit.  When one exits the program, the
   collected jet information will be written out into a `.JET\_CI' file
   which has the same generic name as the input spectral file.

\subsubsection{SCALIM}

{\em To run this program type:}
\begin{quote}
     {\tt HRTS\_IDL>}{\bf SCALIM}, OUT           
\end{quote}   
{\em SCALIM} will produce a .SMG file from a .IMG file by translating the
   .IMG file's floating point data to byte-scaled data.  The optional
   output array, OUT, may be used to obtain within IDL the output .SMG
   data.

      When run, the program will first ask the user for the full name of
   the input (.IMG) file.  It will next ask the user what type of
   intensity scaling should be done (linear, logarithmic or square root).
   The user will then be asked to give upper and lower intensity limits
   for the intensity scaling.  The user will want to have looked at the
   intensity histogram prior to this in order to determine proper limits.
   The next question asks for an intensity minimum for the velocity image.
   (A value of 2 times the lower intensity scaling limit is occasionally
   input here.)  The program then asks for the maximum doppler shift
   for velocity images.  (A value of .13 may be appropriate).  If ISIZ(3)
   is greater that 2 for the input .IMG file, then two more questions are
   asked, namely, what is the intensity minimum for the line-width image
   and what is the maximum line-width for the line-width image.

\subsubsection{SETCRV}

{\em To run this program type:}

\begin{quote}   
     {\tt HRTS\_IDL>}{\bf SETCRV},{\it COEF,FILENAME,FILETYPE}         
\end{quote}   
{\em SETCRV} is used to alter the header of a .CIS or .CDF file in order
   to correct for curvature.  The coefficients stored in COEF, which were
   obtained from {\em STNDRD.PRO}, will be used to modify certain header
   variables in the .CIS/.CDF file (specifically NCRV, CRV(1) and
   SCL(1,1)).  COEF must be a one dimensional array with two elements.
   FILENAME must be a string of characters that denote the name of the
   .CIS/.CDF file, but without the `.CIS' or `.CDF' part.  The procedure
   always assumes that the file to be altered is a .CIS file, unless a
   specific file type is given for the optional FILETYPE parameter.  For
   instance, if one is working with a .CDF file then he will want to give
   `.CDF' as the third parameter.

      In addition to changing the header, this procedure also creates a
   small file which will contain the header information as it was before
   the curvature correction was performed.  As an example, if one should
   enter the following statement:
\begin{verbatim}   

          HRTS_IDL> setcrv,coef,`E2901400'
  
\end{verbatim}
   then the file named `{\tt e02901400.cis}' will have
   its header altered and the old header will be stored in a file named
   `{\tt e02901400h.cis}'.

\subsubsection{STNDRD}

{\em To run this program type:}

\begin{quote}   
     {\tt HRTS\_IDL>}{\bf STNDRD},{\it ARR},{\bf COEF}                
\end{quote}   
{\em STNDRD.PRO} displays the data stored in ARR (which was the output of
{\em ATODATA.PRO)} on the image display screen and puts up vertical lines which 
may be
   moved by using the mouse so as to make an alignment with the spectral
   line in the centre of the image.  When the user makes the alignment, he
   should press $<$return$>$, which will cause the current curvature
   correction coefficients to be stored in COEF, and the procedure will
   then terminate.  SETCRV may then be run to adjust the header of the
   appropriate .CIS file with those two numbers stored in COEF.
      Listed below is a complete sequence used to implement a curvature
   correction:
\begin{verbatim}
                  HRTS_IDL> name='E02901400'
                  HRTS_IDL> atodata,a,name
                  HRTS_IDL> stndrd,a,coef
                  HRTS_IDL> setcrv,coef,name
\end{verbatim}   

\subsubsection{SUMCDF\_IG}

\begin{quote}   
     {\tt HRTS\_IDL>}{\bf SUMCDF\_IG}            
\end{quote}   
{\em SUMCDF\_IG.PRO} is used to set up a file to run {\em SUMCDF\_IG.FOR.}
   Before running SUMCDF\_IG.PRO, the user must create a file named
   SUMCDF\_IG.DAT in his current directory.  This file must contain the
   number of input files listed on its first line.  On each line below
   that must be listed one input file name.  The last line of the file
   must specify the generic name of the output file.  An example is shown
   below:
\begin{center}
\begin{verbatim}   

                  4                                    
                  a00101400.cdf                        
                  a00102400.cdf                        
                  a00201400.cdf                        
                  a00202400.cdf                        
                  a400sum                              
               
\end{verbatim}   
\end{center}

      When this program is run it will first ask the user to enter
   starting and stopping rows.  These row numbers specify a region of
   interest within the data files.  The stopping row must be bigger than
   the starting row by no more than 1024.  The program will then generate
   a plot of the input data versus wavelength for a row midway between the
   starting and stopping rows for each input file.  The user should then
   move the graphics cursor to select a wavelength of interest.  The
   program will then place a thin strip of data onto the TV from each
   input file.  The Y range of the data will be defined by the specified
   starting and stopping rows, and the X range will have have the selected
   wavelength at its centre.  The user must then use the TV cursor to mark
   an equivalent point on each image, starting with the leftmost image.
   The program will then calculate the X and Y offsets, create the
   SUMCDF\_IG.RUN file, and then exit.
\newpage

\section{H\&D Curve Development Programs}

\subsection{DCL level}

\subsubsection{MAKEHIST}

{\em To run this program type:}
\begin{verbatim}   
     HRTS$> MAKEHIST                    
\end{verbatim}
{\em MAKEHIST} will generate a histogram of density values for a specified
   image and stores the result in a .HIS file.  After the program is
   invoked, it first asks for the name of the input density file.  The
   program will then display a screen which contains the following
   information:
\begin{center}
\begin{verbatim}   
            
               Range of input data:   [70 to 1022]           
                                                             
               Before making the histogram, the input data   
                 will be transformed via the equation:      
                                                            
                                  input_data -100           
                 scaled\_data =  -------------------         
                                         750                
                                                            
              Range of scaled data:  [-0.04 to 1.23]        
                                                            
              Low value cutoff:      0.00                   
                                                            
              Histogram bin width:   0.09    (41 bins)      
                                                            
                       Proceed            Abort             
                                                            

\end{verbatim}
\end{center}   

      The screen's cursor can be moved via the arrow keys to select any
   one of the items which are shown in the above illustration in bold
   print.  If the user hits $<$return$>$, then the program will ask for a new
   value of the selected variable.  When the scaling equation, low value
   cutoff, and histogram bin width are appropriate, then the user should
   select `Proceed'.  At this point the program will construct the
   histogram and display it in the following manner:
\begin{center}
\begin{tabular}{||l l l|l l l||}   
\hline

             Bin &   Range  &   Count &   Bin &   Range &    Count \\
              1 &  0.00 - 0.03  &   13 &  23 &  0.66 - 0.69  &   27  \\
              2 &  0.03 - 0.06  &   16 &  24 &  0.69 - 0.72  &   24  \\
              3 &  0.06 - 0.09  &   51 &  25 &  0.72 - 0.75  &   29  \\
              4 &  0.09 - 0.12  &  136 &  26 &  0.75 - 0.78  &   24  \\
              5 &  0.12 - 0.15  &  271 &  27 &  0.78 - 0.81  &   25  \\
              6 &  0.15 - 0.18  &  405 &  28 &  0.81 - 0.84  &   45  \\
              7 &  0.18 - 0.21  &  543 &  29 &  0.84 - 0.87  &   34  \\
              8 &  0.21 - 0.24  &  641 &  30 &  0.87 - 0.90  &   39  \\
              9 &  0.24 - 0.27  &  734 &  31 &  0.90 - 0.93  &   52  \\
             10 &  0.27 - 0.30  &  572 &  32 &  0.93 - 0.96  &   69 \\
             11 &  0.30 - 0.33  &  460 &  33 &  0.96 - 0.99  &   70 \\
             12 &  0.33 - 0.36  &  355 &  34 &  0.99 - 1.02  &   75 \\
             13 &  0.36 - 0.39  &  240 &  35 &  1.02 - 1.05  &   66 \\
             14 &  0.39 - 0.42  &  144 &  36 &  1.05 - 1.08  &   40 \\
             15 &  0.42 - 0.45  &   98 &  37 &  1.08 - 1.11  &   37 \\
             16 &  0.45 - 0.48  &   67 &  38 &  1.11 - 1.14  &   20 \\
             17 &  0.48 - 0.51  &   52 &  39 &  1.14 - 1.17  &    8 \\
             18 &  0.51 - 0.54  &   33 &  40 &  1.17 - 1.20  &    3 \\
             19 &  0.54 - 0.57  &   36 &  41 &  1.20 - 1.23  &    1 \\
             20 &  0.57 - 0.60  &   33 &     &               &      \\
             21 &  0.60 - 0.63  &   30 &     &               &      \\
             22 &  0.63 - 0.66  &   26 &     &               &      \\
\hline
\multicolumn{6}{||l||}{Do you want to impose a cutoff for high values?   Yes  
No} \\ \hline
\end{tabular}
\end{center}

      The user is given an opportunity to impose a high value cutoff on
   the histogram.  If the user selects `Yes' to the question (using the
   arrow and return keys again), then he will be able to select which bin
   should be the last bin.  The count values from the higher bins will be
   added into the chosen final bin.
      The program will then tell the user the name of the output file,
   write the histogram data to that file, and terminate.  The output file
   will be placed in the same directory as the input file, the file title
   will be the title of the input file concatenated with a two character
   string which denotes the bin width, and the file type will be 
   .HIS. These files are the required input for AMOEBA and may themselves be
   examined in IDL via the GETDATA program.

\subsubsection{AMOEBA}

{\em To run this program type:}
\begin{verbatim}
   
           HRTS$> edit/edt files.dat              
           HRTS$>edit/edt amoeba.dat
           HRTS$> AMOEBA
   
\end{verbatim}
{\em AMOEBA} is the most important program in the H\&D curve development
   process.  This program yields five coefficients for an H\&D curve, given
   at least two density histograms and some first-guess values for the
   coefficients.  In order to run the program the user must first see to
   it that the file named FILES.DAT contains certain information regarding
   the histograms (.HIS files) to be used, and that the file AMOEBA.DAT
   contains initial conditions for the coefficients.  Both of these files
   can be examined and altered via the editor.
      The format for the FILES.DAT file is shown below:
\begin{center}
\begin{verbatim}   
                
                    3   | Number of Histograms       
                    1   | Number of Base-Histograms   
                    0   | Chi Square Weighting        
                   .7   | Reference Density           
                   1.4  | Reference Value of Log E    
                   BH0601T05.his                      
                   BH0501T05.his                      
                   BH0301T05.his                      
\end{verbatim}
\end{center}                  
   
      The number of histograms to be used can be any number greater than
   or equal to 2.  That number must be entered on the first line of
   FILES.DAT.  The number of base-histograms can be any number greater
   than or equal to 1 and less than or equal to the number of histograms
   (entered on line \#1).  The number of base-histograms must be entered on
   line \#2.  The third line contains a number which will be interpreted as
   W in the following sum which computes the measurement by which the
   correspondence of actual and predicted histograms is evaluated:
\begin{center}
\begin{displaymath}
 \sum_{i=1}^{NBins} \frac{(Act[i]-Pred[i])^{2}}{[(Act[i]+Pred[i])/2]^{W}}    
\end{displaymath}                
\end{center}   
   where Act[i] is the number of pixels in bin number i for an actual
   histogram, Pred[i] is the corresponding number for an associated
   predicted histogram and NBins is the total number of bins for each
   histogram.
      The fourth line contains a reference density value and the fifth line
   contains a reference value of Log E.  The generated H\&D curves will all
   yield the reference Log E value when evaluated at the reference density
   value.
      Starting on line \#6 is a list of the histogram (.HIS) data files,
   with one filename listed on each line.  The number of filenames must be
   equal to the number of histograms written on line \#1.  Those histograms
   which are intended to be used as bases should be listed first.  For
   instance if the number on line \#2 is `1', then the first listed
   histogram will be used as a base, if that number is `2', then the first
   two listed histograms will be used as bases.
      The file AMOEBA.DAT must also be edited so that it contains an
   initial simplex.  The simplex consists of five sets of the coefficients
   a(1), a(2), a(3) and a(4).  An example of how AMOEBA.DAT should appear
   is shown below.
\begin{center}
\begin{verbatim}      
   
                         
                           1.   1.   .1   1.  
                           1.1  1.   .1   1.  
                           1.   1.1  .1   1.  
                           1.   1.   .11  1.  
                           1.   1.   .1   1.1 
            
\end{verbatim}   
\end{center}
      Each line of the file contains a set of {a(1),a(2),a(3),a(4)}.
   Among the five sets there should be a perturbation of each coefficient,
   or else that coefficient will not be varied when the program runs.  The
   perturbation for the coefficient a(3) should be relatively small in
   order to hopefully prevent that coefficient being assigned a value less
   than 0 as the program runs (if this occurs the program would fail).
   
      Note:  the choice of initial coefficient values is somewhat
          arbitrary, and the program may very well yield a different
          solution for different initial conditions.
   
      When the AMOEBA.FOR program is run, it will read data from FILES.DAT
   and AMOEBA.DAT and then print out actual and predicted histograms for
   the first set of coefficients listed in AMOEBA.DAT.  The program then
   goes through various iterations in which the simplex is moved through
   four dimensional space.  During each iteration an attempt is made to
   change one set of coefficients in order to lower the Chi Square
   difference between actual and predicted histograms.  The following
   illustration shows the type of information displayed on the terminal
   screen to the user as this process is going on.
\begin{center}
\begin{verbatim}   
     
        19        tol: .309                                           
          a(1)      a(2)      a(3)      a(4)      a(5)      Chi Sq.   
       -0.11622   0.55758   0.11826   1.53162  -0.88232   0.42345e+08 
       -0.01432   0.48709   0.12383   1.50887  -1.00033   0.36822e+08 
       -0.15971   0.28770   0.14706   1.48719  -1.07811   0.33944e+08 
        0.07918   0.45969   0.12258   1.46260  -1.03279   0.37795e+08 
       -0.07311   0.27037   0.13193   1.53906  -1.15574   0.31022e+08 
                                                                      
       20        tol: .380                                            
          a(1)      a(2)      a(3)      a(4)      a(5)      Chi Sq.   
       -0.10645   0.01348   0.15754   1.43505  -1.39138   0.25737e+08 
       -0.01432   0.48709   0.12383   1.50887  -1.00033   0.36822e+08 
       -0.15971   0.28770   0.14706   1.48719  -1.07811   0.33944e+08 
        0.07918   0.45969   0.12258   1.46260  -1.03279   0.37795e+08 
       -0.07311   0.27037   0.13193   1.53906  -1.15574   0.31022+e08 
                                                                      
       21        tol: .354                                            
          a(1)      a(2)      a(3)      a(4)      a(5)      Chi Sq.   
        0.10645   0.01348   0.15754   1.43505  -1.39138   0.25737e+08 
       -0.01432   0.48709   0.12383   1.50887  -1.00033   0.36822e+08 
       -0.15971   0.28770   0.14706   1.48719  -1.07811   0.33944e+08 
        0.02200   0.36217   0.13134   1.47757  -1.10147   0.31847e+08 
       -0.07311   0.27037   0.13193   1.53906  -1.15574   0.31022e+08 
                                                                      
\end{verbatim}
\end{center}                                                                      
                                                                      
      There is a table output for each iteration.  The number near the
   top left corner of each table is the iteration number.  The number
   identified as `tol' is a tolerance value which the program evaluates as
   a function of the differences among the five Chi Square values
   associated with the simplex.  In a sense the tolerance may be looked
   upon as an indication of the size of a simplex.  If, after successive
   iterations, the tolerance value gets very small, then the program will
   assume that the simplex has converged on a point and thus the program
   will stop.  This is the optimal form of program termination.  The
   tolerance generally decreases as the iteration number increases, but as
   shown in the above example that is not always the case between
   successive iterations.  The table shows five sets of coefficients and
   the associated measure showing how good those coefficients are when
   used to create predicted histogram(s) from the base histogram(s) in an
   attempt to simulate other actual histograms.
       Listed below are various conditions under which the program either
   will terminate or be terminated by the user:
\begin{itemize}   
         \item  The tolerance goes to a very small number (as described
                above).  This is normal termination.
         \item  The simplex gets `stuck' in the sense that the tolerance
                and all the coefficients become constant for each
                iteration.  In this case the user should halt the
                program with cntl-Y and restart it with new starting
                condition in AMOEBA.DAT which are close to the coefficient
                values used when the simplex got stuck.
         \item  The program terminates under an error condition because
                the value for a(3) has become less than or equal to
                zero.  In this case the user should restart the program
                with different values in AMOEBA.DAT.
         \item  All coefficients in each vertical column of the table
                are the same to within a desired number of digits of
                accuracy.  In this case the user may hit cntl-Y and
                accept the current solution.
\end{itemize}

       Under normal termination there are five sets of coefficients
   displayed.  The user should accept the one of those which has the
   smallest chi square error as being the solution.  It is strongly
   encouraged that the user rerun the program under different starting
   conditions to see if other solutions are obtained which may have a
   lower Chi Square error.

\subsubsection{TESTHND}

{\em To run this program type:}
\begin{verbatim}
     HRTS$> TESTHND              
\end{verbatim}
{\em TESTHND} prints out actual and predicted histograms in which the
   predicted histograms were obtained by applying an H\&D curve to one or
   more base-histograms.  The user can compare the actual and predicted
   histograms to get a qualitative feel about the appropriateness of the
   supplied H\&D curve.  This program also prints out values of Log E which
   are associated by the H\&D curve with certain density values.

       The information regarding details about the histogram files to be
   used will be read in by the program from the file named FILES.DAT
   (which was described in the section for AMOEBA.FOR.)  When the user
   starts to run the program, he will be asked to type in the first four
   coefficients of the H\&D curve (a[1] through a[4]).  a[5] need not be
   input.  The program will then output a table of histograms which may
   look something like the following illustration.
\begin{center}
\begin{tabular}{||l|l|l|l|l|l||}
\hline   
        Histogram Contents: &  &  &  &  & \\                                        
             &     1.act  &    2.act  &    2.pf1  &    3.act   &   3.pf1  \\
          1  &         1  &      488  &      109  &        0 &         0 \\
          2  &        84 &      2901 &       824 &         0 &         0 \\
          3  &       159 &       8791 &      4619  &         1 &        28  \\
          4  &       301 &      18579 &      13054 &          1 &        32  \\
          5  &       528 &      24395 &      25848 &         10 &         39 \\ 
          6  &      1131  &    27319   &   27107    &     54    &     53  \\
          7  &      1695  &    27192   &   26580    &     61    &     53  \\
          8  &      2421  &    26296   &   22352    &     50    &     72  \\
          9  &      3744  &    24580   &   20475    &     50    &    105  \\
         10 &       4744  &    17372   &   14902    &     60    &    105  \\
         11 &       7414  &    15869   &   13574    &    109    &    168  \\
         12 &      10717 &     13235   &   13899    &    151    &    231  \\
\hline
\end{tabular}                                                             
\end{center}                                                                    
       The number of columns in the table will be determined by the total
   number of histograms and the number of base-histograms as read in from
   FILES.DAT. For the above table, the number of histograms was 3 and the
   number of base-histograms was 1.  Each horizontal row of the table
   represents one density bin, in which the leftmost number is the bin
   number.  The title `2.act' designates the actual histogram data for
   histogram \#2 (the second histogram listed in FILES.DAT).  The title
   `2.pf1' designates a simulated version of histogram \#2, which was
   predicted from histogram \#1 (histogram \#1 was used as the base).
       Besides writing the histograms data as shown above to the terminal
   screen, this program also writes that data to a file as well.  This
   file is named PRED.HIS.  This is done so that the user has the ability
   to make plots of the above data in IDL by using the GRHIST.PRO
   procedure.
       TESTHND also outputs sets of density and Log E values such as shown
   in the table below.
\begin{center}
\begin{tabular}{||l l l l l l||}
\hline   
         Density &  Log(E) &   Density &  Log(E)  &    Density &  Log(E) \\
\hline
          0.05 &   0.467   &    0.40  &  1.203    &     1.15  &  1.798  \\ 
          0.10 &   0.741   &    0.50  &  1.272    &     1.30  &  1.996  \\ 
          0.15 &   0.890   &    0.60  &  1.336    &     1.45  &  2.241  \\ 
          0.20 &   0.988   &    0.70  &  1.400    &     1.60  &  2.538  \\ 
          0.25 &   1.069   &    0.80  &  1.470    &     1.75  &  2.897  \\ 
          0.30 &   1.116   &    0.90  &  1.548    &     1.90  &  3.325  \\ 
          0.35 &   1.163   &    1.00  &  1.637    &     2.05  &  3.833  \\ 
\hline
\end{tabular}
\end{center}

\subsection{IDL level}

\subsubsection{GRHIST}

{\em To run this program type:}
\begin{quote}   
     {\tt HRTS\_IDL>}{\bf GRHIST},{\it NOLOG,LINETYPE}              
\end{quote}

{\em GRHIST} will read histogram data from the file PRED.HIS and plot a
   histogram on a graphics terminal.  The procedure always uses the most
   recent version of PRED.HIS that is in your current directory (PRED.HIS
   is generated when the {\em TESTHND.EXE} program is run.)  All arguments to
   this procedure are optional.  If NOLOG is undefined or has a values of
   0, then the plotted graph will have its Y-axis on a logarithmic scale,
   else the Y-axis will be on a linear scale.  

       When {\em GRHIST} runs, it first asks for the number of the histogram
   that you are interested in (this number will typically be in the range
   1 to 3).  The histograms are numbered according to the order they
   were listed in FILES.DAT when the TESTHND program was run.  The
   program next asks if you want to plot the actual or a predicted
   histogram.  If you request a predicted histogram, then the program will
   also ask for the number of the base-histogram from which the predicted
   histogram was generated.  The graph will then be generated and the
   program will terminate.

\subsubsection{GRHND}

{\em To run this program type:}
\begin{quote}   
      {\tt HRTS\_IDL>}{\bf GRHND},{\it A1,A2,A3,A4,DENMAX,LINETYPE}       
\end{quote}   
       This procedure will plot an H\&D curve on a graphics terminal.  The
   graph will consist of Log E values on the X-axis in the range from -1
   to 4, and density values on the Y-axis in the range from 0 to DENMAX
   (if a value for that variable is given, else a default value of 2 will
   be used).  The first four parameters are required and must specify the
   coefficients of the H\&D curve.  The curve will be shifted so that a
   density of .7 corresponds to a Log E value of 1.4.  The sixth
   parameter, LINETYPE, is optional.  If LINETYPE is assigned a value,
   then !P.LINESTYLE will be set to that value and the graph will be drawn
   using OPLOT, otherwise it will be drawn using PLOT.
\newpage

\section{Miscellaneous Programs}

\subsection{DCL level}

\subsubsection{CALC}

{\em To run this program type:}
\begin{verbatim}   
     HRTS$> CALC                    
\end{verbatim}
       {\em CALC} performs a calculator simulation that uses the keyboard's 18
   keypad keys to correspond to the various keys. 
        The user may hit `H' to get help information or `E' to exit the
   program.

\subsubsection{IONEQ}

{\em To run this program type:}
\begin{verbatim}
     HRTS$> IONEQ                  
\end{verbatim}
      This program evaluates ion equilibria by using the tables of Arnaud
   and Rothenflug: 1985, Astron. Astrophys. Suppl. Ser. 60, 425-457. Cubic
   splines are used to interpolate between table entries.  The program
   requires three input values: an element number (e.g. 26 for Fe), an
   ionization state (e.g. 17 for Fe XVII), and a temperature in degrees K.
   Below is a example showing how to use this program:

\begin{verbatim}

               HRTS$> IONEQ
               IONEQ: IZ=element #, IS=ion.state  (Enter 0,0,0 to exit)
               Enter IZ,IS,TEMP:  26,17,2e6
                0.4236
               Enter IZ,IS,TEMP:  0,0,0
               End IONEQ.
               HRTS$>
\end{verbatim}
            Ion equilibria can also be evaluated within a FORTRAN program by
   making a call to IONEQ\$. This function has four arguments, of which the
   first three must be assigned values prior to the call. The arguments
   are as follows:
\begin{itemize}   
\item     1. {\bf IZ}    {\em INTEGER}        element number
\item     2. {\bf ION}   {\em INTEGER}        ion number
\item     3. {\bf TEMP}  {\em FLOATING PT}    temperature in K
\item     4. {\bf IERR}  {\em INTEGER}        output error code (0 = success)
\end{itemize}   
        Below is a sample program that calls this function:

\begin{verbatim}   
                       real ioneq$
                       ans=ioneq$(1,2,20000,ierr)
                       if(ierr.eq.0) print*,ans
                       end

\end{verbatim}

\subsubsection{TMAX}

{\em To run this program type:}
\begin{verbatim}   
     HRTS$> TMAX                   
\end{verbatim}
This program will calculate and print out temperature maximum values.  When
run, it will ask the user to enter a set of four numbers. These numbers will be
interpreted as an element number, (e.g. 26 for Fe.)  an ion, (e.g. 17 for Fe
XVII.) an actual emitting wavelength (Angstroms), and an excitation wavelength
(Angstroms).  The use may repeatedly enter in sets of four numbers.  If
`0,0,0,0' is entered, then the program will exit.

\subsection{IDL level}

\subsubsection{COLSUM}

{\em To run this program type:}

\begin{quote}   
     {\tt HRTS\_IDL>}{\bf c=COLSUM(ARR)}            
\end{quote}   
      This function returns a one dimensional array of which each element
   contains the average of the pixels along a column of the inputed two-
   dimensional array.

\subsubsection{DISPLAY}

{\em To run this program type:}

\begin{quote}   
      {\tt HRTS\_IDL>}{\bf DISPLAY,}{\it ARR,X,Y,FRAC}        
\end{quote}   

This procedure will display an image on a graphics terminal in which all data
points with a value greater than a certain level will be shown by a high
brightness, and all other data points will be shown by a low brightness. The
maximum image size is 1023 x 791. ARR is the only required parameter and must
be a two dimensional array of data.  X and Y, if given, will cause the image to
be placed at the specified location (where 0$<$=X$<$=1023, and 0$<$=Y$<$=791). 
FRAC specifies what fraction of the data points should be marked by a bright
points.  FRAC must be a number between 0.0 and 1.0 (the default value is 0.1).  

\subsubsection{INT}

{\em To run this program type:}
\begin{quote}   
     {\tt HRTS\_IDL>}{\bf INT,ARR,}{\it GR250,ORIENT,X,Y,NOERAS}        
\end{quote}   

{\em INT} represents the contents of a two dimensional array by an image on the
graphics terminal where the brightness of the image at various locations is
related to the numerical value of an associated pixel of the input array.  The
only required variable, ARR, must be a two dimensional array.  GR250 should be
assigned a value of 1 if you are using a GraphOn 250 terminal, else it should
be assigned a value of 0. If ORIENT is assigned a non-zero value, then the
image will be displayed with the first row of data on the bottom, else the
first row of data will be at the top.  If integer values are sent for X and Y,
then those values will be interpreted as the location on the screen where the
lower left corner of the graph will be positioned.  If NOERAS is assigned a
value, then the screen will not be erased before the plot is drawn, otherwise
it will be erased.  

The largest image which this program can display for a non-GraphOn 250 terminal
consists of 300 X-direction pixels and 115 Y-direction pixels.  If the input
array is too big in either dimension, then the program will ask for a starting
pixel for that dimension, and then proceed to display as much of the image as
possible.  As a result of the screen resolution for non-GraphOn 250 terminals,
an image that has the same number of pixels in the X-direction as in the
Y-direction will appear to be twice as high as it is long.  If the user wants
the image to appear square on non-GraphOn 250 terminals, then he should use the
intrinsic CONGRID procedure to stretch out the image by a factor of two in the
X-direction.

For each element of the array, this procedure will associate a 3x3 array of
physical screen pixels.  A certain number of the nine screen pixels will be
bright and the remaining ones will be dark depending on the numerical value of
the array element.  Hence the resulting image will be drawn using a ten level
gray scale.  (Note:  for large arrays this procedure takes a long time to run,
in which case in might be a better idea to use the an image  display unit.)

\subsubsection{POLYFIT}

{\em To run this program type:}   
   
\begin{quote}   
    {\tt HRTS\_IDL>}{\bf POLYFIT,}{\it X,Y,WEIGHT,}YFIT,COEF      
\end{quote}   
{\em POLYFIT} fits a polynomial to a set of {x,y} points, which may be
weighted. If no parameters are given, then the program will ask for the name
of a file which contains the input data in the following format:
\begin{verbatim}                  

                       X1       Y1       Weight1 
                       X2       Y2       Weight2 
                       .        .           .    
                       .        .           .    
                    
\end{verbatim}   

The user may instead have the X, Y, and WEIGHT data transmitted to the program
by sending three arrays as the first three arguments when the program is
invoked.  If an argument is given for X and for Y, but not for WEIGHT, then a
constant weighting of `1' will be used for each data point.  If an arbitrary
variable name is given as the fourth parameter, then that name will reference
(after program termination) an array of predicted y values for each x data
value.  The predicted y values will be produced from the derived polynomial. 
If an arbitrary name is given for the fifth parameter, then that name will be
set to reference an array of the coefficients of the derived polynomial. (Note:
the predicted y values and the coefficients of the derived polynomial will be
displayed on the terminal whether or not a fourth and fifth parameter are
given.)

\subsubsection{POS}

{\em To run this program type:}
\begin{quote}   
      {\tt HRTS\_IDL>}{\bf POS}              
\end{quote}

{\em POS} allows the user to read off values of a graph which has been plotted
on the graphics terminal.  After plotting a graph and typing POS, the user may
hit `C' in order to access the graphics cursor.  The cursor may then be moved
via the arrow keys until it is at a desired location. If the user then hits the
return key, then the {x,y} position of the cursor will be printed on the text
screen.  Hitting `C' again will re-display the graphics screen and re-enable
the graphics cursor so that the process may be repeated.

In addition to this primary application of the program, a number of other
features are available which are described if the user presses the `H' key. 
The `D' command allows the user to set the number of significant digits to be
used in the output data coordinates.  The `F' command may be used to have all
recorded data coordinates be output to a file in addition to going to the text
screen.  The `L' command will toggle on or off a point labelling feature that
allows the point number to be printed on the graphics screen above each newly
selected point.  The `M' command will toggle on or off a point marking feature
that can cause a `+' sign to be placed on the graphics screen to mark each
newly selected point.  The `P' command enables the positioning of the graphics
cursor at a specific X position which the user supplies, or at a specific X,Y
position.  The `S' command is used to set the speed at which the graphics
cursor moves by setting the number of pixels the cursor moves per
arrow-keystroke. The program may be exited by hitting the `E' key.

\subsubsection{ROWSUM}   

\begin{quote}
     {\tt HRTS\_IDL>}{\bf r=ROWSUM}{\it (ARR)}            
\end{quote}   
This function returns a one dimensional array of which each element contains
the average of pixels along a row of the input  two-dimensional array.

\subsubsection{TVALIGN}

{\em To run this program type:}
\begin{quote}   
     {\tt HRTS\_IDL>} {\bf TVALIGN,}{\it ARR1,ARR2}      
\end{quote}   
      
This procedure is used to align two images.  The two required input arguments,
ARR1 and ARR2, must be two-dimensional arrays in which the size of each
dimension is less than or equal to 512. The procedure will begin by putting up
on the image display screen an  image formed from the data from ARR1 in green
and an image from ARR2 in blue. The images are shown using a colour table which
is a two-valued step function.  The user must interact with the procedure at
this point to redefine the colour table so that an alignment can be made later
in the second part of the program in which the two images overlap.  Basically,
one wants both images to appear similar.  The user modifies the colour table by
hitting either the `1' or `2' keys to work on image 1 or 2, and then hitting
the up or down arrow keys to adjust the colour table. When the user is
satisfied, he should hit `E' to exit this part of the program and proceed to
the second part.

In the second part of the program the green and blue images overlap and the
user is given the ability to move the blue image until the two images are
aligned.  The blue image is moved by a vector specified by the user.  The
direction of the vector is defined by hitting one of the four arrow keys, and
the magnitude is given (in pixels) by hitting a number key.  The user can
repeatedly move the blue image until the alignment seems to be good.  The user
should then hit `E' to exit and the program will then print out the X and Y
offsets between the two images.

\subsubsection{TVCOMPARE}

{\em To run this program type:}
\begin{quote}   
     {\tt HRTS\_IDL>} {\bf TVCOMPARE,}{\it A,B,}X,Y          
\end{quote}   

{\em TVCOMPARE} compares two images by using the image display to show both the
images along with their sum and difference, and allows the user to shift image
\#2 with regards to image \#1. A and B are required parameters and each must
denote a two dimensional array. The variables X and Y are optional. If a
variable name is given for each of these arguments, then when the program is
finished those variable names will reference the x and y shift values.  Image
\#2 can be shifted by hitting the `M' key and by using the arrow and return
keys. The user should hit the `E' key to exit the program.

\subsubsection{TVCUT}

{\em To run this program type:}
\begin{quote}   
     {\tt HRTS\_IDL>} {\bf TVCUT,OUT}               
\end{quote}

{\em TVCUT} reads off data from the image display screen for an area defined by 
the user and stores that data in the array OUT. The user must display whatever
image he wants to work with on the image display screen prior to making the 
call to TVCUT.  The one parameter, OUT, is required and must be given an
arbitrary variable name which will point to a two dimensional array of image
data after the program exits.  When this program runs, its first asks the user
to use the mouse to mark a point on the image screen from which the closed loop
(which will define the area of interest) will begin to be drawn.  After moving
the mouse to a point and hitting a mouse button, the program will then ask the
user if he wishes to draw the area by using either the mouse or arrow keys. 
The user should press `M' to use the mouse or `A' to use the arrow keys.  The
user should then draw a loop on the  screen which encloses the desired area. 
This loop must be closed for the program to go on to the next stage.  Since it
is hard to close the loop at the exact point at which it began, this program
has a feature that allows the user to force the loop to close by hitting the
first mouse button if he is using the mouse, or hitting the return key if he is
using the arrow keys.  (It is best to use this feature when the loop is almost,
but not quite, closed.) The program will then extract the data within the loop
and return it in the array OUT.  The array size will be as small as possible
and still contain all the requested data.  Any pixels in the array which
correspond to an area outside of the loop will be assigned a value of zero.

\subsubsection{TVPOS}

{\em To run this program type:}   
\begin{quote}      
     {\tt HRTS\_IDL>} {\bf TVPOS,}{\it DISPLAY}                
\end{quote}   

{\em TVPOS} displays the x and y screen coordinates and the intensity for the
pixel at which the TV cursor is currently located.  While this procedure is
running the user may change the TV cursor position by moving the mouse.  {\em
TVPOS} has one optional argument, DISPLAY, which may be used to specify where
the TV cursor position numbers will be displayed (only in the case in which
there was no zooming done on the TV image).  If DISPLAY is assigned no value,
or if it is assigned a value of 0, then the information will be displayed on
the TV.  If it is assigned a value of 1, then the information will be shown on
the terminal screen.  An assigned value of 2 will cause the information to be
displayed on both the TV and the terminal screen.  If there is a zoomed-in
image on the TV, then the program will place all output on the terminal screen. 
Also, in addition to the screen coordinates and intensity, the data coordinates
will be printed out. The user may exit the program by hitting any keyboard key.

\subsubsection{TVRESTORE}

{\em To run this program type:}
\begin{quote}   
     {\tt HRTS\_IDL>} {\bf TVRESTORE,}{\it FILENAME}         
\end{quote}   
This procedure will display on the  screen an image which was saved via {\em
TVSAVE.} The input file name is a required argument.

\subsubsection{TVSAVE}

{\em To run this program type:}
\begin{quote}
    {\tt HRTS\_IDL>} {\bf TVSAVE,}{\it FILENAME}          
\end{quote}      
This procedure will save an image shown on the  screen into a file, from which
it may be put back onto the image display screen using {\em TVRESTORE.}  The
procedure will ask the user if there is graphics data to be saved in addition
to image data (the user must answer by pressing either `Y' or `N'.)  The output
filename is a required argument.

\subsubsection{TVSTART}

{\em To run this program type:}
\begin{quote}   
     {\tt HRTS\_IDL>} {\bf TVSTART}         
\end{quote}   
{\em TVSTART} allocates the image display unit and defines a red-temperature 
colour table to be used for subsequent image displays. The image display must
be turned  on and be available. {\em TVSTART} requires no arguments.

\typeout{ }
\typeout{ If you want to print this document on a PostScript printer, you}
\typeout{ should use the DVIPSA4 translator program to process the .DVI file}
\typeout{ }

\end{document}
