\documentclass[11pt,nolof,noabs]{starlink}

% -----------------------------------------------------------------------------
% ? Document identification
\stardoccategory    {Starlink User Note}
\stardocinitials    {SUN}
\stardocsource      {sun207.1}
\stardocnumber      {207.1}
\stardocauthors     {J.\,W.\,Palmer}
\stardocdate        {21 February 1996}
\stardoctitle       {AIPS --- Astronomical Image Processing System}
% ? End of document identification
% -----------------------------------------------------------------------------
% ? Document specific \providecommand or \newenvironment commands.
% ? End of document specific commands
% -----------------------------------------------------------------------------
%  Title Page.
%  ===========
\begin{document}
\scfrontmatter

\section{What is AIPS}

The NRAO Astronomical Image Processing System (AIPS) is a software
package for interactive (and, optionally, batch) calibration and
editing of radio interferometric data and for the calibration,
construction, display and analysis of astronomical images made from
those data using Fourier synthesis methods.  It is the principle tool
for display and analysis of both two- and three-dimensional radio
images (\emph{i.e.}, continuum ``maps'' and spectral-line ``cubes''
from the NRAO's Very Large Array (VLA) as well as MERLIN data.

It is also used for astronomical applications that do not involve radio
interferometry.  These uses include the display and analysis of line
and continuum data from large single-dish radio surveys, and the
processing of image data at infrared, visible, ultraviolet and X-ray
wavelengths.

\section{Availability}

The software runs on a range of computer architectures including DEC
Alpha (Digital UNIX), Sun (Solaris) and PC (Linux).

The software is Copyright (C) 1995 by Associated Universities, Inc.,
and is protected by the Free Software Foundation's General Public
License (GPL).  This means it is freely available and no user
agreement is needed.

Source and binary distributions for all platforms are available from
NRAO. Additionally, versions built on Starlink Sun and Alpha machines
can be obtained from the Manchester anonymous ftp site (\texttt{ftp.ast.man.ac.uk}, in \texttt{outgoing/aips}).

\section{Further Information}

Further information, including details of how to obtain and install
the software is available on the WWW at :

\begin{quote}
\url{http://axp2.ast.man.ac.uk:8000/~jwp/aips_home.html}
\end{quote}

The page includes links to the main NOAO site, the Manchester \texttt{ftp}
area (containing both source and binary distributions for Starlink
hardware) and details of the latest patches available.  There is also
a link to the latest AIPS Cookbook, available as Starlink MUD/101.

\end{document}

