\documentstyle{article} 
\pagestyle{myheadings}

%------------------------------------------------------------------------------
\newcommand{\stardoccategory}  {Starlink User Note}
\newcommand{\stardocinitials}  {SUN}
\newcommand{\stardocnumber}    {75.1}
\newcommand{\stardocauthors}   {M D Lawden, M J Bly}
\newcommand{\stardocdate}      {12 December 1990}
\newcommand{\stardoctitle}     {TOOLPACK --- Fortran 77 Software Tools}
%------------------------------------------------------------------------------

\newcommand{\stardocname}{\stardocinitials /\stardocnumber}
\renewcommand{\_}{{\tt\char'137}}     % re-centres the underscore
\markright{\stardocname}
\setlength{\textwidth}{160mm}
\setlength{\textheight}{240mm}
\setlength{\topmargin}{-5mm}
\setlength{\oddsidemargin}{0mm}
\setlength{\evensidemargin}{0mm}
\setlength{\parindent}{0mm}
\setlength{\parskip}{\medskipamount}
\setlength{\unitlength}{1mm}

\begin{document}
\thispagestyle{empty}
SCIENCE \& ENGINEERING RESEARCH COUNCIL \hfill \stardocname\\
RUTHERFORD APPLETON LABORATORY\\
{\large\bf Starlink Project\\}
{\large\bf \stardoccategory\ \stardocnumber}
\begin{flushright}
\stardocauthors\\
\stardocdate
\end{flushright}
\vspace{-4mm}
\rule{\textwidth}{0.5mm}
\vspace{5mm}
\begin{center}
{\Large\bf \stardoctitle}
\end{center}
\vspace{5mm}

Starlink has purchased Release 2 of Toolpack/1 from NAG for its users.
The first release of Toolpack was the result of an international collaborative
effort started in 1979.
The original project aims were twofold:
\begin{itemize}
\item To provide a suite of tools to assist the production, testing,
maintenance, and transportation of medium-sized mathematical software projects
written in standard Fortran 77.
\item To investigate the development of extensible programming support
environments built around integrated tool suites.
\end{itemize}
Readers who are interested in learning about the facilities offered by
Toolpack should read the {\em Toolpack/1 Release 2 Introductory Guide} which
can be obtained from your Site Manager.
If you then wish to use any of the facilities, you should consult the
{\em Toolpack/1 Release 2 Technical Reference Manual}.
This can also be obtained from your Site Manager.

To run Toolpack, the command
\begin{verbatim}
        $ TOOLPACK
\end{verbatim}
should be issued. This will run a command procedure which sets up the names
of the Toolpack tools as DCL symbols. {\tt TOOLPACK} is a DCL symbol. It could 
be added to the users {\tt LOGIN.COM} file.

Users must also define the logical name TIE{\tt{\$}}PFS as follows:
\begin{verbatim}
        $ DEFINE TIE$PFS "DISK:/dir[/dir[/dir]...]"
\end{verbatim}
where {\tt DISK} is the disk, and {\tt /dir[/dir[/dir...]} is the path of 
directories to the location of the files to be worked on. 

For example, the directory
structure:
\begin{verbatim}
        DISK$USER:[STAR.FORTRAN]
\end{verbatim}
would be specified as:
\begin{verbatim}
        $ DEFINE TIE$PFS "DISK$USER:/STAR/FORTRAN/"    
\end{verbatim}
In addition to the standard tools, five command procedures are provided, which
combine the use of several tools to achieve more advanced processing of code.
These command procedures have been given symbols, defined in the TOOLPACK.COM
procedure. They are:
\begin{description}
\item [FFC] -
A command procedure which checks source code for conformance to the full
FORTRAN77 standard using Toolpack/1 tools. Uses normal VMS style filenames.
\item [APT] -
A command procedure which does Arithmetic Precision Transformation, using two
of the Toolpack/1 monolithic tools. Uses normal VMS filenames.
\item [PORT] -
Checks source code for portability. Uses normal VMS filenames.
\item [STRUCT] -
This structures a fortran program to make it "look good". A final polishing
tool. Uses normal VMS filenames.
\item [UNROLL] -
A procedure for "unrolling" Do loops.
\end{description}
\end{document}
