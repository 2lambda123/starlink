\documentstyle[11pt]{article}
\pagestyle{myheadings}

%------------------------------------------------------------------------------
\newcommand{\stardoccategory}  {Starlink User Note}
\newcommand{\stardocinitials}  {SUN}
\newcommand{\stardocnumber}    {153.1}
\newcommand{\stardocauthors}   {V.~Laspias}
\newcommand{\stardocdate}      {20 October 1992}
\newcommand{\stardoctitle}     {RNOTOTEX --- Converting RUNOFF to \LaTeX }
%------------------------------------------------------------------------------

\newcommand{\stardocname}{\stardocinitials /\stardocnumber}
\renewcommand{\_}{{\tt\char'137}}     % re-centres the underscore
\markright{\stardocname}
\setlength{\textwidth}{160mm}
\setlength{\textheight}{230mm}
\setlength{\topmargin}{-2mm}
\setlength{\oddsidemargin}{0mm}
\setlength{\evensidemargin}{0mm}
\setlength{\parindent}{0mm}
\setlength{\parskip}{\medskipamount}
\setlength{\unitlength}{1mm}

%------------------------------------------------------------------------------
% Add any \newcommand or \newenvironment commands here
%------------------------------------------------------------------------------

\begin{document}
\thispagestyle{empty}
SCIENCE \& ENGINEERING RESEARCH COUNCIL \hfill \stardocname\\
RUTHERFORD APPLETON LABORATORY\\
{\large\bf Starlink Project\\}
{\large\bf \stardoccategory\ \stardocnumber}
\begin{flushright}
\stardocauthors\\
\stardocdate
\end{flushright}
\vspace{-4mm}
\rule{\textwidth}{0.5mm}
\vspace{5mm}
\begin{center}
{\Large\bf \stardoctitle}
\end{center}
\vspace{5mm}

%------------------------------------------------------------------------------
%  Add this part if you want a table of contents
%  \setlength{\parskip}{0mm}
%  \tableofcontents
%  \setlength{\parskip}{\medskipamount}
%  \markright{\stardocname}
%------------------------------------------------------------------------------

\section{Introduction}

The program {\bf RNOTOTEX} will read a {\em DEC Standard Runoff} (DSR) format
text file, translate the {\bf RUNOFF} commands to \LaTeX\ commands and write a
corresponding TEX file.  The output file may then be processed by \LaTeX\ and
printed on the local high quality printer.

\section{Execution}

To set up {\bf RNOTOTEX}, type the command:

\begin{verbatim}
      $ RNOTOTEX_SETUP
\end{verbatim}

{\bf RNOTOTEX} requires two files names on the command line---the input {\tt
.RNO} file and the output {\tt .TEX} file. The command:

\begin{quote}\tt
\$ RNOTOTEX[/TEX] {\it input-filename}[.RNO] {\it output-filename}[.TEX]
\end{quote}

reads a {\em Digital Standard Runoff} format file and creates the \LaTeX\
equivalent.

\subsection{Parameters}

\begin{itemize}

\item{\tt input-filename} --- is the name of the input DSR format file.
(The default type is {\tt .RNO}.)

\item{\tt output-filename} --- is the name of the output \LaTeX\ format file.
(The default type is {\tt .TEX}.)

\item{\tt /TEX} --- The {\tt /TEX} qualifier instructs {\bf RNOTOTEX} to
preserve embedded \TeX\ and \LaTeX\ commands in the input DSR file.  Any such
commands found are copied directly to the output {\tt .TEX} file. (The default
is {\tt /NOTEX}.)

\end{itemize}

\subsection{Example}

The following DCL commands convert the DSR file {\it filename}.{\tt RNO} to
{\it filename}.{\tt TEX} and  prints the {\tt .DVI-CAN} file on a Canon laser
printer. The log of the conversion is also printed and deleted.

\begin{quote}\tt
\$ RNOTOTEX filename.RNO \\
\$ LATEX filename.TEX \\
\$ DVICAN filename.DVI \\
\$ PRCN filename.DVI-CAN \\
\$ PRINT/DELETE RNOTOTEX.LOG
\end{quote}

Note that there isn't a one-to-one mapping of DSR commands to \LaTeX\  commands
and that more \LaTeX\ text fits on a page than a {\bf RUNOFF} file. The  first
output may be messy!   The output {\tt .TEX} file must be \index{messy RNOTOTEX
output} edited after examining the first printing.  It may take a few tries to
get the output that you want.

The program was designed for those familiar with both \LaTeX\ and DSR and would
like to upgrade older documents to \LaTeX\ format. However, the program does
not cope very well with the old style Starlink document headers produced with
{\bf RUNOFF}. For complicated {\bf RUNOFF} styles such as these it is easier to
convert just the body of the document and use a purpose built \LaTeX\ style
shell.

\section{Translation Actions}

The program does four tasks.  It will translate {\bf RUNOFF} sectioning
commands to \LaTeX\ sectioning commands, translate {\bf RUNOFF} flags to
\LaTeX\ commands, translate {\bf RUNOFF} page directives to \LaTeX\ page
directives and adjust \LaTeX\ counters by the same amount as {\bf RUNOFF}
counters are to be adjusted.

\subsection{Conversion of Sectioning Commands}

{\bf RNOTOTEX} converts {\bf RUNOFF} sectioning commands to \LaTeX\ sectioning
commands.  The output \LaTeX\ style is always {\tt report} in 10 point font.
The sectioning commands are converted in Table \ref{section-commands}.

\begin{table}[h]
\begin{center}
\begin{tabular}{ll}
{\large\bf RUNOFF }	    & {\LaTeX\ \large\bf Command Generated} \\ \hline
 .HL1 text      & \verb+\chapter{text}+ \\
 .HL2 text      & \verb+\section{text}+ \\
 .HL3 text      & \verb+\subsection{text}+ \\
 .HL4 text      & \verb+\subsubsection{text}+ \\
 .HL5 text      & \verb+\paragraph{text}+ \\
 .HL6 text      & \verb+\subparagraph{text}+ \\ \hline
\end{tabular}
\caption{Conversion of RUNOFF Sectioning Commands\label{section-commands}}
\end{center}
\end{table}

\subsection{Conversion of RUNOFF Flags}

{\bf RUNOFF} flags are converted to \LaTeX\ commands by {\bf RNOTOTEX}.   This
is a direct conversion process, {\em i.e.} a {\bf RUNOFF} underline command is
converted to a \LaTeX\ underline command.  This does not always produce the
prettiest results (as one can use {\bf bold} and {\it italic} fonts in \LaTeX\
), but it is consistent.  Unfortunately, {\bf RNOTOTEX} does not have the
esthetic sense to decide when a font is more appropriate than \index{esthetic
sense} a converted {\bf RUNOFF} command.  The conversion process for {\bf
RUNOFF} flags is Table \ref{flag-commands}.

\begin{table}[h]
\begin{center}
\begin{tabular}{lcl}
{\large\bf RUNOFF Flag } & {\large\bf Flag } & {\LaTeX\ \large\bf Command Generated} \\
{\large\bf Name        } & {\large\bf Char } &		\\ \hline
Accept			& \_		   & \verb!\verb+?+! \\
Bold			& *		   & \verb+{\bf+$\cdots$\verb+}+ \\
Break			& \verb+|+	   & \verb+\\+ \\
Capitalize		& \verb+<+	   & (see below) \\
Comment		        &  !		   & \% \\
Hyphenate		& =		   & \verb+\hyphenate{+$\cdots$\verb+}+ \\
Index			& \verb+>+	   & \verb+\index{+$\cdots$\verb+}+ \\
Lowercase		& $\backslash	$  & (see below) \\
Overstrike		& \%		   & \verb+{\bf+$\cdots$\verb+}+ \\
Period		        & +		   & \verb+\frenchspacing+ \\
Space			& \#		   & \verb*+\ + \\
Subindex		& \verb+>+         & \verb+\index{+$\cdots$\verb+}+ \\
Substitute		& \$		   & {\em not translated } \\
Underline		& \&		   & \verb+\underline{+$\cdots$\verb+}+ \\
Uppercase               & \verb+^+         & (see below) \\ \hline
\end{tabular}
\caption{Conversion of RUNOFF Flags\label{flag-commands}}
\end{center}
\end{table}

Characters following CAPITALIZE, LOWERCASE or UPPERCASE flags are converted
to the appropriate case during the translation process by {\bf RNOTOTEX},
unless they are paired with other flags.  In these other cases, the flags in
{\bf RUNOFF} lock another flag on, such as the {\bf bold}ing flag described
in the DSR manual\cite{dsr-manual} on page 3-6.  {\bf RNOTOTEX} will lock the
appropriate flag to the state it is in and produce the appropriate
\LaTeX\ commands.

For example, {\bf RNOTOTEX} will convert:
\begin{verbatim}
      \^ *These words are in bold face.\*
\end{verbatim}
to
\begin{verbatim}
      {\bf These words are in bold face.}
\end{verbatim}

\subsection{Conversion of RUNOFF Directives}

{\bf RNOTOTEX} converts {\bf RUNOFF} directives to \LaTeX\ directives. {\bf
RUNOFF} allows one to enable or disable flags and change the default flag
character for a flag.  {\bf RNOTOTEX} will recognize these commands and change
the internal parsing setup so that the flag is (or is not) recognized by the
character specified.  See the {\bf RUNOFF} commands ``ENABLE'', ``DISABLE'',
``FLAGS'' and ``NO FLAGS'' described in the RUNOFF manual.

In addition, {\bf RNOTOTEX} will translate {\bf RUNOFF} commands such as
``NOFILL'', ``INDENT'' and ``LIST'' in the manner described in Table
\ref{directive-commands}.

\begin{table}[h]
\begin{center}
\begin{tabular}{ll}
{\large\bf RUNOFF } 	    & {\LaTeX\ \large\bf Command Generated} \\ \hline
.APPENDIX		    & \verb+\appendix+ \\
.BLANK $n$		    & \verb+\vspace{+$n$+\verb+}+ \\
.BREAK			    & \verb+\linebreak+ \\
.CENTRE {\it text }         & \verb+\begin{centering}+ \\
			    &   {\it text } \\
			    & \verb+\end{centering}+ \\
.FOOTNOTE {\it text}	    & \verb+\footnote{+{\it text }\} \\
.INDENT			    & \verb+\indent+ \\
.JUSTIFY		    & \verb+\obeylines \obeyspaces+ \\
.LIST $n$,``q''		    & \verb+\begin{enumerate}+ \\
.LITERAL		    & \verb+\begin{verbatim}+ etc. \\
.NOFILL			    & \verb+\sloppy+ \\
.PAGE			    & \verb+\pagebreak[4]+ \\
.SPACING		    & \verb+\setlength{\baselineskip}{n}+ \\
.SUBTITLE  {\it text}	    & \verb+\pagestyle{myheadings}+ etc. \\
.TITLE {\it text}	    & \verb+\pagestyle{myheadings}+ etc. \\ \hline
\end{tabular}
\caption{Conversion of RUNOFF Directives\label{directive-commands}}
\end{center}
\end{table}

\subsection{Conversion of RUNOFF Counters}

{\bf RNOTOTEX} converts the {\bf RUNOFF} counter changes to \LaTeX\ counter
adjustment changes.  The conversion process in described in Table
\ref{counter-commands}.

\begin{table}[h]
\begin{center}
\begin{tabular}{ll}
{\large\bf RUNOFF}	& {\LaTeX\ \large\bf Command Generated} \\ \hline
.NUMBER APPENDIX $n$   & \verb+\setcounter{\chapter}{+$n$\verb+}+ \\
.NUMBER LEVEL    $s,n$ & \verb+\setcounter{s}{n}+ \\
.NUMBER LIST     $n$   & \verb+\setcounter{\enumi}{n}+ \\
.NUMBER PAGE     $n$   & \verb+\setcounter{\page}{n}+ \\ \hline
\end{tabular}
\caption{Conversion of RUNOFF Numbering\label{counter-commands}}
\end{center}
\end{table}

\section{The Conversion LOG File}

Each time {\bf RNOTOTEX} is executed, it generates the file {\tt RNOTOTEX.LOG}.
It is a record of each conversion used to generate the output \LaTeX\ commands.
It is useful to users less familiar with \LaTeX\ and wish to learn by trial and
error rather than by reading the manual.  It may also be useful to experienced
\LaTeX\ users to quickly identify what \LaTeX\ will produce as output from the
generated TEX file without going through a complete printing.

It is recommended that the log file be printed and deleted for each execution.
For example, the command sequence:

\begin{quote}\tt
\$ RNOTOTEX {\it filename} {\it new\_filename}\\
\$ PRINT/DELETE RNOTOTEX.LOG
\end{quote}

generates the file {\it new\_filemane}.{\tt TEX} from the file {\it
filename}.{\tt RNO}  and prints the log of the conversion process.  The user
can then examine the printed log file and decide upon the editing required to
produce a ``nice'' \LaTeX\ output.

\begin{thebibliography}{99}
\addcontentsline{toc}{section}{Bibliography}


\bibitem{sun9} SUN/9 --- V. Laspias \& M. Bly {\em \LaTeX\ --- A Document
Preparation  System} \\ Starlink User Note 9, Starlink Project


\bibitem{dsr-manual} Digital Equipment Corporation {\em VAX Digital Standard
RUNOFF (DSR) Reference Manual } \\ Order \# AA-LA15A-TE, April 1988

\end{thebibliography}

\end{document}
