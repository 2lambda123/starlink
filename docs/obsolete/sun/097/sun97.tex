\documentstyle {article} 
\markright{SUN/97.2}
\setlength{\textwidth}{153mm}
\setlength{\textheight}{220mm}
\setlength{\oddsidemargin}{3mm}
\setlength{\evensidemargin}{3mm}
\pagestyle{myheadings}

\begin{document}
\thispagestyle{plain}
\noindent
SCIENCE \& ENGINEERING RESEARCH COUNCIL \hfill SUN/97.2\\
RUTHERFORD APPLETON LABORATORY\\
{\large\bf Starlink Project\\}
{\large\bf Starlink User Note 97.2}
\begin{flushright}
M D Lawden\\
2 June 1987
\end{flushright}
\vspace{-4mm}
\rule{\textwidth}{0.5mm}
\vspace{10mm}
\begin{center}
{\Large\bf REXEC --- Introduction \& Comparison with SCAR}
\end{center}
\vspace{10mm}
\section {INTRODUCTION}
REXEC is a relational database management system (DBMS) developed by the
Scientific Databases Section within the Central Computing Division at Rutherford
Appleton Laboratory.
The principal user document is the `R-EXEC Guide and Reference' written by
Brian J Read.
The reference part of this document is accessible via an on-line HELP system.
Some background material is contained in two RAL reports: RAL-86-053 and
RAL-86-084.

SCAR is a DBMS designed principally to facilitate access to existing large
astronomical databases.
It is documented in SUN/70.

This note demonstrates how to set up and use a private database using REXEC
and SCAR.
Its main purpose is to help you start to use REXEC effectively and to compare
its features with those of SCAR.
It is not intended to replace the R-EXEC guide or SUN/70.
Many people will simply want to use existing databases and this problem is
simpler than having to set up your own.
However, as a DBMS is a useful tool for private use, this note deals with
the problem of setting up databases as well as using them.

A good way for a novice to overcome his initial lack of knowledge of and
insight into new software is for him to run through a simple demonstration
example which starts from scratch and can be typed in as shown.
This clarifies what the software does, the concepts behind it, and how to use it
effectively.
The next two sections present such an example for REXEC and SCAR.
The examples do the same job and enable REXEC and SCAR to be compared step by
step.
They are followed by a section which compares the two systems.

Bug reports should be sent to RLVAD::STAR.
Queries should be made direct to Brian Read on 0235-21900 X6492; mail address
BJR@UK.AC.RL.IB.
\section {REXEC Example}
To run this example, copy files REXECDIR:TEST*.* into your own directory
and proceed as described below, typing in the commands as shown.
The files are:
\begin{tabbing}
XXXXXXXX\=TEST2DD.DATXXXX\=X\kill
\>\+TEST1.DAT\>Ship data - set 1\\
TEST2.DAT\>Ship data - set 2\\
TEST1DD.DAT\>Description file for TEST1.DAT\\
TEST2DD.DAT\>Description file for TEST2.DAT\\
TEST.COM\>Command procedure to run example
\end{tabbing}
If you don't want to type in the commands you can execute TEST.COM instead.
The raw data to be manipulated happens to be details of ships; the important
thing is that they are a collection of data in integer, real and character
format.
REXEC assumes a type default of .DAT for raw data files and data description
files.
It stores data it wishes to manipulate in `R-files' which have a type default
of .REX.
These should only be accessed via REXEC commands.

The data flow for the example is shown below:
\setlength{\unitlength}{1mm}
\begin{center}
\begin{picture}(150,90)
\thicklines
\put (0,85){\framebox(30,5){TEST1.DAT}}
\put (40,85){\framebox(30,5){TEST1DD.DAT}}
\put (80,85){\framebox(30,5){TEST2DD.DAT}}
\put (120,85){\framebox(30,5){TEST2.DAT}}
\put (40,75){\framebox(30,5){TEST1}}
\put (80,75){\framebox(30,5){TEST2}}
\put (40,60){\framebox(30,5){TEST1}}
\put (80,60){\framebox(30,5){TEST2}}
\put (40,50){\framebox(30,5){JUNK}}
\put (80,50){\framebox(30,5){=}}
\put (60,35){\framebox(30,5){TEST}}
\put (40,20){\framebox(30,5){=}}
\put (80,20){\framebox(30,5){=}}
\put (40,10){\framebox(30,5){=}}
\put (80,10){\framebox(30,5){TESTCB}}
\put (80,0){\framebox(30,5){RESULT}}
\put (55,85){\vector(0,-1){5}}
\put (95,85){\vector(0,-1){5}}
\put (55,60){\vector(0,-1){5}}
\put (95,60){\vector(0,-1){5}}
\put (75,45){\vector(0,-1){5}}
\put (55,30){\vector(0,-1){5}}
\put (95,30){\vector(0,-1){5}}
\put (55,20){\vector(0,-1){5}}
\put (95,20){\vector(0,-1){5}}
\put (95,10){\vector(0,-1){5}}
\put (55,75){\vector(0,-1){10}}
\put (95,75){\vector(0,-1){10}}
\put (15,85){\line(0,-1){15}}
\put (135,85){\line(0,-1){15}}
\put (15,70){\line(1,0){40}}
\put (95,70){\line(1,0){40}}
\put (55,50){\line(0,-1){5}}
\put (95,50){\line(0,-1){5}}
\put (55,45){\line(1,0){40}}
\put (75,35){\line(0,-1){5}}
\put (55,30){\line(1,0){40}}
\put (140,80){{\bf DEFINE}}
\put (120,66.5){{\bf LOAD}}
\put (120,56.5){{\bf SORT}}
\put (120,43.5){{\bf JOIN}}
\put (120,28.5){{\bf PROJECT}}
\put (120,16.5){{\bf SORT}}
\put (120,6.5){{\bf REPORT}}
\put (10,28.5){{\bf SELECT}}
\put (10,16.5){{\bf SORT}}
\end{picture}
\end{center}
The names in boxes are file names and the branch names in bold-face are command
names.
File type is .REX unless shown otherwise.

To get the raw data into an R-file you must first create the file using the
{\bf DEFINE} command and the appropriate data definition file:
\begin{verbatim}
        $ REXEC DEFINE TEST1DD TEST1
\end{verbatim}
This command will create the R-file TEST1.REX.
The first field `REXEC' is a predefined symbol which causes a command procedure
to be executed which implements the command interface of REXEC.
In the guide the commands are shown starting with the symbol `R'.
This was considered too short for Starlink purposes and so has been changed
to `REXEC'.
{\em You must use `REXEC' instead of `R' to invoke REXEC commands.}

You can examine the field definitions in the created file by:
\begin{verbatim}
        $ REXEC SHOW TEST1
\end{verbatim}
At this stage, file TEST1.REX does not contain any data; only field definitions.
Data is loaded into the file by:
\begin{verbatim}
        $ REXEC LOAD TEST1 TEST1
\end{verbatim}
This adds the data stored in TEST1.DAT to the R-file TEST1.REX.
You can now examine this data on your terminal by:
\begin{verbatim}
        $ REXEC LIST TEST1
\end{verbatim}
File TEST2.DAT contains some further information about the ships specified in
TEST1.DAT.
We want to join these two sets of data together.
Before we can do this, we must store the new data in an R-file.
We use the same method that we used for TEST1:
\begin{verbatim}
        $ REXEC DEFINE TEST2DD TEST2
        $ REXEC LOAD   TEST2   TEST2
        $ REXEC LIST   TEST2
\end{verbatim}
Both sets of data are now in the appropriate format.
However, before we can join them we must make sure they are both sorted on
the same key field.
We will sort them on the NAME field (the ship name):
\begin{verbatim}
        $ REXEC SORT TEST1 JUNK (NAME)
        $ REXEC SORT TEST2 =    (NAME)
\end{verbatim}
We store the output in two temporary files.
The first one is JUNK.REX.
The second one is the {\em current temporary file} specified by the symbol `='.
As there is only one current temporary file, we had to call the first temporary
file by a specific name (JUNK).
Now we can join the data together:
\begin{verbatim}
        $ REXEC JOIN JUNK = TEST
\end{verbatim}
The output is stored in the R-file TEST.REX.
You can examine the field definitions in the new file by:
\begin{verbatim}
        $ REXEC SHOW TEST
\end{verbatim}
and you can examine the actual data by:
\begin{verbatim}
        $ REXEC LIST TEST
\end{verbatim}
Now we have a complete set of raw data in one R-file we can perform some
relational operations on it.
A typical useful operation is to select records which satisfy some criterion.
Thus, let us select those ships with a displacement $>$50000 tons
and display them in order of displacement:
This can be done as follows:
\begin{verbatim}
        $ REXEC SELECT TEST = (DISP>50000)
        $ REXEC SORT   =    = (DISP)
        $ REXEC LIST   =
\end{verbatim}
DISP is the name of the field holding the displacement.
Notice how we have used the current temporary file feature to avoid having to
think up names for the temporary files involved.

Selecting records on fields with character values is more tricky than doing it
on numeric fields.
This is because REXEC delimits character strings with apostrophes and the DCL
interpreter deals with the apostrophe character in a special way.
The safest thing is to let the program prompt you for character strings with
the prompt `R$>$'.
Thus, to list all the passenger ships in the database, do this:
\begin{verbatim}
        REXEC SELECT TEST (
        R>TYPE='P')
\end{verbatim}
(the reply to TYPE is case sensitive, you must use upper case in this instance)
and to list the details of the ship {\em Des Moines,} do this:
\begin{verbatim}
        REXEC SELECT TEST (
        R>NAME='Des Moines')
\end{verbatim}
Another common requirement is to calculate for every record the value of some
parameter which is a function of the record's field values.
Let us calculate values for the following two functions:
\begin{verbatim}
        Cb = 0.9912*DISP/(L*B*D)
        LB = L/B
\end{verbatim}
Cb is the Block Coefficient of the hull calculated from the given values of
Displacement, Length, Beam, and Draught and LB is the Length to Beam ratio.
The appropriate command is {\bf PROJECT:}
\begin{verbatim}
        $ REXEC PROJECT TEST = (NAME (0.9912*DISP/(L*B*D)):CB (L/B):LB)
\end{verbatim}
The parameters enclosed in parentheses are interpreted as follows.
The two blanks separate the specification of the three fields that we want to
create in the output file.
The first is simply a copy of the NAME field.
The second is a new field called CB which is defined in terms of the original
fields by the expression (0.9912*DISP/(L*B*D)).
The third is a new field called LB which is defined in terms of the original
fields by the expression (L/B).
DISP, L, B and D are existing field names.

Finally, let us sort the new file in order of Block Coefficient and display
a report containing the extracted data:
\begin{verbatim}
        $ REXEC SORT    =      TESTCB (CB)
        $ REXEC REPORT  TESTCB        (Average CB LB Error Recordnumbers)
\end{verbatim}
The codes enclosed in parentheses have the following meaning.
`Average CB LB' means display the average value for fields CB and LB; `Error'
means include the standard deviation; `Recordnumbers' means number the records.
You can abbreviate these options to `(A CB LB E R)'.
If you want to store the output permanently in a file, specify a file name
following `TESTCB' in the last command.
This leads to a problem because REXEC will not accept more than eight fields
in a command line.
This is due to a restriction in the number of parameters allowed in a DCL
command procedure.
Once again, the way to overcome it is to enter `(' as the option field and
allow the program to prompt you for the rest of the option list.
Thus, to get a report written to file RESULT.LIS, enter the command:
\begin{verbatim}
        $ REXEC REPORT TESTCB RESULT (
        R>A CB LB E R)
\end{verbatim}
\section {SCAR Example}
This example illustrates how to set up and use a simple database using SCAR.
It does the same processing as the REXEC example so you can compare how REXEC
and SCAR do the same job.
The following SCAR commands are used (the `CAR\_' prefix is omitted):
\begin{tabbing}
XXXXXXXX\=Database functions:XXX\=\kill
\>\+Database functions:\>{\bf CONVERT}, {\bf JOIN}, {\bf SEARCH}, {\bf SORT}\\
Display data:\>{\bf ANALYSE}, {\bf CALC}, {\bf REPORT}
\end{tabbing}
There are many other ways of performing the processing done in this example.
The method used here is not claimed to be the best.

To run the example, copy the following files from REXECDIR into your own
directory and proceed as described below:
\begin{tabbing}
XXXXXXXX\=Database functions:XXX\=\kill
\>\+T1F.DAT\>Ship data - set 1\\
T2F.DAT\>Ship data - set 2\\
DSCFT1F.DAT\>Description file for T1F.DAT\\
DSCFT2F.DAT\>Description file for T2F.DAT\\
SCARDEMO.COM\>Command procedure to run example
\end{tabbing}
Files T1F.DAT and T2F.DAT were prepared using an editor (EDT).
The DSCF files are best prepared using the {\bf FORM} command.
The data flow for the example is show below:
\begin{center}
\begin{picture}(120,100)
\thicklines
\put (20,95){\framebox(30,5){T1F}}
\put (70,95){\framebox(30,5){T2F}}
\put (20,85){\framebox(30,5){T1}}
\put (70,85){\framebox(30,5){T2}}
\put (20,75){\framebox(30,5){T1SS}}
\put (70,75){\framebox(30,5){T2SS}}
\put (45,60){\framebox(30,5){TEST}}
\put (45,50){\framebox(30,5){TESTC}}
\put (20,35){\framebox(30,5){TDP0}}
\put (70,35){\framebox(30,5){TCB0}}
\put (20,25){\framebox(30,5){TDP1}}
\put (70,25){\framebox(30,5){TCB1}}
\put (20,15){\framebox(30,5){TDP2}}
\put (70,15){\framebox(30,5){TCB2}}
\put (70,5){\framebox(30,5){TCB3}}
\put (35,95){\vector(0,-1){5}}
\put (85,95){\vector(0,-1){5}}
\put (35,85){\vector(0,-1){5}}
\put (85,85){\vector(0,-1){5}}
\put (60,70){\vector(0,-1){5}}
\put (60,60){\vector(0,-1){5}}
\put (35,45){\vector(0,-1){5}}
\put (85,45){\vector(0,-1){5}}
\put (35,35){\vector(0,-1){5}}
\put (85,35){\vector(0,-1){5}}
\put (35,25){\vector(0,-1){5}}
\put (85,25){\vector(0,-1){5}}
\put (85,15){\vector(0,-1){5}}
\put (85,5){\vector(0,-1){5}}
\put (35,75){\line(0,-1){5}}
\put (85,75){\line(0,-1){5}}
\put (35,70){\line(1,0){50}}
\put (35,45){\line(1,0){50}}
\put (60,50){\line(0,-1){5}}
\put (110,91.5){{\bf CONVERT}}
\put (110,81.5){{\bf SORT}}
\put (110,68.5){{\bf JOIN}}
\put (85,56.5){{\bf CONVERT}}
\put (110,41.5){{\bf CALC} (BLOCK)}
\put (110,31.5){{\bf CALC} (CB)}
\put (110,21.5){{\bf SORT}}
\put (110,11.5){{\bf CONVERT}}
\put (110,1.5){{\bf ANALYSE}}
\put (0,41.5){{\bf SEARCH}}
\put (0,31.5){{\bf CONVERT}}
\put (0,21.5){{\bf SORT}}
\end{picture}
\end{center}
The names in boxes are file names and the branch names in bold-face are command
names.

First, prepare SCAR for use:
\begin{verbatim}
        $ START_CAR
\end{verbatim}
and get the `\$' prompt back again:
\begin{verbatim}
        SCAR 3.1> SET PROMPT
\end{verbatim}
You can look at the contents of a file at any time by using {\bf REPORT},
thus to look at the contents of T1F:
\begin{verbatim}
        $ CAR_REPORT
        INPUT:=T1F
        MODE:=2
        SELECT?:=N
        HEADER?:=N
\end{verbatim}
Reply `+' to the prompt `Scroll:=' to get more output; `E' to quit.
Use Mode 1 to get printed output (stored in file T1F.CAR).
You can specify the parameter values on the command line:
\begin{verbatim}
        $ CAR_REPORT/INPUT=T1F/MODE=2/SELECT?=N/HEADER?=N
\end{verbatim}
However, it is easier and clearer to wait for the prompts and this is how the
example is shown.

T1F and T2F contain different data for the same ships.
The first job is to join these files and store the data in a single file.
However, we must first convert the data from formatted to unformatted form
since {\bf JOIN} only works on unformatted files:
\begin{verbatim}
        $ CAR_CONVERT
        INPUT:=T1F
        OUTPUT:=T1
        SIMPLE?:=Y
        ASCII?:=N
        INDEX?:=N
        SELECT?:=N
                <4 I lines output>
\end{verbatim}
The $<$4 I lines output$>$ means that four information lines will be output by
the program.
These are self explanatory and will not be displayed in detail in this paper.
The $<>$ syntax is used freely below to indicate where information lines are
output.

SCAR will always write a descriptor file for a specified output file if a such
a descriptor file does not exist.
Thus, in this case the descriptor file DSCFT1.DAT is automatically written.

Files to be joined should be sorted on the same key field.
T1F and T2F are arranged in different order, so T1 and T2 must be sorted on key
field NAME:
\begin{verbatim}
        $ CAR_SORT
        INPUT:=T1
        SIMPLE?:=Y
        OUTPUT?:=T1SS
        KEYS:=NAME
        ASCEND:=Y
        TAG?:=N
                <3 I lines>
\end{verbatim}
A four character output file name is chosen because later on we will identify
fields in such a way that the first four characters of this name are used in the
field name.

Now do the same thing to T2F as was done to T1F:
\begin{verbatim}
        $ CAR_CONVERT        (T2F -> T2)
        $ CAR_SORT           (T2 -> T2SS)
\end{verbatim}
The replies to the prompts are obvious modifications of those shown above.

We are now able to join files T1SS and T2SS to form file TEST containing
all the data in T1F and T2F.
\begin{verbatim}
        $ CAR_JOIN
        INPUT:=T1SS,T2SS
        OUTPUT:=TEST
        SIMPLE?:=Y
        MATCH:=T1SS__NAME.EQ.T2SS__NAME.END.
        OK?:=Y
              <8 I lines>
\end{verbatim}
File TEST is in indexed form and in order to be able to use {\bf SEARCH}
and {\bf CALC} on it we must convert it into a form which contains all the
required fields.
The fields obtained from T1SS will have names formed by prefixing the
original names by `T1SS\_\_' and similarly for T2SS.
Thus the displacement from T1SS will have field name T1SS\_\_DISP and the length
from T2SS will have field name T2SS\_\_L.
We only want one of the NAME fields since they will have the same value in any
one record.
\begin{verbatim}
        $ CAR_CONVERT
        INPUT:=TEST
        OUTPUT:=TESTC
        SIMPLE?:=Y
        ASCII?:=N
        INDEX?:=N
        TITLE:=Full Ship Data
        SELECT?:=Y
                <answer Y to each prompt except on second occasion that
                NAME field is prompted for (8th prompt)>
\end{verbatim}
The next section extracts records for ships with a displacement $>$50000 tons and
prepares a file containing a list of such ships in displacement order.
Select records for ships with displacement $>$50000 tons:
\begin{verbatim}
        $ CAR_SEARCH
        INPUT:=TESTC
        OUTPUT:=TDP0
        SIMPLE?:=Y
        I : KEYFIELD          :NAME
        I : Do you have a RANGE         ?
        ANSWER?:=N
        I : Do you have a QUERY         ?
        ANSWER?:=Y
        QUERY:=T1SS__DISP.GT.50000.END.
        OK?:=Y
                <5 I lines output>
\end{verbatim}
We cannot sort the TDP0 file as it stands since it is an index and {\bf SORT}
cannot find field T1SS\_\_DISP.
We must first convert it to a form which explicitly contains all the fields.
\begin{verbatim}
        $ CAR_CONVERT
        INPUT:=TDP0
        OUTPUT:=TDP1
        SIMPLE?:=Y
        ASCII?:=N
        INDEX?:=N
        TITLE:=Ships with displacement > 50000 tons
        SELECT?:=N
                <4 I lines>
\end{verbatim}
Now sort in order of displacement:
\begin{verbatim}
        $ CAR_SORT
        INPUT:=TDP1
        SIMPLE?:=Y
        OUTPUT:=TDP2
        KEYS:=T1SS__DISP
        ASCEND:=Y
        TAG?:=N
                <3 I lines>
\end{verbatim}
We can now get a list of ships with displacement $>$50000 tons in displacement
order.
\begin{verbatim}
        $ CAR_REPORT
        INPUT:=TDP2
        MODE:=1
                <2 I lines>
        SELECT?:=Y
        HEADER?:=N
                <Select NAME and T1SS__DISP fields>
\end{verbatim}
The output is stored in file TDP2.CAR which may be printed.

We can use {\bf SEARCH} to extract all the passenger ship records by
specifying the query:
\begin{verbatim}
        QUERY:=T1SS__TYPE.EQ."P".END.
\end{verbatim}
However, {\bf SEARCH} is completely defeated by any request for the record
of a ship specified by name, unless that name contains no lower case characters.
Thus, the query:
\begin{verbatim}
        QUERY:=NAME.EQ."Des Moines".END.
\end{verbatim}
fails because SCAR converts the input string `Des Moines' to the string
`DES MOINES' and the comparison with the name string `Des Moines' in the
database fails.
However, a query for ship `QE2' will succeed.
I can't see a way of getting around it.
This problem makes the current version of SCAR unusable for serious
administrative applications.

Now calculate the Block Coefficient (Cb) and Length/Beam (LB) ratio defined
above.
SCAR immediately gives us another problem because it will not accept parentheses
in formulas!
We must therefore make the calculation of Cb in 2 stages:
\begin{verbatim}
        BLOCK = L * B * D
        Cb    = 0.9912 * DISP / BLOCK
\end{verbatim}
First calculate the value of BLOCK for every record in TESTC.
\begin{verbatim}
        $ CAR_CALC
        INPUT:=TESTC
        OUTPUT=TCB0
        KEYWORD:=F
        NAME:=BLOCK
        DEFINE:=T2SS__L*T2SS__B*T2SS__D
        UNIT:=M**3
        KEYWORD=E
                <A report is displayed on the screen>
        Scroll:=+
\end{verbatim}
Now calculate CB.
\begin{verbatim}
        $ CAR_CALC
        INPUT:=TCB0
        OUTPUT:=TCB1
        KEYWORD:=F
        NAME:=CB
        DEFINE:=0.9912*T1SS__DISP/BLOCK
        KEYWORD:=F
        NAME:=LB
        DEFINE:=T2SS__L/T2SS__B
        KEYWORD:=E
                <A report is displayed on the screen>
        Scroll:=+
\end{verbatim}
Now sort the records into CB order.
This works with TCB1 because it contains the CB field although other fields are
held in TESTC.
\begin{verbatim}
        $ CAR_SORT
        INPUT:=TCB1
        SIMPLE?:=Y
        OUTPUT:=TCB2
        KEYS:=CB
        ASCEND:=Y
        TAG?:=N
                <3 I lines>
\end{verbatim}
TCB2 still only contains a small number of fields and points to TESTC for the
rest.
It is convenient to produce a file (TCB3) containing all the fields explicitly.
The fields POINTER, NUMBER and BLOCK are of no interest and should be discarded.
\begin{verbatim}
        $ CAR_CONVERT
        INPUT:=TCB2
        OUTPUT:=TCB3
        SIMPLE?:=Y
        ASCII?:=N
        INDEX?:=N
        TITLE:=Augmented Ship Data
        SELECT?:=Y
                <omit POINTER, NUMBER and BLOCK>
\end{verbatim}
A full report of the joined, augmented and sorted ship data can now be
produced and printed.
\begin{verbatim}
        $ CAR_REPORT
        INPUT:=TCB3
        MODE:=1
                <2 I lines>
        SELECT?:=N
        HEADER?:=N
                <the report appears a record at a time. Answer "+"
                 to the Scroll:= prompt until all data have been
                 processed, then answer "e">
\end{verbatim}
The full report is available for printing in file TCB3.CAR.
The required statistical analysis can be obtained using command {\bf ANALYSE}
and this can be displayed graphically.
The input below shows a typical process:
\begin{verbatim}
        $ CAR_ANALYSE
                <3 I lines>
        STYLE:=1                        (Histogram)
        INPUT:=TCB3
                <3 I lines>
        PTYPE:=1                        (Bar chart)
                <1 I line>
        XAXIS:=CB
        TITLE:=Block Coefficient
                <5 I lines>
                <report on statistical analysis of CB appears on terminal>
        Scroll:=E
        I : Enter bin size
        BIN_SZ:=0.05
        I : Xaxis lower limit
        X_LOW:=0.45
        I : No. of bins required
        NBINS:=10
        I : largest bin pop: 5
        I : Yaxis lower limit
        Y_LOW:=0
        I : Yaxis upper limit
        Y_UPP:=6
        I : Okay to continue?
        OKAY?:=Y
        I : Enter device
                <possible devices displayed>
        DEVICE:=<choose your own>
                <7 I lines>
        PLOTTER:=PRINTRONIX
                <6 I lines>
        SIZE:=4                                        (A4 size)
                <a graph should appear on your terminal>
        STORE?:=Y
        MORE?=N
                <1 I line>
        $ PRINT/PASSALL PRINTRONIX.BIT
\end{verbatim}
The process described above shows what happens when you start from just the
original five files supplied.
If you tried to carry out the process again you would find that SCAR reacted
differently.
This is because of the existence of the DSCF files created during the initial
process.
When an output file is specified, SCAR looks for an associated descriptor file
and if found gets information from it.
This modifies the prompts.
In particular, {\bf CONVERT}, {\bf JOIN} and {\bf SEARCH} only prompt for
INPUT and OUTPUT.

The commands for this demo are stored in file REXECDIR:SCARDEMO.COM.
When executed, this requires that some parameter values be supplied from the
keyboard.
In particular, I can't see how to specify the formula
`0.9912*T1SS\_\_DISP/BLOCK' as a parameter because the `/' is interpreted as a
parameter separator.
\section {A comparison of REXEC and SCAR}
REXEC and SCAR have a lot of similarities in design and functionality.
However, SCAR was designed principally for processing existing astronomical
catalogues while REXEC was designed as a general purpose system.
SCAR is four times larger than REXEC (7100 blocks to 1800 blocks) and in general
SCAR is more powerful and has more facilities.
I personally found SCAR more difficult than REXEC to start using successfully
as it is more complex, more unforgiving and has a tougher user manual to crack.
You will see from the examples that SCAR makes heavier weather of the job than
REXEC and the output is less concise, but you do get a graph.

In REXEC both the text files and the R-files are sequential but the text files
have variable length records while the R-files have fixed length (1024 byte)
records.
I/O for R-files is via Fortran binary direct access statements.
In SCAR the formatted files are sequential with variable length records while
the unformatted files are relative with fixed length records.
SCAR also has index files while REXEC does not.
REXEC only processes R-files while SCAR can do some operations on both
formatted and unformatted files.

REXEC takes the text data description from a separate file to the text data
and stores both sets of information in a single R-file.
SCAR stores the data description in a separate file for both formatted and
unformatted versions and it has naming rules for them.
Only SCAR can edit data in the form in which it is usually processed.
With REXEC you have to edit the text form and them convert this to an R-file.
Only SCAR has a plotting package --- this is a major advantage over REXEC ---
however, an REXEC plotting package is expected.
Both have a HELP system.

SCAR is unable to select records based on keys which are character values
containing lower case letters.
This is a serious defect for administrative type databases.
SCAR doesn't have the temporary file feature of REXEC which makes SCAR more
painful to use.

The command formats differ.
The REXEC command is of the form:
\begin{verbatim}
        $ REXEC command infile(s) outfile (options)
\end{verbatim}
while SCAR has a symbol defined for each command and its parameters use the
INTERIM environment form of specification:
\begin{verbatim}
        $ command/par1=value/par2=value...
\end{verbatim}
The table below shows which commands perform similar functions in REXEC and
SCAR (the `CAR\_' prefix has been omitted).
Specific details can be found in the associated documentation.
\begin{tabbing}
XXXXX\=Database functions:XXX\=JOIN/DIFFERXXXXXXXXXXXXX\=\kill
\>\+{\bf Functional area}\>{\bf SCAR}\>{\bf REXEC}\\
\\
Database functions:\\
\>SEARCH\>SELECT/RANGE\\
\>CONVERT\>PROJECT\\
\>SORT\>SORT\\
\>MERGE\>MERGE\\
\>JOIN/DIFFER\>JOIN\\
\\
Display data:\\
\>REPORT\>REPORT/LIST\\
\>CALCULATE\>PROJECT\\
\>ANALYSE\>PLOT\\
\>PLOT\>-\\
\>-\>GROUP\\
\\
Data definitions:\\
\>FORM\>-\\
\>BINARY/ASCII\>-
\end{tabbing}
The SCAR {\bf PLOT} command is really a version of the CHART program and is not
comparable to the REXEC {\bf PLOT} command (not yet implemented in the VAX/VMS
version).
Both items have a number of commands which are really tools to help the user
perform utility functions and these are not comparable.
They are:
\begin{tabbing}
XXXXXDatabase functions:XXX\=JOIN/DIFFERXXXXXXXXXXXXX\=\kill
\>\+START\_CAR\>DEFINE/LOAD\\
HARDCOPY/WRAP\>COPY/RENAME\\
UPDATE/HELP\>SHOW\\
EXTAPE/MOUNT/DISMOUNT\>UNDO
\end{tabbing}
\section {REFERENCES}
B J Read, `R-EXEC Guide and Reference'; RAL/CCD Division.\\
B J Read, `A Relational Data Handling System for Scientists'; RAL Report
RAL-86-053.\\
B J Read, `Scientific Data Manipulation in a Relational Database System';
RAL Report RAL-86-084.\\
J H Fairclough, `SCAR - Starlink Catalogue Access and Reporting'; SUN/70.
\end{document}
