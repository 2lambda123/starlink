\documentstyle{article}
\pagestyle{myheadings}

%------------------------------------------------------------------------------
\newcommand{\stardoccategory}  {Starlink User Note}
\newcommand{\stardocinitials}  {SUN}
\newcommand{\stardocnumber}    {21.6}
\newcommand{\stardocauthors}   {M D Lawden}
\newcommand{\stardocdate}      {10 May 1989}
\newcommand{\stardoctitle}     {TAPEIO --- Magnetic Tape Routines}
%------------------------------------------------------------------------------

\newcommand{\stardocname}{\stardocinitials /\stardocnumber}
\markright{\stardocname}
\setlength{\textwidth}{160mm}
\setlength{\textheight}{240mm}
\setlength{\topmargin}{-5mm}
\setlength{\oddsidemargin}{0mm}
\setlength{\evensidemargin}{0mm}
\setlength{\parindent}{0mm}
\setlength{\parskip}{\medskipamount}
\setlength{\unitlength}{1mm}

\begin{document}
\thispagestyle{empty}
SCIENCE \& ENGINEERING RESEARCH COUNCIL \hfill \stardocname\\
RUTHERFORD APPLETON LABORATORY\\
{\large\bf Starlink Project\\}
{\large\bf \stardoccategory\ \stardocnumber}
\begin{flushright}
\stardocauthors\\
\stardocdate
\end{flushright}
\vspace{-4mm}
\rule{\textwidth}{0.5mm}
\vspace{5mm}
\begin{center}
{\Large\bf \stardoctitle}
\end{center}
\vspace{5mm}

\section {INTRODUCTION}

This  note  describes  a  set  of  routines  for  performing  physical  I/O
operations on non-VAX-standard magnetic tapes.
The following routines are available:
\begin{description}
\begin{description}
\item[TIO\_OPEN]: assign an I/O channel to a device.
\item[TIO\_SETDEN]: set the recording density for a tape.
\item[TIO\_MOUNT]: mount a tape on a device.
\item[TIO\_DISMT]: dismount a tape from a device.
\item[TIO\_READ]: read a physical block.
\item[TIO\_READX]: read a physical block, optionally suppressing retry.
\item[TIO\_WRITE]: write a physical block.
\item[TIO\_MARK]: write an end-of-file mark.
\item[TIO\_SKIP]: skip forward or backward a given number of tape marks.
\item[TIO\_MOVE]: skip forward or backward a given number of blocks.
\item[TIO\_REWIND]: reposition a tape to the BOT marker.
\item[TIO\_CLOSE]: deassign an I/O channel from a device.
\end{description}
\end{description}
All the routines return a status value indicating the success of the operation.
Currently, this value is the VAX/VMS System Service return status [1], although
this may not be the case in future versions.
Therefore, to facilitate VMS-independent validation of the status value, the
following LOGICAL functions are provided:
\begin{description}
\begin{description}
\item[TIO\_EOF]: test for a physical end-of-file condition.
\item[TIO\_ERR]: test for an error condition.
\end{description}
\end{description}
A further routine is provided that returns the message text associated with a
particular status value:
\begin{description}
\begin{description}
\item[TIO\_GETMSG]: return message text.
\end{description}
\end{description}
Finally, the following LOGICAL function is also provided :
\begin{description}
\begin{description}
\item[TIO\_SENSE]: get information about a tape drive.
\end{description}
\end{description}
Before a process can perform any actual I/O operations on a tape device, an I/O
channel must be assigned using {\bf TIO\_OPEN}.

\section {LINKING}

To use these routines you must specify the object module library
TAPEIO\_DIR:TAPEIO.OLB in the LINK command, eg.
\begin{verbatim}
        $ LINK TAPECOPY,TAPEIO_DIR:TAPEIO/LIB
\end{verbatim}
\section {TAPE MOUNTING}
All tapes to be used by a process must first be mounted using the `FOREIGN'
qualifier, eg.
\begin{verbatim}
        $ ALLOC MTA0: INPUT
        $ ALLOC MTA1: OUTPUT
        $ MOUNT/FOR   INPUT
        $ MOUNT/FOR   OUTPUT
\end{verbatim}
\section {REFERENCES}
\begin{tabbing}
xx\=[2]xx\=\kill
\>[1]\>`VAX/VMS System Services Reference Manual', DEC\\
\>[2]\>`VAX FORTRAN User's Guide', DEC
\end{tabbing}
\appendix
\section {ROUTINE SPECIFICATIONS}
\goodbreak
\rule{\textwidth}{0.3mm}
{\Large {\bf TIO\_CLOSE} \hfill Close Tape Drive \hfill {\bf TIO\_CLOSE}}
\begin{description}
\item [FUNCTION]:
De-assign an I/O channel from a magnetic tape device.
\item [CALL]:
\begin{quote}
{\tt CALL TIO\_CLOSE(iochan,status)}
\end{quote}
\item [INPUT ARGUMENT]:
\begin{tabbing}
descrxxx\=CHARACTERx\=expressionxxx\=\kill
{\em iochan}\>INTEGER\>expression\>I/O channel assigned to device.
\end{tabbing}
\item [OUTPUT ARGUMENT]:
\begin{tabbing}
descrxxx\=CHARACTERx\=expressionxxx\=\kill
{\em status}\>INTEGER\>variable\>Status return value.
\end{tabbing}
\item [NOTES]:
No further I/O operations can be performed on the device unless {\bf TIO\_OPEN}
is called again.
\end{description}
\goodbreak
\rule{\textwidth}{0.3mm}
{\Large {\bf TIO\_DISMT} \hfill Dismount Tape \hfill {\bf TIO\_DISMT}}
\begin{description}
\item [FUNCTION]:
Dismount a tape from a magnetic tape device, optionally unloading it.
\item [CALL]:
\begin{quote}
{\tt CALL TIO\_DISMT(device,unload,status)}
\end{quote}
\item [INPUT ARGUMENTS]:
\begin{tabbing}
descrxxx\=CHARACTERx\=expressionxxx\=\kill
{\em device}\>CHARACTER\>expression\>Name of device to be dismounted (may
be logical).\\
{\em unload}\>LOGICAL\>expression\>True if tape is to be unloaded as well as
dismounted.
\end{tabbing}
\item [OUTPUT ARGUMENT]:
\begin{tabbing}
descrxxx\=CHARACTERx\=expressionxxx\=\kill
{\em status}\>INTEGER\>variable\>Status return value.
\end{tabbing}
\item [NOTES]:
Any channels open on the tape must have been deassigned --- i.e.\ if the tape
has been opened with {\bf TIO\_OPEN}, it must be closed with {\bf TIO\_CLOSE}
before this routine is called.
\end{description}
\goodbreak
\rule{\textwidth}{0.3mm}
{\Large {\bf TIO\_EOF} \hfill Test for EOF \hfill {\bf TIO\_EOF}}
\begin{description}
\item [FUNCTION]:
Test for end-of-file condition.
\item [CALL]: (LOGICAL function)
\begin{quote}
{\tt TIO\_EOF(status)}
\end{quote}
\item [INPUT ARGUMENT]:
\begin{tabbing}
descrxxx\=CHARACTERx\=expressionxxx\=\kill
{\em status}\>INTEGER\>expression\>Status value.
\end{tabbing}
\item [OUTPUT]:
\begin{tabbing}
descrxxx\=CHARACTERx\=expressionxxx\=\kill
{\em tio\_eof}\>LOGICAL\>value\>Set to TRUE if status indicates an end-of-file condition.
\end{tabbing}
\item [NOTES]:
Under VMS this is accomplished by comparing the value with the appropriate
System Service return status (SS\$\_ENDOFFILE).
An end-of-file condition is not regarded as a successful completion of an I/O
operation and is treated as an error by {\bf TIO\_ERR}.
Programs should test for this condition prior to any error checking.
For example:
\begin{verbatim}
    IF (TIO_EOF(STATUS)) THEN
      ...
    ELSEIF (TIO_ERR(STATUS)) THEN
      ...
    ENDIF
\end{verbatim}
\end{description}
\goodbreak
\rule{\textwidth}{0.3mm}
{\Large {\bf TIO\_ERR} \hfill Test for Error \hfill {\bf TIO\_ERR}}
\begin{description}
\item [FUNCTION]:
Test for error conditions.
\item [CALL]: (LOGICAL function)
\begin{quote}
{\tt TIO\_ERR(status)}
\end{quote}
\item [INPUT ARGUMENT]:
\begin{tabbing}
descrxxx\=CHARACTERx\=expressionxxx\=\kill
{\em status}\>INTEGER\>expression\>Status value.
\end{tabbing}
\item [OUTPUT]:
\begin{tabbing}
descrxxx\=CHARACTERx\=expressionxxx\=\kill
{\em tio\_err}\>LOGICAL\>value\>Set to TRUE if status indicates an error condition.
\end{tabbing}
\item [NOTES]:
Under VMS this is accomplished by examining the low-order bit of the status
value.
If the bit is set, the operation completed successfully and {\bf TIO\_ERR} is
set to FALSE.
Non-successful completion is indicated by this bit being clear and
{\bf TIO\_ERR} is set to TRUE.
The actual cause of the error can be determined by calling {\bf TIO\_GETMSG}.
\end{description}
\goodbreak
\rule{\textwidth}{0.3mm}
{\Large {\bf TIO\_GETMSG} \hfill Get Error Message \hfill {\bf TIO\_GETMSG}}
\begin{description}
\item [FUNCTION]:
Return the message text associated with a given status value.
\item [CALL]:
\begin{quote}
{\tt CALL TIO\_GETMSG(status,msgbuf,msglen)}
\end{quote}
\item [INPUT ARGUMENT]:
\begin{tabbing}
descrxxx\=CHARACTERx\=expressionxxx\=\kill
{\em status}\>INTEGER\>expression\>Status value.
\end{tabbing}
\item [OUTPUT ARGUMENTS]:
\begin{tabbing}
descrxxx\=CHARACTERx\=expressionxxx\=\kill
{\em msgbuf}\>CHARACTER\>variable\>Buffer to receive message text.\\
{\em msglen}\>INTEGER\>variable\>Length of message text.
\end{tabbing}
\item [NOTES]:
Currently, the text is obtained from the VMS system message file although this
may not be the case in future versions.
This routine will generally be used in conjunction with {\bf TIO\_ERR}.
For example:
\begin{verbatim}
    CHARACTER*72 MSG
      ...
    CALL TIO_OPEN('TAPE',IOCHAN,STATUS)
    IF (TIO_ERR(STATUS)) THEN
       CALL TIO_GETMSG(STATUS,MSG,LEN)
       WRITE (*,'(X,A)') MSG(:LEN)
    ENDIF
\end{verbatim}
If the user-specified buffer is too small to accommodate the complete message
text, the string will be truncated from the right.
\end{description}
\goodbreak
\rule{\textwidth}{0.3mm}
{\Large {\bf TIO\_MARK} \hfill Write EOF \hfill {\bf TIO\_MARK}}
\begin{description}
\item [FUNCTION]:
Write an end-of-file mark.
\item [CALL]:
\begin{quote}
{\tt CALL TIO\_MARK(iochan,status)}
\end{quote}
\item [INPUT ARGUMENT]:
\begin{tabbing}
descrxxx\=CHARACTERx\=expressionxxx\=\kill
{\em iochan}\>INTEGER\>expression\>I/O channel assigned to device.
\end{tabbing}
\item [OUTPUT ARGUMENT]:
\begin{tabbing}
descrxxx\=CHARACTERx\=expressionxxx\=\kill
{\em status}\>INTEGER\>variable\>Status return value.
\end{tabbing}
\item [NOTES]:
An error condition occurs if the tape is currently positioned past the
end-of-tape marker, or if the EOT region is entered as a result of the
operation.
\end{description}
\goodbreak
\rule{\textwidth}{0.3mm}
{\Large {\bf TIO\_MOUNT} \hfill Mount Tape \hfill {\bf TIO\_MOUNT}}
\begin{description}
\item [FUNCTION]:
Attempt to mount a tape on a named drive, ready for use by other TIO\_ routines.
\item [CALL]:
\begin{quote}
{\tt CALL TIO\_MOUNT(device,status)}
\end{quote}
\item [INPUT ARGUMENT]:
\begin{tabbing}
descrxxx\=CHARACTERx\=expressionxxx\=\kill
{\em device}\>CHARACTER\>expression\>Name of the drive to be used (can be
logical).
\end{tabbing}
\item [OUTPUT ARGUMENT]:
\begin{tabbing}
descrxxx\=CHARACTERx\=expressionxxx\=\kill
{\em status}\>INTEGER\>variable\>Status return value.
\end{tabbing}
\item [NOTES]:
If the drive is not already allocated to the user, this routine will try to
allocate it first.
The tape is mounted `FOREIGN' and `NOASSIST'.
Following execution of this routine, the position of a tape that was already
mounted will be unchanged; a tape that actually had to be mounted will be at
the load point.
\end{description}
\goodbreak
\rule{\textwidth}{0.3mm}
{\Large {\bf TIO\_MOVE} \hfill Skip Tape Blocks \hfill {\bf TIO\_MOVE}}
\begin{description}
\item [FUNCTION]:
Skip forward or backward past a specified number of physical blocks.
\item [CALL]:
\begin{quote}
{\tt CALL TIO\_MOVE(iochan,npb,status)}
\end{quote}
\item [INPUT ARGUMENTS]:
\begin{tabbing}
descrxxx\=CHARACTERx\=expressionxxx\=\kill
{\em iochan}\>INTEGER\>expression\>I/O channel assigned to device.\\
{\em npb}\>INTEGER\>expression\>Number of blocks to skip.
\end{tabbing}
\item [OUTPUT ARGUMENT]:
\begin{tabbing}
descrxxx\=CHARACTERx\=expressionxxx\=\kill
{\em status}\>INTEGER\>variable\>Status return value.
\end{tabbing}
\item [NOTES]:
The number of blocks to be skipped is specified as a signed integer value.
If a positive count is specified, the tape moves forward.
If a negative count is specified, the tape moves backward.
The operation is terminated by a tape mark when the tape moves in either
direction, by the beginning-of-tape marker when the tape moves backward, and by
the end-of-tape marker when the tape moves forward.
If the operation is terminated by a tape mark, an end-of-file status is
returned.
If the movement is forward, the operation will leave the tape positioned just
after the tape mark.
Backward movement leaves the tape positioned just before a tape mark; i.e.\
at the end of a file (unless the BOT marker is encountered).
Consecutive tape marks are treated as such and are not handled as `logical
end-of-volume' which would otherwise terminate a skip operation in the forward
direction.
\end{description}
\goodbreak
\rule{\textwidth}{0.3mm}
{\Large {\bf TIO\_OPEN} \hfill Assign I/O Channel \hfill {\bf TIO\_OPEN}}
\begin{description}
\item [FUNCTION]:
Assign an I/O channel to a magnetic tape device.
\item [CALL]:
\begin{quote}
{\tt CALL TIO\_OPEN(device,iochan,status)}
\end{quote}
\item [INPUT ARGUMENT]:
\begin{tabbing}
descrxxx\=CHARACTERx\=expressionxxx\=\kill
{\em device}\>CHARACTER\>expression\>Device name.\\
\end{tabbing}
\item [OUTPUT ARGUMENTS]:
\begin{tabbing}
descrxxx\=CHARACTERx\=expressionxxx\=\kill
{\em iochan}\>INTEGER\>variable\>I/O channel assigned to device.\\
{\em status}\>INTEGER\>variable\>Status return value.
\end{tabbing}
\item [NOTES]:
The routine does not reposition the tape to the start of the first file, thus
no assumption should be made about its current position.
The program specifies the name of the device on which the tape is mounted.
{\bf TIO\_OPEN} returns the channel number assigned which is used as input to
the routines which perform the actual I/O operations.
The routine must be called for every tape device used by a program before any
I/O operations on them can be performed.
For example:
\begin{verbatim}
    CALL TIO_OPEN('INPUT',INCHAN,STATUS)
    CALL TIO_OPEN('OUTPUT',OUTCHAN,STATUS)
\end{verbatim}
Under VMS, a device name can be a logical name.
Full recursive translation is performed by {\bf TIO\_OPEN} to resolve the
physical name.
Before translation,  the name is stripped of leading and trailing blanks and
converted to uppercase if necessary.
To make programs as flexible as possible, avoid specifying the device name
explicitly as a CHARACTER constant:
\begin{verbatim}
    CALL TIO_OPEN('MTA1:',CHAN,STATUS)
\end{verbatim}
Device names can be specified implicitly by using a CHARACTER variable whose
value is determined at run-time:
\begin{verbatim}
    CHARACTER*63 DEVICE
      ...
    READ (*,'(A)') DEVICE
    CALL TIO_OPEN(DEVICE,CHAN,STATUS)
\end{verbatim}
\end{description}
\goodbreak
\rule{\textwidth}{0.3mm}
{\Large {\bf TIO\_READ} \hfill Read Data Block \hfill {\bf TIO\_READ}}
\begin{description}
\item [FUNCTION]:
Read data from the next block
\item [CALL]:
\begin{quote}
{\tt CALL TIO\_READ(iochan,maxlen,buffer,actlen,status)}
\end{quote}
\item [INPUT ARGUMENTS]:
\begin{tabbing}
descrxxx\=CHARACTERx\=expressionxxx\=\kill
{\em iochan}\>INTEGER\>expression\>I/O channel assigned to device.\\
{\em maxlen}\>INTEGER\>expression\>Max length of transfer (bytes).
\end{tabbing}
\item [OUTPUT ARGUMENTS]:
\begin{tabbing}
descrxxx\=CHARACTERx\=expressionxxx\=\kill
{\em buffer}\>\>array\>Destination buffer.\\
{\em actlen}\>INTEGER\>variable\>Actual length of block (bytes).\\
{\em status}\>INTEGER\>variable\>Status return value.
\end{tabbing}
\item [NOTES]:
The program specifies the maximum number of bytes ({\em maxlen}) to be
transferred from the tape block to the buffer.
Under VMS, this value must lie within the range 14 to 65535 inclusive, otherwise
a status value of SS\$\_BADPARAM is returned.
{\bf TIO\_READ} returns the actual length ({\em actlen}) of the block read.
If {\em actlen} $<$ {\em maxlen}, then {\em actlen} bytes have been read into
the buffer.
If the block length is greater than {\em maxlen}, only {\em maxlen} bytes will
be transferred and the status value will indicate `successful completion' and
not `buffer overflow'.
In both cases the tape will be positioned at the end of the physical block
when the transfer completes.
{\em buffer} can be an array of any type appropriate to the format of the data
being processed.
For character-type data, a BYTE (or LOGICAL*1) array can be used.
Under VAX FORTRAN, a CHARACTER variable (or array) can also be specified, but
must be passed to {\bf TIO\_READ} explicitly `by reference' [2]:
\begin{verbatim}
    CHARACTER*80  CARD
      ...
    CARD=' '
    CALL TIO_READ(IOCHAN,80,%ref(CARD),ACTLEN,STATUS)
\end{verbatim}
Numeric data, such as 16 or 32-bit integers, can be read into INTEGER*2 or
INTEGER*4 arrays, but the byte order may need to be reversed when reading tapes
produced on non-DEC equipment.
\end{description}
\goodbreak
\rule{\textwidth}{0.3mm}
{\Large {\bf TIO\_READX} \hfill Read Data Block (no retry) \hfill {\bf TIO\_READX}}
\begin{description}
\item [FUNCTION]:
Read a physical block from the tape, optionally suppressing retry.
\item [CALL]:
\begin{quote}
{\tt CALL TIO\_READX(iochan,retry,maxlen,buffer,actlen,status)}
\end{quote}
\item [INPUT ARGUMENTS]:
\begin{tabbing}
descrxxx\=CHARACTERx\=expressionxxx\=\kill
{\em iochan}\>INTEGER\>expression\>I/O channel assigned to device.\\
{\em retry}\>LOGICAL\>expression\>True if retry to be attempted on bad records,
false otherwise.\\
{\em maxlen}\>INTEGER\>expression\>Maximum length of transfer (bytes).
\end{tabbing}
\item [OUTPUT ARGUMENTS]:
\begin{tabbing}
descrxxx\=CHARACTERx\=expressionxxx\=\kill
{\em buffer}\>\>array\>Destination buffer.\\
{\em actlen}\>INTEGER\>variable\>Actual length of block (bytes); returned
even if the read\\
\>\>\>operation fails.\\
{\em status}\>INTEGER\>variable\>Status return value.
\end{tabbing}
\item [NOTES]:
This routine will read tape records that have parity or other errors without
the reading and re-reading that occurs when {\bf TIO\_READ} encounters such a
record.
{\em status} will still be returned with an error code if the record is bad.
\end{description}
\goodbreak
\rule{\textwidth}{0.3mm}
{\Large {\bf TIO\_REWIND} \hfill Rewind Tape \hfill {\bf TIO\_REWIND}}
\begin{description}
\item [FUNCTION]:
Reposition a tape to the beginning-of-tape (BOT) marker.
\item [CALL]:
\begin{quote}
{\tt CALL TIO\_REWIND(iochan,status)}
\end{quote}
\item [INPUT ARGUMENT]:
\begin{tabbing}
descrxxx\=CHARACTERx\=expressionxxx\=\kill
{\em iochan}\>INTEGER\>expression\>I/O channel assigned to device.
\end{tabbing}
\item [OUTPUT ARGUMENT]:
\begin{tabbing}
descrxxx\=CHARACTERx\=expressionxxx\=\kill
{\em status}\>INTEGER\>variable\>Status return value.
\end{tabbing}
\item [NOTES]:
Under VMS, the tape is rewound to its load point and the device is left online.
The effect is the same as the DCL command:
\begin{verbatim}
    $ SET MAGTAPE/REWIND.
\end{verbatim}
This routine will generally be used after a call to {\bf TIO\_OPEN} to ensure
that the tape is positioned correctly for subsequent I/O operations.
\end{description}
\goodbreak
\rule{\textwidth}{0.3mm}
{\Large {\bf TIO\_SENSE} \hfill  Interrogate Tape Drive \hfill {\bf TIO\_SENSE}}
\begin{description}
\item [FUNCTION]:
Get information about a tape drive.
\item [CALL]: (LOGICAL function)
\begin{quote}
{\tt CALL TIO\_SENSE(device,test,status)}
\end{quote}
\item [INPUT ARGUMENTS]:
\begin{tabbing}
descrxxx\=CHARACTERx\=expressionxxx\=\kill
{\em device}\>CHARACTER\>expression\>Name of tape drive (can be logical).\\
{\em test}\>CHARACTER\>expression\>The test to be performed.
\end{tabbing}
\item [OUTPUT ARGUMENT]:
\begin{tabbing}
descrxxx\=CHARACTERx\=expressionxxx\=\kill
{\em status}\>INTEGER\>variable\>Status return value.
\end{tabbing}
\item [OUTPUT]:
\begin{tabbing}
descrxxxx\=CHARACTER\=expressionxxx\=\kill
{\em tio\_sense}\>LOGICAL\>value\>Result of the test.
\end{tabbing}
\item [NOTES]:
{\em test} can be one of:
\begin{tabbing}
x\='Odd parity'x\=\kill
\>'Available'\>indicates if device is on-line and available --- it may be
allocated to someone else.\\
\>'Foreign'\>indicates if device is a foreign volume.\\
\>'Mounted'\>indicates if device is mounted.\\
\>'Tape'\>indicates if device is a tape.\\
\>'Lost'\>indicates if current tape position is unknown.\\
\>'Readonly'\>indicates if tape is write protected.\\
\>'EOT'\>indicates tape is past end-of-tape.\\
\>'EOF'\>indicates if last operation hit a file mark.\\
\>'BOT'\>indicates if tape is at load point.\\
\>'Odd parity'\>indicates tape is odd parity.\\
\>'6250'\>indicates tape is 6250 bpi.\\
\>'1600'\>indicates tape is 1600 bpi.\\
\>'800'\>indicates tape is 800 bpi.
\end{tabbing}
The characters of {\em test} may be upper or lower case, and only enough
characters to indicate unambiguously which item is being requested are tested.
So unless the first character is `E', it will be the only one that this
function looks at.
Note that bad status will be returned for a call that tries to get information
like `6250' for a device that is not a tape.
\end{description}
\goodbreak
\rule{\textwidth}{0.3mm}
{\Large {\bf TIO\_SETDEN} \hfill Set Tape Density \hfill {\bf TIO\_SETDEN}}
\begin{description}
\item [FUNCTION]:
Set the recording density for a tape.
\item [CALL]:
\begin{quote}
{\tt CALL TIO\_SETDEN(iochan,dens,status)}
\end{quote}
\item [INPUT ARGUMENTS]:
\begin{tabbing}
descrxxx\=CHARACTERx\=expressionxxx\=\kill
{\em iochan}\>INTEGER\>expression\>I/O channel assigned to device.\\
{\em dens}\>INTEGER\>expression\>Density to be used (800, 1600, or 6250).
\end{tabbing}
\item [OUTPUT ARGUMENT]:
\begin{tabbing}
descrxxx\=CHARACTERx\=expressionxxx\=\kill
{\em status}\>INTEGER\>variable\>Status return value.
\end{tabbing}
\item [NOTES]:
Sets the recording density of a tape.
The tape must already have been opened by {\bf TIO\_OPEN} and is rewound
before the density is changed.
\end{description}
\goodbreak
\rule{\textwidth}{0.3mm}
{\Large {\bf TIO\_SKIP} \hfill Skip Tape Marks \hfill {\bf TIO\_SKIP}}
\begin{description}
\item [FUNCTION]:
Skip forward or backward past a specified number of tape marks.
\item [CALL]:
\begin{quote}
{\tt CALL TIO\_SKIP(iochan,ntm,status)}
\end{quote}
\item [INPUT ARGUMENTS]:
\begin{tabbing}
descrxxx\=CHARACTERx\=expressionxxx\=\kill
{\em iochan}\>INTEGER\>expression\>I/O channel assigned to device.\\
{\em ntm}\>INTEGER\>expression\>Number of tape-marks to skip.
\end{tabbing}
\item [OUTPUT ARGUMENT]:
\begin{tabbing}
descrxxx\=CHARACTERx\=expressionxxx\=\kill
{\em status}\>INTEGER\>variable\>Status return value.
\end{tabbing}
\item [NOTES]:
The number of tape marks to be skipped is passed as a signed integer value.
If a positive count is specified, the tape moves forward.
If a negative count is specified, the tape moves backward.
Only tape  marks (when the tape moves in either direction) and the
beginning-of-tape marker (when the tape moves backward) are counted during a
skip operation.
The BOT marker terminates a backward skip operation, whereas an EOT marker has
no effect in either direction.

A forward skip operation leaves the tape positioned just after a tape mark.
A backward skip operation leaves the tape positioned just before a tape mark,
i.e.\ at the end of a file (unless the BOT marker is encountered).
Consecutive tape marks are treated as such and are not handled as `logical
end-of-volume' which would otherwise terminate a skip operation in the forward
direction.
\end{description}
\goodbreak
\rule{\textwidth}{0.3mm}
{\Large {\bf TIO\_WRITE} \hfill Write Data Block \hfill {\bf TIO\_WRITE}}
\begin{description}
\item [FUNCTION]:
Write data to the next block.
\item [CALL]:
\begin{quote}
{\tt CALL TIO\_WRITE(iochan,buffer,length,status)}
\end{quote}
\item [INPUT ARGUMENTS]:
\begin{tabbing}
descrxxx\=CHARACTERx\=expressionxxx\=\kill
{\em iochan}\>INTEGER\>expression\>I/O channel assigned to device.\\
{\em buffer}\>\>array\>Source buffer.\\
{\em length}\>INTEGER\>expression\>Length of transfer (bytes).
\end{tabbing}
\item [OUTPUT ARGUMENT]:
\begin{tabbing}
descrxxx\=CHARACTERx\=expressionxxx\=\kill
{\em status}\>INTEGER\>variable\>Status return value.
\end{tabbing}
\item [NOTES]:
The program specifies the number of bytes to be transferred from the
buffer to the tape.
Under VMS, this value must lie within the range 14 to 65535 inclusive, otherwise
a status value of SS\$\_BADPARAM is returned.
The source buffer can be an array of any type appropriate to the format of the
data being written.
Under VAX FORTRAN, a CHARACTER variable (or array) can be specified, but must be
passed to {\bf TIO\_WRITE} explicitly `by reference' (see {\bf TIO\_READ}).
\end{description}
\end{document}
