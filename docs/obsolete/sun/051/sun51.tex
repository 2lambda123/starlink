\documentstyle{article} 
\pagestyle{myheadings}
\markright{SUN/51.1}
\setlength{\textwidth}{160mm}
\setlength{\textheight}{240mm}
\setlength{\topmargin}{-5mm}
\setlength{\oddsidemargin}{0mm}
\setlength{\evensidemargin}{0mm}
\setlength{\parindent}{0mm}
\setlength{\parskip}{\medskipamount}
\setlength{\unitlength}{1mm}

\begin{document}
\thispagestyle{empty}
SCIENCE \& ENGINEERING RESEARCH COUNCIL \hfill SUN/51.1\\
RUTHERFORD APPLETON LABORATORY\\
{\large\bf Starlink Project\\}
{\large\bf Starlink User Note 51.1}
\begin{flushright}
C A Clayton\\
1 November 1988
\end{flushright}
\vspace{-4mm}
\rule{\textwidth}{0.5mm}
\vspace{5mm}
\begin{center}
{\Large\bf Submission of NEWS items}
\end{center}
\vspace{5mm}

The purpose of this note is to inform users how to correctly format
and submit NEWS items for distribution around Starlink. An incorrectly
formatted item can appear garbled when inserted into the
NEWS library. At the very least, system managers around Starlink 
will have to waste time reformatting these items before
they can be used. 

\section {Formatting your NEWS item}

NEWS items should be produced according to the following guidelines:

\begin{itemize}

\item Please keep the title of the news item as short as possible 
without losing important detail. The title line should start with a 
`1' in column 1 (see example). The title itself must be a single string of
characters starting in column 3 or greater. If the title contains
more than one word, each word should be separated by a `\_' and
not a space (see example).

\item Every news item {\bf MUST} contain an expiration date. This is the date
after which the NEWS item can be deleted from the NEWS system. In the past,
NEWS items without expiration dates have been left filling up the NEWS system
well beyond their useful lifetime because system managers have been unsure 
whether it is time to delete them. This date should be clearly visible at the
top right-hand corner of the item, above the main text but below the 
title line (see example). 

\item Each line of the main text should be no more than 78 characters long
since the LIBRARIAN utility used to store NEWS items 
indents the main text by two spaces. Longer
lines wrap around and give the NEWS item a chaotic, unprofessional look.

\item The main text of your NEWS item should contain no digits in
the first column. The LIBRARIAN utility will interpret these as key
numbers with potentially catastrophic consequences for your text.

\item If you wish to have a tree structured NEWS item, then this
should be formatted as described in the Librarian Reference manual
(VMS Version 4 -- Vol 7b, LIB-7; VMS Version 5 -- Vol 2b, LIB-5).

\end{itemize}

\section {Submitting your NEWS item}

Once your NEWS item is complete, you should send it to your local
system manager, clearly stating whether the item is just to go
into the local NEWS system or if it is to be displayed around the 
whole of Starlink. Your manager will take care of both local 
installation of the item and, if applicable, distribution to other system 
managers around Starlink. 

\section {Example}

\begin{verbatim}

1 Black_hole_Seminar
                                                Expires 13-12-1988

An extra seminar has been scheduled for next week. On Tuesday 13th
December, Dr. John Blout will describe his controversial work in a
talk entitled ``Star Formation in Black Holes''. Everyone is urged 
to attend this seminar and the lively discussion afterwards at the 
Rose & Crown. 

S.D. Stoddard

\end{verbatim}

\end{document} 
