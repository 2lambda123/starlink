\documentstyle[11pt]{article}
\pagestyle{myheadings}

%------------------------------------------------------------------------------
\newcommand{\stardoccategory}  {Starlink User Note}
\newcommand{\stardocinitials}  {SUN}
\newcommand{\stardocnumber}    {135.2}
\newcommand{\stardocauthors}   {P M Allan}
\newcommand{\stardocdate}      {6 February 1992}
\newcommand{\stardoctitle}     {LOG --- A logical name utility \\ [1ex]Version 1.1}
%------------------------------------------------------------------------------

\newcommand{\stardocname}{\stardocinitials /\stardocnumber}
\renewcommand{\_}{{\tt\char'137}}     % re-centres the underscore
\markright{\stardocname}
\setlength{\textwidth}{160mm}
\setlength{\textheight}{230mm}
\setlength{\topmargin}{-2mm}
\setlength{\oddsidemargin}{0mm}
\setlength{\evensidemargin}{0mm}
\setlength{\parindent}{0mm}
\setlength{\parskip}{\medskipamount}
\setlength{\unitlength}{1mm}

%------------------------------------------------------------------------------
% Add any \newcommand or \newenvironment commands here
%------------------------------------------------------------------------------

\begin{document}
\thispagestyle{empty}
SCIENCE \& ENGINEERING RESEARCH COUNCIL \hfill \stardocname\\
RUTHERFORD APPLETON LABORATORY\\
{\large\bf Starlink Project\\}
{\large\bf \stardoccategory\ \stardocnumber}
\begin{flushright}
\stardocauthors\\
\stardocdate
\end{flushright}
\vspace{-4mm}
\rule{\textwidth}{0.5mm}
\vspace{5mm}
\begin{center}
{\Large\bf \stardoctitle}
\end{center}
\vspace{5mm}

%------------------------------------------------------------------------------
%  Add this part if you want a table of contents
%  \setlength{\parskip}{0mm}
%  \tableofcontents
%  \setlength{\parskip}{\medskipamount}
%  \markright{\stardocname}
%------------------------------------------------------------------------------

\section{Overview}

This document describes a utility to add an item to a VMS logical name search
list. The utility may be joined by others in the future.

\section{Introduction}

Occasionally it is necessary to add an item to a logical name search list.
Surprisingly, there is no simple way of doing this from DCL. The command
procedure {\tt ADD\_TO\_SEARCH\_LIST} has been written to perform this task.
The procedure has four parameters; two mandatory and two optional. These are as
follows:

\begin{itemize}
\item The first parameter is the logical name to be created or modified.
\item The second parameter is the item to be added to the search list.
\item The third (optional) parameter indicates in which logical name table the
search list should be created.
\item The fourth (optional) parameter indicates whether the new item should be
added at the beginning or the end of the current list. The default is to add
the new item at the end. If the fourth parameter is BEGIN then the new item
will be added at the beginning of the list. If the fourth parameter is COPY,
then the search list will be copied from the current logical name table to the
one specified by the third parameter, without change. In this case the second
parameter is ignored.
\end{itemize}

If the logical name does not already exist, then the command procedure will
create it.

Here is an example of building up a search list.

\begin{verbatim}
      $ @UTILITYDIR:ADD_TO_SEARCH_LIST C$INCLUDE EMS_DIR
      $ @UTILITYDIR:ADD_TO_SEARCH_LIST C$INCLUDE CNF_DIR
      $ @UTILITYDIR:ADD_TO_SEARCH_LIST C$INCLUDE CHR_DIR
\end{verbatim}

Typing {\tt SHOW LOGICAL C\$INCLUDE}  will now give

\begin{verbatim}
      "C$INCLUDE" = "EMS_DIR" (LNM$PROCESS_TABLE)
           = "CNF_DIR"
           = "CHR_DIR"

    1  "EMS_DIR" = "STARDISK:[STARLINK.LIB.EMS.STANDALONE]" (LNM$SYSTEM_TABLE)
    1  "CNF_DIR" = "STARDISK:[STARLINK.LIB.CNF]" (LNM$SYSTEM_TABLE)
    1  "CHR_DIR" = "STARDISK:[STARLINK.LIB.CHR]" (LNM$SYSTEM_TABLE)
\end{verbatim}

Note that since the items in the search list are themselves logical names, the
fully recursive translation of those items is also given. The above example is
not realistic in the sense that you could define the search list much more
simply with a single DEFINE command. However, if the individual additions to
the list are spread around several command procedures, then such a sequence is
the easiest way to create the search list.

\section{Different Logical Name Tables}

The third parameter of the command procedure can be given as PROCESS, JOB,
GROUP, SYSTEM or as the name of another logical name table. If this parameter
is specified, then a new logical name will be created in the table specified.
If it is not specified, then logical name is created in the table that it is
already in, if it exists, and in the process table if it does not. This means
that if you wish to add an item to a system search list, but have the result
in your process table, then you must specify PROCESS for the third parameter.

There is one exception to the rule that the new item is added to the current
logical name table or to the one specified by the third parameter. If the user
does not have the required privilege to create or modify the logical name, then
the search list is created in the process logical name table. For example, if a
logical name exists in the system table and you try to modify it without SYSNAM
privilege, then the logical name search list will be copied to your process
logical name table where it will have the new item added.

The third parameter does not affect the logical name tables that get searched
for the translation of the current logical name, if it exists.

\section{Changes from version 1.0}

In version 1.0 of this utility, the function of the fourth parameter was
slightly different. If it was not specified, then the new item was added at the
end of the search list (this behaviour has not changed), but if is was anything
else, then the new item was added to the beginning of the search list. Also,
you could not copy a search list from one logical name table to another
unchanged.

The behaviour is now such that if the fourth parameter not specified, or is
anything other than BEGIN or COPY, then the new item is added at the end of the
search list.

\end{document}
