\documentclass[twoside,11pt]{article}
%
% NAOS - Finding Guide Stars for NAOMI.
%
% Copyright 2001  Starlink, CCLRC.
%
% A.C. Davenhall (Edinburgh), 24/10/00.
%
%------------------------------------------------------------------------------

%
% Set the NAOS and CURSA version numbers.
%
% Note that CURSAversion is the earliest version of CURSA which is
% compatible with NAOSversion, which is not necessarily the latest version
% of CURSA.

\newcommand{\NAOSversion}{1.2~}
\newcommand{\CURSAversion}{6.3~}

%------------------------------------------------------------------------------

% ? Specify used packages
% \usepackage{graphicx}        %  Use this one for final production.
% \usepackage[draft]{graphicx} %  Use this one for drafting.
% ? End of specify used packages

\pagestyle{myheadings}

%------------------------------------------------------------------------------

% Define commands for displaying angles as sexagesimal hours and minutes
% or degrees and minutes.

\newcommand{\tmin}   {\mbox{$^{\rm m}\!\!.$}}
\newcommand{\hm}[3] {$#1^{\rm h}\,#2\tmin#3$}
\newcommand{\dm}[2] {$#1^{\circ}\,#2\raisebox{-0.5ex}{$^{'}$}$}
\newcommand{\arcmin} {\raisebox{-0.5ex}{$^{'}$} }

\newcommand{\arcsec} {$\hspace{-0.05em}\raisebox{-0.5ex}
                     {$^{'\hspace{-0.1em}'}$}
                     \hspace{-0.7em}.\hspace{-0.05em}$}
\newcommand{\tsec}   {\mbox{$^{\rm s}\!\!.$}}
\newcommand{\hms}[4] {$#1^{\rm h}\,#2^{\rm m}\,#3\tsec#4$}
\newcommand{\dms}[4] {$#1^{\circ}\,#2\raisebox{-0.5ex}{$^{'}$}\,#3\arcsec#4$}

%-----------------------------------------------------------------------------
% ? Document identification
% Fixed part
\newcommand{\stardoccategory}  {Starlink User Note}
\newcommand{\stardocinitials}  {SUN}
\newcommand{\stardocsource}    {sun\stardocnumber}
\newcommand{\stardoccopyright} 
{Copyright \copyright\ 2002 Council for the Central Laboratory of the Research Councils}

% Variable part - replace [xxx] as appropriate.
\newcommand{\stardocnumber}    {235.3}
\newcommand{\stardocauthors}   {A.C.~Davenhall}
\newcommand{\stardocdate}      {24 September 2002}
\newcommand{\stardoctitle}     {NAOS --- \\ Finding NAOMI Guide Stars}
\newcommand{\stardocversion}   {Version \NAOSversion}
\newcommand{\stardocmanual}    {User's Manual}
\newcommand{\stardocabstract}
{Observations made with the NAOMI adaptive optics system on the
William~Herschel Telescope (WHT) usually require a guide star located
close to the target object being observed.  This manual describes how to
find such guide stars.  It documents the NAOS package whose purpose is
precisely to find suitable guide stars and also describes various related
items of software.  Details of the potential guide stars are produced as
tabular lists in text files, finding charts and tables in a format
suitable for input into GAIA or CURSA.

\begin{latexonly}
\vspace{5mm}
\end{latexonly}

\begin{center}
{\bf Who Should Read this Document?}
\end{center}

This document is aimed at astronomers who are planning to observe
with the NAOMI adaptive optics system on the WHT and need to find guide
stars for their target objects.}
% ? End of document identification
% -----------------------------------------------------------------------------

% +
%  Name:
%     sun.tex
%
%  Purpose:
%     Template for Starlink User Note (SUN) documents.
%     Refer to SUN/199
%
%  Authors:
%     AJC: A.J.Chipperfield (Starlink, RAL)
%     BLY: M.J.Bly (Starlink, RAL)
%     PWD: Peter W. Draper (Starlink, Durham University)
%
%  History:
%     17-JAN-1996 (AJC):
%        Original with hypertext macros, based on MDL plain originals.
%     16-JUN-1997 (BLY):
%        Adapted for LaTeX2e.
%        Added picture commands.
%     13-AUG-1998 (PWD):
%        Converted for use with LaTeX2HTML version 98.2 and
%        Star2HTML version 1.3.
%      1-FEB-2000 (AJC):
%        Add Copyright statement in LaTeX
%     {Add further history here}
%
% -

\newcommand{\stardocname}{\stardocinitials /\stardocnumber}
\markboth{\stardocname}{\stardocname}
\setlength{\textwidth}{160mm}
\setlength{\textheight}{230mm}
\setlength{\topmargin}{-2mm}
\setlength{\oddsidemargin}{0mm}
\setlength{\evensidemargin}{0mm}
\setlength{\parindent}{0mm}
\setlength{\parskip}{\medskipamount}
\setlength{\unitlength}{1mm}

% -----------------------------------------------------------------------------
%  Hypertext definitions.
%  ======================
%  These are used by the LaTeX2HTML translator in conjunction with star2html.

%  Comment.sty: version 2.0, 19 June 1992
%  Selectively in/exclude pieces of text.
%
%  Author
%    Victor Eijkhout                                      <eijkhout@cs.utk.edu>
%    Department of Computer Science
%    University Tennessee at Knoxville
%    104 Ayres Hall
%    Knoxville, TN 37996
%    USA

%  Do not remove the %begin{latexonly} and %end{latexonly} lines (used by 
%  LaTeX2HTML to signify text it shouldn't process).
%begin{latexonly}
\makeatletter
\def\makeinnocent#1{\catcode`#1=12 }
\def\csarg#1#2{\expandafter#1\csname#2\endcsname}

\def\ThrowAwayComment#1{\begingroup
    \def\CurrentComment{#1}%
    \let\do\makeinnocent \dospecials
    \makeinnocent\^^L% and whatever other special cases
    \endlinechar`\^^M \catcode`\^^M=12 \xComment}
{\catcode`\^^M=12 \endlinechar=-1 %
 \gdef\xComment#1^^M{\def\test{#1}
      \csarg\ifx{PlainEnd\CurrentComment Test}\test
          \let\html@next\endgroup
      \else \csarg\ifx{LaLaEnd\CurrentComment Test}\test
            \edef\html@next{\endgroup\noexpand\end{\CurrentComment}}
      \else \let\html@next\xComment
      \fi \fi \html@next}
}
\makeatother

\def\includecomment
 #1{\expandafter\def\csname#1\endcsname{}%
    \expandafter\def\csname end#1\endcsname{}}
\def\excludecomment
 #1{\expandafter\def\csname#1\endcsname{\ThrowAwayComment{#1}}%
    {\escapechar=-1\relax
     \csarg\xdef{PlainEnd#1Test}{\string\\end#1}%
     \csarg\xdef{LaLaEnd#1Test}{\string\\end\string\{#1\string\}}%
    }}

%  Define environments that ignore their contents.
\excludecomment{comment}
\excludecomment{rawhtml}
\excludecomment{htmlonly}

%  Hypertext commands etc. This is a condensed version of the html.sty
%  file supplied with LaTeX2HTML by: Nikos Drakos <nikos@cbl.leeds.ac.uk> &
%  Jelle van Zeijl <jvzeijl@isou17.estec.esa.nl>. The LaTeX2HTML documentation
%  should be consulted about all commands (and the environments defined above)
%  except \xref and \xlabel which are Starlink specific.

\newcommand{\htmladdnormallinkfoot}[2]{#1\footnote{#2}}
\newcommand{\htmladdnormallink}[2]{#1}
\newcommand{\htmladdimg}[1]{}
\newcommand{\hyperref}[4]{#2\ref{#4}#3}
\newcommand{\htmlref}[2]{#1}
\newcommand{\htmlimage}[1]{}
\newcommand{\htmladdtonavigation}[1]{}

\newenvironment{latexonly}{}{}
\newcommand{\latex}[1]{#1}
\newcommand{\html}[1]{}
\newcommand{\latexhtml}[2]{#1}
\newcommand{\HTMLcode}[2][]{}

%  Starlink cross-references and labels.
\newcommand{\xref}[3]{#1}
\newcommand{\xlabel}[1]{}

%  LaTeX2HTML symbol.
\newcommand{\latextohtml}{\LaTeX2\texttt{HTML}}

%  Define command to re-centre underscore for Latex and leave as normal
%  for HTML (severe problems with \_ in tabbing environments and \_\_
%  generally otherwise).
\renewcommand{\_}{\texttt{\symbol{95}}}

% -----------------------------------------------------------------------------
%  Debugging.
%  =========
%  Remove % on the following to debug links in the HTML version using Latex.

% \newcommand{\hotlink}[2]{\fbox{\begin{tabular}[t]{@{}c@{}}#1\\\hline{\footnotesize #2}\end{tabular}}}
% \renewcommand{\htmladdnormallinkfoot}[2]{\hotlink{#1}{#2}}
% \renewcommand{\htmladdnormallink}[2]{\hotlink{#1}{#2}}
% \renewcommand{\hyperref}[4]{\hotlink{#1}{\S\ref{#4}}}
% \renewcommand{\htmlref}[2]{\hotlink{#1}{\S\ref{#2}}}
% \renewcommand{\xref}[3]{\hotlink{#1}{#2 -- #3}}
%end{latexonly}
% -----------------------------------------------------------------------------
% ? Document specific \newcommand or \newenvironment commands.
% %+
%  Name:
%     SST.TEX

%  Purpose:
%     Define LaTeX commands for laying out Starlink routine descriptions.

%  Language:
%     LaTeX

%  Type of Module:
%     LaTeX data file.

%  Description:
%     This file defines LaTeX commands which allow routine documentation
%     produced by the SST application PROLAT to be processed by LaTeX and
%     by LaTeX2html. The contents of this file should be included in the
%     source prior to any statements that make of the sst commnds.

%  Notes:
%     The commands defined in the style file html.sty provided with LaTeX2html
%     are used. These should either be made available by using the appropriate
%     sun.tex (with hypertext extensions) or by putting the file html.sty
%     on your TEXINPUTS path (and including the name as part of the
%     documentstyle declaration).

%  Authors:
%     RFWS: R.F. Warren-Smith (STARLINK)
%     PDRAPER: P.W. Draper (Starlink - Durham University)
%     MJC: Malcolm J. Currie (STARLINK)
%     DSB: David Berry (STARLINK)
%     TIMJ: Tim Jenness (JAC)

%  History:
%     10-SEP-1990 (RFWS):
%        Original version.
%     10-SEP-1990 (RFWS):
%        Added the implementation status section.
%     12-SEP-1990 (RFWS):
%        Added support for the usage section and adjusted various spacings.
%     8-DEC-1994 (PDRAPER):
%        Added support for simplified formatting using LaTeX2html.
%     1995 October 4 (MJC):
%        Added goodbreaks and pagebreak[3] in various places to improve
%        pages breaking before headings, not immediately after.
%        Corrected banner width.
%     1996 March 7 (MJC):
%        Mark document name on both sides of an sstroutine.
%     2-DEC-1998 (DSB):
%        Added sstattributetype (copied from sun210.tex).
%     2004 August 6 (MJC):
%        Added sstattribute.
%     21-JUL-2009 (TIMJ):
%        Added \sstdiylist{}{} as used when a Parameters section is located that
%        is not "ADAM Parameters".
%     {enter_further_changes_here}

%  Bugs:
%     {note_any_bugs_here}

%-

%  Define length variables.
\newlength{\sstbannerlength}
\newlength{\sstcaptionlength}
\newlength{\sstexampleslength}
\newlength{\sstexampleswidth}

%  Define a \tt font of the required size.
\latex{\newfont{\ssttt}{cmtt10 scaled 1095}}
\html{\newcommand{\ssttt}{\tt}}

%  Define a command to produce a routine header, including its name,
%  a purpose description and the rest of the routine's documentation.
\newcommand{\sstroutine}[3]{
   \goodbreak
   \markboth{{\stardocname}~ --- #1}{{\stardocname}~ --- #1}
   \rule{\textwidth}{0.5mm}
   \vspace{-7ex}
   \newline
   \settowidth{\sstbannerlength}{{\Large {\bf #1}}}
   \setlength{\sstcaptionlength}{\textwidth}
   \setlength{\sstexampleslength}{\textwidth}
   \addtolength{\sstbannerlength}{0.5em}
   \addtolength{\sstcaptionlength}{-2.0\sstbannerlength}
   \addtolength{\sstcaptionlength}{-5.0pt}
   \settowidth{\sstexampleswidth}{{\bf Examples:}}
   \addtolength{\sstexampleslength}{-\sstexampleswidth}
   \parbox[t]{\sstbannerlength}{\flushleft{\Large {\bf #1}}}
   \parbox[t]{\sstcaptionlength}{\center{\Large #2}}
   \parbox[t]{\sstbannerlength}{\flushright{\Large {\bf #1}}}
   \begin{description}
      #3
   \end{description}
}

% Frame attributes fount.  Need to find a way for these to stand out.
% San serif doesn't work by default.  Also without the \rm the
% san serif continues after \sstatt hyperlinks.  Extra braces
% failed to prevent \sstattribute from using roman fount for its
% headings.  The current lash up appears to work, but needs further
% investigation or a TeX wizard.
\newcommand{\sstatt}[1]{\sf #1}
\begin{htmlonly}
  \newcommand{\sstatt}[1]{\large{\tt #1}}
\end{htmlonly}

%  Define a command to produce an attribute header, including its name,
%  a purpose description and the rest of the routine's documentation.
\newcommand{\sstattribute}[3]{
   \goodbreak
   \markboth{{\stardocname}~ --- #1}{{\stardocname}~ --- #1}
   \rule{\textwidth}{0.5mm}
   \vspace{-7ex}
   \newline
   \settowidth{\sstbannerlength}{{\Large {\sstatt #1}}}
   \setlength{\sstcaptionlength}{\textwidth}
   \setlength{\sstexampleslength}{\textwidth}
   \addtolength{\sstbannerlength}{0.5em}
   \addtolength{\sstcaptionlength}{-2.0\sstbannerlength}
   \addtolength{\sstcaptionlength}{-4.9pt}
   \settowidth{\sstexampleswidth}{{\bf Examples:}}
   \addtolength{\sstexampleslength}{-\sstexampleswidth}
   \parbox[t]{\sstbannerlength}{\flushleft{\Large {\sstatt #1}}}
   \parbox[t]{\sstcaptionlength}{\center{\Large #2}}
   \parbox[t]{\sstbannerlength}{\flushright{\Large {\sstatt #1}}}
   \begin{description}
      #3
   \end{description}
}

%  Format the description section.
\newcommand{\sstdescription}[1]{\item[Description:] #1}

%  Format the usage section.
\newcommand{\sstusage}[1]{\goodbreak \item[Usage:] \mbox{}
\\[1.3ex]{\raggedright \ssttt #1}}

%  Format the invocation section.
\newcommand{\sstinvocation}[1]{\item[Invocation:]\hspace{0.4em}{\tt #1}}

%  Format the attribute data type section.
\newcommand{\sstattributetype}[1]{
   \item[Type:] \mbox{} \\
      #1
}

%  Format the arguments section.
\newcommand{\sstarguments}[1]{
   \item[Arguments:] \mbox{} \\
   \vspace{-3.5ex}
   \begin{description}
      #1
   \end{description}
}

%  Format the returned value section (for a function).
\newcommand{\sstreturnedvalue}[1]{
   \item[Returned Value:] \mbox{} \\
   \vspace{-3.5ex}
   \begin{description}
      #1
   \end{description}
}

%  Format the parameters section (for an application).
\newcommand{\sstparameters}[1]{
   \goodbreak
   \item[Parameters:] \mbox{} \\
   \vspace{-3.5ex}
   \begin{description}
      #1
   \end{description}
}

%  Format the applicability section.
\newcommand{\sstapplicability}[1]{
   \item[Class Applicability:] \mbox{} \\
   \vspace{-3.5ex}
   \begin{description}
      #1
   \end{description}
}

%  Format the examples section.
\newcommand{\sstexamples}[1]{
   \goodbreak
   \item[Examples:] \mbox{} \\
   \vspace{-3.5ex}
   \begin{description}
      #1
   \end{description}
}

%  Define the format of a subsection in a normal section.
\newcommand{\sstsubsection}[1]{ \item[{#1}] \mbox{} \\}

%  Define the format of a subsection in the examples section.
\newcommand{\sstexamplesubsection}[2]{\sloppy
\item[\parbox{\sstexampleslength}{\ssttt #1}] \mbox{} \vspace{1.0ex}
\\ #2 }

%  Format the notes section.
\newcommand{\sstnotes}[1]{\goodbreak \item[Notes:] \mbox{} \\[1.3ex] #1}

%  Provide a general-purpose format for additional (DIY) sections.
\newcommand{\sstdiytopic}[2]{\item[{\hspace{-0.35em}#1\hspace{-0.35em}:}]
\mbox{} \\[1.3ex] #2}

%  Format the a generic section as a list
\newcommand{\sstdiylist}[2]{
   \item[#1:] \mbox{} \\
   \vspace{-3.5ex}
   \begin{description}
      #2
   \end{description}
}

%  Format the implementation status section.
\newcommand{\sstimplementationstatus}[1]{
   \item[{Implementation Status:}] \mbox{} \\[1.3ex] #1}

%  Format the bugs section.
\newcommand{\sstbugs}[1]{\item[Bugs:] #1}

%  Format a list of items while in paragraph mode.
\newcommand{\sstitemlist}[1]{
  \mbox{} \\
  \vspace{-3.5ex}
  \begin{itemize}
     #1
  \end{itemize}
}

%  Define the format of an item.
\newcommand{\sstitem}{\item}

%  Now define html equivalents of those already set. These are used by
%  latex2html and are defined in the html.sty files.
\begin{htmlonly}

%  sstroutine.
   \newcommand{\sstroutine}[3]{
      \subsection{#1\xlabel{#1}-\label{#1}#2}
      \begin{description}
         #3
      \end{description}
   }

%  sstattribute. Note the further level of subsectioning.
   \newcommand{\sstattribute}[3]{
      \subsubsection{#1\xlabel{#1}-\label{#1}#2}
      \begin{description}
         #3
      \end{description}
      \\
   }

%  sstdescription
   \newcommand{\sstdescription}[1]{\item[Description:]
      \begin{description}
         #1
      \end{description}
      \\
   }

%  sstusage
   \newcommand{\sstusage}[1]{\item[Usage:]
      \begin{description}
         {\ssttt #1}
      \end{description}
      \\
   }

%  sstinvocation
   \newcommand{\sstinvocation}[1]{\item[Invocation:]
      \begin{description}
         {\ssttt #1}
      \end{description}
      \\
   }

%  sstarguments
   \newcommand{\sstarguments}[1]{
      \item[Arguments:] \\
      \begin{description}
         #1
      \end{description}
      \\
   }

%  sstreturnedvalue
   \newcommand{\sstreturnedvalue}[1]{
      \item[Returned Value:] \\
      \begin{description}
         #1
      \end{description}
      \\
   }

%  sstparameters
   \newcommand{\sstparameters}[1]{
      \item[Parameters:] \\
      \begin{description}
         #1
      \end{description}
      \\
   }

%  sstapplicability
   \newcommand{\sstapplicability}[1]{%
      \item[Class Applicability:]
       \begin{description}
         #1
      \end{description}
      \\
   }

%  sstexamples
   \newcommand{\sstexamples}[1]{
      \item[Examples:] \\
      \begin{description}
         #1
      \end{description}
      \\
   }

%  sstsubsection
   \newcommand{\sstsubsection}[1]{\item[{#1}]}

%  sstexamplesubsection
   \newcommand{\sstexamplesubsection}[2]{\item[{\ssttt #1}] #2\\}

%  sstnotes
   \newcommand{\sstnotes}[1]{\item[Notes:] #1 }

%  sstdiytopic
   \newcommand{\sstdiytopic}[2]{\item[{#1}] #2 }

%  sstimplementationstatus
   \newcommand{\sstimplementationstatus}[1]{
      \item[Implementation Status:] #1
   }

%  sstitemlist
   \newcommand{\sstitemlist}[1]{
      \begin{itemize}
         #1
      \end{itemize}
      \\
   }
%  sstitem
   \newcommand{\sstitem}{\item}

\end{htmlonly}

%  End of sst.tex layout definitions.
%.

 
%+
%  Name:
%     SST.TEX

%  Purpose:
%     Define LaTeX commands for laying out Starlink routine descriptions.

%  Language:
%     LaTeX

%  Type of Module:
%     LaTeX data file.

%  Description:
%     This file defines LaTeX commands which allow routine documentation
%     produced by the SST application PROLAT to be processed by LaTeX and
%     by LaTeX2html. The contents of this file should be included in the
%     source prior to any statements that make of the sst commnds.

%  Notes:
%     The style file html.sty provided with LaTeX2html needs to be used.
%     This must be before this file.

%  Authors:
%     RFWS: R.F. Warren-Smith (STARLINK)
%     PDRAPER: P.W. Draper (Starlink - Durham University)

%  History:
%     10-SEP-1990 (RFWS):
%        Original version.
%     10-SEP-1990 (RFWS):
%        Added the implementation status section.
%     12-SEP-1990 (RFWS):
%        Added support for the usage section and adjusted various spacings.
%     8-DEC-1994 (PDRAPER):
%        Added support for simplified formatting using LaTeX2html.
%     {enter_further_changes_here}

%  Bugs:
%     {note_any_bugs_here}

%-

%  Define length variables.
\newlength{\sstbannerlength}
\newlength{\sstcaptionlength}
\newlength{\sstexampleslength}
\newlength{\sstexampleswidth}

%  Define a \tt font of the required size.
\latex{\newfont{\ssttt}{cmtt10 scaled 1095}}
\html{\newcommand{\ssttt}{\tt}}

%  Define a command to produce a routine header, including its name,
%  a purpose description and the rest of the routine's documentation.
\newcommand{\sstroutine}[3]{
   \goodbreak
   \rule{\textwidth}{0.5mm}
   \vspace{-7ex}
   \newline
   \settowidth{\sstbannerlength}{{\Large {\bf #1}}}
   \setlength{\sstcaptionlength}{\textwidth}
   \setlength{\sstexampleslength}{\textwidth}
   \addtolength{\sstbannerlength}{0.5em}
   \addtolength{\sstcaptionlength}{-2.0\sstbannerlength}
   \addtolength{\sstcaptionlength}{-5.0pt}
   \settowidth{\sstexampleswidth}{{\bf Examples:}}
   \addtolength{\sstexampleslength}{-\sstexampleswidth}
   \parbox[t]{\sstbannerlength}{\flushleft{\Large {\bf #1}}}
   \parbox[t]{\sstcaptionlength}{\center{\Large #2}}
   \parbox[t]{\sstbannerlength}{\flushright{\Large {\bf #1}}}
   \begin{description}
      #3
   \end{description}
}

%  Format the description section.
\newcommand{\sstdescription}[1]{\item[Description:] #1}

%  Format the usage section.
\newcommand{\sstusage}[1]{\item[Usage:] \mbox{}
\\[1.3ex]{\raggedright \ssttt #1}}

%  Format the invocation section.
\newcommand{\sstinvocation}[1]{\item[Invocation:]\hspace{0.4em}{\tt #1}}

%  Format the arguments section.
\newcommand{\sstarguments}[1]{
   \item[Arguments:] \mbox{} \\
   \vspace{-3.5ex}
   \begin{description}
      #1
   \end{description}
}

%  Format the returned value section (for a function).
\newcommand{\sstreturnedvalue}[1]{
   \item[Returned Value:] \mbox{} \\
   \vspace{-3.5ex}
   \begin{description}
      #1
   \end{description}
}

%  Format the parameters section (for an application).
\newcommand{\sstparameters}[1]{
   \item[Parameters:] \mbox{} \\
   \vspace{-3.5ex}
   \begin{description}
      #1
   \end{description}
}

%  Format the examples section.
\newcommand{\sstexamples}[1]{
   \item[Examples:] \mbox{} \\
   \vspace{-3.5ex}
   \begin{description}
      #1
   \end{description}
}

%  Define the format of a subsection in a normal section.
\newcommand{\sstsubsection}[1]{ \item[{#1}] \mbox{} \\}

%  Define the format of a subsection in the examples section.
\newcommand{\sstexamplesubsection}[2]{\sloppy
\item[\parbox{\sstexampleslength}{\ssttt #1}] \mbox{} \vspace{1.0ex}
\\ #2 }

%  Format the notes section.
\newcommand{\sstnotes}[1]{\item[Notes:] \mbox{} \\[1.3ex] #1}

%  Provide a general-purpose format for additional (DIY) sections.
\newcommand{\sstdiytopic}[2]{\item[{\hspace{-0.35em}#1\hspace{-0.35em}:}]
\mbox{} \\[1.3ex] #2}

%  Format the implementation status section.
\newcommand{\sstimplementationstatus}[1]{
   \item[{Implementation Status:}] \mbox{} \\[1.3ex] #1}

%  Format the bugs section.
\newcommand{\sstbugs}[1]{\item[Bugs:] #1}

%  Format a list of items while in paragraph mode.
\newcommand{\sstitemlist}[1]{
  \mbox{} \\
  \vspace{-3.5ex}
  \begin{itemize}
     #1
  \end{itemize}
}

%  Define the format of an item.
\newcommand{\sstitem}{\item}

%% Now define html equivalents of those already set. These are used by
%  latex2html and are defined in the html.sty files.
\begin{htmlonly}

%  sstroutine.
   \newcommand{\sstroutine}[3]{
      \subsection{#1\xlabel{#1}-\label{#1}#2}
      \begin{description}
         #3
      \end{description}
   }

%  sstdescription
   \newcommand{\sstdescription}[1]{\item[Description:]
      \begin{description}
         #1
      \end{description}
      \\
   }

%  sstusage
   \newcommand{\sstusage}[1]{\item[Usage:]
      \begin{description}
         {\ssttt #1}
      \end{description}
      \\
   }

%  sstinvocation
   \newcommand{\sstinvocation}[1]{\item[Invocation:]
      \begin{description}
         {\ssttt #1}
      \end{description}
      \\
   }

%  sstarguments
   \newcommand{\sstarguments}[1]{
      \item[Arguments:] \\
      \begin{description}
         #1
      \end{description}
      \\
   }

%  sstreturnedvalue
   \newcommand{\sstreturnedvalue}[1]{
      \item[Returned Value:] \\
      \begin{description}
         #1
      \end{description}
      \\
   }

%  sstparameters
   \newcommand{\sstparameters}[1]{
      \item[Parameters:] \\
      \begin{description}
         #1
      \end{description}
      \\
   }

%  sstexamples
   \newcommand{\sstexamples}[1]{
      \item[Examples:] \\
      \begin{description}
         #1
      \end{description}
      \\
   }

%  sstsubsection
   \newcommand{\sstsubsection}[1]{\item[{#1}]}

%  sstexamplesubsection
   \newcommand{\sstexamplesubsection}[2]{\item[{\ssttt #1}] #2}

%  sstnotes
   \newcommand{\sstnotes}[1]{\item[Notes:] #1 }

%  sstdiytopic
   \newcommand{\sstdiytopic}[2]{\item[{#1}] #2 }

%  sstimplementationstatus
   \newcommand{\sstimplementationstatus}[1]{
      \item[Implementation Status:] #1
   }

%  sstitemlist
   \newcommand{\sstitemlist}[1]{
      \begin{itemize}
         #1
      \end{itemize}
      \\
   }
%  sstitem
   \newcommand{\sstitem}{\item}

\end{htmlonly}

%  End of "sst.tex" layout definitions.
%.
% ? End of document specific commands
% -----------------------------------------------------------------------------
%  Title Page.
%  ===========
\renewcommand{\thepage}{\roman{page}}
\begin{document}
\thispagestyle{empty}

%  Latex document header.
%  ======================
\begin{latexonly}
   CCLRC / \textsc{Rutherford Appleton Laboratory} \hfill \textbf{\stardocname}\\
   {\large Particle Physics \& Astronomy Research Council}\\
   {\large Starlink Project\\}
   {\large \stardoccategory\ \stardocnumber}
   \begin{flushright}
   \stardocauthors\\
   \stardocdate
   \end{flushright}
   \vspace{-4mm}
   \rule{\textwidth}{0.5mm}
   \vspace{5mm}
   \begin{center}
   {\Huge\textbf{\stardoctitle \\ [2.5ex]}}
   {\LARGE\textbf{\stardocversion \\ [4ex]}}
   {\Huge\textbf{\stardocmanual}}
   \end{center}
   \vspace{5mm}

% ? Add picture here if required for the LaTeX version.
%   e.g. \includegraphics[scale=0.3]{filename.ps}
% ? End of picture

% ? Heading for abstract if used.
   \vspace{10mm}
   \begin{center}
      {\Large\textbf{Abstract}}
   \end{center}
% ? End of heading for abstract.
\end{latexonly}

%  HTML documentation header.
%  ==========================
\begin{htmlonly}
   \xlabel{}
   \begin{rawhtml} <H1> \end{rawhtml}
      \stardoctitle\\
      \stardocversion\\
      \stardocmanual
   \begin{rawhtml} </H1> <HR> \end{rawhtml}

% ? Add picture here if required for the hypertext version.
%   e.g. \includegraphics[scale=0.7]{filename.ps}
% ? End of picture

   \begin{rawhtml} <P> <I> \end{rawhtml}
   \stardoccategory\ \stardocnumber \\
   \stardocauthors \\
   \stardocdate
   \begin{rawhtml} </I> </P> <H3> \end{rawhtml}
      \htmladdnormallink{CCLRC / Rutherford Appleton Laboratory}
                        {http://www.cclrc.ac.uk} \\
      \htmladdnormallink{Particle Physics \& Astronomy Research Council}
                        {http://www.pparc.ac.uk} \\
   \begin{rawhtml} </H3> <H2> \end{rawhtml}
      \htmladdnormallink{Starlink Project}{http://www.starlink.rl.ac.uk/}
   \begin{rawhtml} </H2> \end{rawhtml}
   \htmladdnormallink{\htmladdimg{source.gif} Retrieve hardcopy}
      {http://www.starlink.rl.ac.uk/cgi-bin/hcserver?\stardocsource}\\

%  HTML document table of contents. 
%  ================================
%  Add table of contents header and a navigation button to return to this 
%  point in the document (this should always go before the abstract \section). 
  \label{stardoccontents}
  \begin{rawhtml} 
    <HR>
    <H2>Contents</H2>
  \end{rawhtml}
  \htmladdtonavigation{\htmlref{\htmladdimg{contents_motif.gif}}
        {stardoccontents}}

% ? New section for abstract if used.
  \section{\xlabel{abstract}Abstract}
% ? End of new section for abstract
\end{htmlonly}

% -----------------------------------------------------------------------------
% ? Document Abstract. (if used)
%  ==================
\stardocabstract
% ? End of document abstract

% -----------------------------------------------------------------------------
% ? Latex document Table of Contents (if used).
%  ===========================================

\newpage
\begin{latexonly}
\begin{center}
{\Large\bf Quick Reference}
\end{center}
\end{latexonly}

\begin{htmlonly}
\section*{Quick Reference}
\end{htmlonly}

\subsection*{NAOS}

\begin{enumerate}

  \item To start NAOS type: {\tt naos}

  \item To find potential guide stars for a list of targets:

  \begin{quote}
   {\tt naomiguidestars} {\it targets-list}
  \end{quote}

   where {\it targets-list}\/ is a text file containing a list of targets.
   Details of the targets are entered one per line, and the format of each
   line is:

  \begin{quote}
   $\alpha$ $\delta$ {\it name}
  \end{quote}

   $\alpha$\/ (Right Ascension) and $\delta$\/ (Declination) are mandatory;
   {\it name}\/ is optional.  These items should be separated by one or
   more spaces.

  \item To find potential guide stars for a single object:

  \begin{quote}
   {\tt naomiremote query usno@eso} $\alpha$ $\delta$ {\it radius}
  \end{quote}

  \item Units and formats are as below.

\end{enumerate} 

\subsection*{GAIA}

\begin{enumerate}

  \item To start GAIA type: {\tt gaia \&}

  \item To retrieve a two-dimensional image:

  \begin{tabular}{l}
   {\sf Data-Servers} \\
   {\sf ~~~ Image Servers} \\
   {\sf ~~~~~~ Digitized Sky at ESO} \\
  \end{tabular}

  \item To retrieve and overlay catalogue objects from the PMM:

  \begin{tabular}{l}
   {\sf Data-Servers} \\
   {\sf ~~~ Catalogs} \\
   {\sf ~~~~~~ USNO at ESO} \\
  \end{tabular}

  \item Units and formats are as below.

\end{enumerate}

\subsection*{Units and formats}

\begin{description}

  \item[$\alpha$] (Right Ascension): sexagesimal hours with a colon
   (`:') as a separator, for example: {\tt 16:52:59}

  \item[$\delta$] (Declination): sexagesimal degrees with a colon
   (`:') as a separator, for example: {\tt 2:24:04}

  \item[{\rm The Right Ascension and Declination}] should be for epoch
   and equinox J2000.

  \item[{\rm Height, width and radius:}] minutes of arc.

\end{description}

\newpage
\section*{Accessing this document}

A hypertext version of this document is available.  To access it on
Starlink systems type:

\begin{quote}
{\tt showme ~ sun235}
\end{quote}

On non-Starlink systems access URL:

\begin{quote}
\htmladdnormallink{
{\tt http://www.starlink.rl.ac.uk/star/docs/sun235.htx/sun235.html}}
{http://www.starlink.rl.ac.uk/star/docs/sun235.htx/sun235.html}
\end{quote}

Paper copies can be obtained from the Starlink document librarian,
who can be contacted as follows.

Postal address: \\
\begin{tabular}{l}
The Document Librarian. Starlink Project, Rutherford Appleton Laboratory, 
  Chilton, \\
DIDCOT, Oxfordshire, OX11 0QX, United Kingdom.                \\
\end{tabular}

% \vspace{3mm}

Electronic mail: {\tt ussc@star.rl.ac.uk}


% \vspace{3mm}

Fax: \\
\begin{tabular}{lr}
from within the United Kingdom: &    01235-445-848 \\
from overseas:                  & +44-1235-445-848 \\
\end{tabular}


\section*{Obtaining assistance}

Though NAOS is a separate package, it is closely related to the CURSA
package for manipulating astronomical catalogues and tables.  Consequently,
bugs in NAOS are reported in the same way as those in CURSA; they should
always be sent to username {\tt cursa@star.rl.ac.uk}.  However, you are
welcome to contact me for advice and assistance.  Details of how to do so
are given below.

Postal address: \\
\begin{tabular}{l}
A.C.~Davenhall.  Institute for Astronomy, Royal Observatory, Blackford
Hill, \\
Edinburgh, EH9 3HJ, United Kingdom.  \\
\end{tabular}

% \vspace{4mm}

Electronic mail: {\tt acd@roe.ac.uk}

% \vspace{4mm}
Fax: \\
\begin{tabular}{lr}
from within the United Kingdom: &    0131-668-8416 \\
from overseas:                  & +44-131-668-8416 \\
\end{tabular}

% ? Latex Copyright Statement
\vspace*{\fill}
\stardoccopyright

\newpage
\section*{Revision history}

\begin{enumerate}

  \item 5th March 2001: Version 1. Original version (ACD).

  \item 5th June 2001: Version 2. Modified for release 1.1 of the NAOS
   package.  The changes in version 1.1 are mostly for compatibility
   with CURSA version 6.3 and are not visible to the user.  However,
   each finding chart now includes the name of the target object as a
   title (ACD).

  \item 24th September 2002: Version 3. Modified for release 1.2 of the
   NAOS package.  In this release the software is unchanged, but there are
   some minor corrections to the documentation (ACD).

\end{enumerate}

\cleardoublepage
\begin{latexonly}
  \setlength{\parskip}{0mm}
  \tableofcontents

  \newpage
  \listoffigures
  \listoftables

  \setlength{\parskip}{\medskipamount}
  \markboth{\stardocname}{\stardocname}
\end{latexonly}
% ? End of Latex document table of contents
% -----------------------------------------------------------------------------

\cleardoublepage
\renewcommand{\thepage}{\arabic{page}}
\setcounter{page}{1}

\section{\xlabel{INTRO}\label{INTRO}Introduction}

Observations made with the \htmladdnormallinkfoot{NAOMI}
{http://www.roe.ac.uk/acdwww/naomi/index.html}
adaptive optics system on the William~Herschel Telescope (WHT) on La Palma
usually require a guide star located close to the target object being
observed.  This manual describes how to find such guide stars.  It
documents the NAOS\footnote{NAOS is the common name for $\zeta$ Puppis.
It is simply the transliteration of the Greek for `ship' and its use
as a star name is relatively recent, originating in a Renaissance text
which referred to the Ptolemaic constellation of Argo in this way.  These
details come from a {\it Short Guide to Modern Star Names and Their
Derivations} by P.~Kunitzsch and T.~Smart\cite{KUNITZSCH86}. (In
modern nomenclature Argo has been dismantled in Puppis, Carina, Vela and
Pyxis.)} (NAOMI Stars) package whose purpose is precisely to find suitable
guide stars and also describes various related items of software.
You will normally prepare a list of guide stars at your home institution
prior to travelling to La Palma to use the WHT.  Indeed, the availability
of suitable guide stars will be one factor which determines whether a
given potential target object can be observed with NAOMI.

The characteristics required of guide stars are that they should be within
a given magnitude range and close to the target object; Section~\ref{CHARACT}
gives the details.  The procedure to find guide stars is to search a
reference catalogue to locate stars close to the target object which also
lie in the permitted magnitude range.  A bright guide star is better than
a faint one.  Also, it is better to have a guide star close to the target
than one further away.  However, if searching the reference catalogue
yields several potential guide stars then it usually best to retain all of
them as possibilities for your observing run.  The acceptable separation
between target object and guide star varies with meteorological conditions
and thus the final choice of guide stars is best made at the telescope.

Note that NAOMI can be used in a `self referencing' mode where the
target object itself takes the r\^{o}le of guide star.  This mode is
not considered here.

The structure of the document is as follows:

\begin{description}

  \item[{\rm Part I}] -- tutorial examples,

  \item[{\rm Part II}] -- reference material.

\end{description}

Part I comprises a series of worked examples corresponding to the
tasks that you will usually follow to find guide stars.  Part II gives
a more complete description of NAOS, and in particular the script
{\tt naomiguidestars}.  
The usual method of working is to have a list of potential targets and
to attempt to find guide stars for them (Section~\ref{LIST}).
Alternatively, for single target objects more flexible approaches
are possible (Sections~\ref{SINGLE} and \ref{GAIA}).  Once suitable guide
stars have been found it is possible to produce finding charts and lists
as aids to identifying them at the telescope.

\subsection{Software}

NAOS comprises two Perl scripts.  The scripts require version \CURSAversion
or higher of CURSA (see \xref{SUN/190}{sun190}{}\cite{SUN190}) to be
installed on your system.  Guide stars can also be found with GAIA (see
\xref{SUN/214}{sun214}{}\cite{SUN214}).  NAOS, CURSA and GAIA are available
on all the versions of Unix supported by Starlink: Compaq Alpha/Tru64,
Sun/Solaris and PC/Linux.

\subsection{\label{CATALOG}Catalogues}

The catalogue currently recommended for providing guide stars is the
PMM astrometric catalogue compiled by the 
\htmladdnormallinkfoot{USNO}{http://www.nofs.navy.mil/}, Flagstaff\cite{PMM}.  
NAOS and GAIA access the version of it provided by ESO remotely via the
Internet.

\section{\xlabel{CHARACT}\label{CHARACT}Characteristics of Guide Stars}

Guide stars need to lie within a limited separation from their target
objects and also to be within a given magnitude range.  The requirements
are summarised in Table~\ref{GSCHARACT}.

\begin{table}[htbp]

\begin{center}
\begin{tabular}{lc}
Criterion                           &  Value \\ \hline
Faint limit (V magnitude)           & 14.5   \\
Maximum separation (seconds of arc) & 85     \\
\end{tabular}
\end{center}

\begin{quote}
\caption[Summary of guide star characteristics]{Summary of characteristics
required for guide stars.  The separation is the angular distance between
the target object and guide star.
\label{GSCHARACT} }
\end{quote}

\end{table}

Though guide stars down to a V magnitude of 14.5 can be used, better
results will be obtained with guide stars brighter than magnitude 12.0.
There is no bright limit to the range of magnitudes which can be observed;
if necessary neutral-density filters can be used.  The maximum permissible
separation between target and guide star varies slightly with wavelength.  
However, for the purpose of selecting guide stars, a maximum separation
of 85" is adequate for observations at all the infra-red wavelengths at
which NAOMI will be used.  Also, there is no simple threshold distance below
which the adaptive optics corrections `work' and above which they do not.
Rather, the compensation for atmospheric effects degrades with increasing
separation.  The value quoted in Table~\ref{GSCHARACT} is adequate to
achieve satisfactory compensation under typical meteorological conditions
above the La~Palma site

An additional consideration is that when NAOMI is providing adaptive
optics corrections for observations made with the infra-red camera
\htmladdnormallinkfoot{INGRID}
{http://www.ing.iac.es/Astronomy/instruments/ingrid/index.html},
flat-field calibrations are usually made by `dithering'.  This process
involves shifting the field of view by about 6" in either direction of
either axis.  Either the guide star or the direction of dithering should be
chosen such that dithering does not take the guide star out of the field of
view.

% - Part I ------------------------------------------------------------
\cleardoublepage
\markboth{\stardocname}{\stardocname}
\part{Tutorial Examples}
\markboth{\stardocname}{\stardocname}
\section{\xlabel{START}\label{START}Getting Started}

\begin{center}
{\bf Note that NAOS \NAOSversion requires CURSA version \CURSAversion or
higher.}
\end{center}

No special quotas or privileges are required to run NAOS, though obviously
you need enough disk space to accommodate the various files that it
creates.  NAOS and GAIA access the USNO PMM catalogue (and others)
via the Internet.  The remote access procedure uses the Web protocols.
Consequently, you must run NAOS and GAIA on a computer connected to the
Internet and configured for Web access.

Both NAOS and CURSA are optional Starlink software items. Before proceeding
you should check with your local site manager whether they are installed
at your site, and if not attempt to persuade him to install them.  If they
are installed at your site the following directories should exist:

\begin{quote}
{\tt /star/bin/naos  \\
/star/bin/cursa}
\end{quote}

The procedure to set up for using NAOS is the same on all the variants
of Unix supported by Starlink. Simply type:

\begin{quote}
{\tt naos}
\end{quote}

The following message should appear:

\begin{quote}
{\tt NAOS commands are now available -- (version 1.2).}
\end{quote}

If it does not, then the probable cause is that either NAOS or CURSA is not
installed correctly at your site; check with your local site manager.

Note that though NAOS requires CURSA to be available it is not necessary
to explicitly set up for running CURSA prior to setting up for NAOS.
Conversely, explicitly setting up for using CURSA prior to setting up NAOS
does no harm.

\subsection{Example files}

The various example files mentioned in this document are available in
directory:

\begin{quote}
{\tt /star/examples/naos}
\end{quote}


\newpage
\section{\xlabel{LIST}\label{LIST}Finding Guide Stars for a List of Targets:
Simple Example}

Usually you will have a list of potential target objects for which you
want to try to find guide stars.  NAOS provides the script {\tt
naomiguidestars} for this task.  It is suitable for lists of up to a few
hundred target objects, but not for very large lists.  Potential guide
stars are selected from the version of the USNO PMM catalogue\cite{PMM}
at ESO.

{\tt naomiguidestars} has a number of modes of operation.  In the simplest,
which is described in this example, it is given a list of target objects
and it produces a report and finding charts for them.  Additional options,
are described in Section~\ref{ADDOPT}, below.  The alternatives include
initially just using {\tt naomiguidestars} to generate tables of
potential guide stars, and subsequently using it to generate finding charts
from these tables locally, without further access to the remote catalogue.
Part II includes a more detailed description of {\tt naomiguidestars}.

Proceed as follows.

\begin{enumerate}

  \item The first step is to create a text file containing the coordinates,
   and optionally names, of your potential targets.  Figure~\ref{TARGETS}
   shows an example targets file.  It is available as file {\tt
   targets.lis}.  The format is more-or-less self-explanatory.  However,
   for completeness a description follows.

\begin{figure}[htbp]

\begin{quote}
\begin{verbatim}
16:52:59  2:24:04   NGC 6240
20:07:55  18:42:57  IRAS 20056+1834
19:34:45  30:30:59  BD +30 3639
10:47:50  12:34:57  NGC 3379
14:40:38  53:30:16
14:19:50 -19:28:26  PKS 1417-19
\end{verbatim}
\end{quote}

\caption{Example file of targets \label{TARGETS} }

\end{figure}

   Each target occupies a single line of the file.  Blank lines are not
   permitted; every line must contain a target.  Each line should
   comprise (in the order given): a Right Ascension, Declination and
   optionally a name.  These items should be separated by one or more
   spaces.  The Right Ascension and Declination should be for epoch and
   equinox J2000.  Their units and formats are:

  \begin{description}

    \item[{\rm Right Ascension:}] sexagesimal hours with a colon (`:')
     as the separator,

    \item[{\rm Declination:}] sexagesimal degrees with a colon (`:')
     as the separator.

  \end{description}

   Any characters after the Declination are considered part of the optional
   object name.  If supplied, the name will be used in the report
   generated.  If you have coordinates for your targets for some epoch
   and equinox other than J2000 then you need to convert them to J2000
   prior to assembling the file.  \xref{COCO}{sun56}{}\cite{SUN56} is
   available for this purpose.  Alternatively, if your potential targets are
   in the form of a \xref{CURSA}{sun190}{}\cite{SUN190} catalogue then CURSA
   application {\tt catcoord} can be used.  If you have a target for which
   you know the name but not the coordinates then it may be possible to look
   up the coordinates automatically; see Section~\ref{NAME} for details.

  \item Set up NAOS.  Type:

  \begin{quote}
   {\tt naos}
  \end{quote}

  \item To find guide stars for all the targets in your list simply type:

  \begin{quote}
   {\tt naomiguidestars} ~ {\it file-of-targets}
  \end{quote}

   For example, to process the example list {\tt targets.lis}:

  \begin{quote}
   {\tt naomiguidestars ~ targets.lis}
  \end{quote}

   If you omit the file name it will be prompted for.  The script will
   work through the list, reporting each target as it is processed.  For
   every target for which some potential guide stars were found a table
   of these stars and a finding chart are produced.  A log file and
   report file are also generated.  The names of these latter two files
   are reported when {\tt naomiguidestars} terminates; they are derived
   from the name of the file of targets.  The various output files are
   described in Section~\ref{OUTFILE}, below.  Note that the report file
   is quite wide and the most convenient way to print it is to use the
   \xref{{\tt a2ps}}{sun184}{} utility\cite{SUN184} with the wide file
   option.  For example, to print report {\tt targets.report} on the
   default (postscript) printer:

  \begin{quote}
   {\tt a2ps -nn -w targets.report | lp}
  \end{quote}

\end{enumerate}

\subsection{Tidying up after misadventure}

If {\tt naomiguidestars} aborts due to some misadventure then various
temporary files might be left in your current directory.  You must delete
these files manually.  You will not be able to run {\tt naomiguidestars}
successfully again until you have done so.  The files which might be
created are:

\begin{quote}
{\tt temp*       \\
catpair.script   \\
catselect.script \\
catview.script}
\end{quote}

where `{\tt *}' is a `wild-card' indicating several files.

\subsection{\label{NAME}Obtaining the coordinates of a named object}

If you have a target for which you know the name but not the coordinates,
it might be possible to obtain them using {\tt naomiremote}.  Type:

\begin{quote}
{\tt naomiremote name simbad\_ns@eso} {\it object-name}
\end{quote}

Object names should be entered without spaces.  The case of letters
(upper or lower) is not significant.  Some examples are:

\begin{quote}
{\tt naomiremote name simbad\_ns@eso ngc6240   \\
naomiremote name simbad\_ns@eso iras20056+1834 \\
naomiremote name simbad\_ns@eso bd+303639 \\
naomiremote name simbad\_ns@eso pks1417-19 \\
naomiremote name simbad\_ns@eso mkn477}
\end{quote}

If the name is recognised then the Right Ascension and Declination of the
object are displayed (the Right Ascension is in hours, the Declination is
in degrees and both are for epoch and equinox J2000).  {\tt naomiremote}
is invoking the version of the
\htmladdnormallinkfoot{SIMBAD}{http://simbad.u-strasbg.fr/Simbad}
name-resolver provided by ESO (the basic SIMBAD is maintained by the
\htmladdnormallink{Centre de Donn\'{e}es astronomiques de Strasbourg}
{http://cdsweb.u-strasbg.fr/CDS.html}, CDS).  The technique only works
if the name is recognised by SIMBAD.

\newpage
\section{\xlabel{ADDOPT}\label{ADDOPT}Finding Guide Stars for a List of
Targets: Additional Options}

By default {\tt naomiguidestars} searches the remote USNO PMM catalogue
for potential guide stars, applies defaults for the faint magnitude limit
and maximum separation corresponding to the values in Table~\ref{GSCHARACT},
and produces a table of potential guide stars and finding chart for each
target object.  However, it is possible to modify its behaviour to suit
your purposes.

\subsection{Specifying the maximum separation}

To specify the maximum permissible separation between the target object and
guide star, simply include the required value as an additional argument on
the command line, after the name of the file of targets.  For example:

\begin{quote}
{\tt naomiguidestars ~ targets.lis ~ 2}
\end{quote}

The value should be given in minutes of arc.

\subsection{Specifying the magnitude limits}

In addition to the faint limit given in Table~\ref{GSCHARACT}, {\tt
naomiguidestars} also applies a default bright magnitude limit of 0 V
magnitude, though this will have little effect in practice.  The bright
and faint magnitude limits are specified by the Unix shell environment
variables {\tt NAOMI\_BRIGHT} and {\tt NAOMI\_FAINT}, respectively.  The
values required should be set prior to invoking {\tt naomiguidestars}.  For
example (assuming you are using either the {\tt c} or {\tt tc} shell):

\begin{quote}
{\tt setenv ~ NAOMI\_BRIGHT ~  8 \\
setenv ~ NAOMI\_FAINT       ~  16}
\end{quote}

to set the limits to V magnitudes 8 and 16, respectively.  The environment
variables used by {\tt naomiguidestars} are fully described in
Section~\ref{ENVIR}, below.  Then simply run {\tt naomiguidestars} as
usual, for example:

\begin{quote}
{\tt naomiguidestars ~ targets.lis}
\end{quote}

\subsection{\label{REPROC_T}Reprocessing existing tables}

By default, for each object in a list of targets, {\tt naomiguidestars}
queries the remote catalogue, creates a table of all the potential guide
stars within the specified separation from the target and produces a
finding chart.  However, it is possible to configure it so that it
reprocesses tables of potential guide stars which were created in a
previous run rather than querying the remote catalogue.  The advantage
of this option is that usually it is quicker to reprocess existing
tables than to re-query the remote catalogue.  This option allows you
to, for example, reprocess existing tables with different magnitude limits.
% {\sf or maximum separation from the target (though obviously the new
% maximum separation must be less than that originally used to generate
% the tables).}

On the first run of {\tt naomiguidestars} you might just generate the
tables of potential guide stars and only create finding charts on subsequent
runs.  Suppose that you were using the example list of targets, {\tt
targets.lis}, then you would proceed as follows.

\begin{enumerate}

  \item First set environment {\tt NAOMI\_KEEPFND} to indicate that
   finding charts are not to be retained:

  \begin{quote}
   {\tt setenv ~ NAOMI\_KEEPFND ~  no}
  \end{quote}

   The environment variables used by {\tt naomiguidestars} are described
   fully in Section~\ref{ENVIR}, below.  Then run {\tt naomiguidestars}
   to generate the tables of potential guide stars for each target:

  \begin{quote}
   {\tt naomiguidestars ~ targets.lis}
  \end{quote}

   In addition the log file {\tt targets.log} and report file {\tt
   targets.report} will be created.

  \item Set the bright and faint magnitude limits to be used in the
   finding charts and specify that finding charts are to be produced:

  \begin{quote}
   {\tt setenv ~ NAOMI\_BRIGHT ~  8 \\
   setenv ~ NAOMI\_FAINT       ~  16 \\
   setenv ~ NAOMI\_KEEPFND ~  yes}
  \end{quote}

   Here the bright and faint limits are set to V magnitudes 8 and 16,
   respectively.  Note that the tables generated by {\tt naomiguidestars}
   include all the objects in the catalogue which lie in the specified
   region of sky, irrespective of their brightness.  Thus, the tables can
   be reprocessed using bright and faint limits which are respectively
   brighter and fainter than the originals, without any danger of losing
   stars.

  \item Finally, re-run {\tt naomiguidestars} using {\it the log file
   produced by the previous run as the list of targets:}

  \begin{quote}
   {\tt naomiguidestars ~ targets.log}
  \end{quote}

   The log file includes the file-names of the tables of potential guide
   stars produced in the previous run and these are used in preference
   to querying the remote catalogue.  The format of the input file for
   {\tt naomiguidestars} is fully described in Section~\ref{INFILE}.
   Also note that when {\tt naomiguidestars} queries the remote catalogue
   and finds that there are no stars within the specified separation
   from the target then no table of potential guide stars is created and
   there is no entry for the target in the log file.

\end{enumerate}


\newpage
\section{\xlabel{SINGLE}\label{SINGLE}Finding Guide Stars for a Single Target}

The simplest way of finding guide stars for a single target is to use
the procedure described in Section~\ref{LIST}, above, for multiple
guide stars, but to simply have a list with one entry.  However, it is also
possible to use script {\tt naomiremote}, which allows greater flexibility.
{\tt naomiremote} simply generates a table of objects within a given
separation of the target.  This catalogue can then be imported into GAIA
(see \xref{SUN/214}{sun214}{}\cite{SUN214}) and overlaid on an image which
has suitable WCS (World Coordinate System) information, or manipulated with
CURSA (see \xref{SUN/190}{sun190}{}\cite{SUN190}), for example to produce a
finding chart.

First set up for running the NAOS.  Type:

\begin{quote}
{\tt naos}
\end{quote}

The simplest sort of query finds all the objects in the catalogue
within a given radius of the target coordinate.  Type:

\begin{quote}
{\tt naomiremote query usno@eso} {\it $\alpha$ $\delta$ radius}
\end{quote}

For example:

\begin{quote}
{\tt naomiremote query usno@eso 16:52:59 ~ 2:24:04 ~ 2}
\end{quote}

All the arguments except the first (`{\tt query}') can be omitted and will
be prompted for.  `{\tt usno@eso}' is the name of the catalogue.  The
Right Ascension ($\alpha$) and Declination ($\delta$) should be for epoch and equinox J2000.
Their units and formats are:

\begin{description}

  \item[{\rm Right Ascension ($\alpha$):}] sexagesimal hours with a colon
   (`:') as the separator,

  \item[{\rm Declination($\delta$):}] sexagesimal degrees with a colon (`:')
   as the separator.

\end{description}

The radius should be specified in minutes of arc.  It is also possible to
retrieve only objects in the given region of sky which are brighter than
a specified V magnitude.  Type:

\begin{quote}
{\tt naomiremote query usno@eso} {\it $\alpha$ $\delta$ radius faint-limit}
\end{quote}

For example:

\begin{quote}
{\tt naomiremote query usno@eso 16:52:59 ~ 2:24:04 ~ 2 ~ 15}
\end{quote}

will return only objects brighter than V magnitude 15.  The magnitude
limit is a `hidden' parameter which is never prompted for.  Hence if it
is to be specified all the arguments must be included on the command
line.  A final option is to return only objects in the given region
which are within a specified V magnitude range.  Type:

\begin{center}
{\tt naomiremote query usno@eso} {\it $\alpha$ $\delta$ radius
bright-limit,faint-limit}
\end{center}

For example:

\begin{quote}
{\tt naomiremote query usno@eso 16:52:59 ~ 2:24:04 ~ 2 ~ 13,15}
\end{quote}

will return only objects in the V magnitude range 13 to 15.  Note that
the magnitude range must be supplied as shown: a single comma with no spaces
separating the bright and faint limits.  Again the magnitude range is a
hidden parameter which is never prompted for and hence all the arguments
must be included on the command line.

\subsection{Plotting finding charts}

A finding chart of the retrieved table can be plotted with CURSA.  Note
that CURSA version \CURSAversion or higher is required.  Use applications:

\begin{quote}
{\tt catchartrn  \\
catchart}
\end{quote}

{\tt catchartrn} requires a `graphics translation file' to prescribe how
stars are to be plotted.  A suitable one is provided as file {\tt
usno.grt}.  It shows stars brighter than magnitude 14.5 as filled circles
and fainter ones as open circles.  If you wish to use a different limit
you will need to edit the file.

\subsection{Using a different catalogue}

\begin{quote}
{\bf Note that though {\tt naomiremote} can use several catalogues,
{\tt naomiguidestars} must use the ESO version of the USNO PMM.}
\end{quote}

The USNO PMM catalogue\cite{PMM} is currently recommended for selecting
guide stars for NAOMI.  However, it is also possible to use the HST {\it
Guide Star Catalog}\/ (GSC).  Simply give `{\tt gsc@lei}' as the name of
the catalogue.  For example:

\begin{quote}
{\tt naomiremote query gsc@lei 16:52:59 ~ 2:24:04 ~ 2}
\end{quote}

Here the version of the GSC available in the 
\htmladdnormallinkfoot{LEDAS}{http://ledas-www.star.le.ac.uk/}
service provided by the Department of Physics and Astronomy, University of
Leicester is being used.  If you wish to use CURSA to plot the retrieved
table as a finding chart then the graphics translation file {\tt usno.grt}
cannot be used because the column names are different.  File {\tt gsc.grt}
is provided instead.

% {\sf Add details of the SuperCOSMOS catalogues when they are finalised.}


\newpage
\section{\xlabel{GAIA}\label{GAIA}Finding Guide Stars with GAIA}

This section describes how to use GAIA (see
\xref{SUN/214}{sun214}{}\cite{SUN214}) to find guide stars.  Before
starting the search for guide stars you need to decide the acceptable
magnitude range and separation from the target object (see
Section~\ref{CHARACT} and Table~\ref{GSCHARACT}).  The recommended values
of, respectively, a faint V magnitude limit of 14.5 and a separation of 85",
are usually adequate and will be adopted here.  The example will assume
that the galaxy NGC 6240 is to be observed.  Its J2000 coordinates are:

\begin{center}
% $\alpha = 16:52:58.8, ~~ \delta = +02:24:04$ \\
$\alpha =$ \hms{16}{52}{58}{8}, ~~ $\delta =$ \dms{+02}{24}{04}{0}
\end{center}

In this example an image of the target object will be retrieved from
the on-line version of the DSS (Digitised Sky Survey) at ESO.  However,
you could substitute your own alternative image, held as a local file,
if you so prefer.  It would, however, have to contain appropriate WCS
(World Coordinate System) header information.  Proceed as follows.

\begin{enumerate}

  \item If the coordinates of your target are not for epoch and equinox
   J2000 then you should convert them to this system.  COCO (see
   SUN/56\cite{SUN56}) is available for this purpose.  Though the target
   coordinates can be converted within GAIA it is probably better to
   convert them prior to starting it.

  \item Ensure that your terminal or workstation is configured to
   receive X-output.

  \item Start GAIA.  Type:

  \begin{quote}
   {\tt gaia \&}
  \end{quote}

   The ampersand (`{\tt \&}') is simply to run GAIA as a detached process,
   so that you can continue to issue Unix commands from the command line.
   A window displaying a start-up message should be displayed, shortly
   followed by the main GAIA window.  If these windows do not appear then
   GAIA is not properly installed at your site; in the first instance
   seek assistance from your site manager.

  \item The first step is to retrieve a two-dimensional image of the
   region of sky containing your target object.  Click on the {\sf
   Data-Servers} menu, located towards the right of the menu bar at the
   top of the main window.  Choose the {\sf Image Servers} option and
   then {\sf Digitized Sky at ESO} from amongst the list of servers (it
   will probably be the only choice).  If the {\sf Digitized Sky at ESO}
   is not amongst the choices offered then see
   Section~\ref{RESTORE_CONFIG}, below.

   A window titled {\sf Digitized Sky at ESO (1)} should appear; it allows
   you to specify the region of sky to be retrieved.

  \begin{itemize}

    \item You can simply enter the name of your target object in the {\sf
     Object Name:} box, if you know it.  GAIA will attempt to resolve
     the name of the object and look up its coordinates.

    \item However, you are more likely to enter the coordinates
     directly.  The units and formats required are:

    \begin{description}

      \item[$\alpha$] (Right Ascension): sexagesimal hours with a colon
       (`:') as a separator,

      \item[$\delta$] (Declination): sexagesimal degrees with a colon
       (`:') as a separator.

    \end{description}

     In the present example the values required are:

    \begin{description}

      \item[$\alpha${\rm :}] {\tt 16:52:59}

      \item[$\delta${\rm :}] {\tt 2:24:04}

    \end{description}

    \item You also need to specify the height and width of the region to
     be retrieved.  These values are specified in minutes of arc.  In the
     present example the radius required is 85".  Doubled (width and
     height are twice the radius) and rounded up the value required is 2.9
     minutes of arc.

  \end{itemize}

   When the values are set click on the {\sf Get Image} button.  After
   a few moments the retrieved image should be displayed in the main
   GAIA window.  Click on {\sf Close} to close the {\sf Digitized Sky
   at ESO (1)} window.

  \item You may wish to save the retrieved image as a file for future
   use.  Click on the {\sf File} menu (the leftmost item in the menu bar
   along the top of the main window) and choose the {\sf Save as\ldots}
   item.  A window allowing you to save the image as a file will appear.
   The file will be written in FITS format.

  \item You may wish to adjust the appearance of the displayed image.
   The most likely items to change are {\sf Colors\ldots} and {\sf
   Magnification} in the {\sf View} menu (second from the left in the
   menu bar along the top of the window).  The present example looks
   better if the magnification is set to {\sf x4}.

  \item To select objects from the PMM catalogue and overlay them on the
   image, again click on the {\sf Data-Servers} menu, towards the right of
   the menu bar at the top of the main window.  Choose the {\sf Catalogs}
   option and then {\sf USNO at ESO} from amongst the list of catalogues
   (it will probably be towards the bottom of the list).  If {\sf USNO at
   ESO} is not amongst the choices offered then see
   Section~\ref{RESTORE_CONFIG}, below.

   A window titled {\sf USNO at ESO (1)} should appear; it allows you
   to retrieve objects from the catalogue.  The central position and
   search radius should be already filled in (the values have been obtained
   from the two-dimensional image).  Set the magnitude limits by putting
   values in the appropriate box.  You may also wish to restrict the radius
   (the suggested maximum separation of 85 seconds of arc is about 1.4
   minutes of arc):

  \begin{center}
  \begin{tabular}{ll}
   {\sf Max Radius:}      & 1.4 \\
   {\sf Brightest (min):} & 0 \\
   {\sf Faintest (max):}  & 14.5 \\
  \end{tabular}
  \end{center}

   Then simply click on the {\sf Search} button (in the bottom left of the
   window).  After a couple of moments the selected objects are listed
   in the {\sf Search Results} box and overlaid on the on the image in
   the main window.  Visual inspection of the overlays should allow you
   to decide which objects will make suitable guide stars.  In the example
   a single object, U0900\_09058900, is selected.  If you apply no magnitude
   limits then nineteen objects will be selected, but most of them will be
   too faint for use with NAOMI.

   A useful feature for identifying objects in the list with the
   corresponding plotted symbol is that if you position the cursor over 
   either a plot symbol or a row in the table and click with the left
   mouse button the corresponding row and plot symbol are highlighted.

   To save the selected objects as a text file click on the {\sf File}
   menu in the {\sf USNO at ESO (1)} window (the leftmost item in its
   menu bar) and choose either {\sf Save as\ldots} or {\sf Print\ldots}.
   In both cases a window will appear which allows you to save the list.
   Note that though both options produce text files they are in different
   formats.  When you have finished click the {\sf Close} button to close
   the {\sf USNO at ESO (1)} window.

  \item To close GAIA click on the {\sf File} menu (the leftmost item in
   the menu bar along the top of the main window) and choose the {\sf
   Exit} option.

\end{enumerate}

\subsection{\label{RESTORE_CONFIG}Restoring catalogue access}

It is possible to configure the set of catalogues and sky surveys which
GAIA can access, though the details are not germane here.  However, if
the {\sf Digitized Sky at ESO} survey or {\sf USNO at ESO} catalogue do
not appear in the lists of {\sf Image Servers} and {\sf Catalogs}
respectively then the most likely reason is that the default list of
catalogues and surveys has been substituted with one which does not
include them.  The simplest way to restore access is to revert to using
the default list.  GAIA configuration files are kept in subdirectory {\tt
.skycat} of your top level directory.  To restore the default catalogues
and surveys you should delete (or rename) file:

\begin{center}
\verb+~/.skycat/skycat.cfg+
\end{center}

and then restart GAIA.

% - Part II ----------------------------------------------------------
\cleardoublepage
\markboth{\stardocname}{\stardocname}
\part{Reference Material}
\markboth{\stardocname}{\stardocname}
\section{\xlabel{INFILE}\label{INFILE}naomiguidestars Input File}

The input file for {\tt naomiguidestars} comprises a list of targets
for which potential guide stars are to be obtained, finding charts produced
\emph{etc}.  The details provided for each target may be in one of
two formats, depending on whether either the remote catalogue is to be
queried to obtain potential guide stars or tables of potential guide
stars obtained in a previous run are to be reprocessed.  The two forms
can be mixed in a single file, but usually will not be.

\subsection{Querying a remote catalogue}

Figure~\ref{TARGETS} shows an example targets file in the format for
querying a remote catalogue.  It is available as file {\tt targets.lis}.
The format is more-or-less self-explanatory.  However, for completeness a
description follows.  Each target occupies a single line of the file.
Blank lines are not permitted; every line must contain a target.  Each
line should    comprise (in the order given): a Right Ascension, Declination
and optionally a name.  These items should be separated by one or more
spaces.  The Right Ascension and Declination should be for epoch and
equinox J2000.  Their units and formats are:

\begin{description}

  \item[{\rm Right Ascension:}] sexagesimal hours with a colon (`:')
   as the separator,

  \item[{\rm Declination:}] sexagesimal degrees with a colon (`:')
   as the separator.

\end{description}

Any characters after the Declination are considered part of the optional
object name.  If supplied, the name will be used in the report generated by
{\tt naomiguidestars}.  If you have coordinates for your targets for some
epoch and equinox other than J2000 then you need to convert them to J2000
prior to assembling the file.  \xref{COCO}{sun56}{}\cite{SUN56} is available
for this purpose.  Alternatively, if your potential targets are in the form
of a \xref{CURSA}{sun190}{}\cite{SUN190} catalogue then CURSA application
{\tt catcoord} can be used.  If you have a target for which you know the
name but not the coordinates then it may be possible to look up the
coordinates automatically; see Section~\ref{NAME} for details.

\subsection{\label{REPROC_R}Reprocessing existing files}

Figure~\ref{RTARGETS} shows an example targets file in the format for
reprocessing tables of potential guide stars produced by a previous
run of {\tt naomiguidestars}.  The format is exactly the same as that for
querying a remote catalogue (above) except that there is an additional
item at the start of each line:

\begin{quote}
{\tt \%}{\it table-file-name}
\end {quote}

where {\it table-file-name}\/ is the file-name of the table of potential
guide stars which is to be reprocessed.  The {\it table-file-name}\/
should be separated from the following Right Ascension by one or more
spaces.  Also note the per-cent sign (`{\tt \%}') that immediately
precedes the {\it table-file-name}.  The log file produced by {\tt
naomiguidestars} is in precisely this format and hence can be used to
reprocess tables of potential guide stars.  In fact, Figure~\ref{RTARGETS}
shows the log file produced when the example list of targets, {\tt
targets.lis}, is used to query the remote catalogue.

\begin{center}
\begin{figure}[htbp]

\begin{verbatim}
%usno_eso_165259_22404.tab  16:52:59  2:24:04  NGC 6240
%usno_eso_200755_184257.tab  20:07:55  18:42:57  IRAS 20056+1834
%usno_eso_193445_303059.tab  19:34:45  30:30:59  BD +30 3639
%usno_eso_144038_533016.tab  14:40:38  53:30:16  Target 5
%usno_eso_141950_192826.tab  14:19:50  -19:28:26  PKS 1417-19
\end{verbatim}

\caption{Example file of targets to be reprocessed \label{RTARGETS} }

\end{figure}
\end{center}


\section{\xlabel{OUTFILE}\label{OUTFILE}naomiguidestars Output Files}

The output files produced by {\tt naomiguidestars} include:

\begin{itemize}

  \item a log file,

  \item a report file

\end{itemize}

and, for each target for which potential guide stars were found:

\begin{itemize}

  \item a table of the potential guide stars,

  \item a finding chart.

\end{itemize}

The log and report files are always produced.  Their file-names are derived
from the file-name of the input list of targets and are reported by {\tt
naomiguidestars}.  Unix environment variables can be set to control whether
the tables of potential guide stars and finding charts are produced (see
Sections~\ref{ADDOPT} and \ref{ENVIR}).

\subsection{Log file}

The log file has file-type `{\tt .log}'.  It contains an entry for every
target for which potential guide stars were found.  The details for each
target occupy a single line and comprise: the name of the table of
potential guide stars, the Right Ascension and Declination of the target
and optionally its name.  The log file is, in fact, in precisely the
right format for reprocessing a set of tables of potential guide
stars without re-querying the remote catalogue (see Sections~\ref{REPROC_T}
and \ref{REPROC_R}).  Table~\ref{RTARGETS} is an example.

\subsection{Report file}

The report file has file-type `{\tt .report}'.  It provides a summary
report for each target and includes an entry for all the targets, including
any for which no potential guide stars were found.  The start of the file
gives the options in effect when {\tt naomiguidestars} was run.

For each target the file names of any table of potential guide stars
and finding chart are given, if appropriate.  Details of the individual
potential guide stars are also given.  The quantities tabulated for each
guide star are listed in Table~\ref{REP_TAB}.  The potential guide stars
are listed in order of increasing separation from the target.  The position
angle is measured east from north.  The sign convention is that stars east
of the target have a positive position angle and those west of it a
negative one.

\begin{table}[htbp]

\begin{center}
\begin{tabular}{ll}
Column         & Description              \\ \hline
{\tt Seq num}  &  Sequence number         \\
{\tt SEPN}     & Separation from the target (seconds of arc) \\
{\tt MAG}      & V magnitude (magnitudes) \\
{\tt RA}       & Right Ascension (J2000)  \\
{\tt DEC}      & Declination (J2000)      \\
{\tt RA\_OFF}  & Offset from target in Right Ascension (seconds of arc) \\
{\tt DEC\_OFF} & Offset from target in Declination (seconds of arc) \\
{\tt PA}       & Position angle (degrees) \\
\end{tabular}
\end{center}

\begin{quote}
\caption[Columns tabulated for potential guide stars]{Columns tabulated for
potential guide stars.  See the text for a discussion of the position angle
\label{REP_TAB} }
\end{quote}

\end{table}

The report file is quite wide.  The most convenient way to print it is
to use the {\tt a2ps} utility (see \xref{SUN/184}{sun184}{}\cite{SUN184})
with the wide file option.  For example, to print report {\tt targets.report}
on the default (postscript) printer:

\begin{quote}
{\tt a2ps -nn -w targets.report | lp}
\end{quote}

\subsection{Tables of potential guide stars}

The tables of potential guide stars have file-type `{\tt .tab}'.  They
are written in the tab-separated table (TST) format, which is documented
in \xref{SSN/75}{ssn75}{}\cite{SSN75}.  They can be read into GAIA
(see \xref{SUN/214}{sun214}{}\cite{SUN214}), where they can be overlaid
on a direct image of the target object, provided that it has suitable WCS
(World Coordinate System) information (see Section~\ref{GAIA}).
Alternatively they can be manipulated with CURSA (see
\xref{SUN/190}{sun190}{}\cite{SUN190}).

Each table contains an entry for every star in the reference catalogue
within the specified separation from the target, irrespective of whether
it falls within any magnitude ranges which may have been set.

\subsection{Finding charts}

The finding charts have file-type `{\tt .ps}' and are in postscript format.
Each finding chart is centred on the position of its target object, and this
position is marked with a `gun-sight' open cross.  The name of the target
object is shown beneath the plot.  Stars bright enough to be used as guide
stars are shown as filled circles.  Fainter stars are shown as open circles.
In both cases the radius of the circle scales with magnitude.  


\section{\xlabel{ENVIR}\label{ENVIR}Environment Variables}

Both {\tt naomiguidestars} and {\tt naomiremote} take some input from Unix
shell environment variables and these variables can be used to control their
behaviour.  You can specify some of these environment variables, but there
are others which you should not change.  The variables used be {\tt
naomiguidestars} which you might want to change are listed in
Table~\ref{ENVARS} and described briefly below.

\begin{table}[htbp]

\begin{center}
\begin{tabular}{rl}
Variable             & Description \\ \hline
{\tt NAOMI\_BRIGHT}  & Bright limit for guide stars (magnitudes) \\
{\tt NAOMI\_FAINT}   & Faint limit for guide stars (magnitudes) \\
{\tt NAOMI\_KEEPTAB} & Keep table of guide stars for each object? \\
{\tt NAOMI\_KEEPFND} & Keep finding chart for each object? \\
{\tt NAOMI\_ECHO}    & Echo the commands issued by {\tt naomiguidestars}? \\
\end{tabular}
\end{center}

\caption{{\tt naomiguidestars} environment variables which the user
may configure
\label{ENVARS} }

\end{table}

\begin{description}

  \item[{\tt NAOMI\_BRIGHT}] is the bright magnitude limit for potential
   guide stars (default: `{\tt 0}').

  \item[{\tt NAOMI\_FAINT}] is the faint magnitude limit for potential
   guide stars (default: `{\tt 14.5}').

  \item[{\tt NAOMI\_KEEPTAB}] specifies whether the tables of potential
   guide stars for each target are kept or deleted.  If it is set to
   `{\tt no}' or `{\tt NO}' they are deleted; otherwise they are kept
   (default: `{\tt yes}').

  \item[{\tt NAOMI\_KEEPFND}] specifies whether the finding charts for
   each target are kept or deleted.  If it is set to
   `{\tt no}' or `{\tt NO}' they are deleted; otherwise they are kept
   (default: `{\tt yes}').

  \item[{\tt NAOMI\_ECHO}] specifies whether internal Unix shell commands
   issued by {\tt naomiguidestars} are echoed to the command terminal.
   Normally you will not wish to see such commands and the option is
   usually only used for de-bugging.  If {\tt NAOMI\_ECHO} is set to
   `{\tt yes}' the commands are echoed; otherwise they are not (default:
   `{\tt no}').

\end{description}

{\tt naomiguidestars} also accesses the following environment variables
which you should not normally change.  They are documented here for
completeness.  The ones with names beginning `{\tt CATREM\_}' are also
used by {\tt naomiremote}.

\begin{description}

  \item[{\tt NAOMI\_REMOTE}] The name and full directory specification
   of {\tt naomiremote}.

  \item[{\tt NAOMI\_CURSA}] The directory where the CURSA applications
   are located.

  \item[{\tt NAOMI\_CATALOG}] The remote catalogue used by {\tt
   naomiguidestars}.  The default value of `{\tt usno\\@eso}' should
   not be changed.

  \item[{\tt CATREM\_URLREADER}] {\tt naomiremote} uses a separate program
   to submit the URL constituting a query to the server and return the
   table of results.  This environment variable specifies the program to
   be used.  Currently the options are: {\tt geturl} (the default), a C
   program supplied with CURSA, the
   \htmladdnormallinkfoot{{\tt lynx}}{http://lynx.browser.org/}
   command-line browser or the Java program {\tt UrlReader}, also supplied
   with CURSA.  To use {\tt geturl}, {\tt CATREM\_URLREADER} should be set
   as follows (for the tc shell):

  \begin{quote}
   {\tt setenv ~ CATREM\_URLREADER ~ "/star/bin/cursa/geturl"}
  \end{quote}

   To use {\tt lynx}, the setting should be:

  \begin{quote}
   {\tt setenv ~ CATREM\_URLREADER ~ "lynx -source"}
  \end{quote}

   Or to use {\tt UrlReader} it should be:

  \begin{quote}
   {\tt setenv ~ CATREM\_URLREADER ~ "java  UrlReader"}
  \end{quote}

   In this last case it is also necessary to set the java environment variable
   {\tt CLASSPATH} so that {\tt UrlReader} is picked up in addition to the
   standard Java classes.  For example, I might set:

  \begin{center}
   {\small \tt setenv~CLASSPATH~/star/bin/cursa:/usr/lib/netscape/java/classes}
  \end{center}

   Note that it is necessary to specify the location of {\tt UrlReader} using
   {\tt CLASSPATH} rather than the corresponding Java command-line option
   because the latter appears not to work on Compaq Alpha/Tru64.  Also
   note that the {\tt lynx} browser appears to be faster than {\tt
   UrlReader}.

  \item[{\tt CATREM\_CONFIG}] specifies the configuration file used;
   this configuration file lists the remote catalogues available.  The
   default configuration file supplied with NAOS includes the catalogues
   that you are likely to use when finding guide stars and I recommend
   that you do not change it.  However, if you insist on doing so,
   configuration files are documented in \xref{SSN/75}{ssn75}{}\cite{SSN75}.

  \item[{\tt CATREM\_MAXOBJ}] is the maximum number of objects which a
   table of potential guide stars is allowed to contain.

  \item[{\tt CATREM\_ECHOURL}] Controls whether the URL representing the
   query submitted to the remote server is also displayed to the user.
   In NAOS {\tt CATREM\_ECHOURL} {\it must}\/ be set to `{\tt no}'.

\end{description}


\section{\xlabel{ACCLOC}\label{ACCLOC}Accessing a Local Version of the
PMM Catalogue}

NAOS remotely accesses a version of the USNO PMM catalogue\cite{PMM} 
provided by ESO.  In principle there is no reason why it should not
access a local version of the catalogue held on a computer at your site,
in order to speed up access and avoid network problems.  However,
installing such a local version is not trivial: it is a programming problem
rather than merely a matter of using existing software.

Firstly you need to obtain a copy of the PMM catalogue.  It is available
on CD-ROMs from the USNO, Flagstaff, Arizona and requires about 6.5 Gbytes
of disk space.

To access such a local copy with NAOS you need to write a local server
which NAOS can communicate with.  Such a server typically comprises: a
program to search the catalogue and a CGI script to handle the
communication with NAOS.  The catalogue data format, which the search
program would need to access, is described in the documentation that
comes with the PMM.  \xref{SSN/75}{ssn75}{}\cite{SSN75} describes how to
write a catalogue server CGI script.  Finally, you will need to add an
entry for the catalogue in the NAOS configuration file; again see SSN/75
for details.  It is possible that you might be able to simply borrow and
install the server used by ESO, but this might include obtaining and
installing commercial software and paying for the associated licences.

An alternative approach is to search a local copy of the PMM using a
local program rather than a server connected to NAOS.  However, again,
you would need to write such a program.  The PMM is too large for
searching it with a general-purpose program, such as CURSA application
{\tt catselect}, to be practical (and the catalogue would also need to
be re-formatted into a format which {\tt catselect} could access).


\begin{htmlonly}
\section{Detailed Description of Scripts}

This section gives detailed descriptions of the NAOS scripts.

% \input{sun235help.tex}
%------------------------------------------------------------------------------
\newpage
\sstroutine{
   NAOMIGUIDESTARS
}{
   Find guide stars for NAOMI targets
}{
   \sstdescription{
      naomiguidestars is a script for finding guide stars for target
      objects which are to be observed with the NAOMI adaptive optics
      system on the William Herschel Telescope (WHT) on La Palma.
      naomiguidestars reads an input file listing the equatorial
      coordinates of the targets and searches the USNO PMM catalogue
      to find suitable guide stars for each target.  The version of
      the PMM maintained by ESO is used and it is accessed remotely
      via the Internet.

      naomiguidestars writes log and report files giving details of the
      potential guide stars.  Optionally, it will also create, for each
      target, a finding chart and table of potential guide stars.  These
      tables of guide stars are written in the tab-separated table (TST)
      format and may be imported into GAIA or CURSA.  naomiguidestars
      can also reprocess the tables, for example, using different limits
      for the brightest and faintest magnitude permitted for a guide star.

      naomiguidestars is fully documented in SUN/235.
   }
   \sstusage{
      naomiguidestars file-of-targets [maximum-separation]
   }
   \sstparameters{
      \sstsubsection{
         file-of-targets (read)
      }{
         A file containing the coordinates (and optionally names) of the
         targets for which  guide stars are to be found.  The format of
         the file is described in SUN/235.
      }
      \sstsubsection{
         maximum-separation (read)
      }{
         The maximum separation, in minutes of arc, permitted between
         the target and guide star (default: 1.4 minutes of arc).
      }
   }
   \sstexamples{
      \sstexamplesubsection{
         naomiguidestars  targets.lis
      }{
         Find potential guide stars for the targets listed in the file
         targets.lis.
      }
      \sstexamplesubsection{
         naomiguidestars  targets.lis  2
      }{
         Find potential guide stars for the targets listed in the file
         targets.lis.  The maximum separation permitted between the
         target and guide star is 2 minutes of arc.
      }
   }
   \sstdiytopic{
      Environment Variables
   }{
      This section briefly describes the environment variables used by
      naomiguidestars which you may set.  naomiguidestars uses additional
      environment variables which you should not set and also it internally
      invokes naomiremote, which uses additional environment variables that
      are not documented here.  See SUN/235 for further details.

      NAOMI\_BRIGHT (read)
         The bright magnitude limit for potential guide stars (default: 0).

      NAOMI\_FAINT (read)
         The faint magnitude limit for potential guide stars (default: 14.5).

      NAOMI\_KEEPTAB (read)
         Specifies whether the tables of potential guide stars for each
         target are kept or deleted.  If it is set to `no{\tt '} or `NO{\tt '} they are
         deleted; otherwise they are kept (default: `yes{\tt '}).

      NAOMI\_KEEPFND (read)
         Specifies whether the finding charts for each target are kept or
         deleted.  If it is set to `no{\tt '} or `NO{\tt '} they are deleted; otherwise
         they are kept (default: `yes{\tt '}).

      NAOMI\_ECHO (read)
         Specifies whether internal Unix shell commands issued by
         naomiguidestars are echoed to the command terminal.  Normally you
         will not wish to see such commands and the option is usually only
         used for de-bugging.  If NAOMI\_ECHO is set to `yes{\tt '} the commands
         are echoed; otherwise they are not (default: `no{\tt '}).
   }
}
\newpage
\sstroutine{
   NAOMIREMOTE
}{
   A simple script to query remote catalogues
}{
   \sstdescription{
      naomiremote is a tool for querying remote astronomical catalogues,
      databases and archives via the Internet.  It allows remote
      catalogues to be queried and the resulting table saved as a local
      file written in the Tab-Separated Table (TST) format.  It also
      provides a number of related auxiliary functions.

      naomiremote has several different modes of usage, each providing a
      different function.  The modes are:

      list    - list the catalogues currently available,

      details - show details of a named catalogue,

      query   - submit a query to a remote catalogue and retrieve the results,

      name    - resolve an object name into coordinates,

      help    - list the modes available.
   }
   \sstusage{
      Arguments for naomiremote can be specified on the command line.
      If arguments other than the first are omitted then they will usually
      be prompted for.  The first argument is the mode of operation and
      its value determines the other arguments which are required.  The
      arguments for the various modes are:

       naomiremote list    server-type

       naomiremote details db-name

       naomiremote query   db-name alpha delta radius additional-condition

       naomiremote name    db-name object-name

       naomiremote help

      The individual arguments are described in the `Arguments{\tt '} section.
      If the mode is omitted then  {\tt '}help{\tt '} mode is assumed.

      In addition to the command-line arguments, naomiremote takes some
      input from Unix shell environment variables and these variables can
      be used to control its behaviour.
   }
   \sstparameters{
      \sstsubsection{
         mode (read)
      }{
         The mode in which naomiremote is to be used.  One of: list,
         details, query, name or help.
      }
      \sstsubsection{
         server-type (read)
      }{
         The server type of the catalogues to be listed in {\tt '}list{\tt '} mode.
         One of: all, catalog, archive, namesvr, imagesvr, local or
         directory.  See SSN/76 for further details.
      }
      \sstsubsection{
         db-name (read)
      }{
         The name of the remote catalogue (or database) to be queried.
      }
      \sstsubsection{
         alpha (read)
      }{
         The Central Right Ascension of the query.  The value should be
         for equinox J2000 and given in sexagesimal hours with a colon
         ({\tt '}:{\tt '}) as the separator.  For example: 12:30:00.
      }
      \sstsubsection{
         delta (read)
      }{
         The central Declination of the query.  The value should be for
         equinox J2000 and given in sexagesimal degrees with a colon ({\tt '}:{\tt '})
         as the separator.  Southern Declinations are negative.  For
         example: 30:23:00.
      }
      \sstsubsection{
         radius (read)
      }{
         The radius of the query in minutes of arc.
      }
      \sstsubsection{
         additional-condition (read)
      }{
         Any additional condition applied to the query.  Catalogues vary
         in which, if any, additional queries they support.  See SSN/76
         for further details.
      }
      \sstsubsection{
         object-name (read)
      }{
         The name of an astronomical object which is to be resolved when
         naomiremote is being used in {\tt '}name{\tt '} mode.  It should be entered
         without embedded spaces.  The case of letters (upper or lower) is
         not usually significant.  That is, case is not significant for
         the usual name resolver database, simbad\_ns@eso, and probably
         will not be significant for other name resolvers.  For example:
         NGC3379.
      }
   }
   \sstexamples{
      \sstexamplesubsection{
         naomiremote
      }{
      }
      \sstexamplesubsection{
         naomiremote help
      }{
         List the various modes in which naomiremote may be used.
      }
      \sstexamplesubsection{
         naomiremote list
      }{
         List all the catalogues and databases in the current configuration
         file.
      }
      \sstexamplesubsection{
         naomiremote list namesvr
      }{
         List all the name servers (that is, databases of server type
         {\tt '}namesvr{\tt '}) in the current configuration file.
      }
      \sstexamplesubsection{
         naomiremote details usno@eso
      }{
         Show details of the USNO PMM astrometric catalogue (whose name
         is {\tt '}usno@eso{\tt '}).
      }
      \sstexamplesubsection{
         naomiremote query usno@eso 12:15:00 30:30:00 10
      }{
         Find all the objects in the USNO PMM which lie within ten minutes
         of arc of Right Ascension 12:15:00.0 (sexagesimal hours) and
         Declination 30:30:00.0 (sexagesimal degrees, both J2000).  The
         objects selected will be saved as a catalogue called
         usno\_eso\_121500\_303000.tab created in your current directory.
         This catalogue will be written in the Tab-Separated Table (TST)
         format.
      }
      \sstexamplesubsection{
         naomiremote query usno@eso 12:15:00 30:30:00 10 14,16
      }{
         Find all the objects in the USNO PMM which lie within ten minutes
         of arc of Right Ascension 12:15:00.0 (sexagesimal hours) and
         Declination 30:30:00.0 (sexagesimal degrees, both J2000) which
         also lie in the magnitude range 14 to 16.
      }
      \sstexamplesubsection{
         naomiremote name simbad\_ns@eso ngc3379
      }{
         Find the equatorial coordinates of the galaxy NGC 3379.  The
         coordinates returned are for equinox J2000.
      }
   }
   \sstdiytopic{
      Environment Variables
   }{
      CATREM\_URLREADER (read)
         naomiremote uses a separate program to submit the URL constituting
         a query to the server and return the table of results.  This
         environment variable specifies the program to be used.  See
         SSN/76 for further details.  (Mandatory.)

      CATREM\_CONFIG (read)
         This environment variable specifies the configuration file to be
         used.  It should be set to either the URL (for a remote file) or
         the local file name, including a directory specification (for a
         local file).  Configuration files mediate the interaction between
         naomiremote and the remote catalogue; see SSN/76 for further
         details.  (Mandatory.)

      CATREM\_MAXOBJ (read)
         The maximum number of objects which the returned table is allowed
         to contain.

      CATREM\_ECHOURL (read)
         This environment controls whether the URL representing the query
         submitted to the remote catalogue is also displayed to the user.
         The default is {\tt '}no{\tt '}; to see the URL set CATREM\_ECHOURL\} to {\tt '}yes{\tt '}.
         Seeing the URL is potentially useful when debugging configuration
         files and remote catalogue servers but is not usually required
         for normal operation.
   }
}
%------------------------------------------------------------------------------

\end{htmlonly}


\section{Acknowledgements}

Thanks to Andy~Longmore, Andy~Vick, Martin~Bly, Mike~Read and Peter~Draper
for useful discussions and/or comments on the draft.  Any mistakes, of
course, are my own.


% References ----------------------------------------------------------

\newpage
% \section{References}

% \input{refs.tex}
%------------------------------------------------------------------------------
% \newpage
\addcontentsline{toc}{section}{References}
\begin{thebibliography}{99}

  \bibitem{SUN184} M.J.~Bly, 13 November 1997,
   \xref{SUN/184.2}{sun184}{}: {\it A2PS --- ASCII to PostScript Converter},
   Starlink.

  \bibitem{SSN75} A.C.~Davenhall, 26 July 2000,
   \xref{SSN/75.1}{ssn75}{}: {\it Writing Catalogue and Image Servers for
   GAIA and CURSA}, Starlink.

  \bibitem{SUN190} A.C.~Davenhall, 4 November 2001,
   \xref{SUN/190.10}{sun190}{}: {\it CURSA --- Catalogue and Table
   Manipulation Applications}, Starlink.

  \bibitem{SUN214} P.W.~Draper, N.~Gray and D.S.~Berry, 13 September 2001,
   \xref{SUN/214.9}{sun214}{}: {\it GAIA --- Graphical Astronomy and
   Image Analysis Tool}, Starlink.

  \bibitem{KUNITZSCH86} P.~Kunitzsch and T.~Smart, 1986, {\it Short
   Guide to Modern Star Names and Their Derivations}\, (Otto Harrassowitz:
   Wiesbaden).

  \bibitem{PMM} D.~Monet, A.~Bird, B.~Canzian, H.~Harris, N.~Reid,
   A.~Rhodes, S.~Sell, H.~Ables, C.~Dahn, H.~Guetter, A.~Henden,
   S.~Leggett, H.~Levison, C.~Luginbuhl, J.~Martini, A.~Monet, J.~Pier, 
   B.~Riepe, R.~Stone, F.~Vrba, R.~Walker,
   1996, {\it USNO-SA1.0}, (U.S. Naval Observatory: Washington DC). 
   See also URL: \htmladdnormallink{
   {\tt http://www.nofs.navy.mil/}}{http://www.nofs.navy.mil/}

  \bibitem{SUN56} P.T.~Wallace, 21 June 1995,
   \xref{SUN/56.10}{sun56}{}: {\it COCO --- Conversion of Celestial
   Coordinates}, Starlink.

\end{thebibliography}
%------------------------------------------------------------------------------

\typeout{  }
\typeout{*****************************************************}
\typeout{  }
\typeout{Reminder: run this document through Latex three times}
\typeout{to resolve the references.}
\typeout{  }
\typeout{*****************************************************}
\typeout{  }

\end{document}
