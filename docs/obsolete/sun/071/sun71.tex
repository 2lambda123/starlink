\documentstyle[11pt]{article} 
\pagestyle{myheadings}

%------------------------------------------------------------------------------
\newcommand{\stardoccategory}  {Starlink User Note}
\newcommand{\stardocinitials}  {SUN}
\newcommand{\stardocnumber}    {71.4}
\newcommand{\stardocauthors}   {C.G.Page \& G.R.Mellor\\University of Leicester}
\newcommand{\stardocdate}      {7th April 1992}
\newcommand{\stardoctitle}     {IKONPAINT -- Ikon and GWM window to
                                             Inkjet Hard-copy}
%------------------------------------------------------------------------------

\newcommand{\stardocname}{\stardocinitials /\stardocnumber}
\renewcommand{\_}{{\tt\char'137}}     % re-centres the underscore
\markright{\stardocname}
\setlength{\textwidth}{160mm}
\setlength{\textheight}{230mm}
\setlength{\topmargin}{-2mm}
\setlength{\oddsidemargin}{0mm}
\setlength{\evensidemargin}{0mm}
\setlength{\parindent}{0mm}
\setlength{\parskip}{\medskipamount}
\setlength{\unitlength}{1mm}

%------------------------------------------------------------------------------
% Add any \newcommand or \newenvironment commands here
%------------------------------------------------------------------------------

\begin{document}
\thispagestyle{empty}
SCIENCE \& ENGINEERING RESEARCH COUNCIL \hfill \stardocname\\
RUTHERFORD APPLETON LABORATORY\\
{\large\bf Starlink Project\\}
{\large\bf \stardoccategory\ \stardocnumber}
\begin{flushright}
\stardocauthors\\
\stardocdate
\end{flushright}
\vspace{-4mm}
\rule{\textwidth}{0.5mm}
\vspace{5mm}
\begin{center}
{\Large\bf \stardoctitle}
\end{center}
\vspace{5mm}

%------------------------------------------------------------------------------
%  Add this part if you want a table of contents
%  \setlength{\parskip}{0mm}
%  \tableofcontents
%  \setlength{\parskip}{\medskipamount}
%  \markright{\stardocname}
%------------------------------------------------------------------------------

\section{Purpose} 

IKONPAINT provides ``push-button'' colour hard-copy from the Ikon screen
or from a GWM window to an inkjet printer.
The DEC {\it Companion Colour Printer} (LJ250/LJ252) and the Hewlett-Packard
{\it Paintjet} are both supported.
These printers normally take 8 $\times$ 11 inch fan-fold paper but
single transparency sheets can be hand-fed. 

\section{Changes in version 4.0}
A number of changes have been made in version 4.0:
\begin{itemize}
\item IKONPAINT will now work on a GWM window as well as on the Ikon.
\item On devices with an overlay plane (such as the Ikon) the user is
asked which plane(s) to send to the output. Choices are the base plane,
the overlay plane or both.
\item The resetting of a DEC printer to DEC mode, rather than remaining
in HP mode, after the program has run, is now under the control of a
hidden ADAM parameter. The default is to remain in HP mode.
\end{itemize}

\section{How to use IKONPAINT}

IKONPAINT simply reads the contents of an Ikon screen or GWM window
left on the display by some other program. It is run with the command 
%******* poss change here ********
\begin{verbatim}
      $ IKONPAINT		    
\end{verbatim}
This invokes a short DCL command procedure which ensures that ADAM has
been started. The program then prompts for the graphics device to be
used. Acceptable names are listed in the file GNS\_IDINAMES.

If the device has an overlay plane (such as the Ikon) the program then
prompts for which of the planes are required for output. If the user
specifies 'BASE', or 'OVERLAY' then only that plane is output. If the
user specifies 'BOTH' then a composite image is achieved by
taking the overlay image and substituting base plane pixels where an
overlay pixel has a zero value (i.e. it is transparent at that point). 

IKONPAINT can copy any rectangular region of the screen: if you enter
YES (the default) to the first prompt it asks you to use the mouse to
set the cursor to the two opposite corners of this area\footnote{The GWM
cursor is moved by bringing the workstation cursor close to it.},
pressing the right-hand mouse button at each point, otherwise it dumps
the entire screen area.
The Ikon screen has about 1024 $\times$ 780 pixels, whereas
the printer page of 8 inches by 10 corresponds to 1440 $\times$ 1800
dots. The program computes the maximum magnification that fits the paper
and prompts you to select this by default or to enter another value. You
may enter zero to get another chance of choosing the rectangular region.
The output quality is slightly better for magnification factors which
are integers, but with a higher magnification than the default you will
either lose some of the right-hand side of the chosen area or cause the
printout to use more than one sheet of paper. 

The program then asks you whether to invert white and black.  If selected
this means that white lines on a dark CRT are printed as black lines on
the white paper: IKONPAINT actually examines all colour indices in the
range 0 to 15 (which are used in many packages for line-drawing) and
inverts any shades found to be close to black or white. This also helps
to reduce ink consumption.  Note that other shades including
intermediate shades of grey are not affected. 

If the transparency mode is chosen then a printer control code is
inserted to make the print-head cover each line twice, as recommended
for transparencies. 

The conversion to HP printer control codes can take a minute or more for
a full screen. A temporary file {\tt INKJET.DAT} is written to {\tt
SYS\$SCRATCH:} which may reach several hundred blocks in length. The
program also gives you an estimate of the cost of consumables used: the
ink is rather expensive, often 30 to 50 pence for a highly-coloured
image on plain paper.  Transparency sheets add at least a pound to this
figure. 

When processing is finished the output file is automatically sent to the
queue {\tt SYS\_INKJET} (and deleted on completion).  A printer on a
serial line will take two to four minutes per plot depending on its
complexity. Messages are sent to your terminal when the job is queued
and when output is complete.  Note that if your inkjet printer is on a
queue with a different name then {\tt SYS\_INKJET} should be redefined
to point to it.   If the queue cannot be found then the file will be
left on {\tt SYS\$SCRATCH} and must be printed using the appropriate
queue with the {\tt /PASSALL} option. 

\section{How it works} 

The program uses IDI routines
to read the current device colour tables and then the contents
of the base and overlay planes for the region selected by the cursor (if
this is under 100 pixels in area it is assumed that the user has made a
mistake and the output is cancelled).

The conversion to HP/PCL printer codes is based on the ordered-dither
algorithm described in standard texts (and used in the GKS laser printer
drivers) but with a specially-invented extension to use chromatic
information. This method shades areas of constant colour evenly while
still allowing sharp features to be resolved. IKONPAINT uses a
4$\times$4 grid of dots with 2-fold redundancy to get a pleasing
diagonal dot pattern, so that the unit cell consists of 8 dots.  Each
dot can be white or black or have any of six basic colours: red, yellow,
green, cyan, blue, or magenta. The RGB coordinates of each of the 256
entries in the Ikon colour table are converted to the HSV scheme.  The
{\it hue} is used to select which basic colours to use and what
proportions of each.  The {\it saturation} and {\it value} determine how
much white or black to admix. In theory this can generate up to 256
shades from a palette of 740, but astronomers quite sensibly tend to use
colour tables in which saturated colours predominate, so that printouts
do not often contain more than 150 different shades. 

\section{Some limitations} 

If the image displayed on the device just uses a simple grey-scale
the results are not very impressive because the printer can show no more
than 9 different grey levels (including black and white).  It is best to
use a colour table containing a good range of hues. 

The printer mixes its three primary ink colours (cyan, magenta, and
yellow) to produce red, green, and blue.  Unfortunately the resulting
blue is very dark, while the yellow is rather pale. This means that some
shades are are poor approximation to those shown on the screen,
though different shades can usually be distinguished adequately on the
hard-copy even if they are not very faithful to the original. 

The DEC version of the printer can be set in either DEC or HP mode by a
button on the control-panel panel (the default is set by a DIP switch at
the back). IKONPAINT uses HP mode exclusively: codes are sent to switch
a DEC printer to HP mode and it is left in this mode unless the hidden
parameter 'DECMO' is set to true on the command line. In the latter case
the printer is restored to DEC mode after use but this
always seems to produce an extra form-feed which wastes paper. If the
printer is also used to take direct screen dumps from a Vaxstation using
the VWS then it will be necessary to reset the printer to DEC mode
manually after using IKONPAINT.  VWS screen dumps are not generally very
satisfactory as all shades are rounded to one of the six basic colors.

Lines so narrow that they are only one dot wide in the output cannot be
represented adequately if their shade requires two basic colours to be
mixed. This must be regarded as an intrinsic limition of the
ordered-dither technique.  Lines drawn in the six basic colours or in
black reproduce correctly at all widths, those at least 4 dots wide will
be reproduced in the correct shade. 

\section{References} 

SUN/57: GNS - Graphics Workstation Name Service.

SUN/65: IDI - Image Display Interface.

SUN/130: GWM - X Graphics Window Manager.

\end{document} 
