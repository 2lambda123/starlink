\documentstyle[11pt]{article}
\pagestyle{myheadings}

%------------------------------------------------------------------------------
\newcommand{\stardoccategory}  {Starlink User Note}
\newcommand{\stardocinitials}  {SUN}
\newcommand{\stardocnumber}    {118.10}
\newcommand{\stardocauthors}   {Martin Bly}
\newcommand{\stardocdate}      {20th September 1992}
\newcommand{\stardoctitle}     {Starlink Software on Unix}
%------------------------------------------------------------------------------

\newcommand{\stardocname}{\stardocinitials /\stardocnumber}
\renewcommand{\_}{{\tt\char'137}}     % re-centres the underscore
\markright{\stardocname}
\setlength{\textwidth}{160mm}
\setlength{\textheight}{230mm}
\setlength{\topmargin}{-2mm}
\setlength{\oddsidemargin}{0mm}
\setlength{\evensidemargin}{0mm}
\setlength{\parindent}{0mm}
\setlength{\parskip}{\medskipamount}
\setlength{\unitlength}{1mm}

%------------------------------------------------------------------------------
% Add any \newcommand or \newenvironment commands here
%------------------------------------------------------------------------------

\begin{document}
\thispagestyle{empty}
SCIENCE \& ENGINEERING RESEARCH COUNCIL \hfill \stardocname\\
RUTHERFORD APPLETON LABORATORY\\
{\large\bf Starlink Project\\}
{\large\bf \stardoccategory\ \stardocnumber}
\begin{flushright}
\stardocauthors\\
\stardocdate
\end{flushright}
\vspace{-4mm}
\rule{\textwidth}{0.5mm}
\vspace{5mm}
\begin{center}
{\Large\bf \stardoctitle}
\end{center}
\vspace{5mm}

\setlength{\parskip}{0mm}
\begin{small}
\tableofcontents
\end{small}
\setlength{\parskip}{\medskipamount}
\markright{\stardocname}

\newpage
\section{Introduction}

This note describes how to use the Starlink software items that have been
implemented on Sun SPARC running SunOS and DEC DECstations running
ULTRIX.  It is intended as a supplement to the existing Starlink user notes
describing these packages and information found in those documents (which may
include information about using the package on Unix) is not duplicated here.

The software available at the time this document was issued is listed
in Tables~\ref{applications}, \ref{libraries} and \ref{utilities}.
The timescale for issuing
revisions of Starlink documents is such that there is likely to be more software
available than is listed here; if so, a brief description can be found in the
file

\begin{quote}
{\tt /star\-/man\-/sun118\_addendum}
\end{quote}

Items marked with \dag\ are optional items and may
not be installed on all systems.

\begin{table}[htb]\caption{Starlink applications packages available on
Unix}\label{applications}
\[\begin{tabular}{|l|l|l|}
\hline
Mnemonic &Name &User Note\\
\hline
CCDPACK\dag & CCD data-reduction Package    & SUN/139 \\
DAOPHOT\dag & Stellar Photometry Package    & SUN/42  \\
KAPPA\dag   & Kernel Application Package    & SUN/95 \\
PHOTOM\dag  & Aperture photometry routines  & SUN/45 \\
PISA\dag    & Position, Intensity and Shape Analysis & SUN/109 \\
SPECDRE\dag & Spectroscopy Data Reduction   & SUN/140 \\
TPOINT\dag  & Telescope pointing analysis   & SUN/100 \\
\hline
\end{tabular}\]
\end{table}

\begin{table}[htb]\caption{Starlink libraries available on
Unix}\label{libraries}
\[\begin{tabular}{|l|l|l|}
\hline
Mnemonic &Name &User Note\\
\hline
ADAM/PAR  & ADAM parameter system                  & SG/4    \\
AGI       & Applications Graphics Interface        & SUN/48  \\
ARY       & Accessing ARRAY Data Structures        & SUN/11  \\
CHR       & Character Handling Routines            & SUN/40  \\
CNF       & C/Fortran Interface library            & SGP/5   \\
EMS       & Error Message Service                  & SSN/4   \\
ERR/MSG   & Message and Error Reporting            & SUN/104 \\
FIO       & Fortran I/O                            & SUN/143 \\
GKS       & Graphical Kernel System                & SUN/83  \\
GNS       & Graphic Workstation Name Service       & SUN/57  \\
GRP       & Routines for managing groups of objects & SUN/150  \\
GWM       & X Graphics Window Manager              & SUN/130 \\
HDS       & Hierarchical Data System               & SUN/92  \\

HELP      & Starlink portable HELP library         & SUN/124 \\
IDI       & Image Display Interface                & SUN/65  \\
NAG\dag   & Mathematical subroutine libraries      & SUN/28  \\
NCAR      & NCAR Graphics Utilities                 & SUN/88  \\
NDF       & Routines to access NDFs                & SUN/33  \\
PGPLOT    & Graphics Subroutine Library            & SUN/15  \\
PRIMDAT   & Processing of Primitive Numerical Data & SUN/39  \\
PSX       & POSIX Interface Routines               & SUN/121 \\
REF       & Handling references to HDS objects     & SUN/31  \\
SGS       & Simple Graphics System                 & SUN/85  \\
SLALIB    & Positional Astronomy Library           & SUN/67  \\
SNX       & Starlink NCAR extensions               & SUN/90 \\
TRANSFORM & Coordinate Transformation Facility     & SUN/61  \\
\hline
\end{tabular}\]
\end{table}

\begin{table}[htb]\caption{Starlink utilities available on
Unix}\label{utilities}
\[\begin{tabular}{|l|l|l|}
\hline
Mnemonic &Name &User Note\\
\hline
A2PS        & Formats ascii files for printing on PostScript printers & \\
ASTROM      & Basic Astrometry              & SUN/5  \\
ASURV       & Astronomical survival statistics & SUN/13  \\
COCO        & Celestial coordinate conversion & SUN/56 \\
EMACS\dag   & GNU EMACS editor              & SUN/34  \\
HELP        & Starlink portable HELP system    & SUN/124 \\
NEWS        & Starlink NEWS system             & SUN/51 \\
RV          & Calculate radial velocity components of observer's velocity &
SUN/78 \\
SPT         & Software Porting Tools        & SUN/111 \\
\TeX\dag    & Document Preparation System   & SUN/9   \\
VMSBACKUP   & VMS BACKUP tape reader        & SUN/151 \\
XDISPLAY    & Easy use of remote X-Windows  & SUN/129 \\
\hline
\end{tabular}\]
\end{table}

At present, the Unix Starlink Software requires that you use either
the C-shell or a compatible shell.

Before using any item of Starlink software on Unix it is necessary
to issue the command:

\begin{quote}

{\tt source /star/etc/login}

\end{quote}

This is analogous to the VMS \verb~@SSC:LOGIN~
and should be added to your {\tt .login} file, {\it after} any of your
own entries in {\tt .login} that set environment variables.

{\tt /star/etc/login} calls another file {\tt /star/etc/cshrc}. This file sets
up the aliases required to run certain applications and utilities. The command

\begin{quote}

{\tt source /star/etc/cshrc}

\end{quote}

should be added to you {\tt .cshrc} file so that the aliases required to
run Starlink software are set up as each C--shell is created; environment
variables set in the login shell by {\tt /star/etc/login} are propagated
to child processes but the aliases set by {\tt /star/etc/cshrc} are not
and hence it is necessary to run {\tt /star/etc/cshrc} for every
process from which Starlink software is to be accessed.

The effect of {\tt source /star/etc/login}
on both the SUN and DECstation are given below.

On a SUN:
\begin{quote}
\begin{enumerate}
\item the directory {\tt /star/bin} is added to the definition of the
environment variable {\tt PATH}.
\item the directory {\tt /star/lib} is added to the definition of the
environment variable {\tt LD\_LIBRARY\_PATH}.
\item the directory {\tt /usr/lib/X11} is added to the definition of the
environment variable {\tt LD\_LIBRARY\_PATH}.
\end{enumerate}
\end{quote}

\goodbreak
On a DECstation:
\begin{quote}
\begin{enumerate}
\item the directory {\tt /star/bin} is added to the definition of the
environment variable {\tt PATH}.
\end{enumerate}
\end{quote}

Other environment variables may also need setting for individual packages.

\subsection{Directory Structure}
All the Starlink software is stored under the directory {\tt /star}.
{\tt /star} contains package directories (such as {\tt /star/starlink})
which contain source code and a set of directories containing executable
programs, subroutine libraries and such like. It is these latter
directories that are normally referred to while using the software.

These directories include:
{\renewcommand{\arraystretch}{1.5}
\[\begin{tabular}{ll}
\tt /star/bin & Executable programs and shell scripts\\
\tt /star/help & Help libraries\\
\tt /star/lib & Subroutine libraries\\
\tt /star/include & Fortran and C include files\\
\tt /star/bin/examples & Example programs\\
\tt /star/man & Manual pages and other documentation
\end{tabular}\]}

As the quantity of software grows, sub-directories of these directories
will be added to accommodate the relevant files of large packages but
there will usually be a package initialization script
to make access to these sub-directories automatic.

Starlink manual pages can be accessed via the command {\tt sman} which
functions in the same way as the Unix command {\tt man}. {\tt man} itself
cannot be used directly to access Starlink manual pages in a portable
manner.

\section{Applications packages}

Most applications packages have an initialisation script that must `sourced' to
define aliases for individual applications. {\tt /star\-/etc\-/login} defines
the package name (or a variant of it in cases where the package name is
also the name of an application) as an alias that `sources' the script so
that for example the command:
\begin{quote}\tt
daostart
\end{quote}
defines the commands {\tt daophot, allstar, daogrey, daoplot} and
{\tt daocurs} to run the corresponding
DAOPHOT applications.

\subsection{Package specific information}

If a package listed in
Table~\ref{applications} does not appear in this section then all information
necessary for using the package on Unix can be found in the Starlink document
referenced in the table.

\subsubsection{CCDPACK}

The CCD data-reduction package CCDPACK is made available
by typing:
\begin{verbatim}
     ccdpack
\end{verbatim}
The differences between
the VMS system and the Unix system are small so the VMS documentation
(SUN/139) can be used for the present; any significant differences are
outlined below.
\begin{itemize}
\item Input wildcards. Any filename expansions as
used by the {\tt ls} command ({\tt ?} for single characters, {\tt*} for any
number, {\tt[a-z]} {\sl etc.}) can be used.

\item Background processes -- these are supported by using the routine
{\tt ccdfork}. This performs essentially the same functions as
the VMS command {\tt CCDBATCH}. All you do is write
the appropriate {\tt CCDPACK} commands into a file, run {\tt CCDSETUP}
interactively then give the routine to {\tt ccdfork} which produces a
script which can be forked (nicely into the background) and
forgotten. The arguments to this routine are:
\begin{verbatim}
       name_of_script_file [output_file_name] [new_adam_directory]
\end{verbatim}
({\tt []} means the argument is optional).
Since the CCDPACK commands in your script file will be
interpreted by the C-shell, special characters should be
suitably escaped or protected (see the {\tt ccdexercise} script in
{\tt /star/bin/ccdpack} for examples of how to do this).

\item Log files. The ICL log file system is not yet available on Unix
so an {\sl ad-hoc\/} formatted system has been implemented. Log files
should now be listed using {\tt cat}.
\end{itemize}

\subsubsection{DAOPHOT}

The current release of DAOPHOT also includes the support programs {\tt daogrey,
daoplot} and {\tt daocurs} in addition to {\tt daophot} and {\tt allstar},
the programs which comprised the initial release.

The package initialization command is
\begin{quote}\tt
daostart
\end{quote}

\subsubsection{KAPPA}

The Kernel Application Package (KAPPA) provides general--purpose applications.
It is the backbone of the software reorganization around the ADAM environment
and it applications integrate with other packages such as PHOTOM, PISA and
CCDPACK.
Further details can be found in SUN/95.

The package initialization command is

\begin{quote}\tt
kappa
\end{quote}


\subsubsection{PHOTOM}

PHOTOM, a Stellar Photometry Package is now available on Unix systems.
The PHOTOM applications {\tt photom, photopt, photgrey} are made available
by typing:
\begin{quote}\tt
photomstart
\end{quote}
See SUN/45 for details.

\subsubsection{PISA}

The acronym PISA stands for Position, Intensity and Shape Analysis, and is
the group name for a package of routines that deal with the location and
parameterisation of objects on an image frame.
PISA is fully documented in SUN/109.

The PISA applications are made available
by typing:
\begin{quote}\tt
pisa
\end{quote}

The significant differences from the VMS version are:
\begin{itemize}
\item The inclusion of a very simple image display routine {\tt pisagrey}.
       The (significant) {\tt pisagrey} parameters are
\begin{verbatim}
       pisagrey in drange device xpixs ypixs
\end{verbatim}
       {\tt drange} are the upper and lower data values and {\tt xpixs} and
       {\tt ypixs}
       the range (in pixels) of the sub-set of the data which is to be
       displayed. This routine will be withdrawn when a generic
       display routine becomes available.

\item Extended use of PGPLOT. The {\tt pisaplot} routine now uses PGPLOT and
       the colours of its line plots are those of the PGPLOT reserved
       pens (1--15); these now correspond to the palette numbers
       used by the {\tt palnum} parameter. This may be changed when the KAPPA
       palette facilities become available.

\item The {\tt pisa2scar} routine is not available. It is not known when
       (or if) this application will become available.

\end{itemize}


A demonstration of the package can be seen using the command
{\tt pisa\_demo} after typing {\tt pisa}).
This demonstration requires an image display device.
(Use an empty directory and delete the files created afterwards.)

\subsubsection{SPECDRE}

SPECDRE is a package for spectroscopy data reduction and analysis.
It fills the gap between Figaro and Kappa. Further details can be
found in SUN/140.

The package initialization command is

\begin{quote}\tt
specdre
\end{quote}


\subsubsection{TPOINT}

TPOINT is an interactive telescope pointing analysis system. It allows
data from pointing tests to be input and fitted to various models. The
residuals from the fits can be displayed in a variety of graphical
formats. Further details can be found in SUN/100.

The TPOINT program can be accessed via the command

\begin{quote}\tt
tpoint
\end{quote}


\section{Subroutine libraries}

\subsection{System software versions}

The libraries have been compiled under the system/operating system version
listed in Table~\ref{version}. Problems may be encountered if you are using a
different version of the operating system or compilers, particularly versions
earlier than the ones listed here.
\begin{table}[ht]\caption{Supported software versions}\label{version}
\[\begin{tabular}{|l|l|l|}
\hline
Operating System &Fortran Compiler &C Compiler\\
\hline
Ultrix 4.2  &DEC Fortran 3.0 &DEC C 1.0  \\
Ultrix 4.1  &DEC Fortran 3.0 &DEC C 1.0  \\
SunOs 4.1.2 &Fortran 1.4     &GNU C 1.35 \\
\hline
\end{tabular}\]
\end{table}
\subsection{Include files}

All ``public'' include files are stored in {\tt /star/include} and have the
same name as the logical name used to access them on the VAX but spelt in lower
case. So, for example,
to include the file {\tt SAE\_PAR} used with most Starlink libraries in a
Fortran program, the appropriate include statement would be:
\begin{quote}\tt
INCLUDE '/star/include/sae\_par'
\end{quote}

If you need to port code from the VAX, the {\tt forconv} program (SUN/111)
can be used to convert the {\tt INCLUDE} statements automatically.

\subsection{Linking}

To link with a Starlink library, include {\tt `{\em xxx}\_link`} on
the {\tt f77} command line, where {\em xxx} is the package name. For example,
to compile and link a program with SGS you might use the command below:
\begin{quote}\tt
f77 sgsprog.f -o sgsprog -L/star/lib `sgs\_link`
\end{quote}
Note that the quotes surrounding {\tt sgs\_link} are grave accents (ASCII code
60 hex).\footnote{The construct {\tt `sgs\_link`} results in the command {\tt
sgs\_link} being executed and the output from the command replacing {\tt
`sgs\_link`} on the command line. In this case {\tt sgs\_link} is a shell script
containing an echo command which outputs the names of the libraries needed to
link an SGS program.} On the Sun {\tt -L/star/lib} can be omitted.

\subsection{Library specific information}

If a library listed in
Table~\ref{libraries} does not appear in this section then all information
necessary for using the library on Unix can be found in the Starlink document
referenced in the table.

\subsubsection{AGI}

The location of the AGI database can be controlled by setting the environment
variable {\tt AGI\_USER} to the desired location. If it is not defined, the
database file is placed in your home directory.

\subsubsection{GKS}
The GKS library on Unix conforms to the ISO standard (sometimes referred to
as  GKS 7.4) so programs currently using GKS on Starlink VAXs may need
modification before they will work under Unix. The changes are minor, involving
just a few subroutine interfaces and the numbers assigned to fonts and fill
area hatch styles; full details can be found in the file {\tt
/star\-/starlink\-/lib\-/gks\-/gks72\_to\_gks74.mem}.

\paragraph{Workstations}
The workstation types available are listed in Table~\ref{wkstn}.

\begin{table}\caption{GKS Workstation Types}\label{wkstn}
\[\begin{tabular}{|l|l|}
\hline
Workstation type & Description \\
\hline
3       & WISS Workstation \\
10      & Metafile Input Workstation \\
12      & CGM Input Workstation \\
15      & UNIRAS GKS Metafile Input Workstation \\
16      & SUN GKS Metafile Input Workstation \\
50      & Metafile Output Workstation \\
52      & CGM Output Workstation \\
201     & Tektronix 4010 Workstation \\
203     & Tektronix 4014 Workstation \\
221     & Tektronix 4107 Workstation \\
700     & CalComp 81 Sheet A4 8-Pens Workstation \\
701     & CalComp 81 Sheet A3 8-Pens Workstation \\
702     & CalComp 81 Roll A4 8-Pens Workstation \\
703     & CalComp 81 Roll A3 8-Pens Workstation \\
704     & CalComp 81 Roll (to Wkstn Window) 8-Pens \\
800     & Cifer 2634 workstation \\
801     & Cifer T5 workstation \\
820     & Standard Pericom Monterey WS \\
821     & RAL mods Pericom Monterey WS \\
825	& Pericom 7800 \\
845	& GraphOn 235 \\
1200    & Printronix P300 Lineprinter (one page LANDscape) \\
1201    & Printronix P300 Lineprinter (Max square LANDscape) \\
1202    & Printronix P300 Lineprinter (A4 LANDscape) \\
1203    & Printronix P300 Lineprinter (one page PORTrait) \\
1204    & Printronix P300 Lineprinter (A4 PORTrait) \\
1720    & DEC VT240 Monochrome \\
1721    & DEC VT241 Colour \\
2600    & Canon LPB (landscape) \\
2601    & Canon LPB (portrait) \\
2610    & Canon LPB \TeX (landscape) \\
2611    & Canon LPB \TeX (portrait) \\
2700    & Postscript single A4 page, Portrait \\
2701    & Postscript single A4 page, Landscape \\
2702    & Postscript single A4 page, Port, EPSF \\
2703    & Postscript single A4 page, Land, EPSF  \\
3800--3    & X Workstations \\
3805--8    & X Overlay Workstations\\
\hline
\end{tabular}\]\end{table}

With the exception of the X windows workstation all workstation drivers use the
connection identifier as a Fortran logical unit number for output; workstations
with input devices also use the next logical unit for input. If the output
logical unit is not already open the driver will search for the environment
variable {\tt GCON{\em nn}}\footnote{{\em nn} is always 2 digits and therefore
must have a leading zero if the connection identifier is less than 10.} where
{\em nn} is the connection identifier and use its value as the file name. If it
is not found the default name of {\tt fort.{\em n}} is used.

To plot on your terminal a value of 5 should be used as the connection
identifier as units 5 and 6 are pre-connected to the terminal by the Fortran
I/O system.

The X windows drivers ignores the connection identifier and always uses the
environment variable {\tt DISPLAY} to connect to the X server.

\paragraph{Error Handler}

The error handling routine ({\tt GERHND}) used by default reports errors via
the Starlink error reporting system. All errors are have the same error value
with the symbolic name {\tt GKS\_\_ERROR} (defined in {\tt
/star/include/gks\_err}). The error channel passed to {\tt GOPKS} is only
written to if an internal error occurs in in the GKS library but the channel is
opened when any error is reported, which results in an empty error file
being created. This can be avoided by using channel 6 as the error channel.

An error handling routine that uses the standard error logging routine and
therefore writes error messages to the GKS error channel can be used instead by
including {\tt /star/lib/gerhnd.std.o} when building a program.

\paragraph{Escapes}
There are two escape function supported, one for enabling a single
user action to trigger
both a locator and a choice input request and one for opening a workstation
without clearing the display surface. They are described in {\tt
/star\-/starlink\-/lib\-/gks\-/ral-escape.mem} and {\tt open-escape.mem}
respectively.

\subsubsection{GNS}

GNS workstation names are case sensitive on Unix.

To test the installation of GNS and display a list of the available
workstations, run the program {\tt /star\-/bin\-/examples\-/gnsrun}.

\subsubsection{IDI}
X windows is the only supported IDI device; description files for the X servers
listed in Table~\ref{servers} are stored in {\tt/star\-/etc}. The default
behaviour of IDI programs is to assume that the X~server is of the same type as
the machine on which the library was compiled ({\it i.e.\/} {\tt xsparc} on
a Sun and {\tt xdecst}
on a DECstation). If this is not appropriate, a different X~server type can be
selected by setting the environment variable {\tt IDI\_XDT} to the name of the
appropriate description file. For example, if you are running a program on a
Sun but displaying the windows on a VAXstation, you should issue the
command below before running the program.
\begin{quote}\tt
setenv IDI\_XDT xvaxes
\end{quote}
\begin{table}\caption{IDI X server types}\label{servers}
\[\begin{tabular}{|l|l|}
\hline
\multicolumn{1}{|c|}{Server} &Name\\
\hline
DECstation       &xdecst \\
Sun Sparcstation &xsparc \\
VAXstation       &xvaxes \\
\hline
\end{tabular}\]
\end{table}

\subsubsection{NAG}

At present the following NAG products are available on the Starlink Unix
systems.


\begin{table}[htb]\caption{NAG packages available on Unix}\label{NAG libraries}
\[\begin{tabular}{|l|c|c|}
\hline
NAG Product & SUN Version & DECstation Version\\
\hline
Double Precision Library & 14 & 15 \\
Single Precision Library & 14 & Not available \\
Graphics Library         & 3  & 3  \\
\hline
\end{tabular}\]
\end{table}

To link with the NAG Double Precision library, include {\tt -lnag}
on the {\tt f77} command line. For example, to compile and link a NAG program
you might type:
\begin{quote}
{\tt f77 nagprog.f -o nagprog -L/star/lib -lnag }
\end{quote}

To link with NAG Single Precision library (available on SUN only),
include {\tt -lnagse}
on the {\tt f77} command line. For example, to compile and link a NAG program
you might type:
\begin{quote}
{\tt f77 nagprog.f -o nagprog -L/star/lib -lnagse }
\end{quote}

To link with the NAG Graphics library,
include {\tt -lnaggl -lnag3gks}
on the {\tt f77} command line. For example, to compile and link a NAG program
you might type:
\begin{quote}
{\tt f77 nagprog.f -o nagprog -L/star/lib -lnaggl -lnag3gks }
\end{quote}

A number of files containing information of interest to users of NAG can be
found in {\tt /star\-/man\-/nag}, the most important of which are:
\begin{quote}
\begin{description}
\item[\tt /star/man/nag/un] -- A user note containing system specific information.
\item[\tt /star/man/nag/news] -- News about any new routines in the current release.
\end{description}
\end{quote}

\subsubsection{PGPLOT}

This is the GKS version of PGPLOT and the list of available devices is the same
as for GKS.

The metafile workstations are supported as output devices. Although a
metafile can be plotted on any device, PGPLOT makes extensive use of GKS
enquiry functions to tailor its behaviour. Because of this, the metafile
must be
``targeted'' at a particular device in order that, for example, the individual
strokes of thick lines are correctly spaced. The target device for a metafile
is specified with {\tt /target={\em workstation\/}} qualifier on the workstation
name passed to {\tt PGBEG}. For example, to create a CGM metafile tailored to
the device characteristics of a Tektronix~4010 terminal the appropriate device
specification would be:
\begin{quote}\tt
cgm\_output/target=tek\_4010
\end{quote}
The default target device is a 300~dpi black and white printer with A4 paper in
landscape orientation.

The example programs normally reside in {\tt /star/bin/examples/pgplot}.
However they may be
removed to save disk space. The source code is in {\tt
/star/starlink/lib/pgplot/examples} and the following commands will build an
example program:

\begin{quote}\tt

cp /star/starlink/lib/pgplot/examples/pgex1.f .

f77 -o pgex1 pgex1.f -L/star/lib `pgplot\_link`

\end{quote}

The link file for linking ADAM applications with PGPLOT is called {\tt
pgp\_link\_adam}.

\subsubsection{SGS}

The default choice device has been changed from device 1 to device zero (the
terminal keyboard) and the choice device string is initialized to all the
letters of the alphabet in alphabetical order (so that A returns 1, B returns
2 {\it etc.})

The behaviour of the input routine {\tt SGS\_REQCU} has been changed slightly:
\begin{itemize}
\item A choice number is only returned if the current choice device is set to 0
or 2 (choice device 2 is also the keyboard but the ASCII code of the key
pressed is returned rather than being mapped via the choice string).
\item If the user selects the GKS ``break'' action for the locator device
the choice number is set to a negative number.
\end{itemize}

All errors are reported via the Starlink error reporting system. If {\tt
SGS\_INIT} is used it is recommended that 6 is used for the error channel (see
the section on GKS error reporting above). {\tt SGS\_OPEN} uses channel 6 when
opening GKS instead of channel 22.


\section{Utilities}


\subsection{A2PS}

{\tt a2ps} is a utility for printing ASCII files on postscript printers in
various formats.
For up-to-date information read the man page, that is, type:
\begin{verbatim}
   sman a2ps
\end{verbatim}

\subsection{Starlink portable HELP system}


HELP was changed a few months after initial release. The routine
HLP\_OUTHLP has been withdrawn and replaced by HLP\_HELP which has
one extra argument. The change was needed to assist multi--platform
applications and to avoid problems with shared libraries.

\subsection{Starlink NEWS system}

The command {\tt news -n} results in a listing of new news items.
This latter command  should be included in system-wide login
files, thus all users should be given notice of new
NEWS on login.
Type {\tt news} to see the headings for all the available news items
or {\tt news} {\sl item} for information on a particular item.

Standard Starlink news items can be included in the system
by copying each one to a separate file in
{\tt /star/local/utility/news}. Note that this system currently only
copes with one level of news. Any hierarchical news items
will just be printed out as one item sequentially.
SUN/51 describes the format of standard Starlink news items.
Note also that a file {\tt\$HOME/.newsstat} is created for each user to
enable checking of new news; this should not be deleted
or modified by the user.


\subsection{\TeX\ }

This release of \TeX\ (C Version 3.14t3) includes \LaTeX\ version 2.09 (SUN/9),
Bib\TeX, Sli\TeX (see the \LaTeX user guide by Leslie Lamport), {\tt xdvi}, the
X-windows dvi file previewer (SUN/77) and {\tt dvips}, which converts a dvi
file into a PostScript file which can be printed on any 300 dpi PostScript
laser printer.

There is no package initialisation command for \TeX, the command {\tt tex}
directly runs the \TeX\ program.

It is not necessary to specify the {\tt .tex} or {\tt .dvi} file name
extensions for the input files when using the {\tt latex}, {\tt dvips} and
{\tt xdvi} commands , for example:

\begin{verbatim}
     latex sun118
     xdvi sun118
     dvips sun118
\end{verbatim}

There are Starlink manual pages on \TeX\ and {\LaTeX}  accessible via the {\tt
sman} command. For example:
\begin{quote}\tt
sman latex
\end{quote}


\subsection{VMSBACKUP}

DECstation users should carefully read SUN/151 before using this
utility so that they can take steps to handle unaligned data access errors
which this program may produce on that platform.

A man page describing how to use the package is available via the command
\begin{verbatim}
   sman vmsbackup
\end{verbatim}

\section{Running ADAM programs}

ADAM programs can be compiled and/or linked with the {\tt alink} command. If
a file has the suffix  {\tt .f}, then {\tt alink} will compile
and link the program. If the file has the suffix of {\tt .o}, or
none at all, then {\tt alink} just links the program. For example,

\begin{quote}\tt

alink test1.f

\end{quote}

will compile and link {\tt test1.f}. The resulting executable file is
called {\tt test1}. The link procedures for linking with additional
subroutine libraries
are called {\tt{\em xxx}\_link\_adam}, where {\em xxx\/} is usually
the library acronym.

To run the program {\tt test1}, you must have an interface file called
{\tt test1.ifl}.
Before running your first ADAM program, you must also create a sub-directory
called {\tt adam} in your top--level directory {\tt i.e.}

\begin{quote}\tt

mkdir \$HOME/adam

\end{quote}

For further details on using the Unix version of ADAM, see SUN/144.

\end{document}

