
\documentstyle{article}
\pagestyle{myheadings}

%------------------------------------------------------------------------------
\newcommand{\stardoccategory}  {Starlink User Note}
\newcommand{\stardocinitials}  {SUN}
\newcommand{\stardocnumber}    {103.2}
\newcommand{\stardocauthors}   {A.C. Davenhall and M. Pettini}
\newcommand{\stardocdate}      {21 June 1990}
\newcommand{\stardoctitle}     {APIG --- Absorption Profiles in the Interstellar Gas}
%------------------------------------------------------------------------------

\newcommand{\stardocname}{\stardocinitials /\stardocnumber}
\markright{\stardocname}
\setlength{\textwidth}{160mm}
\setlength{\textheight}{240mm}
\setlength{\topmargin}{-5mm}
\setlength{\oddsidemargin}{0mm}
\setlength{\evensidemargin}{0mm}
\setlength{\parindent}{0mm}
\setlength{\parskip}{\medskipamount}
\setlength{\unitlength}{1mm}

%------------------------------------------------------------------------------
% Add any \newcommand or \newenvironment commands here
%------------------------------------------------------------------------------

\begin{document}
\thispagestyle{empty}
SCIENCE \& ENGINEERING RESEARCH COUNCIL \hfill \stardocname\\
RUTHERFORD APPLETON LABORATORY\\
{\large\bf Starlink Project\\}
{\large\bf \stardoccategory\ \stardocnumber}
\begin{flushright}
\stardocauthors\\
\stardocdate
\end{flushright}
\vspace{-4mm}
\rule{\textwidth}{0.5mm}
\vspace{5mm}
\begin{center}
{\Large\bf \stardoctitle}
\end{center}
\vspace{5mm}

%------------------------------------------------------------------------------
%  Add this part if you want a table of contents
%  \setlength{\parskip}{0mm}
%  \tableofcontents
%  \setlength{\parskip}{\medskipamount}
%  \markright{\stardocname}
%------------------------------------------------------------------------------

\section{Purpose}

APIG, Absorption Profiles in the Interstellar Gas, is a package to
perform analysis of the interstellar
absorption lines detected in the spectra of Galactic and extragalactic
sources. When observed with sufficiently high resolution, these lines
often show a complex structure with multiple components formed in
absorbing regions at different velocities along the line of sight. The
main purpose of the program is to determine the column density of
absorbers along the line of sight from an interactive analysis of the
equivalent widths and profiles of the observed absorption lines. In
cases where the line profiles are resolved the velocity structure
of the absorbing material is also determined

APIG is not a reduction system for astronomical spectra. It assumes
that the raw spectra have already been reduced to produce wavelength
calibrated spectra normalised relative to the background continuum and
that equivalent widths have been determined.

\section{Further information}

APIG is an optional Starlink software item and may or may not be
available at your site. If it is available, then assuming that you have
followed the normal Starlink login procedure, symbol APIGSETUP
should have been set up. By typing this command, a number of further logical
names are set up. The following documents are available in the directory
APIGDOCS:

\begin{description}

  \item [APIGDOCS:USERGUIDE.TEX] user guide.

  \item [APIGDOCS:QUICKREF.TEX] quick reference guide.

  \item [APIGDOCS:EXAMPLE.LIS] example of an APIG session.

\end{description}

The files for the user guide and the quick reference guide contain
Latex source, whereas the example session is a simple text file suitable
for printing directly. Note that the quick reference guide has to be run
through Latex twice, and the user guide {\em three} times, to correctly
paginate the final documents. Alternatively your site manager might
have paper copies of these documents.

The user guide gives full details of how to use APIG and users new to
the package should work with this document, perhaps using it in
conjunction with the example session. The quick reference guide gives
a concise summary of the package and is intended as an {\it
aide-memoire} for more experienced users.

If APIG is not available at your site then see your site manager about
obtaining a copy.

\end{document}
