\documentstyle[11pt]{article}
\pagestyle{myheadings}

%------------------------------------------------------------------------------
\newcommand{\stardoccategory}  {Starlink User Note}
\newcommand{\stardocinitials}  {SUN}
\newcommand{\stardocnumber}    {28.9}
\newcommand{\stardocauthors}   {M J Bly \& M D Lawden}
\newcommand{\stardocdate}      {24 July 1992}
\newcommand{\stardoctitle}     {NAG --- Mathematical Subroutine Library \\[2ex]
				Mk 15}
%------------------------------------------------------------------------------

\newcommand{\stardocname}{\stardocinitials /\stardocnumber}
\renewcommand{\_}{{\tt\char'137}}     % re-centres the underscore
\markright{\stardocname}
\setlength{\textwidth}{160mm}
\setlength{\textheight}{230mm}
\setlength{\topmargin}{-2mm}
\setlength{\oddsidemargin}{0mm}
\setlength{\evensidemargin}{0mm}
\setlength{\parindent}{0mm}
\setlength{\parskip}{\medskipamount}
\setlength{\unitlength}{1mm}

%------------------------------------------------------------------------------
% Add any \newcommand or \newenvironment commands here
%------------------------------------------------------------------------------

\begin{document}
\thispagestyle{empty}
SCIENCE \& ENGINEERING RESEARCH COUNCIL \hfill \stardocname\\
RUTHERFORD APPLETON LABORATORY\\
{\large\bf Starlink Project\\}
{\large\bf \stardoccategory\ \stardocnumber}
\begin{flushright}
\stardocauthors\\
\stardocdate
\end{flushright}
\vspace{-4mm}
\rule{\textwidth}{0.5mm}
\vspace{5mm}
\begin{center}
{\Large\bf \stardoctitle}
\end{center}
\vspace{5mm}

%------------------------------------------------------------------------------
%  Add this part if you want a table of contents
\setlength{\parskip}{0mm}
\tableofcontents
\setlength{\parskip}{\medskipamount}
\markright{\stardocname}
%------------------------------------------------------------------------------

\newpage
\section{Introduction}

{\bf NOTE:}~{\em This note refers mainly to the VAX implementations of the
NAG Library products}.

The NAG library is the primary general purpose mathematical subroutine library
installed by the Starlink project. It is a commercial package and cannot be
distributed by Starlink to non-Starlink sites. The following software is
scheduled for distribution to all Starlink sites:

\begin{description}

\item [NAG FORTRAN LIBRARY] : This is the standard `NAG Library' and is
documented in the {\em NAG Fortran Library Manual --- Mk 15}.

\begin{itemize}

\item {\bf Single precision library} --- Calculations are carried out in
F\_floating format. This enables programs to occupy less space.

\item {\bf Double precision library} --- This is the default library.
Calculations are carried out in D\_floating format.

\item {\bf G\_floating library} --- This should be used when the numerical
range provided by the double precision library is inadequate. The increased
range is paid for a reduction in precision of 3 bits.

\end{itemize}

\item [NAG GRAPHICS LIBRARY] : This provides graphics facilities and is
documented in the {\em NAG Graphics Library Manual --- Mark 3}, and in SUN/29.

\begin{itemize}

\item {\bf Single precision library} --- Calculations are carried out in
F\_floating format. This enables programs to occupy less space.

\item {\bf Double precision library} --- This is the default library.
Calculations are carried out in D\_floating format.

\end{itemize}

\item [NAG ON-LINE INFORMATION SUPPLEMENT] : This is an on-line help system.

\end{description}

The SERC has a general agreement with NAG Ltd covering the provision of the
NAG Fortran and Graphics Libraries, and the On-line Information Supplement for
various hardware platforms. Starlink participates in the agreement, and all
Starlink owned hardware is covered by it. Hardware owned by SERC outside of
Starlink is also covered, and may run the NAG products---including running the
Starlink implementation of the NAG products which differs slightly from
non-Starlink systems.

As newer Marks/Releases/Versions become available, Starlink will update its
implementations. However, implementations on hardware types within Starlink may
not keep in step.

Details of the three floating point formats available are given in
Table~\ref{tab:types}.

\begin{small}
\begin{table}[h]
\begin{center}
\begin{tabular}{|l|l|l|l|l|}
\hline
Library & Fortran type & Bytes & Range & Precision \\
\hline
Single Precision & F\_floating & 4 & $0.29 * 10^{-38}$ to $1.7 * 10^{38}$ &
 23 bits (7 decimal digits) \\
Double Precision & D\_floating & 8 & $0.29 * 10^{-38}$ to $1.7 * 10^{38}$ &
 55 bits (16 decimal digits) \\
G Precision & G\_floating & 8 & $0.56 * 10^{-308}$ to $0.9 * 10^{308}$ &
 52 bits (15 decimal digits) \\
\hline
\end{tabular}
\caption{Data types available in NAG libraries}
\label{tab:types}
\end{center}
\end{table}
\end{small}

Table~\ref{tab:logs} specifies the logical names that may be used to refer
to the libraries and to the directories in which they are held.

\begin{table}
\begin{center}
\begin{tabular}{|l|l|l|l|}
\hline
                     & Precision & Library            & Directory        \\
\hline
NAG Library          & Single    & {\bf NAGS\_LIB}    & {\bf NAGSDIR}    \\
                     & Double    & {\bf NAG\_LIB}     & {\bf NAGDIR}     \\
                     & G         & {\bf NAGG\_LIB}    & {\bf NAGGDIR}    \\
\hline
Graphical supplement & Single    & {\bf NAGRAFS\_LIB} & {\bf NAGRAFSDIR} \\
                     & Double    & {\bf NAGRAF\_LIB}  & {\bf NAGRAFDIR}  \\
\hline
\end{tabular}
\caption{Logical names defined for Starlink's NAG implementation}
\label{tab:logs}
\end{center}
\end{table}

{\bf NOTE:}~{\em The NAG library implementations are an Optional part of the
Starlink Software Collection. As such, part or all of the NAG software may not
be installed at your site---some sites may only have the single precision
implementations, some may have only the double precision implementations. If
the implementation you require is not installed at your site, please see your
Site Manager.}

This note is meant as a starting point for people wishing to use the NAG
library and graphical supplement. It is no substitute for the documentation
provided by NAG, which should be studied by any serious user of the library.

Besides the on-line help system, some documentation is held on-line in the
directory NAGDIR:

\begin{description}
\begin{description}

\item [CALLED.LIS] : For each routine in the Library (including auxiliaries)
this gives a list of those routines and example programs which call it directly
or indirectly.

\item [CALLS.LIS] : A list of routines called directly or indirectly by each
routine in the Library, and by each example program.

\item [ESSINT.DOC] : Essential Introduction to the NAG Fortran Library.

\item [IN.DOC] : Installers' Note.

\item [NEWS.DOC] : News about changes in Mark 15.

\item [SUMMARY.DOC] : Summary of routines and their purpose.

\item [UN.DOC] : Users' Note.

\end{description}
\end{description}

If you only have the single precision version, similar information is
available in NAGSDIR, but the some file names differ by having an `E' on the
end {\it e.g.} {\tt UNE.DOC} is the single precision Users' Note.

The following three files may also be available at your node:

\begin{description}
\begin{description}

\item [EXPT.TLB] : The Fortran source text of the example programs associated
with the routines, as run on a VAX.

\item [EXPD.TLB] : The data required to run the example programs.

\item [EXPR.TLB] : The results produced on a VAX.

\end{description}
\end{description}

These files are text libraries in which the material for individual programs
are stored as library modules. You may find these libraries useful when
investigating particular programs. All these files derive from the master tape
supplied by NAG Ltd.

The Users' Note supplies implementation-specific detail that augments the
information provided in the Library Manual and Introductory Guide. A copy
should be attached to this note (Appendix . \ref{se:usern}. NAG also supply the
following documents in printed form:

\begin{itemize}
\item NAG Fortran Library Manual
\item NAG Fortran Library Introductory Guide
\item NAG Newsletter (two or three times yearly)
\item NAG General Note: Error Bulletin (partly on microfiche).
\end{itemize}

The Introductory Guide consists of extracts from the Manual. Every Starlink
site has a copy of the first two documents. Your Site Manager will tell you
where they are kept.

The changes to the library introduced in Mark 15 are specified in the section
{\em Mark 15 News} in the Guide and Manual.

\section{How to get going}

If you know what you are doing and just want to know where the library is
stored, the link command needed is:
\begin{verbatim}
      $ LINK <prog>, NAG_LIB/LIB
\end{verbatim}
This will link your program with the double precision library. To link with the
single precision library, the link command is:
\begin{verbatim}
      $ LINK <prog>, NAGS_LIB/LIB
\end{verbatim}
and for the G\_Floating Library, the link command is:
\begin{verbatim}
      $ LINK <prog>, NAGG_LIB/LIB
\end{verbatim}

If you want to use one of the other libraries, use the appropriate logical
name as specified in Table~\ref{tab:logs}.

If you don't know what you are doing, proceed as follows:
\begin{itemize}
\item Read the {\em Essential Introduction to the NAG Fortran Library} --- (see
 below).
\item Read the {\em Users' Note} --- (see below).
\item Try out the test described in the next section.
\end{itemize}

Now you will be ready to decide which routines to use. At this stage, your
primary documentation resources are:

\begin{itemize}
\item NAG Fortran Library Manual
\item NAG Fortran Library Introductory Guide
\end{itemize}

The Introductory Guide is best when deciding which routines to use. The Library
Manual is best when coding calls to NAG routines in your programs. If you still
have trouble after reading all this, look at the Error Bulletin for possible
known faults in the routines you are using. If this fails, contact the Starlink
Software Librarian at RAL (RLVAD::STAR), giving full details of your problem.
He may be able to solve it; otherwise, he will report it to NAG (see section
\ref{se:infodesk}).

Note in particular that when you use the double precision version of the
library, variables specified as {\em\bf real} should be specified in your
program as DOUBLE PRECISION; it is a common error to specify them as REAL.

\section{Selecting a version of the library}

The default version of the NAG library is the double precision version.
If you use the routine names specified in the NAG manual and link your program
with the NAG\_LIB library, this is the version you will use.

If you want to use the single precision version, you must change the routine
names by replacing the `F' at the end of the names by an `E', and
you must link your program with the single precision library NAGS\_LIB.

If you want to use the G precision version, use the same routine names as for
the double precision version, but link your program with the G
precision version of the library NAGG\_LIB and use the `/G\_FLOATING' qualifier
when compiling your program.

\section{Other NAG options and services}

At present, Starlink supplies its users with the following NAG products:

\begin{itemize}
\item The NAG Fortran Library in all precisions.
\item The NAG Graphics Library in all precisions.
\item The NAG On-line Information Supplement.
\item {\bf Genstat}, a general statistics package.
\item {\bf Toolpack/1}, a suite of Fortran 77 program development tools.
\end{itemize}

There are many other options and services offered by NAG which Starlink could
provide if there was a significant user demand. These are listed in the section
`Summary of Services' in the Introductory Guide.

The NAG Users Association holds a meeting once a year. This includes formal
technical presentations, software product demonstrations, workshops, and a user
forum. Starlink does not belong to it at present, but if a NAG enthusiast is
prepared to represent Starlink and attend and report on the meetings, this
could be changed; offers invited.

All these things cost money and effort, so you must make out a good case to the
Project Manager if you want Starlink to make them available.

\section{Software Distribution Procedure}

When a new Mark of the NAG library is released, RAL will receive tapes
containing the new libraries, together with updates for the NAG Fortran Library
Manual and a new Introductory Guide. RAL then distributes the libraries to the
other Starlink sites over the network as part of a Starlink Software Change
(SSC). New documentation updates are ordered independently for the other
Starlink sites. You can find out about recent NAG updates by examining
DOCSDIR:NEWS.LIS or the NAG entry in the Starlink Software Index
ADMINDIR:SSI.LIS.

\section{Costs}

The current costs and availability of NAG products may be obtained from:
\begin{tabbing}
xxxxxxxxxx\=tel: \=\kill
\>NAG Ltd\\
\>Wilkinson House\\
\>Jordan Hill Road\\
\>OXFORD, OX2 8DR\\
\>United Kingdom\\
\\
\>Tel: National (0865) 511245\\
\>\>International +44 865 511245\\
\>Telex: 83354 NAG UK G\\
\>Fax: National (0865) 310139\\
\>\>International +44 865 310139
\end{tabbing}

\section{NAG Response Centre}
\label{se:infodesk}

NAG Ltd have setup a Response Centre to co-ordinate queries about NAG software
products and services. This is a useful service for getting help on obscure
problems with the NAG products. If you are stuck and RLVAD::STAR cannot help,
contact the NAG Response Centre who will be glad to help. The Response Centre
may be contacted at:

\begin{tabbing}
xxxxxxxxxx\=xxxxxxxx\=\kill
\>Phone  \>0865 311744 \\
\>Fax    \>0865 311755 \\
\>Email  \>infodesk @ uk.co.nag   (JANET) \\
\end{tabbing}

\section{Attachments}

The following documents are included as appendices to this paper:
\begin{itemize}
\item Essential Introduction to the NAG Fortran Library (Appendix
\ref{se:essin}).
\item Users' Note: NAG Fortran Library, Mark 15 (Appendix \ref{se:usern}).
\item Mark 15 News (Appendix \ref{se:news}).
\end{itemize}

These documents are on-line in NAGDIR (double precision) or NAGSDIR (single
precision).

\appendix

\newpage
\section{Example --- Test Program}
\label{se:exa}

A simple example of how to use a NAG library subroutine is given in the test
program which is stored as {\tt NAGDIR:TEST\_F01ACF.FOR} (double precision) or
{\tt NAGSDIR:TEST\_F01ACE.FOR} (single precision). The input data files are
{\tt NAGDIR:TEST\_F01ACF.DAT} or  {\tt NAGSDIR:TEST\_F01ACE.DAT}, and example
results are stored in {\tt NAGDIR:TEST\_F01ACF.RES} or
{\tt NAGSDIR:TEST\_F01ACE.RES}.

The source for the double precision example is:

\begin{small}
\begin{verbatim}
      *     F01ACF Example Program Text
      *     Mark 14 Revised.  NAG Copyright 1989.
      *     .. Parameters ..
            INTEGER          NMAX, IA, IB
            PARAMETER        (NMAX=8,IA=NMAX+1,IB=NMAX)
            INTEGER          NIN, NOUT
            PARAMETER        (NIN=5,NOUT=6)
      *     .. Local Scalars ..
            DOUBLE PRECISION EPS
            INTEGER          I, IFAIL, J, L, N
      *     .. Local Arrays ..
            DOUBLE PRECISION A(IA,NMAX), B(IB,NMAX), Z(NMAX)
      *     .. External Functions ..
            DOUBLE PRECISION X02AJF
            EXTERNAL         X02AJF
      *     .. External Subroutines ..
            EXTERNAL         F01ACF
      *     .. Executable Statements ..
            WRITE (NOUT,*) 'F01ACF Example Program Results'
      *     Skip heading in data file
            READ (NIN,*)
            READ (NIN,*) N
            WRITE (NOUT,*)
            IF (N.GT.0 .AND. N.LE.NMAX) THEN
               READ (NIN,*) ((A(I,J),J=1,N),I=1,N)
               EPS = X02AJF()
               IFAIL = 1
      *
               CALL F01ACF(N,EPS,A,IA,B,IB,Z,L,IFAIL)
      *
               IF (IFAIL.NE.0) THEN
                  WRITE (NOUT,99999) 'Error in F01ACF. IFAIL =', IFAIL
               ELSE
                  WRITE (NOUT,*) 'Lower triangle of inverse'
                  DO 20 I = 1, N
                     WRITE (NOUT,99998) (A(I+1,J),J=1,I)
         20       CONTINUE
                  WRITE (NOUT,*)
                  WRITE (NOUT,99997) 'Result found in', L, ' iterations'
               END IF
            ELSE
               WRITE (NOUT,99999) 'N is out of range: N = ', N
            END IF
            STOP
      *
      99999 FORMAT (1X,A,I5)
      99998 FORMAT (1X,8F9.4)
      99997 FORMAT (1X,A,I2,A)
            END
\end{verbatim}
\end{small}

The data file for the double precision example is:

\begin{small}
\begin{verbatim}
      F01ACF Example Program Data
        4
          5    7    6    5
          7   10    8    7
          6    8   10    9
          5    7    9   10
\end{verbatim}
\end{small}

The examples can be executed from your private directory by executing the
statement:
\begin{verbatim}
      $ @NAGDIR:TEST_F01ACF
\end{verbatim}
for double precision or:
\begin{verbatim}
      $ @NAGSDIR:TEST_F01ACE
\end{verbatim}
for single precison. The statements execute a command prcedure which compiles,
links and runs the relevent test program. The double precison command procedure
is:

\begin{small}
\begin{verbatim}
      $!+
      $! test_f01acf.com
      $!
      $! M J Bly 21-JUL-1992
      $!
      $! Test of NAG D/P Fortran F01ACF routine for
      $! installation of NAG D/P Fortran Library Mk15.
      $!
      $! Results displayed on terminal, followed by the example results
      $! (those in NAGDIR:TEST_F01ACF.RES)
      $!-
      $!
      $       set noverify
      $       wso = "write sys$output"
      $       wso "Compiling test program..."
      $       fortran nagdir:test_f01acf
      $       wso "Linking test program..."
      $       link test_f01acf,nag_lib/lib
      $       wso "Defining Data stream..."
      $       define/user for005 nagdir:test_f01acf.dat
      $       wso "Running test program..."
      $       wso " "
      $       run test_f01acf
      $       wso " "
      $       wso "Example results for comparison..."
      $       wso " "
      $       type nagdir:test_f01acf.res
      $       wso " "
      $!
      $       exit
\end{verbatim}
\end{small}

When executed, the example programs will display their test results on your
terminal, followed by the example results file. The double precision example
will produce output thus:

\begin{small}
\begin{verbatim}
      $ @nagdir:test_f01acf
      Compiling test program...
      Linking test program...
      Defining Data stream...
      Running test program...

      F01ACF Example Program Results

      Lower triangle of inverse
        68.0000
       -41.0000  25.0000
       -17.0000  10.0000   5.0000
        10.0000  -6.0000  -3.0000   2.0000

      Result found in 2 iterations
      FORTRAN STOP

      Example results for comparison...

      F01ACF Example Program Results

      Lower triangle of inverse
        68.0000
       -41.0000  25.0000
       -17.0000  10.0000   5.0000
        10.0000  -6.0000  -3.0000   2.0000

      Result found in 2 iterations

      $
\end{verbatim}
\end{small}

If this does not work, it could be that the example data file has the wrong
protection. This should be (RE,RWED,RE,RWE). If this is wrong, ask your Site
Manager to change it.

\newpage
\section{Essential Introduction}
\label{se:essin}

\begin{small}
\begin{verbatim}
      ESSENTIAL INTRODUCTION TO THE NAG FORTRAN LIBRARY

      This document is essential reading for any prospective user of the
      Library.

      Contents:

        1. The Library and its Documentation
           1.1. Structure of the Library
           1.2. Structure of the Manual
           1.3. Supplementary Documentation
           1.4. Marks of the Library
           1.5. Implementations of the Library
           1.6. Precision of the Library
           1.7. Library Identification
           1.8. Fortran Language Standards

        2. Using the Library
           2.1. General Advice
           2.2. Programming Advice
           2.3. Error-handling and the Parameter IFAIL
           2.4. Input/output in the Library
           2.5. Auxiliary Routines

        3. Using the Documentation
           3.1. Using the Manual
           3.2. Structure of Routine Documents
           3.3. Specifications of Parameters
                3.3.1. Classification of parameters
                3.3.2. Constraints and suggested values
                3.3.3. Array parameters
           3.4. Implementation-dependent Information
           3.5. Example Programs and Results
           3.6. Summary for New Users
           3.7. Supplementary Documentation
           3.8. Pre-Mark 14 Routine Documents

        4. Contact between Users and NAG

        5. Further Information

        6. References


      1. The Library and its Documentation

      1.1. Structure of the Library

        The NAG Fortran Library is a comprehensive collection of Fortran 77
        routines for the solution of numerical and statistical problems. The
        word `routine' is used to denote `subroutine' or `function'.

        The Library is divided into chapters, each devoted to a branch of
        numerical analysis or statistics. Each chapter has a three-character
        name and a title, e.g.

          D01 - Quadrature

        Exceptionally two chapters (H and S) have one-character names. (The
        chapters and their names are based on the ACM modified SHARE
        classification index [1].)

        All documented routines in the Library have six-character names,
        beginning with the characters of the chapter name, e.g. D01AJF Note
        that the second and third characters are digits, not letters; e.g. 0
        is the digit zero, not the letter O. The last letter of each routine
        name always appears as `F' in the documentation, but may be changed to
        `E' in some single-precision implementations (see Section 1.6).

      1.2. Structure of the Manual

        The NAG Fortran Library Manual is the principal documentation for the
        NAG Fortran Library. It has the same chapter structure as the Library:
        each chapter of routines in the Library has a corresponding chapter
        (of the same name) in the Manual. The chapters occur in alphanumeric
        order. General introductory documents and indexes are placed at the
        beginning of the Manual.

        Each chapter consists of the following documents:

          Chapter Introduction, e.g. Introduction - D01;
          Chapter Contents, e.g. Contents - D01;

        routine documents, one for each documented routine in the chapter. A
        routine document has the same name as the routine which it describes.
        Within each chapter, routine documents occur in alphanumeric order.
        Exceptionally, some chapters (F06, X01, X02), which contain simple
        support routines, do not have individual routine documents; instead,
        all the routines are described together in the Chapter Introduction.

        The Manual has been typeset using the typesetting system TSSD [6].

      1.3. Supplementary Documentation

        In addition to the full Manual, NAG provides the following alternative
        forms of documentation, which may be more convenient to use, but do
        not contain all the information and advice which is provided in the
        full Manual:

        - the Introductory Guide
        - the Concise Reference
        - the On-line Information Supplement

        All these forms of documentation follow the same basic structure
        (ordering, division into chapters) as the Manual. Further details of
        their contents are given in Section 3.7.

      1.4. Marks of the Library

        Periodically a new Mark of the NAG Fortran Library is released: new
        routines are added, corrections or improvements are made to existing
        routines; occasionally routines are withdrawn if they have been
        superseded by improved routines.

        At each Mark, the documentation of the Library is updated. You must
        make sure that your documentation has been updated to the same Mark as
        the Library software that you are using.

        Marks are numbered, e.g. 12, 13, 14. The current Mark is 15.

        The Library software may be updated between Marks to an intermediate
        maintenance level, in order to incorporate corrections. Maintenance
        levels are indicated by a letter following the Mark number, e.g. 15A,
        15B, and so on (Mark 15 documentation supports all these maintenance
        levels).

      1.5. Implementations of the Library

        The NAG Fortran Library is available on many different computer
        systems. For each distinct system, an implementation of the Library is
        prepared by NAG, e.g. the Cray XMP Unicos implementation. The
        implementation is distributed to sites as a tested compiled library.

        An implementation is usually specific to a range of machines (e.g. the
        DEC VAX range); it may also be specific to a particular operating
        system, Fortran compiler, or compiler option (e.g. scalar or vector
        mode).

        Essentially the same facilities are provided in all implementations of
        the Library, but, because of differences in arithmetic behaviour and
        in the compilation system, routines cannot be expected to give
        identical results on different systems, especially for sensitive
        numerical problems.

        The documentation supports all implementations of the Library, with
        the help of a few simple conventions, and a small amount of
        implementation-dependent information, which is published in a separate
        Users' Note for each implementation (see Section 3.4).

      1.6. Precision of the Library

        The NAG Fortran Library is developed in both single precision and
        double precision versions. REAL variables and arrays in the single
        precision version are replaced by DOUBLE PRECISION variables and arrays
        in the double precision version.

        On most systems only one precision of the Library is available; the
        precision chosen is that which is considered most suitable in general
        for numerical computation (double precision on most systems).

        On some systems both precisions are provided: in this case, the double
        precision routines have names ending in `F' (as in the documentation),
        and the single precision routines have names ending in `E'. Thus in
        DEC VAX/VMS implementations:

          D01AJF is a routine in the double precision implementation;

          D01AJE is the corresponding routine in the single precision
                 implementation.

      1.7. Library Identification

        You must know 4 which implementation, which precision 3 and 4 which
        Mark 3 of the Library you are using or intend to use. To find out
        which implementation, precision and Mark of the Library is available
        at your site, you can run a program which calls the NAG Library
        routine A00AAF (or A00AAE in some single precision implementations).
        This routine has no parameters; it simply outputs text to the NAG
        Library advisory message unit (see Section 2.4). An example of the
        output is:

          *** Start of NAG Library implementation details ***
          Implementation title: DEC VAX range (VAX/VMS)
                     Precision: FORTRAN double precision
                  Product Code: (implementation code FLDVV15D)
                          Mark: 15A
          *** End of NAG Library implementation details ***

        (The implementation code can be ignored, except possibly when
        communicating with NAG; see Section 4.)

      1.8. Fortran Language Standards

        All routines in the Library conform to ANSI Standard Fortran 77 [8],
        except for the use of a double precision complex data type (usually
        COMPLEX*16) in some routines in double precision implementations of
        the Library - there is no provision for this data type in the standard.

        Many of the routines in the Library were originally written to conform
        to the earlier Fortran 66 standard [7], and their calling sequences
        contain a few parameters which are not strictly necessary in Fortran
        77.


      2. Using the Library

      2.1. General Advice

        A NAG Fortran Library routine cannot be guaranteed to return
        meaningful results, irrespective of the data supplied to it. Care and
        thought must be exercised in:

        (a) formulating the problem;
        (b) programming the use of library routines;
        (c) assessing the significance of the results.

        The Foreword to the Manual provides some further discussion of points
        (a) and (c); Sections 2.2 to 2.5 are concerned with (b).

      2.2. Programming Advice

        The NAG Fortran Library and its documentation are designed on the
        assumption that users know how to write a calling program in Fortran.

        When programming a call to a routine, read the routine document
        carefully, especially the description of the Parameters. This states
        clearly which parameters must have values assigned to them on entry to
        the routine, and which return useful values on exit. See Section 3.3
        for further guidance.

        The most common types of programming error in using the Library are:

        - incorrect parameters in a call to a Library routine;
        - calling a double precision implementation of the Library from a
          single precision program, or vice versa.

        Therefore if a call to a Library routine results in an unexpected
        error message from the system (or possibly from within the Library),
        check the following:

          Has the NAG routine been called with the correct number of
          parameters?

          Do the parameters all have the correct type?

          Have all array parameters been dimensioned correctly?

          Is your program in the same precision as the NAG Library routines to
          which your program is being linked?

          Have NAG routine names have been modified - if necessary - as
          described in Sections 1.6 and 2.5?

        Avoid the use of NAG-type names for your own program units or COMMON
        blocks: in general, do not use names which contain a three-character
        NAG chapter name embedded in them; they may clash with the names of an
        auxiliary routine or COMMON block used by the NAG Library.

      2.3. Error handling and the Parameter IFAIL

        NAG Fortran Library routines may detect various kinds of error,
        failure or warning conditions. Such conditions are handled in a
        systematic way by the Library. They fall roughly into three classes:

          (i) an invalid value of a parameter on entry to a routine;
         (ii) a numerical failure during computation (e.g. approximate
              singularity of a matrix, failure of an iteration to converge);
        (iii) a warning that although the computation has been completed, the
              results cannot be guaranteed to be completely reliable.

        All three classes are handled in the same way by the Library, and are
        all referred to here simply as errors.

        The error-handling mechanism uses the parameter IFAIL, which occurs in
        the calling sequence of most NAG Library routines (almost always it is
        the last parameter). IFAIL serves two purposes:

         (i) it allows users to specify what action a Library routine should
             take if it detects an error;
        (ii) it reports the outcome of a call to a Library routine, either
             success (IFAIL = 0) or failure (IFAIL 0, with different values
             indicating different reasons for the failure, as explained in
             Section 6 of the routine document).

        For the first purpose IFAIL must be assigned a value before calling
        the routine; since IFAIL is reset by the routine, it must be passed as
        a variable, not as an integer constant. Allowed values on entry are:

        IFAIL =  0: an error message is output, and execution is terminated
                    (hard failure);

        IFAIL = +1: execution continues without any error message;

        IFAIL = -1: an error message is output, and execution continues. The
                    settings IFAIL = 1 are referred to as `soft failure'.

        The safest choice is to set IFAIL to 0, but this is not always
        convenient: some routines return useful results even though a failure
        (in some cases merely a warning) is indicated. However, if IFAIL is
        set to 1 on entry, it is essential for the program to test its value
        on exit from the routine, and to take appropriate action.

        The specification of IFAIL in Section 5 of a routine document suggests
        a suitable setting of IFAIL for that routine.

        For a full description of the error-handling mechanism, see Chapter
        P01.

      2.4. Input/output in the Library

        Most NAG Library routines perform no output to an external file,
        except possibly to output an error message. All error messages are
        written to a logical error message unit. This unit number (which is
        set by default to 6 in most implementations) can be changed by calling
        the Library routine X04AAF.

        Some NAG Library routines may optionally output their final results,
        or intermediate results to monitor the course of computation. All
        output other than error messages is written to a logical advisory
        message unit. This unit number (which is also set by default to 6 in
        most implementations) can be changed by calling the Library routine
        X04ABF. Although it is logically distinct from the error message unit,
        in practice the two unit numbers may be the same.

        All output from the Library is formatted.

        The only Library routines which perform input from an external file
        are a few option-setting routines in Chapter E04: the unit number is a
        parameter to the routine, and all input is formatted.

        You must ensure that the relevant Fortran unit numbers are associated
        with the desired external files, either by an OPEN statement in your
        calling program, or by operating system commands.

      2.5. Auxiliary Routines

        In addition to those Library routines which are documented and are
        intended to be called by users, the Library also contains many
        auxiliary routines. Details of all the auxiliary routines which are
        called directly or indirectly by any documented NAG Library routine,
        are supplied to sites in machine-readable form with the Library
        software.

        In general, you need not be concerned with them at all, although you
        may be made aware of their existence if, for example, you examine a
        memory map of an executable program which calls NAG routines. The only
        exception is that when calling some NAG Library routines, you may be
        required or allowed to supply the name of an auxiliary routine from
        the NAG Library as an external procedure parameter. The routine
        documents give the necessary details. In such cases, you only need to
        supply the name of the routine; you never need to know details of its
        parameter-list.

        NAG auxiliary routines have names which are similar to the name of the
        documented routine(s) to which they are related, but with last letter
        `Z', `Y', and so on, e.g. D01BAZ is an auxiliary routine called by
        D01BAF. In a single precision implementation in which the names of
        documented routines end in `E', the names of auxiliary routines have
        their first three and last three characters interchanged, e.g. BAZD01
        is an auxiliary routine (corresponding to D01BAZ) called by D01BAE.


      3. Using the Documentation

      3.1. Using the Manual

        The Manual is designed to serve the following functions:

        - to give background information about different areas of numerical
          and statistical computation;
        - to advise on the choice of the most suitable NAG Library routine or
          routines to solve a particular problem;
        - to give all the information needed to call a NAG Library routine
          correctly from a Fortran program, and to assess the results.

        At the beginning of the Manual are some general introductory
        documents. The following may help you to find the chapter, and
        possibly the routine, which you need to solve your problem:

          Contents Summary Mark 15 - a structured list of routines in the
                                     Library, by chapter;
          KWIC Index               - a keyword index to chapters and routines;
          GAMS Index               - a list of NAG routines classified
                                     according to the GAMS scheme.

        Having found a likely chapter or routine, you should read the
        corresponding Chapter Introduction, which gives background information
        about that area of numerical computation, and recommendations on the
        choice of a routine, including indices, tables or decision trees.

        When you have chosen a routine, you must consult the routine document.
        Each routine document is essentially self-contained (it may contain
        references to related documents). It includes a description of the
        method, detailed specifications of each parameter, explanations of
        each error exit, remarks on accuracy, and an example program to
        illustrate the use of the routine.

      3.2. Structure of Routine Documents

        Note: at Mark 14 some changes were made to the style and appearance of
        routine documents. If you have a Manual which contains pre-mark 14
        routine documents, you will find that it contains older documents
        which differ in style, although they contain essentially the same
        information. Sections 3.2, 3.3 and 3.5 of this Essential Introduction
        describe the new-style routine documents. Section 3.8 gives some
        details about the old-style documents.

        All routine documents have the same structure, consisting of nine
        numbered sections:

          1. Purpose
          2. Specification
          3. Description
          4. References
          5. Parameters (see Section 3.3 below)
          6. Error Indicators
          7. Accuracy
          8. Further Comments
          9. Example    (see Section 3.5 below)

        In a few documents, Section 5 also includes a description of printed
        output which may optionally be produced by the routine.

      3.3. Specifications of Parameters

        Section 5 of each routine document contains the specification of the
        parameters, in the order of their appearance in the parameter list.

      3.3.1.  Classification of parameters

        Parameters are classified as follows:

          Input: you must assign values to these parameters on or before entry
          to the routine, and these values are unchanged on exit from the routine.

          Output: you need not assign values to these parameters on or before
          entry to the routine; the routine may assign values to them.

          Input/Output: you must assign values to these parameters on or
          before entry to the routine, and the routine may then change these
          values.

          Workspace: array parameters which are used as workspace by the
          routine. You must supply arrays of the correct type and dimension,
          but you need not be concerned with their contents.

          External Procedure: a subroutine or function which must be supplied
          (e.g. to evaluate an integrand or to print intermediate output).
          Usually it must be supplied as part of your calling program, in
          which case its specification includes full details of its
          parameter-list and specifications of its parameters (all enclosed in
          a box). Its parameters are classified in the same way as those of
          the Library routine, but because you must write the procedure rather
          than call it, the significance of the classification is different:

            Input: values may be supplied on entry, which your procedure must
            not change.

            Output: you may or must assign values to these parameters before
            exit from your procedure.

            Input/Output: values may be supplied on entry, and you may or must
            assign values to them before exit from your procedure.

          Occasionally, as mentioned in Section 2.5, the procedure can be
          supplied from the NAG Library, and then you only need to know its name.

          User Workspace: array parameters which are passed by the Library
          routine to an external procedure parameter. They are not used by the
          routine, but you may use them to pass information between your
          calling program and the external procedure.

          Dummy: a simple variable which is not used by the routine. A
          variable or constant of the correct type must be supplied, but its
          value need not be set. (A dummy parameter is usually a parameter
          which was required by an earlier version of the routine and is
          retained in the parameter-list for compatibility.)

      3.3.2. Constraints and suggested values

        The word `Constraint:' or `Constraints:' in the specification of an
        Input parameter introduces a statement of the range of valid values
        for that parameter, e.g.

          Constraint: N > 0.

        If the routine is called with an invalid value for the parameter (e.g.
        N = 0), the routine will usually take an error exit, returning a
        non-zero value of IFAIL (see Section 2.3).

        In newer documents constraints on parameters of type CHARACTER only
        list uppercase alphabetic characters, e.g.

          Constraint: STRING = 'A' or 'B'.

        In practice all routines with CHARACTER parameters will permit the use
        of lower case characters.

        The phrase `Suggested Value:' introduces a suggestion for a reasonable
        initial setting for an Input parameter (e.g. accuracy or maximum
        number of iterations) in case you are unsure what value to use; you
        should be prepared to use a different setting if the suggested value
        turns out to be unsuitable for your problem.

      3.3.3. Array parameters

        Most array parameters have dimensions which depend on the size of the
        problem. In Fortran terminology they have `adjustable dimensions': the
        dimensions occurring in their declarations are integer variables which
        are also parameters of the Library routine.

        For example, a Library routine might have the specification:

          SUBROUTINE <name> (M, N, A, B, LDB)
          INTEGER       M, N, A(N), B(LDB,N), LDB

        For a one-dimensional array parameter, such as A in this example, the
        specification would begin:

          A(N) - INTEGER array.

        You must ensure that the dimension of the array, as declared in your
        calling (sub)program, is at least as large as the value you supply for
        N. It may be larger; but the routine uses only the first N elements.

        For a two-dimensional array parameter, such as B in the example, the
        specification might be:

          B(LDB,N) - INTEGER array.

            On entry: the m by n matrix B.

        and the parameter LDB might be described as follows:

          LDB - INTEGER.                                                 Input

            On entry: the first dimension of the array B as declared in the
            (sub)program from which <name> is called.

            Constraint: LDB  >= M.

        You must supply the first dimension of the array B, as declared in
        your calling (sub)program, through the parameter LDB, even though the
        number of rows actually used by the routine is determined by the
        parameter M. You must ensure that the first dimension of the array is
        at least as large as the value you supply for M. The extra parameter
        LDB is needed because Fortran does not allow information about the
        dimensions of array parameters to be passed automatically to a routine.

        You must also ensure that the second dimension of the array, as
        declared in your calling (sub)program, is at least as large as the
        value you supply for N. It may be larger, but the routine only uses
        the first N columns.

        A program to call the hypothetical routine used as an example in this
        section might include the statements:

          INTEGER AA(100), BB(100,50)
          LDB = 100
          .
          .
          .
          M = 80
          N = 20
          CALL <name>(M,N,AA,BB,LDB)

        Fortran requires that the dimensions which occur in array
        declarations, must be greater than zero. Many NAG routines are
        designed so that they can be called with a parameter like N in the
        above example set to 0 (in which case they would usually exit
        immediately without doing anything). If so, the declarations in the
        Library routine would use the `assumed size' array dimension, and
        would be given as:

          INTEGER        M, N, A(*), B(LDB,*), LDB

        However, the original declaration of an array in your calling program
        must always have constant dimensions, greater than or equal to 1.

        Consult an expert or a textbook on Fortran, if you have difficulty in
        calling NAG routines with array parameters.

      3.4. Implementation-dependent Information

        In order to support all implementations of the Library, the Manual has
        adopted a convention of using bold italics to distinguish terms which
        have different interpretations in different implementations.

        The most important bold italicised terms are the following; their
        interpretation depends on whether the implementation is in single
        precision or double precision:

        real                  means  REAL              or  DOUBLE PRECISION
        complex               means  COMPLEX           or  COMPLEX*16
                                                           (or equivalent)
        basic precision       means  single precision  or  double precision
        additional precision  means  double precision  or  quadruple precision

        Another important bold italicised term is which denotes the relative
        precision to which real floating-point numbers are stored in the
        computer, e.g. in an implementation with approximately 16 decimal
        digits of precision, has a value of approximately10**(-16).

        The precise value of is given by the function X02AJF. Other functions
        in Chapter X02 return the values of other implementation-dependent
        constants, such as the overflow threshold, or the largest
        representable integer. Refer to the X02 Chapter Introduction for more
        details.

        For each implementation of the Library, a separate Users' Note is
        published. This is a short document, revised at each Mark. At most
        installations it is available in machine-readable form. It gives any
        necessary additional information which applies specifically to that
        implementation, in particular:

        - the interpretation of bold italicised terms;
        - the values returned by X02 routines;
        - the default unit numbers for output (see Section 2.4);
        - details of name changes for Library routines (see Sections 1.6 and 2.5).

      3.5. Example Programs and Results

        The example program in the last section of each routine document
        illustrates a simple call of the routine. The programs are designed so
        that they can fairly easily be modified, and so serve as the basis for
        a simple program to solve a user's own problem.

        Bold italicised terms are used in the printed text of the example
        program, to denote precision-dependent features in the code. The
        correct Fortran code must be substituted before the program can be
        run. In addition to the terms real and complex which were explained in
        Section 3.4, the following are used in the example programs:

        Intrinsic Functions: real  means REAL  or DBLE (see Note below)
                             imag  means AIMAG or DIMAG
                             cmplx means CMPLX or DCMPLX
                             conjg means CONJG or DCONJG
        Edit Descriptor:     e     means E     or D    (in FORMAT statements)
        Exponent Letter:     e     means E     or D    (in constants)

        Note: in some implementations, the intrinsic function real with a
        complex argument must be interpreted as DREAL rather than DBLE.

        For each implementation of the Library, NAG distributes the example
        programs in machine-readable form, with all necessary modifications
        already applied. Many sites make the programs accessible in this form
        to users.

        Note that the results from running the example programs may not be
        identical in all implementations, and may not agree exactly with the
        results which are printed in the Manual and which were obtained from
        an Apollo DN3000 double precision implementation (with approximately
        16 digits of precision).

        The Users' Note for your implementation will mention any special
        changes which need to be made to the example programs, and any
        significant differences in the results.

      3.6. Summary for New Users

        If you are unfamiliar with the NAG Library and are thinking of using a
        routine from it, please follow these instructions:

        (a) read the whole of the Essential Introduction;
        (b) consult the Contents Summary or KWIC Index to choose an
            appropriate chapter or routine;
        (c) read the relevant Chapter Introduction;
        (d) choose a routine, and read the routine document. If the routine
            does not after all meet your needs, return to steps (b) or (c);
        (e) read the Users' Note for your implementation;
        (f) consult local documentation, which should be provided by your
            local support staff, about access to the NAG Library on your
            computing system.

        You should now be in a position to include a call to the routine in a
        program, and to attempt to run it. You may of course need to refer
        back to the relevant documentation in the case of difficulties, for
        advice on assessment of results, and so on.

        As you become familiar with the Library, some of steps (a) to (f) can
        be omitted, but it is always essential to:

        - be familiar with the Chapter Introduction;
        - read the routine document;
        - be aware of the Users' Note for your implementation.

      3.7. Supplementary Documentation

        The Introductory Guide contains all the general introductory
        documents, indexes, chapter introductions and chapter contents, from
        the full Manual. It thus gives background information, and advice on
        choosing the most suitable NAG Library routine; but it does not
        contain any detailed specifications of the routines.

        The Concise Reference contains details of the parameter-lists of all
        routines in the Library, and very terse (usually one-line) summaries
        of the specification of each parameter, and of the meaning of each
        error-exit. It is not an adequate substitute for the documentation in
        the full Manual, especially if you are trying to use a routine for the
        first time, but it is intended to be a compact and convenient
        memory-aid for users who have gained some familiarity with the Library.

        The On-line Information Supplement is a machine-based `Help' system,
        which describes the subject areas covered by the Library, advises on
        the choice of routines, and gives essential programming details for
        each documented routine. It contains a machine-readable version of
        Sections 1, 2, 5 and 6 of each routine document.

        The On-Line Information Supplement is a separate product from the
        Library: consult local documentation to see if it is available at your
        site.

      3.8. Pre-Mark 14 Routine Documents

        You need only read this section if you have an updated Manual, which
        contains pre-mark 14 documents.

        You will find that older routine documents appear in a somewhat
        different style, or even several styles if your Manual dates back to
        Mark 7, say. The following are the most important differences between
        the earlier styles and the new style introduced at Mark 14:

        - before Mark 12, routine documents had 13 sections: the extra
          sections have either been dropped or merged with the present
          Section 8 (Further Comments);

        - in Section 5, parameters were not classified as Input, Output and so
          on; the phrase `Unchanged on exit' was used to indicate an input
          parameter;

        - example programs were revised at Mark 12 and again at Mark 14, to
          take advantage of features of Fortran 77: the programs printed in
          older documents do not correspond exactly with those which are now
          distributed to sites in machine-readable form;

        - before Mark 12, the printed example programs did not use bold
          italicised terms; they were written in standard single precision
          Fortran;

        - before Mark 9, the printed example results were generated on an ICL
          1906A (with approximately 11 digits of precision), and between Marks
          9 and 12 they were generated on an ICL 2900 (with approximately 16
          digits of precision);

        - before Mark 13, documents referred to `the appropriate implementation
          document'; this means the same as the `Users' Note for your
          implementation'.


      4. Contact between Users and NAG

        For further advice or communication about the NAG Library, you should
        first turn to the staff of your local computer installation. This
        covers such matters as:

        - obtaining a copy of the Users' Note for your implementation;
        - obtaining information about local access to the Library;
        - seeking advice about using the Library;
        - reporting suspected errors in routines or documents;
        - making suggestions for new routines or features;
        - purchasing NAG documentation.

        Your installation may have advisory and/or information services to
        handle such enquiries. In addition NAG asks each installation mounting
        the Library to nominate a NAG site representative, who may be
        approached directly in the absence of an advisory service. Site
        representatives receive information from NAG about confirmed errors,
        the imminence of updates, and so on, and will forward users' enquiries
        to the appropriate person in the NAG organisation if they cannot be
        dealt with locally. If you are unable to make contact with your local
        site representative, you should write to the address given in the
        Users' Note or to the address given at the head of the Library Manual.

      5. Further Information

        In the NAG Fortran Library Manual, the document Development of NAG
        gives general information about the NAG project, while Summary of
        Services gives details of other NAG products and services, including
        numerical subroutine libraries in Ada, Pascal and Algol 68.

        In addition, references [2], [3], [4], and [5] discuss various aspects
        of the design and development of the NAG Library, and NAG's technical
        policies and organisation.


      6. References

        [1] Collected Algorithms from ACM,
            Index by subject to algorithms, 1960-1976.

        [2] FORD, B.
            Transportable Numerical Software.
            Lecture Notes in Computer Science, 142, pp. 128-140, 1982.

        [3] FORD, B., BENTLEY, J., DU CROZ, J.J. and HAGUE, S.J.
            The NAG Library machine.
            Software Practice and Experience, 9, 1, pp. 65-72, 1979.

        [4] FORD, B. and POOL, J.C.T.
            The Evolving NAG Library Service.
            In: `Sources and Development of Mathematical Software',
            W. Cowell (Ed.).
            Prentice-Hall, Englewood Cliffs, pp. 375-397, 1984.

        [5] HAGUE, S.J., NUGENT, S.M. and FORD, B.
            Computer-based Documentation for the NAG Library.
            Lecture Notes in Computer Science, 142, pp. 91-127, 1982.

        [6] HOPPER, M.J.
            TSSD, a Typesetting System for Scientific Documents.
            United Kingdom Atomic Energy Authority, Harwell,
            Report AERE-R 8574, 1982.

        [7] USA Standard Fortran.
            American National Standards Institute, Publication X3.9, 1966.

        [8] American National Standard Fortran.
            American National Standards Institute, Publication X3.9, 1978.
\end{verbatim}
\end{small}

\newpage
\section{Users Note}
\label{se:usern}

\begin{small}
\begin{verbatim}
                       NAG Fortran Library, Mark 15

                                FLDVV15D

                    DEC VAX/VMS Double Precision

                               Users' Note





                                Contents



      1. Introduction

      2. Availability of Routines

      3. General Information

         3.1. Accessing the Library
         3.2. Interpretation of Bold Italicised Terms
         3.3. Example Programs
         3.4. Explicit Output from NAG Routines
         3.5. User Documentation

      4. Routine-specific Information

      5. Additional Services from NAG

      6. Contact with NAG

      7. NAG Users Association

      1. Introduction

         This document is essential reading for every user of the NAG
         Fortran Library Implementation specified in the title. It provides
         implementation-specific detail that augments the information
         provided in the NAG Fortran Library Manual and Introductory Guide.
         Wherever those manuals refer to the "Users' Note for your
         implementation", the user should consult this note. A tabbed divider
         has been provided near the beginning of Volume 1 of the NAG Fortran
         Library Manual, indicating a suitable place to store the document,
         should you so wish.

         NAG recommends that users read the following minimum reference
         material before calling any library routine:

         (a) Essential Introduction
         (b) Chapter Introduction
         (c) Routine Document
         (d) Implementation-specific Users' Note

         Items (a), (b) and (c) are included in the NAG Fortran Library
         Manual; items (a) and (b) are also included in the NAG Fortran
         Library Introductory Guide; item (d) is this document. Items (a)
         and (d) are also supplied on the distribution medium. Each NAG
         Fortran Library Service site is supplied with at least one copy
         of each of the above. Please ask your NAG Site Contact for further
         information.

      2. Availability of Routines

         All routines listed in the chapter contents documents of the NAG
         Fortran Library Manual, Mark 15 are available in this implementation.
         At Mark 15, 167 new primary ('user-callable') routines have been
         introduced, and 11 deleted. Please consult the document FORTRAN
         MARK 15 NEWS in the manual for lists of these routines and for a list
         of routines scheduled for withdrawal at Mark 16 or later. The file
         NEWS supplied to your site on the library distribution medium (see
         Section 3.5) also gives an outline of the new material available.

      3. General Information

      3.1. Accessing the Library

           To compile a Fortran program, type

               FORTRAN <program>

           where <program> is the name of a .FOR file containing your
           program.

           To link a program with the NAG Library, assuming that the logical
           name NAG has been set up, type

               LINK <program>,NAG/LIB

           where <program> is the name of a .OBJ file containing your
           compiled program.

           To run the program, type

               ASSIGN <data>    FOR005
               ASSIGN <results> FOR006
               RUN <program>

           where <program> is the name of a .EXE file containing your linked
           program, <data> is data to be read from Fortran channel 5, and
           <results> are results to be written to Fortran channel 6.

      3.2. Interpretation of Bold Italicised Terms

           For this double precision implementation, the bold italicised
           terms used in the NAG Fortran Library Manual (and shown here
           as //.....// ) should be interpreted as:

           //real//                 - DOUBLE PRECISION (REAL*8)
           //basic precision//      - double precision
           //complex//              - COMPLEX*16
           //additional precision// - quadruple precision (REAL*16)
           //machine precision//    - the machine precision, see the value
                                      returned by X02AJF in Section 4, below.

           Thus a parameter described as //real// should be declared as
           DOUBLE PRECISION in the user's program. If a routine
           accumulates an inner product in //additional precision//, it
           is using software to simulate quadruple precision.

           In routine documents that have been newly typeset since Mark 12
           additional bold italicised terms are used in the published
           example programs and they must be interpreted as follows:

           //real// as an intrinsic function name - DBLE
           //imag//                               - DIMAG
           //cmplx//                              - DCMPLX
           //conjg//                              - DCONJG
           //e// in constants, e.g. 1.0//e//-4    - D, e.g. 1.0D-4
           //e// in formats, e.g. //e//12.4       - D, e.g. D12.4

           All references to routines in Chapter F07 - Linear Equations
           (LAPACK) use the LAPACK name, not the NAG F07 name. The LAPACK
           name is precision dependent, and hence the name appears in a bold
           italicised typeface.

           The typeset examples use the single precision form of the LAPACK
           name. To convert this name to its double precision form, change the
           first character either from S to D or C to Z as appropriate.
           For example:

           //sgetrf// refers to the LAPACK routine name - DGETRF
           //cpotrs//                                   - ZPOTRS

      3.3. Example Programs

           In the NAG Fortran Library Manual, routine documents that have
           been typeset since Mark 12 present the example programs in a
           generalised form, using bold italicised terms as described in
           Section 3.2.

           In other routine documents, the example programs are in
           single precision and require modification for use with double
           precision routines. This conversion can entail:

           - Changing REAL or COMPLEX type specifications to REAL*8 or
             COMPLEX*16;

           - Changing certain intrinsic function references, e.g. REAL or
             FLOAT to DBLE, ALOG to DLOG, CMPLX to DCMPLX, and so on;

           - Changing real constants to double precision form, e.g. 0.1
             or 0.1E0 to 0.1D0.

           The example programs supplied to a site in machine-readable
           form have been modified as necessary so that they are suitable
           for immediate execution. Note that all the distributed example
           programs have been revised and do not correspond exactly with the
           programs published in the manual, unless the documents have been
           recently typeset. The distributed example programs should be used
           in preference wherever possible.

      3.4. Explicit Output from NAG Routines

           Certain routines produce explicit error messages and advisory
           messages via output units which either have default values or
           can be reset by using X04AAF for error messages and X04ABF for
           advisory messages. (The default values are given in Section 4).
           The maximum record lengths of error messages and advisory
           messages (including carriage control characters) are 80
           characters, except where otherwise specified.

      3.5. User Documentation

           The following machine-readable information files are provided by
           NAG on the library distribution medium. Please consult your local
           advisory service or NAG Site Contact for further details:

           UN       - Users' Note (this document)
           ESSINT   - the Essential Introduction to the NAG Fortran Library
           SUMMARY  - a brief summary of the routines
           NEWS     - an outline of the new and enhanced routines available
                      at Mark 15
           REPLACED - a list of routines available at earlier Marks of the
                      Library but since withdrawn, together with recommended
                      replacements
           CALLS    - a list of routines called directly or indirectly by
                      each routine in the library, and by each example program
           CALLED   - for each routine in the library (including auxiliaries)
                      this gives a list of those routines and example programs
                      which call it directly or indirectly.

           See Section 5 for additional documentation available from NAG.

      4. Routine-specific Information

         Any further information which applies to one or more routines
         in this implementation is listed below, chapter by chapter.

         (a) F01, F04

             The following routines call the function X02CAF for an estimate
             of the amount of actual (as opposed to virtual) store
             available, in order to reduce the amount of paging when solving
             large problems:

             F01BTF      F01BXF      F04AYF      F04AZF

             See X02CAF below for further information.

         (b) G02

             The value of ACC, the machine-dependent constant mentioned in
             several documents in the chapter is approximately 1.0D-13.

         (c) P01

             On hard failure, P01ABF writes the error message
             to the error message unit specified by
             X04AAF, and then STOPs.

         (d) S07 - S21

             The constants referred to in the NAG Fortran Library Manual
             have the following values in this implementation:

                 S07AAF  F(1)   = 3.0D+05
                         F(2)   = 1.0D-14

                 S10AAF  E(1)   = 20.75D+00
                 S10ABF  E(1)   = 88.0D+00
                 S10ACF  E(1)   = 88.0D+00

                 S13AAF  x(hi)  = 85.0D+00
                 S13ACF  x(hi)  = 1.0D+08
                 S13ADF  x(hi)  = 1.0D+09

                 S14AAF  IFAIL  = 1 if X > +34.0D+00
                         IFAIL  = 2 if X < -34.0D+00
                         IFAIL  = 3 if abs(X) < 5.9D-39
                 S14ABF  IFAIL  = 2 if X > 2.03D+36

                 S15ADF  x(hi)  = 9.5D+00
                         x(low) = -6.5D+00
                 S15AEF  x(hi)  = 6.25D+00

                 S17ACF  IFAIL  = 1 if X > 2.1D+09
                 S17ADF  IFAIL  = 1 if X > 2.1D+09
                         IFAIL  = 3 if 0.0 < X <= 3.7D-39
                 S17AEF  IFAIL  = 1 if abs(X) > 2.1D+09
                 S17AFF  IFAIL  = 1 if abs(X) > 2.1D+09
                 S17AGF  IFAIL  = 1 if X > 25.52D+00
                         IFAIL  = 2 if X < -2.2D+11
                 S17AHF  IFAIL  = 1 if X > 25.93D+00
                         IFAIL  = 2 if X < -2.2D+11
                 S17AJF  IFAIL  = 1 if X > 25.84D+00
                         IFAIL  = 2 if X < -5.9D+09
                 S17AKF  IFAIL  = 1 if X > 25.93D+00
                         IFAIL  = 2 if X < -5.9D+09
                 S17DCF  IFAIL  = 2 if abs (Z) < 0.57921826D-35
                         IFAIL  = 4 if abs (Z) or FNU+N-1 > 0.32768000D+05
                         IFAIL  = 5 if abs (Z) or FNU+N-1 > 0.10737418D+10
                 S17DEF  IFAIL  = 2 if imag (Z) > 0.81136554D+02
                         IFAIL  = 3 if abs (Z) or FNU+N-1 > 0.32768000D+05
                         IFAIL  = 4 if abs (Z) or FNU+N-1 > 0.10737418D+10
                 S17DGF  IFAIL  = 3 if abs (Z) > 0.10240000D+04
                         IFAIL  = 4 if abs (Z) > 0.10485760D+07
                 S17DHF  IFAIL  = 3 if abs (Z) > 0.10240000D+04
                         IFAIL  = 4 if abs (Z) > 0.10485760D+07
                 S17DLF  IFAIL  = 2 if abs (Z) < 0.57921826D-35
                         IFAIL  = 4 if abs (Z) or FNU+N-1 > 0.32768000D+05
                         IFAIL  = 5 if abs (Z) or FNU+N-1 > 0.10737418D+10

                 S18ADF  IFAIL  = 2 if 0.0 < X <= 5.9D-39
                 S18AEF  IFAIL  = 1 if abs(X) > 88.0D+00
                 S18AFF  IFAIL  = 1 if abs(X) > 88.0D+00
                 S18CDF  IFAIL  = 2 if 0.0 < X <= 5.9D-39
                 S18DCF  IFAIL  = 2 if abs (Z) < 0.57921826D-35
                         IFAIL  = 4 if abs (Z) or FNU+N-1 > 0.32768000D+05
                         IFAIL  = 5 if abs (Z) or FNU+N-1 > 0.10737418D+10
                 S18DEF  IFAIL  = 2 if real (Z) > 0.81136554D+02
                         IFAIL  = 3 if abs (Z) or FNU+N-1 > 0.32768000D+05
                         IFAIL  = 4 if abs (Z) or FNU+N-1 > 0.10737418D+10

                 S19AAF  IFAIL = 1 if abs(x) >= 49.5D+00
                 S19ABF  IFAIL = 1 if abs(x) >= 49.5D+00
                 S19ACF  IFAIL = 1 if X > 121.06D+00
                 S19ADF  IFAIL = 1 if X > 122.25D+00

                 S21BCF  IFAIL = 3 if an argument < 6.50D-26
                         IFAIL = 4 if an argument >= 1.18D+23
                 S21BDF  IFAIL = 3 if an argument < 1.40D-13
                         IFAIL = 4 if an argument >= 2.19D+12

         (e) X01

             The values of the mathematical constants are:

                 X01AAF (PI)    = 3.1415926535897932
                 X01ABF (GAMMA) = 0.5772156649015329

         (f) X02

             The values of the machine constants are:

             The basic parameters of the model

                 X02BHF =       2
                 X02BJF =      56
                 X02BKF =    -127
                 X02BLF =     127
                 X02DJF =   .TRUE.

             Derived parameters of floating-point arithmetic

                 X02AJF =   1.3877787807814457D-17
                 X02AKF =   2.9387358770557188D-39
                 X02ALF =   1.7014118346046923D+38
                 X02AMF =   5.8774717541114401D-39
                 X02ANF =   2.350988701644576D-38

             Parameters of other aspects of the computing environment

                 X02AHF =   1.0000000000000000D+38
                 X02BBF =       2147483647
                 X02BEF =               16
                 X02CAF =             7680
                 X02DAF =           .FALSE.

             Values returned by pre-Mark 12 routines

                 X02AAF =   1.3877787807814457D-17
                 X02ABF =   2.9387358770557188D-39
                 X02ACF =   1.7014118346046923D+38
                 X02AGF =   5.8774717541114401D-39

             X02CAF

                 This function is called by F01BTF, F01BXF, F04AYF
                 and F04AZF. It does not affect the numerical results
                 returned by those routines, but merely the amount of
                 paging performed when using these routines to operate on
                 large matrices when running on a paged system. The
                 constant value returned by the version of X02CAF in the
                 compiled library should give a performance close to the
                 optimum in almost all circumstances, but users who wish
                 to override this value (e.g. if their jobs are run in a
                 special session with an untypically large amount of
                 actual store available) may do so by supplying their own
                 function of the same name, to be compiled and loaded with
                 their main program. Such a function should have the form:

                 INTEGER FUNCTION X02CAF(X)
                 DOUBLE PRECISION X
                 X02CAF = new value
                 RETURN
                 END

         (g) X04

             The default output units for error and advisory messages
             for those routines which can produce explicit output are both
             Fortran Unit 6.

      5. Additional Services from NAG

         (a) Printed Manuals

             Where a manual has been provided as part of the contract issue,
             this manual is updated automatically at each new release of the
             software, by the supply of a manual update set or a complete new
             manual. If additional manuals have been ordered in the past
             then updates to these manuals may be separately ordered. They
             are NOT sent automatically. Additional complete manuals and the
             manual updates (where relevant) are available at prices published
             in the NAG documentation order form.

         (b) On-line Information Supplement

             To complement the manuals NAG produces an On-line Information
             Supplement which fulfils two roles:

             - it gives key-word-driven guidance on the selection of the
               appropriate NAG routine
             - it gives abbreviated on-line documentation of the NAG routines,
               to enable the user to call the routines and investigate any
               IFAIL messages without recourse to the manual.

         (c) Other Products

             NAG is continually striving to bring the best in mathematical,
             statistical and graphical software to its users. It caters for
             the package user, the software developer and the Fortran, Ada,
             C, Pascal or Algol 68 enthusiast.

             To find out more about the growing range of facilities available
             please contact NAG and ask for a free information pack.

      6. Contact with NAG

         Queries concerning this document or the implementation generally
         should be directed initially to your local Advisory Service.  If
         you have difficulty in making contact locally, you can write to NAG
         directly at one of the supplied addresses.

      7. NAG Users Association

         NAGUA is the NAG Users Association, and membership is open to all
         users of NAG products and services.  As a member of NAGUA your
         organisation would receive our newsletter "NAGUA News" three times
         a year, would receive discounts at the annual conference, and those
         sites with access to electronic mail would be able to participate in
         NAGMAG, our electronic mail digest.

         For an information pack and membership application form, please
         contact the NAGUA Co-ordinator at the supplied address.


      Appendix - supplied addresses

      NAG Ltd                                 NAG Inc
      Wilkinson House                         1400 Opus Place, Suite 200
      Jordan Hill Road                        Downers Grove
      OXFORD  OX2 8DR                         IL 60515-5702
      United Kingdom                          USA

      Tel: +44 865 511245                     Tel: +1 708 971 2337
      Fax: +44 865 310139                     Fax: +1 708 971 2706

      NAG GmbH
      Schleissheimerstrasse 5
      W-8046 Garching bei Munchen
      Deutschland

      Tel: +49 89 3207395
      Fax: +49 89 3207396



      NAGUA Co-ordinator
      NAG Users Association
      PO Box 426
      OXFORD  OX2 8SD
      United Kingdom

      Tel: +44 865 311102
\end{verbatim}
\end{small}

\newpage
\section{Mark 15 News}
\label{se:news}

\begin{small}
\begin{verbatim}
      MARK 15 NEWS

      1. New Features of Mark 15

        Mark 15 represents a further considerable expansion of the NAG Fortran
        Library. It contains a total of 1045 documented routines, of which 167
        are new at this Mark. Two new chapters have been introduced:

        F07 - Linear Equations (LAPACK)
        G12 - Survival Analysis

        Out of 167 new routines, 98 are in the new chapter F07. This new
        chapter includes routines to compute factorizations, solutions, error
        bounds and condition numbers for real and complex linear systems;
        symmetric, Hermitian, positive-definite, general, band and triangular
        systems are provided for.

        35 of the new routines are in the Statistics chapters. They include
        facilities (in the stated chapters) for:

        - further statistical distribution functions (G01)
        - factor analysis and discriminant analysis (G03)
        - improved random number generators (G05)
        - computation of confidence intervals; maximum likelihood estimates
          (G07)
        - generation of a multivariate time series (G05)
        - forecasting from a vector autoregressive moving average model,
          allowing for: transforming and/or differencing of a multivariate
          time series; testing for stationarity and invertibility (G13)

        New routines have been introduced in other chapters of the Library
        for:

        - solution of parabolic partial differential equations with coupled
          differential algebraic systems (D03)
        - solution of symmetric positive-definite Toeplitz systems (F04)
        - computation of matrix norms for different types of matrix (F06)
        - mixed integer LP (H)
        - special functions (S)
        - machine specific numbers (X02)

        For a number of existing routines (mainly in the linear algebra
        chapters) the underlying code has been replaced by calls to the new
        F07 routines or their auxiliaries. The modified versions can achieve
        much better performance on some machines.


      2. New Routines

        For details, please refer to the relevant chapter introductions and
        routine documents. Routines in the F06 and X02 chapters are described
        in the F06 and X02 Chapter Introductions; they do not have individual
        routine documents. A concise summary of the purpose of all documented
        routines in the Library is given in the document Contents Summary,
        Mark 15 (with the exception of routines which have been superseded).

        The following 167 new routines are included in the NAG Fortran Library
        at Mark 15:

          D03PCF      D03PDF      D03PHF      D03PJF      D03PYF      D03PZF
          F04FEF      F04FFF      F04MEF      F04MFF      F06RAF      F06RBF
          F06RCF      F06RDF      F06REF      F06RJF      F06RKF      F06RLF
          F06RMF      F06UAF      F06UBF      F06UCF      F06UDF      F06UEF
          F06UFF      F06UGF      F06UHF      F06UJF      F06UKF      F06ULF
          F06UMF      F07ADF      F07AEF      F07AGF      F07AHF      F07AJF
          F07ARF      F07ASF      F07AUF      F07AVF      F07AWF      F07BDF
          F07BEF      F07BGF      F07BHF      F07BRF      F07BSF      F07BUF
          F07BVF      F07FDF      F07FEF      F07FGF      F07FHF      F07FJF
          F07FRF      F07FSF      F07FUF      F07FVF      F07FWF      F07GDF
          F07GEF      F07GGF      F07GHF      F07GJF      F07GRF      F07GSF
          F07GUF      F07GVF      F07GWF      F07HDF      F07HEF      F07HGF
          F07HHF      F07HRF      F07HSF      F07HUF      F07HVF      F07MDF
          F07MEF      F07MGF      F07MHF      F07MJF      F07MRF      F07MSF
          F07MUF      F07MVF      F07MWF      F07NRF      F07NSF      F07NUF
          F07NVF      F07NWF      F07PDF      F07PEF      F07PGF      F07PHF
          F07PJF      F07PRF      F07PSF      F07PUF      F07PVF      F07PWF
          F07QRF      F07QSF      F07QUF      F07QVF      F07QWF      F07TEF
          F07TGF      F07THF      F07TJF      F07TSF      F07TUF      F07TVF
          F07TWF      F07UEF      F07UGF      F07UHF      F07UJF      F07USF
          F07UUF      F07UVF      F07UWF      F07VEF      F07VGF      F07VHF
          F07VSF      F07VUF      F07VVF      G01DHF      G01EAF      G01EMF
          G01EPF      G01FAF      G01FMF      G01HBF      G01JDF      G01MBF
          G02FCF      G03BAF      G03BCF      G03CAF      G03CCF      G03DAF
          G03DBF      G03DCF      G03ZAF      G05DRF      G05FEF      G05FFF
          G05HDF      G07AAF      G07ABF      G07BBF      G07BEF      G07CAF
          G08ALF      G12AAF      G13DJF      G13DKF      G13DLF      G13DMF
          G13DNF      G13DXF      H02BZF      S21CAF      X02ANF


      3. Withdrawn Routines

        The following routines have been withdrawn from the NAG Fortran
        Library at Mark 15. Warning of their withdrawal was included in the
        Mark 14 Library Manual, together with advice on which routines to use
        instead. The relevant Chapter Introduction documents give more
        detailed guidance.

          Withdrawn Routine       Recommended Replacement
            C02ADF                  C02AFF
            E01ACF                  E01DAF and E02DEF
            F01CDF                  F01CTF
            F01CEF                  F01CTF
            F01CGF                  F01CTF
            F01CHF                  F01CTF
            F01LZF                  F02SWF and F02SXF
            F01QAF                  F01QCF
            F01QBF                  F01QJF
            F02SZF                  F02SYF
            H02BAF                  H02BBF


      4. Routines Scheduled for Withdrawal

        The routines listed below are scheduled for withdrawal from the NAG
        Fortran Library, because improved routines have now been included in
        the Library. Users are advised to stop using routines which are
        scheduled for withdrawal immediately and to use recommended
        replacement routines instead. The relevant chapter introduction
        documents give further guidance, including detailed advice on how to
        change a call to the old routine into a call to the new routine.

        The following routines will be withdrawn at Mark 16:

          Routine scheduled       Recommended
          for withdrawal          Replacement
            C02AEF                  C02AGF
            D03PAF                  D03PCF
            D03PBF                  D03PCF
            D03PGF                  D03PCF
            E02DBF                  E02DEF
            E04HBF                  not needed except with E04JBF
            E04JBF                  E04UCF
            E04KBF                  E04UCF
            F01ACF                  F01ABF
            F01BQF                  F07GDF (SPPTRF/DPPTRF) or
                                    F07PDF (SPPTRF/DSPTRF)
            F02WAF                  F02WEF
            F04AQF                  F07GEF (SPPTRS/DPPTRS) or
                                    F07PEF (SSPTRS/DSPTRS)
            F06QGF                  F06RAF
            F06VGF                  F06UAF
            G01BAF                  G01EBF
            G01BBF                  G01EDF
            G01BCF                  G01ECF
            G01BDF                  G01EEF
            G01CAF                  G01FBF
            G01CBF                  G01FDF
            G01CCF                  G01FCF
            G01CDF                  G01FEF
            G02CJF                  G02DAF and G02DGF
            G05DGF                  G05FFF
            G05DLF                  G05FEF
            G05DMF                  G05FEF
            G08ABF                  G08AGF
            G08ADF                  G08AHF, G08AKF and G08AJF
            G08CAF                  G08CBF
            M01AJF                  M01DAF, M01ZAF and M01CAF
            M01AKF                  M01DAF, M01ZAF and M01CAF
            M01APF                  M01CAF
            X02AAF                  X02AJF
            X02ABF                  X02AKF
            X02ACF                  X02ALF
            X02AGF                  X02AMF

        The following routines have been superseded, but will not be withdrawn
        from the Library until Mark 17 at the earliest. They are being
        retained at Marks 15 and 16 because of their length of life in the
        Library, and to give users a longer time to make the transition to the
        new routines.

          Superseded routine      Recommended Replacement
            F01AAF                  F07ADF (SGETRF/DGETRF) and
                                    F07AJF (SGETRI/DGETRI)
            F01BNF                  F07FRF (CPOTRF/ZPOTRF)
            F01BPF                  F07FRF (CPOTRF/ZPOTRF) and
                                    F07FWF (CPOTRI/ZPOTRI)
            F01BTF                  F07ADF (SGETRF/DGETRF)
            F01BXF                  F07FDF (SPOTRF/DPOTRF)
            F01LBF                  F07BDF (SGBTRF/DGBTRF)
            F01NAF                  F07BRF (CGBTRF/ZGBTRF)
            F03AGF                  F07HDF (SPBTRF/DPBTRF)
            F03AHF                  F07ARF (CGETRF/ZGETRF)
            F03AMF                  see F03 Chapter Introduction
            F04AKF                  F07ASF (CGETRS/ZGETRS)
            F04ALF                  F07HEF (SPBTRS/DPBTRS)
            F04AWF                  F07FSF (CPOTRS/ZPOTRS)
            F04AYF                  F07AEF (SGETRS/DGETRS)
            F04AZF                  F07FEF (SPOTRS/DPOTRS)
            F04LDF                  F07BEF (SGBTRS/DGBTRS)
            F04NAF                  F07BSF (CGBTRS/ZGBTRS)
            G13DAF                  G13DMF
            X02CAF                  not needed except with F01BTF and F01BXF


      5. Routines Revised at Mark 15

      5.1. Linear Algebra Routines

        The following routines have had underlying code replaced to call new
        F07 routines or their auxiliaries. This has resulted in the addition
        of new error exits from these routines.

          F01ADF      F01AEF      F01BNF      F01BPF      F03AAF      F03ABF
          F03ADF      F03AEF      F03AFF      F03AHF      F04AAF      F04ABF
          F04ADF      F04AEF      F04ARF      F04ASF      F04ATF

      5.2. Fast Fourier Transforms

        Some minor revisions have been made to routines for fast Fourier
        transforms in Chapter C06, in order to improve their efficiency.

      5.3. Ordinary Differential Equations

        Some minor revisions have been made to auxiliaries of routines solving
        boundary value problems by shooting methods, namely D02HAF, D02HBF and
        D02SAF, in order to improve reliability.

      5.4. Time Series Analysis

        Some modifications have been made to G13 routines. An algorithmic
        improvement has been incorporated into G13DCF. The matrices computed
        in the argument C of G13DAF and R of G13DSF now return the correct
        results; previously they contained the transpose of the results.
\end{verbatim}
\end{small}

\end{document}
