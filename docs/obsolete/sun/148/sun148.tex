\documentstyle[11pt]{article}
\pagestyle{myheadings}

%------------------------------------------------------------------------------
\newcommand{\stardoccategory}  {Starlink User Note}
\newcommand{\stardocinitials}  {SUN}
\newcommand{\stardocnumber}    {148.1}
\newcommand{\stardocauthors}   {J.\ R.\ Lewis}
\newcommand{\stardocdate}      {6 May 1992}
\newcommand{\stardoctitle}     {SCP --- A Simple CCD Processing Package}
%------------------------------------------------------------------------------

\newcommand{\stardocname}{\stardocinitials /\stardocnumber}
\renewcommand{\_}{{\tt\char'137}}     % re-centres the underscore
\markright{\stardocname}
\setlength{\textwidth}{160mm}
\setlength{\textheight}{230mm}
\setlength{\topmargin}{-2mm}
\setlength{\oddsidemargin}{0mm}
\setlength{\evensidemargin}{0mm}
\setlength{\parindent}{0mm}
\setlength{\parskip}{\medskipamount}
\setlength{\unitlength}{1mm}

%------------------------------------------------------------------------------
% Add any \newcommand or \newenvironment commands here
%------------------------------------------------------------------------------

\begin{document}
\thispagestyle{empty}
SCIENCE \& ENGINEERING RESEARCH COUNCIL \hfill \stardocname\\
RUTHERFORD APPLETON LABORATORY\\
{\large\bf Starlink Project\\}
{\large\bf \stardoccategory\ \stardocnumber}
\begin{flushright}
\stardocauthors\\
\stardocdate
\end{flushright}
\vspace{-4mm}
\rule{\textwidth}{0.5mm}
\vspace{5mm}
\begin{center}
{\Large\bf \stardoctitle}
\end{center}
\vspace{5mm}

\begin{large}
\begin{em}
SCP is an optional item within the Starlink Software Collection. If it is not
installed at your site, please see your Site Manager.
\end{em}
\end{large}


%------------------------------------------------------------------------------
%  Add this part if you want a table of contents
\setlength{\parskip}{0mm}
\tableofcontents
\setlength{\parskip}{\medskipamount}
\markright{\stardocname}
%------------------------------------------------------------------------------

\newpage
\section{Introduction}

This note describes a small set of programs, written at RGO, which deal with
basic CCD frame processing (e.g. bias subtraction, flat fielding, trimming
etc.).  The need to process large numbers of CCD frames from devices such as
FOS or ISIS in order to extract spectra has prompted the writing of routines
which will do the basic hack-work with a minimal amount of interaction from the
user. Although they were written with spectral data in mind, there are no
``spectrum-specific'' features in the software which means they can be applied
to any CCD data.

The programs have been written in accordance with the methods and
conventions of FIGARO (SUN/86) and the Starlink Programming Standard (SGP/16).

SCP is an optional item within the Starlink Software Collection. The programs
will run standalone without the need to invoke the FIGARO startup, but require
that FIGARO itself be installed to ensure the availability of the necessary
shareable images.

SCP will accept any two dimensional FIGARO image or NDF. The output files will
be recognised by any FIGARO or KAPPA application which is designed to handle
them.

\section{How to run the programs}

The SCP programs are made available by running the startup command procedure.
To access the routines issue the command:
\begin{verbatim}
      $ SCP
\end{verbatim}
which runs the startup command procedure.

All of the relevant logical and symbolic definitions should be done for
you. To run a particular program, just issue the name of the program as if
it were a standard FIGARO command. In addition a help file will also be
set up for you. Help on an individual program may be obtained by entering:
\begin{verbatim}
      $ HELP prog
\end{verbatim}
This will give you a list of the program's input parameters as well as an
option to look at the source header comments for the main routine.

\section{Features supported}

The programs can take data of any type including USHORT data which is
masquerading as INTEGER*2. All processing however is done with REAL*4
arithmetic.  Output files will be type REAL*4.

If magic values exist in the data, they are dealt will explicitly. Quality
information is dealt with in the sense that bad quality data is converted to
magic values.

Error arrays are supported.  Input error arrays are updated properly if they
exist and if the user wants error arrays in the output files.  If there are no
input error arrays and the user still wants them in the output file, then
SCP\_PROC creates them from the data and the chip noise parameters.

\section{Package description}

There are currently five routines in the package.  These are SCP\_CHIP,
SCP\_HEAD, SCP\_NOISE, SCP\_PRE and SCP\_PROC. The main routine is SCP\_PROC.
This program was designed to do all the actual work in processing the data with
a minimal number of parameters from the user.  Hence it is very convenient to
use in batch. It reads three text files for its instructions. These are:

\begin{description}

\item [CHIP file]This is set up by the program SCP\_CHIP and describes the
various parameters of the chip used in the observations. The file is in
standard ASCII format and can be edited.  A description of the format and the
parameters is given in section \ref{sec:chip_pars}.

\item [PROC file] This is set up by the program SCP\_PRE and gives SCP\_PROC
its list of commands.  The actual PROC file is a standard ASCII file and
contains both the questions asked and  the answers given in a set format. Again
this file can be edited if desired as long as the format is roughly maintained.
The actual parameters will be described in section \ref{sec:proc_pars}.

\item [IMAGES file]This is set up by the user and lists the frames to be
reduced according to type (bias frame, flat field, etc). The format is quite
simple and will be discussed in section \ref{sec:imfil}.

\end{description}

SCP\_PROC generates a log file called SCP\_PROC.LIS.  This gives all the
information you need regarding the processing of each individual frame.

Because of the current state of affairs with regards to header information in
LPO data files, it is often very difficult to find useful bits of information
such as the exposure time.  For this reason the routine SCP\_HEAD has been
provided.  It looks through a standard list of locations for the exposure time
and (given that it has been found) puts it somewhere where SCP\_PROC can easily
find it.  More on this in section \ref{sec:times}.

Finally, in order to get a proper estimate of the variance for an individual
pixel, it necessary to know the readout noise and a/d conversion for the chip
used.  SCP\_NOISE does a statistical analysis of a series of frames in an
attempt to derive these parameters.  This is important in the event that the
user wishes to do something like optimal extraction where accurate error
estimates are essential. A more complete description of this program is given
in section \ref{sec:noise}.

\section{Individual program descriptions}

\subsection{SCP\_CHIP} \label{sec:chip_pars}

This program sets up the CHIP file for use in SCP\_PROC.
Eventually there should be a standard set of these files in a standard
directory.  Until then, the user will have to set them up for the
individual run. The parameters are:

\begin{description}

\item [CHIPFILE] The name of the output CHIP file. The file extension is
by default .SCP

\item [COMMENT] A comment for the first line of the CHIP file.

\item [CHIPNAME] This the name of the chip.

\item [DIMS] These are the dimensions of the total data array for the chip.
(2 values required)

\item [START, END] These are the starting and ending  points in X and Y for
the actual data. START and END actually define the area of the chip
where the `useful' data occurs, {\it e.g.} away from the bias strip.  They
can also be used for defining a sub-region of the frames to be used if you
are not really interested in the whole data area and would like substantially
smaller output files. (2 values required for each)

\item [BSTART, BEND]  These are the starting and ending points in X and Y  for
the bias strip.  (2 values required for each)

\item [READOUT, EADU] These are the readout noise for the chip in ADUs and the
a/d conversion in electrons per ADU.  READOUT and EADU are only really
necessary if error arrays are ultimately wanted.  If they aren't then dummy
values should be inserted here.

\item [NUMBADC, NUMBADR, NUMBADP] These are the number of bad columns bad rows
and bad individual pixels in the data.  The latter should not have been
included in either of the former.  Once one of these has been entered SCP\_PRE
will prompt for the actual locations. These are used to flag bad areas on the
chip which SCP\_PROC will then either flag as bad or interpolate over.

\end{description}

Here is a sample CHIP file:

\begin{verbatim}
      ******FILE FOR EEV CCD-02-06******
      Chip name! EEV CCD-02-06;
      Dimensions! [400,590];
      Data region! [12:396,1:578];
      Bias strip! [2:6,101:500];
      Readout noise! 10;
      Electrons per ADU! 0.65;
      Bad pixels! C6;300,101;
\end{verbatim}

This is the file for a chip which has dimensions $400 \times 590$.  The data
region runs from $(12,1)$ to $(396,578)$ and the bias strip value is to be
calculated from $(2,101)$ to $(6,500)$. It has a readout noise and a/d
conversion of 10 adu and 0.65 electrons/adu respectively.  It as 1 bad column
(\# 6) and one bad pixel, $(300,101)$.  This file can be edited so long as the
basic format remains unchanged, especially with regards to the punctuation
marks and square brackets.

\subsection{SCP\_HEAD} \label{sec:times}

At the time of writing, there is no standard location within the header of an
LPO file in which to deposit the exposure time.  SCP\_PROC would like to know
what it is, but doesn't want to spend it's whole life searching for it.  Hence
SCP\_HEAD was written.  What it does is to read a list of possible header
locations for the exposure time.  It then tests these locations one by one. If
it finds a valid value, it is deposited in a `standard' place. If the file is a
standard FIGARO file then the location is .OBS.TIME and if it is an NDF then
the location is .MORE.FIGARO.TIME.  The parameters are:

\begin{description}

\item [IMAGEFILE] This is a file with a list of images to be processed. The
info must be in a standard format which is described in section
\ref{sec:imfil}.

\item [HEADFILE] This is the name of a file with the possible locations of the
exposure time information.  There should be a standard file in {\tt SCP\_EXE}
called HEAD.SCP. If you want to create your own, then this is possible too.

\end{description}

\subsection{SCP\_NOISE}\label{sec:noise}

This program calculates the readout noise, a/d conversion and grain noise for
the CCD used in the observations.  The frames should be bias subtracted before
using SCP\_NOISE\footnote{It is worth noting here that the general philosophy
behind this program is based on T.R. Marsh's program PAMNOISE.}.   A quick
summary of the idea behind this routine goes as follows.  The error in counts
in a CCD pixel is governed by Poisson statistics, e.g.

\begin{displaymath}
\sigma^2_x = \langle x\rangle
\end{displaymath}

CCD data also suffers from another source of noise which is caused during the
readout phase, ({\it readout noise}).  This is roughly constant over the whole
chip.  Finally a source of noise caused by random response variations from
pixel to pixel ({\it grain noise}) is present in some two dimensional
detectors.  In the case of a CCD this should be zero.  However, it has been
included here for the sake of completeness.  To find the true variance these
latter factors must be added in quadrature.  Thus for a given pixel the
variance in units of electrons is:

\begin{displaymath}
\sigma^2_{e^-} = \rho^2_{e^-} + \epsilon x + \eta^2 \epsilon^2 x^2
\end{displaymath}

where $\rho$ is the readout noise, $\epsilon$ is the number of electrons
per data number, $x$ is the value of the data at that point and
$\eta$ is the grain noise contribution expressed as a fraction of the
value of the data.  SCP\_NOISE uses data numbers as opposed to electrons,
hence:

\begin{displaymath}
\sigma^2_d = \rho^2_d + x/\epsilon + \eta^2 x^2
\end{displaymath}

where $\rho_d = \rho_{e^-}/\epsilon$.

Technically the independent elements in this analysis are $\epsilon$ and
$\rho_{e^-}$, {\it not} $\rho_d$.  However this program works by fitting a
least squares quadratic in $x$ to the above equation in order to determine the
constants. Thus to use $\rho_{e^-}$ would make $\epsilon$ appear in the
constant term and would therefore couple the constant term to the first order
term.  This calls for another fitting method.  This change will be made
sometime in the future.

SCP\_NOISE first divides the area of interest on the frame into a number of
boxes.  For each pixel in the box, an estimate of the mean and local  variance
is calculated from the surrounding pixels.  This is then averaged over the box.

Because of the nature of the curve, it is often very difficult to define both
the readout and the a/d conversion with the same data set.  The readout comes
from low signal data, whereas the a/d conversion comes from much higher signal
data.  In addition there is the problem of non-uniform illumination.  If
SCP\_NOISE is run on a typical astronomical CCD frame (say a star field) the
smoothing boxes will sometimes contain both low signal regions (say, sky) and
high signal regions (a star in good seeing).  This clearly will not do.  There
are a number of ways in which this can be avoided.  One way is to use a series
of flat field exposures of varying exposure times.  Another would be to have a
single frame in which you could be reasonably sure of very smooth transitions
between low and high signal regions.

The parameters are as follows:

\begin{description}

\item [X\_RNG, Y\_RNG] This is the range in X and Y to include in the analysis.
(2 values required for each)

\item [BOX] This is the size of the smoothing box used to define the `noise'.
(2 values required)

\item [CHANGE] By default the values of the readout and a/d conversion are
varied in the fit and the grain noise is held constant at 0. If this isn't what
is wanted, CHANGE should be set. This parameter isn't prompted for, so this
must be done on the command line.

\item [VARKEY] These are flag which are prompted for when CHANGE has been  set.
Answering 1 will cause a parameter to be varied in the fit and answering 0 will
cause it to be held constant.  The order is READOUT, EADU and GRAIN. (3 values
required)

\item [READOUT, EADU, GRAIN] These are values input by the user if the
particular parameter is to be held constant.

\item [PLOT] If this is set, the a log-log plot of the noise vs. the average
data will be plotted along with the fitted function.

\item [INPUT] This is the name of any number of input files. As the analysis
happens in a loop, each frame which is to be analysed should be entered in
turn.  When no more frames are to be included, a single `*' signals this.

\end{description}

\subsection{SCP\_PRE} \label{sec:proc_pars}

This program sets up the PROC file for SCP\_PROC.  The output file is in an
ordinary ASCII file which is in a set format.  This file can be edited so
long as that format is maintained.  The parameters are fairly large in
number.  If not all options have been accepted, then not all of the
parameters will be prompted for.  Hopefully this should be clear in the
following list.

\begin{description}

\item [PROCFILE] This is the name of the output PROC file.  The default
extension is .SCP.

\item [BIAS, DARK, FLAT, PROG] These are flags which tell SCP\_PROC whether
frames of these types will be processed.

\item [COSMIC] This tells SCP\_PROC to have a go at zapping cosmic rays.

\item [NBINS, CELL\_SIZE, THRESH] If cosmic rays are to be zapped, then these
are the number of bins for background histogramming, the cell size used for the
search, and the threshold in sigma for a cosmic ray event.

\item [DUD] This is true if something is to be done about dud pixels, columns
or rows.

\item [DUD\_FLAG] If something is to be done about dud pixels then they can
either be (1) replaced with an average of their neighbours or (2) just flagged
as bad and ignored.

\item [SUB\_BIAS] This indicates that a bias estimate will be subtracted.

\item [SUB\_FRAME] If a bias is being subtracted then it can either be from a
bias frame or a mean bias from an overscan strip.  SUB\_FRAME is true if  you
want to subtract a bias frame and false if you want to use the  overscan strip
estimate by itself.

\item [BIAS\_FRAME] This has a number of uses.  If bias frames are being
reduced, then this is the name of the output file for the mean bias frame. If a
bias frame is being subtracted, then this is the name of the input bias frame.

\item [OVERSCAN] If a bias frame is being subtracted, it can be scaled by the
estimate in the overscan region in order to take care of any fluctuations in
the mean bias over the course of the night.

\item [FLAT\_DIV] This is the signal to divide any program fields by a flat
field.

\item [IMAGE\_FLAT] If a flat field division is being done, then this can
either be a flat field frame (IMAGE\_FLAT = YES) or a spectral flat field
(IMAGE\_FLAT=N).

\item [FLAT\_FIELD] This has a number of uses. If flat frames are being
processed, then this is the output file for the mean flat field frame. If a
flat field division is being done, then this is the input file.

\item [AXIS, NORD] If a spectral flat field is being done, then  AXIS tells
SCP\_PROC what axis is parallel to the spectral axis. Answer either  1 or 2 for
X or Y axis. NORD is the order of the polynomial to be fit to the spectral flat
field.

\item [DARK\_COR] This is a flag to subtract out dark counts from flat or
program fields.

\item [DARK\_FRAME] This has a number of uses.  If dark frames are being
processed, then this is the name of the output file for the mean dark frame. If
a dark frame is being subtracted, then this is the input file name.

\item [TRIM] If this is true, then the frame is trimmed before output.

\item [SIXTEEN] This is a flag to do the sixteen bit correction.

\item [WANT\_ERR] If error arrays are wanted, then this should be set true.

\item [SUFFIX] The output files have the same name as the input files, save for
a suffix which is defined here. (e.g. for an input file named FILE.DST and a
suffix \_PRO then the output file is FILE\_PRO.DST).

\end{description}

% Newpage command here to make the verbatim bit come out on one page.
% If document is modified, it may need to be removed.

\newpage

The following is a sample PROC file:

\begin{verbatim}
          *** Procedure file ***
       Process bias frames?                  ! Y
       Process flat frames?                  ! N
       Process dark frames?                  ! N
       Process program frames?               ! Y
       Subtract bias?                        ! Y
       Subtract bias frame?                  ! Y
       Do overscan correction?               ! Y
       Mean bias frame                       ! CCD_MBIAS
       Remove cosmic rays?                   ! N
       Number of bins for histogramming      ! 0
       Cell size for CR search               ! 0
       Threshold for CR search               ! 0
       Remove dud pixels?                    ! N
       Dud pixel replacement scheme          ! 0
       Divide by flat field?                 ! N
       Mean flat field frame                 !
       Image flat field?                     ! N
       Order of polynomial                   ! 0
       Axis parallel to spectrum             ! 0
       Do dark count correction?             ! N
       Mean dark count frame                 !
       Do 16 bit correction?                 ! N
       Trim resultant program frames?        ! Y
       Want output error arrays?             ! N
       Suffix for output program frames      ! _PRE
\end{verbatim}

In this case the user has opted to process bias and program frames. The mean
bias frame will be called SCP\_MBIAS. Bias subtraction will be done to the
program frames.  The same mean bias frame which has been scaled to the mean
bias in the overscan regions will be used for the bias estimate.  No cosmetic
touch ups (cosmic ray removal, etc) will be done, no 16 bit correction will be
done and no error arrays will be generated.  The resulting arrays will be
trimmed before output and the output names will be the input names with the
suffix \_PRE added.

\subsection {SCP\_PROC} \label{sec:imfil}

SCP\_PROC does all of the real work involved in processing the frames.
There are only three parameters:

\begin{description}

\item [PROCFILE] This is the procedures file as set up by SCP\_PRE. This
is described in more detail in section \ref{sec:proc_pars}.

\item [CHIPFILE] This is the chip file as set up by SCP\_CHIP (section
\ref{sec:chip_pars})

\item [IMAGEFILE] This is a file with a list of images to be processed. The
info must be in a standard format which is described below.

\end{description}

The format of the IMAGES file is very simple.  There are four types of files
which SCP\_PROC recognises: BIAS, DARK, FLAT and PROGRAM (abbreviated,
strangely enough, as B, D, F and P.)  These file types are not so much a
description of what's in the data, but rather a description of how you want the
data processed.  It may be that you have a series of flat field exposures which
you are going to use in SCP\_NOISE. If you identify them as FLAT in the images
file the SCP\_PROC will try to average them and form a mean flat field out of
them, whereas what you really want is for SCP\_PROC to just subtract the bias
and maybe try and zap the cosmic rays.  In this case you would identify these
frames as PROGRAM.  Ok, having confused the issue, here's a sample IMAGES file.

\begin{verbatim}
      #B
      BIAS1;BIAS2;BIAS3;BIAS4;BIAS5;
      BIAS6;BIAS7;BIAS8;BIAS9;BIAS10;
      #F
      VFLAT1;VFLAT2;
      #D
      DARK1;
      #P
      N5929_V;M31_V;3C273_V;
\end{verbatim}

In this file the user has ten bias frames, two flat fields, one dark frame and
three program frames.  Note that a hash mark is used BEFORE each of the type
designations. Note also that there MUST be a semi-colon after each file name.
If an error message comes up moaning about the IMAGES file, it's more than
likely that it's one of these two things which is missing somewhere. Other than
that, it's all very simple.

A description of how the non-program frames are processed is given in section
\ref{sec:proc_flat}.  The processing of the program frames is straight-forward
and should need no explanation.

\section{Processing Bias, Dark and Flat Frames} \label{sec:proc_flat}

In this section a description is given as to how the non-program files are
processed.  This is not a description of the things like the cosmic ray
rejection algorithm, etc., but rather more of a description of the method  used
in averaging these types of frames.

\subsection{Bias Frames}

The mean bias frame is just a straight forward average over the number of
frames involved, with one exception; if data has been flagged as bad, it is
ignored. Thus:

\begin{displaymath}
\overline{B}_{ij} = \sum_{k=1}^{N_F} {B_{kij} \over N_{Pij}}
\end{displaymath}

where $N_F$ is the number of frames, $N_{Bij}$ is the number frames with a bad
data value at pixel $i,j$ and $N_{Pij} = N_F - N_{Bij}$.

\subsection{Dark and Flat Frames}

Both Dark and Flat frames are first divided by their exposure times.  Then they
are averaged.  If the user has specified that error arrays should be generated,
then the averaging is done by a weighted sum where the weight for an individual
pixel is the reciprocal of its variance.  If no error arrays are desired, then
the weights are uniform.  Thus an output processed dark frame will be in units
of ADUs per pixel per second.  The flat frames have one more step through which
to go.  The average (clipped at the $3\sigma$ level) over the data area of the
mean flat frame is found.  Then the mean frame is normalised by this average.
This is to ensure that the number of counts in any program frame which will be
divided by this flat field is roughly conserved.

\section{A Recipe}

In this section a recipe for using this software is described.

\begin{enumerate}

\item The first thing is to decide whether the noise parameters for the chip
are required. If error arrays are wanted and the noise parameters aren't
already known, then the answer here is clearly `yes'. If not, then skip to
item \ref{it:noerr}.

\item Use SCP\_CHIP to define all the parameters of the chip except the noise
parameters --- these can have dummy values.

\item Create an IMAGES file with the bias frames (if those are  to be used for
the bias estimates) and the flat frames. However, make sure that the flat
frames are listed in the IMAGES file as program frames.

\item Run SCP\_PRE and have it set up a PROC file to subtract the bias and do
any cosmetic things which might be wanted.

\item Run SCP\_HEAD to get the exposure times into the headers of the flat
frames.

\item Run SCP\_PROC to process the flat frames.

\item Run SCP\_NOISE to yield the desired noise parameters.

\item \label{it:noerr} Run SCP\_CHIP. Include the noise parameter estimates if
they are required or dummy values if they are not.

\item Create an IMAGES file with all the frames which are to be reduced.

\item Run SCP\_PRE to create a PROC file with the complete procedure as  it is
wanted.

\item Run SCP\_HEAD to make sure the exposure times are now in all the
frames' headers.

\item Finally run SCP\_PROC to do the final processing.

\end{enumerate}

\end{document}
