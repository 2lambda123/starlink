\documentstyle{article} 
\pagestyle{myheadings}

%------------------------------------------------------------------------------
\newcommand{\stardoccategory}  {Starlink User Note}
\newcommand{\stardocinitials}  {SUN}
\newcommand{\stardocnumber}    {99.4}
\newcommand{\stardocauthors}   {S M Beard}
\newcommand{\stardocdate}      {13 May 1991}
\newcommand{\stardoctitle}     {LIBMAINT --- A tool for library maintenance}
%------------------------------------------------------------------------------

\newcommand{\stardocname}{\stardocinitials /\stardocnumber}
\renewcommand{\_}{{\tt\char'137}}     % re-centres the underscore
\markright{\stardocname}
\setlength{\textwidth}{160mm}
\setlength{\textheight}{240mm}
\setlength{\topmargin}{-5mm}
\setlength{\oddsidemargin}{0mm}
\setlength{\evensidemargin}{0mm}
\setlength{\parindent}{0mm}
\setlength{\parskip}{\medskipamount}
\setlength{\unitlength}{1mm}

%------------------------------------------------------------------------------
% Add any \newcommand or \newenvironment commands here
%------------------------------------------------------------------------------

\begin{document}
\thispagestyle{empty}
SCIENCE \& ENGINEERING RESEARCH COUNCIL \hfill \stardocname\\
RUTHERFORD APPLETON LABORATORY\\
{\large\bf Starlink Project\\}
{\large\bf \stardoccategory\ \stardocnumber}
\begin{flushright}
\stardocauthors\\
\stardocdate
\end{flushright}
\vspace{-4mm}
\rule{\textwidth}{0.5mm}
\vspace{5mm}
\begin{center}
{\Large\bf \stardoctitle}
\end{center}
\vspace{5mm}

%------------------------------------------------------------------------------
%  Add this part if you want a table of contents
%  \setlength{\parskip}{0mm}
%  \tableofcontents
%  \setlength{\parskip}{\medskipamount}
%  \markright{\stardocname}
%------------------------------------------------------------------------------


\section{Introduction}

LIBMAINT is a procedure which simplifies the maintenance of software
which is held as modules in a text/object library pair.
New libraries can be easily created, and modules inserted, extracted,
replaced, examined and printed with fairly simple commands.
Using LIBMAINT for library maintenance also ensures that corresponding
modules within a text/object library pair do not get out of step with
one another.
There are also commands for optimising the disk space used by the
libraries.

LIBMAINT can also be used to maintain modules held in a help library.
Modules can be inserted, extracted, replaced and printed, and the
contents of the library can be examined as if called up from a HELP
command.

The information in this document refers to LIBMAINT version 2.6.
Further information is available by typing;
\begin{verbatim}
      $ LIBMHELP
\end{verbatim}
A hard-copy of the on-line help information may be obtained by typing;
\begin{verbatim}
      $ LIBMAINT LIBMAINT_DIR:LIBMAINT\HELPMODE
      $ LIBPRINT LIBMAINT
\end{verbatim}

LIBMAINT uses the VAX/VMS librarian, which is described in the VAX/VMS
librarian utility manual.

\section{Using Libraries}

Software can be made a lot easier to maintain by storing all the
subroutines used in a text/object library pair.
Each subroutine is stored individually as a module in the text library,
and for each of these a corresponding compiled object module is stored
in the object library.
Most Starlink software is stored in this way.
The main advantages of storing software in this way are:
\begin{itemize}
\item[o] Any subroutine can be easily extracted and edited
individually.
\item[o] When a subroutine is changed, only that particular
subroutine needs to be recompiled.
\item[o] A library of generally-useful subroutines can be used by
more than one program.
\item[o] Libraries enable subroutines with similar functions to be
collected together into recognisable packages.
\end{itemize}
LIBMAINT can make the maintenance of these libraries easier.
For example, a library of subroutines can be created by typing
the following:
\begin{verbatim}
      $ LIBMAINT                       ! Start LIBMAINT
      $ LIBCRE MYLIB                   ! Create a library pair called MYLIB
      $ EDIT MAINPROG.FOR              ! Create the main program
      $ EDIT SUB1.FOR                  ! Create subroutine 1
      $ INSERT SUB1                    ! Insert it into the library
      $ EDIT SUB2.FOR                  ! Create subroutine 2
      $ INSERT SUB2                    ! Insert it into the library
\end{verbatim}
The program created above can be compiled and linked with
the following commands:
\begin{verbatim}
      $ FORTRAN MAINPROG               ! Compile the main program
      $ LINK MAINPROG,MYLIB/LIB        ! Link it with the MYLIB library
\end{verbatim}
For the small library created above, disk space may be saved
by typing:
\begin{verbatim}
      $ LIBCOMP MYLIB                  ! Compress the MYLIB libraries
\end{verbatim}
Subroutine SUB1 can be modified by typing:
\begin{verbatim}
      $ EXTRACT SUB1                   ! Extract SUB1 from the MYLIB library
      $ EDIT SUB1.FOR                  ! Modify it
      $ REPLACE SUB1                   ! Replace it back into the library
\end{verbatim}
Further details on setting up LIBMAINT, and a full list of available
commands, are described in the rest of this document.

\section{Setting up LIBMAINT}

\subsection{The Basic Set Up}

Before using LIBMAINT in any login session, it needs to be set up
by typing the command;
\begin{verbatim}
      $ LIBMAINT
\end{verbatim}
When set up, LIBMAINT will respond with a message;
\begin{verbatim}
      + LIBMAINT: version X.Y (date)
\end{verbatim}
which will indicate the current version of LIBMAINT.

If frequent use is to be made of LIBMAINT, the set-up command can be
included in your LOGIN.COM (provided it comes after executing
SSC:LOGIN.COM). For example :
\begin{verbatim}
      $ @SSC:LOGIN
           "
           "
      $ LIBMAINT
\end{verbatim}
Once set up, DCL symbols will be defined which make all the LIBMAINT
commands available.

\subsection{Setting Up an Initial Default Library}

An initial default library can be set up by specifying it
as a parameter to the LIBMAINT command. {\it i.e.}
\begin{verbatim}
      $ LIBMAINT [default-library]
\end{verbatim}
If no default library is specified it is initially undefined
until set up with DEFLIB or some other LIBMAINT command.

\subsection{Initialising LIBMAINT Switches}

{\it Beginners may skip this section.}

LIBMAINT can be run in several different ways by setting various
switches, which control its action, on or off.
In this way, the action of LIBMAINT can be tailored to suit any
particular user.
The following switches are used:

\newlength{\numlen}
\settowidth{\numlen}{000000000000}
\settowidth{\labelsep}{000}

\begin{list}{}{\setlength{\labelwidth}{\numlen}\setlength{\leftmargin}{\numlen}
\addtolength{\leftmargin}{\labelsep}}

\item[HELPMODE] This controls whether LIBMAINT operates on a help
library or text/object library pair.
When this switch is on, all the LIBMAINT commands operate on a help
library.
When the switch is off, the commands operate on a text/object library
pair.
By default this switch is off, but it may be switched on for maintenance
of a help library.

\item[COMPILING] This controls whether there is automatic compilation of
text modules when they are inserted into a library, and automatic
insertion of the resultant object module into the object library.
By default this is switched on, and should only be switched off in cases
where a text library alone is being maintained.

\item[LOGGING] This controls whether there is automatic logging of the
reasons for the insertion or replacement of each module to a log file.
By default this is switched off, but may be turned on by conscientious
users who require a full record of all modifications.

\item[TYPING] This controls whether LIBMAINT appends the file type to the end
of the name of each module inserted in the text library (except those of
type {\tt .FOR}).
By default this is switched on, and file types will be appended to
module names.
This switch can be switched off for the maintenance of libraries created
by tools other than LIBMAINT, which do not have non-Fortran file types
appended to their module names.

\item[UNDERTAKING] This controls whether underscores are removed from a
module name before attempting to delete it from the text library
(using LIBDEL).
By default this is switched off, and LIBMAINT assumes that module names
in the text and object libraries are identical.
Some pre-VMS version 4 libraries may have text module names which
are the same as the object module names but with underscores removed.
This switch may be turned on to enable LIBMAINT to maintain these
libraries.
\end{list}

The switches can be initialised when LIBMAINT is set up by specifying
either ``\verb+\+switch-name'' as a parameter to the LIBMAINT command to turn a
switch on, or ``\verb+\+NOswitch-name'' to turn it off.
For example;
\begin{verbatim}
      $ LIBMAINT \LOGGING       ! Initialise LIBMAINT with logging ON
      $ LIBMAINT \NOLOGGING     ! Initialise LIBMAINT with logging OFF
\end{verbatim}
The backslash character tells LIBMAINT that a switch is being specified,
and not a default library.
More than one switch may be specified, and each one should be
preceded by a backslash.
It is possible to initialise a default library at the same time
as initialising the switches. {\it e.g.}
\begin{verbatim}
      $ LIBMAINT MYLIB\HELPMODE\LOGGING\NOUNDERTAKING
\end{verbatim}
will define MYLIB as the default library, and initialise HELPMODE
and LOGGING on and UNDERTAKING off.

Full details of the effects of the switches on the function
of each command are described in the on-line help information.
It is best initially to leave these switches at their default
setting.

\section{The LIBMAINT Commands}

LIBMAINT has the following commands.
Only a brief description of each command will be given here.
Full details of each command, together with all the possible ways it can
be modified by qualifiers and the setting of LIBMAINT switches, are
described in the on-line help information;
\begin{verbatim}
      $ LIBMHELP command-name
\end{verbatim}


\settowidth{\numlen}{00000000}
\settowidth{\labelsep}{0000}

\begin{list}{}{\setlength{\labelwidth}{\numlen}\setlength{\leftmargin}{\numlen}
\addtolength{\leftmargin}{\labelsep}}

\item[DEFLIB]
\begin{verbatim}
$ DEFLIB library-spec
\end{verbatim}
Define a default library to be manipulated by the use of
the LIBMAINT commands.

The library-spec is of the form {\tt library-name[.library-type]}, where the
library-type is optional and can be ``{\tt .TLB}'' (specifying a text/object
library pair) or ``{\tt .HLB}'' (specifying a help library).
If a library-type is specified, the HELPMODE switch is set
automatically.

\item[EXTRACT (or LIBEXT)]
\begin{verbatim}
$ EXTRACT module-spec [library-spec] [file-spec]
\end{verbatim}
Extract a specified module from a text (or help) library.
(Users of FIGARO, where there is a symbol clash, may use
the command LIBEXT).

The module-spec is of the form ~{\tt module-name[.file-type]}.
The module name should correspond to the name of the module to be
extracted from the text library, and should also be a legal Files-11
format file name, which is the name of the file to receive the output.

The file-type is optional (defaulting to ``{\tt .FOR}'') and if included it
represents the type of the file to contain the output text module.

The library-spec is optional, and if not specified defaults to the
currently-specified default library.
If a new library is specified, it becomes the new default library.

The file-spec is the name of the file to be created.
It is optional, and if not specified defaults to a file name
derived either from the name of the module or,
if a wild card has been specified for the module name, from the name of
the library. 

{\bf NOTE:}~{\it If the wild card character ``*'' is given instead of a module 
name, then ALL the modules will be extracted from the text library and appended
together into ONE file whose name is specified in file-spec. If file-spec is 
not specified, the file will be called {\tt library-name.file-type}.
To extract modules into separate files, OLBCRE can be used with
the COMPILING switch off.}

\item[INSERT]
\begin{verbatim}
$ INSERT module-spec [library-spec]
\end{verbatim}
Insert a new module into the text and object library
pair (or help library).
If the module already exists in the library, an error message
will be generated.
(Use REPLACE to replace an existing module).

The module-spec is of the form {\tt module-name[.file-type]}.
This should be the name of the file containing the module to be inserted
into the libraries, and will form the module name in the text library.

The file-type is optional (defaulting to {\tt .FOR}) and is used when
deciding which compiler should be used in inserting the module into the
object library.
The following compilers are used;
\begin{center}
\begin{tabular}{ll}
        File-type &      Compiler \\
\\
        .BAS      &      BASIC \\
        .BLI      &      BLISS \\
        .CC       &      C \\
        .FOR      &      FORTRAN (default) \\
        .GEN      &      ``GENERIC'' (optional - see SUN/7) \\
        .MAR      &      MACRO \\
        .MSG      &      MESSAGE \\
        .PAS      &      PASCAL \\
\end{tabular}
\end{center}
The library-spec is optional, and if not specified defaults to the
currently-specified default library.
If a new library is specified, it becomes the new default library.

\item[LIBCOMP]
\begin{verbatim}
$ LIBCOMP [library-spec]
\end{verbatim}
Compress a text/object library pair (or help library) to recover
any unused blocks of disc space allocated but not used.
The resultant libraries are written in normal data-expanded format.

The library-spec is optional, and if not specified defaults to the
currently-specified default library.
If a new library is specified, it becomes the new default library.

\item[LIBCRE]
\begin{verbatim}
$ LIBCRE library-spec
\end{verbatim}
Create a new text and object module library pair (or help
library).

The library-spec is of the form {\tt library-name[.library-type]}, where the
library-type is optional and can be {\tt .TLB} (specifying a text/object
library pair) or {\tt .HLB} (specifying a help library).
If a library-type is specified, the HELPMODE switch is set
automatically.
The library specified becomes the default library.

\item[LIBDEL]
\begin{verbatim}
$ LIBDEL module-spec(s) [library-spec]
\end{verbatim}
Delete the specified module (or modules) from the text and
object module library pair (or help library).
Wild cards may be used.

\item[LIBHELP]
\begin{verbatim}
$ LIBHELP help-string
\end{verbatim}
Display the contents of a help library in the format
obtained by a user typing:
\begin{verbatim}
$ HELP/LIBRARY=help-library  help-string
\end{verbatim}
The library-spec cannot be specified with this command, so the
default help library is used, as specified by DEFLIB.

\item[LIBLIST]
\begin{verbatim}
$ LIBLIST [library-spec]
\end{verbatim}
Display the contents of the text library (or help library).

The library-spec is optional, and if not specified defaults to the
currently-specified default library.
If a new library is specified, it becomes the new default library.

\item[LIBPRINT]
\begin{verbatim}
$ LIBPRINT module-spec(s) [library-spec]
\end{verbatim}
Print a module or modules from the text library (or help
library).

The module specification is of the form {\tt module-name[.file-type]}
and wild cards are allowed.
The module name should correspond to the
name of the module to be printed from the library and should
also be a legal Files-11 format file name [plus file type].
When more than one module is printed, each is printed on
a separate page and a unique file type of the form .LP\_xx is
chosen.

\item[LIBREDUCE]
\begin{verbatim}
$ LIBREDUCE [library-spec]
\end{verbatim}
Compress a text/object library pair (or help library) and
convert them into DCX data-reduced format, so as to minimise the disc
space used.

The library-spec is optional, and if not specified defaults to the
currently-specified default library.
If a new library is specified, it becomes the new default library.
(LIBCOMP can be used to convert libraries back to data-expanded format).

\item[LIBTYPE]
\begin{verbatim}
$ LIBTYPE module-spec(s) [library-spec]
\end{verbatim}
Type on SYS\$OUTPUT a module or modules from the text
library (or help library).

The module specification is of the form {\tt module-name[.file-type]}
and wild cards are allowed.
The usage is as for LIBPRINT except that no file is created.
The output may be interrupted by CTRL/O or CTRL/Y at any
time.

The library-spec is optional, and if not specified defaults to the
currently-specified default library.
If a new library is specified, it becomes the new default library.

\item[OBJLIST]
\begin{verbatim}
$ OBJLIST [library-spec]
\end{verbatim}
Display the contents of the object library.

The library-spec is optional, and if not specified defaults to the
currently-specified default library.
If a new library is specified, it becomes the new default library.

\item[OLBCRE]
\begin{verbatim}
$ OLBCRE [library-spec] [start-module]
\end{verbatim}
Create a new object library from a text library.

OLBCRE extracts all the modules from an existing text library (of the
same name as the object library), compiles each one with the appropriate
compiler and inserts each resultant object module into the new object
library.
The library-spec is optional, and if not specified defaults to the
currently-specified default library.
If a new library is specified, it becomes the new default library.
The start-module parameter may be used to specify at which module OLBCRE
starts, which is particularly useful for restarting an OLBCRE which
failed with a compilation error half way through a library.
By default, OLBCRE starts at the first module in a library.

{\bf NOTE:}~{\it When the LIBMAINT COMPILING switch is off, OLBCRE can be used
for extracting all the modules from a text library into individual
files.}

\item[PREFIX]
\begin{verbatim}
$ PREFIX prefix
\end{verbatim}
Define a prefix to be inserted at the beginning of every
subsequent module name.

This can be used to save typing in cases where
the majority of the modules in a library begin with the
same sequence of characters.

\item[REPLACE]
\begin{verbatim}
$ REPLACE module-spec [library-spec]
\end{verbatim}
Replace a specified module in a text and object library
pair (or help library).

This is exactly the same as the INSERT command, except that if
the specified module already exists a library it will
be replaced by the new version.

\item[SHOLIB]
\begin{verbatim}
$ SHOLIB
\end{verbatim}
Show the library currently set up as the default for the
LIBMAINT commands, and also the file specifications of the text and
object libraries (or help library).

\item[SHOPRE]
\begin{verbatim}
$ SHOPRE
\end{verbatim}
Show the prefix, defined by the PREFIX command, to be added to
the beginning of all module names.

\item[SHOSWITCH]
\begin{verbatim}
$ SHOSWITCH
\end{verbatim}
Show the current settings of all the LIBMAINT switches.

\item[SWITCH]
\begin{verbatim}
$ SWITCH switch-name ON/OFF
\end{verbatim}
Switch the specified LIBMAINT switch on or off.
Details of the LIBMAINT switches have already been described above.
\end{list}

\section{Librarian and Compiler Qualifiers}

It is possible to pass DCL command qualifiers through LIBMAINT
to the VAX/VMS librarian, or the compiler used to compile the
modules, and hence achieve much greater flexibility.
Qualifiers for the librarian may simply be appended
to the ends of the parameters given to LIBMAINT, and begin
with a slash ``/''.
Qualifiers for the compiler are given in a similar way,
but the left-most qualifier should begin with a double slash
``//''.
Full details on the qualifiers which may be used with
each LIBMAINT command may be found in the on-line help
information, e.g.;
\begin{verbatim}
      $ LIBMHELP command-name Library_qualifiers
\end{verbatim}
A few examples are given below to illustrate some
uses for the qualifiers:
\begin{verbatim}
      $ LIBPRINT */SINCE=1-JAN-1987
                  | Librarian     |
                  | qualifier     |
\end{verbatim}
Print all the modules from the current default library which
have been inserted since 1-JAN-1987.
\begin{verbatim}
      $ INSERT MODULE//D_LINES/LIST
                     | Compiler    |
                     | qualifiers  |
\end{verbatim}
Insert module MODULE into the default text/object library pair,
compiling the module with FORTRAN/D\_LINES/LIST.
\begin{verbatim}
      $ INSERT MODULE/SELECTIVE_SEARCH//D_LINES/LIST
                     | Librarian      | Compiler    |
                     | qualifier      | Qualifiers  |
\end{verbatim}
Insert module MODULE into the default text/object library pair,
compiling the module with FORTRAN/D\_LINES/LIST, and defining
the module as a candidate for a selective search.
\begin{verbatim}
      $ OLBCRE MYLIB//DEBUG/NOOPTIMISE/D_LINES/CHECK=ALL/LIST
                    | Compiler qualifiers                   |
\end{verbatim}
Compile all the modules in the library MYLIB, with qualifiers suitable
for debugging, and insert the object modules produced into a new object
library.

\section{Using Text Libraries of Include Files} 

VMS allows Fortran INCLUDE files to be held in a text library.
For instance, a Fortran routine may contain the statement;
\begin{verbatim}
      INCLUDE `(TEST.INC)'
\end{verbatim}
where the string in brackets is actually the name of a module in
a text library.
Suppose the Fortran routine is called ROUTINE.FOR, and the module
TEST.INC is in the text library TEST.TLB. 
When compiling the routine, the command;
\begin{verbatim}
      $ FORTRAN ROUTINE+TEST/LIB
\end{verbatim}
tells the compiler to search the TEST.TLB library for the INCLUDE 
module.
It is possible to maintain libraries of routines which include text
modules in this way with LIBMAINT by clever use of the ``compiler
qualifier'' mechanism described above. 
The name of the text library can be appended to a dummy compiler
qualifier, such as /NOLIST.
If the above routine is to be inserted in a text/object library pair,
the following LIBMAINT command will work:
\begin{verbatim}
      $ INSERT ROUTINE//NOLIST+TEST/LIB
\end{verbatim}
The same also applies to the LIBMAINT commands REPLACE and OLBCRE.
(I am indebted to Richard Prestage for the above suggestion).

\section{USE OF LIBMAINT WITH THE GENERIC COMPILER}
LIBMAINT can be used to maintain generic modules transparently
by means of the GENERIC compiler.
To make use of this compiler requires than a DCL global symbol
GENERIC should be set up to execute it.
LIBMAINT will then use the GENERIC compiler automatically
upon meeting any module of file type {\tt .GEN}.
Full details of the GENERIC compiler may be found in SUN/7.
{\it (N.B. References to SUN/7 in this document refer to versions
dated February 1987 or later).}

\section{Other Library Maintenance Utilities}

Other library maintenance utilities which may be used in conjunction
with LIBMAINT are:


\settowidth{\numlen}{00000000000}
\settowidth{\labelsep}{000}

\begin{list}{}{\setlength{\labelwidth}{\numlen}\setlength{\leftmargin}{\numlen}
\addtolength{\leftmargin}{\labelsep}}

\item[AUTODOC] A utility for automatically editing the ``History'' section
of modules when they are INSERTed and REPLACEd by LIBMAINT (with
LOGGING switched on).
This utility is currently undocumented and unreleased.
(Contact REVAD::JAC if interested).

\item[CHLIB] A utility for carrying out a global substitution on all the
modules in a text library.
This utility is currently undocumented and unreleased, and uses the CH
utility in the software tools.
(Contact REVAD::JAC if interested).

\item[LIBIND] A utility for generating a listing of the modules
within a text library in a format which is guaranteed not
to change in future releases of VMS.
(See SUN/8).

\item[LIBPRE] A utility for extracting all the preamble comments from
all the modules in a text library.
(See SUN/8).

\item[RNLIB] A utility for renaming all the modules in a text library
by changing the prefix which all the modules start with.
This utility is currently undocumented and unreleased.
(Contact REVAD::SMB if interested).

\end{list}

\end{document}
