\documentstyle{article} 
\pagestyle{myheadings}

%------------------------------------------------------------------------------
\newcommand{\stardoccategory}  {Starlink User Note}
\newcommand{\stardocinitials}  {SUN}
\newcommand{\stardocnumber}    {47.3}
\newcommand{\stardocauthors}   {D L Terrett}
\newcommand{\stardocdate}      {14 September 1989}
\newcommand{\stardoctitle}     {TAPECOPY --- Magnetic tape copying utility}
%------------------------------------------------------------------------------

\newcommand{\stardocname}{\stardocinitials /\stardocnumber}
\markright{\stardocname}
\setlength{\textwidth}{160mm}
\setlength{\textheight}{240mm}
\setlength{\topmargin}{-5mm}
\setlength{\oddsidemargin}{0mm}
\setlength{\evensidemargin}{0mm}
\setlength{\parindent}{0mm}
\setlength{\parskip}{\medskipamount}
\setlength{\unitlength}{1mm}

\renewcommand{\_}{{\tt\char'137}}

\begin{document}
\thispagestyle{empty}
SCIENCE \& ENGINEERING RESEARCH COUNCIL \hfill \stardocname\\
RUTHERFORD APPLETON LABORATORY\\
{\large\bf Starlink Project\\}
{\large\bf \stardoccategory\ \stardocnumber}
\begin{flushright}
\stardocauthors\\
\stardocdate
\end{flushright}
\vspace{-4mm}
\rule{\textwidth}{0.5mm}
\vspace{5mm}
\begin{center}
{\Large\bf \stardoctitle}
\end{center}
\vspace{5mm}

\section{Introduction}
TAPECOPY is a utility for duplicating magnetic tapes. The contents of the
original tape are copied to a disk file (the container file) which is then
copied to a new tape to create an exact copy of the original. TAPECOPY can
therefore make copies on a system with only one tape drive. TAPECOPY can also
copy selected files from one tape to another.

The Tape Processing
Utility (TPU, SUN/68) can also copy tapes but requires two tape drives.

Before using TAPECOPY the {\tt TAPECOPY} command must be defined by:
\begin{verbatim}
    $ SET COMMAND TAPECOPY_DIR:TAPECOPY
\end{verbatim}

TAPECOPY recognises three magnetic tape formats:
\begin{description}
\item[Foreign] tapes do not have a label and the files on the tape do
not have names. Files are separated by end-of-file marks and two consecutive
end-of-file marks marks the end of the volume.

\item[ANSI] tapes have one or more volume header labels which contain  the
volume name, volume protection etc. Each file consists of two or more file
header labels, an end-of-file
mark, the file contents, another end-of-file mark and a set of file trailer
labels. The header labels contain the file name and other file attributes
and a sequence number.

The end of-volume is indicated by two consecutive end-of-file marks
immediately following a file trailer label. A file can be empty and so two
consecutive file marks immediately following a file header label does not
indicate the end of volume.
An ANSI tape that has just been initialised with the {\tt INITIALIZE} command
has a volume label followed by a dummy empty file with a blank file name.

ANSI tape volumes can occupy more than one reel of tape but TAPECOPY cannot
handle multi-volume ANSI tapes.

\item[BACKUP] tapes are ANSI format but BACKUP never writes empty files.
\end{description}


\section{Copying a tape}

TAPECOPY is most commonly used to make a copy of a complete tape volume.
The contents of a magnetic tape are copied to disk by:
\begin{verbatim}
    $ MOUNT/FOREIGN/NOWRITE <tape>
    $ TAPECOPY <tape> <diskfile>
    $ DISMOUNT <tape>
\end{verbatim}
where {\tt <tape>} is the name of a magnetic tape device (eg., {\tt MTA1:,
MUA0:}) and {\tt <diskfile>} is any valid disk file specification (which can 
contain a DECnet node specification).

A new tape can then be created by:
\begin{verbatim}
    $ MOUNT/FOREIGN <tape>
    $ TAPECOPY <diskfile> <tape>
\end{verbatim}

If the tape is an ANSI volume which might contain some empty files then {\tt
TAPECOPY/ANSI} should be used when copying from tape to disk so that the empty
file is not mistaken for the end-of-volume.

\section{Command qualifiers}
\subsection{Tape to disk}

The following qualifiers apply to tape-to-disk copy operations:
\begin{description}

\item[\tt/ALLOCATION=] Specifies the number of blocks initially allocated
to the disk container file when it is created.

\item[\tt/ANSI] Enables the correct handling of empty files on ANSI tapes.

\item[\tt/BACKUP] Indicates that the tape is ANSI format but without enabling
the handling of empty files. When used with the {\tt/START=} qualifier the tape
is positioned more quickly because the file header labels do not have to be
checked.

\item[\tt/FILE=(START:m,END:n)] Copies files {\tt m} through {\tt n} only. If
{\tt START} is omitted, the copy starts at the beginning of the tape; if {\tt
END} is omitted the copy continues to the end of the tape.

\item[\tt/RECORD\_SIZE=] Sets the maximum size of the records written to the disk
container file. The default (4096) is the largest record size that can be
accessed over DECnet with the default setting of the RMS network block count.
Using larger records may improve performance.

\item[\tt/EXTENDSIZE=] Specifies the number of blocks by which an output disk
file is extended each time more blocks need to be allocated to the file.

\end{description}

\subsection{Disk to tape}

The following qualifiers apply to disk-to-tape copy operations:

\begin{description}
\item[\tt/APPEND] Positions the tape at end-of-volume before
writing. Note that for an ANSI tape this will result in the files have
incorrect sequence fields unless {\tt/ANSI} or {\tt/BACKUP} is also specified.

\item[\tt/ANSI] Enables the correct handling of empty files and the dummy file
written by {\tt \$INITIALIZE} when positioning at the end-of-volume when used
with the {\tt/APPEND}. File header sequence numbers are modified when required.

\item[\tt/BACKUP] As for {\tt/ANSI} except that empty and dummy files are not
handled correctly enabling the end-of-volume to be found more quickly.

\item[\tt/DENSITY=] Specifies the density of an output tape. This qualifier
only works if the tape deck being used is capable of having its recording
density set by software.
\end{description}

\subsection{All copy operations}

The following qualifiers apply to all copy operations:

\begin{description}
\item[\tt/BLOCKSIZE=] Specifies the maximum block size that can be handled.
Using a larger than necessary can result in TAPECOPY failing because
of insufficient virtual address space.

\item[\tt/BUFFERS=] Specifies the number of buffers to be used for tape
I/O. Increasing the number of buffers may improve performance, particularly
for streaming tape drives.

\item[\tt/DOS11] Selects the optimum values for {\tt/BLOCKSIZE} and
{\tt/BUFFERS} for a DOS format tape.

\item[\tt/NOREWIND] Does not rewind the tape before or after the copy operation.
The tape is left positioned in between the two end-of-file marks at the end of
the volume.

\item[\tt/RT11] Selects the optimum values for {\tt/BLOCKSIZE} and {\tt/BUFFERS}
for an RT11 format tape.

\item[\tt/VERIFY] Rewinds the tape after a copy operation and verifies the
copy against the original. This qualifier cannot be used with {\tt/APPEND},
{\tt/START=} or {\tt/END=}.

\end{description}

\subsection{Container file directory}

The qualifier {\tt/DIRECTORY} list the ``contents'' of a disk container file.
For an ANSI tape the file names and size in blocks of each file is listed; for
non-ANSI tapes the number of blocks and the size of the longest and shortest
block in each file  is listed. TAPECOPY attempts to recognize
the format of the tape by examining the size and format of the first block; it
can be forced to treat the file as ANSI or DOS format by using the {\tt/ANSI}
or {\tt/DOS} qualifiers. No other qualifiers are valid with {\tt/DIRECTORY}.

\subsection{Qualifier defaults}
The default values and limits for the qualifiers that take a numeric value are:
\[\tt\begin{tabular}{|l|r|r|r|}\hline
Qualifier    &Default &Minimum &Maximum \\\hline
ALLOCATION   &2000    &0       &4294967295\\
BLOCKSIZE    &65535   &512     &65535   \\
BUFFERS      &6       &1       &30      \\
DENSITY      &1600    &        &        \\
EXTENDSIZE   &250     &0       &65535   \\
RECORD\_SIZE &4069    &512     &32767   \\
\hline\end{tabular}\]

Valid values for {\tt/DENSITY} are 800, 1600 and 6250.

\section{Selective copying of ANSI tapes}\label{selective}

When copying part of an ANSI or BACKUP tape the volume label is only written to
the disk container file if {\tt/FILE=(START:1)}. An ANSI tape must have a
volume label so the container file must  be appended to an already labelled
tape (using {\tt/APPEND} and {\tt/ANSI} or {\tt/BACKUP}) if a valid ANSI format
tape is to be created. 

If the tape being appended to contains no files (ie. it has just been
initialised) then {\tt/ANSI} must be used so that the dummy file is
overwritten.

If {\tt/NOREWIND} is used with an ANSI or BACKUP tape the file header sequence
numbers will not be correct and the files will not be accessible with
{\tt\$COPY} or {\tt\$DIRECTORY}. However, BACKUP ignores the sequence numbers
and will be able to access save sets. \end{document}
