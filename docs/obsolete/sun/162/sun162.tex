\documentclass[twoside,11pt,nolof]{starlink}

% -----------------------------------------------------------------------------
% ? Document identification
\stardoccategory    {Starlink User Note}
\stardocinitials    {SUN}
\stardocsource      {sun\stardocnumber}
\stardocnumber      {162.1}
\stardocauthors   {A C Davenhall \\
\textit{Department of Physics and Astronomy, University of Leicester} }
\stardocdate        {18 March 1993}
\stardoctitle     {A Guide to\\
                        Astronomical Catalogues, Databases and Archives\\
                        available through Starlink}
\stardocabstract  {
This note is a guide to the astronomical catalogues, databases and
archives which are either available on Starlink computers or which can
be accessed from Starlink computers. It also describes the Starlink
software for manipulating catalogues and tables.
A brief description is given of each facility or software item.
}
% ? End of document identification

% -----------------------------------------------------------------------------
% ? Document-specific \providecommand or \newenvironment commands.
% ? End of document-specific commands
% -----------------------------------------------------------------------------
%  Title Page.
%  ===========
\begin{document}
\scfrontmatter

\section{Introduction\xlabel{introduction}}

This note is a guide to the astronomical catalogues, databases and
archives which are either available on Starlink computers or which can
be accessed from Starlink computers. It also describes the Starlink
software for manipulating catalogues and tables. For each facility or
item of software only a brief description is given. However, where
appropriate, references to further information are included. Inevitably,
a compendium like this one cannot be complete. I have tried to include
the more commonly used datasets and facilities, but, in particular, the
lists of CD-ROMs and facilities overseas are not exhaustive. Such
omissions are in no way pejorative of the facilities missed out. Also,
the document describes only facilities that are currently available; it
does not mention facilities that are still being developed or planned
enhancements to existing facilities.

Because existing facilities evolve and new ones become available I
anticipate that the document will be revised periodically. In order that
it should remain useful and accurate I welcome corrections, comments and
suggestions for additional databases to include. If you have any such
comments please send them to: Clive Davenhall, Department of Physics and
Astronomy, University of Leicester, University Road, Leicester, LE1 7RH.
E-mail: LTVAD::ACD (Starlink DECNET), 19838::ACD (SPAN) or
acd@uk.ac.le.star (JANET).

Finally, a note on lexicography is probably in order. Wherever possible
I have tried to respect the choice of the original authors in spelling
the titles of their catalogues and of institutions in spelling their
names. This approach is the reason for the apparently inconsistent
spelling of `catalogue' and `centre.'

\section{Starlink catalogue manipulation software
\xlabel{starlink_catalogue_manipulation_software}}

The Starlink software collection includes a number of items for
manipulating astronomical catalogues and similar tabular datasets. The
software items available are described below. Mandatory items are
installed at every Starlink node; optional items are installed at the
discretion of the local site manager.

\subsection{SCAR\xlabel{scar}}

The SCAR (Starlink Catalogue Access and Reporting) system is a
general-purpose relational database management system for manipulating
astronomical catalogues and private tables. It was written by the FIIS
group at the Rutherford Appleton Laboratory (RAL). It is a potentially
powerful system which provides virtually all the facilities needed to
manipulate astronomical catalogues. However, it is a complex piece of
software that is difficult to learn and which can be unfriendly and
unforgiving to use.

SCAR is an optional Starlink software item, but if it is installed, it
comes with a set of standard astronomical catalogues. A larger
collection of catalogues in SCAR format are maintained permanently
on-line on STADAT (see Section~\ref{STADAT}, below). Also, it is
possible to add private catalogues.

\textit{Starlink status:} optional

\textit{Starlink support:} best efforts

\textit{Availability:} VAX/VMS only

\textit{Documentation:}

\textit{SCAR --- Beginner's Guide}
(SUN/106) by A.R. Wood, 19th October
1990.

\textit{SCAR --- Starlink Catalogue Access and Reporting}
(SUN/70) by
H.J. Walker, J.H. Fairclough and M.D. Lawden, 1st October 1990.

New users are strongly recommended to start with the \textit{Beginner's
Guide.}

\subsection{CATPAC\xlabel{catpac}}

CATPAC is a replacement for SCAR also developed by the FIIS group at
RAL. It accesses catalogues held in the same format as SCAR, and,
indeed, uses some of the same low-level subroutine libraries. Thus,
CATPAC can access any SCAR catalogue. It is easier to use and more
robust than SCAR, but currently does not provide all the facilities
available in SCAR.

\textit{Starlink status:} optional

\textit{Starlink support:} high

\textit{Availability:} VAX/VMS only

\textit{Documentation:}

\textit{CATPAC --- Catalogue Applications Package}
\xref{(SUN/120}{sun120}{}) by A.R. Wood,
18th August 1992.

\subsection{R-EXEC\xlabel{rexec}}

R-EXEC is a general-purpose scientific relational database management
system developed by the Scientific Databases Section at RAL. It is
primarily used by the geophysics and space physics communities and
contains no specifically astronomical functionality. It is robust,
reliable and reasonably well documented, but is limited by its lack of
astronomical functionality. R-EXEC does not come with a set of
astronomical catalogues. However, a facility to read FITS ASCII tables
has been added.

\textit{Starlink status:} optional

\textit{Starlink support:} best efforts

\textit{Availability: } currently only VAX/VMS and MS-DOS (though the
latter is not distributed by Starlink).

\textit{Documentation: }

A user guide is available: \textit{R-EXEC Guide and Reference}, version
2.4 (RAL-90-085) by B.J. Read, November 1990 (SERC, RAL). Starlink site
managers should hold copies.

\subsection{CHART\xlabel{chart}}

CHART is a package for plotting finding charts. It was originally
developed at the Royal Greenwich Observatory over twenty years ago and
subsequently has been worked on by many people in different
institutions. It is solely concerned with plotting finding charts. CHART
comes with a set of catalogues which are an integral part of the package
and which are written in a very specific format. It is not easy to add
new catalogues into the system.

\textit{Starlink status:} standard

\textit{Starlink support:} little / none

\textit{Availability:} VAX/VMS only

\textit{Documentation:}

\textit{CHART --- Finding chart and stellar data}
(\xref{SUN/32}{sun32}{}) by P.M. Allan,
24th April 1989.


\section{Databases available on CD-ROM
\xlabel{databases_available_on_cdrom}}

The CD-ROM is becoming an increasingly popular medium for distributing
astronomical catalogues and databases. CD-ROMs are a pre-recorded
medium, the same size and shape as audio compact disks, with which they
are electronically and mechanically identical. They are particularly
suited to producing (and subsequently distributing) many copies of a
dataset from a single master, in a manner analogous to conventional
publication or the manufacture of audio disks. The disks are written
in a standard format, ISO 9660, which ensures that they are readable
on the products of a wide variety of manufacturers. Each disk can hold
up to approximately 650Mb of data.

The data stored on the disks may be held in some standard astronomical
format, such as FITS, or in a `one-off' format peculiar to the
individual dataset. The disks sometimes contain software to access and
display their contents. Starlink provides the CDCOPY utility for
accessing CD-ROMs. This utility is described in \textit{CDCOPY - Reading
CD-ROMs}
(SUN/69) by C.G. Page, 24th April 1992.

Many astronomical datasets are now available on CD-ROM. Some of the more
generally useful ones are described below, though the list is by no
means comprehensive. In particular, the selection is biased in favour of
sidereal astronomy and against solar system studies; several sets of
planetary images and other solar system data are also available.
Starlink has no policy to hold copies of CD-ROMs, but most sites will
have copies of at least some of them. In the first instance you should
consult your local site manager.

\subsection{The HST Guide Star Catalog version 1.1
\xlabel{the_hst_guide_star_catalog_version_11}\label{GSC}}

\textit{authors:} B.M. Lasker, C.R. Sturch, B.J. McLean, J.L. Russell,
H. Jenkner and M.M. Shara,

\textit{issue date:} 1st August 1992 (version 1.1)

\textit{institution:} Space Telescope Science Institute. 3700 San Martin
Drive, Baltimore, Maryland 21218, USA.

\textit{number of disks:} 2

\textit{format:} FITS ASCII tables

The HST \textit{Guide Star Catalog} (GSC) was produced by the Space
Telescope Science Institute to support the acquisition of off-axis
guide stars for the Hubble Space Telescope. However, because of its
general usefulness to the astronomical community, it has been made
widely available. The catalogue contains some twenty five million rows
for nearly nineteen million separate astronomical objects brighter than
sixteenth magnitude. It is primarily based on microdensitometer scans of
an all-sky, single epoch collection of Schmidt plates. North of
6$^{\circ}$ the `Quick V' (epoch 1982) survey by the Palomar Schmidt
was used; for southern and equatorial fields the SERC-J (epoch
approximately 1975) and SERC-EJ (epoch approximately 1982) surveys by
the UK Schmidt respectively were used.

The GSC is divided into 9537 zones, bounded by arcs of constant Right
Ascension and Declination. The data for each region are stored as a
separate FITS file.

\textit{History:}

Version 1.1 of the GSC supersedes version 1, which was released on
1st June 1989. The improvements of version 1.1 over version 1 include:
assimilation of data for the brighter stars, better identification of
overlapping objects and incorporation of a number of errata.

\textit{References:}

Note that the following references describe version 1 of the GSC, not
version 1.1.

B.M. Lasker, C.R. Sturch, B.J. McLean, J.L. Russell, H. Jenkner and
M.M. Shara, 1990, \textit{Astron J}, \textbf{99}, pp2019-2058.

J.L. Russell, B.M. Lasker, B.J. McLean, C.R. Sturch and H. Jenkner,
1990, \textit{Astron J}, \textbf{99}, pp2059-2081.

H. Jenkner, B.M. Lasker, C.R. Sturch, B.J. McLean, M.M. Shara and
J.L. Russell, 1990, \textit{Astron J}, \textbf{99}, pp2082-2154.

\subsection{Selected Astronomical Catalogs volume 1
\xlabel{selected_astronomical_catalogs_volume_1}\label{SAC}}

\textit{authors:} L.E. Brotzman and S.E. Gessner.

\textit{issue date:} 1992

\textit{institution:} Astronomical Data Center, National Space Science Data
Center, NASA Goddard Space Flight Center, Greenbelt, Maryland 20771,
USA.

\textit{number of disks:} 2

\textit{format:} FITS ASCII tables (1 disk), ASCII text (1 disk)

This CD-ROM set issued by the American NSSDC contains 114 of the most
widely used astronomical catalogues. It is an extremely useful
compilation containing copies of most of the basic astronomical
catalogues, including astrometric, photometric, spectroscopic and
miscellaneous catalogues of both stellar and nonstellar objects.

One disk contains copies of the catalogues in FITS format, the other has
copies of the catalogues as simple text files. Documentation for each of
the catalogues is included on the disks.

Errata for the catalogues on the disks are published in the \textit{ADC
Electronic News} (see Section~\ref{ADC}, below). To date errata have
appeared in the following issues: July 1992, \textbf{1}, issue 2;
October 1992, \textbf{1}, issue 3; January 1992, \textbf{2}, issue 1.

\subsection{The Catalogue of Principal Galaxies
\xlabel{the_catalogue_of_principal_galaxies}\label{PGC}}

\textit{authors:} G. Paturel, L. Bottinelli, P. Fouqu\'{e} and L.
Gouguenheim

\textit{issue date:} 1992

\textit{Institution:} Observatoire de Lyon, 69561 Saint-Genis Laval Cedex,
France.

\textit{number of disks:} 1

This CD-ROM contains a copy of the \textit{Catalogue of Principal Galaxies}
(PGC). It is a catalogue containing cross-identifications and other
data for 73,098 galaxies. It was compiled from data in the Lyon-Meudon
Extragalactic database (see Section~\ref{EDB}).

\textit{References:}

The CD-ROM and accompanying documentation is available as \textit{PGC-ROM
1992} by G. Paturel, L. Bottinelli, P. Fouqu\'{e} and L. Gouguenheim,
1992, \textit{Monographies de la Base de Donn\'{e}es Extragalactiques},
No. \textbf{4} (Lyon: Observatoire de Lyon).

A brief description is given by G. Paturel, L. Bottinelli and L.
Gouguenheim, 1993, \textit{Bulletin d'Information du Centre de Donn\'{e}es
Stellaires}, no. \textbf{42}, pp33-38.

The paper version of the catalogue is available as \textit{The Catalogue
of Principal Galaxies} by G. Paturel, P. Fouqu\'{e}, L. Bottinelli and
L. Gouguenheim, 1989, \textit{Monographies de la Base de Donn\'{e}es
Extragalactiques}, No. \textbf{1} (3 volumes), (Lyon: Observatoire de
Lyon).

See also G. Paturel, P. Fouqu\'{e}, L. Bottinelli and L. Gouguenheim,
1989, \textit{Astr Astrophys Suppl}, \textbf{80}, pp299-315.

\subsection{Images from the Radio Universe
\xlabel{images_from_the_radio_universe}}

\textit{authors:} J.J. Condon and D. Wells.

\textit{issue date:} 1992

\textit{institution:} National Radio Astronomy Observatory, 520 Edgemont
Road, Charlottesville, Virginia 22903, USA.

\textit{number of disks:} 1

\textit{format:} FITS (images), ASCII text (catalogues)

This CD-ROM contains a number of images of radio-astronomical sources
(in FITS format) and a number of radio source catalogues (held as ASCII
text files). The images and catalogues were drawn from all branches and
frequency ranges of radio astronomy and aim to present some of the
best data currently available. The observations were made with a variety
of radio telescopes and were generously contributed by numerous
institutions in various countries.

\textit{References:}

Documentation, and comprehensive references for the individual datasets,
are included on the disk.

\subsection{IRAS Sky Survey Atlas for High Ecliptic Latitudes
\xlabel{iras_sky_survey_atlas_for_high_ecliptic_latitudes}}

\textit{authors:} S. Wheelock, H. McCallon, F. Boulanger, T.N. Gautier,
C. Oken, T. Chester, J. Chillemi, J. White, D. Kester and D. Gregorich.

\textit{issue date:} November 1991

\textit{institution:} Infrared Processing and Analysis Center, Jet
Propulsion Laboratory, Mail Code 100-22, California Institute of
Technology, Pasadena, California 91125, USA.

\textit{number of disks:} 4

\textit{format:} FITS

The InfraRed Astronomical Satellite (IRAS) conducted an all-sky
survey at infrared wavelengths in 1983. This CD-ROM set is the first
release of the IRAS Sky Survey Atlas (ISSA). It covers high ecliptic
latitudes, with complete coverage for absolute ecliptic latitude,
$|b| > 50^{\circ}$ and some coverage down to $|b| > 40^{\circ}$. This
coverage corresponds to approximately 38\% of the sky.

The atlas comprises a set of images in FITS format. Each image is
500 by 500 pixels square, corresponding to a 12.5$^{\circ}$ field
on the sky. Hence the size of each pixel is 1.5'. The images give the
sky surface brightness after removal of the emission from solar system
dust in four wavebands centred on 12, 25, 60 and 100 micron.

The following caveats apply to using the ISSA. It was designed to
give \textit{differential} photometry after subtraction of the zodiacal
emission. It should not be used for measuring absolute sky surface
brightness. Furthermore, it is intended for the study of extended
structures. For structures detected in the survey with sizes less than
5' either the \textit{IRAS Point Source Catalog}, the \textit{IRAS Faint
Source Survey} or the \textit{IRAS Small Scale Structure Catalog} should
be consulted. Some of these catalogues are included in \textit{Selected
Astronomical Catalogs volume 1} (see Section~\ref{SAC}, above).

\textit{References:}

Further information about the atlas, including important details about
the absolute calibration of IRAS can be found in the \textit{ISSA
Explanatory Supplement}, available from the National Space Science
Data Center, NASA Goddard Space Flight Center, Code 933, Greenbelt,
Maryland 20771, USA.

See also \textit{IRAS Catalogs and Atlases: Explanatory Supplement}, 1988,
Joint IRAS Science Working Group, edited by C.A. Beichman, G.
Neugebauer, H.J. Habing, P.E. Clegg and T.J. Chester (Washington DC:
GPO).

\subsection{Einstein Observatory Data
\xlabel{einstein_observatory_data}\label{EINSTEIN}}

The Einstein Data Center of the Smithsonian Institution Astrophysical
Observatory has produced a number of sets of CD-ROMs containing data
from the \textit{Einstein} Observatory X-ray astronomy satellite (HEAO-B).
The \textit{Einstein} satellite operated from November 1978 to April 1981.
It contained an X-ray imaging telescope which operated in the energy
band 0.2 to 3.5 keV. This telescope had two imaging detectors: the
Imaging Proportional Counter (IPC) and the High Resolution Imager (HRI).
The satellite carried out a pointed, as distinct from scanning or
survey, programme. The sets of disks listed below are available.

\begin{description}

  \item[The Einstein Observatory Catalog of IPC X-ray sources] (1st
   January 1990, FITS and FITS ASCII tables, 3 disks). A set of 3935
   images observed with the IPC and catalogues of the sources detected.

  \item[The Einstein Observatory Database of HRI X-ray Images] (1st June
   1990, FITS,
  \newline 2 disks). A set of 870 images observed with the HRI and a
   catalogue of all the sources detected.

  \item[The Einstein Observatory SLEW Survey] (1st January 1991, FITS
   $n$-dimensional tables extension, 1 disk). A set of event lists
   (details of individual photons detected), and a catalogue of
   detected sources for the \textit{Einstein} SLEW survey. The SLEW survey
   was constructed from data acquired while the IPC was slewing between
   pointed observations.

  \item[The Einstein Observatory Database of HRI Images in Event List
   Format] ~
  \newline (1st January 1992, FITS binary tables, 2 disks). A set of
   event lists for observations made with the HRI and a list of sources
   detected.

  \item[The Einstein Observatory Database of IPC X-ray Images in Event
   List Format] ~
  \newline (1st June 1992, FITS binary tables, 4 disks). A set of
   event lists for observations made with the IPC.

\end{description}

Further information about the \textit{Einstein} Observatory CD-ROMs
can be obtained from: Einstein Data Products Office, High Energy
Astrophysics Division, Harvard-Smithsonian Center for Astrophysics,
60 Garden Street, Mail Stop 3, Cambridge, Massachusetts 01238, USA.
E-mail: 6699::EDPO (SPAN), edpo@cfa.harvard.edu (INTERNET).

\subsection{Einstein Observatory SSS, MPC and FPCS Data Products
\xlabel{einstein_observatory_sss_mpc_and_data_products}}

\textit{issue date:} 1st June 1992

\textit{institution:} High Energy Astrophysics Archive Center, Code 668,
NASA Goddard Space Flight Center, Greenbelt, Maryland 20771, USA.
E-mail: 15735::HEASARC (SPAN),
\newline heasarc@heasrc.gsfc.nasa.gov (INTERNET)

\textit{number of disks:} 1

\textit{format:} FITS binary tables

This CD-ROM contains astronomical data files derived from observations
made with the \textit{Einstein} X-Ray Astronomy Satellite (HEAO-B) using
the Solid State Spectrometer (SSS), the Monitor Proportional Counter
(MPC) and the Focal Plane Crystal Spectrometer (FPCS). The SSS and MPC
data products contain summary spectra and light curves derived from the
Laboratory for High Energy Astrophysics at NASA Goddard Space Flight
Center. The FPCS files were produced at the Center for Space Research
and Department of Physics at the Massachusetts Institute of Technology
and include lists of every detected photon during each observation as
well as summary spectra. This CD-ROM complements the CD-ROMs issued by
the SAO Einstein Data Center which contain data from the \textit{Einstein}
IPC and HRI instruments (see Section~\ref{EINSTEIN}, above).


\section{On-line facilities in the UK
\xlabel{online_facilities_in_the_uk}}

This section describes the database facilities available on-line on
Starlink computers.

\subsection{STADAT
\xlabel{stadat}\label{STADAT}}

STADAT is a microVAX II located at the Rutherford Appleton Laboratory
(RAL) which is primarily used to store catalogues, archives and similar
sorts of data. The facilities available include:

\begin{itemize}

  \item a large collection of catalogues in SCAR format permanently
   available on-line. A \verb-NEWS- item listing the catalogues
   available is distributed to all Starlink sites bi-annually. A service
   is provided to convert, on request, catalogues distributed by the
   INADC (see Section~\ref{INADC}, below) and regions of the \textit{Guide Star Catalog} (see Section~\ref{GSC}, above) into SCAR
   format. Copies of these catalogues may also be provided in their
   original format, if requested,

  \item some aspects of the IUE archive (see Section~\ref{IUE}, below),

  \item the InfraRed Astronomical Satellite (IRAS) Low Resolution
   Spectrograph database, comprising a database of about 170,000
   low-resolution spectra in the wavelength range 7 - 24 micron for
   approximately 40,000 separate sources and an atlas of averaged
   spectra for 5425 sources. Access to this archive is described in
   \textit{IRASLRS --- Obtaining Spectra from the IRAS LRS database}
   (SUN/14) by H. Walker and D. Giaretta, 18th July 1989,

  \item various spectral atlases.

\end{itemize}

Every Starlink node has an account on STADAT and you will usually log on
to STADAT using your node's account. Your site manager should be able to
supply details of the username and password. Alternatively, frequent
users may be allocated their own usernames, and some archives, for
example IUE and IRAS, are accessed from captive accounts. To log on to
STADAT type

\begin{terminalv}
    SET  HOST  STADAT
\end{terminalv}

from your local VAX. Once logged on, type

\begin{terminalv}
    STADAT_HELP
\end{terminalv}

for information on the facilities available. For further information on
the various spectral atlases type

\begin{terminalv}
    @SPECTRAL_ATLASDIR:LOGIN
    ATLAS_HELP
\end{terminalv}

Comments, suggestions, requests for new catalogues etc. should be sent
by E-mail to account STADAT::CATMAIL (Starlink DECNET),
19463::CATMAIL (SPAN) or catmail@uk.ac.rl.stadat (JANET). Because
the STADAT accounts are not personal ones, it is probably less confusing
if you send these mail messages from your personal account on your
local machine.

\subsection{IUE Data Archive
\xlabel{iue_data_archive}\label{IUE}}

The International Ultraviolet Explorer (IUE) satellite was launched in
January 1978 and continues in operation to the present day. It
carries a single telescope which performs ultraviolet spectroscopy in
the wavelength range 1100 - 3300\AA. The satellite has carried out a
pointed rather than survey programme and observing time is allocated to
guest observers. Guest observers retain a right of sole access to their
observations for a period of six months after they were acquired. After
this period has elapsed the observations become publicly available.

In the UK the IUE archives are maintained by the IUE Project in the
Space Science Department at the Rutherford Appleton Laboratory (RAL).
Two archives are available: a main archive which currently contains
approximately 80,000 images and the IULDA, a subsidiary archive of
extracted low-resolution spectra. The main archive is partly maintained
on STADAT (see Section~\ref{STADAT}, above) and partly on the RAL
central IBM machine. The IULDA is maintained on STADAT. To access either
archive you must first log on to STADAT. From your local Starlink VAX
type

\begin{terminalv}
    SET  HOST  STADAT
\end{terminalv}

A captive account is available for the main archive: in response to the
`Username:' prompt type \verb-IUE-; no password is required. For the
IULDA you will usually use your node's STADAT account. See
Section~\ref{STADAT} for details.

For further information contact Chris Lloyd, IUE Project, Space Science
Department, Rutherford Appleton Laboratory, Didcot, Oxfordshire, OX11
0QX. E-mail: RLSAC::CXL (Starlink DECNET), 19465::CXL (SPAN).

\textit{Documentation:}

The main IUE archive is described in \textit{IUEDEARCH --- Access to the
IUE Data Archive}
(\xref{SUN/58}{sun58}{}) by C. Lloyd, 2nd December 1992.

The IULDA is described in \textit{IULDA/USSP --- Accessing the IUE Uniform
Low-Dispersion Archive} (\xref{SUN/20}{sun20}{}) by Jo Murray,
17th October 1990.

Additional information on extracting spectra from IUE data is given in
\textit{IUEDR --- IUE Data Reduction Package (2.0)}
(\xref{SUN/37}{sun37}{}) by Jack
Giddings and Paul Rees, 3rd January 1989.

A more general overview of the IUE archives is given in \textit{The many
faces of the Archive of the International Ultraviolet Explorer
satellite} by W. Wamsteker, 1991, in \textit{Databases and On-Line Data
in Astronomy}, eds. D. Egret and M. Albrecht (Dordrecht: Kluwer)
pp35-46.

\subsection{The Leicester Database System and ROSAT Archive
\xlabel{the_leicester_database_system_and_rosat_archive}\label{LEICS}}

The Leicester Database System (LDS) is maintained as a public facility
on the Leicester Starlink VAX cluster. It is based on the EXOSAT DBMS,
although the system contains many databases, not just those from the
EXOSAT satellite. Most, but not all, of these databases are concerned
with high-energy astronomy. This system was originally developed by ESA
for the EXOSAT satellite data archive, but is now maintained by the
HEASARC at the Goddard Space Flight Center (see Section~\ref{HEASARC},
below).

The LDS currently contains some sixty publicly available databases
occupying 0.25Gb. Additionally, approximately 7.5 Gb of X-ray satellite
data products (images, spectra and light curves) are also available.
These data products can be analysed using software integral to the
system. The system runs on node LTXDB, a MicroVAX II in the Leicester
Starlink cluster. To log on to this node type

\begin{terminalv}
    SET  HOST  LTXDB
\end{terminalv}

from your local Starlink VAX. A captive account is provided to
interrogate the LDS. To access it reply \verb-XRAY- in response to the
`Username:' prompt; no password is required. The first time that you log
on you will be asked to supply a name to identify yourself to the
system. On-line help can be obtained by typing \verb-HELP-.

A brief introduction to the LDS is given in \textit{The EXOSAT Database
Management System}
\newline
(\xref{SUN/127}{sun127}{}), by J. Osborne, 2nd July 1991. This document refers
to additional manuals; these have been widely distributed and most
Starlink site managers will hold copies.

For further information, comments, suggestions etc. contact Julian
Osborne, Department of Physics and Astronomy, University of Leicester,
University Road, Leicester, LE1 7RH. E-mail: LTVAD::JULO (Starlink
DECNET), 19838::JULO (SPAN) or julo@uk.ac.le.star (JANET).

The LDS is also used to access the ROSAT data archive maintained at
Leicester. This archive stores and makes available all the data and
results from the pointed observation programme of the ROSAT satellite.
ROSAT was launched on 1st June 1990 and currently is still in operation.
The main instrument is an X-ray telescope with two alternative
instruments: the Position Sensitive Proportional Counter (PSPC) and the
High Resolution Imager (HRI). The PSPC operates in the energy range
0.1 - 2.4 keV, has a spectral resolution of about 40\% at 0.1 keV and an
angular resolution of approximately 25$"$. The HRI operates over a
similar energy range, has no spectral resolution and a smaller field of
view, but has an angular resolution of approximately 2$"$. An
additional, co-aligned telescope, the Wide Field Camera (WFC), operates
at extreme ultraviolet wavelengths (60 - 200 eV) with an angular
resolution of about 1.4$'$.

Observing time on the satellite is allocated to individual observers.
Observers have sole access to their observations for a proprietary
period of one year and two weeks after the data were dispatched to them.
After this period has elapsed the observations become publicly
available.

A brief description of the ROSAT archive appeared in \textit{The ROSAT
Data Archive} by S. Sembay and M.G. Watson, October 1992, \textit{Starlink Bulletin}, issue \textbf{10}, pp12-13. For further information
contact Steve Sembay, Department of Physics and Astronomy, University
of Leicester, University Road, Leicester, LE1 7RH. E-mail: LTVAD::SSE
(Starlink DECNET), 19838::SSE (SPAN) or sse@uk.ac.le.star (JANET).

\subsection{UK Schmidt, ESO Schmidt and AAT Plate Catalogues
\xlabel{uk_schmidt_eso_schmidt_and_aat_plate_catalogues}}

An extensive photographic Plate Library is maintained at the Royal
Observatory Edinburgh (ROE) and is run as a national facility. This
library contains approximately 15,000 original plates taken with the
UK Schmidt Telescope in Australia, about 6500 original plates taken
using older telescopes operated by the ROE and copies of all the major
photographic sky surveys.

By prior arrangement, the collections may be used either by visiting the
Plate Library at ROE or by obtaining a plate on loan. A complete
catalogue of all the plates taken with the UK Schmidt Telescope is
maintained on-line and a captive account is available to interrogate it.
To access this catalogue from your local Starlink VAX type

\begin{terminalv}
    SET  HOST  REVAD
\end{terminalv}

In response to the `Username:' prompt reply \verb-UKSCAT-; no password
is required. Then follow the instructions. A help file giving a brief
description of the catalogue format is available. To obtain a copy of
this help file, type (from your local VAX)

\begin{terminalv}
    COPY  REVAD::DISK$USER3:[UKSCAT]UKSCAT.DOC  *
\end{terminalv}

and file \verb-UKSCAT.DOC- will be created in your file space. It
requires only eighteen blocks of disk space and is a simple text file
suitable for printing.

Plate catalogues for the Anglo-Australian Telescope (AAT) and ESO
Schmidt Telescope are also maintained, though they are less up-to-date
than the UKST catalogue, and the ROE Plate Library does not have the
original plates for inspection or loan. The AAT catalogue only lists the
photographic plates taken with the AAT; observations recorded digitally
are not included. To access either of these catalogues proceed as for
the UKST catalogue, but in response to the `Username:' prompt reply
either \verb-AATCAT- (for the AAT catalogue) or \verb-ESOCAT- (for the
ESO Schmidt catalogue); again no password is required. Then follow the
instructions.

For further information contact the United Kingdom Schmidt Telescope
Unit, Royal Observatory, Blackford Hill, Edinburgh, EH9 3HJ. E-mail:
REVAD::UKSTU (Starlink DECNET), 19889::UKSTU (SPAN) or
ukstu@uk.ac.roe.star (JANET).


\subsection{La Palma Data Archive
\xlabel{la_palma_data_archive}\label{LAPALMA}}

The Royal Greenwich Observatory (RGO) at Cambridge maintains a data
archive of all the observations acquired with the Isaac Newton group
of optical telescopes in La Palma. This group comprises the Jacobus
Kapteyn Telescope (JKT), the Isaac Newton Telescope (INT) and the
William Herschel Telescope (WHT). The JKT and INT have been in
operation since 1984 and the WHT since 1987; essentially all the
observations acquired with the telescopes have been stored in the
archive.

Access to observations is limited to the original observers who acquired
them for a proprietary period of one year. After this period has
elapsed the observations become publicly available. A catalogue
containing summary details of all the observations in the archive is
maintained on-line. Software is available which allows you to
interrogate this archive to identify observations of interest and to
subsequently request copies of them. This software is run from a captive
account on the Cambridge Starlink cluster. To access it from your local
Starlink VAX type

\begin{terminalv}
    SET  HOST  CAVAD
\end{terminalv}

Alternatively, the appropriate INTERNET address is camv0.ast.cam.ac.uk
or 131.111.69.9. In either case the prompt `Username:' should appear.
Simply reply \verb-ARCQUERY-; no password is required. On-line help
information is provided and a printed user guide is also available.
Copies of the user guide and further information about the archive can
be obtained from Eduard Zuiderwijk, Royal Greenwich Observatory,
Madingley Road, Cambridge, CB3 0EZ. E-mail: CAVAD::EZJ (Starlink
DECNET), 20031::EJZ (SPAN), ezj@uk.ac.cam.ast-star (JANET).

\textit{Documentation:}

\textit{La Palma Data Archive User's Guide} by E. Raimond and G. van
Diepen, June 1989, version 1.0, Isaac Newton Group, La Palma, User
Manual No. XIX.

For a more general discussion, see also \textit{Archives of the Isaac
Newton Group, La Palma and Westerbork Observatories} by E. Raimond,
1991, in \textit{Databases and On-Line Data in Astronomy}, eds. D. Egret
and M. Albrecht (Dordrecht: Kluwer) pp115-124.

\subsection{JCMT Data Archive\xlabel{jcmt_data_archive}}

The Royal Observatory Edinburgh (ROE) maintains a data archive of
observations acquired with the millimetre-wave James Clerk Maxwell
Telescope (JCMT) in Hawaii. The archive contains essentially all the
observations made with the telescope since January 1992.

Access to observations is limited to the original observers who acquired
them for a proprietary period of one year. After this period has
elapsed the observations become publicly available. A catalogue
containing summary details of all the observations in the archive is
maintained on-line. Software is available which allows you to
interrogate this archive to identify observations of interest and to
subsequently request copies of them. The JCMT archive uses the same
software as the La Palma archive (see Section~\ref{LAPALMA}, above),
and the two systems are very similar. This software is run from a
captive account on the ROE Starlink cluster. To access it from your
local Starlink VAX type

\begin{terminalv}
    SET  HOST  REVAD
\end{terminalv}

In response to the `Username:' prompt reply \verb-ARCQUERY-; no password
is required. On-line help information is provided and a user guide and
a quick reference sheet are available. Copies of these documents and
further information about the archive can be obtained from Ko Hummel,
Astronomy Division, Royal Observatory, Blackford Hill, Edinburgh, EH9
3HJ. E-mail: REVAD::KXH (Starlink DECNET), 19889::KXH (SPAN),
kxh@uk.ac.roe.star (JANET).

\textit{Documentation:}

\textit{User's Guide to the Hawaii Telescopes Data Archive}, version 1.1
by A.C. Davenhall, February 1993 (Royal Observatory Edinburgh).

\textit{Quick Reference Sheet for the Hawaii Telescopes Data Archive} by
A.C. Davenhall, June 1992 (Royal Observatory Edinburgh).

Because the JCMT archive uses the same software as the La Palma archive,
the user guide for the latter is also relevant (see
Section~\ref{LAPALMA}, above).

A brief introduction to the JCMT archive is given by K. Hummel and A.C.
Davenhall, March 1993, \textit{The JCMT - UKIRT Newsletter}, no. \textbf{5},
pp22-24.

\subsection{Atomic Databases at Belfast
\xlabel{atomic_databases_at_belfast}}

Two collections of atomic data are available at the Queen's University,
Belfast:

\begin{itemize}

  \item electron impact excitation of atoms and ions,

  \item photo-absorption data for atoms and ions; the so-called
   `opacity project.'

\end{itemize}

The latter collection is more relevant to astronomy, containing, for
example, oscillator strengths. It is a collaborative project involving
institutions in the USA, France, Venezuela and Germany as well as the
UK. It comprises approximately 1 Gb of data held as ASCII text files.

For further information contact, in the first instance, Keith
Berrington, Department of Applied Mathematics, Queen's University of
Belfast, Belfast, BT7 1NN. E-mail:
\newline amg0016@uk.ac.queens-belfast.app-maths.vax1 (JANET).

\textit{Documentation:}

A description of some aspects of this project is given by W. Cunto,
C. Mendoza, F. Ochsenbein and C.J. Zeippen, 1993, \textit{Bulletin
d'Information du Centre de Donn\'{e}es Stellaires}, no. \textbf{42},
pp39-43.

\section{Facilities Overseas\xlabel{facilities_overseas}}

This section describes briefly the astronomical data centres which
collect and distribute catalogues, and then lists some of the more
specialized database facilities available on-line at institutions
overseas. There are many such facilities, and though the selection
includes some of the commonly used ones, it is in some respects
necessarily idiosyncratic. Recently Heinz Andernach of the Instituto
Astrofisica de Canarias (IAC), Tenerife, and Bob Hanisch of the STScI
have compiled documents similar to the present one, though they are
perhaps somewhat more extensive.  These authors have kindly allowed
their documents to be made available on Starlink, and copies are held
in the Local Miscellaneous User Documentation directory (LMUDSDIR) on
node STADAT. To obtain copies you should log on to your local Starlink
VAX and type

\begin{terminalv}
    COPY  STADAT::LMUDSDIR:MUD135.LIS  *
    COPY  STADAT::LMUDSDIR:MUD136.LIS  *
\end{terminalv}

The two documents require a total of about 220 blocks of disk space.
Both are simple text files suitable for printing. The IAC document
contains some details which are specific to that institution.

A good source of further information is \textit{Databases and On-Line Data
in Astronomy}, eds. D. Egret and M. Albrecht, 1991 (Dordrecht: Kluwer).
A list of computer-readable space astronomy archives was compiled by
N.E. White (1987, in Appendix C of \textit{Astrophysics Data Program},
NRA-87-OSSA-11), though it is now a few years out of date. Much useful
background information is contained in \textit{Data in Astronomy} by C.
Jaschek, 1989 (Cambridge University Press, Cambridge). Some general
information about accessing remote hosts across communication networks
is given in Section~\ref{REMOTE}.

\subsection{Data Centres
\xlabel{data_centres}\label{INADC}}

There are a number of so-called `data centres' which exist in order to
collect, collate, preserve and disseminate astronomical data. They will
accept most sorts of astronomical data, but are particularly oriented
towards catalogues and similar sorts of reduced data. Inevitably, many
of these data are now held in a computer readable form. The various data
centres collaborate through the \textit{International Network of
Astronomical Data Centres} (INADC) and swap datasets, operate a common
naming scheme for catalogues etc. Probably the two most important data
centres are the Centre de Donn\'{e}es Stellaires (CDS), in Strasbourg
and the US Astronomical Data Center in Maryland. Both of these
institutions will provide a list of their collections on request. In the
UK a copy of the machine-readable collection is maintained at the
Rutherford Appleton Laboratory and is available through STADAT (see
Section~\ref{STADAT}, above).

\subsubsection{CDS
\xlabel{cds}\label{CDS}}

The Centre de Donn\'{e}es Stellaires (CDS) at the Strasbourg Observatory
was the first astronomical data centre, set up in 1972. In addition to
maintaining a collection of astronomical catalogues and data, the CDS
publishes the bi-annual \textit{Bulletin d'Information du Centre de
Donn\'{e}es Stellaires}, which is the principal publication concerned
with astronomical catalogues and data.

A list of the catalogues currently available and copies of most of them
can be obtained by anonymous ftp (File Transfer Protocol). The details
are as follows.

\textit{INTERNET address: } cdsarc.u-strasbg.fr or 130.79.128.5
\newline \textit{Username:} \verb-anonymous-
\newline \textit{Password:} Type your full name and the name of your
institute.

In order to access this system, you should log on to a local Starlink
Unix system and type either

\vspace{2.0 mm}
\verb:ftp cdsarc.u-strasbg.fr: ~~ or ~~ \verb:ftp 130.79.128.5:
\vspace{2.0 mm}

Then simply supply the details for username and password as indicated
above.  See Section~\ref{REMOTE} for more details of accessing remote
hosts. Once you have logged on, a restricted set of Unix commands
concerned with changing directories and copying files are available to
you. Initially you should set your current directory to the root
directory of the directory tree containing the catalogues, which is
directory \verb-pub/cats-. Note also that some of the larger catalogues
are held in \textit{Unix compressed format}. Further details are given in
\textit{The Archive of astronomical catalogues at CDS} by F. Ochsenbein,
July 1992, \textit{Bulletin d'Information du Centre de Donn\'{e}es
Stellaires}, no. \textbf{41}, pp65-70.

Further information about the CDS can be requested by sending e-mail
to 29588::QUESTION (SPAN), question@fr.u-strasbg.simbad (INTERNET), or
by writing to the Centre de Donn\'{e}es Stellaires, Observatoire
Astronomique, 11, rue de l'Universit\'{e}, 67000 Strasbourg, France.

\subsubsection{US Astronomical Data Center
\xlabel{us_astronomical_data_center}\label{ADC}}

The Astronomical Data Center (ADC) is part of the National Space
Science Data Center (NSSDC) / World Data Center A for Rockets and
Satellites at the NASA Goddard Space Flight Center. An on-line system
is available which provides information on the catalogues held by the
ADC and allows the interactive submission of requests for catalogues.
The details are as follows.

\textit{SPAN address: } 15548
\newline \textit{INTERNET address:} nssdca.gsfc.nasa.gov or 128.183.36.23
\newline \textit{Username:} \verb-NODIS-
\newline \textit{Password:} No password required; captive account.

See Section~\ref{REMOTE} for details of establishing a connection to
a remote host. The first time that you log on you will
be asked to supply your name and other details. You are logged on to a
general service covering all the archives of the NSSDC; select option
ten from the main menu to see the holdings of the ADC.

The ADC publishes \textit{ADC Electronic News} which is distributed by
e-mail. Subscription and back issues are also handled automatically by
e-mail. In order to invoke the various functions you should compose
standard messages (in the format described below) and send them to
account listserver@hypatia.gsfc.nasa.gov (INTERNET). The system will
decode your e-mail address and reply accordingly. This system is
completely automatic, but nonetheless several days can elapse between
dispatch of a message and receipt of a reply. The various messages are
as follows.

\begin{itemize}

  \item to subscribe:
  \begin{terminalv}
    SUBSCRIBE ADCNEWS full name
  \end{terminalv}

   where \verb-full name- is replaced with your full name,

  \item to list the back issues available:
  \begin{terminalv}
    INDEX ADCNEWS
  \end{terminalv}

  \item to retrieve a back issue, for example volume 1, issue 1:
  \begin{terminalv}
    GET ADCNEWS VOLUME1.ISSUE1
  \end{terminalv}

   with obvious extensions for other issues and volumes.

\end{itemize}

Further information about the ADC can be requested by sending e-mail
to 15578::NCF (SPAN), or by writing to the Astronomical Data Center,
National Space Science Data Center, Code 633, NASA, Goddard Space Flight
Center, Greenbelt, Maryland 20771, USA.

\subsection{SIMBAD\xlabel{simbad}}

SIMBAD (Set of Identifications Measurements and Bibliography for
Astronomical Data) is a comprehensive, integrated database for objects
outside the solar system. It is operated by the CDS (see
Section~\ref{CDS}, above). Currently it contains data for
approximately 650,000 stars and about 100,000 non-stellar objects and
is believed to be substantially complete to a visual magnitude of about
fifteen. The type of information stored varies for different sorts of
objects, but includes basic parameters (such as positions and
magnitudes), cross-identifications and bibliographic data. Currently
SIMBAD is the most comprehensive database of this type available.

SIMBAD runs on a Unix workstation at the Strasbourg Observatory. It
can be accessed remotely through either the SPAN or INTERNET networks.
The details are as follows.

\textit{SPAN address: } 29588
\newline \textit{INTERNET address:} simbad.u-strasbg.fr or 130.79.128.4
\newline \textit{Username:} site specific; see below
\newline \textit{Password:} site specific; see below

See Section~\ref{REMOTE} for details of establishing a connection to
a remote host. However, before you can use SIMBAD you need to be
registered as either an individual or as a site (the latter
corresponding to your institution). Once you have registered you will
receive an account name, password and a copy of the user guide. Many
Starlink nodes will already have registered. In the first instance you
should consult your Starlink site manager. He will either be able to
tell you the account name and password for your site, or will arrange
registration. You apply for registration by contacting the director of
the CDS: Dr Michel Cr\'{e}z\'{e}, Director, Centre de Donn\'{e}es
Stellaires, Observatoire Astronomique, 11, rue de l'Universit\'{e},
67000 Strasbourg, France. E-mail: or 29588::CREZE (SPAN),
creze@fr.u-strasbg.simbad (INTERNET). You should note that a charge is
made for the use of SIMBAD.

Other inquiries, requests for help and comments can be sent by e-mail
to account
\newline 29588::QUESTION (SPAN), question@fr.u-strasbg.simbad
(INTERNET).

\textit{Documentation:}

\textit{SIMBAD User's Guide and Reference Manual} SIMBAD III, release 1.2,
April 1992 (Strasbourg: Observatoire Astronomique de Strasbourg).

For a more general description see \textit{The SIMBAD Astronomical
database} by D. Egret, M. Wenger and P. Dubois, 1991, in \textit{Databases
and On-Line Data in Astronomy}, eds. D. Egret and M. Albrecht
(Dordrecht: Kluwer) pp79-88.

\subsection{NASA Extragalactic Database, NED
\xlabel{nasa_extragalactic_database_ned}}

The NASA/IPAC Extragalactic Database (NED) was established at the
Infrared Processing and Analysis Center, California Institute of
Technology in 1987. The purpose of the database is to maintain a
comprehensive archive of published data on galaxies and other
extragalactic objects. The data held for each object includes
designations, basic data, such as positions and magnitudes, and
bibliographic information including abstracts. Currently NED contains
data for over 82,000 objects.

NED is intended to be accessed primarily through communication networks.
The details are as follows.

\textit{INTERNET address:} ned.ipac.caltech.edu
\newline \textit{Username:} \verb-NED-
\newline \textit{Password:} No password required; captive account.

See Section~\ref{REMOTE} for details of establishing a connection to
a remote host. The NED user interface is simple and easy to use,
and on-line help information is available. It is possible to use NED
without a user manual. Problems, questions and comments can be sent by
e-mail to account 15193::NED (SPAN), ned@ipac.caltech.edu (INTERNET).

For further information, contact George Helou, Infrared Processing and
Analysis Center, Jet Propulsion Laboratory, California Institute of
Technology, Pasadena, California 91125, USA. E-mail: 15193::HELOU
(SPAN), helou@ipac.caltech.edu (INTERNET).

\textit{Documentation:}

A general description of NED is given in \textit{The NASA/IPAC
Extragalactic Database} by G. Helou, B.F. Madore, M. Schmitz, M.D.
Bicay, X. Wu and J. Bennett, 1991, in \textit{Databases and On-Line Data in
Astronomy}, eds. D. Egret and M. Albrecht (Dordrecht: Kluwer) pp89-106.

\subsection{Extragalactic Database, EDB
\xlabel{extragalactic_database_edb}\label{EDB}}

The Lyon-Meudon Extragalactic Database (EDB) was created at the
Obsevatoire de Lyon in 1983. Since 1985 it has been managed jointly by
the Observatoire de Lyon and the DERAD'N of the Observatoire de Meudon.
The purpose of the database is to maintain a comprehensive collection
of mean homogeneous data for extragalactic objects. The information
available for each object includes measured parameters, mean homogenized
data and model-dependent parameters. Currently the EDB contains entries
for more than 86,000 normal galaxies. The PGC (see Section~\ref{PGC},
above) was created from the EDB.

On-line access to the EDB is available. The details are as follows.

\textit{INTERNET address:} 134.214.4.7
\newline \textit{Username:} \verb-base- (note: lower case is mandatory)
\newline \textit{Password:} No password required; captive account.

See Section~\ref{REMOTE} for details of establishing a connection to
a remote host. After logging on follow the instructions (which are in
English). Further information can be obtained from the Observatoire de
Lyon, 69561 Saint-Genis Laval Cedex, France.

\textit{Documentation:}

General descriptions of the EDB are given in:

G. Paturel, L. Bottinelli, P. Fouqu\'{e}, and L. Gouguenheim, 1988, in
\textit{Astronomy from Large Databases} eds. F. Murtagh and A. Heck
(Garching: ESO conference and workshop proceedings no. 28) pp435-440.

G. Paturel, L. Gouguenheim and L. Bottinelli, 1992, in
\textit{Astronomy from Large Databases II} eds. A. Heck and F. Murtagh
(Garching: ESO conference and workshop proceedings no. 43) pp429-431.

\subsection{HEASARC
\xlabel{heasarc}\label{HEASARC}}

The High Energy Astrophysics Science Archive Research Center (HEASARC)
at the Goddard Space Flight Center was created by NASA in 1990 as a site
for X-ray and $\gamma$-ray astronomy archival research. The HEASARC
maintains an archive from a number of high-energy astronomy satellites,
including ones that have completed their missions, such as \textit{Einstein}, HEAO 1, HEAO 3, OSO 8, SAS 2 and 3, \textit{Uhuru} and \textit{Vela} 5B, and others which are still in operation or construction, such
as ROSAT, GRO, BBXRT, \textit{Astro}-D and XTE. It is anticipated that
approximately one terabyte of data will be held by 1995.

On-line access to the HEASARC archives is supported. The details are as
follows.

\textit{SPAN address: } 15761
\newline \textit{INTERNET address:} ndadsa.gsfc.nasa.gov or 128.183.36.17
\newline \textit{Username:} \verb-XRAY-
\newline \textit{Password:} No password required; captive account.

See Section~\ref{REMOTE} for details of establishing a connection to
a remote host. The first time that you log on you will be asked to
supply a username to identify yourself to the HEASARC software; this
should consist of your first initial, followed by your surname, without
spaces. On-line help information is available, and can be accessed by
typing \verb-HELP-. Much of the HEASARC software is also available at
Leicester (see Section~\ref{LEICS}, above).

Requests for further information about HEASARC should be sent by e-mail
to
\newline 15761::REQUEST (SPAN), request@ndadsa.gsfc.nasa.gov (INTERNET).
The postal address of the HEASARC is: HEASARC, Code 668, NASA, Goddard
Space Flight Center, Greenbelt, Maryland 20771, USA.

\textit{Documentation:}

A variety of documentation is available from the HEASARC.

A useful introduction to the on-line service is \textit{The HEASARC On-line
Service - An Overview}, by K.L. Rhode, 1992, \textit{Legacy}, no. \textbf{1},
pp21-28.

\textit{Legacy} is the HEASARC journal; copies of it and the other
documentation can be requested by e-mail sent to the account for
requesting further information given above.

\subsection{EINLINE\xlabel{einline}}

The EINLINE on-line service is provided by the Einstein Data Center at
the Smithsonian Astrophysical Observatory. It was originally set up to
make the results of the \textit{Einstein} Observatory catalogue of IPC
X-ray sources available in a form suitable for preparing ROSAT observing
proposals. Subsequently it has grown to include access to many other
\textit{Einstein} data products and tabular data from other wavebands,
including an extensive collection of radio-astronomical catalogues
compiled by Heinz Andernach.

The details of access to this service are as follows.

\textit{SPAN address: } 6714
\newline \textit{INTERNET address:} cfa204.harvard.edu, or
einline.harvard.edu, or 128.103.40.204
\newline \textit{Username:} \verb-einline- (note: lower case is mandatory)
\newline \textit{Password:} No password required; captive account.

See Section~\ref{REMOTE} for details of establishing a connection to
a remote host. After logging on follow the instructions.

Requests for further information should either be sent by e-mail to:
6699::EINLINE (SPAN), einline@cfa.harvard.edu (INTERNET) or by post to:
Einstein Data Products Office, Smithsonian Astrophysical Observatory,
60 Garden Street, MS 12 Cambridge, Massachusetts 01238, USA.

\textit{Documentation:}

Documentation on the various datasets and facilities available may be
down-loaded from the on-line service.

\subsection{DIRA2\xlabel{dira2}}

DIRA2 (Distributed Information Retrieval from Astronomical Data) is a
database package produced by the Astronet Database Group based in
Bologna. It is in use at several Astronet nodes in Italy. The system
comprises a collection of astronomical catalogues, including an
extensive collection of radio-astronomical ones compiled by Heinz
Andernach, and software to query
them.

The system can be accessed remotely; the details are as follows.

\textit{SPAN address: } 37927
\newline \textit{INTERNET address:} bodira.bo.cnr.it or 137.204.51.8
\newline \textit{Username:} \verb-DIRA2-
\newline \textit{Password:} \verb-DIRA2-

See Section~\ref{REMOTE} for details of establishing a connection to
a remote host. After logging on follow the instructions, which are in
English. Further information can be obtained from: M. Nanni, C.N.R.
Instituto di Radioastronomia, Via Irnerio, 46, 40126 Bologno, Italy.
E-mail: 38057::NANNI (SPAN) or nanni@astbo1.bo.cnr.it (INTERNET).

\textit{Documentation:}

A comprehensive 230-page user manual, written in English, is available:
\textit{DIRA2, version 1.1, User Manual}, Astronet Database Group, Italy,
May 1992. To obtain a copy log on to DIRA2 and follow the instructions,
or in case of difficulty contact, in the first instance, M. Nanni at
the above address.

\subsection{Hubble Space Telescope Data Archives
\xlabel{hubble_space_telescope_data_archives}}

Archives of data from the Hubble Space Telescope (HST) are maintained at
both the Space Telescope Science Institute (STScI) in the United States
and at the Space Telescope European Coordinating Facility in Europe.
Currently both institutions run the same archive software, a system
called STARCAT, though in the future they may run different systems.

STARCAT (Space Telescope ARchive CATalogue) allows access to various
sorts of data, as well as HST observations. Some of the information
available includes: standard astronomical catalogues, observation logs
from various satellites and ground-based observatories and spectra
and images from some observation logs (including the HST).

The details for accessing STARCAT at the Space Telescope European
Coordinating Facility are as follows.

\textit{SPAN address: } 28771
\newline \textit{INTERNET address:} stesis.hq.eso.org or 134.171.8.100
\newline \textit{Username:} \verb-starcat-
\newline \textit{Password:} No password required; captive account.

See Section~\ref{REMOTE} for details of establishing a connection to
a remote host. After logging on follow the instructions. In the first
instance, questions, comments or requests for further information should
be sent to Benoit Pirenne, ESA Space Telescope European Coordinating
Facility, European Southern Observatory, Karl Schwarzschild Stra\ss e 2,
D-W8046, Garching bei M\"{u}nchen, Germany. E-mail: 28760::bpirenne
(SPAN), bpirenne@eso.org (INTERNET).

The STScI provides copies of the archive on both VAX/VMS and Unix
systems. The details of accessing these systems are as follows.

\textbf{VAX/VMS archive host}

\textit{INTERNET address:} stdata.stsci.edu or 130.167.1.135
\newline \textit{Username:} \verb-GUEST-
\newline \textit{Password:} \verb-ARCHIVE-

\textbf{Unix archive host}

\textit{INTERNET address:} stdatu.stsci.edu or 130.167.1.148
\newline \textit{Username:} \verb-guest- (note: lower case is mandatory)
\newline \textit{Password:} \verb-archive- (note: lower case is mandatory)

See Section~\ref{REMOTE} for details of establishing a connection to
a remote host. After logging on, a subset of the normal operating system
commands (VMS or Unix, as appropriate) are available for examining files
and navigating directory trees. Additional commands available on both
systems are: \verb-docs- to move to the documents directory, \verb-home-
to return to the login directory and \verb-starcat- to start STARCAT
(then follow the instructions).

A manual and primer are available, though it should be possible to use
STARCAT without them. Requests for copies of the manuals, further
information and assistance should be sent by e-mail to archive@stsci.edu
(INTERNET).

\textit{Documentation:}

\textit{STARCAT User Guide} version 4.52 by A. Micol, B. Pirenne and
D. Durand, 9th July 1992, ST-ECF O-02 series, \textbf{XII} (Garching:
Space Telescope European Coordinating Facility).

\textit{HST Catalogue User Guide} for STARCAT version 4.6 by B. Pirenne and
D. Durand, 30th July 1992, ST-ECF O-02 series, \textbf{XIII} (Garching:
Space Telescope European Coordinating Facility).

General descriptions are given in:

\textit{Data Archive Systems for the Hubble Space Telescope} by E.
Schreier, P. Benvenuti and F. Pasian, 1991, in \textit{Databases and
On-Line Data in Astronomy}, eds. D. Egret and M. Albrecht (Dordrecht:
Kluwer) pp47-58.

\textit{STARCAT: An Interface to Astronomical Databases} by B. Pirenne,
M.A. Albrecht, D. Durand and S. Gaudet, 1992, in \textit{Astronomy from
Large Databases II} eds. A. Heck and F. Murtagh (Garching: ESO
conference and workshop proceedings no. 43) pp447-453.

The future direction of the archive software at the STScI is described
in:

\textit{STARVIEW: The Astronomer's Interface to the Hubble Space Telescope
Archive} by M.D. Johnson, 1992, in \textit{Astronomy from Large Databases
II} eds. A. Heck and F. Murtagh (Garching: ESO conference and workshop
proceedings no. 43) pp75-83.


\section{Distributed Databases\xlabel{distributed_databases}}

Traditionally each data archive exists as a separate, independent
entity. Each has its own software, user interface and data formats.
Combining data from separate archives is difficult if not impossible.
Recently more ambitious systems have been attempted, which allow
uniform access to a number of geographically distributed databases.
Currently two such systems are available: ESIS in Europe and the ADS
in the United States.

\subsection{European Space Information System, ESIS
\xlabel{european_space_information_system_esis}}

The ESA European Space Information System (ESIS) provides access to a
number of astronomy and space physics archives. A prototype version
is currently available. The astronomical databases available through
this prototype include versions of the SIMBAD, IUE, EXOSAT and Hubble
Space Telescope archives. A bibliographic database is also available.

The details of accessing ESIS are as follows.

\textit{SPAN address: } 29617
\newline \textit{INTERNET address:} 192.252.106.127
\newline \textit{Username:} \verb-ESIS-
\newline \textit{Password:} No password required; captive account.

See Section~\ref{REMOTE} for details of establishing a connection to
a remote host. Further information can be obtained from: ISD Help Desk,
ESRIN, P.O. Box 64, 00044 Frascati (RM), Italy. E-mail: 29617::ISDHELP
(SPAN).

\textit{Documentation:}

A general introduction to ESIS is given in \textit{ESIS A Science
Information System} by M.A. Albrecht, 1991, in \textit{Databases and
On-Line Data in Astronomy}, eds. D. Egret and M. Albrecht (Dordrecht:
Kluwer) pp127-138.

\subsection{Astrophysics Data System, ADS
\xlabel{astrophysics_data_system_ads}}

The NASA Astrophysics Data System (ADS) provides uniform access to data
archives from various NASA astronomy satellites, as well as more general
astronomical catalogues. Some of the data collections accessible through
the ADS include: X-ray astronomy (Center for Astrophysics, Smithsonian
Astrophysical Observatory), Infrared astronomy (Infrared Processing and
Analysis Center), Ultraviolet astronomy (IUE Regional Data Analysis
Facility, Boulder), Extreme ultraviolet astronomy (Center for EUV
astrophysics), General astronomical catalogues (National Space Science
Data Center), Ultraviolet astronomy (IUE Regional Data Analysis
Facility, Maryland), High energy astronomy (HEASARC) Hubble Space
Telescope (Space Telescope Science Institute) and HEAO-1 (Pennsylvania
State University).

Until recently the ADS could only be accessed from within the United
States, because of an export restriction on the data encryption software
that it uses. However, this restriction has now (March 1993) been
relaxed and the ADS can be accessed from sites outside the United
States. In order to access the ADS it is necessary to register as an ADS
user and to install special software on your host computer. Versions of
this software are available for a variety of Unix workstations but not
VAX/VMS.

Details of the registration procedure, the configuration required to
run the local ADS software, and general information about the ADS can be
obtained from: ADS User Support, Center for Astrophysics and Space
Astronomy, University of Colorado, Boulder, Colorado 80309, USA. E-mail:
ads@cuads.colorado.edu (INTERNET).

\textit{Documentation:}

General introductions to the ADS are given in:

 \textit{The NASA Astrophysics Data System} by J.R. Weiss and J.C. Good,
1991, in \textit{Databases and On-Line Data in Astronomy}, eds. D. Egret
and M. Albrecht (Dordrecht: Kluwer) pp139-150.

\textit{Astrophysics Data System (ADS)} by S.S. Murray, E.W. Brugel, G.
Eichhorn, A. Ferris, J.C. Good, M.J. Kurtz, J.A. Nousek and J.L. Stoner,
1992, in \textit{Astronomy from Large Databases II} eds. A. Heck and F.
Murtagh (Garching: ESO conference and workshop proceedings no. 43)
pp387-391.

\section{Logging on to remote computers
\xlabel{logging_on_to_remote_computers}\label{REMOTE}}

This section attempts to give some general advice on how to log on to
some of the overseas facilities listed in this document. Accessing
remote computers across communications networks is a notoriously
idiosyncratic business and only general suggestions can be given here;
they may not work in all instances and there may be better alternatives
from your site. If you encounter difficulties you should consult your
site manager in the first instance.

In order to access a remote computer you should first log on to a local
Starlink machine. Here there are two possibilities: a VAX running VMS
or a Unix workstation. For the purposes of network access the various
types of Unix workstation supported by Starlink (SUN, DECstation etc.)
all behave the same. The two most commonly used networks are SPAN and
INTERNET; the details of accessing a remote machine across them differ
from each other and differ depending on whether the starting point was
a VAX/VMS system or a Unix workstation. The various options are set out
below.

\subsection{From VAX/VMS\xlabel{from_vaxvms}}

Log on to a VAX running VMS.

\begin{description}

  \item[SPAN] Type

  \begin{terminalv}
    SET  HOST  span_number
  \end{terminalv}

   for example, to access the HEASARC, type

  \begin{terminalv}
    SET  HOST  15761
  \end{terminalv}

   The login prompt for the remote computer should appear.

  \item[INTERNET] In order to access a computer through INTERNET from a
   Starlink VAX it is necessary to go through a SPAN to INTERNET
   gateway. For example, type

  \begin{terminalv}
    SET  HOST  6913
  \end{terminalv}

   which will log you on to the EAST gateway. A `login:' prompt will
   appear. You should reply with the INTERNET address of your remote
   destination \textit{followed by an exclamation mark}. For example,
   to access NED by this route, you would type

  \begin{terminalv}
    ned.ipac.caltech.edu!
  \end{terminalv}

   (note the exclamation mark at the end). The login prompt for the
   remote computer should now appear. Note that the EAST gateway is
   located in the United States, so it is a somewhat eccentric route to
   access sites in Europe.

\end{description}

\subsection{From Unix\xlabel{from_unix}}

Log on to a Unix workstation.

\begin{description}

  \item[SPAN] It is possible to access a computer through SPAN from a
   Starlink Unix workstation by going through the EAST gateway. However,
   this gateway is located in the United States, making it an
   inefficient route and it should only be used as a last resort. There
   may be preferable alternatives at your site.

   If you do need to use this route the details are as follows. Type

  \begin{terminalv}
    telnet 128.183.104.4
  \end{terminalv}

   At the `login:' prompt type the SPAN address of the host that you
   want to access, \textit{followed by two colons}. For example, to access
   the HEASARC you would type

  \begin{terminalv}
    15761::
  \end{terminalv}

   The login prompt for the remote computer should appear.

  \item[INTERNET] Type

  \begin{terminalv}
    telnet  internet_address
  \end{terminalv}

   (note that lower case is mandatory). For example, to access NED

  \begin{terminalv}
    telnet  ned.ipac.caltech.edu
  \end{terminalv}

   The login prompt for the remote computer should appear.

\end{description}

\subsection{SPAN numbers and names\xlabel{span_numbers_and_names}}

SPAN numbers are usually quoted as simple decimal numbers, and this
format has been used throughout this document. This form is also
required in the \verb-SET HOST- command. However, SPAN numbers are
sometimes given in the hierarchical form $n.m$. In this case, the
simple decimal form can be computed using the expression
$(1024.n) + m$. For example, the hierarchical form of the HEASARC SPAN
number is 15.401, corresponding to the simple number 15761.

In addition to numbers, SPAN nodes may also have names. The names are
simply synonyms for the numbers and may also be used in the
\verb-SET HOST- command. A given SPAN number should always be available,
whereas the corresponding name may or may not have been set up at your
site. In principle, the SPAN name for a site \textit{should} be less likely
to change than its number, though in practice either or both can change.

\subsection{INTERNET numbers and names\xlabel{internet_numbers_and_names}}

INTERNET sites may be identified by either a number or a name. The two
forms are completely equivalent. For example, the corresponding INTERNET
number and name for the HEASARC are 128.183.36.17 and
ndadsa.gsfc.nasa.gov respectively. Either the number or the name may
be used in the \verb-telnet- command. The INTERNET name will usually be
recognized, but the number is a fail-safe mechanism which should always
work.

\section{Acknowlegments\xlabel{acknowlegments}}

The descriptions of the various facilities were mostly abstracted from
the introductory sections of the corresponding documentation. However,
many people have also contributed to this document, either by supplying
information or by offering advice and assistance. It is a pleasure to
thank (in alphabetical order): Heinz Andernach, Blaise Canzian, Jim
Condon, Ko Hummel, Ralph Martin, Brian McLean, Geoff Mellor, David
Morgan, Julian Osborne, Clive Page, Steve Sembay and Alan Wood. Special
thanks are due to Geoff Martin who checked all the network addresses,
account names and passwords. Heinz Andernach and Bob Hanisch kindly
allowed their similar documents to be made available on Starlink.

\typeout{  }
\typeout{***********************************************************}
\typeout{  }
\typeout{Note: in order to process this document ab initio, starting}
\typeout{with only the Latex source file, it needs to be run through}
\typeout{Latex THREE times in order to resolve the references and}
\typeout{table of contents.}
\typeout{  }
\typeout{***********************************************************}
\typeout{  }

\end{document}
