\documentstyle[11pt]{article} 
\pagestyle{myheadings}

%------------------------------------------------------------------------------
\newcommand{\stardoccategory}  {Starlink User Note}
\newcommand{\stardocinitials}  {SUN}
\newcommand{\stardocnumber}    {58.6}
\newcommand{\stardocauthors}   {C Lloyd}
\newcommand{\stardocdate}      {2 December 1992}
\newcommand{\stardoctitle}     {IUEDEARCH --- Access to the IUE data archive}
%------------------------------------------------------------------------------

\newcommand{\stardocname}{\stardocinitials /\stardocnumber}
\renewcommand{\_}{{\tt\char'137}}     % re-centres the underscore
\markright{\stardocname}
\setlength{\textwidth}{160mm}
\setlength{\textheight}{230mm}
\setlength{\topmargin}{-2mm}
\setlength{\oddsidemargin}{0mm}
\setlength{\evensidemargin}{0mm}
\setlength{\parindent}{0mm}
\setlength{\parskip}{\medskipamount}
\setlength{\unitlength}{1mm}

%------------------------------------------------------------------------------
% Add any \newcommand or \newenvironment commands here
%------------------------------------------------------------------------------

\begin{document}
\thispagestyle{empty}
SCIENCE \& ENGINEERING RESEARCH COUNCIL \hfill \stardocname\\
RUTHERFORD APPLETON LABORATORY\\
{\large\bf Starlink Project\\}
{\large\bf \stardoccategory\ \stardocnumber}
\begin{flushright}
\stardocauthors\\
\stardocdate
\end{flushright}
\vspace{-4mm}
\rule{\textwidth}{0.5mm}
\vspace{5mm}
\begin{center}
{\Large\bf \stardoctitle}
\end{center}
\vspace{5mm}

%------------------------------------------------------------------------------
%  Add this part if you want a table of contents
\setlength{\parskip}{0mm}
\tableofcontents
\setlength{\parskip}{\medskipamount}
\markright{\stardocname}
%------------------------------------------------------------------------------

\section {Introduction}

The IUE data archive is supported and maintained by members of the UK IUE 
Project in the Space Science Department at RAL. 
The archive comprises some 1400 tapes, containing about 80000 images,
and a catalogue which gives details of the exposures. New images are continually
added to the archive and the catalogue is updated every month.
All IUE data are formally 
available to the astronomical community about six months after
the observations are made and appear in the catalogue shortly
before release.
The aim is to provide UK astronomers with quick and painless 
access to IUE data.
This document describes how to interrogate the catalogue and extract data from
the archive. There are also two other documents relating to IUE data. SUN/37:
IUEDR---IUE data reduction package,
describes how to extract spectra from IUE data files and SUN/20:
ULDA/USSP---Accessing the IUE ULDA,
describes a subsidiary archive of extracted low-resolution spectra.

\subsection {IUE data}

IUE spectra cover two wavelength ranges; 1100 -- 2000\AA, (the short wavelength,
SW)
and 1900 -- 3300\AA, (the long wavelength, LW).
Both spectrographs have a prime (P) and redundant (R) camera. The SWP camera
has been used throughout the mission while the LWR was used until 1984 and
the LWP since then.
The spectrographs may be used in two resolution modes.
In low resolution (LORES) $\Delta\lambda\sim6\AA$ and high resolution (HIRES)
$\Delta\lambda\sim0.1\AA$.
The spectrographs have two apertures. The 
large aperture (LAP) is 10$\times$20 arcsec and may be used for single,
multiple, trailed or extended images. The small aperture (SAP) is 2 arcsec
in diameter and vignettes the beam.

IUE images are held as a number of files which contain the raw image, some
intermediate partly processed images and the final extracted 1-d spectrum. The
high resolution spectra usually have three files---the RAW, the G/PHOT and the
EXTRACTED. 
The G/PHOT is the geometrically and/or photometrically corrected image, and
which of these is held depends on when the image was last processed. The 
timing of the change over from GPHOT to PHOT is given in Section \ref{change}.
Low resolution spectra 
usually have in addition an (extended) line-by-line file ((E)LBL) 
which is a 2-d image for spectroscopy of extended sources. Early LORES images
also have a rectified image segment file.
LORES images may contain spectra taken through both apertures and a very small
number of images contain spectra taken at both resolutions.
Over the lifetime of the satellite the standard image processing (IUESIPS)
has evolved and although the extracted spectra are calibrated they are not
uniformly processed. Some early images have been reprocessed.

In the UK it is common practice to extract the spectra from the G/PHOT
image using the IUEDR package (SUN/37)
which offers significant advantages over IUESIPS,
particularly for the older HIRES spectra and poorly exposed LORES spectra.
For some work the Uniform Low Dispersion Archive (SUN/20) may be appropriate.
The ULDA is an on-line archive of LORES extracted spectra which have had the 
details
of the targets and exposures verified. However it should be remembered that 
these spectra are {\it not} uniformly processed, and are identical to the 
extracted spectra in the tape archive.

\subsection {Overview of the dearchiving process}

Access to the IUE archive is through three interrelated components.
The captive IUE account on STADAT at RAL, the IBM at RAL and
a small number of global commands which are issued from your own username.
The captive account
is used to interrogate the IUE catalogue and identify the images
of the targets you are interested in, 
and to allocate yourself an output tape to hold your data.
The IUE archive tapes and blank output tapes are held in the IBM tape library.
The global command IUEDEARCH is used to submit the
dearchiving program which runs on the IBM
and copies the required files from archive tapes to an output tape.
A log file, which gives a list of the files copied and other information,
is automatically copied to your account.
You ask for the output tape to be sent to you with the global command
IUEDESPATCH.

\subsection {Network access}

Before using the package ensure that @NET\$DIR:NETSYMB is included in your 
LOGIN.COM file.
To run IUEDEARCH you must have access to CBS mail and the JTMP
file-transfer protocol. This usually requires that you are
connected to the boot node of your cluster.
Your site must also be correctly registered to use JTMP in the NRS tables.

\section {The IUE captive account}

The IUE captive account is a username on the database machine at RAL 
(STADAT::IUE) which is only accessible from a STARLINK machine.
To use the captive account, login to your own username on a STARLINK node
and then:

\begin{quote}
{\tt \$ SET HOST STADAT}
\end{quote}

and login to username IUE (no password is required).
You are asked for a name, which is used for a subdirectory to contain your 
files, and then the system displays a list of options. These give access to the 
catalogue and other 
facilities, and enable a few DCL commands to be issued.
These options are:

\begin{description}
\begin{description}
\item [HELP] - An online help facility which gives brief descriptions of how
to use the various options in IUE.

\item [MAIL] - Gives access to the VMS mail facility.

\item [IUELOG] - Searches the IUE catalogue.

\item [CLASSKEY] - Displays a list of the various classification codes used in the
IUE catalogue.

\item [TYPE] - Same as the DCL command 'TYPE'.
It can be used to view any of the files in the IUE directory. 
Wild card characters can not be used in the TYPE command.

\item [DIR] - Displays the contents of the IUE directory.
It may be used with the same options as the DCL equivalent.

\item [TRANS] - Files which have been created in the IUE directory can be
transferred to one of your own directories with this option.
You will be prompted for the filename, your NRS node name, your username, 
password and the directory
you wish the files to be transferred to.
Wild card characters may be used in the filename to allow several versions of
a file to be transferred at once. MAIL may be used as an alternative.

\item [BETA] - Calculates beta and other angles for targets observed with IUE.

\item [LOG] - End the session and log-out.

\end{description}
\end{description}

Any files you create in the IUE directory will be deleted automatically after
two weeks.
If you wish to keep a files for longer than this they must be transferred to 
one of your own directories using TRANS or MAIL.

\subsection {Consulting the catalogue}

At present consulting the catalogue is possible only through the IUE 
captive account. Select the IUELOG option. 
There is usually a page of information about the latest log update and then
you are prompted for a {\it label} for your files.
The IUELOG user interface has been completely rewritten to enable multiple 
search criteria and bears no relation to the old system.
You should only need to search the MAIN catalogue (the default).
You will be asked:

\begin{quote}
{\tt Enter search key}
\end{quote}

which may be any of the following:

\begin{itemize}

\item {\tt HIRES} selects high resolution images.

\item {\tt LORES} selects low resolution images, these two are mutually
exclusive.
If you do not select either then by default both resolutions are selected.

\item {\tt LWP n, m}

\item {\tt LWR n, m}

\item {\tt SWP n, m}

\item {\tt SWR n, m} where both n and m are optional image numbers specifying a range
of required images. If no range is given then all images are selected.
Only one camera may be specifically selected at a time but by default all 
cameras are selected. 

\item {\tt OC n, m} where n and m (optional) specify a range of object classes
otherwise 
all object classes are selected. A single object class may be specified.

\item {\tt RA hh mm ss.s, hh mm ss.s} where the ra's specify the limits of a range in
right ascension.
The ra's may be truncated to any level. Single digit hours, minutes or seconds
may be given as a single digit or with a preceding zero. Hours, minutes and
seconds {\it must} be separated by spaces and the two limits {\it must} be
separated by a comma. The limits are in the sense, from and increasing to, 
and it is possible to search across the boundary from 24 to 0 hours, although
a warning will be issued.
A single ra may be given which defines the centre of
a search area the size of which is determined by the BOX keyword (see below).

\item {\tt DEC $\pm$dd mm ss, $\pm$dd mm ss} where the dec's specify the limits of 
a range in
declination. The general comments for RA also apply. Positive dec's do not
require a sign.
The RA and DEC keywords may be used completely independently.

\item {\tt BOX n} where n 
specifies the size of the search box in arc minutes around a central
ra and dec. If ra and dec {\it limits} are given then the value of BOX is
irrelevant. There is no dec dependence of the dimension of the box in ra.
The default value of n is 30.

\item {\tt NAME string} where string specifies a target name. The string may include
spaces and wildcard (*) characters. The time taken for a name search, including
wildcards, may be up to 90 seconds.

\item {\tt PROG string} where string specifies a programme identification code.
The string may include wildcard (*) characters.

\item {\tt DATE year day, year day} where the dates define the limits of the search.
The year may be given with or without the preceding 19.
The year and day are separated by spaces and the two limits by a comma.
The year only may be given and a single year may also be selected. 

\item {\tt CLEAR} clears all the search criteria to their default values.

\item {\tt Q} quits the search program.

\item {\tt ?} brief outline of the keywords.

\end{itemize}

The search keys are entered one per line in response to the prompt.
Any combination, order and number of search keys may be used. 
If an error is made simply re-enter the search key at any time.
To initiate the search hit carriage return at the prompt.

The results of the search are stored in a workfile which is named as
appropriate.
The workfiles may be searched (using the old system) but with multiple search
keys this should not be necessary and the new system alone should provide the 
list you want. 
Any of the workfiles may be kept with the KEEP option
and will be saved as {\it label}.IUE.
Initially the workfiles are
in RA order but it is possible to sort them using the ORDER option 
by either date or camera and image number. Enter DATE or CIM as appropriate, or
a null carriage return to exit the order option.
All the KEEPed and ORDERed files are saved as different
generations of [IUE]{\it label}.IUE. 
Note that the label and not the workfile name is used.
The {\it label}.IUE files may be transferred or MAILed to your own account.

\subsection {Allocating an output tape}
\label{allocate}
Before dearchiving IUE images it is necessary to allocate yourself an
output data tape.
At present allocating a tape is possible only from the IUE captive account.
Login and select the ALLOCATE option. In response to the prompts 
give your name and address. 
The name of the tape will be selected automatically and is of the form
WDC{\it nnn}.
The tapes are all 6250 bpi NL (no label).

\section {Global commands}
\label{global}

The main dearchiving commands may be issued globally from your own username on 
your local STARLINK node, without the need to login to the captive account.
To set up the appropriate symbols include:

\begin{quote}
{\tt \$ @STADAT::DISK\$STARDATA:[IUES.CAPACCNT]REMOTESTART}
\end{quote}

in your LOGIN.COM. The commands currently available are:

\begin{itemize}

\item {\tt \$ IUEHELP} - brief outline of the remote commands

\item {\tt \$ IUENEWS} - any news of log updates, revisions etc

\item {\tt \$ IUECOPYSUN} - copies the current version of the SUN58 LaTeX file
to your current directory

\end{itemize}

and

\begin{itemize}

\item {\tt \$ IUEDEARCH}

\item {\tt \$ IUERESUBMIT}

\item {\tt \$ IUEDESPATCH}

\end{itemize}

which are described in detail in the appropriate sections below.

\section {Dearchiving images}

\subsection {Preparing and submitting the job}

First allocate yourself an output tape (see Section \ref{allocate}).
IUEDEARCH can only be run from a machine that has both a link to JANET and a
DECnet connection to Starlink. You must also be on the machine that has 
access to CBS mail and JTMP,
which is usually the boot node
of your cluster. Some nodes are set up differently.
Dearchiving jobs are submitted from your own username using the 
global command:

\begin{quote}
{\tt \$ IUEDEARCH}
\end{quote}

If IUEDEARCH is not recognised check that you have the remotestart procedure 
set up correctly (see Section \ref{global}) or if it continues to fail
contact RLVAD::IUES.
IUEDEARCH gives a page or so of information about bug fixes or updates and
then you will be prompted for the following:

\begin{quote}
{\tt Please give your name: }surname first, then your initials.

{\tt and your establishment: } one line

{\tt Please give the output tape name: } WDC{\it nnn}

{\tt Please give combination of codes for type of files required }
\newline
{\tt ? gives the list of codes: }
\end{quote}

Any combination of files may be requested
(eg. GPHOT image or/and EXTRACTED etc).
If you do not know the codes, type `?' for a list (see also Section \ref{codes}).
Some information relating to
the image is available in a more understandable form in the header of the 
EXTRACTED spectrum, so for HIRES it is usual to dearchive the G/PHOT and the
EXTRACTED files. On tapes received from NASA since May 1992 the order of the
files has been different to before, and inconsistent. To circumvent this problem
all the files on the relevant tapes will be copied, irrespective of the codes
given.

\begin{quote}
{\tt Please give the number of the first file on the output tape: }
\end{quote}

Initially this will be 1 but if 
you have copied 10 files to this tape previously, you would type 11.
If you give a number other than 1 you will be asked to give the
amount of tape already used
which is given in the WDC{\it nnn}.REQ of the previous job; see below.
The program keeps track of the amount of output tape used, so the tape is used
efficiently and the IBM job should not crash when the tape is full.

\begin{quote}
{\tt Type image number, or LIST for input file or}
\newline
{\tt END to finish}

{\tt Type camera and image number: }
\end{quote}

The camera name and image number 
may be entered with or without a space and in upper or lower case,
eg. SWP1234, SWP 1234,
swp1234 or swp 1234 will all work. For details of list directed input see
Section \ref{list}.

Check that the object identification is correct; the default is YES so you may 
simply hit carriage return. If there are an unusual number of files for an
image you will be asked if you want each file individually. 

Repeat for all the required images and enter END to finish.
Alternatively, the program will stop when it estimates that the output tape will
be filled by the data.
IUEDEARCH checks the release date of all images requested so you may not be
allowed access to some images.

\begin{quote}
{\tt Enter the NRS code for your machine. (? for list)}
\end{quote}

A list is available form the program if you type `?',
and also in SUN/36 and on the back of the STARLINK bulletin.

Finally you will be asked:

\begin{quote}
{\tt Do you want to submit this job? ([Y]/N)} the default is YES.
\end{quote}

If for any reason you do not want to submit the job, you can stop the job from 
being
submitted by replying `N' to the prompt at this stage.
If the job fails or you do not submit the job, the relevant files remain in your
directory and may be submitted later using IUERESUBMIT. 

On submitting the job you will receive the following messages:

\begin{quote}
\tt
File DISK\$USER1:[IUES]WDC001.JOB;1 Copied
\newline
Task 69856 has been queued
\newline
...submitting WDC001
\newline
Mail request 69857 has been queued
\newline
Mail request 69858 has been queued
\end{quote}

\rm
If you receive the message:

\begin{quote}
{\tt /TRANSFER/ unrecognized command}
\end{quote}

it probably means you do not 
have @NET\$DIR:NETSYMB in your LOGIN.COM file.

If you receive:

\begin{quote}
\begin{verbatim}
%DCL-W-IVVERB, unrecognized command verb - check validity and spelling
 \SEND\
\end{verbatim}
\end{quote}

you do not have access to JTMP which probably means that you are not connected 
to the appropriate machine on your cluster. It is usually accompanied by:

\small
\begin{quote}
{\tt \%MAIL-E-ERRACTRNS, error activating transport CBS
\newline
\%LIB-E-ACTIMAGE, error activating image \$1\$DUA1:[STARSEC.NOCBS]NOCBS\_MAILSHR.EXE;
\newline
-RMS-E-FNF, file not found}
\end{quote}
\normalsize

or something like:

\begin{quote}
{\tt \%MAIL-E-NOSUCH\_PROTOCOL, To use CBS mail on this machine, send mail to
\newline
ARMV1::CBS\%<NRS node name>::<username>}
\end{quote}

IUEDEARCH creates several files in your current directory;
WDC{\it nnn}.JOB, WDC{\it nnn}.JIN, WDC{\it nnn}.REQ and some temporary files.
The .JOB file is the batch job which is submitted to the IBM at your request.
The .JIN file is used by JTMP to submit the .JOB file and the .REQ file contains
a list of the requested images (the old REQU.FIL).
A copy of .REQ is sent to the IUE data archive for dearchiving 
statistics, and so you may be notified
when any bad tapes you required are replaced.
The order of the files on the output tape will almost certainly
be different from the order in which you
entered them.

When the job has been accepted an initial mail message is sent by the 
FTP\_manager.
When the dearchiving job has completed on the IBM four more mail messages
are issued and examples of these and a variety of error messages 
are described in Section \ref{jtmp}.

\subsection {List directed input}
\label{list}

Instead of giving the image numbers individually it is now possible to give
the filename of a list containing the required image numbers. When asked
for the camera and image number, reply:

\begin{quote}
{\tt LIST}
\end{quote}

and then give the filename of the list at the prompt.
The list may be in either of two formats. The program will extract the camera 
and image numbers from an IUELOG
listing complete with all the header information, or as an alternative 
a simple typed list may be used, as in the example below, which
follows the rules for entering the image numbers individually. No {\tt END}
is required to terminate the list.

\begin{quote}
\begin{verbatim}
LWR1234
swp6789
swp12345
lwp 23456
lwp 3456
\end{verbatim}
\end{quote}

When running
in list directed mode the program issues no information about the images and
assumes that the target identifications are correct. All the information that
you would normally see goes instead to the .REQ file.
If there is any ambiguity
the program will copy a file rather than not.

\subsection {Waiting for the job to run}

If you have access to CMS at RAL, you can check the progress of your job by
logging in and typing BATCH Q AMGHK{\it nnn} 
where WDC{\it nnn} is your output tape.
Do not submit a new job using the same output tape until you are sure the
previous job has finished.

\subsection {Output from IUEDEARCH}

The output of the IBM job should appear in your default directory as 
a file called WDC{\it nnn}.OUT.
It contains several pages of the IBM log file and then for each file that has
been copied five lines from the image header for identification. 
You should examine the .OUT file for errors and to identify these you should
see Section \ref{out} {\it The .OUT file and IBM errors},
and if necessary report
any problems with a copy of the file to RLVAD::IUES.
Output from different jobs using the same tape appear in your default directory
as different generations of WDC{\it nnn}.OUT.
The turn-around time for a dearchiving job may be as little as 5 minutes or
several hours depending on the size of the job and the work load of the machine.
If this file appears almost immediately then the job has most probably
failed for some reason.
If your output does not appear within a day or so you should also report it.
Your output tape is retained at RAL until you request it 
so you may append data from further
dearchiving jobs. 

\subsection {When things go wrong}

If most of the file copying is successful but
for various reasons some files are not copied correctly then do not resubmit the
whole job again but re-dearchive the missing images. This is a simple matter
if the image numbers are held in a file. If some images still refuse
to be copied then again mail the .OUT file to RLVAD::IUES.
If the bulk of the job failed or it is not clear what happened then mail the
.OUT file to RLVAD::IUES.

\subsection {Submitting another job}

Now that you have checked the job has run and inspected the output in your
files, you may if you wish run another job to copy a further set of files to
the same output tape. You will need to know the number of files on the tape
and an estimate of the amount of tape used, from the previous WDC{\it nnn}.REQ.
Obviously the amount given in the .REQ
file assumes that all the files were copied successfully and so represents an
upper limit.
Do not submit another job using the same output tape until you have checked 
that the previous one has finished. 

\subsection {Resubmitting a job}
\label{resubmit}

If the file transfer to the IBM is not successful
JTMP re-queues the job and tries a limited number of
times automatically before giving up.
If you do not receive mail from the JTMP\_manager confirming that the job
has been sent you may examine the FTP queue with the command:

\begin{quote}
{\tt \$ LIST}
\end{quote}

If necessary you can resubmit the job later, without
going through IUEDEARCH again. There may be other circumstances when the job as
submitted is correct but it has failed.
To resubmit a job use the global command:

\begin{quote}
{\tt \$ IUERESUBMIT}
\end{quote}

By default IUERESUBMIT will
pick up the last .JOB file you created in the current directory
but this may be changed at the prompt:

\begin{quote}
{\tt Which job do you want to resubmit [WDC001]?:}
\end{quote}

If you are unsure of the file
names you may type `?' and the procedure will list the .JOB files in the 
current directory. There is no default.
On submitting the job the procedure issues the message:

\begin{quote}
{\tt...submitting WDC001}
\end{quote}

and the JTMP messages should be received as above.

\subsection {Requesting the output tape}
When you have all the required data on tape you may request the tape to be
despatched to you by using the global command:

\begin{quote}
{\tt \$ IUEDESPATCH tapename}
\end{quote}

which sends a standard message to RLVAD::IUES and the IBM tape library.
If the tapename is omitted it is prompted for.
Your tape will then be despatched to you by post and
you will receive an acknowledgement that it has been sent.

\subsection {Error messages}

IUEDEARCH is intended to be robust and the messages it issues are self 
explanatory. If it can be made to fail under normal circumstances or its
behaviour is unclear then
please report the problem to RLVAD::IUES.

\section {The .OUT file and IBM errors}
\label{out}

The handling of archive tapes on the IBM is relatively complex. Many of the 
archive tapes are old and unreliable, so to preserve the data and also for
operational reasons the open reel tapes are being copied as they are
used on to 3480 cartridges. Most of the cartridges are now held almost 
`on-line' in a StorageTek device but the open tapes are held in one of the
tape libraries. Most of the problems that occur with the tapes stems from a
tape being in the wrong place.

The .OUT file divides into two main sections. The first part contains largely
unintelligible tape setup and mounting, and tape to cartridge copying 
information. The second part also contains extracts of the system log but it
is mostly IUE archive tape copying information.

If an archive tape is not available in the local library it will be 
{\it rejected} by the system and problems may occur in copying files from other
tapes. Search the .OUT file for `REJECTED' and report to RLVAD::IUES.
The relevant section of log file looks like:

\begin{quote}
{\small
\tt
**SETMNT** SETUP TAPE 184 NAS100 REAL NOMOUNT
\newline
** VOLSER NAS100 REJECTED BY TMS
\newline
*** COPY FAILED FROM NAS100 TO 8N0100
\newline
File not open
\newline
DMSTPJ113S TAP4(184) not attached
}
\end{quote}

\rm
The {\tt COPY FAILED} line may also occur on other occasions and may be ignored.

The start of the program output should look like:

\begin{quote}
{\small
\tt
  IUE DEARCHIVING PROGRAM VERSION 2.1 FOR SLAC BATCH
\newline
  ==================================================
\newline
INITIALIZING OUTPUT TAPE
\newline
 Dumping ...
\newline
 PROFILE  EXEC     B1
\newline
 ***  OPENED WDC002 DENSITY       6250
}
\end{quote}

\rm
or if files have been appended to the output tape:

\begin{quote}
{\small
\tt
  IUE DEARCHIVING PROGRAM VERSION 2.1 FOR SLAC BATCH
\newline
  ==================================================
\newline
SKIPPING FORWARD         54  FILES
\newline
***  OPENED WDC343 AT 6250
}
\end{quote}

\rm
Each file which has been correctly copied will have 5 lines listed from the 
header and a message saying which file on the output tape it has been copied
to:

\begin{quote}
{\small
\begin{verbatim}
HEADER FOR LWR 3122
===================

                         0001000107681536   1 2 0221 3122            1  C
       *    *        *   *   *     9*      *   *  * * * * * *     *  2  C
                                                                     3  C
                                                                     4  C
        NO LAMP-EXP TIME=9 SECS                                      5  C
**********************************************
*** LWR 3122 HAS BEEN COPIED TO FILENO 25
**********************************************
\end{verbatim}
}
\end{quote}

If the file does not correspond to the one expected, the erroneous header
is listed and an error message given.

\begin{quote}
{\small
\begin{verbatim}
HEADER FOR LWP13155
===================

                         0001000101662048   1 2 022014124            1  C
     99*   4*8SEP-1  *   *   *    20*      *   *  * * * * * *     *  2  C
 HD31295,LWR14124,LORES,SLAP,0M10S,16:52:25,0M10S,16:55:17           3  C
 020908,SPREP,MAXG,LOREAD,EA115,WEHRSE-HECK                          4  C
 36,330,FU,13.5,4 MIN-HTR                                            5  C
**********************************************
*** TAPE FILE WAS      LWR 14124
*** SHOULD HAVE BEEN   LWP 13155
*** PLEASE MAIL THIS FILE TO RLVAD::IUES,
*** SPECIFYING SUBJECT -INDEX ERROR-
**********************************************
***  ERROR ON TAPE OPERATION
***  IRC =        0 INFO =        1
***     -----------------------------
*** *** THIS FILE HAS NOT BEEN COPIED ***
***     -----------------------------
***  WILL PROCEED TO NEXT FILE
**********************************************
\end{verbatim}
}
\end{quote}

Sometimes there are tape errors when reading the header or the data part of
the file. 

\begin{quote}
{\small
\begin{verbatim}
HEADER FOR SWP 3552
===================

**********************************************
***   ERROR READING LABEL WIL015 FILE 119
***   SWP 3552 AT LINE    1
***   MAIL THIS OUTPUT TO RLVAD::IUES
**********************************************
***   ERROR ON TAPE OPERATION
***   IRC =   -10000 INFO =        3
***     -----------------------------
*** *** THIS FILE HAS NOT BEEN COPIED ***
***     -----------------------------
***   WILL PROCEED TO NEXT FILE
**********************************************


HEADER FOR LWR 3122
===================

                         0001000107680768   1 2 0221 3122            1  C
       *    *        *   *   *     9*      *   *  * * * * * *     *  2  C
                                                                     3  C
                                                                     4  C
        NO LAMP-EXP TIME=9 SECS                                      5  C
**********************************************
***   ERROR READING WIL015 FILE 131
***   LWR 3122 AT LINE       32
**********************************************
***   ERROR ON TAPE OPERATION
***   IRC =   -10000 INFO =        3
***     -----------------------------
*** *** THIS FILE HAS NOT BEEN COPIED ***
***     -----------------------------
***   WILL PROCEED TO NEXT FILE
**********************************************
\end{verbatim}
}
\end{quote}

...and sometimes the file cannot be opened:

\begin{quote}
{\small
\begin{verbatim}
**********************************************
***   ERROR OPENING NAS069 FILE 281 LWR 3462
***   IRC =    -1001  INFO =        3
***   INPUT NAME = CAR2NL
*********THIS FILE WILL NOT BE COPIED*********
\end{verbatim}
}
\end{quote}

If in any doubt mail the .OUT file to RLVAD::IUES.

\section {JTMP mail messages}
\label{jtmp}

On submitting the job, there is an initial message from JTMP:

\begin{quote}
{\small
\tt
From:	JTMP\_Manager  4-OCT-1991 10:26:28.91
\newline
To:	IUES        
\newline
CC:	
\newline
Subj:	JTMP Report for Task : 0000055166

Task reference: 0000055166
\newline
Report from UK.AC.RUTHERFORD.IBM-B
\newline
Parent Task type : EXECUTION
\newline
Task has been accepted.
}
\end{quote}

\rm
When the job is {\it completed}, these 4 messages are issued:

1. The job file was copied successfully to the IBM:

\begin{quote}
{\small
\tt
From:	JTMP\_Manager  4-OCT-1991 12:22:13.63
\newline
To:	IUES        
\newline
CC:	
\newline
Subj:	JTMP Report for Task : 0000055166

Task reference: 0000055166
\newline
Report from UK.AC.RUTHERFORD.IBM-B
\newline
Task type : DOCUMENT DISPOSITION
\newline
Task successfully created.
\newline
Target site : UK.AC.RUTHERFORD.STARLINK
}
\end{quote}

\rm
2. It understands the job and accepts it:

\begin{quote}
{\small
\tt
From:	JTMP\_Manager  4-OCT-1991 12:22:29.88
\newline
To:	IUES        
\newline
CC:	
\newline
Subj:	JTMP Report for Task : 0000055166

Task reference: 0000055166
\newline
Report from local host
\newline
Arrival report message: 
\newline ~JTMP task analysed and accepted
}
\end{quote}

\rm
3. Dearchiving program has finished and created one (log) file:

\begin{quote}
{\small
\tt
From:	JTMP\_Manager  4-OCT-1991 12:22:38.10
\newline
To:	IUES        
\newline
CC:	
\newline
Subj:	JTMP Report for Task : 0000055166

Task reference: 0000055166
\newline
Report from UK.AC.RUTHERFORD.IBM-B
\newline
Parent Task type : EXECUTION
\newline
Task terminated normally.
\newline
Number of descriptors spawned : 1
}
\end{quote}

\rm
4. WDC{\it nnn}.OUT file successfully transferred:

\begin{quote}
{\small
\tt
From:	JTMP\_Manager  4-OCT-1991 12:22:41.68                                   
\newline
To:	IUES        
\newline
CC:	
\newline
Subj:	JTMP Report for Task : 0000055166

Task reference: 0000055166
\newline
Report from UK.AC.RUTHERFORD.IBM-B
\newline
Task type : DOCUMENT DISPOSITION
\newline
Task transferred successfully.
\newline
Receiving site : UK.AC.RUTHERFORD.STARLINK
}
\end{quote}

\rm
\subsection {JTMP errors}
In the following case the .OUT was not copied, probably because there was not 
enough file space:

\begin{quote}
{\small
\tt
From:	JTMP\_Manager  2-JAN-1992 17:46:10.96
\newline
To:	IUES        
\newline
CC:	
\newline
Subj:	JTMP Report for Task : 0000057094

Task reference: 0000057094
\newline
Report from UK.AC.RUTHERFORD.IBM-B
\newline
Task type : DOCUMENT DISPOSITION
\newline
Task has terminated abnormally.
\newline
Number of descriptors spawned : 0
\newline
Report message : 
\newline
Transfer terminated by UK.AC.RUTHERFORD.STARLINK. JTMP file delivery failure 
\newline
- reason:  \%RMS-E-CRE, ACP file create failed.
}
\end{quote}

\rm
No explanation for this one yet, but it seems like a transmission error:

\begin{quote}
{\small
\tt
From:	JTMP\_Manager 20-FEB-1992 11:15:30.38
\newline
To:	IUES        
\newline
CC:	
\newline
Subj:	JTMP Report for Task : 0000060839

Task reference: 0000060839
\newline
Report from UK.AC.RUTHERFORD.IBM-B
\newline
Task type : DOCUMENT DISPOSITION
\newline
Task has terminated abnormally.
\newline
Number of descriptors spawned : 0
\newline
Report message : 
\newline
Transfer rejected by UK.AC.RUTHERFORD.STARLINK. Unacceptable attribute on SFT. 
\newline
Invalid value in command:  "initial\_restart\_mark" value 0 offered, but mark 
\newline
acknowledgement not agreed. Binary\_word\_size ignored. VAX/VMS FTP (80) Version 
}
\end{quote}

\rm
The following
facility not found error probably indicates that your site is not correctly
listed in the national NRS tables as a JTMP user:

\begin{quote}
{\small
\tt
From:	JTMP\_Manager  4-MAR-1992 15:16:51.37
\newline
To:	ANON
\newline
CC:	
\newline
Subj:	Task failure

Task reference: 23260
\newline
Report from local host

Your task for UK.AC.RUTHERFORD.IBM-B has failed
\newline
Further information :
\newline
Facility not found
}
\end{quote}

\rm
The following 
and variants of the invalid command received seem to indicate a random
transmission problem which is overcome by simply resubmitting the job. It also
can occur if there is an error in the node name. If this problem persists delete
the WDC{\it nnn}.JIN;* files and run IUERESUBMIT giving your node name at the prompt.
See Section \ref{resubmit}.

\begin{quote}
{\small
\tt
From:	JTMP\_Manager 14-MAY-1992 14:51:55.25
\newline
To:	ANON
\newline
CC:	
\newline
Subj:	Task failure

Task reference: 3562
\newline
Report from local host

Your task for UK.AC.RUTHERFORD.IBM-B has failed
\newline
Further information :
\newline
invalid command received!/- !AC
\newline
Invalid value in command:  Value "0822" returned for attribute 
\newline
"delimiter\_preservation" outside offered range.
}
\end{quote}

\rm
\section {Change over from GPHOT to PHOT}
\label{change}

The timing of the change over from GPHOT to PHOT images was different for
the two ground stations and different dispersions and is given in the table.
It should also be remembered that some early images may have been reprocessed
at a later date and have a different set of output files.

\begin{center}
\begin{tabular}{rcc} 
	& Goddard & Vilspa \\
LORES	& ~3 Nov 1980 & 10 Mar 1981 \\
HIRES	& 10 Nov 1981 & 11 Mar 1981 \\

\end{tabular}
\end{center}

\section{File Codes}
\label{codes}

\begin{center}
\begin{tabular}{ll} 
R & Raw image file\\
G & GPHOT or PHOT image file\\
E & Extracted spectrum file\\
L & (Extended) line-by-line file\\
S & Image segment file\\
A & All files \\

\end{tabular}
\end{center}

\end{document}
