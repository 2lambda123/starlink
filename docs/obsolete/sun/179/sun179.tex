\documentstyle[11pt]{article}
\pagestyle{myheadings}

%------------------------------------------------------------------------------
\newcommand{\stardoccategory}  {Starlink User Note}
\newcommand{\stardocinitials}  {SUN}
\newcommand{\stardocnumber}    {179.3}
\newcommand{\stardocauthors}   {Rhys Morris \\ Grant Privett}
\newcommand{\stardocdate}      {13 July 1995}
\newcommand{\stardoctitle}     {IRAF --- Image Reduction Analysis Facility}
%------------------------------------------------------------------------------

\newcommand{\stardocname}{\stardocinitials /\stardocnumber}
\markright{\stardocname}
\setlength{\textwidth}{160mm}
\setlength{\textheight}{230mm}
\setlength{\topmargin}{-2mm}
\setlength{\oddsidemargin}{0mm}
\setlength{\evensidemargin}{0mm}
\setlength{\parindent}{0mm}
\setlength{\parskip}{\medskipamount}
\setlength{\unitlength}{1mm}

\begin{document}
\thispagestyle{empty}
CCL / {\sc Rutherford Appleton Laboratory} \hfill {\bf \stardocname}\\
{\large Particle Physics \& Astronomy Research Council}\\
{\large Starlink Project\\}
{\large \stardoccategory\ \stardocnumber}
\begin{flushright}
\stardocauthors\\
\stardocdate
\end{flushright}
\vspace{-4mm}
\rule{\textwidth}{0.5mm}
\vspace{5mm}
\begin{center}
{\Large\bf \stardoctitle}
\end{center}
\vspace{5mm}

\setlength{\parskip}{0mm}
\tableofcontents
\setlength{\parskip}{\medskipamount}
\markright{\stardocname}

\newpage
\section{Introduction}

IRAF is a powerful and comprehensive environment for reducing and
analyzing optical astronomical data.  The system was developed by the
IRAF group at National Optical Astronomy Observatory (NOAO) in Tucson,
Arizona, who also maintain it (including the production of versions for
new types of computer) and answer queries from users. It has its own
file format, programming language, command language and on-line help
documentation.  The package can be installed on a number of platforms:
Solaris (Beta-test version), SunOS, Ultrix and VMS are supported by the
IRAF group, and there is a version for Alpha machines under OSF/1 which
is maintained by the Space Telescope Science Institute (STScI). This
Starlink User Note is intended as a brief introduction and a pointer to
IRAF's own documentation.  A walk through a short IRAF session is
included as an example.

\subsection{IRAF and Starlink}

IRAF is one of several major software environments which are available
to Starlink users and upon which entire data-analysis campaigns can be
based.  Other examples include AIPS, IDL and MIDAS, as well as
Starlink's own large collection of infrastructure tools and application
packages and the various forms of the FIGARO system.  Each has its own
particular strengths and special capabilities;  IRAF and AIPS are the
most comprehensive overall.

The excellent support provided by IRAF's home institute (with which
Starlink maintains close contact) means that comparatively little
national UK Starlink support is called for and provided;  the same is
true for the other overseas environments mentioned above.

IRAF is a sound choice for many Starlink data-analysis users,
especially where compatibility with overseas collaborators is a
requirement.  The choice is harder for those wishing to develop major
applications of their own, who may be reluctant to adopt IRAF's
non-industry-standard \verb|SPP| programming language or who feel
uncomfortable with the limitations of the ``flat'' IRAF data formats.
Those who need formal guarantees of future support should also be very
wary about committing themselves to any package which is not under UK
control.  However, IRAF is an important weapon in the armoury of the
average Starlink data-analysis user and likely to remain so for some
years.

An important part of Starlink's present software plans is to enhance
``interoperability'' with other environments, IRAF being of special
importance.  Starlink applications can already share data with other
packages via FITS files, but in the case of IRAF this data-interchange
capability will be extended by enabling Starlink applications to read
and write IRAF data files.  This will mean that a user can run IRAF in
one window on his or her screen and a Starlink package ({\it
e.g.}\ CCDPACK) in another, processing the same datasets with each.  A
further capability, currently being investigated, would be to run
Starlink applications from the IRAF command-line just as if they were
native IRAF applications.

\section{Documentation}

\subsection*{On-line Documentation}

There is comprehensive on-line help available for each task in IRAF.
If, for example, you want information on the IRAF task ``{\tt
imcombine}" then you would type ``{\tt help imcombine}" while running
IRAF. If you are not sure what a task is called, you can search for all
occurrences of a particular keyword in the help files using the {\tt
references} task. Typing ``{\tt ref combine}'' should turn up the {\tt
imcombine} task mentioned earlier. Command abbreviation is allowed to
an unambiguous minimum. Should you wish to study one of the on-line
help pages in detail, you can produce a printable plain ASCII file
using a command line of the form:

{\tt help imcombine device=text > dump\_file }

Print {\tt dump\_file} and read it at your leisure.


\subsection*{Paper Documentation}

The paper documentation for IRAF is archaic;  most of the documentation
effort goes into the on-line documentation. There are some sets
of relevant documents available from the Starlink document
librarian:  the {\it IRAF User Handbook}\/, Volume 1A -- the
IRAF System (Starlink MUD/104), and the {\it IRAF User Handbook}\/,
Volume 2B -- User's Guides / KPNO Cookbooks (Starlink MUD/105).

Volume 1A contains:

\begin{itemize}

\item{The IRAF Data Reduction and Analysis System -- a system overview.}

\item{An introduction to the IRAF command language.}

\item{IRAF test procedures.}

\item{IRAF directory trees.}

\item{IMFORT -- a FORTRAN interface to IRAF.}

\end{itemize}

Volume 2B, the ``cookbook'', contains documents
describing various packages in IRAF. For example, the document on the
CCD reduction package, {\tt CCDRED}, is included as is a document on
the spectroscopy package {\tt ONEDSPEC}.

Starlink MUD/154, the STScI {\it STSDAS Users Guide}\/, contains a useful
introduction to IRAF which is well worth reading. The rest of the
document describes an optional add-on package to IRAF called STSDAS
which is for the analysis of data from the Hubble Space Telescope. See
later in this document for more about this package and other packages
that can be added to IRAF.

\subsection*{WWW Documentation}

IRAF now maintain a World\,Wide\,Web (WWW) home page via which
the user may obtain the latest copies of the IRAF FAQ, the IRAF newsletter,
bug reports and other documentation. This may be accessed via URL
{\tt http://iraf.noao.edu/}.

\section{File Formats}

The IRAF file format is usually referred to as the OIF or ``Old IRAF
Format'' (although there is no ``New IRAF Format'' yet) and it consists of
two files. One file contains header information such as image
dimensions and any lines of FITS type information from the original
observation, and has a {\tt .imh} suffix. The other file contains the
actual data values and has a {\tt .pix} suffix.

The header file contains the full pathname of the pixel file so the
pixel file can be in a different directory or even on a separate
disk. This means that you cannot copy IRAF files around from place to
place or from machine to machine and expect them to work -- a common
source of difficulties for new users! There are tools for editing the
header file should you get into a mess, but in general images should
always be copied and renamed using the {\tt imcopy} and {\tt imrename}
tasks within IRAF. If you need to transfer data from one machine to
another, then you can use the {\tt rfits} and {\tt wfits} tasks in the
{\tt dataio} package to create disk FITS files which can usually be
moved from one machine to another.

STSDAS format images are recognized by most IRAF tasks;  this format is
used for storing Hubble Space Telescope data.  Again there are two
files:  one file is the
header file, and has the extension {\tt .XXh}, where X can be any
alphanumeric character, while the other contains the actual data
numbers and has a {\tt .XXd} suffix;  these are usually kept together
in the same directory. STSDAS files have the advantage that the header
file is an ASCII file and can be read on a terminal without using IRAF
at all. This is not true of {\tt .imh} files as some of the header
information is stored in binary form. In a future release of IRAF,
disk FITS files with the {\tt .fit} extension will be recognized as
well.

\section{The IRAF Package Structure}

IRAF's ``tasks'', as the individual programs are sometimes called,
are divided into packages. See the example session, later, for the
packages available when you start up IRAF. Each of the packages may
contain tasks or further subpackages. For example, loading the {\tt
noao} package by typing ``{\tt noao}'', leads to further subpackages,
while loading the {\tt dataio} package makes available tasks for
importing and exporting data to and from IRAF. You cannot use any task
until you have loaded the package which contains it. Typing ``{\tt
bye}'' unloads the most recently loaded package.

There are documents available describing some of the individual
packages in Volume 2A of the {\it IRAF User Handbook}\/ (MUD/105,
the ``cookbook'' volume) while the STSDAS package is described
in the {\it STSDAS User Guide}\/ (Starlink MUD/154).

\section{Running IRAF for the First Time}

A few things need to be done before a user can run IRAF. It is usual
to create a subdirectory called {\tt iraf} for storing IRAF files; do
this and then move to that directory. You now need to type ``{\tt mkiraf}'' to
set things up. This creates a file called {\tt login.cl} and a
subdirectory to your IRAF directory called {\tt uparm}. The former is
an initialization file for IRAF and the latter is a directory used for
storing parameter information such as default values for commands.

You will need to edit {\tt login.cl} so that the variable {\tt home}
contains your correct disk name for your IRAF home directory. You will
also need to set the {\it imdir} variable to the directory where you intend
to keep your pixel files. If you are not sure about this it is
probably best to set this to ``{\tt HDR\$}'' so that the pixel file is
created in the same directory as the header file. Running {\tt mkiraf} also
prompts you for a terminal type: if you run IRAF from an {\tt xterm}
window, you could reply ``{\tt vt100}'' or ``{\tt xterm}.''
This is stored in the {\tt login.cl} file, and you
can always edit it to something different later.

If you wish to customize IRAF you should create a file called {\tt
loginuser.cl} in which you can define things like variables to
contain the names of frequently used directories. It is worth doing
this as sometimes a major revision of IRAF requires all users to
delete all old parameter files and type ``{\tt mkiraf}'' to get a new {\tt
login.cl}.  Any customizations will then be lost unless you put them in
a {\tt loginuser.cl} file.

The IRAF command language is started by typing ``{\tt cl}'';  a welcome
message appears and you will find yourself at the {\tt cl>} prompt.
By the time you read this, there may be an IRAF GUI (Graphical User
Interface) available. To use this, you would go through the set-up
procedure as above and type ``{\tt xcl}'' instead of ``{\tt cl}''
to run IRAF.

\section{Useful Tips}

If you are not sure of the task name to perform any particular
function then, as mentioned before, you could do a keyword search of
all the help files using the {\tt references} task. Once you think you
have a found a task that suits your needs you can read the help file
for it; you do not need to load any packages for this as all the help
files are accessible at all times. The help file will tell you which
packages you need to load to use the task.

Once you have found your task and loaded all packages necessary to use
it, you can either get the parameter list from the help files, or
if you just want a quick reminder of the order, you could use
{\tt lpar} to list the parameters of the task.  Similar to {\tt lpar}
is the {\tt epar} task which allows you to edit the parameter list for
a task. For example, if you would like the {\tt imdelete} task to
ask for confirmation before deleting any images, you could edit its
parameter list so that it does so.

\newpage

%===================================================================
% The next two pages are meant to be opposite each other. Future  ||
% modifiers of this document please take note.                    ||
%===================================================================

\section{An IRAF Session}

This is a simple IRAF session; each step is explained in the text on
the next page.

{\footnotesize
\begin{verbatim}

/home/pollux/rahm[pollux]rahm: cd ~/iraf
/home/pollux/rahm/iraf[pollux]rahm: cl
setting terminal type to vt100...
    NOAO SUN/IRAF Revision 2.10.3BETA Thu Aug 18 17:39:59 MST 1994
    This is the BETA version of Sun/IRAF V2.10.3 for Solaris 2.3.

    Welcome to IRAF.  To list the available commands, type ? or ??.  To get
    detailed information about a command, type `help command'.   To  run  a
    command  or  load  a  package,  type  its name.   Type  `bye' to exit a
    package, or `logout' to get out of the CL.   Type `news'  to  find  out
    what is new in the version of the system you are using.   The following
    commands or packages are currently defined:

      apropos     images.     noao.       proto.      system.
      dataio.     language.   obsolete.   softools.   tables.
      dbms.       lists.      plot.       stsdas.     utilities.
cl> images
      blkavg        geomap        imdelete      imstack       median
      blkrep        geotran       imdivide      imstatistics  minmax
      boxcar        gradient      imgets        imsum         mode
      chpixtype     hedit         imheader      imsurfit      register
      convolve      hselect       imhistogram   imtranspose   rotate
      fit1d         imarith       imlintran     laplace       sections
      fmedian       imcombine     imrename      lineclean     shiftlines
      fmode         imcopy        imshift       listpixels    tv.
      gauss         imdebug.      imslice       magnify
im> imstat r136
#               IMAGE      NPIX      MEAN    STDDEV       MIN       MAX
                 r136    262144     1.659     1.698        0.       27.
im> lpar imarith
     operand1 = ""              Operand image or numerical constant
           op = "*"             Operator
     operand2 = "100"           Operand image or numerical constant
       result = ""              Resultant image
       (title = "")             Title for resultant image
     (divzero = 0.)             Replacement value for division by zero
     (hparams = "")             List of header parameters
     (pixtype = "")             Pixel type for resultant image
    (calctype = "")             Calculation data type
     (verbose = no)             Print operations?
       (noact = no)             Print operations without performing them?
        (mode = "ql")
im> imarith r136 - 1.659 new_image
im> imstat new_image
#               IMAGE      NPIX      MEAN    STDDEV       MIN       MAX
            new_image    262144  1.873E-4     1.698    -1.659     25.34
im> logout
/home/pollux/rahm/iraf[pollux]rahm:
\end{verbatim}
}
\newpage
\subsection{Explanation of the IRAF Session}

IRAF must always be started from the directory
where the {\tt login.cl} and the {\tt loginuser.cl} files are
kept, otherwise any customizations of IRAF will be lost. Once in your
IRAF home directory, IRAF is started by typing ``{\tt cl}'' (in
lowercase -- IRAF is case-sensitive) and the
IRAF welcome message appears. The version of IRAF shown is version
2.10.3, which was the most recent at the time of writing. Usually the
minor version number increments two or three times a year as minor bug
fixes are applied. The major version number changes less frequently.

Of the top-level packages available on start-up, {\tt apropos} is not
a package: it is a task from STSDAS which performs a similar task to
the IRAF {\tt references} task in searching through the helpfiles for
a keyword. The basic IRAF will not contain this task, but it is part
of the set-up at the Cardiff Starlink node,
where this document was written.

The {\tt images} package is loaded and the tasks in that package are
listed.  The task {\tt imstat} is selected to get the statistics of an
image. The task {\tt imarith} is a multipurpose application
for performing arithmetic
operations. Operands 1 and 2 can be either images or constants while
the operator argument decides the arithmetic operation to be
performed. The argument list for {\tt imarith} is quite lengthy, so the {\tt
lpar} task is used to list the parameters for the task. This is very
useful if you do not have any paper documentation and cannot remember
exactly what an argument is called. The bracketed parameters are
optional, while the other parameters are compulsory and must be specified in
the correct order.

We use {\tt imarith} to subtract a constant from the image to create
a new image called {\tt new\_im}. Finally we use {\tt imstat} again on the
new image to verify that the operation worked.

IRAF is closed down by typing ``{\tt logout}'', which leaves you back in
the directory in which you started.

\section{The IRAF FAQ}

The single simplest piece of advice for users of IRAF is that they
should obtain a copy of the excellent IRAF FAQ (Frequently Asked Questions)
file without delay.  This document, available both from NOAO
(where it is maintained)
and from the Starlink IRAF Support programmer, contains a wealth
of information and usually provides a solution to user queries.
It is a good idea to have a copy permanently available in the terminal room
or on disk wherever IRAF is being used.


\section{IRAF Scripts and Batch Jobs}

A series of IRAF commands can be executed by putting them in a
file. The IRAF command language, {\tt cl}, is also a programming language
and contains many features which allow you to write programs to help
you with repetitive data reduction tasks. Although in general it is
not possible to pass parameters from one task to another, some tasks
do have output parameters so that the result of a calculation can be
stored and accessed by another task. Unfortunately the {\tt imstat}
task does not have an output parameter for storing the mean value of
the image, but it is possible to write the output to a file using the
UNIX-like redirection abilities of the {\tt cl} and then read in only the
parts we are interested in. This is done in the script {\tt doit.cl}
which is shown below.

The commands below are typed into a file called {\tt doit.cl}, but before we
can run the script we must tell IRAF that it is a ``foreign task''. The
command line to do this is:

{\tt task \$doit=home\$doit.cl}

This tells IRAF that typing ``{\tt doit}'' will execute the commands in
the file {\tt doit.cl}, which is stored in the IRAF home directory.

The file {\tt doit.cl} contains the following commands:

{\footnotesize
\begin{verbatim}

# IRAF script to calculate a value and subtract
# it from an image.

# Load images package.
images

# Initialize the cl list variable.
list = " "

# Store a temporary file name in string s3.
s3 = "store_file"

# Redirect imstat output to temporary file.
imstat "star_cluster" > (s3)

# Associate temporary file with list variable.
list = (s3)

# Read first line and discard it.
i = fscan (list, s2)

# Read second line, the parameter we want is "y".
i= fscan (list,s1,x,y,z)

# Use imarith to subtract y from the image.
imarith ("star_cluster","-",y,"new_image")

# Look at statistics of new image to see if it
# worked.
imstat ("new_image")

# Change protection on the image created
# and delete it.
unprotect("new_image")
delete ("new_image", verify=no)

# Delete temporary file.
delete ("store", verify=no)

# Print an output message
print(``subtracted'',y,''from image star_cluster'')

\end{verbatim}
}

As with any unfamiliar programming language, the first reaction is one
of trepidation as you wonder what is going on. No variables were
declared at the start of this program although variables are used
throughout the program for parameter passing. The {\tt cl} has some built-in
variables which do not need to be declared. There are three string
variables named {\tt s1} to {\tt s3}, three real variables
{\tt x}, {\tt y} and {\tt z}, three
integers {\tt i}, {\tt j} and {\tt k},
three booleans and other more esoteric ones. Type
``{\tt lpar cl}'' for a full list of the parameters available for
immediate use. Other variables can be used but they need to be
declared: see the manual for {\tt cl} for information on how to do
this. List-type parameters are useful, they are usually associated
with files, and they allow you to use {\tt fscan} and {\tt fprint}
for reading and
writing to files -- these are similar to functions found in the C
language.

Put simply, the script calculates the average pixel
value for the file {\tt star\_cluster} and then subtracts it from the
image. Without direct parameter passing this is not
straightforward but can be done via a temporary file.
To this end, the output of {\tt imstat} is
re-directed to a file.  A list parameter is associated with the file. Since
the first line read back contains only the column headings, this line
is discarded, as are all of the variables read from the second line
apart from the mean value which is stored in the real type variable
{\tt y}. This
is used in the {\tt imarith} task and subtracted from the input image to
yield a new image.

Quoted parameters are passed to the tasks as they are; the brackets
around some of the parameter lists indicate that the {\tt cl} language is
meant to interpret these parameters as variables.  Programming in {\tt cl}
can be tricky sometimes: putting ``{\tt echo=yes}'' at the start of
the script means that all commands are echoed to the terminal as they
are executed, useful for debugging.  The file name
``star\_cluster'' was of course `hardwired' into the script; a more
typical application would operate on every file in a list, or loop
until some condition is satisfied. The {\tt cl} can support loops of various
types -- see the {\tt cl} manual for more information.

Sometimes it is useful to run a batch job on a UNIX machine which
starts up IRAF in the background, executes some IRAF tasks and then
quits IRAF and returns to the operating system. This can be done as
follows:

{\footnotesize
\begin{verbatim}
#!/bin/csh
# Make sure a C shell is used
# Run IRAF from the UNIX shell, taking parameters
# up until the word END.
cl <<END
images
imstat( "star_cluster")
task \$doit = home\$doit.cl
doit
logout
END
\end{verbatim}
}

The script above first ensures that a C shell is used, if it is run as
an executable shell script, then IRAF is started and fed with
commands. Notice how the {\tt \$} signs need to be escaped in the task
definition to stop the UNIX shell from interpreting them as shell
variables.


\section{Image Display}

The IRAF image display capabilities were originally developed under
the {\it sunview}\/ window environment which has now been superseded by
X windowing systems. Under this old window environment the
display tools for IRAF were quite good: there was an image display
tool called {\tt imtool} and a graphics display window called
{\tt gterm}. X window versions of these tools called {\tt ximtool} and
{\tt xgterm} have been under development in NOAO for some time,
and the final versions are expected in one of the next IRAF releases.
Alternatively, the {\it SAOimage}\/ program may be used to display
images from IRAF. Simply start up {\tt saoimage} on the same machine that
you are running IRAF on:  an appropriate command line would be:

{\tt saoimage -quiet -imtool \&}

SAOimage can display IRAF images without having to start up
IRAF:  typing ``{\tt saoimage image.imh}'' should do the trick. Note
that the full name of the image including the suffix {\tt .imh} is
required when using SAOimage from outside IRAF.


\section{Obtaining and Installing IRAF}

This section is probably of no interest to a IRAF user.
It is intended for
system managers who may have to install IRAF,
so skip this section if you are not a system manager.

Installing IRAF is fairly straightforward. A username {\tt iraf} needs
to created whose home directory is $\langle${\it dir}$\rangle${\tt
/local}, where $\langle${\it dir}$\rangle$ stands for the place you
have decided to install IRAF. On UNIX, you would use the {\tt
uncompress} and {\tt tar} programs to unpack the installation kit for
your machine and operating system into the right place in your file
system.

The installation kit for IRAF can be obtained in a number of ways.
Starlink can supply a tape with the 3 {\tt tar} files appropriate to your
machine along with the STSDAS package if it is required. NOAO also
will prepare a tape for you, but they will charge for postage and
packaging. There is an IRAF {\tt ftp} network archive.

 The IRAF network archive contains installation kits for different
machines in separate directories. It is accessible on the Internet
using the {\tt ftp} program or any of the other `network browsing' tools now
available. The Internet name of the archive is {\tt iraf.noao.edu},
and the
number is {\tt 140.252.1.1}. You log on to the archive using the username
``{\tt anonymous}'' and a password composed of your e-mail address. For
SunOS, for example, the installation instructions for IRAF version
2.10.2 can be found in the directory {\tt /iraf/v210/SOS4/}. Go to
that directory and you will find a list of files similar to that shown
below.

{\footnotesize
\begin{verbatim}
ftp>cd iraf/v210/SOS4
250 CWD command successful.
ftp> ls -l
200 PORT command successful.
150 ASCII data connection for /bin/ls (131.251.45.11,1901).
total 5068
-rw-rw-r--  1 212         19785 Dec  3  1992 README
-rw-r--r--  1 212          3338 Dec  8 03:38 README.solaris
drwxr-sr-x  2 root         1536 Jul  9  1992 as.sos4.gen
drwxr-xr-x  2 root         1024 Nov 21  1992 ib.sos4.f68
drwxr-xr-x  2 root         1024 Nov 21  1992 ib.sos4.fpa
drwxr-xr-x  2 root         1024 Nov 21  1992 ib.sos4.spc
drwxr-xr-x  2 root         1024 Nov 21  1992 nb.sos4.f68
drwxr-xr-x  2 root         1024 Nov 21  1992 nb.sos4.fpa
drwxr-xr-x  2 root         1024 Nov 21  1992 nb.sos4.spc
-rw-r--r--  1 root       947593 Jul 10  1992 patch0.tar.Z
-rw-r--r--  1 root      2039731 Jul 24  1992 patch1.tar.Z
-rw-r--r--  1 root      1243869 Nov 21  1992 patch2.tar.Z
-rw-r--r--  1 root       722167 Aug 12  1992 phsibin.403.tar.Z
-rw-r--r--  1 root        14332 Jul 17  1992 suniraf.ms.Z
-rw-r--r--  1 root        33842 Jul 17  1992 suniraf.ps.Z
-rw-r--r--  1 root        33560 Jul  9  1992 sunsmg.ms.Z
-rw-r--r--  1 root        74699 Jul  9  1992 sunsmg.ps.Z
-rw-r--r--  1 root         1904 Jun 26  1992 zzmake
226 ASCII Transfer complete.

\end{verbatim}
}

The README file contains a brief guide to installing IRAF, while more
complete instructions can be found in {\tt suniraf.ps.Z}, which is the
guide to installing IRAF, and {\tt sunsmg.ps.Z}, which is the guide to
managing an IRAF system. Both documents should be read before
attempting to install IRAF. Use the UNIX {\tt uncompress} program
to recover the PostScript files from their compressed forms. Remember
to use binary mode {\tt ftp} to retrieve compressed files.

To install IRAF on a SPARCstation running SunOS, you would require 3
files: {\tt as.sos4.gen}, {\tt nb.sos4.spc.Z} and {\tt ib.sos4.spc}.
The file {\tt as.sos4.gen.Z} is found in
{\tt as.sos4.gen}, a subdirectory
of the directory shown above. It is split into 38
files: each one, apart from the last part, is 512,000 bytes long. This
means that if the file transfer should fail, you would have some of
the files and could continue retrieving the files later. The file
fragments can be concatenated into one file as is detailed in the
{\tt README} file. This process has to repeated for each of the 3 big files
mentioned above.

Installing IRAF does not entail any compilation; this has already been
done at NOAO. The file {\tt as.sos4.gen} is a {\tt tar} file containing all
the text and program sources for IRAF. The files {\tt ib.sos4.spc} and
{\tt nb.sos4.spc} contain the standard IRAF binaries and the binaries
for the {\tt noao} package both of which are compiled for a
SPARCstation.  For a DECStation, the corresponding binaries would be
{\tt as.dsux.gen.Z}, {\tt ib.dsux.mip} and {\tt nb.dsux.mip.Z} and the
installation instructions would be found in the directory {\tt
/iraf/v210/DSUX}.

In summary, before installing IRAF you should obtain and read the
following documents:

\begin{enumerate}
\item{The IRAF installation guide}

\item{The IRAF management guide}

\item{The README file for your particular machine}

\item{The REGISTRATION file}

\end{enumerate}

If you are obtaining IRAF from the network archive, these documents
should be retrieved and read.  If your IRAF kit comes from Starlink,
make sure you ask for these documents. The
{\tt REGISTRATION} file is important as it
contains a form to be filled in and returned to NOAO by e-mail, so that
you can be added to the mailing list for the {\it IRAF Newsletter}\/ which is
published once or twice a year.

Finally, before you can put IRAF on your system, you must have around
{\bf 83\,Mbytes} of disk space available to receive it.

\section{IRAF Directory Structure}

Some users have experienced difficulties when changing from one site running
IRAF to another. There are a number of potential reasons for this, among which
is differences in directory structure. Clearly, the flexibility with which
IRAF can be installed (in particular for sites with several versions of the
software installed) is very useful to system managers. However, if possible,
care should be taken to minimise unnecessary differences between sites by
sticking to the IRAF default choices.

The recommended structure is to have "some directory" (in the IRAF documentation
examples {\tt /iraf}, but {\tt /iraf} can just as easily be a symbolic link to
any directory for the iraf files, so keep this consistent to avoid problems)
for all of the IRAF directories.  These would be e.g.

{\footnotesize
\begin{verbatim}

/iraf/iraf                   - the iraf source tree (i.e. the as.* files)
/iraf/irafbin                - a directory for the bin directories
/iraf/irafbin/bin.sparc      - core system SunOS sparc binaries (the ib.*)
/iraf/irafbin/noao.bin.sparc - NOAO package sparc binaries
/iraf/extern                 - a directory for external pkgs (e.g. TABLES)

\end{verbatim}
}

Even on cross mounted disks an {\tt /iraf} symbolic link would still allow the
path {\tt /iraf/iraf} to be valid on all machines even if the files were elsewhere.
The {\tt irafbin} structure above means that the symbolic links used in the
IRAF tree for the bin directories will be correct when the distribution
files are unpacked.

Users needing to know what the path is they can find the
definition in the {\tt /usr/include/iraf.h} file for that system, since
this should be on all systems running IRAF (because the install script
should be run on all nodes) it works even if there's no IRAF login account
on that machine.


\section{Using IRAF on Alpha Hardware}

Paul Collison, the Starlink system manager at Oxford, has spent some
time working with and modifying the STScI port of IRAF for DEC Alpha
machines. As a result of this he has prepared a document LSN/1.1 (OXF)
describing much of what he (and some of his users) have done. This
document will be of great use to system managers with Alpha systems.

Further information of the use of IRAF at Oxford can be obtained from the
World\,Wide\,Web on URL
{\tt http://www-astro.physics.ox.ac.uk/iraf/}.

\section{STSDAS}

STSDAS is a separate suite of software programs which can be installed
as as a subpackage to IRAF. Again it can be obtained by several
means. Starlink can produce a tape, the STScI
can send you a tape if you pay some transport costs or
there is a network archive from which you can obtain the files at any
time of day or night. The Internet address of this site is
``stsci.edu'', and the Internet number is {\tt 130.167.1.2}.

You can
log on to the archive using ftp with the anonymous login procedure
described earlier for IRAF and the software for STSDAS can be found in
the directory {\tt /software/stsdas/v1.3}. It is available as a {\tt tar}
file, a compressed {\tt tar} file or in 512,000-byte chunks which can be
re-assembled to create the compressed {\tt tar} file. There is a comprehensive
installation guide called {\tt InstallGuide.ps} in the directory {\tt
/software/stsdas/v1.3/doc/manager}. STSDAS needs to be compiled on
your own system; this can take several hours on a SPARCstation, so it
is best done overnight.

Finally, before installing STSDAS, you should have about {\bf
85\,Mbytes} of disk space available.
This could be reduced if you were sure you were
not going to use any of the instrument specific tasks in STSDAS and
could delete parts of the system accordingly.

\section{PROS}

{\tt PROS} is a package for dealing with X-ray data; an e-mail contact
address is given at the end of this document. After installation this
introduces the {\tt xray} and {\tt euv} packages into the top level of the
IRAF package hierarchy.

\section{Useful Addresses}

Both IRAF and STSDAS have a `Hotseat' service. This is an e-mail
address for any problems you may have with the software. They are
usually pretty fast at responding, but do bear in mind the time
difference between the UK and Baltimore or Tucson.

Starlink also provides user support for IRAF. This currently
takes the form of one of the contract Application Programmers being
available to answer IRAF queries as they arise. This
programmer is also a good
source of new copies of the IRAF FAQ -- probably the single most useful
IRAF document.

\begin{verbatim}
   iraf@noao.edu            Address for IRAF problems and queries
   iraf-request@noao.edu    Address for ordering IRAF
   hotseat@stsci.edu        Address for STSDAS problems
   rsdc@cfa.harvard.edu     Address for PROS queries
   ussc@star.rl.ac.uk       Starlink software librarian
   pmc@astro.ox.ac.uk       Paul Collison, author of LSN/1.1 (OXF)
   gjp@astro.cf.ac.uk       Starlink IRAF user support
\end{verbatim}

This information is subject to change.  Consult your Starlink Site
Manager in cases of difficulty.
\end{document}

