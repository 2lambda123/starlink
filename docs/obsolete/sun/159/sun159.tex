\documentstyle[11pt]{article}
\pagestyle{myheadings}

%------------------------------------------------------------------------------
\newcommand{\stardoccategory}  {Starlink User Note}
\newcommand{\stardocinitials}  {SUN}
\newcommand{\stardocnumber}    {159.1}
\newcommand{\stardocauthors}   {M. McSherry\footnote{Prepared for release
 by M D Lawden.}}
\newcommand{\stardocdate}      {19 January 1993}
\newcommand{\stardoctitle}     {XRP --- SMM XRP Data Analysis \& Reduction
Package}
%------------------------------------------------------------------------------

\newcommand{\stardocname}{\stardocinitials /\stardocnumber}
\renewcommand{\_}{{\tt\char'137}}     % re-centres the underscore
\markright{\stardocname}
\setlength{\textwidth}{160mm}
\setlength{\textheight}{230mm}
\setlength{\topmargin}{-2mm}
\setlength{\oddsidemargin}{0mm}
\setlength{\evensidemargin}{0mm}
\setlength{\parindent}{0mm}
\setlength{\parskip}{\medskipamount}
\setlength{\unitlength}{1mm}

\begin{document}
\thispagestyle{empty}
SCIENCE \& ENGINEERING RESEARCH COUNCIL \hfill \stardocname\\
RUTHERFORD APPLETON LABORATORY\\
{\large\bf Starlink Project\\}
{\large\bf \stardoccategory\ \stardocnumber}
\begin{flushright}
\stardocauthors\\
\stardocdate
\end{flushright}
\vspace{-4mm}
\rule{\textwidth}{0.5mm}
\vspace{5mm}
\begin{center}
{\Large\bf \stardoctitle}
\end{center}
\vspace{5mm}

\setlength{\parskip}{0mm}
\tableofcontents
\setlength{\parskip}{\medskipamount}
\markright{\stardocname}

\newpage

\normalsize

\newpage

\section{Overview}

\subsection{Introduction}

This manual describes the various programs that are  available for the
reduction and analysis of XRP data. These programs have been developed under
the VAX operating system. The original programs are resident on a
VaxStation~3100 at the Solar Data Analysis Center (NASA/GSFC Greenbelt MD).
Further information about these programs can be obtained by contacting
15833::ZARRO on DECNET (formerly SPAN).

Some programs are to be run from DCL level, while others are to be run from
IDL.  A short description of XRP data files and how to  access them is given in
the following pages. XRP data files consist of both header and data
information. The header is usually contained in the first record of the data
file, and contains information describing the data.  The header and data are
generally written as unformatted records and cannot be edited with EDT or any
other text editor.

\subsection{Description of the XRP Instrument}

The primary objective of the X-Ray Polychromator (XRP) on the Solar Maximum
Mission (SMM) was the study of the physical conditions in the corona for a
variety of solar phenomena. This was achieved through the use of two
complementary instruments: the Bent Crystal Spectrometer (BCS) and the Flat
Crystal Spectrometer (FCS). The XRP acquired data during the peak of solar cycle
21 (February to November 1980), during its decline (following the in-orbit
repair of SMM in April 1984), and through the rise phase of cycle 22 (until
October 1989).

\subsubsection{The BCS Instrument}

The characteristics of the BCS are listed in Table A. The BCS was designed to:
\begin{itemize}
\item	 monitor the spatially integrated fluxes in 8 high-temperature soft
	  X-Ray emission lines produced by the flare plasma;

\item	 Measure line broadening and shifts for flare dynamics;

\item	 Measure temperature, emission measure, abundances, and equilibrium
	  states to characterise the physical conditions in the flare plasma.

\end{itemize}

The BCS had eight bent Bragg crystal spectrometers with position-sensitive
detectors. It was collimated to 6 arcmin FWHM and was capable of changing its
time resolution from 11~s to 0.5~s or less by using dynamic memory and
sacrificing spectral coverage or resolution. The BCS was sensitive to the
high-temperature ($T_e$ $>$ 10 MK) component of the flare plasma. Most of the
BCS data were taken in the standard flare mode, which recorded spectra every 11
or 7.6~s (depending on the number of operational channels) until a specified
threshold was reached. In flare mode, the integration time decreased to
3.6~s.

\begin{center}
                     Table A: BCS Characteristics
\bigskip
\begin{tabular}{||l l l l l ||}
\hline
{\bf Channel} &  {\bf Ion Stage} &   {\bf Wavelength}  &   {\bf Peak $T_e$} &
{\bf Resolution} \\
{\bf Number}  &                  &         (A)         &   (MK)  &(E/$\delta E$)  \\
\hline
   1  &    Ca XIX          &  3.165 - 3.231   &    35     &        5000  \\
   2  &    Fe (inner shell)&  1.928 - 1.945   &     2     &       14000  \\
   3  &    Fe (inner shell)&  1.839 - 1.947   &     2     &        4500  \\
   4  &    Fe XXV          &  1.840 - 1.984   &    50     &        4500  \\
   5  &    Fe XXV          &  1.866 - 1.879   &    50     &       10000  \\
   6  &    Fe XXV          &  1.854 - 1.867   &    50     &       10000  \\
   7  &    Fe XXV          &  1.842 - 1.855   &    50     &       10000  \\
   8  &    Fe XXVI         &  1.769 - 1.796   &    60     &        8000  \\
\hline
\end{tabular}
\end{center}

After 1980, BCS channels 2 and 8 no longer functioned. Most of the BCS analysis
software has been written for Channels 1 (the Ca XIX channel) and 4 (the
low-resolution Fe XXV channel).

\subsubsection{The FCS Instrument}

The FCS characteristics are listed in Table B. The FCS was designed to:
\begin{itemize}
\item	 Image soft X-rays emitted over a wide range of coronal
	  temperatures (from 2-70~MK, i.e., from coronal active regions to
	  flares) so as to find the flare site in relation to its parent active
	  region and follow the evolution of both;

\item	 Obtain high resolution spectra from 1.5-23\AA to determine plasma
	  temperatures, emission measure, density, abundances, and dynamics;

\item	 Monitor the evolution of X-ray fluxes with up to 0.256~s time
          resolution.
\end{itemize}

The FCS was a seven-channel Bragg crystal spectrometer built on a raster
mechanism. The FCS contained an optical sensor in channel 8 (with a small known
offset from the X-ray detectors) which could be used for image co-registration.
The FCS isolated specific areas in the corona for study through its narrow
collimation  (12-16~arcsec FWHM response, depending on the detector),
producing modest spatial resolution images or well-resolved X-ray spectra. At
the ``home position'' (HP) of the wavelength drive, each X-ray channel could
monitor the emission from a prominent soft X-ray line. By scanning its
wavelength drive, the FCS could access an even wider range of soft X-ray lines.
The FCS could image an area up to 7 arcmin square in up to six bright spectral
lines simultaneously, make rapid scans over a wide wavelength range or at high
resolution across individual line complexes, or combine the two modes to make
spectroscopic observations over a predetermined area.

The FCS could interchange time resolution, spatial coverage and sampling,
and/or spectral coverage and sampling. The FCS was run in a large number of
operational modes that took advantage of its capabilities.  Most observations
included imaging an area of several square arcmin at HP or in a bright line in
a single channel, such as Fe~XVII at 15.01\AA in Channel 1; images were built
up  from boustrophedonic raster scans (with successive rows in alternate
directions), with a typical exposure at each pixel of about 1 second. Thus,
for example, a 4 x 4 arcmin image with 10~arcsec pixels took about 10~mins to
acquire; a 3 x 3 arcmin image with 10~arcsec pixels or a 4 x 4 arcmin image
with 15 arcsec pixels took about 5~mins.  A standard observing sequence began
with a survey image to locate the brightest pixel in a given line, then
centered on a series of (typically smaller) rasters around that pixel, or
recorded time histories at that pixel, or initiated scans of the wavelength
drive there or at a set of adjacent pixels. Receipt of a flare flag, generally
from the BCS, could interrupt the default mode and replace it with one or more
branching sequences in response to the flare, including maps to update the
bright pixel and attempts to initiate wavelength scans if specified criteria
were met.

About 400 FCS observations using the wavelength drive (``crystal runs'') were
executed during the mission. These included scans of individual profiles at one
or more spatial locations, taking 10-30~s each; scans of line complexes
(such as density-diagnostic triplets), taking on order of 1-2~mins; and long
spectral scans of all the accessible wavebands, taking on order of 10-20~mins.
Several sequences combined images in bright lines at several wavelength drive
positions with scans of the corresponding line profiles at the brightest pixel.
Some of the FCS sequences (designed to study active regions or quiescent solar
structures) were run in a ``standalone mode''. Most of the remaining sequences
(designed to study flares) were triggered by a flare flag from the BCS.

\begin{center}
                     Table B: FCS Characteristics
\medskip
\begin{tabular}{||l l l l l l l ||}
\hline
{\bf Channel} & {\bf HP Ion} & {\bf Wavelength} & {\bf Peak $T_e$} &  {\bf Scan
Extent} & {\bf Resolution} &    {\bf  FOV}  \\
{\bf Number} &  {\bf Stage}  &   (A)   &    (MK)     &   (A)    & ($E/\delta E$)
&  (arcsec)\\
\hline
   1 &    O VIII & 18.969  &      3  &   13.1 - 22.4  &    1000  &        16  \\
   2 &    Ne IX  & 13.447  &      4  &   10.5 - 14.9  &    2500  &        15  \\
   3 &    Mg XI  &  9.169  &      6  &    7.3 - 10.1  &    8000  &        14  \\
   4 &    Si XIII &  6.649  &     10 &     4.9 -  7.6 &    5500  &        14  \\
   5 &    S XV   &  5.039  &     16  &    3.6 -  5.8  &   10500  &        13  \\
   6 &    Ca XIX &  3.173  &     35  &    2.4 -  3.6  &    8000  &        12  \\
   7 &    Fe XXV &  1.850  &     50  &    1.4 -  2.1  &    25000 &        14  \\
\hline
\end{tabular}
\end{center}

FCS Channel 6 (Ca XIX) was misaligned with respect to the other detectors, so
that Ca XIX was offset from HP. After 1980, this channel no longer functioned.
After February 1985, engineering data was output into the telemetry
for Channel 6. In early 1985, FCS Channel 2 (Ne~IX) was lost, so that coverage
of the Ne~IX complex was available only at the short wavelength end of Channel
1. Soon thereafter, the Channel 2 telemetry was replaced with telemetry from
the FCS optical sensor \# 8 to allow higher spatial sampling of its output. By
1987, the FCS thin-window detectors began to run out of gas, so that they were
used only intermittently in a ``campaign mode''.  By May 1989, only the
high-energy FCS channels remained, so that the FCS became restricted to flare
studies. After December 1987, the wavelength drive experienced increasing
problems. No FCS spectroscopic sequences were run successfully after April
1988.

For further details of the XRP instrument design and operational capabilities,
see Acton et al., Solar Physics, 65, 53-71, 1980. Instrumental considerations
such as calibration, sensitivity, alignment, fluorescence, and background are
discussed by Bentley (Ph.D. thesis, Department of Physics and Astronomy,
University College London, 1986).

The XRP was built by a consortium of three groups: Lockheed Palo Alto Research
Laboratory (now Lockheed Solar and Astrophysics Laboratory); Mullard Space
Science Laboratory; and Rutherford Appleton Laboratory. The research carried
out with the XRP data has been supported mainly by NASA contracts and grants,
UK-SERC grants, and the Lockheed Independent Research Programme.

\newpage

\subsection{Description of XRP Data Files}

The XRP software described in this manual operates on a variety of
file types. Different file types contain different levels of processed
data ranging from raw detector count rates to actual solar physical parameters.
These file types are listed below.

\begin{center}
\begin{tabular}{||l l||}
\hline
       {\bf File type} & {\bf Description}\\
\hline
                   &            \\

 {\em BCSDA} & {\bf Bent Crystal Spectrometer Data ---} Detector count rates \\
                    & for the eight BCS channels. \\
                    &            \\

 {\em FCSDA} & {\bf Flat Crystal Spectrometer Data ---} Detector count rates \\
                    & for the eight FCS channels. \\
                    &            \\

 {\em BCSIX} & {\bf Bent Crystal Spectrometer Index  ---} An index file \\
                   & providing pointers to the data in the BCSDA file.\\
                   &                                \\

 {\em FCSIX} & {\bf Flat Crystal Spectrometer Index ---} An index file  \\
                    & providing pointers to the data in the FCSDA file.\\
                    &  \\

 {\em ENG} & {\bf Engineering ---} An engineering log file containing detector \\
     &   parameters (e.g. FCS crystal temperature, crystal address, etc). \\
     &  \\

 {\em SPS} & {\bf Spacecraft Pointing Subcom ---} A spacecraft pointing
log \\
                  & containing attitude information (e.g. pitch, yaw, and roll) \\
                    &                               \\

 {\em BDA} & {\bf BCS Data ---} The BCSDA and BCSIX files combined. \\
                   & \\

 {\em FDA} & {\bf FCS Data ---} The FCSDA and FCSIX files combined. \\
                         &                             \\

 {\em BRR} &    A keyed-access file (VMS only) containing BCS spectra \\
                         &  selectively extracted from BDA files \\
                         &  \\

 {\em FIS} & {\bf FCS Images and Spectra ---} An unformatted file containing \\
                         & image and/or spectra extracted from FDA files.\\
                         &                             \\

 {\em SPC} & {\bf Spectral Data ---} A keyed-access file consisting of\\
                         & model Calcium and Iron line spectra fitted to data in\\
                         & the BRR files. \\
                         &                             \\

 {\em FIT} & {\bf Fitted parameters ---} An ASCII file containing various \\
                        & physical parameters (e.g. $T_e$, $EM$, $v_{turb}$) \\
                         & derived from spectral fits to the BRR spectra. \\
                         &                             \\
\hline
\end{tabular}
\end{center}

\newpage

The BCSDA and FCSDA files are considered {\it old format} files in the sense
that the software to read them is no longer supported. These files (together
with the ENG and SPS files)  are created by the REFORMAT program, which copies
XRP data from exabyte tape to disk.  The naming convention of these files is
{\bf BCSDAHHMM.DOY}, {\bf FCSDAHHMM.DOY} etc where:

\begin{itemize}
\item     hh  --- start hour of data
\item     mm  --- start minute of data (approx)
\item     doy --- day of year of data
\end{itemize}

The BDA and FDA files are considered {\it new format} files and are created  by
the programs BOTON and FOTON, respectively. The latter programs  combine the
index and data information into a single new file and also switch  the naming
convention to {\bf BDAYYMMDD.HHMM} and {\bf FDAYYMMDD.HHMM} where:

\begin{itemize}
\item     yy  --- year of data
\item     mm  --- month of data
\item     dd  --- day of data
\item     hh  --- start hour of data
\item     mm  --- start minute of data (approx)
\end{itemize}

FOTON will include the engineering and pointing information in the FDA file.
BOTON does not include this information in the BDA file. A set of FDA and BDA
files has been created  for approximately 400 FCS spectroscopic observations
and over 700 BCS observations for which the count rate in the Ca~XIX channel
exceeded  a threshold of 250 counts/sec (equivalent to a GOES mid-class C
event). These files comprise the XRP Selected Data Archive (SDA) and are
written onto two sides of a WORM optical disk which is stored at the SDAC.

The FIS and BRR files  are processed files that contain selected spectra (for
BRR) or images and spectra (for FIS). These files are created from the FDA and
BDA files using the programs MKFIS and MKBRR, respectively. The bulk of the IDL
software described in this manual operates on these files. The following
flowchart illustrates the steps taken in going from the FCSDA and BCSDA files
to the FIS and BRR files:

\clearpage
\setlength{\unitlength}{1.0pt}
\begin{figure}
\begin{center}
\begin{picture}(150,300)
\thicklines
\put (-16,450){\Large \bf XRP Package Flow Chart}
\put (-20,370){\framebox(200,45){\Huge REFORMAT \small (F/I)}}
\put (-180,230){\framebox(190,35){\Large \it BOTON \small (F)}}
\put (170,230){\framebox(160,35){\Large \it FOTON \small (F)}}
\put (185,110){\framebox(120,30){\large \it MKFIS \small (F)}}
\put (-50,110){\framebox(90,25){\large \it PLOTBDA \small (I)}}
\put (-150,110){\framebox(90,25){\large \it MKBRR \small (F)}}
\put (55,-10){\framebox(75,20){\large \it LISFIS \small (F)}}
\put (145,-10){\framebox(75,20){\large \it PLOTFIS \small (I)}}
\put (255,-10){\framebox(75,20){\large \it FTEMAP \small (I)}}
\put (255,-70){\framebox(75,20){\large \it PLOTFTE \small (I)}}
\put (-150,40){\framebox(75,20){\large \it CALBRR \small (F)}}
\put (-60,40){\framebox(75,20){\large \it PLOTBRR \small (I)}}
\put (30,40){\framebox(75,20){\large \it LISBRR \small (F)}}
\put (-150,-10){\framebox(75,20){\large \it SPECFIT \small (F)}}
\put (-150,-70){\framebox(80,20){\large \it PLOTFIT \small (I)}}
\put (-200,-120){\framebox(85,25){\large \it BCS \small (I)}}
\put (140,-120){\framebox(85,25){\large \it FCS \small (I)}}
\put (-12,370){\line(0,-1){40}}
\put (173,370){\line(0,-1){39}}
\put (-12,330){\line(-1,0){148}}
\put (173,330){\line(1,0){140}}
\put (190,330){\vector(0,-1){63}}
\put (165,300){\circle{30}}
\put (153,297){fcsda}
\put (190,300){\line(-1,0){9}}
\put (229,330){\vector(0,-1){63}}
\put (210,300){\circle{25}}
\put (203,297){eng}
\put (229,300){\line(-1,0){7}}
\put (-40,330){\vector(0,-1){63}}
\put (-18,300){\circle{30}}
\put (-32,297){bcsix}
\put (-40,300){\line(1,0){6}}
\put (-160,330){\vector(0,-1){63}}
\put (-180,300){\circle{30}}
\put (-192,297){bcsda}
\put (-160,300){\line(-1,0){4}}
\put (270,330){\vector(0,-1){63}}
\put (250,300){\circle{25}}
\put (244,297){sps}
\put (270,300){\line(-1,0){7}}
\put (312,330){\vector(0,-1){63}}
\put (292,300){\circle{28}}
\put (283,297){fcsix}
\put (312,300){\line(-1,0){6}}
\put (-105,230){\line(0,-1){27}}
\put (-10,230){\vector(0,-1){93}}
\put (-105,186){\circle{30}}
\put (-117,183){BDA}
\put (-169,186){\line(1,0){49}}
\put (-89,186){\line(1,0){79}}
\put (-105,170){\vector(0,-1){33}}
\put (250,230){\vector(0,-1){88}}
\put (278,186){\circle{30}}
\put (266,183){FDA}
\put (250,186){\line(1,0){12}}
\put (-170,230){\vector(0,-1){322}}
\put (-140,86){\circle{27}}
\put (-152,83){BRR}
\put (-126,86){\line(1,0){20}}
\put (290,110){\line(0,-1){28}}
\put (290,50){\vector(0,-1){38}}
\put (290,-10){\vector(0,-1){38}}
\put (315,-30){\circle{25}}
\put (309,-33){fte}
\put (290,-30){\line(1,0){13}}
\put (290,110){\vector(-2,-1){191}}
\put (290,110){\vector(-1,-1){97}}
\put (290,110){\vector(-1,-2){101}}
\put (-106,110){\vector(0,-1){48}}
\put (-106,110){\vector(3,-2){70}}
\put (-106,110){\vector(4,-1){185}}
\put (291,66){\circle{30}}
\put (282,63){FIS}
\put (-106,40){\vector(0,-1){28}}
\put (-85,25){\circle{20}}
\put (-92,22){cal}
\put (-106,25){\line(1,0){11}}
\put (-132,-10){\vector(0,-1){38}}
\put (-153,-30){\circle{25}}
\put (-160,-33){spc}
\put (-141,-30){\line(1,0){9}}
\put (-91,-10){\vector(0,-1){38}}
\put (-69,-30){\circle{25}}
\put (-76,-33){fit}
\put (-82,-30){\line(-1,0){9}}
\put (-100,-160){\em Square boxes refer to command routines; the circles are the input/output files.}
\put (-100,-175){\em BCS programs are listed on the left; FCS programs on the right.}
\put (-100,-190){\em Fortran programs are denoted by (F); IDL programs by (I)}
\end {picture}
\end{center}
\end{figure}
\clearpage
\setlength{\unitlength}{1mm}
\newpage

\subsection{Summary of XRP Software}

\subsubsection{Fortran Data-reduction Programs}

\begin{description}
\item [BCS Programs:]\hfill
\begin{description}
\item[BOTON --- ]          Creates BDA files from BCSDA and BCSIX files.
\item[MKBRR --- ]          Creates BRR files from BDA files.
\item[LISTBRR --- ]         Lists the contents of a BRR file.
\item[CALBRR --- ]         Performs a wavelength calibration for BRR spectra.
\item[SPCFIT --- ]        Fits BCS channel 1 or 4 with model spectra.
\item[BCSPRO --- ]         Sets up a command file to run MKBRR, CALBRR, SPCFIT.
\end{description}
\item [FCS Programs:]\hfill
\begin{description}
\item[FOTON --- ]          Creates FDA files from FCSDA, FCSIX, ENG and SPSub files.
\item[MKFIS --- ]          Creates FIS files from FDA files.
\item[LISTFIS --- ]         Lists the contents of a FIS file.
\end{description}
\end{description}

\subsubsection{IDL Data-analysis Programs}

\begin{description}
\item [BCS Programs:]\hfill
\begin{description}
\item[PLOTBDA --- ]       Plots lightcurves and spectra from BDA files.
\item[PLOTBRR --- ]       Plots BRR spectra.
\item[PLOTFIT --- ]       Plots FIT files from BCSPRO.
\end{description}
\item [FCS Programs:]\hfill
\begin{description}
\item[PLOTFIS --- ]       Plots FIS spectral data files.
\item[RDFIS --- ]         Reads FIS image files.
\item[RUST --- ]         Reads and unpacks data and pixel coordinates arrays
 for images saved in a structure variable created by RDFIS.
\item[FLUXCAL --- ]       Calibrates FCS counts/sec to flux
 (photons $s^{-1} cm^{-2}$) in a given channel.
\item[FTEMAP --- ]        Computes the isothermal temperature and column
 emission from the ratios of the FCS ion Maps.
\item[PLOTFTE --- ]       Plots temperature and emission measure maps saved in
 FTE.
\item[TEMPCAL --- ]       Plots emissivity ratios.
\item[EMRAT --- ]         Plots emissivity ratios.
\end{description}
\end{description}

\subsubsection{IDL X-Windows/Widget Programs}

\begin{description}
\item[FCS --- ]          Creates a widget interface to run FCS software.
\item[BCS --- ]          Creates a widget interface to run BCS software.
\item[LISTBCS --- ]      Creates a widget interface to the  BCS catalog of
 flares.
\item[SCANPATH --- ]     Reads and extracts procedures within the XRP library.
\end{description}

\subsection{Getting started --- some examples for new users}

This section shows you how to get started.
Logical names are defined which allow you to access one .FIS and one .BDA file
within the demo directory XRP\_DIR:[XRP.DBASE.DEMO].
Commands which you should type are shown after the `XRPIDL$>$' prompt.
\begin{itemize}
\item Enter the XRP package:
\begin{verbatim}
    $ XRP
    XRP$> idl                                    ! Enter IDL environment
\end{verbatim}
\item Examine a .FIS file via an X-Windows image display:
\begin{verbatim}
    XRPIDL> FCS
\end{verbatim}
\item Examine a .FIS file image via a non-X-Windows display:
\begin{verbatim}
    XRPIDL> PLOTFIS
\end{verbatim}
\item Examine a .FIS file spectra via a non-X-Windows display:
\begin{verbatim}
    XRPIDL>PLOTFIS
\end{verbatim}
\item Examine a .BDA file via an X-windows image display:
\begin{verbatim}
    XRPIDL> BCS
\end{verbatim}
\item Examine a .BDA file via a non-X-Windows display:
\begin{verbatim}
    XRPIDL> PLOTBDA
\end{verbatim}
\end{itemize}

\subsubsection{REFORMAT}

{\em To run this program type:}
\begin{verbatim}
    XRP$> REFORMAT
\end{verbatim}
 This program consists of a command procedure designed to reformat XRP data.
 It runs several other programs in sequence namely DUMPEXB, RFTCTL, SORTEXB,
 and BOTON or FOTON (each step can be skipped if necessary).
 The program can be divided up into four Phases:
\begin{itemize}
\item	{\bf Phase 1: }
	The IDL-V2 procedure DUMPEXB extracts the relevant data from a XRP raw data Exabyte tape.
	The data are stored in the files assigned to the logicals DMPFL1 and DMPFL2 (in that order).
	The DUMPEXB actions are registered on a file COPYEXB.TXT
	If the requested data are already present in existing dump files this phase can be skipped.
\item	{\bf Phase 2: }
	The F77 executable RFTCTL is run to define the exact time range and type of data to be
	reformatted. This information is stored on the file SCRTNspec, where `spec' stands for some
	label specified by the user; usually it will be in the format HHMM.DOY (i.e. Day of Year, Hours
	and Minutes of the start time of the requested data period).
	If the proper SCRTN file is already known to exist, this phase can be skipped.
\item	{\bf Phase 3: }
	The actual reformatting is preceded by:
	1. Selection of the directory where the reformatted files should be put.
	2. Making sure that the dumpfile with the requested data is assigned to DMPFL1; this may involve
	   interchanging the assignments of the logicals DMPFL1 and DMPFL2.
	The F77 executable SORTEXB is run next; this does the actual reformatting and will produce output
	files FCSDAspec, FCSIXspec, BCSDAspec, BCSIXspec, SPSUBspec and ENGspec.
	If these files already exist, this phase can be skipped.
\item	{\bf Phase 4: }
	The procedure FOTON is run; this combines the FCSDA,FCSIX,ENG
        and SPS files into an FDA file. Or if you require BCS data
        BOTON is run combining BCSDA and BCSIX files into a BDA file.
\end{itemize}
        As the program is proceeding through its four phases the
	user will be prompted for a number of responses (default choices are give between brackets and
	are selected by hitting return):
\begin{itemize}
\item	{\bf Phase 1: }
This phase commences with the question:
\begin{verbatim}
        Run DUMPEXB to read XRP raw exabyte tape.
        If DUMPEXB is skipped, reformatting continues
        using the data currently stored in the dump files

 Run DUMPEXB (YES/NO) [def=NO]?
\end{verbatim}
You must enter YES the first time in order to read and copy the data. This copy
is performed by the IDL procedure DUMPEXB which extracts the relevant data (in
24 hour blocks) from exabyte to magnetic disk.  The data will be stored in two
dump files that are assigned automatically to the logicals DMPFL1 and  DMPFL2.
(The second file is used in case the requested data interval spans midnight.)
If the requested data are already present in existing dump files, then this
phase can be skipped by answering NO.

If YES, then you must first load  the exabyte tape corresponding to
the month containing the required data. You will then be asked to enter the
number corresponding to the tape drive in which the exabyte is loaded. DUMPEXB
will mount this drive and then present the following
menu:
\begin{verbatim}
        select option:
                0 - dump file
                1 - skip files
                2 - rewind input tape
                9 - stop
        option ?
\end{verbatim}
Before copying any data, you must first skip the requisite number of files to
reach the start record of the required day on tape. Option 1  allows you to
skip a specified number of files (days) on the tape.  The day of year (DOY)
currently available for dumping is given on the line  preceding the menu
(starting with the word LABEL). If  the DOY indicated on the  LABEL line is N,
then skipping of M files will take you to the start of  DOY N+M. Selection of
option 1 will be followed by a prompt for the number of  files (days) to be
skipped. At most five files can be skipped in one go.

Option 0 will dump one or two files (one file contains one day of data)  into
the dump files.  Selection of option 0 will be followed by a prompt for the
number of  files (days) to be dumped. One or two files (days) can be dumped;
the first will be dumped into  DMPFL1, the second into DMPFL2. To conserve
space, it is recommended that only one file (day) be dumped at a time,
unless the interval of interest extends into the subsequent day.

Option 2 will rewind the tape and option 9 will exit DUMPEXB. All choices
(including typing errors) except 9 will eventually return  you to the menu.

\item{\bf Phase 2}:

After DUMPEXB has completed copying data onto disk, you can extract
the time interval of interest from the created dump files by
answering YES to the following question:
\begin{verbatim}
        Information about the data to be reformatted is stored in
        a SRTCN* file.  If the required SRTCN* file is available
        already, answer NO.

  Select new set of reformat times (YES/NO) [def=YES]?
\end{verbatim}
The answer NO will take you to phase 3. The answer YES will be
followed by a number of prompts describing how you would like the data
reformatted. The first prompt is for a file ID to be used in naming the output
reformatted files:
\begin{verbatim}
        File ID [hhmm.DOY] ?
\end{verbatim}
There is no default for this question. You will be prompted until you
have entered some ID. The most convenient choice is an eight letter ID,  like
0510.123 for data starting on 05:10 UT on day of year 123. Next, you will be
asked  whether or not you want BCS and/or FCS data,  and for a start and stop
time for each data set. This information will be stored in the file
SRTCNhhmm.doy. It is very important that you do not select a time interval
that is longer than  two hours as this will cause integer overflows in the
software. Hence, for time intervals exceeding two hours,  it is recommended to
partition the intervals into several sub-intervals of less than two hours
duration.

\item{\bf Phase 3}:

This is the actual reformatting phase. You will be asked:
\begin{verbatim}
        If the required reformatted files exist already,
        you do not have to run the reformatter SORTEXB;
        Answer NO to skip to the FDA/BDA file production

  Reformat most recent selected times [def=YES]?
\end{verbatim}
If the reformatted files already exist, it is possible to skip to
phase  4 by answering NO here. The answer YES will be followed by a prompt for
the name of the directory where you want to store the reformatted files (the
default is the current directory):
\begin{verbatim}
        Device and directory (^Z=Exit) ? [def=DIR]
\end{verbatim}
REFORMAT will switch to the indicated directory. If the move is
unsuccessful (e.g. non-existent  directory), a message is displayed and the
above question appears again. If necessary, pressing Ctrl-Z will abort the
procedure. If the required data have been written by DUMPEXB into the dumpfile
assigned to DMPFL2 then it will  be necessary to interchange the assignments of
DMPFL1 and DMPFL2. You will only need to flip assignments if the start time of
the data interval occurs after midnight. This  option is given in the next
question:
\begin{verbatim}
        Flip logical assignments FILE1 and FILE2 (YES/NO) ? [NO]
\end{verbatim}
Using the time interval and file type information in SRTCNhhmm.DOY,
program  SORTEXB will  produce FCSDA, FCSIX, BCSDA, BCSIX, ENG, and SPSUB files
--- each with the hhmm.doy ID incorporated into the filename. The ENG and SPS
files contain engineering (e.g. FCS crystal address, temperature) and
spacecraft pointing  (pitch, yaw, and roll) information, respectively.

\item{\bf Phase 4}:

After SORTEXB has completed, the FOTON program will combine newly
created FCSDA, FCSIX, ENG, and SPS files into an FDA file.  If you requested
BCS data, then the BOTON program will combine the BCSDA and BCSIX files
into a BDA file.  In case the REFORMAT procedure was interrupted during any of
the previous phases, you may be asked to re-enter the file ID in order to
complete the FDA and BDA file creation. Upon completion, you will be returned
to your working directory where you will find the reformatted files and the BDA
and FDA files. You may want to delete the COPYEXB.DAT and SRTCNhhmm.doy files
which contain information about the times of the reformatted data.

\end{itemize}

\newpage

\section{Fortran Data Reduction Programs}

This section describes the XRP Fortran data reduction programs.
These Fortran programs are used primarily for the conversion of one file type
to another, or for  the extraction of selected data products from files. In
general, they are executed by typing a DCL symbol that is defined in the
LOGIN.COM file for the XRP account. The programs are described in the
chronological order in which they are likely to be used.

\subsection{BCS Programs}

\subsubsection{BOTON}

{\em To run this program type:}
\begin{verbatim}
     XRP$> BOTON `filid'
\end{verbatim}
{\em BOTON} creates BDA files from BCSDA and BCSIX files. Although
   BOTON is run automatically during the REFORMAT procedure, it can be
   executed manually as shown above.
   Where `filid' is the hhmm.doy extension of the BCSDA and BCSIX files
   produced by REFORMAT. If the reformatted files are not in the current
   directory, then the full directory and filename must be entered on the command
   line. BOTON will prompt for these inputs if `filid' is not entered. Note that
   the BDA file id is different from that of the BCSDA file. The BDA file contains
   spectra (at the original observed time resolution) for all the available BCS
   channels. These spectra can be examined with the IDL program PLOTBDA.

\subsubsection{MKBRR}

  {\em MKBRR} will produce a BRR file, using a BDA file as input.
   The BRR files are keyed-access ISAM files that contain selected spectra that have been
   accumulated  for a specified time and corrected for crystal curvature effects.
   These files are the starting point for the derivation of flare parameters
   (temperature, emission measure, velocity) from the BCS spectral data.

    MKBRR can be run interactively or by using command line switches.
    The following is a series of examples of the program running in different
    modes:
\begin{center}
                   ***	Interactive mode    ***
\newline	(specify no switches, i.e. no "/" symbols should appear)
\end{center}
\begin{verbatim}
    $ MKBRR
    $ MKBRR 800629.1810
\end{verbatim}
  Within the  interactive mode the user will be asked the following questions:
\begin{verbatim}
   * Enter BDA File id or name [yymmdd.hhmm]

     In default mode :,
        -  Channels 1 & 4 are processed,
        -  20 sec is assumed for data gathering interval,
        -  Crystal curvature corrections applied.

   * Use default mode ? [def=y]
\end{verbatim}
Answering YES (after entering the file name) will invoke the default
mode. In this mode, MKBRR extracts spectra for channels 1 (Ca~XIX) and 4
(Fe~XXV), and accumulates the data into approximately 20~s intervals. Curvature
corrections are also applied. These corrections essentially straighten the
spectra, and have been calibrated only for channels 1 and 4. Answering NO
will allow you to override these defaults.

The non-interactive mode allows more control over the BRR output file.
There are 6 switches available listed in the following table.
    They may be included in any order.
    If the /fil: switch is present, MKBRR will run in non-interactive mode.
    If the /fil: switch is NOT present, the other switches will have no effect.
  An example of their usage is as follows:
\begin{center}
                   ***	Non-interactive mode    ***
\end{center}
\begin{verbatim}
    $ MKBRR /fil:800629.1810/chn:1/int:15/sta:18,25/end:18,30/cnt:4000,1
\end{verbatim}
In this example, channel~1 Ca~XIX  data in the BDA file with file id
800629.1810 will be accumulated into 15s (or greater)  intervals starting at
1825~UT and ending at 1830~UT. The accumulation interval for each spectrum
will be increased until a 4000 total count level is reached in channel 1.  The
spectra within the BRR file can be examined by using the IDL program PLOTBRR.
In practice, you would use MKBRR to extract BCS channels 1 and/or 4.
The remaining Fortran BCS analysis programs that are described in this section
do not operate on any of the other channels.

\newpage

\begin{center}
\begin{tabular}{|l l|}
\hline
{\bf Optional switches} & {\bf Description}  \\
\hline
              &    \\

	/fil: & This switch specifies the file id of the BDA file.\\
              & \\
	      &	If this switch is present, MKBRR is will ask no questions \\
	      &	(appropriate for running in batch mode) and by default, \\
	      &	the crystal curvature corrections will be applied. \\
              &    \\
	      &	In addition, if this switch is NOT present,  \\
	      &	the /chn:, /int:, /sta:, and /end: switches are ignored. \\
              &    \\
	/chn: &	Use this switch to specify which channel to process.  \\
	      &	If this switch is absent, MKBRR writes channel 1 and 4 data. \\
              & Can also specify multiple channels as /chn:1,4 (no blanks). \\
              & This switch is ignored if the /fil: switch is absent. \\
              &  \\
	/int: &	Integration time (integer) in secs.  The default is to  \\
              &  accumulate spectra with 20s time resolution. \\
              &  This switch is ignored if the /fil: switch is absent. \\
              &   \\
	/cnt: &	Sets the (total) count level (integer) which must be  \\
              &  accumulated.  The integration time will be the minimum  \\
              &  specified with the /int: switch.  If the specified counts  \\
              &  is not detected, the integration time is increased until  \\
              &  desired count rate is reached.  The format is /cnt:level,chan \\
              &  such as /cnt:1000,1 (no blanks allowed).  If the channel  \\
              &  number is omitted, the lowest selected channel is used.  \\
              &  If the /cnt: switch is supplied, /int:0 be used.  In this  \\
              &  case the shortest integration times which supply the desired  \\
              &  total count rate in the specified channel will be used.  If  \\
              &  multiple channels are selected, the same integration time will \\
              &   used for the other channels. \\
              &  This switch is ignored if the /fil: switch is absent. \\
              &  \\
	/sta: &	The start time in hh,mm,ss,dd,mm,yy format (from minutes  \\
              &  onward the quantities are optional). The omission of this switch  \\
              &  will cause MKBRR to begin with the earliest available data. \\
              &  \\
	/end: &	The end time in hh,mm,ss,dd,mm,yy format. The omission of this \\
	      &	switch will cause MKBRR to process until the end of the  \\
              &  BDA file. \\
              &  \\
\hline
\end{tabular}
\end{center}

\newpage

\subsubsection{LISTBRR}

{\em To run this program type:}
\begin{verbatim}
     XRP$> LISTBRR
\end{verbatim}
  {\em Listbrr} lists the contents of a BRR file. The user is asked
   only to supply a file name (it is not necessary to type the BRR prefix).
   After which the program lists the file as shown below:
\begin{verbatim}
 Summary of BRR file: SYS$DATA:[XRP.DATA]BRR850123.0657;12

 Total number of spectra:  292   146   0   0 146   0   0   0   0

 Chan      Mean time          sdgi    num     dw        w0     rate Corrected
   1    7  5 10  23  1 85   921.600     1  3.05E-04  3.23E+00     4    1 0
   1    7 19 25  23  1 85   783.360     3  3.05E-04  3.23E+00     4    1 0
   1    7 26 28  23  1 85    65.280     5  3.02E-04  3.24E+00    64    1 1
   1    7 27 15  23  1 85    30.720     7  3.11E-04  3.24E+00   141    1 1
   1    7 27 45  23  1 85    19.200     9  3.01E-04  3.24E+00   221    1 1
   1    7 28  9  23  1 85    15.360    11  3.02E-04  3.24E+00   279    1 1
   1    7 28 25  23  1 85    15.360    13  3.02E-04  3.24E+00   329    1 1
   1    7 28 38  23  1 85    11.520    15  3.00E-04  3.24E+00   379    1 1
   1    7 28 50  23  1 85    11.520    17  3.05E-04  3.24E+00   405    1 1
\end{verbatim}

\subsubsection{CALBRR}

{\em To run this program type:}
\begin{verbatim}
     XRP$> CALBRR `filid' [/vel] or [/novel]
\end{verbatim}
This program wavelength calibrates the BCS channel~1 (Ca XVIII-XIX) and/or
channel~4 (Fe XXII-XXV) spectra contained in a BRR file. Because of the nature
of a Bragg crystal spectrometer, the relationship between detector bin and
wavelength scale is not absolute, and depends upon the relative  position of
the  emitting source within the spectrometer field of view.  CALBRR calibrates
this scale within the rest frame of the observer by fitting key lines within
the spectra, and referencing their centroids to known laboratory wavelengths.

The `filid' is the usual file id of the BRR file to be processed.
The /vel switch will implement the fitting of a secondary blueshifted
component to the Ca~XIX and Fe~XXV $w$ resonance lines in channels 1 and 4,
respectively. By default, this switch is on. In cases where a blueshifted
component is unlikely to be pronounced (such as in a limb flare), it is
recommended that blueshift fitting be turned off by using the /novel switch.
CALBRR automatically updates the wavelength scale  of individual spectra in the
BRR file. If `filid' is not entered, then CALBRR will run interactively and
prompt for assorted inputs. The results of CALBRR are written in the summary
ASCII file  CALyymmdd.hhmm, which inherits the file id of the BRR file.

\subsubsection{SPCFIT}

{\em To run this program type:}
\begin{verbatim}
     XRP$> SPCFIT
\end{verbatim}
{\em SPCFIT} fits theoretical profiles  to  BCS channel~1 (Ca XVIII-XIX) and/or
channel~4 (Fe XXII-XXV) spectra contained in a  BRR file. The electron
temperature, emission measure, nonthermal  broadening, and line-of-sight upflow
velocity of the soft X-ray source are determined from  these fits.
This program has three control switches.
\begin{verbatim}
             XRP$> SPCFIT /fil:filid /chn:1 /vel:3
\end{verbatim}
\begin{center}
\begin{tabular}{|l l|}
\hline
{\bf Optional switches} & {\bf Description}  \\
\hline
              &    \\
        /fil: & Input BRR file id. \\
              &  \\
        /chn: & optional switch. If absent, both channels 1 and 4 are fit. \\
              &  \\
        /vel: & Optional switch [def $=$ 3]. \\
              & This controls how SPCFIT will fit a single or a \\
              & 2-component spectrum. \\
              &    \\
\hline
\end{tabular}
\end{center}
It is important that the spectra for the requested channel be extracted first
by MKBRR and wavelength calibrated by CALBRR before being processed by SPCFIT.
Note that  SPCFIT (like CALBRR) will process only channels 1 and/or 4. In
addition, SPCFIT   will fit only spectra with sufficiently good statistics.
Hence, the program  may skip low count rate spectra observed very early during
the flare impulsive  phase or late during the decay phase.

The /vel switch is an optional input that controls how  fitting of a secondary
blueshifted component will be handled. CALBRR will have  already made a first
approximation at fitting this component if the /vel  switch was used. A
detailed description of the velocity model used by SPCFIT is  given in Fludra
{\it et al.} (1989, {\it Ap.J.}, {\bf 344}, 991). Basically, the  model assumes
that the doppler broadening width of the secondary component is  proportional
to its blueshift velocity from the primary  $w$ resonance  line of Ca XIX (or
Fe~XXV).
This switch can assume the following values:
\begin{itemize}

\item 0: do not fit a blueshifted component. This is the stationary model.

\item 1: use the blueshift results from CALBRR as final, and do not iterate
further;

\item 2: use the blueshift results from CALBRR. Iterate further on the
intensity of the blueshifted component, but keep it's velocity fixed; and

\item 3: use the blueshift velocity and intensity from CALBRR
as first guesses and iterate further. This is the default option.

\end{itemize}
The input files are as follows:
\begin{description}
\item [BRRxxxx.xxx] ---	The BCS data is read from the BRR files.
The files must have been updated by the CALBRR program before SPCFIT will
attempt to fit the spectra.
\item [CALxxxx.xxx] --- The program searches for this file in the current
directory, and then in the directory of the BRR file if the BRR file is not in
the current directory.
\end{description}
The output files are as follows:
\begin{description}
\item [FITxxxx.xxx] --- Results are written to an Ascii file called
FITyymmdd.hhmm, which has the same file id as the input BRR file.
Contains the fitted parameters (temperature, emission measure, doppler
broadening, and blueshift) derived from the theoretical profiles fitted to the
Ca~XIX or Fe~XXV spectra.
These parameters can be examined by using the IDL program PLOTFIT.
\item [SPCxxxx.xxx] --- Is a keyed-access isam file called SPCyymmdd.hhmm, which
has the same file id as the input BRR file created.
This file contains the final synthesized theory spectra.
These spectra can be examined using the IDL program PLOTFIT.
\end{description}

\subsubsection{BCSPRO}

{\em To run this program type:}
\begin{verbatim}
     XRP$> BCSPRO `filid' INT
\end{verbatim}
This program facilitates the task of fitting BCS channel 1 and 4 spectra in BDA
files by creating a command file to run MKBRR, CALBRR, and SPCFIT in batch
mode. It can be run non-interactively by typing the above command.
Where `filid' is the file id of the input BDA file, and INT is the
integration time in seconds. The default integration time is 20~s, which is
optimum for producing spectra of sufficiently high signal to noise to allow
good quality fits of theoretical models.  BCSPRO will then submit a batch job
to run the above three programs in succession. The name of the batch command
file is: BCSPROhhmm.dmp, where hhmm is the file extension of the input file id.

If the file id is omitted, then the program will run interactively
and the following prompts will appear:
\begin{quote}
\begin{verbatim}
  *****  XRP BCSPRO PROGRAM  V2.1  Mar 1990        *****

  Three operations may be performed:

  1) MKBRR         - Write BRR file (read BDA file)
  2) CALBRR        - Calculate DW,W0, write CAL file
  3) SPCFIT        - Fit Te,Td,EM     write FIT file

  Enter 1 or 1,3 for: 1+2+3
        2 or 2,3 for: 2+3
        1,1      for: 1, etc.

           [def=1,3]:
\end{verbatim}
\end{quote}
At this point, you can control which programs to run by entering the
appropriate number. For example, entering 1 will initiate all three programs in
succession. You will then be asked the following series of questions:
\begin{verbatim}
    * Enter the BDA file id or filename:
    * Enter the integration time (sec) [def=20]:
    * Change the start or stop times ? [def=N]:

    The command file = BCSPROhhmm.dmp
\end{verbatim}
The submit instruction looks like:
\begin{verbatim}
    $ SUBMIT/NOPRINT/ID/NOTIFY/NOLOG/RESTART BCSPROhhmm.DMP

    * Submit BCSPROhhmm.dmp file as a batch job? [def = y]:
\end{verbatim}
Accordingly, you enter the BDA file id, the integration time,  and
whether or not you wish to change the start and end times for processing.  By
default, BCSPRO will adopt the start and end times of the data in the BDA file.
The final question asks whether you wish to submit the subsequent program
execution as a batch job. For both the interactive and non-interactive modes,
the completion of the batch job will be signalled by an e-mail message sent to
your login account.

\subsection{FCS Programs}

\subsubsection{FOTON}

{\em To run this program type:}
\begin{verbatim}
     XRP$> FOTON `filid'
\end{verbatim}
{\em FOTON} creates FDA files by combining the FCS data in FCSDA and FCSIX files,
together with the engineering ENG files and pointing SPS files. It is run
automatically at the end of the reformat process. Or it may also be run manually
by typing the command as shown above.
Where `filid' is the hhmm.doy extension of the FCSDA and FCSIX files.
reformatted files are not in the current directory, then the full directory and
filename must be entered on the command line. FOTON will prompt for these
inputs if `filid' is not entered.  It is very important that the ENG and SPS
files exist for FOTON to create a complete and correct FDA file.  Although
FOTON will create an FDA  file without these ancillary files, the subsequent
analysis software may  crash if the engineering and pointing data are missing
from the FDA file. Note that the FDA file id is different from that of the
input FCSDA file.

\subsubsection{MKFIS}

{\em To run this program type:}
\begin{verbatim}
     XRP$> MKFIS `filid'
\end{verbatim}
{\em MKFIS} creates FIS files from FDA files.  The FIS file contains  the
available images and spectra observed with the  FCS during the reformatted data
interval.  The parameter 'filid' is the yymmdd.hhmm extension of  the FDA file.
If the FDA file is not in the current directory, then the full  directory and
filename must be entered on the command line. MKFIS will prompt  for these
inputs if 'filid' is not entered. Note that MKFIS may often not  process
incomplete image or spectral modes (such as modes interrupted by  spacecraft
night of South Atlantic Anomaly [SAA]). In addition to creating a FIS  file,
MKFIS will output a log file: FISyymmdd\_E.hhmm. The log file can usually  be
ignored.

\subsubsection{LISTFIS}

{\em To run this program type:}
\begin{verbatim}
     XRP>$ LISTFIS
\end{verbatim}
  This program will list the contents of a FIS file. To begin with the user is asked if
  the output is to be sent to a file (default is NO). After which the user types
  1 for a brief listing or 2 for a full listing, The program then prompts for
  the file name (at this point you may then type the full directory specification of
  the file, depending on its current location.) you may also type `?' for help or
  use `*' as a wildcard. The program will then print the contents of the file
  for a brief listing as shown below to either the terminal or a file.
\begin{verbatim}
 Data    Time    Date   Mode    #Yx#Z(dY)   St. Add.  Exp.         ModeID
 Set                                       Add  Step (DGI) Pts. Reg.
   1  14:45:50  14/ 7/86  RAST 16x16( 3)   6353   0     4    1  1 182 1.0.24.0
   2  14:50:17  14/ 7/86  RAST  5x 5( 1)   6353   0     4    1  1   2 1.0.24.0
   3  14:51:41  14/ 7/86  SS    1x 1( 0)    528   3     1 3325  1   0 3.0.28.1
   4  15:06:01  14/ 7/86  SS    1x 1( 0)    528   3     1 3325  1   0 3.0.28.1
   5  15:20:20  14/ 7/86  SS    1x 1( 0)    528   3     1 3325  1   0 3.0.28.1
   6  15:34:39  14/ 7/86  SS    1x 1( 0)    528   3     1 3325  1   0 3.0.28.1


 Data    Time    Date   Mode     Initial    Spacecraft       #Samp #Bad  %
 Set                              Y  Z    Pitch  Yaw  Roll
   1  14:45:50  14/ 7/86  RAST    -23   23  -82 -970     0.0  256   0 100.0
   2  14:50:17  14/ 7/86  RAST      2   -5  -82 -970     0.0   25   0 100.0
   3  14:51:41  14/ 7/86  SS        3   -7  -82 -970     0.0 3325   6  99.8
   4  15:06:01  14/ 7/86  SS        3   -7  -81 -970     0.0 3325   1 100.0
   5  15:20:20  14/ 7/86  SS        3   -7  -82 -970     0.0 3325   0 100.0
   6  15:34:39  14/ 7/86  SS        3   -7  -75 -970     0.0 33252386  28.2
\end{verbatim}

\newpage

\section{IDL Data Analysis Programs}

This section describes IDL (version 2) programs that can be used
to read, display, and analyze XRP data.

\subsection{BCS Programs}

\subsubsection{PLOTBDA}

{\em PLOTBDA} plots BCS lightcurves and/or spectra contained in BDA files
created BOTON.   The program will first prompt for the
BDA filename and directory location. If you can't remember the full BDA file
name, you can use wild characters (e.g. BDA*.*) and a menu of all the available
BDA files on the current (default) disk  will be listed for you.

After selecting the file, you will be asked to enter the channel numbers to
extract. Because of IDL memory limitations, it is unwise to request more than
three channels. Typically, you would enter 1,4,7 to read the Ca~XIX, and the
Fe~XXV low and high resolution channel data (\S~1.2.1 Table 1-1), respectively.
You will then be asked for the maximum accumulation time over which to sum the
data. The default is 10~s, which means that the BCS data in the selected
channels will be summed until the total integration time per spectrum is
approximately 10~secs. Note that, because the actual BCS integration times are
usually non-integer (e.g. 3.6~s around the flare peak, and 7.2~s during the
decay), the final accumulation time will rarely be exactly equal to the
requested accumulation time.

After reading and accumulating the requested channel data, PLOTBDA will ask you
to enter a channel number to plot. This question will be skipped if you
initially extracted only one channel. A lightcurve of the selected channel will
then appear. You will then be asked whether you wish to plot spectra. Answering
YES (or Y) will produce a vertical cursor (ignore the horizontal cursor). You
will be asked to mark the start and end times (on the lightcurve plot) during
which you would like to extract and accumulate spectra.   Hitting any key will
automatically select the spectrum nearest the time where the vertical cursor is
located. Hitting a key once, then moving the cursor to another position on the
lightcurve, and then hitting a key again will cause PLOTBDA to select all
spectra in between the start and end cursor positions and  plot an average of
the selected spectra. Of course, hitting a key twice without moving the cursor
will cause PLOTBDA to plot a single spectrum nearest the cursor position.
PLOTBDA will ask next whether you wish to make a hardcopy of the plotted
lightcurve and/or spectrum. Answering YES (or Y) will generate a Postscript
plot file (IDL.PS) that will be printed on the available output device.
Finally, PLOTBDA will ask whether you wish to repeat the whole plot process.

PLOTBDA has several optional command line arguments and keyword parameters that
provide  access to the plot arrays and more control over the plot. (N.B.
although command line arguments are optional, they must be entered in the
correct order. Keywords can be entered in any order).

\noindent {\underbar{Calling sequence:}}

\noindent PLOTBDA,{\it T,Y,W,Z}

\noindent {\underbar{Outputs:}

\noindent{\it T} -- the time array (in seconds since the start of day) for the
plotted lightcurve.

\noindent{\it Y} -- the count rate array for the plotted lightcurve.

\noindent{\it W} -- the bin (or wavelength) array for
the spectrum selected using the cursor.

\noindent{\it Z} -- the corresponding spectrum array (counts per sec per bin).

\noindent {\underbar{Keywords:}}

\noindent{\it LOG} -- requests that the lightcurve be plotted on a $log_{10}$
scale.

\noindent {\it EBAR} -- requests that uncertainties be overplotted on the
lightcurve.

\noindent {\it WAVE} -- requests that BCS detector addresses be
converted to their nominal wavelength (\AA) values.

\noindent {\it BIN} -- a value by which to rebin the spectrum array
                   (e.g. bin=2 will double bin the spectrum).

\subsubsection{PLOTBRR}

PLOTBRR plots BCS lightcurves and/or spectra contained in BRR files
created by MKBRR. The program is analogous to PLOTBDA
except that you will not be asked to enter the channels to extract nor the
accumulation time. The latter processing was performed by MKBRR. The program
will first prompt for the BRR filename and directory location. After reading
the file, it will then list the available channels, their associated ions
(usually Ca~XIX and Fe~XXV),  and the total number of spectra in each channel.
You will then be asked to select a channel to plot.

PLOTBRR will first plot the lightcurve for the selected channel. It will then
ask whether you wish to plot spectra. From this point on, the program will
proceed exactly as PLOTBDA. In addition, PLOTBRR has the same optional command
line arguments and keyword parameters as PLOTBDA.

\noindent {\underbar {Calling sequence:}}

\noindent {PLOTBRR,{\it T,Y,W,Z}

\subsubsection{PLOTFIT}

{PLOTFIT plots the results of the BCS analysis program SPCFIT. In particular, it
will read the FIT and SPC files created by SPCFIT and provide time history
plots of the electron temperature, emission measure, nonthermal broadening, and
upflow velocity deduced from fits of theoretical profiles to the BCS Ca~XIX
(ch~1) and Fe~XXV (ch~4) detectors. PLOTFIT commences with the following menu
of available items to plot:
\begin{verbatim}
    * enter:
         1 : for electron temperature
         2 : for primary component emission measure
         3 : for primary component nonthermal broadening
         4 : for secondary component blueshift velocity
         5 : for secondary/primary component emission measure ratio
    ---->
\end{verbatim}
After you have entered the item number, you will be prompted for the
name of the FIT file and it's directory location.  After reading the selected
FIT file, PLOTFIT will print the start and end times of the file. The default
action is to plot the time history of the selected parameter between these
start and end times. You can override these defaults by entering your own start
and end times explicitly in the format: hhmm:ss.msec, where `hhmm' is the hour
and minute (e.g. 0750), and `ss.msec' is the second and millisecond (e.g.
23.123). The latter two time elements are optional. You will then be asked to
enter the channel number. The default is channel 1. If you enter channel 4 and
you did not process channel 4 data with SPCFIT, then you will be later asked to
re-enter the appropriate channel number.

After the selected time plot appears, you will be asked whether you wish to
examine the fitted spectra that were computed by SPCFIT during the derivation
of the fitted parameters. Answering YES (or Y) will invoke a cursor that you
can use to mark the time on the plot  when you wish to view the fitted BCS
spectrum.  PLOTFIT will read the observed and fitted spectra nearest the time
of the cursor from the BRR and SPC files, respectively,  associated with the
FIT file. The BRR and SPC files must be in the same directory as the FIT file
for this feature to work. A plot of the observed BCS spectrum with the
corresponding fitted spectrum superposed will appear. The final two questions
asked by PLOTFIT are whether you desire a hardcopy of the screen plots that you
have generated, and whether you wish to repeat the program.

\noindent {\underbar{Calling sequence:}}

\noindent PLOTFIT,{\it T,Y,E,W,Z,ZF}


\noindent {\underbar{Outputs:}}

\noindent{\it T} -- the time array (in seconds since the start of day) for the
plotted lightcurve.

\noindent{\it Y} -- the selected fitted parameter.

\noindent{\it E} -- the uncertainty of the fitted parameter.

\noindent{\it W} -- the wavelength array for the spectrum selected using the cursor.

\noindent{\it Z} -- the observed spectrum array nearest the time of the cursor.

\noindent{\it ZF} -- the fitted spectrum array nearest the time of the cursor.


\noindent {\underbar{Keywords:}}

\noindent{\it LOG} -- requests that the fitted parameter to
be plotted on a $log_{10}$ scale.

\subsection{FCS Programs}

\subsubsection{PLOTFIS}

{\em PLOTFIS} plots images and/or spectra from FIS files  created by the program
MKFIS. The program will first  prompt you to enter I or S according to whether
you to wish to plot images or spectra. The default is  to plot images. You will
then be asked to enter the FIS file name and it's directory location. If you
selected images, then PLOTFIS will read all the image modes from the selected
FIS file and present a menu of the times of each mode. You can choose a time by
using a mouse in X-windows, or entering the number of the mode on a regular
terminal. You will then be presented with a menu of available channels for the
selected mode time. An image of the selected channel ion will then be plotted.
The image will be plotted as a contour on a non X-windows terminal. You will
then be asked whether you wish to make a hard copy plot, and  whether you wish
to repeat the program. It is not necessary to read the FIS file again if you
wish to plot a different image --- all the available images are saved in memory
after the initial read. However, you must read the file again if you wish to
plot spectra.

If you selected spectra, then (after entering the FIS file name)
you will be presented with the following menu:
\begin{itemize}
\item  Display Data Blocks
\item  Select ALL Data Blocks
\item  Enter a List of Blocks to Extract
\item  Enter a Range of Blocks to Extract
\item  Enter a Type of Block to Extract
\item  Quit
\end{itemize}
\noindent Because of memory limitations, it is recommended that you do not read
all the FIS spectral modes at one time.  Hence, you should begin by displaying
all the available modes and then selecting a particular mode to plot. After
making a mode selection, PLOTFIS will list the available channels and proceed
as described earlier for images. If you choose to repeat the program after
making a hard copy (and you wish to plot a different spectral mode) then you
must remember to read the FIS file again. Unlike images, all the available
spectra in the FIS file are not saved in memory -- only the most recent
spectrum is saved.

\noindent {\underbar{Calling sequence:}}

\noindent PLOTFIS,{\it X,Y,IMG}

\noindent {\underbar{Optional Outputs:}}

\noindent{\it X} -- the wavelength array of the selected FCS spectrum.

\noindent{\it Y} -- the count rate (per bin) array of the selected FCS spectrum.

\noindent{\it IMG} -- a structure (see RDFIS) whose elements contain count
rate and pointing data for the selected FCS image.

\noindent {\underbar{Keywords:}}

\noindent {\it BIN} -- a value by which to rebin the spectrum or image data.

\noindent {\it NOBKG} -- inhibit automatic background subtraction of FCS image.

\noindent {\it CONT} -- plot image as a contour.

\subsubsection{RDFIS}

{\em RDFIS} reads images FIS files and returns the data in a structure variable that
can be used by other programs. The program prompts first for the FIS file name
and it's directory location.  It will then read all the image modes from the
selected FIS file and present a menu of the times of each mode.  You can choose
a time by using a mouse in X-windows, or entering the number of the mode on a
regular terminal. You will then be presented with a menu of available channels
for the selected mode time.  The selected image will be returned in a structure
variable  IMG which contains the following tags:

\begin{itemize}

\item  IMG.DATA  = a 1-d float array containing the image count rates
\item  IMG.NX    = the number of columns in the image
\item  IMG.NY    = the number of rows in the image
\item  IMG.XP    = 1-d float array with the x-coordinates of the data pixels
\item  IMG.YP    = 1-d float array with the y-coordinates of the data pixels
\item  IMG.ID    = string name identifying the image
\item  IMG.TIME  = string time of data in UT format (e.g. 87/11/12, 0300:12)
\item  IMG.EXPS  = exposure time per pixel (secs)
\item  IMG.PITCH = spacecraft pitch (arcsecs)
\item  IMG.YAW   = spacecraft yaw (arcsecs)
\item  IMG.ROLL  = spacecraft roll (degrees clockwise from solar north)

\end{itemize}

The pitch and yaw variables  are the pointing  coordinates of the spacecraft
boresite. Positive pitch and yaw are measured in arcsecs  south and east,
respectively,  relative to the center of the solar disk.  The cartesian
coordinates of the FCS images are also in arcsecs  relative to the solar
origin, however their sense is reversed such that solar north  and west are
positive. To avoid structure size conflicts in IDL, it was necessary  to store
the 2-d images as a 1-d array. To reconstruct the original 2-d data  array, one
performs the following simple steps in IDL:
\begin{verbatim}
   NX=IMAGE.NX & NY=IMAGE.NY & DATA=FLTARR(NX,NY) & DATA(0)=IMG.DATA(0:NX*NY-1)
\end{verbatim}
A similar operation is performed to restore the arrays  containing
the coordinates of each data pixel. The procedure RUST will automatically
restore the FCS image and pointing data.

\noindent {\underbar{Calling sequence:}}

\noindent RDFIS,{\it IMG, CHNUM, BACK}

\noindent {\underbar{Outputs:}}

\noindent{\it IMG} -- a structure whose elements contain count
rate and pointing data for the selected FCS image.

\noindent {\underbar{Optional Outputs:}}

\noindent{\it CHNUM} -- the FCS channel number of the selected image.

\noindent{\it BACK} -- a structure whose elements contain background count
rate and pointing data for the selected FCS image.

\noindent {\underbar{Keywords:}}

\noindent {\it NOBKG} -- inhibit automatic background subtraction of FCS image.

\subsubsection{RUST}

{\em RUST} will read and unpack the data and pixel  coordinate arrays for an
image saved in a structure variable created by RDFIS.

\noindent {\underbar{Calling sequence:}}

\noindent RUST,{\it IMG, DATA, XP, YP, DX, DY, XC, YC}

\noindent {\underbar{Outputs}}

\noindent{\it IMG} -- input structure whose elements contain count
rate and pointing data for the selected FCS image.

\noindent {\it DATA} -- output 2-d data array.

\noindent {\it XP} -- output cartesian coordinates of image pixels in
x-direction (+ W).

\noindent {\it YP} -- output cartesian coordinates of image pixels in
y-direction (+ N).

\noindent {\underbar{Optional Outputs}}

\noindent {\it DX} -- average spacing between pixels in x-direction.

\noindent {\it DY} -- average spacing between pixels in y-direction.

\noindent {\it XC} -- x-coordinate of center of image.

\noindent {\it YC} -- y-coordinate of center of image.


\subsubsection{FLUXCAL}

FLUXCAL converts FCS count rates  to photon number flux
\hbox{(photons~cm$^{-2}$~s$^{-1}$)}. The procedure approximates the FCS line  profile
by a symmetric Voigt function and converts the input count rate (which
normally corresponds to the intensity at the peak of the line) to a total
intensity  by integrating the area beneath the Voigt profile.  To compute this
integral,  the program requires the total (thermal plus nonthermal) line
doppler width. You will be prompted for this width in velocity units of
km~s$^{-1}$, or you can enter it via the keyword DOPP (see below). The default
value is 100~km~s$^{-1}$.  A  typical value is somewhere between 50 and
100~km~s$^{-1}$ in active regions, and  above 200~km~s$^{-1}$ in flares. In
some cases, you may already have available  the total integrated line intensity
(e.g.\/ from actually fitting a line  profile). Under these circumstances, it is
not necessary to use the Voigt  profile approximation. To inform the program of
this case, simply use the TOTAL keyword  and the input rate will be treated as
a total  (wavelength-integrated) intensity -- the DOPP keyword will be ignored
in this case. Note that the FCS profile often sits on a ``pedestal''
fluorescence background that is dependent upon peak intensities in other FCS
detectors. The pedestal correction is {\it not} applied in this program. Except
in the case of a strong (high C-level) flare, this background is generally
negligible.

The following is a sample session with FLUXCAL:
\begin{quote}
\begin{verbatim}
XRPIDL> fluxcal
* enter FCS line counts per second: 100
* enter FCS channel no [1-7]: 1
* enter FCS XTAL address of observation [def=6563]: 6563
% FLUXCAL: count rate is assumed to be at line peak
* enter total line doppler width [def = 50 km/s]: 100
* line flux (ph cm-2 s-1) =   9.86e+04
\end{verbatim}
\end{quote}
In this example, the program FLUXCAL computes the FCS line flux
for an observed count rate of 100~counts~per~sec in channel~1, with
the crystal positioned at address 6563 (i.e. the home position).
A doppler velocity of 50~km~s~$^{-1}$ is assumed for the line width.

\noindent{\underbar{Calling Sequence:}}

FLUXCAL, {\it RATE, CHAN, ADD, FLUX, DOPP=DOPP, /TOTAL}

\noindent{\underbar{Inputs:}}

 {\it RATE} -- FCS counts per sec at line peak (i.e. line center).

 {\it CHAN} -- FCS channel (between 1 and 7) of the observed line.

 {\it ADD} -- FCS crystal address (e.g. 6563 for the home position).

\noindent{\underbar{Outputs:}}

{\it FLUX} -- total line flux (photons~cm$^{-2}$~s$^{-1}$).

\noindent{\underbar{Keywords:}}

{\it DOPP} --  total line doppler width (thermal plus nonthermal) in km~s$^{-1}$.

{\it TOTAL} -- informs FLUXCAL that RATE is a total count rate.

\subsubsection{FTEMAP}

FTEMAP computes isothermal temperature and  column emission maps from pairs
of FCS images. The program begins by  prompting for the FIS file which
contains the FCS images in different channels. It will read all the available
images from the FIS file and list their corresponding times, ion
identifications, and mode characteristics (e.g.\ number of pixels, pixel size,
exposure time, spacecraft pointing, etc). To compute a temperature map, you
must choose  images observed in two different ions (e.g. Mg~XI and O~VIII) and
as close as possible in time (preferably the same time). FTEMAP will use the
FLUXCAL procedure to compute the flux for each pixel in the two ion maps, and
derive a temperature (in MK) and column emission measure (cm$^{-5}$) by
interpolating the ratios of these fluxes onto the ratio of corresponding ion
emissivity functions obtained from the Mewe tabulations.

After selecting an ion pair, the following menu will appear:
\begin{quote}
\begin{verbatim}
-------------------------------------------------------------
* DEFAULT MODE IS TO RUN ASSUMING:
-------------------------------------------------------------
1: Meyer coronal abundances
2: Nonthermal line doppler width = 50 km/s
3: Range of temperatures searched = 2 - 10 MK
4: Pixels with count rates below half sigma rejected
5: Adjacent pixels not averaged
6: Calculation results saved in XDR SAVE file: FTEyymmdd.hhmm
--------------------------------------------------------------
* do you wish to modify any of these assumptions [def=n]?
\end{verbatim}
\end{quote}
These assumptions have the following explanations:
\begin{itemize}
\item the Meyer coronal abundances will be used to scale the
emissivity-temperature functions for the selected ion pair. You can enter your
own abundances (relative to atomic hydrogen) via this option;

\item a nonthermal doppler velocity width of 50~km~s$^{-1}$ will be
assumed to reconstruct the line profile when FLUXCAL computes the total line
flux. You can enter a different velocity with this option. Note that FTEMAP
will add (in quadrature) the nonthermal width to the thermal  width (from the
derived temperature) to produce a total line width;

\item in some cases, the temperature solution obtained from line ratios
can be multivalued. You can ``window'' single-temperature solutions by varying
the temperature range with this option;

\item in some cases, the ratio of two noisy low signal-to-noise pixels can
result in a seemingly valid temperature value. To correct for this effect,
FTEMAP will compute the standard deviation of the count rates in the two
selected maps and reject all data values that fall below half a $\sigma$. You
can change this cutoff threshold via this option;

\item by default, FTEMAP will not smooth or average ion maps. With this
option, you can choose to rebin the FCS count rates in order to improve
signal-to-noise at the expense of degraded spatial resolution;

\item the temperature and column emission measure maps (together with the
corresponding flux maps for the two ions used  to derive them) will be saved as
structures in a portable XDR IDL save file: FTEyymmdd.hhmm, where the file name
is inherited from the file id of the original input FIS file. You can choose a
different output file name with this option.
\end{itemize}

After FTEMAP has completed its calculations, it will print the following menu:
\begin{quote}
\begin{verbatim}
ENTER:

  1) to QUIT FTEMAP
  2) to REPEAT calculation
  3) to PLOT ION1 flux map
  4) to PLOT ION2 flux map
  5) to PLOT TEMPERATURE map
  6) to PLOT COLUMN EMISSION MEASURE map
  7) to SAVE results in FTExxxxxx.xxxx
  8) to HARD COPY latest map
\end{verbatim}
\end{quote}
The required responses to these choices are obvious. Note that, if no valid
temperatures are found for any of the pixels in the selected ion pair, then
options 5--8 will not appear.

\noindent{\underbar{Calling Sequence:}}

FTEMAP, TEMP, EMISS, FLUX1, FLUX2

\noindent{\underbar{Optional Outputs:}}

{\it TEMP} -- temperature (10$^6$~K) array.

{\it EMISS} -- column emission measure (cm$^{-3}$) array. The column emission
measure is computed from the volume emission measure (deduced from the total
line flux) divided by the FCS pixel area size.

{\it FLUX1} -- flux (photons~cm$^{-2}$~s$^{-1}$) array for the first selected
ion.

{\it FLUX2} -- flux (photons~cm$^{-2}$~s$^{-1}$) array for the second selected
ion.

\subsubsection{PLOTFTE}

PLOTFTE plots maps of temperature, emission measure, and fluxes saved in FTE
saveset files created by FTEMAP. The program commences by asking for the name
and directory location of the FTE file. It restores the contents of the file
and presents the following (sample) menu:
\begin{quote}
\begin{verbatim}
* following images are available in:
FTE850123.0657
-----------------------------------------------------
 1) TEMP (MK)
 2) EM (10^28 CM-5)
 3) Mg XI (PH CM-2 S-1)
 4) O VIII (PH CM-2 S-1)
-----------------------------------------------------
* enter image number to plot [def = 1]:

* hardcopy [def=n]?
* another plot [def=n]?
\end{verbatim}
\end{quote}
In this example, you have the choice of plotting maps of
temperature, emission measure,  and Mg~XI and O~VIII fluxes that are saved in
the file FTE850123.0657. After plotting the selected map, PLOTFTE
will give you the options of making a hardcopy and returning to the
above menu for a new plot.

\noindent{\underbar{Calling Sequence:}}

PLOTFTE, TEMP, EMISS, FLUX1, FLUX2, COLOR=COLOR, /CONT

\noindent{\underbar{Optional Outputs:}}

{\it TEMP} -- temperature (10$^6$~K) array.

{\it EMISS} -- column emission measure (cm$^{-3}$) array. The column emission
measure is computed from the volume emission measure (deduced from the total
line flux) divided by the FCS pixel area size.

{\it FLUX1} -- flux (photons~cm$^{-2}$~s$^{-1}$) array for the first selected
ion.

{\it FLUX2} -- flux (photons~cm$^{-2}$~s$^{-1}$) array for the second selected
ion.

\noindent {\underbar{Keywords:}}

{\it COLOR} -- color table value (e.g. 0 for B/W [def], 3 for red-temperature).

{\it CONT} -- plot maps as contours (user is prompted for contour level values).

\subsubsection{TEMCAL}

TEMCAL calculates temperature (isothermal) and emission measure from
intensity ratios of two FCS lines. The following sample session
illustrates the usage of this program.
\begin{quote}
\begin{verbatim}
XRPIDL> temcal

* enter two FCS detector nos [def =  1, 3]
---> 1, 3

* enter corresponding line IDs from Mewe list [H for help; def = 752, 306]
---> 752, 306

* ion = O VIII
* central wavelength =  18.97
* O VIII MEYER abundance used = 2.47e-04
* ion = MG XI
* central wavelength =   9.17
* MG XI MEYER abundance used = 3.73e-05

* change ion abundances [def=n]? n
* enter corresponding observed counts/sec: 100,200
* are these integrated line rates [def = n]? n
* enter non-thermal Doppler velocity in km/s [def = 50]: 50
* enter range for temperature solution [def = 2, 10]:

% TEMCAL: -- temperature iterations in progress ---

-----------------------------------------
          Te       Log10(EM)
       (10^6 K)     (cm-3)
-----------------------------------------
         4.9        47.4
-----------------------------------------

* do you wish to plot G(T) ratios [def=y]? y
* enter temperature range (in MK) for plot [def=2,20]: 2,20
* hardcopy [def=n]?
\end{verbatim}
\end{quote}
In the above example, TEMCAL computes temperature and emission measure using an
O~VIII count rate of 100 in FCS channel 1, and a Mg~XI count rate of 200 in FCS
channel 3. The numbers 752 and 306 correspond to the identifications of the
O~VIII and Mg~XI emissivity functions in the Mewe line tabulations. By default,
TEMCAL will normalize these functions to the Meyer coronal abundances. You  can
override this default by answering YES (or Y) to the change ion abundances
question. By default, TEMCAL assumes that  the input count rates are peak line
center values and will use FLUXCAL to compute an integrated line intensity. If
the input rates are integrated values, then you can skip the integration step
by answering YES (or Y) to the next question. If the rates are peak values,
then  TEMCAL needs the nonthermal Doppler width (km~s$^{-1}$)  in order to
reconstruct the line profiles.  Lastly, TEMCAL requires a temperature range
within which to  confine its calculation. Such a range is necessary in case of
multiple solutions.

TEMCAL iterates the line flux ratios until the emission measures deduced from
the individual line fluxes are identical. If a converged solution is not
achieved, you will be given the option of varying the temperature range for the
calculation. After printing the converged solution, TEMCAL provides the option
of plotting the ratio of the ion emissivity functions to the terminal and a
hard copy device.

\noindent{\underbar{Calling Sequence:}}

TEMCAL, TEMP, EMISS

\noindent{\underbar{Optional Outputs:}}

{\it TEMP} -- temperature (10$^6$~K).

{\it EMISS} -- volume emission measure (cm$^{-5}$).

\subsubsection{EMRAT}

{\em To run this program type:}
\begin{verbatim}
    XRPIDL> EMRAT
\end{verbatim}
{\em EMRAT} plots emissivity ratios. It reads the emissivity functions for a
specified line pair from the Mewe et al catalog. The temperature range common to
the two lines is then plotted.
When the program is first run it asks to user if they wish to plot the ratio as
the abscissa the default is no as this would not be required in most cases.
The user is then asked to enter the MEWE catalog numbers of the two lines
they are interested in. (the default lines are 1st 326 and 2nd 756)
After this two plots are drawn one contains a plot of Emissivity ratio against
temperature for the specified lines the other is a plot of the emissivities of each
each line against temperature.
The user then has the option of selecting a point on the plot in order to
determine its value. If this is not required a further menu will appear as
follows:
\begin{itemize}
\item  To stop
\item  Change the temperature range plotted
\item  Change the temperature range plotted allow full range available
\item  Select different lines
\item  Plot emissivity of a single line
\item  Change flux values
\item  Plot emission function
\item  Toggle the plotting abscissa
\item  Toggle whether cursor input is asked for after plotting
\item  Hard copy to laser (PostScript)
\item  Start new logging file
\item  Replot the screen (return to default)
\item  Plot emissivity ratio
\item  Plot emissivities
\end{itemize}
and the various options can be selected as required.

\newpage

\section{X-Window/Widget Programs}

\subsection{IDL Level}

This chapter describes the IDL/widget programs that
enable reading and displaying of BCS and FCS data. These programs
provide the user with an \hbox{X-Windows/Motif} graphical user interface (GUI) to
many of the programs described previously.

\subsubsection{BCS}

The IDL/widget interface to BCS lightcurves and spectra is
invoked by typing:
\begin{verbatim}
    XRPIDL> BCS
\end{verbatim}
\noindent The program accepts BDA files as input and assumes that the logical
BCS\_DATA points to the directory  location of the BDA files. If this logical
is not defined, then you will be asked initially to  enter the directory
location of the BDA files that you wish to plot. Figure 4-1
shows
the main widget display produced by BCS. A listing of the BDA files in the
current  working directory appears under the listing:
\newline
\newline
\centerline{\bf SELECT FROM THE FOLLOWING FILES}
\newline
\newline
\noindent You can select from this list by using the
mouse (and hitting the leftmost button). By default, BCS will read channel~1
spectra, integrating  with 10~s time resolution. These values  can be changed
at any time by clicking on the desired channels in the
\hbox{\bf CHANNEL BUTTONS}
menu, and sliding the \hbox{\bf ACCUMULATION TIME} slide bar to the desired
integration time value (where 0 corresponds to the minimum available time
resolution).  Because of memory limitations, it is not recommended that you
read more than two channels at a time.

The main widget display contains additional buttons which have the following
functions:
\begin{itemize}

\item{\it Spawn To OS} -- will temporarily suspend the main IDL widget
program and spawn to the current operating system (VMS or UNIX).

\item{\it Change Directory} -- will permit the user to switch
to another working directory, and also to select files from the BCS
Selected Data Archive (\S~1.5) by choosing a date.

\item{\it Help} -- will print some very brief help information.

\item{\it Replot Latest File} -- will replot the lightcurve from
the latest file that has been read into memory.

\item{\it Quit} -- quit and exit BCS.

\end{itemize}
Upon reading the selected BDA file, BCS will plot
lightcurves for each of the selected channels. Recall that the
lightcurves are derived by summing individual spectra over address bin.
The lightcurve plot will appear in a separate widget window.

Figure 4-2 shows a sample BCS lightcurve window with the following application
buttons:
\begin{itemize}

\item{\it Log Scale} -- if set, then plot the lightcurve using a
log$_{10}$ scale. \

\item{\it Oplot Errors} -- if set, then plot the lightcurve with the square root
uncertainties.

\item{\it Hard Copy} -- make a hardcopy of the current plot.

\item{\it Integrate Spectra} -- integrate BCS
spectra during the time interval defined by the two
most recent cursor positions on the lightcurve plot.

\item{\it Write Data} -- write the BCS count rate and universal time arrays
to a data file (ASCII or IDL/XDR saveset).

\item{\it Zoom Lightcurve} -- zoom in on lightcurve
during the time interval defined by the two
most recent cursor positions on the lightcurve plot.

\item{\it Done} -- exit the lightcurve application and return to main calling widget.

\end{itemize}
After the lightcurve is plotted, the message:
\newline
\newline
\centerline{\bf USE CURSOR TO SELECT TIME OF SPECTRUM}
\newline
\newline
will flash. At this point, you can view a BCS spectrum for a
particular time by clicking the mouse button at the corresponding time position
on the lightcurve. The selected spectrum will be plotted
in a new widget window.
Figure 4-3 shows a sample BCS spectrum window with
the following application buttons:
\begin{itemize}

\item{\it Double Bin} -- if set, then spectrum will be plotted
with double binning
(i.e.\ count rates in two adjacent address bins averaged into a single bin).

\item{\it Wavelength Axis} -- if set, then the spectrum
will be plotted using a nominal wavelength (\AA) axis
(i.e., uncorrected for position of the source relative to the dispersion axis).

\item{\it Fit Options} -- opens a sub-menu of functions (Gaussian,
Voigt) to fit to a zoomed portion of the spectrum. (This facility
is still under development at time of writing).

\item{\it Write Data} -- write the BCS count rate and address/wavelength
arrays
to a data file (ASCII or IDL/XDR saveset).

\item{\it Zoom Spectrum} -- zoom in on a portion of spectrum
between address/wavelength values defined by the two most recent cursor
positions on the spectrum plot. Zooming must be performed prior to fitting.

\item{\it Done} -- exit the spectrum application and return to main calling widget.
\end{itemize}

\subsubsection{FCS}

The IDL/widget interface to FCS images and spectra is a direct analog
of the BCS widget program.
It is invoked by typing:
\begin{verbatim}
    XRPIDL> FCS
\end{verbatim}
The program accepts FIS files as input and assumes that the logical
FCS\_DATA points to the directory  location of the FIS files. If this logical
is not defined, then you will be asked initially to  enter the directory
location of the FIS files that you wish to plot. Figure 4-4
shows the main widget display produced by FCS.  The main widget
contains application buttons (SPAWN, CHANGE, HELP, QUIT) that have the same
functions as in the BCS widget program. A listing of the FIS files in the
current working directory appears under the heading:
\newline
\newline
\centerline{\bf SELECT FROM THE FOLLOWING FILES}
\newline
\newline
You can select from this list by using the mouse (and hitting the leftmost button).

The FCS instrument operated in two modes: imaging and spectral. You must first
select between these two modes by clicking on the corresponding buttons that
are adjacent to the FIS file selection window. The program will read the header
of the selected FIS file and list the times of the selected mode, with imaging
and spectral modes identified by the labels RAST and SS, respectively,  under
the heading:
\newline
\newline
\centerline{\bf SELECT FROM THESE FCS TIMES} \noindent
\newline
\newline
(The labels: IS, MSS, and MSSR denote spectral modes  corresponding to integral
scans, multiple spectroscopic scans, and multiple spectroscopic scans in a
raster, respectively). After you select a mode time, the program  will list
the channel information for the FCS detectors that were switched on during the
time of the mode. This information will appear under the \hbox{\bf CHANNEL
BUTTONS} menu in the form of channel number and corresponding ion transition.
Detectors that were off during the selected time are not selectable. Clicking
on a valid channel will produce an image or spectrum window plot depending on
the selected mode type.

Figure 4-5 shows a sample FCS image window. The image is plotted as a contour
with
a solar heliographic grid. The application buttons: HARD COPY, WRITE DATA, and
DONE
have the same functions as in the BCS programs. The remaining buttons have the
following definitions:

\begin{itemize}

\item{\it Double Bin Image } -- if set, then image will be plotted
with double binning
(i.e. count rates in two adjacent pixels averaged into a single bin).

\item{\it Contour Image} -- if set, then plot the image as a contour.
Otherwise, plot as color image.

\item{\it Gang Images} -- if set, then divide the plot window into
four quadrants and plot successive images in separate quadrants.

\item{\it Heliographic Grid} -- if set, then overplot a solar
heliographic grid coordinate system.

\item{\it Log Scale Image} -- if set, then plot the image
using log$_{10}$ scaling.

\item{\it Overplot Images} -- if set, then successive images will
be overplotted as contours on the first image.

\item{\it Subtract Background} -- if set, then subtract detector background
from image.

\item{\it Tools} -- opens a submenu of tools to customize the color table;
check and correct the spacecraft pointing. (These tools are still under
development at the time of writing).
\end{itemize}

Figure 4-6 shows a sample FCS spectrum window.
The application
buttons in the spectrum window have the same functions as those described
previously, with the following additional button:

\begin{itemize}
\item{\it Flux Axis} -- plot spectrum in units of
photons~cm$^{-2}$~s$^{-1}$~\AA$^{-1}$ at the detector.
\end{itemize}

\subsubsection{LISTBCS}

LISTBCS provides a simple widget interface to the BCS catalog of flares in
which the total count rate in the Ca~XIX channel exceeded 18 counts~per~sec.
The catalog is approximately 90\% complete at the time of writing.
The program is invoked by:
\begin{verbatim}
    XRPIDL> LISTBCS
\end{verbatim}
Figure 4-7 shows a sample display produced by LISTBCS. The events are
listed by start and stop time (as determined by Ca~XIX count rates), as well as
peak Ca~XIX and Fe~XXV count rates (and corresponding times).
The application buttons in this display have the following definitions:
\begin{itemize}
\item{\it Select Date} -- permits the user to
choose the year, month, and day for which to search the BCS catalog.
\item{\it Print Listing} -- print the listing on a hardcopy device.
\item{\it Quit} -- exit the program.
\end{itemize}

\subsubsection{SCANPATH}

SCANPATH is a useful facility for reading and extracting procedures in XRP
libraries and directories (as well as other procedures) that are present in the
IDL path. The program works on VMS and UNIX systems and is invoked by typing:
\begin{verbatim}
    XRPIDL> SCANPATH
\end{verbatim}
Figure 4-8 shows a sample display produced by SCANPATH. A listing of
available libraries and directories appears under the heading:
\newline
\newline
\centerline{\bf SELECT FROM THE FOLLOWING DIRECTORIES/LIBRARIES}
\newline
\newline
SCANPATH will first list the names of the procedures contained in
the library/directory that you select from this list. SCANPATH will then list
the text of the procedure that you select from the list of procedure names.

The application buttons in SCANPATH have the following functions:
\begin{itemize}
\item{\it Quit} -- exit the program.

\item{\it Print} -- prints the listed procedure on a hardcopy device.
\smallskip

\item{\it Extract} -- extract the listed procedure and save it
in the working directory with a .TXT extension.

\item{\it All} -- lists the entire procedure.

\item{\it Doc Only} -- lists only the documentation header found between
the ;+ and ;- delimiters in the procedure.
\end{itemize}
\end{document}
