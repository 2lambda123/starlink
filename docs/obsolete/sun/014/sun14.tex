\documentstyle{article} 
\pagestyle{myheadings}

%------------------------------------------------------------------------------
\newcommand{\stardoccategory}  {Starlink User Note}
\newcommand{\stardocinitials}  {SUN}
\newcommand{\stardocnumber}    {14.1}
\newcommand{\stardocauthors}   {H Walker and D Giaretta}
\newcommand{\stardocdate}      {18 July 1989}
\newcommand{\stardoctitle}     {IRASLRS --- Obtaining spectra from the IRAS LRS database}
%------------------------------------------------------------------------------

\newcommand{\stardocname}{\stardocinitials /\stardocnumber}
\markright{\stardocname}
\setlength{\textwidth}{160mm}
\setlength{\textheight}{240mm}
\setlength{\topmargin}{-5mm}
\setlength{\oddsidemargin}{0mm}
\setlength{\evensidemargin}{0mm}
\setlength{\parindent}{0mm}
\setlength{\parskip}{\medskipamount}
\setlength{\unitlength}{1mm}

\begin{document}
\thispagestyle{empty}
SCIENCE \& ENGINEERING RESEARCH COUNCIL \hfill \stardocname\\
RUTHERFORD APPLETON LABORATORY\\
{\large\bf Starlink Project\\}
{\large\bf \stardoccategory\ \stardocnumber}
\begin{flushright}
\stardocauthors\\
\stardocdate
\end{flushright}
\vspace{-4mm}
\rule{\textwidth}{0.5mm}
\vspace{5mm}
\begin{center}
{\Large\bf \stardoctitle}
\end{center}
\vspace{5mm}

\section{Introduction}
The IRAS LRS (Low Resolution Spectrograph) database contains about 
170~000 spectra for about 40~000 sources. 
The LRS Atlas contains the ``best'' averaged spectra for 5425 sources, so 
there are a lot of new spectra; some good, some rubbish.
The LRS operates from around 7$\mu$m to 24$\mu$m, using a blue and red
channel, with some overlap around 12$\mu$m to 13$\mu$m. 
There are 100 sample points in each channel.
The LRS is a slitless spectrograph, so extended sources and multiple sources
confuse the blue and the red channels.
During the IRAS mission, whenever a sufficiently bright point source 
triggered the algorithm, the spectrum was extracted from the LRS datastream
and stored in the LRS dataset. 
This means that, although the source may not be in the Point Source Catalog
(but in the Rejects file) the spectra are still in the dataset. 

The program to average the spectra, and output them, started life as a batch
program on an old IBM main-frame. 
Parts of this program are the original pieces used to create the LRS Atlas,
and by setting flags appropriately you could recreate the Atlas. 
The original program was written by E. Raimond, F. Olnon, K. Lugtenberg. 
The program was modified to run ``interactively'' on a Vax (at Groningen)
by P. te Lintel, and he added a radio astronomy plotting package, POPS, to it.
The present copy was sent to H. Walker (at NASA Ames) by P. Wesselius, and
now, having been loaded onto STADAT, the Ames command procedures have been 
modified by D. Giaretta.
This warns you that the package is fragile, and temperamental.
There is a complete user manual available, in the [iraslrs] directory.

\section{The IRASLRS captive account}

Users should log into the IRASLRS account on STADAT, through their Starlink 
node.
\begin{center}
ONLY ONE USER MAY BE LOGGED IN AT A TIME 
\end{center}
the login command procedure checks this and may log you out 
immediately if someone is already using the account.

There are a limited number of things you can do in the captive account, you
should move the spectra to your own account for detailed analysis.
\begin{verbatim}
    Options:
        To run the LRS database access program.........................LRS
        To reformat ASCII output to MONGO/DIPSO compatible files....CHANGE
        To rename a file created by LRS.............................RENAME
        To run MONGO to quick-look at the spectra....................MONGO
        To transfer a file to your own directory (using CBS).........TRANS
        To delete a file (fname.ftype;version)......................DELETE
        For help and information......................................HELP
        To use the mail facility......................................MAIL
        For a directory listing........................................DIR
        To type a file (on the terminal)..............................TYPE
        To print a file (on the lineprinter).........................PRINT
        To logoff......................................................LOG
\end{verbatim}

\section{Getting spectra}

Type:
\begin{verbatim}
    $ LRS       --- initiates the access program
    > ZED       --- initiates the editor
\end{verbatim}
There are 9 main pages of menu items.
The program is set up for maximum access and retrieval of spectra, so let
most of the flags stay at their present setting.
To move through the menu, either hit $<$ RETURN $>$ to go to the next page,
or type in the page number.
Typing ``EXIT'' leaves the editor, and when you wish, the program.
To move to an appropriate parameter type {\bf F PARAMETER}, then the parameter
can be set, if needed.
To get files out, to be plotted or analysed by other programs, type
\begin{verbatim}
    > F ASCII
    > ASCII YES
    > BINAIR NO
\end{verbatim}

\section{Selecting positions}
The (present) recommended way to extract spectra is by RA and DEC.
Up to 18 positions can be interrogated at one search.
The coordinate system can be changed using CRDSYS, to say GA for galactic
(but never tried it).
To input positions, for example, type
\begin{verbatim}
    > F CPOS
    > CPOS ON
\end{verbatim}
You are moved to a new menu page requiring up to 18 search circle definitions. 

\begin{verbatim}
    > RAD1 2.0     --- defines the search circle for the 1st source as 2.0 arcmin
       < RETURN >  --- moves to the next menu page, which contains positions
    > RA1 18 35 12 --- either input like this or with `:' between pieces
    > DEC1 38:44:00 
    > EXIT         --- to leave this page of the menu
    > EXIT         --- to leave the ZED editor
    > LRS          --- to start the retrieval section
    > EXIT         --- to leave LRS
\end{verbatim}

\section{Output data}
You will be informed of the program's success in finding spectra around the
specified position, by the logging of additional spectra to the SCANS set 
in DISK.
The position of each scan is given, in degrees, for galactic and equatorial
coordinates.
The first spectrum listed in each set is the averaged spectrum, followed by
the individual spectra making up the average.
You may have several sources extracted for one given position.
Normally, you will find 2 -- 3 spectra for each source, from the 2 -- 3
HCONs of the mission.
For example, with IRAS 18352+3844 (Vega), 2 average spectra are extracted,
one with 3 spectra and one with 4 spectra. 
The spectra are stored sequentially in the program's DISK file (for later
use with POPS), and so there are now 9 spectra in DISK.
Another run of the program will add more scans to the DISK file, so regular
tidying up is required.
When the scans are output to your directory, using the ASCII option, a name
is created for each spectrum scan from the coordinates, so that it looks like
an IRAS name, e.g.\ for Vega we get A01\_18352P3844.SPE, A02\_18352P3844.SPE,
A03\_18351P3845.SPE, etc.
If the names of 2 different spectra are the same (because the coordinates are
identical), which is the case for Vega, two version numbers are created, so 
look carefully, and rename before purging, e.g.\ A01\_18352P3844.SPE;2 and 
A01\_18352P3844.SPE;1 are two different averaged spectra.  
The same type of processor is used to combine the spectra as was used to
combine sightings for the Point Source Catalog.

There are 100 samples from the blue channel and 100 samples from the red 
channel output. 
Of these, lines 30 -- 74 and 127 -- 175 in your file contain useful data.
Beyond this, to red and blue in each channel, are data which should be used for
noise and baseline calculations. 
The first column in the file gives the wavelength in microns, the second
column in the flux in watts.m$^{-2}$.$\mu$m$^{-1}$, i.e. not what is 
displayed in the LRS Atlas.

\section{Post-processing}

There is information about the spectra extracted, in SCAN.LOG.
This can be very complicated to appreciate without knowing the IRAS satellite
intimately, but some flags may be worth checking if you are uncertain about
the quality of your spectra. 
The LRS had 5 detectors, 3 for the blue channel and 2 for the red channel.
Part of the SCAN.LOG output for Vega is shown below (the far right side of
the printed output).
\begin{verbatim}
            D SSRR XXDG CCBB AABS GGJJ DIR     refer to individual spectra
            G 1212 12GC 1212 12BV 1212

            SSRRJJ  ACC DET1 DET3 DET4 FIRST   refer to average spectrum
            121212  REJ DET2      DET5 DIR  #

            * -++- ---+ +--- ---- ---k         individual spectra for 1st source
            * -++- ---+ +--- ---- ---l
            * -++- ---+ +--- ---- ---m

            --**--    3    1    1    2 126443  average spectrum for 1st source
            0.99      0    1         1

            * -++- +--+ +--- ---- ---n         individual spectra for 2nd source
            * -++- +--+ +--- ---- ---q
            * ---- +--+ +**- ---- -++t
            * -++- +--+ +--- ---- --+w

            --**--    4    0    4    0 126446  average spectrum for 2nd source
            1.60      0    0         4

    - means not set, + means set but do not reject, * means reject

    SS  skip flags for blue and red channels. These flags are set
    12  when any of the rest of the flags are set leading to the 
        rejection of that piece of spectrum for the average.
    RR  survey/LRS flux density ratio out of bounds
    12  

    DG  deglitch count too high (per second in 5 second period)
    GC  in high source density region (i.e. near galactic plane)
 
   ACC  in the average, how many spectra were accepted,
   REJ  and how many rejected
\end{verbatim}
There is a post-processor to reformat the data for use in DIPSO or MONGO.
The first 10 wavelengths for each part are 0.0 in the .SPE files.
The post-processor alters these to the first non-zero wavelength in that 
section.
This allows plotting to be done with defaulted axes, but also maintains the
information about baselines contained in those first few channels.

\subsection{DIPSO}
There is a post-processor to alter the output files so that they can be 
read by DISPO. Type:
\begin{verbatim}
    $ CHANGE
\end{verbatim}
to initiate this, and answer the prompts. 
All the spectra you have retrieved can be changed by using {\tt A*.SPE} as 
answer to the file name query.
The output files have extension ``DIPSO''.

\subsection{MONGO}
There is a post processor to alter the output files so that they can be 
read by MONGO. Type:
\begin{verbatim}
    $ CHANGE
\end{verbatim}
to initiate this, and answer the prompts. 
All the spectra you have retrieved can be changed by using {\tt A*.SPE} as 
answer to the file name query.
The output files have extension ``MONGO''.

An example of the MONGO commands required to plot these spectra is as follows.

\begin{tabular}{ll}
MONGO72$>$ TERMINAL GRAPHPACK      & for a Graphon - list available\\
MONGO72$>$ LOCATION 30 215 15.5 163& defines the plotting area so you can \\
                                   & read the y-axis scale\\
MONGO72$>$ DATA A01\_18352P3844.MONGO& load in the spectrum to be plotted\\
MONGO72$>$ XCOLUMN 1               &  \\
MONGO72$>$ YCOLUMN 2               &  \\
MONGO72$>$ LIMITS                  & this is to find the yscale of the plot \\
MONGO72$>$ BOX                     & \\
MONGO72$>$ LIMITS 7 24 --0.4E--12 3.4E--12& use the given yscale, and add the\\
                                   & known useful xscale\\
MONGO72$>$ ERASE                   & clear the screen for the new plot\\
MONGO72$>$ BOX                     &  \\
MONGO72$>$ LINES 30 74             & select the good blue channels \\
MONGO72$>$ XCOLUMN 1               & and plot \\
MONGO72$>$ YCOLUMN 2               &  \\
MONGO72$>$ CONNECT                 &  \\
MONGO72$>$ LINES 127 175           & select the good red channels and plot\\
MONGO72$>$ XCOLUMN 1               &  \\
MONGO72$>$ YCOLUMN 2               &  \\
MONGO72$>$ CONNECT                 &  \\
MONGO72$>$ XLABEL $\backslash$gl (in microns)& label the axes\\
MONGO72$>$ YLABEL w.m$\backslash$$\backslash$u--2$\backslash$$\backslash$
  d.$\backslash$gmm$\backslash$$\backslash$u--1$\backslash$$\backslash$d& \\
MONGO72$>$ LIST                    & prepare the commands for hardcopy\\ 
MONGO72$>$ DELETE 3 6              & remove the investigation for yscale\\
MONGO72$>$ TERMINAL CANON          & select appropriate hardcopy device\\
MONGO72$>$ PLAYBACK                & play the commands above to it\\
MONGO72$>$ HARDCOPY                & get hardcopy plot of spectrum\\
MONGO72$>$ END                     & leave Mongo\\
\end{tabular}

These commands produced the first plot at the end of this document.

\subsection{Renaming output files}
RENAME allows one to rename the .SPE or .MONGO or .DIPSO files to
something else.

\section{Transferring data}

The command
\begin{verbatim}
    $ TRANS
\end{verbatim}
allows one to TRANSFER data files back to ones own node for further work.
\end{document}
