\documentstyle {article} 
\markright{SUN/74.3}
\setlength{\textwidth}{153mm}
\setlength{\textheight}{220mm}
\setlength{\oddsidemargin}{3mm}
\setlength{\evensidemargin}{3mm}
\pagestyle{myheadings}

\begin{document}
\thispagestyle{plain}
\noindent
SCIENCE \& ENGINEERING RESEARCH COUNCIL \hfill SUN/74.3\\
RUTHERFORD APPLETON LABORATORY\\
{\large\bf Starlink Project\\}
{\large\bf Starlink User Note 74.3}
\begin{flushright}
W F Lupton\\
27th April 1987
\end{flushright}
\vspace{-4mm}
\rule{\textwidth}{0.5mm}
\vspace{9mm}
\begin{center}
{\Large\bf DSCL --- Starlink Command Language}
\end{center}
\vspace{9mm}
\tableofcontents
\markright{SUN/74.3}
\newpage
This paper contains the internal documentation on DSCL stored in file
ASPDIR:FULLDSCL.LIS.
It is issued as a SUN so that it is visible in the Starlink documentation
indexing system.
The layout has been revised by M D Lawden but the text mostly derives from 1981
and so may no longer be fully up-to-date.
\section {INTRODUCTION}
At the moment (1981), we have the Starlink Subroutine  Interface  (SUN/4) but no
command language.
This means each application program must run in isolation from all others and
the unfortunate user is saddled with the endless specification of parameters and
answering of prompts.
Clearly, some sort of interim command language is needed which will allow one
program to communicate its results to another one.
DSCL is a response to this need.

Prime considerations in the design of this system have been both that existing
Starlink programs should not need any modification at all to run within the
system and that DSCL should not need any re-configuration in order to allow new
applications to be `plugged-in'.

The first version of DSCL was implemented in DCL.
However, it ran too slowly to be useful in real circumstances.
In this version, as much as possible has been converted to FORTRAN and several
new features have been implemented.

Although DSCL can be used by anyone for specific applications, it has been
produced primarily to aid in the creation of ASPIC, an interim set of image
processing applications (ASPIC stands for AStronomical PICtures).
A basic introduction to ASPIC and DSCL is given in SUN/23 which should be
read in conjunction with this paper.
\section {SPECIFICATIONS}
In section 3 of SUN/23 on DSCL no rules were given, merely examples of usage.
This section should be used as a reference to determine command formats.
\subsection {Command Line}
The syntax of the command line is (square brackets indicate optional fields):
\begin{verbatim}
        [<commchar>][<commname>][ <parval1>[ <parval2>[..]]]
\end{verbatim}
where:
\begin{description}
\begin{description}
\item [$<$commchar$>$] is an optional `*', `@' or `\$' which indicates that the
command is an applications program, DSCL procedure, or DSCL or DCL command
respectively.
It is not normally needed.
\item [$<$commname$>$] is the program, procedure, DSCL or DCL command name.
If omitted (this is tested for by checking whether $<$parval1$>$ contains an equals
sign) then LET is assumed: see section 2.3.
\item [$<$parvaln$>$] is of the form [$<$param$>$][=][$<$value$>$].
For the interpretation of this field see the separate section on command line
parameters (2.2.1).
\end{description}
\end{description}
The spaces separating parameters may in fact be multiple spaces or tabs, just as
in DCL.
The same is true in DSCL procedures.

Command qualifiers can appear anywhere after the command name, preceded by a
slash (/).
At present, the only places where a qualifier is used are on compiling and
executing DSCL procedures (sections 2.4.2 and 2.4.3).
\subsubsection {Searching procedures}
Whether the command is interpreted as a request to run an applications program,
execute a procedure or perform a DCL or DSCL command depends on whether
$<$commname$>$ is the name of a program, a procedure or not.

Unless a $<$commchar$>$ is given, it is first assumed that it is a program, so a
search is made first for $<$commname$>$.EXE, then, failing that, for
DSCL\_EXEDIR:$<$commname$>$.EXE.
(DSCL\_EXEDIR is a logical name which points to the directory where generally
available applications programs live.)
If both searches fail, an attempt is made to translate $<$commname$>$ as a logical
name.
If this is successful, $<$commname$>$ is replaced with the name of the file in the
translation.
Thus, if logical name G2 has translation DISK\$USER2:[WFL.ENV]GEN2D.EXE, a
command of G2 is very nearly (see below) the same as a command of GEN2D issued
from DISK\$USER2:[WFL.ENV].

If the search for an applications program fails a procedure is searched for,
firstly a compiled version, then a source version.
The search order is the same as for applications programs, except that the
`system' directories are pointed to by logical names DSCL\_SCCDIR and
DSCL\_SCLDIR.

If a procedure is not found, it is assumed that the command is a DCL or DSCL
command and the command line (minus the $<$commchar$>$ if it was given) is passed to
DCL.

As can be seen from the above, there is a certain overhead in executing DCL and
DSCL commands.
If commonly used sets of such commands are put into DSCL procedures this
overhead is overcome since the above searches are carried out at compile time.

If logical name G2 had translation DISK\$USER2:[WFL.ENV]GEN2D.SCC, typing G2
would be taken as a request to execute the DSCL procedure whose compiled version
is held in DISK\$USER2:\-[WFL.\-ENV]\-GEN2D.\-SCC.
{\em No actual check is made to ensure that this file exists}, it is your
responsibility to set up sensible logical names.
(See SUN/23 for a discussion of the effect of packages on the search path ---
MDL.)
\subsection {Application Programs}
Every application program will have a connection file which details the
program parameters.
Once DSCL has located the program, it searches separately for the connection
file and, if it is present, determines which of the parameters are already
defined.

Because the program and connection file are searched for separately, a logical
name can be used to allow the same program to be used with several different
connection files, eg.\ if logical name QDISP has translation
DSCL\_EXEDIR:ADISP.EXE, but there is a file DSCL\_CONDIR:QDISP.CON, then there
need be only one copy of the executable image, but QDISP might allow more
connection file defaults (say).
\subsubsection {Definition of program parameters}
Program parameters are initially undefined.
They may become defined by being:
\begin{itemize}
\item Set equal to a literal value or the value of another parameter using the
LET command
\item Set by explicit call to WRKEYx in an application program
\end{itemize}
The value of a defined parameter can be retrieved and manipulated using DCL
facilities.
Parameter $<$param$>$ of program $<$prog$>$ is stored in a DCL logical name called:
\begin{verbatim}
        <prog>_<param>  ,eg:

        STATS_IMAGE
        GLOBAL_INPUT
        FFT_RINPUT
\end{verbatim}
Logical names may be manipulated within DSCL procedures by using the DCL
F\$LOGICAL lexical function.
Also, whenever a parameter becomes defined through use of the DSCL LET command,
a DCL global symbol is set up in {\em addition} to the logical name.
This is for the convenience of users as symbols are more convenient to use in
procedures than are logical names.
These logical names and global symbols remain in force until either cancelled
using the CLEAR command or until you log out.

Every DSCL command line is in fact read by DCL and consequently any DCL symbol
substitution will automatically be performed.
This is really a free bonus --- the only real benefits from symbol substitution
come in procedures which are dealt with in Section 2.4.

There is a special sort of program parameter called a GLOBAL PARAMETER which
may be thought of as a parameter to an application program called GLOBAL.
Such a parameter is referred to as  \_$<$param$>$ (or GLOBAL\_$<$param$>$) and is used
to provide a `last ditch' value for program parameters of the same name.
If a parameter is undefined but there is a global parameter of the same name,
DSCL will use the global value, but this will {\em not} cause the undefined
parameter to become defined.
\paragraph {Command line parameters}
A parameter value specified on the command line will always override any
previous value.
This will be a commonly used method and it is important to allow a flexible
syntax.
For this reason, command line parameters may be specified by KEYWORD or by
POSITION.
The order assigned to parameters is the same as that in the connection file.
DSCL {\em never} reads past the first ERROR type parameter in the connection
file.
Thus, the $<$parval$>$ of section 2.1 can have the following forms:
\begin{verbatim}
    <param>=<value>    normal form; parameter specified by keyword

    [=]<value>         positional parameter
\end{verbatim}
There must be {\em no} extraneous spaces.
This is a necessary restriction, since to permit spaces would be to permit
ambiguity.
An equals sign surrounded by spaces is a positionally-specified null parameter
value so `$<$param$>$ = $<$value$>$' is three positional parameters, {\em not}
one keyword parameter!
\paragraph {Accessing the stack}
If a frame parameter has value `\$', this indicates that the frame is
to be read from or written to the image stack (depending on whether it is an
input or output frame).
In some circumstances the user may ask for an input frame from an empty stack.
If this happens, DSCL will ask which file to put on the stack.

NB: because it is DSCL, not the application program, which handles the stack,
it is {\em not} permissible to reply `\$' to a prompt which has come from a
program since it will be taken literally!
\paragraph {Further details}
As stated above, DSCL reads the program connection file and determines which of
the program parameters are defined.
If a parameter is defined, the value is passed to the program just as
though it had been specified on the command line (although a command line value
takes precedence over a previously defined value).

If a value is defined then, just as described in SUN/4, DSCL attempts a
recursive logical name translation on it.
Thus, the parameter mechanism can be used to create an EQUIVALENCE between
parameters.
For example, suppose that we do the following:
\begin{verbatim}
        STATS_IMAGE=_ADD_OUT
\end{verbatim}
The underscore before ADD is to prevent a possible translation of ADD\_OUT
(ie.\ exactly as in DCL).
The actual value of STATS\_IMAGE is `ADD\_OUT' (ie.\ the underscore is removed).

Whenever we run STATS, DSCL will determine that STATS\_IMAGE has value
`ADD\_OUT'.
It will then translate ADD\_OUT recursively until translation fails, so if, for
example, the value of ADD\_OUT is `\$', STATS will read its input from the
stack.

It is advisable to include the underscore before ADD in the definition of
STATS\_IMAGE, since otherwise ADD\_OUT might be translated at the first stage
and we wish to defer translation to as late as possible.
It would be possible to prevent the LET command (section 2.3.1) from
translating, but the more normal interpretation of:
\begin{verbatim}
        STATS_IMAGE=ADD_OUT
\end{verbatim}
would be to set STATS\_IMAGE to the CURRENT value of ADD\_OUT.
\subsection {DSCL Commands}
DSCL commands are implemented as foreign DCL commands and are thus used in
exactly the same way as DCL commands.
They handle definition, manipulation, clearing and display of both parameters
and stack and provide a sophisticated HELP facility.
The internal workings are documented elsewhere.
Here we are concerned only with the user interface.
\subsubsection {Parameter commands}
You can use LET, LOOK and CLEAR to handle parameters.
Each is discussed separately.
\paragraph {Setting parameters}
The syntax of the LET command is:
\begin{verbatim}
        [LE[T] ]<progparam>{ }<value>
                           {=}
\end{verbatim}
There are several rules to be stated here so we will give a list of examples
demonstrating each rule at least once and then we will list the rules:
\begin{tabbing}
XXXX\=PLOTxTITLE=POWERxSPECTRUMXXX\=\kill
\>Command\>Effect\\
\\
\>LE CADD\_SCALAR=2\>CADD\_SCALAR = 2\\
\>LET CADD\_SCALAR 2\>as above\\
\>CADD\_SCALAR=2\>as above\\
\>CADD\_OUT=STATS\_IMAGE\>CADD\_OUT = value of STATS\_IMAGE. (Note\\
\>\>the underscore implies a parameter name.\\
\>\>If STATS\_IMAGE is undefined, the rhs is\\
\>\>taken literally)\\
\>PLOT\_TITLE=POWER SPECTRUM\>PLOT\_TITLE = POWER (the space delimits\\
\>\>--- see rules below)\\
\>PLOT\_TITLE=POWER\_SPECTRUM\>PLOT\_TITLE = value of parameter\\
\>\>SPECTRUM of program POWER. If this is\\
\>\>undefined, the literal value is used\\
\>PLOT\_TITLE=\_POWER\_SPECTRUM\>PLOT\_TITLE = text `POWER\_SPECTRUM'.\\
\>\>This is safer\\
\>GLOBAL\_OUT=\$\>GLOBAL\_OUT = \$\\
\>\_OUT=\$\>as above\\
\>OUT=\$\>as above
\end{tabbing}
and the rules are:
\begin{enumerate}
\item If the `LET' is omitted, the equals sign must be included and no spaces
can surround it.
\item If the $<$value$>$ contains an underscore, it will be assumed to be a
parameter name, unless it is preceded by an underscore, in which case the
leading underscore is removed.
If the parameter is undefined, the literal value is taken.
\item The $<$value$>$ may not contain any spaces.
This implies that to set a parameter to the value of a global parameter, the
`GLOBAL' {\em must} be included in the parameter name.
\end{enumerate}
\paragraph {Looking at parameters}
The syntax of the LOOK command (for parameters) is:
\begin{verbatim}
        LOO[K] [P[ARAM] ]<progparam>
\end{verbatim}
Again, we give a list of examples, followed by rules:
\begin{tabbing}
XXXX\=PLOTxTITLE=POWERxSPECTRUMXXX\=\kill
\>Command\>Effect\\
\\
\>LOO PARAM CADD\_SCALAR\>displays value of CADD\_SCALAR\\
\>LOOK P CADD\_SCALAR\>as above\\
\>LOO CADD\_SCALAR\>as above\\
\>LOOK CADD\>displays all of CADD's parameters\\
\>LOOK \_IN\>displays global parameter IN\\
\>LOOK GLOBAL\>displays values of all global parameters\\
\>\>accessed since login\\
\>LOOK STATS\>displays all of STATS's parameters\\
\end{tabbing}
and the rules are:
\begin{enumerate}
\item `PARAM' is only needed when $<$progparam$>$ could be confused with
`PARAM' or `STACK', ie.\ very seldom.
\item Writing the $<$progparam$>$ as $<$a$>$\_$<$b$>$, there are three cases:
\begin{itemize}
\item $<$a$>$ and $<$b$>$ both non-null - look at param b of program a.
\item $<$b$>$ null $<$a$>$ non-null - look at all params of program a.
\item $<$a$>$ null or `GLOBAL' $<$b$>$ not - look at global symbol b.
\end{itemize}
If there is no underscore, behaviour is as for the second case above (bearing
rule 1 in mind)
\item `PARAM' may be abbreviated to one character.
\end{enumerate}
\paragraph {Clearing parameters}
The syntax and operation of the CLEAR command are exactly like that of the LOOK
command; see above.
\subsubsection {Stack commands}
You can use CLEAR, DUPE, LASTX, POP, PUSH, RCL, LOOK, STORE and SWAP to handle
the stack.
They are tabulated below, followed by notes on points of interest:
\begin{tabbing}
XXXX\=PLOTxTITLE=POWERxSPECTRUMXXX\=\kill
\>Syntax\>Effect\\
\\
\>CL[EAR] S[TACK]\>Clears the entire stack by  performing an\\
\>\>appropriate number of POP operations.\\
\>\>LASTX will become undefined.\\
\>DU[PE]\>Duplicates the top entry on the stack\\
\>LA[STX]\>RCL's the most recently POPped frame\\
\>PO[P]\>Removes the top stack entry\\
\>PU[SH] $<$filename$>$\>Puts $<$filename> on top of the stack\\
\>RC[L] $<$filename$>$\>Exactly like PUSH\\
\>LOO[K] S[TACK]\>Displays the contents of the stack\\
\>STOR[E] $<$filename$>$\>Stores the top entry on the stack in\\
\>\>$<$filename$>$ (really just renames it)\\
\>SW[AP]\>Swaps the two top stack entries
\end{tabbing}
and the notes are:
\begin{enumerate}
\item Any attempt to do something illegal results in an error message  (eg.\
POPping an empty stack).
If the error is one which might affect the safe execution of a procedure, an
error status is communicated to the environment (which will cause immediate
exit from a procedure).
For example, an error in CLEAR STACK is {\em not} deemed fatal (the only
possible error is that the stack is already empty), whereas one in LET {\em is}.
\item POP does not normally cause files to be deleted.
Files are {\em never} deleted unless their names begin `WORK' (as is the case
for all files which are created automatically by DSCL).
\end{enumerate}
\subsubsection {Procedure commands}
The CALL command has been withdrawn since procedures can, in all cases, be
initiated implicitly:
\begin{verbatim}
        <procname>[ <param1>[ <param2>[...]]]
\end{verbatim}
$<$procname$>$ must have no filetype specified.
The searching procedure outlined in section SUN/23 is used to locate the
procedure and then it is compiled if necessary and executed.
Parameters are exactly as for DCL command procedures.

When DSCL is first entered, it searches for a procedure called STARTUP.
If such a file is found then it is executed.
This therefore allows the user the equivalent of a LOGIN.COM file in which
common definitions etc.\ can be effected.
Procedures are dealt with in more detail in Section 2.4.
\subsubsection {Help command}
The syntax of the HELP command is:
\begin{verbatim}
        H[ELP][ <item>[ <subitem1>[ <subitem2>[...]]]]
\end{verbatim}
To obtain a list of available items, HELP with no parameters.
To obtain a list of $<$subitem$>$s, refer to the output from HELP $<$item$>$.
A list of all available help on a particular subject can be obtained by typing
`HELP $<$subject$>$\ldots'.
If you abbreviate an item you will get help on all items matching the
abbreviation.
For example:
\begin{verbatim}
        H ASPIC
        HE FFT PARAMETERS
        HELP ARGS...
        HEL LET
\end{verbatim}
Help is available on all ASPIC programs, all DSCL commands, and various `general
topics' such as `ASPIC', `DSCL' and `AUTHORS'.
If you type HELP, you get a list of all available information (it's just like
DCL HELP in operation).

The help available on ASPIC programs is compiled automatically by stripping
certain of the comments from the program source.
In addition the connection file is listed, as is information on the author,
keywords (eg.\ PLOT, ARGS, 3D,\ldots) and the date that the HELP entry was made.

You can also use the HELP command to obtain help on DCL commands, but you
{\em cannot} abbreviate the names of DCL commands on which you want help.
This is unfortunate, but unavoidable.
\subsection {Procedures}
A DSCL procedure is similar in concept to a DCL command procedure.
In fact, after a little compilation, DSCL procedures are executed using the DCL
`@' command (although users do not see this) and this is why their parameters
work exactly as do those for DCL command procedures.
\subsubsection {Creating procedures}
Procedure sources reside in files of type `SCL'.
You can use an editor (eg.\ EDT) from within DSCL to create and modify procedure
files.

A procedure can contain any DSCL command and any DCL command with one very
slight restriction.
The `LET' of the DSCL parameter assignation command cannot be omitted in a
procedure.
This is because there is a potential ambiguity over the meaning of a command
such as:
\begin{verbatim}
        SCALAR=2000
\end{verbatim}
Does this mean `Set the DCL local symbol SCALAR equal to 2000' or `Set the DSCL
GLOBAL parameter SCALAR equal to 2000'?

The leading `\$' required in DCL command procedures should not be included in
DSCL procedures.
The first character of a procedure command has the same interpretation as in an
ordinary DSCL command line.
\subsubsection {Compiling procedures}
If you create a procedure and then immediately execute it, it will be compiled
before it is run and the compiled version will be put into a file of type `SCC'.
If you subsequently alter the procedure source, you must either delete the
compiled version (which is in the same directory as the source) or else compile
it `manually' using the COMPILE command:
\begin{verbatim}
        COM[PILE] <procname>
\end{verbatim}
$<$procname$>$ must not have a filetype specified.

Normally it is possible to decide, at run-time, whether a procedure is to echo
DSCL commands at the terminal as they are executed.
This means that there is a certain (small) overhead in running a procedure where
the commands are not being echoed.
If a `/NOECHO' qualifier is appended to $<$procname$>$, the echoing instructions
will not be compiled into the procedure and echoing will thus be disabled, eg.
\begin{verbatim}
        COM STARTUP /NOECHO
\end{verbatim}
`NOECHO' can be abbreviated to `NOEC'.
\subsubsection {Executing procedures}
Procedures are executed merely by typing the procedure name as a DSCL command.
As for application programs, first of all the current directory is searched,
then the ASPIC directories, then the logical name tables.
It is perfectly legal to include a disk/directory specification if you want to
execute somebody else's procedure.
For example, if the current directory is [RGOWFL] and the logical name GO has
translation STARTUP, the following are all equivalent:
\begin{verbatim}
        [RGOWFL]STARTUP
        STARTUP
        GO      (but not [RGOWFL]GO!)
\end{verbatim}
If an `/ECHO' qualifier is given (anywhere following the procedure name), DSCL
commands will be echoed as they are executed (unless the procedure was compiled
with the `/NOECHO' qualifier).
For nested procedures, each procedure preserves its own level of echoing.
`ECHO' can be abbreviated to `EC'.

Up to eight parameters may be passed to a procedure, separated, just as in DCL,
by spaces.
Within the procedure they are known as P1, P2,\ldots,P8 and are handled as DCL
local symbols.
The following simple example demonstrates how to clear the ARGS and display an
image, centred and zoomed as required:
\begin{verbatim}
        ACLEAR
        ADISP 'P1' = = = =
        AZOOM 256 256 'P2' 'P3'
\end{verbatim}
The parameters are optional.
If specified, the first is the input image and the other two are the factors for
zooming.
If omitted and undefined, the programs prompt for them.
If the procedure is called ARGS, the call might look like:
\begin{verbatim}
        ARGS GAUSS 8 8
\end{verbatim}
If we knew the size of the image --- perhaps program STATS writes parameters
NAXIS1 and NAXIS2 --- we could dispense with the last two parameters and use DCL
integer arithmetic to determine the appropriate zoom factors:
\begin{verbatim}
        ACLEAR
        ADISP 'P1' = = = =
        STATS ADISP_INPUT
        MAX=STATS_NAXIS1
        IF STATS_NAXIS2.GT.MAX THEN MAX=STATS_NAXIS2
        Z=2*(256/MAX)
        IF Z.GT.14 THEN Z = 14
        IF Z.LT.1  THEN Z = 1
        AZOOM 256 256 'Z' 'Z'
\end{verbatim}
The argument to the procedure is passed through to the program ADISP.
Strictly, since it is a parameter to the procedure, it should be thought of as
ARGS\_P1 (or whatever) not as ADISP\_INPUT, but that will have to come later, if
ever!

This will, of course, be slower than a program dedicated to the task, but it is
quicker to write and requires no knowledge of the ARGS library routines.
Provided the right primitive facilities are  available (such as clearing the
ARGS, displaying an image, adding two arrays or calculating a Fourier
Transform), a  DSCL  procedure will often be a far better solution than yet
another application program.

The DCL global symbol DSCL\_NSTACK holds the number of entries currently on the
stack.
It is possible to retrieve the names of the frames on the stack but is probably
not a good idea.
\subsubsection {Control structures in procedures}
Because all DCL commands can be used in DSCL procedures, DCL control statements
can be used to evaluate logical expressions and take conditional branches to
labels.
Admittedly DCL is not conducive to structured programming, but the basic
facilities are available.
An example of symbol usage was given in Section 2.4.2.

The following trivial example pushes ten files, named FILE1, FILE2,\ldots,FILE10
onto the stack.
It is, of course, mostly DCL:
\begin{verbatim}
        COUNT=0 NEXT_COUNT:
        COUNT=COUNT+1
        PUSH FILE'COUNT'
        IF COUNT.LT.10 THEN GOTO NEXT_COUNT
\end{verbatim}
\subsection {DCL Commands}
It has been stated that `most DCL commands' can be used in DSCL and that `all
DCL commands' can be used in DSCL procedures.
What exactly does this mean and what are the restrictions?
\begin{itemize}
\item You cannot take advantage of DCL's habit of prompting for missing
parameters.
Outside DSCL you can type `COPY' and be prompted `From:' and `To:'.
Within DSCL you cannot.
\item It is only possible to use DCL symbol assignments, IF, GOTO and labels in
procedures.
\end{itemize}
\subsection {Other Observations}
Several important points should be noted:
\subsubsection {Leaving DSCL}
The recommended way to leave DSCL is to execute the DCL STOP command:
\begin{verbatim}
        STOP
\end{verbatim}
Use of cntrl/Y will only return you to DSCL, not to DCL.
EXIT will have the same effect as STOP (but not, luckily, in a procedure!).
In an emergency, typing cntrl/Y {\em six} or more times without intervening
commands will always get you back to DCL.
\subsubsection {Precise action of the stack}
A stack is a simple concept: what goes in last comes out first.
However, in order to make the stack operate in a natural fashion, some
complication must be introduced into its operation and users should be aware of
this.
Although frames are referred to as being `on the stack', the stack is merely a
list of frame filenames and does not itself hold any frame data.

For each application program, DSCL makes a note of how many input frames come
from the stack and how many output frames are to go on to it.
When the program has run, DSCL makes the assumption that if there are no output
frames on the stack, then any input frames there should be left untouched, and
conversely that if there are any output frames on the stack, then all input
frames on it should be removed.
This is obviously `sensible' in that if an output frame has been produced, it is
likely to be derived from the input frames and so they should no longer be
needed.
In SUN/23 we saw that this allows DSCL to be used like an HP calculator.

Note that removal from the stack does not in general cause deletion of the frame
file.
It is deleted only if the filename begins `WORK', which is the case if DSCL put
it there in the first place.
Users should therefore avoid having frames in files whose names begin `WORK'
unless they want them to be deleted when POPped.
\section {EXAMPLES}
The following is a fairly genuine example of a DSCL procedure.
It is only fairly genuine because the absolute minimum is done in the
`specific application program' part of it (and even this could just about have
been done using existing ASPIC programs).
For this reason it is slow to execute, but it does work.
In practice, although convolution and display would be kept separate, a lot more
of the algorithm would be packaged into the central application program.

The problem is to deconvolve a spectrum using a known point spread function
using a method of successive convolution converging (in theory!) to the inverse
filter solution.
The method employed is due to Jansson (Huang) and the example comes from Dave
Pike.

The following is the procedure.
It requires three arguments which are spectrum, PSF, and relaxation parameter
--- 0.5 is a good guess! --- respectively.
After each iteration the restored spectrum is displayed and the user can
change the relaxation parameter.
The final iteration is left on the stack.
\begin{verbatim}
        IF P3.EQS."" THEN GOTO ERROR
        PUSH 'P2'
        STATS $
        CDIV STATS_TOTAL $ $
        DUPE
        POP
        PUSH 'P1'
        STATS $
        SPECMAX := 'STATS_MAX'
        JUMP_HERE:
        DUPE
        DUPE
        LASTX
        CONVOLVE
        POP
        STATS $
        CDIV STATS_MAX $ $
        CMULT 'SPECMAX' $ $
        PUSH 'P1'
        SWAP
        SUB $ $ $
        SWAP
        STATS $
        CDIV STATS_MAX $ $
        [CDP]MBRP 'P3' $ $
        MULT $ $ $
        ADD $ $ $
        LINPLOT $ =
        INQUIRE P3 "NEW RELAXATION PARAMETER (NULL TO STOP)"
        IF P3.EQS."" THEN GOTO END
        SWAP
        DUPE
        POP
        SWAP
        GOTO JUMP_HERE
        END:
        SWAP
        POP
        EXIT
        ERROR:
        WRITE SYS$OUTPUT "EX - USAGE: EX <SPECT> <PSF> <RELAX PARM>"
        EXIT
\end{verbatim}
It calls one procedure, CONVOLVE, which takes two images from the stack and puts
their convolution back on it (both the real and imaginary inverse FT images are
left, but we hope that the imaginary one is essentially zero!).
This involves three Fourier Transforms and would easily qualify for an ASPIC
program of its own!
\begin{verbatim}
        IF DSCL_NSTACK.LT.2 THEN GOTO ERROR
        FFT $ = $ $ = = F
        STORE KEEPCONBI
        POP
        STORE KEEPCONBR
        SWAP
        LASTX
        SWAP
        FFT $ = $ $ = = F
        STORE KEEPCONAI
        POP
        STORE KEEPCONAR
        MULT $ $ $
        SWAP
        LASTX
        MULT $ $ $
        ADD $ $ $
        RCL KEEPCONAR
        STORE WORKCONAR
        RCL KEEPCONBR
        STORE WORKCONBR
        MULT $ $ $
        RCL KEEPCONAI
        STORE WORKCONAI
        RCL KEEPCONBI
        STORE WORKCONBI
        MULT $ $ $
        SUB $ $ $
        SWAP
        FFT $ $ $ $ = = T
        EXIT
        ERROR:
        WRITE SYS$OUTPUT "CONVOLVE - STACK HAS FEWER THAN TWO ELEMENTS"
\end{verbatim}
Note the use of temporary names not beginning `WORK' to prevent popped images
from being deleted.
Perhaps it is possible to do it {\em all} on the stack, but if so, the solution
probably resembles that of the Tower of Hanoi!
\section {FUTURE DEVELOPMENTS}
Several additional features could be added at little expense of effort:
\subsection {Further Procedure Commands}
It would be useful to be able to submit a DSCL procedure on a batch queue.
This would be very easy to implement.
Another possibility, only slightly less easy to implement, is the creation of a
subprocess (or even a detached one) to perform some computationally intensive
task, such as the calculation of a Fourier Transform.
The only difficulty is if the subprocess is to be allowed to request input from
its parent (this could, I think, be a real difficulty).
\subsection {Multiple Commands on the line}
This, again, would be easy to implement but is probably not worth it.
\subsection {Stacking of Program Parameters}
If it were possible to stack the values of program parameters on entry to a
procedure this would be very useful.
\section {REFERENCES}
\begin{itemize}
\item SUN/4 `INTERIM --- Starlink software environment'.
\item SUN/23 `ASPIC --- Image processing programs (1)'.
\item Digital Command Language User's Guide.
\item 	Picture Processing and Digital Filtering [1975], Springer-Verlag.
\end{itemize}
\appendix
\section {COMMAND FORMATS}
\begin{quote}
\begin{verbatim}
Command       Format

Program       [<commchar>]<progname>[ <parval1>[ <parval2>[...]]]
                (<parvaln> is of the form [<param>][=][<value>])
Procedure     <procname>[/EC[HO]][ <param1>[ <param2>[...]]]
DSCL command  Behave as DCL commands
DCL command   As normal
CLEAR         CL[EAR] [P[ARAM] ]<progparam>
              CL[EAR] S[TACK]
                (<progparam> is of the form <program>[_<param>])
COMPILE       COM[PILE] <procname>[/NOEC[HO]]
DUPE          DU[PE]
HELP          H[ELP][ <item>[ <subitem1>[ <subitem2>[...]]]]
LASTX         LA[STX]
LET           [LE[T] ]<progparam1>=<progparam2>
                (In most circumstances `LET' can be omitted.)
LOOK          LOO[K] [P[ARAM] ]<progparam>
              LOO[K] S[TACK]
                (<progparam> is of the form <program>[_<param>])
POP           PO[P]
PUSH          PUS[H] <filename>
RCL           RC[L] <filename>
STORE         STOR[E] <filename>
SWAP          SW[AP]
\end{verbatim}
\end{quote}
\end{document}
