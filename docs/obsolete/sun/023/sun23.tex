\documentstyle[11pt]{article}
\pagestyle{myheadings}

%------------------------------------------------------------------------------
\newcommand{\stardoccategory}  {Starlink User Note}
\newcommand{\stardocinitials}  {SUN}
\newcommand{\stardocnumber}    {23.14}
\newcommand{\stardocauthors}   {M D Lawden \& D S Berry}
\newcommand{\stardocdate}      {17 September 1991}
\newcommand{\stardoctitle}     {ASPIC --- Image processing programs (1)}
%------------------------------------------------------------------------------

\newcommand{\stardocname}{\stardocinitials /\stardocnumber}
\markright{\stardocname}
\setlength{\textwidth}{160mm}
\setlength{\textheight}{230mm}
\setlength{\topmargin}{-2mm}
\setlength{\oddsidemargin}{0mm}
\setlength{\evensidemargin}{0mm}
\setlength{\parindent}{0mm}
\setlength{\parskip}{\medskipamount}
\setlength{\unitlength}{1mm}

\begin{document}
\thispagestyle{empty}
SCIENCE \& ENGINEERING RESEARCH COUNCIL \hfill \stardocname\\
RUTHERFORD APPLETON LABORATORY\\
{\large\bf Starlink Project\\}
{\large\bf \stardoccategory\ \stardocnumber}
\begin{flushright}
\stardocauthors\\
\stardocdate
\end{flushright}
\vspace{-4mm}
\rule{\textwidth}{0.5mm}
\vspace{5mm}
\begin{center}
{\Large\bf \stardoctitle}
\end{center}
\vspace{5mm}

\setlength{\parskip}{0mm}
\tableofcontents
\setlength{\parskip}{\medskipamount}
\markright{\stardocname}

\section {INTRODUCTION}

ASPIC is a collection of image processing programs.
It is not a monolithic package of the usual kind, nor even a tight group of
interacting programs like SPICA.
The common link is that they were all written using INTERIM (the first Starlink
software environment: see SUN/4), are held in a single directory (except for
the EDRS and PERIODS packages which are stored in subdirectories) and share a
common HELP library.
Minimal effort is needed to add new programs to the system.
A summary of the available ASPIC programs is given in section 2.

The programs may be run using the RUNSTAR command (see SUN/4).
However, this was found to be very tedious and it offers only primitive
inter-program communication.
Consequently, another command language called DSCL has been implemented.
This provides an on-line HELP facility both for ASPIC programs and DSCL itself,
a procedure mechanism for creating new applications, an image stack for
intermediate image storage, and a convenient way of handling parameters by
position, keyword or default which saves a lot of typing.
Communication between programs is also improved by having a database of
information about the images displayed on the ARGS.
This makes it possible to separate the display function from manipulative
functions.
Any program held in any directory may be run by DSCL with no further action
except that the directory name is required if the program does not reside in
the current default directory.
An introduction to DSCL is given in section 3, to the HELP system in section 4
and to the ARGS database in section 5.

The word `package' is used in two different senses in this paper.
The first refers to a collection of associated programs which are suited to
a specific application area or were derived from a previously existing software
package.
The second (distinguished from now on by beginning with a capital P) refers to
a set of programs in a specific directory which is activated by the
DSCL `GO' command and have their own environment and HELP.
The only Packages which exist in the currently installed version of ASPIC are
EDRS, PER and CHART although some sites may also have installed extra local
packages.
The EDRS Package corresponds to the {\bf EDRS} program class (see below) and
the PER Package corresponds to the {\bf Periods} class.
The CHART Package corresponds to the CHART software item (see SUN/32) which
is installed separately from the ASPIC programs and is not considered in this
paper.
The use of Packages is described in more detail in section 6.

This paper only provides an introduction to ASPIC and DSCL.
More detailed information on some of the ASPIC programs is contained in SUN/24
and on DSCL in SUN/74.
A lot of information is also stored on-line where it may be accessed using the
HELP facility or extracted for printing using the ASPIC program FILEDOC.
A demonstration of ASPIC is shown in section 7.
A tutorial introduction to ASPIC is contained in SG/1, which also contains
extensive documentation on some of the more popular programs.

This paper is a major revision of SUN/23.9 and also incorporates the material
in the old SUN/53 which is now withdrawn.
The original version of SUN/23 was written by Dave King and Ken Hartley while
they were working at RGO.
DSCL was developed by William Lupton at RGO and the ASPIC programs were written
by many programmers throughout the Starlink network.

ASPIC is in the process of being replaced by a set of application packages
based on the new Starlink environment called ADAM (see SUN/94) which should
replace the INTERIM environment.
This will take a long time and ASPIC should remain available for several years
to come.

\section {APPLICATION PROGRAMS}

ASPIC was initially created as a set of image processing primitives to handle
the basic manipulations of images which are an essential part of any
astronomical processing.
Early releases concentrated on image arithmetic, standard filters,
expansion/contraction/selection/combination of images, displaying and
manipulating images on the ARGS and other devices.
Later releases added to this sound framework many new astronomy-specific
applications.
Note in the following that images are often referred to as {\em frames}.
A frame (or `BDF file') is the data structure used to store images in the
INTERIM environment.

ASPIC programs have been grouped into 21 classes.
These are used in the program list in appendix B and in the on-line HELP system.
The classes are as follows:
\begin{description}
\item [ARGS]:
These programs either display images on the Sigmex ARGS or modify an existing
display.
The ARGS assigned to ARGS\_DEVICE is used.
They all make use of entries in the ARGS database so that they can
share information about the size and location of the visible images.
Look-up-table operations are described under class {\bf LUT}.
\item [Arithmetic]:
These programs perform scalar operations on frames (in most cases they
need not be 2D images) and array operations on a pixel by pixel basis.
These are not matrix operations.
\item [Astrometry]:
These programs are logical extensions of the {\bf EDRS} (X,Y) routines which
input and manipulate lists of (RA,DEC).
They allow transformations between an ARGS pixel element and its position
on the sky.
They must be used in conjunction with the {\bf EDRS} programs XYKEY, XYCUR,
XYFIT, XYTRAN and CINVERT.
Their use is described in more detail in LUN/42 (RGO).
\item [Compress]:
These programs allow compression of the intensity range of images so that they
can be packed into as few bits as possible for rapid transfer over the network.
They should not be confused with those in class {\bf Size} which actually
change the dimensions of an image.
\item [Details]:
These programs extract various details from images.
They cover such things as descriptor items, sections, positions of details
in an image, and the histogram.
\item [Display]:
These programs perform various types of display, excluding the `usual' image
display on the ARGS (see {\bf ARGS} class).
Nevertheless, many of them use the ARGS in graphics mode.
Some of the programs in the {\bf Details} class also display graphical
information.
See also SUN/24.
\item [Docs]:
These programs create program documentation or extract it from the HELP
library.
See also FILEDOC (section 4.2).
\item [EDRS]:
These programs comprise the Electronography Data Reduction System which is a
Package designed for the reduction and analysis of large format astronomical
images written by Rodney Warren-Smith at Durham.
In its original form it specialised in the reduction of electronographic data
but was built around a set of utility programs which were widely applicable
to astronomical images from other sources.
The programs align and calibrate images, handle lists of (X,Y) positions, apply
linear geometrical transformations and do some stellar photometry.
The Package has been extended and no restrictions in its application to other
areas should arise.
The {\bf GRASP} class programs offer extended versions of some of these
programs and others which fit into the same overall scheme.
To use an {\bf EDRS} program from within DSCL, type `GO EDRS' to enter the
Package environment or prefix the program name with `EDRS:'.
Within the Package environment, access the {\bf EDRS} on-line HELP library by
typing `HELP'.
A user manual is available from Warren-Smith in Durham; see also SUN/24.
In 1988, {\bf EDRS} was updated by David Berry to support GKS 7 graphics, and
image handling on the Digisolve IKON displays.
\item [EDRSX]:
This is an extension to the {\bf EDRS} package written by David Berry.
It was originally produced to make possible more versatile analysis of IRAS
images than was otherwise available.
EDRSX provides facilities for converting images into and out of EDRS format,
accessing RA and DEC information stored with IRAS images, and for doing several
standard image processing operations such as displaying image histograms and
statistics, Fourier transforms, etc.
These enable such operations to be performed as estimation and subtraction of
non-linear backgrounds, de-striping of IRAS images, modelling of image
features, easy aligning of separate images, etc.
\item [FC]:
These programs allow up to three images to be combined together and displayed as
a single {\em False Colour} picture.
They can be applied to B,V,R or intensity/velocity images.
The techniques used are described in LGP/3 (RGO).
\item [Filter]:
These programs perform some of the many types of noise filtering described in
the literature which do not make use of Fourier techniques.
Other non-Fourier filter programs are available in the {\bf EDRS} and
{\bf GALPHOT} classes.
\item [Fourier]:
These are basic Fourier transform programs.
More sophisticated programs are available in the {\bf EDRS} and {\bf GALPHOT}
classes.
\item [GALPHOT]:
These programs perform photometry of bright galaxies and were written by
Clive Davenhall at ROE.
They do absolute photometric calibration, axis extraction, background fitting
and normalization, contour mapping, ellipse fitting, equivalent profiling,
intensity conversion, object removal, profile analysis, and smoothing.
See also SUN/24.
\item [Geometry]:
These programs define and apply geometrical transformations to images.
The transformations are non-linear; for linear transformations see the
{\bf EDRS} class.
\item [GRASP]:
These programs comprise the Giant Raster and Stellar Photometry package which
is an extension of the {\bf EDRS} Package with some more elaborate and varied
profile fitting, including multiple star fitting, together with some programs to
manipulate and display the results and to convert PDS scans into calibrated
images.
Some of them use 16-bit data to save storage of REAL and INTEGER incarnations.
This is particularly important for the large images which this package was
designed to handle (2000x2000 images have been successfully processed).
The package was prepared by Alan Penny while he was working at RGO.
\item [Input]:
These programs allow various forms of data input into images.
Users who are interested in authentic astronomical data should use PUTSTAR in
{\bf GRASP}.
\item [LUT]:
These programs prepare and load look-up-tables into the ARGS.
Each is a 3x256 {\em BDF} file and is loaded into the ARGS assigned to
ARGS\_DEVICE.
\item [Miscellaneous]:
These programs do not fit into any of the other classes but perform many
important functions.
See also SUN/24.
\item [Periods]:
These programs comprise a Package (called PER) designed to search for periods in
observations which may be spaced at regular or irregular intervals.
They were built to handle photometric or radial velocity observations but are
not restricted to these applications.
They will {\em not} work on very large data sets such as may be produced by
high speed photometers, and will {\em not} handle aperiodic phenomena which
cannot be analysed as being composed of a number of periodic components.
To use any of these programs from within DSCL, type `GO PER' to enter the
Package environment or prefix the program name with `PER:'.
Within the Package environment, `HELP' will contain information about these
programs.
The Package was put together by Norman Walker and Ken Hartley at RGO.
See also SUN/24.
\item [Photometry]:
These programs extract photometric information from frames of pixel data.
See also SUN/24.
\item [Polarimetry]:
These programs manipulate and reduce frames of polarimetric data.
See also SUN/24.
\item [Size]:
These programs perform expansion and contraction of images, selection of parts
of an image, and merging of several images.
\end{description}
\subsection {Feedback}
Users are encouraged to send any comments about the performance --- efficiency
and accuracy --- of these programs to RLVAD::STAR.
The same should be done with bug reports, suggestions for modifications to
existing programs, inclusion of new programs and deletion of `useless' programs.
Authors of individual programs should only be contacted in case of difficulty
using a program or to find out about the details of a particular algorithm.

\section {DSCL}

\subsection {Basic Facilities}
DSCL aims to prevent as much redundant program parameter specification as
possible.
It provides the following facilities:
\begin{itemize}
\item Values of parameters can be set and the definitions remain in force
until cancelled or until logout.
\item Frames may be stored on a stack and programs can be instructed to
read or write their data from or to it.
A set of stack-handling commands is provided.
\item Commands are provided to display and delete program parameter definitions
and stack contents.
\item Most DCL commands (all except for symbol assignments, labels, IF and
GOTO) can be used in DSCL.
The procedure facility allows a file containing any DSCL or DCL command to be
executed making full use of all DCL symbols, conditional tests, etc.
If you have a procedure called STARTUP, it is executed automatically when you
enter DSCL.
\item Help on DSCL command syntax, DSCL procedures and ASPIC programs etc.\
can be obtained by the HELP command which operates just like DCL HELP and
provides, in addition, help on DCL commands.
\item All commands may be abbreviated to their briefest unambiguous form.
DSCL commands will prompt for their parameters if these are omitted (DCL ones
won't).
\end{itemize}
\subsection {Restrictions}
It is a DCL restriction that command procedures cannot be nested more than
eight deep.
This will not cause problems unless user procedures are nested more than six
deep.
\subsection {Simple Use of DSCL}
To enter DSCL from DCL type the command:
\begin{quote}
{\tt \$ DSCL}
\end{quote}
The DSCL prompt is:
\begin{quote}
{\tt Dscl$>$}
\end{quote}
This is an invitation to input one of the following:
\begin{enumerate}
\item The name of an application program,
\item The name of a DSCL procedure,
\item A DSCL or DCL command
\end{enumerate}
with an optional argument list.
In cases 1 and 2 a search is carried out for the executable image file or
procedure file.
The order of searching is:
\begin{enumerate}
\item Current directory,
\item Local ASPIC directory (defined by logical name DSCL\_LOCDIR),
\item Package directory (DSCL\_PAKDIR),
\item ASPIC directory (ASPDIR),
\item Logical name tables.
\end{enumerate}
Thus, it will not normally be necessary to specify a disk or directory name.
If one is specified, no searching is done.
Note that only one Package directory is searched and any
programs or procedures in this directory may be referenced by name only.
Programs or procedures in other Packages can only be referenced by prefixing the
name by the full file description of the directory in which they are stored or
an equivalent logical name.

On receiving a command, DSCL assumes first of all that an application program
is to be run.
It is only when the search for an executable image file fails that it looks for
a DSCL procedure file, and it is only when this second search fails that it
tries to execute the command as a DSCL or DCL command.
This means that there is a possible ambiguity where, for example, a
procedure has the same name as an application program.
For this reason, it is possible to begin any DSCL command with a special
character to indicate what sort of command it is.
These special characters are:
\begin{verbatim}
        Application program     -     *
        DSCL procedure          -     @
        DSCL or DCL command     -     $
\end{verbatim}
Clearly the presence or absence of a `*' makes no difference.
Note that these special characters affect only the `jump-in' point for the
search.
If the command starts `@' and no procedure is found, an attempt is still
made to execute it as a DSCL or DCL command.

To get out of DSCL type:
\begin{quote}
{\tt STOP}
\end{quote}
\subsection {Program Parameters}
Suppose that you have a program ADD which adds together two frames.
It has three parameters:
\begin{description}
\item [IN1]: the first input frame.
\item [IN2]: the second input frame.
\item [OUT]: the output frame.
\end{description}
The normal way of running this program outside DSCL, perhaps during its
development, would be to type:
\begin{quote}
{\tt \$ RUNSTAR ADD/IN1=SALT/IN2=PEPPER/OUT=CRUET}
\end{quote}
(inserting your own values), or:
\begin{quote}
{\tt \$ RUNSTAR ADD}
\end{quote}
(responding to the prompts issued by the system).

When you communicate with DSCL, these parameters are known as ADD\_IN1, ADD\_IN2
and ADD\_OUT and their values are of relevance only to program ADD.
The first time you run ADD none of the parameters are defined, so if you
merely type:
\begin{verbatim}
     ADD
\end{verbatim}
the program will prompt you for them:
\begin{verbatim}
     IN1:=SALT
     IN2:=PEPPER
     OUT:=CRUET
\end{verbatim}
Alternatively, you could have typed:
\begin{verbatim}
      ADD IN1=SALT IN2=PEPPER OUT=CRUET
\end{verbatim}
These both have exactly the same effect, as does:
\begin{verbatim}
      ADD SALT PEPPER CRUET
\end{verbatim}
You could also type:
\begin{verbatim}
      ADD_IN1=SALT
      ADD_IN2=PEPPER
      ADD_OUT=CRUET
      ADD
\end{verbatim}
Here we have defined values for all three parameters and DSCL will remember
them for future use.
DSCL never remembers parameter values unless you ask it to, thus you can keep
track of what is going on.
You can always use the LOOK command in cases of doubt:
\begin{verbatim}
      LOOK ADD
\end{verbatim}
In this case the following output would result:
\begin{verbatim}
      ADD_IN1 = SALT
      ADD_IN2 = PEPPER
      ADD_OUT = CRUET
\end{verbatim}
Typing `CLEAR ADD' will cancel all the above definitions (they can be cancelled
individually as well).

All this would be wonderful were ADD the only program in the world.
This may be the case at present (it isn't in fact), but we can hope for a
better state of affairs in the future.
Suppose we also have a program STATS which has a single parameter IMAGE which
is an input frame whose size, mean data value etc.\ are to be calculated.
We might want to run ADD and then use the output from it as the input to STATS.
One way of doing this is:
\begin{verbatim}
      ADD_OUT=POWER
      ADD REALSQ IMAGSQ
      STATS ADD_OUT
\end{verbatim}
In fact we know that the value of ADD\_OUT is POWER so we could just have typed
`STATS POWER'.
However, in some circumstances we might know only the parameter name.
Suppose we have a program LIMITS which takes a 1D image, displays it, and gets
the user to define a left and right cutoff using the cursor.
These values go into a parameter LIMS and, if they are ever needed again, can be
retrieved as in the following example:
\begin{verbatim}
      LIMITS INPUT=M57.HIS
      RESCALE INPUT=M57 INLIMS=LIMITS_LIMS OUTPUT=$ OUTLIMS=0,255
\end{verbatim}
For an explanation of the mysterious `\$' read on.
\subsection {The Image Stack}
For many applications it is far more convenient to use the stack than
continually to have to think up new temporary names.
A `\$' as a parameter value (provided it is a frame) means that
the frame is to come from or go to the stack.
Note that because it is DSCL rather than the application program that handles
the stack, it is {\em not} possible to respond `\$' to a parameter prompt from
an application program.

ADD is a good example of such an application.
If we type:
\begin{verbatim}
      ADD $ $ $
\end{verbatim}
then it is assumed that both input frames are to come from the stack and that
the output frame is to go back to it.
DSCL will ensure that the appropriate stack house-keeping is carried out.
Thus, using the stack, DSCL behaves like an HP calculator.

Here is an example of a procedure which calculates a power spectrum from the
real and imaginary parts of a Fourier Transform.
It is self-explanatory and gives a taste of the power afforded through the use
of DCL in DSCL procedures:
\begin{verbatim}
      !      Procedure PSPECT
      !
      !      Take real and imaginary format frames from stack
      !      and evaluate power spectrum on stack
      !
            DUPE                  ! duplicate top of stack
            MULT $ $ $            ! square one image
            SWAP                  ! bring other to top of stack
            DUPE                  ! duplicate top of stack
            MULT $ $ $            ! square other image
            ADD $ $ $             ! calculate the power spectrum
\end{verbatim}

\section {ASSISTANCE}

Most of the detailed documentation of ASPIC and DSCL is held on-line and may be
accessed through a HELP command; the same information may also be printed.
The appendices to this paper give full lists of DSCL commands and ASPIC
programs.
\subsection {HELP}
The HELP library is grouped into classes which correspond to the sections of
Appendix B.
It also contains information on the following topics:
\begin{description}
\item [ASPIC]: Some brief notes on ASPIC.
\item [DSCL]: Detailed information about DSCL under various headings.
\item [HELP]: How to use the HELP system.
\item [LOCAL]: These are programs which are only available at the local site and
cannot be accessed by DSCL unless prefaced by a locally-defined directory name.
Some sites may adopt other strategies such as defining a `local' DSCL command
LDSCL.
\end{description}
Because of the way HELP libraries work, if a program should be in more than one
class then a second copy of the documentation must be stored.
In general this has not been done so be aware that programs documented in one
class may be of use in other contexts.
\subsubsection {ASPIC HELP}
The ASPIC HELP library and the standard VMS library may both be searched by
using a modified version of the standard VMS HELP.
Within DSCL, typing `HELP' initiates an interactive session in exactly the same
way as the standard VMS HELP.
However, there are three differences from the standard HELP:
\begin{itemize}
\item The first library to be searched is the ASPIC HELP library. The structure
of this means that:
\begin{description}
\item [Topics]: are the {\em classes} of ASPIC programs.
\item [Sub-topics]: are the individual programs.
\item [Sub-sub-topics]: are {\em date}, {\em author} and {\em parameters}.
\end{description}
This means that it is no-longer possible to ask for HELP on an individual
program; you must either specify a class or use the command:
\begin{verbatim}
      HELP * program
\end{verbatim}
where the `*' means search all classes.
For any prompt, a response of $<$CR$>$ will take you up a single
level in the hierarchy and a response of `?' (with {\em no} $<$CR$>$) will list
the options available at that stage.
\item The VMS library is also available at the same time.
When at the `topic' level, simply prefacing any command with `@VMS' will search
that library; using `@ASP' will return to the ASPIC library.
\item It is possible to page output to most terminals.
It has not been possible to confirm that this procedure works on every terminal
in Starlink, but it does on all the terminals on which it has been tried.
Output may be interrupted when a page has been output by responding `?'.
\item For the effect on HELP of activating Packages, see section 6.5.
\end{itemize}
The command WHERE allows you to find out how to find out about an ASPIC
program.
It works as follows:
\begin{verbatim}
      WHERE program
\end{verbatim}
will respond with the string `HELP topic program' where the topic is the one
appropriate to the selected program (or procedure) name.
\begin{verbatim}
      WHERE
\end{verbatim}
an alphabetical list of {\em all} ASPIC programs is output.
\begin{verbatim}
      WHERE/OUTPUT=LIST
\end{verbatim}
the same information is written to the file LIST.LIS.
\begin{verbatim}
      WHERE/PRINT
\end{verbatim}
the same information will be printed but not stored.
\subsection {FILEDOC}
A procedure has been written to create documentation in a form suitable for
printing.
The command may be used in several forms:
\begin{verbatim}
      FILEDOC
\end{verbatim}
will write a classified list of all ASPIC programs into the file INFO.LIS.
\begin{verbatim}
      FILEDOC class program [parameters,author,date]
\end{verbatim}
will write the requested information into a suitable file and tell you its
name.
\begin{verbatim}
      FILEDOC *...
\end{verbatim}
will write the complete ASPIC HELP information into a file.
The resulting output occupies about 200 pages when printed so this should not be
done unless it is essential.
\subsection {FULLDSCL}
The command:
\begin{verbatim}
      $ PRINT ASPDIR:FULLDSCL.LIS
\end{verbatim}
will print a much fuller account of DSCL than the one contained here.
There are also more examples and a description of the mechanism for creating
and running procedures.
DSCL is also documented in SUN/74.

\section {ARGS DATABASE}

ASPIC separates the processing and manipulation functions from the display
function.
This is possible because each time ADISP is used to display an image, an
entry is made in a small database.
AZOOM, for example, can find out where the last image displayed was located on
the ARGS screen and hence give sensible run-time defaults.
Likewise, routines which use the ARGS cursor `know' whether the point selected
was inside an image and can also return the co-ordinates of the selected point
in `array' co-ordinates, not just in ARGS co-ordinates.

No programs have been implemented which read image data back from the ARGS
memory.
Consequently, programs which need to access the data, such as SLICE, must
prompt for the image name.
Provided the image whose name is given is of the same size as the one displayed
it need not be the one displayed.
Thus, a contrast enhanced or smoothed version may be displayed but a slice may
be taken out of the raw data.
However, program LUTSTORE reads a look-up-table back from the ARGS and stores it
as a BDF file.

\section {APPLICATION PACKAGES}

The Applications Package concept can be used not only as a Starlink-wide or
local ASPIC facility, but also on an individual user basis.
The following sub-sections describe the DSCL command to initiate searching of
the Package directory, defining the Package directory and the effects on the
HELP facility.
\subsection {Initiating the Package}
When DSCL is started, logical name DSCL\_PAKDIR is set to `NULL' and
therefore the searching of directories defaults to the original structure.
The new Package will only be used if specifically requested by the user.
This ensures that DSCL/ASPIC is upwards compatible and no changes are required
to existing ASPIC programs and procedures.
The searching of the Package directory is initiated by the following command:
\begin{quote}
{\tt GO {\em Package name}}
\end{quote}
Where {\em Package name} is a logical name which translates into the directory
where the Package programs are stored.
It is possible to GO directly from searching one Package to searching another
without returning to the normal DSCL state.
In order to return to the normal DSCL state, ie.\ no searching of a Package
directory, type:
\begin{quote}
{\tt GO DSCL}
\end{quote}
Typing `STOP' at any stage will cause an orderly close down of DSCL and a return
to VMS.
\subsection {Package names}
The Package name is a logical name which may be set in one of three places.
The logical names of any Starlink supported Packages will be set up in file
ASPDIR:ASPPAK.COM.
Similarly, for any locally supported Packages they will be set up in the file
DSCL\_LOCDIR:ASPPAK.COM.
These two files will be executed when DSCL is started.
Note that they will not redefine any existing logical names, so you can set
a logical name in your own LOGIN.COM file to point to a directory which
contains your own version of a Starlink or locally supported Package.
This is a very useful facility for modifying and testing an existing Package
without inconveniencing users of this supported Package.
You can also define a logical name or redefine an existing logical name once
DSCL has been started.
\subsection {Package initialisation}
A Package may wish to have its own initialisation routine to preset certain
variables, etc.
This can be done automatically by including a file INIT.SCL in the Package
directory.
When the GO command is executed the Package directory is searched for a file
INIT.SCL which, if found, is executed.
\subsection {Package prompts}
When Package searching is in operation, the normal `DSCL$>$' prompt will be
replaced by a prompt `{\em Package name}$>$' unless the Package name is longer than
7 characters, in which case the prompt is formed from the first 7 characters of
the Package name.
Therefore, if you issue several GO commands the prompt will always tell you
which Package directory is currently being searched.
\subsection {Package HELP}
In DSCL's default state the searching order for the HELP libraries is the ASPIC
HELP library, followed by the VMS HELP library.
When you issue a GO command, a search will be made in the Package
directory for a file HELPLIB.HLB which, if found, will be made the first HELP
library to be searched, followed by the ASPIC HELP library and finally the VMS
HELP library.
This means that Packages can have their own HELP system which is totally
independent of the ASPIC HELP system but is only available when the GO command
has been issued.
It is not essential for a Package to have a HELP library; if one is
not present then DSCL's default state HELP system will still be available.
\subsection {Package closedown}
When a Package closes down because we are GOing to another Package or returning
to the normal DSCL level, the Package may wish to execute its own tidying up
routine.
This can be done automatically by including a file TIDY.SCL in the Package
directory.
Therefore, when GOing to another Package or to DSCL, the current Package
directory is searched for a file TIDY.SCL which, if found, is executed.
\subsection {DSCL procedures}
The GO command may also be used in DSCL procedures, provided the logical names
for the Packages have been set up when the procedure was compiled.
\subsection {Example}
Suppose a user with default directory DISK\$USER1:[USER] has modified some
ASPIC programs or written his own programs for his particular data reduction
system and stored them in a subdirectory [USER.\-ASPIC].
His data consists of a number of observation runs stored in the subdirectories
[USER.OBS1], [USER.OBS2], etc.
In order to use the Package facility he could put the following statement in
his LOGIN.COM file:
\begin{quote}
{\tt ASSIGN DISK\$USER1:[USER.ASPIC] MYPAK}
\end{quote}
Then, in whichever subdirectory he runs DSCL and wishes to use his ASPIC
programs (stored in the subdirectory [USER.\-ASPIC]) he only has to type:
\begin{quote}
{\tt GO MYPAK}
\end{quote}
This will change the DSCL prompt to `MYPAK$>$' and enable searching of the
subdirectory [USER.\-ASPIC] before searching the main ASPIC directory.
Since the Package name is a logical name, the user can always refer
directly to his own programs while running DSCL in its default state, or within
another Package by prefixing the program name with the logical name (eg.\
MYPAK:progname).
This will be useful if the user uses another Package but still wishes to refer
to a program or procedure in MYPAK.

\section {DEMONSTRATION}

A self-documenting demonstration procedure has been devised which shows some
application programs in action and illustrates some features of DSCL.
It also shows how DSCL command procedures can be used for repeated operations.
To run it, type:
\begin{verbatim}
      $ DSCL
\end{verbatim}
followed by:
\begin{verbatim}
      DEMO/ECHO
\end{verbatim}
The /ECHO qualifier ensures that each line of the procedure, including the
comments, is written to the terminal as it is executed.
When it has finished, leave DSCL by typing:
\begin{verbatim}
      STOP
\end{verbatim}
The procedure displays and manipulates a 256x256 image of the Horsehead Nebula
obtained by scanning a UKSTU plate on the PDS microdensitometer.
The procedure is held in file ASPDIR:\-DEMO.\-SCL and its `compiled' version
in file ASPDIR:DEMO.SCC.

\section {IMPLEMENTATION}

If the ASPIC directory is moved or installed at a new site, the logical names
defined in ASPDIR:ASPAK.COM may need changing, as well as those defined in
LSSC:STARTUP.COM.
Also, check that ASPDIR:ASPINST.COM is acceptable.

\section {REFERENCES}

\begin{description}
\item [SG/1]: ASPIC Guide.
\item [SUN/4]: INTERIM --- Starlink software environment.
\item [SUN/24]: ASPIC --- Image processing programs (2).
\item [SUN/32]: CHART --- Finding chart and stellar data system.
\item [SUN/74]: DSCL --- An interim Starlink command language.
\item [SUN/94]: ADAM --- Starlink software environment.
\item [LGP/3 (RGO)]: Some comments on displaying images.
\item [LUN/42 (RGO)]: A simple interactive astrometry package.
\end{description}
\appendix

\section {DSCL COMMAND SUMMARY}

\begin{quote}
\begin{verbatim}
Program        [<commchar>]<progname>[ <parval1>[ <parval2>[...]]]
                 (<parvaln> is of the form [<param>][=][<value>])
Procedure      <procname>[/EC[HO]][ <param1>[ <param2>[...]]]
DSCL command   Behave as DCL commands
DCL command    As normal
CLEAR          CL[EAR] [P[ARAM] ]<progparam>
               CL[EAR] S[TACK]
                 (<progparam> is of the form <program>[_<param>])
COMPILE        COM[PILE] <procname>[/NOEC[HO]]
DUPE           DU[PE]
HELP           H[ELP][ <item>[ <subitem1>[ <subitem2>[...]]]]
LASTX          LA[STX]
LET            [LE[T] ]<progparam1>=<progparam2>
                 (In most circumstances `LET' can be omitted.)
LOOK           LOO[K] [P[ARAM] ]<progparam>
               LOO[K] S[TACK]
                 (<progparam> is of the form <program>[_<param>])
POP            PO[P]
PUSH           PUS[H] <filename>
RCL            RC[L] <filename>
STORE          STOR[E] <filename>
SWAP           SW[AP]
\end{verbatim}
\end{quote}
\newpage

\section {CLASSIFIED LIST OF PROGRAMS}

This appendix lists every documented ASPIC program and procedure grouped into 21
classes.
These correspond to topics in the HELP library and are arranged in alphabetical
order.

Some programs can be run either directly (by specifying the program name) or
indirectly (by specifying a procedure which runs the program).
In such cases the name of the program is made by following the name of the
procedure by the character `P'.
For example, {\bf INDPIX} is a procedure which runs program {\bf INDPIXP}.
In the list below, such procedure/program combinations are indicated by
following the procedure name by the character `*', eg.\ {\bf INDPIX*}.
\subsection {ARGS}
\begin{description}
\item [ABLINK]: Display two images on the ARGS and allow them to be registered
and blinked under trackerball control.
\item [ABLOCK]: Display a ramp on the ARGS.
\item [ACLEAR]: Clear the ARGS display.
\item [ADHC]: Automatic high-contrast display of an image.
\item [ADISOV]: Disable (thereby making the data invisible) an overlay plane in
the ARGS.
\item [ADISP]: Display an image on the ARGS with automatic scaling of intensity
values if required.
\item [ADISP3]: Variation of ADISP which displays in turn the z-planes in a
3D image.
\item [AEROV]: Erase ARGS overlays in a specified bit plane.
\item [AFLASH]: Faster version of ADISP because it does not do any intensity
scaling.
\item [AFRAME]: Display a frame and graticule round the latest image displayed
on the ARGS.
\item [AGBLINK]: Display two images on the ARGS; just like ABLINK except two
distinct lookup tables may be used, one with each image.
\item [ALIST]: List parameters for all images held in the ARGS database and
hence displayed on the ARGS by ASPIC programs.
\item [APAN]: Pan and zoom the ARGS display; return final position of cursor.
\item [APANG]: Pan and zoom the ARGS display with independent scaling in X and
Y; return final position of cursor.
\item [APIC]: Display an I*2 image on the ARGS and allow the scale and zero
point to be changed interactively to produce the best looking image.
It is faster than other display programs and allows quick investigation of the
range of intensity values.
Can pan and zoom.
\item [ARESET]: Reset all ARGS functions.
\item [ATEXT]: Place text on the ARGS display.
\item [AUCUR]: Allow use of a variable size and shape cursor.
\item [AZOOM]: Zoom the ARGS about a defined position.
\item [BLINKER]: Use after ABLINK to continue blinking with new zoom
factors and centre.
\item [CDEMO]: Demonstration of a square cursor; allows zoom of an ARGS image
so that a selected region fills the screen.
\item [DATIM]: Display current date and time as characters on the ARGS if
needed for recording.
\item [GBLINKER]: More general version of BLINKER which allows different
lookup tables to be used with the two images being blinked.
\item [GREYCELL]: Produce a grey scale representation of a Starlink image file
on raster graphics devices known to GKS 7.2.
Vector graphics devices, such as Tektronix 4010 emulators, cannot be used.
\item [ICBLINK]: Allow blinking of images with distinct look-up-tables using
integer images.
\item [ICDISP]: Variation on ADISP which specifically displays 16-bit images.
It also allows compression of an image so that it will fit on the ARGS screen.
\item [UNZOOM]: Set the ARGS zoom factors to 1 and 1 on the centre of the
screen.
\end{description}
\subsection {ARITHMETIC}
\begin{description}
\item [ADD]: Add two frames.
\item [ADDMSK]: Combine two images where effect is defined by 2 masks
created by POLIFILLA.
\item [BITMASK]: Logical AND of a frame with a scalar.
\item [CADD]: Add a scalar to a frame.
\item [CDIV]: Divide a frame by a scalar.
\item [CMULT]: Multiply a frame by a scalar.
\item [CPOW]: Raise each element of a frame to a power.
\item [CSUB]: Subtract a scalar from a frame.
\item [DIV]: Divide a frame by a second frame.
\item [DIVFF]: Divide a frame by a frame, preserving the scaling.
\item [EXP]: Exponentiate each element of a frame.
\item [FLIPMSK]: Interchange 0's and 1's in a mask image.
\item [LOG]: Natural logarithm of each element of a frame.
\item [MULT]: Multiply two frames, element by element.
\item [SQRT]: Replace each element by its square root.
\item [SUB]: Subtract the second frame from the first.
\end{description}
\subsection {ASTROMETRY}
\begin{description}
\item [CCIRD]: Set the transformation coefficients calculated by CINVERT to the
input for RDTOXY.
\item [CFICI]: Set the transformation coefficients calculated by XYFIT to the
input for CINVERT.
\item [CFITR]: Set the transformation coefficients calculated by XYFIT to the
input for XYTRAN.
\item [CFIXY]: Set the transformation coefficients calculated by XYFIT to the
input for XYTORD.
\item [COORDS]: Allow up to 500 points in the current image displayed on the
ARGS to be identified using the cursor and stored as an output image.
\item [CURFIT]: Calculate the astrometric position and photometric parameters
for a stellar image identified using the box-cursor on the ARGS.
\item [RDKEY]: Keyboard input of a list of right ascensions and declinations.
\item [RDLIST]: Type/print a stored list of right ascensions and declinations.
\item [RDTOXY]: Convert a list of right ascensions and declinations into crosses
superimposed on an image displayed on the ARGS.
\item [STARXY]: Measure astrometric (X,Y) positions of user selected stars in an
image and save them in a file.
\item [TESTFIT]: Calculate the astrometric position and photometric parameters
of a stellar image for which approximate coordinates are input by the user.
\item [XYTORD]: Convert cursor defined positions on the ARGS into a list of
right ascensions and declinations.
\end{description}
\subsection {COMPRESS}
\begin{description}
\item [CONFLEV]: Rescale an image using the background level and noise variance.
\item [PACK]: Pack an image which has been scaled from 0 to 1 (thresholded) or
0 to 3 (confidence levels) by CONFLEV.
\item [UNPACK]: Unpack images created by PACK (PACK and UNPACK may be used to
compress images for transmission over the network).
\end{description}
\subsection {DETAILS}
\begin{description}
\item [CCDCON]: Change the dimensions of an image by resetting the values of
the NAXIS1 and NAXIS2 descriptors.
\item [COG]: Show the centre of gravity in (X,Y) of an image.
\item [CURVAL]: Display the coordinates and value of a cursor selected pixel
within an image.
\item [CYDSCR]: Copy a complete frame descriptor to another frame.
\item [DESCR]: Show one or all of the descriptors of a frame.
\item [HIST*]: Histogram a patch of an array.
\item [HISTPLOT]: 1D histogram plot of a Starlink image.
\item [IMSIZE]: Determine the dimensionality and size of a frame.
\item [INSPECT]: Perform similar functions to PEEP, WRHIST and STATS on several
selected regions of an image.
\item [ISLICE]: The equivalent of SLICE after use of ICDISP.
\item [MEAN*]: Calculate the mean from a patch of an array.
\item [MEDBOX*]: Calculate the median from a patch of an array.
\item [MSLICE]: Variant of SLICE which allows multiple slices to be obtained
but which stores none of them.
\item [MULCON]: Produce a contour map of a frame, or a series of contour maps
of selected regions of a frame.
Output files produced are suitable for a Versatec.
\item [SECTOR]: Define, display and store the radial profile in a cursor-defined
sector of an image displayed on the ARGS.
\item [SLICE]: Define, display and store a 1D cursor-defined slice through an
image displayed on the ARGS.
\item [STAR]: Find location, size and intensity of a star assumed to occupy
most of an image.
\item [STATS]: Show dimensions, range of values and so on of a frame.
\item [TBXY]: Return the location of a cursor-defined point in an image
displayed on the ARGS.
\item [TBXY2]: Return the location of a pair of cursor-defined points in an
image displayed on the ARGS.
\item [WRDSCR]: Write new items as descriptors into an existing frame.
\item [WRHIST]: Form, store and/or write out the histogram of a frame.
\end{description}
\subsection {DISPLAY}
\begin{description}
\item [AHARDCOPY]: Copy the current ARGS pixel store onto the Printronix
lineprinter.
\item [ANNOTASP]: Annotate an image displayed on the ARGS with text strings,
arrows or scale-length bars.
\item [APLOT*]: Plot an image on the ARGS with suitable scaling so that the whole
image fits onto the whole screen.
\item [APLOTQ]: Similar to APLOT but fits the image into one quadrant of the
ARGS screen.
\item [APLOTRNG]: Plot an intensity wedge on the ARGS with a scale.
\item [BONW]: Set the ARGS background to white with black lines when using the
ARGS in vector rather than image mode.
\item [CLRQUAD]: Clear a given quadrant on the ARGS.
\item [CONTOUR]: Draw a contour map of an image on one of several devices.
\item [DOODLE]: Create images on the ARGS suitable for photographing to make
slides.
Allows images, text, lines and arrows to be displayed in a flexible way to
convert a {\em picture} into a {\em diagram}.
\item [GREYLASER]: Produce a grey-scale representation of an image file on
a Canon LBP-8 A2 laser printer (see SUN/24).
\item [GREYSCALE]: Extended version of VERGREY which also produces greyscale
pictures on the Printronix printers.
\item [HIDE]: Draw various forms of hidden-line plot.
\item [LIMITS]: Draw a graph of a 1D frame and ask for cursor selection of two
points.
\item [LINPLOT]: Draw the graph of a 1D frame on one of several devices.
\item [LIST]: List part of a frame.
\item [MOVIE]: Display a movie made up of 1D plots of successive rows of a
2D frame.
Plot up to 500 pixels in the X-direction with a delay between each row
controlled by the trackerball.
\item [PEEP]: Type a 9x9 section of an image.
\item [PLOTQTH]: Threshold an image and plot it on an ARGS quadrant.
\item [THRESH]: Define a threshold value using the ARGS trackerball; the effect
is visible on the ARGS.
\item [VERGREY]: Generate pseudo-greyscale output for the Versatec.
\item [WONB]: Set the ARGS to white lines on a black background when it is used
in vector mode.
\end{description}
\subsection {DOCS}
\begin{description}
\item [CREDOC]: Guide user through the creation of program documentation
(calls the program DOCIT).
\item [EDDOC]: Allow user to edit program documentation before it is put into a
program.
\item [FILEDOC]: Generate the HELP available on any topic in a form
suitable for printing.
\item [PUTDOC]: Put a documentation file created by CREDOC into the source of
the program in the form of `C+' comments (calls program STUFF and the command
procedure EDITIT).
\item [UNDOC]: Remove the documentation from the start of a program if major
changes are required (calls the program STRIP).
\end{description}
\subsection {EDRS}
\begin{description}
\item [ARGPIC]: Display images on the ARGS with autoscaling and choice of
screen position. Can also display images on a Digisolve IKON display.
\item [ARITH]: Scalar arithmetic.
\item [AVERAGE]: Average all valid regions of an image.
\item [BDFKEY]: Read image data from a terminal.
\item [BIN]: Average an image using rectangular bins.
\item [BLANK]: Blank out a range of lines or columns in an image.
\item [BOXFILTER]: Local averaging of an image.
\item [CENTROID]: Find the centre of fiducial marks.
\item [CINVERT]: Invert a bi-linear transformation.
\item [COLLAPSE]: Produce a 1D image by collapsing a polygonal section of an
image along the x or y axis.
\item [CONT]: Plot contour map of an image leaving areas containing invalid
pixels blank.
\item [CUT]: Apply upper and lower intensity cuts to an image.
\item [DESCRIPT]: Make image descriptor items available as program parameters.
\item [FFCLEAN]: Remove blemishes from a flat image.
\item [FILTSPEC]: Calculate resultant spectral energy distribution when a source
with a known continuum spectrum is observed through combinations of optical
components each of which has its own spectral transmission function.
\item [IMGARITH]: Image arithmetic.
\item [ITFCORR]: Apply intensity transfer function correction to an image.
\item [ITFGEN]: Generate an intensity transfer function.
\item [ITFPLOT]: Plot an intensity transfer function.
\item [ITOR]: Convert an I*2 format image containing a scale factor and zero
level in its descriptor into a R*4 image.
\item [LINEFIT]: Fit linear functions to each line in an image.
\item [MASK]: Mask one image with another so that the first image shows where
it is valid but the second image shows through in the regions where the first
is invalid.
\item [MATHS]: Apply arithmetic and mathematical functions to input images and
constants.
\item [NOPROMPT]: Cancel automatic prompting mode set by PROMPT.
\item [NORMALIZE]: Calculate the scaling required between two images of the
same object with different exposure times so that they have (on average) the
same data values.
\item [NSTACK]: Stack images to improve signal/noise and remove blemishes.
\item [PIXFILL]: Replace invalid regions of an image with a smooth function.
\item [PIXMAP]: Move the pixels in an image to new positions so as to put them
in a 2D set of bins.
\item [PIXUNMAP]: Generalized resampling of a 1D or 2D image at a set of points
defined by specified coordinate images.
\item [PROFILE]: Extract and plot sections through images.
\item [PROMPT]: Activate automatic prompting.
\item [RESAMPLE]: Resample an image according to a bi-linear position
transformation.
\item [RESCALE]: Alter the scale and zero descriptors and reset the invalid
pixel flag.
\item [SEGMENT]: Copy polygonal segments of one image to another.
\item [STARFIT]: Determine the best profile parameters for the stars in an
image.
\item [STARMAG]: Perform stellar photometry by fitting a model star profile.
\item [SURFIT]: Produce image defined by a polynomial or bi-cubic spline surface
which is a least-squares fit to an input image.
\item [TRCONCAT]: Calculate the combined effect of applying two successive
6-parameter linear position transformations to a set of (X,Y) coordinates.
\item [TRIM]: Change the size of an image by selecting a square or rectangular
region to copy into another image.
\item [VIEW]: Produce a file suitable for printing containing an array of the
integers stored in a square section of an image.
\item [XYCOEFF]: Generate a set of position transformation coefficients from a
specified shift, rotation and magnification.
\item [XYCUR]: Create a file of (X,Y) positions from an image displayed
on an ARGS or IKON.
\item [XYCURA]: Extended version of XYCUR.
\item [XYFIT]: Fit a bi-linear transformation between two sets of positions.
\item [XYKEY]: Enter and store a list of (X,Y) positions from a keyboard.
\item [XYKEYA]: Like XYKEY with more extended facilities.
\item [XYLIST]: Type a set of (X,Y) positions.
\item [XYLISTA]: Type up to 20 parameters in the same format as XYLIST.
\item [XYTRAN]: Apply a linear transformation to a set of (X,Y) positions.
\end{description}
\subsection {EDRSX}
\begin{description}
\item[BDFGEN]: Produces a BDF image or XY list from data stored in a text file.
A wide range of text file formats can be handled.
\item[CONVOLVE]: Convolves an image with a point spread function defined by
another image.
N.B.\ for large images this program is {\em very} slow; use procedure BIGCONV
for such images.
\item[CRDDBLUR]: Performs in-scan smoothing of CRDD using the in-scan
point spread function of a specified IRAS band.
\item[CRDDSAMPLE]: Subtracts linear baselines from a CRDD file to make the
data consistent with a given image.
\item[CRDDTRACE]: Graphically displays data streams from individual IRAS
detectors.
\item[DATARANGE]: Displays various statistics of an image, including cumulative
histogram points, and makes them available for use by other programs.
\item[DRAWSCAN]: Draws a coloured box around a single IRAS scan which forms
part of a mosiaced image displayed on an ARGS or IKON.
\item[EDRSIN]: Converts a 2d image stored in any BDF data format into an image
usable by EDRS and EDRSX, and ensures all the descriptors are set up properly.
\item[FIXINVAL]: Sets all invalid (or blank) pixels in an image, to a given
valid value.
\item[FOURIER]: Produces the Fourier transform of an image.
\item[HISTOGRAM]: Plots a histogram of image pixel values from an image on any
GKS 7.2 device.
\item[IMGEDIT]: Allows inspection and alteration of individual pixels within
an image, and creation of images containing data typed in by the user.
\item[IMGSTACK]: Stacks several pre-aligned images together with optional
weighting.
Precisely the same as the EDRS program NSTACK except that high data values are
never saturated.
\item[IRASBACK]: Produces a ``background'' image onto which IRAS images can be
automatically ``pasted'', in such a way that the resulting mosaic fits exactly
onto the background image without any bits being lost over the edge.
\item[IRASCOEF]: Calculates the linear transformation coefficients required to
align two IRAS images.
The coefficients produced can be passed on to program IRASSHIFT to shift one of
the images into alignment with the other.
\item[IRASCORR]: Corrects image descriptors to take into account a linear
transformation of pixel positions.
\item[IRASDSCR]: Given the RA and DEC of two pixels in a non-IRAS image,
calculates the descriptors necessary for IRAS programs to position the image on
the sky.
Assumes the image pixels are square.
\item[IRASIN]: Takes an EDRS image with IRAS descriptors and produces an image
which can be used with I\_PICV or I\_CONTOUR.
\item[IRASSHIFT]: Resamples an IRAS image so as to move the features to
different locations within the image.
The shift is described by 6 coefficients as produced by IRASCOEF for instance.
Descriptors defining the position of the image on the sky are updated to take
account of the shift.
\item[ITFHIST]: Allows an image to be modified so that its histogram of pixel
values matches that of a reference image as closely as possible.
\item[MATCHBACK]: Subtracts offsets from overlapping images to minimise
the difference between mean data value in the overlapp regions.
\item[MEMCRDD]: Maps multiple CRDD files using a Maximum Entropy Method.
\item[NDFOUT]: Converts an image from an EDRS .BDF file to an NDF .SDF file.
\item[RADEC]: Converts a list of pixel co-ordinates to RA and DEC, or
vice-versa.
\item[SCATTER]: Produces a scatter plot of pixel values in one image
against those in another.
\item[SOURCEFIT]: Fits Gaussian-like sources to a set of merged features in an
image.
\item[XYPLOT]: Plots positions held in an XY list on an image overlay device.
\end{description}
In addition to these, there are several DCSL command procedures which use EDRSX
and EDRS program, as examples of what can be done with these packages.
These procedures use the features of DSCL and DCL to run programs automatically
according to a preset recipe (see SUN/74).
The user should look at these procedures for examples of how to use EDRS and
EDRSX.
They are in files with file extension .SCL and in directory EDRSX (e.g.\
EDRSX:DESTRIPE.SCL):
\begin{description}
\item [BIGCONV]: Convolves two images by multiplying their Fourier transforms.
{\em Much} quicker than program CONVOLVE when using large images.
\item [BLINK]: Blinks between two images displayed on an ARGS allowing one to
be moved relative to the other while blinking is in progress.
\item [DESTRIPE]: For destriping IRAS images (see cookbook help on destriping
for more information).
\item [IRASSTACK]: Aligns and stacks separate IRAS scans.
\item [MASKGEN]: Used to create masks for destriping IRAS images.
\item [MEMASK]: Creates a mask for use with the ANALYSE facility in program
MEMCRDD.
\item [SKYSUB]: Calculates a polynomial background to an image and subtracts
it off the image.
\item [WEIGHTGEN]: Used to create weight images for destriping IRAS images.
\end{description}
The following utilities also exist :
\begin{description}
\item[BATCH]: Submits a DSCL command to run in batch with ALL currently defined
logical names and symbols copied over from the current process to the batch job.
The DSCL command thus runs in batch just as it would at the terminal.
\item[ZAPTAB]: Removes tabs from a text file replacing them with sufficient
spaces to maintain the same appearance.
Any tab positions can be set up but by default EDT tab spacing is assumed.
This program can be run from DCL using the RUN command as well as from DSCL.
\end{description}
\subsection {FC}
\begin{description}
\item [FC]: Perform the `standard' false colour and colour enhancement
operations in sequence.
\item [FCDISP]: Display a false colour image with the correct LUT and zoom.
\item [FCPACK]: Form a single false colour image from three (R,G,B) input
images.
\item [FCSAT]: Perform enhancement on the `saturation' image.
\item [FCSCALE]: Allow rescaling of three input images to a common background
and exposure.
\item [FCTHSI]: Convert a set of R,G,B images to a set of H,S,I (Hue,
Saturation, Intensity) images.
\item [FCTRGB]: Convert a set of H,S,I images back to R,G,B.
\item [LOG]: Logarithmic intensity enhancement.
\item [TWOTONE]: Take two 2D Starlink images and form a single coded
`pseudo-colour' image.
\item [VELINT]: Form a composite image from a velocity and an intensity image.
\item [VIDISP]: Display an image created by VELINT as a colour coded velocity
map but with intensities represented as well.
\end{description}
\subsection {FILTER}
\begin{description}
\item [BLURR]: Generate a 5x5 frame containing a Gaussian profile.
\item [CONV]: Convolve an image with a second (smaller) one as may have been
generated by BLURR or PSFEST.
\item [FINULS]: Filter image noise using local statistics.
\item [LAPLACE]: Form the difference between an image and some multiple of its
Laplacian.
\item [MEDIAN]: Apply a median filter at each point in an image.
\item [MEM]: Maximum entropy deconvolution program for 2D images.
\item [MODAL]: Apply a modal filter to discrete boxes in an image and then use
linear interpolation for intermediate values.
\item [MODE]: Apply a modal filter at each point of an image.
\item [NITPIK]: Remove small defects from an image.
\item [NORMFILT]: Apply a convolution and normalization to an image as an aid to
star detection when using a matched filter.
\item [ODDHDC1]: Apply an odd-order hierarchical convolution to a 1D image.
\item [ODDHDC2]: Apply an odd-order hierarchical convolution to a 2D image.
\item [RHDC1]: Apply a reduced odd-order hierarchical convolution to a 1D
image.
\item [RHDC2]: Apply a reduced odd-order hierarchical convolution to a 2D
image.
\item [SMOOTH]: Smooth an image by a Gaussian or top hat convolution.
\item [USMASK]: Form an `un-sharp' masked image by subtracting a smoothed
version of an image from the original.
\end{description}
\subsection {FOURIER}
\begin{description}
\item [COSBELL]: Apply a cosine-bell function to a 1D or 2D image.
\item [DFFTASP]: Efficient 2D FFT program.
\item [FFT]: Apply a fast Fourier transform to a 1D or 2D image.
\item [FILDEF]: Create a 1D or axi-symmetric 2D filter.
\item [FTCONJ]: Complex conjugation of a Hermitian transform.
\item [PSFEST]: Point-spread-function (star profile) from an image.
\item [PSPEC]: Power spectrum from the Fourier transform.
\end{description}
\subsection {GALPHOT}
\begin{description}
\item [ABOXASP]: Define and save a series of rectangular boxes drawn round
selected features in an image.
\item [BOXASP]: Define a single rectangular box round a selected feature in an
image.
\item [CNTEXT*]: Store details of contours found in a predefined region of an
image.
They can then be used for fitting ellipses.
\item [COLDIF]: Generate a colour difference profile from two profiles
extracted along the same axis of a nebular image in different colours.
\item [ELLPLTP]: Plot previously found contours with superimposed ellipses
which have been fitted using FITELLP.
\item [ELLPRTP]: Print details of fitted ellipses.
\item [EQPROFASP]: Compute equivalent profile, with associated photometric
parameters, of a bright galaxy within an image in the system of de Vaucouleurs.
\item [FITBAKASP]: Fit a polynomial to the background and divide it into the
original image giving normalized intensities.
\item [FITELLP]: Fit an ellipse to a set of extracted contours and store them
for input to ELLPLTP and ELLPRTP.
\item [JONESASP]: Smoothing program which uses different degrees of smoothing
for the central parts of a galaxy and the fainter outer parts.
\item [MAGCNTASP]: Produce absolutely calibrated contours in magnitudes per
square arcsec.
\item [PARPLTP]: Extract any two parameters from the ellipses which have been
fitted by FITELLP and plot one against the other.
\item [PECALBASP]: Absolute photometric calibration of a 2D image which is held
as intensity normalized to a sky background of 1.0.
\item [PRFDECASP]: Determine the parameters of the bulge and disk components in
the observed profile of a galaxy.
\item [PRFLOGASP]: Convert a profile held as normalized intensities into log of
intensity above the sky.
\item [PRFPLTASP]: Plot an extracted galaxy profile.
\item [PRFPRTASP]: Produce a printed listing of a fitted profile and its
associated parameters.
\item [PRFVISASP]: Visual comparison of an extracted profile with profiles
computed from a model with fixed parameters.
\item [PRIAXEASP]: Extract the profiles along two orthogonal axes centred on a
galaxy.
\item [STAREMASP]: Interactively remove stars and blemishes from an image.
\end{description}
\subsection {GEOMETRY}
\begin{description}
\item [FLIP]: Invert an image with respect to a horizontal or vertical axis.
\item [MOVE]: Move an image by a non-integral shift in X and Y.
\item [ROT3D]: Rotate the axes of a 3D image from (X,Y,Z) to (Z,Y,X).
Designed for use in conjunction with STK23 to convert a set of 2 spectra (such
as long slit IPCS) into a TAURUS-compatible cube.
\item [STK23]: Combine a set of 2D images into a 3-D cube.
\end{description}
\subsection {GRASP}
\begin{description}
\item [ALIGN]: Align an image with a reference image.
\item [APERCUR]: Aperture photometry using a circular aperture positioned on an
ARGS image by the cursor.
\item [APERFOT]: Simple aperture photometry.
\item [APERMAG]: Aperture photometry of all the stars in a list of positions.
\item [ASXY]: Calculate mean, standard deviation, normalized standard deviation
and number of valid pixels in a selected area of an image.
\item [BATFLAT]: A procedure which runs FLAT as a batch job.
\item [BATIMSTAC]: A procedure which runs IMSTACK as a batch job.
\item [BATLORMUL]: A procedure which runs LORMUL as a batch job.
\item [BATLORSIM]: A procedure which runs LORSIM as a batch job.
\item [BATPDSIM]: A procedure which runs PDSIM as a batch job.
\item [FLAT]: A procedure which runs FLATTEN.
\item [FLATTEN]: Flat fields an image using an un-aligned flat field and a
co-ordinate transformation file to allow for the misalignment.
\item [GAUFIT]: Estimate the Gaussian profile that best fits the stars in an
image.
\item [GAUMAG]: Estimate the magnitudes of stars in an image by reading an
(X,Y) list of positions and fitting a full 2D Gaussian profile to each star.
\item [ICBLINK]: A variation on BLINK which blinks between two 16-bit images.
The contrast may be altered at any time.
\item [ICDISP]: A variation of ADISP which displays a 16-bit image but also
allows several optional scalings to be performed.
\item [ICFLASH]: Display an image on the ARGS, scaling it around sky level to
show faint features.
Compresses the image if a side is $>$ 512.
\item [IFLASH]: Display an image on the ARGS.
Does't do any intensity scaling so is much faster than ADISP.
If the image is $>$ 512 it is sampled down to less.
\item [IJOIN]: Join up to 25 2D images.
\item [IMANIC]: Select part of a 2D 16-bit image.
\item [IMERGE]: Merge up to 25 16-bit images.
\item [IMSTACK]: Stack images together which need not be aligned; it should be
used before PSTACK and IMGARITH are run, wherever possible.
\item [INSPECT]: Similar to STATS, but works on a selected region of a 16-bit
image.
\item [ISLICE]: Version of SLICE which works on 16-bit images.
\item [ITFGENA]: Generate a conversion table to correct for PDS and emulsion
non-linearity.
\item [LORCUR]: Measure cursor-selected stars by fitting full 2D Lorentz
profiles, including an option to handle overlapping stars.
\item [LORFIT]: Estimate the Lorentzian profile that best fits the stars in an
image.
\item [LORMUL]: Measure magnitudes of all stars in an (X,Y) list, including
possible overlapping stars, using a full 2D Lorentz profile.
\item [LORSIM]: Measure magnitudes of stars by fitting full 2D Lorentz profiles
to isolated stars whose positions are given in an (X,Y) list.
\item [MAGAV]: Average the magnitudes found by any of the GRASP programs,
rejecting stars which have poor fit parameters.
\item [MAGCOR]: Correct the magnitudes found by any of the GRASP programs for
the effects of measuring machine aperture density averaging.
\item [MAGDIAG]: Combine several magnitude lists, apply colour equations and
zero points, then plot and store colour-magnitude or colour-colour diagrams.
\item [MAGRMS]: Plot the run of RMS error with magnitude for the results from
any of these photometric programs.
\item [PDSCOR]: Get clear areas measured by a segmented PDS raster and work
out PDS drift.
Fill 1D array with a {\em clear} value for each line of the main raster.
Display the levels of the reference scans and the fit and run of the mode of
the main scan values which are near the central scan values image.
\item [PDSIM]: Take the PDS main image and reference area scans from PDSMULTI,
estimate the variable zero level, subtract it and correct for PDS and emulsion
non-linearities.
\item [PDSMULTI]: Take a PDS tape created by the MULTI-mode and produce Starlink
images.
\item [PDSRASTER]: Read a nine track PDS tape of rasters written in the
Scansalot format from the Forth system and create a Starlink image.
\item [PDSRIPPLE]: Eliminate the zero-point ripple on PDS produced images.
\item [PRAPERCUR]: Format output from APERCUR for printing.
\item [PRAPERMAG]: Format output from APERMAG for printing.
\item [PRGAUMAG]: Format output from GAUMAG for printing.
\item [PRLORCUR]: Format output from LORCUR for printing.
\item [PRLORMUL]: Format output from LORMUL for printing.
\item [PRLORSIM]: Format output from LORSIM for printing.
\item [PRMAGS]: Format an EDRS-type list of positions and magnitudes for
printing.
\item [PSTACK]: Add images which are mutually aligned and normalized; it is a
variation of NSTACK.
\item [PUTSTAR]: Create a realistic artificial star and galaxy field containing
profiles and noise which emulate electronic or photographic detectors.
\item [SATCOR]: Apply the electronograph saturation law to an image or
de-saturate an image.
\item [XYARITH]: Make a new (X,Y) list by performing linear arithmetic
operations on any of the parameters stored in two existing (X,Y) lists.
\item [XYCHART]: Plot a realistic looking star map; star identifications may be
put by the stars.
\item [XYCURA]: Create a new list of (X,Y) positions from part or all of an
existing list using the ARGS cursor.
\item [XYCURB]: Create a new list of (X,Y) positions by using the ARGS cursor to
correct the positions in an existing list.
\item [XYCUT]: Extract a portion of an (X,Y) list and write it to a new file.
You select the section of columns to be extracted.
Identifiers and headers are copied over correctly.
\item [XYDRAW]: Draw a graph of the first two parameters (usually X and Y) in
any (X,Y) list.
\item [XYDRAWA]: Draw from (X,Y) lists.
The graph is a smooth curve (using a cubic spline) joining the points.
\item [XYEDIT]: Convert the normal (binary) version of an (X,Y) list into an
ASCII version, and vice versa, so that (X,Y) lists may be created or edited
using one of the standard editors.
\item [XYFITA]: Generate the coefficients for a linear transformation between
one set of (X,Y) positions and another.
\item [XYJOIN]: Join together two EDRS (X,Y) files.
\item [XYKEYA]: Extended version of the EDRS program XYKEY allowing keyboard
entry of items to an (X,Y) list.
\item [XYLISTA]: Create a printable listing of any (X,Y) list in F12.4 format.
\item [XYMATCH]: Take two (X,Y) lists and create two new files containing
either the stars from each list which have corresponding identifiers, sorted
in the same order, or the entries which have no corresponding identifier.
\item [XYMULTA]: Take two (X,Y) lists and makes a third by copying the first
one, replacing the second parameter with the multiple of the second parameter
of the first and second files.
Suitable for multiplying two response functions.
\item [XYPMATCH]: Take two (X,Y) lists and make two new ones which are copies
of the old ones but only contain entries which match in (X,Y) position.
\item [XYPRNT]: Make a neat listing of an (X,Y) file.
\item [XYRENUM]: Take an (X,Y) file and renumber the identifiers from 1 to N.
\item [XYSORT]: Sort an (X,Y) list into ascending or descending order of one of
the parameters contained in it.
\item [XYSTAT]: Find statistics for 1 or 2 sets of numbers taken from (X,Y)
lists.
\item [XYWEED]: Take an (X,Y) list and extract those entries which have values
for one of the parameters present inside or outside a defined range.
\item [XYWEEDA]: Take an (X,Y) list and extract those entries whose (X,Y) lie
in a pixel of the reference image which has a non-zero value.
\end{description}
\subsection {INPUT}
\begin{description}
\item [FITSIN]: Convert FITS data on magnetic tape into Starlink frames.
\item [IPCSIN]: Convert IPCS data on magnetic tape into Starlink frames.
(see SUN/3).
\item [IPOLYGON]: A 16 bit integer version of POLYFILLA.
\item [MANYG]: Create a 2D test image containing a background, as many Gaussian
images as required, and some noise.
\item [PDSIN]: Convert PDS data on magnetic tape into Starlink frames.
\item [POLYFILLA]: Cursor definition on the ARGS of a new image made up from
one or more polygons.
The interiors are set to 1, the exteriors to 0.
\item [TOSTAR]: Convert RGODR-format images into Starlink frames.
\item [TYPEIN]: Keyboard input to a 1D frame.
\end{description}
\subsection {LUT}
\begin{description}
\item [COLCYCLE]: Generate a colour table consisting of cyclic replicas of an
original colour table.
\item [COLSEL]: Select colours from a pre-defined palette; a standard palette
is supplied by default.
\item [COLSLICE]: Generate a monochromatic ramp colour table.
\item [E2DCOL]: Generate the default E2D colour table.
\item [HEATCOL]: Generate a stepped pseudo {\em heat sequence} colour table.
\item [HEATCON]: Generate a continous pseudo {\em heat sequence} colour table.
\item [HSICOL]: User definition of a palette for input to COLSEL if the default
is not acceptable.
\item [INVARG]: Invert current colour table on the ARGS.
\item [INVCOL]: Invert colour table held as a {\em BDF} file.
\item [LUTCOL]: Write a standard colour LUT to the ARGS.
\item [LUTCONT]: Fill the ARGS colour table with discrete values to give a
contour-like display.
\item [LUTE]: Interactive tuning of a LUT using the push-buttons.
\item [LUTFC]: Load the `false-colour' LUT (as used in the opening
demonstration at RAL).
\item [LUTGREY]: Load the standard grey LUT.
\item [LUTLIN]: Interactive manipulation of a linear (or logarithmic) LUT using
the ARGS cursor.
\item [LUTREAD]: Read any pre-defined LUT and write it to the ARGS.
\item [LUTROT]: Cycle a LUT using a trackerball.
\item [LUTSET]: Fill part of the ARGS colour table with values linearly
interpolated between two RGB sets.
\item [LUTSTORE]: Store the look-up-table currently held in the ARGS as a
Starlink frame.
It can subsequently be read back into the ARGS using LUTREAD.
\item [LUTTWEAK]: Interactive manipulation of a LUT; three different LUTs are
available.
\item [RINGCOL]: Generate a continuous purple, blue, green, red, purple colour
table.
\item [SPECOL]: Generate a continuous blue, green, red colour table.
\item [SWEEP]: Rotate the current ARGS colour table.
\item [VARGREY]: Generate a grey colour table with user defined end points.
\item [ZEBRA]: Generate a {\em pseudo contour} colour table.
\end{description}
\subsection {MISCELLANEOUS}
\begin{description}
\item [ASAVE]: Save the current ARGS pixel store in a BDF file.
\item [CALHELP]: Short help program for {\em CALIBRATION} package.
\item [CALIB]: Convert an image using calibrations in table LOOKUP generated
by programs in {\em CALIBRATION} package.
\item [CATCOPY]: Copy part of the Solar Atlas from tape to disk.
\item [COLFIX*]: Change a specified column of an image to a given value.
\item [COSFIT]: Produce COSMOS intensity conversion lookup table LOOKUP from
known conversions in LEVELS.
\item [DISKFIL*]: Remove a disk-shaped area from an FFT modes image.
\item [EDITLEV]: Edit existing LEVELS file interactively.
\item [EDITTAB]: Edit existing TABLE file interactively.
\item [ENTERTAB]: Enter 2D table of reals from keyboard.
\item [FITSHEAD]: Extract header from standard FITS format tape.
\item [FITSOUT]: Write standard FITS format tape.
\item [FRINGE]: Remove the fringe pattern which is characteristic of some CCD
images.
\item [GETLEV]: Enter known calibration levels by ARGS box cursor on a
stepwedge frame.
\item [GRID]: Incorporate a black and white grid into an image.
\item [HIFREQ*]: Eliminate high frequency components from an FFT mode image
produced by DFFTASP.
\item [HISTMATCH]: Rescale an image so that its histogram has a prescribed form.
\item [IAMEDRS]: Convert a file of IAM parameterised data to EDRS (X,Y) list
format.
\item [INCARN]: Extract a specified incarnation from an image.
\item [INDPIX*]: Change an indicated pixel to a defined value.
\item [LINCONU]: Read a Starlink image (as unsigned 16 bit integers) together
with the descriptor items BSCALE and BZERO and convert the data into REAL
values using: \newline OUTPUT=INPUT*BSCALE+BZERO.
\item [LINCONV]: Read a Starlink image (as 16 bit integers) together with the
descriptor items BSCALE and BZERO and convert the data into REAL values
using:\newline OUTPUT=INPUT*BSCALE+BZERO.
\item [LINFIX*]: Change an indicated row of an image to a defined value
(cf.\ COLFIX).
\item [LISTLEV]: List LEVELS file on lineprinter.
\item [LOFREQ*]: Eliminate low frequencies from an FFT mode image produced by
DFFTASP.
\item [LSEE]: Simple to use image display package (now almost never used).
\item [MERGE]: Patch several images together (with or without overlap) to form
a mosaic.
\item [MODHELP]: Summarize the functions of the pixel modification programs
in E2D (formerly known as MODPIXASP).
\item [OUTSET*]: Set all pixels in an image outside a circle to a defined value.
\item [PARDIF]: Crude estimate of the partial derivative of an image.
\item [PATCH]: Replace selected regions of an ARGS displayed image with a
smooth or noisy patch.
\item [PCT]: Generate the principle component transformation on a set of up to 4
frames.
\item [POLFIT]: Produce intensity conversion table LOOKUP from known conversions
in LEVELS using a polynomial fit.
\item [PRINTTAB]: Print 2D table on lineprinter.
\item [REALFR]: Tidy up an image so that only the REAL incarnation is left (it
needs some work space within your quota).
\item [RECFIL*]: Set pixels inside a selected rectangle to a defined value.
\item [RTOI]: Tiny procedure which may be of use within procedures to
convert parameters from REAL (as written by WRKEY) into INTEGER.
It also sets a global parameter.
\item [SETLEV]: Produce calibration LEVELS file by keyboard entry.
\item [SIMPAR]: Generate control parameter file for STARSIMP.
\item [SINCFIL*]: Multiply an FFT image by a {\em sinc} function.
\item [SLAFIT]: Generate intensity conversion table LOOKUP from LEVELS using
a slalom fit devised by John Cooke.
\item [SPLFIT]: Generate intensity conversion table LOOKUP from LEVELS using a
spline fit.
\item [STARSIM*]: Generate a frame containing  an artificial star field with
user defined parameters.
\item [SYMDIS*]: Remove 4 symmetrically placed disk-shaped areas from an FFT
image.
\item [TARGET]: Overlay markers showing the brightest stars in an artificial
image on top of the image displayed on the ARGS.
\item [THRESHOLD]: Zeroise values of an image which lie outside specified
lower and upper limits.
\item [TYPETAB]: Display 2D table on terminal.
\item [WAVEGET]: Copy part of the Solar Spectrum Atlas into a Starlink image.
\item [ZAPLIN]: Replace a vertical or horizontal strip in an image to remove
defects.
\end{description}
\subsection {PERIODS}
\begin{description}
\item [CONCAT]: Concatenate two or more input datasets.
\item [DETREND]: Fit a polynomial to the raw observations to remove any trends;
reduce the mean value to zero.
\item [EXTEND]: Unconstrained extrapolation from known data into the unknown
(dangerous).
\item [EXTRACT]: Select part of a dataset by extracting all samples whose epochs
lie between two given values.
\item [FFTPOW]: Compute the power spectrum of data sampled at equal intervals.
\item [FILLGAP]: Fill gaps between datasets which are themselves regularly
sampled and with gaps which are an integral number of samples wide using an
auto-regressive model based on the whole of the data.
\item [FLAG]: Flag points which are to be removed from a dataset using a graphics
cursor.
\item [FOLD]: Fold the data at a given period, find a mean curve and subtract
it from the data.
\item [FREQ]: Set parameter GLOBAL\_FREQ to mean frequency.
\item [LIN]: Estimate amplitude and phase of up to 20 sine-waves using a
{\em linear} least squares fit.
\item [LISTPAR]: Display a formatted version of a parameter file on the
terminal.
\item [MEMPOW]: Compute the maximum entropy estimate of the {\em true} power
spectrum of up to 50 datasets.
\item [NON]: Non-linear least squares fit of a set of sine waves starting with
good initial extimates.
\item [PDM]: {\em Phase Dispersion Methods} --- several variations on the
{\em string} method for estimating the possible frequencies present on a
dataset are available.
\item [PERIOD]: Set parameter GLOBAL\_FREQ to mean period.
\item [PLOTDAT]: Plot datasets.
\item [PLOTFIT]: Plot data created from a parameter file to be plotted over a
dataset.
\item [PLOTPS]: Plot the power spectrum computed by POWER.
\item [POWER]: Compute (optionally) the power spectrum and/or the window
function of the photometric data.
\item [PW]: {\em Pre-Whitening} --- Remove the variation specified by a set of
parameters from an existing dataset.
\item [RESTORE]: Write an unformatted BDF file as a formatted DAT file.
\item [SELECT]: Select one or more subsets from a given dataset and store as
separate datasets.
\item [SPLEQ]: Resample a dataset at equal time intervals.
\item [STORE]: Convert part of a formatted DAT file into an unformatted BDF
file.
\item [SYNTH]: Generate sinusoidal data with user-defined periods and noise
levels at the same epochs as an input data set.
\end{description}
\subsection {PHOTOMETRY}
\begin{description}
\item [APERASP]: Perform all the tasks related to aperture photometry.
\item [ARGSCIR]: Display a circular cursor on the ARGS and return its position
in array units and size.
\item [ARGSCUR]: Display an ARGS cursor and return its position in array units.
\item [CLRLIM]: Clear histogram limits in IAMHIS, SETLIM and GETLIM.
\item [GAUSFIT]:  Produce astrometric positions and photometric parameters for
a number of stars for which approximate positions are known using a Gaussian
fit.
\item [GETLIM*]: Return results to IAMHISP and SETLIMP.
\item [GETPAR]: Set IAM parameter PAR using the cursor.
\item [GETSKY]: Set IAM parameter SKY in master connection file IAMANALP.CON.
\item [GETTHR]: Set IAM parameter THRLD in master connection file IAMANAL.CON.
\item [IAMANAL]: Analyse image and find all objects and their parameters, given
various thresholds.
\item [IAMHELP]: Short help listing for IAM suite of programs.
\item [IAMHIS*]: Compute intensity histogram of image and plot on given ARGS
quadrant.
\item [IAMPR]: Output image parameters from IAM.
\item [IGJOB]: Produce Versatec plot file of IAM images.
\item [IGPLOT]: Plot IAM image on ARGS quadrant.
\item [SETALL*]:  Change IAM parameters.
\item [SETAREA*]: Set IAM area cut.
\item [SETLIM*]: Set zero point and highest histogram bin value.
\item [SETMAG*]: Set IAM sky background.
\item [SETPER*]: Set threshold parameter of histogram.
\end{description}
\subsection {POLARIMETRY}
\begin{description}
\item [CVPLOT]: Produce annotated contour map of an image with corresponding
polarization vectors overlayed.
\item [POLAR]: Generate total polarization and polarization angle from sky
subtracted Q and U polarization frames.
\item [PRPLOT]: Produce plot of polarization vectors from total polarization
and polarization angle data.
\item [STOKES]: Generate Stokes parameters Q and U from total polarization
and polarization angle.
\end{description}
\subsection {SIZE}
\begin{description}
\item [CMPRS]: Compress an image by integer factors in X and Y.
\item [COMPAVE]: Compress an image by averaging adjacent elements.
\item [COMPICK]: Compress an image by selecting elements in a regular grid.
\item [EXPAND]: Expand part of an image by a {\em sinc} interpolation.
\item [IMANIC]: Integer variant of MANIC.
\item [MANIC]: M and N image conversion --- pick part of a 1D, 2D or 3D frame
and convert it into a 1D, 2D or 3D frame.
\item [MERGE]: Combine two or more images with weighting to form a mosaic.
\item [PICK]: Pick a 2D subset from a 2D image using the ARGS cursor {\em or} a
keyboard.
MANIC handles much more general cases.
\item [PIXDUPE]: Expand an image by pixel duplication.
\item [SQORST]: Either squashes or expands an `unknown' image into a defined
shape by performing bi-linear interpolation.
\end{description}
\newpage

\section {ALPHABETICAL LIST OF PROGRAMS}

Some programs and procedures exist in ASPDIR which are not mentioned in appendix
B since I can find no documentation for them (in particular they are not
mentioned in the on-line HELP).
Of these, the following appear to be programs:\\
\vspace{20mm}
{\scriptsize
\begin{tabbing}
BBBBBBBBB - 88xx\=BBBBBBBBB - 88xx\=BBBBBBBBB - 88xx\=BBBBBBBBB - 88xx\=BBBBBBBBB - 88xx\=\kill
ATAN\>DOCIT\>IDISP\>LUTHIST\>PLOTIT\>STUFF\\
DEPD\>HEXFILE\>INPD\>MYPD\>STRIP\>TAN
\end{tabbing}}
\vspace{20mm}
and the following appear to be things like tools and tests:\\
\vspace{20mm}
{\scriptsize
\begin{tabbing}
BBBBBBBBB - 88xx\=BBBBBBBBB - 88xx\=BBBBBBBBB - 88xx\=BBBBBBBBB - 88xx\=BBBBBBBBB - 88xx\=\kill
ASPTEST\>COPYDOC\>DEMO\>PROGDOC\>TEMPORARY
\end{tabbing}}
\vspace{20mm}
The 440 documented programs mentioned in appendix B are shown oveleaf in
alphabetical order together with the number of the subsection in which they are
described.

\newpage
\begin{center}
{\Large\bf ASPIC PROGRAMS}
\end{center}

{\scriptsize
\begin{tabbing}
BBBBBBBBB - 88xx\=BBBBBBBBB - 88xx\=BBBBBBBBB - 88xx\=BBBBBBBBB - 88xx\=BBBBBBBBB - 88xx\=\kill
 ABLINK - 1                              \>CNTEXT* - 13
   \>FLAG - 19                               \>
 LIN - 19                                \>PERIOD - 19
   \>STARFIT - 8                             \\
 ABLOCK - 1                              \>COG - 5
   \>FLAT - 15                               \>
 LINCONU - 18                            \>PICK - 22
   \>STARMAG - 8                             \\
 ABOXASP - 13                            \>COLCYCLE - 17
   \>FLATTEN - 15                            \>
 LINCONV - 18                            \>PIXDUPE - 22
   \>STARSIM* - 18                           \\
 ACLEAR - 1                              \>COLDIF - 13
   \>FLIP - 14                               \>
 LINEFIT - 8                             \>PIXFILL - 8
   \>STARXY - 3                              \\
 ADD - 2                                 \>COLFIX* - 18
   \>FLIPMSK - 2                             \>
 LINFIX* - 18                            \>PIXMAP - 8
   \>STATS - 5                               \\
 ADDMSK - 2                              \>COLLAPSE - 8
   \>FOLD - 19                               \>
 LINPLOT - 6                             \>PIXUNMAP - 8
   \>STK23 - 14                              \\
 ADHC - 1                                \>COLSEL - 17
   \>FOURIER - 9                             \>
 LIST - 6                                \>PLOTDAT - 19
   \>STOKES - 21                             \\
 ADISOV - 1                              \>COLSLICE - 17
   \>FREQ - 19                               \>
 LISTLEV - 18                            \>PLOTFIT - 19
   \>STORE - 19                              \\
 ADISP - 1                               \>COMPAVE - 22
   \>FRINGE - 18                             \>
 LISTPAR - 19                            \>PLOTPS - 19
   \>SUB - 2                                 \\
 ADISP3 - 1                              \>COMPICK - 22
   \>FTCONJ - 12                             \>
 LOFREQ* - 18                            \>PLOTQTH - 6
   \>SURFIT - 8                              \\
 AEROV - 1                               \>CONCAT - 19
   \>GAUFIT - 15                             \>
 LOG - 2,10                              \>POLAR - 21
   \>SWEEP - 17                              \\
 AFLASH - 1                              \>CONFLEV - 4
   \>GAUMAG - 15                             \>
 LORCUR - 15                             \>POLFIT - 18
   \>SYMDIS* - 18                            \\
 AFRAME - 1                              \>CONT - 8
   \>GAUSFIT - 20                            \>
 LORFIT - 15                             \>POLYFILLA - 16
   \>SYNTH - 19                              \\
 AGBLINK - 1                             \>CONTOUR - 6
   \>GBLINKER - 1                            \>
 LORMUL - 15                             \>POWER - 19
   \>TARGET - 18                             \\
 AHARDCOPY - 6                           \>CONV - 11
   \>GETLEV - 18                             \>
 LORSIM - 15                             \>PRAPERCUR - 15
   \>TBXY - 5                                \\
 ALIGN - 15                              \>CONVOLVE - 9
   \>GETLIM* - 20                            \>
 LSEE - 18                               \>PRAPERMAG - 15
   \>TBXY2 - 5                               \\
 ALIST - 1                               \>COORDS - 3
   \>GETPAR - 20                             \>
 LUTCOL - 17                             \>PRFDECASP - 13
   \>TESTFIT - 3                             \\
 ANNOTASP - 6                            \>COSBELL - 12
   \>GETSKY - 20                             \>
 LUTCONT - 17                            \>PRFLOGASP - 13
   \>THRESH - 6                              \\
 APAN - 1                                \>COSFIT - 18
   \>GETTHR - 20                             \>
 LUTE - 17                               \>PRFPLTASP - 13
   \>THRESHOLD - 18                          \\
 APANG - 1                               \>CPOW - 2
   \>GREYCELL - 1                            \>
 LUTFC - 17                              \>PRFPRTASP - 13
   \>TOSTAR - 16                             \\
 APERASP - 20                            \>CRDDTRACE - 9
   \>GREYSCALE - 6                           \>
 LUTGREY - 17                            \>PRFVISASP - 13
   \>TRCONCAT - 8                            \\
 APERCUR - 15                            \>CREDOC - 7
   \>GRID - 18                               \>
 LUTLIN -17                              \>PRGAUMAG - 15
   \>TRIM - 8                                \\
 APERFOT - 15                            \>CSUB - 2
   \>HEATCOL - 17                            \>
 LUTREAD - 17                            \>PRIAXEASP - 13
   \>TWOTONE - 10                            \\
 APERMAG - 15                            \>CURFIT - 3
   \>HEATCON - 17                            \>
 LUTROT - 17                             \>PRINTTAB - 18
   \>TYPEIN - 16                             \\
 APIC - 1                                \>CURVAL - 5
   \>HIDE - 6                                \>
 LUTSET - 17                             \>PRLORCUR - 15
   \>TYPETAB - 18                            \\
\end{tabbing}

\vspace{-8mm}

\begin{tabbing}
BBBBBBBBB - 88xx\=BBBBBBBBB - 88xx\=BBBBBBBBB - 88xx\=BBBBBBBBB - 88xx\=BBBBBBBBB - 88xx\=\kill
 APLOT* - 6                              \>CUT - 8
   \>HIFREQ* - 18                            \>
 LUTSTORE - 17                           \>PRLORMUL - 15
   \>UNDOC - 7                               \\
 APLOTQ - 6                              \>CVPLOT - 21
   \>HIST* - 5                               \>
 LUTTWEAK - 17                           \>PRLORSIM - 15
   \>UNPACK - 4                              \\
 APLOTRNG - 6                            \>CYDSCR - 5
   \>HISTMATCH - 18                          \>
 MAGAV - 15                              \>PRMAGS - 15
   \>UNZOOM - 1                              \\
 ARESET - 1                              \>DATARANGE - 9
   \>HISTOGRAM - 9                           \>
 MAGCNTASP - 13                          \>PROFILE - 8
   \>USMASK - 11                             \\
 ARGPIC - 8                              \>DATIM - 1
   \>HISTPLOT - 5                            \>
 MAGCOR - 15                             \>PROMPT - 8
   \>VARGREY - 17                            \\
 ARGSCIR - 20                            \>DESCR - 5
   \>HSICOL - 17                             \>
 MAGDIAG - 15                            \>PRPLOT - 21
   \>VELINT - 10                             \\
 ARGSCUR - 20                            \>DESCRIPT - 8
   \>IAMANAL - 20                            \>
 MAGRMS - 15                             \>PSFEST - 12
   \>VERGREY - 6                             \\
 ARITH - 8                               \>DESTRIPE - 9
   \>IAMEDRS - 18                            \>
 MANIC - 22                              \>PSPEC - 12
   \>VIDISP - 10                             \\
 ASAVE - 18                              \>DETREND - 19
   \>IAMHELP - 20                            \>
 MANYG - 16                              \>PSTACK - 15
   \>VIEW - 8                                \\
 ASXY - 15                               \>DFFTASP - 12
   \>IAMHIS* - 20                            \>
 MASK - 8                                \>PUTDOC - 7
   \>WAVEGET - 18                            \\
 ATEXT - 1                               \>DISKFIL* - 18
   \>IAMPR - 20                              \>
 MASKGEN - 9                             \>PUTSTAR - 15
   \>WEIGHTGEN - 9                           \\
 AUCUR - 1                               \>DIV - 2
   \>ICBLINK - 1,15                          \>
 MATHS - 8                               \>PW - 19
   \>WONB - 6                                \\
 AVERAGE - 8                             \>DIVFF - 2
   \>ICDISP - 1,15                           \>
 MEAN* - 5                               \>RADEC - 9
   \>WRDSCR - 5                              \\
 AZOOM - 1                               \>DOODLE - 6
   \>ICFLASH - 15                            \>
 MEDBOX* - 5                             \>RDKEY - 3
   \>WRHIST - 5                              \\
 BATCH - 9                               \>DRAWSCAN - 9
   \>IFLASH - 15                             \>
 MEDIAN - 11                             \>RDLIST - 3
   \>XYARITH - 15                            \\
 BATFLAT - 15                            \>E2DCOL - 17
   \>IGJOB - 20                              \>
 MEM - 11                                \>RDTOXY - 3
   \>XYCHART - 15                            \\
 BATIMSTAC - 15                          \>EDDOC - 7
   \>IGPLOT - 20                             \>
 MEMPOW - 19                             \>REALFR - 18
   \>XYCOEFF - 8                             \\
 BATLORMUL - 15                          \>EDITLEV - 18
   \>IJOIN - 15                              \>
 MERGE - 18,22                           \>RECFIL* - 18
   \>XYCUR - 8                               \\
 BATLORSIM - 15                          \>EDITTAB - 18
   \>IMANIC - 15,22                          \>
 MODAL - 11                              \>RESAMPLE - 8
   \>XYCURA - 8,15                           \\
 BATPDSIM - 15                           \>EDRSIN - 9
   \>IMERGE - 15                             \>
 MODE - 11                               \>RESCALE - 8
   \>XYCURB - 15                             \\
 BDFGEN - 9                              \>ELLPLTP - 13
   \>IMGARITH - 8                            \>
 MODHELP - 18                            \>RESTORE - 19
   \>XYCUT - 15                              \\
 BDFKEY - 8                              \>ELLPRTP - 13
   \>IMGEDIT - 9                             \>
 MOVE - 14                               \>RHDC1 - 11
   \>XYDRAW - 15                             \\
 BIGCONV - 9                             \>ENTERTAB - 18
   \>IMGSTACK - 9                            \>
 MOVIE - 6                               \>RHDC2 - 11
   \>XYDRAWA - 15                            \\
 BIN - 8                                 \>EQPROFASP - 13
   \>IMSIZE - 5                              \>
 MSLICE - 5                              \>RINGCOL - 17
   \>XYEDIT - 15                             \\
 BITMASK - 2                             \>EXP - 2
   \>IMSTACK - 15                            \>
 MULCON - 5                              \>ROT3D - 14
   \>XYFIT - 8                               \\
\end{tabbing}
\vspace{20mm}
{\em Continued....}

\newpage
\begin{center}
{\Large\bf ASPIC PROGRAMS - Continued}
\end{center}

\begin{tabbing}
BBBBBBBBB - 88xx\=BBBBBBBBB - 88xx\=BBBBBBBBB - 88xx\=BBBBBBBBB - 88xx\=BBBBBBBBB - 88xx\=\kill
 BLANK - 8                               \>EXPAND - 21
   \>INCARN - 18                             \>
 MULT - 2                                \>RTOI - 18
   \>XYFITA - 15                             \\
 BLINK - 9                               \>EXTEND - 19
   \>INDPIX* - 18                            \>
 NITPIK - 11                             \>SATCOR - 15
   \>XYJOIN - 15                             \\
 BLINKER - 1                             \>EXTRACT - 19
   \>INSPECT - 5,15                          \>
 NON - 19                                \>SECTOR - 5
   \>XYKEY - 8                               \\
 BLURR - 11                              \>FC - 10
   \>INVARG - 17                             \>
 NOPROMPT - 8                            \>SEGMENT - 8
   \>XYKEYA - 8,15                           \\
 BONW - 6                                \>FCDISP - 10
   \>INVCOL - 17                             \>
 NORMALIZE - 8                           \>SELECT - 19
   \>XYLIST - 8                              \\
 BOXASP - 13                             \>FCPACK - 10
   \>IPCSIN - 16                             \>
 NORMFILT - 11                           \>SETALL* - 20
   \>XYLISTA - 8,15                          \\
 BOXFILTER - 8                           \>FCSAT - 10
   \>IPOLYGON - 16                           \>
 NSTACK - 8                              \>SETAREA* - 20
   \>XYMATCH - 15                            \\
 CADD - 2                                \>FCSCALE - 10
   \>IRASBACK - 9                            \>
 ODDHDC1 - 11                            \>SETLEV - 18
   \>XYMULTA - 15                            \\
 CALHELP - 18                            \>FCTHSI - 10
   \>IRASCOEF - 9                            \>
 ODDHDC2 - 11                            \>SETLIM* - 20
   \>XYPMATCH - 15                           \\
 CALIB - 18                              \>FCTRGB - 10
   \>IRASCORR - 9                            \>
 OUTSET* - 18                            \>SETMAG* - 20
   \>XYPRNT - 15                             \\
 CATCOPY - 18                            \>FFCLEAN - 8
   \>IRASDSCR - 9                            \>
 PACK - 4                                \>SETPER* - 20
   \>XYRENUM - 15                            \\
 CCDCON - 5                              \>FFT - 12
   \>IRASIN - 9                              \>
 PARDIF - 18                             \>SIMPAR - 18
   \>XYSORT - 15                             \\
 CCIRD - 3                               \>FFTPOW - 19
   \>IRASSHIFT - 9                           \>
 PARPLTP - 13                            \>SINCFIL* - 18
   \>XYSTAT - 15                             \\
 CDEMO - 1                               \>FILDEF - 12
   \>IRASSTACK - 9                           \>
 PATCH - 18                              \>SKYSUB - 9
   \>XYTORD - 3                              \\
 CDIV - 2                                \>FILEDOC - 7
   \>ISLICE - 5,15                           \>
 PCT - 18                                \>SLAFIT - 18
   \>XYTRAN - 8                              \\
 CENTROID - 8                            \>FILLGAP - 19
   \>ITFCORR - 8                             \>
 PDM - 19                                \>SLICE - 5
   \>XYWEED - 15                             \\
 CFICI - 3                               \>FILTSPEC - 8
   \>ITFGEN - 8                              \>
 PDSCOR - 15                             \>SMOOTH - 11
   \>XYWEEDA - 15                            \\
 CFITR - 3                               \>FINULS - 11
   \>ITFGENA - 15                            \>
 PDSIM - 15                              \>SPECOL - 17
   \>ZAPLIN - 18                             \\
 CFIXY - 3                               \>FITBAKASP - 13
   \>ITFHIST - 9                             \>
 PDSIN - 16                              \>SPLEQ - 19
   \>ZAPTAB - 9                              \\
 CINVERT - 8                             \>FITELLP - 13
   \>ITFPLOT - 8                             \>
 PDSMULTI - 15                           \>SPLFIT - 18
   \>ZEBRA - 17                              \\
 CLRLIM - 20                             \>FITSHEAD - 18
   \>ITOR - 8                                \>
 PDSRASTER - 15                          \>SQORST - 22
   \\
 CLRQUAD - 6                             \>FITSIN - 16
   \>JONESASP - 13                           \>
 PDSRIPPLE - 15                          \>SQRT - 2
   \\
 CMPRS - 22                              \>FITSOUT - 18
   \>LAPLACE - 11                            \>
 PECALBASP - 13                          \>STAR - 5
   \\
 CMULT - 2                               \>FIXINVAL - 9
   \>LIMITS - 6                              \>
 PEEP - 6                                \>STAREMASP - 13
   \\
\end{tabbing}}
\end{document}
