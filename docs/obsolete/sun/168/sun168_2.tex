\documentstyle[11pt]{article} 
\pagestyle{myheadings}

%------------------------------------------------------------------------------
\newcommand{\stardoccategory}  {Starlink User Note}
\newcommand{\stardocinitials}  {SUN}
\newcommand{\stardocnumber}    {168.2}
\newcommand{\stardocauthors}   {Clive G. Page\\University of Leicester}
\newcommand{\stardocdate}      {2nd November 1993}
\newcommand{\stardoctitle}     {JED --- the Text Editor}
%------------------------------------------------------------------------------

\newcommand{\stardocname}{\stardocinitials /\stardocnumber}
\renewcommand{\_}{{\tt\char'137}}     % re-centres the underscore
\markright{\stardocname}
\setlength{\textwidth}{160mm}
\setlength{\textheight}{230mm}
\setlength{\topmargin}{-2mm}
\setlength{\oddsidemargin}{0mm}
\setlength{\evensidemargin}{0mm}
\setlength{\parindent}{0mm}
\setlength{\parskip}{\medskipamount}
\setlength{\unitlength}{1mm}

%------------------------------------------------------------------------------
% Add any \newcommand or \newenvironment commands here
%------------------------------------------------------------------------------

\begin{document}
\thispagestyle{empty}
SCIENCE \& ENGINEERING RESEARCH COUNCIL \hfill \stardocname\\
RUTHERFORD APPLETON LABORATORY\\
{\large\bf Starlink Project\\}
{\large\bf \stardoccategory\ \stardocnumber}
\begin{flushright}
\stardocauthors\\
\stardocdate
\end{flushright}
\vspace{-4mm}
\rule{\textwidth}{0.5mm}
\vspace{5mm}
\begin{center}
{\Large\bf \stardoctitle}
\end{center}
\vspace{5mm}

%------------------------------------------------------------------------------
%  Add this part if you want a table of contents
%  \setlength{\parskip}{0mm}
%  \tableofcontents
%  \setlength{\parskip}{\medskipamount}
%  \markright{\stardocname}
%------------------------------------------------------------------------------
\section{What is JED?}

JED is a text-editor which provides a reasonably good emulation of EDT on
Unix systems, together with a few extra facilities such as cut-and-paste
of rectangular regions, unlimited undo, horizontal scroll, and display of
two or more windows. The author of JED, John Davis of Harvard University,
permits it to be used freely on non-commercial systems provided his {\tt
copyright} file is included in each distribution.

Most people who are used to EDT on VAXes seem to be able to use JED without
much difficulty, but a few differences are worth noting:-

\begin{itemize}

\item JED only emulates the keypad-and-screen mode of EDT: various GOLD--key
sequences have been defined (as described below) to make other functions
available.

\item To exit from JED use {\tt GOLD E} (because as usual with Unix
control/Z just suspends the process).  Alternatively use {\tt GOLD Q}
(for quit), or press the DO key and enter either {\tt exit} or {\tt
quit}.  These all do the same thing.  If any buffer has been modified JED
asks you whether to re-write it to disc.

\item The HELP key just shows a keypad diagram together with a brief
listing of the additional GOLD--key assignments.

\item The FILL function ({\tt GOLD 8}) fills the current paragraph and does
{\em not} require a mark to be set.  See below for further details.

\item The function to enter special characters, SPECINS or {\tt GOLD 3},
is slightly simpler than with EDT: it prompts for the decimal value of
the ASCII character required.

\end{itemize}

\subsection{New features in version 0.93}

The most visible change is that marked sections are now show in inverse
video, as in EDT.  Word-wrapping is improved and cut-and-paste operations
are easier as the text is put in a visible buffer called {\em cut\_buf}.
The new sequence {\tt GOLD K} moves all lines containing a specified
string to another buffer. Two new command-line switches ({\tt -f and -t})
have been added.

\section{Additional GOLD-key sequences}

\begin{center}
\begin{tabular}{|ll|}
\hline
Key-seq. & Effect \\
\hline
{\tt GOLD A} & ASCII codes of the next 10 characters are shown\\
{\tt GOLD B} & Buffer/window options... (see section 3.4) \\
{\tt GOLD C} & Centres current line using the {\tt WRAP} value \\
{\tt GOLD D} & Date inserted at current position (and date/time shown in minibuffer)\\
{\tt GOLD E} & Exits - if any buffers modified asks whether to update disc file \\
{\tt GOLD F} & Finds string at start of line (prompts for string) \\
{\tt GOLD G} & Global search and replace (prompts for both strings) \\
{\tt GOLD I} & Inserts file at current position (prompts for name) \\
{\tt GOLD K} & Prompts for string: moves lines containing this to another
buffer \\
{\tt GOLD L} & Goes to specified line-number (prompts for number) \\
{\tt GOLD M} & Mode settings displayed and can be altered (see section 3.6) \\
{\tt GOLD N} & Narrows (centres) current paragraph (see section 3.7) \\
{\tt GOLD O} & Occurrences of string listed with line-numbers (prompts for string)\\
{\tt GOLD Q} & Quits -- same as exit. \\
{\tt GOLD R} & Rectangular cut/paste options... (see section 3.8) \\
{\tt GOLD S} & Spawns command to operating system (some restrictions) \\ 
{\tt GOLD T} & Toggles terminal width 80/132 columns, adjusts {\tt WRAP} value\\
{\tt GOLD U} & Undo - restores buffer state undoing preceding changes one by one\\
{\tt GOLD W} & Writes current buffer to file (prompts for name) \\
{\tt GOLD X} & Exchanges position of mark and current position \\
{\tt GOLD} {\em space}   & Shows current position and mode settings \\
{\tt GOLD $\downarrow$}  & Selects the next window as the active one\\
{\tt GOLD $\uparrow$  }  & Expands current window by one line \\
{\tt GOLD $\rightarrow$} & Scrolls screen right by half its width \\
{\tt GOLD $\leftarrow$ } & Scrolls screen back to the left by half its width \\
{\tt GOLD GOLD}          & Toggles keypad application/numeric modes.\\
\hline
\end{tabular}
\end{center}

\section{Special Features}

\subsection{Command-line Switches}

\begin{tabular}{ll}
{\tt -f} & sets Fortran mode: tabs stops set at columns 7,10,13... \\
{\tt -g} {\em n} & go to line number {\em n} \\
{\tt -n} & ignore the {\tt .jedrc} file (to get default rather than 
customized settings) \\
{\tt -s} {\em string} & search for string \\
{\tt -t} & sets TAB to 0 so tabs show as {\verb+^I+}\\
{\tt -2} & split window \\
\end{tabular}

Note that the {\tt -f} and {\tt -t} switches must be given after the file
name.

For example to load file {\tt prog.f} into one window with the cursor at
line number 123, and then load {\tt prog.lis} into the second window
positioned at the first occurrence of the string {\em myfunc} you could
use:
\begin{verbatim}
      % jed prog.f -g 123 -2 prog.lis -s myfunc
\end{verbatim}

\subsection{Back-up and Auto-save files}

If a new file is created with the same name as one that already exists
(for example when exiting), JED will rename the original file by
appending a $\sim$ to its name so that the original contents are
preserved. In addition, while editing is in progress JED saves the
contents of each buffer in a disc file after every 300 changes (which may
be every 300 keystrokes if you are simply entering text). These auto-save
files, which have names of the form {\tt
\#}{\em filename}{\tt \#}, are deleted when JED exits
normally, but may be useful in the event of a crash.

\subsection{Status Line}

The status line, shown in inverse video beneath each window, shows some
status flags, the JED version number, the file or buffer name, the
current line-number, the total line count.  The status flags start as a
set of hyphens: the meaning of other symbols that may appear there is as
follows:

\begin{tabular}{cl}
{\tt **}   & this buffer has been modified \\
{\tt \%\%} & this buffer is read-only \\
{\tt m}    & a mark has been set \\
{\tt d}    & disc file has been updated since the buffer was created \\
{\tt +}    & the undo facility is enabled for this buffer. \\
\end{tabular}

\subsection{Using Multiple Buffers and Windows}

JED can handle two or more windows at once, and display in them either
different parts of the same file or two or more different files.  You
can, of course, cut and paste from one window to another. The {\tt
GOLD B} key prompts for a number of options.

\begin{center}
\begin{tabular}{|ll|}
\hline
Key-sequence & Buffer/window action \\
\hline
{\tt GOLD B 2} & Splits current window into two.\\
{\tt GOLD B 1} & Restores single window (the current one).\\
{\tt GOLD B G} & Gets a file into the current buffer (prompts for name).\\
{\tt GOLD B L} & Lists all existing buffers.\\
{\tt GOLD B S} & Selects a buffer by name (prompts for it).\\
\hline
\end{tabular}
\end{center}

{\tt GOLD} down-arrow selects the next window in turn as the active one,
while {\tt GOLD} up-arrow expands the current window by one line.

\subsection{Undo}

A powerful undo facility is provided using {\tt GOLD U} (also mapped to
the UNDO key on a Sparcstation and F14 on a VT terminal).  JED keeps
track of buffer modifications and undoes the changes one by one when you
press the UNDO key.  This can be used to reverse complex changes such as
{\em fill} or {\em sort}.

\subsection{Modes and Customizing}

A few variables exist to control editing modes. These are set by default
to give the closest emulation of EDT, but may be changed either using the
{\tt GOLD M} options or by providing your own {\tt .jedrc} customisation
file (see below).

{\tt WRAP\_INDENTS} is 0 by default.  If set to 1, when the RETURN
key is pressed the new line will be indented the the same point as the
previous one.  This can be useful when entering Fortran or C source-code.

{\tt CASE\_SEARCH} is 0 by default, giving case-insensitive searches.  
If set to 1 all searches become case-sensitive.

{\tt TAB} determines how tab characters in the text are displayed on
the screen. If {\tt TAB} is non-zero then it appears as if there are tab
stops every {\tt TAB} columns across.  Otherwise each tab character is
shown as {\verb+^I+}.  The default TAB value is 8, which corresponds to the
way most VT terminals are set up, except if the {\tt -t} switch is used.

{\tt WRAP} specifies the right-hand margin for text entry, filling, and
centring text. When you enter text which crosses the {\tt WRAP} boundary
and press the space bar the excess words are moved to the following line.
The default {\tt WRAP} value is 75.  To avoid the temporary disappearance
of text when you are typing continuously it is advisable to choose a
value slightly smaller than the physical screen width. The {\tt WRAP}
value is adjusted when the screen-width is changed using {\tt GOLD T}.

These initial defaults may be changed by providing a file called {\tt
.jedrc} which, if put in your {\em home} directory, will affect all JED
sessions.  The format of the assignments should be noted: a space is
required before the equals sign but not after it. Comments may be entered
after a semi-colon.  For example:

\begin{verbatim}
      ; an example .jedrc file for a Fortran programmer
      72 =WRAP    ; show Fortran statement margin violation by wrapping
      0 =TAB      ; show any actual tab characters as ^I
      1 =WRAP_INDENT  ; auto-indent mode is set
\end{verbatim}

The tab characters in a file may be converted to the appropriate set of
spaces using the current value of {\tt TAB} by selecting a range of text,
pressing the DO key, and then entering the {\tt untab} command.

\subsection{Paragraph Fill and Centre}

Text formatting operations work differently from those in EDT. The FILL
command ({\tt GOLD 8}) fills the current paragraph, and does {\em not}
use the selected range.  A paragraph is defined as any set of lines
ending with a blank line (or a line starting with either {\tt \%} or 
\verb+\+ to assist in filling Latex files). If the
first line of the paragraph was inset from the left-hand margin the whole
paragraph is indented by the same amount.  The {\tt WRAP} value defines
the maximum line-length.

To centre the current line use {\tt GOLD C}, while ({\tt GOLD N}) centres
the current paragraph, with margins set by the indentation of the first
line.  In both cases the centre is half the {\tt WRAP} value.

\subsection{File-name Completion}

Whenever JED prompts for a file-name you can enter a partial name and
press the TAB key to get JED to complete it.  If the name is not unique
then press the space-bar at least twice to cycle among them.

\subsection{Editing Wide Files and Binary Files}

JED imposes no intrinsic limit on the length of lines it can handle, but
it is always awkward to edit lines wider than the screen. If your
terminal can manage 132-column mode then you can use {\tt GOLD T} to
toggle the width. The value of the {\tt WRAP} variable is automatically
incremented/decremented by 52 when you do this. Another option is to use
{\tt GOLD} right-arrow to scroll the screen sideways in chunks of half a
screen-width.  Not surprisingly the {\tt GOLD} left-arrow reverses this.

It is also possible to use JED to examine or modify binary files.  Note
that the {\tt GOLD A} sequence lists the decimal ASCII codes of the next
10 characters.  If, however, you plan to modify a binary file it is
sensible to set the {\tt ADD\_NEWLINE} variable to zero before starting,
otherwise it will add a line-feed character to the last line of the file
(if it does not already have one).

\subsection{Special Characters}

The SPECINS key ({\tt GOLD 3}) may be used to enter non-standard characters: it
prompts for the decimal code.  Alternatively to enter control codes first
press the F14 key (marked ESCAPE on many VT terminals) and then the
required control key.  This even works for control/C and control/Z and
may be used within a search string.

\subsection{Rectangular Column Cut/Paste}

First define the rectangle by pressing SELECT (or the key-pad dot key) at
one corner and move the cursor to the opposite corner.  Then {\tt GOLD R
C} will copy the rectangle to a buffer, or {\tt GOLD R D} delete it, i.e.
move it to a buffer.  A subsequent {\tt GOLD R I} will insert the saved
rectangle at the current point.  Other options are {\tt GOLD R B} to
blank rectangle (overwrite with spaces), or {\tt GOLD R S}
to shift text over by inserting a blank rectangle of the size specified.

\subsection{Sorting Lines}

JED can sort a series of lines into ascending order using the ASCII
collating sequence.  First select the rectangle containing the sort keys
(press SELECT at one corner and move to the other corner), then press the
DO key and at the prompt enter the {\tt sort} command. Sorting is done by
interpreted code and is rather slow for regions of more than a few
hundred lines.

\section{SUN Keyboard Support}

The original Sparcstation keyboard (type-4) has no separate arrow keys,
so that F9 through F12 have been mapped to act as up, down, left, and
right-arrows, respectively.  The new type-5 (PC-style) keyboard available
for the Sparc Classic provides a full set of arrow keys and a separate
application keypad.  The GOLD key is the one at the top left of the
key-pad in both cases, though called {\tt Num-lock} on the type-5
keyboard. Unfortunately there is still one key missing on the right-most
strip, as the one marked {\tt +} covers the space occupied by both {\tt
-} and {\tt ,} on the VT keypad. This {\tt +} key supports the
delete-character function.  The delete-word function is supported on the
type 5 keyboard by the {\tt scroll-lock} key at the top.  Other useful
equivalences are noted below.

\begin{center}
\begin{tabular}{|l|l|l|}
\hline
Key/function & Sparc-key \\
\hline
DO & Copy \\
SELCT & Open \\
REMOVE & Cut \\
INS HERE & Paste \\
FIND & Find \\
UNDO & Undo \\
HELP & Help \\
Search/replace & f6 \\
Toggle 80/132 cols & f7 \\
\hline
\end{tabular}
\end{center}

If the keypad of your terminal does not behave correctly, it may not have
the right key-mapping set up for it.

\section{Installation}

The instructions below assume that the JED executable will be put in
{\tt /usr/local/bin} with the supporting files in {/usr/local/lib/jed}. If
this is not so, modify them accordingly.

(1) Move the {\tt jed093.tar} file to {\tt /usr/local/lib} and un-tar it: it will
extract to a sub-directory called {\tt jed093}.

(2) The package comes with a executables for SunOs, Solaris, Ultrix, and
OSF1, these are called e.g. {\tt jed.solaris}.  Copy and rename the
appropriate one to {\tt /usr/local/bin}.

(3) Rename your old jed sub-directory (if any) to something else, and rename
the new {\tt jed093} sub-directory to {\tt /usr/local/lib/jed}.

(4) If the jed support files are indeed in this directory then the
environment variable {\tt JED\_LIBRARY} is no longer needed, else set it
to point to the sub-directory that you have used.

(5) Copy the user-note file {\tt sun168.tex} to the documentation directory.

If you need to re-build the package for any reason, then a makefile is
provided which can be used as follows.  The procedure for Solaris is more
messy and explained in file AAA.solaris provided in the package.

\begin{tabular}{ll}
OSF/1: & {\tt make -f makefile.unx} \\
Ultrix: & {\tt make -f makefile.unx} \\
SunOs:   & {\tt make -f makefile.unx CC=gcc} \\
Irix:    & {\tt make -f makefile.unx CCFLAGS="-DPOSIX"} \\
\end{tabular}

Note: if users have existing files called {\tt .jedrc} they may be
incompatible with this version and should be deleted or renamed.

\section{Software Support}

The kernel is written in C and implements a Forth-like language called
S-lang in which most higher-level features of the EDT emulation are
programmed.  Reports of problems (or suggestions for enhancements) may be
sent to {\tt cgp@star.le.ac.uk} or {\tt LTVAD::CGP}.  These will, where
appropriate, be passed on to the author, who has already fixed several 
problems and provided many enhancements.

As indicated by the current version number (0.93-2B), the author is still
actively developing the code.  Future versions are likely to be available
by anonymous ftp, but anyone collecting a new version of JED directly
should note that the Starlink-specific S-lang code provided here will
probably not work, as from version 0.94 onwards S-lang will use 
infix rather than postfix notation. 

The author's copyright notice is reproduced here.
\begin{quote}
\copyright 1992 by John E. Davis  ({\tt davis@amy.tch.harvard.edu})

Permission to use, copy, modify, and distribute this software and its
documentation for any non-commercial purpose and without fee is hereby
granted, provided that the above copyright notice appear in all copies
and that both that copyright notice and this permission notice appear in
supporting documentation.

This file is provided AS IS with no warranties of any kind.  The author
shall have no liability with respect to the infringement of copyrights,
trade secrets or any patents by this file or any part thereof.  In no
event will the author be liable for any lost revenue or profits or other
special, indirect and consequential damages. 
\end{quote}
\end{document}
