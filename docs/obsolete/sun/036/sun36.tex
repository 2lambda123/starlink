\documentstyle{article}
\pagestyle{myheadings}

%------------------------------------------------------------------------------
\newcommand{\stardoccategory}  {Starlink User Note}
\newcommand{\stardocinitials}  {SUN}
\newcommand{\stardocnumber}    {36.14}
\newcommand{\stardocauthors}   {P M Allan\\D L Terrett}
\newcommand{\stardocdate}      {11 October 1990}
\newcommand{\stardoctitle}     {Starlink Networking}
%------------------------------------------------------------------------------

\newcommand{\stardocname}{\stardocinitials /\stardocnumber}
\markright{\stardocname}
\setlength{\textwidth}{160mm}
\setlength{\textheight}{240mm}
\setlength{\topmargin}{-5mm}
\setlength{\oddsidemargin}{0mm}
\setlength{\evensidemargin}{0mm}
\setlength{\parindent}{0mm}
\setlength{\parskip}{\medskipamount}
\setlength{\unitlength}{1mm}

\begin{document}
\thispagestyle{empty}
SCIENCE \& ENGINEERING RESEARCH COUNCIL \hfill \stardocname\\
RUTHERFORD APPLETON LABORATORY\\
{\large\bf Starlink Project\\}
{\large\bf \stardoccategory\ \stardocnumber}
\begin{flushright}
\stardocauthors\\
\stardocdate
\end{flushright}
\vspace{-4mm}
\rule{\textwidth}{0.5mm}
\vspace{5mm}
\begin{center}
{\Large\bf \stardoctitle}
\end{center}
\vspace{5mm}

\section{Introduction}

The Starlink project uses the Joint Academic Network (JANET) to provide
communications between the Starlink VAXs and to other systems in the academic
community.
The main uses of the network are to provide access for remote users (i.e.\ users
not situated at a Starlink site), communication between users by electronic
mail, and the transfer of modestly sized files (i.e.\ software and documentation
rather than data).
It is not intended that the network should be used for users at one node to
access facilities at another
(apart from using the database machine at RAL)
and it is not normal for a user to have usernames
on more than one node, although in certain special circumstances this may
happen.

\section{JANET and the Coloured Books Software}

JANET is an X.25 packet switched network and provides three main facilities:
\begin{itemize}
\item File transfer
\item Electronic mail
\item Remote login
\end{itemize}
File transfer and electronic mail are not defined by the X.25 standard, and are
implemented according to the `Coloured Books Protocols', a UK standard.
The name `JANET software' is sometimes used to refer to the Coloured Books
software, but JANET is really just the hardware component of the network.
The Coloured Books software provides four main functions:
\begin{itemize}
\item TRANSFER (for file transfer)
\item CBS-MAIL (for sending electronic mail)
\item PAD (for beginning a remote login)
\item LIST (for showing the status of TRANSFER and CBS-MAIL jobs)
\end{itemize}
An extra component of the Coloured Books software is the Red Book `Job Transfer
and Manipulation Protocol (JTMP)'. This provides a versatile method of
submitting jobs to remote computers, however, the software is not available on
all Starlink machines (see appendix~\ref{who-has-JTMP} for availability). The
remote computer need not be a VAX, and a common use is to submit large jobs to
a Cray type computer from a VAX. The JTMP software is not particularly easy to
use, but the commands to perform a particular task can be packaged up into a
library for easier use. It is also possible to use the TRANSFER command to
perform simple remote job submission.

The commands referred to above are for VMS systems; they are documented briefly
in appendix~\ref{cbs} and more fully in the DEC Coloured Books and Red Book
manuals.

In order to use any of these facilities, the name of the remote node must be
provided. These names are the subject of an official (non-Starlink)
registration scheme and are commonly known as NRS~names. The names are created
in a hierarchical manner and while they can seem long and unwieldy, it has to
be possible to specify the name of any computer, anywhere in the world. For
example, the Rutherford Appleton Laboratory Starlink project VAX has a name of
UK.AC.\-RUTHERFORD.\-STARLINK, which has a shorter form of UK.AC.RL.STAR. On
systems which fully support the NRS scheme, only that part of the name which
differs from the name of the host system need be used. For example, from within
the UK academic community RL.STAR is all that is required, and within the RAL
site STAR is sufficient. The NRS names of the Starlink systems are given in
table~\ref{nrs}.

To make network calls from a VAX you must be authorized to access the site
you wish to contact; this has already been done for all the Starlink
systems, but for others you may need to contact your system manager.

\begin{table}[p]
\caption{NRS names of Starlink sites}\label{nrs}
\small
\begin{center}\begin{tabular}{l@{\hspace{1cm}}l}
Armagh& UK.AC.QUEENS-BELFAST.ARMAGH.STARLINK\\
& UK.AC.QUB.ARM.STAR\\[\medskipamount]
Belfast& UK.AC.QUEENS-BELFAST.PHYSICS.STARLINK\\
& UK.AC.QUB.PHY.STAR\\[\medskipamount]
Birmingham& UK.AC.BIRMINGHAM.SPACE--RESEARCH.STARLINK\\
& UK.AC.BHAM.SR.STAR\\[\medskipamount]
Cambridge& UK.AC.CAMBRIDGE.ASTRONOMY.STARLINK\\
& UK.AC.CAM.AST--STAR\\[\medskipamount]
Cardiff& UK.AC.CARDIFF.ASTRONOMY.VAX1\\
& UK.AC.CF.ASTRO.V1\\[\medskipamount]
Durham& UK.AC.DURHAM.STARLINK\\
& UK.AC.DUR.STAR\\[\medskipamount]
Hatfield Polytechnic & Not yet known \\[\medskipamount]
Imperial College& UK.AC.IMPERIAL-COLLEGE.PHYSICS.STARLINK\\
& UK.AC.IC.PH.STAR\\[\medskipamount]
Jodrell Bank& UK.AC.MANCHESTER.JODRELL--BANK.STARLINK\\
& UK.AC.MAN.JB.STAR\\[\medskipamount]
Keele& UK.AC.KEELE.PH.STARLINK\\
& UK.AC.KL.PH.STAR\\[\medskipamount]
Kent & UK.AC.UKC.STARLINK \\
& UK.AC.UKC.STAR \\[\medskipamount]
Lancashire Polytechnic& UK.AC.LANCASHIRE--POLY.STARLINK\\
& UK.AC.LANCSP.STAR\\[\medskipamount]
Leicester& UK.AC.LEICESTER.STARLINK\\
& UK.AC.LE.STAR\\[\medskipamount]
Manchester& UK.AC.MANCHESTER.ASTRONOMY.STARLINK\\
& UK.AC.MAN.AST.STAR\\[\medskipamount]
Oxford& UK.AC.OXFORD.ASTROPHYSICS\\
& UK.AC.OX.ASTRO\\[\medskipamount]
Queen Mary and Westfield College& UK.AC.QMW.STARLINK\\
& UK.AC.QMW.STAR\\[\medskipamount]
RAL (Project) & UK.AC.RUTHERFORD.STARLINK\\
& UK.AC.RL.STAR\\[\medskipamount]
RAL (Astrophysics) & UK.AC.RUTHERFORD.STARLINK.ASTROPHYSICS\\
& UK.AC.RL.STAR.AST\\[\medskipamount]
RAL (Database) & UK.AC.RUTHERFORD.STARLINK-DATABASE\\
& UK.AC.RL.STADAT\\[\medskipamount]
ROE& UK.AC.ROE.STARLINK\\
& UK.AC.ROE.STAR\\[\medskipamount]
St Andrews& UK.AC.ST--ANDREWS.STAR\\
& UK.AC.ST--AND.STAR \\[\medskipamount]
Southampton& UK.AC.SOUTHAMPTON.PHYSICS.ASTRONOMY\\
& UK.AC.SOTON.PHASTR\\[\medskipamount]
Sussex & UK.AC.SUSSEX.STARLINK\\
& UK.AC.SUSX.STAR \\[\medskipamount]
UCL& UK.AC.UCL.STARLINK\\
& UK.AC.UCL.STAR
\end{tabular}\end{center}\end{table}

In order to use the TRANSFER command, you must know a username and password on
the remote system.
All Starlink systems have a username NETUSR with password NETUSR that can be
used for copying files from remote sites.
This username has the same access to files as normal users but actual logins are
disabled.

The username of every Starlink user is kept in a file called:
\begin{quote}
{\tt ADMINDIR:USERNAMES.LIS}
\end{quote}
on every node; this file is updated every month.
An up-to-date list of all users on a particular node is kept on that node in a
file called:
\begin{quote}
{\tt LADMINDIR:USERNAMES.LIS}
\end{quote}
This file is updated every time a new user is put on the system. The site
manager (or someone temporarily acting on his or her behalf) can always be
contacted by mail to OPER (not SYSTEM, which may not be looked at for several
days).

When copying files from one user to another (whether across the network or just
locally), always `pull' the files rather than `pushing' them, i.e.\ copy the
files to your own directory, not to the other user's directory as this will
result in the files having the wrong ownership.

The terminal access provided by X.25 networks is designed for simple `dumb'
terminal access and is not particularly suitable for full screen editors or
graphics, especially on a low speed and often congested network such as JANET.
Screen editors and graphics can be made to work correctly but the response times
often become unacceptably long.
However, commands exist for altering the behaviour of your terminal and these
can be used to get the best service possible from the network.
A short guide to logging into VAX/VMS systems over an X.25 network can be found
in appendix~\ref{pad}.

\subsection{Transferring large files}
Although it is preferable to send large files on magnetic tape, sometimes it
is necessary to use TRANSFER. In such a case it is very important to send the
data in as compact a form as possible. Every Starlink site has a file
compression utility for compressing and decompressing files. The
compression program is called LZCMP and the decompression program is called
LZDCM. These utilities can give a spectacular reduction in the size of a
file. A compression factor of two is not unusual and a factor of five has
certainly been achieved. These programs are described in SUN/25. If you must
send large files over the network, calculate how long it will take before you
do it. Transfers that would take many hours should be sent overnight to avoid
network congestion.

\section{DECnet}

In addition to offering PAD, CBS-MAIL and TRANSFER, Starlink also runs its
nodes as a DECnet network, using JANET to provide the communications links.
Certain other (non-Starlink) VAXs are also part of this DECnet network, some
using JANET to provide the links and others using ethernet local area networks.
For the latter, DECnet may be the only network access provided. DECnet provides
a somewhat richer set of facilities than Coloured Books at the expense of a
greater load on both the network and the host systems. In particular, not all
nodes have direct connections to all other nodes and so excessive use of DECnet
may also impose a load on the routing nodes in the network.

The extra facilities provided are:
\begin{itemize}
\item Remote file access
\item VMS mail
\item Remote execution of SHOW commands
\item PHONE and TALK
\end{itemize}
Remote file access means that a program running on one node can access a file on
another node just as if it were on the local system.
This means that, for example, remote directories can be listed with the
DIRECTORY command, and remote files can be copied (with COPY) with all the
normal wild card facilities available.
It is even possible to `SET DEFAULT' to a directory on a remote system.

In spite of their ease of use, these facilities should be used with discretion
as the effect on the remote system is similar to having an extra user logging
in.
When single files are being copied, the TRANSFER command should always be used
as it never involves routing traffic through a third node, and it will also
recover from network failures transparently and without having to restart the
transfer from the beginning of the file.
The TRANSFER command can also be issued regardless of whether the remote system
is reachable at the time.

A SHOW command can be executed on a remote node with the NETSHOW command.
NETSHOW is described in appendix~\ref{shownet}.

PHONE and TALK allow you to have conversations with other users (local as well
as remote).
PHONE is part of VMS and is described in the VMS HELP but requires a VT 
compatible terminal.
TALK will work with any sort of terminal and is described in
appendix~\ref{shownet}.
Be aware that TALK can be very irritating for other users if their work is
constantly interrupted by your messages; it is more polite to use MAIL.
The DECnet node names of the Starlink sites are shown in 
table~\ref{DECnet-node-names}.
\begin{table}[htbp]
\caption{DECnet node names}\label{DECnet-node-names}
\begin{center}\begin{tabular}{l@{\hspace{1.5cm}}l}
Armagh & ARVAD\\
Belfast & QUVAD\\
Birmingham & BHVAD\\
Cambridge & CAVAD\\
Cardiff & CARDIF\\
Durham & DUVAD\\
Hatfield Polytechnic & HATVAD\\
Imperial College & ICVAD\\
Jodrell Bank & JBVAD\\
Kent & KENVAD\\
Keele & KLVAD\\
Lancashire Polytechnic & LPVAD\\
Leicester & LTVAD\\
Manchester & MAVAD\\
Oxford & OXVAD\\
QMW & QMCMV\\
RAL (Project) & RLVAD\\
RAL (Astrophysics) & RLSAC\\
RAL (Database) & STADAT\\
ROE & REVAD\\
St.\ Andrews & SASTAR\\
Southampton & SOTON\\
Sussex & SUSTAR\\
UCL & ZUVAD
\end{tabular}\end{center}\end{table}

Remember that these names apply only to DECnet. In general they have no
meaning outside the Starlink network, although they can be used in connection
with SPAN (section~\ref{SPAN}), as this is based on DECnet. When giving your
network address to anyone but a Starlink user, be sure to quote the NRS name
(for example ME at UK.AC.ROE.STAR).

The command SHOWNET (a Starlink feature distinct from the VMS command SHOW
NETWORK) gives all the readily available information on the accessibility
of all the nodes in the DECnet network.

\subsection{VAXclusters}
Many Starlink sites have more than one VAX configured as a
VAXcluster. This has implications for network access. When several VAXs are
configured as a VAXcluster, each VAX has its own node name, but you can also
refer to the cluster as a whole by means of the `cluster alias'. When you are
communicating with a cluster in a way that does not require a specific machine
to perform a task then you should use the cluster alias. Sending a MAIL message
is the most common example of this.

If you are doing something that requires the cooperation of a particular
machine, then you must use the particular node name. For example, if you are
using the TALK command, then you are TALKing to someone who is logged onto a
particular machine, so you must give the node name of the machine that they are
using. 

Take note that using the cluster alias when you should use a particular node
name can cause confusion. For example, if you were to type the command
\begin{quote}
{\tt \$ SET HOST MAVAD}
\end{quote}
twice, there is no guarantee that you would be logged into the same machine
each time.

The DECnet node names
given above are actually cluster aliases for those sites that have clusters.
If you want to find the node names of the machines in the Manchester VAXcluster
(for example), type the command
\begin{quote}
{\tt \$ NETSHOW MAVAD CLUSTER}
\end{quote}
The full list of node names is not given here because the information can
change on a relatively short timescale.
\subsection{Proxies}
If you have an account at more than one Starlink node you will probably
want to do things like a directory listing of your files on the remote
node. When you do something like
\begin{quote}
{\tt \$ DIR RLVAD::}
\end{quote}
you will get a directory listing of the default network account on the
remote node, which is pretty useless. You can type
\begin{quote}
{\tt \$ DIR RLVAD::DISK\$USER1:[ABC]}
\end{quote}
(if you are user ABC), but this is more typing and will not work if you
have protected your files on the remote node. There are two ways around
this. The first is to type
\begin{quote}
{\tt \$ DIR RLVAD"ABC ABCSPASSWORD"::DISK\$USER1:[ABC]}
\end{quote}
This requires even more typing and unfortunately requires you to type your
password on the screen. A method that is generally better is to have a proxy
login set up at the remote node. If this is done, the remote node knows that
calls coming from user ABC at a certain node should be logged in as user ABC on
that node. The usernames do not have to be the same at each end (though they
usually are), but each proxy is specific to a certain user at a certain site.
If user PMA at Manchester has a proxy username of PMA at Jodrell Bank, then it
is not necessarily true that user PMA at Jodrell Bank has a proxy of PMA at
Manchester.

There are some security implication of having proxy logins. Although you do
not have to type your password (in fact {\it because} you do not have to
type your password), if your account is hacked on your local machine, your
files on the remote machine are immediately accessible to the hacker. On
account of this, privileged users should not allow proxy logins. This is
not just paranoia; more than one system disk has been corrupted by such
means. Proxies have to be set up by the manager of the remote machine. If
the manager considers the provision of a proxy login to be too much of a
security loophole then he or she may refuse to set one up.

\section{Comparing DECnet and Coloured Books}
\label{post-mail}

If you want to communicate with a computer on JANET, then you will generally
have to use the Coloured Books software (PAD, CBS-MAIL and TRANSFER). The Starlink
VAXs and computers connected to the SPAN network (see section \ref{SPAN}) can
be contacted using either Coloured Books software or DECnet, although the
DECnet access to SPAN is a little restricted. When you have the choice of these
two methods, it is useful to know the advantages and disadvantages of each
method. There are essentially two aspects to consider, sending mail and
transferring files.

The advantage of using VMS MAIL rather than CBS-MAIL is that the connection to the
remote computer is made immediately so that you {\it know} that the message has
been delivered successfully. However, if the remote computer is down, then the
message cannot be delivered. If you use CBS-MAIL, then the message is queued for
later delivery, so it is generally a more robust method, although occasionally
messages sent using CBS-MAIL just disappear for no very good reason.

When considering file transfers the choice is between COPY and TRANSFER.
TRANSFER is nearly always the better solution since, like CBS-MAIL, it will
recover from network failures. If there is a hiccough in the network connection
when using COPY, the transfer is aborted. The problem with TRANSFER is that you
can only copy one file at a time. You cannot use wild cards. If you need to
transfer many files, a useful way is to put all of the files in a BACKUP
saveset at the remote computer and then TRANSFER that single file (you need to
specify TRANSFER/CODE=FAST, and ideally you should compress the saveset with
LZCMP). However, this presumes that you (or a colleague) are able to log on to
the remote computer to make the saveset. If this is not possible, then you can
copy multiple files with wild card facilities using the JTMP software,
although as noted earlier, not all sites have this available.


\section{Network Access Account}

Every Starlink site should have set up a special account with username
`STARLINK'. This allows a visitor to call another machine using the `SET HOST'
command but prevents him from using DCL commands. To use the facility, login
with username STARLINK and reply to the prompt `Node?' with either the DECnet
node name (e.g.\  RLVAD) or `{\em number}/X' where {\em number} is the DTE
address of the JANET machine you wish to call.

\section{Dialup lines}

Terminal access to some nodes from the public telephone system is available if
you have access to a suitable modem. This information tends to change fairly
often and it is advisable to contact the site manager before using these
services.

The `V' numbers used below have the following meanings.
\begin{itemize}
\item V21 = 300/300 baud, full duplex
\item V22 = 1200/1200 baud, full duplex
\item V22bis = 2400/2400 baud, full duplex
\item V23 = 1200/75 baud, full duplex
\end{itemize}
Some modems also provide error correction facilities. At present these all use
the MNP protocol. 

Some modems have their access controlled by a password. This may be set in the
modem or in the equipment to which it is connected (e.g.\ terminal server or
PACX). You should contact the local system manager for details.
\newlength{\numlen}
\settowidth{\numlen}{xxxx000--000--0000}
\settowidth{\labelsep}{000}

\subsection{Birmingham}
\begin{list}{}{\setlength{\labelwidth}{\numlen}\setlength{\leftmargin}{\numlen}
\addtolength{\leftmargin}{\labelsep}}
\item[021--472--7387] V22 connected to BH34A (microVAX 3400).
\item[021--414--3724] V22 `secure dialback modem' connected to School Gandalf
PACX. Contact system manager for details.
\end{list}

\subsection{Durham}
\begin{list}{}{\setlength{\labelwidth}{\numlen}\setlength{\leftmargin}{\numlen}
\addtolength{\leftmargin}{\labelsep}}
\item[091--374--2133] V21/V23 (8 bit, no parity) . Connects to 
Gandalf PACX. Enter STAR to connect to STarlink DECserver. 
\end{list}

\subsection{Jodrell Bank}
\begin{list}{}{\setlength{\labelwidth}{\numlen}\setlength{\leftmargin}{\numlen}
\addtolength{\leftmargin}{\labelsep}}
\item[0477--71324] V21/V23/V22/V22bis (autobaud), connected to a DECserver
200.
\item[0477--71548] V21/V23/V22/V22bis (autobaud), connected to a Spiderport
terminal server. Type OPEN JBVAD for the Vax.
\end{list}

\subsection{Manchester}
\begin{list}{}{\setlength{\labelwidth}{\numlen}\setlength{\leftmargin}{\numlen}
\addtolength{\leftmargin}{\labelsep}}
\item[061--273--5730] V21/V23/V22/V22bis (autobaud), MNP 5, eight bits, 
no parity, connected to a DECserver 200.
\end{list}

\subsection{QMW}
\begin{list}{}{\setlength{\labelwidth}{\numlen}\setlength{\leftmargin}{\numlen}
\addtolength{\leftmargin}{\labelsep}}
\item[071--980--7100] V21 (8 lines). Queen Mary and Westfield College Computer
Centre. This gives access to a PAD and QMW can be called as {\tt QMW.STAR}.
\item[071--981--7331] V23 (4lines). Ditto.
\item[071--388--2333] V21/V22/V22bis, UCL Computer Centre (UCLCC). This
gives access to the UCL `data exchange' or DCX. Starlink may be accessed by
specifying {\tt STARLINK} when asked {\tt Which Service?} and then in
response to {\tt Username:} specify {\tt STARLINK} which will then prompt
for {\tt Node?} to which you reply {\tt QMCMV}.
\end{list}

\subsection{RAL}
\begin{list}{}{\setlength{\labelwidth}{\numlen}\setlength{\leftmargin}{\numlen}
\addtolength{\leftmargin}{\labelsep}}
\item[0235--44--6951] PAD line V21/V23.
\item[0235--44--6952] PAD line V22/V22bis.
\end{list}

\subsection{ROE}
\begin{list}{}{\setlength{\labelwidth}{\numlen}\setlength{\leftmargin}{\numlen}
\addtolength{\leftmargin}{\labelsep}}
\item[031--668--8365] V21/V22/V23, connected to a DECserver 200;
in response to the {\tt Local} prompt, enter {\tt C RE3500}.
\item[031--668--8364] V21, connected to a DECserver 200;
in response to the {\tt Local} prompt, enter {\tt C RE3500}.
\end{list}

\subsection{UCL}
\begin{list}{}{\setlength{\labelwidth}{\numlen}\setlength{\leftmargin}{\numlen}
\addtolength{\leftmargin}{\labelsep}}
\item[071--388--2333] V21/V22/V22bis, UCL Computer Centre (UCLCC).
This gives access to the UCL `data exchange' or DCX.
Starlink may be accessed by specifying {\tt :24} or {\tt STARLINK} when asked
for {\tt Service}.
\item[071--831--6181] V22, University of London Computer Centre (ULCC).
This gives access to a PAD and UCL (or anywhere else) can be called
by DTE number (2005002 for UCL.STAR).
\end{list}

\section{Overseas Access}

This section describes facilities which are not under the control of the
Starlink project and the information presented here may be incorrect, out of
date or incomplete.
Futhermore the existence of these facilities does not imply that you are
entitled to use them.

\subsection{Names and addresses}

If you want to send electronic mail to someone on a remote computer you need
to know his username and the node name of his computer. The most
comprehensive list of usernames and node names may be available in the files
\begin{quote}
{\tt LDOCSDIR:ASTROPERSONS.LIS\\
LDOCSDIR:ASTROPLACES.LIS\\
LDOCSDIR:ASTROPOSTAL.LIS}
\end{quote}
These files are provided by the RGO and are regularly updated, but one can
never hope for perfect accuracy as the information is constantly changing.

You should be aware that a computer might have several possible network
addresses depending on the route that you use to send a message. For example,
the RAL (project) Starlink VAXcluster has a DECnet cluster alias of RLVAD,
whereas the JANET NRS name is UK.AC.RL.STAR. To add to the confusion, foreign
BITNET sites might need to address this site as STAR.RL.AC.UK. When sending a
mail message abroad it is very useful if you can include all possible return
addresses that you know to work.

Extra information, such as how to address users at the end of exotic network
chains, is stored in the file:
\begin{quote}
{\tt RLVAD::LADMINDIR:NET\_HINT.LIS.}
\end{quote}
There is also a VAXnotes conference called NETWORKS on the RAL Starlink
microVAX 3500
(node name RLSTAR) about network names and addresses.

\subsection{Operator driven Mail services}
\subsubsection{Australia}
\label{oper-mail}

Starlink and the Anglo-Australian Observatory run a daily mail service between
the UK and Australia.
The AAO operators will distribute mail to:
\begin{itemize}
\begin{itemize}
\item AAO Staff
\item Visitors to the AAO
\item CSIRO Radiophysics
\item Mount Stromlo Observatory
\end{itemize}
\end{itemize}
The UK operators will distribute to:
\begin{itemize}
\begin{itemize}
\item Any Starlink user
\item People who work at a Starlink site
\item Visitors from AAO
\item A user on any JANET system
\end{itemize}
\end{itemize}
Note that (i) when someone from the AAO is visiting the UK, messages from
the AAO must include a site name as the UK
operators do not know the itinerary of such visitors, (ii) mail addressed to a
user on a non-Starlink system must be accompanied by a complete network
address---the UK operator only has lists of Starlink users.
Mail will not be forwarded to networks other than JANET.

Users who have to correspond with the AAO on a regular basis should obtain
the necessary authorization from their site manager to send mail directly
with CBS-MAIL. CSIRO Radio physics can also be mailed directly
(AUSTRALIA.CSIRO.RPEPPING) and Mount Stromlo can be mailed via BITNET or NSFnet
(OZ.ANU.MSO).

Details of how to use this service can be found in appendix~\ref{aao-mail}.

\subsubsection{La Palma Observatory}

The RGO operates a mail service to the Isaac Newton Group of Telescopes on
La~Palma. Messages should be sent to {\tt CAVAD::LPMAIL} (via VMS MAIL) or
{\tt LPMAIL} at {\tt UK.AC.CAM.AST-STAR} (via CBS-MAIL)
with the subject field containing the name of the intended recipient.

\subsection{SPAN/ESPIN}

\label{SPAN}
The Space Physics Analysis Network is a large DECnet network operated by NASA;
the European Space Information Network operated by ESA is functionally part of
the same network.

Terminal access and some limited mail facilities are available from JANET; a
short guide to what is available and how to use them can be obtained by:
\begin{quote}
{\tt \$ TRANSFER/USERNAME="CYM//194" -\\
\hspace*{30mm}RL.IB::"SPAN GUIDE" -\\
\hspace*{30mm}SPAN\_GUIDE.LIS}
\end{quote}
If you have any problems or queries you should send CBS-MAIL to {\tt CYM}
at {\tt UK.AC.RL.IB}.

\subsubsection{The ESIS gateway and the database microVAX}
\label{RLESIS}
At the beginning of 1988, a microVAX~II was installed at RAL to provide access
to large catalogues (see SUN/30). The DECnet name of this machine is STADAT
(for STArlink DATabase). A secondary function of this machine is to provide
access to SPAN. More recently another machine (RLESIS) has been installed for
providing network access to the European Space Information System (ESIS). This
also provides access to SPAN. You may or may not be able to access SPAN
directly from your own Starlink system. If you are in DECnet area 19 (ask your
system manager about this), then you will have full access to SPAN. If not, you
will have to access SPAN via RLESIS and STADAT as described here.

VAXs not in DECnet area 19 have limited access to SPAN nodes via RLESIS. For
example, from any Starlink node you can send mail to user SHARP on SPAN node
DRACO (a VAX at NOAO) by the commands
\begin{quote}
{\tt \$ \underline{MAIL}\\[\medskipamount]
MAIL> \underline{SEND RESULTS.TXT}\\
To:     \underline{RLESIS::DRACO::SHARP}\\
Subj:   \underline{Here are the results you wanted}\\[\medskipamount]
MAIL> \underline{EXIT}\\[\medskipamount]
\$}
\end{quote}
Mail can be sent from SPAN to Starlink in a similar way but
it is quite likely that the name RLESIS has not been defined on many SPAN
nodes in which case RLESIS should be replaced by the node number 19527.

This form of routing (known as poor man's routing) does not provide all of the
normal routing functions and consequently not all DECnet operations will work.
The major thing that does not work is `SET HOST', i.e. you cannot do
SET~HOST~RLESIS::DRACO. If you want to log onto a SPAN node you should first
log onto RLESIS and then `SET HOST' to the machine that you want. RLESIS
provides a menu system for doing this when you log on as user ESIS (no
password).

To log into a Starlink node that is not in area 19 from SPAN, you have to log
into STADAT first; however, you cannot use your general node username because
these usernames can only be used {\em from} a Starlink node. Therefore a
username called STARLINK (no password) has been created which can be used from
any SPAN node and can be used to SET~HOST to any Starlink node. If STADAT is
not defined on the remote SPAN node, you should type SET~HOST~19463 instead.
Fortunately, Starlink is migrating all of its systems to DECnet area 19 so that
these routing problems will eventually disappear. 

It should be possible to send mail to computers that are SPAN nodes by using
CBS-MAIL, although the node name would not be that same as the SPAN node name.
See section \ref{post-mail} for the merits of each method.

\subsection{EARN/BITNET/NORTHNET}
\label{bitnet}

EARN (European Academic Research Network) is an IBM based (but with non-IBM
nodes as well) network with a store and forward mail system.
EARN and its USA (BITNET) and Canadian (NORTHNET) equivalents are functionally
the same network.
The Central Computing Division IBM mainframe at RAL is an EARN node and acts as
a mail and file transfer gateway between EARN and JANET.

Mail can be sent to a user on an EARN, BITNET or NORTHNET node with CBS-MAIL
to \\ {\tt CBS\%EARN-RELAY::nodename::username}.
For example, to send a message stored in a file called MESSAGE.TXT to user
TINTIN at EARN node FRIAP51 the following command would be used:
\begin{quote}
{\tt \$ MAIL MESSAGE.TXT CBS\%EARN-RELAY::FRIAP51::TINTIN}
\end{quote}
A guide to using the gateway can be copied from the IBM system at RAL by:
\begin{quote}
{\tt \$ TRANSFER/USERNAME="NETSERV//193" -\\
\hspace*{30mm}RL.IB::"JANET EARNGATE" -\\
\hspace*{30mm}EARNGATE.LIS}
\end{quote}
This guide also describes how to access many other information files including
a list of all EARN and BITNET nodes.
If you have any problems or queries you should send CBS-MAIL to {\tt SUPPORT}
at {\tt UK.AC.RL}.

EARN has gateways to many other networks including NSFNET, UUCP, SUNET and
UNINET. For a complete list and information on how to access them see the
gateway user's guide.

EARN is a `store and forward' network and a message may have to be staged on
several systems before reaching its destination; therefore if a node is down for
any reason messages can suffer a considerable delay, sometimes measured in days!
Also, because the network is IBM based, messages are translated from
ASCII (the character code used on VAX's) to EBCDIC and this can result in
`exotic' characters, such as \{ ~\} and~$\backslash$ being lost or corrupted.

\subsection{IPSS}

The International Packet Switched Stream is the international public X.25
network; and there are gateways between JANET and IPSS at RAL, MCC and ULCC.
Charges are levied and therefore access has to be strictly controlled.

If you are logged in on a computer which is outside JANET (for example if you
are outside the UK) but which has access to IPSS, you can, if locally
authorized, use IPSS to log in to any system on JANET for which you have
authorization.
The DTE number of the PSS gateway at RAL and MCC is 234223519191 and at ULCC is
234219200100 and you should make an X.29 call to these numbers with the
sub-address 69; consult local experts on how to do this.
The gateway will prompt for your authorization code and the address which you
want to call.
For calls from IPSS to JANET no authorization is required and all that is needed
is the address you want to call, preceded by a point (.).
The address can either be a DTE number or a site mnemonic, however, the site
mnemonics that are known to the gateway are the old SERCnet names of computers
rather than NRS names, and these old names no longer work.

The DTE numbers of all the Starlink systems are listed in
appendix~\ref{sercnames}.
For example to call the RAL Starlink VAX the string:
\begin{quote}
{\tt .00000000940010}
\end{quote}
should be typed in response to the gateway prompt.

To call PSS or IPSS hosts from JANET, authorization is required; if you do not
have a personal account, your site manager may have an account to which you can
have temporary access.

A guide to using the gateway can be obtained by:
\begin{quote}
{\tt \$ TRANSFER/USERNAME=NEWS -\\
\hspace*{30mm}JANET.NEWS::"NEWS.GATEWAYS.PSS.GUIDE" -\\
\hspace*{30mm}PSSGUIDE.LIS}
\end{quote}

A few overseas institutions (e.g.\ AAO, ESOC) run the Coloured Books Software so
TRANSFER and CBS-MAIL can also be used, provided you have the appropriate
authorization.
Note that for the AAO a daily mail transfer is made which does not require any
authorization; see section~\ref{oper-mail}.

\subsection{Internet}

The `Internet' is a generic name for the TCP/IP networks which all evolved out
of the US DOD ARPAnet. The original ARPAnet was divided  into networks such as
MILNET (for military traffic), NSFnet (for Universities and similar) and others
but, as the subnetworks are all connected transparently, the Internet still
appears as a single entity. In fact, as the host name addressing is  allocated
on a worldwide basis, Internet sites are now common outside the USA and is
growing rapidly in Europe (outside the UK that is!).

The UK has had a gateway into the Internet for some years. Originally this was
sited at UCL and subject to very strict `per user' vetting. The gateway
(UK.AC.NSFNET-RELAY) is now sited at ULCC and has a fast transatlantic line.
Improvements since moving to ULCC include free and unrestricted transfer of
mail to and from JANET registered sites and a `guestftp' facility which allows
transfer of Internet files (via the relay computers disks) to JANET sites.
Terminal access to Internet sites is still subject to authorisation and 
intending users should mail for advice to {\tt liaison at UK.AC.NSFNET-RELAY}. 
The NETWORKS VAXnotes conference has some hints on using the `guestftp'
facility.

At the time of writing (October 1990) NSFNET-RELAY is being moved from a VAX
(UK.AC.NSFNET-RELAY.VAX) to a SUN (UK.AC.NSFNET-RELAY.SUN). Mail should still
be sent to the generic name NSFNET-RELAY while the `guestftp' service runs on
the SUN only. There are also plans to allow JANET users to transfer files to
and from the Internet without having to login to the gateway.

\subsection{USENET}

USENET is a world wide network of mostly university computer science UNIX
systems using the Unix to Unix Communications protocol (UUCP). A gateway
between USENET and JANET is run by the University of Kent. The network uses
the public telephone system and therefore authorization is required before
the gateway can be used. There are standing charges to to paid for using this
service (roughly \pounds 50 per quarter), so it is not as cheap as using EARN.
More information about the gateway can be obtained
by sending an empty CBS-MAIL message to {\tt information} at {\tt UK.AC.UKC}.

On account of the charges involved in using this gateway, many Starlink sites
do not subscribe to it. Unfortunately the gateway cannot just be ignored. Mail
that is sent from abroad frequently tries to get into the UK via this gateway.
If this happens, the intended recipient of the mail message is sent a message
from Kent saying that a mail message has arrived for him, but that it will not
be forwarded to him as he does not subscribe to the gateway. The sender of the
mail message is not informed that the message did not get through, so both
people are left wondering what is going on. If you get involved in a situation
like this, suggest to your foreign colleague that they try to explicitly send
the mail to you via the EARN gateway at RAL. Alternatively, if they can access
SPAN, then using that should allow the mail to get through.

\appendix

\section{TRANSFER, PAD and POST}
\label{cbs}

These three commands are an implementation of the `coloured books' protocols;
a UK standard.
Before using any of them they must be defined by:
\begin{quote}
{\tt \$ @NET\$DIR:NETSYMB}
\end{quote}
This command can conveniently be inserted in your login command file
(LOGIN.COM), although some sites have this as part of the standard login
sequence.
\subsection{The TRANSFER and LIST commands}
The TRANSFER command enables you to transfer a file between computers.
It uses a software package called FTP (File Transfer Protocol).
It is best illustrated by some examples.
\newtheorem{example}{Example}[subsection]
\begin{example}\rm
To copy a file from a remote computer to your local computer:
\begin{quote}
{\tt \$ TRANSFER/USERNAME=NETUSR,NETUSR -\\
\hspace*{25mm}RL.STAR::STARDISK:[STARLINK.DOCS]SUN36.LIS -\\
\hspace*{25mm}SUN36.LIS}
\end{quote}
Note the use of the special username/password NETUSR, which avoids knowing
anyone's password on the remote machine.
(This is only available when the remote machine is run by Starlink).
Note that you cannot use wild cards (``$*$'' or ``\%'') in the file names.
\end{example}
\begin{example}\rm
You can also copy a file from your computer to a remote one.
This would normally only be appropriate where the account on the remote machine
is your own.
In general it should never be necessary for the recipient to divulge his
password to you: he should `pull' the file instead.

To copy the file PROGRAM.FOR from your default directory on your local computer to
directory DISK\$USER1:[ME] at ROE.
\begin{quote}
{\tt \$ TRANSFER\\
\%\_Input filename ? PROGRAM.FOR\\
\%\_Output filename ? ROE.STAR::DISK\$USER1:[ME]PROGRAM.FOR\\
\%\_Remote Username ? ME\\
\%\_Remote Username password ? WHATEVER\\
File Transfer 35 has been queued}
\end{quote}
The parameter values can be included on the command line as follows:
\begin{quote}
{\tt \$ TRANSFER/USERNAME=ME,WHATEVER PROGRAM.FOR -\\
\hspace*{25mm}ROE.STAR::DISK\$USER1:[ME]PROGRAM.FOR}
\end{quote}
\end{example}
\subsubsection{Qualifiers}
The above examples show the use of the `/USERNAME' qualifier.
There are 26 options available which may be selected by qualifiers.

The following are likely to be the most generally useful:
\begin{description}
\begin{description}
\item[/ACCOUNT=``{\em accountname password\/}'']
\hspace{5mm} This is needed for file transfers with the RAL mainframe computers
(colloquially referred to as `the IBM').
You will be prompted for the {\em accountname} and {\em password} to be used if
they have not been specified.
You may need to ask an IBM user to disclose his password.
Do not reciprocate by telling him yours---the Starlink NETUSR username is
intended to avoid this.
\item[/CODE=FAST]
\hspace{5mm} Use this when transferring executable image files or save sets
between VAX's.
It may also be needed for other file formats.
We recommend you always use this qualifier when transferring files between
two VAXs as it speeds up the transfer.
\item[/INFORM]
\hspace{5mm} You will be sent a MAIL message telling you that a transfer attempt
has finished and whether or not it was successful.
You will always be told of an unsuccessful completion, even if you do not use
this qualifier.
\item[/USERNAME={\em username,password}]
\hspace{5mm} Specifies the username and password required to login to the
remote computer.
\end{description}
\end{description}
More information can be found by typing HELP TRANSFER or by consulting the
Coloured Books User Guide; your site manager will have a copy.
\subsubsection{LIST}
The LIST command allows you to monitor requested transfers.
To obtain a list of your transfers in the queue, type:
\begin{quote}
{\tt \$ LIST}
\end{quote}
To obtain a list of all the entries in the queue, type:
\begin{quote}
{\tt \$ LIST/ALL}
\end{quote}
To obtain detailed information on a particular transfer, type:
\begin{quote}
{\tt \$ LIST/ENTRY={\em queue\_entry\_number}/FULL}
\end{quote}
\subsection{PAD}
PAD (Packet Assembler Disassembler) is a program that allows you to login to
remote computers from your local VAX.
The following example shows how to call ROE.STAR (the ROE Starlink VAX)
from your local node:
\begin{quote}
{\tt \$ \underline{PAD}}
\end{quote}
This command invokes the PAD program.
The program responds with a message and a prompt:
\begin{quote}
{\tt PAD>}
\end{quote}
You respond with a request for the desired computer:
\begin{quote}
{\tt PAD>\underline{CALL ROE.STAR}\\
Call connected}
\end{quote}
You will now receive the usual `username' and `password' prompts as if you were
logging into your local VAX.
Once logged in successfully, you can use the remote computer as if you were
logged into it locally.

You end the session by logging out in the normal way.
Then you will get the PAD prompt again and can call another node, set a
parameter, or exit from the PAD program:
\begin{quote}
{\tt PAD>\underline{EXIT}}
\end{quote}
If you have any difficulty in clearing a call to another node, try the
following:
\begin{quote}
{\tt \underline{<CTRL/P>}\\
PAD>\underline{CLEAR}}
\end{quote}
You can invoke the PAD prompt at any time while running the PAD program
by typing control P.
\subsection{POST and CBS-MAIL}
POST and CBS-MAIL are two different ways of achieving the same end, i.e.\
sending a Coloured Books mail message.
Although it is generally more convenient use CBS-MAIL for interactive use
there is no need to change command procedures that use the POST command.
The major advantage of Coloured Books mail over VMS mail is that you can reach 
more computers. VMS mail uses DECnet and so is restricted to machines that are
logically connected to the Starlink DECnet, although these days that includes
SPAN and HEPnet. Coloured Books mail will in principle allow you to get to any
computer that is connected to a network anywhere in the world, although in some
cases the address that you have to type can be very complicated.

The only difference between CBS-MAIL and the POST command is the user
interface. The mail is sent out in the same way and in principle, the recipient
cannot tell which method you used to send it. Like TRANSFER, you can use the
LIST command to check the progress of your message.

To send a CBS-MAIL message you need to send mail to 
{\tt CBS\%$<$NRS node name$>$::$<$username$>$}.
For example,
\begin{quote}
{\tt \$ \underline{MAIL}\\[\medskipamount]
MAIL> \underline{SEND TEST.LIS}\\
To:     \underline{CBS\%ROE.STAR::OPER}\\
Subj:   \underline{This is a test of CBS-MAIL}\\[\medskipamount]
MAIL> \underline{EXIT}\\[\medskipamount]
\$}
\end{quote}
The particularly useful feature of this interface is that you can reply to
incoming Coloured Book mail as the software works out the return address for
you. You can even use this to send mail to Bitnet. In this case you should send
mail to {\tt CBS\%EARN-RELAY::$<$bitnet node name$>$::$<$username$>$}. Again,
you can reply to incoming Bitnet mail.

If you prefer, you can send Coloured Books mail using the POST command. 
The following example shows how to send mail to username OPER at ROE using POST,
where the message is typed in directly on your terminal:
\begin{quote}
{\tt \$ \underline{POST}\\
\$\_ Remote username(s) : \underline{OPER}\\
\$\_ At which site ?    : \underline{ROE.STAR}\\
Enter your message below. Press CTRL/Z when complete:\\
\underline{TESTING POST FROM RL.STAR ME TO ROE.STAR OPER}\\
\underline{<CTRL/Z>}\\
Mail request 434 has been queued}
\end{quote}
This message could have been stored previously in a file, say TEST.LIS, and sent
as follows:
\begin{quote}
{\tt \$ \underline{POST TEST.LIS}\\
\$\_ Remote username(s) : \underline{OPER}\\
\$\_ At which site ?    : \underline{ROE.STAR}\\
Mail request 435 has been queued}
\end{quote}
You can send messages to more than one username by typing a list of usernames,
separated by commas.

\subsubsection{Using CBS-MAIL on VAXclusters}

The above section assumes that you are logged onto a computer that is licensed
to run the Coloured Books software. At the present time, most Starlink sites
consists of a VAXcluster rather than just a single VAX computer. On account of
the high cost of the software, Coloured Books is only licensed to run on a
single computer in each cluster. This is usually the boot node (or one of the
boot nodes if you have a dual microVAX 3400 system), but this need not be the
case. Having the software licensed for only a single CPU is not a great
problem, but it can be annoying when reading a mail message that was sent via
Coloured Books, on a machine that is not licensed to run Coloured Books, to be
unable to reply to the message without logging into another computer. This is
sufficiently annoying that many people log into  the machine that has Coloured
Books when reading mail just in case they need it. In fact this is not
necessary as you can send Coloured Books mail from any node in a cluster by
employing a trick of DECnet. Suppose that your cluster consists of two nodes, a
boot node boot MAV1 and a satellite MAV2, and that it has a cluster alias of
MAVAD. This is of course part of the Manchester cluster. MAV1 runs the Coloured
Books software, but you are logged onto MAV2. You can send a Coloured Books
mail message as follows:

\begin{quote}
{\tt \$ \underline{MAIL}\\[\medskipamount]
MAIL> \underline{SEND}\\
To:     \underline{MAV1::CBS\%ROE.STAR::OPER}\\
Subj:   \underline{This is a test of CBS-MAIL on a satellite}\\[\medskipamount]
etc.\\}
\end{quote}

The point to note is that you send the mail message to node MAV1, which
forwards the mail to the intended recipient using Coloured Books. This does not
break the conditions of the software licence, but it does come into the class
of `nifty tricks', so is prone to breaking when the software is changed.
However, as something that costs nothing, it is well worth using.

There are some imperfections with this method of using Coloured Books from a
satellite that you should be aware of. When you receive a normal CBS mail
message, it says it has come from someone like {\tt
CBS\%UK.AC.\-MANCHESTER.\-ASTRONOMY.\-STARLINK::PMA} (the computer generates
the long form of the name). This means that if you read this message on a
machine that does not run Coloured Books, you cannot simply type {\tt REPLY} to
reply to the mail message, however, you can send a new message using the
technique given above. It is of course possible to use the short form of the
name if you know it.

If you do send a mail message using this technique, it will appear to come from
{\tt CBS\%UK.AC.\-MANCHESTER.\-ASTRONOMY.\-STARLINK::"MAVAD::PMA"}. The
recipient of the message (FRED at Leicester, let's say) can reply directly to
this message if they are reading it on the machine that runs Coloured Books,
but you will then get a reply that appears to come from {\tt
MAVAD::CBS\%UK.AC.\-LEICESTER.\-STARLINK::FRED}.  It may or may not be possible
to {\tt REPLY} to this message. What is even more confusing is that, depending
on the way that the network is set up at your particular site, it may work at
sometimes and not at other times. In fact any attempt to bat messages back and
forth using {\tt REPLY} is likely to be doomed to failure in the end.
Consequently, if you receive a message that look like it has been sent in this
fashion, it would be as well to {\tt SEND} a new message rather than try {\tt
REPLYing} to it. Despite these warnings, if you always send a new message,
rather than try replying to old ones, you should have no problems. 

\section{Interaction of X.29 terminals protocols with VAX/VMS}
\label{pad}

X.29 is the CCITT recommendation for the terminal protocol using an X.25
network.
When using a screen editor (and this includes command line editing provided by
VMS) the echoing of characters on the screen has to be under complete control
of the editor program.
The only way to achieve this on an X.25 network is for each character to be
transmitted across the network as it is typed and the echo to be transmitted
back.
This results in two packets traversing the network for each key pressed, which is
extremely inefficient use of the network.
If the public packet switched system is involved it is also very expensive.
When editing facilities are not required, it is far more efficient for the
terminal to provide the echoing and for data to be transmitted only when
carriage return (and some other special characters) are typed.
This mode of working is selected by the command:
\begin{quote}
{\tt \$ SET TERMINAL/LOCAL\_ECHO}
\end{quote}
Before selecting this mode, it is important to ensure that the appropriate set
of characters causes the current line to be transmitted.
This is done by:
\begin{quote}
{\tt \$ SET TERMINAL/X29/PARAMETER=(FORWARD=126)}
\end{quote}
This only needs to be done once during a terminal session; the /X29 qualifier
must come before /PARAMETER.
To enable screen editing use the command:
\begin{quote}
{\tt \$ SET TERMINAL/NOLOCAL\_ECHO}
\end{quote}
If you are connected via a network which does not fully support X.29, such as a
campus local area network, this may not work correctly.
If your terminal displays symptoms such as no echo appearing until carriage
return is pressed, you must work in LOCAL\_ECHO mode.
The default mode when logging into a VAX running VMS Version 4 is NOLOCAL\_ECHO,
although this may be changed to LOCAL\_ECHO by the system wide login command
procedure set up by the system manager.

\section{Additional DECnet facilities}
\label{shownet}

There are three DECnet commands provided by Starlink as an extension to
the standard DEC utilities.
\subsection{SHOWNET}
The command SHOWNET will list all the DECnet nodes and their current state,
either reachable or unreachable.
For nodes which are not Starlink systems this information may not always be
available.
\subsection{TALK}
TALK is a program that allows you to display a message on another user's
terminal.
It has two modes of operation:
\begin{quote}
{\tt \$ TALK {\em user message}}
\end{quote}
which sends a single line, and a conversational mode:
\begin{quote}
{\tt \$ TALK {\em user}\\
<message ......\\
\hspace{16mm}.......>\\
<CTRL/Z>}
\end{quote}
in which every line typed is sent until control Z is typed.
{\em user} (which will be prompted for if not given) can be either the
username of someone who is logged on or the device name of a terminal (e.g.\
TTA3, RTA3 etc.\ ).
If a username is used and that person is logged on to more than one terminal,
only one terminal will receive the message.
For example:
\begin{quote}
{\tt \$ TALK PTW Your meeting is about to start}
\end{quote}
will cause a message similar to:
\begin{quote}
{\tt From RLSTAR::LTA10: DLT          12:18:13\\
YOUR MEETING IS ABOUT TO START}
\end{quote}
to appear on PTW's terminal.
The message appears in upper case because it has been processed by DCL.
To send a message containing lower case or special characters the message must
be enclosed in quotation marks, eg:
\begin{quote}
{\tt \$ TALK PTW "Your meeting has started!"}
\end{quote}
In conversational mode the message is read directly by the program and it is
transmitted without alteration.

The username or device name can include a DECNET node name (e.g.\ MAV1::PMA) in
which case the message will sent to the user on the remote node. This means
that it is not necessary to log on to a remote node to talk to someone on that
node. Note that you must use the specific node name, not the cluster alias.

Many people find TALK annoying---it interrupts whatever they are doing and
demands an instant reply. TALK messages can also get lost. If the
screen gets refreshed at just the wrong moment then the message will be
overwritten. MAIL is preferable for most purposes.
\subsection{NETSHOW}
NETSHOW is a command that allows you to execute the equivalent of the DCL SHOW
command on a remote Starlink node.
The form of the command is:
\begin{quote}
{\tt \$ NETSHOW {\em nodename item}}
\end{quote}
where {\em item} is any of the parameters that can be given to the SHOW command.
For example:
\begin{quote}
{\tt \$ NETSHOW RLSTAR USER}
\end{quote}
will list all the users logged on to RLSTAR.
If {\em item} contains a `/' character (as in for example, SHOW MEMORY/POOL)
the item must be enclosed in quotation marks thus:
\begin{quote}
{\tt \$ NETSHOW DUVAD "MEMORY/POOL"}
\end{quote}
Commands which involve interaction with the terminal such as SHOW
PROCESS/CONTINUOUS will not work nor will commands which are executed entirely
by the command interpreter such as SHOW TIME and SHOW TRANSLATION.
NETSHOW is a Starlink specific utility and although it makes use of DECnet, it
will not function on non-Starlink computers.

\section{DTE numbers of Starlink Nodes}
\label{sercnames}

On rare occasions you may find that you have to use the explicit DTE for a
Starlink node. These are:

\begin{center}\begin{tabular}{l@{\hspace{1cm}}l}
Armagh & 000011500001 \\
Belfast & 000011200401 \\
Birmingham & 000020013301 \\
Cardiff & 000060305101 \\
Cambridge & 000008005001 \\
Durham & 000014000300 \\
Hatfield Polytechnic & not yet known \\
Imperial College & 000005106004 \\
Keele & 000010302000 \\
Kent & 000049200300 \\
Jodrell Bank & 000010124001 \\
Lancashire Polytechnic & 000014500032 \\
Leicester & 000021213000 \\
Manchester & 00001010900111 \\
Oxford & 000050251100 \\
QMW & 000005121062 \\
RAL (Project) & 00000000940010 \\
RAL (Astrophysics) & 00000000940011 \\
RAL (Database) & 00000000940012 \\
ROE &00000700200155 \\
St.Andrews & 000007007061 \\
Southampton & 000050315020 \\
Sussex & not yet known \\
UCL & 000002005002 
\end{tabular}\end{center}

\section{AAO/Starlink MAIL service}
\label{aao-mail}

Starlink and the Anglo-Australian Observatory jointly operate a service which
allows the exchange of MAIL messages between users of the Starlink and the AAO
VAXs.
At each end, messages are sent to a holding account and are later transmitted
en masse.
As with the telex service, a daily turnaround is possible, governed mainly by
the large difference in local time.

The X.25 communication route is Starlink: JANET - PSS - IPSS - MIDAS - AUSTPAC
- AAO.
The PAD, CBS-MAIL and TRANSFER software is used in both directions.
\subsection{How to send messages to the AAO}
All you have to do is to send your message to account AAOMAIL at RAL, specifying
to whom the message has to go in the `Subj:'.

For example, here is one way to send a message prepared beforehand in file
WHINGE.TXT to someone at Epping called Bruce Billabong:
\begin{quote}
{\tt \$ \underline{MAIL}\\[\medskipamount]
MAIL> \underline{SEND WHINGE.TXT}\\
To:     \underline{RLVAD::AAOMAIL}\\
Subj:   \underline{for Bruce Billabong}\\[\medskipamount]
MAIL> \underline{EXIT}\\[\medskipamount]
\$}
\end{quote}
Because the messages are sent to a holding account (and sometimes are for
individuals without accounts on the VAX), automatic transmission direct to the
addressee is not possible, and an operator at each end has to handle
distribution.
Be specific with names, especially with visiting astronomers---initials or
first names may not be enough.
A full character set can be used---but don't use fancy escape sequences and
non-printable characters.
\subsection{How to receive messages from the AAO}
You don't have to do anything---if a reply is sent it will be forwarded to your
Starlink account by the local operator.
Note, however, that it is not possible to use the REPLY command in the MAIL
utility---you will simply send the reply to the operator's account on Starlink
and it will not go to Australia.
Use SEND to transmit your reply to AAOMAIL.
\subsection{How to send messages from the AAO}
If you are visiting the AAO, ask for current operating instructions for the
AAO/Starlink MAIL service.
At present the procedure is simply for you to MAIL your message to the account
UKMAIL if you are at Epping, or to the account EPPINGMAIL if you are at the AAT.
\subsection{Reliability}
Under normal circumstances the service is operated on every working day.
Occasionally---every few weeks---a day will be missed for one reason or another;
the VAX either end might be down, there might have been network problems in PSS,
the gateway might be down, there may be a public holiday, no operator
may be available.
If longer lapses occur, node managers are informed.
The service is very reliable, but really urgent things should be sent by telex,
or acknowledgment requested.
Garbling and bad characters do not occur at all.
\subsection{Cost}
The AAO pay for the UK:Australia traffic, and Starlink pays for the
Australia:UK traffic.
It currently costs the AAO 14 cents per minute they are connected, and
1.2 cents per segment (64 characters).
The costs to Starlink are roughly comparable---10p per minute and 0.4p per
segment.
Messages should be concise, though telegraphese can be avoided.
The production of multiple copies should be left to the remote operator.
Very short papers and programs may be sent whole, though whenever possible it is
better to send appropriate extracts.
\subsection{Timing}
Messages for Australia will arrive on the next working day if sent to UKMAIL by
about midnight UK time.
If the recipient in Australia replies the same day (by about 18:00), the reply
will be picked up by the UK operator and distributed at about 09:00 UK time on
the same date.
Messages for Coonabarabran are forwarded the same day and a reply, providing it
is sent to the EPPINGMAIL account by about 15:00, will be received in the UK on
the same day.
These times vary somewhat due to local daylight saving time conventions.
\subsection{Security}
Normal unprivileged users cannot read the AAO--Starlink traffic.
Privileged software staff could access messages but do not in practice do so
unless there is a problem.
The operators see the messages.

Hardcopy of the traffic in both directions is kept at the AAO, but access is
restricted.
No hardcopy is kept in the UK.

\section{JTMP availability}
\label{who-has-JTMP}

The JTMP software is available as follows:

\begin{table}[htbp]
\caption{Availability of JTMP}
\begin{center}\begin{tabular}{l@{\hspace{1.5cm}}l}
Armagh & yes\\
Belfast & yes\\
Birmingham & yes\\
Cambridge & yes\\
Cardiff & no\\
Durham & yes\\
Hatfield Polytechnic & no\\
Imperial College & yes\\
Jodrell Bank & yes\\
Kent & no\\
Keele & no\\
Lancashire Polytechnic & no\\
Leicester & yes\\
Manchester & yes\\
Oxford & yes\\
QMW & yes\\
RAL (Project) & yes\\
RAL (Astrophysics) & yes\\
RAL (Database) & no\\
ROE & yes\\
St.\ Andrews & yes\\
Southampton & no\\
Sussex & no\\
UCL & yes
\end{tabular}\end{center}\end{table}

\end{document}
