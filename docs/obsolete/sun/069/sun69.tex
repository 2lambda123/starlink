\documentstyle[11pt]{article}
\pagestyle{myheadings}

%------------------------------------------------------------------------------
\newcommand{\stardoccategory}  {Starlink User Note}
\newcommand{\stardocinitials}  {SUN}
\newcommand{\stardocnumber}    {69.4}
\newcommand{\stardocauthors}   {Clive G. Page\\University of Leicester}
\newcommand{\stardocdate}      {19th April 1993}
\newcommand{\stardoctitle}     {CDCOPY --- Reading CDROMs}
%------------------------------------------------------------------------------

\newcommand{\stardocname}{\stardocinitials /\stardocnumber}
\renewcommand{\_}{{\tt\char'137}}     % re-centres the underscore
\markright{\stardocname}
\setlength{\textwidth}{160mm}
\setlength{\textheight}{230mm}
\setlength{\topmargin}{-2mm}
\setlength{\oddsidemargin}{0mm}
\setlength{\evensidemargin}{0mm}
\setlength{\parindent}{0mm}
\setlength{\parskip}{\medskipamount}
\setlength{\unitlength}{1mm}

%------------------------------------------------------------------------------
% Add any \newcommand or \newenvironment commands here
%------------------------------------------------------------------------------

\begin{document}
\thispagestyle{empty}
SCIENCE \& ENGINEERING RESEARCH COUNCIL \hfill \stardocname\\
RUTHERFORD APPLETON LABORATORY\\
{\large\bf Starlink Project\\}
{\large\bf \stardoccategory\ \stardocnumber}
\begin{flushright}
\stardocauthors\\
\stardocdate
\end{flushright}
\vspace{-4mm}
\rule{\textwidth}{0.5mm}
\vspace{5mm}
\begin{center}
{\Large\bf \stardoctitle}
\end{center}
\vspace{5mm}

\section{Introduction}

The compact disc is an increasingly popular medium for the distribution
of astronomical data, especially for large files such as source
catalogues and images. CDs can be pressed cheaply in quantity and hold
around 600 megabytes each.  ISO 9660 specifies a standard physical
layout and file structure which is almost universally used (although DEC
is a notable exception). The advent of recordable CD-ROM may provide us
with an inexpensive and convenient archiving medium.

Although CD-ROM readers such as DEC's RRD40 are physically capable of
reading these discs, VMS cannot (at present) cope with the ISO 9660 file
structure. DEC have produced an alternative device driver in an attempt
to overcome this limitation but it does not handle the file
characteristics satisfactorily, and the software is unreliable and
unsupported. There are rumours that the next major release of VMS will
include proper support for ISO standard CD-ROMs. Until then, the {\tt
CDCOPY} program described here is probably the best alternative for
VAX/VMS users. It can copy files from ISO 9660 discs to other devices
and provides several options for format conversion on the fly.

Version 3.6 of {\tt CDCOPY} fixes several minor bugs and in particular
supports the IRAS survey CD-ROMs which have image arrays with unity as
the first dimension. Unfortunately {\tt KAPPA} still cannot handle axes
described by spaced arrays (the natural translation for FITS axis
information) -- but {\tt ASTERIX} will work on these (advt).
I am grateful to James Albinson for helping to test some of the latest
modifications.

\section{Running CDCOPY}

If the software has been installed correctly and you have executed the
Starlink {\tt LOGIN.COM} file then the name {\tt CDCOPY} will be set as
a DCL symbol and the logical name {\tt CD\_DEV} will be defined as the
name of a CD reader on your system. If your system has more than one CD
reader then this default may not be suitable and you may need to use the
physical device name in these commands, or redefine the logical {\tt
CD\_DEV} to point to the required unit.  It is not necessary to use
explicit {\tt ALLOCATE} or {\tt MOUNT} commands before using the
program, but use of {\tt ALLOCATE} and {\tt DEALLOCATE} may be desirable
to avoid clashes with other users attempting concurrent access to the
same drive.

{\tt CDCOPY} {\em must} be used as a VMS foreign command with a switch
after the command-name to select the operation required; the input and
output file specifications follow on the same line and both have
defaults.  For example:
\begin{verbatim}
$ CDCOPY/DIR
$ CDCOPY/TEXT  [LISTS.VMS]SRCS.LIS  TEMP.TXT
\end{verbatim}
The switch may be abbreviated to a slash plus a single letter. Upper and
lower-case letters are everywhere treated as equivalent.

The input specification may include a device, directory path, and
filename. It is designed to resembles a normal VMS specification but
there are some restrictions. The default input device is {\tt CD\_DEV}
which may be omitted if that logical name has been defined appropriately.
The default directory path is {\tt [}\hspace{1mm}{\tt ]} which
represents the (nameless) root directory on the disc; the path may
include {\tt *} as a wild-card, but the notation {\tt [...]} is only
supported at the top level (to allow a search of the entire disc). The
filename part may include wild-card characters {\tt *} and {\tt \%} to
allow a number of files (of the same type) to be copied in one
operation.

If the output specification is omitted then {\tt /VOL, /PATH} and {\tt
/DIR} options send their output to \verb+SYS$OUTPUT:+ (normally your
terminal), while the copy options create files in your current working
directory; by default the file has the same name as on the CD-ROM. A
logical name or directory specification may be used to select another
destination; an alternative output file-name may be given if you are
only copying one file. Note that wild-card characters are {\em not}
allowed in the output specification.

\section{Listing the Volume Header and Path Table}

There are two options to help you find your way around a disc:
{\tt CDCOPY/VOL} lists the volume header information and {\tt CDCOPY/PATH}
lists the path table (every subdirectory path on the disc). For example:
\goodbreak
\begin{verbatim}
$ CDCOPY/VOL
CDCOPY version 3.6
 Descriptor type               1
 Standard identifier    CD001
 Descriptor version            1
 System identifier
 Volume identifier      USA_AURA_STSI_GSC1_0002
 Volume space size       1208610
 Volume set size               1
 Volume sequence No.           1
 Logical block size          512
 Path table size             190
 Path table location          72
 Volume set              USA_AURA_STSI_GSC1_0002
 Publisher               SPACE TELESCOPE SCIENCE INSTITUTE
 Data prep.              SPACE TELESCOPE SCIENCE INSTITUTE
 Application
 Copywrite file
 Abstract file
 Bibliog. file
 Creation date/time     1989060100000000
\end{verbatim}

\goodbreak
\begin{verbatim}
$ CDCOPY/PATH
CDCOPY version 3.6
Volume path table
  #  address    path
   1      80
   2      88  GSC
   3      96  TABLES
   4     104  GSC.S0730
   5     160  GSC.S1500
   6     216  GSC.S2230
   7     272  GSC.S3000
   8     324  GSC.S3730
   9     372  GSC.S4500
  10     416  GSC.S5230
  11     456  GSC.S6000
  12     488  GSC.S6730
  13     512  GSC.S7500
  14     532  GSC.S8230
\end{verbatim}
The example shown above comes from volume 2 of the first issue of the
NASA Guide Star Catalog; many path tables are much longer than this.
Note that the first path is the (un-named) root.  The {\tt address} is
the initial sector number of the file which is only provided
for diagnosic purposes.

\section{Listing Directories}

Often the next step will be to use {\tt CDCOPY/DIR} to list one or more
sub-directories.  The default input specification is {\tt CD\_DEV:[ ]*}
which lists all files in the root directory.  The output will be sent to
your terminal unless you name an output file.

Examples of use:
\begin{verbatim}
$ CDCOPY/DIR
$ CDCOPY/DIR  DKB0:[GSC.S1500]1%%%.*
$ CDCOPY/DIR  [GSC.*]*.TXT     ALL.TEXT
\end{verbatim}
The following example shows a directory on the ADC catalog disc, volume
1 number 2.
\goodbreak
\begin{verbatim}
$CDCOPY/DIR [NONSTELL.GALAXIES.IRASGAL]
CDCOPY version 3.6
Directory of CD_DEV:[NONSTELL.GALAXIES.IRASGAL]*
    #  ----filename----             ---date/time----   length  rtyp cc reclen
    3  GALQSO.DOC                   1991-10-03 17:41    38458     0  0     0
    4  GALQSO.FIT                   1991-10-08 11:14  3916800     0  0     0
    5  SUPPL                    dir 1991-10-08 10:50     2048     0  0     0
\end{verbatim}

The sequential numbering starts at 3 because files 1 and 2 are the
directory itself and its parent.  The letters ``{\tt dir}'' at the end
of the file-name identifies a file which is a sub-directory, such as {\tt
SUPPL} in the example above.

{\tt length} is the file size in bytes (divide by 512 to get VMS
blocks).

{\tt rtyp} is the record type: 0 means no information; 1 means
fixed-length of {\tt reclen} bytes; other values indicate
variable-length records.

{\tt cc} shows the carriage-control type for text files: 0 means no
information or fixed length, 1 means Fortran-style, 2 means embedded
carriage-control characters.  {\tt CDCOPY} should handle all these cases
and others.

\section{Copying Files}
A switch must be used to specify the transfer format.  If you forget
which switches are available then the command {\tt CDCOPY} alone gives a
brief summary.  A switch may be abbreviated to the first letter but it must
appear directly after the command word and not after the input or output
specifications as is permitted in internal VMS commands.

Wild-card characters {\tt *} and {\tt \%} are permitted in the input
file specification to allow a number of files (of the same type) to be
copied in one operation, but not in the output specification.
The output device is usually a disc but text files may be
copied to your terminal by using {\tt TT:} as the output device.

\begin{description}
\item {\tt CDCOPY/BINARY} produces an unformatted direct-access file
with a record length of 512 bytes. It may be used on all types of file
but you may need to write your own software to process the results.

\item {\tt CDCOPY/FITS} copies FITS files to disc, producing unformatted
direct-access files with records of 2880 bytes.  No byte swapping is
performed, this is the responsibility of any software that reads the
disc files.

\item {\tt CDCOPY/GSC} is designed to transfer the FITS tables in the
NASA Guide Star Catalog to text files, but should work on any FITS ASCII
table file which is under 5 megabytes in length. The output text file
includes the all the fits header lines as well as the tabular data. If
used on a FITS file which is not an ASCII table it will just copy the
header lines.  If the file contains more than one ASCII table in
sequence than all will be copied to the same output file, each preceded
by its own extension header.

\item {\tt CDCOPY/HDS} is designed to copy the image files on the SAO
Einstein Observatory discs to HDS files, but should work with any FITS
image files with 8-bit or 16-bit image arrays {\em (but note that it does
not work on FITS tables, only images)}. If FITS parameter {\tt BSCALE}
is unity and {\tt BZERO} is zero then 8 or 16-bit data will produce a
{\tt DATA\_ARRAY} of {\tt \_UBYTE} or {\tt \_WORD} type respectively in
order to reduce output file size, otherwise the scaling factors will be
used to create a {\tt \_REAL} array.  The header values are put in the
{\tt .MORE.FITS} structure as 80-character strings.  Where possible
standard NDF {\tt AXIS} arrays are also created to allow use with {\tt
ASTERIX} or {\tt KAPPA}.  The celestial coordinates of images from the
Einstein discs (or others with similar FITS components) will also be
copied to the {\tt MORE.ASTERIX.HEADER} structure (and will be precessed
to equinox 2000 if necessary).

\item {\tt CDCOPY/TEXT} transfers text to formatted sequential files.
The program is designed to handle fixed record lengths (if
correctly described in the file header) or lines of varying length which
end with carriage-return or line-feed or both.  No other control
characters are removed, so tabs and form-feeds should be preserved. The
VMS file will have {\it list} style carriage-control (unless {\it
Fortran} style is specified in the CD-ROM extended attribute block.

\end{description}
A few examples are shown here:
\begin{verbatim}
$ CDCOPY/TEXT  README.TXT  TT
$ CDCOPY/TEXT  [LISTS.VMS]SRCS.LIS
$ CDCOPY/BIN   [SOFTWARE.VMS.BINARIES]SAOIMAGE.EXE
$ CDCOPY/FITS  [ASTROM.SAO]*.FIT      DISK$TEMP:[ABC]
$ CDCOPY/GSC   $2$DKB0:[GSC.*]526%.*  SYS$SCRATCH
$ CDCOPY/HDS   [DATA.07H]I0700S05.XIA  .SDF
\end{verbatim}
Note that the first example copies a text file to the terminal (device
{\tt TT}), while the last one produces an HDS file called {\tt I0700S05.SDF}
to conform to the Starlink convention for file extensions.

\section{Limitations}
{\tt CDCOPY} mounts a disc specifying VMS {\em foreign} format which
restricts access to one user at a time. If CDCOPY finishes normally it
will dismount the disc automatically (but without unloading it since
this causes trouble in some drives).  If, however, you abort CDCOPY
(or it fails in some way) it may be necessary to use an explicit VMS
{\tt DISMOUNT} command to dismount the disc afterwards.  You should not
try to swap discs while VMS still has a disc mounted.

The {\tt /BINARY, /TEXT} and {\tt /FITS} options will copy files of any
length, but the {\tt /HDS} and {\tt /GSC} options rely on reading the
whole file into virtual memory before format conversion, so they are
currently restricted to files no more than 5 megabytes in size.  This
limit could be increased by altering a parameter in the source-code and
re-compiling, but the program might then use excessive amounts of
virtual memory.

Most CD-ROM drives are rather slow, being based on those in domestic
hi-fi systems.  Typical copying speeds are 1.5 to 2 megabytes/minute, but
your mileage may vary. Elaborate error-correction is employed (each byte
of information is represented by at least 17 bits on the disc) so
reading errors are rare if not entirely unknown.

\end{document}
