\documentstyle{article}
\pagestyle{myheadings}

%------------------------------------------------------------------------------
\newcommand{\stardoccategory}  {Starlink User Note}
\newcommand{\stardocinitials}  {SUN}
\newcommand{\stardocnumber}    {131.1}
\newcommand{\stardocauthors}   {D J Rawlinson, C A Clayton}
\newcommand{\stardocdate}      {12 August 1991}
\newcommand{\stardoctitle}     {Printing on a PostScript\footnote{PostScript
is trademark of Adobe Inc.} laser printer.}
%------------------------------------------------------------------------------

\newcommand{\stardocname}{\stardocinitials /\stardocnumber}
\renewcommand{\_}{{\tt\char'137}}     % re-centres the underscore
\markright{\stardocname}
\setlength{\textwidth}{160mm}
\setlength{\textheight}{240mm}
\setlength{\topmargin}{-5mm}
\setlength{\oddsidemargin}{0mm}
\setlength{\evensidemargin}{0mm}
\setlength{\parindent}{0mm}
\setlength{\parskip}{\medskipamount}
\setlength{\unitlength}{1mm}

%------------------------------------------------------------------------------
% Add any \newcommand or \newenvironment commands here
%------------------------------------------------------------------------------

\begin{document}
\thispagestyle{empty}
SCIENCE \& ENGINEERING RESEARCH COUNCIL \hfill \stardocname\\
RUTHERFORD APPLETON LABORATORY\\
{\large\bf Starlink Project\\}
{\large\bf \stardoccategory\ \stardocnumber}
\begin{flushright}
\stardocauthors\\
\stardocdate
\end{flushright}
\vspace{-4mm}
\rule{\textwidth}{0.5mm}
\vspace{5mm}
\begin{center}
{\Large\bf \stardoctitle}
\end{center}
\vspace{5mm}

%------------------------------------------------------------------------------
%  Add this part if you want a table of contents
%  \setlength{\parskip}{0mm}
%  \tableofcontents
%  \setlength{\parskip}{\medskipamount}
%  \markright{\stardocname}
%------------------------------------------------------------------------------
\section{Introduction}

The document describes how to print text, graphics and \TeX\/ style output on a
PostScript laser printer using a Starlink-distributed PostScript print
symbiont. Throughout the document it is assumed that your PostScript laser
printer queue is called {\tt SYS\_PS}. This may not be the case at your site
and you should consult your Site Manager it you are in any doubt as to which
queue to use.

\section{Printing Text}

To print an ASCII text file on a PostScript printer using the print symbiont
you need simply issue the command:

\begin{verbatim}

          $ PRINT/QUE=SYS_PS filename

\end{verbatim}

Note that this is precisely the same command you would use to print a text file
on a line printer, with the exception that you are specifying an alternative
queue. An ASCII text file that is sent directly to a PostScript printer will be
interpreted by the printer as a PostScript program that it should execute.
Hence, to print a text file, it is necessary to first encapsualte it in a piece
of PostScript code that instructs the printer that the file you have sent is
data to be printed and not PostScript code to be executed. This is done
transparently for you by the PostScript print symbiont.

There are a number of optional FORMS available with the print symbiont. These
print files using different templates. For example, the form HEADERS prints
the filename/date/page number at the top of the page. Parameters can be passed
to these forms to specify, for example, the number of copies, font size in
points etc.

The following forms are defined for text printing. Parameters are passed as
follows on the print command:

\begin{verbatim}
          $ PRINT/QUE=SYS_PS/FORM=form_name/PARAM=(parameter=value,...)
\end{verbatim}
\begin{tabbing}
XXXXXX\=XXXXXXXXXXXXXXXXXXXXX\= \kill
\>{\bf\underline{Form}}            \>{\bf\underline{Parameters}}\\
\\
\>HEADERS   \>Print file with filename and page number headers.\\
            \>\>{\it ncopies:}   Number of copies. (default: 1)\\
            \>\>{\it fsize:  }   Font size in points. (default: 10)\\
            \>\>{\it wide:   }   If true set landscape mode. (default: false)\\
            \>\>{\it title:  }   If false suppress the page headers. (default:
true)\\
            \>\>{\it ncolumns:}  Number of columns. (default: 1)\\
            \>\>{\it linelimit:} Max number of lines per page. (default 66)\\
\\
\>NOHEADERS \>  Same as HEADERS except default title to false.\\
            \>\>(NOHEADERS is the default)\\
\\
\>LANDSCAPE \>  Same as HEADERS except default wide to true, fsize to 8.\\
\\
\>2UP       \>  Same as LANDSCAPE except default ncolumns to 2.\\
\\
\end{tabbing}

For example,

\begin{verbatim}
  $ PRINT/QUE=SYS_PS/FORM=HEADERS/PARAM=(NCOPIES=4,FSIZE=14) STUFF.TXT
\end{verbatim}
-- Prints 4 copies of the file STUFF.TXT with a font size of 14 (points)

\section{Printing PostScript files}

PostScript files can be printed (executed) using the form POST. For example,

\begin{verbatim}
  $ PRINT/QUE=SYS_PS/FORM=POST SCREENDUMP.PS
\end{verbatim}

Those wishing to delve deeper into the workings of PostScript may want to
make use of the /NOTE option. In the example below, the first line sent to
the printer will be ``debug true def''. This will be executed before the
user's PostScript file.

\begin{verbatim}
  $ PRINT/QUE=SYS_PS/FORM=POST/NOTE="/debug true def" USER.PS
\end{verbatim}

\section{Printing Starlink graphics}

To print Starlink graphics on the PostScript printer, you must select
one of the GKS devices from the following list when running
the software that produces the plot.

\begin{itemize}
\item  POSTSCRIPT\_P    Postscript 300 dpi A4 portrait
\item  POSTSCRIPT\_L    Postscript 300 dpi A4 landscape
\item  PSCRIPT\_PTEX    TeX Postscript (portrait)
\item  PSCRIPT\_LTEX    TeX Postscript (landscape)
\end{itemize}

The PostScript file produced (GKS\_72.PS) can then either be printed with the
/FORM=POST qualifier in the case of the POSTSCRIPT\_P and POSTSCRIPT\_L
options

\begin{verbatim}

   $ PRINT/QUE=SYS_PS/FORM=POST GKS_72.PS

\end{verbatim}

or included in a \TeX\/ document, in the case of the PSCRIPT\_PTEX and
PSCRIPT\_LTEX options.


\section{Printing \TeX\/ and \LaTeX\/ documents}

\TeX\/ and \LaTeX \/ output (.DVI files) are processed ready for printing on a
Canon LBP II/III laser printer running in native Canon mode  using the command
DVICAN. This produces files of the form .DVI-CAN. These files cannot be printed
on a PostScript printer; the .DVI files must instead be processed using the
DVIPSA4 command which produces .DVI-PS PostScript files. This type of file can
then be sent to the printer with the command:

\begin{verbatim}

   $ PRINT/QUE=SYS_PS/FORM=POST REPORT.DVI-PS

\end{verbatim}

For a list of possible qualifiers to the DVIPSA4 command, please refer to
SUN/34.


\section{Printing other types of file}

The print symbiont has forms that allows several other popular format files to be
printed on a PostScript laser printer. Note that not all of these forms have
been tested by Starlink.

\begin{tabbing}
XXXXXX\=XXXXXXXXXXXXXXXXXXXXX\= \kill
\>{\bf\underline{Form}}            \>{\bf\underline{Parameters}}\\
\\
\>PAINT     \>  Print a MacPaint image.\\
            \>\>{\it ncopies:}   Number of copies. (default: 1)\\
            \>\>{\it scale:}     Scale factor. (default: 4, recommended: 1-4)\\
\\
\>BITIMAGE \>  Print an uncompressed image.\\
       \>\>{\it ncopies:}  Number of copies. (default: 1)\\
       \>\>{\it scale:  }  Scale factor. (default: 1)\\
       \>\>{\it xsize:  }  Number of pixels in a row. (default: 512)\\
       \>\>{\it ysize:  }  Number of pixel columns. (default: 512)\\
       \>\>{\it pxlwid: }  Number of bits in a pixel, default 1, values: 1,2,4,8\\
\\
\>ZETA \>  Print a zeta plot file.\\
       \>\>{\it ncopies:}  Number of copies. (default: 1)\\
       \>\>{\it scale:  }  Scale factor. (default: 1)\\
       \>\>{\it zFont:  }  false - selects Helvetica font for internal character\\
       \>\>         true -  (default) uses Zeta characters\\
       \>\>{\it penscale:} Scaling factor for pen point (default: 1)\\
       \>\>{\it usepens:}  false - (default) ignore alternate pen select\\
\end{tabbing}

\end{document}
