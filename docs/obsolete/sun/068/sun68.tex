\documentstyle{article}
\pagestyle{myheadings}

%------------------------------------------------------------------------------
\newcommand{\stardoccategory}  {Starlink User Note}
\newcommand{\stardocinitials}  {SUN}
\newcommand{\stardocnumber}    {68.3}
\newcommand{\stardocauthors}   {D L Terrett}
\newcommand{\stardocdate}      {10 May 1989}
\newcommand{\stardoctitle}     {TPU --- Tape Processing Utility}
%------------------------------------------------------------------------------

\newcommand{\stardocname}{\stardocinitials /\stardocnumber}
\renewcommand{\_}{{\tt\char'137}}     % re-centres the underscore
\markright{\stardocname}
\setlength{\textwidth}{160mm}
\setlength{\textheight}{240mm}
\setlength{\topmargin}{-5mm}
\setlength{\oddsidemargin}{0mm}
\setlength{\evensidemargin}{0mm}
\setlength{\parindent}{0mm}
\setlength{\parskip}{\medskipamount}
\setlength{\unitlength}{1mm}

%------------------------------------------------------------------------------
% Add any \newcommand or \newenvironment commands here
%------------------------------------------------------------------------------

\begin{document}
\thispagestyle{empty}
SCIENCE \& ENGINEERING RESEARCH COUNCIL \hfill \stardocname\\
RUTHERFORD APPLETON LABORATORY\\
{\large\bf Starlink Project\\}
{\large\bf \stardoccategory\ \stardocnumber}
\begin{flushright}
\stardocauthors\\
\stardocdate
\end{flushright}
\vspace{-4mm}
\rule{\textwidth}{0.5mm}
\vspace{5mm}
\begin{center}
{\Large\bf \stardoctitle}
\end{center}
\vspace{5mm}

%------------------------------------------------------------------------------
%  Add this part if you want a table of contents
%  \setlength{\parskip}{0mm}
%  \tableofcontents
%  \setlength{\parskip}{\medskipamount}
%  \markright{\stardocname}
%------------------------------------------------------------------------------

\section{Introduction}

TPU is a utility, originally developed for AOIPS (Atmospheric and Oceanographic
Information Processing System), for the manipulation, examination and copying of
magnetic tapes of any format.
TPU sees a magnetic tape as only a series of files or records of unknown number
and size.
When TPU was designed, the main objective was to give the user maximum
flexibility and not lock the user into some mode where it is supposed that there
is some `normal' or `standard' procedure to be followed when processing the
tape.
The need for something like TPU surfaced when it was quickly apparent that
formatted tapes, such as FLX or PIP tapes, with some physical or logical defect
were often impossible to duplicate because the appropriate utility would abort
as soon as such an error was encountered without giving the user any say in the
matter.
Hence, the objective of TPU to allow the user maximum control over operations,
especially in the case of tape errors.

TPU duplicates all the functionality of MTCOPY and MTDUMP (SUN/22) as well as
performing many additional operations.
It assumes a magnetic tape deck has only one density and so it cannot detect
tape densities.

\section{INITIALIZATION}

TPU is run with the command:

\begin{verbatim}
      $ RUN TPU_DIR:TPU
\end{verbatim}

It then prompts for the name of the INPUT tape.
The reply can be either the physical name of a tape deck (eg MTA0) or a generic
name (eg MT) in which case TPU will select a free deck and ask you to load the
tape.
The tape need not be mounted before running TPU but it doesn't matter if it has
(provided that it has been mounted /FOREIGN).
It will then prompt for the OUTPUT tape (if the system has more than one deck)
and then the density at which it is to be written.
A carriage return instead of a tape name means that there will be no output
tape.
From this point on the tapes will be referred to as INPUT and OUTPUT and never
by their physical names.
TPU then prompts for the first command, responding with a `?' will cause all the
TPU commands to be listed.
Most commands then prompt for additional information.

For all operations in which TPU physically reads or writes a record, the
maximum record size is 64,000 bytes and the minimum record size is 17 bytes.

If the user responds with a `CTRL Z' sequence to a prompt, TPU will either exit
or return to the main command dispatch mode.
When TPU exits, the tapes are left mounted.

\section{TPU COMMANDS}

There are four functional command areas in TPU.
They are POSITIONING, EXAMINATION, DUPLICATION and MISCELLANEOUS.


The POSITIONING commands are:

\begin{description}
\begin{description}
\item [SKIP~~~]	Skip records or files.
\item [FLET~~~]	Find logical end-of-tape.
\item [REWIND~]	Rewind a tape.
\end{description}
\end{description}

The EXAMINATION commands are:

\begin{description}
\begin{description}
\item[FDEN~~~~]	Find the density of a tape.
\item[FDIR~~~~]	Produce a directory of the tape structure.
\item[DMP~~~~~]	Dump tape records or files.
\item[COMPARE~] Compare the contents of two tapes
\end{description}
\end{description}

The DUPLICATION commands are:

\begin{description}
\begin{description}
\item[COPY~~~~]	Copy tape-to-tape until logical end-of-tape.
\item[C2T2~~~~]	Copy specified records/files tape-to-tape.
\item[CTD~~~~~]	Copy designated files tape-to-disk.
\item[IBM~~~~~]	Reads or creates an IBM EBCDIC tape. Some understanding of IBM
tape formats helpful.
\end{description}
\end{description}


The MISCELLANEOUS commands are:

\begin{description}
\begin{description}
\item[WEOF~~~~]	Write an end-of-file record.
\item[WLEOT~~~]	Write two successive end-of-file records.
\item[DEN~~~~~]	Set the density of a tape.
\item[RED~~~~~]	Redirect the INPUT and OUTPUT tapes.
\item[GEN~~~~~]	Generate a data test tape.
\item[ERASE~~~]	Erase tape
\item[EX~~~~~~]	Exit TPU
\item[HELP~~~~]	Display valid TPU functions
\end{description}
\end{description}

A detailed discussion of each command follows.

\subsection{POSITIONING commands}

There are three positioning commands.

The SKIP command will skip forward or backward over a specified number of files
or records.
If no skip count is given, a count of one is assumed.
Whether files are records are to be skipped will be prompted for.
The SKIP command will not go past the DEC standard logical end-of-tape (two
successive end-of-file records).
In keeping with the design objective of allowing the user full control, TPU will
not inhibit any command that causes the tape to continue forward movement even
though the last command may have caused logical end-of-tape to be found.

The FLET command will search the tape in a forward direction looking for the DEC
standard logical end-of-tape (two successive end-of-file records).

The REWIND command will rewind the designated tape.
If two tapes are in use (INPUT and OUTPUT), both may be rewound with a single
command.

\subsection{EXAMINATION commands}

There are three examination commands.

The FDEN command will, by doing tape reads, try to determine the density of a
tape.
The densities attempted, in turn, are 800, 1600 and 6250 BPI.
If the tape successfully responds to one of these densities, the density of the
tape is so set.
This whole operation is somewhat redundant as `automatic' determination of tape
density is a normal feature of VAX/VMS, but is useful for verifying that a tape
has been written at the desired density.
It is important to realise that many of the newer tape drives at
Starlink sites set the operating tape density in hardware on the
front panel of the tape drive. If there is a difference between the
operating density as indicated by the tape drive and either
TPU or VMS itself, then the tape drive will be correct.

The FDIR command will provide what TPU calls a `formatted directory' of a tape.
This command will read the tape from its current position to the logical or
physical end-of-tape.
During this read, a one line report reflecting the composition of each file is
listed.
This report tells how many records there were in the file, and the minimum and
maximum record size (in bytes).
When the end-of-tape is reached, a short summary report on the number of files,
records, sizes, etc, is produced.
Optionally, TPU may be directed to perform the tape reads in the DATACHECK mode.
In READ with DATACHECK, each record is read from tape into memory, the tape is
back-spaced and the record is read again and the data read is compared to the
data in memory.
If no differences are found, the next read is performed.
This is an excellent method of verifying tapes.

The DMP command will read and dump to the terminal or line printer the specified
number of tape records.
The format of the dump is is selectable as OCTAL, ASCII or HEXADECIMAL byte.

\subsection{DUPLICATION commands}

There are four duplication commands.

The COPY command will copy the contents of the INPUT tape to the OUTPUT tape.
The copy is performed from the current position of each tape until the
end-of-tape is detected.
Both the INPUT and OUTPUT tape can be copied in the DATACHECK mode as described
under the FDIR command.
The copy may be directed to either stop or ignore tape parity errors.
If directed not to stop on parity error, the position of the error (file and
record number) will be reported.

The C2T2 command copies a specified number of records or files from the INPUT
drive to the OUTPUT drive.
NB: After copying records/files, use WLEOT to indicate position of the logical
end-of-tape.

The COMPARE command does a record by record comparison of two tapes.
The comparisons continue until a difference is found.
The assumption here is that the comparison is done to validate copies etc.
Once a difference is found, you are prompted for next action.
The comparisons can proceed as before or you can skip to next file and continue
from there or you can abort (exit) the operation.
COMPARE does not rewind tapes, the comparison starts wherever tapes are
positioned.
The file and record numbers are relative to where the comparison begins.

The CTD command is a copy of the INPUT tape to a specified disk file.
It is assumed that the tape files to be copied are in some `normal' format as
generated by RMS or some similar file control system.
Various forms of binary data files, such as object or executable images, cannot
be properly processed.
In other words, DEC type text files are expected.
There are four modes in which the copy may be done.
They are:

\begin{itemize}
\item
Mode 1 - The specified number of records/files are copied to a named file in the
current user directory.
If there are existing copies of the named file already in the directory, the
next higher version number is used.
Disk files are never deleted.
If a record/file count is not specified, the copy goes from the current tape
position to the next end-of-file record.
A Mode 1 copy will span tape files.
If, during the copy, an end-of-file record is encountered, it is skipped and the
copy continues with a message to the terminal so advising.

\item
Mode 2 - The specified number of records/files are copied to ascending versions
of the named disk file on a one-to-one basis.
In this mode, each record or file (depending on whether the copy is of records
or files) is copied to a single disk file.
If, for example, there are 10 files on the tape and the named file is
`FOOBAR.DAT', then all of the records in the first tape file go into disk file
`FOOBAR.DAT;1' while all the records in the second tape file go into disk file
`FOOBAR.DAT;2', and so on.
Mode 2 is specified by delimiting the file name prompt with an asterisk ``*'',
eg, FOOBAR.DAT*.
This mode allows records/files to be copied from tape to individual disk files
without the user specifying each disk file name.
Sort of a `wild-card' tape-to-disk copy.

\item
Mode 3 - Same as Mode 1 except no file name is specified and TPU uses the
default of FOR001.DAT.

\item
Mode 4 - Same as Mode 2 except no file name is specified and TPU uses the
default of FOR001.DAT.
Mode 4 copy is specified by responding to the file name prompt with only an
asterisk ``*'' character.
\end{itemize}

The IBM command will either read a file from an IBM format tape and put it into
a named disk file or `dump' a named disk file onto tape in IBM format.
Both IBM Standard Label and unlabeled tapes are handled.

\subsection{MISCELLANEOUS commands}

There are six miscellaneous commands.

The WEOF command writes an end-of-file record on the specified tape.

The WLEOT command writes two successive end-of-file records to the specified
tape which creates a logical end-of-tape.

The ERASE command will erase a tape to the physical End-of-Tape (to the EOT
marker) or else a specified number of inches (approximately).
You can run into problems if you erase large (30+ feet) portions of tape which
precede portions of the tape that are to be subsequently read.
This might be the case if you start at BOT and erase and then follow with
writing to the tape.
Another case would be to write to the tape, erase some tape, and then write
more.
The problem arises because most tape units have a timeout function which is used
mainly to detect runaway tape.
If a tape read (or skip) is given and the tape is positioned in front of a large
section of erased tape, an error will occur if data is not found before the
timeout expires.
In normal tape processing this is most likely a fatal error, but TPU will ask if
you want to continue and if you know there is data further down the tape the
operation will eventually succeed.

The DEN command sets the specified tape to the specified density.
The possible densities are 800, 1600 and 6250 BPI.
It is important to realise that many of the newer tape drives at
Starlink sites set the operating tape density in hardware on the
front panel of the tape drive. If there is a difference between the
operating density as indicated by the tape drive and either
TPU or VMS itself, then the tape drive will be correct.
Note that if the first operation performed on the tape is a read, it will
switch back to density at which it was last written.

The RED command `swaps' the designation of the INPUT and OUTPUT tapes.
The INPUT becomes the OUTPUT and the OUTPUT becomes the INPUT.

The GEN command generates a `test' tape.
The number of records/files and the data pattern of each tape character (byte)
is specified by the user.
Mainly used as a convenient method of making test tapes to check out TPU.

The EX command causes a TPU exit.
If TPU is in the main command dispatch mode, an exit to the system is made.
If TPU is in a specific command execution mode (eg, COPY), an exit back to the
main command dispatch mode is made.

\pagebreak	% deliberate page break here to get example on one page

\section{EXAMPLE}
\begin{verbatim}
$ RUN TPU_DIR:TPU

       **** TPU - VAX/VMS Tape Processing Utility V3.2 *****

 Enter Input tape drive> MT
%MOUNT, Please mount device  MTA0:
%MOUNT,  mounted on MTA0:
%MOUNT, operator request canceled - mount completed successfully
 The current tape density is 1600
 Enter OUTPUT tape drive>
 ** Enter TPU function > FDIR

** WARNING - Next few questions refer only to INPUT tape MT:

 Stop at [L]ogical or [P]ysical end-of-tape (Default is "L") > L
 Stop on INPUT tape parity error (Default is "N") [Y/N]? > N
 WRITE output to [L]ineprinter of [T]erminal (Default is "T")? > T
 Read with datacheck [Y/N]? > N
 File    1 has     23 recs, <minrec size=  80> <maxrec size =  80>
 File    1 has    143 recs, <minrec size=  80> <maxrec size =  80>
 File    1 has    343 recs, <minrec size=  80> <maxrec size =  80>
 File    1 has    209 recs, <minrec size=  80> <maxrec size =  80>
 File    1 has    354 recs, <minrec size=  80> <maxrec size =  80>
 Logical end of tape detected.
 End of formatted directory operation...

 Total of 1072 records in 5 file(s).
 Smallest record length is 80 bytes.
 Largest record length is 80 bytes.

 ** Enter TPU function > REWIND
 ** Enter TPU function > SKIP

** WARNING - Next few questions refer only to INPUT tape MT:

 Enter number of records or files to skip > 3
 Enter type of skip ("?" will list options) > ?
 Enter FF for forward skip of files.
 Enter BF for backward skip of files.
 Enter FR for forward skip of records.
 Enter BR for backward skip of records.
 Enter type of skip ("?" will list options) > FF
 A total of 3 records/files were skipped
 ** Enter TPU function > CTD
 Copy tape [R]ecords or tape [F]iles? (Default is "Files") > F
 Enter number of tape FILES to copy  (Default = copy to next EOF) >
 Enter disk filename (Default is FOR001.DAT) > TEST.FOR
1 tape file transferred to disk file "TEST.FOR.

 Do more tape to disk copying? > N
 ** Enter TPU function > EXIT
 TPU exiting
 $

\end{verbatim}

\end{document}
