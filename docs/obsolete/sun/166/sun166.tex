\documentstyle[11pt]{article}
\pagestyle{myheadings}

%------------------------------------------------------------------------------
\newcommand{\stardoccategory}  {Starlink User Note}
\newcommand{\stardocinitials}  {SUN}
\newcommand{\stardocnumber}    {166.2}
\newcommand{\stardocauthors}   {Rhys Morris}
\newcommand{\stardocdate}      {19 January 1994}
\newcommand{\stardoctitle}     {SAOIMAGE --- Astronomical image display}
%------------------------------------------------------------------------------

\newcommand{\stardocname}{\stardocinitials /\stardocnumber}
\renewcommand{\_}{{\tt\char'137}}     % re-centres the underscore
\markright{\stardocname}
\setlength{\textwidth}{160mm}
\setlength{\textheight}{230mm}
\setlength{\topmargin}{-2mm}
\setlength{\oddsidemargin}{0mm}
\setlength{\evensidemargin}{0mm}
\setlength{\parindent}{0mm}
\setlength{\parskip}{\medskipamount}
\setlength{\unitlength}{1mm}

%------------------------------------------------------------------------------
% Add any \newcommand or \newenvironment commands here
%------------------------------------------------------------------------------

\begin{document}
\thispagestyle{empty}
SCIENCE \& ENGINEERING RESEARCH COUNCIL \hfill \stardocname\\
RUTHERFORD APPLETON LABORATORY\\
{\large\bf Starlink Project\\}
{\large\bf \stardoccategory\ \stardocnumber}
\begin{flushright}
\stardocauthors\\
\stardocdate
\end{flushright}
\vspace{-4mm}
\rule{\textwidth}{0.5mm}
\vspace{5mm}
\begin{center}
{\Large\bf \stardoctitle}
\end{center}
\vspace{5mm}

\section{Introduction} 

SAOimage is an astronomical image display program which works on
computers with X-window displays. It allows you to manipulate images
in a number of ways, see the changes applied, and when you are  happy
with the result, produce a hard copy on a Postscript
printer. An example of the output is included with this document.

This document is a quick introduction to SAOimage. More
information can be found in the manual page using `{\tt sman saoimage}'
or in the SAOimage manual (MUD/140).

Send bug reports to the software librarian (ussc@star.rl.ac.uk).
The Starlink version runs under both VMS and
Unix so you should indicate which machine you are using if a problem
occurs.  In this document, the program will be referred to as
`SAOimage', although you would always type `{\tt saoimage}' at your
computer.

\section{History}

The original author of the program was Michael VanHilst of the
Smithsonian Astrophysical Observatory (hence the name
SAOimage). Later, other people contributed work to the project, and it
is now a sizable piece of C code, well over 30,000 lines in
all. SAOimage can deal with a number of file formats and can now read
Starlink NDF image files. The routines required to do this were added
by Rhys Morris for Starlink.

\section{Documentation} 

There is a comprehensive manual describing SAOimage (MUD/140),
which is also distributed in \LaTeX\ form with the SAOimage source
code in the `doc' subdirectory. A Unix `{\tt man}' page is also provided
which can be examined using {\tt sman saoimage}.

\section{Supported File Formats}

\begin{itemize} 
\item{Starlink NDF images.}
\item{IRAF .imh images.} 
\item{Disk FITS files.} 
\item{Disk files containing 1/2/4 byte signed/unsigned integers or 4/8~byte 
floating point numbers.} 
\end {itemize}

\subsection{Notes on File Formats} 

Both primitive and simple NDFs are supported. The `{\tt -ndf}' flag may
be used to denote that the image is a Starlink NDF, but SAOimage will
assume this if the file has a {\tt .sdf} suffix (file extension). 
STSDAS (Space Telescope Science Data Analysis System) {\tt .hhh} images
are not supported, but these can be converted into {\tt .imh} files
within IRAF. As with all disk FITS reading programs --- SAOimage is not
foolproof, but it works well with non-exotic disk FITS files. When reading a
file of unformatted data, you must specify the dimensions of the image
on the command line, it is possible to skip over header bytes at the
start of the image using the `{\tt -skip}' option. Data from a different
machine which stores its numbers in a different format can be
displayed without conversion by instructing SAOimage to swap bytes
using the `{\tt -bswap}' option.

\section{Using SAOimage} 

Using SAOimage is fairly intuitive. The program is started up by
typing `{\tt saoimage}' in a window followed by any appropriate flags
and an image file name. You may need to set the DISPLAY environment
variable if you are running SAOimage on a remote machine. You can use
the Starlink utility {\tt xdisplay} to do this. Either devote a window
to running SAOimage or run it in the background using an {\tt \&} at
the end of the command line. The latter frees the window, although
messages from SAOimage may still appear in the parent window unless
you use the `{\tt -quiet}' switch on the start-up command line. Here
are some example command lines.

\begin{verbatim}
      saoimage orion.sdf
\end{verbatim}

\begin{verbatim}
      saoimage -i2 650 1125 -bswap -rotate 1 field2.r &
\end{verbatim}

The first command line simply tells SAOimage to start up and display
the NDF {\tt orion.sdf}. The second example reads two---byte signed
integer data from the file {\tt field2.r}, and assembles it into a 650
by 1125 image. Bytes are swapped using the `{\tt -bswap}' switch, and
the image is rotated by 90 degrees using the `{\tt -rotate 1}'
options, (`{\tt -rotate 2}' would rotate by 180 degrees). Putting an
`{\tt \&}' at the end of the command line runs the program in the
background. Have a look at the manual page or the SAOimage manual for
a complete list of command line options.

Once your image is displayed, the mouse is used to `press' SAOimage
buttons. Moving the mouse around with a mouse button depressed, called
dragging the mouse, in the biggest window --- the data window ---
allows you to manipulate your image interactively. Try starting up
SAOimage with an image now. Use the mouse to press the `{\tt Color}'
button, then the `{\tt cmap}' button, then the `{\tt B}' button. Now
drag the mouse around the biggest window with a button depressed. This
shows you how SAOimage can be used to modify a false colour
display. Click the mouse once on the colour bar at the bottom of
SAOimage, and you will see a representation of the colour table
currently in use, you can now see how dragging the mouse vertically
and horizontally changes the colour table.

Should you wish to load in another image or change your original command
line, you press the `{\tt etc}' button until you see a button marked
`{\tt new}' as the bottom leftmost button. Pressing this button allows
you to type in a new command line. If you cannot find the `{\tt QUIT}'
button, then a swift control-c in the controlling window will stop
SAOimage --- or if running in the background, you will have to find the
process identifier number for SAOimage using a command such as `{\tt ps
-aux $|$ grep saoimage}' and then use the `{\tt kill}' command to stop
the SAOimage process. But this should rarely be necessary.

If your version of SAOimage refuses to display NDFs, then it could be
because you are using another, non---Starlink, version of SAOimage which
is present on your system. The Starlink version has a start-up message
which mentions Starlink. If there is no such message, you may have to
rearrange your PATH environment variable to put {\tt /star/bin}
earlier --- or you could define an alias to SAOimage.

This version of SAOimage is based on version 1.07 of SAOimage dating from
October 1992. An example of the hardcopy produced, which is annotated with
some text, is included in the file {\tt /star/docs/sun166.ps} which should
accompany this document.

\section{Image Manipulation}

Once you have your image in SAOimage, there are a large number of ways
in  which you can enhance (or degrade!) your images --- these are listed
below. You can:

\begin{itemize}

\item{Scale between limits using `-min $\langle$value$\rangle$' and 
`-max $\langle$value$\rangle$' switches.}

\item{Magnify, zoom and pan around the image.}

\item{Perform histogram equalisation.}

\item{Display using log and square root scaling.}

\item{Use false colour with in-built or user specified colour tables.
Clicking on the colour bar at the bottom of SAOimage pops up a
graphical representation of the RGB (Red, Green, Blue) values making
up the current colour table. This can be interactively adjusted with
the mouse or a previously saved colour table can be loaded from a
file.}

\item{Stretch the contrast.}

\item{Change Gamma ($\gamma$) parameter to give non-linear contrast, so that
the display assigns more shades of grey to the darker or lighter end of
the scale. This may suit the eye better that a linear scale.}

\item{Type text on to an image.}

\end {itemize}

\section{Other Features}

\begin{itemize}

\item{Blinking --- several images can be stored in SAOimage and you can blink
between them. To store an image, press the {\tt Scale} button,
pressing the {\tt blink} button with the left mouse button associates
the current image with the left mouse button. Load another image
using the {\tt new} button, scale it as required, and press blink with
another mouse button. An image can be associated with each mouse
button and displayed by pressing the appropriate button. But you cannot
change the scaling while blinking the images have to be re-loaded in
order to do that.}

\item{Regions --- a description of a region of the image can be saved to an
ASCII file. This can be used for flagging regions as bad, or examining only
the counts within a certain region. This feature is mainly to be used in
conjunction with the IRAF and PROS (X-ray) packages.}

\item{Tracking --- The pixel coordinates and value under the mouse pointer and
the region around it can be displayed and constantly updated as your mouse
wanders around the image. Use  the `{\tt track}' and `{\tt coord}' buttons to
toggle these on and off.}

\item{IRAF image display --- SAOimage can also be used as a display for IRAF
--- you need to use the `{\tt -imtool}' flag to put it into the right mode}

\item{Hard copy output can be saved to a file by setting the environment 
variable R\_DISPOSE to ``mv \%s piccy.ps''. This stores SAOimage's 
output in the file ``piccy.ps''.}

\end{itemize}

\section{Problems and Limitations}

\begin{itemize}

\item{Colour tables --- Starting SAOimage may affect the colours already
present on your screen as it tries to reserve some of the available
colours for its own use. So beware, especially those of you with gaudy
background images on your workstations.}

\item{Full filenames --- The full filename must be used, so ADAM users must
type the {\tt .sdf} at the end of  a filename. This is no inconvenience if you
use the filename completion facilities offered with many shells. In the t-shell
for example, if you type enough letters to uniquely specify the file, then
pressing the `tab' key will complete the filename for you.}

\item{Numbers as filenames --- If your filename is a number, then you have to
use the `{\tt -name}' switch to  inform SAOimage that it is a filename not a 
parameter.}

\item{Start-up errors --- If you do not have IRAF installed, then the error
messages ``{\tt ERROR: unable to open /dev/imt1o}'' and ``{\tt Error:
No remote input possible}'' will appear on starting SAOimage --- this
is nothing to worry about.}

\item{Portability --- This version of SAOimage requires some Starlink 
libraries, so there may be problems on a non---Starlink machine. It may
be necessary to recompile without the {\tt -DNDF} preprocessor directive
flag to remove all mention of Starlink.}

\item{Filling the image --- SAOimage does not automatically scale your image
to fill the window, so you will have to zoom in to examine small images.}

\item{False colour --- only false colour is supported, real colour is not
possible in SAOimage.}

\item{Black and White output --- if you want a hardcopy of your carefully
coloured image you will have to use another X-window utility to
produce colour hardcopies.}  

\item{Hardcopies on VMS --- a Postscript file is produced with a name
of the form {\tt \_PS$\langle$number$\rangle$.TMP} which can be printed on a
Postscript printer.}

\item{Disk FITS on a DECstation --- A disk FITS file created on a SUN
containing an image of reals or doubles cannot be read on a DECstation
as the floating point storage form is different. Other number formats
such as two-byte unsigned integers are unaffected.}

\item{Missing fonts --- When starting up, especially on VMS, SAOimage may
complain about missing fonts. If this means that you cannot read the
pixel values, then get your system manager to install the required
fonts.}

\item{Scaling --- If a FITS file is loaded in first, then SAOimage
remembers the zero point offset and scale factors from the FITS
keywords {\tt BZERO} and {\tt BSCALE} and erroneously uses them
for all subsequent arrays of whatever format. Quit SAOimage and reload
the image if this problem occurs.}

\end{itemize} 

\section{Internal Workings}

This section is probably of no interest to the SAOimage user, it is intended
for people who may have to modify and maintain the package.

A file format is recognized by SAOimage as one of the supported types
either by the file extension or by a command line flag. For example, a
{\tt .imh} suffix would suggest an IRAF image and a {\tt -fits} flag
tells SAOimage to expect a disk FITS file. Once the file type is
recognized, two routines specific to each file type are required. One
needs to return the dimensions and storage format of the data, and one
to return the actual data for display. The routines to do this for NDF
images are called {\tt init\_ndf()} and {\tt load\_ndf()} respectively
and are contained in the file {\tt readndf.c}.

These C routines call FORTRAN routines {\tt openndf()} and {\tt mapndf()} 
via the CnF
library. The FORTRAN routines use the HDS library to get at the data in the
NDFs.  NDF routines could not be used at the time of writing as they would
require going through the ADAM parameter system --- this would make it
difficult to use SAOimage on other file formats. The FORTRAN routines
mentioned are in files with the same names as the subroutines. At present,
{\tt mapndf()} always maps the NDF to an array of 4 byte REALs regardless 
of the
the original NDF data type.  This is also true on 64bit machines as this is
a FORTRAN array.

SAOimage carries information about the image being displayed in a global
structure called img, the template for which is defined in the file
hfiles/image.h as imageRec. The routine {\tt init\_ndf()} has the job of
inserting various values into this structure so that this information can
be propagated around the program. The file hfiles/constant.h contains an 
entry for
each kind of image format supported, the following line was added.
\begin{quote}
{\tt \#define SOP\_NDF         0x0100          /* Starlink data format */}
\end{quote}
There are similar entries for FITS, IRAF and the array formats. This
entry and the entries regarding NDFs in imgread.c are not in the original
source code, so these will have to be added should a superior version of
SAOimage come along with new source code. Also, the code in {\tt openndf()} 
and
{\tt mapndf()} could be greatly simplified using the {\tt ndf\_open()} 
routine instead of
the numerous HDS calls.

The routine {\tt parse\_filetype()} in {\tt cmdimage.c} and 
{\tt check\_image()} in
{\tt imgcheck.c} were altered so that the {\tt -ndf} flag and the {\tt .sdf} 
file
suffix were added to the list of recognized file types. C
pre---processor ``\#ifdef NDF'' directives surround all the optional file
format code, so that support for FITS, for example, could be removed
by compiling without the -DFITS flat which defines FITS for the
pre---processor. Support for NDF format was added in the same way as
for FITS and OIF (Old IRAF Format) file types.

The source code is the same on the VMS version as in the Unix
versions. On VMS the executable is built by typing {\tt @make} while a
Unix build is performed using the usual {\tt mk build} system.

\end{document}
