\documentstyle{article} 
\pagestyle{myheadings}

%------------------------------------------------------------------------------
\newcommand{\stardoccategory}  {Starlink User Note}
\newcommand{\stardocinitials}  {SUN}
\newcommand{\stardocnumber}    {43.1}
\newcommand{\stardocauthors}   {K Shortridge}
\newcommand{\stardocdate}      {20 January 1989}
\newcommand{\stardoctitle}     {EDFITS --- Copy FITS Tapes}
%------------------------------------------------------------------------------

\newcommand{\stardocname}{\stardocinitials /\stardocnumber}
\renewcommand{\_}{{\tt\char'137}}
\markright{\stardocname}
\setlength{\textwidth}{160mm}
\setlength{\textheight}{240mm}
\setlength{\topmargin}{-5mm}
\setlength{\oddsidemargin}{0mm}
\setlength{\evensidemargin}{0mm}
\setlength{\parindent}{0mm}
\setlength{\parskip}{\medskipamount}
\setlength{\unitlength}{1mm}

\begin{document}
\thispagestyle{empty}
SCIENCE \& ENGINEERING RESEARCH COUNCIL \hfill \stardocname\\
RUTHERFORD APPLETON LABORATORY\\
{\large\bf Starlink Project\\}
{\large\bf \stardoccategory\ \stardocnumber}
\begin{flushright}
\stardocauthors\\
\stardocdate
\end{flushright}
\vspace{-4mm}
\rule{\textwidth}{0.5mm}
\vspace{5mm}
\begin{center}
{\Large\bf \stardoctitle}
\end{center}
\vspace{5mm}

\section{Introduction}

EDFITS is a program used at AAO to copy and edit FITS tapes.  Since it is
designed to be used with original data tapes, it has a slightly unusual
style which places a heavy emphasis on making as sure as possible that you
are not doing anything rash---such as mounting an input tape with a write 
ring fitted---and that you understand the consequences of the commands
given.  So, for example, after most commands the effect that the command will
have is described and you are asked to confirm the command before it is 
executed.  Apart from its ability to produce usefully formatted header listings
for a variety of flavours of FITS tapes, EDFITS provides little that cannot
be done with standard tape manipulation programs such as MTCOPY.  However,
you may find its particular style to your taste, and may come to appreciate
the way it makes it that little bit more difficult to mess up a FITS tape
copy.

\section{Running the program}

EDFITS may be run from any terminal, but it has a user interface that makes
use of the VAX SMG\$ screen management routines, and so tends to look odd
and to feel uncomfortable if used on a terminal that these routines do not
support.   This does not necessarily restrict it to DEC terminals---non-Digital
terminals may be made to run SMG\$ programs; your system manager will know
which terminals are supported and which are not.

EDFITS works with one input tape and one output tape.  Its main function is
to allow you to copy selected files from the input tape onto the output tape.
Its other capabilities---allowing you to move around on either tape, for
example, or make a directory of the tape contents---are really just to help
with this main function. The tapes do not need to be MOUNTED before the 
program is run (you can set the output tape density from within EDFITS, so
MOUNT/DEN= is not needed), but it is probably a good idea to ALLocate the 
drives first,
e.g.
\begin{verbatim}
    $ ALL MTA0:
    $ ALL MTA1:
\end{verbatim}

To run EDFITS, just give the command
\begin{verbatim}
    $ EDFITS
\end{verbatim}

If this fails, EDFITS has probably not been installed on your machine.  Speak
severely to your system manager.

\section{Initial setup}

EDFITS leads you by the hand through the initial stages of determining
which tape drives are to be used for the input and output tapes.  

Firstly, an introductory message is displayed and you are invited to press
RETURN to indicate that you have read it.  The introductory message makes the
important point that you can invariably get more help at any time by entering
`?' in response to a prompt.  If you feel so inclined, try entering `?'
at this point instead of just pressing RETURN; you will get an idea of the
(irritating?) way EDFITS works.

Then, you are prompted for the name of the input tape drive.  This can be 
the actual name, for example \_MTA0: (the `:' is optional, as is the `\_'),
or a logical name.  EDFITS will complain if the tape has a write ring.  If
the tape is not at the load point (perhaps having been left there from a
previous run of EDFITS or some other program), it will point this out and 
give you the option of rewinding it.  It helps EDFITS a little if you start 
with the tapes at the load point, since it can then keep track of the actual
tape positions, but this is not necessary.

Next, you are prompted for the name of the output tape drive.  Again, this
can be the actual name or a logical name.  EDFITS will accept `NONE' at this
point, indicating that you do not in fact have an output tape---this might be
the case if want only to examine the input tape rather than copy it.  EDFITS
will complain if the output tape does not have a write ring.  As for the
input tape, EDFITS will offer to rewind a tape that is not at the load point.

EDFITS will then give you the option of initialising the output tape.  Note
that EDFITS is fussy about its output tapes and will NEVER (really, 
NEVER\footnote{Well, HARDLY ever.  The only exceptions are when a tape is 
initialised or terminated by the TERM command})
write to an output tape without first locating the two successive file marks
that it assumes indicate the end of the data already on the tape.  This means
that it will not be able to write to a blank tape unless it is initialised
first (initialising involves rewinding the tape and writing two file marks).
So if you have a blank tape, or one that you want to re-use completely, you
must initialise it at this stage.  Protective as ever of its output tapes,
EDFITS will make you confirm that you want the tape initialised, and will
then ask you the tape density to be used.  If you declined to rewind a tape
that was not at the load point, EDFITS assumed you must have data on it and
bypasses the initialisation option.

\section{Selecting the tape type}

The next question is a little unusual.  Although the FITS standard defines
how a FITS tape is to be written, and the rules for how to specify keywords
in the FITS headers, it leaves the choice of all bar a few standard keywords
up to the writer of the tape.  EDFITS---a mere copying program---is not
concerned with the meaning of the various keywords, but there is one case
where they are of some importance.  When EDFITS does what it calls a
`Partial Directory' of a tape, where it generates a one line summary of 
each FITS image on the tape, it needs to know which of the FITS keywords to
include in the listing and how to format their values.

A control file should be available that describes to EDFITS the details of
a number of various flavours of FITS tapes.  If it can find this, EDFITS
will then list the types it knows about and will ask you to select one of
these.  If you neither know nor care about the optimum choice, choose one of
the more general sounding options and remember that it only affects the
relatively unimportant `PD' command, not any of the tape copying.

\section{Main command node}

EDFITS now moves to its main command `node', putting up a menu of commands,
prompting `Enter command' and waiting for you to select one of the commands
offered.  The menu shows the minimum acceptable abbreviation for each command
(this part of the command name is in capitals), and a brief summary of the
syntax.

Many commands have no parameters at all.  Those that could sensibly
apply to either tape have a parameter shown in the menu as `I/O'.  You actually
specify this parameter as either `INPUT' or `OUTPUT'---or an abbreviation
of these, usually just `I' or `O'---to indicate which tape you intend to 
be used.  If you only have one tape this parameter becomes optional.  Some
commands have a numeric parameter: `SKip', for example, needs to know
not only what tape to skip on, but how many files to skip, and is shown
as:
\begin{verbatim}
    SKip I/O #
\end{verbatim}
where `\#' indicates that a number will be required.  You have to give
the parameters for any command at the same time as you give the command
itself: if you just say `SKIP', EDFITS will complain that you have missed
parameters, but will not prompt you for them in the way VMS would.  Instead
you end up back at the `Enter command' prompt and have to give the command
in full, this time including the parameters.

The commands will now be listed individually:

\begin{description}

\item [COPY] --- Copy images.

Copy specified number of file from input tape to output

Syntax:   `COpy  \#'     

where \# is some positive number.

COPY will copy the specified number of files from the input tape to 
the output tape.  The copy will start from the current position of the 
input tape.

These files will be appended to the files already on the output
tape.

\item [CT] --- Copy to an image.

Copy up to specified file from input tape to output

Syntax:  `CT \#'       

where \# is some input tape file number

CT will copy files from the input tape to the output tape, up to 
and including the specified file. The copy will start at the current
input tape position.    

These files will be appended to the files already on the output
tape.  

\item [DUPE] --- Duplicate tapes.

Copy files from current position on input tape

Syntax:    `DUpe'

DUPE will copy all the files, from the current position of the
input tape to the end of data on the input tape, to the output
tape.   Note that the input tape is NOT rewound first.

These files will be appended to the files already on the output
tape.  

If the input tape is rewound, and the output tape has just been
initialised, DUPE will produce a complete copy of the input tape.

\item [EXIT] --- Leave EDFITS.

Exit from EDFITS.

Syntax:  `Exit'

The EXIT command leaves EDFITS.  The tapes are rewound and logically
dismounted,  although they are left physically loaded on the drives.
(What is done is the equivalent of a \$DISMOUNT/NOUNLOAD DCL command.)
This means that if you restart EDFITS - or any other program - the tapes
will still be available, but not at their current positions.

\item [FD] --- Full directory.

Full directory of input or output tape.

Syntax:  `FD  tape'      

where `tape' is INPUT or OUTPUT

The specified tape is rewound.  EDFITS then runs through all the
files on it, listing the full FITS header for each in a file.  When
the last file is read, the listing file is automatically printed.
The tape will be left at the end of the last file on it.  

Note that FD produces a lot of paper output, and is really rather
wasteful.  In most cases, PD should be used instead, particularly if
EDFITS knows about the particular flavour of FITS tape you are using.
If it does not, you should consider producing your own type description
file, as described later.

\item [FEnd] --- Find end.

Find end of output tape.

Syntax:   `FEnd'             

only applies to output tape.

EDFITS will position the output tape at the end of data marker - ie
at the end of the last file on the tape.  This is done automatically
at the start of any copy command, but can be done separately to find
how many files are on the output tape (if the tape position is known.)

Ye of little faith may use FEND prior to a DUPE or COPY command if it 
makes ye feel happier to see the tape positioned at the end of its 
existing data before you give the command that will write to it.  

\item [HElp] --- Describe command.

Describe an EDFITS command.

Syntax:  `Help command'  

where command is any EDFITS command.

HELP will describe the action of any of the commands that may be
given at the EDFITS main node.  The command name may be abbreviated.
HELP with no command specified will redisplay the main node menu.

\item [INIT] --- Initialise tape.

Rewind and Initialise Output Tape.

Syntax:  `INit'            

can only be used on the output tape.

The output tape will be rewound and an and of data marker (two
successive filemarks) will be written. This will make it impossible
to read any data already on the tape, so be CERTAIN that you really
want to do this. You have to explicitly confirm this command with a
`YES' response to the next prompt.  `NO' will abort the operation.

\item [LIst] - List header.

List header of next file on tape.

Syntax:   `List  tape'    

where `tape' can be INPUT or OUTPUT

EDFITS will read the header of the next file on the specified tape
and list it on the terminal. It will then backspace over that header, 
leaving the tape positioned exactly as it was.

\item [NOw] --- Give positions.

Show current tape positions.

Syntax:   `Now'

EDFITS displays the current positions of the input and output
tapes - assuming they are known.  Note that if EDFITS is started
with tapes that are not at the load point, it cannot know their
positions until they are rewound (you may not wish to do this, of
course).

\item [NPR] --- No parity error retry.

No Parity error Retry.

Syntax:   `NPR'

If EDFITS gets a parity error when reading the input tape, it can
attempt to re-read the record a number of times before accepting
that it has a solid parity error.  The `NPR' command disables the 
re-reading of records with parity errors, and is the best option if 
you know that your tape is full of solid parity errors, since the
continual re-reading takes a long time and may stress the tape
unduly.

\item [PD] --- Partial directory.

Partial directory of input or output tape.

Syntax:  `PD  tape'      

where `tape' is INPUT or OUTPUT

The specified tape is rewound.  EDFITS then runs through all the
files on it, producing a disk file containing a single line for 
each tape file, summarising the contents of its FITS header. When
the last file is read, the listing file is automatically printed.
The tape will be left at the end of the last file on it.

For details on which keywords are used for the one line image summaries,
see the section describing the way the EDFITS type file is used.

\item [PR] --- Re-read on parity error

Parity error Retry

Syntax:  `PR'

If EDFITS gets a parity error when reading the input tape, it can
attempt to re-read the record a number of times before accepting
that it has a solid parity error.  The `PR' command enables the 
re-reading of records with parity errors, and is the best option if 
you think your tape has only a few errors, or if they are marginal,
since a marginal record may in fact be read properly on one of the 
retries.

\item [QUIT] --- Leave EDFITS.

Exit from EDFITS.

Syntax:  `Quit'

The QUIT command leaves EDFITS.  The tapes are left at their present
positions, so you can restart EDFITS as you left it, should you need to.

\item [RW] --- Rewind tape.

Rewind the input or output tape.

Syntax:  `RW  tape'      

where `tape' is INPUT or OUTPUT

The specified tape will be rewound.  If a tape was not at the load
point when EDFITS was started, its position is unknown.  Rewinding 
a tape whose position is unknown is the only way of making sure
EDFITS knows the position of the tape, and this can be useful.

\item [SKip] --- Skip images.

Skip files on input or output tape.

Syntax:   `SKip  tape  \#'    

where `tape' is INPUT or OUTPUT and \# is a positive or negative number.

EDFITS skips over the specified number of files on the specified tape.
A positive number will cause the tape to be skipped forwards, and a
negative number will cause a backwards skip.  The tape will be left 
positioned at the start of a file.  LIST can be used to check that
it is positioned where you wanted it.

\item [ST] --- Skip to image.

Skip to a specified file on a tape.

Syntax:    `ST  tape  \#'      

where `tape' is INPUT or OUTPUT and \# is a file number.

EDFITS will skip to the start of the specified file on the specified
tape.  LIST can be used to check that it is positioned where you wanted
it.  The position of the tape must be known for ST to work.

\item [TERM] --- Terminate tape.

Write an end of data marker on output tape.

Syntax:   `TErm'            

can only be used for the output tape.

An end of data marker (two successive filemarks) will be written on the
output tape at its current position.  This will make it impossible to
read any data further up the tape, so be CERTAIN the tape is in the
place you want.    You have to explicitly confirm this command with a
`YES' response to the next prompt.  `NO' will abort the operation.
Please note: TERM is ONLY intended to be used to tidy up bad tapes.

\end{description}

\section{Using EDFITS with non-FITS tapes}

The only parts of EDFITS that expressly assume that the input or output
tape is in FITS format are the listing commands `LIST', `PD' and `FD'.
The copying and positioning commands just treat the tapes as having a
number of files and a double file mark to indicate the end of
data on the tape.  Any tape that looks like this can be copied using EDFITS.
This includes, for example, VAX standard tapes---indeed, EDFITS is the 
program its author uses to copy VMS BACKUP tapes.  The ability to suppress
the re-reading of records with parity errors makes EDFITS a useful utility
for making a copy of tapes with large numbers of parity errors.  EDFITS
internal buffer sizes mean that it cannot handle tapes with records longer
than 8192 bytes.

When using non-FITS tapes it helps if the `NONE' option is selected when
you are prompted for the tape type.  If EDFITS thinks it is copying a FITS
tape and it gets a parity error on input, it will force the output record
length to be the standard FITS length of 2880 bytes if the input record 
seems to be close to but not exactly that length.  This is an extremely
minor consideration and EDFITS will normally copy any non-FITS tape 
quite happily no matter what type it was told it was.

\section{Logical names}

EDFITS makes use of two files which it refers to by the logical names
`EDFITS\_HELP' and `EDFITS\_TYPES'.  

EDFITS\_HELP should point to the file containing the various help and
information blocks that are displayed during the running of EDFITS.  Normally
this should be the file supplied with EDFITS as EDFITS.INF, and there
should be little need to modify this file or to override the logical name.

EDFITS\_TYPES should point to the file containing the specifications for the
various flavours of FITS tapes that EDFITS needs to know about for the
`Partial Directory' command.  The file originally supplied for this
purpose with EDFITS, EDFITS.TYP, contains a number of type descriptions
that may be useful as they stand but are more likely to be used as
templates as described later.  This means that this file may be modified
by the system manager, or alternatively users may create their own versions
and use them by redefining the EDFITS\_TYPES logical name.

Installing EDFITS consists only of making these files available, defining these
two logical names and defining the EDFITS command itself.  EDFITS needs no
particular privileges or quotas.

\section{Describing a new type of FITS tape}

EDFITS is told the details of which keywords to use for a partial directory
of a tape through the type description file whose logical name is EDFITS\_TYPES.
To add a new type to EDFITS' repertoire this file must be edited to include
the description of the new type.  The description lists the keywords to be
used in the partial directory, the output columns to be used for each, the
column headings, and any formatting to be performed on the value associated
with each keyword.

The rules governing the format of this file are as follows:

\begin{itemize}

\item Any line whose first character is `*' is treated as a comment.    
Blank lines are ignored.
                                                                     
\item The type described first will become the default type.

\item A new type description is begun with a line that has a character
other than `*' in the first column.  The first word is the name
of the type.  Anything else on the line is regarded as just
descriptive text.  This line should then be followed by lines
describing the keywords to be used.

\item A keyword description line has up to four parts, of which the 
last two are optional.  They are separated by blanks.

The first part is the name of the keyword itself.  In addition
to the actual names of FITS keywords, the following are also
recognised:
\begin{description}
\item{\%FILE} is the file number on the tape.  Note that this refers only
to which physical file on the tape is being processed, which is all that 
EDFITS knows about.  
\item{\%BLANK} is a comment line - cols 1-8 in the FITS line are blank.
\item{\%SIZE} is used for a formatted description of the image size.  This
is based on the values of the NAXIS, NAXISn keywords.
\item{\%ERROR} is used to indicate where error messages should be placed.  This
refers to errors generated by EDFITS itself while processing the keywords.
\end{description}
The second part indicates the columns to be used.  Columns n
through m may be specified as `n-m', or `n m' or `n -m'.
Column numbers must be in the range 1-132.

The third part indicates the heading for the field in the output
file.  If it includes blanks it must be enclosed in quotes.

The fourth part is a conversion specifier.  Normally, EDFITS can
tell from the FITS header itself how to list the keyword value, but
the following special cases are recognised:
\begin{description}
\item[LIT] indicates that all the FITS line past column 9 is to be
used.  \%BLANK, HISTORY and COMMENT should all be LIT.
\item[D-$>$DMS]  indicates that the keyword is a decimal degree value that
should be listed in +dd:mm:ss format.  Declination values
are often like this.
\item[D-$>$HMS] indicates that the keyword is a decimal degree value that
should be divided by 15 and listed in hh:mm:ss format.
Right ascension values are often like this.
\end{description}

For example, the following is a description used for the AAO standard FITS
format:
\begin{quote}
\begin{tt}
AAO\hspace{10ex}New AAO standard tape format
\begin{tabbing}
xxxxxxxxx \= xxx-xxx \= `xxxxxxxxxxx' \= xxxxxx \kill
\%FILE    \> 2-8     \> `File \#' \\
OBJECT    \> 10-25   \> `Object name' \\
\%SIZE    \> 27-42   \> `Size' \\
RUN       \> 44-49   \> `Run \#' \\
MEANRA    \> 51-60   \> `R.A.'        \> D->HMS \\
MEANDEC   \> 62-70   \> `Dec.'        \> D->DMS \\
INSTRUME  \> 71-85   \> `Instrument' \\
\%BLANK   \> 87-120  \> `Comments'    \> LIT \\
COMMENT   \> 87-120  \> `Comments'    \> LIT \\
HISTORY   \> 87-120  \> `Comments'    \> LIT \\
\%ERROR   \> 87-120
\end{tabbing}
\end{tt}
\end{quote}
Note that this definition uses the same field for any of the three possible
ways of inserting a comment line into a FITS header---the blank keyword, the
COMMENT keyword and the HISTORY keyword.  If more than one of these are
present in any header processed, only one will be included in the output
line, and error text uses the same field in the output line.
\end{itemize}

\section{Undocumented features}

EDFITS has a few features that are described as `undocumented'.  Clearly,
since you are reading their documentation, this is a misnomer.  The term is
used to refer to features of EDFITS that one is not normally expected to use,
but which may be useful in some circumstances.  You should think carefully
before using these.
\begin{description}

\item[No Input Tape]:

You can reply `NONE' when prompted for the input tape.
Running EDFITS with no input tape seems foolish---and usually is.  It is
used mainly when testing the program, and probably has no normal application.

\item[Write-enabled Input Tapes]:

Sometimes, you want to override the way
EDFITS will reject an input tape that has a write ring fitted.  Let it be
on your head if you overwrite your irreplaceable data, but you can specify
the input tape as, for example, `MTA0:/WRITEOK'.  The WRITEOK qualifier
may be abbreviated.
\item[/LEAVE on Output]:

If you specify the output tape as, for example,
`MTA1/LEAVE', then EDFITS will bypass any positioning of the output tape
at the setup stage.  This is sometimes useful for output tapes that
already have a lot of data files. If you know, for example, that the first
thing you are going to do is a directory of the output tape, then the 
normal way EDFITS locates the end of the tape data during setup is both
time-consuming and unnecessary.  Using /LEAVE has no particularly dreadful
consequences, since EDFITS will always find the end of tape before writing
new files to it, but it goes against the `safety-first' policy of the 
EDFITS design.
\end{description}
\end{document}
