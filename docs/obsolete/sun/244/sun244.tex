\documentclass[twoside,11pt]{article}

% ? Specify used packages
\usepackage{graphicx}        %  Use this one for final production.
% \usepackage[draft]{graphicx} %  Use this one for drafting.
% ? End of specify used packages

\pagestyle{myheadings}
\raggedbottom

% -----------------------------------------------------------------------------
% ? Document identification
% Fixed part
\newcommand{\stardoccategory}  {Starlink User Note}
\newcommand{\stardocinitials}  {SUN}
\newcommand{\stardocsource}    {sun\stardocnumber}
\newcommand{\stardoccopyright}
{Copyright \copyright\ 2001 Council for the Central Laboratory of the Research Councils}

% Variable part - replace [xxx] as appropriate.
\newcommand{\stardocnumber}    {244.1}
\newcommand{\stardocauthors}   {Mark Taylor}
\newcommand{\stardocdate}      {30 October 2001}
\newcommand{\stardoctitle}     {TREEVIEW - Hierarchical data viewer}
\newcommand{\stardocversion}   {1.0}
\newcommand{\stardocmanual}    {User's Manual}
\newcommand{\stardocabstract}  {
TREEVIEW is a graphical tool for examination of hierarchical data structures.
It knows all about the Starlink HDS and NDF file formats, as well
as being able to do basic examination of XML documents, 
FITS files, Zip or Jar files, and filesystem directory trees.
}
% ? End of document identification
% -----------------------------------------------------------------------------

% +
%  Name:
%     sun.tex
%
%  Purpose:
%     Template for Starlink User Note (SUN) documents.
%     Refer to SUN/199
%
%  Authors:
%     AJC: A.J.Chipperfield (Starlink, RAL)
%     BLY: M.J.Bly (Starlink, RAL)
%     PWD: Peter W. Draper (Starlink, Durham University)
%
%  History:
%     17-JAN-1996 (AJC):
%        Original with hypertext macros, based on MDL plain originals.
%     16-JUN-1997 (BLY):
%        Adapted for LaTeX2e.
%        Added picture commands.
%     13-AUG-1998 (PWD):
%        Converted for use with LaTeX2HTML version 98.2 and
%        Star2HTML version 1.3.
%      1-FEB-2000 (AJC):
%        Add Copyright statement in LaTeX
%     {Add further history here}
%
% -

\newcommand{\stardocname}{\stardocinitials /\stardocnumber}
\markboth{\stardocname}{\stardocname}
\setlength{\textwidth}{160mm}
\setlength{\textheight}{230mm}
\setlength{\topmargin}{-2mm}
\setlength{\oddsidemargin}{0mm}
\setlength{\evensidemargin}{0mm}
\setlength{\parindent}{0mm}
\setlength{\parskip}{\medskipamount}
\setlength{\unitlength}{1mm}

% -----------------------------------------------------------------------------
%  Hypertext definitions.
%  ======================
%  These are used by the LaTeX2HTML translator in conjunction with star2html.

%  Comment.sty: version 2.0, 19 June 1992
%  Selectively in/exclude pieces of text.
%
%  Author
%    Victor Eijkhout                                      <eijkhout@cs.utk.edu>
%    Department of Computer Science
%    University Tennessee at Knoxville
%    104 Ayres Hall
%    Knoxville, TN 37996
%    USA

%  Do not remove the %begin{latexonly} and %end{latexonly} lines (used by
%  LaTeX2HTML to signify text it shouldn't process).
%begin{latexonly}
\makeatletter
\def\makeinnocent#1{\catcode`#1=12 }
\def\csarg#1#2{\expandafter#1\csname#2\endcsname}

\def\ThrowAwayComment#1{\begingroup
    \def\CurrentComment{#1}%
    \let\do\makeinnocent \dospecials
    \makeinnocent\^^L% and whatever other special cases
    \endlinechar`\^^M \catcode`\^^M=12 \xComment}
{\catcode`\^^M=12 \endlinechar=-1 %
 \gdef\xComment#1^^M{\def\test{#1}
      \csarg\ifx{PlainEnd\CurrentComment Test}\test
          \let\html@next\endgroup
      \else \csarg\ifx{LaLaEnd\CurrentComment Test}\test
            \edef\html@next{\endgroup\noexpand\end{\CurrentComment}}
      \else \let\html@next\xComment
      \fi \fi \html@next}
}
\makeatother

\def\includecomment
 #1{\expandafter\def\csname#1\endcsname{}%
    \expandafter\def\csname end#1\endcsname{}}
\def\excludecomment
 #1{\expandafter\def\csname#1\endcsname{\ThrowAwayComment{#1}}%
    {\escapechar=-1\relax
     \csarg\xdef{PlainEnd#1Test}{\string\\end#1}%
     \csarg\xdef{LaLaEnd#1Test}{\string\\end\string\{#1\string\}}%
    }}

%  Define environments that ignore their contents.
\excludecomment{comment}
\excludecomment{rawhtml}
\excludecomment{htmlonly}

%  Hypertext commands etc. This is a condensed version of the html.sty
%  file supplied with LaTeX2HTML by: Nikos Drakos <nikos@cbl.leeds.ac.uk> &
%  Jelle van Zeijl <jvzeijl@isou17.estec.esa.nl>. The LaTeX2HTML documentation
%  should be consulted about all commands (and the environments defined above)
%  except \xref and \xlabel which are Starlink specific.

\newcommand{\htmladdnormallinkfoot}[2]{#1\footnote{#2}}
\newcommand{\htmladdnormallink}[2]{#1}
\newcommand{\htmladdimg}[1]{}
\newcommand{\hyperref}[4]{#2\ref{#4}#3}
\newcommand{\htmlref}[2]{#1}
\newcommand{\htmlimage}[1]{}
\newcommand{\htmladdtonavigation}[1]{}

\newenvironment{latexonly}{}{}
\newcommand{\latex}[1]{#1}
\newcommand{\html}[1]{}
\newcommand{\latexhtml}[2]{#1}
\newcommand{\HTMLcode}[2][]{}

%  Starlink cross-references and labels.
\newcommand{\xref}[3]{#1}
\newcommand{\xlabel}[1]{}

%  LaTeX2HTML symbol.
\newcommand{\latextohtml}{\LaTeX2\texttt{HTML}}

%  Define command to re-centre underscore for Latex and leave as normal
%  for HTML (severe problems with \_ in tabbing environments and \_\_
%  generally otherwise).
\renewcommand{\_}{\texttt{\symbol{95}}}

% -----------------------------------------------------------------------------
%  Debugging.
%  =========
%  Remove % on the following to debug links in the HTML version using Latex.

% \newcommand{\hotlink}[2]{\fbox{\begin{tabular}[t]{@{}c@{}}#1\\\hline{\footnotesize #2}\end{tabular}}}
% \renewcommand{\htmladdnormallinkfoot}[2]{\hotlink{#1}{#2}}
% \renewcommand{\htmladdnormallink}[2]{\hotlink{#1}{#2}}
% \renewcommand{\hyperref}[4]{\hotlink{#1}{\S\ref{#4}}}
% \renewcommand{\htmlref}[2]{\hotlink{#1}{\S\ref{#2}}}
% \renewcommand{\xref}[3]{\hotlink{#1}{#2 -- #3}}
%end{latexonly}
% -----------------------------------------------------------------------------
% ? Document specific \newcommand or \newenvironment commands.
% ? End of document specific commands
% -----------------------------------------------------------------------------

%+
%  Name:
%     SST.TEX

%  Purpose:
%     Define LaTeX commands for laying out Starlink routine descriptions.

%  Language:
%     LaTeX

%  Type of Module:
%     LaTeX data file.

%  Description:
%     This file defines LaTeX commands which allow routine documentation
%     produced by the SST application PROLAT to be processed by LaTeX and
%     by LaTeX2html. The contents of this file should be included in the
%     source prior to any statements that make of the sst commnds.

%  Notes:
%     The style file html.sty provided with LaTeX2html needs to be used.
%     This must be before this file.

%  Authors:
%     RFWS: R.F. Warren-Smith (STARLINK)
%     PDRAPER: P.W. Draper (Starlink - Durham University)

%  History:
%     10-SEP-1990 (RFWS):
%        Original version.
%     10-SEP-1990 (RFWS):
%        Added the implementation status section.
%     12-SEP-1990 (RFWS):
%        Added support for the usage section and adjusted various spacings.
%     8-DEC-1994 (PDRAPER):
%        Added support for simplified formatting using LaTeX2html.
%     {enter_further_changes_here}

%  Bugs:
%     {note_any_bugs_here}

%-

%  Define length variables.
\newlength{\sstbannerlength}
\newlength{\sstcaptionlength}
\newlength{\sstexampleslength}
\newlength{\sstexampleswidth}

%  Define a \tt font of the required size.
\latex{\newfont{\ssttt}{cmtt10 scaled 1095}}
\html{\newcommand{\ssttt}{\tt}}

%  Define a command to produce a routine header, including its name,
%  a purpose description and the rest of the routine's documentation.
\newcommand{\sstroutine}[3]{
   \newpage
   \label{#1}
   \goodbreak
   \rule{\textwidth}{0.5mm}
   \vspace{-7ex}
   \newline
   \settowidth{\sstbannerlength}{{\Large {\bf #1}}}
   \setlength{\sstcaptionlength}{\textwidth}
   \setlength{\sstexampleslength}{\textwidth}
   \addtolength{\sstbannerlength}{0.5em}
   \addtolength{\sstcaptionlength}{-2.0\sstbannerlength}
   \addtolength{\sstcaptionlength}{-5.0pt}
   \settowidth{\sstexampleswidth}{{\bf Examples:}}
   \addtolength{\sstexampleslength}{-\sstexampleswidth}
   \parbox[t]{\sstbannerlength}{\flushleft{\Large {\bf #1}}}
   \parbox[t]{\sstcaptionlength}{\center{\Large #2}}
   \parbox[t]{\sstbannerlength}{\flushright{\Large {\bf #1}}}
   \begin{description}
      #3
   \end{description}
}

%  Format the description section.
\newcommand{\sstdescription}[1]{\item[Description:] #1}

%  Format the usage section.
\newcommand{\sstusage}[1]{\item[Usage:] \mbox{}
   \begin{description}
      {\ssttt \item #1}
   \end{description}
}
% \newcommand{\sstusage}[1]{\item[Usage:] \mbox{}
% \\[1.3ex]{\raggedright \ssttt #1}}

%  Format the invocation section.
\newcommand{\sstinvocation}[1]{\sloppy \item[Invocation:]\hspace{0.4em}{\tt #1}}
%\newcommand{\sstinvocation}[1]{\item[Invocation:]\hspace{0.4em}{\tt #1}}

%  Format the arguments section.
\newcommand{\sstarguments}[1]{
   \item[Arguments:] \mbox{} \\
   \vspace{-3.5ex}
   \begin{description}
      #1
   \end{description}
}

%  Format the returned value section (for a function).
\newcommand{\sstreturnedvalue}[1]{
   \item[Returned Value:] \mbox{} \\
   \vspace{-3.5ex}
   \begin{description}
      #1
   \end{description}
}

%  Format the parameters section (for an application).
\newcommand{\sstparameters}[1]{
   \item[Parameters:] \mbox{} \\
   \vspace{-3.5ex}
   \begin{description}
      #1
   \end{description}
}

%  Format the examples section.
\newcommand{\sstexamples}[1]{
   \item[Examples:] \mbox{} \\
   \vspace{-3.5ex}
   \begin{description}
      #1
   \end{description}
}

%  Define the format of a subsection in a normal section.
\newcommand{\sstsubsection}[1]{ \item[{#1}] \mbox{} \\}

%  Define the format of a subsection in the examples section.
\newcommand{\sstexamplesubsection}[2]{\sloppy \item{\ssttt #1} \mbox{} \\ #2 }
%\newcommand{\sstexamplesubsection}[2]{\sloppy
%\item[\parbox{\sstexampleslength}{\ssttt #1}] \mbox{} \vspace{1.0ex}
%\\ #2 }

%  Format the notes section.
\newcommand{\sstnotes}[1]{\item[Notes:] \mbox{} \\[1.3ex] #1}

%  Provide a general-purpose format for additional (DIY) sections.
\newcommand{\sstdiytopic}[2]{\item[#1:] \mbox{} \\[1.3ex] #2}
%\newcommand{\sstdiytopic}[2]{\item[{\hspace{-0.35em}#1\hspace{-0.35em}:}]
%\mbox{} \\[1.3ex] #2}

%  Format the implementation status section.
\newcommand{\sstimplementationstatus}[1]{
   \item[{Implementation Status:}] \mbox{} \\[1.3ex] #1}

%  Format the bugs section.
\newcommand{\sstbugs}[1]{\item[Bugs:] #1}

%  Format a list of items while in paragraph mode.
\newcommand{\sstitemlist}[1]{
  \mbox{} \\
  \vspace{-3.5ex}
  \begin{itemize}
     #1
  \end{itemize}
}

%  Define the format of an item.
\newcommand{\sstitem}{\item}

%% Now define html equivalents of those already set. These are used by
%  latex2html and are defined in the html.sty files.
\begin{htmlonly}

%  sstroutine.
   \newcommand{\sstroutine}[3]{
      \subsection{#1\xlabel{#1}-\label{#1}#2}
      \begin{description}
         #3
      \end{description}
   }

%  sstdescription
   \newcommand{\sstdescription}[1]{\item[Description:]
      \begin{description}
         #1
      \end{description}
      \\
   }

%  sstusage
   \newcommand{\sstusage}[1]{\item[Usage:]
      \begin{description}
         {\ssttt #1}
      \end{description}
      \\
   }

%  sstinvocation
   \newcommand{\sstinvocation}[1]{\item[Invocation:]
      \begin{description}
         {\ssttt #1}
      \end{description}
      \\
   }

%  sstarguments
   \newcommand{\sstarguments}[1]{
      \item[Arguments:] \\
      \begin{description}
         #1
      \end{description}
      \\
   }

%  sstreturnedvalue
   \newcommand{\sstreturnedvalue}[1]{
      \item[Returned Value:] \\
      \begin{description}
         #1
      \end{description}
      \\
   }

%  sstparameters
   \newcommand{\sstparameters}[1]{
      \item[Parameters:] \\
      \begin{description}
         #1
      \end{description}
      \\
   }

%  sstexamples
   \newcommand{\sstexamples}[1]{
      \item[Examples:] \\
      \begin{description}
         #1
      \end{description}
      \\
   }

%  sstsubsection
   \newcommand{\sstsubsection}[1]{\item[{#1}]}

%  sstexamplesubsection
   \newcommand{\sstexamplesubsection}[2]{\item[{\ssttt #1}] #2}

%  sstnotes
   \newcommand{\sstnotes}[1]{\item[Notes:] #1 }

%  sstdiytopic
   \newcommand{\sstdiytopic}[2]{\item[{#1:}] #2 }

%  sstimplementationstatus
   \newcommand{\sstimplementationstatus}[1]{
      \item[Implementation Status:] #1
   }

%  sstitemlist
   \newcommand{\sstitemlist}[1]{
      \begin{itemize}
         #1
      \end{itemize}
      \\
   }
%  sstitem
   \newcommand{\sstitem}{\item}

\end{htmlonly}

%  End of "sst.tex" layout definitions.
%.

%  Title Page.
%  ===========
\renewcommand{\thepage}{\roman{page}}
\begin{document}
\thispagestyle{empty}

%  Latex document header.
%  ======================
\begin{latexonly}
   CCLRC / \textsc{Rutherford Appleton Laboratory} \hfill \textbf{\stardocname}\\
   {\large Particle Physics \& Astronomy Research Council}\\
   {\large Starlink Project\\}
   {\large \stardoccategory\ \stardocnumber}
   \begin{flushright}
   \stardocauthors\\
   \stardocdate
   \end{flushright}
   \vspace{-4mm}
   \rule{\textwidth}{0.5mm}
   \vspace{5mm}
   \begin{center}
      {\LARGE\textbf{\stardoctitle \\ [2.5ex]}}
   \end{center}
   \vspace{5mm}

% ? Add picture here if required for the LaTeX version.
% \begin{center}
% \includegraphics[scale=0.6]{sun244treefront.eps}
% \end{center}
% ? End of picture

% ? Heading for abstract if used.
   %\vspace{10mm}
   \begin{center}
      {\Large\textbf{Abstract}}
   \end{center}
% ? End of heading for abstract.
\end{latexonly}

%  HTML documentation header.
%  ==========================
\begin{htmlonly}
   \xlabel{}
   \begin{rawhtml} <H1 ALIGN=CENTER>\end{rawhtml}
      \stardoctitle
   \begin{rawhtml} </FONT></H1> \end{rawhtml}

% ? Add picture here if required for the hypertext version.
% \begin{center}
%    \htmladdimg{treefront.gif}
% \end{center}
% ? End of picture

   \begin{rawhtml} <P> <I> \end{rawhtml}
   \stardoccategory\ \stardocnumber \\
   \stardocauthors \\
   \stardocdate
   \begin{rawhtml} </I> </P> <H3> \end{rawhtml}
      \htmladdnormallink{CCLRC / Rutherford Appleton Laboratory}
                        {http://www.cclrc.ac.uk} \\
      \htmladdnormallink{Particle Physics \& Astronomy Research Council}
                        {http://www.pparc.ac.uk} \\
   \begin{rawhtml} </H3> <H2> \end{rawhtml}
      \htmladdnormallink{Starlink Project}{http://www.starlink.rl.ac.uk/}
   \begin{rawhtml} </H2> \end{rawhtml}
   \htmladdnormallink{\htmladdimg{source.gif} Retrieve hardcopy}
      {http://www.starlink.rl.ac.uk/cgi-bin/hcserver?\stardocsource}\\

%  HTML document table of contents.
%  ================================
%  Add table of contents header and a navigation button to return to this
%  point in the document (this should always go before the abstract \section).
  \label{stardoccontents}
  \begin{rawhtml}
    <HR>
    <H2>Contents</H2>
  \end{rawhtml}
  \htmladdtonavigation{\htmlref{\htmladdimg{contents_motif.gif}}
        {stardoccontents}}

% ? New section for abstract if used.
  \section{\xlabel{abstract}Abstract}
% ? End of new section for abstract
\end{htmlonly}

% -----------------------------------------------------------------------------
% ? Document Abstract. (if used)
%  ==================
\stardocabstract
% ? End of document abstract

% -----------------------------------------------------------------------------
% ? Latex Copyright Statement
%  =========================
\begin{latexonly}
\newpage
\vspace*{\fill}
\stardoccopyright
\end{latexonly}
% ? End of Latex copyright statement

% -----------------------------------------------------------------------------
% ? Latex document Table of Contents (if used).
%  ===========================================
\newpage
\begin{latexonly}
  \setlength{\parskip}{0mm}
  \tableofcontents
  \setlength{\parskip}{\medskipamount}
  \markboth{\stardocname}{\stardocname}
\end{latexonly}
% ? End of Latex document table of contents
% -----------------------------------------------------------------------------

% Icon definitions.
\newcommand{\iconOpen}
    {\latexhtml{`+'}{\htmladdimg{Plus.gif}}}
\newcommand{\iconClose}
    {\latexhtml{`-'}{\htmladdimg{Minus.gif}}}
\newcommand{\iconCascade}
    {\latexhtml{`++'}{\htmladdimg{PlusPlus.gif}}}
\newcommand{\iconExcise}
    {\latexhtml{`-\,-'}{\htmladdimg{MinusMinus.gif}}}
\newcommand{\iconSplitNone}
    {\latexhtml{`\texttt{O}'}{\htmladdimg{Frame.gif}}}
\newcommand{\iconSplitBelow}
    {\latexhtml{`\texttt{=}'}{\htmladdimg{SplitHorizontal.gif}}}
\newcommand{\iconSplitBeside}
    {\latexhtml{`\texttt{||}'}{\htmladdimg{SplitVertical.gif}}}
\newcommand{\iconExit}
    {\latexhtml{`$\times$'}{\htmladdimg{exit1.gif}}}
\newcommand{\iconHelp}
    {\latexhtml{`\texttt{?}'}{\htmladdimg{Help2.gif}}}

\newcommand{\iconHandleOpen}
    {\latexhtml{`o-'}{\htmladdimg{handle_open1.gif}}}
\newcommand{\iconHandleClosed}
    {\latexhtml{`P'}{\htmladdimg{handle_closed1.gif}}}

\newcommand{\iconAryZero}
    {\latexhtml{ARY}{\htmladdimg{cell.gif}}}
\newcommand{\iconAryOne}
    {\latexhtml{ARY}{\htmladdimg{row.gif}}}
\newcommand{\iconAryTwo}
    {\latexhtml{ARY}{\htmladdimg{Sheet.gif}}}
\newcommand{\iconAryThree}
    {\latexhtml{ARY}{\htmladdimg{Sheets.gif}}}
\newcommand{\iconDir}
    {\latexhtml{DIR}{\htmladdimg{defaultClosed2.gif}}}
\newcommand{\iconFile}
    {\latexhtml{File}{\htmladdimg{defaultLeaf2.gif}}}
\newcommand{\iconFits}
    {\latexhtml{FIT}{\htmladdimg{star_bul.gif}}}
\newcommand{\iconFrame}
    {\latexhtml{FRM}{\htmladdimg{axes42.gif}}}
\newcommand{\iconHdu}
    {\latexhtml{HDU}{\htmladdimg{TileCascade.gif}}}
\newcommand{\iconTable}
    {\latexhtml{TBL}{\htmladdimg{table10.gif}}}
\newcommand{\iconNdf}
    {\latexhtml{NDF}{\htmladdimg{star_gold.gif}}}
\newcommand{\iconSkyframe}
    {\latexhtml{SKY}{\htmladdimg{axes62.gif}}}
\newcommand{\iconStructure}
    {\latexhtml{HDS}{\htmladdimg{closed.gif}}}
\newcommand{\iconWcs}
    {\latexhtml{WCS}{\htmladdimg{world1.gif}}}
\newcommand{\iconZipentry}
    {\latexhtml{ZIP}{\htmladdimg{squishfile2.gif}}}
\newcommand{\iconZipfile}
    {\latexhtml{ZPE}{\htmladdimg{squishdir2.gif}}}
\newcommand{\iconHistory}
    {\latexhtml{HIS}{\htmladdimg{book2.gif}}}
\newcommand{\iconHistoryRecord}
    {\latexhtml{HRE}{\htmladdimg{mini-doc.gif}}}
% \newcommand{\iconNdx}
%     {\latexhtml{NDX}{\htmladdimg{jsky2.gif}}}
% \newcommand{\iconHdx}
%     {\latexhtml{HDX}{\htmladdimg{box7.gif}}}

\newcommand{\iconXmlDocument}
    {\latexhtml{XML}{\htmladdimg{xml_doc.gif}}}
\newcommand{\iconXmlElement}
    {\latexhtml{ELE}{\htmladdimg{xml_el.gif}}}
\newcommand{\iconXmlComment}
    {\latexhtml{COM}{\htmladdimg{xml_comm.gif}}}
\newcommand{\iconXmlPi}
    {\latexhtml{XPI}{\htmladdimg{xml_pi.gif}}}
\newcommand{\iconXmlCdata}
    {\latexhtml{CDA}{\htmladdimg{xml_txt.gif}}}
\newcommand{\iconXmlEref}
    {\latexhtml{XER}{\htmladdimg{xml_eref.gif}}}
\newcommand{\iconXmlString}
    {\latexhtml{TXT}{\htmladdimg{xml_txt.gif}}}


% Figure display.
\newcommand{\showpicture}[1]{
   \begin{quote} 
   \latexhtml{\includegraphics[scale=0.6]{sun244#1.eps}}{\htmladdimg{#1.gif}}
   \end{quote}}
%\newcommand{\showpicture}[1]{
%   \begin{quote} 
%   \latexhtml{\fbox{Picture #1 here}}{\htmladdimg{#1.gif}}
%   \end{quote}}

\cleardoublepage
\renewcommand{\thepage}{\arabic{page}}
\setcounter{page}{1}

% ? Main text

\section{Introduction\xlabel{introduction}}

Treeview gives a convenient way to display and explore 
hierarchical data structures;  
it provides similar functionality to the \xref{HDSTRACE}{sun102}{}
and \xref{NDFTRACE}{sun95}{NDFTRACE} tasks but with a graphical 
interface (though it can be used in text mode if desired).
It also knows about 
many other data structures such as file system directory
hierarchies, zip files, FITS files and XML documents.
% , and HDX containers.

The program is installed as a Starlink package in the normal way;
it requires installation of the JNIAST package 
(\texttt{uk.ac.starlink.ast.*} classes and associated shared libraries)
and of a J2SE Java
Runtime Environment (JRE) at version 1.3 or higher in order to run.
Installation installs a script in \texttt{\$INSTALL/bin/treeview}
which is used to run the application.

As long as the installation binary directory is on your path, 
you can just type
\begin{quote}
\begin{verbatim}
% treeview item [item...]
\end{verbatim}
\end{quote}
where each \texttt{item} is the name of a file, directory, or HDS or NDF 
component.
Following a short delay, a window will pop up showing the item or
items you named, and you can point and click to see the 
images and what is inside them.
Navigation and usage is pretty intuitive
(especially if you have encountered a java JTree widget before, which
is the basis of the visual display), and you may well find you can
see how it works by playing with it.
If you want more details on the workings however, 
you can read the following sections.
\begin{latexonly}

\it Note: 
This document includes representations of icons used in the Treeview GUI.
In the \LaTeX\ version they are represented using rather crude textual
versions.  If you are keen to know what they look like, you may be
better off with the HTML version.
\end{latexonly}

\section{The Treeview window}

Suppose you wish to display all the NDFs in the current directory
starting with the string `reduced\_'.  You could invoke Treeview like this:
\begin{quote}
\begin{verbatim}
% treeview reduced_*.sdf
\end{verbatim}
\end{quote}
which would bring up a window looking something like this:
\showpicture{treeshut}
As you can see, the window is divided into several parts,
described in the following sections.

\subsection{Tree display}

The main area to the left of the window displays the tree with the
data nodes in it.  Each node is shown on a single line,
and is represented by an icon,
followed by its name, followed by some brief descriptive information.
Additionally, there may be a {\em handle} 
\begin{htmlonly}
like this \iconHandleClosed\  
\end{htmlonly}
at the left of the line
if this node can be opened up to see inside.
In the example shown in the previous section, 
each icon is a little star, indicating that the nodes
have been recognised as NDF structures, the names are as given,
and the descriptive information gives the pixel bounds of the NDFs.
These nodes do have handles, since Treeview knows how to understand
the data inside them.
The different types of nodes which Treeview knows about are listed
in section \ref{sec:nodetypes}.

Nodes with handles can be opened up to display their {\em children}.
The easiest way to do this is to click on the handle itself with
the mouse, though various other actions will do the trick, 
including double-clicking on the name, and if the node is 
{\em selected} (see \ref{sec:detail}) pressing the return key or clicking
the \iconOpen\ icon in the toolbar.  Once opened, they can be
closed in a similar way.  The child nodes behave like their
parents, and can be opened and closed in the same fashion.
\begin{htmlonly}
When a node is open the handle is rotated to look like this
\iconHandleOpen.
\end{htmlonly}

Here is a what the window looks like when a couple of the nodes
have been opened:
\showpicture{treeopen}

\subsubsection{Popup menu}

If you activate a popup menu on one of the displayed nodes in the
tree display (this is normally done by pressing the right mouse button)
you will in most cases see a menu which offers you two types of option:
\begin{description}
\item[Reload]\mbox{}\\
This forgets what it knows about the node in question and its children, 
reloads it.  It can be used to refresh information about the state
of an object which has changed since it was first read by Treeview.
{\it Note: reload currently does not always work for HDS-type nodes}.
\item[Alter-egos]\mbox{}\\
The remaining options present different ways of looking at the node.
For instance an NDF file may also be considered as an HDS container file
or as a normal file.  By selecting one of these, the node will be 
replaced by the alternative view.
\end{description}
Some types of node cannot be reloaded or viewed as alteregos in this
way - you will get a beep rather than a popup menu in this case.


\subsection{\label{sec:detail}Detail display}

The panel to the right of the tree display initially shows a short 
help message.  However, for the most part, this panel is used to
display further details about the node which is currently
{\em selected}.

The easiest way to select a node is to click on its name in the 
tree display.  Other methods include using the arrow keys on the
keyboard to move the selection up and down.  There can be at most
one selected node at any time.
When a node is selected, its line in the tree display panel is
highlighted, and further information about it is shown 
in the detail display panel.  What information this is will 
depend on the kind of node it is. 

At the least, an overview panel will be shown giving at least the
node's name and type.  It may also include other information 
about the contents of the node according to what Treeview knows
about its structure, though hierarchical information of the sort
that can be seen by expanding it in the tree panel will generally
not be shown.  In some cases a limited amount of the data content 
itself may be shown, for instance the detail overview for an HDS 
data array node will show the values of the first few elements of the 
data array itself, but if there are too many they will not all be given.
Here is a simple example of a selected element and its overview:
\showpicture{selected}

In some cases, where there are various ways you might want to look
at the data of a node, a set of tabs will be shown at the top of
the details panel.  The Overview one will be shown by default, but
to look at the other ones, click the tab you want and you will see
the relevant display.  For instance, if you are looking at a 
two-dimensional NDF, you can switch between the summary overview 
and a drawing of the area represented by the data grid in its
various World Coordinate System frames.
Here is an example of such a tabbed detail panel:
\showpicture{withgrids}

If help text is shown, for instance on startup or if you have
selected it from the menu, the text will be displayed in the
detail display panel.

The divider which separates the detail display panel from the
tree panel can be dragged by moving the mouse while holding down
the button.  This enables the relative sizes of the two panels
to be changed.  The entire window can be resized in the usual way.

Although the detail display panel by default appears to the right of
the tree panel, you can change its position according to taste, so that it 
appears below the tree panel, or not at all.  This can be done using
the \iconSplitNone, \iconSplitBeside\ and \iconSplitBelow\ 
icons in the toolbar, 
or the {\em View} menu, or on the \texttt{treeview} command line.

% Define actions which will appear in both the Menu and Toolbar subsections.
\newcommand{\actionOpen}{
   Brings up a dialog box in which you can choose a new file
   to add as a top-level node of the tree.  When you select
   a file and click the `Add node' button in the selection
   dialog it will be added as the bottom line in the tree
   display window.
}
\newcommand{\actionExit}{
   Exits Treeview without further ado.
}
\newcommand{\actionBeside}{
   The details panel is shown to the right of the tree panel.
}
\newcommand{\actionBelow}{
   The details panel is shown below the tree panel.
}
\newcommand{\actionNone}{
   The entire window is used for the tree panel;
   no details display is shown.
}
\newcommand{\actionCollapse}{
   The selected node is closed, so that its children are no longer
   visible.  This option is not available if no node is currently 
   selected, or if the selected node is already closed.
}
\newcommand{\actionExpand}{
   The selected node is opened, so that its children become 
   visible.  This option is not available if no node is currently 
   selected, or if the selected node is already open.
}
\newcommand{\actionRCollapse}{
   The selected node, and all its children, and all its children's 
   children\ldots\ are closed, and if it is opened again its children
   will be treated as if they had not been encountered before, and
   re-read.  This will be slower, but may result in more up-to-date
   information about the nodes.
   Since the child nodes are forgotten during a recursive collapse,
   this may be useful if viewing very large data structures if 
   memory needs to be freed up.
   This option is not available if no node is currently 
   selected.
}
\newcommand{\actionRExpand}{
   The selected node, and all its children, and all its children's
   children\ldots\ are opened.  This enables you to see the entire
   contents of a given node.  If performed on a node containing a
   lot of data, e.g.\ a directory containing many HDS files, it
   can result in many nodes being added to the tree which may be slow.
   This option is not available if no node is currently
   selected.
}
\newcommand{\actionRCollapseAll}{
   All the children of all the top-level nodes, and all their children,
   and all their children's children\ldots\ are collapsed.
   This returns the tree state to what it was when the application started
   (apart from any nodes added using the File|Open menu).
   Children of the top-level nodes will be re-read if they are opened
   again, though note the top-level nodes themselves 
   will not.
}
\newcommand{\actionRExpandAll}{
   All the children of all the top-level nodes, and all their children,
   and all their children's children\ldots\ are expanded.
   This means that every node at every level within the top-level nodes
   will be visible.  If some or all of the top-level nodes represent
   large data structures, this can be rather a long job.
   This option should be used with care.
}
\newcommand{\actionHelp}{
   A short help message is displayed in the details display panel.
   This has the side-effect of deselecting any currently
   selected node.
}
\newcommand{\actionSun}{
   This document (SUN/244) is displayed in the details display panel.
   This has the side-effect of deselecting any currently
   selected node.
}

\subsection{Toolbar}

The toolbar provides one-click access to several of the menu options.
The actions provided are as follows\latexhtml{
   {\it (Sorry about the icons in the \LaTeX\ version)}}{}:
\begin{description}
\item[\iconExit] \actionExit
\item[\iconSplitBeside] \actionBeside
\item[\iconSplitBelow] \actionBelow
\item[\iconSplitNone] \actionNone
\item[\iconClose] \actionCollapse
\item[\iconOpen] \actionExpand
\item[\iconExcise] \actionRCollapse
\item[\iconCascade] \actionRExpand
\item[\iconHelp] \actionHelp
\end{description}

\subsection{Menu bar}

The menu bar operates in the usual way, and contains the following menu
items:
\begin{description}
\item[File menu]\mbox{}\\
   \begin{description}
   \item[Open] \actionOpen
   \item[Exit] \actionExit
   \end{description}
\item[View menu]\mbox{}\\
   This menu allows you to choose where the detail display window is displayed,
   if at all.  Only one of the radio buttons can be selected at any one
   time:
   \begin{description}
   \item[Details beside] \actionBeside
   \item[Details below] \actionBelow
   \item[No details] \actionNone
   \end{description}
\item[Tree menu]\mbox{}\\
   \begin{description}
   \item[Collapse selected] \actionCollapse
   \item[Expand selected] \actionExpand
   \item[Recursive collapse selected] \actionRCollapse
   \item[Recursive expand selected] \actionRExpand
   \item[Recursive collapse all] \actionRCollapseAll
   \item[Recursive expand all] \actionRExpandAll
   \end{description}
\item[Help menu]\mbox{}\\
   \begin{description}
   \item[Show help text] \actionHelp
   \item[Show user document] \actionSun
   \end{description}
\end{description}
      
Most of the actions available from the menu are also available from
the toolbar and/or in other ways.

\section{\label{sec:nodetypes}Types of node}

Treeview works by examining files and other data and trying to see
if they appear to be the kind of data it knows about.
It then constructs the most specific type of node it can 
for each file (or whatever it is) that it has encountered.
Below is a list of the types of node which it knows about, 
along with the 
\latexhtml{three-letter code used in the text mode output}
          {icon used in the display},
the {\em Eligibility\/} (type of node which Treeview will have a 
go at analysing to see if it is the node type in question)
and a list of {\em Detail views\/} available in the detail
panel when such a node is selected.
\newcommand{\datanode}[4]{
\item[#1] #2 
\begin{quote}
{\em Eligibility: } #3\\
{\em Detail views: } #4
\end{quote}}
\begin{itemize}
\datanode{\iconNdf}
   {An \xref{NDF}{sun33}{} data structure.}
   {Any HDS object, path name, or file with extension \texttt{.sdf}}
   {Overview, WCS grids for 2-d arrays, Image view for 2-d arrays}
\datanode{\iconWcs}
   {The WCS (World Coordinate System) component of an NDF data structure.
    This is an AST \xref{FrameSet}{sun211}{FrameSet} which contains 
    all the coordinate information concerning its data array.}
   {A 1-d array of strings in the WCS component of an NDF data structure}
   {Overview, Text view, XML view, FITS view}
\datanode{\iconAryOne}
   {A data array.  This may represent either a simple HDS data array, 
    an \xref{ARY}{sun11}{} structure, or an NDArray.
    \latexhtml{}{The icons \iconAryOne, \iconAryTwo and \iconAryThree\
    are used to represent arrays with one, two, or more than two 
    dimensions respectively.}}
   {Any HDS object, path name, or file with extension \texttt{.sdf}}
%     or array component of an HDX object}
   {Overview, Array data, Graph view for 1-d arrays, 
    Image view for 2-d arrays}
\datanode{\iconStructure}
   {An \xref{HDS}{sun92}{} component of some kind.
    \latexhtml{}{The icon \iconAryZero\ is used if the component is scalar
    and hence has no internal structure.}}
   {Any HDS object, path name, or file with extension \texttt{.sdf}}
   {Overview, Array data, Graph view for 1-d arrays, 
    Image view for 2-d arrays}
\datanode{\iconDir}
   {A directory within the filesystem.}
   {Any file}
   {Overview only}
\datanode{\iconFile}
   {A normal file within the filesystem.}
   {Any file}
   {Overview, File text if it looks like a text file}
\datanode{\iconFits}
   {A FITS file.}
   {Files with extensions
    \texttt{.fit}, \texttt{.dst}, \texttt{.fits}, \texttt{.fts}, 
    \texttt{.lilo}, \texttt{.lihi}, \texttt{.silo},
    \texttt{.sihi}, \texttt{.mxlo}, \texttt{.mxhi}, \texttt{.rilo},
    \texttt{.rihi}, \texttt{.vdlo}, or \texttt{.vdhi}}
   {Overview only}
\datanode{\iconHdu}
   {A FITS generic Header plus Data Unit.}
   {HDUs found inside FITS files}
   {Overview, Header cards}
\datanode{\iconTable}
   {A FITS table HDU.}
   {HDUs found inside FITS files}
   {Overview, header cards, table view}
\datanode{\iconAryOne}
   {A FITS image HDU.
    \latexhtml{}{The icons \iconAryOne, \iconAryTwo and \iconAryThree\
    are used to represent arrays with one, two, or more than two
    dimensions respectively.}}
   {HDUs found inside FITS files}
   {Overview, Header cards, Array data, Graph view for 1-d arrays,
    WCS grid for 2-d arrays, Image view for 2-d arrays,
    Slice view for >2-d arrays}
\datanode{\iconSkyframe}
   {An AST \xref{SkyFrame}{sun211}{SkyFrame} object;
    this is a special type of AST Frame.}
   {AST objects found inside the WCS component of NDFs}
   {Overview, Text view, XML view}
\datanode{\iconFrame}
   {An AST \xref{Frame}{sun211}{Frame} object.}
   {AST objects found inside the WCS components of NDFs}
   {Overview, Text view, XML view}
\datanode{\iconZipfile}
   {A ZIP file, which is an archive that can contain, possibly compressed, 
   a number of other files.  Note that a JAR file is a special kind of
   ZIP file -- Treeview makes no distinction.}
   {Files with extensions \texttt{.zip} or \texttt{.jar}}
   {Overview only}
\datanode{\iconZipentry}
   {An entry, which may represent a file or directory,
    in a ZIP or JAR archive.}
   {Items found within a Zip archive}
   {Overview, Entry text if it looks like a text file}
\datanode{\iconHistory}
   {The History component of an NDF}
   {An HDS object with the name HISTORY}
   {Overview only}
\datanode{\iconHistoryRecord}
   {Records within the History component of an NDF}
   {Children of an NDF History component}
   {Overview only}
% \datanode{\iconHdx}
%    {An HDX container object}
%    {XML file}
%    {Overview only}
% \datanode{\iconNdx}
%    {An NDX object}
%    {NDX-type child of an HDX container object}
%    {Overview, Normalised XML}
\datanode{\iconXmlDocument}
   {An well-formed XML document}
   {Any file}
   {Overview, Full text}
\datanode{\iconXmlElement}
   {An element within an XML document}
   {A child of an XML document or element}
   {Overview, Full text}
\datanode{\iconXmlComment}
   {A comment within an XML document}
   {A child of an XML document or element}
   {Overview only}
\datanode{\iconXmlPi}
   {An XML processing instruction}
   {A child of an XML document or element}
   {Overview only}
\datanode{\iconXmlCdata}
   {An XML CDATA marked section}
   {A child of an XML element}
   {Overview only}
\datanode{\iconXmlEref}
   {An XML entity reference}
   {A child of an XML element}
   {Overview only}
\datanode{\iconXmlString}
   {An XML text node}
   {A child of an XML element}
   {Overview only}
\end{itemize}


\section{Text mode}

Treeview can be invoked to display the structure of objects
in a textual form instead of using a graphical interface, 
by using the \texttt{-text} flag.
In this case it opens all the nodes recursively, so that the
structure is fully visible.  This can result in a lot of output
if the object you are examining (e.g.\ a directory containing
many data files) has a lot of strucure.  The result is simply
written to the screen.  In text mode only the tree structure
is written, the data shown in the detail panel is not accessible.

Here is the output you would get from examining a normal NDF in this way.
\begin{quote}
\begin{verbatim}
% treeview -text mosaic
  + [NDF] mosaic  ( -17:154, 2:202 )
    - [HDS] TITLE  <_CHAR*19>  "Output from MAKEMOS"
    + [ARY] DATA_ARRAY  ( -17:154, 2:202 )  <_REAL>
      - [HDS] DATA  ( 172, 201 )  <_REAL>
      - [HDS] ORIGIN  ( 2 )  <_INTEGER>
    + [ARY] VARIANCE  ( -17:154, 2:202 )  <_REAL>
      - [HDS] DATA  ( 172, 201 )  <_REAL>
      - [HDS] ORIGIN  ( 2 )  <_INTEGER>
    + [WCS] WCS  5 frames; current domain "CCD_REG"
      - [FRM] GRID  (2 axes) "2-d coordinate system"
      - [FRM] PIXEL  (2 axes) "2-d coordinate system"
      - [FRM] AXIS  (2 axes) "2-d coordinate system"
      - [FRM] CCD_GEN  (2 axes) "Alignment of CCDGENERATE test data"
      - [FRM] CCD_REG  (2 axes) "Alignment by REGISTER"
    + [HDS] MORE  <EXT>
      - [HDS] FITS  ( 16 )  <_CHAR*80>
      + [HDS] CCDPACK  <CCDPACK_EXT>
        - [HDS] DEBIAS  <_CHAR*24>  "Tue Jan 16 16:52:34 2001"
        - [HDS] FLATCOR  <_CHAR*24>  "Tue Jan 16 16:52:42 2001"
        - [HDS] CURRENT_LIST  <_CHAR*18>  "reduced_data1.off"
\end{verbatim}
\end{quote}

The text mode output of Treeview somewhat resembles that of 
\xref{HDSTRACE}{sun102}{}, but differs in that the elements of
HDS data arrays are not listed at all, while all elements,
rather than just the first one, of HDS structure arrays are listed.

\newpage
\appendix
\section{Reference Section}

% \include{treeview}
\sstroutine{
   treeview
}{
   Display a hierarchical structure graphically or textually
}{
   \sstdescription{
      Treeview is a utility for viewing hierarchical data structures
      such as Starlink NDF and HDS files, XML documents, jar files,
      directory structures, and other items.  In normal use it displays
      a graphical user interface which initially shows a set of top-level
      nodes, each of which can be {\tt '}opened{\tt '} to see the structure inside,
      or {\tt '}selected{\tt '} to display more detail in various formats
      depending on the nature of the item being examined.
      An intuitive point and click interface is used, and detailed
      help is available from within the tool if required.
      If the {\tt '}-text{\tt '} flag is used, a textual view of the object is
      printed to standard output instead.

      Items to be viewed are given on the command line.  These will
      often be filenames, but in general they represent a set of
      strings which the utility will try to turn into {\tt '}nodes{\tt '} -
      it will try one node type after another until it succeeds.
      If it cannot make any kind of node it will fail with an error
      message.  The order of preference in which it tries to
      construct nodes from strings on the command line can
      be altered using flags if required, but by default it is:
      \sstitemlist{

         \sstitem
            NDF structure

         \sstitem
            WCS component of an NDF

         \sstitem
            HDS object

         \sstitem
            XML document

         \sstitem
            zip/jar file

         \sstitem
            FITS file

         \sstitem
            normal file

      }
      The HDS-like nodes can be given as container file names or HDS paths
      (so that the .sdf extension may be included or omitted).

      Flags are provided to modify the order of construction if required.
      When interpreting the commmand line arguments, treeview keeps
      an ordered list of node types, which starts off as above.
      If a flag specifying one of the node types is encountered, the
      corresponding type is brought to the head of the list, and
      if the {\tt '}-strict{\tt '} flag is encountered the list is cleared.
      The order in which the flags and item strings are encountered on the
      line is significant.  In this way it is possible to specify exactly
      what kind of node you would like to make from a given string.
      Modifying the node type preference list is not often necessary,
      but it can be useful in the case of name clashes or to view an
      item of one sort in its aspect as another.
   }
   \sstusage{
      treeview [flags] item [item ...]
   }
   \sstparameters{
      \sstsubsection{
         item [item ...]
      }{
         One or more items must be named on the command line.  These
         will often be filenames, but in general they represent a set
         of strings which the utility will try to turn into {\tt '}nodes{\tt '}
         for display.  If one item is given it will be opened so that
         its children are visible in the GUI, but if more than one
         are given they will initially be displayed closed.
         If any of the items listed does not exist (i.e. cannot be
         made into a node), then treeview will exit with an error
         message.
      }
      \sstsubsection{
         -text
      }{
         If this flag is specified, instead of starting a graphical
         interface, a fully expanded view of the item(s) named on
         the command line is printed to standard output.  Since all
         nodes are recursively expanded, the amount of output may
         be quite large.
      }
      \sstsubsection{
         -split(x$|$y$|$0)
      }{
         These flags control the initial orientation of the tree and
         detail viewing panels in the GUI.  Specifying {\tt '}-splitx{\tt '}
         (the default) causes the tree and detail display panels to
         appear beside each other.  Specifying {\tt '}-splity{\tt '} causes the
         detail panel to appear below the tree panel.  Specifying
         {\tt '}-split0{\tt '} displays only the tree panel.  The orientation can
         be changed while the GUI is running using toolbar buttons.
      }
      \sstsubsection{
         -help
      }{
         Supplying -help (or any unrecognised flag) will display a short
         usage message.
      }
      \sstsubsection{
         -debug
      }{
         This flag is intended for debugging and not recommended for
         general use.  It generates verbose output about attempts made
         to construct nodes from strings.
      }
      \sstsubsection{
         -strict
      }{
         Clears the node type list.  To make sense, it must be followed
         by at least one of the node type specific flags.
      }
      \sstsubsection{
         -ary
      }{
         Adds ARY to the head of the node type list.
         Subsequent items will by preference be turned into
         ARY data structure nodes if possible.
      }
      \sstsubsection{
         -file
      }{
         Adds File to the head of the node type list.
         Subsequent items will by preference be turned into
         ordinary file or directory nodes if possible.
      }
      \sstsubsection{
         -fit
      }{
         Adds FITS to the head of the node type list.
         Subsequent items will by preference be turned into
         FITS file nodes if possible.
      }
      \sstsubsection{
         -ndf
      }{
         Adds NDF to the head of the node type list.
         Subsequent items will by preference be turned into
         NDF structure nodes if possible.
      }
      \sstsubsection{
         -hds
      }{
         Adds HDS to the head of the node type list.
         Subsequent items will by preference be turned into
         HDS object nodes if possible.
      }
      \sstsubsection{
         -wcs
      }{
         Adds WCS to the head of the node type list.
         Subsequent items will by preference be turned into
         World Coordinate System description nodes if possible.
      }
      \sstsubsection{
         -xml
      }{
         Adds XML to the head of the node type list.
         Subsequent items will by preference be turned into
         XML document nodes if possible.
      }
      \sstsubsection{
         -zip
      }{
         Adds Zip to the head of the node type list.
         Subsequent items will by preference be turned into
         ZIP file nodes if possible.
      }
   }
   \sstexamples{
      \sstexamplesubsection{
         treeview $*$.sdf
      }{
         This will bring up a GUI displaying all the HDS data files in
         the current directory.
      }
      \sstexamplesubsection{
         treeview -text ngc1038.more
      }{
         This will write to the screen a fully expanded view of the
         .MORE component (the extensions) of an NDF in the file
         ngc1038.sdf.
      }
      \sstexamplesubsection{
         treeview -split0 jpackage.jar
      }{
         This displays the jar archive {\tt '}jpackage.jar{\tt '} in a window which
         has a tree display but no detail display panel.
      }
      \sstexamplesubsection{
         treeview myndf.wcs -hds myndf.wcs
      }{
         This examines the WCS component of the named NDF twice, first
         in the default way, and then looking at it preferentially as
         an HDS object.  The first node displayed will be as a WCS
         frameset, and the second one will be as the HDS array of
         character data which encodes the WCS information in the NDF
         data structure.
      }
   }
}

\end{document}
