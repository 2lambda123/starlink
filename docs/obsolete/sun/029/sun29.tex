\documentstyle{article}
\pagestyle{myheadings}

%------------------------------------------------------------------------------
\newcommand{\stardoccategory}  {Starlink User Note}
\newcommand{\stardocinitials}  {SUN}
\newcommand{\stardocnumber}    {29.1}
\newcommand{\stardocauthors}   {M J Bly}
\newcommand{\stardocdate}      {28 May 1991}
\newcommand{\stardoctitle}     {NAGRAF --- NAG Graphics Library Mk 3.}
%------------------------------------------------------------------------------

\newcommand{\stardocname}{\stardocinitials /\stardocnumber}
\renewcommand{\_}{{\tt\char'137}}     % re-centres the underscore
\markright{\stardocname}
\setlength{\textwidth}{160mm}
\setlength{\textheight}{240mm}
\setlength{\topmargin}{-5mm}
\setlength{\oddsidemargin}{0mm}
\setlength{\evensidemargin}{0mm}
\setlength{\parindent}{0mm}
\setlength{\parskip}{\medskipamount}
\setlength{\unitlength}{1mm}

%------------------------------------------------------------------------------
% Add any \newcommand or \newenvironment commands here
%------------------------------------------------------------------------------

\begin{document}
\thispagestyle{empty}
SCIENCE \& ENGINEERING RESEARCH COUNCIL \hfill \stardocname\\
RUTHERFORD APPLETON LABORATORY\\
{\large\bf Starlink Project\\}
{\large\bf \stardoccategory\ \stardocnumber}
\begin{flushright}
\stardocauthors\\
\stardocdate
\end{flushright}
\vspace{-4mm}
\rule{\textwidth}{0.5mm}
\vspace{5mm}
\begin{center}
{\Large\bf \stardoctitle}
\end{center}
\vspace{5mm}

%------------------------------------------------------------------------------
%  Add this part if you want a table of contents
%  \setlength{\parskip}{0mm}
%  \tableofcontents
%  \setlength{\parskip}{\medskipamount}
%  \markright{\stardocname}
%------------------------------------------------------------------------------

\section{Introduction}

The NAG Graphics Library is a library of versatile plotting
routines which supercedes the {\em ``NAG Graphical Supplement''\/} which was an
addition to the {\em ``NAG Fortran Library''\/}.
At the introduction of the Fortran Library Mk 14, the Graphical
Supplement was released as a stand-alone Graphics Library, Mk 3.

\section{NAG Graphics on Starlink}

The  NAG Graphics Library Mk 3 has three components:
\begin{itemize}
\item The Graphical Subroutines Library
\item The Graphical Interface Library
\item The Fortran Routines Library
\end{itemize}

\subsection{The Graphical Subroutines Library}

This contains the high-level plotting routines that a
user calls to make use of the NAG Graphics Library.

\subsection{The Graphical Interface Library}

The Graphical Interface Library contains the routines which the high-level
routines call to do the work, and which in turn call the underlying graphics
package. On Starlink, this is GKS.

\subsection{The Fortran Routines Library}

The NAG Graphics Library needs a subset of the full NAG Fortran Library
routines. The Fortran Routines Library contains this subset.

To save disk space at Starlink sites, the Fortran Routines Library has been
omitted, and the full Fortran Library should be used.

\subsection{Precision}

The NAG Libraries (both Fortran Mk 14 and Graphics Mk 3) are installed on
Starlink machines as an Optional Item. If your site does not have the NAG
Libraries, see your Site Manager.

Starlink has both Double Precision and Single Precision versions of the
NAG Graphics Library Mk 3. Each version has its own high-level Graphical
Subroutines Library and  Graphical Interface Library, and each should use the
relevant version of the full NAG Fortran Library.

\subsubsection{Double Precision}

The Double Precision version of the NAG Graphical Subroutines Library is
GRLIB.OLB in directory NAGRAFDIR. For easier reference, the Library has the
logical name NAGRAF\_LIB.

The low level Graphical Interface Library is NAG3GKS.OLB, in the same directory
as the Graphical Subroutines. This Library has the logical name NAG3GKS\_LIB.

The Graphical Subroutines call the Graphics Interface routines at Double
Precision but the calls to the GKS routines are Single Precision.

To link using the Double Precision NAG Graphics Library, use the following:
\begin{verbatim}
      $ LINK <program>,NAGRAF_LIB/L,NAG3GKS_LIB/L,NAG_LIB/L,@GKS_DIR:LINKG
\end{verbatim}
Alternatively, a Link Options file NAGD.OPT, which contains all the NAG Double
Precision Libraries can be used. It has the logical name NAGD, so link using:
\begin{verbatim}
      $ LINK <program>,NAGD/OPT,@GKS_DIR:LINKG
\end{verbatim}

\subsubsection{Single Precision}
The Single Precision version of the NAG Graphical Subroutines Library is
GRSLIB.OLB in directory NAGRAFSDIR. For easier reference, the Library has the
logical name NAGRAFS\_LIB.

The low-level Graphical Interface Library is NAG3GKSS.OLB, in the same directory
as the Graphical subroutines. This Library has the logical name NAG3GKSS\_LIB.

The Graphical Subroutines call the Graphical Interface routines at Single
Precision. Double precision calculations are used internally, but the calls to
the GKS routines are Single Precision.

To link using the Single Precision NAG Graphics Library, use the following:
\begin{verbatim}
      $ LINK <program>,NAGRAFS_LIB/L,NAG3GKSS_LIB/L,NAGS_LIB/L,@GKS_DIR:LINKG
\end{verbatim}
Alternatively, a Link Options file NAGD.OPT, which contains all the NAG Single
Precision Libraries can be used. It has the logical name NAGS, so link using:
\begin{verbatim}
      $ LINK <program>,NAGS/OPT,@GKS_DIR:LINKG
\end{verbatim}

\subsection{GKS Interface}

NAG do not supply a routine to initialize GKS and open a workstation, so users
must make a specific call to GKS {\em ``BEFORE''\/} calling any NAG routines.
A standard example which allows the Interface Test routines and Example
programs (See Section - Information) may be found in NAGRAFDIR:XXXXXX.FOR.

The following code fragment may be used as a guide for initialising the GKS
system. It initialises GKS to use a Canon laser printer.
\begin{verbatim}
      C
            INTEGER BUFA, CONID, ERRFIL, WKID, WTYPE
      C
      C	Use Canon Laser Printer  ( wtype 2600 )
      C
            ERRFIL = 6
            CONID =  0
            WKID = 1
            WTYPE = 2600
      C
      C Open GKS
      C
            CALL GOPKS (ERRFIL, BUFA)
      C
      C Open Workstation
      C
            CALL GOPWK (WKID, CONID, WTYPE)
      C
      C Activate Workstation
      C
            CALL GACWK (WKID)
      C
\end{verbatim}

Details of GKS workstation identifiers supported by Starlink may be found in
SUN/83 : GKS - Graphical Kernal System (7.2)

\section{Information}

There are various elements of information available for the NAG Graphics
Library.

\subsection{Manuals}

Each site should have a copy of {\em ``NAG Graphics Library Mark 3''\/}, a two
volume NAG manual which describes in detail the NAG Graphics Library and its
routines.

\subsection{On-Line Information}

Each version of the NAG Graphics Library has with it lists of which routines
call each routine, and which routines are called by each routine. There is also
a brief summary of each high-level routine, and the NAG Users' note.

For the Double Precision version, the lists are in NAGRAFDIR and are:
\begin{description}
\item[CALLED.DOC] -- A list of all the NAG routines and example programs called
by each NAG routine in the NAG Graphics Library.
\item[CALLS.DOC] -- A list of routines called by each high-level routine in the
NAG Graphics Library, and by the example programs.
\item[SUMMARY.DOC] -- A brief summary of each high-level routine.
\item[UN.DOC] -- Users' Note.
\end{description}

For the Single Precision version, the lists are in NAGRAFSDIR and are:
\begin{description}
\item[CALLEDE.DOC] -- A list of all the NAG routines and example programs called
by each NAG routine in the NAG Graphics Library.
\item[CALLSE.DOC] -- A list of routines called by each high-level routine in the
NAG Graphics Library, and by the example programs.
\item[SUMMARYE.DOC] -- A brief summary of each high-level routine.
\item[UNE.DOC] -- Users' Note.
\end{description}

Users are advised to read the Users Note for the version they are using, and to
note that the names of the various libraries mentioned therein are not the same
as the names used on Starlink. The Text of the Double Precision version of the
Users' Note is appended as Appendix A

\section{Testing}
The Starlink installation of the NAG Graphics Library contains example programs
and data for each version. Each set of programs and data is contained in a
separate text library in the appropriate directory. Example programs and data
may be extracted using the VMS LIBRARY command.

There is also an interface test program for each version, together with its
test data.

The Double Precision versions are in NAGRAFDIR:
\begin{description}
\item[ITESTPGM.FOR] -- Interface test program.
\item[ITESTDAT.DAT] -- Interface test data.
\item[NAG3TESTFOR.TLB] -- Example programs.
\item[NAG3TESTDAT.TLB] -- Example data.
\end{description}

The Single Precision versions are in NAGRAFSDIR:
\begin{description}
\item[ITESTPGME.FOR] -- Interface test program.
\item[ITESTDATE.DAT] -- Interface test data.
\item[NAG3TESTFORS.TLB] -- Example programs.
\item[NAG3TESTDATS.TLB] -- Example data.
\end{description}

The Interface Test program and the example programs require that a routine
called XXXXXX.FOR is supplied which opens an appropriate plotting device in the
normal GKS manner (see above). An example of XXXXXX.FOR is contained in
NAGRAFDIR.

\newpage
\begin{center}
{\bf APPENDIX A}
\end {center}

\begin{verbatim}
                          NAG Graphics Library, Mark 3

                                   GLDVV03D

                          DEC VAX/VMS Double Precision

                                  Users' Note

                                   Contents

     1. Introduction

     2. NAG Graphical Software

     3. General Information

        3.1. Accessing the High-level Library and Interfaces
        3.2. Initialisation of Underlying Plotting Package
        3.3. Output from Graphics Library Routines
        3.4. Interpretation of Bold Italicised Terms
        3.5. Example Programs
        3.6. User Documentation

     4. Routine-specific Information

        4.1. J06AQF
        4.2. Constants

     5. Interface-specific Information

        5.1. GINO-F
        5.2. DEC ReGIS
        5.3. Hewlett Packard HPGL

     6. Additional Services from NAG

     7. Contact with NAG

     8. NAG Users Association
\end{verbatim}
\newpage
\begin{verbatim}
     1. Introduction

        This document is essential reading for every user of the NAG
        Graphics Library Implementation specified in the title. It
        provides implementation-specific detail that augments the
        information provided in the NAG Graphics Library Handbook.

        NAG recommends that users read the following minimum reference
        material before calling any library routine:

        (a) Using the NAG Graphics Library
        (b) Chapter Introduction
        (c) Routine Document
        (d) Implementation-specific Users' Note

        Items (a), (b) and (c) are included in the NAG Graphics Library
        Handbook; item (d) is this document. Each NAG Graphics Library
        Service site is supplied with at least one copy of each of the
        above. Please ask your NAG Site Contact for further information.

     2. NAG Graphical Software

        The NAG Graphics Library is the successor product to the NAG
        Graphical Supplement which, as its name suggests, was originally
        designed for use as a supplementary product to the NAG Fortran
        Library. During recent years it has become increasingly clear that
        Supplement routines are being used in areas far beyond those
        originally envisaged; NAG graphics routines are now being included
        within other products such as Genstat 5, being ported to PC-based
        machines etc. It is with these considerations in mind that NAG
        decided to re-launch the NAG Graphical Supplement as the NAG Graphics
        Library.

        The Graphics Library (which has been labelled as Mark 3 for
        historical consistency) can be considered as an updated, consistent
        enhancement of the Graphical Supplement which can be used
        independently of the NAG Fortran Library. This means that it may be
        licensed separately from the NAG Fortran Library; this allows, for
        example, a site to mount (subject to the appropriate licences)
        NAG graphical routines on a machine which does not have the Fortran
        Library or to use graphical routines with the NAG Workstation
        Library etc. Existing users will not be affected other than
        noticing a change of product name.

        The NAG Graphics Library can be thought of as consisting of three
        distinct parts; high-level graphics routines, low-level routines
        and support routines. When calling the NAG Graphics Library to draw
        a contour map or graph you are invoking what are termed
        'high-level' or package-independent graphics routines. These are
        written in terms of a small set of primitive, or low-level,
        routines known collectively as the NAG Graphical Interface. NAG
        Graphical Interface routines in turn call routines in a locally
        available plotting package, and perform such functions as drawing
        a line or changing the current text font. The support routines are
        a subset of the routines available in the NAG Fortran Library, and
        hence are optional if that Library is available at an installation,
        but are needed otherwise to ensure that the NAG Graphics Library
        is complete.

        There is a version of the NAG Graphical Interface for each plotting
        package (or protocol) supported. The Interface translates drawing
        requests into a form which can be accepted by the underlying
        plotting software. As an appropriate interface is selected at link
        time only one version of an application program need be written and
        maintained. Irrespective of the plotting software used, all routines
        in the NAG Graphical Interface have the same parameter sequence.

     3. General Information

     3.1. Accessing the High-level Library and Interfaces

          A user will need to link his compiled program with the following
          libraries (in the given order):

          - NAG Graphics Library high-level routines (NAG$LIBJ06.OLB)
          - NAG Graphical Interface Library (e.g. TEXTG27.OLB)
          - NAG Fortran support routines (NAG$LIBFOR.OLB or the NAG
            Fortran Library)
          - underlying Graphics package (e.g. GINO-F version 2.7)

          For example, to compile, link and execute the example program
          J06BZF.FOR using data J06BZF.DAT and with the libraries available in the
          same directory (or accessible via logical names), the steps would be:

          - to compile the program, type

            FORTRAN J06BZF

          - to link the program, using the GINO-F 2.7 version of the
            NAG Graphical Interface and the NAG Fortran support routines,
            type

            LINK J06BZF, NAG$LIBJ06/LIB, TEXTG27/LIB, NAG$LIBFOR/LIB, GINO/LIB

          - to link the program, using the GINO-F 2.7 version of the
            NAG Graphical Interface and the full NAG Fortran Library,
            type

            LINK J06BZF, NAG$LIBJ06/LIB, TEXTG27/LIB, NAG$LIBRARY/LIB, GINO/LIB

          - to run the program, type

            ASSIGN/USER_MODE J06BZF.DAT FOR005
            RUN J06BZF

     3.2. Initialisation of Underlying Plotting Package

          Usually the first step when using a graphics package is to
          initialise the system and/or select a particular graphical
          output device. Users must make an explicit call to the local
          graphics software to perform this step, before calling any
          NAG graphical routine.

          Each interface appendix in the NAG Graphics Library Handbook
          gives recommended initialisation statements. However, for
          generality, the example programs contain the dummy initialisation
          statement:

              CALL XXXXXX

          Users who adapt the example programs for their own use must
          replace this statement by an appropriate initialisation call if
          local versions of XXXXXX have not been established at installation
          time.

     3.3. Output from Graphics Library Routines

          It is important to note that a file must always be attached to the
          current error and warning message units (see routines J06VAF and
          J06VBF), even when graphs are the only expected output. This is
          because warning messages may be output by some interface routines
          if they are called with invalid parameters and error messages may
          be output by some high-level routines when an error condition is
          detected. Additionally, if the Lineprinter interface is in use then
          plots are output on the current advisory message unit (see routine
          J06VBF).

          Command sequence output may also be produced by the DEC ReGIS,
          HPGL and Adobe PostScript interfaces (see routine J06VCF). Some
          routines for contouring or the plotting of the solution of ordinary
          differential equations use option-setting facilities. Routine J06VEF
          specifies the unit number of any corresponding option-setting SAVE
          file.

     3.4. Interpretation of Bold Italicised Terms

          For this implementation, the bold italicised terms used in the
          NAG Graphics Library Handbook (and shown here as //.....// )
          should be interpreted as:

          //basic precision//                    - double precision
          //real// as a type declarator          - DOUBLE PRECISION
          //real// as an intrinsic function name - DBLE
          //e// in constants, e.g. 1.0//e//-4    - D, e.g. 1.0D-4
          //e// in formats, e.g. //e//12.4       - D, e.g. D12.4

          Thus a parameter described as //real// should be declared as
          DOUBLE PRECISION in the user's program.

     3.5. Example Programs

          The example programs published in the NAG Graphics Library
          Handbook are presented in a generalised form, using bold
          italicised terms as described in Section 3.4. The programs
          supplied to a site in machine-readable form have been modified
          as necessary to run on this system. They are suitable for
          immediate execution, except for initialisation of the underlying
          plotting package (see Section 3.2). Users of DEC ReGIS, HPGL or
          PostScript interfaces may also need to include a call to J06VCF
          to direct the resultant command sequences to a chosen Fortran
          unit.

     3.6. User Documentation

          The following machine-readable information files are provided by
          NAG on the library release medium. Please consult your local
          advisory service or NAG Site Contact for further details:

              UN      - Users' Note (this document)
              SUMMARY - a brief summary of the routines
              CALLS   - a list of routines called directly or indirectly
                        by each routine in the library, and by each
                        example program
              CALLED  - for each routine in the library (including
                        auxiliaries) this gives a list of those routines
                        and example programs which call it directly or
                        indirectly.

     4. Routine-specific Information

     4.1. J06AQF

          Users are warned that axis annotation produced by J06AQF may
          show variations in tick mark positioning and numbering on
          different systems. This is due to differences in machine
          precision.

     4.2. Constants

          Graphics Library routines make use of the following
          implementation-dependent constants:

          J06VAF GIVES NERR  =     6  (error message unit number)
          J06VBF GIVES NADV  =     6  (advisory message unit number)
          J06VCF GIVES NCOMM =     7  (command sequence unit number)
          J06VEF GIVES NSAV  =     8  (save file unit number)

          X02AJF =   1.3877787807814457E-17  (the machine precision)
          X02AKF =   2.9387358770557188E-39  (the smallest positive model number)
          X02ALF =   1.7014118346046923E+38  (the largest positive model number)
          X02AMF =   5.8774717541114401E-39  (the safe range parameter)
          X02BBF =       2147483647  (largest positive integer)
          X02BEF =               16  (precision in decimal digits)
          X02DAF =                F  (indicates how underflow is handled)

     5. Interface-specific Information

     5.1. GINO-F

          GINO-F uses * (asterisk) as the default string escape character
          to be used in conjunction with the GINO routine CHASTR. The GINO-F
          NAG Graphical Interface resets the escape character to !
          (exclamation). If you wish to change to another escape
          character, e.g. ? (query) then the following statement should
          be included in your program after initialisation with NAG
          routine J06WAF:

              CALL STRESC('?')

          If the version of GINO-F available at your site is older than
          release 2.7c then you should use the alternative routine:

              CALL CHAESC(1H?)

     5.2. DEC ReGIS

          The form and standard use of the ReGIS instruction set does not
          lend itself to easy incorporation into Fortran code. It has,
          therefore, been necessary to develop a set of emulator routines
          for this Interface. This should not affect the user, except
          that the names used for these routines should not be
          duplicated. Details of the 'reserved names' can be found in the
          DEC ReGIS Interface Appendix to the NAG Graphics Library Handbook.

     5.3. Hewlett Packard HPGL


          The form and standard use of the  HPGL instruction set does not
          lend itself to easy incorporation into Fortran code. It has,
          therefore, been necessary to develop a set of emulator routines
          for this Interface. This should not affect the user, except
          that the names used for these routines should not be
          duplicated. Details of the 'reserved names' can be found in the
          Hewlett Packard HPGL Interface Appendix to the NAG Graphics Library
          Handbook.

     6. Additional Services from NAG

        (a) Printed Manuals

            Where a manual has been provided as part of the contract issue,
            this manual is updated automatically at each new release of the
            software, by the supply of a manual update set or a complete new
            manual. If additional manuals have been ordered in the past
            then updates to these manuals may be separately ordered. They
            are NOT sent automatically. Additional complete manuals and the
            manual updates (where relevant) are available at prices published
            in the NAG documentation order form.

        (b) On-line Information System

            To complement the manuals NAG produce an On-line Information
            System which fulfils two roles:

            - it gives key-word-driven guidance on the selection of the
              appropriate NAG routine
            - it gives abbreviated on-line documentation of the NAG routines,
              to enable the user to call the routines and investigate any
              IFAIL messages without recourse to the manual.

        (c) Other Products

            NAG is continually striving to bring the best in mathematical,
            statistical and graphical software to its users. It caters for
            the package user, the software developer and the Fortran, Ada,
            C, Pascal or Algol 68 enthusiast.

            To find out more about the growing range of facilities available
            please contact NAG and ask for a free information pack.

     7. Contact with NAG

        Queries concerning this document or the implementation generally
        should be directed initially to your local Advisory Service.  If
        you have difficulty in making contact locally, you can write to NAG
        directly, at one of the addresses given on the back of the title
        page in the NAG Graphics Library Handbook or this document.

     8. NAG Users Association

        NAGUA is the NAG Users Association, and membership is open to all
        users of NAG products and services.  As a member of NAGUA your
        organisation would receive our newsletter "NAGUA News" three times
        a year, would receive discounts at the annual conference, and those
        sites with access to electronic mail would be able to participate in
        NAGMAG, our electronic mail digest.

        For an information pack and membership application form, contact

           Caroline Foers
           NAGUA Co-ordinator
           NAG Users Association
           PO Box 426
           OXFORD
           United Kingdom   OX2 8SD

           Tel:   (0865) 311102
                  International +44 865 311102

\end{verbatim}
\end{document}
