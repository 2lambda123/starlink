\documentstyle[11pt]{article}
\pagestyle{myheadings}

% -----------------------------------------------------------------------------
% ? Document identification
\newcommand{\stardoccategory}  {Starlink User Note}
\newcommand{\stardocinitials}  {SUN}
\newcommand{\stardocsource}    {sun62.5}
\newcommand{\stardocnumber}    {62.5}
\newcommand{\stardocauthors}   {M.\, Denby \& P.\, McGale}
\newcommand{\stardocdate}      {5 December 1995}
\newcommand{\stardoctitle}     {WFCPACK --- ROSAT Wide Field Camera data 
                                reduction}
\newcommand{\stardocversion}   {Version 6}
\newcommand{\stardocmanual}    {User's Manual}
% ? End of document identification
% -----------------------------------------------------------------------------

\newcommand{\stardocname}{\stardocinitials /\stardocnumber}
\markright{\stardocname}
\setlength{\textwidth}{160mm}
\setlength{\textheight}{230mm}
\setlength{\topmargin}{-2mm}
\setlength{\oddsidemargin}{0mm}
\setlength{\evensidemargin}{0mm}
\setlength{\parindent}{0mm}
\setlength{\parskip}{\medskipamount}
\setlength{\unitlength}{1mm}

% -----------------------------------------------------------------------------
%  Hypertext definitions.
%  ======================
%  These are used by the LaTeX2HTML translator in conjunction with star2html.

%  Comment.sty: version 2.0, 19 June 1992
%  Selectively in/exclude pieces of text.
%
%  Author
%    Victor Eijkhout                                      <eijkhout@cs.utk.edu>
%    Department of Computer Science
%    University Tennessee at Knoxville
%    104 Ayres Hall
%    Knoxville, TN 37996
%    USA

%  Do not remove the %\begin{rawtex} and %\end{rawtex} lines (used by 
%  star2html to signify raw TeX that latex2html cannot process).
%\begin{rawtex}
\makeatletter
\def\makeinnocent#1{\catcode`#1=12 }
\def\csarg#1#2{\expandafter#1\csname#2\endcsname}

\def\ThrowAwayComment#1{\begingroup
    \def\CurrentComment{#1}%
    \let\do\makeinnocent \dospecials
    \makeinnocent\^^L% and whatever other special cases
    \endlinechar`\^^M \catcode`\^^M=12 \xComment}
{\catcode`\^^M=12 \endlinechar=-1 %
 \gdef\xComment#1^^M{\def\test{#1}
      \csarg\ifx{PlainEnd\CurrentComment Test}\test
          \let\html@next\endgroup
      \else \csarg\ifx{LaLaEnd\CurrentComment Test}\test
            \edef\html@next{\endgroup\noexpand\end{\CurrentComment}}
      \else \let\html@next\xComment
      \fi \fi \html@next}
}
\makeatother

\def\includecomment
 #1{\expandafter\def\csname#1\endcsname{}%
    \expandafter\def\csname end#1\endcsname{}}
\def\excludecomment
 #1{\expandafter\def\csname#1\endcsname{\ThrowAwayComment{#1}}%
    {\escapechar=-1\relax
     \csarg\xdef{PlainEnd#1Test}{\string\\end#1}%
     \csarg\xdef{LaLaEnd#1Test}{\string\\end\string\{#1\string\}}%
    }}

%  Define environments that ignore their contents.
\excludecomment{comment}
\excludecomment{rawhtml}
\excludecomment{htmlonly}
%\end{rawtex}

%  Hypertext commands etc. This is a condensed version of the html.sty
%  file supplied with LaTeX2HTML by: Nikos Drakos <nikos@cbl.leeds.ac.uk> &
%  Jelle van Zeijl <jvzeijl@isou17.estec.esa.nl>. The LaTeX2HTML documentation
%  should be consulted about all commands (and the environments defined above)
%  except \xref and \xlabel which are Starlink specific.

\newcommand{\htmladdnormallinkfoot}[2]{#1\footnote{#2}}
\newcommand{\htmladdnormallink}[2]{#1}
\newcommand{\htmladdimg}[1]{}
\newenvironment{latexonly}{}{}
\newcommand{\hyperref}[4]{#2\ref{#4}#3}
\newcommand{\htmlref}[2]{#1}
\newcommand{\htmlimage}[1]{}
\newcommand{\htmladdtonavigation}[1]{}

%  Starlink cross-references and labels.
\newcommand{\xref}[3]{#1}
\newcommand{\xlabel}[1]{}

%  LaTeX2HTML symbol.
\newcommand{\latextohtml}{{\bf LaTeX}{2}{\tt{HTML}}}

%  Define command to re-centre underscore for Latex and leave as normal
%  for HTML (severe problems with \_ in tabbing environments and \_\_
%  generally otherwise).
\newcommand{\latex}[1]{#1}
\newcommand{\setunderscore}{\renewcommand{\_}{{\tt\symbol{95}}}}
\latex{\setunderscore}

%  Redefine the \tableofcontents command. This procrastination is necessary 
%  to stop the automatic creation of a second table of contents page
%  by latex2html.
\newcommand{\latexonlytoc}[0]{\tableofcontents}

% -----------------------------------------------------------------------------
%  Debugging.
%  =========
%  Remove % on the following to debug links in the HTML version using Latex.

% \newcommand{\hotlink}[2]{\fbox{\begin{tabular}[t]{@{}c@{}}#1\\\hline{\footnotesize #2}\end{tabular}}}
% \renewcommand{\htmladdnormallinkfoot}[2]{\hotlink{#1}{#2}}
% \renewcommand{\htmladdnormallink}[2]{\hotlink{#1}{#2}}
% \renewcommand{\hyperref}[4]{\hotlink{#1}{\S\ref{#4}}}
% \renewcommand{\htmlref}[2]{\hotlink{#1}{\S\ref{#2}}}
% \renewcommand{\xref}[3]{\hotlink{#1}{#2 -- #3}}
% -----------------------------------------------------------------------------
% ? Document specific \newcommand or \newenvironment commands.
% ? End of document specific commands
% -----------------------------------------------------------------------------
%  Title Page.
%  ===========
\renewcommand{\thepage}{\roman{page}}
\begin{document}
\thispagestyle{empty}

%  Latex document header.
%  ======================
\begin{latexonly}
   CCLRC / {\sc Rutherford Appleton Laboratory} \hfill {\bf \stardocname}\\
   {\large Particle Physics \& Astronomy Research Council}\\
   {\large Starlink Project\\}
   {\large \stardoccategory\ \stardocnumber}
   \begin{flushright}
   \stardocauthors\\
   \stardocdate
   \end{flushright}
   \vspace{-4mm}
   \rule{\textwidth}{0.5mm}
   \vspace{5mm}
   \begin{center}
   {\Huge\bf  \stardoctitle \\ [2.5ex]}
   {\LARGE\bf \stardocversion \\ [4ex]}
   {\Huge\bf  \stardocmanual}
   \end{center}
   \vspace{5mm}

% ? Heading for abstract if used.
   \vspace{10mm}
   \begin{center}
      {\Large\bf Abstract}
   \end{center}
% ? End of heading for abstract.
\end{latexonly}

%  HTML documentation header.
%  ==========================
\begin{htmlonly}
   \xlabel{}
   \begin{rawhtml} <H1> \end{rawhtml}
      \stardoctitle\\
      \stardocversion\\
      \stardocmanual
   \begin{rawhtml} </H1> \end{rawhtml}

% ? Add picture here if required.
% ? End of picture

   \begin{rawhtml} <P> <I> \end{rawhtml}
   \stardoccategory \stardocnumber \\
   \stardocauthors \\
   \stardocdate
   \begin{rawhtml} </I> </P> <H3> \end{rawhtml}
      \htmladdnormallink{CCLRC}{http://www.cclrc.ac.uk} /
      \htmladdnormallink{Rutherford Appleton Laboratory}
                        {http://www.cclrc.ac.uk/ral} \\
      \htmladdnormallink{Particle Physics \& Astronomy Research Council}
                        {http://www.pparc.ac.uk} \\
   \begin{rawhtml} </H3> <H2> \end{rawhtml}
      \htmladdnormallink{Starlink Project}{http://star-www.rl.ac.uk/}
   \begin{rawhtml} </H2> \end{rawhtml}
   \htmladdnormallink{\htmladdimg{source.gif} Retrieve hardcopy}
      {http://star-www.rl.ac.uk/cgi-bin/hcserver?\stardocsource}\\

%  HTML document table of contents. 
%  ================================
%  Add table of contents header and a navigation button to return to this 
%  point in the document (this should always go before the abstract \section). 
  \label{stardoccontents}
  \begin{rawhtml} 
    <HR>
    <H2>Contents</H2>
  \end{rawhtml}
  \renewcommand{\latexonlytoc}[0]{}
  \htmladdtonavigation{\htmlref{\htmladdimg{contents_motif.gif}}
        {stardoccontents}}

% ? New section for abstract if used.
  \section{\xlabel{abstract}Abstract}
% ? End of new section for abstract
\end{htmlonly}

% -----------------------------------------------------------------------------
% ? Document Abstract. (if used)
%   ==================

The programs described in this document provide the mechanism for
producing ASTERIX (HDS) datasets from Wide Field Camera data collected
during the pointed phase of the ROSAT mission. A sort program generates
ASTERIX datasets ({\em e.g.\/} time series, images, {\em etc}.,) from
pre-processed event data supplied as part of the Rosat WFC Observation
Datasets (RWODs). An exposure program corrects these datasets allowing
for instrument characteristics. In addition a simple database manager
allows an index of RWODs to be maintained and searched.  The programs
make use of the ADAM and ICL environment and a number of ICL procedures
have been provided to perform some of the more commonly required
operations such as extracting background subtracted lightcurves.

WFCPACK is supported under Digital Unix on DEC Alpha AXP, and under
Solaris 2.x on Sun SPARC.  There is no longer a VAX/VMS version.  The
old VAX RWOD which consisted of several VMS format specific files has
been replaced with a single HDS file that is easily transportable
between the various UNIX flavours.

% ? End of document abstract
% -----------------------------------------------------------------------------
% ? Latex document Table of Contents (if used).
%  ===========================================
 \newpage
 \begin{latexonly}
   \setlength{\parskip}{0mm}
   \latexonlytoc
   \setlength{\parskip}{\medskipamount}
   \markright{\stardocname}
 \end{latexonly}
% ? End of Latex document table of contents
% -----------------------------------------------------------------------------
\newpage
\renewcommand{\thepage}{\arabic{page}}
\setcounter{page}{1}

\section{Getting Started}
\label{sec:staring}

Assuming that WFCPACK is installed at your site, it may be started 
by executing the setup commands:

\begin{quote} {\tt
\% aststart \\
\% wfcstart}
\end{quote}

The tasks are callable from the command line by simply typing the 
task name in response to the UNIX prompt, {\em e.g.}:

\begin{quote}{\tt
\% wfcsort}
\end{quote}

Alternatively, in order to make use of the facilities provided by ICL,
the tasks together with a series of ICL procedures may be loaded into
the user's ICL session.  This is achieved by means of the following ICL
command:

\begin{quote}
{\tt ICL\verb+>+ wfcpack}
\end{quote}

\section{The ROSAT WFC Observation Dataset (RWOD)}
\label{sec:rwod}

The RWOD is a set of two data files distributed to the ROSAT WFC observer.
Each RWOD contains data from a single pointed phase observation in the 
form of a tar file.  The software described in this document does not 
require the data to be stored in a predefined directory.  After starting 
ICL, the data on an RWOD may be recovered to disk as follows:

\begin{verbatim}
    ICL>default location                     ! The target directory
    ICL>wfcdisk /dev/rmt0h 141887 location   ! Replace with the tape drive and 
                                             ! the tar file required
\end{verbatim}

The files which are essential for the sorting software to work are:

\begin{description}
\begin{description}
\item [{\tt mcf\_nn.sdf}] The WFC Master Calibration File, version {\tt nn} 
{\em e.g.}, 12. 
\item [{\tt \verb+<+RWOD\verb+>+.sdf}] Events, housekeeping, attitude data file
\end{description}
\end{description}

{\tt mcf\_nn.sdf} defines the time history of the WFC calibration
for the sorting software. {\tt wfcdisk} checks if the tar file contains
a more recent version of the master calibration file than that pointed
to by the environment variable {\tt CAL\_WFC\_MASTER}.  If it does,
check with your system manager regarding updating to the latest
version.

{\tt \verb+<+RWOD\verb+>+.sdf} is an HDS file whose structures contain
data arrays that define:

\begin{enumerate}

\item the pointing and rate of change of the pointing over consecutive
time  intervals,

\item the instrument counters (see Section \ref{sec:programs:wfchk}) at
8 secs. resolution,

\item the satellite location and instrument pointing at 60 second
intervals, and

\item a list of event parameters ordered in both time and spatial senses.

\end{enumerate}

For each event the following values are stored:

\begin{itemize}

\item The telemetered coordinates of the event on the surface of the
detector (in the range 0-511)

\item The linearised coordinates of the event relative to the nominal
optical axis of the instrument (in the range -2.5 to 2.5 deg and to 9
arc sec resolution)

\item The local coordinates of the event relative to the nominal
pointing direction shown in the file header (in the range -2.5 to 2.5
deg and to 6 arc secs resolution)

\item The time tag of the event in seconds relative to the base time
shown in the file header (BASE\_MJD) to 1/32 sec resolution.

\end{itemize}

A summary of the contents of an observation may be obtained by showing
the file header records using the {\tt wfcdbm} program (described in
section \ref{sec:programs:wfcdbm}).  Below is an example of the header
records from a typical RWOD:

\begin{verbatim}
 ROR Sequence #          : B50026    
 Axis Ra                 :  359.38
 Axis Dec                :   20.92
 Roll                    :   -8.50
 POP_FLAGS               : 00000B03
 Target                  : MKN335 (Day 46, ROR 150026 offset)      
 Observer                : WCALIB                                  
 Instrument              : WFC                                     
 Base date               : 16-JUL-90  
 Base MJD                : 48088.8086
 End MJD                 : 48088.8369
 Total Events            :    20879
 File Revision Level     : 5.2 

 Slot        Stime            Etime        Durn.  Flt  Fmt  Zm  Win   HT
        d:hh:mm:ss       d:hh:mm:ss

    1   0:00:02:59       0:00:07:46         288.  UV         z       161
    2   0:00:08:03       0:00:40:51        1968.  P1         z       161
\end{verbatim}

A fixed element of the header defines a number of general observation
parameters:

\begin{itemize}

\item The Base MJD is the (nominal) start time of the observation and
serves as the origin for the event time tags.

\item The Axis centre is the (nominal) observation pointing and serves
as the origin for the event local coordinate frame.

\item The Roll is the angle, in degrees, between Celestial North and
the Wide Field Camera elevation (defined in a positive clockwise sense
when viewing the sky).

\end{itemize}

Note that all ROSAT coordinates are for equinox J2000.0.

The fixed header is followed by information for a variable number of
'slots'. Each time interval during which events were collected and the
instrument configuration ({\em e.g.}, the filter wheel position, HTs)
remained constant defines a slot. (Slots are clipped whenever the angle
between the instrument viewing direction and the zenith exceeds a
preset threshold of 130 degrees.)  For each slot the following
information is displayed:

\begin{itemize}

\item The start and end time in units of days, hours, mins, secs after
Base MJD.

\item The slot duration in secs.

\item The filter configuration ({\em e.g.} S1A, P1, {\em etc.}).

\item The telemetry format. Screening at the pre-processing stage
rejects data except when the telemetry format was in normal or
high-speed modes. When the telemetry is in normal mode this field is
blank, otherwise the {\tt Hi} flag is displayed. The dead-time
characteristics of the data are different in the two modes, there being
a 200 count per second limit in normal mode and a 408 count per second
limit in high-speed mode.

\item The Zoom flag. The WFC has an enhanced spatial resolution mode in
which events from the central quarter of the detector only are
processed and the linear resolution is doubled. The field is blank for
non-zoomed data, otherwise the {\tt z} flag is displayed.

\item The Window flag. Events from pre-set regions of the detector may
have been discarded before insertion in the telemetry stream. When the
window mode is in operation (denoted by a {\tt w} flag in the slot
summary) events from a rectangular area of the detector have been
excluded, leaving events from outside the window unaffected. For
inverted window mode (denoted by an {\tt i} flag in the summary) the
event selection is the reverse of the above ({\em i.e.} events have
been accepted from within the window and rejected from outside).

\item The detector HT voltage setting. Whilst it is not anticipated
that this will change, it should be noted that the instrument
characteristics may change with different HT settings.

\end{itemize}

\section{The WFCPACK Programs}
\label{sec:programs}

The programs involved are:

\begin{description}
\begin{description}
\item [{\tt wfcdbm}]   Data Base Manager
\item [{\tt wfcsort}]  Sorts events into images, time series, {\em etc.}
\item [{\tt wfcexp}]   Exposure corrects a binned data set
\item [{\tt wfcposx}]  Gives exposure at a point in the FOV
\item [{\tt wfchk}]    Sorts HK rates to an HDS container
\item [{\tt wfcslots}] Selects a set of filter defined time slots
\item [{\tt wfcslmrg}] Merges a pair of time slot files
\item [{\tt wpsscorr}] Exposure corrects a PSS source search list
\item [{\tt wfchelp}]  Help information on WFCPACK programs and ICL procedures
\end{description}
\end{description}

These programs are described below in terms of the parameters which
they require from the user.  Each program may be started by typing the
name in response to the ICL prompt together with on--line parameters,
if required, in the KEYWORD= format.

\begin{quote}
{\tt ICL\verb+>+wfcsort RA=270. DEC=66.6 \verb+\\+}
\end{quote}

All programs have context sensitive help which may be obtained by
replying with the {\tt `?'} character to any program prompt.

\subsection{WFCDBM}
\label{sec:programs:wfcdbm}

Provides update and search facilities for all RWODs held at a particular
institute.

\subsubsection{Parameters}

\begin{description}

\item [CMND] 
A number of commands, specified by the CMND parameter, are
available and are specified below:

\begin{verbatim}
     (C)reate      Create an empty Index file
     (U)pdate      Update the index with a new RWOD
     (S)earch      Search for datasets matching various criteria.
     (H)ead        Display the header records from an RWOD.
     (G)roup       Associate a group of slots within an RWOD.
\end{verbatim}

\item[WFCF   ]
The name of the input RWOD. The default extension for the file is {\tt .sdf}.

\item[LISTF   ]
The name of a file to hold listing o/p. The default is TERMINAL.

\item[INDF   ]
The name the Index file which holds information on all the RWODs
in the library. If this file does not already exist it should be created 
using the CREATE option. Thereafter the index file may be 
updated and searched.

\item[SMJD   ]
The start MJD for a search.

\item[EMJD   ]
The end MJD for a search.

\item[RAL    ]
The Right ascension lower limit for a search (decimal degs)

\item[RAH    ]
The Right ascension upper limit for a search (decimal degs)

\item[DECL   ]
The Declination lower limit for a search (decimal degs)

\item[DECH   ]
The Declination upper limit for a search (decimal degs)

\item[PI   ]
Any unique string which identifies the Principal Investigator

\item[TARGET   ]
Any unique string which identifies the target

\item[SELECT1]
The search function is performed on a series of selection criteria
which are entered in response to the repeated SELECTn prompts. The valid 
requests are:

\begin{verbatim}
     (M)jd      select on an MJD window
     (R)a       select on an RA window
     (D)ec      select on a  DEC window
     (P)i       select on a  Principal Investigator string
     (T)arget   select on a  Target string
     (E)xit     combine selection criteria and perform the search
\end{verbatim}

\item[GROUP  ]
The sequence number of a group of slots to be associated together.
(There will be as many sequence numbers for an observation as there are
different filter configurations). The slots are present in the RWOD
event header records and the command groups together slots with the
same filter at the focus. The slot list is returned in the SLOTS
parameter. If the number of groups present in the data is exceeded by
the GROUP command SLOTS is returned as `-999'

\item[SLOTS  ]
A text string which specifies which observation slots have been
associated. The text string is constructed by the GROUP command and may
be used to associate slots with the same filter without a priori
knowledge of the contents of the RWOD event header records.

\end{description}

\subsection{WFCSORT}
\label{sec:programs:wfcsort}

This task is used to sort pre-processed events from the RWOD into 
ASTERIX (HDS) datasets.

\subsubsection{Parameters}

\begin{description}

\item[WFCF]
The name of the input RWOD.  The default extension for the file is {\tt .sdf}.

\item[SHOW]
Requests that a sub-set of the information from the header records
is displayed at the user terminal. In a situation where many
operations are performed on the same RWOD the user may wish to reply
`Y' to the first occurrence of the prompt but `N' to all subsequent
occurrences.

\item[OUTPUT]
The name of the output dataset. The default extension for the file is 
{\tt .sdf} and should NOT be provided by the user.

\item[DTYPE]
The type of dataset to be produced by the sort. Valid types are

\begin{verbatim}
     (I)mage   to create an ASTERIX sky coord image dataset
     (R)aw     to create a detector coord image
     (L)in     to create a linearised detector coord image
     (T)ime    to create a time series dataset
     (E)vent   to create an event dataset.
\end{verbatim}

\item[RA]
The Right Ascension of the sort field centre (decimal degs). By default this
is set to the axis RA value  given in the header.

\item[DEC]
The Declination of the sort field centre (decimal degs). By default this
is set to the axis DEC value in the header.

\item[DAZ]
The half extent of the square field to be sorted (degs), 
the default value is 3 degrees.

\item[INRAD]
In time series mode INRAD specifies the inner radius of an annulus
from which to select events. The radius should be specified in arcminutes
and may be set to zero in order to select from a circular region.

\item[OUTRAD]
In time series mode OUTRAD specifies the outer radius of an annulus
from which to select events. The radius should be specified in arcminutes.

\item[NXPIX]
The number of pixels along a side of the image. For the default field
half extent of 3 degrees NXPIX = 256 would give a pixel width of approx
1.4 arcminutes.

\item[SLOTS]
A text string which specifies which observation slots to include in
the sort. (These are the slots which are displayed when SHOW = Y is selected).
Specify the slots in the format i-j,k,l-m. The slots selected
should all have the same filter at the focus. If a slot value of `0' is
selected the selection mode changes and the user is prompted for the 
filter type via the FILTER parameter.

\item[START\_T]
The start time of a selection window which will be combined with the
slot and discrimination windows. The unit may be either seconds from
the BASE\_MJD given in the RWOD event header or an absolute MJD (in the
latter case the MJD value should be preceded by an `M').  The default
value is the seconds offset of the start of the first selected slot.

It is also possible to input the start and end times from a text file,
with one time pair per record. In this case the time pairs should be
specified in MJDs, each preceded by the `M' character and separated by
a space. When performing background subtraction of two concentric time
series it is important that the two time series be co-binned and so
START\_T should be chosen to be the same for both.

\item[END\_T ]
The end time of a selection window which will be combined with the slot
and discrimination windows. The unit may be either seconds from the
BASE\_MJD given in the RWOD event header or an absolute MJD (in the
latter case the MJD value should be preceded by an `M'). The default
value is the seconds offset of the end of the last selected slot. It is
also possible to input the start and end times from a text file, with
one time pair per record. In this case the time pairs should be
specified in MJDs, each preceded by the `M' character and separated by
a space.

\item[USEF]
The user may optionally specify the time windows from which events
will be selected by means of a simple text file.

\item[SFILE]
The name of a text file from which the time windows will be read.
Each record in the file should contain a pair of start and stop MJDs
separated by a space and with each MJD preceded by the letter `M'.

\item[NBINS]
The number of bins into which the time extent of the sort 
is to be divided. If a value `0' is given the program prompts for the
time duration for each bin (see TBIN).

\item[TBIN]
If NBIN is specified as 0 then TBIN should be set to the required
time duration of a bin (in seconds).

\item[FILTER]
A text string definition of the filter to use. This is selected from
the second column of the following table:

\begin{verbatim}
      MCF #        Filter        Material

      1             UV           UV CAL
      2             S2B          Le/Be
      3             P1           Le/Al
      4             S1B          Le/C
      5             OPQ          Le/Al
      6             S2A          Le/Be
      7             P2           Sn/Al
      8             S1A          Le/C/B4C
 
      (MCF = Master Calibration File)
\end{verbatim}

\item[REJECT]
Wide Field camera data is sometimes contaminated by background features
whose exact nature and cause are not well understood. It is however
possible to improve the quality of badly affected data by choosing to
reject data on a series of constraints. Data may be excluded from a
sort operation by selecting one of the LEVS or VIEW options described
below. Alternatively the NONE option should be selected to retain all
the events.

\item[LEVS]
Rejects data when the limited event rate was above a user defined
threshold user is then prompted for the LEVS threshold to use, a typical
rate which gives noticeable improvement in a large number of cases is
50 c/s.

\item[VIEW]
Rejects data when the Velocity-View (or ram) angle or the solar zenith
angle were below certain limits. The limits are set as hidden defaults
of the program but may be overridden by giving values for the VELVIEW
and/or SOLZEN parameters on the command line (or using the PROMPT
qualifier on the command line).

\end{description}

\subsection{WFCEXP}
\label{sec:programs:wfcexp}

This program takes as input an ASTERIX dataset (time series or image)
whose data array contains detected events. It creates
a dataset whose data array contains event flux (c/s) normalised
to the on-axis response of the instrument. The flux calculation takes
into account the following instrument dependencies:

\begin{itemize}

\item Instrument dead time effects (event rates which are needed for
the calculation are part of the standard observation dataset).

\item Instrument vignetting effects. The Instrument pointing is derived
from the RWOD attitude structure. The vignetting at points within the
field is derived from the filter and detector information stored in the
WFC Calibration File (with the environment variable name {\tt
CAL\_WFC\_MASTER}).

\item Optionally the program may correct the fluxes in a dataset for
the degradation of the detector efficiency. The time history of the
efficiency is stored in the calibration file and is used to determine
an interpolated value at the epoch appropriate to the dataset. The user
is warned if the dataset epoch is beyond the end of the current time
history in the calibration file. In this event the latest value for
detector efficiency is used.

\item Optionally the program may correct the fluxes in a one
dimensional dataset for the fraction of the point response function
which fell outside the source centred circle from which events were
originally accepted by the {\tt wfcsort} program.

\end{itemize}

\subsubsection{Parameters}

\begin{description}

\item[INP]
The name of an input ASTERIX dataset. The dataset may be either a time series
or an image.

\item[OUT]
The name of an output ASTERIX dataset. The dataset is cloned from the input and 
then the raw count array is replaced by an exposure corrected 
count/sec array.

\item[EFFCOR]
Selects correction of fluxes for the degradation of detector efficiency.

\item[PSFCOR]
Selects correction of fluxes in a one-dimensional dataset for the point
spread fraction.

\item[WFCF]
This hidden parameter, {\em i.e.}, user must specify it if
required as it is not prompted for, can be used to specify the {\em e.g.}, 
a time--series' parent RWOD.  ({\tt wfcexp} normally finds the RWOD name
automatically.)

\end{description}

\subsection{WFCPOSX}
\label{sec:programs:wfcposx}

This program takes as input an ASTERIX image dataset as produced by
{\tt wfcsort} and reports exposure information for any location within
the image.  The quoted effective exposure is the value by which a raw
count derived at the location should be divided in order to produce a
count rate corrected to the axis of the instrument at the epoch of the
data set.  In addition the program displays the efficiency of the
detector for the epoch of the data set relative to the launch
efficiency. This relative efficiency may be used to increase a measured
flux to that appropriate to a non-degraded detector. The exposure
calculation works in a similar manner to that in the {\tt wfcexp}
program described above.

\subsubsection{Parameters}

\begin{description}

\item[INP]
The name of an input ASTERIX image dataset.

\item[RA]
The right ascension of the point in the FOV at which exposure information
is required. (decimal degrees).

\item[DEC]
The declination of the point in the FOV at which exposure information
is required. (decimal degrees).

\end{description}

\subsection{WFCHK}
\label{sec:programs:wfchk}

This program extracts the counters present in the housekeeping rates
structure of the RWOD together with the pointing information present in
the aspect structure into a series of arrays in an HDS file. (Note that
for flexibility this file does not obey the ASTERIX standard format.
This means that an extension must be given when giving filenames to
ASTERIX, {\em e.g.}: {\tt gdraw \verb+<+file\verb+>+.TEVS
\verb+<+device\verb+>+}).

The following parameters are stored in the HDS file (in arrays
of the same names):

\begin{description}
\begin{description}
\item [{\tt TEVS}] Total events output by the detector preamplifiers
\item [{\tt VEVS}] Valid events, {\em i.e.}, TEVS within pulse height limits
\item [{\tt LEVS}] Limited events, {\em i.e.}, VEVS within the electronic f.o.v.
\item [{\tt AEVS}] Accepted events, {\em i.e.}, LEVS entered into telemetry queue.
\item [{\tt FEVS}] Final events, {\em i.e.}, AEVS telemetered to ground
\item [{\tt RA}]   Pointing direction RA (degrees)
\item [{\tt DEC}]  Pointing direction DEC (degrees)
\end{description}
\end{description}

The program gives the option of binning directly from the RWODs' HKR
structure ({\em i.e.}, using the file start and end MJDs and intrinsic
bin size of 8 secs) or using the bin structure from an existing time
series dataset.  In this latter mode the user may create arrays of
background counts aligned with an existing bin structure. (BUT note
that the binning mechanism is spot sample at the bin centre {\em i.e.},
summing over the bin duration is not performed.)

\subsubsection{Parameters}

\begin{description}

\item[USE]
Requests that the binning information should be extracted from an
existing time series.

\item[INPUT]
If USE is selected the user is prompted for the name of a time series
dataset.

\item[WFCF]
The name of the input RWOD.

\item[OUTPUT]
The name of the created HDS file.

\end{description}

\subsection{WFCSLOTS}
\label{sec:programs:wfcslots}

This program is used to select time slots with the WFC in a particular
filter configuration. The time slots are written to a text file with
each record containing one pair of start and end times (MJDs). The file
may be merged (using the {tt wfcslmrg} program) with a similar file created
from XRT data and in this way common time slots, when the two instruments
were simultaneously collecting data, may be defined and used to control
event selection in the {\tt wfcsort} program.

\subsubsection{Parameters}

\begin{description}

\item[WFCF]
The name of the input RWOD.  The default extension for the file is {\tt .sdf}.

\item[SHOW]
Requests that a sub-set of the information from the RWOD event header records 
is displayed at the user terminal. In a situation where many
operations are performed on the same event file the user may wish to 
reply `Y' to the first occurrence of the prompt but `N' to all subsequent
occurrences.

\item[SLOTS]
A text string which specifies which observation slots to include in
the sort. (These are the slots which are displayed when SHOW = Y is selected).
Specify the slots in the format i-j,k,l-m. The slots selected
should all have the same filter at the focus. If a slot value of `0' is
selected the selection mode changes and the user is prompted for the 
filter type via the FILTER parameter.

\item[SFILE]
The name of a text file into which the time windows will be written.
Each record in the file contains a pair of start and stop MJDs
separated by a space and with each MJD preceded by the letter `M'.

\item[FILTER]
A text string definition of the filter to use. This is selected from
the second column of the following table:

\begin{verbatim}
      MCF #        Filter        Material

      1             UV           UV CAL
      2             S2B          Le/Be
      3             P1           Le/Al
      4             S1B          Le/C
      5             OPQ          Le/Al
      6             S2A          Le/Be
      7             P2           Sn/Al
      8             S1A          Le/C/B4C
 
      (MCF = Master Calibration File)
\end{verbatim}

\end{description}

\subsection{WFCSLMRG}
\label{sec:programs:wfcslmrg}

The program is used to merge time slot files into a single file of common
time slots. The merged file may then be used to control event selection in 
the {\tt wfcsort} program.

\subsubsection{Parameters}

\begin{description}

\item[SFILE1]
The name of the first time slot file to merge.

\item[SFILE2]
The name of the second time slot file to merge.

\item[SOFILE]
The name of the output merged time slot file.

\end{description}

\subsection{WPSSCORR}
\label{sec:programs:wpsscorr}

Converts raw flux in a source search results file produced by {\tt pss} to 
``on--axis'' counts per second by correcting for vignetting, dead time,
and degradation of detector efficiency.  The source list files can be
examined by using the ASTERIX program {\tt ssdump}.

\subsubsection{Parameters}

\begin{description}

\item[SLIST]
The name of the {\tt pss} source search list file.

\item[CLIST]
The name of the file that will contain the corrected counts.  (The input
file is also copied.)

\end{description}

\subsection{WFCHELP}
\label{sec:programs:wfchelp}

Although help is available for all of the WFCPACK programs by typing
{\tt `?'} in reply to any prompt, {\tt wfchelp} collates this information
into an easy to search package. Details on the supplied ICL procedures
(see Section \ref{sec:procedures}) are also given.

\subsection{An example}
\label{sec:programs:example}

The following example shows how {\tt wfcsort} and {\tt wfcexp} may be
used to create and correct a time series from an RWOD. Parameter
prompts and typical user responses are shown.

\begin{verbatim}
   % aststart                          ! Define necessary UNIX symbols etc.
   % wfcstart			       

   ICL>wfcpack                         ! Load up WFC ICL procedures

   ICL>wfcsort
       WFCF - 015079                   ! Select an RWOD file
       SHOW - Y                        ! Display the header records
       DTYPE - TIME                    ! Select a time series o/p
       RA - 123.45                     ! Set the Ra (decimal degs)
       DEC - 67.89                     ! Set the Dec
       INRAD - 0                       ! Select a circular region on source
       OUTRAD - 6                      ! Circle radius = 6 arcmin
       SLOTS - 0                       ! Select by filter rather than by slots
       FILTER - P1                     ! Set the filter
       NBINS - 0                       ! Select by bin size rather than number
       TBIN - 5.                       ! Set the bin width to 5 secs
       REJECT - NONE                   ! Don't do event discrimination
       OUTPUT - tim_p1                 ! O/p Asterix dataset
 
   ICL>wfcexp
       INP - tim_p1                    ! I/p raw time series
       OUT - tim_p1c                   ! O/p corrected time series

\end{verbatim}

\section{The ICL procedures}
\label{sec:procedures}

The {\tt wfcpack} procedure defines a set of ICL procedures which perform
commonly required series of operations with minimum user interaction.
The procedures described here are:

\begin{description}
\begin{description}
\item [{\tt wfcdisk}]   Load data from an RWOD onto disk
\item [{\tt wfcauto} ]  Standard analysis of a series of RWODs
\item [{\tt wfcbop}]    Maximise signal to noise in producing images
\item [{\tt wfccounts}] Gives counts in an image at a specified position
\item [{\tt wfcimage}]  Auto imaging of the events from an RWOD
\item [{\tt wfclight}]  Produce a light curve at a point in the current image
\item [{\tt wfcspc}]    Produce a WFC spectrum at a point in the current image
\item [{\tt wfcsub}]    Subtracts two contemporary time series
\end{description}
\end{description}

The procedures have been written to exploit the NOCHECKPARS switch in
ICL so that if command line arguments are omitted then they are
prompted for.  The procedures are described below in terms of the
command line arguments which they require (shown in angle
parentheses).

\subsection{WFCDISK $<$DEVICE$>$ $<$TARFILE$>$ $<$TARGET$>$}
\label{sec:procedures:wfcdisk}

The procedure copies an RWOD tar file from tape to the users target
directory.  Each tar file contains a copy of the WFC calibration file
and a check is made of its version number.  The user is informed if it
is more recent than the one currently being pointed to by the {\tt
CAL\_WFC\_MASTER} environment variable.  If it is, see the your system
manager first before overwriting the older master calibration file.

\subsection{WFCAUTO $<$INDEX$>$ $<$RWOD$>$}
\label{sec:procedures:wfcauto}

The procedure produces raw images and exposure maps for the specified
RWOD for each filter configuration stored in the index file
created and managed by the {\tt wfcdbm} program.


\subsection{WFCIMAGE $<$RWOD$>$ $<$FILT$>$ $<$OUTF$>$}
\label{sec:procedures:wfcimage}

The procedure creates a full field, 256 pixel, image of the
data specified by the RWOD argument for the filter specified by the
FILT argument. An ASTERIX image is formed in the file specified
by the OUTF argument. In addition an exposure corrected image
and exposure field are created in files called $<$OUTF$>$\_c
and $<$OUTF$>$\_x respectively.

\subsection{WFCBOP $<$RWOD$>$ $<$FILT$>$ $<$LEVS$>$}
\label{sec:procedures:wfcbop}

The procedure aids optimising the background rate (LEVS) threshold
selection when producing images from the specified RWOD for the filter
FILT. {\tt wfcbop} computes (in a field-averaged sense) the optimum
LEVS threshold (optimum for maximising S:N for a point source, assuming
Gaussian statistics).  The procedure uses {\tt wfcsort} and ASTERIX
routines and creates several temporary, working files, named {\tt
boptemp*.sdf}, on the current default directory. These files are
deleted on successful completion.

\subsection{WFCCOUNTS}
\label{sec:procedures:wfccounts}

This procedure requests the location of source and background boxes from
the currently displayed image.  The source position is input via the
keyboard.  It then returns the number of source counts corrected for
background and instrument psf --- {\tt wfcposx} can be used to
calculate exposure at the source location.  {\tt wfccounts} serves as
a quick check on the significance of a source or on estimating upper
limit values. The ASTERIX program {\tt pss} provides much more secure
methods for source searching (a whole image) and deriving upper
limits.  Both {\tt wfccounts} and {\tt pss} work only on raw images
{\em i.e.}, not corrected for exposure.

\subsection{WFCLIGHT $<$TBIN$>$ $<$OUTF$>$}
\label{sec:procedures:wfclight}

The procedure is intended for interactive use only. It requests the
location of source and background boxes from the currently displayed
image and produces a background subtracted, exposure corrected, light
curve for the source in the file specified by the argument OUTF. The
background can be estimated from either an annulus around the source or
from an offset position. The bin size in seconds is specified via the 
TBIN argument.

\subsection{WFCSPC $<$OUTF$>$ $<$MERGE$>$ $<$XTIM$>$}
\label{sec:procedures:wfcspc}

The procedure is intended for interactive use only. It requests the
location of source and background boxes from the currently displayed
image and produces a background subtracted, exposure corrected,
spectral dataset for the source in the file specified by the argument
OUTF.  The background can be estimated from either an annulus around
the source or from an offset position. The procedure may optionally
arrange to collect data from periods only when the WFC and XRT systems
were collecting data simultaneously. If this option is required the
MERGE option should be set to `YES' and the XTIM argument should
contain the name of a simple text file in which each record contains a
pair of MJDs between which the XRT was collecting data. The MJDs should
be separated by a space and each preceded by the letter `M'.

\subsection{WFCSUB $<$SRCF$>$ $<$BKDF$>$ $<$OUTF$>$}
\label{sec:procedures:wfcsub}

The procedure subtracts a background light curve, specified by the
BKGF argument, from a source light curve, specified by the SRCF
argument. The subtracted light curve is formed in a file specified
by the OUTF argument. The two input light curves must have been
exposure corrected. The fluxes are normalised by the areas of the
two original boxes from which events were sorted, then subtracted.

\section{Known Problems}
\label{sec:problems}

The ASTERIX command {\tt arithmetic}, which is used in many of the
above ICL procedures, can cause problems when it is repeatedly called
from within ICL.  Type: {\tt wfcfix} from the ICL command prompt, if
procedures, or {\tt arithmetic}, are failing to execute properly.


\section{Points of Contact}
\label{sec:contact}

The VMS version of WFCPACK was written by Mike Denby ({\tt md@star.le.ac.uk}).
It was ported to UNIX by Jeremy Ashley ({\tt jka@star.le.ac.uk}) and Paul 
McGale ({\tt pam@star.le.ac.uk}).  A further point of contact is John Pye 
({\tt pye@star.le.ac.uk}).

\end{document}
