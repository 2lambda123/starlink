\documentstyle{article} 
\pagestyle{myheadings}

%------------------------------------------------------------------------------
\newcommand{\stardoccategory}  {ADAM Release Note}
\newcommand{\stardocinitials}  {ARN}
\newcommand{\stardocnumber}    {16.1}
\newcommand{\stardocauthors}   {A J Chipperfield}
\newcommand{\stardocdate}      {1 December 1989}
\newcommand{\stardoctitle}     {ADAM --- Release 1.6}
%------------------------------------------------------------------------------

\newcommand{\stardocname}{\stardocinitials /\stardocnumber}
\markright{\stardocname}
\setlength{\textwidth}{160mm}
\setlength{\textheight}{240mm}
\setlength{\topmargin}{-5mm}
\setlength{\oddsidemargin}{0mm}
\setlength{\evensidemargin}{0mm}
\setlength{\parindent}{0mm}
\setlength{\parskip}{\medskipamount}
\setlength{\unitlength}{1mm}

%------------------------------------------------------------------------------
% Add any \newcommand or \newenvironment commands here
%------------------------------------------------------------------------------

\font\tt=CMTT10 scaled 1095
\renewcommand{\_}{{\tt\char'137}}

\begin{document}
\thispagestyle{empty}
SCIENCE \& ENGINEERING RESEARCH COUNCIL \hfill \stardocname\\
RUTHERFORD APPLETON LABORATORY\\
{\large\bf Starlink Project\\}
{\large\bf \stardoccategory\ \stardocnumber}
\begin{flushright}
\stardocauthors\\
\stardocdate
\end{flushright}
\vspace{-4mm}
\rule{\textwidth}{0.5mm}
\vspace{5mm}
\begin{center}
{\Large\bf \stardoctitle}
\end{center}
\vspace{5mm}
%------------------------------------------------------------------------------
%  Add this part if you want a table of contents
  \setlength{\parskip}{0mm}
  \tableofcontents
  \setlength{\parskip}{\medskipamount}
  \markright{\stardocname}
%------------------------------------------------------------------------------
\newpage
\section{SUMMARY}
This is a complete release of ADAM. 
It includes version 1.5 of ICL, some of the improvements to the parameter 
system which were agreed at the 1989 ADAM Workshop, and a number of bug and 
deficiency fixes.
The release also prepares the way for the switch to the Version 2 Message
System (see Sections \ref{messys}, \ref{msp}, \ref{azuss} and \ref{adamnet}).

It will not generally be necessary to re-link applications for this release but
applications linked with this release will not run with an earlier one.
There was, however, a bug in the DTASK library (see Section \ref{dtask})
which will be corrected by re-linking.

The full release requires about 42000 blocks of disk storage and includes a
mini-release which can be extracted and put up separately.
The mini-release requires about 13100 blocks and allows the system to be run
and tasks to be developed.

\section{INSTALLATION}
Full installation instructions are given in SSN/44 and the Starlink Software 
Change Notice.\\
\begin{tabular}{ll}
{\bf N.B.} & 1. A number of changes have been made to the installation 
instructions in SSN/44.\\
& 2. SYSLOGNAM.COM has been changed, so any site specific versions must
also be changed.\\
\end{tabular}

\section{NEW FEATURES IN THIS RELEASE}

\subsection{ICL}
\subsubsection{Summary}
ICL version 1.5 incorporates a number of changes recommended by the 1989
ADAM workshop. The main changes are summarized below:
\begin{itemize}
\item Real variables are now represented in double precision.
\item SAVEINPUT command saves previous lines of input in a file.
\item `\$' symbol can be used in place of the DCL command.
\item New functions ELEMENT and FILE\_EXISTS.
\item The behaviour of the \verb+~+ continuation character has been 
   tidied up.
\item It is no longer necessary to have a space in expressions of 
   the form keyword=(variable) on the ADAM command line.
\item DIR command, and callable FIGARO have been removed from ICL.
\item New command CREATEGLOBAL creates an ADAM global parameter.
\item Sub-topics on HELP command.
\item Revised login command files.
\item Revised User Guide.
\end{itemize}
See also Section \ref{bugfix} for a list of fixed bugs.

\subsubsection{Details}                                                        
\begin{description}
\item[Real variables now in double precision]
All real variables are now represented in double precision (about 16
decimal digits of precision). 

In order to ensure upward compatibility, the default behaviour when performing
un-formatted conversions of real variables to strings is to convert with six
digits of precision as in the old version. This ensures that the resulting
strings have the same size as previously.

Full precision conversion to strings can be obtained using formatted conversion
(with the colon operator) or by controlling the precision of un-formatted
conversions using a new command SET PRECISION n. Here n is the number of digits
of precision and can range from 1 to 16.

\item[SAVEINPUT command]
This command has the form:
\begin{quote}
\begin{verbatim}
SAVEINPUT n filename
\end{verbatim}
\end{quote}
and causes the last n lines of input to be saved in the file with name
filename. If filename is omitted, the file SAVEINPUT.ICL in the default
directory is used. If n is omitted, the entire contents of the input buffer
are saved.

\item[`\$' symbol in place of DCL command]
The `\$' symbol, if it occurs as the first non-blank character on a line, is
interpreted as equivalent to the DCL command. Thus:
\begin{quote}
\begin{verbatim}
$TYPE SAVEINPUT.ICL
\end{verbatim}
\end{quote}
is equivalent to:
\begin{quote}
\begin{verbatim}
DCL TYPE SAVEINPUT.ICL
\end{verbatim}
\end{quote}

\item[New functions]
Two new functions are available:
\begin{itemize}
\item ELEMENT provides the equivalent of the F\$ELEMENT lexical function in DCL.
\begin{quote}
ELEMENT({\em n,delim,string})
\end{quote}

returns the {\em n}\/th element of {\em string}, where string is divided into 
elements using the delimiter character {\em delim}. 
If the element does not exist, it returns the delimiter character. 
The first element in the string is numbered zero.

For example:
\begin{quote}
\begin{verbatim}
ELEMENT(3,'/','Mon/Tue/Wed/Thu/Fri/Sat/Sun')
\end{verbatim}
\end{quote}
returns `Thu'.

\item FILE\_EXISTS has the form:
\begin{quote}
FILE\_EXISTS({\em filename})
\end{quote}
and is a logical function returning TRUE if the file of name {\em filename}
exists.
\end{itemize}

\item[Continuations]
The behaviour of the \verb+~+ continuation character has been improved. 
The character may be used in direct input as well as in files. 
Output files and LIST output of procedures, will now appear formatted as they 
were input, including continuations.

There is a maximum of 10 continuations on a single statement, as well as a
maximum statement length of 300 characters.
Note that ADAM inter-task communications impose tighter restrictions on
task parameter string lengths.

\item[Formatting of keyword=(variable)]
On an ADAM command line it is no longer necessary to have a space after the
equal sign on expressions of the form keyword=(variable).

\item[Removal of DIR and callable FIGARO]
The DIR command has been removed from the language. It was something of an
anomaly as it was the only DCL command provided in this way. The DIR command
was exactly equivalent to \$DIR or DCL DIR, and can therefore be provided
if required by including the line:
\begin{quote}
\begin{verbatim}
DEFSTRING DIR DCL DIR
\end{verbatim}
\end{quote}
in the LOGIN.ICL file.

Previous versions of ICL included direct access to callable Figaro. This has
now been removed. Equivalent behaviour can be obtained by defining callable
Figaro commands using the DEFUSER facility in the form.
\begin{quote}
\begin{verbatim}
DEFUSER SPLOT BIGFIG FIGARO
\end{verbatim}
\end{quote}
where SPLOT is the command being defined, BIGFIG is the logical name for the
callable Figaro shareable image, and FIGARO is the subroutine to be called
within that image.

\item[CREATEGLOBAL command]
This command creates an ADAM global parameter of a specified primitive type.
It has the form.
\begin{quote}
\begin{verbatim}
CREATEGLOBAL name type
\end{verbatim}
\end{quote}
where name is the name of the global parameter to be created, and type
is one of the HDS primitive type names (\_REAL, \_INTEGER, \_LOGICAL, 
\_DOUBLE or \_CHAR*n). Any previous parameter of the same name is deleted.

\item[Sub-topics on HELP]
The HELP command will now accept sub-topics.
For example:
\begin{quote}
\begin{verbatim}
ICL> HELP SCAR GETTING_STARTED
\end{verbatim}
\end{quote}

\item[Login Command Files]
The `system' ICL login command files, ICLDIR:LOGIN.ICL and ADAM\-LOG\-IN\-.ICL
have been modified to implement the Starlink applications package organization
described in SSN/64.
\item[User Guide]
See Section \ref{docs}.
\end{description}

\subsection{Parameter System -- SUBPAR and LEX}
\subsubsection{Summary}
A number of improvements have been implemented. Many of them were agreed at
the 1989 ADAM Workshop.
The changes are:
\begin{itemize}
\item Interface file search path.
\item Efficient monolith initialization.
\item Consistency in saving `current' values. 
\item RESET affects vpath.
\item Revised action for CANCELLED parameters.
\item Sensible behaviour in the event of unallocated parameter positions.
\item Improved formatting of prompt values.
\item Tabs treated as spaces.
\item Simplified code for saving GLOBAL values.
\item Symbolic names used in setting up the LEX decision table.
\end{itemize}

\subsubsection{Details}
\begin{description}
\item[Interface File Search Path]
The interface file search path has been implemented as described in the
ADAM\_DEVELOPMENT VAX Notes conference.
Basically, if a JOB logical name search path, ADAM\_IFL, is defined, it will be
used to find the interface file.
If ADAM\_IFL is not defined, or an interface file is not found using it, an
attempt will be made to find one in the directory in which the task executable
image was found.
The following subroutines were modified to implement the change:
\begin{quote}
SUBPAR\_ACTIV\\
SUBPAR\_ACTDCL\\
SUBPAR\_ACTSHR\\
SUBPAR\_FINDIF (New subroutine)
\end{quote}
and an additional SUBPAR Status value has been defined:
\begin{quote}
SUBPAR\_\_IFNF - failed to find interface module
\end{quote}
\item[Monolith Initialization]
Monoliths no longer create parameter storage components in their .SDF files
for all their tasks when they are loaded but only as each task is activated.
The modification was developed by Keith Shortridge.
The following subroutines were modified to implement the change:
\begin{quote}
SUBPAR\_ACTIV\\
SUBPAR\_ACTDCL\\
SUBPAR\_ACTSHR\\
SUBPAR\_FINDACT
\end{quote}

\item[`Current' Values]
Names supplied as parameter values will now become `current' immediately and
not only if the parameter is still active on task deactivation.
This results in the same behaviour for parameters whether they are specified
as primitive values or as names.
One disadvantage is that names which are unsuitable for one reason or another
may become current values and be offered as prompt values on re-tries.

The following subroutines were modified to implement the change:
\begin{quote}
SUBPAR\_DEACT\\
SUBPAR\_GETHDS\\
SUBPAR\_GETNAME\\
SUBPAR\_CURSAV (New subroutine)
\end{quote}

\item[RESET affects vpath]
The special keyword RESET will now cause `current' to be ignored on vpath as
well as ppath fields in the interface file.

The following subroutines were modified to implement the change:
\begin{quote}
SUBPAR\_FINDHDS\\
SUBPAR\_GETNAME
\end{quote}

\item[Action on CANCELLED]
If a parameter is in the CANCELLED state, any attempt to obtain a parameter
value will totally ignore the vpath and prompt for the value.
This will result in modified behaviour in the case of vpath `current' or
`noprompt'.

The following subroutines were modified to implement the change:
\begin{quote}
SUBPAR\_FINDHDS\\
SUBPAR\_GETNAME
\end{quote}

\item[Simplified Saving of GLOBALS]
DAT\_ERASE and DAT\_COPY have been used to simplify the code for making
global associations at task deactivation.

The following subroutine has been modified:
\begin{quote}
SUBPAR\_DEACT
\end{quote}

\item[Unallocated Parameter Positions]
A fix in this release means that if parameter position {\em n} is unallocated,
the {\em n}\/th non-keyword parameter on the command line will be ignored.
(An attempt to provide this fix in the Release 1.5 contained an error, so it
was withdrawn.)

Note that this situation should not really be allowed (see Section \ref{bugs}).

The following subroutine was modified to implement the change:
\begin{quote}
SUBPAR\_CMDLINE
\end{quote}

\item[Prompt Value Formats]
The use of the new CHR library and HDS type conversion may result in changed 
formats for numerical prompt values.

The following subroutine has been modified to use HDS type conversion:
\begin{quote}
SUBPAR\_HDSASS
\end{quote}

\item[Tabs]
Tabs are now treated as `space' on command lines.
The following subroutine has been modified to implement the change:
\begin{quote}
LEX\_CMDLINE
\end{quote}

\item[LEX Constants]
LEX\_CMDLINE has been modified to use the LEX symbolic constants defined in
LEX\_DIR:LEX.PAR.
Symbol LEX\_\_NULL has been added to LEX.PAR.
\end{description}

\subsection{Error Include Files}
All `include' files defining symbolic names for error status values have been 
re-generated using the latest version of the ERRGEN utility to produce standard 
Fortran 77 code.
There should be no change in the error values generated.

\subsection{SYSLOGNAM.COM}
\label{syslog}
Changes have been made to SYSLOGNAM.COM as follows:
\begin{itemize}
\item Logical names ADAM\_PACKAGES, LADAM\_PACKAGES and ADAMLOGIN are defined
(see SSN/44 and SSN/64).
\item Logical names MSP\_SHR and AZUSS are defined.
\item Shared images MSP\_SHR and AZUSS are installed.
\item The commented out section re-defining LNM\$TEMPORARY\_MAILBOX has been
removed.
\end{itemize}

\subsection{ADAMSTART}
\begin{description}
\item[ADAM Version Number]
ADAMSTART will now display the ADAM Version number as a definitive check on
which release is in use at a site.
This implies that a new version of ADAMSTART will be released with every
release of ADAM, whether complete or partial.

\item[User-defined ADAM\_USER]
To allow users to define their own preferred location for parameter files etc.,
ADAM\-START will only create the [.ADAM] directory and define {\em job} logical 
name ADAM\-\_USER if ADAM\-\_USER is not already defined as a {\em job} logical 
name.
If ADAM\_USER is so defined, the assigned directory is assumed to exist.

The existence of GLOBAL.SDF in ADAM\_USER will still be ensured in both cases.

This scheme also raises the possibility of `sets' of task and global parameter
values; something which has been seen as desirable.

\item[Prevention of Repeated ADAMSTART]
To enable users to prevent the undoing of changes to symbols and logical names
etc. made after ADAM\-START is obeyed, a logical name, ADAM\$\_INITIAL\-IZED,
will be tested on entry to ADAMSTART.
If its equivalence name is `TRUE', then ADAM\-START will do nothing, apart from
displaying the ADAM version number; otherwise it will proceed.

On exit from ADAMSTART, if the equivalence name of ADAM\$\-\_INITIAL\-IZED is
`FALSE', a {\em process} logical name ADAM\$\-\_INITIALIZED will be DEFINED to
be `TRUE'; otherwise it will not be affected.

Thus if ADAM\$\-\_INITIAL\-IZED is undefined, ADAMSTART will proceed as now.

If it is required that ADAMSTART will operate once only, 
ADAM\$\-\_INITIAL\-IZED must be defined as `FALSE' before ADAMSTART is obeyed.
This process could be initiated by a {\em system} logical name permanently set 
to `FALSE', it would continue with the {\em process} logical name.

\item[ADAM\_INC Defined]
The definition of logical name ADAM\_INC has been moved from ADAMDEV to
ADAMSTART so that the SMS control table include files can be accessed without
obeying ADAMDEV (See Section \ref{smssystab}).
\end{description}

\subsection{Graphics}
\begin{description}
\item[GNS]
The transfer vector of the ADAM graphics shared image now includes the GNS
Graphics Workstation Naming Service subroutines (see SUN/57).
The latest versions of SGS and GKS, which use GNS are also incorporated but,
for the time being, the Device Dataset method of obtaining device names will
be retained for ADAM.
It is expected to be withdrawn soon.

\item[SGS]
\begin{itemize}
\item The way in which ADAM tells SGS to use the inherited status mechanism 
has been revised.
The following subroutines have been modified to implement the change:
\begin{quote}
SGS\_\$HSTAT (Removed from SGSPAR)\\
SGS\_\$ISTAT\\
SGS\_ASSOC
\end{quote}
\item The operation of SGS\_ASSOC has been improved in that:
\begin{enumerate}
\item The ACMODE parameter is now checked fully.
\item The ACMODE parameter is now checked outside the loop attempting to open
the device.
\end{enumerate}
\item Additional possible errors have been added to SGS\_\$ERR and SGSERR.MSG
to bring them in to line with the Starlink release of SGS, particularly with
the inclusion of GNS.
A catch-all error, SGS\_\_SGSER has also been added.
\item SGS\_\$ERR will now stack error reports using ERR\_REP rather than 
outputting them immediately with ERR\_OUT.
\end{itemize}

\item[Linking]
The graphics shared image has been re-linked to include the latest versions
of SGS and GKS which now use the GNS system.
The GNS subroutines have been included in the transfer vector.

Linking with the graphics object libraries is now done according to documented
Starlink method, i.e. using @SGS\_DIR:SGSLINK.COM. 
This is true for both building the shared image and the `noshr' task linking.

The following files have been modified to implement the change:
\begin{quote}
SHARE\_DIR:ADAMGRAPH7.MAR\\
SHARE\_DIR:ADAMGRAPH7.COM\\
LIB\_DIR:LINKNOSHR.OPT\\
A\_DIR:ANOSHR.COM\\
A\_DIR:MNOSHR.COM\\
D\_DIR:DNOSHR.COM\\
D\_DIR:CDNOSHR.COM
\end{quote}
\end{description}

\subsection{CHR}
The shared image is now built using the Starlink release of the CHR character
handling routines (see SUN/40) and the ADAM CHR directory has been removed.
This may result in some format changes for displayed numbers.

Subroutines CHR\_STRDEC and CHR\_LDBLNK are still included in the transfer
vector but will be withdrawn at the next release.

\subsection{FIO}
\label{fio}
A new subroutine FIO\_READF has been written; it reads a block from the tape 
but does not return the used length of the buffer.
This makes reading much faster when the used length is not required.
FIO\_READ has been implemented using the new routine.
This is a technique used by the SCAR version of FIO.
It is hoped that SCAR will now be modified to use the ADAM version of FIO;
however, the name of the new routine has been changed, from FIO\_\$READ, to
meet Starlink standards.

Subroutines FIO\_\$CHKFD, FIO\_GUNIT and FIO\_PUNIT have been tidied up.

\subsection{MESSYS}
\label{messys}
Because of a recently discovered problem with the Version 2 inter-process 
message system, MESSYS, which uses Jeremy Bailey's Message System Primitive 
system, MSP, rather than mailboxes, the switch has not yet been made.
However, the source of the  new system has been included in sub-directory 
MESSYSV2 of LIB\_DIR and sub-directory NEWSHARE contains the procedures 
necessary for creating the ADAM Version 2 shared images.
Furthermore, the version of MESSYSMSG copied to the mini-release directories 
is actually the new version but will do for either.
It is hoped that this will encourage observatories in particular to exercise
the new system.

If it is required to communicate between processes under different usernames
in a group using the new system, GROUP privilege is required and the logical 
name GBL\$MSP\_username must be defined to be the common global section name.
Processes not in the same group cannot communicate with each other.

The hoped-for improvement in speed has not materialized but the new system 
opens the way to a number of other improvements in ADAM.
The new message system is fully described in ASN/2.

\subsection{MSP}
\label{msp}
Version 2.1 of MSP is included.
This includes improvements arising from a code walkthrough held at ROE in June.

The shared image MSP\_SHR is also INSTALLed by SYS\-LOGNAM\-.COM; this will
ease the conversion to MESSYS V2.
A document describing MSP may be found in directory MSP\_DIR of the full
release.

\subsection{AZUSS}
\label{azuss}
AZUSS is a shared image of User System Services which is required by MSP.
It was written by Graham Bothwell at the AAO and revised by Lew Waller to 
improve security.
AZUSS must be INSTALLed; this is done by SYSLOGNAM.COM.
A document describing the AZUSS routines may be found in directory AZUSS\_DIR
of the full release.

\subsection{ADAMNET}
\label{adamnet}
The ADAM Networking Task, ADAMNET, is released in ANT\_DIR.
It requires the Version 2 MESSYS and will not be run as part of a standard 
Starlink installation anyway until various security problems are resolved.
ADAMNET is described in ASN/2.

\subsection{SMS}
\label{smssystab}
\begin{description}
\item[SMSSYSTAB]
The SMS control table include file, SMSSYSTAB.INC, which defines the `system'
components, has been copied to the ADAM\_INC directory.
It can be included into SMS control tables using the `include file' facility.
A modified version, SMSICL\-SYSTAB\-.INC, is also provided for use with SMSICL
control tables; this version does not provide help.
SMSSYSTAB.SCT, previously held in directory ADAM\_TEST, has been withdrawn;
there are some differences.

\item[Demonstration Control Tables]
Control tables AON004\-.SCT and AON004ICL\-.SCT may be found in directory
ADAM\_TEST (defined after `\$ ADAMDEV' has been obeyed).
AON004\-.SCT is the example given in AON/4 and AON004ICL\-.SCT is the same 
example modified for use with SMSICL.
To demonstrate SMS, type:
\begin{quote}
\begin{verbatim}
$ ADAMSTART
$ ADAMDEV
$ DEFINE SMSTABLE ADAM_TEST:AON004.SCT
$ DEFINE AON004 ADAM_TEST:AON004
$ SMS
\end{verbatim}
\end{quote}
or, to demonstrate SMSICL, type:
\begin{quote}
\begin{verbatim}
$ ADAMSTART
$ ADAMDEV
$ DEFINE SMSTABLE ADAM_TEST:AON004ICL.SCT
$ SMSICL
\end{verbatim}
\end{quote}
The demonstration can then proceed using the keystrokes defined in AON/4.

These demonstration tables replace TESTTAB.SCT which was formerly in
ADAM\-\_TEST
\end{description}

\subsection{Documentation}
\label{docs}
SSN/45.8 and ARN/16.1 describe ADAM release 1.6.
Note that both are now \LaTeX\ documents. 
The ARNs will be retained for historical purposes as SSN/45 gets replaced at
each new release.

The following documents have been re-issued:
\begin{description}
\item[SSN/44.6] {\it ADAM --- Installation Guide} -- reflects the change to MSP 
for inter-task communication and the revised use of the ICL login command file
feature.
\item[APN/9.4] {\it ADAM Programmer's Guide to the FIO Package} -- Describes the
new routine FIO\_READF (see Section \ref{fio}).
\item[AED/14.3] {\it Using the MSG and ERR Systems} -- Improved subroutine
descriptions; MSG\_MARK and MSG\_RLSE added.
\item[ICL User's Guide] -- Additional functions and commands; minor 
clarifications and corrections.
The document source, ICL.TEX and ICL.TOC can be found in ICLDIR or hardcopies
obtained from The Starlink Software Librarian.
\end{description}

The following document has been released in the Starlink System Note series:
\begin{description}
\item[SSN/64.1] {\it ADAM --- Organization of Applications Packages} --
describes the way in which Starlink will set up ADAM applications packages.
\end{description}

The summaries, ADAM\_DOCS:CONTENTS.LIS, FULLDOCS.LIS and NEWDOCS.LIS, have been 
updated. 
Related documents, not classified as ADAM documents have been included in 
FULLDOCS and NEWDOCS.

\section{BUGS FIXED}
\label{bugfix}
\subsection{ICL}
\begin{itemize}
\item In ICL/SMS a cached load of a task caused by a `from' 
   statement in an SMS switch field did not work properly. 
   This has now been fixed.
\item A double space preceding the parameter part of an ADAM 
   command caused a leading space to be passed as part of the 
   parameter string. Leading spaces are now removed in these 
   cases.
\item Files created by the SAVE operation were created with a 
   maximum record size of 133 bytes. This has been increased 
   to allow for the maximum ICL statement length of 300 
   characters.
\item Process names longer than 15 characters are truncated. 
   Previously an ADAM task would fail to load if its process 
   name was too long.
\item DEFHELP will now work properly with more than one key name.
   (e.g. DEFHELP {\em name helplib key1 key2 key3}\/)
\item DEFUSER, DEFSTRING, DEFPROC and DEFHELP now no longer output
   a `command redefined' message if the redefinition is the same 
   as the original definition. Previously only DEFINE had this 
   behaviour.
\item A bug, which caused the ICL command KILLW to hang if the task had
  crashed, has been fixed.
\item A bug, which caused STKOVFLOW error while processing loops within
  loops, has been fixed.
\item A bug, which caused ICL to hang if the \$ICL command specified
  a command file which contained EXIT following a task run, has been fixed.
\item A bug, which caused corruption of a procedure argument if it was used
  to pass a parameter value to a task, has been fixed.
\item ICL variables no longer need to be defined before being used to receive
  an output parameter value from a task.
\item A bug, which caused error condition LOGZERNEG if the second argument
  of the SNAME function was zero, has been fixed.
\item A number of typing errors in the ICL help file have been corrected.
\end{itemize}

\subsection{Parameter System -- SUBPAR and LEX}
\begin{itemize}
\item Shared image monoliths are now correctly re-activated. (SUBPAR\_ACTSHR)
\item The parameter system will no longer report an error if $<$return$>$ is 
hit in response to a prompt and there is no prompt value; it will just 
re-prompt. (SUBPAR\_PROMPT)
\item A bug, which caused problems when updating global parameters if the global
file was corrupted, has been corrected. (SUBPAR\_DEACT)
\item Special characters are now allowed in literals in arrays. (LEX\_CMDSET)
\end{itemize}

\subsection{MSG}
\begin{itemize}
\item A bug, which caused an attempt at recursion if `\$' was found in a token
value, has been corrected. (MSG\_FORM)
\end{itemize}

\subsection{SMS}
\begin{itemize}
\item The cursor is now restored on message output. (SMS\_GETUSRKEY and
SMS\_GETUSRLIN)
\item The keystroke reading has been re-implemented using QIOs instead of SMG
to overcome the problem of incomplete escape sequences which has appeared
with VMS version 5.
The SMG version will be saved and restored if the problem is fixed.

The following routines have changed:
\begin{quote}
SMS\_ASTHDLR\\
SMS\_SACTV\\
SMS\_KRESTART\\
SMS\_SDEAC\\
SMS\_SETBUFF\\
SMS\_CLRBUFF\\
SMS\_DODCL\\
SMS\_SMSESCIN (New routine)\\
SMS\_READ\_KEYSTROKE (New routine)\\
SMSCOMAST  (New common block)
\end{quote}
\end{itemize}

\subsection{UFACE}
\begin{itemize}
\item PROCESS logical names which form part of the executable image name will
be translated before calling SYS\$CREPRC with the image name. 
This corrects an unfortunate side effect of the modification in ADAM version
1.5 to allow privileged tasks to be run.
The UFACE libraries in UFACE\_DIR, SMSUFACE\_DIR and ICL\_DIR have been 
updated. (UFACE\_LOAD)
\end{itemize}

\subsection{DTASK}
\label{dtask}
\begin{itemize}
\item A bug, which caused a Dtask which does some rescheduling and which gets
sent more than 32767 OBEY commands to crash with a stack dump indicating integer
overflow in DTASK\_RESCHED, has been corrected. (DTCOMMON, DTASK\_OBEY)
\end{itemize}

\section{CLEANUP}
\subsection{Items Removed}
\begin{description}
\item[CHR]         Replaced by Starlink stand-alone library
\item[ENGINT]      Obsolete
\item[TEL]         UKIRT specific and obsolete
\item[CAMAC]       Obsolete
\end{description}

\subsection{Candidates for Removal}
It is proposed to remove the following items in the near future:
\begin{description}
\item[ADAMCL] This has almost entirely been replaced by ICL and is no longer
properly supported.
It is therefore proposed that ADAMCL be withdrawn at the next release.
CLITASK (ADAMCL running as a task) will be withdrawn at the same time as will
command language monoliths.

\item[MESDEFNS] This `include' file was part of the old MESSYS but is not part
of the new one and should therefore go when Version 2 comes into use.
However, it is thought that it may be used by some tasks to define INFINITE
and EF\_APP01 etc.
Can anyone comment?
\end{description}


\section{KNOWN BUGS}
\label{bugs}
The following bugs have been reported but have not yet been fixed:
\begin{itemize}
\item Arrays in interface files - Arrays specified as defaults in interface 
files must be enclosed in parentheses.
Square brackets (the normal parameter system syntax) are not accepted.
\item Parameter POSITION Fields - AED/3 says that every parameter should be 
allocated a position but the interface file parsing permits POSITION fields to 
be omitted.  
It should not be necessary to define a position for every parameter, especially
if there are a large number of them, but it must be a mistake to have
non-contiguous positions allocated; therefore it is proposed to check for this
condition during interface file parsing in a future release.
\item HDS - There is a bug which causes a crash if an attempt is made to 
create a file smaller than the disk cluster size.
CLITASK - Reports \%PAR-I-NOUSR if an attempt is made to use it.
\item There is a bug which causes long error messages to be truncated, and in
some cases to produce garbage at the end.
\end{itemize}
\end{document}
