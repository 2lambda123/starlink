\documentstyle{article} 
\pagestyle{myheadings}

%------------------------------------------------------------------------------
\newcommand{\stardoccategory}  {ADAM Release Note}
\newcommand{\stardocinitials}  {ARN}
\newcommand{\stardocnumber}    {16-1.1}
\newcommand{\stardocauthors}   {A J Chipperfield}
\newcommand{\stardocdate}      {19 January 1990}
\newcommand{\stardoctitle}     {ADAM --- Release 1.6-1}
%------------------------------------------------------------------------------

\newcommand{\stardocname}{\stardocinitials /\stardocnumber}
\markright{\stardocname}
\setlength{\textwidth}{160mm}
\setlength{\textheight}{240mm}
\setlength{\topmargin}{-5mm}
\setlength{\oddsidemargin}{0mm}
\setlength{\evensidemargin}{0mm}
\setlength{\parindent}{0mm}
\setlength{\parskip}{\medskipamount}
\setlength{\unitlength}{1mm}

%------------------------------------------------------------------------------
% Add any \newcommand or \newenvironment commands here
%------------------------------------------------------------------------------

\font\tt=CMTT10 scaled 1095
\renewcommand{\_}{{\tt\char'137}}

\begin{document}
\thispagestyle{empty}
SCIENCE \& ENGINEERING RESEARCH COUNCIL \hfill \stardocname\\
RUTHERFORD APPLETON LABORATORY\\
{\large\bf Starlink Project\\}
{\large\bf \stardoccategory\ \stardocnumber}
\begin{flushright}
\stardocauthors\\
\stardocdate
\end{flushright}
\vspace{-4mm}
\rule{\textwidth}{0.5mm}
\vspace{5mm}
\begin{center}
{\Large\bf \stardoctitle}
\end{center}
\vspace{5mm}
%------------------------------------------------------------------------------
%  Add this part if you want a table of contents
  \setlength{\parskip}{0mm}
  \tableofcontents
  \setlength{\parskip}{\medskipamount}
  \markright{\stardocname}
%------------------------------------------------------------------------------

\section{SUMMARY}
This is a partial release of ADAM.
It consists mainly of bug fixes to the Parameter System, ICL, and the
DTASK library, and some improvements to operational aspects of MAG/MIO.

It will not generally be necessary to re-link applications for this release but
the fix to the DTASK library (see Section \ref{dtask}) will not be incorporated
unless they are re-linked.

The full release requires about 42000 blocks of disk storage and includes a
mini-release which can be extracted and put up separately.
The mini-release requires about 13200 blocks and allows the system to be run
and tasks to be developed.

\section{INSTALLATION}
Full installation instructions are given in SSN/44 and the Starlink Software 
Change Notice.

\section{NEW FEATURES IN THIS RELEASE}

\subsection{MAG and MIO}
\label{mag}
\begin{itemize}
\item The subroutines MAG\_ALOC, MAG\_DEAL, MAG\_MOUNT and MAG\_DISM have been 
re-written to exit with status set in the case of most errors.
Errors which will result in another attempt to obtain a value are:
\begin{itemize}
\item Device not described in a DEVDATASET component.
\item No such physical device on the machine.
\end{itemize}

\item The errors MAG\_\_LCERR (Unable to open DEVDATASET) and MAG\_\_UNKDV
(Unable to locate device in DEVDATASET) are distinguished. (MAG\_\_\$CPTDS)

\item The error messages associated with status values have been improved.
(MAGERR)

\item The severity bits are ignored in converting VMS to ADAM codes. 
(MIO\_\$CODE)

\item Logical names for MIO system development are now defined after obeying
ADAM\_SYS:SYSDEV. (LIB\_DIR:LOGICAL.COM)
\end{itemize}

\subsection{Documentation}
SSN/45.9 and ARN/16-1.1 describe ADAM release 1.6-1.
ARNs will be retained for reference as SSN/45 gets replaced at each new release.

The following documents have been re-issued:

\begin{description}
\item[APN/1.4] {\it ADAM Programmer's Guide to the MAG Package} -- reflects
the changes described in Section \ref{mag}.
\end{description}

The summaries, ADAM\_DOCS:CONTENTS.LIS, FULLDOCS.LIS and NEWDOCS.LIS, have been 
updated. 

\subsection{ADAMSTART}
ADAMSTART.COM is modified to display the latest ADAM version number.

\section{BUGS FIXED}
\subsection{ICL}
ICL Version 1.5-1 is issued with this release.
\begin{itemize}
\item A bug, which caused ICL to fail if a procedure defined by DEFPROC caused
another procedure defined by DEFPROC to be loaded, has been fixed.

\item A bug, which caused incorrect string concatenation if the elements to be
concatenated were numerical variables, has been fixed.

\item A bug, which caused the logical name SYS\$INPUT to be re-defined for the
DCL subprocess after a procedure was run in the subprocess, has been fixed.
(SYS\$INPUT and SYS\$OUTPUT are now defined prior to each DCL command sent.)

\item Minor errors in the help library have been fixed.

\item Character string parameter truncation -
ICL can now GET the complete 132 characters available
for a task's character string parameter. A corresponding change has been
made to DTASK\_GET in ADAM release 1.6-1.
Previously DTASK\_GET limited the length to 80 characters and ICL to 60.
\end{itemize}

\subsection{Parameter System -- SUBPAR}
\begin{itemize}
\item A bug, which caused the task.SDF not to be updated when a parameter
value was obtained via vpath global, has been fixed. (SUBPAR\-\_HDSASS)

\item A bug, which caused endless prompting if a task run directly from DCL
in batch prompted without a prompt value and was provided with insufficient 
input, has been fixed. A maximum of five prompts will now occur before
returning with PAR\_\_NULL status. (SUBPAR\-\_PROMPT)

\item The used length, rather than the declared length, of the message is
now transmitted. (SUBPAR\-\_WRITE)

\item Several routines have been modified to guard against the HDS `hanging
locators' problem which has been the cause of several obscure bug reports.
(SUBPAR\-\_ASSOC,
SUBPAR\-\_CREAT,
SUBPAR\-\_CRINT,
SUBPAR\-\_CUR\-LOC,
SUBPAR\-\_CUR\-NAME, 
SUBPAR\-\_CUR\-SAV,
SUBPAR\-\_DAT\-FIND,
SUBPAR\-\_GET\-HDS,
SUBPAR\-\_HDS\-ASS,
SUBPAR\-\_HDS\-LOCS,
SUBPAR\-\_HDS\-OPEN,
SUBPAR\-\_NAME\-ASS,
SUBPAR\-\_VAL\-ASS)

\item Some temporary work files are deleted from the directory.

\end{itemize}

\subsection{DTASK}
\label{dtask}
\begin{itemize}
\item DTASK\_GET has been modified to handle a 132 character value.
\end{itemize}

\subsection{LINKNOSHR}
LINKNOSHR.OPT, used by the `noshareable' linking procedures has been 
corrected to remove MSP\_SHR which will not be required until the Version 2 
message system is in use.

\subsection{ADAMNET}
The ADAMNET program has been altered to make the maximum message size
C\-\_MAXMSG\-\_LEN. (ADAMNET.FOR)

\section{CLEANUP}
\subsection{Candidates for Removal}
Since release 1.6, there have been strong representations from JAC to retain
ADAMCL and other associated items as part of the standard release until
they have time to convert their systems to use ICL.
In view of this, it is no longer proposed to withdraw ADAMCL in the near 
future.
However, the level of support for it is likely to be minimal.

\end{document}
