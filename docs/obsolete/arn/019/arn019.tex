\documentstyle{article}
\pagestyle{myheadings}

%------------------------------------------------------------------------------
\newcommand{\stardoccategory}  {ADAM Release Note}
\newcommand{\stardocinitials}  {ARN}
\newcommand{\stardocnumber}    {19-4.1}
\newcommand{\stardocauthors}   {A J Chipperfield}
\newcommand{\stardocdate}      {13 September 1991}
\newcommand{\stardoctitle}     {ADAM --- Release 1.9-4}
%------------------------------------------------------------------------------

\newcommand{\stardocname}{\stardocinitials /\stardocnumber}
\markright{\stardocname}
\setlength{\textwidth}{160mm}
\setlength{\textheight}{240mm}
\setlength{\topmargin}{-5mm}
\setlength{\oddsidemargin}{0mm}
\setlength{\evensidemargin}{0mm}
\setlength{\parindent}{0mm}
\setlength{\parskip}{\medskipamount}
\setlength{\unitlength}{1mm}

%------------------------------------------------------------------------------
% Add any \newcommand or \newenvironment commands here
%------------------------------------------------------------------------------

\font\tt=CMTT10 scaled 1095
\renewcommand{\_}{{\tt\char'137}}

\begin{document}
\thispagestyle{empty}
SCIENCE \& ENGINEERING RESEARCH COUNCIL \hfill \stardocname\\
RUTHERFORD APPLETON LABORATORY\\
{\large\bf Starlink Project\\}
{\large\bf \stardoccategory\ \stardocnumber}
\begin{flushright}
\stardocauthors\\
\stardocdate
\end{flushright}
\vspace{-4mm}
\rule{\textwidth}{0.5mm}
\vspace{5mm}
\begin{center}
{\Large\bf \stardoctitle}
\end{center}
\vspace{5mm}
%------------------------------------------------------------------------------
%  Add this part if you want a table of contents
%  \setlength{\parskip}{0mm}
%  \tableofcontents
%  \setlength{\parskip}{\medskipamount}
%  \markright{\stardocname}
%------------------------------------------------------------------------------

\section{SUMMARY}
This is a partial release of ADAM to update Version 1.9-3.
Full installation instructions are given in the Starlink Software
Change Notice.

The purpose of the release is to change references to the setup procedures
for HDS and REF in line with the reorganization of those libraries.
A new procedure is added to ease the task of starting up an ADAM session.

The opportunity is also taken to update ICL and SMS with some enhancements
and bug fixes, mainly provided by Chris Mayer from JACH.

It will not be necessary to re-link tasks for this release.

\section{NEW FEATURES IN THIS RELEASE}

\subsection{ADAMSTART}
ADAMSTART.COM is modified to:
\begin{itemize}
\item Display the latest ADAM version number.
\item Set a PROCESS logical name ADAM\$\_INITDONE to `TRUE'. This may be
tested by other procedures to determine whether or not ADAMSTART has
already been obeyed.
\end{itemize}

\subsection{ADAMDEV}
The setup for ADAM application development has been modified as follows:
\begin{itemize}
\item Now obeys HDS\_DEV to set up logical names {\em etc.} for HDS
application development.\\
(ADAMDEV.COM, APPLOG.COM)
\item Now defines symbol REF to be `REF\_DEV'.
This is a temporary measure to enable user's procedures to continue
working but they should be modified in accordance with SUN/31.3 as soon as
possible.

{\em Note that the definition of REF will be removed in the near future.}\\
(ADAMDEV.COM)
\item ADAM copies of DAT\_PAR, DAT\_ERR, DAT\_SYS and CMP\_ERR are
no longer used. The versions in HDS\_DIR (HDS\_TOP\_DIR) are used.\\
(Modified APPLOG.COM).
\end{itemize}

\subsection{SYSDEV}
The setup for ADAM system development has been modified as follows:
\begin{itemize}
\item REF\_DIR is assumed to be set as a SYSTEM logical name.\\
(Modified DIR.COM)
\item Logical names for HDS are set by obeying HDS\_TOP\_DIR:LOGICAL.COM as
before. However, that procedure no longer defines logical names for REF so
reference to REF is removed from the comment in LIB\_DIR:LOGICAL.COM.\\
(Modified LOGICAL.COM)
\item HDS\_DIR is now defined as a SYSTEM logical name pointing to the
top-level directory for HDS, therefore a new logical name, HDS\_KERNEL, is used
for ADAM system development to point to the HDS kernel directory.\\
(Modified: DIR.COM, ADAMSHARE.COM, LINKNOSHR.OPT)
\end{itemize}
{\em Note that further changes in this area are expected when the next version
of HDS is released.}

\subsection{The ADAM Procedure}
A new procedure, ADAM\_DIR:ADAM.COM, has been written to merge the sequence
of instructions formerly required to start up an ADAM session.
A symbol, ADAM, is defined by SSC:LOGIN.COM to run the procedure.
\begin{quote} \begin{verbatim}
$ ADAM command
\end{verbatim} \end{quote}
may be issued to replace the sequence:
\begin{quote} \begin{verbatim}
$ ADAMSTART
$ ICL
ICL> command
\end{verbatim} \end{quote}

Notes:
\begin{itemize}
\item \verb%command% is optional and may be any valid ICL command, {\em e.g.}
KAPPA.
\item Any other initialisation procedure specified via the ICL\_LOGIN
logical name will still be executed beforehand.
\item Invocations of ADAM.COM other than the first in a process will not obey
ADAMSTART.
\end{itemize}

\subsection{ICL}
\begin{itemize}
\item ICL Version 1.5-5 is released - the login banner has been changed
appropriately.\\
(Modified: LOGIN.ICL, ADAMLOGIN.ICL, INFO.LIS)
\item A new command, SET [NO]SETPARS is implemented. It allows ICL to be
prevented from checking the number of parameters supplied to procedures.
Thus procedures may be written which prompt the user for any required parameters
which are not supplied on the command line.\\
(Modified: ICLDEF.PAS, ICLMAIN.PAS, ICLPROC.PAS).
\item The help library has been updated to include the new commands.\\
(Modified: ICLHELP.HLP/.HLB)
\end{itemize}

\subsection{SMS}
SMS has been modified to support user scrolling of the scrolling region.
The scrolling keys are only accessible after pressing the SELECT  or PERIOD
key and the following are defined:

\begin{tabbing}
gold/right\_arrow \= - \= \kill
\verb%<crtl>N% \> - \> scroll to next line\\
\verb%<crtl>P% \> - \> scroll to previous line\\
prev\_screen \> - \> scroll up by one screen\\
next\_screen \> - \> scroll down by one screen\\
left\_arrow  \> - \> scroll one column to the left\\
right\_arrow \> - \> scroll one column to the right\\
\\
gold/up\_arrow \> - \> scroll to top of display buffer\\
gold/down\_arrow \> - \> scroll to bottom of display buffer\\
gold/left\_arrow \> - \> scroll to far left of buffer\\
gold/right\_arrow \> - \> scroll to far right of buffer\\
\end{tabbing}

Note the following points:
\begin{itemize}
\item The `gold' key is PF1, a `/' indicates press the keys in sequence
and \verb%<ctrl>% means press at the same time as the following letter.
\item None of these keys get into the command buffer so can be freely mixed with
any other input you may be entering at the command prompt.
\item Currently the previous 100 lines of SMS output are available in this way
and the width of the buffer is 132 characters.
\item On pressing return and thus exiting from command mode, the display
automatically scrolls to the bottom of the buffer.
\end{itemize}
(Modified: SMS\_\-GETUSRLIN, SMS\_\-KEYCHAR, SMS\_\-WRSCRL, SMS\_\-SCROLL\-VIEW,
SMS\_\-GET\-USR\-KEY, SMS\_\-DODCL, SMS\-DEFNS, SMS\-COM\-SMG, INFO.LIS\\
New include file: SMSCOMVIEW.)

\subsection{Documentation}
SSN/45.13 and ARN/19-4.1 describe ADAM release 1.9-4. The ARN will remain when
the SSN is superseded.

The summaries, ADAM\_DOCS:0CONTENTS.LIS, FULLDOCS.LIS and NEWDOCS.LIS, have
been updated, as has the COPYDOCS procedure.

\section{BUGS FIXED}

\subsection{ICL}
\begin{itemize}
\item A bug, which caused ICL to fail with ``Error opening file'' after a
large number of LOAD commands, has been fixed.\\
(Modified: ICLMAIN.PAS)
\item LOAD files are now opened `READONLY' to prevent inter-user conflicts.\\
(Modified: ICLMAIN.PAS)
\item A bug, which caused incorrect behaviour if \verb%<CR>% or \verb%?% was
given in response to parameter prompts when ICL was in screen mode, has been
fixed. (There is still a problem if \verb%<TAB>% is typed.)\\
(Modified: UFACE\_ASKPARAM)
\end{itemize}

\subsection{SMS}
A bug, which could cause SMS to get into an infinite loop, has been corrected.\\
(Modified: SMS\_GETUSRKEY)

\end{document}
