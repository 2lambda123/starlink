\documentstyle{article} 
\pagestyle{myheadings}

%------------------------------------------------------------------------------
\newcommand{\stardoccategory}  {ADAM Release Note}
\newcommand{\stardocinitials}  {ARN}
\newcommand{\stardocnumber}    {18.1}
\newcommand{\stardocauthors}   {A J Chipperfield}
\newcommand{\stardocdate}      {18 September 1990}
\newcommand{\stardoctitle}     {ADAM --- Release 1.8}
%------------------------------------------------------------------------------

\newcommand{\stardocname}{\stardocinitials /\stardocnumber}
\markright{\stardocname}
\setlength{\textwidth}{160mm}
\setlength{\textheight}{240mm}
\setlength{\topmargin}{-5mm}
\setlength{\oddsidemargin}{0mm}
\setlength{\evensidemargin}{0mm}
\setlength{\parindent}{0mm}
\setlength{\parskip}{\medskipamount}
\setlength{\unitlength}{1mm}

%------------------------------------------------------------------------------
% Add any \newcommand or \newenvironment commands here
%------------------------------------------------------------------------------

\font\tt=CMTT10 scaled 1095
\renewcommand{\_}{{\tt\char'137}}

\begin{document}
\thispagestyle{empty}
SCIENCE \& ENGINEERING RESEARCH COUNCIL \hfill \stardocname\\
RUTHERFORD APPLETON LABORATORY\\
{\large\bf Starlink Project\\}
{\large\bf \stardoccategory\ \stardocnumber}
\begin{flushright}
\stardocauthors\\
\stardocdate
\end{flushright}
\vspace{-4mm}
\rule{\textwidth}{0.5mm}
\vspace{5mm}
\begin{center}
{\Large\bf \stardoctitle}
\end{center}
\vspace{5mm}
%------------------------------------------------------------------------------
%  Add this part if you want a table of contents
  \setlength{\parskip}{0mm}
  \tableofcontents
  \setlength{\parskip}{\medskipamount}
  \markright{\stardocname}
%------------------------------------------------------------------------------

\section{SUMMARY}
This is a {\it complete} release of ADAM available as either a {\it full}
or {\it mini-release}.
It will not be necessary to re-link tasks for this release unless the bug
correction to DTASK\_CANCEL is required.

The main improvement is the provision of better help facilities. In particular,
the method of producing multi-line help on parameter prompts in ICL is 
extended to SMS and to tasks running directly from DCL or as ICL shareable 
monoliths. The facility to remain in the help system is also provided as is 
paging of the help output.
SMS is improved to provide a help window which is used for multi-line parameter
help and by the ICL `HELP' command.

The full release requires about 43000 blocks of disk storage and includes a
mini-release which can be extracted and put up separately.
The mini-release requires about 14200 blocks and allows the system to be run
and tasks to be developed.

\section{INSTALLATION}
Full installation instructions are given in SSN/44 and the Starlink Software 
Change Notice.

\section{NEW FEATURES IN THIS RELEASE}

\subsection{SUBPAR}
\begin{itemize}
\item Modifications are made to implement the parameter help system described
in Appendix \ref{parhelp}.
The following routines/files have been changed: SUB\-PAR\-\_INP\-UT, \_REQU\-EST, 
\_PROMPT, \_PROMP\-TCL, \_LOAD\-IFC, SUB\-PAR.\-CMN.
The following new routines have been written: SUBPAR\-\_SPLIT\-VAL, \_SEL\-HELP,
\_OPUT, \_IPUT.
\item SUBPAR\_\_MAXLIMS, the size of the `data lists' in the SUBPAR common
block, has been increased from 350 to 500.
File changed: SUBPAR.CMN.
\end{itemize}

\subsection{PARSECON}
\label{parse}
\begin{itemize}
\item Additional fields {\em helpkey} and {\em helplib} are provided to
extend the parameter help facilities as described in the revised AED/3.
The following routines/files have been changed: PARSE\-CON\-\_TAB\-INIT,
\_READ\-IFL,
\_DUMP\-IFC, \_NEW\-PAR, \_TOK\-TYP, \_SET\-VP and PARSE\-CON\-.CMN
The following new routines/files have been written: PARSE\-CON\-\_SET\-HKEY, 
\_SET\-HLIB, PARSE\-CON2\-.CMN, PARSE\-CON3\-.CMN
\item Error reporting has been improved. To facilitate this, the input
buffer and the current interface/action and parameter names are now stored in a 
common block. All available information is then output in the event of an
error.
(PARSERR, PARSE\-CON\-\_GET\-TOK, \_READ\-IFL, \_NEW\-PAR, \_PAR\-END(new),
\_FAC\-END(new), \_ACT\-END,
\_ERROR, PARSE\-CON\-.CMN, PARSE\-CON3\-.CMN)
\item Recovery from errors is now more sophisticated. 
A new state is set depending upon the state in which the error occurred and
parsing will now continue after some errors which previously caused exit with 
status set.
(PARSE\-CON\-\_READ\-IFL, \_GET\-TOK, \_RE\-SET(new))
\item The interface file syntax has been rationalised, in particular, fewer 
things now have to be enclosed in quotes. The allowed syntax is fully described
in the revised AED/3\@. (PARSECON\_TABINIT)
\end{itemize}

\subsection{ICL}
\label{icl}
ICL Version 1.5-3 is issued with this release.
Improvements have been made in the following areas:
\begin{itemize}
\item Symbol table handling now avoids efficiency problems with large
tables. (ICLPARSE)
\item ICL now calls routine UFACE\_WRHELP to output help. This enables SMS
to handle ICL help output sensibly (see Section \ref{sms})\@. (ICLTERM, ICLEXT)
\item The enhanced parameter help system described in Appendix \ref{parhelp}
has been implemented; so has paging of ICL help.
(ICL\-TERM, ICL\-EXT, UFACE\_\-ASK\-PARAM, \_WRHELP\-(new), \_IPUT\-(new), 
\_OPUT\-(new) UFACE\_\-CMN).
\item A new command, SET HELPFILE, has been added (see Section \ref{sms})\@.
The new command has been added to the ICL help file.
\item Improvements have also been made to the parameter system re-initialisation
required for shareable monoliths. Locators to the top-level of the
monolith parameter files are kept so that the file does not need to be
re-opened every time. (SUBPAR\_ACTSHR)
\end{itemize}

\subsection{SMS}
\subsubsection{User-interface}
\label{sms}
A number of modifications have been made by Chris Mayer (JACH::CJM)\@.
The purpose of the modifications is:
\begin{itemize}
\item To enable command line recall. (SMS\_GETUSRLIN)
\item To enable better handling of ICL help.
In order to implement this:
\begin{itemize}
\item A special `help' window is used which disappears when help output is
complete. Help library searching is provided and the output is `paged'\@.
(SMS\_COMNINIT, \_LOAD\-TABLE, \_WRHELP, common blocks SMSCOMSMG, SMSCOMFLD)
\item Modifications have been made to ICL.
The changes involve the addition of a SET HELPFILE command and the provision 
of a UFACE\_WRHELP subroutine (see Section \ref{icl}).
\end{itemize}

\item To decouple SMS's help system from that of the underlying command
language. The SMS (ADAM) help file can now be set using the new {\em config}
parameter {\em helpfile} (see Appendix \ref{smshelp}).
(SMS\-\_SYS\-CONFIG)
\item An additional {\em +config+} parameter, {\em refresh} is provided to
refresh the screen from the SCT\@. (SMS\-\_SYS\-CONFIG)
\item Handle the spawn command correctly so that SMS and the
subprocess are not expecting input at the same time and
the screen is repainted on logging out from the sub-process. (SMS\_DOCMDLINE)
\item Update the SMS help library with the latest changes. (SMS\_HELPFILE)
\item Fix known bugs (see Section \ref{smsbugs}).
\end{itemize}

Additionally, SMS has been extended to handle multi-line parameter help
(see Appendix \ref{parhelp}). The new help display is used for this
purpose. (SMS\_ASKPARAM, UFACE\_WRHELP)

\subsubsection{Control Table Files}
\begin{itemize}
\item The template SMS Control Table `system include' (SMSYSTAB.INC and
SMSICLSYSTAB.INC) files have been modified to:

\begin{itemize}
\item Make use of the new {\em +config+ helpfile} parameter.
\item In the case of SMSICLSYSTAB, to restore correct SMS help and auto-help
working.
To this end, an additional VMS help file SMS\_HELPFILE is provided in
ADAM\_HELP.
The new help display window is used for SMS help.
\item Slightly modify the position of the SMS help and configuration menus.
\item Remove reference to the AON004 help file.
\end{itemize}

\item An additional template `system include' file (FIXED.SCT),  which will 
work with either SMS or SMSICL, is provided in ADAM\_INC.
It specifies a non-default help window and a help menu which uses `??' rather
than `ENTER' to obtain full help (SMSICL will not use its new help display in
this case).

\item The example control tables (AON004.SCT and AON004ICL.SCT) in ADAM\_TEST
(see SG/4 section 15.2) have been improved to:
\begin{itemize}
\item Use suitably modified versions of the `system include' files.
\item Add menu option to run an A-task (TRACE).
This can be used to demonstrate parameter help with SMS (ref. Section
\ref{trace}).
\item In the case of AON004ICL, restore auto-help and SMS help.
\end{itemize}
\end{itemize}

\subsection{Adamcl}
The Adamcl UFACE\_ASKPARAM routine has been modified to accept the revised
parameter request message format (see Appendix \ref{parhelp}).
Note however that multi-line parameter help is not available with Adamcl and
that {\em helpkey} and {\em helplib} interface file fields will be ignored.

\subsection{TRACE}
\label{trace}
The interface module of TRACE, the HDS structure tracing utility, has been
modified to specify multi-line help for the parameters.
This is not actually very useful but serves to demonstrate the new parameter
help facilities.
A suitable help file been created -- the source is in ADAMEXE\_SOURCE and the
library in ADAM\_HELP.

\subsection{HDS}
The shared image is built with the latest version of the HDS kernel library
available at RAL. 
It contains major internal revisions but should be functionally the same as 
previous versions, apart from an additional subroutine, DAT\_WHERE.
{\em This version has not yet been released}, therefore any attempt to rebuild 
the system must take this into account, possibly removing DAT\_WHERE from the
transfer vector.

\subsection{PROCEDURES etc.}
\subsubsection{ADAMSTART}
ADAMSTART.COM is modified to:
\begin{itemize}
\item Display the latest ADAM version number.
\item Use F\$TRNLNM rather than F\$LOGICAL to translate all names
\end{itemize}

\subsubsection{LIB\_DIR:LOGICAL}
\begin{itemize}
\item PARSECON2\_CMN and PARSECON3\_CMN are defined (see Section \ref{parse}).
\item Logical names required for ICL linking are defined.
\end{itemize}

\subsubsection{ADAMSHARE}
\begin{itemize}
\item Minor id increased for version 1.8
\item CHR\_STRDEC and CHR\_LDBLNK are no longer linked into the shared image.
\item New common blocks PARESECON2, PARSECON3 and SUBPARTERM.
\item SUBPAR\_SELHELP and SUBPAR\_SPLITVAL are added to the transfer vector.
\item Dummy transfers are defined in place of CHR\_STRDEC and CHR\_LDBLNK
(see Section \ref{cleanchr}).
\end{itemize}

\subsubsection{ADAMGRAPH7}
\begin{itemize}
\item Minor id increased for version 1.8
\item GNS\_IANG, GNS\_IANI, GNS\_IGAG, GNS\_IIAI and GNS\_ITWCG are added to 
the transfer vector.
\end{itemize}

\subsubsection{LINKNOSHR etc.}
The order of library search in the `noshareable' link commands has been altered
to cope with new releases of the Starlink graphics libraries which use EMS for
error reporting.
(LINKNOSHR.OPT, ANOSHR.COM, MNOSHR.COM, DNOSHR.COM, CDNOSHR.COM)

\subsection{Documentation}
SSN/45.11 and ARN/18.1 describe ADAM release 1.8.
ARNs will be retained for reference as SSN/45 gets replaced at each new release.

The following document has been re-issued:
\begin{description}
\item[AED/3.4] {\it ADAM -- Interface Module Reference Manual} -- various
improvements and additions. In particular describing the enhanced parameter
help fields. 
Also converted to \LaTeX.
\end{description}

The summaries, ADAM\_DOCS:0CONTENTS.LIS, FULLDOCS.LIS and NEWDOCS.LIS, have 
been updated. 

\section{BUGS FIXED}
\subsection{ICL}
The following bugs have been fixed.
\begin{itemize}
\item The SMSICL `from adamcl' switch facility did not work. (ICLINTERP)
\item Incorrect typing of un-initialised variables used to receive parameter 
values returned from tasks if the variable was an argument of a procedure.
{\em Note that if the variable has already been given a value, it 
still must be a suitable type.}
(ICLDEF)
\item Exponentiation of a negative number failed. (ICLINTERP)
\item The global parameter file was not closed if the component specified
by a GETGLOBAL or GETPAR command did not exist. This led to values
apparently not being updated. (ICLADAM)
\item A failure to open additional ({\em e.g.}\ interface) files after
multiple invocations of tasks in different shareable monoliths.
(SUBPAR\_ACTSHR)
\item There were a number of typos in the ICLHELP file. (ICLHELP.HLP/.HLB)
\end{itemize}

\subsection{Parameter System}
The following bugs in SUBPAR have been fixed.
\begin{itemize}
\item Array bounds were exceeded, usually causing a spurious HDS\_\_TYPIN
error, if more than 32 parameters were defined for a task.
The checking has been improved and two new errors defined:
\begin{description}
\item[SUBPAR\_\_XMXPOS] too many command line parameter positions allocated.
\item[SUBPAR\_\_PNOTAL] command line parameter position used is not allocated.
\end{description}
Thus it is no longer possible to put a parameter value on the command line and
have it ignored because no parameter had that position allocated to it.
(SUBPAR\_CMDLINE, SUBERR.MSG etc.)
\item Dynamic defaults were remembered across invocations of tasks if the
task remained loaded. (SUBPAR\_DEACT, \_DEF1x, \_HDSDYN, \_INPUT, 
PARSECON\_NEWPAR)
\item If a new value for a cancelled parameter was requested and a valid value
was not given in response to the prompt, no error message was output before
the first re-prompt. (SUBPAR\_FINDHDS, \_GETNAME)
\item The number of attempts to get a valid value in response to a prompt
before status PAR\_\_NULL was returned could be anything up to five.
It will now be consistently five. (SUBPAR\_FINDHDS, SUBPAR\_GETNAME)
\end{itemize}

\subsection{LEX}
The following bug in LEX has been fixed.
\begin{itemize}
\item A filename starting with digits followed by E or D was not recognised.
\end{itemize}

\subsection{SMS}
\label{smsbugs}
The following bugs in SMS have been fixed.
\begin{itemize}
\item ICL could hang if a very large number of procedures were being loaded. 
The bug was only seen when two invocations of SMSICL were running but of 
course using the same SDF file.
It was due to the default of 32 for the maximum size of the  HDS working page 
list being too low.
HDS\_TUNE is now called to set it to 256. (SMS\_MAIN)
\item The value of NEWOPTION was being retained between calls to
SMS\_MOVCURMEN.
This caused SMS to crash with a divide by zero error in some circumstances.
(SMS\_MOVCURMEN)
\item If in command line mode and a message came in whilst typing a command, 
subsequent characters would appear in the scrolling region.
(SMS\_GETUSRLIN)
\item For a parameter list occurring as the last item in a control table,
the `from' lines were ignored.
(SMS\_LOADTABLE)
\item One-line help messages were not cleared up correctly after displaying
full help.
\end{itemize}

\subsection{MSP}
The following bugs in MSP have been corrected:
\begin{itemize}
\item A problem occurred when a process was killed without running its exit 
handler (as happens if you log out with the task still running). 
This caused the task entries to be left in the MSP data structures as long 
as the global section existed (which could be all the time, {\em e.g.}\ if 
running ADAMNET)\@.
The problem occurs if another task tries to get a path to the killed task 
before it is reloaded. 

An earlier correction for this condition was flawed, causing an infinite loop.
This has now been fixed and GET\_TASK\_QUEUE will now return status 
MSP\_\_NOSUCHTASK under these circumstances.
(MSP\_GET\_TASK\_QUEUE)

\item The variable TASK\_LIST was not page-aligned. This caused a problem when
linking non-shareable. (MSP.PAS)
\end{itemize}

\subsection{DTASK}
An acknowledgement was not sent to terminate the OBEY transaction if it was
from the same process as the CANCEL. This is now done, provided that the
action is cancelled. (DTASK\_CANCEL)

\subsection{FIO/RIO}
The following bugs in FIO/RIO have been corrected.
\begin{itemize}
\item If FIO/\-RIO\-\_ASSOC failed after having obtained a value for the 
specified parameter, FIO/\-RIO\-\_CANCL would fail to cancel the parameter 
value. 
Hence it was not possible to obtain a different value by calling the ASSOC 
routine again. (FIO/\-RIO\-\_ASSOC, FIO/\-RIO\-\_CANCL)
\item FIO/\-RIO\-\_ASSOC and FIO/\-RIO\-\_CANCL were not case-insensitive 
about the parameter name. (FIO\_\$FNDFP)
\end{itemize}

\section{CLEANUP}
\subsection{CHR}
\label{cleanchr}
As forewarned at the last release,
the temporary alternative names CHR\-\_STRDEC and CHR\-\_LDBLNK for routines 
CHR\-\_DCWRD and CHR\-\_LDBLK have been removed.
Tasks which are already linked will continue to work, but it will not now
be possible to link tasks which call the old names.
A direct name substitution in the code is possible.

\appendix
\newpage
\section{Parameter Help Enhancements}
\label{parhelp}
To meet requirements expressed in the ADAM\_DEVELOPMENT conference and
elsewhere, the following modifications to the parameter help system have
been implemented

A parameter may now have two help specifiers defined in the interface file.
\begin{description}
\item[help] - a one-line help string (as now).
For the time being, the one-line help specifier may in the form 
proposed and implemented by Jeremy to provide `multi-line' help for
ICL\@.
That is:
\begin{quote}
\begin{verbatim}
help '%library key1 key2 etc.'
\end{verbatim}
\end{quote}
Single-line help is not expected to be used in important packages.
   

\item[helpkey] - to specify a help library and a module within it at 
which `multi-line' help on the parameter may be found.

For example:   
\begin{quote}
\begin{verbatim}
helpkey 'kappa_dir:kappa kappa add object'
\end{verbatim}
\end{quote}

Note that `multi-line' is a reference to how many lines of help there 
could be, rather than how many there are.

If the specifier is of the form:
\begin{quote}
\verb%helpkey *% or \verb%helpkey '*'% 
\end{quote}
a default value of \verb%interface_name PARAMETERS parameter_name% will be
used, where \verb%interface_name% is the name specified in the current 
INTERFACE field.
\end{description}
The response to the parameter prompt may be ? or ?? and the effect will
depend upon the particular user-interface in use and which of the help
specifiers are given. The following table shows the order in which the 
options will be tried by the user-interfaces ICL, DCL and SMS. Adamcl
will ignore any {\em helpkey} specifier.
\begin{center}
\begin{tabular}{|c|c|} \hline
Response  & Options \\
\hline
? & help, helpkey, apology\\
?? & helpkey, help, apology\\
\hline
\end{tabular}
\end{center}

Notes.
\begin{enumerate}
\item The `help' option uses the interface file {\em help} field if there is
one. Note that this could produce either one-line or multi-line help.

\item The `helpkey' option uses the interface file {\em helpkey} field if
there is one.

\item The `apology' option will be selected if no earlier one is available.
It consists of outputting an apologetic message. 

\item For ICL and DCL, if multi-line help is output in response to `?', the 
user is prompted again immediately; if it is output in response to `??',
the user will be left in the help system and prompted for further keys.
This means that single-line help plus immediate return from multi-line
help is not available. SMS will always leave the user in the help system.

\item Multi-line help will be paged.

\item If the specified help library cannot be found or opened by the user,
an error message will be output.

\item Other messages may appear from the help system itself.
\end{enumerate}

\subsection{Further Extensions}
An additional interface file field {\em helplib} is provided.
The intention is to specify a top level for all succeeding helpkey specifiers 
until another {\em helplib} field is found. 
When the interface file is parsed, the specified string is inserted (with 
an intervening space) in front of any string specified in subsequent 
{\em helpkey} fields. 
The {\em helplib} specifier can be a blank string to nullify an earlier one.

\subsection{Implementation}
Both the {\em help} and {\em helpkey} specifiers are included in the parameter 
request message sent to the user-interface. 
This has become possible with the recently increased maximum inter-task
message length.

\newpage
\section{SMS Help Enhancements}
\label{smshelp}
To configure the new VMS help display, the following extra {\em fields} entries
are permitted.
\begin{center}
\begin{tabular}{ll}
help\_width  {\em n} & the width of the help display in columns\\
help\_height  {\em n} & the height of the help display in rows\\
help\_leftcol {\em n} & the column position of the top left hand corner of the
                   help display\\
help\_toprow  {\em n} & the row position of the top left hand corner of the help
                   display\\
\end{tabular}
\end{center}
If no values are specified then the system defaults to
\begin{center}
\begin{tabular}{lr}
help\_width   & 78\\
help\_height  & 22\\
help\_leftcol  & 2\\
help\_toprow   & 2\\
\end{tabular}\\
\end{center}
This will pass the full screen over to VMS help ( a border is applied around
the display) and so enable the maximum amount of information to be seen at
any one time. 

The new {\em config} parameters allow actions such as the following
\begin{quote}
\begin{verbatim}
action set_help
+config+ helpfile <filename>
\end{verbatim}
\end{quote}
where \verb%<filename>% is a valid ADAM help file or a logical name that 
points to one.
\begin{quote}
\begin{verbatim}
action refresh
+config+ refresh
\end{verbatim}
\end{quote}

\end{document}
