\documentstyle{article}
\pagestyle{myheadings}

%------------------------------------------------------------------------------
\newcommand{\stardoccategory}  {ADAM Release Note}
\newcommand{\stardocinitials}  {ARN}
\newcommand{\stardocnumber}    {17.1}
\newcommand{\stardocauthors}   {A J Chipperfield}
\newcommand{\stardocdate}      {1 May 1990}
\newcommand{\stardoctitle}     {ADAM --- Release 1.7}
%------------------------------------------------------------------------------

\newcommand{\stardocname}{\stardocinitials /\stardocnumber}
\markright{\stardocname}
\setlength{\textwidth}{160mm}
\setlength{\textheight}{240mm}
\setlength{\topmargin}{-5mm}
\setlength{\oddsidemargin}{0mm}
\setlength{\evensidemargin}{0mm}
\setlength{\parindent}{0mm}
\setlength{\parskip}{\medskipamount}
\setlength{\unitlength}{1mm}

%------------------------------------------------------------------------------
% Add any \newcommand or \newenvironment commands here
%------------------------------------------------------------------------------

\font\tt=CMTT10 scaled 1095
\renewcommand{\_}{{\tt\char'137}}

\begin{document}
\thispagestyle{empty}
SCIENCE \& ENGINEERING RESEARCH COUNCIL \hfill \stardocname\\
RUTHERFORD APPLETON LABORATORY\\
{\large\bf Starlink Project\\}
{\large\bf \stardoccategory\ \stardocnumber}
\begin{flushright}
\stardocauthors\\
\stardocdate
\end{flushright}
\vspace{-4mm}
\rule{\textwidth}{0.5mm}
\vspace{5mm}
\begin{center}
{\Large\bf \stardoctitle}
\end{center}
\vspace{5mm}
%------------------------------------------------------------------------------
%  Add this part if you want a table of contents
  \setlength{\parskip}{0mm}
  \tableofcontents
  \setlength{\parskip}{\medskipamount}
  \markright{\stardocname}
%------------------------------------------------------------------------------

\section{SUMMARY}
This is a {\it complete} release of ADAM available as either a {\it full}
or {\it mini-release}.
The change most obvious to users is that the graphics environment level
routines now use the Graphics Workstation Name Service (GNS) to define graphics
devices.
This may require operational changes (see Section 3.1).
DEVDATASET is now used only for magnetic tape devices.

The release also contains bug fixes and minor enhancements to other facilities,
most notably the switch to the EMS based MSG and ERR facilities.

It will not generally be necessary to re-link tasks for this release but tasks
must be re-linked to make use of the increased command line length (see
Section 3.3).

The full release requires about 43000 blocks of disk storage and includes a
mini-release which can be extracted and put up separately.
The mini-release requires about 13500 blocks and allows the system to be run
and tasks to be developed.

\section{INSTALLATION}
Full installation instructions are given in SSN/44 and the Starlink Software
Change Notice.

\section{NEW FEATURES IN THIS RELEASE}

\subsection{SGS and GKS}
\begin{itemize}
\item The subroutines GKS\_ASSOC and SGS\_ASSOC now use the Graphics
Workstation Name Service (GNS) to translate user-friendly names into GKS
workstation type and connection identifier.
DEVDATASET is no longer used for graphics devices.
This may result in the user-friendly names of some devices being changed, but
it is possible to retain the old names by specifying them as logical
names (see SUN/57).
\item Error reporting from the environment level routines has been improved,
bringing it into line with the recommendations of SUN/104.
\item The GKS and SGS test programs have been improved.
\item The interface file for the test program SGSX1 has been improved.
\item The following consequent changes have also been made:
\begin{itemize}
\item ADAM\_DIR:DEVICE7.PRC and DEVICE7.SDF have graphics device definitions
removed.
DEVICE7.PRC is also renamed DEVICE7.ICL.
\item CRPDEV is removed from the directories and from the system `build'
procedures.
\end{itemize}
\end{itemize}

\subsection{GNS}
Subroutines GNS\_FILTG, GNS\_FILTI, GNS\_GWNI and GNS\_TNI have been added to
the ADAMGRAPH7 shared image transfer vector.

\subsection{SUBPAR}
\begin{itemize}
\item The maximum length of a command line which may be handled by SUBPAR
and MESSYS has been increased to 444 characters (ICL still imposes a limit
of 300).
(SUBPAR\_CMDLINE)
\item The modifications provided by John Fairclough to improve number to
character conversion have been incorporated.
(SUBPAR\_GET0C)
\item SUBPAR\_\_MAXLIMS, the size of the `data lists' in the SUBPAR common
block, has been increased from 300 to 350.
\end{itemize}

\subsection{MESSYS}
PTH\_LEN and MSG\_LEN have been increased to  accommodate a 512-byte message
so that longer command lines may be passed. This allows a message `value'
of 444 bytes.
(MES\-DEFNS\-.FOR and DDMSG\-.FOR)

This change required the re-compilation of the DTASK, TASK, ADAM,
UFACE (standard, ICL and SMS), SUBPAR, SMS and ANT libraries.
Any other tasks which include MESDEFNS will need to be re-compiled.

\subsection{MESSYS V2}
\begin{itemize}
\item PTH\_LEN and MSG\_LEN have been increased as for Version 1 MESSYS.
(MESDEFNS.FOR, DDMSG.FOR)
\item MSP now handles the separation of one user's tasks from another's
(MESSYS\_PREFIX)
\end{itemize}

\subsection{MSG, ERR and EMS}
The shared image, ADAMSHARE, has been built with the `ADAM' versions of the
separate Starlink release of the ERR and MSG systems based upon EMS, the
low-level Error Message Service. That release will coincide with this release.
For descriptions of these facilities, see SUN/104 and SSN/4.

The MSG and ERR libraries have been removed from the ADAM directories.

\subsection{SMS}
\begin{itemize}
\item CL\_MAXENT has been increased from 1500 to 2000. (SMSDEFNS.FOR)
\item The VT200 keys have been added. (SMS\_ESCIN)
\end{itemize}

\subsection{HDS}
The shared image is built with the latest version of the HDS kernel library
available at RAL.
It contains major internal revisions but should be functionally the same as
previous versions, apart from an additional subroutine, DAT\_WHERE.
{\em This version has not yet been released}, therefore any attempt to rebuild
the system must take this into account, possibly removing DAT\_WHERE from the
transfer vector.

\subsection{MSP}
\begin{itemize}
\item Improvements have been made in the area of recovery after task exit.
\item The use of logical names to give user/group/system divisions has been
simplified.
\end{itemize}

\subsection{ANT}
\begin{itemize}
\item A fixed prefix is now used in MSP names for ANT. (ANT\_MSPNAME)
\item An exit handler is provided. (ANT\_EXIT)
\item The ADAMNET program has been altered to make the maximum message size
C\-\_MAXMSG\-\_LEN. (ADAMNET.FOR)
\item ADAMNET declares the new ANT exit handler. (ADAMNET.FOR)
\end{itemize}

\subsection{NBS}
NBS version R2\_3\_1 is released.

\subsection{Procedures etc.}
\subsubsection{System Building Procedures}
Changes to the system building procedures have been made to incorporate
the changes described above.
Files affected are:
\begin{description}
\item[In SHARE\_DIR] ADAMSHARE.COM/MAR, ADAMGRAPH7.COM/MAR.
\item[In LIB\_DIR] LINKNOSHR.OPT, DIR.COM, LOGICAL.COM
\item[In ADAM\_ADAMEXE] BUILDEXE.COM, BUILDIFC.COM, COPYEXE.COM
\end{description}

\subsubsection{SYSLOGNAM}
The coding of SYSLOGNAM.COM and SYSLOGNAM.MINI has been improved.

\subsubsection{ADAMSTART}
ADAMSTART.COM is modified to display the latest ADAM version number.

\subsubsection{ADAMDEV}
\begin{itemize}
\item A symbol, REF, is defined for use with th REF facility.
\item F\$TRNLNM is used in place of F\$LOGICAL.
\end{itemize}

\subsection{Documentation}
SSN/45.10 and ARN/17.1 describe ADAM release 1.7.
ARNs will be retained for reference as SSN/45 gets replaced at each new release.

The following new documents are relevant:
\begin{description}
\item[SUN/104.1] {\it MSG and ERR --- Message and Error Reporting}
\item[SSN/4] {\it EMS --- Error Message Service}
\end{description}

The following documents have been re-issued:
\begin{description}
\item[APN/2.3] {\it ADAM Programmer's Guide to the GKS Package} -- reflects
the changes described above. Also converted to \LaTeX.
\item[APN/3.3] {\it ADAM Programmer's Guide to the SGS Package} -- reflects
the changes described above. Also converted to \LaTeX.
\item[ASN/1.2] {\it ADAM --- Organization of Graphics} -- converted to
\LaTeX and updated.
\item[ASN/2.2] {\it ADAM V2 Message System} -- updated
\item[SSN/44.7] {\it ADAM --- Installation Guide} -- reflects
the changes to DEVDATASET described above.
\end{description}

The following document has been withdrawn:
\begin{description}
\item[AED/14] {\it Using the MSG and ERR Systems}
\end{description}

The summaries, ADAM\_DOCS:0CONTENTS.LIS, FULLDOCS.LIS and NEWDOCS.LIS, have
been updated.

\section{BUGS FIXED}
\subsection{ICL}
ICL Version 1.5-2 is issued with this release.
\begin{itemize}
\item A bug, which caused ICLSMS to fail with the error PAS-F-STRASGLEN if
it attempted to obey an ICL command line more than 80 characters long, has been
corrected.
300 characters will now be accepted.
\item A bug, which caused ICL to hang if no EXIT command was supplied in a
batch job, has been fixed.
\item The full parameter string, from a maximum command line of 300 characters,
will now be passed to ADAM tasks.
\end{itemize}

\subsection{Command line parsing -- LEX}
\begin{itemize}
\item A bug, which caused command lines to be mis-interpreted if they contained
specifiers of the form `\verb%keyword=#%', where `\verb%#%' could be various
non-alphanumeric characters, has been fixed. (LEX\-\_CMDSET)
\end{itemize}

\subsection{LINKNOSHR}
LINKNOSHR.OPT, used by the `noshareable' linking procedures has been
corrected to remove MSP\_SHR which will not be required until the Version 2
message system is in use.

A further problem involving the ARGS\_FUDGE PSECT has also been corrected.

\subsection{TASK}
An bug in the string concatenation routine has been fixed.
(TASK\_CNCAT)

\section{CLEANUP}
\subsection{Candidates for Removal}
The temporary alternative names CHR\_STRDEC and CHR\_LDBLNK for routines
CHR\_DCWRD and CHR\_LDBLK will be removed in the next release.

\end{document}
