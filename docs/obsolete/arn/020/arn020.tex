\documentstyle[11pt]{article}
\pagestyle{myheadings}

%------------------------------------------------------------------------------
\newcommand{\stardoccategory}  {ADAM Release Note}
\newcommand{\stardocinitials}  {ARN}
\newcommand{\stardocnumber}    {20-2.1}
\newcommand{\stardocauthors}   {A J Chipperfield}
\newcommand{\stardocdate}      {1 February 1993}
\newcommand{\stardoctitle}     {ADAM --- Release 2.0-2}
%------------------------------------------------------------------------------

\newcommand{\stardocname}{\stardocinitials /\stardocnumber}
\markright{\stardocname}
\setlength{\textwidth}{160mm}
\setlength{\textheight}{230mm}
\setlength{\topmargin}{-2mm}
\setlength{\oddsidemargin}{0mm}
\setlength{\evensidemargin}{0mm}
\setlength{\parindent}{0mm}
\setlength{\parskip}{\medskipamount}
\setlength{\unitlength}{1mm}

%------------------------------------------------------------------------------
% Add any \newcommand or \newenvironment commands here
%------------------------------------------------------------------------------

\font\tt=cmtt10 scaled 1095
\renewcommand{\_}{{\tt\char'137}}

\begin{document}
\thispagestyle{empty}
SCIENCE \& ENGINEERING RESEARCH COUNCIL \hfill \stardocname\\
RUTHERFORD APPLETON LABORATORY\\
{\large\bf Starlink Project\\}
{\large\bf \stardoccategory\ \stardocnumber}
\begin{flushright}
\stardocauthors\\
\stardocdate
\end{flushright}
\vspace{-4mm}
\rule{\textwidth}{0.5mm}
\vspace{5mm}
\begin{center}
{\Large\bf \stardoctitle}
\end{center}
\vspace{20mm}
\begin{center}
{\Large\bf Summary}
\end{center}
This is a `partial' release of ADAM -- its main features are:
\begin{itemize}
\item Enhanced ICL:
\begin{itemize}
\item Much improved error reporting and recovery when tasks fail -- notably
when they fail to load.
\item Setting default directory for loaded tasks.
\item Improved access to information.
\item Optional passing of unrecognized commands to the DCL subprocess.
\end{itemize}
\item Enhanced parameter system:
\begin{itemize}
\item ACCEPT at a prompt.
\item Any string as a `name'.
\item Improvements when `putting' array parameter values to the parameter file.
\item A conceptual split of output into message and error streams.
\item Re-use of dynamic storage.
\end{itemize}
\item Enhancements to SMS.
\item Use of the portable help system.
\item Bugs fixed in the Parameter System, ICL, MESSYS, TASK, UFACE and SMS.
\end{itemize}

N.B. TASKS WHICH HAVE NOT BEEN RE-LINKED WITH ADAM V2 SHOULD NOW BE RE-LINKED
-- apart from not providing a number of useful enhancements, the old built-in
fixed-part is no longer valid for all cases.

\newpage
%------------------------------------------------------------------------------
%  Add this part if you want a table of contents
  \setlength{\parskip}{0mm}
  \tableofcontents
  \setlength{\parskip}{\medskipamount}
  \markright{\stardocname}
%------------------------------------------------------------------------------

\section{INSTALLATION}
Full installation instructions are given in SSN/44 and the Starlink Software
Change Notice.

The full system requires about 38000 blocks of disk storage and includes a
mini-system which can be extracted and put up separately. The mini-system
requires about 12000 blocks and allows tasks to be run and  developed.

\section{NEW FEATURES OF ICL}
Version 1.6 of ICL is released. The new and changed features are described
below. SG/5 will be re-issued separately in the near future.

\subsection{Control of ADAM Tasks}
The task control logic of ICL has been extensively rewritten to cure various
problems. ICL now uses the VMS termination mailbox facility to accurately
track the failures of tasks initiated by this incarnation of ICL. This
mechanism does not work with detached tasks but they are entered into the task
tables when first used. Users should get more accurate error messages when
tasks abort and will not experience the long delays seen at present.\\
(ICLADAM.PAS)
\subsection{Improved DCL Subprocess Control}
It is now possible to send a CTRL/C to a DCL subprocess command
({\em e.g.}\ MAIL)
correctly and without interfering with ICL (which sometimes used to hang).
Also a CTRL/Y typed to the DCL subprocess will abort only the subprocess and
not ICL.\\
(ICLIO.PAS)

\subsection{Error Handling}
\label{errs}
A number of improvements to error handling within ICL have been made.
\begin{itemize}
\item ICL now uses EMS internally and will process EMS
errors from HDS correctly.\\
(Various modules)
\item Internally ICL now holds its own error status and any status it can
derive from external functions in one global location.
This can be accessed by the user in
one of several formats using a new function, SYSERR (see
Section \ref{syserr}). The ICL exception
mechanism and exception names are left unchanged for compatibility reasons.
Use
of SYSERR after an exception should provide more information on what
occurred.\\
(All modules)
\item Numerous other improvements to error messages to improve clarity and
accuracy.\\
(Mostly ICLADAM.PAS)
\end{itemize}

\subsection{Miscellaneous Other Improvements to ICL}
\label{iclmisc}
\begin{itemize}
\item ICL.SDF is now called ICL\_\verb%<process_id>%.SDF and
created in
ADAM\_USER by default (or in old location of SYS\$SCRATCH if ADAM\_USER is not
defined). The file is deleted at ICL exit.
This should avoid conflicts when several copies of ICL are run from
one username.\\
(ICLMAIN.PAS)
\item The DECW\$DISPLAY logical name will be copied into the LNM\$JOB table
on ICL startup.
This avoids a problem whereby graphics could work from tasks started
from DCL but not from ICL.\\
(Routine ICL\_INIT in ICLMAIN.PAS)
\item ICL may now optionally use portable help libraries (ref.\ SUN/124)
both for the HELP command and to provide multi-line parameter
help in response to ? or ?? at parameter prompts.

If the given library file extension is .SHL, the portable system will be used.
If the library file extension is omitted or .HLB, the system will look for a
file with the given name and extension .SHL. If one is found, the portable
system will be called to use it: if not, VMS help will be used.

In the interests of portability, library names may include logical names in
either of the following forms:
\begin{quote} \begin{verbatim}
logical_name:filename
\end{verbatim} \end{quote}
or
\begin{quote} \begin{verbatim}
$logical_name/filename
\end{verbatim} \end{quote}
where \verb%filename% is optional.\\
(See also Section \ref{uface}, UFACE)
\end{itemize}

\subsection{Modified Commands}
\begin{description}
\item[DEFAULT] may now be used to set the default directory for loaded tasks.
This involves a related change to the DTASK fixed part
(see Section \ref{dtask}).
\begin{quote} \begin{verbatim}
DEFAULT [directory [taskname]]
\end{verbatim} \end{quote}
will set the default directory for task \verb%taskname% to \verb%directory%.

If \verb%taskname% is omitted, the default directory for ICL, the DCL
subprocess and any cached tasks will be set.

If \verb%directory% is omitted, the current default directory for ICL will be
displayed.\\
(ICLPROC.PAS, ICLIO.PAS and DTASK library)
\item[GET and PUTNBS]
will now handle variable length strings.\\
(ICLNBS.PAS)
\item[SET (NO)MESSAGES]
SET NOMESSAGES will cause fewer messages than before and will allow DEFSTRINGs
(for example) to be redefined silently.\\
(ICLUSER.PAS)
\item[PROCS]
The PROCS  command will output the list in alphabetical order and a selective
listing of procedures is now possible.
\begin{quote} \begin{verbatim}
PROCS [string]
\end{verbatim} \end{quote}
If \verb%string% is specified, only those procedures whose names begin with
`\verb%string%' will be listed.\\
(ICLMAIN.PAS, ICLPROC.PAS)
\item[KILL and KILLW]
Users can now refer to tasks by their DEFINEd command name rather
than their full name ({\em e.g.} KILL KAPPA rather than
KILL KAPPA\_DIR:KAPPA).\\
(ICLADAM.PAS)
\item[DEFSTRING, DEFUSER, DEFPROC]
Any conflict between the command names defined by these commands or
a defined PROCedure name will be reported, although the new definition
will remain.
These commands can now be `undone' by re-entering them with only the first
(command) parameter.\\
(ICLUSER.PAS)
\item[PROC]
Any conflict between the PROCedure name and a command name defined by
DEFSTRING, DEFUSER or DEFPROC  will be reported.\\
(ICLMAIN.PAS)
\item[SET PROMPT] The tokens \$DATE and \$TIME may now be included in the
string defining the required prompt. The date and time at the time of the
prompt will be substituted.
\begin{quote} \begin{verbatim}
SET PROMPT '$DATE $TIME ICL >'
\end{verbatim} \end{quote}
(ICLMAIN.PAS)
\item[EDIT] In addition to editing procedures, this command may now be used to
edit normal files without creating a subprocess. If the parameter contains `.',
it is assumed to be a filename rather than a procedure, {\em e.g.\ }
\begin{quote} \begin{verbatim}
ICL> EDIT MYDIR:MYFILE.TXT   ! edits a file
ICL> EDIT MYFILE             ! edits procedure MYFILE
\end{verbatim} \end{quote}
The editor selected by default or the SET EDITOR command will be used.

There are some differences from using the editors in DCL:
\begin{enumerate}
\item Only one parameter may be given.
\item The editors assume a change has been made so always create a new file.
\end{enumerate}
(ICLPROC.PAS)
\end{description}

\subsection{New Commands}
\begin{description}
\item[STRINGS]
This will list, in alphabetical order, the strings defined by DEFSTRING,
DEFUSER, DEFHELP or DEFPROC.
\begin{quote} \begin{verbatim}
STRINGS [string]
\end{verbatim} \end{quote}
If \verb%string% is specified, only those strings whose names begin with
`\verb%string%' will be listed.\\
(ICLUSER.PAS)
\item[VERSION]
Prints the ICL version number and creation date.\\
(ICLPROC.PAS)
\item[SET (NO)AUTODCL]
Unrecognized ICL commands will, by default, be passed through to DCL. To avoid
spelling mistakes accidentally starting the DCL subprocess the user is reminded
what to do if the DCL subprocess is not active.
The default is AUTODCL.\\
(ICLPROC.PAS)
\item[SET (NO)COMMENTS] SET NOCOMMENTS will cause ICL not to save comments
when procedures are loaded. In large systems this can cut down considerably
the memory required and the number of page faults.
The default is SET COMMENTS so there is no change in behaviour by default.\\
(ICLMAIN.PAS, ICLDEF.PAS)
\item[SET (NO)CTRLZ]
If SET NOCTRLZ is in effect it should be impossible to escape from ICL by
typing CTRL/Z. In this case an explicit EXIT is required.
The default is CTRLZ.\\
(ICLPROC.PAS)
\end{description}

\subsection{New Function SYSERR()}
\label{syserr}
The global status code can be returned as a number or any (or all) of the
translation of this number through LIB\$SYS\_GETMSG
(See also Section \ref{errs}.)
\begin{quote} \begin{verbatim}
ERR = SYSERR([part])
\end{verbatim} \end{quote}
\verb%part% may be:
\begin{description}
\item[omitted or 'default'] A string in the form {\em fac\_\_ident} will be
returned.
\item['full'] The translated message is returned in full.
\item['facility'] The facility name for the message is returned.
\item['ident'] The ident for the message is returned.
\item['severity'] The severity of the message is returned.
\item['value'] The status code is returned as an integer.
\end{description}
It should be borne in mind that use of this facility may not be fully portable
and some packages do not have messages associated with their status values.\\
(ICLFUNC.PAS)

\subsection{ICL Procedures {\em etc.}}
\begin{description}
\item[LOGIN.ICL, ADAMLOGIN.ICL]
The login banner has been changed to display the new version number, 1.6.
\item[ICLHELP] The changes described above have been reflected in the ICL help
library.\\
(ICLHELP.HLP, ICLHELP.HLB)
\item[LINK\_ICL] The procedure for linking ICL has been changed to pick up all
libraries {\em etc.\ } from the appropriate source directory -- previously some
libraries were picked up from ADAM\_LIB.
The HLP shareable library is also included in the link.
\item[SMSLINK] The procedure for linking SMSICL has been renamed to SMSLINK\_ICL
and changed to pick up ICLSMS from the ICL object library and certain UFACE
routines from the ICL UFACE library rather than the SMS UFACE library (see
Section \ref{uface}).
Furthermore, the image produced is named SMSICL.EXE rather than just SMS.
\item[COMPILE\_ICL] Allow for revised format for ICL library.
\end{description}

\section{NEW FEATURES OF DTASK}
\label{dtask}
\subsection{CONTROL Context}
The task fixed part has been modified to cater for a new type of context
message, namely CONTROL context. Currently the only CONTROL context message
is to set the default directory for the task.\\
(Modified: DTASK\_DTASK, \_GSOC, DTASK\_ERR, ADAMDEFNS\\
New: DTASK\_CONTROL)

\subsection{Include Files}
\begin{itemize}
\item The names of the DTASK include files have been changed to DTASK\_xxx.FOR.
This is to assist in the porting exercise.\\
(DTASK.PAR, DTASK.CMN, DTASK.SYS)
\item An additional constant, DTASK\_\_SYSNORM is defined (see Section
\ref{port}).\\
(DTASK\_SYS)
\end{itemize}

\subsection{Revised Error Messages}
Several error messages have been split to provide neater displays than are
provided by the simple wrapping employed by ERR.\\
(DTASK\_DTASK)

\subsection{Portability}
\label{port}
To enable the same code to be used as far as possible on both VMS and Unix
systems, the following changes have been made:
\begin{itemize}
\item The task of obtaining the parameter section of the command line when a
task is activated direct from the operating system is separated out into a new
(system-specific) routine.\\
(Modified DTASK\_DCLTASK, DTASK\_ERR; New DTASK\_GTCMD)
\item The task of including the message associated with a status value within
an error message is now done in two stages. The special syntax \verb%^STATUS%
is not used.
In this way the fact that there is no message associated with status
values on Unix can be handled whilst retaining the existing behaviour on VMS.
A new (system-specific) routine is provided to set a normal token to the
appropriate message.
In the Unix version, the message will be the status value.\\
( Modified DTASK\_DCLTASK, \_DTASK, \_CANCEL, \_COMSHUT,
\_RESCHED, \_SUBSID; New DTASK\_ESETK)
\item An additional constant, DTASK\_\_SYSNORM is defined in DTASK\_SYS.
It is set to 1, the value of VMS SS\$\_NORMAL. The application reschedule
request is tested against this constant, rather than SS\$\_NORMAL, at the end
of an obey action to trap the special case which may occur with some tasks on
VMS.\\
(DTASK\_OBEY, \_OBEYDCL, DTASK\_SYS)
\end{itemize}

\section{NEW FEATURES OF THE PARAMETER SYSTEM}
\subsection{ACCEPT at a Prompt}
It is now possible to reply to a prompt with `\verb%\%'.
The effect will be to take the suggested (prompt) value for the current
parameter and for all remaining unset parameters.

{\em Note that this overrides the VPATH for the remaining parameters.}\\
(Modified: SUBPAR\_INPUT)

\subsection{Strings as Names}
SUBPAR\_GETNAME will now accept a quoted string as the name to be obtained.
This is primarily to allow lists of names to be supplied for the forthcoming
GRP library.
It also allows dynamic defaults set using SUBPAR\_DEF0C.

{\em Note that the above does not apply to names to be associated by DAT\_ASSOC
or NDF\_ASSOC.}

Names specified as default in the interface file may now be enclosed in
quotes. How the string is interpreted will depend on how the parameter value
is obtained.\\
(Modified: SUBPAR\_GETNAME, \_CHECKNAME, PARSECON\_CONVERT)

\subsection{Putting Array Parameter Values}
It is no longer necessary for the parameter file component associated with a
parameter to be the correct size for a `put'. The PAR\_PUT routines
will now change the size if associated with the parameter file -- they will
not change the size of an external object.

This makes it possible to avoid being prompted {\em and} have global parameters
updated by specifying:

\begin{quote} \begin{verbatim}
vpath   default
default n
\end{verbatim} \end{quote}
where \verb%n% is a constant of an appropriate type, in the interface file,
Appendix C of SUN/115 will be modified to reflect this shortly.\\
(Modified: SUBPAR\_PUT1/V/N.GEN)

\subsection{Message Streams}
This version of ADAM is linked with version 1\_2-3 of EMS which will be
released at the same time. (See SUN/104.5 for details of changes to
MSG/ERR/EMS.)

The ADAM versions of MSG and ERR have been changed to output their messages
using two new routines, SUBPAR\-\_WRMSG and SUBPAR\-\_WRERR, respectively.
Currently both these routines call SUBPAR\-\_WRITE so there is effectively no
change.

A new routine, SUBPAR\_EFLSH, has been written to flush error messages from
within SUBPAR -- it uses SUBPAR\_WRERR
Various routines have been modified to use SUBPAR\_EFLSH.\\
(New: SUBPAR\_WRMSG, \_WRERR, \_EFLSH\\
(Modified: SUBPAR\_GETNAME, \_FINDHDS, \_GET0C/D/I/L/R)

\subsection{Dynamic Default Storage}
Pointers to the common block areas used for storing dynamic defaults
{\em etc.\ }
at run time are reset at SUBPAR de-activation to avoid overflowing the
available space after multiple task invocations, particularly with large
monoliths.\\
(Modified: SUBPAR\_RDIF, \_DEACT, \_DATDEF, \_DEF1C/D/I/L/R, \_HDSDYN, \_INPUT,
\_GETNAME, SUBPAR\_CMN, PARSECON\_NEWPAR)

\subsection{Portable Help System}
\label{porthelp}
When a task is run direct from DCL, the parameter system provides the same
optional use of the portable help system for multi-line parameter help as is
provided by UFACE for user interfaces.
Existing interface files will continue to work, using the portable help system
if a .SHL file is found. (See Section \ref{iclmisc} for more details.)\\
(Modified: SUBPAR\_WRHELP, SUBPAR\_ERR\\
New: SUBPAR\_PWHLP, \_VWHLP, \_NAMETR, \_HLPEX)

\subsection{Futures}
Some additions have been made to the SUBPAR global constants and common blocks
in preparation for the MIN/MAX system.\\
(Modified: SUBPAR\_PAR, SUBPAR\_CMN)

\section{OTHER CHANGES TO ADAM}

\subsection{UFACE}
\subsubsection{ICL UFACE}
\label{uface}
\begin{itemize}
\item Changes have been made to handle the new (CONTROL) context.\\
(UFACE\_MSGINFO, \_SEND)
\item Changes have been made to define termination mailboxes for the improved
task failure reporting system. This includes additional arguments in two
routines.\\
(UFACE\_LOAD, \_LOADD)
\item An improvement has been made to send only the `used part' of message
strings.\\
(UFACE\_SEND)
\item The returned value will no longer be displayed if it is the same as
the sent value when an error occurs on SET or GET contexts.
(UFACE\_SEND)
\item UFACE\_SEND will now correctly handle information (error) messages
returned from the task during transactions it handles (GET, SET {\em etc}).\\
(UFACE\_SEND)
\item The system will now handle help from either portable help or VMS help
libraries. For details of the help library selection process see Section
\ref{iclmisc}.\\
(Modified: UFACE\_WRHELP, \_IPUT, \_OPUT,\\
New: UFACE\_PWHLP, \_VWHLP, \_SCRNSZ)
\item The AST handler is changed to use MESSYS\_ASTMSG rather than
MESSYS\_ASTINT which is now obsolete.\\
(UFACE\_ASTHDLR)
\end{itemize}
\subsubsection{SMS UFACE}
\begin{itemize}
\item The changed calling sequence for UFACE\_LOAD and UFACE\_LOADD implies
changes to ADAMCL, which is effectively frozen.
As a consequence of this, the SMS versions of these have not been changed but
the procedure (in ICL\_DIR) for linking
ICLSMS has been modified to take the LOAD routines from ICL UFACE.\\
(ICL\_DIR:SMSLINK\_ICL.COM)
\item The changes listed for UFACE\_SEND in the ICL UFACE library have been
made to the SMS UFACE library. In addition, because the SMS version now
calls SMS\_INFORM to handle returned information messages, it has diverged from
the other two versions (which call UFACE\_INFORM).\\
(UFACE\_SEND)
\end{itemize}
\subsubsection{ADAMCL UFACE}
\begin{itemize}
\item The modifications to UFACE\_SEND has been applied (but note that ADAMCL
{\em etc.}\ have not been re-linked to use it, nor will the new context be
used).\\
(UFACE\_SEND)
\item Error codes UFACE\_\_IFNF and UFACE\_\_HLPER have been added for use
with the help system and
the message associated with the status UFACE\_\_ILLCONTEXT has been
changed to allow for the new context.\\
(UFACERR.MSG/.OBJ/.FOR)
\item The AST handler is changed to use MESSYS\_ASTMSG rather than
MESSYS\_ASTINT which is now obsolete.\\
(UFACE\_ASTHDLR)
\end{itemize}
{\em Note that ADAMCL etc.\ have not been re-linked to include the above
changes.}

\subsection{SMS}
The following changes have been made to the SMS library.
\begin{description}
\item[Pre-entry] The pre-entry actions for switch fields are now carried out
BEFORE any  values for parameters are obtained using the `from' statement.
This potentially allows the latter values to be updated by carrying out an
action of some sort.  This change could affect the behaviour of existing systems
but is desired by both UKIRT and JCMT who are believed to be the only serious
users of SMS, so has been included.\\
(SMS\_NEWSWITCH)
\item[New Configuration Parameters] There are two new configuration parameters.
They may be set using the configuration switch (type keypad `,') or by the
+config+ command in a control table. The initial settings preserve the existing
behaviour.
\begin{description}
\item[pf4\_prompt]
If turned on, this causes a prompt `Do you mean to ABORT this menu? /N/' if
PF4 is keyed.
If the default N is returned the menu is not exited. If `Y' is entered then
the menu is exited.  This is to stop `illegal' use of PF4 (for instance where
PF1 exit does essential things).
\item[topexit\_prompt] If turned off, this suppresses the `Do you mean to exit
from SMS ? (y/n)' prompt on exiting from the top menu.
\end{description}
(SMSCOMTIO, SMS\_COMNINIT, \_SYSCONFIG, \_PFKEY, \_RUN)
\item[EDIT] The ICL EDIT command is treated as a `special' command in the same
way as HELP. This means deassigning the terminal from SMS so that the editor
can use it. The terminal is re-assigned to SMS and the screen refreshed on
exit from the editor.\\
(SMS\_DOCMDLINE)
\item[UFACE Equivalent] The equivalent routine to UFACE\_MSGINFO for SMS has
been changed in line with UFACE.\\
(SMS\_MSGINFO)
\item[Stopping HDS] As the ICL exit handler now uses HDS (to delete the ICL.SDF
file), the call to HDS\_STOP in SMS has been removed.\\
(SMS\_MAIN)
\item[AST Handler] The AST handler is changed to use MESSYS\_ASTMSG rather than
the obsolete MESSYS\_ASTINT.\\
(SMS\_ASTHDLR)
\item[HELPSYS] An additional possible status value from HELPSYS is allowed
for.\\
(SMS\_BRIEFHELP)
\item[SCT Include files] The new configuration parameters have been added to
the configuration switch in the include files defining the `system' part of
control tables.
The changes are also reflected in the modified include files
AON004.INC and AON004ICL.INC in directory ADAM\_TEST.\\
(FIXED.SCT, SMSSYSTAB.INC, ICL\_DIR:SMSICLSYSTAB.INC)
\item[SMS\_HELPFILE] Both the ADAM HELPSYS and VMS HELP versions of
SMS\_HELPFILE have been updated. The source file and a HELPINFO file describing
how to produce the help files have been moved from [ADAM.HELP] to SMS\_DIR
\item[SMSLINK] SMSLINK.COM is renamed SMSLINK\_ADAMCL.COM and produces
executable image SMSADAMCL.EXE.
The procedure has also been updated to use the ISD\_MAX link option.
\end{description}

\subsection{TASK}
\begin{itemize}
\item The new context name (CONTROL) has been added to the options.\\
(TASK\_GET\_CONTEXTNAME)
\item The names of the TASK include files have been changes to TASK\_xxx.FOR.
This is to assist in the porting exercise.\\
(TASK.PAR, TASK.CMN)
\item Several unnecessary files have been removed from the TASK directory.\\
(DRIVER.OBJ, TASKTEST.OBJ, TASKTEST.IFC)
\end{itemize}

\subsection{MESSYS}
\begin{itemize}
\item MESSYS\_ASTINT is now actually obsolete. For the moment it has been
re-implemented using MESSYS\-\_ASTMSG.
\item MESSYS\_CLRESCH and MESSYS\-\_REMNAME are now obsolete and removed from
the library. The ADAMSHARE transfer vector for MESSYS\-\_CLRESCH now uses the
`obsolete' feature (see Section \ref{shr}).

THIS MEANS THAT ANY TASKS WHICH HAVE NOT BEEN RE-LINKED WITH ADAM V2.0 OR
LATER, SHOULD NOW BE RE-LINKED AS THE OLD BUILT-IN FIXED-PART IS NO LONGER
VALID IN ALL CASES.
\item The MESSYS parameter file is renamed to MESSYS\_PAR.FOR. At the same
time,
parameter MESSYS\_\_NETNAME (= 'ADAMNET'), the root name of network processes,
is defined. The new parameter with '\_2', '\_3' and '\_4' appended is used
to define the names of the four possible systems.\\
(MESSYS\_PAR.FOR, MESSYS\_INIT, LOGICAL.COM)
\end{itemize}

\subsection{FIO/RIO}
The FIO/RIO library has been released as a separate Starlink software item.
The new release has changed and enhanced functionality and uses a shareable
image with different linking instructions.

The existing system will remain within the ADAM release for a short while to
allow users to make any necessary changes to their software. The old linking
instruction will link with the old library but ADAMDEV will obey FIO\_DEV to
set up logical names FIO\_PAR and FIO\_ERR pointing to the new system (if it
is installed).
The FIO shareable image will be searched automatically by ALINK, MLINK and
ICLMLINK.

For full details of the new facilities, see SUN/143.

\subsection{SHR}
\label{shr}
A new library, SHR,  has been created in SHARE\_DIR to contain routines
for use in building shareable images containing obsolete or not-yet-implemented
routines. The principle is that the shareable images will be built without
defining universal symbols for the missing routine so that it will not be
possible to link with the new image. However, if a task was linked with a
previous image which did define the symbol, and the symbol is accessed, some
error messages will be reported and a bad status returned.\\
(SHARE\_DIR:SHR.OLB/.TLB, SHR\_ERR.MSG/.FOR/.OBJ)

\subsection{AZUSS}
A dummy version of the AZUSS shareable image has been created to satisfy
requests from sites which do not find it convenient to have to INSTALL the
shareable images, particularly with privilege. The dummy image contains only
a dummy for AZ\$SNDAST which, if it is called, will report some error messages
and return a bad status. It makes use of the new SHR library.

The dummy image can be used with any tasks which do not require to trigger an
AST in another task. To use it:
\begin{quote} \begin{verbatim}
$ DEFINE/x AZUSS ADAM_EXE:AZUSS
\end{verbatim} \end{quote}
Where \verb%x% is JOB, GROUP or SYSTEM.\\
(AZUSS\_DIR:AZUSS\_DUM.MAR/.COM, ADAMEXE\_DIR:AZUSS\_DUM.EXE)

\subsection{Procedures {\em etc.}}
\begin{description}

\item[ADAMSTART.COM] Changes are:
\begin{itemize}
\item The ADAM version number has been updated. So that
only a single change to the procedure is required to update the version
number, the latest number is set as a symbol and the symbol used wherever the
version number is needed in the procedure,
{\em i.e.\ }:
\begin{itemize}
\item In the displayed messages.
\item In setting the ADAM\_CHANGES command.
\item In setting the ADAM\$\_VERSION logical name.
\end{itemize}

The version number is set to `2.0-2'.

\item SMS and SMSICL are redefined to pick up the renamed executable images,
SMS\-ADAMCL and SMSICL.
\end{itemize}

\item[ADAMSHARE]
ADAMSHARE.COM and ADAMSHARE.MAR are altered to reflect:
\begin{itemize}
\item An increased minor id.
\item An additional common block for SUBPAR.
\item Use of the portable help system shareable library.
\item The PSX library.
\item Additional routines SUBPAR\_WRERR, \_WRMSG, \_NAMETR and \_HLPEX.
\item Additional routines ICL\_TERM\_CANCIO and ICL\_TERM\_ENABLE\_CTRLC
(in ICLTERMADAM).
\item The new `obsolete' feature and MESSYS\_CLRESCH declared obsolete.
\end{itemize}

\item[LINKNOSHR.OPT]
The options file used for linking tasks `non-shareable' has been changed:
\begin{itemize}
\item In line with ADAMSHARE.COM.
\item To link with the EMS shareable image to avoid conflicts with other
libraries.
\end{itemize}

\item[APPLOG.COM, LOGICAL.COM and DIR.COM]
The procedures for setting up process logical names for development have been
changed as follows:
\begin{itemize}
\item Logical names for FIO/RIO are no longer defined.
\item The names of include files for DTASK and TASK have been changed.
\item .FOR is added to the definition for MESSYS\_ERR.
\end{itemize}
\item[TASK\_LINK.COM] The procedure which does the work for ALINK etc.\ has been
modified to use DELETE/NOCONFIRM during cleanup.
\item[DEVDATASET] TMSCP device names (node\_name\$device\_name) appropriate for
RAL have been added to the DEVICE7 files.
\end{description}

\subsection{Tasks}
Several tasks released with ADAM in [ADAM.RELEASE.EXE] which had not been
re-linked since before ADAM V2.0 have now been re-linked. This is necessary to
allow the ICL `DEFAULT' command to work with them.\\
(LISTLOG, SDFCREATE, TAPECREATE, TAPEDISM, TAPEMOUNT)

\subsection{ADAM\_HELP}
A number of unnecessary files have been deleted from the help source directory
[ADAM.HELP]. SMS\_HELPFILE.HLP and INFO.LIS have been moved to SMS\_DIR.\\
(SMS\_HELPFILE.HLP, ADAMCL.AHF, ADAMCL.HEL, ADAMCL.LIS, CHECKTASK.HLP)

\subsection{Documentation}
\label{docs}
\begin{itemize}
\item SSN/45 and ARN/2.0-2 describe ADAM release 2.0-2.
\item The ADAM\_CHANGES help library has been updated
with the information contained in this document. The information for earlier
releases is
retained in the library.
\item The following applicable Starlink documents have been updated since the
last ADAM release.
\begin{itemize}
\item SG/4 ADAM -- The Starlink Software Environment.
\item SUN/104 MSG and ERR -- Message and Error Reporting Systems.
\item SSN/4 EMS -- Error Message Service.
\end{itemize}
\item The following applicable Starlink documents have been released since the
last ADAM release:
\begin{itemize}
\item SUN/143 FIO/RIO -- FORTRAN File I/O Routines.
\end{itemize}
\item The following document from the old ADAM series has been withdrawn:
\begin{itemize}
\item APN/9 ADAM Programmers Guide to the FIO/RIO Package.
\end{itemize}
\item The following applicable Starlink document has been withdrawn:
\begin{itemize}
\item SUN/35 ICL -- Interactive Command Language: An Introduction.
\end{itemize}
\item The summaries, 0CONTENTS.LIS, FULLDOCS.LIS and NEWDOCS.LIS in
ADAM\-\_DOCS, have been updated.
\end{itemize}

\section{BUGS FIXED}

\subsection{SUBPAR}
\begin{itemize}
\item A zero length command line is now correctly checked for.\\
(SUBPAR\_CMDLINE)
\item A bug which caused command line array values for `internal' parameters
to be ignored has been corrected.\\
(SUBPAR\_CMDLINE)
\item A number of inconsistencies in handling arrays declared as `internal'
have been corrected.
\begin{itemize}
\item Formerly, if PAR\_GET1/N/Vx was called for an internal parameter and no
value (or a scalar value) was preset, the `current' (last used) value would be
used if the relevant component of the interface file existed. Any dynamic or
static default would be ignored. This has been corrected.\\
(SUBPAR\_FINDHDS)
\item The order of precedence in which the defaults are taken for `internal'
parameters obtained by PAR\_GET1/N/Vx has been switched to be the same as that
for `internal' parameters obtained by PAR\_GET0x, {\em i.e.} dynamic, static.\\
(PARSECON\_SETVP)
\item A scalar value may be given for an `internal' parameter obtained by
PAR\_GET1x. This is the same as for `normal' parameters.\\
(SUBPAR\_GET1.GEN)
\end{itemize}
\item 1-d suggested (prompt) values will be enclosed in [ ]. This allows
character arrays to be handled correctly.\\
(SUBPAR\_CONVALS)
\item A bug which prevented `internal' parameters taking dynamic defaults which
were specified in a type different from the declared type of the parameter, has
been corrected.\\
(SUBPAR\_FETCHC/D/I/L/R)
\item A bug which prevented global parameters being used as suggested values
for names on Unix has been corrected.\\
(SUBPAR\_NAMEASS)
\item Messages exceeding the maximum size which may be output to SYS\$OUTPUT
when running tasks direct from DCL are now truncated before attempting to
WRITE them.
Failures on the Fortran WRITE are trapped and status SUBPAR\_\_OPTER returned.\\
(SUBPAR\_WRITE, SUBPAR\_ERR)
\item A bug which caused an HDS object slice such as ARRAY(x,y1:y2) to be
interpreted as ARRAY(1:x,y1:y2) instead of ARRAY(x:x,y1:y2), has been
corrected.\\
(SUBPAR\_HDSARR)
\item A bug which caused RESET to have no effect on the VPATH for some
name-type parameters ({\em e.g.\ }FIO filenames) has been corrected.\\
(SUBPAR\_GETNAME)
\item Numerous other minor changes to error reports and layout
{\em etc.\ }have been made.
\end{itemize}

\subsection{PARSECON}
A bug which prevented task names in INTERFACE or MONOLITH statements in the
interface file being of the form `D' or `E' possibly followed by
an integer, has been corrected. The mod also fixes a problem on Suns whereby a
single-letter name could not be specified.\\
(PARSECON\_TOKTYP)

\subsection{TASK}
A bug which caused the TASK encode and decode routines to fail to cooperate
over LOGICAL values has been corrected. The solution makes the `L' routines
different from the other types so GENERIC can now only be used for types C,D,I
and L.\\
(TASK\_ENC0L/1L/NL, TASK\_DEC0L/1L/NL, TASK\_VAL0L/1L/NL, TASKDEV.COM)

\subsection{ICL}
\begin{itemize}
\item Failure to initialize the message system is now reported correctly
and will cause an ICL abort.\\
(ICL.PAS)
\item Correction to the code such that spurious (non-fatal) "invalid HDS
locator" messages are eliminated.\\
(ICLDEF.PAS)
\item Various points of failure (ICL crashes) caused by zero length character
strings have been eliminated.\\
(ICLMAIN.PAS, ICLDEF.PAS)
\item A bug in the processing of quoted strings ({\em e.g.\ } CONTOUR ABSLAB=
"X (AXIS)") has been fixed.\\
(ICLPARSE.PAS)
\item ICL will now correctly handle error messages returned from the task
during SEND SET and SEND GET contexts. Furthermore, the returned value will
not be displayed on error if it is the same as the value sent.\\
(See section \ref{ufbugs} on UFACE\_SEND)
\end{itemize}

\subsection{MESSYS}
A bug which prevented ADAMNET\_4 being used has been corrected.\\
(MESSYS\_PATH)

\subsection{UFACE}
\label{ufbugs}
\begin{itemize}
\item The check on whether there was sufficient subprocess quota to start another
task was out by one. This has been corrected in both ICL and SMS UFACE
libraries.\\
(UFACE\_CHECKQUOT)
\item Information (error) messages returned from tasks during SET and GET
contexts will now be correctly handled.\\
(UFACE\_SEND)
\end{itemize}

\subsection{SMS}
\begin{itemize}
\item `\verb%+config+ refresh%' in an action will now correctly
refresh the SMS screen.\\
(SMS\_DORFSH)
\item A number of small bugs have been fixed - most (all?) were places where
zero-length strings could potentially create problems.\\
(SMS\_GETUSRLIN, \_INFORM, \_MOVECURS, \_MSGINFO, \_SUBSPARS,\\
\_SWITEXT, \_WRSCRL, \_WRUSER )
\end{itemize}

\subsection{ADAM\_TEST}
The `DEFINE TRACE' command in the example SMS control table has been
corrected.\\
(AON004.SCT)

\subsection{HELPSYS}
A bug in the ADAM help system has been fixed. It caused access violation if
the variable supplied to receive the brief help string was not long enough.
This caused SMS to crash occasionally when running CGS4.\\
(HELPSYS\_BRIEFGET, HELPMSG)

\section{PROPOSED DEVELOPMENTS}
\subsection{Parameter System}
It is intended that another ADAM release will be made in the near future to
include a number of improvements to the parameter system, notably:
\begin{itemize}
\item The ability to specify MIN or MAX as a parameter value, in which case
the system will take a specified minimum or maximum value.
\item The ability to specify the null value `!' on the command line.
\item PAR routines with the functionality of the AIF routines which have been
used by several packages but never officially released for use.
\end{itemize}

\subsection{STRING}
The STRING library is being considered for removal. It seems appropriate that
its general purpose functions should be transferred to portable code in the
CHR library and other more specific functions be incorporated into other
ADAM system packages.

\end{document}
