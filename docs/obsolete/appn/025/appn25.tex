\documentstyle{article}
\pagestyle{myheadings}

%------------------------------------------------------------------------------
\newcommand{\stardoccategory}  {ADAM Portability Project Note}
\newcommand{\stardocinitials}  {Starlink/APPN}
\newcommand{\stardocnumber}    {25.0}
\newcommand{\stardocauthors}   {B D Kelly}
\newcommand{\stardocdate}      {29 March 1994}
\newcommand{\stardoctitle}     {The Unix low-level ADAM libraries}
%------------------------------------------------------------------------------

\newcommand{\stardocname}{\stardocinitials /\stardocnumber}
\markright{\stardocname}
\setlength{\textwidth}{160mm}
\setlength{\textheight}{230mm} % changed from 240
\setlength{\topmargin}{-2mm}   % changed from -5
\setlength{\oddsidemargin}{0mm}
\setlength{\evensidemargin}{0mm}
\setlength{\parindent}{0mm}
\setlength{\parskip}{\medskipamount}
\setlength{\unitlength}{1mm}

%------------------------------------------------------------------------------
% Add any \newcommand or \newenvironment commands here
%------------------------------------------------------------------------------

\begin{document}
\thispagestyle{empty}
SCIENCE \& ENGINEERING RESEARCH COUNCIL \hfill \stardocname\\
ROYAL OBSERVATORY EDINBURGH\\
{\large\bf Computing Section\\}
{\large\bf \stardoccategory\ \stardocnumber}
\begin{flushright}
\stardocauthors\\
\stardocdate
\end{flushright}
\vspace{-4mm}
\rule{\textwidth}{0.5mm}
\vspace{5mm}
\begin{center}
{\Large\bf \stardoctitle}
\end{center}
\vspace{5mm}

%------------------------------------------------------------------------------
%  Add this part if you want a table of contents
%  \setlength{\parskip}{0mm}
%  \tableofcontents
%  \setlength{\parskip}{\medskipamount}
%  \markright{\stardocname}
%------------------------------------------------------------------------------

\section {Summary}

The low-level libraries (exh, atimer and msp) necessary to support ADAM
multitasking on Unix are described.


\section {Introduction}

The three libraries described here are necessary for implementing various
properties of adam tasks required when multitasking or when writing
software for instrumentation. The libraries are

\begin{itemize}
\item atimer - allows a library to specify a function to be called after a delay
\item exh - allows a library to specify a shutdown routine
\item msp - provides intertask communication primitives
\end{itemize}


\section {task exit handler, exh}

The "exit handler" is actually a signal handler responding to

\begin{itemize}
\item SIGHUP
\item SIGINT
\item SIGQUIT
\item SIGKILL
\item SIGTERM
\item SIGTRAP
\item SIGABRT
\item SIGEMT
\item SIGFPE
\end{itemize}

and it is assumed that adam tasks will normally exit as a result of one of
these. The application interface to exh consists of one routine:-
\begin{verbatim}

/*+  EXH_DECLARE - declare an exit handler */

void exh_declare
(
void (*exh)(),           /* handler routine (given) */
int *status              /* global status (given and returned) */
);

\end{verbatim}

exh\_declare() takes the address of a function as an argument. Its
operation is to push the address onto a stack. When one of the triggering
signals occurs, exh pops the functions of its stack one by one and
executes them. Exit routines are, therefore, executed in the reverse
order of declaration. The implication is that any library which uses
lower-level libraries should initialise them and give them a chance to
declare their exit routines before it declares its own routine.

exh\_declare() can return a single bad status EXH\_\_MEMFAIL, corresponding
to malloc() failing to get space for the new entry.

\section {exh implementation}

The exh stack consists of a linked list of structures containing pointers
to functions. The first time in, exh\_declare() uses signal() to declare
exh\_execute() as the signal handler for the signals listed above.
exh\_declare() uses malloc() to create new space for a new list entry as
needed. When one of the signals occurs, exh\_execute() is invoked by the
operating system, and it pops entries off the stack and invokes the
routines whose addresses are present.

\section {timer facility, atimer}

The atimer application interface consists of two routines, allowing a
caller to specify an application routine to be called after a time
interval specified in milliseconds. The caller also has to specify a
timer identifier which can be used to cancel a previously initiated
timer. If the timed interval is allowed to complete, the specified
application function is called and is passed the timer identifier as its
sole argument.

Note that the given timer identifier has to be unique within the process
(cf the VMS \$setimr facility).


\begin{verbatim}

/*+  ATIMER_CANTIM - remove an event from the timer queue */

void atimer_cantim
(
int timerid,    /* timeout identifier (given) */
int *status     /* global status (given and returned) */
);



/*+  ATIMER_SETTIMR - add an event to the timer queue */

void atimer_settimr
(
int delay,             /* time in millisecs (given) */
int timerid,           /* timer request identifier (given) */
void (*func)(),        /* address of associated routine (given) */
int *status            /* global status (given and returned) */
);

\end{verbatim}


\section {atimer implementation}

The timer facility, atimer, provides the necessary adam millisecond timer
by using the setitimer() facility. atimer maintains a linked list of
timer requests, each containing the extra time delay required for this
entry after delivery of the previous entry. Items are added to or removed
from the queue by searching for the relevant point, performing the
insertion or deletion (with associated malloc() or free()) and modifying
the time interval stored in the following list entry.

\section {message system primitives msp}

msp provides the low-level adam intertask communication facility. As
such, it is intended to support the higher levels of intertask
communication which implement the adam protocols. The result is that msp
must not be called as a separate facility within adam tasks.

The call interface of msp consists of the following routines.


\begin{verbatim}


/*+  MSP_CLOSE_TASK_QUEUE - close communications with another task */

void msp_close_task_queue
(
sendq_type qid,      /* a send queue to the other task */
int *status          /* global status (given and returned) */
);

/*+  MSP_CREATE_LOCALQ - create a queue for local messages */

void msp_create_localq
(
sendq_type *sendq,   /* created send queue (returned) */
receiveq_type *qid,  /* created receive queue (returned) */
int *status          /* global status (given and returned) */
);

/*+  MSP_CREATE_RECEIVEQ - create a queue for receiving messages */

void msp_create_receiveq
(
receiveq_type *qid,  /* created queue identifier (returned) */
int *status          /* global status (given and returned) */
);

/*+  MSP_DELETE_QUEUE - delete a queue */

void msp_delete_queue
(
receiveq_type qid,  /* identifier of queue to be deleted (given) */
int *status         /* global status (given and returned) */
);

/*+  MSP_ENTER_TASK - register this task with MSP */

void msp_enter_task
(
char *task_name,         /* name of this task (given) */
receiveq_type *commandq, /* command queue for this task (returned) */
int *status              /* global status (given and returned) */
);



/*+  MSP_GET_TASK_QUEUE - get the command queue of a named task */

void msp_get_task_queue
(
char *task_name,    /* name of task (given) */
sendq_type *qid,    /* task command queue (returned) */
int *status         /* global status (given and returned) */
);


/*+  MSP_RECEIVE_MESSAGE - receive a message on one of a list of queues */

void msp_receive_message
(
receiveq_type *qarr,  /* array of queue identifiers (given) */
int nqueues,          /* number of queues (given) */
int waitflag,         /* wait flag (given) */
int maxlen,           /* maximum length of message (given) */
char msgbody[],       /* received message (returned) */
int *actlen,          /* size of received message (returned) */
receiveq_type *qid,   /* identifier of queue used (returned) */
sendq_type *replyq,   /* reply queue for message (returned) */
int *status           /* global status (given and returned) */
);

/*+  MSP_SEND_MESSAGE - send a message on a queue */

void msp_send_message
(
char msgbody[],        /* message to be sent (given) */
int msglen,            /* length of message to be sent (given) */
sendq_type sendq,      /* queue identifer to be used (given) */
receiveq_type replyq,  /* reply queue to be associated with the message
                          (given) */
int *status            /* global status (given and returned) */
);

\end{verbatim}

A process which uses msp must have the environment variable ADAM\_USER
created to translate to a directory which msp can use for creating and
deleting "rendezvous" files. The first call to msp has to be to
msp\_enter\_task, providing msp with the name by which this process is to
be known to other processes. msp\_enter\_task returns a queue (the "command
queue") on which unsolicited messages from other tasks can be received.
Communication to another task is initiated using msp\_get\_task\_queue,
which returns a queue for sending messages to the target task's command
queue. Note that queues for sending and receiving messages are
fundamentally different and have different C types.

msp\_create\_queue creates additional queues on which messages can be
received. Messages are sent to another task using msp\_send\_message, and
providing the queue to be used for sending and one of this task's receive
queues which the other task can reply to. msp\_receive\_message allows a
process to attempt to receive a message on any one of a provided array of
receive queues. If the argument waitflag is set to 1, msp\_receive\_message
will wait until a message is received. If waitflag is zero,
msp\_receive\_message will return immediately if there are no messages on
any of the specified queues. msp\_receive\_message returns the identifier
of the receive queue actually used, and also the send queue which can be
used to return a message to the other task.

msp\_delete\_queue can be used to delete a receive queue created by
msp\_create\_queue.

msp\_close\_task\_queue can be used to close communications with another
task. If more than one send queue has been obtained to the other task,
msp\_close\_task\_queue should only be called once, specifying one of the
queues.

msp\_create\_localq is used to create a pair of send/receive queues which
enable a process to send messages to itself (eg from within an event
handler).


\section {msp example}

The following provides the source code of two programs which communicate
using msp. These serve both as an example and as a timing test.


\begin{verbatim}

/*  MASTER.C - send messages to SLAVE and collect replies */

#include <string.h>
#include <stdio.h>
#include <signal.h>
#include "../sae/sae_par.h"
#include "msp_par.h"
#include "msp.h"

int main()
{
   int status;
   sendq_type slavecom;
   receiveq_type commq;
   receiveq_type replyq;
   receiveq_type queues[1];
   char answer[512];
   int actlen;
   receiveq_type used;
   sendq_type slaveq;
   int j;


   status = SAI__OK;
   msp_enter_task ( "master", &commq, &status );
   msp_get_task_queue ( "slave", &slavecom, &status );
   msp_create_receiveq ( &replyq, &status );

   for ( j=0; j<10000; j++ )
   {
      msp_send_message ( "master calling", 15, slavecom, replyq, &status );
      queues[0] = replyq;
      msp_receive_message ( queues, 1, 1, 512, answer, &actlen, &used,
        &slaveq, &status );
      msp_receive_message ( queues, 1, 1, 512, answer, &actlen, &used,
        &slaveq, &status );
      if ( status != SAI__OK )
      {
         break;
      }
   }

   msp_delete_queue ( replyq, &status );

   if ( status != SAI__OK )
   {
      printf ( "master: bad status = %d\n", status );
   }
   else
   {
      answer[actlen] = '\0';
      printf ( "master: received - %s\n", answer );
   }

/*   Trigger exit handler */

   kill ( getpid(), SIGINT );

}



/*  SLAVE.C - receive messages and return two acknowledgements */

#include <string.h>
#include <stdio.h>
#include <signal.h>
#include "../sae/sae_par.h"
#include "msp_par.h"
#include "msp.h"

int main()
{
   int status;
   sendq_type mastercom;
   receiveq_type qarr[1];
   int actlen;
   receiveq_type qid;
   char message[512];
   int j;
   receiveq_type replyq;


   status = SAI__OK;
   msp_enter_task ( "slave", &replyq, &status );
   qarr[0] = replyq;

   for ( j=0; j<10000; j++ )
   {
      msp_receive_message ( qarr, 1, 1, 512, message, &actlen, &qid,
        &mastercom, &status );
      msp_send_message ( "slave answering", 16, mastercom, replyq, &status );
      msp_send_message ( "slave answering", 16, mastercom, replyq, &status );

      if ( status != SAI__OK )
      {
         break;
      }
   }


   if ( status != SAI__OK )
   {
      printf ( "slave: bad status = %d\n", status );
   }
   else
   {
      message[actlen] = '\0';
      printf ( "slave: received size %d mess %s\n", actlen, message );
      printf ( "slave: connection %hd queue %hd\n", mastercom.connection,
        mastercom.ack_queue );
   }

/*   Trigger exit handler */

   kill ( getpid(), SIGINT );

}


\end{verbatim}



\section {msp implementation}

Intertask communication is implemented within msp by using TCP/IP STREAM
sockets. These are point-to-point connections, so each pair of tasks
communicating are connected, and the concept of a number of message
queues available between the tasks is implemented by multiplexing over
the single connection. Actual message queues are implemented within the
receiving task as an array of pointers to linked lists. Messages read
from a socket are inspected for the queue number which they have to
carry, and are appended to the relevant queue. A "sendq" on the other
hand, identifies both the socket which the message has to be written on,
and the number to identify the receiving queue in the other task.

msp\_enter\_task operates by finding a free TCP port number in the range
5001 to 5099. It then creates a file with a name of the form
\$ADAM\_USER/<taskname>\_<portnum> and signifies to the socket system that
it is available for connections on the selected port number.
msp\_get\_task\_queue searches the \$ADAM\_USER directory for the file
matching the requested task name, and so can deduce the target task's
port number. It can then connect to the target task. The socket
connection opened between the two tasks can then be used for passing
messages. By convention, the command queue of a task is always queue
number zero, enabling communications to start.

msp\_create\_localq is distinct in its operation. It uses socketpair() to
create a pair of unix sockets, thereby giving the single process access
to both ends of the communication.

\section {Conclusion}

This summarises the low-level adam libraries. In general these are
provided to enable the higher-level adam libraries to be implemented and
so should not be called directly from an adam task's application code

\end {document}
