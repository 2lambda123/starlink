\documentstyle{article} 
\pagestyle{myheadings}

%------------------------------------------------------------------------------
\newcommand{\stardoccategory}  {ADAM Portability Project Note}
\newcommand{\stardocinitials}  {Starlink/APPN}
\newcommand{\stardocnumber}    {27.0}
\newcommand{\stardocauthors}   {B D Kelly}
\newcommand{\stardocdate}      {8 June 1994}
\newcommand{\stardoctitle}     {The Unix ADAM Message System}
%------------------------------------------------------------------------------

\newcommand{\stardocname}{\stardocinitials /\stardocnumber}
\markright{\stardocname}
\setlength{\textwidth}{160mm}
\setlength{\textheight}{230mm} % changed from 240
\setlength{\topmargin}{-2mm}   % changed from -5
\setlength{\oddsidemargin}{0mm}
\setlength{\evensidemargin}{0mm}
\setlength{\parindent}{0mm}
\setlength{\parskip}{\medskipamount}
\setlength{\unitlength}{1mm}

%------------------------------------------------------------------------------
% Add any \newcommand or \newenvironment commands here
%------------------------------------------------------------------------------

\begin{document}
\thispagestyle{empty}
SCIENCE \& ENGINEERING RESEARCH COUNCIL \hfill \stardocname\\
ROYAL OBSERVATORY EDINBURGH\\
{\large\bf Computing Section\\}
{\large\bf \stardoccategory\ \stardocnumber}
\begin{flushright}
\stardocauthors\\
\stardocdate
\end{flushright}
\vspace{-4mm}
\rule{\textwidth}{0.5mm}
\vspace{5mm}
\begin{center}
{\Large\bf \stardoctitle}
\end{center}
\vspace{5mm}

%------------------------------------------------------------------------------
%  Add this part if you want a table of contents
%  \setlength{\parskip}{0mm}
%  \tableofcontents
%  \setlength{\parskip}{\medskipamount}
%  \markright{\stardocname}
%------------------------------------------------------------------------------

\section {Summary}

The Adam Message System (AMS) library, which implements the Adam
intertask commications protocol under unix is described, along with its
FORTRAN-callable interface (FAMS). There is also a compatibility library,
MESSYS, which is FORTRAN-callable and provides the same interface as
VAX-Adam.



\section {Introduction}

The Adam Message System enables communications to be opened between adam
tasks, commands to be sent and received and acknowledgements to be sent
and received. Provided suitable Adam network processes are loaded, the
communication can take place across networks. In the absence of adamnet
processes, the communication is restricted to Adam tasks within a single
machine. The C-callable interface is implemented by the ams library. A
FORTRAN-callable interface is provided called FAMS. See APPN/26 for a
summary of the layering of the system.


\section {Communications directory}

Adam communications are opened between tasks by name. The name lookup is
provided by each task making an entry in a directory pointed to by the
environment variable ADAM\_USER. This variable has, therefore, to be
defined to point to an existing directory before any Adam tasks are loaded.

\section {task initialisation}

When a task initialises AMS by calling ams\_init(), it specifies its name
as a character string (say, "slave"). This results in a file being
created in the ADAM\_USER directory compounded of the name and an
identifying number (eg. \$ADAM\_USER/slave\_5001). Another task can then
open communications by searching ADAM\_USER for the right form of name and
using the identifying number to open communications.


\section {task exit}

When a task receives a signal causing it to exit, the message system exit
handler is envoked via the Adam exh facility. The result is that messages
are sent to any other Adam processes which have been in communication
with the exiting process, informing them of its removal. In addition the
task's file is removed from ADAM\_USER.

If the task cannot exit normally for some reason, the file in ADAM\_USER
may get left behind. In this case the file must be deleted explicitly
before the task is loaded again - if this is not done the task will
refuse to load and will give the reason why.


\section {opening communications}

Once a task has initialised itself successfully, it can open
communications with another initialised task by using ams\_path(). The
path number returned can then be used for further communications.


\section {sending a command}

Having obtained a path, a task can use ams\_send() to send a command to
the task identified by that path. If ams\_send() succeeds, it returns a
messid which remains valid for the duration of the command.


\section {getting expected replies}

Once a command is in progress between two tasks, they can receive replies
specific to that command, as identified by (path,messid) by using
ams\_getreply. Completion of the command and freeing of the associated
messid occurs when the task returns a message status which is not one of

\begin{itemize}
\item MESSYS\_\_PARAMREQ
\item MESSYS\_\_PARAMREP
\item MESSYS\_\_INFORM
\item MESSYS\_\_SYNC
\item MESSYS\_\_SYNCREP
\item MESSYS\_\_TRIGGER
\item DTASK\_\_ACTSTART
\end{itemize}

ams\_getreply() ignores all messages not associated with the specified
(path,messid) except those generated by the ams\_extint() function.


\section {receiving a command}

Any message, including commands or replies associated with an existing
command, can be received using ams\_receive(). The path and messid
associated with the command/reply are returned.


\section {sending expected replies}

A reply associated with an existing command has to be sent using
ams\_reply(), specifying the relevant path and messid. The command is
terminated by specifying the message status to be something other than

\begin{itemize}
\item MESSYS\_\_PARAMREQ
\item MESSYS\_\_PARAMREP
\item MESSYS\_\_INFORM
\item MESSYS\_\_SYNC
\item MESSYS\_\_SYNCREP
\item MESSYS\_\_TRIGGER
\item DTASK\_\_ACTSTART
\end{itemize}


\section {sending local messages}

A set of routines are provided to enable a task to send messages to itself.
ams\_kick() is intended to enable main-line code to queue a message which
it can subsequently collect by calling ams\_receive(). The other functions
are intended only for calling from within signal handlers. ams\_resmsg()
ams\_astmsg and ams\_astint() generate messages to be read by ams\_receive()
but ignored by ams\_getreply(). ams\_extint() generates a message which can
be received either by ams\_receive or ams\_getreply (its main purpose is to
allow user interfaces to be implemented).


\section {example}

The example consists of a pair of C programs called master and slave.
They should be run in background by

\begin{verbatim}
..> slave.a &
..> master.a &
\end{verbatim}

The code for master is:

\begin{verbatim}
#include <stdio.h>
#include <sys/types.h>
#include <sys/time.h>
#include <string.h>
#include <signal.h>
#include <errno.h>
#include <unistd.h>

#include "sae_par.h"
#include "ams_adampar.h"
#include "dtask_err.h"             /* dtask error codes */

#include "../messys/messys_err.h"
#include "../messys/messys_par.h"
#include "../messys/messys_dd.h"

#include "ams_sys.h"
#include "ams_mac.h"
#include "ams_struc.h"
#include "ams.h"

int main()
{
   int outmsg_status;
   int outmsg_function;
   int outmsg_context;
   int outmsg_length;
   char outmsg_name[32];
   char outmsg_value[MSG_VAL_LEN];
   int inmsg_status;
   int inmsg_context;
   int inmsg_length;
   char inmsg_name[32];
   char inmsg_value[MSG_VAL_LEN];
   int status;
   int path;
   int messid;
   int j;

   status = 0;
   outmsg_status = SAI__OK;
   outmsg_function = MESSYS__MESSAGE;
   outmsg_context = OBEY;
   outmsg_length = 16;

   strcpy ( outmsg_name, "junk" );
   strcpy ( outmsg_value, "master calling" );

   ams_init ( "master", &status );
   if ( status != SAI__OK )
   {
      printf ( "master - bad status after ams_init\n" );
   }
   ams_path ( "slave", &path, &status );
   if ( status != SAI__OK )
   {
      printf ( "master - bad status after ams_path\n" );
   }
   else
   {
      printf ( "master - path set up ok\n" );
   }

   for ( j=0; j<1000; j++ )
   {
      ams_send ( path, outmsg_function, outmsg_status, outmsg_context, 
        outmsg_name, outmsg_length, outmsg_value, &messid, &status );
      ams_getreply ( MESSYS__INFINITE, path, messid, 32, MSG_VAL_LEN, 
        &inmsg_status, &inmsg_context, inmsg_name, &inmsg_length, 
        inmsg_value, &status );
      ams_getreply ( MESSYS__INFINITE, path, messid, 32, MSG_VAL_LEN, 
        &inmsg_status, &inmsg_context, inmsg_name, &inmsg_length, 
        inmsg_value, &status );
   }
   if ( status != 0 )
   {
      printf ( "master: bad status = %d\n", status );
   }
   else
   {
      printf ( "master: received - %s\n", inmsg_value );
   }

   kill ( getpid(), SIGINT );
   return 0;
}


\end{verbatim}

The code for slave is:

\begin{verbatim}


#include <stdio.h>
#include <time.h>
#include <sys/types.h>
#include <sys/time.h>
#include <string.h>
#include <signal.h>
#include <errno.h>
#include <unistd.h>

#include "sae_par.h"
#include "ams_adampar.h"
#include "dtask_err.h"             /* dtask error codes */

#include "../messys/messys_err.h"
#include "../messys/messys_par.h"
#include "../messys/messys_dd.h"

#include "ams_sys.h"
#include "ams_mac.h"
#include "ams_struc.h"
#include "ams.h"

int main()
{
   int outmsg_status;
   int outmsg_function;
   int outmsg_context;
   int outmsg_length;
   char outmsg_name[32];
   char outmsg_value[MSG_VAL_LEN];
   int inmsg_status;
   int inmsg_function;
   int inmsg_context;
   int inmsg_length;
   char inmsg_name[32];
   char inmsg_value[MSG_VAL_LEN];

   int status;
   int path;
   int messid;
   int j;

   status = 0;
   outmsg_status = SAI__OK;
   outmsg_function = MESSYS__MESSAGE;
   outmsg_context = OBEY;
   outmsg_length = 16;

   strcpy ( outmsg_name, "junk" );
   strcpy ( outmsg_value, "slave replying" );

   ams_init ( "slave", &status );
   if ( status != 0 )
   {
      printf ( "slave: failed init\n" );
   }
   for ( j=0; j<1000; j++ )
   {
      ams_receive ( MESSYS__INFINITE, 32, MSG_VAL_LEN, &inmsg_status, 
        &inmsg_context, inmsg_name, &inmsg_length, inmsg_value, &path, 
        &messid, &status );

      outmsg_status = DTASK__ACTSTART;

      ams_reply ( path, messid, outmsg_function, outmsg_status, 
        outmsg_context, outmsg_name, outmsg_length, outmsg_value, 
        &status ); 

      outmsg_status = SAI__OK;
      ams_reply ( path, messid, outmsg_function, outmsg_status, 
        outmsg_context, outmsg_name, outmsg_length, outmsg_value, 
        &status ); 
   }
   if ( status != 0 )
   {
      printf ( "slave: bad status = %d\n", status );
   }
   else
   {
      printf ( "slave: received - %s\n", inmsg_value );
   }

   kill ( getpid(), SIGINT );
   return 0;
}



\end{verbatim}




\section {the ams routines}

The following lists ams.h.


\begin{verbatim}
/*+  AMS_ASTINT - send an ASTINT message from a signal handler */

void ams_astint
(
int *status         /* global status (given and returned) */
);

/*+  AMS_ASTMSG - send an ASTMSG from a signal handler */

void ams_astmsg
(
char *name,       /* name of the action to be rescheduled (given) */
int length,       /* number of significant bytes in value (given) */
char *value,      /* message to be passed to main-line code (given) */
int *status       /* global status (given and returned) */
);

/*+  AMS_EXTINT - send an EXTINT message from a signal handler */

void ams_extint
(
int *status         /* global status (given and returned) */
);

/*+  AMS_GETREPLY - receive a message on a specified path, messid */

void ams_getreply
(
int timeout,              /* timeout time in milliseconds (given) */
int path,                 /* pointer to the path (given) */
int messid,               /* message number of incoming message (given) */
int message_name_s,       /* space for name (given) */
int message_value_s,      /* space for value (given) */
int *message_status,      /* message status (returned) */
int *message_context,     /* message context (returned) */
char *message_name,       /* message name (returned) */
int *message_length,      /* length of value (returned) */
char *message_value,      /* message value (returned) */
int *status               /* global status (given and returned) */
);

/*+  AMS_INIT - initialise ams */

void ams_init
(
char *own_name,      /* name of this task (given) */
int *status
);

/*+  AMS_KICK - send a message to this task's kick queue */
void ams_kick 
(
char *name,       /* name of the action to be rescheduled (given) */
int length,       /* number of significant bytes in value (given) */
char *value,      /* message to be passed to application code (given) */
int *status       /* global status (given and returned) */
);

/*+  AMS_PATH - get a communications path to another task */

void ams_path
(
char *other_task_name,  /* name of task to which path is required (given)
                           */
int *path,              /* pointer to the path (returned) */
int *status             /* global status (given and returned) */
);

/*+  AMS_PLOOKUP - look up a taskname given a path to it */

void ams_plookup 
( 
int path,             /* the path number (given) */
char *name,           /* the task name (returned) */
int *status           /* global status (given and returned) */
);

/*+  AMS_RECEIVE - receive any incoming message */

void ams_receive 
( 
int timeout,              /* timeout time in milliseconds (given) */
int message_name_s,       /* space for name (given) */
int message_value_s,      /* space for value (given) */
int *message_status,      /* message status (returned) */
int *message_context,     /* message context (returned) */
char *message_name,       /* message name (returned) */
int *message_length,      /* length of value (returned) */
char *message_value,      /* message value (returned) */
int *path,                /* path on which message received (returned) */
int *messid,              /* message number of incoming message (returned) */
int *status               /* global status (given and returned) */
);

/*+  AMS_REPLY - send a message on a given path, messid */

void ams_reply
(
int path,               /* the path number for communicating with the other 
                           task (given) */
int messid,             /* the number identifying the transaction (given) */
int message_function,   /* message function (given) */
int message_status,     /* message status (given) */
int message_context,    /* message context (given) */
char *message_name,     /* message name (given) */
int message_length,     /* length of value (given) */
char *message_value,    /* message value (given) */
int *status             /* global status (given and returned) */
);

/*+  AMS_RESMSG - send a message to this task's  reschedule queue */

void ams_resmsg
(
int length,        /* number of significant bytes in value (given) */
char *value,       /* message to be passed to main-line code (given) */
int *status        /* global status (given and returned) */
);

/*+  AMS_SEND - send a message on a given path */

void ams_send
(
int path,               /* pointer to the path (given) */
int message_function,   /* message function (given) */
int message_status,     /* message status (given) */
int message_context,    /* message context (given) */
char *message_name,     /* message name (given) */
int message_length,     /* length of value (given) */
char *message_value,    /* message value (given) */
int *messid,            /* message number issued by this task (returned) */
int *status             /* global status (given and returned) */
);


\end{verbatim}

\section {the FAMS routines}

The FORTRAN-callable interface to ams is implemented by the FAMS library,
the calling sequences for which are listed below. The library is
implemented in C, using the CNF/F77 facilities.


\begin{verbatim}

/*+  FAMS_ASTINT */

SUBROUTINE FAMS_ASTINT ( STATUS )

INTEGER STATUS         ! global status (given and returned)    


/*+  FAMS_ASTMSG */

SUBROUTINE FAMS_ASTMSG ( NAME, LENGTH, VALUE, STATUS )

CHARACTER*(*) NAME     ! name of action (given)
INTEGER LENGTH         ! length of message in VALUE (given)
CHARACTER*(*) VALUE    ! string to be sent (given)
INTEGER STATUS         ! global status (given and returned)    


/*+  FAMS_EXTINT */

SUBROUTINE FAMS_EXTINT ( STATUS )

INTEGER STATUS         ! global status (given and returned)    


/*+  FAMS_GETREPLY */

SUBROUTINE FAMS_GETREPLY ( TIMEOUT, PATH, MESSID, MESSAGE_STATUS,
  MESSAGE_CONTEXT, MESSAGE_NAME, MESSAGE_LENGTH, MESSAGE_VALUE, STATUS )

INTEGER TIMEOUT               ! timeout in millisec (given)
INTEGER PATH                  ! path to task (given)
INTEGER MESSID                ! message identifier (given)
INTEGER MESSAGE_STATUS        ! message status field (returned)
INTEGER MESSAGE_CONTEXT       ! GET, SET, OBEY or CANCEL (returned)
CHARACTER*(*) MESSAGE_NAME    ! name field for message (returned)
INTEGER MESSAGE_LENGTH        ! length of value string (returned)
CHARACTER*(*) MESSAGE_VALUE   ! value string (returned)
INTEGER STATUS                ! global status (given and returned)    


/*+  FAMS_INIT */

SUBROUTINE FAMS_INIT ( OWN_NAME, STATUS )

CHARACTER*(*) OWN_NAME        ! name of this task (given)
INTEGER STATUS                ! global status (given and returned)    


/*+  FAMS_KICK */

SUBROUTINE FAMS_KICK ( NAME, LENGTH, VALUE, STATUS )
(
CHARACTER*(*) NAME     ! name of action (given)
INTEGER LENGTH         ! length of message in VALUE (given)
CHARACTER*(*) VALUE    ! string to be sent (given)
INTEGER STATUS         ! global status (given and returned)    


/*+  FAMS_PATH */

SUBROUTINE FAMS_PATH ( OTHER_TASK_NAME, PATH, STATUS )

CHARACTER*(*) OTHER_TASK_NAME  ! name of other task (given)
INTEGER PATH                   ! path to other task (returned)
INTEGER STATUS                 ! global status (given and returned)    


/*+  FAMS_PLOOKUP */

SUBROUTINE FAMS_PLOOKUP ( PATH, NAME, STATUS )

INTEGER PATH)          ! path to another task (given)
CHARACTER*(*) NAME     ! name of other task (returned)
INTEGER STATUS         ! global status (given and returned)    


/*+  FAMS_RECEIVE */

SUBROUTINE FAMS_RECEIVE ( TIMEOUT, MESSAGE_STATUS, MESSAGE_CONTEXT,
MESSAGE_NAME, MESSAGE_LENGTH, MESSAGE_VALUE, PATH, MESSID, STATUS )

INTEGER TIMEOUT               ! timeout in millisec (given)
INTEGER MESSAGE_STATUS        ! message status field (returned)
INTEGER MESSAGE_CONTEXT       ! GET, SET, OBEY or CANCEL (returned)
CHARACTER*(*) MESSAGE_NAME    ! name field for message (returned)
INTEGER MESSAGE_LENGTH        ! length of value string (returned)
CHARACTER*(*) MESSAGE_VALUE   ! value string (returned)
INTEGER PATH                  ! path to task (returned)
INTEGER MESSID                ! message identifier (returned)
INTEGER STATUS                ! global status (given and returned)    


/*+  FAMS_REPLY */

SUBROUTINE FAMS_REPLY ( PATH, MESSID, MESSAGE_FUNCTION, MESSAGE_STATUS,
  MESSAGE_CONTEXT, MESSAGE_NAME, MESSAGE_LENGTH, MESSAGE_VALUE, STATUS )
(
INTEGER PATH                  ! path to task (given)
INTEGER MESSID                ! message identifier (given)
INTEGER MESSAGE_FUNCTION      ! function flag (usually MESSYS__MESSAGE)
                              ! (given)
INTEGER MESSAGE_STATUS        ! message status field (given)
INTEGER MESSAGE_CONTEXT       ! GET, SET, OBEY or CANCEL (given)
CHARACTER*(*) MESSAGE_NAME    ! name field for message (given)
INTEGER MESSAGE_LENGTH        ! length of value string (given)
CHARACTER*(*) MESSAGE_VALUE   ! value string (given)
INTEGER STATUS                ! global status (given and returned)    


/*+  FAMS_RESMSG */

SUBROUTINE FAMS_RESMSG ( LENGTH, VALUE, STATUS )

INTEGER LENGTH                ! length of message in VALUE (given)
CHARACTER*(*) VALUE           ! message to be sent (given)
INTEGER STATUS                ! global status (given and returned)    


/*+  FAMS_SEND */

SUBROUTINE FAMS_SEND ( PATH, MESSAGE_FUNCTION, MESSAGE_STATUS,
  MESSAGE_CONTEXT, MESSAGE_NAME, MESSAGE_LENGTH, MESSAGE_VALUE, MESSID,
  STATUS )

INTEGER PATH                  ! path to task (given)
INTEGER MESSAGE_FUNCTION      ! function flag (usually MESSYS__MESSAGE)
                              ! (given)
INTEGER MESSAGE_STATUS        ! message status field (given)
INTEGER MESSAGE_CONTEXT       ! GET, SET, OBEY or CANCEL (given)
CHARACTER*(*) MESSAGE_NAME    ! name field for message (given)
INTEGER MESSAGE_LENGTH        ! length of value string (given)
CHARACTER*(*) MESSAGE_VALUE   ! value string (given)
INTEGER MESSID                ! message identifier (returned)
INTEGER STATUS                ! global status (given and returned)    

\end{verbatim}

\end {document}
