\documentstyle{article}
\pagestyle{myheadings}

%------------------------------------------------------------------------------
\newcommand{\stardoccategory}  {ADAM Portability Project Note}
\newcommand{\stardocinitials}  {Starlink/APPN}
\newcommand{\stardocnumber}    {24.1}
\newcommand{\stardocauthors}   {B D Kelly\\
B K McIlwrath\footnote{Rutherford Appleton Laboratory}\\B V McNally}
\newcommand{\stardocdate}      {10 December 1993}
\newcommand{\stardoctitle}     {Specifying the Unix version of MSP}
%------------------------------------------------------------------------------

\newcommand{\stardocname}{\stardocinitials /\stardocnumber}
\markright{\stardocname}
\setlength{\textwidth}{160mm}
\setlength{\textheight}{230mm} % changed from 240
\setlength{\topmargin}{-2mm}   % changed from -5
\setlength{\oddsidemargin}{0mm}
\setlength{\evensidemargin}{0mm}
\setlength{\parindent}{0mm}
\setlength{\parskip}{\medskipamount}
\setlength{\unitlength}{1mm}

%------------------------------------------------------------------------------
% Add any \newcommand or \newenvironment commands here
%------------------------------------------------------------------------------

\begin{document}
\thispagestyle{empty}
SCIENCE \& ENGINEERING RESEARCH COUNCIL \hfill \stardocname\\
ROYAL OBSERVATORY EDINBURGH\\
{\large\bf Computing Section\\}
{\large\bf \stardoccategory\ \stardocnumber}
\begin{flushright}
\stardocauthors\\
\stardocdate
\end{flushright}
\vspace{-4mm}
\rule{\textwidth}{0.5mm}
\vspace{5mm}
\begin{center}
{\Large\bf \stardoctitle}
\end{center}
\vspace{5mm}

%------------------------------------------------------------------------------
%  Add this part if you want a table of contents
%  \setlength{\parskip}{0mm}
%  \tableofcontents
%  \setlength{\parskip}{\medskipamount}
%  \markright{\stardocname}
%------------------------------------------------------------------------------

\section {Summary}

The Unix version of the Adam low-level intercommunications library (MSP)
is specified and its implementation outlined. In addition the methods
for implementing the Adam requirement for timers and the relationship of
ICL with its terminal handling process are described.


\section {Introduction}

Preliminary Unix versions of ICL and intertask communications facilities
were delivered by RAL Informatics division to Starlink. During the week
6-10 December 1993, BKM visited ROE to deliver an alpha-release of the
resulting system and to carry out a comprehensive review of various
design decisions taken during its implementation. This document
summarises the result of this review, and the intention is to enable the
delivery of a beta-release of the system by 14 February 1994.

\section {Problems with the alpha-release}

The alpha version of unix MSP routes all messages via a server process.
As compared with a direct task-to-task implementation, this doubles the
message traffic and also (possibly more seriously) doubles the context
switching and the consequent "ungreed". There ARE advantages to the
method, in that it provides the nameserver facility required to enable
one Adam process to establish connections with another.

The alpha version has certain undesirable features in the areas of
layering, meaning that

\begin{itemize}
\item The timer facility uses MSP
\item The timer facility uses knowledge about the MSP implementation
\item ICL uses MSP
\item ICL uses knowledge about the MSP implementation
\item ICL uses knowledge about the MESSYS implementation
\end{itemize}


\section {Adam Layering Principles}

One principle which has proved very important in the past is that Adam
intertask communication is defined by the interface to the MESSYS
library. The actual implementation of MESSYS has changed from time to
time, and features such as networking have been added to it without
forcing changes to be made to the code outside the MESSYS layer.

A corollary of this is that no Adam code outside MESSYS knows of the
existence (let alone the implementation) of MSP. (Note that in this
context, ADAMNET processes are inside the MESSYS layering.)

This principle is emphasized when one considers that there is
speculation about switching to the DRAMA IMP library for low-level
communications at some time in the future. The conclusion is that the
implementation of the timer facility and of certain details of ICL
should be changed.


\section {MSP Specification}

The MSP interface was specified at the AAO Workshop on Data Acquisition
Environments, 1985. The specification was based on a detailed analysis
of the requirements of an implementation layer beneath MESSYS. One
facility was added to MSP "in case it was needed". This facility is
represented by the MSP\_ENABLE\_AST function, which tells MSP that if a
message arrives on a certain message queue, then an AST should be
delivered to the queue owner. This facility has never been needed by
MESSYS, and therefore should be removed from the specification of unix
MSP.

The VAX/VMS version of MESSYS is written in FORTRAN. The unix version
was translated to C by the programmers from RAL Informatics division.
This has an implication for the actual call arguments to the MSP
functions, and the obvious decision is to make them C-style arguments.
The resulting set of calls is as follows.

\begin{verbatim}

/*+  MSP_CREATE_QUEUE - create a queue and return its identifier */
void msp_create_queue
(
int *qid,           /* created queue identifier (returned) */
int *status         /* global status (given and returned) */
);

/*+ MSP_DELETE_QUEUE - delete a queue */
void msp_delete_queue
(
int qid,            /* identifier of queue to be deleted (given) */
int *status         /* global status (given and returned) */
);

/*+ MSP_ENTER_TASK - register this task with MSP */
void msp_enter_task
(
char *task_name,    /* name of this task (given) */
int *status         /* global status (given and returned) */
);

/*+ MSP_GET_TASK_NAME - get the task name associated with a queue */
void msp_get_task_name
(
int qid,            /* queue identifier (given) */
int task_name_len,  /* length of buffer for name (given) */
char *task_name,    /* name of task (returned) */
int *status         /* global status (given and returned) */
);

/*+ MSP_GET_TASK_QUEUE - get the command queue of a named task */
void msp_get_task_queue
(
char *task_name,    /* name of task (given) */
int *qid,           /* task command queue (returned) */
int *status         /* global status (given and returned) */
);

/*+ MSP_RECEIVE_MESSAGE - receive a message on one of a list of queues */
void msp_receive_message
(
int *qarr,          /* array of queue identifiers (given) */
int nqueues,        /* number of queues (given) */
int wait,           /* wait flag (given) */
int maxlen,         /* maximum length of message (given) */
char msgbody[],     /* received message (returned) */
int *actlen,        /* size of received message (returned) */
int *qid,           /* identifier of queue used (returned) */
int *replyq,        /* reply queue for message (returned) */
int *status         /* global status (given and returned) */
);

/*+ MSP_SEND_MESSAGE - send a message on a queue */
void msp_send_message
(
char msgbody[],     /* message to be sent (given) */
int msglen,         /* length of message to be sent (given) */
int sendq,          /* queue identifer to be used (given) */
int replyq,         /* reply queue to be associated with the message
                       (given) */
int *status         /* global status (given and returned) */
);

\end{verbatim}


\section {MSP implementation}

A prototype of MSP for Unix was implemented by John Cooke and reported
at the 1991 Adam workshop. This took the approach of mapping the concept
of an MSP queue onto a Unix datagram socket. This results in a very
small system, as MSP simply becomes a wrap-around for the socket
library. The socket names (specifically the task names) are entered into
the filing system. Intrinsic message exchange was acceptably fast in
this system (3.5 msec for a one-way message on a SUN3/80) but queue
creation and deletion was unacceptably slow. An upgrade of this code
with optimisation achieved by simply caching deleted queues appears to
be the best way forward in the short term.


\section {Timer implementation}

The implementation of the timer facility can be made independent of
intertask communication by having each task use its own timer. The basic
system facility used is the itimer facility, set to deliver a signal on
timer expiry. The rest of the timer is based on the implementation by
BKM of timers on the alpha Adam-unix release. The basic principle of
operation is as follows.

There is a data structure which holds a queue of wait-times. The
main-line code cancels the itimer if necessary, then modifies the queue
depending on whether timer requests are being added or removed, and then
starts the itimer for the next interval on the queue. When the timer
signals, the signal is caught by a signal handler which modifies the
queue, restarts the itimer for the next interval, and executes an
application-specific function whose address was stored on the queue with
the wait-time.


\section {ICL terminal handling}

The Unix version of ICL relies on a separate process to perform terminal
handling, and so has to perform terminal input and output via some form
of intertask communication. Currently there is one terminal handling
process for ICL, called ICL\_IO. An alternative handler, ICLmenus, is
under development. It is accepted as a requirement that the terminal
handler process should be specified as part of the command line which
initiates ICL.

There are two aspects to the communications problem, corresponding to
the two directions of data flow.

Messages into ICL (essentially ICL commands coming from the terminal
handler) have to be seen by ICL as Adam messages, which ICL waits for
using ADAM\_RECEIVE. This enables ICL to wait for receipt of
communications both from the terminal handler and from Adam tasks.

Messages into the terminal handler have to make it possible for the
handler to wait for receipt of messages from ICL or user input on the
terminal. In the case of ICL\_IO this is character input, for ICLmenus it
is X events.

The solution is that on startup the terminal handler initialises itself
into the Adam message system and performs ADAM\_RECEIVE. ICL uses
ADAM\_SENDONLY to send an OBEY to the terminal handler, which also sets
up a return route.

ICL and the i/o process use a private pipe or socket for subsequent
communications from ICL to the terminal handler. This mechanism allows
the use of select() or the setup of an X callback using XtAppAddInput().
The i/o process sends commands to ICL using ADAM\_TRIGGER, putting the
command in the "value string". This has the side effect of also allowing
any Adam task to return a command to ICL.

\section {Conclusion}

An acceptably straightforward route involving a limited amount of work
has been found for progressing from the alpha release of the Unix Adam
multitasking facilities.

\end {document}

