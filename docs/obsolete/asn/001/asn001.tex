\documentstyle{article}
\pagestyle{myheadings}

%------------------------------------------------------------------------------
\newcommand{\stardoccategory}  {ADAM System Note}
\newcommand{\stardocinitials}  {ASN}
\newcommand{\stardocnumber}    {1.2}
\newcommand{\stardocauthors}   {A J Chipperfield}
\newcommand{\stardocdate}      {1 March 1990}
\newcommand{\stardoctitle}     {Organization of Graphics in ADAM}
%------------------------------------------------------------------------------

\newcommand{\stardocname}{\stardocinitials /\stardocnumber}
\markright{\stardocname}
\setlength{\textwidth}{160mm}
\setlength{\textheight}{240mm}
\setlength{\topmargin}{-5mm}
\setlength{\oddsidemargin}{0mm}
\setlength{\evensidemargin}{0mm}
\setlength{\parindent}{0mm}
\setlength{\parskip}{\medskipamount}
\setlength{\unitlength}{1mm}

%------------------------------------------------------------------------------
% Add any \newcommand or \newenvironment commands here
\catcode`\$=12 \catcode`\_=12
\font\tt=CMTT10 scaled 1095
\renewcommand{\_}{{\tt\char'137}}
%------------------------------------------------------------------------------

\begin{document}
\thispagestyle{empty}
SCIENCE \& ENGINEERING RESEARCH COUNCIL \hfill \stardocname\\
RUTHERFORD APPLETON LABORATORY\\
{\large\bf Starlink Project\\}
{\large\bf \stardoccategory\ \stardocnumber}
\begin{flushright}
\stardocauthors\\
\stardocdate
\end{flushright}
\vspace{-4mm}
\rule{\textwidth}{0.5mm}
\vspace{5mm}
\begin{center}
{\Large\bf \stardoctitle}
\end{center}
\vspace{5mm}

%------------------------------------------------------------------------------
%  Add this part if you want a table of contents
%  \setlength{\parskip}{0mm}
%  \tableofcontents
%  \setlength{\parskip}{\medskipamount}
%  \markright{\stardocname}
%------------------------------------------------------------------------------
\section{INTRODUCTION}
This document describes the way in which GKS, SGS and DIAGRAM have
been incorporated into ADAM.
{\em Note that DIAGRAM is now virtually obsolete.}
The document also includes sections on
the high-level graphics package NCAR which is not fully integrated
with ADAM but can be used.
The Applications Graphics Interface, AGI (see SUN/48), will provide
more versatile access to graphics within ADAM.

\section{GENERAL}
Each package has been organized within ADAM to :
\begin{itemize}
\item Incorporate the GKS 7.2 versions of DIAGRAM, SGS and GKS
\item Clearly separate the environment and stand-alone layers of each
of the packages to facilitate the update of the stand-alone parts with new
releases from Starlink.
\end{itemize}
The directory structure for each package is shown in Figure \ref{pic}.

\begin{figure}[h]
\begin{center}
\begin{picture}(70,25)
\put(15,5){\makebox(0,0){[.SSC]}}
\put(55,5){\makebox(0,0){[.TEST]}}
\put(35,20){\makebox(0,0){[ADAM.LIB.{\it pkg}\/]}}
\put(15,13){\line(1,0){40}}
\put(15,13){\line(0,-1){4}}
\put(55,13){\line(0,-1){4}}
\put(35,13){\line(0,1){4}}
\end{picture}
\caption{Graphics Directory Structure}
\label{pic}
\end{center}
\end{figure}
Logical names {\it pkg}\_ENV and {\it pkg}\_DIR are assigned
to the top level and [.SSC] directories respectively.
{\it pkg}\_ENV contains the environment layer, ie subroutines handling
activation and de-activation of the package under ADAM, the
association between the package and physical devices, and the reporting of
errors to ADAM.

Note that this is not consistent with other ADAM {\it pkg}\_DIR logical names
where {\it pkg}\_DIR is the top level.

In addition to the true environment layer subroutines, {\it pkg}\_ENV
contains any subroutines which need to be altered from the version
contained in the Starlink stand-alone libraries. Inclusion of the correct
version of these subroutines is ensured by the order of searching the
libraries. The changes required are described below under each package
heading. The environment specific subroutines are contained in libraries named
{\it pkg}\/PAR.OLB and {\it pkg}\/PAR.TLB.

{\it pkg}\_DIR contains those parts of the Starlink Software Collection's
stand-alone release of the package which are required.
For Starlink sites, the logical names SGS\_DIR and GKS\_DIR be assigned to the
actual stand-alone system directories to save disk space.

A slight deviation from the above occurs in the case of the DIAGRAM package
and its subsidiary DIP.
DIP\_ENV and DIP\_DIR are still defined separately
but they are both assigned to the DIAGRAM directories. Furthermore, in order
to incorporate the ADAM status values in DIA\_ERR and DIP\_ERR, the Starlink
stand-alone versions must
be re-compiled. The stand-alone libraries cannot be used as they will result
in unrecognized status values.

The [.TEST] directory contains test programs for the package.
They may mostly be compiled, linked and run in the normal way.
Some deliberate errors are produced so check before getting too worried.

\section{GKS ADAM IMPLEMENTATION}

\subsection{THE ENVIRONMENT LAYER}
 Environment level GKS subroutine names are of the
form GKS\_name. They handle:
\begin{itemize}
\item The activation and de-activation of GKS under ADAM.
\item The connection between GKS and the physical devices.
\item The allocation of `graphics descriptors' which tabulate the connection
between ADAM parameters and GKS workstations.
\item The reporting of GKS errors to ADAM.
\end{itemize}

\subsection{THE STAND-ALONE LAYER}
This corresponds with the standard Starlink release although the standard
error handling subroutines GERHND and GERLOG will not be used (see below).
The ISO standard (G) form of name is used for GKS kernel subroutines throughout
the ADAM implementation
of the graphics systems.This is because SGS and NCAR both use the standard form.
If for any reason it is
essential to use the GKS\_ form in an application, try including
GKS\_DIR:GKSLIBG/LIB in the link command.
{\em No guarantees are given. The wrong error handler may be loaded}

\subsection{ERROR HANDLING}
The GKS standard allows the user to replace the GKS error handling subroutine
GERHND and this is done in the ADAM implementation. The ADAM version of the
subroutine is a modification of the GKS error handler used by
the standard release of SGS.
It sets a flag in common to indicate to the environment layer (or SGS) that
an error has occurred
and then calls the ADAM GKS error logging subroutine GKS\_ERLOG regardless
of whether or not inherited status mode is in use.
GKS\_ERLOG is a modification of the standard error logger and uses a
modification of the RAL/ICL implementation internal subroutine GKGEM
(renamed GKS\_KGEM) to obtain the standard GKS error message text from a
direct access file defined by the logical name GKS\_EMF.
Currently this is the method used by the RAL
implementation of GKS.
In case the GKS release style of error message file changes, the program
GKSEMF.FOR and its input data file GKSEMF.DAT are included in GKS\_ENV.

Having obtained the text of the error message, GKS\_ERLOG reports the error
to ADAM using ERR\_REP so that the message may be output or annulled in the
usual way.

An alternative, simpler, error logger is also provided in case the sophisticated
one fails to work after a GKS update.
The alternative is held in the module GKS\_ERLOGB in GKS\_ENV:GKSPAR.TLB.
In order to incorporate it into the object library, it must be extracted and
compiled and the resultant object file used to replace GKS\_ERLOG in
GKS\_ENV:GKSPAR.OLB.
{\em Note that if the complete library is re-built, care must be exercised to
ensure that the primary error logger and not the fallback one is included in
the object library.}

GKS environment layer subroutines use subroutine GKS\_GSTAT to interrogate
the error flag set by GERHND.

\section{SGS ADAM IMPLEMENTATION}

\subsection{THE ENVIRONMENT LAYER}
Handles:
\begin{itemize}
\item The activation and de-activation of SGS under ADAM.
\item The connection between SGS and the physical devices.
\item The allocation of `zone descriptors' which tabulate the connection
between ADAM parameters SGS zones.
\item The reporting of SGS errors to ADAM.
\end{itemize}

\subsection{THE STAND-ALONE LAYER}
This corresponds with the standard Starlink release of SGS although the
following subroutines will not be used (ADAM versions in the environment
layer libraries will be used).
\begin{itemize}
\item GERHND The ADAM GKS version described above is used.
\item SGS\_$ERR The ADAM version reports the error
using ERR\_REP and sets the status to the ADAM error status corresponding
with the input error number.
\item SGS\_$ISTAT The ADAM version does nothing. It doesn't matter anyway as
SGS\_$HSTAT does nothing.
\end{itemize}
Note that the cursor sampling routines SGS\_ENSCU,SGS\_SAMCU and SGS\_DISCU
cannot be used with GKS 7.2 at the moment. They are included in the
ADAMGRAPH shared image but the corresponding GKS routines will give an
error message.

\subsection{ERROR HANDLING}
SGS replaces the standard GKS error handler with one of
its own. However, as the ADAM GKS error handler does for both SGS and GKS
under ADAM that is used in preference.
The subroutine SGS\_$GKERR is used to check for GKS error under SGS (The
text of any GKS error will already have been reported but not output).
SGS errors are reported via subroutine SGS\_$ERR which is altered as described
above.

\section{DIAGRAM AND DIP}

\subsection{ENVIRONMENT LAYER}
Handles:
\begin{itemize}
\item The activation and de-activation of DIAGRAM under ADAM.
\item The connection between DIAGRAM and the physical devices. This is done
via the appropriate SGS environment layer subroutines.
\item The allocation of `diagram descriptors' which tabulate the connection
between ADAM parameters DIAGRAM diagrams.
\item The reporting of DIAGRAM and DIP errors to ADAM.
\end{itemize}

\subsection{THE STAND-ALONE LAYER}

\subsubsection{DIAGRAM}
The standard Starlink release of DIAGRAM must be re-compiled using
the ADAM error numbers defined in DIA\_ENV:DIAERR.FOR .
The Starlink text library has
subroutines grouped into larger modules whereas the current ADAM version
has them separately in conformity with other ADAM packages.

\subsubsection{DIP}
The standard Starlink release of DIP must be re-compiled using
the ADAM error numbers defined in DIP\_ENV:DIPERR.FOR .
The Starlink text library has
subroutines grouped into larger modules whereas the current ADAM version
has them separately in conformity with other ADAM packages.
The current Starlink release of DIAGRAM/DIP uses GKS 6.2. In order to convert
it to GKS 7.2, the three subroutines comprising module DIPZ in library DIP
i.e. DIP\_CLONE, DIP\_$UPDTE and DIP\_$SELH must be replaced by the versions
contained in DIP\_ENV

\section{NCAR}
NCAR, the high level graphics package distributed by the National Center for
Atmospheric Research in Boulder, Colorado has not yet been integrated with
ADAM.

\section{PROGRAMMING}
Under each section, reference is made to the relevant Starlink User Note
(SUN). This document briefly describes the environment level programming,
APN/2 and APN/3 give more information on the GKS and SGS environment-level
routines.

\subsection{GKS}
GKS\_ASSOC performs GKS initialization if necessary via GKS\_ACTIV which
calls GOPKS. It ensures that the workstation specified by the associated
parameter is activated performing the functions GOPWK and GACWK.

GKS\_ANNUL will just close a workstation,
GKS\_CANCL will close a workstation and also cancel the parameter,
and GKS\_DEACT will close all open workstations and GKS.

Task caching means that GKS, workstations and parameters
may or may not remain active between invocations of a task so correct
closedown procedures should always be adopted to ensure consistent results.

Example:
\begin{quote} \begin{verbatim}
CALL GKS_ASSOC
     -
GKS kernel calls
     -
CALL GKS_CANCL
CALL GKS_DEACT
\end{verbatim} \end{quote}
For information on programming using GKS, see SUN/83.

\subsection{SGS}
SGS\_ASSOC performs SGS initialization if necessary via SGS\_ACTIV which calls
SGS\_INIT. It then call SGS\_OPNWK rather than SGS\_OPEN.

SGS\_ANNUL will just close a workstation,
SGS\_CANCL will close a workstation and also cancel the parameter,
and SGS\_DEACT will close all open workstations and SGS.

Task caching means that SGS, workstations and parameters
may or may not remain active between invocations of a task so correct
closedown procedures should always be adopted to ensure consistent results.

Example:
\begin{quote} \begin{verbatim}
CALL SGS_ASSOC
     -
SGS plotting calls etc.
     -
CALL SGS_CANCL
CALL SGS_DEACT
\end{verbatim} \end{quote}
For information on programming using SGS, see SUN/85.

\subsection{DIAGRAM}
DIA\_ASSOC performs DIAGRAM initialization if necessary via DIA\_ACTIV which
calls SGS\_ASSOC.

DIA\_ANNUL will just close a workstation,
DIA\_CANCL will close a workstation and also cancel the parameter,
and DIA\_DEACT will close all open workstations and DIAGRAM.

Task caching means that DIA, workstations and parameters
may or may not remain active between invocations of a task so correct
closedown procedures should always be adopted to ensure consistent results.

Example:
\begin{quote} \begin{verbatim}
CALL DIA_ASSOC
     -
DIA plotting calls etc.
     -
CALL DIA_CANCL
CALL DIA_DEACT

\end{verbatim} \end{quote}
For information on programming using DIAGRAM, see SUN/54

\subsection{NCAR}
For the time being
NCAR can be used by treating applications as GKS or SGS applications
(See above). The NCAR library can be searched by including it as a private
library in the ADAM link command:

For example:
\begin{quote} \begin{verbatim}
ALINK application,NCAR_DIR:NCARLIB/LIB
\end{verbatim} \end{quote}
Assuming that NCAR has been installed according to Starlink instructions.

One major problem with this is that NCAR utilities have a habit of obeying
Fortran STOP in the event of errors. This causes confusion for the environment.

For information on programming using NCAR, see SUN/88, SUN/90

\end{document}
