%
% An Introduction to IRAF.
%
% Copyright 1999  Starlink, CCLRC.
%
% A.C. Davenhall (Edinburgh), 3/9/98.
%

\documentclass[twoside,11pt]{article}

% ? Specify used packages
\usepackage{graphicx}        %  Use this one for final production.
% \usepackage[draft]{graphicx} %  Use this one for drafting.
% ? End of specify used packages

\pagestyle{myheadings}

% -----------------------------------------------------------------------------
% ? Document identification
% Fixed part
\newcommand{\stardoccategory}  {Starlink Guide}
\newcommand{\stardocinitials}  {SG}
\newcommand{\stardocsource}    {sg\stardocnumber}

% Variable part - replace [xxx] as appropriate.
\newcommand{\stardocnumber}    {12.2}
\newcommand{\stardocauthors}   {R.~Morris, G.J.~Privett \& A.C.~Davenhall}
\newcommand{\stardocdate}      {2nd December 1999}
\newcommand{\stardoctitle}     {An Introduction to IRAF}
\newcommand{\stardocabstract}
{This document is an introduction to the IRAF software package and
its use on Starlink systems.  IRAF is a powerful and comprehensive
environment for reducing and analysing astronomical data.  It was
developed at the National Optical Astronomy Observatories, Tucson and is
in widespread use around the world.  IRAF has its own data file format,
command language, on-line help system and programming language.  It is a
modular system.  The basic core, which is always present, provides general
facilities for image processing and data reduction.  For more specialised
tasks additional packages are loaded to augment the core system.
Software for processing most sorts of astronomical data is available
for the IRAF environment.

\latex{\vspace{5mm}}

\begin{center}
{\bf Who Should Read this Document?}
\end{center}

This document is aimed at people who are considering whether or not
to use IRAF or who are using it for the first time.}
% ? End of document identification
% -----------------------------------------------------------------------------

% +
%  Name:
%     sg.tex
%
%  Purpose:
%     Template for Starlink Guide (SG) documents.
%     Refer to SUN/199
%
%  Authors:
%     AJC: A.J.Chipperfield (Starlink, RAL)
%     BLY: M.J.Bly (Starlink, RAL)
%
%  History:
%     17-JAN-1996 (AJC):
%        Original with hypertext macros, based on MDL plain originals.
%     16-JUN-1997 (BLY):
%        Adapted for LaTeX2e.
%        Added picture commands.
%     {Add further history here}
%
% -

\newcommand{\stardocname}{\stardocinitials /\stardocnumber}
\markboth{\stardocname}{\stardocname}
\setlength{\textwidth}{160mm}
\setlength{\textheight}{230mm}
\setlength{\topmargin}{-2mm}
\setlength{\oddsidemargin}{0mm}
\setlength{\evensidemargin}{0mm}
\setlength{\parindent}{0mm}
\setlength{\parskip}{\medskipamount}
\setlength{\unitlength}{1mm}

% -----------------------------------------------------------------------------
%  Hypertext definitions.
%  ======================
%  These are used by the LaTeX2HTML translator in conjunction with star2html.

%  Comment.sty: version 2.0, 19 June 1992
%  Selectively in/exclude pieces of text.
%
%  Author
%    Victor Eijkhout                                      <eijkhout@cs.utk.edu>
%    Department of Computer Science
%    University Tennessee at Knoxville
%    104 Ayres Hall
%    Knoxville, TN 37996
%    USA

%  Do not remove the %begin{latexonly} and %end{latexonly} lines (used by 
%  star2html to signify raw TeX that latex2html cannot process).
%begin{latexonly}
\makeatletter
\def\makeinnocent#1{\catcode`#1=12 }
\def\csarg#1#2{\expandafter#1\csname#2\endcsname}

\def\ThrowAwayComment#1{\begingroup
    \def\CurrentComment{#1}%
    \let\do\makeinnocent \dospecials
    \makeinnocent\^^L% and whatever other special cases
    \endlinechar`\^^M \catcode`\^^M=12 \xComment}
{\catcode`\^^M=12 \endlinechar=-1 %
 \gdef\xComment#1^^M{\def\test{#1}
      \csarg\ifx{PlainEnd\CurrentComment Test}\test
          \let\html@next\endgroup
      \else \csarg\ifx{LaLaEnd\CurrentComment Test}\test
            \edef\html@next{\endgroup\noexpand\end{\CurrentComment}}
      \else \let\html@next\xComment
      \fi \fi \html@next}
}
\makeatother

\def\includecomment
 #1{\expandafter\def\csname#1\endcsname{}%
    \expandafter\def\csname end#1\endcsname{}}
\def\excludecomment
 #1{\expandafter\def\csname#1\endcsname{\ThrowAwayComment{#1}}%
    {\escapechar=-1\relax
     \csarg\xdef{PlainEnd#1Test}{\string\\end#1}%
     \csarg\xdef{LaLaEnd#1Test}{\string\\end\string\{#1\string\}}%
    }}

%  Define environments that ignore their contents.
\excludecomment{comment}
\excludecomment{rawhtml}
\excludecomment{htmlonly}

%  Hypertext commands etc. This is a condensed version of the html.sty
%  file supplied with LaTeX2HTML by: Nikos Drakos <nikos@cbl.leeds.ac.uk> &
%  Jelle van Zeijl <jvzeijl@isou17.estec.esa.nl>. The LaTeX2HTML documentation
%  should be consulted about all commands (and the environments defined above)
%  except \xref and \xlabel which are Starlink specific.

\newcommand{\htmladdnormallinkfoot}[2]{#1\footnote{#2}}
\newcommand{\htmladdnormallink}[2]{#1}
\newcommand{\htmladdimg}[1]{}
\newenvironment{latexonly}{}{}
\newcommand{\hyperref}[4]{#2\ref{#4}#3}
\newcommand{\htmlref}[2]{#1}
\newcommand{\htmlimage}[1]{}
\newcommand{\htmladdtonavigation}[1]{}

% Define commands for HTML-only or LaTeX-only text.
\newcommand{\html}[1]{}
\newcommand{\latex}[1]{#1}

% Use latex2html 98.2.
\newcommand{\latexhtml}[2]{#1}

%  Starlink cross-references and labels.
\newcommand{\xref}[3]{#1}
\newcommand{\xlabel}[1]{}

%  LaTeX2HTML symbol.
\newcommand{\latextohtml}{{\bf LaTeX}{2}{\tt{HTML}}}

%  Define command to re-centre underscore for Latex and leave as normal
%  for HTML (severe problems with \_ in tabbing environments and \_\_
%  generally otherwise).
\newcommand{\setunderscore}{\renewcommand{\_}{{\tt\symbol{95}}}}
\latex{\setunderscore}

%  Redefine the \tableofcontents command. This procrastination is necessary 
%  to stop the automatic creation of a second table of contents page
%  by latex2html.
\newcommand{\latexonlytoc}[0]{\tableofcontents}

% -----------------------------------------------------------------------------
%  Debugging.
%  =========
%  Remove % on the following to debug links in the HTML version using Latex.

% \newcommand{\hotlink}[2]{\fbox{\begin{tabular}[t]{@{}c@{}}#1\\\hline{\footnotesize #2}\end{tabular}}}
% \renewcommand{\htmladdnormallinkfoot}[2]{\hotlink{#1}{#2}}
% \renewcommand{\htmladdnormallink}[2]{\hotlink{#1}{#2}}
% \renewcommand{\hyperref}[4]{\hotlink{#1}{\S\ref{#4}}}
% \renewcommand{\htmlref}[2]{\hotlink{#1}{\S\ref{#2}}}
% \renewcommand{\xref}[3]{\hotlink{#1}{#2 -- #3}}
%end{latexonly}
% -----------------------------------------------------------------------------
% ? Document specific \newcommand or \newenvironment commands.
% ? End of document specific commands
% -----------------------------------------------------------------------------
%  Title Page.
%  ===========
\renewcommand{\thepage}{\roman{page}}
\begin{document}
\thispagestyle{empty}

%  Latex document header.
%  ======================
\begin{latexonly}
   CCLRC / {\sc Rutherford Appleton Laboratory} \hfill {\bf \stardocname}\\
   {\large Particle Physics \& Astronomy Research Council}\\
   {\large Starlink Project\\}
   {\large \stardoccategory\ \stardocnumber}
   \begin{flushright}
   \stardocauthors\\
   \stardocdate
   \end{flushright}
   \vspace{-4mm}
   \rule{\textwidth}{0.5mm}
   \vspace{5mm}
   \begin{center}
   {\Huge\bf  \stardoctitle}
   \end{center}
   \vspace{5mm}

% ? Add picture here if required for the LaTeX version.
%   e.g. \includegraphics[scale=0.3]{filename.ps}
% ? End of picture

% ? Heading for abstract if used.
   \vspace{10mm}
   \begin{center}
      {\Large\bf Abstract}
   \end{center}
% ? End of heading for abstract.
\end{latexonly}

%  HTML documentation header.
%  ==========================
\begin{htmlonly}
   \xlabel{}
   \begin{rawhtml} <H1> \end{rawhtml}
      \stardoctitle
%      \stardocversion\\
%      \stardocmanual
   \begin{rawhtml} </H1> \end{rawhtml}

% ? Add picture here if required for the hypertext version.
%   e.g. \includegraphics[scale=0.7]{filename.ps}
% ? End of picture

   \begin{rawhtml} <P> <I> \end{rawhtml}
   \stardoccategory \stardocnumber \\
   \stardocauthors \\
   \stardocdate
   \begin{rawhtml} </I> </P> <H3> \end{rawhtml}
      \htmladdnormallink{CCLRC}{http://www.cclrc.ac.uk} /
      \htmladdnormallink{Rutherford Appleton Laboratory}
                        {http://www.cclrc.ac.uk/ral} \\
      \htmladdnormallink{Particle Physics \& Astronomy Research Council}
                        {http://www.pparc.ac.uk} \\
   \begin{rawhtml} </H3> <H2> \end{rawhtml}
      \htmladdnormallink{Starlink Project}{http://www.starlink.ac.uk/}
   \begin{rawhtml} </H2> \end{rawhtml}
   \htmladdnormallink{\htmladdimg{source.gif} Retrieve hardcopy}
      {http://www.starlink.ac.uk/cgi-bin/hcserver?\stardocsource}\\

%  HTML document table of contents. 
%  ================================
%  Add table of contents header and a navigation button to return to this 
%  point in the document (this should always go before the abstract \section). 
  \label{stardoccontents}
  \begin{rawhtml} 
    <HR>
    <H2>Contents</H2>
  \end{rawhtml}
  \newcommand{\latexonlytoc}[0]{}
  \htmladdtonavigation{\htmlref{\htmladdimg{contents_motif.gif}}
        {stardoccontents}}

% ? New section for abstract if used.
  \section{\xlabel{abstract}Abstract}
% ? End of new section for abstract
\end{htmlonly}

% -----------------------------------------------------------------------------
% ? Document Abstract. (if used)
%  ==================
\stardocabstract
% ? End of document abstract
% -----------------------------------------------------------------------------
% ? Latex document Table of Contents (if used).
%  ===========================================
  \newpage
  \vspace{3cm}

  \subsection*{Revision History}

  \begin{enumerate}

    \item 30th December 1998: Version 1.  Original version based on
     the earlier document SUN/179.3, 13th July 1995 by Rhys~Morris
     and Grant~Privett (ACD).

    \item 2nd December 1999: Version 2.  Corrected a mistake in the
     start-up instructions.  Also improved the links to the on-line
     documentation.

  \end{enumerate}

  \vspace{10cm}
  \copyright \underline{1999} Starlink, CCLRC

  \cleardoublepage
  \begin{latexonly}
    \setlength{\parskip}{0mm}
    \latexonlytoc

%   \newpage
%    \listoftables
%    \newpage
   \vspace{5cm}
   \listoffigures

    \setlength{\parskip}{\medskipamount}
    \markboth{\stardocname}{\stardocname}
  \end{latexonly}
% ? End of Latex document table of contents
% -----------------------------------------------------------------------------

% - Part I --------------------------------------------------------------------
\cleardoublepage
\newpage
\renewcommand{\thepage}{\arabic{page}}
\setcounter{page}{1}
\markboth{\stardocname}{\stardocname}
\part{Overview}
\markboth{\stardocname}{\stardocname}
\section{\xlabel{INTRO}\label{INTRO}Introduction}

This document is an introduction to the IRAF (Image Reduction and
Analysis Facility) software package and its use on Starlink systems.
It is aimed at people who are considering whether or not to use IRAF or
who are using it for the first time.

IRAF is a powerful and comprehensive environment for reducing and
analysing astronomical data.  It was developed at the National Optical
Astronomy Observatories (NOAO), Tucson and is now in widespread use around
the world.  IRAF has its own data file format, command language, on-line
help system and programming language.  It is a modular system.  The basic
core, which is always present, provides general facilities for image
processing and data reduction.  For more specialised tasks additional
packages are loaded to augment the core system.  Software for processing
most sorts of astronomical data is available for the IRAF environment.
IRAF is available on all the variants of Unix currently supported by
Starlink: Digital Alpha/Digital Unix, SUN/Solaris and PC/Linux.

IRAF is one of several major software environments which are available
to Starlink users and upon which entire data-analysis campaigns can be
based.   Some other examples include: Starlink's own large collection of
utilities and application packages, the various incarnations of
the Figaro package (\xref{SUN/86}{sun86}{}\cite{SUN86}), AIPS
(\xref{SUN/207}{sun207}{}\cite{SUN207}) and MIDAS.  Each package has its
own particular strengths and special capabilities;  IRAF, AIPS and
Figaro are probably the most extensive.

IRAF is a sound choice for many Starlink data-analysis users,
especially where compatibility with overseas collaborators is a
requirement.  The choice is harder for those wishing to develop major
applications of their own, who may be reluctant to adopt IRAF's
non-standard programming language, {\tt SPP}, or who feel that the
`flat' IRAF data format is inadequate for their purposes (see
Section~\ref{FORMATS} for further details).  Those who need formal
guarantees of future support should also be very wary about committing
themselves to any package which is not under UK control.  However, IRAF
is an important weapon in the armoury of the average Starlink
data-analysis user and likely to remain so for some years.

The structure of this document is:

\begin{description}

  \item[{\rm Part I}] -- an overview of IRAF,

  \item[{\rm Part II}] -- some simple worked examples.

\end{description}

We suggest that you read through the overview to gain an impression
of what IRAF is about and then try some of the worked examples.  If
you have a particular aversion to reading manuals you could proceed
straight to the worked examples, though we do not recommend this.

The presence of IRAF is not mandatory at Starlink sites, though in
practice it is available at most sites.  You should check the
arrangements at your site with your local site manager.
If IRAF is not available at your site it is possible to obtain a copy
via the Internet; the following section gives the details.


\section{\xlabel{OBTAIN}\label{OBTAIN}Obtaining IRAF}

You can retrieve a copy of IRAF via the Internet by anonymous ftp.  It
can be downloaded from either the original NOAO site in Tucson or from a
mirror (that is, an identical copy) maintained by Starlink at the Rutherford
Appleton Laboratory.  The latter is quicker and more convenient for sites
in the UK.  The Starlink mirror is updated from the NOAO original every
night and hence can be no more than twenty-four hours out of date.  The
access details are as follows.

\begin{description}

  \item[NOAO, Tucson] ~
  \newline URL: \htmladdnormallink{ {\tt http://iraf.noao.edu/} }
   {http://iraf.noao.edu/}

   {\bf Anonymous ftp}

  \begin{tabular}{ll}
   site:      & {\tt ftp.noao.edu} \\
   directory: & {\tt /iraf}        \\
  \end{tabular}

   Retrieve file {\tt README} for further details.

  \item[Starlink mirror] ~
  \newline URL: \htmladdnormallink{ {\tt http://www.starlink.ac.uk/iraf/} }
   {http://www.starlink.ac.uk/iraf/}

   {\bf Anonymous ftp}

  \begin{tabular}{ll}
   site:      & {\tt ftp.starlink.rl.ac.uk} \\
   directory: & {\tt /pub/mirrors/iraf}     \\
  \end{tabular}

   Retrieve file {\tt README} for further details.

\end{description}

There are additional mirrors in Germany and Japan.
Installing IRAF is a non-trivial process and will almost certainly
require the assistance of your site manager.  Some hints on obtaining
and installing IRAF are given in Appendix~\ref{INSTALL}.  Postscript
versions (suitable for printing) of all the IRAF manuals mentioned in
this document can also be retrieved from the NOAO IRAF site and the
Starlink mirror.  In practice you are more likely to use the IRAF sites
to obtain documents than a complete installation.

If you are not familiar with the ftp utility then ask a colleague or
your site manager for assistance.  Alternatively, the books by
Krol\cite{KROL92} and Thomas\cite{THOMAS95} include good descriptions.


\section{\xlabel{DOC}\label{DOC}Documentation}

A great deal of \htmladdnormallink{documentation}
{http://iraf.noao.edu/iraf/web/docs/docmain.html}
is available for IRAF.  Most of the
documentation comprises detailed manuals for the various individual,
specialised packages.  However, some more general documentation is
available and, in particular, there is an introductory guide for new
users: {\it A Beginner's Guide to Using IRAF}\/\cite{BARNES93}.
IRAF is a complex and in some ways non-intuitive system and it is well
worth taking the time and trouble to learn the basics of its operation
before attempting a complicated data-reduction task.  The {\it
Beginner's Guide}\, is an accessible and thorough document and is a good
place to start.  Even if you are already familiar with IRAF it is still
worthwhile having a look at the {\it Beginner's Guide}\, because it may
well still contain useful information which is new to you.  Another
useful document is {\it A User's Introduction to the IRAF Command
Language}\/\cite{SHAMES86} which describes the IRAF command language,
{\tt cl}.

Copies of these documents can be retrieved from the IRAF 
\htmladdnormallinkfoot{Web}
{http://www.starlink.ac.uk/iraf/web/docs/docmain.html}
or anonymous ftp sites (see Section~\ref{OBTAIN} for details).  The
directory specifications given below are relative to the `top-level' IRAF
directory given in Section~\ref{OBTAIN}.  The {\it Beginner's Guide}\, is
available as file:

\begin{quote}
{\tt pub/beguide.ps.Z}
\end{quote}

The {\it Introduction to the IRAF Command Language}\, is available
as file:

\begin{quote}
{\tt iraf/docs/cluser.ps.Z}
\end{quote}

Both of these files are in compressed postscript format.  Remember
to set ftp to binary mode prior to retrieving them.  Once retrieved,
they should be decompressed using Unix command {\tt uncompress} (sic)
and can then be printed on a postscript printer.  Most IRAF documents
(but not the {\it Beginner's Guide}\/) are kept in directory {\tt
iraf/docs}.  A complete, up-to-date list of documents is kept in file:

\begin{quote}
{\tt iraf/docs/README}
\end{quote}

which is a simple ASCII text file.

\subsection{On-line documentation}

\htmladdnormallink{The IRAF Web sites}{http://www.starlink.ac.uk/iraf/}
contain a great deal of useful information.  See Section~\ref{OBTAIN} for
the URLs.

\subsection{\label{FAQ}Frequently Asked Questions}

An excellent, up-to-date list of Frequently Asked Questions (FAQs)
is maintained for IRAF.  It is, perhaps, the single most useful
IRAF document and you should familiarise yourself with it; it will
probably contain the answers to most of your initial queries about
IRAF.  An on-line version is available at the following URLs:

\begin{tabular}{ll}
NOAO, Tucson:    & \htmladdnormallink{ {\tt 
    http://iraf.noao.edu/iraf/web/faq/FAQ.html} }
   {http://iraf.noao.edu/iraf/web/faq/FAQ.html}      \\
Starlink mirror: & \htmladdnormallink{ {\tt
    http://www.starlink.ac.uk/iraf/web/faq/FAQ.html} }
   {http://www.starlink.ac.uk/iraf/web/faq/FAQ.html}  \\
\end{tabular}

A simple ASCII text version (suitable for printing) can be retrieved
by anonymous ftp from file:

\begin{quote}
{\tt iraf/FAQ}
\end{quote}

\subsection{Newsletter\label{NEWSLETTER}}

A {\it Newsletter}\, describing recent enhancements and developments
of IRAF is published intermittently.  To subscribe to this {\it
Newsletter}\, send an e-mail message to username {\tt iraf@noao.edu}
asking to be added to the distribution list and including your
e-mail and postal address.  Alternatively use ftp to retrieve the
registration file:

\begin{quote}
{\tt iraf/REGISTER}
\end{quote}

and return it to username {\tt iraf-requests@noao.edu}.

Back issues of the {\it Newsletter}\, are 
\htmladdnormallink{available on the IRAF Web pages}
{http://www.starlink.ac.uk/iraf/web/irafnews/newslet.html}.
A \htmladdnormallink{table-of-contents}
{http://www.starlink.ac.uk/iraf/ftp/iraf/docs/TOC_news.txt}
for all the issues can be retrieved by anonymous ftp from file:

\begin{quote}
{\tt iraf/docs/TOC\_news.txt}
\end{quote}

and is available from the Web pages.


\section{\xlabel{VERSION}\label{VERSION}IRAF Version Numbers}

Currently the most recent major release of IRAF (at least for the three
variants of Unix supported by Starlink) is version 2.11.  However, the
previous version, 2.10, is still in use at some sites.  There are
significant differences between versions 2.10 and 2.11, particularly in
the data formats which they support (see Section~\ref{FORMATS}).  These
differences will be alluded to at various points in the present document.


\section{\xlabel{FORMATS}\label{FORMATS}Data File Formats}

The basic IRAF data file format is the so-called Old IRAF Format (OIF).
Despite this name there is no `new' IRAF format.  Most IRAF
tasks use the OIF format.  Some brief details of it are given below.

In IRAF version 2.11 and subsequent versions most tasks can also
access data files in the widely-used international standard
FITS\footnote{The original FITS format was proposed by Wells {\it et
al.}\/\cite{WELLS81}.  However, it has been developed and enhanced over
the years.  The FITS standard is now maintained and documented by the
FITS Support Office of the Astrophysics Data Facility at the NASA
Goddard Space Flight Center (see URL: \htmladdnormallink{ {\tt
http://www.gsfc.nasa.gov/astro/fits/fits\_home.html}}
{http://www.gsfc.nasa.gov/astro/fits/fits\_home.html}).
Though FITS is basically an astronomical format it is sometimes
mentioned in books about standard image formats.  See, for example,
{\it Graphics File Formats}\, by Kay and Levine\cite{KAY95}.}
(Flexible Image Transport System) format.

Other formats are used for more specialised purposes.  For example,
the STF format is used by the STSDAS package for storing HST data,
the QPOE format is used to hold photon event lists (which are
usually produced by X-ray astronomy satellites) and the PLIO format
for pixel lists (which flag individual pixels in an image).

The IRAF {\tt dataio} package contains tasks for converting between
numerous different formats.

\subsection{Old IRAF Format}

A dataset, typically an image or spectrum, in the Old IRAF Format (OIF)
format consists of two files: a header file which has file-type {\tt
.imh} and a pixel file which has file-type {\tt .pix}.  These files
are usually stored in standard locations in your IRAF directories.
The {\tt .pix} files are large and can be stored in either the same
directory as the header files or in a different directory.  The header
file contains the dimensions of the data array, the name of the pixel file
and optionally any auxiliary information describing the image, expressed
as a series of keywords and their values.  These OIF keywords deliberately
have a syntax similar to FITS keywords.  The image dimensions and name of
the pixel file are stored in a binary format, but the keywords are stored
as simple ASCII characters.  Thus, you can check the values of the keywords
by listing a header file on the terminal.

There are several points about the OIF format which you should be aware
(and beware) of.

\begin{enumerate}

  \item IRAF version 2.10 supports only the original OIF (that is, version 1
   of the format).  However, version 2.11 also supports an enhanced version
   of the format (version 2).

  \item In version 1 of the format the {\tt .pix} file differs on
   different types of computer.  That is, if you are using IRAF version
   2.10 then a data file written on a SUN workstation cannot be read on a
   Digital Alpha or vice versa.  Thus, in this case, it is sensible to carry
   out all your IRAF work on the same sort of machine.

  \item Version 2 of the format is independent of the host operating
   system.  That is, if you are using IRAF version 2.11 then a data file
   written on one type of computer can be read on another type.

  \item This change is backwards but not forwards compatible.  IRAF version
   2.11 will read files in either version of the format, but version 2.10
   only reads files in version 1 of the format.  If you need to write a
   data file with version 2.11 and subsequently read it with version
   2.10 then one approach is to write it as a FITS file.  Alternatively,
   version 2.11 can be made to write files in version 1 of the format by
   setting an appropriate environment variable:

  \begin{quote}
   {\tt set ~ oifversion=1}
  \end{quote}

  \item In either version of the format, the header file contains the full
   name and directory specification of the pixel file.  Consequently, IRAF
   data files cannot be manipulated (copied, renamed or deleted) using
   standard Unix operating system commands; there are special commands
   within IRAF ({\tt imcopy}, {\tt imrename} and {\tt imdelete} respectively)
   which must be used for these tasks.  This point is covered fully in the
   {\it Beginner's Guide}\/\cite{BARNES93}.

\end{enumerate}

\subsection{Space Telescope Format}

The Space Telescope Format (STF) is used for storing Hubble Space
Telescope (HST) data.  It was introduced for the STSDAS package
(see Section~\ref{STSDAS}), but is now understood by most IRAF
tasks.  Like the OIF format, an STF dataset comprises two files:
one is the header file, and has the extension
{\tt .}{\it xx}\/{\tt h}, where {\it x}\, can be any
alphanumeric character, while the other contains the actual data
values and has a {\tt .}{\it xx}\/{\tt d} suffix.  The most usual file
extensions are {\tt .hhh} for header files and {\tt .hhd} for data files.
These files are usually kept together in the same directory.  STSDAS files
have the advantage that the header file is an ASCII file and can be read on
a terminal without using IRAF at all (which is not true of {\tt .imh} files;
see above).


\section{\xlabel{STRUCTURE}\label{STRUCTURE}The IRAF Package Structure}

IRAF's `tasks', as the individual programs are usually called, are
divided into packages.  Each of the packages may contain tasks or further
subpackages. For example, loading the {\tt noao} package by typing `{\tt
noao}', leads to further subpackages, while loading the {\tt dataio}
package makes available tasks for importing and exporting data to and
from IRAF. You cannot use any task until you have loaded the package
which contains it. Typing `{\tt bye}' unloads the most recently loaded
package.

Documentation for the various specialised packages is available from
the IRAF \htmladdnormallinkfoot{Web pages}
{http://www.starlink.ac.uk/iraf/web/docs/docmain.html}
and anonymous ftp sites (see Section~\ref{OBTAIN} for details).


\section{\xlabel{PACKAGES}\label{PACKAGES}Some IRAF Packages}

A couple of the more common IRAF packages are described briefly
below.

\subsection{\label{STSDAS}STSDAS}

STSDAS (Space Telescope Science Data Analysis System) is a package
for analysing data  acquired with the Hubble Space Telescope (HST).
It was developed at the Space Telescope Science Institute (STScI) in
Baltimore.  Further information is available at URL:

\begin{quote}
\htmladdnormallink{ {\tt http://ra.stsci.edu/STSDAS.html} }
   {http://ra.stsci.edu/STSDAS.html}
\end{quote}

including details of retrieving the software and documentation.  Note
that installing STSDAS requires about 85 Mbytes of free disk space.
Also note that STSDAS requires the TABLES package for manipulating
tables to be installed.  TABLES is also available from the STScI.
Inquiries, queries, bug reports etc. about STSDAS can be sent by e-mail
to the HST Help Desk at username {\tt help@stsci.edu}.

\subsection{PROS}

PROS (Post-Reduction Off-line Software) is a package for analysing
X-ray astronomy data.  It is not specific to any one X-ray astronomy
satellite and, in particular, it has facilities to handle data from
the EINSTEIN, ROSAT, ASCA and AXAF satellites.  It can also be used
for non-X-ray data and is particularly suitable for generating pixel
masks.  It was developed at the Smithsonian Astrophysical Observatory,
Cambridge, Massachusetts.  Further information is available at URL:

\begin{quote}
\htmladdnormallink{ {\tt http://hea-www.harvard.edu/PROS/pros.html} }
   {http://hea-www.harvard.edu/PROS/pros.html}
\end{quote}

including details of retrieving the software and documentation.
Inquiries, queries, bug reports etc. about PROS can be sent by e-mail
to the PROS user support hotline at username {\tt
hotseat@cfa.harvard.edu}.


\section{\xlabel{IRAFSTAR}\label{IRAFSTAR}IRAF and Starlink}

Most Starlink software is written using the Starlink software environment
and stores data in the N-dimensional Data Format (NDF;
\xref{SUN/33}{sun33}{}\cite{SUN33}).  A substantial degree of
inter-operability is possible between this software and IRAF.  

At the most basic level Starlink provides the CONVERT package 
(\xref{SUN/55}{sun55}{}\cite{SUN55}) which is a set of applications
for converting data files between a number of different data formats.
It includes applications for converting an NDF format file to one in
the IRAF OIF format (see Section~\ref{FORMATS}) and vice versa.  You
should be aware, however, that the NDF format is more capable, and can
contain more auxiliary information, than the IRAF OIF format.  When
an NDF file is converted to the OIF format such information is lost.
In particular, the NDF format has standard ways of representing arrays
of errors associated with arrays of data and many applications in the
Figaro package process and propagate these errors.  Error arrays are
not accommodated in the OIF format and are lost when an NDF is converted
to this format.

In addition to explicitly converting NDF files, most Starlink
applications which read and write NDF files can be configured to
read and write IRAF OIF files instead.  Again see
\xref{SUN/55}{sun55}{}\cite{SUN55} for details.

Some Starlink packages have been enhanced so that they can be run from
the IRAF command language, {\tt cl}.  When used in this way the Starlink
packages behave like `native' IRAF packages loaded on top of the basic
IRAF; they read and write data files in the native IRAF format and their
parameters can be set and inspected using the usual mechanisms of the
IRAF command language.  Tasks in the Starlink packages can be intermixed
freely with the standard IRAF tasks (again, just like `native' IRAF
packages).  There are a few differences between Starlink and `native'
IRAF tasks, principally in the way that graphics are displayed; see
\xref{SUN/217}{sun217}{}\cite{SUN217} for details.

The principal Starlink packages which have been enhanced in this way
are Figaro, KAPPA and GAIA.  Their use from the IRAF command language is
described in \xref{SUN/217}{sun217}{}\cite{SUN217} and
\xref{SUN/221}{sun221}{}\cite{SUN221} and Section~\ref{STARSESS} gives
a worked example.  The details of the technical legerdemain which
allows applications written for the Starlink software environment to
be run from the IRAF {\tt cl} will not be of interest to most users.
However, for completeness it is documented in
\xref{SSN/35}{ssn35}{}\cite{SSN35}.  The Figaro, KAPPA and GAIA packages
themselves are documented in \xref{SUN/86}{sun86}{}\cite{SUN86},
\xref{SUN/95}{sun95}{}\cite{SUN95} and \xref{SUN/214}{sun214}{}\cite{SUN214}
respectively.


\section{\xlabel{ASSIST}\label{ASSIST}Obtaining Assistance}

For queries about how IRAF is installed at your site or assistance
with obtaining and installing it you should consult your local
site manager.

If you have a problem with or query about using IRAF then your first
port of call should be the 
\htmladdnormallink{FAQ}{http://www.starlink.ac.uk/iraf/web/faq/FAQ.html}
(see Section~\ref{FAQ}), which will often have the answer.

The IRAF Project in Tucson provides a support service for users.  If
your query is not covered by the FAQ then you can send it in an e-mail
message to username {\tt iraf@noao.edu}.  The response is usually pretty
swift, but do bear in mind the time difference between the UK and Tucson.

Because the support provided by the IRAF Project is excellent, little
additional support by Starlink is either called for or provided.
However, there is a Starlink contact for assistance with IRAF problems;
you should send an e-mail message to username {\tt iraf@star.rl.ac.uk}.
This contact is particularly appropriate if you encounter problems
inter-operating Starlink and IRAF software.


% - Part II -------------------------------------------------------------------
\cleardoublepage
\markboth{\stardocname}{\stardocname}
\part{Worked Examples}
\markboth{\stardocname}{\stardocname}
\section{\xlabel{INTRO_EX}\label{INTRO_EX}Introduction}

This part of the document provides a few simple, introductory examples
of using IRAF.  The examples are:

\begin{itemize}

  \item starting IRAF,

  \item a simple IRAF session,

  \item using Starlink packages from IRAF,

  \item using scripts and batch jobs.

\end{itemize}

The best way to use the examples is to work through them in the order
that they are given.  All the data files, scripts etc. referred to in
the examples are available in directory:

\begin{quote}
{\tt /star/examples/sg12}
\end{quote}

\subsection{Additional examples}

The examples given here form only the briefest of introductions and
there are many facilities which are not described.  There are further
examples, and more detailed descriptions, in the {\it Beginner's
Guide}\/\cite{BARNES93}.

There are also a set of exercises which demonstrate how to reduce
various sorts of data.  They can retrieved by anonymous ftp (see
Section~\ref{OBTAIN} for details) from file:

\begin{quote}
{\tt iraf/misc/exer2102.tar.Z}
\end{quote}

The file is a compressed tar archive.  Remember to set ftp to binary
mode prior to retrieving it.  The file should be decompressed using Unix
command {\tt uncompress} (sic).  Instructions for installing the
examples are given in file 
\htmladdnormallink{{\tt exer2102.readme}}
{ftp://iraf.noao.edu/iraf/misc/exer2102.readme}
(which is included in the tar archive).


\newpage
\section{\xlabel{START}\label{START}Starting IRAF}

Starting IRAF for the first time is non-trivial and hence this example
is provided as an illustration of the procedure.  {\it The example is
broadly correct, and works in Edinburgh.  However, some of the details,
in particular the file locations, may be different at your site.  You
should check the details at your site with your site manager.}

An additional complication is that IRAF has various ways of displaying
two-dimensional images (such as the CCD frames), none of which can be
considered standard.  Some of the common alternatives are to use either
SAOTNG, its predecessor SAOIMAGE (see \xref{SUN/166}{sun166}{}\cite{SUN166})
or XIMTOOL.  SAOTNG and SAOIMAGE are separate image display applications
which are not part of IRAF (though, obviously, they can display IRAF images).
SAOIMAGE works correctly with IRAF version 2.10, but there seem to be
problems with using it with version 2.11.  Hence SAOTNG is used
in the current example.  It is also worth noting that when using IRAF you
will not always need to display two-dimensional images, though in practice
you usually will.  Consequently setting up SAOTNG is integrated with the
instructions here.  The IRAF {\it Beginner's Guide}\/\cite{BARNES93} covers
starting IRAF for the first time, but not all the details of setting up an
image display.

The procedure is as follows.

\begin{enumerate}

  \item In order to use IRAF you need a `base' or `root' directory
   which will contain various configuration files, sub-directories
   and probably your IRAF data files.  This directory is your `IRAF
   home directory' and you need to decide where it is going to be located
   and what it is going to be called.  For example, we might decide
   that our IRAF directory is going to be:

  \begin{quote}
   {\tt /home/acd/iraf}
  \end{quote}

   This directory will be used in this example, though if you are
   working through it you should substitute your own chosen IRAF
   directory.

  \item You will probably need to add a one or more lines to your shell
   login script.  If you run the `tc' shell then this script is file {\tt
   .tcshrc}; if you run the `c' shell it is file {\tt .cshrc}.  In either
   case it is located in your home directory.  The details will depend on
   how IRAF has been set up at your site.  At Edinburgh the following line
   should be added:

  \begin{quote}
   {\tt source ~ /net/reaxp02/packages2/iraf211/jbin/login}
  \end{quote}

   See your site manager for the details at your site.  It is probably most
   convenient to put these lines at the end of the file.  
   
  \item Either log out and log in again or {\tt source} your {\tt
   .tcshrc} (or {\tt .cshrc}) file.

  \item If you have not already done so, create your own IRAF directory and
   move to it (for example, {\tt mkdir iraf} followed by {\tt cd iraf}).

  \item Type:

  \begin{quote}
   {\tt mkiraf}
  \end{quote}

   You will be prompted for the terminal type; reply {\tt xgterm}.
   Subdirectory {\tt uparm} will be created in your IRAF directory,
   as will file {\tt login.cl}.  The latter is the IRAF start up script.
   It contains various options and defaults which you can change, but
   you should not need to do so.

  \item Customisations can be made in file {\tt login.cl}, but are better
   put in a separate file, which must be called {\tt loginuser.cl}.  (It
   is better to put customisations in {\tt loginuser.cl} than in {\tt
   login.cl} because the latter is overwritten when {\tt mkiraf} is
   executed whereas the former is not.)  An example {\tt loginuser.cl}
   file is available as file:

  \begin{quote}
   {\tt /star/examples/sg12/loginuser.cl}
  \end{quote}

   You can simply copy this file to your IRAF directory.  (It configures
   IRAF to handle the large headers often associated with fibre spectroscopy
   data and specifies that the pixel files associated with bulk datasets,
   such as images or spectra,  are to be kept in subdirectory {\tt pixels}
   of the IRAF home directory.  Section~\ref{PIXELS} gives some notes on
   how to specify the directory in which the pixel files are to be kept.)

  \item Start SAOTNG ready for use with IRAF.  Type:

  \begin{quote}
   {\tt saotng \&}
  \end{quote}

   The `{\tt \&}' is, of course, simply to run SAOTNG as a detached
   process, which is more convenient.

  \item Create an {\tt xgterm} window:

  \begin{quote}
   {\tt xgterm \&}
  \end{quote}

   A new window with a distinctive appearance is created.  You will use
   this window to run IRAF.  You can run IRAF from your normal login
   window, but using an {\tt xgterm} window allows you to make better
   use of the workstation or X-terminal graphics.

   Depending on the configuration options in place at your site the
   {\tt xgterm} window might be created without a scroll bar.  If so,
   and you want to add a scroll bar (which is often useful), then from
   the {\tt xgterm} window hold down the {\tt Ctrl} key and simultaneously
   click on the right mouse button.  A window containing  a list of options
   should appear; simply click on the {\tt scrollbar} item and a scroll
   bar should be added to the window.

  \item Move to the {\tt xgterm} window and type:

  \begin{quote}
   {\tt cl}
  \end{quote}

   An introductory message should be displayed and the prompt should
   change to `\verb-cl>-'.  You are now running the IRAF command language
   and ready to issue IRAF commands.  See the {\it Beginner's
   Guide}\/\cite{BARNES93} for an introduction to the commands available
   and Section~\ref{SESSION} for a simple example.

   As a quick check that the image display is working correctly type:

  \begin{quote}
   {\tt display ~ dev\$pix ~ 1}
  \end{quote}

   An IRAF test image of the spiral galaxy M51 should appear in the SAOTNG
   display.

   Type {\tt logout} to leave the IRAF command language and return to the
   Unix shell prompt.

\end{enumerate}

\subsection{\label{SUBSQ_IRAF}Subsequent IRAF sessions}

Once you have set up for using IRAF it is not necessary to repeat the
entire foregoing procedure every time you use IRAF.  For subsequent
IRAF sessions in which you will use an image display you need repeat
only steps 7, 8 and 9.  For sessions in which an image display is
not needed just repeat steps 8 and 9.

\subsection{\label{PIXELS}Specifying the directory to contain the pixel
files}

The directory which is to contain the pixel files associated with bulk
datasets (such as images or spectra) is specified by the value of
environment variable {\tt imdir}.  For example, the sample {\tt loginuser.cl}
customisation file contains the line:

\begin{quote}
{\tt set ~ imdir=HDR\$pixels/}
\end{quote}

which specifies that the pixel files are to be kept in subdirectory {\tt
pixels} of your IRAF home directory.  Note that the trailing `{\tt /}' is
important.  Here {\tt HDR\$} denotes your IRAF home directory, which
contains the header files.  If you wanted to keep the pixel files with
the header files in your IRAF home directory you would simply specify:

\begin{quote}
{\tt set ~ imdir=HDR\$}
\end{quote}


\newpage
\section{\xlabel{SESSION}\label{SESSION}A Simple IRAF Session}

This example demonstrates a simple session using IRAF.  It covers:
converting a FITS image to the IRAF OIF format, displaying an image
and calculating statistics for an image.  It also introduces the
IRAF on-line help system and displaying and setting IRAF task
parameters.

The FITS image used in this example is available as file:

\begin{quote}
{\tt /star/examples/sg12/ccdimage.fit}
\end{quote}

If you intend to work through the example using this image you should make
a copy of it in your IRAF directory.  Alternatively, you may prefer
to use a FITS image of your own.

\begin{enumerate}

  \item Start IRAF and SAOTNG as described in the previous example
   (Section~\ref{START}).  Subsequent commands should be issued from
   the window which is running the IRAF command language (that is, the
   one with the prompt set to `\verb-cl>-').

  \item The first step is to convert the FITS image to the IRAF OIF
   format (see Section~\ref{FORMATS}).  If you are using IRAF version 2.11
   this step is not strictly necessary as this version of IRAF can access
   FITS files directly.  However, it is included here in order to
   keep the example consistent for versions 2.10 and 2.11.  In addition,
   you might find the OIF format more convenient for subsequent operations.
   You should also note that the command to perform the conversion is
   slightly different in the two versions.  FITS file {\tt ccdimage.fit}
   (or an alternative) should be available in your IRAF directory.
   Type either:

  \begin{description}

    \item[{\rm IRAF version 2.10:}] ~

    \begin{quote}
     {\tt rfits ~ ccdimage.fit ~ 1 ~ ccdframe}
    \end{quote}

    \item[{\rm IRAF version 2.11:}] ~

    \begin{quote}
     {\tt rfits ~ ccdimage.fit ~ 0 ~ ccdframe}
    \end{quote}

  \end{description}

   A line similar to the following should be displayed:

  \begin{quote}
   {\tt File: ccdframe  KAPPA - Cadd          Size = 384 x 256}
  \end{quote}

   File {\tt ccdframe.imh} should be created in your IRAF directory
   and file {\tt ccdframe.pix} in the pixels directory (probably
   subdirectory {\tt pixels} if you have used the {\tt loginuser.cl}
   file supplied with this document).  From the IRAF command language 
   you can show the files in the current directory with command {\tt
   dir}, change to a new directory with {\tt cd} and list a text
   file with {\tt cat}; these commands behave similarly to the
   equivalent Unix commands.  You can use these commands to check
   that the header and pixel files have been created.  When you are
   satisfied that all is well, make your IRAF directory your current
   directory again.

  \item To show a summary of the file, including its provenance, type:

  \begin{quote}
   {\tt imhead ~ ccdframe}
  \end{quote}

  \item To display an image using SAOTNG simply type:

  \begin{quote}
   {\tt display ~ ccdframe ~ 1}
  \end{quote}

   The image should be displayed in the SAOTNG window.  You can
   use the various buttons in the SAOTNG window to customise the
   display.

  \item To display simple statistics about the image type:

  \begin{quote}
   {\tt imstat ~ ccdframe}
  \end{quote}

   Output similar to the following should be listed:

  \begin{verbatim}
#               IMAGE      NPIX      MEAN    STDDEV       MIN       MAX
             ccdframe     98304      508.     308.3      330.    31988.
  \end{verbatim}

  \item On-line help information is available for all the IRAF tasks.
   To obtain help on {\tt imstat} simply type:

  \begin{quote}
   {\tt help ~ imstat}
  \end{quote}

   A description of the command and its various parameters will be
   displayed.  Some brief details of the IRAF on-line help are given in
   Section~\ref{OHELP}, below.  So far, we have been specifying information
   to IRAF tasks by giving mandatory parameters on the command line.  IRAF
   tasks also have so-called `hidden' parameters which adopt a default value
   if you do not specify them.  You can show the current value of all the
   parameters of a task using the command {\tt lpar}.  For example:

  \begin{quote}
   {\tt lpar ~ imstat}
  \end{quote}

   The output should be something like:

  \begin{verbatim}
       images = "ccdframe"      Images
      (fields = "image,npix,mean,stddev,min,max") Fields to be printed
       (lower = INDEF)          Lower cutoff for pixel values
       (upper = INDEF)          Upper cutoff for pixel values
    (binwidth = 0.1)            Bin width of histogram in sigma
      (format = yes)            Format output and print column labels?
        (mode = "ql")           
  \end{verbatim}

   The parameters enclosed in brackets are hidden (in this case this
   is all of them except {\tt images}).

  \item By default {\tt imstat} displayed the name of the image, the
   number of pixels, the mean, standard deviation and minimum and
   maximum values.  However, the help information revealed that other
   options were available and that these were controlled by the value
   of parameter {\tt fields}.  Suppose that we wished to display the
   mean, standard deviation, skewness and kurtosis\footnote{The
   skewness and kurtosis are standard statistical quantities.  See,
   for example, Wall\cite{WALL79}.}.  It is necessary to set parameter
   {\tt fields} appropriately, which is achieved by typing:

  \begin{quote}
   {\tt imstat.fields ~ = ~ "mean,stddev,skew,kurtosis"}
  \end{quote}

   If {\tt imstat} is run again:

  \begin{quote}
   {\tt imstat ~ ccdframe}
  \end{quote}

   the output should now be:

  \begin{verbatim}
#      MEAN    STDDEV      SKEW  KURTOSIS
       508.     308.3     62.03     4728.
  \end{verbatim}

   In addition to setting parameter values from the command line they
   can also be edited using task {\tt epar}; see the {\it Beginner's
   Guide}\/\cite{BARNES93} for further details.

  \item Finally, type:

  \begin{quote}
   {\tt logout}
  \end{quote}

   To leave IRAF.

\end{enumerate}

\subsection{\label{OHELP}On-line help}

On-line help on individual tasks is available from the IRAF command
language.

\begin{enumerate}

  \item To obtain help on an individual task on simply type:

  \begin{quote}
   {\tt help} ~ {\it task-name}
  \end{quote}

   from the IRAF command line.  For example:

  \begin{quote}
   {\tt help ~ imstat}
  \end{quote}

   and the help text for task {\tt imstat} is displayed.  The bottom
   line of the window gives a summary of the commands available (type
   {\tt q} to quit and return to the IRAF \verb-cl>- prompt).

  \item If you are looking for tasks to perform some function but are
   not sure what the task names are, which packages they are in, or indeed
   whether they exist, try typing:

  \begin{quote}
   {\tt apropos} ~ {\it topic}
  \end{quote}

   For example, if you were looking for commands to mosaic several CCD
   frames together you might type:

  \begin{quote}
   {\tt apropos ~ mosaic}
  \end{quote}

   This command searches all the available IRAF help databases and lists
   a summary of all the tasks with (in this example) the word `mosaic'
   in their title.  Note, however, that {\tt apropos} is part of STSDAS
   and hence is only available if STSDAS is installed.

\end{enumerate}


\newpage
\section{\xlabel{STARSESS}\label{STARSESS}Using Starlink Packages From IRAF}

This example shows a simple session using a Starlink package from
within IRAF.  It covers loading the package and calculating statistics
for an image.

Several Starlink packages are available from within IRAF, of which the
principal ones are KAPPA (see \xref{SUN/95}{sun95}{}\cite{SUN95}) and
Figaro (see \xref{SUN/86}{sun86}{}\cite{SUN86}).  When used from within
the IRAF command language these packages behave in most respects like
`native' IRAF packages: they can read and write IRAF OIF format data
files and their parameters can be displayed and controlled using the
usual mechanisms of the IRAF command language.  Tasks in the Starlink
packages can be intermixed freely with the standard IRAF tasks (again,
just like `native' IRAF packages).  There are a few differences between
Starlink and `native' IRAF tasks, principally in the way that graphics are
displayed; see \xref{SUN/217}{sun217}{}\cite{SUN217} for details.

This example uses the Figaro application {\tt istat} to display the
statistics of an image in order to compare and contrast it with the
standard IRAF application {\tt imstat} used in the previous example.
The IRAF OIF image {\tt ccdframe} which was used in the previous
example will also be used again here.

\begin{enumerate}

  \item If you are not already running IRAF then start it (see
   Section~\ref{START}).  Subsequent commands should be issued from
   the window which is running the IRAF command language (that is, the
   one with the prompt set to `\verb-cl>-').

  \item The first step is to load the Figaro package.  Simply type:

  \begin{quote}
   {\tt figaro}
  \end{quote}

   A list similar to the following should be displayed:

  \begin{center}
  \begin{verbatim}
   abconv      cset        fscrunch    ilog        lxset       sfit
   abline      cspike      fwconv      image       lyset       slice
   adjoin      delobj      gauss       imult       mask        soft
\end{verbatim}$\vdots$\end{center}

   Also the prompt should be reset to `\verb-fi>-'.  The list comprises
   all the commands in the Figaro package.  If instead of the list of
   commands you get an error message like:

  \begin{quote}
   {\tt ERROR: task `figaro' not found}
  \end{quote}

   then Figaro has not been installed in the version of IRAF available
   at your site.  Section~\ref{STARINSTALL} below gives a few hints on
   how to proceed in this case.

   However, assuming that the package loaded successfully, then all the
   Figaro tasks are now available, in addition to the basic IRAF tasks.
   Here Figaro is behaving just like a `native' IRAF package.  For any
   IRAF package, native or Starlink, you can:

  \begin{itemize}

    \item load it by typing the package name,

    \item remove it by typing {\tt bye},

    \item obtain help on it by typing {\tt help} {\it package name}.

  \end{itemize}

   Any number of packages can be loaded simultaneously.  All the
   commands in all the loaded packages are available.  Packages are
   hierarchical, with packages having sub-packages inside 'em, and so {\it
   ad infinitum}\, (as the poet almost had it).  Typing {\tt bye} removes
   the most recently loaded package or sub-package.  For example, type:

  \begin{quote}
   {\tt bye}
  \end{quote}

   Figaro will be removed and the prompt will be reset to \verb-cl>-.

  \item Now reload the Figaro package by typing {\tt figaro}.  Once the
   package is loaded simple statistics of an image can be displayed by
   typing:

  \begin{quote}
   {\tt istat ~ ccdframe}
  \end{quote}

   Output similar to the following should be listed:

  \begin{verbatim}
Y-range 1 to 256
X-range 1 to 384
Total (over 98304.0 pixels) = 4.9943E+7
Max   = 31988    in pixel (332,108)
Min   = 330      in pixel (1,256)
Mean  = 508.05
Sigma = 308.263
  \end{verbatim}

   Compare this output with that produced by {\tt imstat} in the previous
   example.  As in this case, Starlink applications are usually
   somewhat more verbose than standard IRAF tasks.

  \item On-line help information is available for the Figaro tasks,
   as it is for the standard ones.  To obtain help on {\tt istat} simply
   type:

  \begin{quote}
   {\tt help ~ istat}
  \end{quote}

   A description of the command and its various parameters will be
   displayed.  Again as for the standard tasks, the parameters of
   Figaro applications can be shown in the usual way:

  \begin{quote}
   {\tt lpar ~ istat}
  \end{quote}

   The output should be something like:

  \newpage
  \begin{verbatim}
        image = ""              (IMage) Name of image to examine
       ystart =                 *! (YStart) First Y value to be used
         yend =                 *! (YEnd) Last Y value to be used
       xstart =                 *! (XStart) First X value to be used
         xend =                 *! (XEnd) Last X value to be used
       (pass2 = no)             (PAss2) Pass twice through data for better sigm
      (median = no)             (MEDian) Calculate median as well as other stat
  (stat_total = 0.)             
    (stat_max = 0.)             
    (stat_min = 0.)             
   (stat_mean = 0.)             
  (stat_sigma = 0.)             
   (stat_size = 0.)             
   (stat_xmax = 0.)             
   (stat_xmin = 0.)             
   (stat_ymax = 0.)             
   (stat_ymin = 0.)             
 (stat_xstart = 0.)             
   (stat_xend = 0.)             
 (stat_ystart = 0.)             
   (stat_yend = 0.)             
 (stat_median = 0.)             
        (mode = "ql")           
  \end{verbatim}

  \item In addition to the quantities calculated automatically {\tt istat}
   may also optionally calculate the median value of the image.  Type:

  \begin{quote}
   {\tt istat.median ~ = ~ yes}
  \end{quote}

   Now run {\tt istat} again:

  \begin{quote}
   {\tt istat ~ ccdframe}
  \end{quote}

   The output should now be:

  \begin{verbatim}
Y-range 1 to 256
X-range 1 to 384
Total (over 98304.0 pixels) = 4.9943E+7
Max   = 31988    in pixel (332,108)
Min   = 330      in pixel (1,256)
Mean  = 508.05
Sigma = 308.263
Median  494
  \end{verbatim}

   Note that the median is now calculated and displayed.

\end{enumerate}

\subsection{\label{STARINSTALL}Installing Starlink software in IRAF}

IRAF installations at Starlink sites will usually have the additional
Starlink packages installed.  However, if they appear not to be
available at your site then this section gives a few hints on how to
proceed.

The directory {\tt /star/iraf} should contain a subdirectory for every
Starlink IRAF package present at your site.  Check that this directory
exists and that there is a subdirectory for the package that you are
interested in.  If the required subdirectory is absent then ask your
site manager to install the package.

If the directory is present but the package is still not available
when you start IRAF the most likely explanation is that the package
is not known to IRAF as an external package.  An entry for the package
should be added to the {\tt hlib\$extern.pkg} file.  Your site manager
will have to add this entry.  \xref{SSN/35}{ssn35}{}\cite{SSN35},
Section~4.8 contains the necessary details.

As an interim measure, perhaps to establish the cause of the problem,
you can add an entry to your own {\tt login.cl} file.  For example,
for Figaro add the following lines:

\begin{quote} \begin{verbatim}
reset   figaro            = starlink$figaro/
task    figaro.pkg        = figaro$figaro.cl
\end{verbatim} \end{quote}

These lines must be added just before the {\tt user} package definition.
The {\tt helpdb} definition can be extended by a command like:

\begin{quote} \begin{verbatim}
reset helpdb = (envget("helpdb") // ",figaro$helpdb.mip")
\end{verbatim} \end{quote}

For other packages you would add equivalent lines, substituting the
name of the package for {\tt figaro}.


\newpage
\section{\xlabel{SCRIPTS}\label{SCRIPTS}IRAF Scripts and Background Jobs}

A series of IRAF commands can be executed as a script by putting them
in a file and then running the file. The IRAF command language, {\tt cl},
is also a programming language and contains many features which allow
you to write scripts to perform repetitive tasks automatically.  It is
also possible to run IRAF scripts as Unix background jobs.  This
facility is particularly useful if you want to run lengthy tasks
at low priority without tying up your terminal.  This example is a
simple introduction to IRAF scripts and background jobs.

The scripts used in this example is available as files:

\begin{quote}
{\tt /star/examples/sg12/submean.cl} \\
{\tt /star/examples/sg12/background.csh}
\end{quote}

If you intend to work through the example you should copy these files
to your IRAF directory.  The IRAF OIF image {\tt ccdframe}, which was
used in the previous examples, should also be available.

File {\tt submean.cl} is a script of IRAF commands.  It is shown
in Figure~\ref{SUBMEAN}.  It calculates the mean value of all the
pixels in an image and then generates a new image in which the mean
value is subtracted from every pixel.

\begin{figure}[htbp]

\begin{verbatim}
# IRAF script to calculate a mean and subtract
# it from an image.

# Load images package.
images

# Initialize the cl list variable.
list = " "

# Store a temporary file name in string s3.
s3 = "temp.file"

# Configure imstat to display only the mean
imstat.format=no
imstat.fields="mean"

# Redirect imstat output to temporary file.
imstat "ccdframe" > (s3)

# Associate temporary file with list variable.
list = (s3)

# Read the value in the file into parameter 'x'
i= fscan (list,x)

# Use imarith to subtract x from the image.
imarith ("ccdframe","-",x,"newframe")

# Reconfigure imstat and then look at statistics of
# new image to see if the subtraction worked.
imstat.format=yes
imstat.fields="image,npix,mean,stddev,min,max"

imstat ("newframe")

# Delete the temporary file.
delete ("temp.file", verify=no)

# Print an output message
print("Subtracted ",x,"to create file newframe.")
\end{verbatim}

\caption{Example IRAF script
\label{SUBMEAN} }

\end{figure}

As with any unfamiliar programming language, your first reaction is 
probably one of trepidation as you wonder what is going on.  No variables
were declared at the start of this script although variables are used
throughout the program for parameter passing. The {\tt cl} has some built-in
variables which do not need to be declared. There are three string
variables named {\tt s1} to {\tt s3}, three real variables
{\tt x}, {\tt y} and {\tt z}, three
integers {\tt i}, {\tt j} and {\tt k},
three booleans and other more esoteric ones. Type
`{\tt lpar cl}' for a full list of the parameters available for
immediate use. Other variables can be used but they need to be
declared: see the {\tt cl} manual\cite{SHAMES86} for information on how to
do this. List-type parameters are useful, they are usually associated
with files, and they allow you to use {\tt fscan} and {\tt fprint}
for reading and
writing to files.  These functions are similar to the equivalent functions
in the C language.

The command {\tt imstat} is used to calculate the mean.  Some IRAF
tasks have output parameters which allow their results to be stored
and subsequently accessed by another task.  Unfortunately, {\tt imstat}
does not have any such output parameters.  Instead, it is configured
so that the only output that it generates is the mean value, without
any annotation.  This output is redirected to temporary file {\tt
temp.file} using the Unix-like output redirection capabilities of
the IRAF {\tt cl}.

The {\tt fscan} function is used to read the value from the file and
store it in variable {\tt x}.  The {\tt imarith} task (from the {\tt
images} package) is then used to subtract the mean value from each
pixel and create a new image {\tt newframe}.  Finally, {\tt imstat} is
reconfigured to display its default output and the statistics for image
{\tt newframe}  displayed as a check that the script has worked correctly.

Quoted parameters are passed to the tasks as they are; the brackets
around some of the parameter lists indicate that the {\tt cl} language is
meant to interpret these parameters as variables.  Programming in {\tt cl}
can be tricky sometimes: putting `{\tt echo=yes}' at the start of
the script means that all commands are echoed to the terminal as they
are executed, which can be useful for debugging.  The file name
`{\tt ccdframe}' was of course hardwired into the script; a more
typical application would operate on every file in a list, or loop
until some condition is satisfied. The {\tt cl} can support loops of
various types.  See {\it A User's Introduction to the IRAF Command
Language}\/\cite{SHAMES86} for more information.

The procedure to run the script is as follows.

\begin{enumerate}

  \item First you must make the script known to IRAF.  From the \verb-cl>-
   prompt type:

  \begin{quote}
   {\tt task \$submean=home\$submean.cl}
  \end{quote}

   `{\tt home}' is specifying that the script file resides in your
   root IRAF directory.

  \item You can now simply run the script by typing:

  \begin{quote}
   {\tt submean}
  \end{quote}

   Output similar to the following should appear:

  \begin{verbatim}
#               IMAGE      NPIX      MEAN    STDDEV       MIN       MAX
             newframe     98304  3.044E-5     308.3     -178.    31480.
Subtracted 508.046 to create file newframe.
  \end{verbatim}

   You can check that image {\tt newframe} has been created by typing
   {\tt dir} and display it with SAOIMAGE by typing:

  \begin{quote}
   {\tt display ~ newframe ~ 1}
  \end{quote}

  \item Finally you should tidy-up by deleting the new image:

  \begin{quote}
   {\tt imdelete ~ newframe}
  \end{quote}

   Note that the script will not work correctly if {\tt newframe}
   already exists, so you must delete it before running the script
   again.

\end{enumerate}

\subsection{Running a background job}

It is possible to run IRAF scripts as background jobs.  It is necessary
to prepare a Unix shell script which starts IRAF and then runs the
IRAF script.  File {\tt background.csh} is an example shell script.
It is shown in Figure~\ref{BACKGROUND}.  This example invokes IRAF
and uses it to run script {\tt submean.cl}.

\newpage
\begin{figure}[htbp]

\begin{verbatim}
#!/bin/csh
# Make sure a C shell is used
# Run IRAF from the UNIX shell, taking parameters
# up until the word END.
cl  <<END  >log.file
images
task \$submean = home\$submean.cl
submean
logout
END
\end{verbatim}

\caption{Example shell script to run IRAF as a background job
\label{BACKGROUND} }

\end{figure}

The script first ensures that a C shell is used, if it is run as
an executable shell script, then IRAF is started and fed with
commands. Notice how the {\tt \$} signs need to be escaped in the task
definition to stop the UNIX shell from interpreting them as shell
variables.  The output is written to file {\tt log.file}.  There is
further information about shell scripts in \xref{SC/4: {\it C-shell
Cookbook}}{sc4}{}\/\cite{SC4}).

The script is run from the Unix shell prompt.  Simply type:

\begin{quote}
{\tt background.csh \&}
\end{quote}

Output will be written to file {\tt log.file}.  Remember that script
{\tt submean.cl} will only run successfully if image {\tt newframe}
does not already exist.


\newpage
\appendix
\section{\xlabel{INSTALL}\label{INSTALL}Obtaining and Installing IRAF}

This appendix gives a few brief notes about obtaining and installing
IRAF.  A typical Starlink user is unlikely to be able to carry out a
complete IRAF installation.  The assistance of your site manager is
likely to be required for at least some of the stages and he will
probably carry out the complete installation.

IRAF is available for all the variants of Unix currently supported by
Starlink (and on various other operating systems).  A basic IRAF
installation, without any external packages, requires about 100
Mbytes of disk space, though the actual amount varies somewhat between
the different variants of Unix.  However, about half this space can,
if necessary, be reclaimed by deleting source files once the installation
is complete.

A copy of IRAF can be retrieved by anonymous ftp from NOAO or the
Starlink mirror.  The \htmladdnormallink{IRAF Web pages}
{http://www.starlink.ac.uk/iraf/web/iraf-homepage.html}
(see Section~\ref{OBTAIN}) give the necessary details.  Alternatively
the NOAO can send you a copy on tape or CD-ROM, though a small charge
is made.  Again see the Web pages for details.  There is a separate
distribution of IRAF for each version of Unix supported and there
is no compilation involved in making the installation (though source
is supplied).

Before installing IRAF you should obtain and read the following
documents:

\begin{enumerate}

  \item The IRAF installation guide,

  \item The IRAF management guide,

  \item The {\tt README} file for your particular machine,

  \item The {\tt REGISTRATION} file.

\end{enumerate}

All these documents can be retrieved from the IRAF Web pages.  There
are different versions of the \htmladdnormallink{first three documents}
{http://www.starlink.ac.uk/iraf/web/docs/igsm.html}
corresponding to each version of Unix supported and you should ensure
that you are reading the correct version for the system on which you
intend to install IRAF.

The {\tt REGISTRATION} file is important as it contains a form to be
filled in and returned to NOAO by e-mail, so that you can be added to
the mailing list for the {\it IRAF Newsletter}\, (see
Section~\ref{NEWSLETTER}).


\newpage
\addcontentsline{toc}{section}{Acknowledgements}
\section*{Acknowledgements}

We are grateful to Dave~Bowen for advice and assistance with using
IRAF and useful comments on the draft version of the document.
Nigel~Hambly also gave several helpful tips on using IRAF.  Martin~Bly,
Alan~Chipperfield and Rodney~Warren-Smith provided useful comments on the
draft.

Any mistakes, of course, are our own.

% References -----------------------------------------------------------

% \input{refs.tex}

\newpage
\addcontentsline{toc}{section}{References}
\begin{thebibliography}{99}

  \bibitem{BARNES93} J.~Barnes, 1993, {\it A Beginner's Guide to
   Using IRAF}\, (National Optical Astronomy Observatories: Tucson).
   File: {\tt beguide.ps.Z}.  See Section~\ref{DOC} for details of
   obtaining IRAF documents.

  \bibitem{SSN35} A.J.~Chipperfield, 1997, \xref{SSN/35.3}{ssn35}{}: 
   {\it IRAFSTAR -- The IRAF/Starlink Inter-operability Infrastructure}\,
   (Starlink).

  \bibitem{SUN217} A.J.~Chipperfield, 1997, \xref{SUN/217.2}{sun217}{}:
   {\it Running Starlink Applications from IRAF CL}\, (Starlink).

  \bibitem{SUN221} A.J.~Chipperfield, 1997, \xref{SUN/221.1}{sun221}{}:
   {\it KAPPA for IRAF}\, (Starlink).

  \bibitem{SUN95} M.J.~Currie and D.S.~Berry, 1999,
   \xref{SUN/95.14}{sun95}{}: {\it KAPPA -- Kernel Application Package}\,
   (Starlink).

  \bibitem{SC4} M.J.~Currie, 1998, \xref{SC/4.2}{sc4}{}:
   {\it C-shell Cookbook}\, (Starlink).

  \bibitem{SUN55} M.J.~Currie, G.J.~Privett, A.J.Chipperfield,
   D.S.~Berry and A.C.~Davenhall, 1999, \xref{SUN/55.11}{sun55}{}:
   {\it CONVERT -- A Format-conversion Package}\, (Starlink).

  \bibitem{SUN214} P.W.~Draper, 1999, \xref{SUN/214.5}{sun214}{}:
   {\it GAIA -- Graphical Astronomy and mage Analysis Tool}\, (Starlink).

  \bibitem{KAY95} D.C.~Kay and J.R.~Levine, 1995, {\it Graphics File
   Formats}, second edition
  \newline (Windcrest/McGraw-Hill: New York).  See in particular
   Chapter 18, pp235-244.

  \bibitem{KROL92} E.~Krol, 1992, {\it The Whole Internet User's Guide
   and Catalog}\, (O'Reilly and Associates Inc: Sebastopol, California).

  \bibitem{SUN166} R.~Morris and G.J.~ Privett, 1966,
   \xref{SUN/166.4}{sun166}{}: {\it SAOIMAGE -- Astronomical Image
   Display}\, (Starlink).

  \bibitem{SUN207} J.W.~Palmer, 1996, \xref{SUN/207.1}{sun207}{}:
   {\it AIPS -- Astronomical Image Processing System}\, (Starlink).

  \bibitem{SHAMES86} P.M.B.~Shames and D.~Tody, 1986, {\it
   A User's Introduction to the IRAF Command Language}, Version 2.3,
   (National Optical Astronomy Observatories: Tucson).  File: {\tt
   cluser.ps.Z}.  See Section~\ref{DOC} for details of obtaining IRAF
   documents.

  \bibitem{SUN86} K.T.~Shortridge, H.~Meyerdierks, M.J.~Currie,
   M.J.~Clayton, J.~Lockley, A.C.~Charles, A.C.~Davenhall, M.B.~Taylor,
   T.~Ash, T.~Wilkins, D.~Axon, J.W.~Palmer and A.~Holloway, 1999,
   \xref{SUN/86.17}{sun86}{}: {\it FIGARO -- A General Data Reduction
   System}\, (Starlink).

  \bibitem{THOMAS95} B.J.~Thomas, 1995, {\it The Internet for
   Scientists and Engineers}\, (SPIE -- The International Society
   for Optical Engineering: Bellingham, Washington).

  \bibitem{SUN33} R.F.~Warren-Smith, 1998, \xref{SUN/33.6}{sun33}{}:
   {\it NDF -- Routines for Accessing the Extensible N-Dimensional Data
   Format}\, (Starlink).

  \bibitem{WALL79} J.V. Wall, 1979, `Practical Statistics for
   Astronomers', {\it Q. J. R. Astron. Soc}, {\bf 20},
   pp138-152.

  \bibitem{WELLS81} D.C.~Wells, E.W.~Greisen, and R.H.~Harten, 1981,
   {\it Astron. Astrophys. Suppl}, {\bf 44}, pp363-370.

\end{thebibliography}

\typeout{  }
\typeout{*****************************************************}
\typeout{  }
\typeout{Reminder: run this document through Latex three times}
\typeout{to resolve the references.}
\typeout{  }
\typeout{*****************************************************}
\typeout{  }

\end{document}
