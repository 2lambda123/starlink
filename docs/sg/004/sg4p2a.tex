\part{FOR USERS}
\label{P_users}

\chapter{Applications Packages}
\label{C_applic}

When you run ICL and type:

\begin{small}
\begin{verbatim}
    ICL> HELP PACKAGES
\end{verbatim}
\end{small}

you will be shown two lists of packages.
The first contains {\em Standard} packages which are available at all
Starlink sites.
The other contains {\em Options} which are made available at a site only
on request.

This Chapter gives a brief overview of each package, while Chapter~\ref{C_rsapp}
describes the specific commands available within each package.
Further information can be obtained from their associated Starlink User Note
(SUN) which is referenced at the top of each section.
Starlink software as a whole is described in
\xref{SUN/1}{sun1}{}.

An important part of the rationalization of Starlink software which the coming
of ADAM made possible concerns data structures.
The Hierarchical Data System (HDS) (see Chapter \ref{C_HDS}) is very flexible,
and is capable of creating an infinite variety of data structures.
Without recommending some standard structure there would be a danger of
programmers writing applications which could not read each other's data.
If the standard is also implemented in a small number of routines, the
restrictions imposed by the standard also make programming easier.
The main unifying theme of Starlink applications is the standard data format
defined by Starlink; this is the Extensible $N$-dimensional Data Format ---
NDF --- described in Section \ref{S_NDF}.
This is centred on an $n$-dimensional data array that can store most
astronomical data such as spectra, images and spectral-line data cubes.
The NDF may also contain such information as title, axis labels and units, and
error and quality arrays.
There is also a place to store ancillary data associated with the data array.
These could be information about the original observing set-up, such as airmass
during the observation or temperature of the detector; there may be calibration
data or results produced during processing, for example spectral line fits.
Groups of related parameters not defined by the NDF format itself are held in
{\em extensions}\@.

A key component of Starlink software is KAPPA (Kernel Application Package).
This does not process non-standard extensions, but neither does it lose them
--- it copies them to any NDFs which it creates.
Other application packages may be able to process some, but not all, extensions.
It is hoped that such packages will use KAPPA applications as templates, in
procedures, or directly as appropriate.

Each section below contains a sketch of how to use the application being
described.
In these:

\begin{small}
\begin{verbatim}
    $ command
\end{verbatim}
\end{small}

means ``issue the command from DCL'', while:

\begin{small}
\begin{verbatim}
    ICL> command
\end{verbatim}
\end{small}

means issue it from ICL.
Remember that before using any of these packages, the commands:

\begin{small}
\begin{verbatim}
    $ @SSC:LOGIN
    $ ADAMSTART
\end{verbatim}
\end{small}

should have been executed.
Likewise, to run ICL you should type:

\begin{small}
\begin{verbatim}
    $ ICL
\end{verbatim}
\end{small}

to get the ICL prompt.

The packages described in this chapter are listed below ordered by function.
This should help you find a package which is appropriate for a particular
purpose.
The packages are then described in alphabetical order on separate pages.

\vspace*{5mm}

\begin{quote}

\begin{quote}
\begin{description}

\item [Image Analysis \& Photometry] \mbox{}
\begin{description}
\item [KAPPA] --- Kernel applications
\item [DAOPHOT] --- Stellar photometry
\item [PHOTOM] --- Aperture photometry
\item [PISA] --- Object finding and analysis
\end{description}

\item [Spectroscopy] \mbox{}
\begin{description}
\item [FIGARO] --- General spectral reduction
\item [SPECDRE] --- Spectroscopy data reduction
\end{description}

\item [Specific Wavelengths] \mbox{}
\begin{description}
\item [ASTERIX] --- X-ray data analysis
\end{description}

\item [Specific Instruments] \mbox{}
\begin{description}
\item [CCDPACK] --- CCD data reduction
\item [IRCAM] --- Infrared camera data reduction
\end{description}

\item [Polarimetry] \mbox{}
\begin{description}
\item [TSP] --- Time series and polarimetry analysis
\end{description}

\item [Database Management] \mbox{}
\begin{description}
\item [SCAR] --- Catalogue data base system
\end{description}

\item [Utilities] \mbox{}
\begin{description}
\item [CONVERT] --- Data format conversion
\item [SST] --- Simple software tools
\end{description}

\end{description}
\end{quote}

\end{quote}

\newpage

\section{ASTERIX --- X-ray data analysis}

\vspace{-11mm}

\hfill [\xref{SUN/98}{sun98}{}]

\vspace{5mm}

This is a collection of programs to analyse astronomical data in the X-ray
waveband.
Many of the programs are general purpose and are capable of analysing any data
in the correct format.
It is instrument independent and currently has interfaces to the Exosat and
Rosat instruments.

ASTERIX data are stored in HDS files and are therefore compatible with all ADAM
packages.
There are basically two different types: binned and event datasets.
{\em Binned data} (e.g.\ time series, spectra, images) are stored in files whose
structure is based on the Starlink standard NDF format
(\xref{SGP/38}{sgp38}{}).
Data errors (stored in the form of variances) and quality are catered for.
{\em Event data sets} store information about a set of photon `events'.
Each event will have a set of properties, e.g.\ X position, Y position, time,
raw pulse height.

The input data are first processed by an instrument interface.
Event data are then processed and binned, and then the binned data are
processed.
Finally, graphical output is generated.

The commands may be classified as follows:
{\small
\begin{quote}
\begin{itemize}
\item Interface to a particular instrument (EXOSAT, ROSAT, etc).
\item Event dataset and binned dataset processing.
\item Data conversion and display.
\item Mathematical manipulations.
\item Time series analysis.
\item Image processing.
\item Spectral analysis.
\item Statistical analysis.
\item Data quality analysis.
\item HDS editor.
\item Source searching.
\item Graphical and textual display.
\end{itemize}
\end{quote}
}

{\em To run ASTERIX:}\hfill
\begin{quote}
\begin{description}

\item [from ICL:]\hfill

\begin{small}
\begin{verbatim}
   $ ASTSTART
   $ ICL
   ICL> HELP ASTERIX
   ICL> ASTERIX
   ICL> ASTHELP
   ICL> (any ASTERIX command)
\end{verbatim}
\end{small}


\item [from DCL:]\hfill

\begin{small}
\begin{verbatim}
   $ ASTSTART
   $ ASTHELP
   $ (most, but not all, ASTERIX commands)
\end{verbatim}
\end{small}

\end{description}
\end{quote}

{\em Demonstration:}\hfill
\begin{quote}
\begin{description}
\item [] A demonstration session is described in
\xref{SUN/98}{sun98}{}.
\end{description}
\end{quote}

\newpage

\section{CCDPACK --- CCD data reduction}

\vspace{-11mm}

\hfill [\xref{SUN/139}{sun139}{}]

\vspace{5mm}

A package to perform the initial processing of CCD data.
Its main advantage over previous methods is its enhanced functionality.
It processes large amounts of data easily and efficiently, with a minimum of
effort on the user's part.

It includes routines for performing:
{\small
\begin{quote}
\begin{itemize}
\item Bias calibration data preparation.
\item Bias subtraction.
\item Flash and dark calibration data preparation.
\item Flash and dark-count correction.
\item Flatfield data preparation.
\item Flatfield correction.
\end{itemize}
\end{quote}
}
The following features are of particular note:
{\small
\begin{quote}
\begin{itemize}
\item Accesses lists of NDFs, using wildcards and by using names
 stored in text files.
\item Logs the progress of a reduction sequence.
\item Processes all non-complex numeric HDS data types.
\item Supports many data-combination techniques.
\item Takes full account of statistical uncertainty using variance production
 and propagation.
\end{itemize}
\end{quote}
}

{\em To run CCDPACK:}\hfill
\begin{quote}
\begin{description}

\item [from ICL:]\hfill

\begin{small}
\begin{verbatim}
   ICL> HELP CCDPACK
   ICL> CCDPACK
   ICL> <individual commands>
   ICL> HELP <command>
\end{verbatim}
\end{small}


\item [from DCL:]\hfill

\begin{small}
\begin{verbatim}
   $ CCDPACK
   $ CCDHELP
   $ (any CCDPACK commands)
\end{verbatim}
\end{small}

\end{description}
\end{quote}

\newpage

\section{CONVERT --- Data format conversion}

\vspace{-11mm}

\hfill [\xref{SUN/55}{sun55}{}]

\vspace{5mm}

This package converts data between the Starlink standard $n$-dimensional data
format (NDF) and other formats.
Currently, it can handle three data formats:

{\small
\begin{quote}
\begin{itemize}
\item DIPSO.
\item FIGARO (version 2).
\item INTERIM (BDF).
\end{itemize}
\end{quote}
}

{\em To run CONVERT:}\hfill
\begin{quote}
\begin{description}

\item [from ICL:]\hfill

\begin{small}
\begin{verbatim}
   ICL> CONVERT
   ICL> (any of the CONVERT programs)
\end{verbatim}
\end{small}

\item [from DCL:]\hfill

\begin{small}
\begin{verbatim}
   Simply type the name of the conversion program you want.
\end{verbatim}
\end{small}

\end{description}
\end{quote}

{\em Demonstration:}\hfill
\begin{quote}
\begin{description}
\item [] This will convert an NDF file into BDF format, as used by the
earlier Interim environment.

\begin{small}
\begin{verbatim}
   ICL> CONVERT
   ICL> NDF2BDF
   NDF - Name of NDF to be converted > adam_examples:image
   BDF - BDF filename > image
   ICL>
\end{verbatim}
\end{small}

A file IMAGE.BDF will have been created in your current directory.
\end{description}
\end{quote}

\newpage

\section{DAOPHOT --- Stellar photometry}

\vspace{-11mm}

\hfill [\xref{SUN/42}{sun42}{}]

\vspace{5mm}

A {\em stellar photometry} package written by Peter Stetson at the Dominion
Astrophysical Observatory, Victoria, B.C., Canada and adapted for use under
ADAM.
It performs the following tasks:

{\small
\begin{quote}
\begin{itemize}
\item Finding objects.
\item Aperture photometry.
\item Obtaining the point spread function.
\item Profile-fitting photometry.
\end{itemize}
\end{quote}

}
Profile fitting in crowded regions is performed iteratively, which improves the
accuracy of the photometry.
It does not directly use an image display (which aids portability), although
three additional routines allow results to be displayed on an image device.
It uses image data in NDF format.

{\em To run DAOPHOT:}\hfill
\begin{quote}
\begin{description}

\item [from ICL:] \hfill

\begin{small}
\begin{verbatim}
   Cannot be run from ICL.
\end{verbatim}
\end{small}

\item [from DCL:] \hfill

\begin{small}
\begin{verbatim}
   $ DAOPHOT
   Command: HELP
   Command: ...
   Command: EXIT
\end{verbatim}
\end{small}

\end{description}
\end{quote}
N.B. If you enter `DAOPHOT' just to see what happens, you will be disappointed
to receive the message {\tt `Value unacceptable --- please re-enter'.}
You will need to read the documentation and user manual before you can
make any progress with this program.

\newpage

\section{FIGARO --- General spectral reduction}

\vspace{-11mm}

\hfill [\xref{SUN/86}{sun86}{}]

\vspace{5mm}

This is a {\em general data reduction} system written by Keith Shortridge at
Caltech and the AAO.
Most people find it of greatest use in the reduction of spectroscopic data,
though it also has powerful image and data cube manipulation facilities.
Starlink recommends FIGARO as the most complete spectroscopic data reduction
system in the Collection.
Examples of its facilities are:

{\small
\begin{quote}
\begin{itemize}
\item Analyse absorption lines interactively.
\item Aperture photometry.
\item Calibrate B stars.
\item Calibrate flat fields.
\item Calibrate using flux calibration standards.
\item Calibrate wavelengths of spectra.
\item Correct S-distortion.
\item Extract spectra from images and images from data cubes, and insert
 spectra into images and images into data cubes.
\item Extract spectra from images taken using optical fibres.
\item Fit Gaussians to lines in a spectrum interactively.
\item Generate and apply a spectrum of extinction coefficients.
\item Input, output, and display data.
\item Look at the contents of data arrays, other than graphically.
\item Manipulate complex data structures (mainly connected with Fourier
 transforms).
\item Manipulate data arrays `by hand'.
\item Manipulate images and spectra (arithmetic and more complicated).
\item Process data taken using FIGS (the AAO's Fabry-Perot Infra-Red Grating
Spectrometer).
\item Process echelle data, in particular the UCL echelle in use at the AAO.
\end{itemize}
\end{quote}
}

At present, a number of related packages are bundled with FIGARO.
In future, these may be released as separate items.
They include:
\begin{description}

\item [TWODSPEC]\index{TWODSPEC} \hfill [\xref{SUN/16}{sun16}{}]

This reduces and analyses long-slit and optical-fibre array spectra.
A number of its functions are useful outside the area of spectroscopy.
The main application areas are:

{\small
\begin{quote}
\begin{itemize}

\item Line profile analysis; LONGSLIT analyses calibrated long-slit spectra.
For example, it can fit Gaussians, either manually or automatically, in batch.
It can handle data with two spatial dimensions, such as TAURUS data.
\mbox{FIBDISP} provides further options useful for such data, although it is
primarily designed for fibre array data.
An extensive range of options is available, especially for output.
\item Two-dimensional arc calibration.
\item Geometrical distortion correction; S-distortion and Line curvature.
\item Conversion between FIGARO and IRAF data formats.
\item Display programs.
\item Removing continua.
\end{itemize}
\end{quote}
}

\end{description}

{\em To run FIGARO:}\hfill
\begin{quote}
\begin{description}

\item [from ICL:]\hfill

\begin{small}
\begin{verbatim}
   ICL> HELP FIGARO
   ICL> FIGARO
   ICL> HELP FIGARO CLASSIFIED
   ICL> (any FIGARO commands)
\end{verbatim}
\end{small}

\item [from DCL:]\hfill

\begin{small}
\begin{verbatim}
   $ FIGARO
   $ HELP FIGARO
   $ HELP FIGARO CLASSIFIED
   $ (any FIGARO commands)
\end{verbatim}
\end{small}

\end{description}
\end{quote}

\newpage

\section{IRCAM --- Infrared camera data reduction}

\vspace{-11mm}

\hfill [\xref{SUN/41}{sun41}{}]

\vspace{5mm}

This package reduces, displays, and analyses 2-dimensional images from the
{\em UKIRT infrared camera} (IRCAM).
The image data reduction facilities available are:

{\small
\begin{quote}
\begin{itemize}
\item Mathematical and statistical operations.
\item Size changing and mosaicking.
\item Inspection.
\item Interpolation.
\item Smoothing.
\item Feature enhancement.
\item Bad pixel removal.
\item Polarimetry.
\item Median filtering of flat-fields.
\end{itemize}
\end{quote}
}

The graphics and image display facilities available are:

{\small
\begin{quote}
\begin{itemize}
\item Image display of various types (PLOT, CONTOUR, NSIGMA, RANPLOT).
\item Display cursor position and value.
\item Colour control.
\item Line graphics such as 1-dimensional cuts/slices through images and
 contour maps.
\item Annotation.
\end{itemize}
\end{quote}
}

{\em To run IRCAM:}\hfill
\begin{quote}
\begin{description}

\item [from ICL:]\hfill

\begin{small}
\begin{verbatim}
   Cannot be run from ICL (it uses the older ADAMCL).
\end{verbatim}
\end{small}

\item [from DCL:]\hfill

\begin{small}
\begin{verbatim}
   $ IRCAM_SETUP
   $ IRCAM_CLRED
     (you are then asked to say where your data are
      and which plotting device you want to use)
   Ircam-CLRED : > ?
     (produces a list of commands)
   Ircam-CLRED : > (select commands)
       .
   Ircam-CLRED : > EXIT
\end{verbatim}
\end{small}

\end{description}
\end{quote}

\newpage

\section{KAPPA --- Kernel applications}

\vspace{-11mm}

\hfill [\xref{SUN/95}{sun95}{}]

\vspace{5mm}

The Kernel Application Package runs under ADAM, using the NDF data format, and
provides {\em general-purpose applications}.
It is the backbone of the software reorganization around the ADAM environment,
and its applications integrate with other packages such as PHOTOM, PISA, and
FIGARO.
It is usable as a single large program from the ADAM command language ICL, or
as individual applications from DCL.

It handles bad pixels, and processes quality and variance information within
NDF data files.
Although oriented towards image processing, many applications will work on
NDFs of arbitrary dimension.
Its graphics are device independent.
Currently, KAPPA has about 140 commands and provides the following facilities
for data processing:

{\small
\begin{quote}
\begin{itemize}
\item Generation of NDFs and ASCII tables by up-to-date FITS readers.
\item Generation of test data, and NDF creation from ASCII files.
\item Setting NDF components.
\item Arithmetic, including a powerful application that handles expressions.
\item Editing pixels and regions, including polygons and circles, and
 re-flagging bad pixels by value or by median filtering.
\item Configuration changing: flip, rotate, shift, subset, dimensionality.
\item Image mosaicking; normalization of NDF pairs.
\item Compression and expansion of images.
\item Filtering: box, Gaussian, and median smoothing; very efficient Fourier
 transform, maximum-entropy deconvolution.
\item Surface fitting.
\item Statistics, including ordered statistics, histogram, pixel-by-pixel
 statistics over a sequence of images.
\item Inspection of image values.
\item Centroiding of features, particularly stars; stellar PSF fitting.
\item Detail enhancement via histogram equalization, Laplacian convolution,
 edge enhancement via a shadow effect, thresholding.
\end{itemize}
\end{quote}
}

There are also many applications for data visualization:

{\small
\begin{quote}
\begin{itemize}
\item Use of the graphics database, AGI, to pass information about pictures
 between applications.
 Facilities for the creation, labelling and selection of pictures, and
 obtaining world and data co-ordinate information from them.
\item Image and greyscale plots with a selection of scaling modes and many
 options such as axes.
\item Creation, selection, saving and manipulation of colour tables and
 palettes (for axes, annotation, coloured markers and borders).
\item Snapshot of an image display to hardcopy.
\item Blinking and visibility of image-display planes.
\item Line graphics: contouring, including overlay; columnar and hidden-line
 plots of images; histogram; line plots of 1-d arrays, and multiple-line plots
 of images; slices through an image.
 There is some control of the appearance of plots.
\end{itemize}
\end{quote}
}

{\em To run KAPPA:}\hfill
\begin{quote}
\begin{description}
\item [] See Chapter \ref{C_tour} for a demonstration of KAPPA.
\end{description}
\end{quote}

\newpage

\section{PHOTOM --- Aperture photometry}

\vspace{-11mm}

\hfill [\xref{SUN/45}{sun45}{}]

\vspace{5mm}

This performs aperture photometry.
It has two basic modes of operation:

{\small
\begin{quote}
\begin{itemize}
\item Using an interactive display to specify the positions for the
 measurements.
\item Obtaining those positions from a file.
\end{itemize}
\end{quote}
}

The aperture is circular or elliptical, and the size and shape can be varied
interactively on the display, or by entering values from the keyboard or
parameter system.
The background sky level can be sampled interactively by manually positioning
the aperture, or automatically from an annulus surrounding the object.

{\em To run PHOTOM:}\hfill
\begin{quote}
\begin{description}

\item [from ICL:]\hfill

\begin{small}
\begin{verbatim}
    ICL> HELP PHOTOM
    ICL> PHOTOM
    IN - NDF containing input image /@ramp1/ > adam_examples:image
    COMMAND - PHOTOM /'Values'/ > H (for help)
    COMMAND - PHOTOM /'Values'/ > (an option from the menu)
    COMMAND - PHOTOM /'Values'/ > E
\end{verbatim}
\end{small}

\item [from DCL:]\hfill

\begin{small}
\begin{verbatim}
    $ RUN PHOTOM_DIR:PHOTOM
    IN - NDF containing input image /@ramp1/ > adam_examples:image
    .
    . (as above)
\end{verbatim}
\end{small}

\end{description}
\end{quote}

\newpage

\section{PISA --- Object finding and analysis}

\vspace{-11mm}

\hfill [\xref{SUN/109}{sun109}{}]

\vspace{5mm}

The Position, Intensity and Shape Analysis package, PISA, locates and
parameterizes objects in an image frame.
The core of the package is a routine which performs image analysis on a
2-dimensional data frame.
It searches for objects having a minimum number of connected pixels above a
given threshold, and extracts the image parameters (position, intensity, shape)
for each object.
The parameters can be determined using thresholding techniques, or an analytical
stellar profile can be used to fit the objects.
In crowded regions, deblending of overlapping sources can be performed.

The package derives from the APM IMAGES routine originally written by Mike
Irwin at the University of Cambridge to analyse output from the Automatic
Photographic Measuring system.

{\em To run PISA:}\hfill
\begin{quote}
\begin{description}

\item [from ICL:]\hfill

\begin{small}
\begin{verbatim}
    ICL> HELP PISA
    ICL> PISA
    ICL> <individual commands>
    ICL> HELP <command>
\end{verbatim}
\end{small}

\item [from DCL:]\hfill

\begin{small}
\begin{verbatim}
    $ PISA
    $ <individual commands>
    $ HELP <command>
\end{verbatim}
\end{small}

\end{description}
\end{quote}

{\em Demonstration:}\hfill
\begin{quote}
\begin{description}
\item [] This example performs isophotal analysis with deblending of overlapped
images on a frame containing a mixture of stars and galaxies.
The results are then plotted on a suitable device.

\begin{small}
\begin{verbatim}
   ICL> PISA
        Welcome to PISA ...
   ICL> PISAFIND
   IN - NDF containing input image /.../ > PISA_DIR:FRAME
    Analysing whole image
   MINPIX - Minimum pixel size for images (typically 4-16) > 6
   METHOD - Intensity analysis ( 0=Isophotal, 1=Total, 2=Profile ) /0/ > 0
    Estimated background level =   492.2
    Background standard deviation =     7.4
   BACKGROUND - Background (global sky) value /492.17/ >
   THRESH - Threshold for analysis (data units) /18.61135/ >

    Total number of positive images = 118
    The results have been written to PISAFIND.DAT
        and PISASIZE.DAT
   ICL> PISAPLOT
   RESULTS - File of PISAFIND parameterised data /@PISAFIND.DAT/ >
   DEVICE - Name of graphics device /@IKON/ >
   ICL>
\end{verbatim}
\end{small}

\end{description}
\end{quote}

\newpage

\section{SCAR --- Star catalogue database system}

\vspace{-11mm}

\hfill [\xref{SUN/70}{sun70}{}, \xref{106}{sun106}{}]

\vspace{5mm}

The Starlink Catalogue Access and Reporting system is a relational database
management system.
It was designed principally for extracting information from astronomical
catalogues, but it can be used to process any data stored in relational form.
A large number of catalogues are available, including the IRAS catalogues.
For general database requirements, REXEC may be preferable.
SCAR can perform the following functions:

{\small
\begin{quote}
\begin{itemize}
\item Extract data from a catalogue using selection criteria.
\item Manipulate data using various statistical and plotting routines.
\item Output data from a catalogue.
\item Put a new catalogue into the database.
\item Search catalogues and generate reports on what has been found.
\item Sort, merge, join, and difference catalogues.
\item Plot sources in a gnomonic (tangent plane) or Aitoff (equal area)
 projection.
\item Analyse the fields of a catalogue by scatterplot and histogram.
\item Calculate new fields.
\end{itemize}
\end{quote}
}

A distinctive feature of SCAR is the use of index files
which you can create and which contain pointers to rows in one or more
catalogues.
This is a compact and flexible method of accessing catalogues;
for example, a very large catalogue may be physically ordered by declination,
but you can create an index giving access to it ordered by flux.

{\em To run SCAR:}\hfill
\begin{quote}
\begin{description}

\item [from ICL:]\hfill

\begin{small}
\begin{verbatim}
   $ SCARSTART
   .
   ICL> HELP SCAR
   ICL> SCAR
   ICL> SCAR_HELP  (for help on SCAR)
   ICL> CAR_HELP   (for help on CAR commands)
   ICL> CAT_HELP   (for help on catalogues)
   ICL> (any SCAR commands)
\end{verbatim}
\end{small}

\item [from DCL:]\hfill

\begin{small}
\begin{verbatim}
   $ SCARSTART
   $ (any SCAR commands)
\end{verbatim}
\end{small}

\end{description}
\end{quote}

{\em Demonstration:}\hfill
\begin{quote}
\begin{description}
\item [] A script is available which demonstrates some of the features of SCAR.
It can be invoked by typing:

\begin{small}
\begin{verbatim}
   ICL> LOAD SCAR_DOC_DIR:SCAR_SCRIPT
\end{verbatim}
\end{small}

It is suggested that you run this in an empty directory so that you can
identify the files which have been created.
\end{description}
\end{quote}

\newpage

\section{SPECDRE --- Spectroscopy data reduction}

\vspace{-11mm}

\hfill [\xref{SUN/140}{sun140}{}]

\vspace{5mm}

A package for spectroscopy data reduction and analysis.
It fills the gap between FIGARO and KAPPA --- on the one hand, all its routines
conform with Starlink's concept of bad values and variances, on the other hand,
they offer spectroscopy applications hitherto available only in FIGARO.
In general, it can work on data sets with seven or less axes.
Often an application will take just a one-dimensional subset as a spectrum.
In most cases the spectroscopy axis can be any axis, but for some applications
must be the first axis.

{\em To run SPECDRE:}\hfill
\begin{quote}
\begin{description}

\item [from ICL:]\hfill

\begin{small}
\begin{verbatim}
   ICL> HELP SPECDRE
   ICL> SPECDRE
   ICL> <individual commands>
   ICL> HELP <command>
\end{verbatim}
\end{small}

\item [from DCL:]\hfill

\begin{small}
\begin{verbatim}
   <not specified>
\end{verbatim}
\end{small}

\end{description}
\end{quote}

\newpage

\section{SST --- Simple software tools}

\vspace{-11mm}

\hfill [\xref{SUN/110}{sun110}{}]

\vspace{5mm}

The Simple Software Tools package helps produce software and documentation,
with particular emphasis on ADAM programming using Fortran~77.
It performs fairly simple manipulations of software, and also tackles some of
the commonly encountered problems which are not catered for in the more
sophisticated commercial software tools (such as FORCHECK and VAXset)
available on Starlink.

The main purpose of the first version is to extract information from subroutine
`prologues', and to format it to produce various forms of user documentation.
A simple source-code and comment statistics tool is also included.

There are five applications:

{\small
\begin{quote}
\begin{itemize}
\item Convert `old-Style' ADAM/SSE prologues to `new-style' ones.
\item Produce \LaTeX\ documentation.
\item Produce Help libraries.
\item Produce STARLSE package definitions.
\item Produce source-code statistics.
\end{itemize}
\end{quote}
}

{\em To run SST:}\hfill
\begin{quote}
\begin{description}

\item [from ICL:]\hfill

\begin{small}
\begin{verbatim}
   ICL> HELP SST
   ICL> SST
   ICL> HELP <any SST command>
   ICL> (any of the SST commands)
\end{verbatim}
\end{small}

\item [from DCL:]\hfill

\begin{small}
\begin{verbatim}
   $ SST
   $ (any of the SST commands)
\end{verbatim}
\end{small}

\end{description}
\end{quote}

{\em Demonstration:}\hfill
\begin{quote}
\begin{description}
\item [] This simple demonstration will create a file containing statistics
about source and comment lines in a Fortran program; the one used happens to
be the one which is run in the demonstration.

\begin{small}
\begin{verbatim}
   ICL> SST
   ICL> FORSTATS
   IN - Input file(s) /'*.FOR'/ > CONVERT_DIR:CONVERT.FOR
   .
   . (messages from FORSTATS)
   .
   ICL> $ TYPE FORSTATS.LIS
   .
   . (the output which was written by FORSTATS)
   .
   ICL>
\end{verbatim}
\end{small}

\end{description}
\end{quote}

\newpage

\section{TSP --- Time-series and polarimetry analysis}

\vspace{-11mm}

\hfill [\xref{SUN/66}{sun66}{}]

\vspace{5mm}

This is a data reduction package for time-series and polarimetric data.
These facilities are missing from most existing data reduction packages which
are usually oriented towards either spectroscopy or image processing or both.
Currently TSP can process the following data:

{\small
\begin{quote}
\begin{itemize}
\item Spectropolarimetry obtained with the AAO Pockels cell
 spectropolarimeter in conjunction with either IPCS or CCD detectors.
\item Time series polarimetry obtained with the Hatfield Polarimeter
 at either UKIRT or AAT.
\item Time series polarimetry obtained with the University of Turku
 UBVRI polarimeter.
\item Five channel time series photometry obtained with the Hatfield
 polarimeter at the AAT in its high speed photometry mode.
\item Time series infrared photometry obtained with the AAO Infrared
 Photometer Spectrometer (IRPS).
\item Time series optical photometry obtained using the HSP3 high speed
photometry package at the AAT.
\end{itemize}
\end{quote}
}

{\em To run TSP:}\hfill
\begin{quote}
\begin{description}

\item [from ICL:]\hfill

\begin{small}
\begin{verbatim}
   ICL> HELP TSP
   ICL> TSP
   ICL> (any TSP command)
\end{verbatim}
\end{small}

\item [from DCL:]\hfill

\begin{small}
\begin{verbatim}
   Cannot be run from DCL.
\end{verbatim}
\end{small}

\end{description}
\end{quote}

{\em Demonstration:}\hfill
\begin{quote}
\begin{description}
\item [] The following example shows the use of the PPLOT command to plot a
polarization spectrum.
The SN1987A data file is included with the software, so you can use this
command to check that TSP is working.

\begin{small}
\begin{verbatim}
    ICL> PPLOT
    Loading TSP_DIR:TSP into xxxxTSP
    INPUT - Stokes Data to Plot > TSP_DIR:SN1987A
    BINERR - Error per bin (per cent) /0.1/ >
    AUTO - Autoscale Plot /YES/ >
    LABEL - Label for plot /''/ > SN1987A  1987 Sep 2
    DEVICE - Plot Device > IKON
    ICL>
\end{verbatim}
\end{small}

\end{description}
\end{quote}
