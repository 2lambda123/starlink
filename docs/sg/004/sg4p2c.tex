\chapter{Running Applications}
\label{C_runap}
\markboth{Users}{\stardocname}

\section{ADAM applications}
\label{S_aa}
\markboth{Users}{\stardocname}

An ADAM application is a program which uses ADAM subroutines.
From the user's point of view, the most important of these is the Parameter
system since this controls (in conjunction with the user language) the
interaction between the program and the user.
The Parameter system uses a file called an {\em Interface File} which stores
information about a program's parameters.

Many users will only use applications which have been written by someone else.
In this case, everything necessary to enable them to be used should already
have been carried out.
This includes the preparation of the interface files and the definition of
commands.
All you need do is carry out a specified startup procedure.
However, if you have written your own application, you will need to do
some preparation of your own before using it.

\section{The running process}
\label{S_runpro}

The process of running an ADAM application can be summarized as follows
\begin{quote}
\begin{description}

\item [Read the documentation] ---

I know it's hard, but it often really does help to read about the program
you intend to use.
Program documentation often tells you what the program does, what data it
uses, and what its parameters are and how to specify them.
Why not try it?

\item [ADAMSTART] ---

Don't forget to enter {\small\tt ADAMSTART} --- people do.

\item [Select a command language] ---

The options are DCL, ICL, and SMS at the moment.
ICL is the one we recommend.
SMS is usually used only for data acquisition and is not covered in this
guide.
Some packages won't run under some languages --- Chapter~\ref{C_applic} tells
you which ones.

\item [Start the package] ---

Some packages need to have various definitions and logical names defined before
you can use them.
For example, before you can use ASTERIX you must enter the DCL symbol
{\small\tt ASTSTART}.
Once again, Chapter~\ref{C_applic} tells you what to do for specific
applications.

\item [Select programs] ---

Most applications contain many programs.
You must select the programs you want to use.
For applications provided by other people, this will normally involve just
typing the name of the program.
However, if you have written your own application, you will need to define
the appropriate commands required to run your programs.
Applications which have large numbers of programs can be executed more
efficiently if they are grouped together into a {\em Monolith}.
These are considered below.

\item [Specify parameter values] ---

Once you start running applications you will need to control them.
This is done by specifying parameter values --- names of data files, values,
options, and so on.
This is the most complex part of the process and is discussed in
Chapter~\ref{C_prompts}.

\end{description}
\end{quote}

\section{Selecting programs}
\label{S_selprog}

You normally select programs just by typing their names.
However, for this to work, the program name must have been defined as a DCL
symbol or an ICL command.
Also, there might be a delay the first time you run a program in an application.
This is usually because a large {\em Monolith} is being loaded.

\subsection{Defining commands to run programs}

To run programs under VMS, their names are defined as DCL symbols so that typing
the name runs the program.
The symbol definitions are executed by running a command procedure.

Similarly, to run programs under ICL, their names are defined as ICL commands
using the DEFINE command described in Chapter~\ref{C_commands}.
Usually, the command definitions are executed by running an ICL procedure which
contains them.
Before a program can be run under ICL, a command to run it must be defined.

\subsection{Monoliths}

Programs are sometimes stored together in structures called {\em monoliths}
which are loaded for execution as a single entity.
In this case a command must be defined for each program in the monolith.
When you use an application package, the commands needed to execute its programs
are normally defined for you when you start up the package.
Thus:

\begin{small}
\begin{verbatim}
    ICL> KAPPA
\end{verbatim}
\end{small}

will cause all the commands required to run the KAPPA programs to be defined.
What happens is that a file called {\small\tt KAPPA\_DIR:KAPPA.PRC}\footnote{
The file type `.PRC' is now obsolete.  The normal file type for command files
is `.ICL'.} containing the required DEFINE commands is automatically input to
ICL for execution.
The commands look like this:

\begin{small}
\begin{verbatim}
    define add       kappa_dir:kappa
    define aperadd   kappa_dir:kappa
            ...
    define zaplin    kappa_dir:kappa
\end{verbatim}
\end{small}

There are also commands to display the initial messages output by the package.
{\small\tt KAPPA.PRC} is an example of an ICL {\em Command File} which is
described in more detail in Chapter~\ref{C_iclprog}.
Notice that all the commands refer to the same executable image (the monolith),
and that the directory in which this is to be found is specified explicitly so
ADAM\_EXE is not used.

Monoliths are useful because they enable lots of programs to be loaded within
a single process creation.
Since process creation is quite slow under VMS, you save time by doing it
only once.
The advantage of monoliths is that they reduce the time required to create
processes.
The disadvantage is the time required for the loading a large monolith
instead of a small program; however, you only have to do it once.
