\chapter{Specifying Parameter Values}
\label{C_prompts}

\section{Introduction}
\label{S_p2dint}

ADAM programs are controlled by specifying parameter values.
You can specify them in two ways:

\begin{quote}
\begin{description}

\item [On a command line] ---

In this case, the value will be processed by a command language interpreter
(DCL, ICL, or SMS) before being passed to the parameter system.
So, for example, if you were talking to ICL you could include ICL functions
in your values.

\item [In response to a prompt] ---

In this case you are talking directly to the program through the ADAM parameter
system, and you could not use ICL functions.

\end{description}
\end{quote}

The thing to remember is that when you respond to prompts you are talking
directly to the ADAM parameter system, whereas when you specify values on the
command line, they will be processed by the language interpreter and then passed
to the parameter system.

Usually, it is not necessary to specify parameters on the command line --- you
can just respond to prompts.
However, there are three reasons why you may wish to do so:
\begin{itemize}
\item To avoid waiting for prompts.
\item To set values for parameters which normally are not prompted.
\item To use a command safely in a procedure.
\end{itemize}

\section{Command line specification}
\label{S_cls}

This section refers to ICL commands.
However, you can run ADAM programs by using DCL commands.
These suffer from the following restrictions compared to ICL:
\begin{itemize}
\item It is not possible to use ICL expressions.
\item It is not possible to return parameter values to the command language, and
hence on to applications.
\item Single quotes cannot be used, except where they are needed to force
expansion of DCL symbols.
Double quotes can be used.
\item It is not possible to use an ! unless it is in a character string
enclosed in double quotes.
Normally this character indicates a comment.
\end{itemize}
There are two ways in which you can specify parameter values in an ICL command:
\begin{itemize}
\item By using a command which has been defined to run an ADAM program.
 (The definition is done by a DEFINE command.)
\item By using the SETPAR command.
(You can also store a parameter value in an ICL variable by using the
analogous GETPAR command.)
\end{itemize}
The first way is the most usual and will now be considered further.

You will remember that a program's parameters may be assigned values on the
command line:

\begin{small}
\begin{verbatim}
    ICL> command   p1   p2   ...
\end{verbatim}
\end{small}

where {\tt command} is a commmand that has been defined to run the program.
This method is particularly useful for running programs in VMS batch mode,
and for specifying values for parameters that would otherwise be defaulted.
Normally, command-line specification will prevent prompting for a parameter's
value unless there is an error in the given value, say giving a character
string for a parameter of type \_REAL.

There are two ways in which parameter values may be specified on the command
line:
\begin{itemize}
\item by keyword
\item by position
\end{itemize}
The two forms may not be mixed.

The keyword method is usually preferable because the position method has these
disadvantages:
\begin{quote}
\begin{itemize}
\item Values must be given for {\em every} parameter preceding the
last parameter for which a value is required; values for parameters after
this one can be omitted.
\item Some programs have many parameters and it is tedious to enter all the
intermediate values between the ones you want to define, and difficult to
remember them all.
\item Some parameters may not have defined positions because they are normally
defaulted.
\end{itemize}
\end{quote}

\subsection{Keyword method}

In the keyword method, a parameter is identified by a {\em keyword} (specified
in the program's interface file) and a value is assigned to it using the syntax:

\begin{small}
\begin{verbatim}
    keyword=value
\end{verbatim}
\end{small}

The keyword is the name by which the parameter is known to the user.
This is usually, but not necessarily, the same as the parameter name.

In some cases (such as when setting a switch) the keyword itself is all that
needs to be specified:
Here is an example:

\begin{small}
\begin{verbatim}
    ICL> DEFPIC CURPIC FRACTION=0.4
\end{verbatim}
\end{small}

The keyword and data type corresponding to a particular parameter is specified
in the program's interface file:

\begin{small}
\begin{verbatim}
    interface DEFPIC
      parameter FRACTION
        keyword 'FRACTION'
        type _REAL
        ...
      parameter CURPIC
        keyword 'CURPIC'
        type _LOGICAL
        ...
    endinterface
\end{verbatim}
\end{small}

Here, the keyword {\tt CURPIC} is associated with a logical type parameter ---
just specifying its name sets its value to `TRUE', specifying `NOCURPIC'
sets its value to `FALSE'.
The keyword {\tt FRACTION} is associated with a real type parameter which is
given the value 0.4.
Keyword forms may appear in any order on the command line, and are ignored
when calculating the `position' of other parameters.

\subsection{Position method}

In the position method, the position of a value on the command line identifies
the parameter which is being given that value.
Here is an example:

\begin{small}
\begin{verbatim}
     ICL> ADD HISIMAGE HERIMAGE THEIRIMAGE
\end{verbatim}
\end{small}

The parameter corresponding to a particular position is specified in the
program's interface file:

\begin{small}
\begin{verbatim}
    interface ADD
      parameter IN1
        position 1
        ...
      parameter IN2
        position 2
        ...
      parameter OUT
        position 3
        ...
      parameter TITLE
        position 4
        ...
    endinterface
\end{verbatim}
\end{small}

Here, the first parameter IN1 is given the value {\tt HISIMAGE},
the second parameter IN2 is given the value {\tt HERIMAGE}, and the third
parameter OUT is given the value {\tt THEIRIMAGE}.
The fourth parameter TITLE is not given a value.

\subsection{ICL variables and functions}

Command-line specification can be used to put the values of ICL variables
and functions into programs.
A common situation where this is useful is when processing a sequence of files
in the same way.
For example, consider this ICL procedure:

\begin{small}
\begin{verbatim}
    PROC MULTISTAT
      LOOP FOR I=1 TO 10
        FILE = '@' & SNAME('REDX',I,2)
        STATS2D INPIC=(FILE) AGAIN=F XSTART=1 YSTART=1 XSIZE=62 YSIZE=58
      END LOOP
    END PROC
\end{verbatim}
\end{small}

This obtains the statistics of ten images named {\small\tt REDX01},
{\small\tt REDX02}, \dots, {\small\tt REDX10}.
The parameter with keyword INPIC is assigned the value of the ICL variable FILE.
Variable FILE is in parentheses because syntactically it is an ICL expression.
The ICL function SNAME is used to construct the appropriate file name, which must
be preceded by an `{\tt @}'.

Here is another example of the use of ICL variables in commands which uses KAPPA
to `flat field' a series of CCD frames:

\begin{small}
\begin{verbatim}
    PROC FLATFIELD
      INPUT Which flat field frame?: (FF)
      FF = '@' & (FF)
      INPUTI Number of frames:  (NUM)
      LOOP FOR COUNT=1 TO (NUM)
        INPUT Enter frame to flat field: (IMAGE)
        TITLE = 'Flat field of ' & (IMAGE)
        IMAGE = '@' & (IMAGE)
        IMAGEOUT = (IMAGE) & 'F'
        PRINT Writing to (IMAGEOUT)
        DIV IN1=(IMAGE) IN2=(FF) OUT=(IMAGEOUT) TITLE=(TITLE)
      END LOOP
    END PROC
\end{verbatim}
\end{small}

Here, the parameters of the DIV program are given values stored in ICL string
variables.

\section{Prompts and suggested values}
\label{S_pd}

If program parameters are not given values on the command line, values
will be prompted for (unless this is suppressed).
If the value supplied is not acceptable, the prompt will be repeated.
If no acceptable value is supplied after five prompts, a `Null value'
is given to the program.

The general form of a prompt is:

\begin{small}
\begin{verbatim}
    <keyword> - <prompt> /<suggestion>/ >
\end{verbatim}
\end{small}

However, the {\small\tt /<suggestion>/} may not appear.

\subsection{Prompts without a suggested value}

The simplest type of prompt is one where the suggested value is omitted, for
example:

\begin{small}
\begin{verbatim}
    LOW - Lower limit for data >
\end{verbatim}
\end{small}

In this case, if you just press `carriage return', the prompt will be repeated
up to five times, and then a null value will be given to the program:

\begin{small}
\begin{verbatim}
    LOW - Lower limit for data >
    LOW - Lower limit for data >
    LOW - Lower limit for data >
    ...
\end{verbatim}
\end{small}

You should specify the value you want, for example:

\begin{small}
\begin{verbatim}
    LOW - Lower limit for data > 1
\end{verbatim}
\end{small}

This value will be accepted, and then the next prompt will be displayed.

\subsection{Prompts with a suggested value}

When a prompt includes a suggested value, you can override it by entering a
value of your own.
However, when this prompt appears again during a subsequent use of the program,
the suggested value may or may not be the same as the value you last entered.
Sometimes the current (last) value of the parameter is used, sometimes a fixed
default value is used, and sometimes a dynamic default value is calculated
by the program from the values of other parameters.
It all depends on the characteristics of the program and its interface file.
Parameter XDIM of program CREFRAME is an example of a parameter that uses the
current value:

\begin{small}
\begin{verbatim}
    XDIM - x dimension of output array /64/ > 109
    ...
    XDIM - x dimension of output array /109/ > 45
    ...
    XDIM - x dimension of output array /45/ > 16
    ...
\end{verbatim}
\end{small}

Current values of parameters are stored in a file in ADAM\_USER, so they
persist between ADAM sessions.
The file should not be deleted unless the old values are not required.

Parameter TYPED of program CREFRAME is an example of a parameter that always
offers the same suggested value, no matter what the current value is:

\begin{small}
\begin{verbatim}
    TYPED - Type of data to be generated /'GS'/ > RA
    ...
    TYPED - Type of data to be generated /'GS'/ > BL
    ...
    TYPED - Type of data to be generated /'GS'/ > GN
    ...
\end{verbatim}
\end{small}


\subsection{Global parameters}

Normally, a program's parameter values are only accessible by that program.
However, {\em global parameters} can be shared by many programs.
The use of global parameters as suggested values can reduce the need for typing
responses to prompts.
For example, in this case:

\begin{small}
\begin{verbatim}
    ICL> creframe
    ...
    OUTPIC - Image for output data > ramp4
    ICL> look
    INPIC - Image to be inspected/@ramp4/ >
    ...
\end{verbatim}
\end{small}

the frame created by CREFRAME can be displayed by LOOK without having to
retype its name.
This is because parameter {\small\tt OUTPIC} of program CREFRAME and parameter
{\small\tt INPIC} of program LOOK are both associated (in their interface files)
with the global parameter {\small\tt GLOBAL.DATA\_ARRAY}.
Thus, parameter {\small\tt INPIC} offers as a suggested value the object name
which was input as a value for the {\small\tt OUTPIC} parameter of a different
program.

In ICL you can create a global parameter using the CREATGLOBAL command, set its
value using the SETGLOBAL command, and get its value using the GETGLOBAL
command.

\subsection{Missing prompts}

Sometimes a prompt for a parameter value will not appear, even if you haven't
specified a value on the command line.
This is because the parameter system looks for a value for a parameter in a
number of places, as specified in a `search path' in the program's interface
file.
This feature is described in Chapter~\ref{C_parsys}.
A prompt will only appear if it cannot find a value in the search path, or if
the search path specifically asks the parameter system to generate a prompt.
This ability to supply values automatically enables programs with many options
to avoid burdening the user unnecessarily with large numbers of prompts.
If you want to supply your own value in these cases, you must specify a value
on the command line, or demand to be prompted (see below).

When prompts don't appear, the program documentation may tell you what the
unprompted parameters are, what options are available, and what the suggested
values are.
Another way is to look at the online help on a program's parameters; for
example:

\begin{small}
\begin{verbatim}
     ICL> kaphelp creframe param *
\end{verbatim}
\end{small}

gives details of all the parameters of the KAPPA program CREFRAME.

\subsection{Keywords --- RESET, PROMPT, ACCEPT}

Another way in which prompts and suggested values can be controlled is by use
of the keywords RESET, PROMPT, and ACCEPT in ICL commands.

\paragraph{RESET:}\hfill

The RESET keyword causes the suggested value of all parameters (apart from those
already specified before it on the command line) to be set to the original
values specified by the program and interface file.
For example, consider the prompt for the value of the XDIM parameter in the
following successive executions of the CREFRAME program:

\begin{small}
\begin{verbatim}
    ICL> creframe
    XDIM - x dimension of output array /64/ > 10
    ...
    ICL> creframe
    XDIM - x dimension of output array /10/ > 25
    ...
    ICL> creframe reset
    XDIM - x dimension of output array /64/ >
    ...
    ICL>
\end{verbatim}
\end{small}

When the CREFRAME program is executed for the first time, the suggested value of
the {\small\tt XDIM} parameter is `64'.
In the first prompt above, the value of {\small\tt XDIM} is set to `10'.
When CREFRAME is executed for the second time, the suggested value of
{\small\tt XDIM} has changed to `10', i.e.\ it takes its `current value' which
was set during the previous execution of CREFRAME.
In the third execution of CREFRAME we have included the keyword RESET in the
command line.
The effect is that when the value of parameter {\small\tt XDIM} is prompted
for, the suggested value has reverted to `64'.

RESET may be combined with the keywords PROMPT and ACCEPT described below.

\paragraph{PROMPT:}\hfill

The PROMPT keyword forces a prompt to appear for every program parameter not
specified on the command line.
For example, program CREFRAME has a parameter called {\small\tt TITLE} which
normally takes the value `KAPPA -- Creframe' without prompting.
However, if you use the keyword PROMPT in the command line, a prompt for
TITLE will be made:

\begin{small}
\begin{verbatim}
    ICL> creframe prompt
    ...
    TITLE - Title for output array /'KAPPA - Creframe'/ > `My title'
    ICL>
\end{verbatim}
\end{small}

\paragraph{ACCEPT:}\hfill

The ACCEPT keyword forces the parameter system to accept the suggested
value for {\em every} program parameter.
This must be used with care because some parameters may not have suggested
values and need a value to be specified in order to work properly.
For example, CREFRAME must have a value specified for parameter
{\small\tt OUTPIC}, the name of the output data object.
If you ran the program like this:

\begin{small}
\begin{verbatim}
    ICL> creframe accept
\end{verbatim}
\end{small}

the program would fail because it does not know where to write the output
data.
However, if you ran the program like this:

\begin{small}
\begin{verbatim}
    ICL> creframe outpic=ramp accept
\end{verbatim}
\end{small}

The program would generate an output image using suggested values for all the
parameters except {\small\tt OUTPIC}, and write the output to file
{\small\tt RAMP.SDF}.

Sometimes the keyword ACCEPT can be used alone.
For example, you could follow the above command by the command:

\begin{small}
\begin{verbatim}
    ICL> look accept
\end{verbatim}
\end{small}

and the central {\tt 7x7} array of the image created by CREFRAME would be
displayed on your terminal without any parameter values being prompted for.

The symbol `\verb+\+' has the same effect as ACCEPT, thus:

\begin{small}
\begin{verbatim}
    ICL> look \
\end{verbatim}
\end{small}

would have the same effect as the previous example --- and is much quicker
to type.

\section{Value formats}
\label{S_sv}

When specifying parameter values it is important to remember whether you are
talking to the ICL command interpreter or to the ADAM parameter system.
The differences are highlighted in Section~\ref{S_comp}

\subsection{ICL command parameters}

Parameter value specifications in an ICL command can take three forms:
\begin{itemize}
\item An expression enclosed in parentheses. E.g.

\begin{small}
\begin{verbatim}
    (2)        (X)         (Y*SIN(THETA))
\end{verbatim}
\end{small}

\item A string enclosed in quotes. E.g.

\begin{small}
\begin{verbatim}
    `String'               "Filtered and calibrated"
\end{verbatim}
\end{small}

\item Any sequence of characters not including a space, quote, comma or left
 parenthesis. E.g.

\begin{small}
\begin{verbatim}
    NGC1635    X           PI*2
\end{verbatim}
\end{small}

\end{itemize}
The first form is used to pass the value of an ICL expression to a command, or
to give a command a variable in which to return a value.
The other two forms both pass a string.

We can illustrate the three forms by using the PRINT command which prints its
parameter values on the terminal:

\begin{small}
\begin{verbatim}
    ICL> X=1.234
    ICL> PRINT (X)
    1.234000
    ICL> PRINT 'HELLO'
    HELLO
    ICL> PRINT HELLO
    HELLO
\end{verbatim}
\end{small}

{\em In many cases you do not need to use the quoted form of a string because
the simpler form will work.}
You need the quoted form for those cases in which you need to delimit a
{\em single} string containing a left parenthesis or spaces.
Consider the following example which contains a PRINT command which has several
parameters:

\begin{small}
\begin{verbatim}
    ICL> X=2
    ICL> PRINT The Square Root of (X) is (SQRT(X))
    The Square Root of          2 is 1.414214
\end{verbatim}
\end{small}

Since spaces are parameter separators, `\verb+The+', `\verb+Square+',
`\verb+Root+', and `\verb+of+' are all received by PRINT as separate parameters.
However, PRINT simply concatenates all its parameters with a space between each
pair.
Thus, the output written by PRINT is the string, just as you typed it.
Many other ICL commands which accept strings work in this way.
This means that strings with {\em single} spaces may not need quotes when used
as command parameters.
However, in general this is {\em not} true of commands which run ADAM programs
when strings are specified on the command line (see below).

The rules for specifying ICL command parameters can be summarized as follows:
\begin{itemize}
\item To pass the value of an expression or the name of a variable, enclose it
 in parentheses.
\item Anything not in parentheses is passed as a string, or as a sequence of
 strings if it contains spaces or commas.
\item Any string which does not fit these restrictions can be passed by placing
 it in quotes.
 Quotes may be included in a string by typing two consecutive quotes.
\end{itemize}

\subsection{ADAM program parameters}

Values can be specified for ADAM program parameters in an ICL command line and
in response to prompts.
However, some values are only acceptable when given in response to prompts;
they are meaningless to ICL.

\paragraph{Values acceptable in commands, and in response to prompts:}\hfill

\begin{quote}
\begin{description}
 \item [Numbers] --- These can be Integer, Real, or Double Precision, and are
  entered in the usual Fortran format.
  If necessary, the parameter system will perform the appropriate type
  conversion and truncation required to store the value in the form specified
  by the programmer.
 \item [Logicals] --- These can be represented by the words TRUE, FALSE,
  YES, NO, T, F, Y, N; regardless of case.
  ICL logical expressions can also be used, but only in commands.
 \item [Strings] --- These are strings of characters similar to the Fortran
  CHARACTER*n type.
  Strings can be enclosed in single (') or double ('') quotes.
  Quotes may be omitted where there is no ambiguity (see below).
  A special case of a String is an object or logical {\em Name}.
  These need careful treatment and are discussed below.
 \item [Arrays] --- Arrays of any of the above types may be represented by
  a list of values enclosed in square brackets, e.g.\ [1, 2, 3].
  Two dimensional arrays may be represented as [[1, 2], [3, 4]] etc.
\end{description}
\end{quote}

\paragraph{Values acceptable only in response to prompts:}\hfill

\begin{quote}
\begin{description}
\item [$<$CR$>$] ---
 Accept the suggested value.
\item [$<$TAB$>$] ---
 The suggested value is put in the input buffer.
 You can then edit it using the normal keyboard editing facilities before
 entering the value in the normal way.
\item [!] ---
 Return a STATUS value indicating `Null Parameter Value'.
 This will often cause a program to abort, but it can  force a suitable
 default value to be used when this is the most sensible action.
\item [!!] ---
 Abort the program, by convention.
\item [?] ---
 The parameter system will display the text specified in the `help' field of
 the parameter specification in the program's interface file.
 If this field is missing, a blank line will be output.
 You will be reprompted for the parameter, for example:

\begin{small}
\begin{verbatim}
    MODE - Method for selecting contour heights /'FR'/ > ?
      Options are: FR = explicit list; AU = automatic selection;
      AR = equal-area; LI = linear; MA = magnitude
    MODE - Method for selecting contour heights /'FR'/ >
\end{verbatim}
\end{small}

\item [??] ---
 This puts you into an interactive Help session, providing access to
 information about more than just the parameter being prompted for.
 It is used like the ordinary VMS Help facility.
 When you exit (by pressing $<$CR$>$ in response to a {\em Topic?} prompt) you
 will be returned to the original parameter prompt.
\end{description}
\end{quote}

\paragraph{Strings:}\hfill

You must be careful when specifying string values.
A string which contains a space or comma, or which begins with a left
parenthesis, should be enclosed in single or double quotes.
For instance, if you specify the string `Sum of 2 images' in an ICL command,
you could type:

\begin{small}
\begin{verbatim}
    ICL> TESTC `Sum of 2 images'
\end{verbatim}
\end{small}

in which case the parameter system would be handed the string `Sum of 2 images'
as the value of the first parameter of TESTC.
However, if you type:

\begin{small}
\begin{verbatim}
    ICL> TESTC Sum of 2 images
\end{verbatim}
\end{small}

the parameter system will think there are four parameters `\verb+Sum+',
`\verb+of+', `\verb+2+', and `\verb+images+'.

Quotes are not needed when a string is input in response to a prompt.
For example:

\begin{small}
\begin{verbatim}
    ICL> TESTC
    X - x value/'Default'/ > Sum of 2 images
\end{verbatim}
\end{small}

will successfully give parameter X the value `Sum of 2 images' (provided
X is of type \_CHAR or LITERAL).

The use of single quotes (') is dangerous when used from DCL as DCL
will interpret them as indicating symbol substitution.
Thus, when your input will be processed by the DCL command interpreter,
double quotes (") should be used.

\paragraph{Object names:}\hfill

Strings are used to specify the names of files, data objects, and devices
(graphics devices, tape drives etc.).
Components of data objects are specified by a dot notation, e.g.\ RAMP1.TITLE
refers to the TITLE component of the image RAMP1.

Normally, a name can be typed without quotes.
However, ambiguities can occur: a file name such as [ABC.DEF]FILE begins
with a `[' character and may be confused with an array specification.
Such ambiguities can be resolved by prefixing the name with an `@' character,
e.g.\ @[ABC.DEF]FILE means the file of this name.
As another example, in the prompt line:

\begin{small}
\begin{verbatim}
    PLTITL - Plot title /' '/ > @ADAM_USER:GALAXY.MYTITLE
\end{verbatim}
\end{small}

the parameter PLTITL is given the value of the string contained in the object
MYTITLE in {\tt GALAXY.SDF}.
If the `{\tt @}' were omitted, the string
`{\tt ADAM\_USER:GALAXY.MYTITLE}' would be used as the value of PLTITL.
Note that the file extension (.SDF) should not be included when giving the
name of a data file, otherwise the data system will look for the component
SDF within the object GALAXY.MYTITLE.

\paragraph{Logical names:}\hfill

Logical names must be defined with the /JOB qualifier.
Thus, if your IMAGE data files are stored in IMAGEDIR alias
{\tt DISK\$USER1:[XYZ.IRCAM.IMAGES]}:

\begin{small}
\begin{verbatim}
    $ DEFINE/JOB IMAGEDIR DISK$USER1:[XYZ.IRCAM.IMAGES]
\end{verbatim}
\end{small}

will enable you to respond to a prompt thus:

\begin{small}
\begin{verbatim}
    INPIC - Input image /@ramp4/ > imagedir:ngc1365
\end{verbatim}
\end{small}


\section{Comparison of ICL and ADAM parameters}
\label{S_comp}

Some values which are acceptable when specified as a parameter value in an
ICL command are {\em not} acceptable when used in response to prompts from
the ADAM parameter system.
These unacceptable values are ICL functions.
Thus, suppose program TESTR has been written to read a real number and print
it out.
We could do the following:

\begin{small}
\begin{verbatim}
    ICL> DEFINE TESTR TESTR
    ICL> TESTR (TAND(60))
    ICL> TESTR prints 1.732051
    ICL>
\end{verbatim}
\end{small}

However, an attempt to use the ICL function TAND in a response to a prompt
would be rejected:

\begin{small}
\begin{verbatim}
    ICL> TESTR
    X - x value/1.732051/ > (TAND(60))
    %RMS-F-SYN, file specification syntax error
    X - x value/1.732051/ >
\end{verbatim}
\end{small}

You will have to specify a value which is acceptable to the ADAM parameter
system.

Chapter~\ref{C_simpap} shows some example ADAM programs which read a
parameter value and display it on the user's terminal.
These can be used to examine how ICL and the ADAM parameter system process input
values.
Program TESTI reads a parameter value of type `\_INTEGER'.
Here is how this program responds to various inputs specified first on the
command line and then in response to a prompt:
\begin{center}
\begin{tabular}{|l|l|l|}
\hline
Input & \multicolumn{2}{c|}{Output} \\ \cline{2-3}
& Command line & Prompt \\
\hline
5 & 5 & 5 \\
5.6 & 6 & 6 \\
-1234 & -1234 & -1234 \\
tand(60) & File not found & File not found \\
(tand(60)) & 2 & file specification syntax error \\
(\%B100110) & 38 & file specification syntax error \\
\hline
\end{tabular}
\end{center}
In this table the first column shows the input value, the second shows the
value output by the program when the input value is given on the command
line, and the third shows the value output when the input is given in response
to a prompt.
Thus, the second and third columns show the parameter value presented to the
program, or the error status if no legal value could be obtained.

Notice that:
\begin{itemize}
\item The input value `5.6' is rounded up to `6' by the parameter system before
 being given to the program.
 This is because the value is obtained by calling routine PAR\_GET0I which
 converts an input value to integer type.
\item The ICL function `tand(60)' must be enclosed in parentheses if it is to be
 recognised as an expression, otherwise it is interpreted as an object name
 and the system goes looking for its container file.
\item The expressions `(tand(60))' and `(\%B100110)' are recognised as
 expressions when input on the command line, but not when input in response to
 prompts.
 When it is recognised, the value of tand(60) is rounded up from 1.732051 to 2.
\end{itemize}
The next table shows the response of program TESTR which reads the value of
a parameter of type `\_REAL' from the environment and displays this value on
the terminal:
\begin{center}
\begin{tabular}{|l|l|l|}
\hline
Input & \multicolumn{2}{c|}{Output} \\ \cline{2-3}
& Command line & Prompt \\
\hline
5 & 5 & 5 \\
5.6 & 5.6 & 5.6 \\
1.23D12 & 1.23E12 & 1.23E12 \\
(tand(60)) & 1.732051 & file specification syntax error \\
true & File not found & File not found \\
'true' & Conversion error & Conversion error \\
\hline
\end{tabular}
\end{center}
Notice that in this case the input values 5.6 and (tand(60)) are not rounded to
integers as the program is expecting real numbers.

The next table shows the behaviour of program TESTL which is like the previous
programs but whose parameter is of type `\_LOGICAL':
\begin{center}
\begin{tabular}{|l|l|l|}
\hline
Input & \multicolumn{2}{c|}{Output} \\ \cline{2-3}
& Command line & Prompt \\
\hline
5 & Conversion error & Conversion error \\
TRUE & TRUE & TRUE \\
T & TRUE & TRUE \\
n & FALSE & FALSE \\
(lge(a,b)) & TRUE & File not found \\
\hline
\end{tabular}
\end{center}
Acceptable inputs for logical values are:

\begin{small}
\begin{verbatim}
    TRUE  true   YES  yes    T    t    Y    y
    FALSE false  NO   no     F    f    N    n
\end{verbatim}
\end{small}

Notice that the ICL function `lge' is acceptable when specified on the
command line, but not in response to a prompt.

The final table shows the input to and output from program TESTC, which
is the same as the previous programs but has a parameter value of type
`\_CHAR':
\begin{center}
\begin{tabular}{|l|l|l|}
\hline
Input & \multicolumn{2}{c|}{Output} \\ \cline{2-3}
& Command line & Prompt \\
\hline
5.6 & 5.6 & 5.6 \\
tand(60) & tand(60) & tand(60) \\
(tand(60)) & 1.73205080 & (tand(60)) \\
yogi bear & yogi & yogi bear \\
'yogi bear' & yogi bear & yogi bear \\
"yogi bear" & yogi bear & yogi bear \\
(yogi bear) & Right parenthesis expected & (yogi bear \\
\hline
\end{tabular}
\end{center}
Notice that:
\begin{itemize}
\item For `command line' input, ICL expressions are {\em evaluated} and the
result is converted into a string before being given to the program.
For `prompt' input, the expressions are not evaluated but are just regarded
as strings of characters.
\item Spaces in strings separate parameters in `Command line' input when the
string is not enclosed in quotes or double quotes.
\end{itemize}
If you read Chapter~\ref{C_simpap} you will be able to create your own
versions of the TESTx programs and try out various inputs on the system
yourself.
