\chapter{The Data System}
\label{C_datsys}
\markboth{Programmers}{\stardocname}

In this chapter, a bottom-up approach is adopted --- the low-level packages are
described first, followed by the higher-level ones.

\section{HDS --- Hierarchical data system}
\label{S_usehds}
\markboth{Programmers}{\stardocname}

\subsection{Symbolic names and Include files}

Symbolic names should be used for important constant values in HDS programs to
make them clearer and to insulate them from possible future changes in their
values.
These symbolic names are defined by Fortran `include' files as shown in the
examples below.
The following include files are available: 

\begin{quote}
\begin{description}

\item [SAE\_PAR] ---
This file is not actually part of HDS, but it defines the global symbolic
constant SAI\_\_OK (the value of the status return indicating success) and will
be required by nearly all routines which call HDS.
It should normally be included as a matter of course.\footnote{Due to an
historical anomaly on VAX/VMS systems, the file SAE\_PAR also contains
definitions for the DAT\_\_\, symbolic constants which should properly reside
in DAT\_PAR.
To prevent multiple definitions occurring, DAT\_PAR is therefore an empty file
on VAX/VMS systems.
Thus, if SAE\_PAR has been included, the further inclusion of DAT\_PAR is
optional (but only on VAX/VMS).
Its inclusion is recommended, however, because this allows the same software
to be used on other systems without change.}

\item [DAT\_PAR] ---
Defines various symbolic constants for HDS.
These should be used whenever the associated value is required (typically this
is when program variables are defined)

\item [DAT\_ERR] ---
Defines symbolic names for the error status values returned by the DAT\_ 
and HDS\_ routines.

\item [CMP\_ERR] ---
Defines symbolic names for the additional error status values returned by
the CMP\_ routines.
\end{description}
\end{quote}

The symbolic names of these include files may be used directly on VAX/VMS
systems, but on Unix systems an explicit directory specification for the file
is normally also required, and the file name should appear in lower case.
Thus, to include the DAT\_PAR file on a VAX/VMS system, the following code
would be used:

\begin{small}
\begin{verbatim}
      INCLUDE 'DAT_PAR'
\end{verbatim}
\end{small}

whereas on a Unix system, the following code is required:

\begin{small}
\begin{verbatim}
      INCLUDE '/star/include/dat_par'
\end{verbatim}
\end{small}

(/star/include is a standard directory containing include files on Starlink
machines running Unix.)

If it is necessary to test for specific error conditions, the appropriate
include file and symbolic names should be used in your program.
Here is an example of how to use these symbols: 

\begin{quote}

\begin{small}
\begin{verbatim}
      INCLUDE 'SAE_PAR'
      INCLUDE 'DAT_PAR'
      INCLUDE 'DAT_ERR'
      ...
      CHARACTER*(DAT__SZLOC) LOC1, LOC2
      CHARACTER*(DAT__SZNAM) NAME
      INTEGER STATUS
      ...
*  Find a structure component.
      CALL DAT_FIND(LOC1, NAME, LOC2, STATUS)

*  Check the status value returned.
      IF (STATUS .EQ. SAI__OK) THEN
        <normal action>
      ELSE IF (STATUS .EQ. DAT__OBJNF) THEN
        <take appropriate action for object not found>
      ELSE
        <action on other errors>
      ENDIF
\end{verbatim}
\end{small}

\end{quote}

\subsection{Creating objects}
\label{S_creating}

To fix ideas, look at the example data structure in Fig~\ref{F_exndf}. 
This is actually one form of NDF structure, but for the purposes of this chapter
it will be treated as if it were simply an arbitrary HDS structure, i.e.\ we
will use HDS routines rather than NDF routines to process it.
The following notation is used to describe each object: 

\begin{small}
\begin{verbatim}
                      NAME[(dimensions)] <TYPE> [value]
\end{verbatim}
\end{small}


Note that scalar objects have no dimensions and that each level down the
hierarchy is indented. 

\begin{figure}[htb]

\begin{small}
\begin{verbatim}
       DATASET <NDF>
          DATA_ARRAY(512,1024)     <_UBYTE>     0,0,0,1,2,3,255,3,...
          LABEL                    <_CHAR*80>   'This is the data label'
          AXIS(2) <AXIS>
             AXIS <AXIS>
                DATA_ARRAY(512)    <_REAL>      0.5,1.5,2.5,...
                LABEL              <_CHAR*30>   'Axis 1'
             AXIS <AXIS>
                DATA_ARRAY(1024)   <_REAL>      5,10,15.1,20.3,...
                LABEL              <_CHAR*10>   'Axis 2'
\end{verbatim}
\end{small}

\caption{A simple NDF structure.}
\label{F_exndf}
\end{figure}

This example exhibits several of the most important properties of HDS data
objects.

\begin{itemize}
\item Both structures and primitives are present in the structure.
\item Scalar and non-scalar objects are present.
\item You can have arrays of structures, {\em e.g.}\, the AXIS component is a
 vector structure (with two elements).
\end{itemize}

The following code will create the structure in Fig~\ref{F_exndf}:

\begin{quote}

\begin{small}
\begin{verbatim}
      INCLUDE 'SAE_PAR'
      INCLUDE 'DAT_PAR'
      CHARACTER*(DAT__SZLOC) NLOC, ALOC, CELL
      INTEGER DIMS(2), STATUS
      DATA DIMS /512, 1024/

      CALL HDS_START(STATUS)

*  Create a container file with a top level scalar object of type NDF.
      CALL HDS_NEW('dataset', 'DATASET', 'NDF', 0, 0, NLOC, STATUS)

*  Create components in the top level object.
      CALL DAT_NEW(NLOC, 'DATA_ARRAY', '_UBYTE', 2, DIMS, STATUS)
      CALL DAT_NEWC(NLOC, 'LABEL', 80, 0, 0, STATUS)
      CALL DAT_NEW(NLOC, 'AXIS', 'AXIS', 1, 2, STATUS)

*  Create components in the AXIS structure...

*  Get a locator to the AXIS component.
      CALL DAT_FIND(NLOC, 'AXIS', ALOC, STATUS)

*  Get a locator to the array cell AXIS(1).
      CALL DAT_CELL(ALOC, 1, 1, CELL, STATUS)

*  Create internal components within AXIS(1) using the CELL locator.
      CALL DAT_NEW(CELL, 'DATA_ARRAY', '_REAL', 1, DIM(1), STATUS)
      CALL DAT_NEWC(CELL, 'LABEL', 30, 0, 0, STATUS)

*  Annul the cell locator
      CALL DAT_ANNUL(CELL, STATUS)

*  Do the same for AXIS(2).
      CALL DAT_CELL(ALOC, 1, 2, CELL, STATUS)
      CALL DAT_NEW(CELL, 'DATA_ARRAY', '_REAL', 1, DIM(2), STATUS)
      CALL DAT_NEWC(CELL, 'LABEL', 10, 0, 0, STATUS)
      CALL DAT_ANNUL(CELL, STATUS)

*  Access objects which have been created.
      ...

*  Tidy up
      CALL DAT_ANNUL(ALOC, STATUS)
      CALL HDS_CLOSE(NLOC, STATUS)
      CALL HDS_STOP(STATUS)

      END
\end{verbatim}
\end{small}

\end{quote}

The following points should be borne in mind:
\begin{itemize}

\item The structure created is in no way static --- new objects can be added or
existing ones deleted at any level without disturbing what already exists.
(Remember, however, that there are very strong rules about what can and cannot
be put into an NDF structure.
In general, the NDF routines of Section \ref{S_progndf} should be used to
manipulate NDFs.)

\item No primitive values have been stored yet --- that will be done next. 

\end{itemize}

Here are some notes on particular aspects of this example: 

\begin{quote}
\begin{description}

\item [DAT\_\_SZLOC] ---
This is one of the constants mentioned in Section \ref{S_usehds}, which is 
defined in the include file DAT\_PAR and specifies the length in characters of
all HDS locators.
Similar constants, DAT\_\_SZNAM and DAT\_\_SZTYP, specify the maximum lengths of
object names and types. 

\item [STATUS] ---
HDS routines conform to Starlink error handling conventions and use
{\em inherited status checking}.

\item [HDS\_NEW] ---
A container file called `dataset' is created (HDS provides the default file
extension of `.SDF').
A scalar structure called DATASET with a type of NDF is created within this
file, and a locator, NLOC, is associated with this structure.
It is usually convenient, although not essential, to make the top-level object
name match the container file name, as here.

\item [DAT\_NEW/DAT\_NEWC] ---
These routines create new objects within an object --- they are not
equivalent to HDS\_NEW because they don't have any reference to the
container file, only to a higher level structure.
Two variants are used simply because the character string length has to be
specified when creating a character object and it is normally most convenient
to provide this via an additional integer argument.
However, DAT\_NEW may be used to create new objects of any type, including
character objects.
In this case the character string length would be provided via the type
specification, {\em e.g.}\ `\_CHAR$*$15' (a character string length of one
is assumed if `\_CHAR' is specified alone). 

\item [DAT\_FIND] ---
After an object has been created, it is necessary to associate a locator with
it before values can be inserted; this routine performs this function. 

\item [DAT\_CELL] ---
There are several routines for accessing components of objects.
This one obtains a locator to a scalar object (structure or primitive) within
a non-scalar object like a vector. 

\item [HDS\_CLOSE] ---
This is used to close the container file and to annul the locator passed to
it.

\end{description}
\end{quote}

\subsection{Writing and reading objects}

Having created a structure, the next step will usually be to put some values
into it.
This can be done by using the {\bf DAT\_PUT} and {\bf DAT\_PUTC} routines.
For example, the main data array in the above example could be filled with
values as follows: 

\begin{quote}

\begin{small}
\begin{verbatim}
      BYTE IMVALS(512, 1024)
      CHARACTER*(DAT__SZLOC) LOC

*  Put data from array IMVALS into the object DATA_ARRAY.
      CALL DAT_FIND(NLOC, 'DATA_ARRAY', LOC, STATUS)
      CALL DAT_PUT(LOC, '_UBYTE', 2, DIMS, IMVALS, STATUS)
      CALL DAT_ANNUL(LOC, STATUS)

*  Put data from character constant to the object LABEL.
      CALL DAT_FIND(NLOC, 'LABEL', LOC, STATUS)
      CALL DAT_PUTC(LOC, 0, 0, 'This is the data label', STATUS)
      CALL DAT_ANNUL(LOC, STATUS)
\end{verbatim}
\end{small}

\end{quote}
Because this sort of activity occurs quite often, {\em packaged} access routines
have been provided (see Section~\ref{S_package}) for the programmer.

A complementary set of routines also exists for getting data from objects back
into program arrays or variables; these are the {\bf DAT\_GET} routines.
Again, packaged versions exist and are often handy in reducing the number of
subroutine calls required.

\subsection{Accessing objects by mapping}
\label{S_mapping}

Another technique for accessing the data values stored in primitive HDS objects
is termed `mapping'.\footnote{This terminology derives from the facility
originally provided by VAX/VMS for {\em mapping} the contents of files into the
computer's memory, so that they appear as if they are arrays of numbers
directly accessible to a program.
Although HDS still exploits this technique when appropriate, other techniques
are also used internally so that HDS no longer depends on the use of file
mapping (which some operating systems do not provide).
The terminology remains in use, however.}
An important advantage is that it removes a size restriction imposed by having
to declare fixed size program arrays to hold data.
This simplifies software, so that a single routine can handle objects of
arbitrary size without recourse to accessing subsets. 

HDS provides mapped access to primitive objects via the {\bf DAT\_MAP} routines.
Essentially, {\bf DAT\_MAP} will return a pointer to a region of the computer's
memory in which the object's values are stored.
This pointer can then be passed to another routine using the VAX Fortran `\%VAL'
facility.\footnote{This VAX extension to Fortran~77 is also supported by the
other implementations of Fortran for which HDS is available.}
An example will illustrate this: 

\begin{quote}

\begin{small}
\begin{verbatim}
      INTEGER PNTR, EL

*  Map the DATA_ARRAY component of the NDF structure as a vector of type
*  _REAL (even though the object is actually a 512 x 1024 array whose
*  elements are of type _UBYTE).
      CALL DAT_FIND(NLOC, 'DATA_ARRAY', LOC, STATUS)
      CALL DAT_MAPV(LOC, '_REAL', 'UPDATE', PNTR, EL, STATUS)

*  Pass the "array" to a subroutine.
      CALL SUB(EL, %VAL(PNTR), STATUS)

*  Unmap the object and annul the locator.
      CALL DAT_UNMAP(LOC, STATUS)
      CALL DAT_ANNUL(LOC, STATUS)

      END

*  Routine which takes the LOG of all values in a REAL array.
      SUBROUTINE SUB(N, A, STATUS)
      INCLUDE 'SAE_PAR'
      INTEGER N, STATUS
      REAL A(N)
      IF (STATUS .NE. SAI__OK) RETURN

      DO 1 I = 1, N
         A(I) = LOG(A(I))
 1    CONTINUE

      END
\end{verbatim}
\end{small}

\end{quote}
This example illustrates two features of HDS which we haven't yet mentioned:

\begin{quote}
\begin{description}

\item [Vectorisation] ---
It is possible to force HDS to regard objects as vectors, irrespective of
their true dimensionality.
This facility was useful in the above example as it made the subroutine SUB
much more general in that it could be applied to any numeric primitive object. 

\item [Automatic type conversion] ---
The program can specify the data type it wishes to work with and the program
will work even if the data are stored as a different type.
HDS will (if necessary) automatically convert the data to the type required by
the program.\footnote{This will work even if the object was originally created
on a different computer which formats its numbers differently.}
This useful feature can greatly simplify programming --- simple programs can
handle all data types.
Automatic conversion works on reading, writing and mapping. 

\end{description}
\end{quote}

Note that once a primitive has been mapped, the associated locator cannot be
used to access further data until the original object is unmapped.

\subsection{Mapping character data}
\label{S_charmapping}

Although the above example used a numeric type of `\_REAL' to access the data,
HDS allows any primitive type to be specified as an access type, including
`\_CHAR'.
It gives you a choice about how to determine the length of the character
strings it will map.
You may either specify the length you want explicitly, {\em e.g:}

\begin{small}
\begin{verbatim}
      CALL DAT_MAPV(LOC, '_CHAR*30', 'READ', PNTR, EL, STATUS)
\end{verbatim}
\end{small}


(in which case HDS would map an array of character strings with each element 
containing 30 characters) or you may leave HDS to decide on the length
required by omitting the length specification, thus:

\begin{small}
\begin{verbatim}
      CALL DAT_MAPV(LOC, '_CHAR', 'READ', PNTR, EL, STATUS)
\end{verbatim}
\end{small}


In the latter case, HDS will determine the number of characters actually
required to format the object's values without loss of information.
It uses decimal strings for numerical values and the values `TRUE' and `FALSE'
to represent logical values as character strings.
If the object is already of character type, then its actual length will be
used directly.
The routine {\bf DAT\_MAPC} also operates in this manner.

You should consult \xref{SUN/92}{sun92}{} for details of how to use this
facility on different machines.
It is one of the areas where it is very difficult to produce a mechanism which
works properly on all machines. 

\subsection{Copying and deleting objects}

HDS can also copy and delete objects. 
Routines {\bf DAT\_COPY} and {\bf DAT\_ERASE} will recursively copy and erase
all levels of the hierarchy below that specified in the subroutine call: 

\begin{quote}

\begin{small}
\begin{verbatim}
      CHARACTER*(DAT__SZLOC) OLOC

*  Copy the AXIS structure to component AXISCOPY of the structure located
*  by OLOC (which must have been previously defined).
      CALL DAT_FIND(NLOC, 'AXIS', ALOC, STATUS)
      CALL DAT_COPY(ALOC, OLOC, 'AXISCOPY', STATUS)
      CALL DAT_ANNUL(ALOC, STATUS)

*  Erase the original AXIS structure.
      CALL DAT_ERASE(NLOC, 'AXIS', STATUS)
\end{verbatim}
\end{small}

\end{quote}
Note that the locator to the AXIS object has been annulled before attempting to
delete it.
This whole operation can also be done using {\bf DAT\_MOVE}:
\begin{quote}

\begin{small}
\begin{verbatim}
      CALL DAT_MOVE(ALOC, OLOC, 'AXISCOPY', STATUS)
\end{verbatim}
\end{small}

\end{quote}

\subsection{Subsets of objects}

The routine {\bf DAT\_CELL} accesses a single element of an array.
An example was shown in Section \ref{S_creating}.
The routine {\bf DAT\_SLICE} accesses a subset of an arbitrarily dimensioned
object.
This subset can then be treated as if it were an object in its own right.
For example:

\begin{quote}

\begin{small}
\begin{verbatim}
      CHARACTER*(DAT__SZLOC) SLICE
      INTEGER LOWER(2), UPPER(2)
      DATA LOWER / 100, 100 /
      DATA UPPER / 200, 200 /

*  Get a locator to the subset DATA_ARRAY(100:200,100:200).
      CALL DAT_FIND(NLOC, 'DATA_ARRAY', LOC, STATUS)
      CALL DAT_SLICE(LOC, 2, LOWER, UPPER, SLICE, STATUS)

*  Map the subset as a vector.
      CALL DAT_MAPV(SLICE, '_REAL', 'UPDATE', PNTR, EL, STATUS)
\end{verbatim}
\end{small}

\end{quote}
In contrast to {\bf DAT\_SLICE}, routine {\bf DAT\_ALTER} makes a permanent
change to a non-scalar object.
The object can be made larger or smaller, but only in the last dimension.
This function is entirely dynamic, {\em i.e.}\, it can be done at any time,
provided the object is not mapped for access.
Note that {\bf DAT\_ALTER} works on both primitives and structures.
It is important to realise that the number of dimensions cannot be changed by
{\bf DAT\_ALTER}. 

\subsection{Temporary objects}

Temporary objects of any type and shape may be created by using the {\bf
DAT\_TEMP} routine.
This returns a locator to the newly created object, and this may then be
manipulated just as if it were an ordinary object (in fact a temporary
container file is created with a unique name to hold all such objects, and this
is deleted when {\bf HDS\_STOP} is executed at the end of the program).
This is often useful for providing workspace for algorithms which may have to
deal with large arrays. 

\subsection{Enquiries}

One of the most important properties of HDS is that its data files are
self-describing.
This means that each object carries with it information describing all its
attributes (not just its value), and these attributes can be obtained by means
of enquiry routines.
An example will illustrate:

\begin{quote}

\begin{small}
\begin{verbatim}
      PARAMETER (MAXCMP=10)
      CHARACTER*(DAT__SZNAM) NAME(MAXCMP)
      CHARACTER*(DAT__SZTYP) TYPE(MAXCMP)
      INTEGER NCOMP, I
      LOGICAL PRIM(MAXCMP)

*  Enquire the names and types of up to MAXCMP components...

*  First get the total number of components.
      CALL DAT_NCOMP(NLOC, NCOMP, STATUS)

*  Now index through the structure's components, obtaining locators and the
*  required information.
      DO 1 I = 1, MIN(NCOMP,MAXCMP)

*  Get a locator to the I'th component.
         CALL DAT_INDEX(NLOC, I, LOC, STATUS)

*  Obtain its name and type.
         CALL DAT_NAME(LOC, NAME(I), STATUS)
         CALL DAT_TYPE(LOC, TYPE(I), STATUS)

*  Is it primitive?
         CALL DAT_PRIM(LOC, PRIM(I), STATUS)
         CALL DAT_ANNUL(LOC, STATUS)
 1    CONTINUE
\end{verbatim}
\end{small}

\end{quote}
Here, {\bf DAT\_INDEX} is used to get locators to objects about which (in
principle) we know nothing.
This is just like listing the files in a directory, except that the order in
which the components are stored in an HDS structure is arbitrary (so they
won't necessarily be accessed in alphabetical order). 

\subsection{Packaged routines}
\label{S_package}

HDS includes families of routines which provide a more convenient method of
accessing objects than the basic routines.
For instance, members of the family {\bf DAT\_PUT$*$} write values of specific
type and dimensionality, and the {\bf DAT\_GET$*$} routines read similar values.
Thus {\bf DAT\_PUT0I} will write a single INTEGER value to a scalar primitive,
and {\bf DAT\_GET1R} will read the value of a vector primitive and store it in a
REAL program array.
There are no {\bf DAT\_GET2x} routines; all dimensionalities higher than one
are handled by {\bf DAT\_GETNx} and {\bf DAT\_PUTNx}. 

Another family of routines are the CMP routines.
These access components of the `current level'.
This usually involves: 

\begin{itemize}

\item FINDing the required object and getting a locator to it.

\item Performing the required operation, {\em e.g.}\ PUTting some value into it.

\item ANNULling the locator.

\end{itemize}

The CMP routines package this sort of operation, replacing three or so
subroutine calls with one.
The naming scheme is based on the associated DAT routines.
An example is shown below. 

\begin{quote}

\begin{small}
\begin{verbatim}
      CHARACTER*80 DLAB
      INTEGER DIMS(2)
      REAL IMVALS(512, 1024)
      DATA DIMS / 512, 1024 /

*  Get REAL values from the DATA_ARRAY component.
      CALL CMP_GETNR(NLOC, 'DATA_ARRAY', 2, DIMS, IMVALS, DIMS, STATUS)

*  Get a character string from the LABEL component and store it in DLAB.
      CALL CMP_GET0C(NLOC, 'LABEL', DLAB, STATUS)
\end{verbatim}
\end{small}

\end{quote}

\section{REF --- References to HDS objects}
\label{S_ref}
\markboth{Programmers}{\stardocname}

Reference objects are HDS objects which store references to other HDS data
objects.
They act as pointers to data, rather than storing data themselves.

The REF package allows reference objects to be created and written, and it
allows locators to referenced objects to be obtained.
The routines are listed in Section~\ref{R_REF}.

The referenced object may be defined as {\em internal\/}, in which case it is
assumed to be within the same container file as the reference object itself,
even if the reference object is copied to another container file.
In that case the reference must point to an object which has the same pathname
within the new file as it had in the old one.
References which are not {\em internal\/} will point to a named container file.

Reference objects may be copied and erased using DAT\_COPY and DAT\_ERASE\@.
Care must be taken when copying reference objects or referenced objects,
otherwise the reference may no longer point to the referenced object.

Referenced objects must exist at the time the reference is made or used.

\subsection{Uses}

Two main uses for this package are foreseen:
\begin{itemize}
\item To maintain a catalogue of data objects.
\item To avoid duplicating a large data object.
\end{itemize}
As an example of the second use, suppose that a large object is logically
required to form part of a number of other objects.
To avoid duplicating the common object, the others may contain a reference to
it.
For example:

\begin{quote}

\begin{small}
\begin{verbatim}
   Name                type                  Comments

DATA                DATA_SETS
  .SET1             SPECTRUM
     .AXIS1         _REAL(1024)          Actual axis data
     .DATA_ARRAY    _REAL(1024)
  .SET2             SPECTRUM
     .AXIS1         REFERENCE_OBJ        Reference to DATA.SET1.AXIS1
     .DATA_ARRAY    _REAL(1024)
  .SET3             SPECTRUM
     .AXIS1         REFERENCE_OBJ        Reference to DATA.SET1.AXIS1
     .DATA_ARRAY    _REAL(1024)  
\end{verbatim}
\end{small}

\end{quote}
Then a piece of code which handles structures of type SPECTRUM, which would
normally contain the axis data in .AXIS1 (as SET1 does), could be modified as
follows to handle an object .AXIS1 containing either the actual axis data or
a reference to the object which does contain the actual axis data.

\begin{quote}

\begin{small}
\begin{verbatim}
*  LOC1 is a locator associated with a SPECTRUM object; obtain locator to AXIS data
      CALL DAT_FIND(LOC1, 'AXIS1', LOC2, STATUS)

*  Modification to allow AXIS1 to be a reference object; check type of object
      CALL DAT_TYPE(LOC2, TYPE, STATUS)
      IF (TYPE .EQ. 'REFERENCE_OBJ') THEN
          CALL REF_GET(LOC2, 'READ', LOC3, STATUS)
          CALL DAT_ANNUL(LOC2, STATUS)
          CALL DAT_CLONE(LOC3, LOC2, STATUS)
          CALL DAT_ANNUL(LOC3, STATUS)
      ENDIF

*  End of modification.  LOC2 now locates the axis information wherever it is.
\end{verbatim}
\end{small}

\end{quote}
This code has been packaged into the subroutine {\bf REF\_FIND} which can be
used instead of DAT\_FIND in cases where the component requested may be a
reference object.

When a locator which has been obtained in this way is finished with, it should
be annulled using REF\_ANNUL rather than DAT\_ANNUL. 
This is so that, if the locator was obtained via a reference, the HDS\_OPEN for
the container file may be matched by an HDS\_CLOSE.
{\em Note that this should only be done when any other locators derived from
the locator to the referenced object are also finished with.}

\section{NDF --- Extensible n-dimensional data format}
\label{S_progndf}
\markboth{Programmers}{\stardocname}

The programmer's manual (\xref{SUN/33}{sun33}{}) for NDF runs to 199 pages.
It is impossible to produce a useful summary of it, and pointless to reproduce
the whole of it here.
Instead, we give you an overview of how NDFs are related to HDS, what is in an
NDF, and the sort of functions provided by the NDF routines.
A realistic, fully commented example program is also given, which should at
least give you an idea of how the routines are used.
The routines are listed in Section~\ref{R_NDF}.

\subsection{Relationship with HDS}

The NDF file format is based on HDS, and NDF data structures are stored in HDS
container files.
However, this does not necessarily mean that all applications which can read
HDS files can also handle data stored in NDF format. 

To understand why, you must appreciate that HDS provides only a rather low-level
set of facilities for storing and handling astronomical data.
These include the ability to store primitive data objects (such as arrays of
numbers, character strings, {\em etc.}) in a convenient and self-describing
way within {\em container files}.
However, the most important aspect of HDS is its ability to group these
primitive objects together to build larger, more complex structures. 
In this respect, HDS can be regarded as a construction kit which higher-level
software can use to build even more sophisticated data formats.

The NDF is a higher-level data format which has been built in this way out of
the more primitive facilities provided by HDS.
Thus, in HDS terms, an NDF is a data structure constructed according to a
particular set of conventions to facilitate the storage of typical
astronomical data (such as spectra, images, or similar objects of higher
dimensionality). 

While HDS can be used to access such structures, it does not contain any of
the interpretive knowledge needed to assign astronomical meanings to the
various components of an NDF, whose details can become quite complicated. 
In practice, therefore, it is cumbersome to process NDF data structures using
HDS directly. 
Instead, the NDF access routines are provided.
These `know' how NDF data structures are built, so they can hide the details
from writers of astronomical applications.
This results in a subroutine library which deals in higher-level concepts
more closely related to the work which typical astronomical applications
need to perform, and which emphasises the data concepts which an NDF is
designed to represent, rather than the details of its implementation. 

\subsection{Data format}

The simplest way of regarding an NDF is to view it as a collection of those
items which might typically be required in an astronomical image or spectrum. 
The main part is an $N$-dimensional array of {\em data\/} (where $N$ is 1 for a
spectrum, 2 for an image, {\em etc.}), but this may also be accompanied by a
number of other items which are conveniently categorised as follows: 

\begin{quote}

\begin{small}
\begin{center}
\begin{tabular}{rl@{ --- }l}
    {\em Character components:} & {\bf TITLE} & NDF title\\
                                & {\bf LABEL} & Data label\\
                                & {\bf UNITS} & Data units\\[1ex]
        {\em Array components:} & {\bf DATA}  & Data pixel values\\
                                & {\bf VARIANCE} & Pixel variance estimates\\
                                & {\bf QUALITY} & Pixel quality values\\[1ex]
{\em Miscellaneous components:} & {\bf AXIS}  & Coordinate axes\\
                                & {\bf HISTORY} & Processing 
                                                  history\footnotemark\\[1ex]
              {\em Extensions:} & {\bf EXTENSION} & Provides extensibility
\end{tabular}
\end{center}
\end{small}

\end{quote}
\footnotetext{The {\em history\/} component is not fully supported by the
present version of the NDF access routines.} 

The names of these components are significant, since they are used by the
NDF access routines to identify the component(s) to which certain operations
should be applied.\footnote{Note that the name `DATA' used by the NDF\_
routines to refer to an NDF's {\em data\/} component differs from the actual
name of the HDS object in which it is stored, which is `DATA\_ARRAY'.} 
The following describes the purpose and interpretation of each component in 
slightly more detail.

{\large \em Character components:}

\begin{quote}
\begin{description}

\item[{\bf TITLE}] --- This is a character string whose value is intended for
general use as a heading for such things as graphical output;
{\em e.g.}\ `M51 in good seeing'. 

\item[{\bf LABEL}] --- This is a character string whose value is intended to be
used on the axis of graphs to describe the quantity in the NDF's {\em data\/}
component; {\em e.g.}\ `Surface brightness'. 

\item[{\bf UNITS}] --- This is a character string whose value describes the
physical units of the quantity stored in the NDF's {\em data\/} component;
{\em e.g.}\ `J/(m$**$2$*$Ang$*$s)'. 

\end{description}
\end{quote}

{\large \em Array components:}

\begin{quote}
\begin{description}

\item[{\bf DATA}] --- This is an $N$-dimensional array of pixel values
representing the spectrum, image, {\em etc.}\ stored in the NDF. 
This is the only NDF component which must {\bf always} be present.
All the others are optional. 

\item[{\bf VARIANCE}] --- This is an array of the same shape and size as the
{\em data\/} array, and represents the measurement errors or uncertainties
associated with the individual {\em data\/} values. 
If present, these are always stored as {\em variance\/} estimates for each
pixel. 

\item[{\bf QUALITY}] --- This is an array of the same shape and size as the
{\em data\/} array which holds a set of unsigned byte values. 
These are used to assign additional `quality' attributes to each pixel (for
instance, whether it is part of the active area of a detector). 
Quality values may be used to influence the way in which the NDF's {\em data\/}
and {\em variance\/} components are processed, both by general-purpose software
and by specialised applications. 

\end{description}
\end{quote}

{\large \em Miscellaneous components:}

\begin{quote}
\begin{description}

\item[{\bf AXIS}] --- This represents a group of {\em axis\/} components which
may be used to describe the shape and position of the NDF's pixels in a
rectangular coordinate system. 
The physical units and a label for each axis of this coordinate system may 
also be stored.
(Note that the ability to associate {\em extensions\/} with an NDF's {\em
axis\/} coordinate system, although described in \xref{SGP/38}{sgp38}{}, is
not yet available via the NDF access routines.)

\item[{\bf HISTORY}] --- This may be used to keep a record of the processing
history.
If present, this component should be updated by any applications which
modify the data structure. 
Support for this component is not yet provided by the NDF access routines.

\end{description}
\end{quote}

{\large \em Extensions:}

\begin{quote}
\begin{description}

\item[{\bf EXTENSIONs}] --- These are user-defined HDS structures associated
with the NDF, and are used to give the data format flexibility by allowing it
to be extended. 
Their formulation is not covered by the NDF definition, but a few simple
routines are provided for accessing and manipulating named extensions, and
for reading and writing the values of components stored within them. 

\end{description}
\end{quote}

\subsection{Routines}

The NDF access routines are listed in Section~\ref{R_NDF}.
They perform the following types of operation on NDF data structures: 

\begin{small}
\begin{quote}
\begin{itemize}
\item Obtaining access, both input and output.
\item Creating and deleting.
\item Enquiring about attributes, including shape and size.
\item Enquiring about attributes of components. 
\item Reading, writing, and resetting component values.
\item Enquiring about (and flagging) the presence of {\em bad\/} pixels in
 components.
\item Accessing and handling {\em quality\/} information.
\item Modifying attributes (including shape and size) and those of components
 (such as numeric type). 
\item Reading, writing, and resetting the values of {\em axis\/} arrays and
 other {\em axis\/} components.
\item Modifying the attributes of {\em axis} components (such as the numeric
 type of {\em axis\/} arrays). 
\item Controlling the propagation of components to output data structures.
\item Creating, deleting, and enquiring about extensions, and obtaining
 access to components stored within extensions. 
\item Controlling the propagation of extensions to output data structures. 
\item Selecting and managing {\em sections\/} which refer to subsets or 
 super-sets.
\item Merging attributes to match the processing capabilities of specific
 applications.
\item Importing, finding, and copying objects held in container files.
\item Constructing messages.
\item Controlling access.
\end{itemize}
\end{quote}
\end{small}

Programming support for these routines, including on-line help, is also
provided by the Starlink language-sensitive editor STARLSE.

\subsection{Example program}
\label{S_addexample}

The following application adds two NDF data structures pixel-by-pixel.
It is a fairly sophisticated `add' application which will handle both the {\em
data\/} and {\em variance\/} components, as well as coping with NDFs of any
shape and data type.

\begin{quote}

\begin{small}
\begin{verbatim}
      SUBROUTINE ADD(STATUS)
*+
*  Name:
*     ADD

*  Purpose:
*     Add two NDF data structures.

*  Description:
*     This routine adds two NDF data structures pixel-by-pixel to produce a new NDF.

*  ADAM Parameters:
*     IN1 = NDF (Read)
*        First NDF to be added.
*     IN2 = NDF (Read)
*        Second NDF to be added.
*     OUT = NDF (Write)
*        Output NDF to contain the sum of the two input NDFs.
*     TITLE = LITERAL (Read)
*        Value for the title of the output NDF.  A null value will cause
*        the title of the NDF supplied for parameter IN1 to be used instead.
*-
      
*  Type Definitions:
      IMPLICIT NONE              ! No implicit typing

*  Global Constants:
      INCLUDE 'SAE_PAR'          ! Standard SAE constants
      INCLUDE 'NDF_PAR'          ! NDF_ public constants

*  Status:
      INTEGER STATUS             ! Global status

*  Local Variables:
      CHARACTER*(13) COMP        ! NDF component list
      CHARACTER*(NDF__SZFTP) DTYPE ! Type for output components
      CHARACTER*(NDF__SZTYP) ITYPE ! Numeric type for processing
      INTEGER EL                 ! Number of mapped elements
      INTEGER IERR               ! Position of first error (dummy)
      INTEGER NDF1               ! Identifier for 1st NDF (input)
      INTEGER NDF2               ! Identifier for 2nd NDF (input)
      INTEGER NDF3               ! Identifier for 3rd NDF (output)
      INTEGER NERR               ! Number of errors
      INTEGER PNTR1(2)           ! Pointers to 1st NDF mapped arrays
      INTEGER PNTR2(2)           ! Pointers to 2nd NDF mapped arrays
      INTEGER PNTR3(2)           ! Pointers to 3rd NDF mapped arrays
      LOGICAL BAD                ! Need to check for bad pixels?
      LOGICAL VAR1               ! Variance component in 1st input NDF?
      LOGICAL VAR2               ! Variance component in 2nd input NDF?
*.
*  Check inherited global status.
      IF (STATUS.NE.SAI__OK) RETURN

*  Begin an NDF context.
      CALL NDF_BEGIN

*  Obtain identifiers for the two input NDFs.  These involve calls to the 
*  parameter system and may be resolved from the interface file, the command
*  line, or by prompting the user.
      CALL NDF_ASSOC('IN1', 'READ', NDF1, STATUS)
      CALL NDF_ASSOC('IN2', 'READ', NDF2, STATUS)

*  Trim their pixel-index bounds to match.  This selects the largest common
*  set of pixels from the two input arrays.
      CALL NDF_MBND('TRIM', NDF1, NDF2, STATUS)

*  Create a new output NDF based on the first input NDF.  Propagate the axis
*  and quality components, which are not changed.  This program does not
*  support the units component.
      CALL NDF_PROP(NDF1, 'Axis,Quality', 'OUT', NDF3, STATUS)

*  See if a variance component is available in both input NDFs and generate
*  an appropriate list of input components to be processed.
      CALL NDF_STATE(NDF1, 'Variance', VAR1, STATUS)
      CALL NDF_STATE(NDF2, 'Variance', VAR2, STATUS)
      IF (VAR1.AND.VAR2) THEN
         COMP = 'Data,Variance'
      ELSE
         COMP = 'Data'
      END IF

*  Determine which numeric type to use to process the input arrays and set an
*  appropriate type for the corresponding output arrays.  This program supports
*  integer, real and double-precision arithmetic.  ITYPE says what type should
*  be used for the processing;  DTYPE is the type needed for the output data
*  (identified by NDF3) and so is passed on to NDF_STYPE
      CALL NDF_MTYPE('_INTEGER,_REAL,_DOUBLE',
     :                NDF1, NDF2, COMP, ITYPE, DTYPE, STATUS)
      CALL NDF_STYPE(DTYPE, NDF3, COMP, STATUS)

*  Map the input and output arrays.  Note that the identifier NDF3 produced by
*  NDF_PROP is used for the output data, which must be in WRITE access mode.
      CALL NDF_MAP(NDF1, COMP, ITYPE, 'READ', PNTR1, EL, STATUS)
      CALL NDF_MAP(NDF2, COMP, ITYPE, 'READ', PNTR2, EL, STATUS)
      CALL NDF_MAP(NDF3, COMP, ITYPE, 'WRITE', PNTR3, EL, STATUS)

*  Merge the bad pixel flag values for the input data arrays to see if checks
*  for bad pixels are needed.  The first argument `.TRUE.' says that this
*  application can handle bad pixels (if it were .FALSE. and bad pixels were
*  present the STATUS would be set to an error value).  The fifth argument
*  `.FALSE.' says not to check whether there actually are any bad pixels present.
      CALL NDF_MBAD(.TRUE., NDF1, NDF2, 'Data', .FALSE., BAD, STATUS)

*  Select the appropriate routine for the data type being processed and add the data
*  arrays.  Note that the arithmetic is done by one of the VEC_ routines in PRIMDAT
*  which handle bad pixels and any arithmetic errors, such as overflow, for you.
      IF (STATUS.EQ.SAI__OK) THEN
         IF (ITYPE.EQ.'_INTEGER') THEN
            CALL VEC_ADDI(BAD, EL, %VAL(PNTR1(1)),
     :                     %VAL(PNTR2(1)), %VAL(PNTR3(1)),
     :                     IERR, NERR, STATUS)

         ELSE IF (ITYPE.EQ.'_REAL') THEN
            CALL VEC_ADDR(BAD, EL, %VAL(PNTR1(1)),
     :                     %VAL(PNTR2(1)), %VAL(PNTR3(1)),
     :                     IERR, NERR, STATUS)

         ELSE IF (ITYPE.EQ.'_DOUBLE') THEN
            CALL VEC_ADDD(BAD, EL, %VAL(PNTR1(1)),
     :                     %VAL(PNTR2(1)), %VAL(PNTR3(1)),
     :                     IERR, NERR, STATUS)
         END IF

*  Flush any messages resulting from numerical errors.
         IF (STATUS.NE.SAI__OK) CALL ERR_FLUSH(STATUS)
      END IF

*  See if there may be bad pixels in the output data array and set the output
*  bad pixel flag value accordingly.  NERR is the number of errors detected by
*  the VEC_ADDx routine.
      BAD = BAD .OR. (NERR.NE.0)
      CALL NDF_SBAD(BAD, NDF3, 'Data', STATUS)

*  If variance arrays are also to be processed (i.e. added), then see if bad
*  pixels may be present in the variance arrays.
      IF (VAR1.AND.VAR2) THEN
         CALL NDF_MBAD(.TRUE., NDF1, NDF2, 'Variance', .FALSE., BAD,
     :                  STATUS)

*  Select the appropriate routine to add the variance arrays.
         IF (STATUS.EQ.SAI__OK) THEN
            IF (ITYPE.EQ.'_INTEGER') THEN
               CALL VEC_ADDI(BAD, EL, %VAL(PNTR1(2)),
     :                        %VAL(PNTR2(2)), %VAL(PNTR3(2)),
     :                        IERR, NERR, STATUS)

            ELSE IF (ITYPE.EQ.'_REAL') THEN
               CALL VEC_ADDR(BAD, EL, %VAL(PNTR1(2)),
     :                        %VAL(PNTR2(2)), %VAL(PNTR3(2)),
     :                        IERR, NERR, STATUS)

            ELSE IF (ITYPE.EQ.'_DOUBLE') THEN
               CALL VEC_ADDD(BAD, EL, %VAL(PNTR1(2)),
     :                        %VAL(PNTR2(2)), %VAL(PNTR3(2)),
     :                        IERR, NERR, STATUS)
            END IF

*  Flush any messages resulting from numerical errors.
            IF (STATUS.NE.SAI__OK) CALL ERR_FLUSH(STATUS)
         END IF

*  See if bad pixels may be present in the output variance array and set the
*  bad pixel flag accordingly.
         BAD = BAD .OR. (NERR.NE.0)
         CALL NDF_SBAD(BAD, NDF3, 'Variance', STATUS)
      END IF

*  Obtain a new title for the output NDF, by way of the parameter system.
      CALL NDF_CINP('TITLE', NDF3, 'Title', STATUS)
      
*  End the NDF context.
      CALL NDF_END(STATUS)

*  If an error occurred, then report context information.
      IF (STATUS.NE.SAI__OK) THEN
         CALL ERR_REP('ADD_ERR',
     :   'ADD: Error adding two NDF data structures.', STATUS)
      END IF

      END
\end{verbatim}
\end{small}

\end{quote}

The following is a possible interface file for the above application:

\begin{quote}
\begin{small}
\begin{verbatim}
   interface ADD
      parameter IN1                 # First input NDF
         position 1
         prompt   'First input NDF'
      endparameter
      parameter IN2                 # Second input NDF
         position 2
         prompt   'Second input NDF'
      endparameter
      parameter OUT                 # Output NDF
         position 3
         prompt   'Output NDF'
      endparameter
      parameter TITLE               # Title for output NDF
         type     'LITERAL'
         prompt   'Title for output NDF'
         vpath    'DEFAULT'
         default  !
      endparameter
   endinterface
\end{verbatim}
\end{small}
\end{quote}
