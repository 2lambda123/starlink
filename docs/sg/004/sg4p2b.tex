\chapter{ICL --- Interactive Command Language}
\label{C_ugicl}
\markboth{Users}{\stardocname}

\section {Introduction}
\label{S_iclint}
\markboth{Users}{\stardocname}

ICL is the most commonly used ADAM command language.
It is designed to provide a programmable user interface to an astronomical
data-reduction or data-acquisition system.
In some ways, ICL is similar to a high level programming language such as
Fortran or Pascal, but it has some important differences:
\begin{itemize}
\item It is an {\em interactive\/} language ---
 It provides a complete environment for entering, editing and debugging
 programs in much the same way that Basic does, rather than relying on external
 editors, compilers and so on. 
\item It is a {\em command\/} language ---
 One of its main uses is to enable you to type commands with few format
 restrictions.
 For example, it can be used to run the FIGARO data reduction system and it is
 possible to type FIGARO commands in almost exactly the same format as was
 previously used from DCL.
\item It is a {\em programming} language ---
 However it is intended only for writing relatively simple and straightforward
 programs.
 Its requirements differ from those of most modern programming languages which
 are designed for the needs of big software projects.
 ICL is designed to make simple programs easy to write.
\end{itemize}

The definitive reference manual and user's guide is \xref{SG/5}{sg5}{}.
The language is summarised in Chapter~\ref{C_rsicl}.
ICL also has an on-line Help system --- simply type HELP.

\section {Starting and stopping ICL}
\label{S_useicl}
\markboth{Users}{\stardocname}

ICL is started by typing:

\begin{small}
\begin{verbatim}
    $ ICL
\end{verbatim}
\end{small}

(Always remember that you must previously have executed SSC:LOGIN and
ADAMSTART.)

ICL is stopped by issuing the EXIT command:

\begin{small}
\begin{verbatim}
    ICL> EXIT
\end{verbatim}
\end{small}

If things are hopeless you can, of course, abandon ship by typing Ctrl/Y.

\section{Modes}
\label{S_modes}
\markboth{Users}{\stardocname}

ICL can be used in two different {\em modes}:
\begin{itemize}
\item Direct Mode
\item Procedure Mode
\end{itemize}
In {\em Direct Mode}, statements are executed immediately after they are typed
in.
In {\em Procedure Mode}, statements are first entered into a procedure, and
subsequently executed by running that procedure.
Another difference between the modes is that Control Statements (see below) can
only be executed in Procedure Mode.

\section{Statements}
\label{S_stmnts}
\markboth{Users}{\stardocname}

The {\em statements} of the language can be grouped into two sets:
\begin{itemize}
\item Direct Statements
\item Control Statements
\end{itemize}
{\em Direct statements} are executed immediately and can be entered in either
mode.
{\em Control statements} affect the order in which statements are executed
and can only be executed in Procedure mode.
You can also enter {\em Comments} which are lines whose first non-blank
character is `\{'.
Comments are ignored by the language interpreter.
If desired, they can be terminated by `\}', but this isn't necessary and is
a matter of taste.
Here is an example of a comment:

\begin{small}
\begin{verbatim}
    { This is a comment
\end{verbatim}
\end{small}

Although comments can be entered in Direct mode, their real purpose is to
document procedures written in Procedure mode.

When entering statements in ICL, all the normal command line editing facilities
available in DCL may be used.
The Up arrow or ctrl/B keys may be used to recall previous commands, and the
Down arrow key will step to the next command.
Any command back to the start of the ICL session may be recalled.

A command may take more than one line.
A tilde (\verb+~+) symbol at the end of a line is used to indicate that the
command continues on the next line.

\section{Direct statements}
\label{S_dirstm}
\markboth{Users}{\stardocname}

There are three types of direct statements:

\begin{description}

\item [Immediate Statement:] \mbox{}

 This is used to make ICL do simple calculations.
 It consists of an equals sign (\verb+=+) followed by an expression, and causes
 the value of the expression to be printed on the terminal.
 For example:

\begin{small}
\begin{verbatim}
    ICL> =1+2+3
             6
    ICL> =SQRT(2)
    1.414214
    ICL>
\end{verbatim}
\end{small}


\item [Assignment Statement:] \mbox{}

 This is exactly the same as the Fortran assignment statement and is used to
 assign a value to an ICL variable.
 For example:

\begin{small}
\begin{verbatim}
    ICL> PI=3.1415926    
    ICL> =2*pi           
    6.283185
    ICL>
\end{verbatim}
\end{small}

 Note that the case of letters doesn't matter.
 `PI' and `pi' are the same variable.
 In practice, names like `PI' are the names of HDS objects which store the
 constant values.

\item [Command:] \mbox{}

A command consists of a command name followed (optionally) by one or more
parameter specifications:
\begin{quote}
    command\_name [parameter\_specification] \ldots
\end{quote}
The parameter specifications may be separated by commas, or by one or more
spaces.
The forms that parameters can take are described in Chapter~\ref{C_prompts}.
Here are some examples of commands:

\begin{small}
\begin{verbatim}
    PRINT Single Root is (-C/B)
    QUADRATIC 7.2,18,4.6
    TESTR 5.8
\end{verbatim}
\end{small}

Commands are a very important part of ICL and several types exist.
You can't tell which type a command is just by looking at it; you have to
know how it is defined.
Commands are described in more detail in Chapter~\ref{C_commands}.
\end{description}

All three types of direct statement make use of {\em expressions}.
These are described next.

\section{Expressions}
\label{S_expr}
\markboth{Users}{\stardocname}

One of the basic syntactical units of the ICL language is the {\em Expression}.
Expressions are built up by operating on {\em values} using {\em operators}:
\begin{quote}
VALUE {\em operator} VALUE {\em operator} VALUE \ldots
\end{quote}
Here are some examples of expressions:

\begin{small}
\begin{verbatim}
    2
    X
    X+5
    Y*SIN(THETA)
\end{verbatim}
\end{small}

The value of an expression belongs to one of the following four
{\em types}:
\begin{itemize}
\item Real --- same as Fortran DOUBLE PRECISION type.
\item Integer --- same as Fortran INTEGER type.
\item Logical --- same as Fortran LOGICAL type.
\item String --- strings of characters, similar to Fortran CHARACTER*n type.
\end{itemize}
The two constituents of expressions --- {\em value} and {\em operator} --- will
now be considered in more detail:
\begin{description}
\item [Value] ---
A value can be represented in three ways:
\begin{description}
\item[Constant:] \mbox{}

{\em Real} and {\em Integer} constants are represented by numbers written in
the same formats that are accepted in Fortran.
{\em Integer} constants may also be entered in binary, octal, or hexadecimal
format by preceeding the value with \verb+%B+, \verb+%O+, or \verb+%X+
respectively.

{\em Logical} constants are written as \verb+TRUE+ or \verb+FALSE+.
Note that they are not delimited by decimal points as in Fortran.

{\em String} constants consist of any sequence of characters enclosed in either
single or double quotes.
Two consecutive quote symbols in a string are used to represent a single quote.
String constants are the one place in ICL where the case of letters is
significant.

Here are some examples of constants:

\begin{small}
\begin{verbatim}
    Real:          1.234E-5              3.14159 
    Integer:       123      %B100110     %O377      %Xffff
    Logical:       TRUE     FALSE
    String:        'This is a string'    "So is this"      ''
\end{verbatim}
\end{small}

The last example defines a string of zero length (valid in ICL, but not in
Fortran).

\item[Variable:] \mbox{}

Variables are represented by names composed of characters which may be letters,
digits, or the underscore character (\verb+_+).
The first character must be a letter.
The first 15 characters of a variable name are significant.

An important difference between ICL and Fortran is in the handling
of variable {\em types}.
In Fortran, each variable name has a unique type associated with it, and this
type is either derived implicitly from the first letter of the name, or is
explicitly specified in a declaration.
In ICL, variable names do not have types --- only values have types.
A variable gains a type when it is assigned a value.
This type can change when a new value is assigned to it.
Thus we can have the following series of assignments making the variable X
an integer, real, logical, and string in sequence:

\begin{small}
\begin{verbatim}
    ICL> X = 123      
    ICL> X = 123.456    
    ICL> X = TRUE         
    ICL> X = 'String'   
\end{verbatim}
\end{small}

This approach to variable types means that we do not have to declare the
variable names we use, which helps to keep programs simple\footnote{The
disadvantage is that ICL cannot usually spot cases where we accidentally
mistype a variable name, as can languages which enforce declaration of variables
(such as Fortran with the IMPLICIT NONE directive).
This is not thought to be a serious problem for the relatively simple programs
for which ICL is intended.}.

\item[Function:] \mbox{}

ICL provides a variety of standard functions.
They are written as in Fortran, and all the standard Fortran 77 generic
functions which are relevant to the ICL data types are provided with the same
names.
Thus SIN, COS, TAN, ASIN, ACOS, ATAN, ATAN2, LOG, LOG10, EXP, SQRT and ABS are
all valid functions.
A complete list of functions is given in Chapter~\ref{C_rsicl}.
\end{description}
\item[Operator] ---

The operators which are used to build up expressions are listed in the following
table in order of priority:

\begin{small}
\begin{verbatim}
    1 (highest)              **
    2                        *   /
    3                        +   -
    4                        =   >   <   >=  <=  <>  :
    5 (lowest)               NOT   AND   OR   &
\end{verbatim}
\end{small}

The order of evaluation of expressions is determined by the priority of the
operators.
The rules are the same as those in Fortran\footnote{But different from those
in Pascal.}, with arithmetic operators having the highest priority and logical
operators the lowest.
This means that a condition such as \( 0 < X+Y \leq 1 \) can be expressed in
ICL as follows:

\begin{small}
\begin{verbatim}
    X+Y > 0  AND  X+Y <= 1
\end{verbatim}
\end{small}

without requiring any parentheses.
However, in general it is good practice to use parentheses to clarify any
situation in which the order of evaluation is unclear.

The `\&' operator performs string concatenation.
If one of its operands is a numeric value, it will be converted to a string.
The `:' operator is used for formatting numbers into character strings as
described in Section~\ref{S_IO}
\end{description}
In evaluating expressions, ICL will freely apply type conversion to its operands
in order to make sense of them.
This means not only that integers will be converted to reals when required, but
also that strings will be converted to numbers when possible.
A string can be converted to a number if the value of the string is itself a
valid ICL expression; for example, the string \verb+'1.2345'+ has a numeric
value.
The string \verb-'X+1'- has a numeric value if X is currently a numeric
variable (or if X is a string which has a numeric value).
