\chapter{HDS/NDF --- The Data System}
\label{C_HDS}

\section{HDS --- Hierarchical data system}
\label{S_HDS}

HDS --- the Hierarchical Data System --- is one of the most powerful features of
ADAM.
It is implemented as a set of subroutines which are of much more interest to
the programmer than the user of the programs.
Nevertheless, as a user it is necessary for you to know something of the
system in order to make the best use of your data.
HDS is about storing astronomical data in a compact, flexible and efficient
way.
It recognises that observations are often complex --- possibly consisting of a
data array (in 1, 2, 3, or even more dimensions), together with variable
amounts of ancillary data --- calibrations, errors, telescope and instrument
information, observing conditions, and so on.
The way HDS handles this complexity bears some similarities to the way VMS
handles directories and files.

\begin{figure}[htb]
\begin{center}
\begin{picture}(150,85)
\thicklines
\put (20,70){\framebox(30,5){SYSTEM}}
\put (0,55){\framebox(30,5){ACCOUNT}}
\put (40,55){\framebox(30,5){ACCOUNT}}
\put (40,45){\framebox(30,5){Directory}}
\put (20,30){\framebox(30,5){File}}
\put (60,30){\framebox(30,5){File}}
\put (100,30){\framebox(30,5){Container File}}
\put (100,20){\framebox(30,5){Object}}
\put (80,5){\framebox(30,5){Object}}
\put (120,5){\framebox(30,5){Object}}
\put (15,65){\vector(0,-1){5}}
\put (55,65){\vector(0,-1){5}}
\put (55,55){\vector(0,-1){5}}
\put (35,40){\vector(0,-1){5}}
\put (75,40){\vector(0,-1){5}}
\put (115,30){\vector(0,-1){5}}
\put (95,15){\vector(0,-1){5}}
\put (135,15){\vector(0,-1){5}}
\put (90,32.5){\vector(1,0){10}}
\put (100,32.5){\vector(-1,0){10}}
\put (35,70){\line(0,-1){5}}
\put (55,45){\line(0,-1){5}}
\put (115,20){\line(0,-1){5}}
\put (15,65){\line(1,0){40}}
\put (35,40){\line(1,0){40}}
\put (95,15){\line(1,0){40}}
\put (15,55){\line(0,-1){1}}
\put (15,53){\line(0,-1){1}}
\put (15,51){\line(0,-1){1}}
\put (20,80){\bf VMS File System}
\put (110,55){\bf HDS}
\thinlines
\multiput(65,1)(2,0){46}{\line(1,0){1}}
\multiput(157,1)(0,2){24}{\line(0,1){1}}
\multiput(157,49)(-2,0){31}{\line(-1,0){1}}
\multiput(95,49)(0,-2){14}{\line(0,-1){1}}
\multiput(95,21)(-2,0){15}{\line(-1,0){1}}
\multiput(65,21)(0,-2){10}{\line(0,-1){1}}
\end{picture}
\caption{The relationship between VMS and HDS.}
\label{HDS_rel}
\end{center}
\end{figure}

\subsection{Data objects}
\label{S_datobj}

HDS files are known as {\em container files} and by default have the extension
`.SDF'.
They contain {\em data objects} which will be referred to simply as
{\em objects} when the context makes clear what sort of object it is.
An object is an entity which contains data or other objects.
This is the basis of the hierarchical nature of HDS and is analogous to the
VMS concepts of {\em file} and {\em directory} --- a directory can contain files
and directories which can themselves contain files and directories and so
on (Figure~\ref{HDS_rel}).
An object possesses the following attributes:
\begin{itemize}
\item Name
\item Type
\item Shape
\item State
\item Group
\item Value
\end{itemize}
HDS allows great freedom in specifying names and types, but standards have been
laid down (see Section~\ref{S_NDF}) to encourage portability of data and
applications.

\paragraph{Name:}\hfill

An object is identified by its {\em name}.
This must be unique within its own container object.
This is in contrast to VMS where different files in the same directory may be
distinguished by their version numbers.
A name is written as a character string containing any printing characters.
Spaces, tabs and so on are ignored and alphabetic characters are capitalised.
There are no special rules governing the first character (i.e.\ it can be
numeric).

When referring to components of objects, the following syntax is used:

\begin{small}
\begin{verbatim}
    A.B.C...
\end{verbatim}
\end{small}

where C is a component of B, and B is a component of A, which is the top-level
object in the container file.
Some specific examples of names are given at the end of this section.

\paragraph{Type:}\hfill

The {\em type} of an object falls into one of two {\em classes}:
\begin{itemize}
\item Primitive
\item Structure
\end{itemize}
Structure objects contain other objects called {\em components}.
Primitive objects contain only numeric, character, or logical values.
Objects in the different classes will be referred to simply as {\em structures}
and {\em primitives}, while the more general term {\em object} will refer to
either a {\em structure} or a {\em primitive}.
Structures are analogous to VMS directories --- they can contain a part of the
hierarchy below them.
Primitives are analogous to VMS files --- they are at the bottom of any branch
of the structure.

The primitive types defined in HDS are shown in Table~\ref{HDS_Types}.

\begin{table}[htb]
\begin{center}
\begin{tabular}{|l|l|l|}
\hline
{\em HDS Type} & {\em VAX Fortran Type} & {\em Length in Bits}\\
\hline
\_INTEGER & INTEGER & 32\\
\_REAL & REAL & 32\\
\_DOUBLE & DOUBLE PRECISION & 64\\
\_LOGICAL & LOGICAL & 32\\
\_CHAR[\raisebox{-0.6ex}{\rm*}N] & CHARACTER\raisebox{-0.6ex}{\rm*}N & 8\raisebox{-0.6ex}{\rm*}N\\
\hline
\_UBYTE & BYTE & 8\\
\_BYTE & BYTE & 8\\
\_UWORD & INTEGER\raisebox{-0.6ex}{\rm*}2 & 16\\
\_WORD & INTEGER\raisebox{-0.6ex}{\rm*}2 & 16\\
\hline
\end{tabular}
\caption{The HDS primitive data types.}
\label{HDS_Types}
\end{center}
\end{table}

The first five of these types are referred to as {\em standard data types}.
The \_UBYTE type provides a value range of 0 to 255; the \_UWORD type provides a
value range of 0 to 65535.
The others are as for Fortran 77.
Examples of structure types are IMAGE, SPECTRUM, INSTR\_RESP {\em etc.}
Their names don't begin with an underscore, so the system and the programmer can
easily distinguish between primitives and structures.
A {\em type} is written as a character string with the same rules as for
{\em name}, except that an asterisk can only appear if the first character is an
underscore (i.e.\ it is a primitive), and also a type can be blank.

\paragraph{Shape:}\hfill

Every object has a {\em shape} or dimensionality.
This is described by an integer (the number of dimensions) and an integer array
(the size of each dimension).
A {\em scalar}, for example a single number, has by convention a dimensionality
of zero, i.e.\ number of dimensions is 0.
A {\em vector} has a dimensionality of 1, i.e.\ number of dimensions is 1 and
the first element of the dimension array contains the size of the vector.
An {\em array} refers to an object with 2 or more dimensions; currently a
maximum of 7 dimensions are allowed.
Objects may be referred to as {\em scalar primitives} or
{\em vector structures} and so on.

\paragraph{State:}\hfill

The {\em state} of an object specifies whether or not its value is defined.
In routines it is represented as a LOGICAL variable where .TRUE. means defined
and .FALSE. means undefined.

\paragraph{Group:}\hfill

In order to access an object, it is first necessary to obtain a {\em locator\/},
a sort of pointer which can then be used to address the object.
When the program no longer needs to access the object, the locator should be
{\em annulled}.
A locator is analogous to a Fortran logical unit number (but is actually a
character variable, not an integer).
Any number of locators can be active simultaneously.
The {\em group} attribute is used to form an association between locators so
that they can be annulled together.
A group is written as a character string whose rules of formation are the same
as for {\em name}.

\paragraph{Value:}\hfill

When an object is first created it contains no value, somewhat like
an empty file.
It must be given a value in a separate operation.
A value can be a scalar, vector, or an array.
The scalar or the elements of the vector or array must all be of the same
type and can be primitives or structures.
The rules for handling character values are the same as for Fortran 77, i.e.\
character values are padded with blanks or truncated from the right depending on
the relative length of the program value and the object.

\paragraph{Illustration:}\hfill

To fix ideas, look at the example of an NDF data structure in Figure 8.2.
The following notation is used to describe each object:

\begin{small}
\begin{verbatim}
                      NAME(dimensions)  TYPE  VALUE
\end{verbatim}
\end{small}

where `\verb+(dimensions)+' only appears when describing vectors or arrays.
Each level down the hierarchy is indented.

Suppose an object with this structure were stored in a (container) file called
EXAMPLE.SDF, then we can refer to components of this object by names such as:

\begin{small}
\begin{verbatim}
      EXAMPLE.DATA_ARRAY            an array of type _REAL
      EXAMPLE.QUALITY.BADBITS       an unsigned scalar of type _BYTE
      EXAMPLE.MORE.FIGARO.TIME      a scalar of type _REAL
\end{verbatim}
\end{small}

and so on.

\section {NDF --- Extensible n-dimensional data format}
\label{S_NDF}

A major preoccupation of Starlink since its inception has been to design a data
storage format which is both standard and yet which can accommodate most of
the things which one might wish to store.
(This is a weak point with most software environments in astronomical use at
present.)
One of the practical problems with unfettered HDS is that it is {\em too\/}
flexible.
The solution adopted, NDF (Extensible $N$-dimensional-Data Format), provides
a more limited set of designs, but still implemented using HDS.
It is described in awesome detail in \xref{SGP/38}{sgp38}{}.

In essence, NDF defines a set of standard data objects.
Not all of them must be present in an NDF object, but no others will be
processed.
Non-standard items are handled in a standard way by using self-contained
{\em extensions}.
There are defined locations for items such as the main data array, axes, title,
units {\em etc.}
The only mandatory item is the main data array; all other items are optional!

All this means that the {\em user} can be certain that no properly written
application will mess up his data, and there is a very good chance
that all the useful information will be properly used.
(For the {\em programmer}, the huge advantage of this system is that he
doesn't need to know the details of the format at all!
A comprehensive set of routines is available to access the standard components
of an NDF.
These are described in Section~\ref{R_NDF} and (more fully) in
\xref{SUN/33}{sun33}{}.)

\subsection{The structure of an NDF object}
\label{S_ndfstruc}

ADAM\_EXAMPLES:EXAMPLE.SDF is file containing an NDF object which contains all
the standard NDF components and also has a Figaro extension.
Such a file is often referred to as an `NDF file', or even as just an `NDF'.
The structure of the file, as revealed by:

\begin{small}
\begin{verbatim}
   ICL > TRACE ADAM_EXAMPLES:EXAMPLE
\end{verbatim}
\end{small}

is shown in Figure 8.2.

\begin{quote}

\begin{small}
\begin{verbatim}

EXAMPLE  <NDF>

   DATA_ARRAY(856)  <_REAL>       *,0.2284551,-2.040089,
                                  ... 820.8976,570.0729,*,449.574
   TITLE          <_CHAR*30>      'HR6259 - AAT fibre data'
   LABEL          <_CHAR*20>      'Flux'
   UNITS          <_CHAR*20>      'Counts/s'
   QUALITY        <QUALITY>       {structure}
      BADBITS        <_UBYTE>        1
      QUALITY(856)   <_UBYTE>        1,0,0,0,0,1,0,0,0,0,0,0,0,0,0,0,0,0,
                                     ... 0,0,0,0,0,0,0,0,0,0,0,0,0,0,0,0,0,1,0

   VARIANCE(856)  <_REAL>         2.1,0.1713413,1.5301,34.38378,42.35531,
                                  ... 615.6732,427.5547,353.9127,337.1805
   AXIS(1)        <AXIS>          {structure}

   Contents of AXIS(1)
      DATA_ARRAY(856)  <_REAL>       3847.142,3847.672,3848.201,3848.731,
                                     ... 4298.309,4298.838,4299.368,4299.897
      LABEL          <_CHAR*20>      'Wavelength'
      UNITS          <_CHAR*20>      'Angstroms'

   HISTORY        <HISTORY>       {structure}
      CREATED        <_CHAR*30>      '1990-DEC-12 08:21:02.324'
      CURRENT_RECORD  <_INTEGER>     3
      RECORDS(10)    <HIST_REC>      {array of structures}

      Contents of RECORDS(1)
         TEXT           <_CHAR*40>      'Extracted spectrum from fibre data.'
         DATE           <_CHAR*25>      '1990-DEC-19 08:43:03.08'
         COMMAND        <_CHAR*30>      'FIGARO V2.4 FINDSP command'


   MORE           <EXT>           {structure}
      FIGARO         <EXT>           {structure}
         TIME           <_REAL>         1275
         SECZ           <_REAL>         2.13
\end{verbatim}
\end{small}

\end{quote}
\begin{center}
Figure 8.2: An example of the internal structure of an NDF file.
\end{center}
Of course, this is only an example format.
There are various ways of representing some of the components.
These {\sl variants\/} are described in \xref{SGP/38}{sgp38}{}.

The components of an NDF are described below.
The names (in bold type) are significant as they are used by the NDF
access routines to identify the components.

\begin{quote}
\begin{description}

\item[{\bf DATA\_ARRAY}] --- the main data array is the only component which
 {\em must} be present in an NDF.
 In the case of EXAMPLE.SDF, this component is a 1-d real array with 856
 elements.
\item[{\bf TITLE}] --- the  character string {\tt'HR6259 - AAT fibre data'}
 describes the contents of the NDF.
 The TITLE might be used as the title of a graph {\it etc.}
\item[{\bf LABEL}] --- the character string {\tt'Flux'} describes the
 quantity represented in the NDF's main data array.
 The LABEL is intended for use on the axes of graphs {\it etc.}
\item[{\bf UNITS}] --- this character string describes the physical units of
 the quantity stored in the main data array, in this case, {\tt'Counts/s'}.
\item[{\bf QUALITY}] --- this component is used to indicate the quality of
 each element in the main data array.
 The quality structure contains a quality array and a BADBITS value, both of
 which {\sl must\/} be of type \_UBYTE.
 The quality array has the same shape and size as the main data array, and is
 used in conjunction with the BADBITS value to decide the quality of a pixel
 in the main data array.
 In the example the BADBITS component has value 1.
 QUALITY normally works by taking the {\em bit-wise AND} of BADBITS with
 each element of the QUALITY array.
 Thus, an odd value in the QUALITY array indicates a bad value, while an even
 value identifies a good pixel.
\item[{\bf VARIANCE}] --- the variance array is the same shape and size as the
 main data array and contains the errors associated with the individual data
 values.
 These are stored as {\sl variance\/} estimates for each pixel.
\item[{\bf AXIS}] --- this structure may contain axis information for any
 dimension of the NDF's main array.
 In this case, the main data array is only 1-d, therefore only the AXIS(1)
 structure is present.
 This structure contains the actual axis data array, and also label and units
 information.
\item[{\bf HISTORY}] --- this component provides a record of the processing
 history of the NDF.
 Only the first of three records is shown in the example.
 This indicates that the spectrum was extracted from fibre data using the
 Figaro FINDSP command on 19th December 1990.
 (Support for the history component is not yet provided by the NDF access
 routines.)
\item[{EXTENSIONs}] --- the purpose of extensions is to store non-standard
 items.
 \verb+EXAMPLE.SDF+ began life as an old-style (DST) Figaro file\footnote{The
 file was converted to an NDF using the CONVERT command DST2NDF.} which
 contained values for the airmass and exposure time associated with the
 observations.
 These are stored in the Figaro extension, and the intention is that the
 Figaro applications which use these values will know where to find them.
\end{description}
\end{quote}
