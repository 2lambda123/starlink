\chapter{The Parameter System}
\label{C_parsys}
\markboth{Programmers}{\stardocname}

The Parameter system is a central feature of the ADAM system architecture.
It is a unifying concept which binds the various ADAM `facilities' together
and permeates many aspects of ADAM.
Both users and programmers need to understand it clearly if they are to use
ADAM properly.

A list of the routines which implement the system is given in
Section~\ref{R_Parameter}.

\section{Parameters and objects}
\markboth{Programmers}{\stardocname}

Programs are usually written to be as general as possible in order to maximize
their usefulness.
Thus, you would not write a program to display a specific image but would make
it accept {\em any} image stored in one or more specified formats.
Because of this, at run time the program needs to know the name of the image
to be displayed.
The `name of the image' is a {\em parameter} of the program.
More complicated programs may have many parameters controlling their actions.
Thus, your display program may need to know the format of the input image,
the device on which it is to be displayed, the colour mappings to be used in
the display, types of axes, scaling methods, and so on.
All these things are parameters.
When an ADAM program is run, the task of the ADAM parameter system is to
associate the program's parameters with objects in the outside world.

The word `object' is rather overworked in ADAM, so you have to be aware of the
context in which it is used.
`Object' is often used to mean a `data object', {\em i.e.}\, an HDS structure
accessed through the data system.
When talking about parameters, the word `object' has a wider meaning;
in fact it can mean absolutely anything outside itself that a program wishes
to refer to by name --- a data object, a hardware device, a message, a graphics
database, a file {\em etc.}

\begin{figure}[htb]
\begin{center}
\begin{picture}(120,110)
\thicklines
\put (50,40){\framebox(50,10){\small User Interface}}
\put (50,50){\framebox(50,20){ \small \bf PARAMETER SYSTEM}}
\put (75,30){\line(0,1){10}}
\put (60,75){\line(0,1){5}}
\put (90,75){\line(0,1){5}}
\put (75,70){\line(0,1){5}}
\put (60,75){\line(1,0){30}}
\put (75,25){\oval(20,10)}
\put (60,85){\oval(20,10)}
\put (90,85){\oval(20,10)}
\put (53.5,84.25){\small Program}
\put (83.5,86){\small Interface}
\put (86.5,82){\small File}
\put (67,25.8){\small Command}
\put (67.5,22.5){\small Language}
\put (75,107){\circle{4}}
\put (75,98){\line(0,1){7}}
\put (75,102){\line(1,1){5}}
\put (75,102){\line(-1,1){5}}
\put (75,98){\line(1,-1){5}}
\put (75,98){\line(-1,-1){5}}
\put (20,99){\bf Parameters}
\put (90,99){\em Programmer}
\put (75,15){\circle{4}}
\put (75,6){\line(0,1){7}}
\put (75,10){\line(1,1){5}}
\put (75,10){\line(-1,1){5}}
\put (75,6){\line(1,-1){5}}
\put (75,6){\line(-1,-1){5}}
\put (23,10){\bf Objects}
\put (90,10){\em User}
\put (30,20){\vector (0,1){70}}
\put (30,90){\vector (0,-1){70}}
\put (10,55){Association}
\end{picture}
\caption{The Parameter System --- Communication between User and Programmer.}
\label{F_parsys}
\end{center}
\end{figure}

Figure~\ref{F_parsys} is a schematic illustration of the function of the ADAM
Parameter System.
Its main role is to provide a means of communication between the user and the
programmer, and in particular to associate objects chosen by the user with
a program's parameters chosen by the programmer.
The programmer communicates by using ADAM routines in his program and parameter
specifications in his interface file.
The user communicates by using a command language such as ICL.

You must be careful to distinguish between a `parameter name' (used to identify
which parameter is being talked about), and a `parameter value' (the value a
given parameter has at a particular moment).
The word `parameter' is often used loosely in both senses.
However, the precise meaning is normally obvious from the context.

\section{Associating objects with parameters}
\markboth{Programmers}{\stardocname}

The programmer has two ways to associate an object with a program parameter:
\begin{itemize}
\item Directly, using `PAR' routines.
\item Indirectly, using `ASSOC' routines.
\end{itemize}

\paragraph{Direct association:}\hfill

The direct method uses the set of PAR\_GET and PAR\_PUT routines introduced in
Chapter~\ref{C_simpap}.
It is called `direct' because a call to a single routine causes the parameter
value to be read into a program variable, or written from a program variable.
Thus, in the program TESTR illustrated in Chapter~\ref{C_simpap} the call:

\begin{small}
\begin{verbatim}
    CALL PAR_GET0R('X', XVALUE, STATUS)
\end{verbatim}
\end{small}

causes the parameter system to obtain a real scalar value for parameter `X' and
store it in program variable `XVALUE'.
Similarly, a call such as:

\begin{small}
\begin{verbatim}
    CALL PAR_PUT0R('Y', YVALUE, STATUS)
\end{verbatim}
\end{small}

causes the parameter system to write the value stored in program variable
`YVALUE' into the data object associated with parameter `Y'.

The direct method only works for parameters whose values are scalars, vectors,
or n-dimensional arrays of one of the five standard HDS primitive data types:
\begin{quote}
\begin{itemize}
\item \_INTEGER
\item \_REAL
\item \_DOUBLE
\item \_LOGICAL
\item \_CHAR
\end{itemize}
\end{quote}
Values of other types ({\em e.g.}\,\ graphics plots, non-standard data types)
must be associated with parameters by the indirect method.

\paragraph{Indirect association:}\hfill

The various ADAM subroutine libraries enable you to access and manipulate
objects of various kinds.
The indirect method of accessing parameters first `associates' an object of an
appropriate type with a parameter by calling a special `ASSOC' routine
provided for that library.
This routine provides the program with an `identifier'.
From then on, this identifier is used to specify the object, rather than the
parameter name.

The usual form\footnote{The ASSOC routines for the FIO and RIO libraries are
slightly unusual in that they need two extra parameters: the format, FM, and
the record size, RS. These are placed after MODE.} of the ASSOC routine is:

\begin{small}
\begin{verbatim}
    CALL fac_ASSOC(PAR, MODE, ID, STATUS)
\end{verbatim}
\end{small}

where:
\begin{quote}
\begin{description}
\item [fac] is the facility name of the library.
\item [PAR] is the parameter name.
\item [MODE] is the access mode permitted to the object.
\item [ID] is the identifier used to identify the object in other routines in
 the library.
\item [STATUS] is the status return.
\end{description}
\end{quote}
Table~\ref{T_assoc} shows the names of the ASSOC routines for the ADAM
subroutine libraries, together with the name of the object identifier `ID' used
in the documentation, and the type of object addressed.
\begin{table}[htb]
\centering
\begin{tabular}{|l|l|l|l|}
\hline
{\em Subroutine library} & {\em ASSOC routine} & {\em Object identifier} &
 {\em Object type} \\
\hline
NDF,DAT,CMP,HDS,REF,TRN & DAT\_ASSOC    & LOC    & Data object \\
GKS                     & GKS\_ASSOC    & WKID   & GKS workstation \\
SGS                     & SGS\_ASSOC    & ZONE   & SGS zone \\
IDI                     & IDI\_ASSOC    & DISPID & Display device \\
AGI                     & AGI\_ASSOC    & PICID  & Picture in database \\
FIO,RIO                 & FIO\_ASSOC    & FD     & File \\
MAG                     & MAG\_ASSOC    & TD     & Tape drive \\
ADC                     & ADC\_ASSOC    & CD     & Catalogue \\
NCAR,SNX,PGPLOT,GNS,NAG & not specified &        & \\
PAR,MSG,ERR,EMS         & not needed    &        & \\
\hline
\end{tabular}
\caption{ASSOC routines for ADAM subroutine libraries.}
\label{T_assoc}
\end{table}

Once an object has been associated with a parameter, its identifier is
used to access the object.
Suppose, for example, that parameter `IMAGE' is a data object which contains
a pixel array stored in a component named `DATA\_ARRAY'.
This array can be accessed by the code:

\begin{small}
\begin{verbatim}
    CALL DAT_ASSOC('IMAGE', 'READ', LOC, STATUS)
    CALL DAT_FIND(LOC, 'DATA_ARRAY', LOCA, STATUS)
    CALL DAT_GETNR(LOCA, NDIM, DIMX, VALUE, DIM, STATUS)
\end{verbatim}
\end{small}

in which the first call associates an HDS object with the parameter `IMAGE'
and makes the locator `LOC' point to it.
The second call finds the component `DATA\_ARRAY' in this image and makes
another locator `LOCA' point to it.
The third call stores the value of the array in variable `VALUE'.

Most facilities also have ANNUL routines which are used to stop the identifier
pointing to the object, thus:

\begin{small}
\begin{verbatim}
    CALL DAT_ANNUL(LOC, STATUS)
\end{verbatim}
\end{small}

will cause LOC to stop pointing to the object associated with parameter `IMAGE'.
The object will still be associated with parameter `IMAGE', but you won't be
able to access it.
It is considered good practice to annul identifiers when you have finished
using their objects.
However, the system tidies up after you when your program finishes so you
needn't worry about it.

The ASSOC and CANCL routines (described later) are often referred to as
`Parameter Routines' because they are added to a `stand-alone' subroutine
library in order to integrate it with the ADAM parameter system.
The equivalent `stand-alone' routines are OPEN and CLOSE.
In ADAM programs the OPEN routine is usually replaced by the ASSOC routine.
For example, when the SGS graphics library is used outside ADAM, the first
call to one of its routines would be:

\begin{small}
\begin{verbatim}
    CALL SGS_OPEN(WKSTN, ZONE, STATUS)
\end{verbatim}
\end{small}

which would cause the graphics workstation specified by WKSTN to be opened and
an SGS zone specified by ZONE to be made available for plots.
However, an ADAM program which used SGS would not use this call, but would
replace it with:

\begin{small}
\begin{verbatim}
    CALL SGS_ASSOC(PAR, MODE, ZONE, STATUS)
\end{verbatim}
\end{small}

Here, the workstation is specified through the parameter system, so PAR
replaces WKSTN in the calling sequence.
ZONE and STATUS have the same function as before, but a new variable, MODE, is
introduced to provide the program with the ability either to clear or to retain
the display screen contents.

\paragraph{Parameter values:}\hfill

Your program relies on the parameter system to provide a value for a parameter.
This process depends on a `search path' defined in the interface file, as
will be explained later.
However, your program can influence the value provided by suggesting a `dynamic
default' to the parameter system.
This is done by a set of `PAR\_DEF' commands.
For example:

\begin{small}
\begin{verbatim}
    CALL PAR_DEF0R('X', 3.4, STATUS)
\end{verbatim}
\end{small}

will set a dynamic default of 3.4 for parameter `X'.

The set of PAR\_DEF routines is analogous to the PAR\_GET and PAR\_PUT sets,
except that there is no `V' form, {\em i.e.}\, you cannot set a dynamic
default for an array mapped as a vector; you have to use the explicit array
form.

\section{Interface files}

An interface file for a single program has the following form:

\begin{small}
\begin{verbatim}
      interface PROGNAME
         Parameter specification
         Parameter specification
         ...
         Message specification
         Message specification
         ...
      endinterface
\end{verbatim}
\end{small}

The body of the file comes between the `interface' and `endinterface'
statements, and consists of a sequence of `Parameter specifications' and
`Message specifications'.
Each interface file has a unique name (\verb+PROGNAME+) which should be the same
as the name of the program with which it is associated.
Comments may be introduced by the `\#' character --- any characters following
`\#' on a line will be ignored.

\paragraph{Minimal interface files:}\hfill

The example interface file shown in Section~\ref{S_Inter} was quite large
considering the simplicity of its associated program.
Is all this complexity really essential?
The answer is `No' --- the example given was made complex in order to show off
the facilities of ADAM, and not because of any intrinsic complexity in
using ADAM.
Suppose you write the following abbreviated ADAM program to read in a real
number and write it out on your terminal:

\begin{small}
\begin{verbatim}
            SUBROUTINE READ(STATUS)
            IMPLICIT NONE
            INCLUDE 'SAE_PAR'
            INTEGER STATUS
            REAL XVALUE
      *.........................................................................
            CALL PAR_GET0R('X', XVALUE, STATUS)
            CALL MSG_SETR('X', XVALUE)
            CALL MSG_OUT('MESS', 'Value read for X is ^X', STATUS)
            END
\end{verbatim}
\end{small}

Now, suppose you don't prepare an interface file for this program at all and
try to run it using ICL in the usual way.
This is what would happen:

\begin{small}
\begin{verbatim}
    ICL> define read read
    ICL> read
    Loading READ into xxxxREAD
    Failed to read interface module
    ...
\end{verbatim}
\end{small}

When I tried it, the system hung and I had to use cntrl/Y to escape.
So, you can't get away with not storing an interface file because the system
just won't work without one.
However, suppose you store one with no information on the program parameter X,
{\em i.e.}\, an interface file containing the lines:

\begin{small}
\begin{verbatim}
    interface READ
    endinterface
\end{verbatim}
\end{small}

If you try to run the program again it still won't work, but you will get
a different error message:

\begin{small}
\begin{verbatim}
    ADAMERR   %PARSECON, requested parameter name not defined
\end{verbatim}
\end{small}

However, at least the system won't hang.
Now let's go one step further and put an empty specification for parameter X in
the interface file:

\begin{small}
\begin{verbatim}
    interface READ
      parameter X
      endparameter
    endinterface
\end{verbatim}
\end{small}

This time the program will work, even though we haven't stored a specification
for the message parameter `MESS':

\begin{small}
\begin{verbatim}
    ICL> read
    X   > 1.2
    Value read for X is 1.2
    ICL>
\end{verbatim}
\end{small}

So, the shortest interface file you can get away with is one which contains
empty specifications for all the program parameters except message parameters
(such as MESS) which do not need to be specified at all.
Thus, the simplest way of testing a program you have compiled and linked is to
prepare an interface file containing empty specifications for its non-message
parameters, then use ICL to define a command to run it, and then type in that
command.
You will be prompted for a value for each of the program's parameters.
If you try to specify values on the command line by position, they will be
ignored and you will be prompted anyway.
However, you can specify them on the command line by keyword, using the
parameter name as the keyword.

You can see now that the function of the interface file is to provide you with
the ability to modify or enhance the user interface if required.
The facilities available from the interface file are to:
\begin{itemize}
\item Control the data type of a value associated with a parameter.
\item Control the access to a parameter value.
\item Control the format of the command line.
\item Control the appearance of prompt and help lines.
\item Control the search for a parameter value.
\item Control the contents of messages.
\end{itemize}
These facilities are considered in more detail in the next sections where the
fieldnames which are used to specify these controls are described.

\paragraph{Search path:}\hfill

An interface file can be accessed through a search path.
If a {\em job} logical name search path, ADAM\_IFL, is defined, it will be used
to find the interface file.
If ADAM\_IFL is not defined, or if an interface file is not found using it, an
attempt will be made to find one in the directory in which the program
executable image was found.

\paragraph{Compilation:}\hfill

An interface file can exist in two forms.
The first form is an ordinary text file which can be edited easily.
This is used when a program is being developed.
The file should have the same name as the program, but with a file extension of
`.IFL'.
When the user interface for a program has become relatively stable, the
interface file should be compiled and stored as an `.IFC' file --- this is the
second form.
This is done as follows:

\begin{small}
\begin{verbatim}
    $ COMPIFL progname
    COMPIFL: successful completion
    $
\end{verbatim}
\end{small}

where `\verb+progname+' is the name of the interface file.
(You must have previously entered the ADAMSTART and ADAMDEV commands.)
The system can extract information from interface files stored in the second
form faster than from files stored in the first form, so this improves
efficiency.

\section{Parameter specifications}
\markboth{Programmers}{\stardocname}

The different types of fieldname provided in an interface file for controlling
the user interface fall naturally into various groups which have a particular
function, as shown in Table~\ref{T_fields}.
\begin{table}[htb]
\centering
\begin{tabular}{|l|l|l|} \hline
{\em Group} & {\em Fieldname} & {\em Function} \\
\hline
Data type \& access & {\bf TYPE} & Data type \\
& {\bf PTYPE} & Device type \\
& {\bf ACCESS} & Access mode \\
\hline
Command line & {\bf POSITION} & Position in command line \\
& {\bf KEYWORD} & Keyword in command line \\
\hline
Prompt & {\bf KEYWORD} & Keyword in prompt \\
& {\bf PROMPT} & Prompt text \\
& {\bf HELP} & On-line help text \\
& {\bf HELPKEY} & On-line help text \\
& {\bf PPATH} & Prompt-value search path \\
& {\bf ASSOCIATION} & Associated global parameter \\
& {\bf DEFAULT} & Default prompt-value \\
\hline
Value & {\bf VPATH} & Parameter-value search path \\
& {\bf ASSOCIATION} & Associated global parameter \\
& {\bf DEFAULT} & Default parameter-value \\
& {\bf IN} & Valid value set \\
& {\bf RANGE} & Valid value range \\
\hline
\end{tabular}
\caption{Groups of fieldnames used in parameter specifications in interface
 files.}
\label{T_fields}
\end{table}
Some of the fieldnames have slightly different functions in different contexts
and so belong to more than one group.
The groups and their fieldnames are described in more detail below.
Each fieldname description is headed by a line with the format:
\begin{quote}
\begin{description}
\item [FIELDNAME] {\tt fieldvalue}
\end{description}
\end{quote}
This shows how this fieldname is specified.

\begin{description}

\item [Data type \& access group] ---

These fieldnames specify the data type of the parameter, and how it should be
accessed:

\begin{description}

\item [TYPE] {\tt data-type}

This specifies the {\em type}\, of a {\em Normal}\, parameter, {\em i.e.}\,
any parameter except a device type.
It may used by the system to check that a supplied parameter value is of a
type that the application program is expecting.
It also allows automatic conversion between the primitive data types.
It is important to distinguish between the type of a parameter and the type
of the value to be obtained from it by the application.

The \verb%data-type% may be:
\begin{itemize}
\item An HDS primitive type ({\em i.e.}\, \_INTEGER, \_REAL, \_DOUBLE,
\_LOGICAL, \_CHAR, \_UBYTE, \_BYTE, \_UWORD, \_WORD).
\item LITERAL --- formerly used to specify that the parameter is of type \_CHAR,
 and to force the parameter system to accept unquoted character strings as
 character values rather than HDS object names.
 This is now the standard behaviour for parameters of type \_CHAR, so data-type
 LITERAL is no longer required.
\item A `user-defined' type (IMAGE, SPECTRUM {\em etc.}).
\item UNIV --- used by convention when the parameter type is unimportant.
 Actually, this is treated the same as other `user-defined' types.
\item A `facility-recommended' type (TAPE for MAG, FILENAME for FIO {\em etc.}).
 These are used by convention --- any user-defined type will do.
\end{itemize}

If the TYPE field is omitted, type UNIV is assumed.

{\em Example:}

\begin{small}
\begin{verbatim}
    parameter INPIC1
      type    image
\end{verbatim}
\end{small}

An acceptable value for parameter \verb+INPIC1+ is specified as being of type
\verb+image+.

\item [PTYPE] {\tt device}

This specifies that a parameter is of the {\em Device} type.

Most parameters are associated with objects whose {\em values} belong to some
data type, {\em e.g.}\, \_REAL.
These are known as {\em Normal} parameters.
However, some parameters are associated with a hardware {\em device},
{\em e.g.}\, a magnetic tape drive or a graphics terminal.
These are known as {\em Device} parameters.
The value of a device parameter is a device name.

{\em Example:}

\begin{small}
\begin{verbatim}
    parameter PLOTTER
      ptype   device
\end{verbatim}
\end{small}

The parameter \verb+PLOTTER+ is specified as being of device type.

\item [ACCESS] {\tt access-mode}

This specifies the {\em mode of access} the program requires to a parameter
value.
Permitted values for the `\verb+access-mode+' field are:
\begin{quote}
\begin{description}
\item [read] --- read-only access permitted.
\item [write] --- write-only access permitted.
\item [update] --- read and write access permitted.
\end{description}
\end{quote}
Thus, if a program does not modify the value associated with a parameter,
specify `read' access.
If the value is to be written, specify `write' access.
If the value is to be modified, specify `update' access.
The default value is `update'.

{\em Example:}

\begin{small}
\begin{verbatim}
    parameter INPIC1
      access  'read'
\end{verbatim}
\end{small}

The parameter \verb+INPIC1+ is specified as being restricted to `\verb+read+'
access.

\end{description}

\item [Command line group] ---

These fieldnames specify how a parameter should be specified on a command line:

\begin{description}

\item [POSITION] {\tt ipos}

This specifies the {\em position} on a command line where a value for a
parameter may appear.

The `\verb+ipos+' field is an integer.
A value of `1' specifies the first parameter position, and so on.

This fieldname must be given for every parameter capable of being specified by
position on the command line; there is no default value.
Each such parameter must have a different position number.
Positions allocated must form a contiguous set.
If this fieldname is omitted, then a keyword must be specified whenever the
parameter is used on a command line.

{\em Example:}

\begin{small}
\begin{verbatim}
    parameter  INPIC1
      position 1
\end{verbatim}
\end{small}

The parameter \verb+INPIC1+ is associated with the first parameter position on
the command line.

\item [KEYWORD] {\tt name}

This specifies the name by which a parameter can be identified on the command
line, and by which the parameter is known to the user.

The `\verb+name+' field specifies the keyword.
This may differ from the parameter name.

This fieldname separates a program's view of its parameters from a user's view
of them.
It is possible for a programmer to re-write a program using completely
different parameter names, but the command seen by a user can be kept the same
as before by just leaving the keywords as they were.
Conversely, a user's view of a program can be changed by just changing the
keyword specifications in the interface file.

If a keyword is not specified for a parameter, the system will use the
parameter name as the keyword in the prompt and on the command line.

{\em Example:}

\begin{small}
\begin{verbatim}
    parameter INPIC1
      keyword 'PIC1'
\end{verbatim}
\end{small}

The keyword for parameter \verb+INPIC1+ is specified as being \verb+PIC1+.
Thus, on the command line we could enter:

\begin{small}
\begin{verbatim}
    ICL> ADD PIC1=ramp1
\end{verbatim}
\end{small}

to specify that the input image is \verb+ramp1+.

\end{description}

\item [Prompt group] ---

These fieldnames determine the appearance of the prompt for a parameter value.
A prompt has the following syntax:
\begin{quote}
{\em keyword -- prompt-string /suggested-value/}~\verb+>+
\end{quote}
The following fieldnames are involved:

\begin{description}

\item [KEYWORD] {\tt name}

The keyword forms the first part of the prompt.
This fieldname is described above in the command line group.

\item [PROMPT] {\tt prompt-string}

This specifies the {\em prompt-string} field in the prompt for a parameter
value.

This field is a character string with a maximum size of about 80 characters ---
it is not possible to give a single figure which is correct in all cases.
The prompt-string can also be set by the program using the PAR\_PROMT routine.

If this fieldname is not specified, the prompt-string field is omitted from the
prompt.

{\em Example:}

\begin{small}
\begin{verbatim}
    parameter INPIC1
      prompt  'First input image'
\end{verbatim}
\end{small}

The prompt-string for parameter \verb+INPIC1+ is specified as
`\verb+First input image+'.

\item [HELP] {\tt help-text}

This gives information about the meaning of the parameter.

The `\verb+help-text+' field is a character string containing the information.
The maximum size of the field is about 132 characters --- it is not possible
to give a single figure which is correct in all cases.

If the user is prompted for a parameter value, a possible reply is `?'.
The system responds to this by displaying the `\verb+help-text+' on the
terminal and re-prompting the user for a value.

{\em Example:}

\begin{small}
\begin{verbatim}
    parameter INPIC1
      help    'Name of IMAGE structure containing first input data array'
\end{verbatim}
\end{small}

This defines the help text for parameter \verb+INPIC1+.

\item [HELPKEY] {\tt filename key1 key2 ...}

This specifies a help file, and module within it, at which {\em multi-line}
help text for the parameter may be found.
The text to be displayed is stored in the module defined by
\verb+key1 key2 ...+ in the help file \verb+filename+.
(The keys represent the usual hierarchical keywords used to access the help
system).
This is replacing the earlier {\em one-line} help provided by the \verb+HELP+
specification.

If \verb+filename key1 key2 ...+ is replaced by \verb+*+ or \verb+'*'+, the
default values:

\begin{small}
\begin{verbatim}
    interface_name PARAMETERS parameter_name
\end{verbatim}
\end{small}

will be used, where \verb+interface_name+ is the name specified in the current
INTERFACE field.
Use of this default implies that the library has been specified using
the \verb+HELPLIB+ specification.

{\em Example:}

\begin{small}
\begin{verbatim}
    interface DISP
        helplib 'disp_dir:disp disp parameter'
        parameter IMAGE
            helpkey 'image'
\end{verbatim}
\end{small}


\item [PPATH] {\tt prompt-value-resolution-path}

This specifies the search path to be followed to obtain a value for the
{\em suggested-value} field in the prompt.
This search path is usually referred to as the `ppath'.

If a suggested value is required, the system looks at the `ppath' specification,
picks out the first specifier, and tries to find a value from this source.
If a value is not found, the next specifier is extracted and another search is
made.
This process continues until either a value is found, or the search path is
exhausted.

The `\verb+prompt-value-resolution-path+' field consists of a set of specifiers
separated by commas.
The valid specifiers are the same as for `vpath' (see below), except that the
`{\bf prompt}' and `{\bf noprompt}' specifiers are invalid in this context and
should not be used.

If `ppath' is not specified, a default search path of `dynamic,default' is
used.

{\em Example:}

\begin{small}
\begin{verbatim}
    parameter INPIC1
      ppath   'global,current'
\end{verbatim}
\end{small}

This specifies a `ppath' of `\verb+global+' followed by '\verb+current+'
for parameter INPIC1.

\item [ASSOCIATION] {\tt association-specification}

See below in the `Value' group.

\item [DEFAULT] {\tt default-value}

See below in the `Value' group.

\end{description}

\item [Value group] ---

These fieldnames determine the value given to a parameter, and check that this
satisfies given criteria:

\begin{description}

\item [VPATH] {\tt parameter-value-resolution-path}

This specifies the search path to be followed when searching for a parameter
value.
This search path is usually referred to as the `vpath'.

A value can come from many different sources, and these sources may need to be
searched in different orders.
The `\verb+parameter-value-resolution-path+' field specifies these sources
and the search order --- this constitutes the search-path.
The field consists of specifiers separated by commas.
The valid specifiers are:
\begin{quote}
\begin{description}
\item [current] --- use the current value of the parameter.
\item [default] --- use the default specified in the interface file.
\item [dynamic] --- use the dynamic default specified by the program.
\item [global] --- use the value of the associated global parameter.
\item [noprompt] --- do not prompt the user for a value.
\item [prompt] --- prompt the user and obtain a value.
\end{description}
\end{quote}
If the value of a parameter is not specified on the command line, the system
looks at the `vpath' specification, picks out the first specifier, and tries
to find a value from this source.
If a value is not found, the next specifier is extracted and another search is
made.
This process continues until either a value is found, or the search path is
exhausted.
If the search path is exhausted the user is prompted for a value, unless
the `noprompt' specifier has been given, in which case the system returns a
PAR\_\_NULL status value to the program.

{\em Example:}

\begin{small}
\begin{verbatim}
    parameter INPIC1
      vpath   'dynamic,current'
\end{verbatim}
\end{small}

This specifies a `vpath' of `\verb+dynamic+' followed by '\verb+current+'
for parameter `\verb+INPIC1+'.

\item [ASSOCIATION] {\tt association-specification}

This associates a program parameter with a global parameter.

If the system comes across the `global' specifier in a vpath or ppath, this
fieldname tells it which global parameter to use as a source for the value.

The `\verb+association-specification+' field is a character string which
specifies the global parameter.
This string has two parts:

\begin{small}
\begin{verbatim}
    <association-operator><parameter-specification>
\end{verbatim}
\end{small}

The `\verb+association-operator+' field must one of the following:
\begin{quote}
\begin{description}
\item [$<$--] (read-only).
\item [$<$--$>$] (read-write).
\item [--$>$] (write-only).
\end{description}
\end{quote}
`write' means that if the task completes successfully, the current value of the
parameter will be copied to the global parameter.

The `\verb+parameter-specification+' field must be the name of a global
parameter.

This fieldname only needs to be specified if the specifier `global' is used in
the ppath or vpath for the parameter.

{\em Example:}

\begin{small}
\begin{verbatim}
    parameter     INPIC1
      vpath       'prompt'
      ppath       'global'
      association '<->global.data_array'
\end{verbatim}
\end{small}

The global parameter `\verb+global.data_array+' will be used as the
suggested-value in the prompt.
The parameter value will be obtained by prompting if it is not specified on
the command line.

\item [DEFAULT] {\tt default-value}

This specifies one or more default values for a parameter.

The `\verb+default-value+' field is a string of alphanumeric characters
containing items to be used as values for the parameter.
The values may be any expression acceptable to the parameter system.
This fieldname must not be specified before the `type' fieldname has been
declared.

{\em Example:}

\begin{small}
\begin{verbatim}
    parameter OTITLE
      type    '_CHAR'
      vpath   'default'
      default 'KAPPA - Add'
\end{verbatim}
\end{small}

The default value for parameter \verb+OTITLE+ is `\verb+KAPPA - Add+'.
The vpath specifies that this will be used as the parameter value if one
is not specified on the command line.

\item [IN] {\tt set-of-values}

This specifies a set of valid values for a parameter.

The `\verb+set-of-values+' field consists of a list of constants separated by
commas.
These constants are the valid values.

Checking is carried out when a program attempts to get a value for a parameter.
If the supplied value does not belong to the valid set, either the status
SUBPAR\_\_OUTRANGE is returned, or the system attempts to prompt for the value.
No checking occurs when writing a value.

This fieldname should only be used for scalar parameters and must be specified
after `type'.
A parameter cannot have both a `\verb+range+' and an `\verb+in+' field.

{\em Example:}

\begin{small}
\begin{verbatim}
    parameter BOXSIZ
      type    '_INTEGER'
      in      3, 5, 7, 9, 11, 13, 15
\end{verbatim}
\end{small}

The only valid values for parameter \verb+BOXSIZ+ are 3, 5, 7, 9, 11, 13, 15.

\item [RANGE] {\tt range-of-values}

This specifies a range of valid values for a parameter.

The `\verb+range-of-values+' field consists of two constants separated by a
comma.
These constants specify the range boundaries.
This fieldname has the same characteristics as the `\verb+in+' field described
above.

{\em Example:}

\begin{small}
\begin{verbatim}
    parameter SIGMA
      type    '_REAL'
      range   0.1, 5.0
\end{verbatim}
\end{small}

The only valid values for parameter \verb+SIGMA+ lie in the range 0.1 to 5.0.

\end{description}
\end{description}

\section{Message specifications}

If a program uses the Message or Error systems it can specify one or more
`Message Parameters'.
Text can be associated with these parameters in the interface file by using a
`Message Specification'.
However, this is not essential since text specified in the program will be
used by default.

The Message specification has the form:

\begin{small}
\begin{verbatim}
      message PARNAME
         fieldname fieldvalue(s)
         ...
      endmessage
\end{verbatim}
\end{small}

where `\verb+PARNAME+' is the Message Parameter name which is used in the call
to an ERR or MSG routine.

Currently only one fieldname is supported:
\begin{quote}
\begin{description}
\item [TEXT] {\tt string}

This specifies the text to be associated with the message parameter.

The `\verb+string+' field contains the text to be used.
The string may be anything acceptable to the ERR or MSG routine ({\em i.e.}\,
it may include tokens {\em etc.}).

{\em Example:}

\begin{small}
\begin{verbatim}
    message MESS
      text  'TESTR prints ^X'
    endmessage
\end{verbatim}
\end{small}

This will replace the message defined in the program with the message specified
by `\verb+text+'.
\end{description}
\end{quote}

\section{Monoliths}

When a set of programs are grouped into a monolith, their interface files
should be concatenated into a single interface file as follows:

\begin{small}
\begin{verbatim}
     monolith MONNAME
         interface PROGNAME1
          ...
         endinterface
         interface PROGNAME2
          ...
         endinterface
         ...
      endmonolith
\end{verbatim}
\end{small}

An example is given in Section~\ref{S_Mono}.

\section{Parameter states}
\markboth{Programmers}{\stardocname}

A simple program will get values for its parameters, do something with them,
and then finish.
However, a programmer may wish to obtain a sequence of values for a single
parameter, and a user may wish to indicate that no parameter value should
be used.
To deal with these requirements, ADAM uses the concept of {\em Parameter State}.
A parameter can be in one of the four states shown in Table~\ref{T_states}.
This shows how the various states are entered.

\begin{table}[htb]
\centering
\begin{tabular}{|l|l|l|}
\hline
{\em State} & {\em Method of entry} & {\em Method of getting a value} \\
\hline
{\bf GROUND} & Program entry & Follow the vpath in the interface file \\
{\bf ACTIVE} & GET, PUT, ASSOC routines & Use the current value \\
{\bf ANNULLED} & Input value of `!' & Value not set; \verb+STATUS=PAR__NULL+ \\
{\bf CANCELLED} & CANCL routines & Prompt the user for a value \\
\hline
\end{tabular}
\caption{Parameter states and values.}
\label{T_states}
\end{table}

In simple programs, each parameter goes from GROUND to ACTIVE state when it is
given a value and stays in that state until the programs exits.
The significance of the parameter state is its effect on what happens when a
program asks for a value for a parameter; this is also shown in
Table~\ref{T_states}.
If a program asks repeatedly for the value of a parameter, as in:

\begin{small}
\begin{verbatim}
    CALL PAR_GET0R('X', XVALUE, STATUS)
    ...
    CALL PAR_GET0R('X', XVALUE, STATUS)
    ...
\end{verbatim}
\end{small}

the first call will get a value for `X' by following the vpath.
However, the second and succeeding calls will obtain the same value as obtained
by the first call since this is the `current value' and the vpath is not
used.
In particular, the user will not be prompted for another value.
However, if the programmer `cancels' the parameter by calling PAR\_CANCL before
each attempt to get a value:

\begin{small}
\begin{verbatim}
    CALL PAR_GET0R('X', XVALUE, STATUS)
    ...
    CALL PAR_CANCL('X', STATUS)
    CALL PAR_GET0R('X', XVALUE, STATUS)
    ...
    CALL PAR_CANCL('X', STATUS)
    ...
\end{verbatim}
\end{small}

the user will be prompted for each call of PAR\_GET0R.
This technique is useful when you have finished processing one object associated
with a parameter, and want to process a different object within the same
program.

Each subroutine library which has been integrated with the ADAM system has
a CANCL routine, identical in form to PAR\_CANCL, with which to cancel the
association of a parameter with an object previously made by the ASSOC routine.
For example:

\begin{small}
\begin{verbatim}
    CALL DAT_ASSOC('X', 'READ', LOC, STATUS)
      <do something with the first object associated with parameter X>
    CALL DAT_CANCL('X', STATUS)
    CALL DAT_ASSOC('X', 'READ', LOC, STATUS)
      <do something with a new object associated with parameter X>
    CALL DAT_CANCL('X', STATUS)
\end{verbatim}
\end{small}

The CANCL routines are different from the ANNUL routines considered earlier.
For example, in:

\begin{small}
\begin{verbatim}
    CALL DAT_ASSOC('X', 'READ', LOC, STATUS)
      <do something with the first object associated with parameter X>
    CALL DAT_ANNUL(LOC, STATUS)
    CALL DAT_ASSOC('X', 'READ', LOC, STATUS)
      <do something with the same object associated with parameter X>
    CALL DAT_ANNUL(LOC, STATUS)
\end{verbatim}
\end{small}

the second DAT\_ASSOC will associate the same object with parameter X as the
first one did, because DAT\_ANNUL does not break the association.
Notice that the first parameter of ANNUL routines is the identifier used to
address the object, whereas the first parameter of CANCL routines is the
parameter name.

The ANNUL routines have nothing to do with the `ANNULLED' state of a program
parameter; it's just an unfortunate clash of words.
