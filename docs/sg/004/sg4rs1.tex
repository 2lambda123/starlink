\part{REFERENCE}
\label{P_refsect}

\chapter{ICL}
\label{C_rsicl}
\pagestyle{myheadings}
\markboth{Reference}{\stardocname}

\small

\section{Syntax}

This section gives a summary of the ICL language syntax.
Square brackets imply that the enclosed part of the construct is optional.

\begin{description}

\item [Operational Mode] ---

\begin{description}
\item [Direct:] \mbox{}

Statements are executed as they are entered.

\item [Procedure:] \mbox{}

Statements are first entered into a procedure, then the procedure is
executed.
\end{description}

\item [Data types] ---

\begin{description}
\item[Real]
\item[Integer]
\item[Logical]
\item[String]
\end{description}

\item [Expression] ---

\begin{verbatim}
    value [operator value operator value ...]
\end{verbatim}

\item [Value] ---

\begin{description}
\item [Constant:] \mbox{}
\begin{description}
\item[Real] - \verb%       1.234E-5,  3.14159%
\item[Integer] - \verb@    123,  %B100110,  %O377,  %Xffff@
\item[Logical] - \verb%    TRUE,  FALSE%
\item[String] - \verb%     'This is a string',  "So is this"%
\end{description}
\item [Variable:] \mbox{}
\begin{description}
\item[Name] - Letter followed by letters, digits, or underscores (15 significant)
\end{description}
\item [Function:] \mbox{}

See below.
\end{description}

\item [Operator] ---

\begin{quote}
\begin{tabular}{cl}
     1 &            \verb%**% \\
     2 &            \verb%*  /% \\
     3 &            \verb%+  - % \\
     4 &            \verb%=  >  <  >=  <=  <>  :% \\
     5 &            \verb%NOT  AND  OR  &%
\end{tabular}
\end{quote}
\begin{quote}
\begin{description}
\item[\&] performs string concatenation.
\item[:] is formatting operator (X:n:m is X converted to an n character string
 with m decimal places).
\end{description}
\end{quote}

\item [Statement] ---

Normally one statement per line.
A tilde (\verb%~%) at the end of a line indicates that the statement continues
on the next line.

\begin{description}
\item [Direct Statement:] \mbox{}

These statements can be used in any operational mode.
\begin{description}
\item [Comment] -

Any line starting with the characters `\{' or `;'.
\item [Immediate Statement] -

= expression {\em (prints value of expression)}
\item [Assignment Statement] -

variable = expression {\em (assignment to variable)}
\item [Command] -

command\_name p1 p2 \ldots {\em (command or procedure call)}

In commands or procedure calls, parameters are separated by a comma or spaces.
Expressions or variables should be enclosed in parentheses.
\end{description}

\item [Control Statement:] \mbox{}

These statements can only be used in the Procedure operational mode.
\begin{description}

\item [IF Statement] -

\begin{verbatim}
    IF expression
      statements
    [ELSE IF expression
      statements]
    [ELSE
      statements]
    END IF
\end{verbatim}

\item [LOOP Statement] -

\begin{verbatim}
    LOOP
      statements
    END LOOP

    [LOOP] FOR var = exp1 TO exp2 [STEP exp3]
      statements
    END LOOP

    [LOOP] WHILE expression
      statements
    END LOOP
\end{verbatim}
You can escape from a loop via a BREAK statement.

\newpage

\item [PROCEDURE Statement] -

\begin{verbatim}
    [HIDDEN] PROC name p1 p2 ....
      statements
    END PROC
\end{verbatim}

\item [EXCEPTION Statement] -

\begin{verbatim}
    EXCEPTION name
      statements
    END EXCEPTION
\end{verbatim}

\end{description}
\end{description}

\item [Command Search Path] ---

\begin{enumerate}
\item User defined commands (defined by DEFSTRING, DEFUSER, DEFPROC)
\item User written procedures (defined by PROC)
\item ICL commands to control ADAM tasks (ALOAD, etc)
\item ADAM commands (defined by DEFINE)
\item Other ICL built-in commands (see below)
\end{enumerate}
\end{description}

\section{Commands}

The commands which are likely to be of most use in data analysis are classified
below.
Commands relevant mainly to data acquisition are omitted.
A full list is contained in \xref{SG/5}{sg5}{}.
\begin{description}

\item [Information \& Escape] ---

\begin{quote}
\begin{description}
\item[HELP] :
 Display on-line documentation.
\item[EXIT] :
 Exit from ICL.
\end{description}
\end{quote}

\item [Defining user commands] ---

\begin{quote}
\begin{description}
\item[DEFSTRING] :
 Associate a command with an equivalence string.
\item[DEFINE] :
 Define a command to run an ADAM task.
\item[DEFUSER] :
 Associate a command with a user written subroutine.
\item[DEFPROC] :
 Associate a command with a procedure.
\end{description}
\end{quote}

\item [I/O] ---

\begin{description}
\item [Terminal I/O:] \mbox{}
\begin{description}
\item[PRINT] :
 Output to the terminal.
\item[INPUT] :
 Input a string.
\item[INPUTI] :
 Input integers.
\item[INPUTL] :
 Input logical values.
\item[INPUTR] :
 Input real numbers.
\end{description}
\item [Screen mode:] \mbox{}
\begin{description}
\item[SET SCREEN] :
 Set screen I/O mode.
\item[SET NOSCREEN] :
 Set normal I/O mode.
\item[SET ATTRIBUTES] :
 Set attributes for text written with the LOCATE command.
\item[LOCATE] :
 Write to screen at specified position.
\item[CLEAR] :
 Clear lines in a specified range.
\end{description}
\item [Keyboard facilities:] \mbox{}
\begin{description}
\item[KEY] :
 Define an equivalence string for a key.
\item[KEYTRAP] :
 Specify trapping of a key.
\item[KEYOFF] :
 Turn off trapping of a key.
\end{description}
\item [File I/O:] \mbox{}
\begin{description}
\item[CREATE] :
 Create a file and open it for output.
\item[OPEN] :
 Open an existing file for input.
\item[APPEND] :
 Open an existing file for appending.
\item[CLOSE] :
 Close a file opened with CREATE, OPEN, or APPEND.
\item[WRITE] :
 Write to a file.
\item[READ] :
 Read a string from a file.
\item[READI] :
 Read integers from a file.
\item[READL] :
 Read logical values from a file.
\item[READR] :
 Read real numbers from a file.
\end{description}
\end{description}

\item [Access to DCL] ---

\begin{description}
\item [General:] \mbox{}
\begin{description}
\item[\$] :
 Issue a DCL command.
\item[SPAWN] :
 Spawn a subprocess to issue one or more DCL commands.
\item[DEFAULT] :
 Set/Show default directory.
\end{description}
\item [Managing tape drives:] \mbox{}
\begin{description}
\item[ALLOC] :
 Allocate a device.
\item[DEALLOC] :
 Deallocate a device.
\item[MOUNT] :
 Mount a tape.
\item[DISMOUNT] :
 Dismount a tape.
\end{description}
\end{description}

\item [Parameters] ---

\begin{quote}
\begin{description}
\item[SETPAR] :
 Set the value of a parameter where the program has an associated command.
\item[GETPAR] :
 Get the value of a parameter where the program has an associated command.
\item[CREATEGLOBAL] :
 Create a global parameter.
\item[SETGLOBAL] :
 Set the value of a global parameter.
\item[GETGLOBAL] :
 Get the value of a global parameter into an ICL variable.
\end{description}
\end{quote}

\item [Procedures] ---

\begin{description}
\item [Listing and editing:] \mbox{}
\begin{description}
\item[LIST] :
 List a procedure.
\item[PROCS] :
 List procedure names.
\item[VARS] :
 List procedure variables.
\item[EDIT] :
 Edit a procedure.
\item[SET EDITOR] :
 Change the editor (TPU, EDT, or LSE) used by EDIT.
\end{description}
\item [Saving and loading:] \mbox{}
\begin{description}
\item[SAVE] :
 Save a procedure.
\item[LOAD] :
 Accept commands from a saved procedure.
\item[DELETE] :
 Delete a procedure.
\end{description}
\item [Tracing execution:] \mbox{}
\begin{description}
\item[SET TRACE] :
 Switch tracing of procedures ON.
\item[SET NOTRACE] :
 Switch tracing of procedures OFF.
\end{description}
\end{description}

\item [Errors and Exceptions] ---

\begin{quote}
\begin{description}
\item[SIGNAL] :
 Signal an ICL exception.
\end{description}
\end{quote}

\item [Help system] ---

\begin{quote}
\begin{description}
\item[DEFHELP] :
 Define the source of help information.
\end{description}
\end{quote}

\item [Miscellaneous] ---

\begin{quote}
\begin{description}
\item[TASKS] :
 List all loaded tasks.
\item[SAVEINPUT] :
 Save previous input lines in a text file.
\item[SETPRECISION] :
 Set the number of decimal digits precision for unformatted conversions
 of real values to strings.
\end{description}
\end{quote}
\end{description}

\section{Functions}

The following functions are available for use in expressions (N.B.\ The `$*$'
symbol in the {\em Type} column stands for `Real or Integer').

\begin{description}

\item [Mathematical Functions] ---

\begin{tabbing}
VARIABLE(proc,X)xxxx\=Logicalxxx\=\kill
{\em Name} \> {\em Type} \> {\em Definition} \\
\\
SIN(X) \> Real \> Trig. functions (x in radians). \\
COS(X) \> Real \\
TAN(X) \> Real \\
SIND(X) \> Real \> Trig. functions (x in degrees). \\
COSD(X) \> Real \\
TAND(X) \> Real \\
ASIN(X) \> Real \> $\sin^{-1} x \; where -1 \leq x \leq 1 $;
 $ -\pi/2 \leq result \leq \pi/2$.\\
ACOS(X) \> Real \> $\cos^{-1} x \; where -1 \leq x \leq 1 $;
 $ 0 \leq result \leq \pi$.\\
ATAN(X) \> Real \> $\tan^{-1} x $; $ -\pi/2 \leq result \leq \pi/2$.\\
ATAN2(X1,X2) \> Real \> $\tan^{-1} (x_{1}/x_{2}) $; $ -\pi < result \leq \pi $.\\
ASIND(X) \> Real \> Inverse trig functions in degrees. \\
ACOSD(X) \> Real \\
ATAND(X) \> Real \\
ATAN2D(X1/X2) \> Real \\
SINH(X) \> Real \> Hyperbolic functions. \\
COSH(X) \> Real \\
TANH(X) \> Real \\
ABS(X) \> * \> $\mid x \mid$.\\
DIM(X1,X2) \> * \> Positive difference between $x_1$ and $x_2$. \\
EXP(X) \> Real \> $e^{x}$. \\
LOG(X) \> Real \> $\ln x$.\\
LOG10(X) \> Real \> $\log_{10} x $.\\
MIN(X1,X2,...) \> * \> Minimum of two or more arguments. \\
MAX(X1,X2,...) \> * \> Maximum of two or more arguments. \\
MOD(X1,X2) \> * \> Remainder when $x_1$ is divided by $x_2$. \\
SIGN(X1,X2) \> * \> Transfer of sign $\mid x_1 \mid$ Sign $x_2$. \\
SQRT(X) \> Real \> $\sqrt{x}$.\\
\end{tabbing}

\item [Formatting and String Handling Functions] ---

\begin{tabbing}
VARIABLE(proc,X)xxxx\=Logicalxxx\=\kill
BIN(I,n,m) \> String \> Integer I formatted in binary into an n char string \\
 \> \> with m significant digits. \\
DEC(I,n,m) \> String \> Integer I formatted in decimal into an n char string \\
 \> \> with m significant digits. \\
HEX(I,n.m) \> String \> Integer I formatted in hexadecimal into an n char string \\
 \> \> with m significant digits. \\
OCT(I,n,m) \> String \> Integer I formatted in octal into an n char string \\
 \> \> with m significant digits. \\
DECL(S) \> Real \> Declination in radians from string in degrees, minutes, seconds. \\
DEC2S(R,NDP,SEP) \> String \> String in DMS with separator SEP and NDP decimal places \\
 \> \> on seconds, from Dec in radians. \\
RA(S) \> Real \> Right Ascension in radians from a string in hours, minutes, seconds. \\
RA2S(R,NDP,SEP) \> String \> String in HMS with separator SEP and NDP decimal places \\
 \> \> on seconds, from RA in radians. \\
CHAR(I) \> String \> Character with ASCII value I. \\
ICHAR(S) \> Integer \> ASCII value of first character of string S. \\
INDEX(S1,S2) \> Integer \> Position of first occurrence of string S2 in string
 S1.\\
LEN(S) \> Integer \> Length of string. \\
ELEMENT(I,DELIM,S) \> String \> Ith element of delimited string. \\
SUBSTR(S,n,m) \> String \> Substring of S beginning at n of length m. \\
SNAME(S,n,m) \> String \> Name derived by concatenating string S with integer n \\
 \> \>          formatted into m characters including leading zeros. \\
UPCASE(S) \> String \> String S converted to upper case. \\
LGE(S1,S2) \> Logical \> True if S1 $\geq$ S2 (ASCII collating sequence). \\
LGT(S1,S2) \> Logical \> True if S1 $>$ S2 (ASCII collating sequence). \\
LLE(S1,S2) \> Logical \> True if S1 $\leq$ S2 (ASCII collating sequence). \\
LLT(S1,S2) \> Logical \> True if S1 $<$ S2 (ASCII collating sequence). \\
\end{tabbing}

\item [Bitwise Logical Operations] ---

\begin{tabbing}
VARIABLE(proc,X)xxxx\=Logicalxxx\=\kill
IAND(I1,I2) \> Integer \> Bitwise AND. \\
IOR(I1,I2) \> Integer \> Bitwise OR. \\
IEOR(I1,I2) \> Integer \> Bitwise Exclusive OR. \\
INOT(I) \> Integer \> Bitwise Complement. \\
\end{tabbing}

\item [Type Conversion and Inquiry Functions] ---

\begin{tabbing}
VARIABLE(proc,X)xxxx\=Logicalxxx\=\kill
FLOAT(I) \> Real \> Integer I converted to real. \\
IFIX(X) \> Integer \> X converted to integer by truncation. \\
INT(X) \> Integer \> X converted to integer by truncation. \\
NINT(X) \> Integer \> Nearest integer to X. \\
INTEGER(X) \> Integer \> X converted to integer. \\
LOGICAL(X) \> Logical \> X converted to logical. \\
REAL(X) \> Real \> X converted to real.\\
STRING(X) \> String \> X converted to string. \\
TYPE(X) \> String \> The type of X (`REAL', `INTEGER', `LOGICAL', \\
 \> \>`STRING', or `UNDEFINED'). \\
UNDEFINED(X) \> Logical \> TRUE if X is undefined. \\
\end{tabbing}

\item [Miscellaneous Functions] ---

\begin{tabbing}
VARIABLE(proc,X)xxxx\=Logicalxxx\=\kill
DATE() \> String \> Current date. \\
TIME() \> String \> Current time. \\
GETNBS(S) \> Any \> Value of noticeboard item. \\
GET\_SYMBOL(S) \> String \> Value of DCL symbol. \\
INKEY() \> Integer \> Key value of last key trapped (Screen mode only). \\
KEYVALS(S) \> Integer \> Value of key with name S. \\
OK(stat) \> Logical \> True if VMS Status OK. \\
FILE\_EXISTS(S) \> Logical \> True if file S exists. \\
RANDOM(I) \> Real \> Random number between 0 and 1 from seed variable I. \\
VARIABLE(proc,X) \> Any \> Returns value of variable X of procedure proc.\\
\end{tabbing}
\end{description}

\section{Exceptions}

\begin{center}
\begin{tabular}{ll}
{\em Name} & {\em Description} \\
\\
ADAMERR  &  An error has occurred in an ADAM task. An associated message will
   be output. \\
ASSNOTVAR  &  An assignment has been made to a procedure formal parameter
   which does \\
   & not correspond to a variable in the procedure call. \\
CLOSEERR  &  Error closing text file. \\
CONVERR  &  Error converting RA or Dec to string or vice versa. \\
CTRLC  &  A Control-C has been entered on the terminal.\\
DEVERR  &  Error allocating or mounting device.\\
EDITERR  &  Attempt to use the LSE or TPU editors when they are not available.\\
EOF & End of file encountered on text file operation. \\
FIGERR  &  Error in a FIGARO program, or FIGARO not available. \\
FLTDIV  &  Floating point division by zero. \\
FLTOVF  &  Floating point overflow. \\
IFERR  &  The expression in an IF or ELSE IF statement does not evaluate to a
  logical value. \\                              
INTOVF  &  Integer overflow. \\
INVARGMAT  &  Invalid argument to a mathematical function.  \\
INVSET  &  Invalid SET command.  \\
LOGZERNEG  &  Logarithm of zero or negative number.  \\
NBSERR  &  Error in GETNBS or PUTNBS. \\
OPENERR  &  Error opening text file. \\
OPNOTLOG  &  Operands of a logical operator (AND, OR {\em etc.}) are not
   logical values.  \\
OPNOTNUM  &  Operands of a formatting operation (:) are not numeric. \\
PROCERR  &  Unrecognized procedure or command name.  \\
READERR &  Error reading from text file. \\
RECCALL  &  Attempt to make a recursive call of a procedure. \\
SCREENERR  &  Error in screen mode I/O. \\
SQUROONEG  &  Square root of negative number. \\
STKOVFLOW  &  ICL's stack has overflowed. \\
STKUNDFLOW  &  ICL stack underflow --- if this occurs it indicates an internal
   error in ICL. \\
TOOFEWPARS  &  Not enough parameters for a function or command. \\
TOOMANYPARS  &  Too many parameters for a procedure or command. \\
UNDEFVAR  &  Attempt to use an undefined variable --- {\em i.e.} one that has
   not yet had a value assigned.  \\
UNDEXP  &  Undefined Exponentiation.  \\
USERERR  &  Error accessing a routine defined using DEFUSER, or error during
   such a routine. \\
WHILEERR & The expression in a WHILE statement does not evaluate to a logical
   value. \\                    
WRITERR  &  Error writing to text file.
\end{tabular}
\end{center}
