\chapter*{Acknowledgements}

The initial revision of the original version of this document was carried out
by Ken Hartley.
It was then taken over by Mike Lawden who put it into its final form.

The source documents that have been used are extensively referenced within
the text.
We would like to thank the authors of these documents for making them
available.
We would also like to thank the many people who made comments on the
working drafts of this guide.

\chapter*{Preface\label{C_pref}}

   \begin {quote} {\small {Whoever has lived long enough to find out
                  what life is, knows how deep a debt of gratitude
                  we owe to ADAM.} \hfill {\small \bf Mark Twain} }
  \end{quote}

\section*{ADAM}
\label{S_preadam}

ADAM --- the acronym originally stood for the Astronomical Data Acquisition
Monitor --- had its beginnings in the Royal Observatories and is now the
cornerstone of software support for the UK astronomical community.
This document focuses on its role within Starlink for data-reduction and
analysis, for which it is well suited.
However, it should not be forgotten that it is perhaps even more important
for data acquisition at the UK's overseas observatories, where it has no
competitors.

ADAM has three major components --- applications; subroutine libraries
to help build applications; and command languages to run applications.
The libraries can be thought of as a toolkit, or a collection of building
blocks for constructing applications.
{\em Lego,}\, the children's toy, provides a good model --- because care was
taken to define the interface properly (size and spacing of knobs and holes) it
is guaranteed that any piece will fit onto any other piece, even when that
connection was not envisaged when the original piece was designed.

The command language forms the working environment of the user, and the
libraries themselves are sometimes seen as the software environment in which
programmers work.
However, these are both subsidiary to the applications which are available to
help astronomers to carry out their research.

\section*{History}
\label{S_histadam}

Within a year of Starlink being set up, the {\em Interim Starlink Environment}
was delivered and working.
It was always recognized that the facilities it offered were limited --- it
consisted of only a handful of routines to handle parameters and a simple
image format.
The DSCL command language was soon added.

Work started at once on the definitive Starlink Software Environment (SSE).
For various reasons --- insufficient manpower and the untimely death of its
chief architect being the main ones --- a satisfactory product was never
delivered.
In the meantime, the author of DSCL had moved into the La Palma software team
at RGO and built ADAM to control instrumentation for the INT and JKT on
La Palma, using Perkin-Elmer computers.
ROE then adopted ADAM for instrument control on UKIRT and ported it to VAX/VMS.
It used the Hierarchical Data System and SGS/GKS graphics packages developed
at RAL as part of the SSE project.
ROE also re-implemented the SSE parameter system.
Though closely compatible with the SSE, the new version performed much better,
in part because character handling was made more efficient (but at some cost in
future compatibility with Unix and C).

By 1985 it was clear that Starlink was unlikely to have the resources needed
to provide what the community required.
The Project therefore recommended adoption of the IRAF system from the USA.
Workshops were held in late 1985 and early 1986, which were attended by
astronomers representing all major groups of Starlink users.
After a thorough review of many systems (including IRAF, MIDAS, AIPS, variants
of the Interim Environment, IDL, LUCID, ADAM and many more) the community
decided that ADAM offered the best overall prospects.
It was proposed that further development and support would be on a `best
endeavours' basis by all the interested parties.
These proposals were endorsed by the Starlink User Committee.

Subsequently the VAX version of ADAM was adopted as the basis of instrument
control at AAT, for the WHT on La Palma, and the JCMT on Mauna Kea.
In 1989 ICL was adopted as the command language for ADAM, replacing the
earlier ADAMCL.
The need for a more formal method of supporting ADAM was recognized and so in
1990 the ADAM Support Group was established within the Starlink project at RAL.

It is now recognized that the cost/performance ratio of RISC (Reduced
Instruction Set Chip) machines means that they currently provide
typically three times the performance of VMS systems for the same price.
This has inevitably led to growing numbers of astronomers using workstations ---
particularly Suns and DECstations --- running varieties of the Unix Operating
System.
In order to ensure that the investment in Starlink application software is not
wasted, and the community continues to have the enviable coherence which
Starlink provides, the decision was taken in 1991 to remove all the dependence
on VMS from ADAM.
The result is a {\em portable} version of ADAM.
This version has proved to perform very well on Unix systems.

\section*{The importance of ADAM}
\label{S_impadam}

The above history shows that ADAM is essential for every major telescope
which is funded through SERC.
Its role in Starlink means that all UK observational astronomy depends on ADAM.
The spread of Unix to big machines (Cray, IBM, massively parallel systems and
so on) allied with a portable system means that even the computational
modellers in the community can contemplate using ADAM, especially when high
bandwidth communications make it feasible to distribute work between a
local workstation and a remote supercomputer.

ADAM was chosen by the community and is now supported by SERC.
It was chosen because it was the best possible option at that time.
It is fully under the control of the UK community and is not dependent on any
one individual.
It has been designed to handle any astronomical data and is guaranteed to be
supported on all Starlink hardware, and graphics devices in particular.
The existence of a portable version means that it can be made available on
almost any hardware platform.

Chapter 4 describes many packages built using ADAM, ranging from large,
general-purpose packages such as FIGARO and KAPPA, through those aimed at
specific branches of astronomy ({\it e.g.}\ ASTERIX, TSP), to those more
concerned with software and data handling ({\it e.g.}\ SST, CONVERT).
New astronomical applications are being added all the time.
It is hoped that this Guide will lead more people to use them and encourage
users to write their own.
