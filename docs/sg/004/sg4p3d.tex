\chapter{Programming Standards, Conventions and Tools}
\label{C_standards}

The most striking features of the {\em Starlink Software Collection} are
its size (in March 1990 it contained about 1 million lines of code, half a
million comment lines, and another half million blank lines) and the number of
programmers who have contributed to it.
These are both its strength and its weakness.
In particular, it presents a growing maintenance problem.
The only hope of keeping it under control is by `keeping up standards'.
This includes both the sense of `better' rather than `worse', and also `this
is what BSI, ANSI, ISO \ldots have to say on the subject'.

While nobody wants to stifle a programmer's creativity, you have to recognise
that software constructed using the guidelines discussed here is likely to be
a great deal easier to maintain than that written as the spirit moves you.
Software which is very personal to its creator is likely to suffer one of
two fates: it will not be used by anyone else, or the original programmer
will have the task of maintaining it for life.

{\em Software portability} has always been regarded by the Starlink Project as
important, but has, in fact, been of marginal interest to the majority of
users.
Now, however, the spread of Starlink to Unix-based systems means that
portability is a very real problem for a growing number of users.

The aim of these standards is to achieve software which, while providing the
best possible applications, is also portable, maintainable by anyone, and
integrates with the other software in the Collection.
The software tools provided should make it easier to produce compliant
code than not!

\section{Language standards}
\label{S_langstand}

\subsection{Fortran}

\xref{SGP/16}{sgp16}{} offers advice on how to write application programs for
Starlink, embodied in a set of general rules.
As a programmer you should bear in mind the needs of the person who will
eventually be responsible for maintaining your code.
You are strongly encouraged to read
\xref{SGP/16}{sgp16}{} if you have not already done so.

ADAM has a number of special requirements which may mean that one of the
general rules has to be reinterpreted --- in some cases strengthened, in
others relaxed.
There are, in addition, several new rules which do not have to be obeyed when
writing non-ADAM applications.
Here is a summary of the extra rules which apply to ADAM applications, as
compared with ordinary free-standing programs:

\begin{itemize}
\item Initialise variables.

It is {\bf essential} that variables are initialised.
Even the VAX's initialisation to 0 cannot be relied on as the task may or
may not be reloaded between invocations.
DATA statements must only be used to initialise data which will not change.

\item Use the ADAM standard prologues

ADAM standard prologues differ in some respects from the Starlink
standard, allowing less freedom but giving more opportunity for
the automatic production of documentation and help files.
Standard prologues exist for subroutines and interface files.

\item Output messages via the MSG routines.

Message output must be done using the ADAM Message system (MSG) subroutines.

\item Don't use \$, \%, $\wedge$ in messages.

The non-Fortran~77 characters \$, \% and $\wedge$ are
used as escape characters in the ADAM message system.

\item Report errors via the ERR routines.

Error reporting must be done using the ADAM Error system (ERR) subroutines,
and the ADAM Error Strategy should be employed.

\item Set STATUS on failure.

All programs which fail must return to the
environment with an error status value set.
This enables the environment to detect the failure so that users can write
procedures which take appropriate action.

\item When setting STATUS, generate a message.

If a subroutine is entered with STATUS=SAI\_\_OK
but, during execution, sets the STATUS (other than by calling another ADAM
routine), an appropriate error message must be generated using ERR\_REP.

\item Some routines have $>6$ character names.

Some of the ADAM environment package subroutines have names and prefixes
greater than 6 characters.
Where it is necessary to call these, the general rules must be relaxed.

\item Get parameters with the PAR routines etc.

All program parameters must be obtained using the Parameter system PAR
or pkg\_ASSOC subroutines.

\item Use symbols when testing for bad pixels.

A REAL or DOUBLE PRECISION variable may be equated to its corresponding
bad-pixel value, though explicit bad-pixel values, e.g.\ -32767, are banned.
The parameters VAL\_\_BADx, where x corresponds to the data type, must be
used.

\item Avoid Fortran input/output.

Use the environment facility packages MAG, FIO etc.\ wherever possible.
If it is necessary to use Fortran I/O, obtain and release logical unit numbers
using FIO\_GUNIT and FIO\_PUNIT.

\item Use symbolic names.

Status values and package constants are given symbolic names such as
PAR\_\_NULL by INCLUDE files for each package.
These symbolic names should be used on every occasion that the constant is
required.

\item RETURN is permissible when testing status.

The RETURN statement is allowed in the form:

\begin{small}
\begin{verbatim}
    IF (STATUS.NE.SAI__OK) RETURN
\end{verbatim}
\end{small}

as the first executable statement in a subroutine.
This avoids an extra, unhelpful IF clause and indentation.
Alternatively, use a GO~TO n, where line n is a CONTINUE statement immediately
preceding the END statement.

\item In generic routines use only the standard tokens.

The preprocessor for generic routines supports special tokens used by the
\mbox{ASTERIX} package (\xref{SUN/98}{sun98}{}), as well as ones for general
use.
Use only the standard tokens.

\item PAR\_\_ABORT status (!!) must abort the application.

An application must terminate if the {\it abort} response (!!) is made when a
parameter has been requested.
Note that this rule does not mean that the application has to test for the abort
status after every parameter is obtained;  the inherited status will look after
that.
What matters is the appearance to the user of the application, who should:
\begin{itemize}
\item not be re-prompted for the parameter,
\item not be prompted for further parameters, and
\item not receive additional error messages merely because the status was not
 OK.
\end{itemize}
An abort does not absolve the programmer from ensuring that the application
 closes down in an orderly fashion.
\end{itemize}

\subsection{C}

In spite of the first rule (use Fortran) in the Starlink Application
Programming Standard,
\xref{SGP/16}{sgp16}{}, there is a Starlink C Programming Standard ---
\xref{SGP/4}{sgp4}{}.
Its general style and philosophy are very similar to the Fortran standard,
though obviously many of the rules are only directly relevant to one of the
languages.

The suggestions which it contains are made with maintainability, portability
and efficiency in mind.
Several of them are made because certain code constructs will have different
effects in different C implementations.
In some cases the ambiguity is resolved by the ANSI standard.
However, it may be some time before all C implementations meet the standard,
and in any case it is preferable to avoid code which suggests more than one
possibility to the human reader.

\subsection{Fortran/C interface}

There are always problems when writing programs in a mixture of Fortran and C.
Each vendor offers his own solution, but the problem for Starlink is to
try to provide a portable solution.
SGP/5 describes two packages to help with this problem:
\begin{quote}
\begin{description}
\item [F77] --- is a set of C macros to handle the Fortran/C subroutine
 linkage.
\item [CNF] --- is a set of C functions to handle the difference between
 Fortran and C character strings.
\end{description}
\end{quote}
This software is currently available on VAX/VMS systems, Sun SPARC systems, and
DEC systems running Ultrix/RISC (typically, DECstations).

\subsection{Posix interface}

You may need to use C simply to call one of its run-time library routines,
to allocate memory for example.
A Fortran library called PSX has been provided to enable you to avoid doing
this (see \xref{SUN/121}{sun121}{}).
In many programs, it will remove the need to write any C code at all.

PSX allows you to use the functions provided by the Posix and X/Open libraries.

\section{Prologue standards}
\label{S_standprols}

Starlink has defined a number of standard prologues as a starting point for
programming.
They are important because they provide a minimal level of source-code
documentation, and they can also be processed into on-line and off-line
documentation by software tools.
These prologues are normally accessed through STARLSE (see Section
\ref{S_softools}).
However, they are also available as files which you can edit.
These can be found in \verb+STARLSE_DIR+ in files with the following names:
\begin{quote}
\begin{description}
\item [ATASK.PRO] --- A-task template.
\item [SUB.PRO] --- subroutine template.
\item [FUNC.PRO] --- function template.
\item [BLOCK.PRO] --- block data routine template.
\item [MON.PRO] --- monolith template.
\item [IFL.PRO] --- interface file template.
\end{description}
\end{quote}
As an example of a complete template, here is BLOCK.PRO (blank lines have been
removed to save space):

\begin{quote}

\begin{small}
\begin{verbatim}
      BLOCK DATA {routine_name}
*+
*  Name:
*     {routine_name}
*  Purpose:
*     {routine_purpose}
*  Language:
*     {routine_language}
*  Type of Module:
*     BLOCK DATA
*  Description:
*     {routine_description}
*  Notes:
*     {routine_notes}...
*  Side Effects:
*     {routine_side_effects}...
*  Implementation Deficiencies:
*     {routine_deficiencies}...
*  {machine}-specific features used:
*     {routine_machine_specifics}...
*  {DIY_prologue_heading}:
*     {DIY_prologue_text}
*  References:
*     {routine_references}...
*  Keywords:
*     {routine_keywords}...
*  Copyright:
*     {routine_copyright}
*  Authors:
*     {author_identifier}: {authors_name} ({affiliation})
*     {enter_new_authors_here}
*  History:
*     {date} ({author_identifier}):
*        Original version.
*     {enter_changes_here}
*  Bugs:
*     {note_any_bugs_here}
*-
*  Type Definitions:
      IMPLICIT NONE              ! No implicit typing
*  Global Constants:
      [standard_SAE_constants]
      INCLUDE '{global_constants_file}' ! [global_constants_description]
*  Global Variables:
      INCLUDE '{global_variables_file}' ! [global_variables_description]
*        {global_name}[dimensions] = {data_type} ({global_access_mode})
*           [global_variable_purpose]
*  Local Constants:
      {data_type} {constant_name} ! [constant_description]
      PARAMETER ( {constant_name} = {cons} )
*  Local Variables:
      {data_type} {name}[dimensions] ! [local_variable_description]
*  Global Data:
      DATA {data_elm} / {data_values}... /
*.
      END
\end{verbatim}
\end{small}

\end{quote}
This contains {\em everything} you might need, though many of the items will
not always be needed.

\section{Software tools}
\label{S_softools}

Software tools are programs which help you write programs.
Of the ones described below, STARLSE and SST are specific to ADAM programs.
The others can be used for any type of program.

\subsection{FORCHECK}

\vspace{-10mm}

\hfill [\xref{SUN/73}{sun73}{}]

\vspace{2mm}

A Fortran verifier and programming aid installed on the Starlink central
facilities computer (STADAT).
It checks code for conformance to the ANSI standard X3.9--1978.
However, it can also deal with non-standard code and by default accepts VAX
Fortran.
It:

\begin{small}
\begin{quote}
\begin{itemize}
\item Checks inter-module consistency by checking that the number, type, and
 size of elements in both sub-program argument lists and COMMON blocks are the
 same throughout a program.
\item Identifies recursive calls and misuse of arguments.
\item Identifies `clutter': the unused variables, COMMON blocks, INCLUDE
 files, and code fragments which accumulate in old programs, and which make
 maintenance such a time-consuming and costly task.
\item Composes cross-reference charts for constants, variables, COMMONs,
 INCLUDE files, sub-programs, and I/O.
\end{itemize}
\end{quote}
\end{small}

Automatically composed documentation of this type is an invaluable addition
to system documentation.
As often as not, it is the only reliable source of information about old
programs.

\subsection{GENERIC}

\vspace{-10mm}

\hfill [\xref{SUN/7}{sun7}{}]

\vspace{2mm}

Preprocesses a {\it generic} Fortran subroutine --- one written so as to apply
to several different data types --- into one routine per data type, and
concatenates them into a file.
This can then be compiled to produce an object module.

\subsection{LIBMAINT}

\vspace{-10mm}

\hfill [\xref{SUN/99}{sun99}{}]

\vspace{2mm}

Simplifies the maintenance of software held as modules in a source/object
library pair, and of modules in a Help library.
New libraries can be created and modules inserted, extracted, replaced,
examined, and printed with simple commands.
It ensures that corresponding modules within a source/object library pair do not
get out of step with one another, and it can optimise the disk space used by
libraries.

\subsection{LIBX}

\vspace{-10mm}

\hfill [\xref{SUN/8}{sun8}{}]

\vspace{2mm}

Contains two tools for use with libraries.
The first outputs a list of all modules in a library, and is useful for building
more elaborate utility procedures.
The second extracts preamble comments from all the modules in a Fortran
source library.

To some extent they duplicate facilities in LIBMAINT.
The latter is a very powerful system but, unavoidably, is vulnerable to
changes in the formats of the reports which the DEC Librarian utility produces.
The LIBX facilities, though limited in what they do, use only published
interfaces and should survive new VMS releases; they are also fast.

\subsection{SPAG}

\vspace{-10mm}

\hfill [\xref{SUN/63}{sun63}{}]

\vspace{2mm}

SPAG stands for `Spaghetti Unscrambler'.
It re-orders blocks of Fortran statements in such a way that the structure of
the code is improved, while remaining logically equivalent to the original
program.
The result improves the readability and maintainability of badly-written
Fortran programs.
On Starlink, it may be used to convert unstructured Fortran 77 into the
structured and indented VAX Fortran required by the Starlink Programming
Standard.
It can also be used to update Fortran 66 code.
It is marketed by Polyhedron Software and is available only on the Starlink
central facilities machine, STADAT.

\subsection{SST}

\vspace{-10mm}

\hfill [\xref{SUN/110}{sun110}{}]

\vspace{2mm}

The Simple Software Tools package was described in Chapter \ref{C_applic}.
However, its importance can best be appreciated when programming.
In particular, three of the tools --- PROHLP, PROLAT, and PROPAK --- process
prologues in useful ways:

\begin{small}
\begin{quote}
\begin{description}

\item [PROHLP] --- converts prologues into on-line Help text.
Suppose you have source code in file PROG.FOR and you are building up the Help
library MYLIB.HLB, then:

\begin{verbatim}
    ICL> PROHLP PROG.FOR PROG.HLP
    ICL> $ LIBRARY/HELP MYLIB.HLB PROG.HLP
    ICL> $ DELETE PROG.HLP;*
\end{verbatim}

will add more material.

\item [PROLAT] --- converts prologues to \LaTeX\/ form, which can be used as
either a stand-alone document, or incorporated into another document.

\item [PROPAK] --- converts prologues from the code of a subroutine package
into a form that STARLSE can use.
STARLSE can be `taught' about a subroutine library, so that simply typing
the name of a routine (or just the first few characters) followed by
ctrl/E will expand it into either a list of routines (if the string entered
is not unique), or into a call to that routine ready for its arguments to be
filled in.
This removes a common source of error.

\end{description}
\end{quote}
\end{small}

\subsection{STARLSE}

\vspace{-10mm}

\hfill [\xref{SUN/105}{sun105}{}]

\vspace{2mm}

This is a `Starlink Sensitive' editor based on the \mbox{VAX} Language
Sensitive Editor \mbox{LSE}.
It helps you write portable Fortran~77 software in a standard Starlink style.

If you normally use a standard screen editor (like EDT), you will probably find
that using an \mbox{LSE}-based editor for the first time will slow you down
considerably.
All those new keys to remember!
However, once you get used to it, you will start to realise how much time
you used to spend doing simple things like moving the cursor, formatting
prologues, indenting lines of code and going to fetch essential
documentation --- all things which \mbox{STARLSE} can do far more
efficiently.

One of the strengths of \mbox{LSE}-based editing is that it allows
you to pick and choose --- to use the features that you personally find
time-saving, while still being able to type directly over anything which you
find too fussy.
In this respect, \mbox{STARLSE} can be regarded as a sort of interactive
`Manual of Style' which provides guidelines on layout and programming
standards when you need reminding of them, but lets you work unhindered once
you know what you are doing.

Perhaps the most important component of \mbox{STARLSE} is a version of the
Fortran~77 programming language called \mbox{STARLINK\_FORTRAN}, which
defines the style of programming that the editor supports.
This language is used by default for files of type \mbox{.FOR} and
\mbox{.GEN}.

The language is based on the Starlink Application Programming Standard and
contains only a small number of approved extensions to Fortran~77.
It is therefore much simpler and easier to use than the \mbox{VAX}~Fortran
language which comes with `native' \mbox{LSE} (in fact, nearly 80\% of
\mbox{VAX}~Fortran consists of extensions to the Fortran~77 standard!).
In \mbox{STARLINK\_FORTRAN}, the number of available options and ambiguous
abbreviations is greatly reduced, and most of the common language constructs
can be produced simply by typing a short token, like `DO' or `IF', followed
by \mbox{ctrl-E}.

The most important features of the language are:

\begin{small}
\begin{quote}
\begin{itemize}
\item Templates and Prologues.
\item Subroutine definitions.
\item On-line Help.
\item Alias definitions.
\item ADAM programming constructs.
\item Symbolic constants, error codes, and Include files.
\item Enumerated type codes.
\item Tokens and Menus.
\end{itemize}
\end{quote}
\end{small}

For further information, refer to
\xref{SUN/105}{sun105}{} and the DEC LSE User Guide and
Reference Manual.

\subsection{TOOLPACK}

\vspace{-10mm}

\hfill [SUN/75]

\vspace{2mm}

Toolpack/1 (release 2) is a suite of software tools designed to support the
Fortran programmer.
In this context, a `software tool' is a utility program to assist in the
various phases of constructing, analysing, testing, adapting, or maintaining
a body of Fortran software.
Typically, the input to such a tool is your Fortran source code.
The tool processes this and produces output that may have one or both of the
following forms:

\begin{small}
\begin{quote}
\begin{itemize}
\item A report that gives an analysis of the input program, e.g.\ a summary of
the types of statements used; this type of tool is called a static analyser.
\item A modified version of the input program; in this case, the tool is
called a transformer.
An example is a formatter which improves the appearance of the code.
\end{itemize}
\end{quote}
\end{small}

In some cases the input may be test data, documentation, or a report generated
by a previously applied tool.
Tools that assist directly in preparing documents are usually called
documentation generation aids.
These and other tools serving utility functions all have an important role
to play and so, even if they do not process a program directly, they are
still regarded as programming aids.

Further examples of the software tools provided include:

\begin{small}
\begin{quote}
\begin{itemize}
\item A text editor with Fortran 77 oriented features.
\item A transforming tool that standardises the declarative part of a Fortran
program.
\item An instrumenter that modifies the program by inserting monitoring
and other control statements.
The instrumented program is then compiled and executed, and data is gathered
that is used to generate reports.
Execution of an instrumented program is an example of dynamic analysis.
\end{itemize}
\end{quote}
\end{small}

If you want to know more about TOOLPACK, read the `Introductory Guide'
available from your Site Manager.
