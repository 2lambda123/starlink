\part{INTRODUCTION}

\chapter{Introduction}
\label{C_intro}

\section{Who is this document for?}
\label{S_who}

This Guide is for anyone interested in Starlink software.
Its purpose is to provide people with a single document that describes the
ADAM software environment in sufficient detail for them to understand what it
is, what it can do, and where to find more information.
It is only for people doing data analysis.
Features of ADAM which are particularly relevant to data acquisition are not
covered.

\section{What is ADAM?}
\label{S_what}

This question is often asked, and various equally valid answers are possible.
This Guide is based on the view that ADAM is about {\em astronomical
applications software} written in a particular way.
Specifically, it is about applications written using the {\em ADAM subroutine
libraries}.
It has usually been found that when a collection of application programs has
been developed, the need is perceived for a {\em command language} to invoke
them and to offer useful features over and above those provided by the
operating system command language.
These, then, are the three major components of ADAM: {\bf Applications,
Subroutine libraries, Command language(s)}\@.
However, there are other aspects to ADAM.

Firstly, it is about a certain style in constructing applications, sometimes
called a software {\em architecture}, and about coding {\em standards}.
ADAM also stands for {\em professional} software of high quality.
It means that attention is given to {\em supportability}, {\em
portability}, and {\em performance}.

Secondly, it is concerned with a consistent way of handling the storage of
large quantities of {\em astronomical data} from all possible sources.

Many ADAM libraries can be used in isolation to write `non-ADAM' applications.
This practice is encouraged as it makes it more likely that such applications
will fit in better with existing ADAM applications.

It must be emphasized that this document refers to the VMS version of the
software being discussed.
There will be minor differences when ADAM is ported to other systems.
Subsequent versions of this document will specify those differences.

\section{The structure of this document}
\label{S_strucdoc}

Part I presents the fundamental information you need to know before you can
use ADAM successfully.
It also give you a guided tour in which ADAM is used to process simple data
in simple ways.
The idea is to give you a feel for what the system is like, without drowning
in details.

Part II begins by surveying the applications software that is available for use.
It then describes in detail the main language that is used to run ADAM programs,
namely ICL.
It ends by introducing the data system (HDS/NDF) that is at the heart of ADAM.

Part III shows how to use ADAM subroutine libraries to create new applications.
It begins by describing some simple ADAM programs, and how to compile, link,
and run them.
ADAM programs are associated with something called an Interface File which
describes the parameters which control the program --- this is also described.
Another important consideration for ADAM programmers are the standards and
conventions which are recommended when writing programs, and the tools which
are available to help you do it, and a chapter is devoted to this.
Next, the ADAM libraries are surveyed, and then the individual systems are
described in more detail in separate chapters.

Part IV is a reference section containing a summary of the main command
language, ICL, and lists of the precise contents of ADAM application packages
and subroutine libraries.
The number of individual programs and subroutines is so large (about 1500 of
each) that it is helpful to have them organised in functional groups and listed
concisely in a consistent format without (in the case of subroutines) the
clutter of parameter lists.
This should enable you to understand in detail just what the packages and
libraries can do.
When you start to use the subroutines in your programs, you will need their
associated documentation in order to establish the required parameters.

\chapter{Getting Started}
\label{C_getstart}

This chapter gets you started with ADAM.
If possible, you should try out the examples on your terminal so you can
verify that what is claimed to happen does happen, and you can get a feel for
the ADAM system.
Any terminal will do --- you don't need image displays or graphics devices.

If you are totally unfamiliar with computers you should read
the introductory guide to your local computer system (for example, {\em
Introduction to VMS}~), or better still get the Starlink site manager or a
local user to show you.
In the rest of this manual you are assumed to be familiar with basic VMS
terms and concepts.

\section{Preliminaries}
\label{S_prelim}

Before you switch on your terminal, let us explain how to interpret the
examples.
The basic principle we use is to show lines as they actually appear on your
terminal screen.
When the computer wants you to type something it normally displays a prompt.
For example, the VAX VMS operating system displays the prompt:

\begin{small}
\begin{verbatim}
    $
\end{verbatim}
\end{small}

(We indent lines which appear on your terminal screen.)
You respond to this prompt by typing something that VMS will understand.
For example:

\begin{small}
\begin{verbatim}
    $ DIR
\end{verbatim}
\end{small}

will cause the contents of your default directory to be displayed on the
screen.
What you actually type are the characters `DIR', followed by the `carriage
return' key\footnote{Sorry, but there are still people around who complain
that you never told them this.}.
We will use the word `enter' to mean `type and then press carriage return'.
A displayed line will consist partly of characters output by the
computer, and partly of characters typed in by you.
We do not attempt to differentiate between the two by typographical conventions;
you will find it obvious which is which when you try out the examples.
Sometimes the only key you need to press is the `carriage return' key.
Usually, this means that you want to accept a default value offered by
the parameter system.
For example, you might come across a line like:

\begin{small}
\begin{verbatim}
    XDIM - x dimension of output array /64/ >
\end{verbatim}
\end{small}

This is output by the computer, and the last character `$>$' is an invitation
for you to type something.
If the line appears like that without any further comment, it means that
you respond by pressing `carriage return'.
Finally, you will find that in most cases you can type input characters in
upper or lower case --- it doesn't matter which; they are interpreted as
being the same.

Sometimes you need to press a key while holding down the `CTRL' key.
This is the way you send `control codes' to the computer.
This is indicated typographically by the convention `ctrl/x', where `x' is
the other key you press.
For example, `ctrl/Z' means `press the Z key while holding down the CTRL key'.

\section{Quotas}
\label{S_quotas}

You must have sufficient VMS process quotas for running ADAM.
Typical quotas required are as follows:
\begin{center}
\begin{tabular}{|lr|lr|}
\hline
CPU limit: & Infinite & Direct I/O limit: & 18 \\
Buffered I/O byte count quota: & 65400 & Buffered I/O limit: & 18 \\
Timer queue entry quota: & 10 & Open file quota: & 74 \\
Paging file quota: & 49000 & Subprocess quota: & 10 \\
Default page fault cluster: & 64 & AST quota: & 23 \\
Enqueue quota: & 100 & Shared file limit: & 0 \\
Max detached processes: & 0 & Max active jobs: & 0 \\
JTQUOTA: & 3072 & & \\
\hline
\end{tabular}
\end{center}
The paging file quota given here is large enough to allow three large
applications to be loaded simultaneously.
You can find out what your quotas are by:

\begin{small}
\begin{verbatim}
    $ SHOW PROC/QUOTA
\end{verbatim}
\end{small}

You may run out of paging file quota when using big programs or data sets.
This is usually indicated by the system failing to do what you asked it to do
and displaying some (probably incomprehensible) message on the screen.
Consult your Site Manager if you need to get your quotas increased.

\section{Starting up ADAM}
\label{S_startadam}

If you have not already executed the Starlink command procedure SSC:LOGIN.COM
(which defines symbols like ADAMSTART and ICL), do that first:

\begin{small}
\begin{verbatim}
    $ @SSC:LOGIN
\end{verbatim}
\end{small}

Since you need to do this before using any Starlink software, it is worth
putting it in your personal LOGIN.COM file.

The first thing to do when you want to use ADAM is to type in the following
command:

\begin{small}
\begin{verbatim}
    $ ADAMSTART
    *****************************************
    * ADAM Version 2.0-1 has been installed *
    *  Type ADAM_CHANGES for information    *
    *****************************************
      ADAM version 2.0-1 available
    $
\end{verbatim}
\end{small}

(If this command causes problems, then ADAM hasn't been installed properly at
your site --- see your Site Manager.
You may, of course, find that you are using a different version of ADAM, so you
may get a slightly different response when you try it.)
This command sets up various logical names and commands needed to use ADAM.
Also, if you haven't executed this command before, it will create a subdirectory
called ADAM in your login directory and give it the logical name
ADAM\_USER\footnote{
ADAMSTART will only create the [.ADAM] directory and define the {\em job}
logical name ADAM\-\_USER if ADAM\-\_USER is not already defined as a
{\em job} logical name.
If ADAM\_USER is so defined, the assigned directory is assumed to exist.}.
You can see what's in it by typing:

\begin{small}
\begin{verbatim}
    $ DIR ADAM_USER
\end{verbatim}
\end{small}

Notice that it contains a file called GLOBAL.SDF; this will hold global
parameter values.
(In general, the behaviour of programs and the files they read and write can
be modified by specifying parameters, of which much more will be said later.
Global parameters are parameters which are not specific to any one application.)
The file type qualifier `.SDF' is characteristic of ADAM data files.
They are written by a component of ADAM called HDS, which stands for
`Hierarchical Data System'.
ADAM\_USER is used by the system to hold files which it creates automatically.
It should be cleared occasionally by using the VMS PURGE and DELETE commands
in the usual way.
Provided that no parameter values need retaining between sessions, everything
can be deleted at the end of a session, but ADAMSTART must be re-run to
initialize things again before ICL is used.

\section{Choice of command language}
\label{S_comlang}

As an ADAM user you have a choice of at least three ways of running ADAM
programs:
\begin{itemize}
\item directly from DCL (the `VMS' command language)
\item from ICL (the `ADAM' command language)
\item from SMS.
\end{itemize}
ICL is recommended for most purposes because it gives flexibility and good
performance when you are using many commands.
It also offers you an easy, yet efficient, way of writing simple procedures,
and a comprehensive scientific calculator (with the ability to feed the results
directly into programs).
It also allows you to run multiple applications at the same time.
Most of this Guide will assume that ICL is being used.

DCL gives fewer facilities --- particularly in terms of command line
specification --- but in some circumstances it can be quicker.
Naturally, it also gives immediate access to all DCL facilities.
For this reason it is often quickest to test a new program under DCL; the
normal switching between editor, compiler and program execution could mean
frequent overheads in either stopping and starting ICL or having to execute
DCL commands from ICL.

SMS (`Simple Menu System') is primarily used for data-acquisition, though
there is no reason why it should not be used for data-analysis.
Its main advantage is that it requires little typing, and gives maximum
efficiency when a few programs are being run which have lots of
parameters but only a few change for each run.
Its main drawbacks are that it is quite difficult to set up the description of
what is to appear in the various menus and how they are to look, and that this
has not been done for the normal Starlink application packages.

SMS is not considered further in this guide.
If you wish to investigate it, read chapter 11 of the ICL User's Guide
(\xref{SG/5}{sg5}{}).

\section{Trying out the command languages}
\label{S_trycl}

The program TRACE, used below, looks at the contents of a data file of the
type handled by the Hierarchical Data System (HDS).

\subsection{ICL}

You start up the ICL command language by typing:

\begin{small}
\begin{verbatim}
   $ ICL
\end{verbatim}
\end{small}

This loads the ICL program which will interpret the commands you type.
After it has warmed up (a few seconds), it displays some introductory
information:

\begin{small}
\begin{verbatim}
   Interactive Command Language   ---   Version 1.5-7

      - Type HELP package_name for help on specific Starlink packages
      -   or HELP PACKAGES for a list of all Starlink packages
      - Type HELP [command] for help on ICL and its commands

   ICL>
\end{verbatim}
\end{small}

You may find that the system responds differently when {\em you} try it
because it may have been updated since we wrote this.
Anyway, it's all self-explanatory.
The last line is the prompt that ICL gives when it wants you to enter something.
Like the `\$' prompt, the `\verb+ICL>+' prompt is output by the computer; you do
not type it in yourself.

You can find out what packages are available at your site by entering:

\begin{small}
\begin{verbatim}
    ICL> HELP PACKAGES

    PACKAGES

        The following ADAM applications packages are available from Starlink:

        Standard Packages:
         CONVERT   -  Data format conversion.
         FIGARO    -  General data-reduction.
         KAPPA     -  Image processing.
         SPECDRE   -  Spectroscopy Data Reduction
         SST       -  Simple Software Tools
         TSP       -  Time series and polarimetry data analysis.

        Option Packages:
         ASTERIX   -  X-ray data analysis.
         CCDPACK   -  CCD data reduction
         DAOPHOT   -  Stellar photometry.
         PHOTOM    -  Aperture photometry.
         PISA      -  Position, Intensity and Shape Analysis
         SCAR      -  Catalogue access and reporting.

        Additional information available:

        ASTERIX    CCDPACK   CONVERT    DAOPHOT    FIGARO     KAPPA
        MISCELLANEOUS        PHOTOM     PISA       SCAR       SPECDRE   SST
        TSP

    PACKAGES Subtopic?
\end{verbatim}
\end{small}

Now you can enter the name of one of the packages to get information on it.
Use it exactly like the VMS HELP system.
Keep pressing `carriage return' to get back to the `ICL$>$' prompt.

One of the things you can do with ICL is get answers to calculations; for
example, if you input:

\begin{small}
\begin{verbatim}
    ICL> X=SQRT(2)
    ICL> PRINT Result=(X)
\end{verbatim}
\end{small}

ICL will print the answer:

\begin{small}
\begin{verbatim}
    Result= 1.414214
    ICL>
\end{verbatim}
\end{small}

You can exit from ICL at any time by entering the command:

\begin{small}
\begin{verbatim}
    ICL> EXIT
    $
\end{verbatim}
\end{small}

The `\$' prompt indicates that you are now talking to DCL.
To use ICL again you will have to type:

\begin{small}
\begin{verbatim}
    $ ICL
\end{verbatim}
\end{small}

again, and this will generate the same introductory lines as before.

When you have an ICL$>$ prompt, simply type:

\begin{small}
\begin{verbatim}
   ICL> TRACE
\end{verbatim}
\end{small}

followed by $<$CR$>$ when you get the prompt:

\begin{small}
\begin{verbatim}
   OBJECT --- Object to be examined /@ADAM_USER:GLOBAL/ >
\end{verbatim}
\end{small}

This will show you the structure and contents of the ADAM\_USER:GLOBAL.SDF
file.

Once again, to leave ICL and return to DCL simply type:

\begin{small}
\begin{verbatim}
   ICL> EXIT
   $
\end{verbatim}
\end{small}


The above sequence of operations is encapsulated in the single command
\verb+ADAM+.
Therefore, the easiest way of running \verb+TRACE+ is to type:

\begin{small}
\begin{verbatim}
    $ ADAM TRACE
\end{verbatim}
\end{small}

which leaves you in ICL and in a position to try out any of the other ICL
commands.

\subsection{DCL}

When using DCL, the normal way to run program TRACE is simply to type:

\begin{small}
\begin{verbatim}
   $ TRACE
\end{verbatim}
\end{small}

and then press $<$CR$>$ when you receive the prompt:

\begin{small}
\begin{verbatim}
   OBJECT --- Object to be examined /@ADAM_USER:GLOBAL/ >
\end{verbatim}
\end{small}

Once again, the contents and structure of the file will be displayed, only this
time you will be left in DCL with a `\$' prompt.

\chapter {A Guided Tour}
\label{C_tour}

This Chapter contains simple illustrations of how to use ADAM applications from
the ICL command language.

One of the best ways of learning how to use ADAM is to play with one of the
applications packages.
There are many of these and they are described in Chapter \ref{C_applic}.
The KAPPA package is used below because it conforms to ADAM conventions and
has a lot of useful commands for processing data.

Remember, before you attempt to use ADAM applications from ICL you must
initialise ADAM logical names:

\begin{small}
\begin{verbatim}
    $ ADAMSTART
\end{verbatim}
\end{small}

and start up the ICL language interpreter:

\begin{small}
\begin{verbatim}
    $ ICL
\end{verbatim}
\end{small}


\section {Starting KAPPA}
\label{S_startkap}

KAPPA is made available from within ICL by typing:

\begin{small}
\begin{verbatim}
    ICL> KAPPA
\end{verbatim}
\end{small}

This loads KAPPA and defines the commands required to run its programs.
It also generates the following output:

\begin{small}
\begin{verbatim}
     Help key KAPPA redefined

     --     Initialised for KAPPA     --
     --    Version 0.8, 1991 August   --

     Type HELP KAPPA or KAPHELP for KAPPA help

    ICL>
\end{verbatim}
\end{small}

Don't worry about these messages for the moment.
If you now enter:

\begin{small}
\begin{verbatim}
    ICL> HELP KAPPA
\end{verbatim}
\end{small}

you will get an
introduction to the KAPPA Help system:

\begin{small}
\begin{verbatim}
    HELP

       This is the KAPPA online help system.  It invokes the VMS HELP Facility
       to display information about a KAPPA command or topic.  If you need
       assistance using this help library, enter "Using_help" in response to
       the "Topic?" prompt.  If you need more information about getting help
       about KAPPA from the ICL level, then enter "Command-line_help".
\end{verbatim}
\end{small}

followed by the prompt:

\begin{small}
\begin{verbatim}
    Topic?
\end{verbatim}
\end{small}

This is an invitation for you to ask for further information.
To find out what subjects are available you simply enter a `?'.
You may find the following topics particularly useful :
\begin{quote}
\begin{description}
\item [Classified\_commands] --- displays a list of subject areas as subtopics.
Each subtopic lists all KAPPA applications in that classification, and gives
their functions.
\item [Summary] --- gives a brief description of the function of each
application within KAPPA.
\item [Using\_Help] --- tells you how to use this particular Help system.
\item [$<$command$>$ param] --- describes all the parameters of the particular
KAPPA  {\tt <command>}.
Subtopics give further details about individual parameters, including their
command line position.
\end{description}
\end{quote}
You can either enter a subtopic, or press $<$CR$>$ to get back the `Topic?'
prompt.
Then you can enter `?' to get the list of topics displayed again, or just
press $<$CR$>$ to get back to the `ICL$>$' prompt.
You can escape from the Help system at any level by entering ctrl/Z.

Now have a look in your ADAM\_USER directory:

\begin{small}
\begin{verbatim}
    ICL> DIR ADAM_USER
    PROCERR    Procedure DIR not recognized
    ICL>
\end{verbatim}
\end{small}

The attempt to use the DIR command fails\footnote{It is intended to change
this soon.} because it is only recognized by the DCL command interpreter, and
you are talking to the ICL command interpreter.
What you need to do is to tell ICL that this is a DCL command.
You can do this by preceding the command with a `\$' character, thus:

\begin{small}
\begin{verbatim}
    ICL> $ DIR ADAM_USER
\end{verbatim}
\end{small}

ADAM now creates a subprocess in which to execute the DCL command processor.
This is the reason for the delay and the message:

\begin{small}
\begin{verbatim}
    Creating DCL subprocess
\end{verbatim}
\end{small}

This subprocess is only created once per session; any further DCL commands you
issue will be run in the DCL subprocess without any further action from you.

\section{Creating test images}
\label{S_creimag}

You can now use KAPPA to create some small images which you can then display
and manipulate.
The simplest way to create an image is to use command CREFRAME, which
you can type in lower case if you like:

\begin{small}
\begin{verbatim}
    ICL> creframe
\end{verbatim}
\end{small}

After a loading message, the first thing that happens is that you are invited
to specify a value for one of the program parameters:

\begin{small}
\begin{verbatim}
    XDIM - x dimension of output array /64/ >
\end{verbatim}
\end{small}

(If you have used CREFRAME before, you might find the `\verb+/64/+' in the
above
line comes out as something else.)
This line is typical of the way in which ADAM asks for {\em parameter values.}
The first field, in this case `XDIM', is the parameter keyword --- the name by
which the parameter is known to the user.
Then, after the `\verb+-+' delimiter, comes the second field, in this case
`\verb+x dimension of output array+'.
This is a brief description of what the parameter means.
The third field, in this case `\verb+/64/+', shows a suggested value for the
parameter.
If you just press $<$CR$>$ in response to the prompt, the suggested value will
be taken as the value you wish to specify for that parameter.
Sometimes a suggested value will not be displayed in a prompt.
In this case you {\em must} specify a value.

For the purposes of this example it is sensible to choose a small array size,
say 10$\times$10.
Thus, the suggested value of `64' is not what you want and you must specify
the value you require by entering the value `10':

\begin{small}
\begin{verbatim}
    XDIM - x dimension of output array /64/ > 10
\end{verbatim}
\end{small}

The parameter system then accepts this value as the value of the parameter XDIM.
Once you press $<$CR$>$, the system will present you with a prompt for another
parameter value:

\begin{small}
\begin{verbatim}
    YDIM - y dimension of output array /64/ >
\end{verbatim}
\end{small}

This prompt is similar to the previous one, and the suggested value is `64' as
before.
Specify the value `10' again:

\begin{small}
\begin{verbatim}
    YDIM - y dimension of output array /64/ > 10
\end{verbatim}
\end{small}

The system now presents you with the following lines:

\begin{small}
\begin{verbatim}
    GS = Gaussians, RR = Random 0 to 1, RL = Random from Min to Max,
    RA = Ramp across image, FL = Flat, BL = Blank,
    GN = Gaussian noise with standard deviation about the mean,
    RP = Poissonian noise about mean.
    TYPED - Type of data to be generated /'GS'/ >
\end{verbatim}
\end{small}

The first four indicate the codes you can enter in response to the parameter
prompt and the meanings attached to them.
The last is a prompt in the same format as the previous prompts with a
suggested value of `GS'.
In this case, the input required is a character string rather than an integer
--- you can usually tell what type of input is required from the format of
the suggested value.
For this illustration, generate an array that is simple in structure to make
it easy to follow what is going on.
Choose an image in the form of a `ramp' --- this is an array whose
pixel values rise or fall steadily from left to right or from top to bottom.
The code is `RA':

\begin{small}
\begin{verbatim}
    TYPED - Type of data to be generated /'GS'/ > ra
\end{verbatim}
\end{small}

You may wonder why the suggested value is surrounded by quotes but the supplied
value is not.
In fact, the supplied value should strictly be \verb+'RA'+, but it is
permissible to omit the quote characters as long as there is no ambiguity in
doing so\footnote{This is not the right place to become embroiled in a
discussion of what is a quite complicated subject --- see
Chapter~\ref{C_prompts}.}.

The next line to appear is:

\begin{small}
\begin{verbatim}
    LOW - Lower limit for data >
\end{verbatim}
\end{small}

This time the suggested value field (the `/.../' bit) is missing.
Whether it is present or not depends on how the programmer has used the
parameter system and whether or not he thinks a suggested value is appropriate.
Because no suggested value is given, you need to enter a value --- use
`1' for the lower limit and `10' for the upper limit:

\begin{small}
\begin{verbatim}
    LOW - Lower limit for data > 1
    HIGH - Upper limit for data > 10
\end{verbatim}
\end{small}

The next prompt is:

\begin{small}
\begin{verbatim}
    DIRN - Direction of ramping /1/ >
\end{verbatim}
\end{small}

This time you have suggested value again, but what does the number mean?
You can find out by entering `?':

\begin{small}
\begin{verbatim}
    DIRN - Direction of ramping /1/ > ?
\end{verbatim}
\end{small}

The system will explain that `1' generates pixels which increase from left to
right, `2' generates pixels which increase from right to left, `3' bottom to
top, and `4' top to bottom.
Choose the suggested value `1' by just pressing $<$CR$>$:

\begin{small}
\begin{verbatim}
    DIRN - Direction of ramping /1/ >
\end{verbatim}
\end{small}

The next prompt to appear is:

\begin{small}
\begin{verbatim}
    OUTPIC - Image for output data >
\end{verbatim}
\end{small}

Here there is no suggested value, so you must enter your own.
The response required is a name for the generated image; call it RAMP1:

\begin{small}
\begin{verbatim}
    OUTPIC - Image for output data > ramp1
    ICL>
\end{verbatim}
\end{small}

After `ramp1' is entered the program generates the required image, so there
may be a slight delay.
The `ICL$>$' prompt indicates that the CREFRAME command has completed its task
and has returned control to the command language processor.
The generated image will be called `RAMP1' and will be stored in a file called
RAMP1.SDF in your default directory --- take a look, you will find it is there.

If you now generate three more 10$\times$10 images containing different types
of ramp, you can display and process them in various ways.
Each time you want to create a new image using CREFRAME, you must enter the name
of the program as a command:

\begin{small}
\begin{verbatim}
    ICL> creframe
    XDIM - x dimension of output array /10/ >
\end{verbatim}
\end{small}

Notice that the second and all subsequent times you run an application program,
there is no delay while it is loaded; the response is almost immediate.
This is one of the main advantages of running applications from ICL.
Notice also that the suggested value has changed from `64' to `10' --- the
parameter system has remembered the value you last entered for XDIM; this is
called its `current value'.
(The current values of parameters used by KAPPA commands are stored in the file
KAPPA.SDF in your ADAM\_USER directory.)
Thus, you can just press $<$CR$>$ to accept the suggested value.
Generating the next image is going to be easy, isn't it?

\begin{small}
\begin{verbatim}
    YDIM - y dimension of output array /10/ >
    GS = Gaussians, RR = Random 0 to 1, RL = Random from Min to Max,
    RA = Ramp across image, FL = Flat, BL = Blank,
    GN = Gaussian noise with standard deviation about mean,
    RP = Poissonian noise about mean.
    TYPED - Type of data to be generated /'GS'/ >
\end{verbatim}
\end{small}

Wait a minute --- last time you specified `RA' as the value of TYPED, but the
suggested value has remained `GS' as it was originally.
In this case, the suggested value `GS' is generated by the program and the
current value is not used.
Once again, it all depends on how the parameter system has been used by the
programmer --- this subject will be considered in more detail in
Chapter~\ref{C_parsys}.
All you can do is keep a careful eye on the suggested values.
Here you must specify `RA' explicitly again in order to generate a ramp:

\begin{small}
\begin{verbatim}
    TYPED - Type of data to be generated /'GS'/ > ra
    LOW - Lower limit for data > 1
    HIGH - Upper limit for data > 10
    DIRN - Direction of ramping /1/ > 2
    OUTPIC - Image for output data > ramp2
    ICL>
\end{verbatim}
\end{small}

This time DIRN has been set to `2' and the image  called `RAMP2'.

In a similar way, you can now generate images called `RAMP3' and `RAMP4' with
DIRN set to 3 and 4 respectively.
Notice that the suggested value for DIRN is similar to that for TYPED in that
it always has the same value rather than the last used value.

\section{Examining images}
\label{S_examimag}

Now that you have generated four images, you can play with them.
In order to avoid getting involved with graphics devices at this stage, only
commands which will run on any terminal will be used.
The first thing to do is to examine their pixel values.
There are several ways of doing this with KAPPA; one way is to use the LOOK
command:

\begin{small}
\begin{verbatim}
    ICL> look
    INPIC - Image to be inspected/@ramp4/ >
\end{verbatim}
\end{small}

The suggested value for the name of the image to be looked at is
`\verb+@ramp4+'.
The `\verb+@+' character means `an object with the name of', thus
`\verb+@ramp4+' means `an object with the name of ramp4'.
In this case the system assumes that any character string you type is the name
of a data object; therefore you do not need to precede the name with an
`\verb+@+' symbol.
In the example, the system realizes that the value of INPIC should be the name
of an image and it remembers that the name of the last image referred to was
`ramp4'.
However, look at `ramp1' first:

\begin{small}
\begin{verbatim}
    INPIC - Image to be inspected/@ramp4/ > ramp1
\end{verbatim}
\end{small}

LOOK now accesses `ramp1' and displays the following information:

\begin{small}
\begin{verbatim}
    Title = KAPPA - Creframe
    Array is 10 by 10 pixels
\end{verbatim}
\end{small}

The title of the image --- `\verb+KAPPA - Creframe+' --- was generated by the
CREFRAME program, and the array size is 10$\times$10 as expected.
This is followed by the next prompt:

\begin{small}
\begin{verbatim}
    CHOICE - Peep, Examine or List /'Peep'/ >
\end{verbatim}
\end{small}

These are just different ways of listing the pixel values.
The suggested value is `Peep', so try that by typing $<$CR$>$.
You can also accept the suggested values for the next two prompts which specify
the central point of the part of the image you are going to peep at:

\begin{small}
\begin{verbatim}
    XCEN - x centre pixel index of 7x7 box /5/ >
    YCEN - y centre pixel index of 7x7 box /5/ >
\end{verbatim}
\end{small}

The `Peep' option will display the pixel values within a 7$\times$7 box centred
on pixel (XCEN, YCEN).
Notice that the program has chosen the centre of the image as the suggested box
centre.
All the required parameter values have now been entered, so the program proceeds
to display the specified box:

\begin{small}
\begin{verbatim}
          2         3         4         5         6         7         8

    8     2         3         4         5         6         7         8
    7     2         3         4         5         6         7         8
    6     2         3         4         5         6         7         8
    5     2         3         4         5         6         7         8
    4     2         3         4         5         6         7         8
    3     2         3         4         5         6         7         8
    2     2         3         4         5         6         7         8

    ANOTHER - Another inspection ? /YES/ >
\end{verbatim}
\end{small}

Pixel indices are numbered from bottom left.
In this case the pixel values are equal to their column number.
Unlike CREFRAME, you are not returned to ICL but are given a chance to peep
at another part of the image --- these decisions are made by the programmer.
As the image you are peeping at is highly predictable, do not bother to
examine it further but escape from the program by entering:

\begin{small}
\begin{verbatim}
    ANOTHER - Another inspection ? /YES/ > n
    ICL >
\end{verbatim}
\end{small}

When a program asks for a `YES' or `NO' value, you can get away with
`Y', `True', and `T' for `YES'; and `N', `False', and `F' for `NO'; regardless
of case.

You can use the LOOK command to examine the pixel values of the other images
you have created in a similar way to that shown above.
The `List' option gives you the opportunity to store the output in a file:

\begin{small}
\begin{verbatim}
    ICL> look
    INPIC - Image to be inspected/@ramp1/ >
    Title = KAPPA - Creframe
    Array is 10 by 10 pixels

    CHOICE - Peep, Examine or List /'PEEP'/ > L

    XLOW - x start pixel index of sub-array /1/ >
    YLOW - y start pixel index of sub-array /1/ >
    XSIZE - x size of sub-array /10/ >
    YSIZE - y size of sub-array /10/ >
    FILENAME - Name of listing file /@LOOKOUT.LIS/ >
    Listing to LOOKOUT.LIS

    ANOTHER - Another inspection ? /YES/ > n
    ICL>
\end{verbatim}
\end{small}

Notice in this case that the parameter prompts after that for `CHOICE' differ
from those encountered before.
This is because you have chosen the `List' option instead of the `Peep' option,
and the program needs different information before it can proceed.
Accept the suggested values for all the parameters until you come to `ANOTHER'
again.
This makes things quick and easy for you.
The program writes the pixel listing to the file LOOKOUT.LIS which is written
in your default directory.
You should, therefore, be able to display it on the screen using the DCL `TYPE'
command.

\begin{small}
\begin{verbatim}
    ICL> $ type lookout
\end{verbatim}
\end{small}

will cause the contents of file LOOKOUT.LIS to be displayed on your terminal
(in a folded form --- not shown here).

\section{ICL procedures}
\label{S_iclprocs}

So far we have used ICL to do simple things like calculate the value of a simple
formula and load and use KAPPA.
However, ICL can also be used as a programming language in which to write
procedures, just as commands of DCL can be stored in a command procedure.
This is a very convenient way to extend the range of applications available to
you.

Suppose you want to add together (pixel by pixel) the four images you have
created to produce a new composite image.
This is the sort of problem that can be solved using an ICL {\em procedure}.
ICL has a lot of functions available, one of which, `SNAME', generates
sequential names:

\begin{small}
\begin{verbatim}
    SNAME(string,n,m)
\end{verbatim}
\end{small}

produces a name which is the concatenation of the `string' with the
integer $n$.
A third parameter $m$ specifies a minimum number of digits for the
numeric part of the name. Leading zeros are inserted if necessary to produce at
least $m$ digits, thus:

\begin{small}
\begin{verbatim}
    SNAME('RUN',3,1)      has the value   RUN3
    SNAME('IPCS',17,3)    has the value   IPCS017
\end{verbatim}
\end{small}

An ICL procedure can be written using this to add a whole series of images
together:

\begin{small}
\begin{verbatim}
    ICL> PROC RAMPADD N
    RAMPADD> { Add images RAMP1 to RAMPN to form SUM
    RAMPADD>   ADD RAMP1 RAMP2 SUM TITLE='Sum of 2 images'
    RAMPADD>   LOOP FOR I=3 TO (N)
    RAMPADD>     TITLE = 'Sum of' & I:2 & ' images'
    RAMPADD>     ADD SUM (SNAME('RAMP',I,1)) SUM TITLE=(TITLE)
    RAMPADD>   END LOOP
    RAMPADD> END PROC
    ICL>
\end{verbatim}
\end{small}

The line starting \verb+PROC+ tells ICL you wish to write a procedure called
\verb+RAMPADD+ which is to have one parameter called \verb+N+.
This will use the KAPPA application \verb+ADD+ to
add successive
images together in the new image \verb+SUM+.

You can save this procedure for future use by entering:

\begin{small}
\begin{verbatim}
    ICL> save rampadd
\end{verbatim}
\end{small}

To run this procedure, use its name as a command and specify the value of the
parameter.
Thus, to add together the images RAMP1, RAMP2, RAMP3, and RAMP4 enter:

\begin{small}
\begin{verbatim}
    ICL> rampadd 4
    ICL>
\end{verbatim}
\end{small}

An ADD statement is executed three times within the procedure, and a suitable
title is generated for each image which is created.
If you use LOOK to `Peep' at the new image called `SUM', you will see that
every pixel has the value `22' as you would expect.
Note that four SUM.SDF files have been created.
You should PURGE or DELETE them.

End your ICL session by exiting from ICL in the normal way:

\begin{small}
\begin{verbatim}
    ICL> exit
    $
\end{verbatim}
\end{small}

You should now be fairly familiar with the style in which ADAM programs are
used.
More examples of how to use KAPPA are given in
\xref{SUN/95}{sun95}{}.
ICL procedures are considered in more detail in Chapter~\ref{C_iclprog}.
