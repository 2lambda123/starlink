\chapter{ICL Commands}
\label{C_commands}

\section{Commands}
\label{S_commands}

In the last chapter we introduced the {\em Command} as one of the Direct
Statements of the ICL language.
You will remember that the format was:
\begin{quote}
command\_name [parameter\_specification] \ldots
\end{quote}
There are two main types of commands:
\begin{itemize}
\item Built-in commands
\item User-defined commands
\end{itemize}
The difference between them is that built-in commands are always available
as they are an intrinsic part of the language, while user-defined commands
are only available if you have defined them yourself in some way.
The two types will be considered separately in the following two sections.

If you define commands, or use an application that defines commands, there is
a danger of ambiguity since you may give a new meaning to an existing command
name.
Just for the record, here is the search-path that ICL uses when it searches
for command definitions:
\begin{quote}
\begin{itemize}
\item User-defined: DEFSTRING, DEFUSER, DEFPROC
\item Procedures: PROC
\item Built-in commands to control ADAM tasks: ALOAD, \ldots
\item Applications: DEFINE
\item Other Built-in
\end{itemize}
\end{quote}
Don't worry about those command names, you will learn about them latter.

\section{Built-in commands}
\label{S_bic}

The built-in commands are listed in Chapter~\ref{C_rsicl}.
Here is a classification:
\begin{quote}
\begin{description}
\item [Information \& Escape] --- Chapter~\ref{C_ugicl}\\
HELP, EXIT
\item [Defining user commands] --- Chapter~\ref{C_commands}\\
DEFSTRING, DEFINE, DEFUSER, DEFPROC
\item [I/O] --- Chapter~\ref{C_commands}
\begin{description}
\item [Terminal] \hfill \\
PRINT, INPUTt
\item [Screen mode] \hfill \\
SET SCREEN, SET NOSCREEN, SET ATTRIBUTES, LOCATE, CLEAR
\item [Keyboard facilities] \hfill \\
KEY, KEYTRAP, KEYOFF
\item [Files] \hfill \\
CREATE, OPEN, APPEND, CLOSE, WRITE, READt
\end{description}
\item [Access to DCL] --- Chapter~\ref{C_commands}\\
\$, SPAWN, DEFAULT, \\
ALLOC, DEALLOC, MOUNT, DISMOUNT
\item [Parameters] --- Chapter~\ref{C_prompts}\\
SETPAR, GETPAR, \\
CREATEGLOBAL, SETGLOBAL, GETGLOBAL
\item [Procedures] --- Chapter~\ref{C_iclprog}\\
LIST, PROCS, VARS, EDIT, SET EDITOR, \\
SAVE, LOAD, DELETE, \\
SET TRACE, SET NOTRACE
\item [Errors \& Exceptions] --- Chapter~\ref{C_iclprog}\\
SIGNAL
\item [HELP system] --- Chapter~\ref{C_iclprog}\\
DEFHELP
\end{description}
\end{quote}
All these commands are described in the chapter specified.

\section{User-defined commands}
\label{S_udc}

\begin{center}
\begin{tabular}{|l|l|}
\hline
DEFSTRING & Associate command with equivalence string \\
DEFINE    & Define command to run a program \\
DEFUSER   & Define command to run a subroutine \\
DEFPROC   & Define command to run an ICL procedure \\
\hline
\end{tabular}
\end{center}

There are three types of command which you can use to define your own commands:
\begin{itemize}
\item To define commands which stand for strings --- DEFSTRING.
\item To define commands which run programs --- DEFINE, DEFUSER.
\item To define commands which run ICL procedures --- DEFPROC.
\end{itemize}
These are described briefly below:
\begin{quote}
\begin{description}

\item [DEFSTRING] --- associates a command with an equivalence string:

\begin{small}
\begin{verbatim}
    DEFSTRING  command  equivalence-string
\end{verbatim}
\end{small}

when you enter {\small\tt command}, it is as if you had entered
{\small\tt equivalence-string}.

\item [DEFINE] --- defines a command which causes a program to be executed:

\begin{small}
\begin{verbatim}
    DEFINE  command  program-name
\end{verbatim}
\end{small}

When you enter {\small\tt command}, program {\small\tt program-name} is
executed.

\item [DEFUSER] --- defines a command which causes a compiled subroutine to
be executed:

\begin{small}
\begin{verbatim}
    DEFUSER  command  image-name
\end{verbatim}
\end{small}

When you enter {\small\tt command}, the shareable image {\small\tt image-name}
is executed.

\item [DEFPROC] --- defines a command which causes an ICL procedure to be
executed:

\begin{small}
\begin{verbatim}
    DEFPROC  command  file
\end{verbatim}
\end{small}

When you enter {\small\tt command}, the ICL procedure stored in file.ICL is
compiled (first time only) and executed.

\end{description}
\end{quote}

The DEFINE command is explained further in Chapter~\ref{C_runap}, and the
definition and use of ICL procedures is explained in Chapter~\ref{C_iclprog}.

\section{Input/Output commands}
\label{S_IO}

The I/O commands fall into three classes:
\begin{itemize}
\item Terminal
\item File
\item Screen mode
\end{itemize}
These are described in the following sub-sections.

\subsection{Terminal}

\begin{center}
\begin{tabular}{|l|l|}
\hline
INPUT  & Input string from terminal \\
INPUTI & Input integers from terminal \\
INPUTL & Input logicals from terminal \\
INPUTR & Input reals from terminal \\
\hline
PRINT  & Output to terminal \\
\hline
\end{tabular}
\end{center}

\paragraph{Input:}\hfill

The commands for terminal input are INPUT, INPUTR, INPUTI, and INPUTL.
Command INPUT reads a line of text from the terminal into a string variable.
An example is:

\begin{small}
\begin{verbatim}
    ICL> INPUT Enter your name> (X)
    Enter your name>Mike
    ICL> PRINT (X)
    Mike
    ICL>
\end{verbatim}
\end{small}

The last parameter must specify a variable in which the input will be stored
and must, therefore, be in parentheses.
The earlier parameters form a prompt string.

Commands INPUTR, INPUTI, and INPUTL are used to input real, integer, and logical
values.
A single command may be used to supply values for more than one variable, so
these commands have the general form:

\begin{small}
\begin{verbatim}
    ICL> INPUTR  Prompt  (X)  (Y)  (Z)  ...
\end{verbatim}
\end{small}

For these three commands, only the first parameter is used to provide the
prompt string, so if it has spaces in it, the string must be enclosed in quotes.
For example:

\begin{small}
\begin{verbatim}
    ICL> INPUTR 'Give values of X and Y > ' (X) (Y)
    Give values of X and Y > 2.3 8.9
    ICL> PRINT (X) (Y)
    2.300000 8.900000
    ICL>
\end{verbatim}
\end{small}

INPUTL will accept values of TRUE, FALSE, YES, and NO in either upper or
lower case, as well as abbreviations.

\paragraph{Output:}\hfill

You can output to the terminal by using the PRINT command:

\begin{small}
\begin{verbatim}
    PRINT  p1  p2  ...
\end{verbatim}
\end{small}

The parameters are concatenated and printed on the terminal.
For example

\begin{small}
\begin{verbatim}
    ICL> X=2
    ICL> PRINT The square root of (X) is (SQRT(X))
    The square root of          2 is 1.414214
\end{verbatim}
\end{small}

Another way of writing to the terminal is by using an Immediate statement:

\begin{small}
\begin{verbatim}
    ICL> =1+2+3
             6
\end{verbatim}
\end{small}


\paragraph{Formatting Output:}\hfill

While Fortran regards formatting of numbers for output as part of an output
operation, ICL performs formatting using an operator (:) which produces a string
result from a numeric operand.
Thus, if I is an integer variable, the expression \verb+I:5+ has as its value
the string which is produced by converting I with a field width of 5 characters.
It is equivalent to an I5 format in Fortran.
Similarly, if X is a real variable, the expression \verb+X:10:4+ produces the
value of X formatted in a Fortran F10.4 format (i.e.\ a field width of 10
characters with 4 decimal places).
The ICL formatting is not precisely equivalent to the Fortran form because ICL
will extend the field width if a number is too large to fit in the requested
width.
For example:

\begin{small}
\begin{verbatim}
    ICL> =1.234567:5:2
     1.23
    ICL> =12.34567:5:2
    12.35
    ICL> =123.4567:5:2
    123.46
    ICL> =123456.7:5:2
    123456.70
    ICL>
\end{verbatim}
\end{small}

Integers can also be formatted in binary, octal, decimal, or hexadecimal formats
using the functions BIN, OCT, DEC, and HEX.
These have the form \verb+HEX(X,n,m)+ which would return a string of n
characters containing the number X with m significant digits.
Parameters n and m may be omitted, in which case they default to the number of
digits needed to represent a full 32 bit word.
Using these forms together with constants in various bases, ICL can be used to
perform conversions between bases.
For example:

\begin{small}
\begin{verbatim}
    ICL> =%Xffff
          65535
    ICL> =hex(65535)
     0000FFFF
    ICL> =oct(%XFF,5,5)
    00377
    ICL>
\end{verbatim}
\end{small}

\subsection{Screen mode}

\begin{center}
\begin{tabular}{|l|l|}
\hline
SET SCREEN    & Select screen mode \\
SET NOSCREEN  & Select normal mode \\
SET ATTRIBUTES & Set attributes for text written with LOCATE \\
\hline
LOCATE         & Write to screen at specified position \\
CLEAR          & Clear range of lines \\
\hline
\end{tabular}
\end{center}

Screen mode allows more control over the terminal screen than is possible
in the normal mode.
It also allows more control over the use of the keyboard.
Screen mode is implemented using the DEC screen management (SMG\$) routines
of the run time library, and will work on any terminal compatible with these
routines.

Screen mode is selected by the SET SCREEN command.
In this mode, the terminal screen is divided into an upper fixed region and a
lower scrolling region.
The size of the scrolling region may be specified by an optional parameter
to SET SCREEN.

\begin{small}
\begin{verbatim}
    ICL> SET SCREEN 10
\end{verbatim}
\end{small}

will select 10 lines of scrolling region.
The size of the scrolling region can be changed by further SET SCREEN commands.
SET NOSCREEN is used to leave screen mode and return to normal mode.

Standard terminal I/O operations work exactly as normal in the scrolling
region of the screen.
However, an additional facility is the ability to examine text which has
scrolled off the top of the scrolling region.
The Next Screen and Prev Screen keys on a VT200 (or ctrl/N, ctrl/P on other
terminals) may be used to move through the text.
Any output since screen mode was started is viewable in this way.

To write to the fixed part of the screen, the command LOCATE is used:

\begin{small}
\begin{verbatim}
    ICL> LOCATE 6 10   This text will be written starting at Row 6 Column 10
\end{verbatim}
\end{small}

The first two parameters specify the row and column at which the text will
start.
The remaining parameters form the text to be written.

The SET ATTRIBUTES command provides further control over text written with
the LOCATE command.
This command has a parameter string composed of any combination of the letters
R (Reverse Video), B (Bold), U (Underlined), F (Flashing) and D (Double Size).
These attributes apply to all LOCATE commands until the next SET ATTRIBUTES
command.
SET ATTRIBUTES with no parameter gives normal text.

The CLEAR command is used to clear all or part of the fixed region of the
screen.
For example:

\begin{small}
\begin{verbatim}
    ICL> CLEAR 6 10
\end{verbatim}
\end{small}

clears lines 6 to 10 of the screen.

\subsection{Keyboard facilities}

\begin{center}
\begin{tabular}{|l|l|}
\hline
KEY            & Define equivalence string for key \\
KEYTRAP        & Specify trapping of key in a procedure \\
KEYOFF         & Turn off trapping of key in a procedure \\
\hline
\end{tabular}
\end{center}

The KEY command may be used to define an equivalence string for any key
on the keyboard.
For example:

\begin{small}
\begin{verbatim}
    ICL> KEY  PF1  PROCS#
\end{verbatim}
\end{small}

defines the PF1 key so that it issues the PROCS command.
The \# character is used to indicate a Return character in the equivalence
string.

The name of a main keyboard key may be specified as a single character or
as an integer representing the ASCII code for the key.
Keypad and function keys are specified by names as follows:

\begin{small}
\begin{verbatim}
   Keypad keys             PF1, PF2, PF3, PF4, KP0, KP1, KP2, KP3, KP4, KP5,
                           KP6, KP7, KP8, KP9, ENTER, MINUS, COMMA, PERIOD.

   Function Keys (VT200)   F6, F7, F8, F9, F10, F11, F12, F13, F14, HELP,
                           DO, F17, F18, F19, F20.

   Editing Keypad (VT200)  FIND, INSERT_HERE, REMOVE, SELECT, PREV_SCREEN, NEXT_SCREEN.

   Cursor Keys             UP, DOWN, LEFT, RIGHT.
\end{verbatim}
\end{small}

The KEYTRAP command and the INKEY() function allow an ICL procedure to test
for keyboard input during its execution, without having to issue an INPUT
command and thus wait for input to complete.

KEYTRAP specifies the name of a key to be trapped.

KEYOFF turns off the trapping of a specified key.

INKEY() is an integer function which returns zero if no key has been pressed, or
the key value if a key has been pressed since the last call.
The key value is the ASCII value for ASCII characters, or a number between 256
and 511 for keypad and function keys.

KEYVAL(S) is a function which obtains the value from the key name.

Here is an example of the use of these commands and functions within a
procedure:

\begin{small}
\begin{verbatim}
    { Trap ENTER and LEFT and RIGHT arrow keys

    KEYTRAP ENTER
    KEYTRAP LEFT
    KEYTRAP RIGHT

    LOOP
      K = INKEY()
      IF K = KEYVAL('ENTER') THEN
      .
      ELSE IF K = KEYVAL('LEFT') THEN
      .
      ELSE IF K = KEYVAL('RIGHT') THEN
      .
      ELSE
      .
      ENDIF
    END LOOP
\end{verbatim}
\end{small}

\subsection{File}

\begin{center}
\begin{tabular}{|l|l|}
\hline
CREATE   & Create new file and open for output \\
OPEN     & Open existing file for input \\
APPEND   & Open existing file for output, append text \\
CLOSE    & Close file \\
\hline
READ     & Read line from file \\
READI    & Read integers from file \\
READL    & Read logicals from file \\
READR    & Read reals from file \\
\hline
WRITE    & Write to file \\
\hline
\end{tabular}
\end{center}
ICL can read and write text files.
In order to access such files, they must first be opened using one of the
commands CREATE, OPEN, or APPEND, for example:

\begin{small}
\begin{verbatim}
    ICL> CREATE MYFILE
\end{verbatim}
\end{small}

will create a file called {\small\tt MYFILE} and open it for output ---
{\small\tt MYFILE} is the name used within ICL for the file.
In this case, the file will appear in your default directory as
{\small\tt MYFILE.DAT}.
However, the file name may be specified explicitly by adding a second parameter
to the commands, for example:

\begin{small}
\begin{verbatim}
    ICL> OPEN  INFILE  DISK$DATA:[ABC]FOR008.DAT
\end{verbatim}
\end{small}

opens an existing file for input and the file is known internally as
{\small\tt INFILE}.

The APPEND command opens an existing file for output.
Anything written to it is appended to the existing contents.

A line of text is written to a file with the WRITE command.
WRITE is similar to PRINT, the only difference being that its first parameter
specifies the internal name of the file to which the data will be written.

Text is read from files with the commands READ, READR, READI, and READL.
These are analogous to the INPUT commands for terminal input.
The first parameter specifies the internal name of the file.
READ reads a line of text into a single string variable.
The other commands read one or more real, integer, or logical values.

When a file is no longer required it may be closed using the CLOSE command
which has a single parameter: the internal name of the file.

The following example defines a procedure which uses these commands to read a
file containing three real numbers in free format and output the same numbers
as a formatted table.

\begin{small}
\begin{verbatim}
    PROC REFORMAT
    { Open input file and create output file
      OPEN INFILE
      CREATE OUTFILE
    { Copy lines from input to output
      LOOP
        READR INFILE (R1) (R2) (R3)
        WRITE OUTFILE (R1:10:2) (R2:10:2) (R3:10:2)
      END LOOP
    END PROC
\end{verbatim}
\end{small}

Note that no specific test for completion of the loop is included.
When an end-of-file condition is detected on the input file, the procedure will
exit and return to the {\small\tt ICL>} prompt with an appropriate
message\footnote{A tidier exit can be arranged by using an exception handler
for the EOF exception, see Chapter~\ref{C_iclprog}.}.

\section{DCL commands and VMS processes}
\label{S_dclvms}

\subsection{Executing DCL commands}

\begin{center}
\begin{tabular}{|l|l|}
\hline
\$       & Execute a DCL command in \$'s subprocess \\
SPAWN    & Execute a DCL command in new subprocess \\
\hline
\end{tabular}
\end{center}

When using ICL, it is frequently useful to be able to access features of
Digital's command language DCL.
Typical operations we may want to do include listing directories, copying
files, allocating tape drives, and mounting tapes.
The ICL command `\$' allows any DCL command to be issued from inside ICL.
Its form is simply:

\begin{small}
\begin{verbatim}
    ICL> $ dcl_command
\end{verbatim}
\end{small}

where \verb+dcl_command+ is any command we could issue from the DCL `\$' prompt.
For example:

\begin{small}
\begin{verbatim}
    ICL> $ COPY *.SDF DATADIR:*.SDF
    ICL> $ RUN MYPROGRAM
\end{verbatim}
\end{small}

There is one restriction --- you must use a complete DCL command.
You can't, for example, just type {\small\tt \$ COPY} and expect DCL to prompt
you for the two file specifications.
Apart from this, any command acceptable to DCL can be issued in this way.

There is a way around the restriction on prompts mentioned above.
This is to use the command `SPAWN' rather than the command `DCL'.
For example:

\begin{small}
\begin{verbatim}
    ICL> SPAWN COPY
    _From: *.SDF
    _To: DATADIR:*.SDF
\end{verbatim}
\end{small}

in which case you get the `\_From:' and `\_To:' prompts, just as you do in
normal DCL.
The disadvantage of SPAWN is that it is slower.
This is because SPAWN creates a new subprocess to run each command, whereas
DCL creates a permanent subprocess in which all commands are run.

SPAWN has another use --- by just typing SPAWN you can get a DCL `\$' prompt
from which a series of DCL commands can be executed.
LOGOUT is used to return control to ICL.

\subsection{Processes and subprocesses}

In VMS, a program runs in a {\em process}.
If that program wishes to run another program or do something else, it must
first create a {\em subprocess}.
ICL is such a program and it runs application programs.
The applications thus run in subprocesses, and so do DCL commands issued from
ICL.
This is the heart of many of the sometimes unexpected properties of ICL.

VMS is not particularly efficient at starting up processes, so loading a
program takes time.
ADAM deals with this through a structure called a monolith (discussed in
Chapter~\ref{C_runap}).

The worst problems with subprocesses arise when something goes wrong.
Uninhibited use of such things as ctrl/C to kill an application can have serious
consequences for ICL, resulting in a loss of context and the consequent waste
of time getting back to where you started.
Ctrl/C returns you to the ICL prompt, while ctrl/Y takes you right out of ICL
and back into DCL.
A better way to get out of a program is to enter the {\em abort} response `!!'
when prompted for a parameter (see Chapter \ref{C_prompts}.)

Note that ctrl/C will only break out of a program and return you to ICL if the
program is run as a subprocess of ICL.
This is usually the case, but will not be if the command being executed was
defined using DEFUSER (as in ASTERIX for example).

\subsection{Changing your default directory}

\begin{center}
\begin{tabular}{|l|l|}
\hline
DEFAULT & Set default directory \\
\hline
\end{tabular}
\end{center}

You can change your default directory by using the DCL command `SET DEFAULT'.
This will change the default directory of the DCL subprocess, but not of the
process running ICL.
Thus, an additional ICL command DEFAULT has been provided.
This changes the default directory of both the process running ICL and the DCL
subprocess (if one exists).
The format for specifying the directory is exactly the same as that accepted
by the equivalent DCL command.

\subsection{Managing tape drives}

\begin{center}
\begin{tabular}{|l|l|}
\hline
ALLOC    & Allocate a device\\
DEALLOC  & Deallocate a device\\
MOUNT    & Mount a device\\
DISMOUNT & Dismount a device\\
\hline
\end{tabular}
\end{center}

Similar problems with processes occur when allocating and mounting tape drives.
The DCL command `ALLOCATE' will allocate the device to the DCL subprocess.
This {\em may} be what you want; for example, if you are going to use another
DCL command (such as `BACKUP') to read or write the tape.
However, if a tape is to be processed using a FIGARO command, say, it must be
allocated to the process running ICL.
A set of commands has been provided for this purpose.

ALLOC allocates a device and may specify a generic name; the name of the device
actually allocated will be returned in the optional second parameter.
For example:

\begin{small}
\begin{verbatim}
    ICL> ALLOC MT
    _MTA0: Allocated
    ICL> ALLOC MT (DEVICE)
    _MTA1: Allocated
    ICL> =DEVICE
    _MTA1:
\end{verbatim}
\end{small}

DEALLOC deallocates a device.

MOUNT performs a MOUNT/FOREIGN at the tape's initialised density.
It does not provide the many qualifiers of the DCL command.
There are several additional optional parameters for some of these commands.

DISMOUNT has an optional parameter which is used to specify that the tape be
dismounted without unloading:

\begin{small}
\begin{verbatim}
    ICL> DISMOU MTA1 NOUNLOAD
\end{verbatim}
\end{small}
