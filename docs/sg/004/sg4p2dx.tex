\chapter{Programming in ICL}
\label{C_iclprog}

\section{Control statements}
\label{S_cs}

These statements provide a means of defining ICL procedures and of controlling
the flow of execution within them.
There are four of them:
\begin{quote}
\begin{description}
\item [PROC] --- Defines a procedure.
\item [IF] --- Provides a decision-making structure.
\item [LOOP] --- Provides a looping structure.
\item [EXCEPTION] --- Provides an error-handling structure.
\end{description}
\end{quote}
Unlike direct statements, they can only be used in a procedure and are not
accepted in direct mode.

\subsection{IF}

The IF statement is essentially the same as the block IF of Fortran.
It has the following general form:

\begin{small}
\begin{verbatim}
    IF expression
        statements
    ELSE IF expression
        statements
    ELSE IF expression
        statements
    ...
    ELSE
        statements
    END IF
\end{verbatim}
\end{small}

The expressions (which must give logical values) are evaluated in turn until
one is found to be true; the following statements are then executed.
If none of the expressions are true, the statements following ELSE are executed.

Every IF statement must begin with IF and end with END IF (or ENDIF).
The ELSE IF (or ELSEIF) and ELSE clauses are optional, so the simplest IF
statement would have the form:

\begin{small}
\begin{verbatim}
    IF expression
        statements
    ENDIF
\end{verbatim}
\end{small}

The following example procedure illustrates the use of the IF statement, and
shows how they may be nested inside each other:

\begin{small}
\begin{verbatim}
    PROC QUADRATIC A,B,C
    { A Procedure to find the roots of the quadratic equation
    { A*X**2 + B*X + C = 0
       IF A=0 AND B=0
          PRINT The equation is degenerate
       ELSE IF A=0
          PRINT Single Root is (-C/B)
       ELSE IF C=0
          PRINT The roots are (-B/A) and 0
       ELSE
          RE = -B/(2*A)
          DISCRIMINANT = B*B - 4*A*C
          IM = SQRT(ABS(DISCRIMINANT)) / (2*A)
          IF DISCRIMINANT >= 0
             PRINT The Roots are (RE + IM) and (RE - IM)
          ELSE
             PRINT The Roots are complex
             PRINT (RE) +I* (IM) and
             PRINT (RE) -I* (IM)
          ENDIF
       ENDIF
    END PROC
\end{verbatim}
\end{small}


\subsection{LOOP}

The LOOP statement is used to execute repeatedly a group of statements.
It has three different forms:
\begin{itemize}
\item LOOP
\item LOOP FOR
\item LOOP WHILE
\end{itemize}

\paragraph{LOOP:}\hfill

This is the simplest form of looping structure:

\begin{small}
\begin{verbatim}
    LOOP
        statements
    END LOOP
\end{verbatim}
\end{small}

This form sets up an infinite loop.
Fortunately, an additional statement (BREAK) can be used to terminate the loop;
BREAK would normally appear inside an IF statement within the loop.
For example:

\begin{small}
\begin{verbatim}
    PROC COUNT
    { A procedure to print the numbers from 1 to 10
       I = 1
       LOOP
          PRINT (I)
          I = I+1
          IF I>10
             BREAK
          ENDIF
       ENDLOOP
    ENDPROC
\end{verbatim}
\end{small}

\paragraph{LOOP FOR:}\hfill

As it is frequently required to loop over a sequential range of numbers, a
special form of the LOOP statement is provided for this purpose.
It has the following form:

\begin{small}
\begin{verbatim}
    LOOP FOR variable = expression1 TO expression2 [STEP expression3]
        statements
    END LOOP
\end{verbatim}
\end{small}

This form is essentially equivalent to the DO loop in Fortran.
The expressions specifying the range of values for the control variable are
rounded to the nearest integer so that the variable always has an integer value.
Using this form of the LOOP statement you can simplify the previous example as
follows:

\begin{small}
\begin{verbatim}
    PROC COUNT
    { A procedure to print the numbers from 1 to 10
       LOOP FOR I = 1 TO 10
          PRINT (I)
       ENDLOOP
    ENDPROC
\end{verbatim}
\end{small}

Note that there is an optional STEP clause in the LOOP FOR statement.
If this is not specified, a STEP of 1 is assumed.
The STEP clause can be used to specify a different value.
A step of $-1$ must be specified to get a loop which counts down from a high
value to a lower value.
For example:

\begin{small}
\begin{verbatim}
    LOOP FOR I = 10 TO 1 STEP -1
\end{verbatim}
\end{small}

will count down from 10 to 1.

\paragraph{LOOP WHILE:}\hfill

The third form of LOOP statement specifies loops which terminate on a condition.
It has the form:

\begin{small}
\begin{verbatim}
    LOOP WHILE expression
        statements
    END LOOP
\end{verbatim}
\end{small}

The expression is evaluated each time round the loop and if it has the logical
value TRUE, the statements which form the body of the loop are executed.
If it has the value FALSE, execution continues with the statement following
END LOOP.
Using this form you can write yet another version of the COUNT procedure:

\begin{small}
\begin{verbatim}
    PROC COUNT
    { A procedure to print the numbers from 1 to 10
       I = 1
       LOOP WHILE I<=10
          PRINT (I)
          I = I+1
       ENDLOOP
    ENDPROC
\end{verbatim}
\end{small}

In the above case, the LOOP WHILE form is more complicated than the LOOP FOR
form.
However, LOOP WHILE can be used to express more general forms of loop
where the termination condition is something derived inside the loop.
An example is a program which prompts you for an answer to a question
and has to keep repeating the prompt until a valid answer is received:

\begin{small}
\begin{verbatim}
    FINISHED = FALSE
    LOOP WHILE NOT FINISHED
       INPUT Enter YES or NO:  (ANSWER)
       FINISHED = ANSWER = 'YES' OR ANSWER = 'NO'
    END LOOP
\end{verbatim}
\end{small}

\section{Procedures and command files}
\label{S_pcf}

\begin{center}
\begin{tabular}{|l|l|}
\hline
LIST        & List a procedure \\
\hline
EDIT        & Edit a procedure \\
SET EDITOR  & Change editor used by EDIT \\
\hline
SAVE        & Save a procedure \\
LOAD        & Accept commands from a saved procedure \\
DELETE      & Delete a procedure \\
\hline
PROCS       & List procedure names \\
VARS        & List procedure variables \\
\hline
SET (NO)TRACE   & Switch tracing of procedures on/off \\
\hline
\end{tabular}
\end{center}

An ICL procedure is like a subroutine in Fortran.
It allows you to write a sequence of ICL statements which can later be run with
a single command.
The procedure may have parameters which are used to pass values to the procedure
and return values from it.

\subsection{Defining procedures --- PROC}

To define a procedure, type a PROC command which specifies the name of the
procedure and the names of its parameters.
ICL then returns a new prompt using the name of the procedure (rather than
\verb+ICL>+) to show that you are in the procedure entry phase of procedure mode.
The statements that make up the procedure are then entered, followed by an
\verb+END PROC+ or \verb+ENDPROC+ to mark the end of the procedure.
For example:

\begin{small}
\begin{verbatim}
    ICL> PROC SQUARE_ROOT X
    SQUARE_ROOT> { An ICL procedure to print the square root of a number
    SQUARE_ROOT> PRINT The Square Root of (X) is (SQRT(X))
    SQUARE_ROOT> END PROC
    ICL>
\end{verbatim}
\end{small}


\subsection{Running procedures}

To run the procedure you have entered, use its name as an ICL command and add
any parameter values required:

\begin{small}
\begin{verbatim}
    ICL> SQUARE_ROOT (2)
    The Square Root of         2 is 1.414214
\end{verbatim}
\end{small}

\subsection{Listing procedures --- LIST}

The LIST command lists a procedure on the terminal.
Just type `LIST', followed by the name of the procedure you want to list:

\begin{small}
\begin{verbatim}
    ICL> LIST SQUARE_ROOT

        PROC SQUARE_ROOT X
        { An ICL procedure to print the square root of a number
        PRINT The Square Root of (X) is (SQRT(X))
        END PROC

    ICL>
\end{verbatim}
\end{small}

\subsection{Editing procedures --- EDIT, SET EDITOR}

Typing in procedures directly is fine for very simple procedures, but for
anything complex it is likely that some mistakes will be made.
When this happens, it will be necessary to {\em edit} the procedure.
Editing can be done from within ICL using standard editors.
For example the command:

\begin{small}
\begin{verbatim}
    ICL> EDIT SQUARE_ROOT
\end{verbatim}
\end{small}

can be used to edit the SQUARE\_ROOT procedure.
By default the TPU editor is used.
It is also possible to select the EDT or LSE editors using the SET EDITOR
command.
For example, to select EDT, type:

\begin{small}
\begin{verbatim}
    ICL> SET EDITOR EDT
\end{verbatim}
\end{small}

When editing a procedure there are two possible options:
\begin{itemize}
\item You can leave the name of the procedure (specified in the PROC statement)
 unchanged, but edit the code.
 This creates a new version of the procedure which replaces the old one
 {\em when you exit from the editing session} --- in other words, the old
 version will still be used until you terminate the edit.
\item You can change the name of the procedure by editing the PROC statement.
 This creates a new procedure with the new name, and leaves the old procedure
 unchanged.
\end{itemize}
It is possible to create procedures from scratch using the editors.
However, it is recommended that procedures be typed in directly to start with.
The advantage is that during direct entry, any syntactic errors will be detected
immediately.
Thus, if you mistype the PRINT line in the above example you get an error
message as follows:

\begin{small}
\begin{verbatim}
    SQUARE_ROOT> PRINT The Square Root of (X is (SQRT(X))
    PRINT The Square Root of (X is (SQRT(X))
                                ^
    Right parenthesis expected
\end{verbatim}
\end{small}

The error message consists of the line in which the error was detected, a
pointer which indicates where in the line ICL had got to when it found something
was wrong, and a message indicating what was wrong.
In this case it encountered the \verb+'is'+ string when a right parenthesis was
expected.
Following such an error message, you can use the command line editing facility
to correct the line and reenter it.
If the same error occurred during procedure entry using an editor, the error
message would only be generated at the time of exit from the editing session,
and it would be necessary to edit the procedure again to correct it.

\subsection{Direct execution of statements during procedure entry}

It is sometimes useful to have a statement executed directly while entering
a procedure.
When using the direct entry method, this can be done by prefixing the command
with a `\%' character.
For example:

\begin{small}
\begin{verbatim}
    ICL> PROC SQUARE_ROOT X
    SQUARE_ROOT> %HELP
\end{verbatim}
\end{small}

would give you on-line help information you might need to complete the
procedure.
If the `\%' was omitted, the HELP command would be included as part of the
procedure.

\subsection{Saving, loading and deleting procedures --- SAVE, LOAD, DELETE}

Procedures created by direct entry will only exist for the duration of an ICL
session.
If you need to keep them longer, they need to be saved in a disk file.
This is achieved by means of the SAVE command.
To save your SQUARE\_ROOT procedure on disk you would use:

\begin{small}
\begin{verbatim}
    ICL> SAVE SQUARE_ROOT
\end{verbatim}
\end{small}

To load it again, probably in a subsequent ICL session, you would use the LOAD
command.

\begin{small}
\begin{verbatim}
    ICL> LOAD SQUARE_ROOT
\end{verbatim}
\end{small}

The SAVE command causes the procedure to be saved in a file with name
SQUARE\_ROOT.ICL in the current default directory.
LOAD will load the procedure from the same file, also in the default directory.
However, with LOAD it is possible to specify an alternative directory if
required:

\begin{small}
\begin{verbatim}
    ICL> LOAD DISK$USER:[ABC]SQUARE_ROOT
\end{verbatim}
\end{small}

If you have many procedures, you may not want to save and load them all
individually.
It is possible to save {\em all} the current procedures using the command:

\begin{small}
\begin{verbatim}
    ICL> SAVE ALL
\end{verbatim}
\end{small}

This saves them in a single file with the name SAVE.ICL.
They may then be reloaded with the command:

\begin{small}
\begin{verbatim}
    ICL> LOAD SAVE
\end{verbatim}
\end{small}

The SAVE ALL command is rarely used, however, because it is executed
automatically when you exit from ICL.
This ensures that ICL procedures do not get accidentally lost because you
forget to save them.

Procedures can be deleted with the DELETE command.
Just follow the command name with the name of the procedure to be deleted:

\begin{small}
\begin{verbatim}
    DELETE  name
\end{verbatim}
\end{small}


\subsection{Variables}

Any variable used within a procedure is completely distinct from a variable of
the same name used outside the procedure or within a different procedure, as
can be seen in the following example:

\begin{small}
\begin{verbatim}
    ICL> X=1
    ICL> PROC FRED
    FRED> X=1.2345
    FRED> =X
    FRED> END PROC
    ICL> FRED
    1.234500
    ICL> =X
            1
    ICL>
\end{verbatim}
\end{small}

When you run the procedure FRED you get the value of the variable X within the
procedure.
Then, typing `=X' gives the value of X outside the procedure, which has remained
unchanged during execution of the procedure.
This feature has the consequence that you can use procedures freely without
having to worry about any possible side effects on variables outside them.

The situation is exactly the same as that in Fortran where variables in a
subroutine are local to the subroutine in which they are used.
In Fortran, the COMMON statement is provided for use in cases where it is
required to extend the scope of a variable over more than one routine.
ICL does not have a COMMON facility, but does provide an alternative mechanism
for accessing variables outside their scope using the command VARS and the
function VARIABLE.

\subsection{Finding out what is available --- VARS, PROCS}

The VARS command lists all the variables of a procedure.
It has one parameter, which is the name of the procedure.
If the parameter is omitted, the outer level variables, i.e.\ those that are
not part of any procedure, are listed.
Thus, after the previous example you would get:

\begin{small}
\begin{verbatim}
    ICL> VARS FRED
                     X  REAL     1.2345000000000E+00
    ICL> VARS
                     X  INTEGER          1
    ICL>
\end{verbatim}
\end{small}

\paragraph{VARIABLE():}\hfill

The function VARIABLE gives the value of a specified variable in a specified
procedure, for example:

\begin{small}
\begin{verbatim}
    ICL> =VARIABLE(FRED,X)
    1.234500
    ICL>
\end{verbatim}
\end{small}

and thus allows a variable belonging to a procedure to be accessed outside
that procedure.

Note that the variables belonging to a procedure continue to exist after it
finishes execution, and if the procedure is executed a second time they will
retain their values from the first time through the procedure on entry to the
procedure for the second time.

\paragraph{PROCS:}\hfill

To find the names of all the current procedures, use the `PROCS' command:

\begin{small}
\begin{verbatim}
    ICL> PROCS

    SQUARE_ROOT

    ICL>
\end{verbatim}
\end{small}


\subsection{Tracing execution --- SET (NO)TRACE}

The commands SET TRACE and SET NOTRACE switch ICL in and out of trace mode.
When in trace mode, each statement executed will be listed on the terminal.
Trace mode is very useful for debugging procedures.
The commands can be issued either from direct mode to turn on tracing for the
entire execution of a procedure, or inserted in the procedure itself, making
it possible to trace just part of its execution.

\subsection{Command files}

You have seen that the LOAD command `loads' a procedure that had previously
been stored in a file by a SAVE command.
What

\begin{small}
\begin{verbatim}
    ICL> LOAD SQUARE_ROOT
\end{verbatim}
\end{small}

actually does is to read ICL commands from the file SQUARE\_ROOT.ICL
{\em and obey them}.
In the case of SQUARE\_ROOT, these commands define a procedure of the same
name.
However, there is nothing to stop us storing any ICL commands you like in a
file of type .ICL, and then you can load and execute them using the LOAD
command.
The PROC statement is designed only for creating procedures, so for creating
general .ICL files you must use a normal editor.
These general .ICL files are called {\em Command Files}.

A common use of command files is to store a set of command definitions
for a monolith.
You have already seen an example of this in file KAPPA\_DIR:KAPPA.PRC.
When you startup the KAPPA package, this command:

\begin{small}
\begin{verbatim}
    ICL> LOAD KAPPADIR:KAPPA.PRC
\end{verbatim}
\end{small}

is automatically executed, and this causes the command definitions it
contains to become effective.
An example of setting up your own command file is given in Section~\ref{S_Mono}.

A subtlety to be aware of is the difference between storing commands within
a procedure in a command file, and just storing the commands.
For example, if you wanted to store the command definitions for your TESTx
programs (see Chapter~\ref{C_simpap}) in a procedure (given that these programs
had all been put into a monolith called TEST), you could do this from within ICL
as follows:

\begin{small}
\begin{verbatim}
    ICL> PROC TEST
    TEST> { Define commands to run the TESTx programs
    TEST> DEFINE TESTC TEST
    TEST> DEFINE TESTI TEST
    TEST> DEFINE TESTL TEST
    TEST> DEFINE TESTR TEST
    TEST> END PROC
    ICL> SAVE TEST
\end{verbatim}
\end{small}

In a later ICL session you could load this procedure from the file TEST.ICL
in which it had been stored:

\begin{small}
\begin{verbatim}
    ICL> LOAD TEST
\end{verbatim}
\end{small}

This command would read the commands from TEST.ICL and obey them, and this
would have the effect of defining the procedure TEST.
{\em However, the DEFINE commands within the procedure would not be
executed until the procedure was executed.}
Thus, the commands TESTC etc would not be recognised until you had run the
procedure as follows:

\begin{small}
\begin{verbatim}
    ICL> TEST
\end{verbatim}
\end{small}

However, if you used your favourite editor to store the following ICL commands
in the command file TEST.ICL (independently of any ICL session):

\begin{small}
\begin{verbatim}
    { Define commands to run the TESTx programs
    DEFINE TESTC TEST
    DEFINE TESTI TEST
    DEFINE TESTL TEST
    DEFINE TESTR TEST
\end{verbatim}
\end{small}

in your next ICL session you could execute these commands by typing:

\begin{small}
\begin{verbatim}
    ICL> LOAD TEST
\end{verbatim}
\end{small}

The difference from the previous method is that the commands TESTC etc are
now defined and can be used immediately without having to execute a procedure.

\subsection{Running ICL as a batch job}
\label{S_iclbatch}

It is sometimes useful to run one or more ICL procedures as a batch job in VMS.
It is quite easy to set this up by using a parameter with the ICL command to
specify a file from which commands will be taken:

\begin{small}
\begin{verbatim}
    $ ICL filename
\end{verbatim}
\end{small}

This form of the command is equivalent to typing ICL and then typing:

\begin{small}
\begin{verbatim}
    ICL> LOAD filename
\end{verbatim}
\end{small}

Note that a LOAD file may include direct commands as well as procedures.
In order to create a Batch job, you must set up a file which contains all the
procedures wanted, a command (or commands) to run them, and an EXIT command to
terminate the job.
Here is the file for a simple Batch job to print a table of square roots using
your earlier example procedure:

\begin{small}
\begin{verbatim}
    PROC SQUARE_ROOT X
    { An ICL procedure to print the square root of a number
       PRINT The Square Root of (X) is (SQRT(X))
    END PROC

    PROC TABLE
    { A procedure to print a table of square roots of numbers from 1 to 100
       LOOP FOR I=1 TO 100
          SQUARE_ROOT (I)
       END LOOP
    END PROC

    { Next, the command to run this procedure

    TABLE

    { And then an EXIT command to terminate the job

    EXIT
\end{verbatim}
\end{small}

This file can be generated using the EDIT command from DCL.
If the procedures have already been tested from ICL, it is convenient to use
a SAVE ALL command (or exit from ICL) to save them, and then edit the SAVE.ICL
file to add the additional direct commands.
Suppose this file is called TABLE.ICL, then to create a batch job, a command
file is needed which could be called TABLE.COM and would contain the following:

\begin{small}
\begin{verbatim}
    $ ICL TABLE
    $ EXIT
\end{verbatim}
\end{small}

It might also need to contain a SET DEF command to set the appropriate
directory, or a directory specification on the TABLE file name if it is
not in the top level directory.

To submit the job to the batch queue, the following command is used:

\begin{small}
\begin{verbatim}
    $ SUBMIT/KEEP TABLE
\end{verbatim}
\end{small}

The /KEEP qualifier specifies that the output file for the batch job is
to be kept.
This file will appear as TABLE.LOG in your top level directory and will contain
the output from the batch job.
An /OUTPUT qualifier can be used to specify a different file name or directory
for it.

\section{Exceptions}
\label{S_excep}

\begin{center}
\begin{tabular}{|l|l|}
\hline
SIGNAL      & Signal an ICL exception \\
\hline
\end{tabular}
\end{center}

Error conditions and other unexpected events are referred to as
{\em Exceptions}.
When such a condition is detected in direct mode, a message is output.
For example, if you enter a statement which results in an error:

\begin{small}
\begin{verbatim}
    ICL> =SQRT(-1)
    SQUROONEG   Square Root of Negative Number
    ICL>
\end{verbatim}
\end{small}

you get a message consisting of the name of the exception (SQUROONEG) and
a description of the nature of the exception.
A full list of ICL exceptions is given in Chapter~\ref{C_rsicl}.

If the error occurs within a procedure, the message contains a little more
information.
For example, if you use your square root procedure with an invalid value, you
get the following messages:

\begin{small}
\begin{verbatim}
    ICL> SQUARE_ROOT (-1)
    SQUROONEG   Square Root of Negative Number
    In Procedure: SQUARE_ROOT
    At Statement: PRINT  The Square Root of (X) is (SQRT(X))
    ICL>
\end{verbatim}
\end{small}

If one procedure is called by another, the second procedure will also be listed
in the error message.
For example, if you run the following procedure:

\begin{small}
\begin{verbatim}
    PROC TABLE
    { Print a table of Square roots from 2 down to -2
       LOOP FOR I = 2 TO -2 STEP -1
         SQUARE_ROOT (I)
       END LOOP
    END PROC
\end{verbatim}
\end{small}

you get:

\begin{small}
\begin{verbatim}
    ICL> TABLE
    The Square Root of 2 is 1.414214
    The Square Root of 1 is 1
    The Square Root of 0 is 0
    SQUROONEG   Square Root of Negative Number
    In Procedure: SQUARE_ROOT
    At Statement: PRINT  The Square Root of (X) is (SQRT(X))
    Called by: TABLE
    ICL>
\end{verbatim}
\end{small}


\subsection{Exception handlers}

It is often useful to be able to modify the default behaviour on an error
condition.
You may not want to output an error message and return to the ICL$>$ prompt, but
rather to handle the condition in some other way.
This can be done by writing an {\em Exception Handler}.
Here is an example of an exception handler in the SQUARE\_ROOT procedure:

\begin{small}
\begin{verbatim}
    PROC SQUARE_ROOT X
    { An ICL procedure to print the square root of a number
       PRINT The Square Root of (X) is (SQRT(X))
       EXCEPTION SQUROONEG
    { Handle the imaginary case
          SQ = SQRT(ABS(X))
          PRINT The Square Root of (X) is (SQ&'i')
       END EXCEPTION
    END PROC
\end{verbatim}
\end{small}

Now running the TABLE procedure gives:

\begin{small}
\begin{verbatim}
    ICL> TABLE
    The Square Root of 2 is 1.414214
    The Square Root of 1 is 1
    The Square Root of 0 is 0
    The Square Root of -1 is 1i
    The Square Root of -2 is 1.414214i
    ICL>
\end{verbatim}
\end{small}

The exception handler has two effects.
First, the code contained in the exception handler is executed when the
exception occurs.
Second, the procedure exits normally to its caller (in this case TABLE) rather
than aborting execution completely and returning to the ICL$>$ prompt.

Exception handlers should be placed after the normal code, but before the END
PROC statement.
There may be any number of exception handlers in a procedure, each for a
different exception.
The exception handler begins with an EXCEPTION statement specifying the
exception name, and finishes with an END EXCEPTION statement.
Between these may be any ICL statements, including calls to other procedures.

An exception handler does not have to be in the procedure causing the exception,
but could be in a procedure further up the chain of calls.
In your example you could put an exception handler for SQUROONEG in TABLE rather
than in SQUARE\_ROOT:

\begin{small}
\begin{verbatim}
    PROC TABLE
    { Print a table of Square roots from 2 down to -2
       LOOP FOR I = 2 TO -2 STEP -1
          SQUARE_ROOT (I)
       END LOOP
       EXCEPTION SQUROONEG
          PRINT 'Can''t handle negative numbers - TABLE Aborting'
       END EXCEPTION
    END PROC
\end{verbatim}
\end{small}

giving:

\begin{small}
\begin{verbatim}
    ICL> TABLE
    The Square Root of 2 is 1.414214
    The Square Root of 1 is 1
    The Square Root of 0 is 0
    Can't handle negative numbers - TABLE aborting
    ICL>
\end{verbatim}
\end{small}

Below is an example of a pair of procedures which use an exception handler for
floating point overflow in order to locate the largest floating point number
allowed on the system.
Starting with a value of 1, this is multiplied by 10 repeatedly until floating
point overflow occurs.
The highest value found in this way is then multiplied by 1.1 repeatedly until
overflow occurs, then by 1.01 etc:

\begin{small}
\begin{verbatim}
    PROC LARGE  START, FAC, L
    { Return in L the largest floating point number before
    { overflow occurs when START is repeatedly multiplied by FAC.
       L = START
       LOOP
          L = L * FAC
       END LOOP
       EXCEPTION FLTOVF
    { This exception handler doesn't have any code - it just
    { causes the procedure to exit normally on overflow.
       END EXCEPTION
    END PROC

    PROC LARGEST
    { A Procedure to find the largest allowed floating point number on the system.
       FAC = 10.0
       LARGE  1.0, (FAC), (L)
       LOOP WHILE FAC > 0.00000001
          LARGE (L), (1.0+FAC), (L)
          FAC = FAC/10.0
       END LOOP
       PRINT  The largest floating point number allowed is (L)
    END PROC
\end{verbatim}
\end{small}


\subsection{Keyboard aborts}

One exception which is commonly encountered is that which results when a
ctrl/C is entered on the terminal.
This results in the exception CTRLC and may therefore be used to abort execution
of a procedure and return ICL to direct mode.
However, an exception handler for CTRLC may be added to a procedure to modify
the behaviour when a ctrl/C is typed.

\subsection{SIGNAL}

The exceptions described up to now have all been generated internally by the
ICL system, or, in the case of CTRLC, initiated by the user.
It is also possible for ICL procedures to generate exceptions which may be used
to indicate error conditions.
This is done by the SIGNAL command which has the form:

\begin{small}
\begin{verbatim}
    SIGNAL  name  text
\end{verbatim}
\end{small}

where \verb+name+ is the name of the exception, and \verb+text+ is the message
text associated with the exception.
The exception name may be any valid ICL identifier.
Exceptions generated by SIGNAL work in exactly the same way as the standard
exceptions listed in \xref{SG/5}{sg5}{}.
An exception handler will be executed if one exists, otherwise an error message
will be output and ICL will return to direct mode.

One use of the SIGNAL command is as a means of escaping from deeply nested
loops.
The BREAK statement can be used to exit from a single loop, but is not
applicable if two or more loops are nested.
In these cases, the following structure could be used:

\begin{small}
\begin{verbatim}
    LOOP
       LOOP
          LOOP
             ...
             IF FINISHED
                SIGNAL ESCAPE
             END IF
             ...
          END LOOP
       END LOOP
    END LOOP

    EXCEPTION ESCAPE
    END EXCEPTION
\end{verbatim}
\end{small}

where the exception handler again contains no statements, but simply exists
to cause normal procedure exit, rather than an error message, when the
exception is signalled.

\section{ICL login files}
\label{S_icllf}

If you frequently need to define commands to run particular programs, it is
convenient to define them in an {\em ICL login file} which will be loaded
automatically each time ICL is started up.
An ICL login file works in exactly the same way as a DCL login file.
ICL uses the logical name ICL\_LOGIN to locate this file, so store the
required definitions in a file called LOGIN.ICL in your top level directory,
and put the following definition in your DCL LOGIN.COM file:

\begin{small}
\begin{verbatim}
   $ DEFINE ICL_LOGIN DISK$USER:[ABC]LOGIN.ICL
\end{verbatim}
\end{small}

where DISK\$USER:$[$ABC$]$ needs to be replaced by the actual directory used.
This command file will then be loaded automatically whenever you start up ICL
and can include procedures, definitions of commands, or indeed any valid ICL
command.
Below is an example of an ICL login file which illustrates some of the
facilities which may be used:

\begin{small}
\begin{verbatim}
    { ICL Login File

    { Define TYPE command
    DEFSTRING  T(YPE)  DCL TYPE

    { Define EDIT command
    HIDDEN PROC EDIT name
       IF INDEX(name,'.') = 0
          #EDIT (name)
       ELSE
          $ EDIT (name)
       ENDIF
    END PROC

    { Login Message
    PRINT
    PRINT   Starting ICL at (TIME()) on (DATE())
    PRINT
\end{verbatim}
\end{small}


\subsection{Hidden procedures}

The definition of the EDIT command shown in the login file above is done using
a {\em Hidden procedure}.
Since EDIT is an ICL command to edit procedures, if you just used DEFSTRING to
define EDIT as DCL EDIT, you would lose the ability to edit ICL procedures ---
the EDIT command would always edit VMS files.
The procedure used to redefine EDIT gets around this by testing for the
existence of a dot in the name of the file to be edited using the INDEX
function.
If a dot is present, it assumes that a VMS file is being edited and issues the
command `\$ EDIT (name)'.
If no dot is present, it assumes that an ICL procedure is being edited and
the command `\#EDIT (name)' is issued.
The \# character forces the internal definition of EDIT to be used, rather than
the definition currently being defined.

The procedure is written as a {\em hidden} procedure, indicated by the word
HIDDEN preceding PROC.
A hidden procedure works in exactly the same way as a normal procedure, but it
does not appear in the listing of procedures produced by a PROCS statement,
nor can it be edited, deleted, or saved from within ICL.
It is convenient to make all procedures in your login file hidden procedures
so that they do not clutter your directory of procedures and cannot be
deleted accidentally.

\section{Extending on-line help}
\label{S_eolh}

\begin{center}
\begin{tabular}{|l|l|}
\hline
HELP        & Display on-line documentation \\
DEFHELP     & Define the source of Help information \\
\hline
\end{tabular}
\end{center}

ICL includes a HELP command which provides on-line documentation on ICL
itself.
Using the DEFHELP command it is possible to extend this facility to access
information on the commands you have added.
In order to do this, you need to create a help library in the normal format used
by the VMS help system.
This is described in the VAX/VMS documentation for the Librarian utility.
You can then specify topics from this library which will be available using
the ICL HELP command by using a command of the form:

\begin{small}
\begin{verbatim}
    DEFHELP EDIT LIBRARY.HLB
\end{verbatim}
\end{small}

This will cause a:

\begin{small}
\begin{verbatim}
    HELP EDIT
\end{verbatim}
\end{small}

command to return the information on EDIT in help library LIBRARY.HLB, rather
than in the standard ICL library.

\section{Example procedures}
\label{S_exproc}

This final section lists some examples of ICL procedures taken from the KAPPA
manual (\xref{SUN/95}{sun95}{}).
They should help you understand how to use the programming facilities of
ICL correctly.
Many of the commands used in the procedures are KAPPA applications.

\paragraph{Unsharpmask:}\hfill

Suppose you have a series of commands to run on a number of files.
You could create a procedure to perform all the stages of the processing,
deleting the intermediate files that it creates.

\begin{small}
\begin{verbatim}
     PROC UNSHARPMASK NDFIN CLIP NDFOUT

     { Clip the image to remove the cores of stars and galaxies above
     { a nominated threshold.
        THRESH (NDFIN) TMP1 THRHI=(CLIP) NEWHI=(CLIP) \

     { Apply a couple of block smoothings with boxsizes of 5 and 13
     { pixels.  Delete the temporary files as we go along.
        BLOCK TMP1 TMP2 5
        $ DELETE TMP1.SDF;0
        BLOCK TMP2 TMP3 13
        $ DELETE TMP2.SDF;0

     { Multiply the smoothed image by a scalar.
        CMULT TMP3 0.8 TMP4
        $ DELETE TMP3.SDF;0

     { Subtract the smoothed and renormalised image from the input image.
     { The effect is to highlight the fine detail, but still retain some of the
     { low-frequency features.
        SUB (NDFIN) TMP4 (NDFOUT)
        $ DELETE TMP4.SDF;0
     END PROC
\end{verbatim}
\end{small}

\paragraph{Multistat:}\hfill

A common use of procedures is likely to be duplicate processing for several
files.
Here is an example procedure that does that.
It uses some intrinsic functions which look just like Fortran.

\begin{small}
\begin{verbatim}
     PROC MULTISTAT

     { Prompt for the number of NDFs to analyse.  Ensure that it is positive.
        INPUTI Number of frames:  (NUM)
        NUM = MAX(1, NUM)

     { Find the number of characters required to format the number as
     { a string using a couple of ICL functions.
        NC = INT(LOG10(NUM)) + 1

     { Loop NUM times.
        LOOP FOR I=1 TO (NUM)

     { Generate the name of the NDF to be analysed via the ICL function SNAME.
          FILE = '@' & SNAME('REDX',I,NC)

     { Form the statistics of the image.
          STATS NDF=(FILE)
        END LOOP
     END PROC
\end{verbatim}
\end{small}

If NUM is set to 10, the above procedure obtains the statistics of the images
named REDX1, REDX2, \dots REDX10.
The {\small ICL} variable FILE is in parentheses because its value is to be
substituted into parameter NDF.
There is a piece of syntax to note which often catches people out.
Filenames passed via {\small ICL} variables, such as FILE in the above example,
must be preceded by an {\tt @}.

\paragraph{Flatfield:}\hfill

Here is another example, which could be used to flat field a series of CCD
frames.
Instead of executing a specific number of files, you can enter an arbitrary
sequence of NDFs.
When processing is completed a !! is entered rather than an NDF name, and that
exits the loop.
Note the {\tt \~{}} continuation character.
(It's not required but it's included for pedagogical reasons.)
\pagebreak[3]

\begin{small}
\begin{verbatim}
     PROC FLATFIELD

     { Obtain the name of the flat-field NDF.  If it does not have a
     { leading @ insert one.
        INPUT Which flat field frame?: (FF)
        IF SUBSTR(FF,1,1) <> '@'
           FF = '@' & (FF)
        END IF

     { Loop until there are no further NDFs to flat field.
        MOREDATA = TRUE
        LOOP WHILE MOREDATA

     { Obtain the frame to flat field.  Assume that it will not have
     { an @ prefix. Generate a title for the flattened frame.
          INPUT Enter frame to flat field (!! to exit): (IMAGE)
           MOREDATA = IMAGE = '!!'
           IF MOREDATA
              TITLE = 'Flat field of ' & (IMAGE)
              IMAGE = '@' & (IMAGE)

      { Generate the name of the flattened NDF.
              IMAGEOUT = (IMAGE) & 'F'
              PRINT Writing to (IMAGEOUT)

      { Divide the image by the flat field.

              DIV IN1=(IMAGE) IN2=(FF) OUT=(IMAGEOUT) ~
                  OTITLE= (TITLE)
           ENDIF
        END LOOP
     END PROC
\end{verbatim}
\end{small}

\paragraph{Colstar:}\hfill

Some {\small KAPPA} applications, particularly the statistical ones, produce
output parameters which can be passed between applications via {\small ICL}
variables.
Here is an example to draw a perspective histogram centred about a star in a
nominated data array from only the star's approximate position.
The region about the star is stored in an output NDF file.
Note, in a procedure meant to be used in earnest, there would be checks that
input and output names begin with an {\tt @}.

\begin{small}
\begin{verbatim}
     PROC COLSTAR FILE,X,Y,SIZE,OUTFILE

     {+
     {  Arguments:
     {     FILE = FILENAME (Given)
     {        Input NDF containing one or more star images.
     {     X = REAL (Given)
     {        The approximate x position of the star.
     {     Y = REAL (Given)
     {        The approximate y position of the star.
     {     SIZE = REAL (Given)
     {        The half-width of the region about the star's centroid to be
     {        plotted and saved in the output file.
     {     OUTFILE = FILENAME (Given)
     {        Output primitive NDF of 2*%SIZE+1 pixels square (unless
     {        constrained by the size of the data array or because the location
     {        of the star is near an edge of the data array.
     {-

     { Search for the star in a 21x21 pixel box.  The centroid of the
     { star is stored in the ICL variables XC and YC.
        CENTROID INPIC=(FILE) XINIT=(X) YINIT=(Y) XCEN=(XC) YCEN=(YC) ~
          MODE=INTERFACE SEARCH=21 MAXSHIFT=14

     { Convert the co-ordinates to pixel indices.
        IX = NINT(XC + 0.5)
        IY = NINT(YC + 0.5)

     { Find the upper and lower bounds of the data array to plot. Note
     { this assumes no origin information is stored in the data file.
        XL = MAX(1, IX - SIZE)
        YL = MAX(1, IY - SIZE)
        XU = MAX(1, IX + SIZE)
        YU = MAX(1, IY + SIZE)

     { Create a new IMAGE file centred on the star.
        PICK2D INPIC=(FILE) OUTPIC=(OUTFILE) XSTART=(XL) YSTART=(YL) ~
           XFINISH=(XU) YFINISH=(YU)

     { Draw a perspective histogram around the star on the current
     { graphics device.
        COLUMNAR IN=(OUTFILE)

     { Exit if an error occurred, such as not being to find a star
     { near the supplied position, or being unable to make the plot.
        EXCEPTION ADAMERR
           PRINT Unable to find or plot the star.
        END EXCEPTION
     END PROC
\end{verbatim}
\end{small}

\paragraph{Fancylook:}\hfill

This creates a fancy display of an image with axes and a key showing data
values.
Note the need to give an expression combining the $x$-$y$ bounds of the key to
the LBOUND and UBOUND parameter arrays.

\begin{small}
\begin{verbatim}
     PROC FANCYLOOK NDF

     { Find the extent of the current picture.
        GDSTATE NCX1=(FX1) NCX2=(FX2) NCY1=(FY1) NCY2=(FY2) NOREPORT

     { Display the image with axes using the most-ornate font.
        DISPLAY (NDF) MODE=PE AXES FONT=NCAR COSYS=D SCALOW=(LOW) SCAHIGH=(HIGH) \

     { Find the extent of the image picture.
        PICIN NCX1=(DX1) NCX2=(DX2) NCY1=(DY1) NCY2=(DY2) NOREPORT

     { Determine the widths of the borders.
        XL = DX1 - FX1
        XR = FX2 - DX2
        YB = DY1 - FY1
        YT = FY2 - DY2

     { Only plot a key if there is room.
        IF MAX(XL, XR, YB, YT) > 0.0

     { Determine which side has most room for the key, and derive the
     { the location of the key. First, see if the key is vertical.
           IF MAX(XL,XR) >= MAX(YB,YT)
              WIDTH = MIN(0.4*MAX(XL,XR), 0.25*(DX2-DX1))
              HEIGHT = MIN(6.0*WIDTH, 0.7*(DY2-DY1))
              IF XL > XR
                 XK1 = DX1 - 1.5 * WIDTH
                 XK2 = DX1 - 0.5 * WIDTH
              ELSE
                 XK1 = DX2 + 0.5 * WIDTH
                 XK2 = DX2 + 1.5 * WIDTH
              ENDIF
              YK1 = 0.5 * (DY2 + DY1 - HEIGHT)
              YK2 = 0.5 * (DY2 + DY1 + HEIGHT)
           ELSE

     { Deal with horizontal key.
              WIDTH = MIN(0.4 * MAX(YB,YT), 0.25 * (DY2-DY1))
              HEIGHT = MIN(6.0 * WIDTH, 0.7 * (DX2-DX1))
              IF YB > YT
                 YK1 = DY1 - 1.5 * WIDTH
                 YK2 = DY1 - 0.5 * WIDTH
              ELSE
                 YK1 = DY2 + 0.5 * WIDTH
                 YK2 = DY2 + 1.5 * WIDTH
              ENDIF
              XK1 = 0.5 * (DX2 + DX1 - HEIGHT)
              XK2 = 0.5 * (DX2 + DX1 + HEIGHT)
           ENDIF

     { Draw the key to fit within the current picture annotating with
     { the scaling used in DISPLAY.
           LUTVIEW LOW=(LOW) HIGH=(HIGH) LBOUND=[(XK1&','&YK1)] ~
             UBOUND=[(XK2&','&YK2)] MODE=XY
        ENDIF
     END PROC
\end{verbatim}
\end{small}
