\documentclass[noabs,11pt,nolof]{starlink}

%------------------------------------------------------------------------------
\stardoccategory    {Starlink System Note}
\stardocinitials    {SSN}
\stardocnumber      {8.1}
\stardocauthors     {R.F. Warren-Smith}
\stardocdate        {6th March 1991}
\stardoctitle       {Naming Conventions for Accessing \\[1ex]
                                Starlink Subroutine Libraries}
%------------------------------------------------------------------------------
% Add any \providecommand or \newenvironment commands here
%------------------------------------------------------------------------------

\begin{document}
\scfrontmatter

\section{Introduction}

This document defines a set of naming conventions to be used for accessing
Starlink subroutine libraries on VAX/VMS machines.
The purpose of these conventions is to:

\begin{itemize}

\item Provide a uniform interface for software developers who use Starlink
libraries.

\item Simplify the process of linking with Starlink libraries.

\item Facilitate the use of shareable images, so that enhancements to
libraries can be made available without the need to re-link the calling
software.

\item Increase the modularity of libraries, and hence allow them to be
maintained and developed more easily by separate individuals.

\end{itemize}

The new arrangements are based on the use of VMS shareable images and on the
definition of a new uniform set of VMS logical names and DCL symbols.
A library may comply with the naming conventions described in this document
by providing its facilities via logical names and symbols having the form
specified in later sections.
However, no restriction is placed on other names or symbols which may be
provided, so an existing library may be upgraded simply by defining any new
names or symbols required, along with the associated files if necessary.
Names and symbols which already exist to serve the same purpose may be
left in place so that software that depends on them will continue to work.


\section{Use of Shareable Images}

As far as possible, Starlink subroutine libraries should make their
executable code available to users in the form of VMS shareable images.
This approach confers a number of advantages, in particular the ability for
other software to assimilate new features when the library is upgraded
without the need to re-link the calling software.
The linking procedures themselves are also simplified by removing the need
to know about, or to specify, all the sub-libraries (or sub-sub-libraries
\emph{etc.}) which may be required.

However, shareable images place a responsibility on the developer of a
subroutine library to ensure that any modifications which may be made are
compatible with all previously distributed versions of the library.
Any changes which would require the calling software to be re-linked will
take a long time to propagate through all affected software and should
always be avoided.
In particular, the need to ensure upward-compatibility of shareable image
\emph{transfer vectors} must be appreciate --- the VAX Linker Utility
Manual should be consulted for details.

In some cases, the use of a shareable image may not be possible or
appropriate.
For instance, the library's executable files may have been provided
commercially, or it may be desirable to allow the user to replace certain
library routines with private versions -- something which cannot be done
in quite the same way with a shareable image\footnote{This facility can
still be provided by placing the routine to be replaced on its own in a
separate shareable image.
A user can then provide an alternative version of this image and access it
by assigning the appropriate logical name to it at run time.}
(of course the standard, but under-used technique of passing a subroutine
or function as an argument to another routine, so that the user can
specify which routine should be called at a lower level, is not affected
in any way by shareable libraries).
In such cases, the prescription given for the use of shareable libraries in
this document may not apply, but the spirit of the naming convention
described here should be followed as far as possible.


\section{Stand-alone and ADAM Libraries}

Starlink subroutine libraries should be viewed as separate items of software
which are intended to work together, as part of a complete programming
environment.
To function together effectively they must adhere to various conventions,
and the ease with which they can be integrated into a single application
will depend on the extent of the conventions which they share.

In some situations, a highly ``Starlink-specific'' implementation of a
library may be required.
By making full use of Starlink conventions, such a system can be made
particularly easy to use in a Starlink environment.
In other situations, however, a more general-purpose implementation might be
better, especially if it may need to be used on non-Starlink machines where
other supporting software will not be available.
To satisfy these requirements, it is anticipated that many Starlink
libraries will be made available in two ``compatibility levels''.

The ``\emph{stand-alone}'' version of a library will be a general-purpose
implementation which is intended for use on its own, separate from any
pre-defined programming environment.
Such a library may call the stand-alone versions of other Starlink
libraries, but this dependence should generally be kept to a minimum so that
the maximum freedom of use and portability is achieved.
The stand-alone version of a library is provided for use by applications
which do not intend to integrate closely with any particular programming
environment.

Conversely, the \emph{ADAM} version of a library will be closely bound to the
Starlink ADAM programming environment, which allows a closer degree of
cooperation between separate libraries (in particular through the use of the
ADAM parameter system).
A more integrated appearance can then be provided for the user, along with
additional facilities, so that the ADAM version of a library will usually
contain extra routines which are not available in the stand-alone version.
This version of a library should always be used by applications which make
use of ADAM facilities.


\section{Library Names and Prefixes}

The naming conventions described in this document are based upon the use of
a unique name for each subroutine library together with a simple unique
abbreviation of it.
This abbreviation is commonly referred to as the \emph{library prefix},
although both it and the full library name are used as prefixes when
constructing logical names and DCL symbols associated with the library.

In the remainder of this document, the library name:

\begin{terminalv}
LIBRARY
\end{terminalv}

and its abbreviation:

\begin{terminalv}
LIB
\end{terminalv}

are used to denote these, with the understanding that they should be
replaced with the specific library name or prefix in question.

Whenever a library name or prefix is used to construct a compound name it
should be separated from the subsequent part of the name with a single
underscore character `\_'.
This follows the established Starlink practice for naming subroutines and
functions.\footnote{The names of symbolic constants differ slightly in using
a double underscore `\_\_'.}

\paragraph{Library names.}
The library name LIBRARY is the name or acronym used to identify
the library as a software item in the Starlink Software Index.\footnote{Some
existing libraries (\emph{e.g.}\ PAR) are not distributed as separate
software items, but all new libraries should be designed with separate
distribution in mind.}
Thus examples of library names are:

\begin{quote}
\begin{center}
\begin{tabular}{ll}
CHR & \emph{Character Handling Routines} \\
EMS & \emph{Error and Message Service} \\
HDS & \emph{Hierarchical Data System} \\
IDI & \emph{Image Display Interface} \\
PRIMDAT & \emph{Primitive Data Processing} \\
SGS & \emph{Simple Graphics System}
\end{tabular}
\end{center}
\end{quote}

Library names should be no more than 8 characters long,\footnote{Some
existing libraries do not adhere to these conventions, but they should be
observed by all new software.} should begin with an alphabetic character and
should contain only alphanumeric characters (no underscores are allowed).
Existing library names should not be re-used (see the Starlink Software
Index for a list of names currently in use) and potentially confusing names
should be avoided.
In particular, names or terms which are used for other purposes (\emph{e.g.}\
command names), or which have potentially ambiguous meanings, should not be
used as library names.

\paragraph{Library prefixes.}
The library prefix LIB should consist of three alphabetic
characters\footnotemark[\thefootnote] chosen, where possible, to be an
abbreviation of the library name.
If the library name itself consists of just three characters, then both the
name and its abbreviation should be the same.
The prefixes for the libraries above are:

\begin{quote}
\begin{center}
\texttt{CHR, EMS, HDS, IDI, PRM, SGS}
\end{center}
\end{quote}

The library prefix is intended for use not only as part of the naming
conventions described in this document, but also as a prefix for the
subroutines, functions and symbolic constants within the library (according
to the Starlink subroutine naming conventions described in SGP/16).
Thus, routines in the SGS library have names such as:

\begin{quote}
\begin{center}
\texttt{SGS\_LINE, SGS\_SHTX,} \emph{etc.}
\end{center}
\end{quote}

In some libraries, more than one routine prefix may be used to sub-divide
the routines into separate facilities.
In such cases the library prefix should be chosen from amongst the
routine prefixes in use.
For instance, the HDS library currently uses all the following routine prefixes,
some of them internally:

\begin{quote}
\begin{center}
\texttt{CMP, DAT, DAU, EXC, HDS, PRO, REC}
\end{center}
\end{quote}

but only ``HDS'' is used as the library prefix.

To avoid conflicts, routine prefixes which are already in use are documented
in an appendix to SGP/16 which is periodically updated.
A note of prefixes allocated since the last revision of SGP/16 is held at
RAL and a new prefix may be reserved by contacting any member of the
Starlink project team at RAL.

\section{Library Directory Names}

For each library, a logical name of the following standard form:

\begin{terminalv}
LIBRARY_DIR
\end{terminalv}

should be defined to identify the directory in which user-accessible files
reside.
All other logical name and symbol definitions associated with the library
should be made in terms of this single directory name, so that if the
software is moved to another location, only the assignment of this name need
be changed.

The library directory name should be defined in the system logical name
table by means of a suitable command in the SSC:STARTUP.COM file, such as:

\begin{terminalv}
$ DEFINE/SYSTEM LIBRARY_DIR <disk>:<directory>
\end{terminalv}

This command will then be executed at system startup.
Other software may test for the existence or non-existence of the directory
name in order to determine whether the library is installed and available
for use.
The choice of name for the actual directory in which the software resides
will be made by the Starlink Software Librarian, while the name of the disk
will be determined by the System Manager who installs it.


\section{Library Startup Files}
\label{ss:librarystartup}

It is intended that applications which call Starlink libraries may be run
without further preparation once the Starlink login file has been executed
using the command:\footnote{Except that ADAM applications will also require
the \$ADAMSTART command.}

\begin{terminalv}
$ @SSC:LOGIN
\end{terminalv}

However, additional logical name definitions (\emph{e.g.}\ for include files,
\emph{etc.}) will usually be needed if software development is planned.
Each library should therefore provide a DCL command file whose purpose is to
define any logical names and/or DCL symbols needed when developing software
which calls the library.

This file should reside in the library directory and a DCL global symbol of
the following standard form:

\begin{terminalv}
LIB_DEV
\end{terminalv}

should be defined in order to execute it.
This symbol should be defined by means of a suitable command in the Starlink
login file SSC:LOGIN.COM, such as:

\begin{terminalv}
$ LIB_DEV :== @LIBRARY_DIR:LIB_DEV
\end{terminalv}

where LIB\_DEV.COM is the command file in question (this file name is
recommended for future use but existing files which serve the same purpose
need not be changed).

Other software may test for the existence or non-existence of the LIB\_DEV
symbol in order to determine whether a library is installed in a form which
allows software development.

A software developer wishing to make use of the library's facilities should
first execute the command:

\begin{terminalv}
$ LIB_DEV
\end{terminalv}

to set up the appropriate development environment.
A sequence of such commands might typically be placed in a LOGIN.COM file in
order to define a programming environment which is used frequently.
Pre-defined programming environments such as ADAM may also make use of these
commands in their own startup procedures.
To facilitate this type of use, a library's startup file should ensure that
if the command issued is of the form:

\begin{terminalv}
$ LIB_DEV NOLOG
\end{terminalv}

then the presence of ``NOLOG'' as the first parameter will suppress the
output of any informational message.
Otherwise, library startup files may optionally send messages to SYS\$OUTPUT
when they are executed in order to identify themselves.

\section{Shareable Image Names}
\label{ss:imagenames}

The files which contain the executable code for a library's shareable
images should reside in the library directory and should be assigned logical
names of the following standard form in order to identify them:

\begin{quote}
\begin{tabular}{rl}
& LIB\_IMAGE\\
\emph{and/or} & LIB\_IMAGE\_ADAM
\end{tabular}
\end{quote}

The first of these names refers to the stand-alone version of the library
and the second to the ADAM version; they are required at run-time to
identify the relevant shareable images to other executable images linked
against them.
If either version of the library does not exist, then the corresponding
logical name should be left undefined.
Other software may test for the existence or non-existence of either of these
logical names in order to determine the availability of a particular version
of the library.

In cases where there is no difference between the stand-alone and ADAM
shareable images, both logical names may refer to the same image file.
In general, however, these names will refer to different files, and it
should always be assumed that the images will differ.
In particular, it should be assumed that their transfer vectors will be
incompatible, so that they may not be interchanged once an application has
been linked with a particular version.

The logical names identifying a library's shareable images should be defined
in the system logical name table at system startup by means of suitable
commands in the SSC:STARTUP.COM file, such as:

\begin{terminalv}
$ DEFINE/SYSTEM LIB_IMAGE      LIBRARY_DIR:LIB_IMAGE
$ DEFINE/SYSTEM LIB_IMAGE_ADAM LIBRARY_DIR:LIB_IMAGE_ADAM
\end{terminalv}

where LIB\_IMAGE.EXE and LIB\_IMAGE\_ADAM.EXE are the shareable image files
in question (these file names are recommended for future use but existing
files need not be changed).


\section{Linker Options Files}

Each library should provide linker options files (with a file type of .OPT) to
be used by software developers when linking against the library.
These files should reside in the library directory and should be assigned
logical names of the following standard form in order to identify them:

\begin{quote}
\begin{tabular}{rl}
& \texttt{LIB\_LINK} \\
\emph{and/or} & \texttt{LIB\_LINK\_ADAM}
\end{tabular}
\end{quote}

The first of these allows linking with the stand-alone version of the
library, and the second with the ADAM version.
As usual, if either version of the library does not exist, then the
corresponding logical name should be left undefined.

Normally the options file for linking with a shareable image will refer
directly to that image, which should be identified by its logical name
according to the conventions in \S\ref{ss:imagenames}, \emph{e.g:}

\begin{terminalv}
LIB_IMAGE/SHARE
\end{terminalv}

or

\begin{terminalv}
LIB_IMAGE_ADAM/SHARE
\end{terminalv}

In cases where there is no difference between the stand-alone and ADAM
executable images, both versions of the linker options file should
nevertheless still be provided, and each options file should access the
image through the appropriate logical name.

Users of linker options files should always assume that the stand-alone and
ADAM linking processes will differ.
Thus, a stand-alone application might be linked with a particular library as
follows:

\begin{terminalv}
$ LIB_DEV               ! Done initially, normally in a LOGIN.COM file
$ FORTRAN MYPROG
$ LINK MYPROG,LIB_LINK/OPT
\end{terminalv}

while an ADAM application would be linked thus:

\begin{terminalv}
$ LIB_DEV               ! Done initially, normally in a LOGIN.COM file
$ FORTRAN MYPROG
$ ALINK MYPROG,LIB_LINK_ADAM/OPT
\end{terminalv}

Note that only those libraries explicitly referenced in the application need
be considered when constructing a linker command line.
Any sub-libraries called at lower levels will be included automatically
through having previously been linked into those shareable images which \textbf{are} referenced.

In practice, explicit reference to startup commands or linker options files
will not normally be required in order to access libraries which form part
of the ADAM programming environment, since the appropriate ADAM startup and
linking procedures will access all relevant libraries automatically.
However, newly developed libraries, or those which are not part of ADAM, may
still be accessed in ADAM applications by referring to them explicitly, as
above.
The number of libraries which are made available as part of the ADAM
environment is likely to increase as a result of standardising the methods
of library access, but the relevant ADAM documentation should be consulted
for a complete list of these.

The logical names identifying the linker options files should be defined in
the process logical name table by means of suitable commands in the library
startup file (\S\ref{ss:librarystartup}), such as:

\begin{terminalv}
$ DEFINE/NOLOG LIB_LINK      LIBRARY_DIR:LIB_LINK
$ DEFINE/NOLOG LIB_LINK_ADAM LIBRARY_DIR:LIB_LINK_ADAM
\end{terminalv}

where LIB\_LINK.OPT and LIB\_LINK\_ADAM.OPT are the linker options files in
question (these file names are recommended for future use but existing files
need not be changed).


\section{Include File Names}

The convention for naming and accessing include files from Fortran is
already well-established within ADAM and, to a lesser degree, within
non-ADAM Starlink software.
Subroutine libraries may provide Fortran include files for public use and
should define logical names to access them as follows:

\begin{quote}
\texttt{LIB\_ERR} --- \emph{Error codes associated with the library} \\
\texttt{LIB\_PAR} --- \emph{Other constants defined by PARAMETER statements}
\end{quote}

Users of include files should always refer to them by their logical names,
\emph{e.g:}

\begin{terminalv}
INCLUDE 'LIB_ERR'
INCLUDE 'LIB_PAR'
\end{terminalv}

No distinction is drawn between ADAM and stand-alone versions of include
files because, in practice, they will always be the same (there being no
difficulty if some of the constants defined in these files are simply not
used by one or other versions of the library).

The naming of include files which do not fit into either of the above
categories is arbitrary.
However, logical names should always be provided for accessing them, and the
library prefix LIB\_ should be used at the start of each logical name.

\newpage
\appendix
\section{Summary}

This appendix summarises the definitions and files which should normally
form part of a Starlink subroutine library, indicating which features form a
mandatory part of the library naming conventions and which are optional.

\subsection{The SSC:STARTUP.COM File}

Definitions such as the following will normally be required in
SSC:STARTUP.COM (the Starlink system startup
file\footnote{Note, however, that Starlink software items which are
considered ``optional'' have their startup commands placed in a \emph{local\/} version of the Starlink STARTUP.COM and LOGIN.COM files which
reside in the LSSC: directory.}):

\begin{small}
\begin{terminalv}
$ DEFINE/SYSTEM LIBRARY_DIR    <disk>:<directory>         ! Library directory name
$ DEFINE/SYSTEM LIB_IMAGE      LIBRARY_DIR:LIB_IMAGE      ! Stand-alone shareable image
$ DEFINE/SYSTEM LIB_IMAGE_ADAM LIBRARY_DIR:LIB_IMAGE_ADAM ! ADAM shareable image
\end{terminalv}
\end{small}

These logical name definitions are mandatory, except that if either image
file does not exist, then the corresponding logical name should be left
undefined.
Any existing logical name definitions for the same files may remain in place.

The image files must reside in the library directory and the file names
given above are a recommendation for future use.

\subsection{The SSC:LOGIN.COM File}

A definition such as the following will normally be required in
SSC:LOGIN.COM (the Starlink login file\footnotemark[\thefootnote]):

\begin{terminalv}
$ LIB_DEV :== @LIBRARY_DIR:LIB_DEV
\end{terminalv}

This global symbol definition is mandatory unless the library is installed
in a form which does not permit software development, in which case it
should be left undefined.
Existing symbol definitions for the same purpose may remain in place.
If the value ``NOLOG'' is supplied as the first parameter of the startup
procedure, then any informational messages must be suppressed.
Such messages are otherwise optional.

The library startup file must reside in the library directory and the name
given above is a recommendation for future use.

\subsection{The Library Startup File}

Definitions such as the following will normally be required in the library
startup file which is executed by the \$LIB\_DEV command:

\begin{terminalv}
$ DEFINE/NOLOG LIB_LINK      LIBRARY_DIR:LIB_LINK      ! Stand-alone link options
$ DEFINE/NOLOG LIB_LINK_ADAM LIBRARY_DIR:LIB_LINK_ADAM ! ADAM link options
$ DEFINE/NOLOG LIB_ERR       LIBRARY_DIR:LIB_ERR       ! Error codes
$ DEFINE/NOLOG LIB_PAR       LIBRARY_DIR:LIB_PAR       ! Other constants
$ DEFINE/NOLOG LIB_?????     LIBRARY_DIR:LIB_?????     ! Any other include files
\end{terminalv}

These logical name definitions are mandatory, except that if any of the
files do not exist, then the corresponding name should be left undefined.
Existing logical name definitions for the same files may remain in place.

The files must reside in the library directory and the names given above are
a recommendation for future use.

\subsection{Overall Summary}

The following is an overall summary of the files which a library may need to
provide and the logical names or symbols to be used to refer to them:

\begin{center}
\begin{tabular}{|l|l|l|l|}
\hline
\textbf{File/Directory} & \textbf{Purpose} & \textbf{Name/Symbol} & \textbf{Defined in...} \\
\hline \hline
\texttt{<disk>:<directory>} & \emph{Library directory}
   & \texttt{LIBRARY\_DIR} & \texttt{SSC:STARTUP.COM} \\
\texttt{LIB\_IMAGE.EXE} & \emph{Stand-alone shareable image}
   & \texttt{LIB\_IMAGE} & \texttt{SSC:STARTUP.COM} \\
\texttt{LIB\_IMAGE\_ADAM.EXE} & \emph{ADAM shareable image}
   & \texttt{LIB\_IMAGE\_ADAM} & \texttt{SSC:STARTUP.COM} \\
\texttt{LIB\_DEV.COM} & \emph{Library startup file}
   & \texttt{LIB\_DEV} & \texttt{SSC:LOGIN.COM} \\
\texttt{LIB\_LINK.OPT} & \emph{Stand-alone link options file}
   & \texttt{LIB\_LINK} & \texttt{LIBRARY\_DIR:LIB\_DEV.COM} \\
\texttt{LIB\_LINK\_ADAM.OPT} & \emph{ADAM link options file}
   & \texttt{LIB\_LINK\_ADAM} & \texttt{LIBRARY\_DIR:LIB\_DEV.COM} \\
\texttt{LIB\_ERR.FOR} & \emph{Error codes (include file)}
   & \texttt{LIB\_ERR} & \texttt{LIBRARY\_DIR:LIB\_DEV.COM} \\
\texttt{LIB\_PAR.FOR} & \emph{Other constants (include file)}
   & \texttt{LIB\_PAR} & \texttt{LIBRARY\_DIR:LIB\_DEV.COM} \\
\texttt{LIB\_?????.FOR} & \emph{Any other include files}
   & \texttt{LIB\_?????} & \texttt{LIBRARY\_DIR:LIB\_DEV.COM} \\
\hline
\end{tabular}
\end{center}


\end{document}
