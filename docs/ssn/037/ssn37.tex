\documentclass[11pt]{article}
\pagestyle{myheadings}

% -----------------------------------------------------------------------------
% ? Document identification
\newcommand{\stardoccategory}  {Starlink System Note}
\newcommand{\stardocinitials}  {SSN}
\newcommand{\stardocnumber}    {37.1}
\newcommand{\stardocsource}    {ssn\stardocnumber}
\newcommand{\stardocauthors}   {C A Clayton}
\newcommand{\stardocdate}      {30th May 1996}
\newcommand{\stardoctitle}     {Starlink Security}
% ? End of document identification
% -----------------------------------------------------------------------------

\newcommand{\stardocname}{\stardocinitials /\stardocnumber}
\markright{\stardocname}
\setlength{\textwidth}{160mm}
\setlength{\textheight}{230mm}
\setlength{\topmargin}{-2mm}
\setlength{\oddsidemargin}{0mm}
\setlength{\evensidemargin}{0mm}
\setlength{\parindent}{0mm}
\setlength{\parskip}{\medskipamount}
\setlength{\unitlength}{1mm}

% -----------------------------------------------------------------------------
%  Hypertext definitions.
%  ======================
%  These are used by the LaTeX2HTML translator in conjunction with star2html.

%  Comment.sty: version 2.0, 19 June 1992
%  Selectively in/exclude pieces of text.
%
%  Author
%    Victor Eijkhout                                      <eijkhout@cs.utk.edu>
%    Department of Computer Science
%    University Tennessee at Knoxville
%    104 Ayres Hall
%    Knoxville, TN 37996
%    USA

%  Do not remove the %begin{latexonly} and %end{latexonly} lines (used by 
%  star2html to signify raw TeX that latex2html cannot process).
%begin{latexonly}
\makeatletter
\def\makeinnocent#1{\catcode`#1=12 }
\def\csarg#1#2{\expandafter#1\csname#2\endcsname}

\def\ThrowAwayComment#1{\begingroup
    \def\CurrentComment{#1}%
    \let\do\makeinnocent \dospecials
    \makeinnocent\^^L% and whatever other special cases
    \endlinechar`\^^M \catcode`\^^M=12 \xComment}
{\catcode`\^^M=12 \endlinechar=-1 %
 \gdef\xComment#1^^M{\def\test{#1}
      \csarg\ifx{PlainEnd\CurrentComment Test}\test
          \let\html@next\endgroup
      \else \csarg\ifx{LaLaEnd\CurrentComment Test}\test
            \edef\html@next{\endgroup\noexpand\end{\CurrentComment}}
      \else \let\html@next\xComment
      \fi \fi \html@next}
}
\makeatother

\def\includecomment
 #1{\expandafter\def\csname#1\endcsname{}%
    \expandafter\def\csname end#1\endcsname{}}
\def\excludecomment
 #1{\expandafter\def\csname#1\endcsname{\ThrowAwayComment{#1}}%
    {\escapechar=-1\relax
     \csarg\xdef{PlainEnd#1Test}{\string\\end#1}%
     \csarg\xdef{LaLaEnd#1Test}{\string\\end\string\{#1\string\}}%
    }}

%  Define environments that ignore their contents.
\excludecomment{comment}
\excludecomment{rawhtml}
\excludecomment{htmlonly}

%  Hypertext commands etc. This is a condensed version of the html.sty
%  file supplied with LaTeX2HTML by: Nikos Drakos <nikos@cbl.leeds.ac.uk> &
%  Jelle van Zeijl <jvzeijl@isou17.estec.esa.nl>. The LaTeX2HTML documentation
%  should be consulted about all commands (and the environments defined above)
%  except \xref and \xlabel which are Starlink specific.

\newcommand{\htmladdnormallinkfoot}[2]{#1\footnote{#2}}
\newcommand{\htmladdnormallink}[2]{#1}
\newcommand{\htmladdimg}[1]{}
\newenvironment{latexonly}{}{}
\newcommand{\hyperref}[4]{#2\ref{#4}#3}
\newcommand{\htmlref}[2]{#1}
\newcommand{\htmlimage}[1]{}
\newcommand{\htmladdtonavigation}[1]{}

% Define commands for HTML-only or LaTeX-only text.
\newcommand{\html}[1]{}
\newcommand{\latex}[1]{#1}

% Use latex2html 98.2.
\newcommand{\latexhtml}[2]{#1}

%  Starlink cross-references and labels.
\newcommand{\xref}[3]{#1}
\newcommand{\xlabel}[1]{}

%  LaTeX2HTML symbol.
\newcommand{\latextohtml}{{\bf LaTeX}{2}{\tt{HTML}}}

%  Define command to re-centre underscore for Latex and leave as normal
%  for HTML (severe problems with \_ in tabbing environments and \_\_
%  generally otherwise).
\newcommand{\setunderscore}{\renewcommand{\_}{{\tt\symbol{95}}}}
\latex{\setunderscore}

% -----------------------------------------------------------------------------
%  Debugging.
%  =========
%  Remove % from the following to debug links in the HTML version using Latex.

% \newcommand{\hotlink}[2]{\fbox{\begin{tabular}[t]{@{}c@{}}#1\\\hline{\footnotesize #2}\end{tabular}}}
% \renewcommand{\htmladdnormallinkfoot}[2]{\hotlink{#1}{#2}}
% \renewcommand{\htmladdnormallink}[2]{\hotlink{#1}{#2}}
% \renewcommand{\hyperref}[4]{\hotlink{#1}{\S\ref{#4}}}
% \renewcommand{\htmlref}[2]{\hotlink{#1}{\S\ref{#2}}}
% \renewcommand{\xref}[3]{\hotlink{#1}{#2 -- #3}}
%end{latexonly}
% -----------------------------------------------------------------------------
% ? Document-specific \newcommand or \newenvironment commands.
% ? End of document-specific commands
% -----------------------------------------------------------------------------
%  Title Page.
%  ===========
\renewcommand{\thepage}{\roman{page}}
\begin{document}
\thispagestyle{empty}

%  Latex document header.
%  ======================
\begin{latexonly}
   CCLRC / {\sc Rutherford Appleton Laboratory} \hfill {\bf \stardocname}\\
   {\large Particle Physics \& Astronomy Research Council}\\
   {\large Starlink Project\\}
   {\large \stardoccategory\ \stardocnumber}
   \begin{flushright}
   \stardocauthors\\
   \stardocdate
   \end{flushright}
   \vspace{-4mm}
   \rule{\textwidth}{0.5mm}
   \vspace{5mm}
   \begin{center}
   {\Large\bf \stardoctitle}
   \end{center}
   \vspace{5mm}

% ? Heading for abstract if used.
%   \vspace{10mm}
%   \begin{center}
%      {\Large\bf Abstract}
%   \end{center}
% ? End of heading for abstract.
\end{latexonly}

%  HTML documentation header.
%  ==========================
\begin{htmlonly}
   \xlabel{}
   \begin{rawhtml} <H1> \end{rawhtml}
      \stardoctitle
   \begin{rawhtml} </H1> \end{rawhtml}

% ? Add picture here if required.
% ? End of picture

   \begin{rawhtml} <P> <I> \end{rawhtml}
   \stardoccategory\ \stardocnumber \\
   \stardocauthors \\
   \stardocdate
   \begin{rawhtml} </I> </P> <H3> \end{rawhtml}
      \htmladdnormallink{CCLRC}{http://www.cclrc.ac.uk} /
      \htmladdnormallink{Rutherford Appleton Laboratory}
                        {http://www.cclrc.ac.uk/ral} \\
      \htmladdnormallink{Particle Physics \& Astronomy Research Council}
                        {http://www.pparc.ac.uk} \\
   \begin{rawhtml} </H3> <H2> \end{rawhtml}
      \htmladdnormallink{Starlink Project}{http://star-www.rl.ac.uk/}
   \begin{rawhtml} </H2> \end{rawhtml}
   \htmladdnormallink{\htmladdimg{source.gif} Retrieve hardcopy}
      {http://star-www.rl.ac.uk/cgi-bin/hcserver?\stardocsource}\\

%  HTML document table of contents. 
%  ================================
%  Add table of contents header and a navigation button to return to this 
%  point in the document (this should always go before the abstract \section). 
  \label{stardoccontents}
  \begin{rawhtml} 
    <HR>
    <H2>Contents</H2>
  \end{rawhtml}
  \htmladdtonavigation{\htmlref{\htmladdimg{contents_motif.gif}}
        {stardoccontents}}

% ? New section for abstract if used.
%  \section{\xlabel{abstract}Abstract}
% ? End of new section for abstract

\end{htmlonly}

% -----------------------------------------------------------------------------
% ? Document Abstract. (if used)
%  ==================
% ? End of document abstract
% -----------------------------------------------------------------------------
% ? Latex document Table of Contents (if used).
%  ===========================================
% \newpage
\begin{latexonly}
   \setlength{\parskip}{0mm}
   \tableofcontents
   \setlength{\parskip}{\medskipamount}
   \markright{\stardocname}
\end{latexonly}
% ? End of Latex document table of contents
% -----------------------------------------------------------------------------
\newpage
\renewcommand{\thepage}{\arabic{page}}
% \setcounter{page}{1}

\section{\label{introduction}\xlabel{introduction}Introduction} 

There is already a very large amount of documentation on Unix security,
indeed too much for your average site manager to find time to read. The
purpose of this document is not to duplicate that documentation but
instead to discuss just a few topics which are considered to be
particularly relevant to Starlink sites and which are not covered to an
appropriate depth elsewhere. For example, issues relating to network
security are well documented elsewhere but concerns such as disaster
recovery are not. References to the more complete guides on Unix
security can be found on the Starlink security WWW page at URL:

\begin{quote}{\tt
\htmladdnormallink{http://star-www.rl.ac.uk/\~{}cac/security}
{http://star-www.rl.ac.uk/\~{}cac/security}}
\end{quote}

The references listed there are not repeated here in order to extend the
lifetime of this document. Suggestions for new links from these WWW pages
are always welcome.

The issues covered here are those about which queries have been
received from Site Managers. I have also tried to make this guide short
and pragmatic, listing practical tips and actions without a full blown
description of the background, which can be found elsewhere. It is
assumed that you know how to use basic security related commands such
as {\tt chmod}. This guide is also not intended to be a ``Hacker's
Handbook''. Hence, I have been deliberately vague in several
areas. For further information, contact the author.

There is no such thing as ``total security''. In our environment,
setting security too tight can make life very difficult for users, to
the extent of being counter-productive, not to mention making you very
unpopular. However, we must make our system as secure as possible to
avoid the worse alternative of having your system compromised.  A good
strategy is to set security as tight as you can without significantly
affecting the usability of the system and to check the file system
regularly for signs of security lapses. Both of these topics are
discussed in the present paper. You might like to let students at your
site know you do this checking to dissuade them of trying anything.
These checks can be run regularly and automatically using {\tt cron}
and any lapses can mailed back to you. Hence, security lapse monitoring
need not involve you in any extra work once it has been set up.

Security is not just about keeping out hackers. It also covers security
against user error and hardware failures (\emph{e.g.}, backups), and theft.
A section of disaster recovery is included in this paper.

I have attempted to make this document Unix-flavour neutral. In a few
places I have given examples for Solaris simply because that is what I
am familiar with. However, most of the contents apply equally to any
other flavour of Unix, such as Digital Unix or Linux.

% I have also assumed that the potential hacker is not a current user
% of your system but an outside agent. In practise this may not be the case
% but your users are likely to attempt non-destructive hacking. The author
% can provide advise on how to guard against being hacked by current
% users of the system if you feel this is necessary.

\subsection{What is in this document?}

\begin{itemize}
\item Some basic steps for securing your system
\item A system for checking for security lapses
\item Disaster recovery procedures
\item Trusted host security issues
\item References to existing documentation and security tools (via WWW)
\end{itemize}

\subsection{What is not in this document?}

\begin{itemize}
\item Basic Unix Security
\item Descriptions of security loopholes
\item Physical security of hardware
\item Many network security issues
\end{itemize}

%------------------------------------------------------------------------

\section{\label{securing_your_system}\xlabel{securing_your_system}Securing your system}

This is not a description on how to weld your system shut. 
Such actions are likely to reduce the usability of your system
to the extent that your users complain. Instead emphasis is placed
on plugging up loopholes and avoiding bad practise. 

The following list is far from complete. Only a small selection of the more
important issues are considered. Site managers wishing further
recommendations on securing their system are referred to the
documentation available via the \htmladdnormallink{Starlink security WWW page}{http://star-www.rl.ac.uk/\~{}cac/security}.  For
example, the Sun White Paper ``A practical Guide to Solaris Security''
is a must for all Unix managers, Solaris or otherwise.


\subsection{Security patches}

The news group {\tt comp.security.announce} contains announcements
about security problems (and fixes). Install all security related
patches for the OS that you are running. Sun also maintain a list of
security related patches on their WWW SunSolve service (link available
from the \htmladdnormallink{Starlink security WWW pages).}{http://star-www.rl.ac.uk/\~{}cac/security}

\subsection{Single user boot security}

Ensure that your systems require that a password must be given when
booted single user, if possible. Additionally, remote login as root
should be disabled.  Root logins can still be accomplished, but users
must first login as a user and then {\tt su} to root. This is done for
logging and accountability purposes. Both are the default on Solaris
2.x but the former is not for Digital Unix. 

\subsection{Passwords}

\subsubsection{Cracking passwords}

It is not difficult for a potential hacker to get access to your
password information. I will not go into details here for obvious
reasons. This information can then be ``cracked'' using even a modest
computer in a relatively short amount of time. The cracking procedure
usually involves comparing your encrypted password with encryptions of
dictionary words and variations on this theme. Thus, easy passwords
such as ``galaxy'' can be discovered in this way almost instantly. A
less crackable password is one containing gibberish such as
non-dictionary words and one or more non-alphabetic characters.

We recommend that you run a cracking program on your own password file
(for the sole purpose of ensuring that your users don't have easy to
crack passwords) and asking users with passwords which you crack to
change them for more sophisticated ones. Details of the availability of
such software can be found on the
\htmladdnormallink{WWW.}{http://star-www.rl.ac.uk/\~{}cac/security}

These WWW pages also give access to a script for producing passwords
which have proven to be uncrackable but which are still memorable.
This is done by constructing a password from parts of other words.  The
results is very secure and yet easy to remember since it is
pronounceable (not that you would ever speak it!) and the words used
are given at the time of generation.

You should be aware that programs for cracking passwords are not very
complex. You can do it in 32 lines of C. More sophisticated versions are
considerably longer but can crack even some very obscure passwords.

\subsubsection{Old and new accounts} 

This is an area for potential security lapses. Make sure that new
accounts have passwords on them as soon as they are created and ensure
that this temporary password is unique and non-trivial. Although
you may ask the user to log on and change it immediately, some users
continue to use the password that you set for them. Hence, you could
end up with several accounts with the password ``newuser'' or something
similar.

When users leave, their accounts should be removed or at the very least
disabled. If the remote user needs to use it again, it can easily be
re-enabled. Old accounts are a risk partly because the password on them
is not being changed regularly and partly because users no longer
at your site still have access to your system possibly unnecessarily.

\subsubsection{Password management features}

You might like to consider using any password management features
offered by your operating system. For example, Solaris 2.x allows
password aging (enables the administrator to set an expiration date for
passwords), password qualification (checks that the password passed
certain simple criteria to make it more difficult for someone else to
guess) and account expiration. One can also disallow the reuse of an
old password.

\subsection {NFS}

Network security is a vast topic in itself and is covered already in
other documents (see WWW pages). It is recommended that you read one of
those documents.  Here I restrict myself to pointing out the importance of
one particular NFS mount option. You should mount
user disks with the {\tt nosuid} option (ignore and {\tt `s'} bits set
in the file permissions). Why is this important? If you were to mount
up a disk from a remote machine with an untrusted root account, you
could end up allowing users on your machine to run scripts as root.

\subsection{File system permissions}

\subsubsection{File system permission gotchas}
Here are a couple of ``gotchas'' which are worth bearing in mind.

\begin{itemize}
\item To delete a file, you just need ``write access'' to the directory
in which it is in and not the file. 
\item Always send a new user a welcoming mail message. The mail directory
has public write access (to allow mail delivery) but consequently if
a user does not already have a mail file there, then another user could
create that file before they receive any mail and place a ``spoof'' message 
in it.
\end{itemize}

\subsubsection{The dreaded {\tt `s'} bit}

Any user is allowed to alter the user accounts file {\tt /etc/passwd}
by using the {\tt passwd} command even though this file is at the heart
of your system's security. This is achieved through the {\tt `s'} bit being
set on the program {\tt /usr/bin/passwd}. Setting the {\tt `s'} bit on
certain files is vital to the operation of your system but this
facility is unfortunately well known about and potentially one of the
most dangerous security problems on any Unix system.

Later in this paper you will be shown how to look for files with the
{\tt `s'} bit set.

A user with access to the root account for just a moment can use this
facility to completely compromise your system. Similarly, be careful
when running commands for a user as root and either have `.' last in
your path or not at all, to avoid root running a user's enhanced
version of a system command instead of the real version. The rules of
thumb are:

\begin{itemize}
\item Don't run code as root unless you have to
\item Don't run setuid programs unless you have to. If it is necessary
to run such a program or script, take a look at it first. 
\end{itemize}

If for some reason you think you need to write a setuid script, try and think
of a work around. If you cannot then don't make it a script, as many shells
can be forced to go interactive. It is far better to write a program.

\subsection{The Prom password}

On most flavours of Unix, it is possible to boot off of a CD-ROM and
gain root access to a system. This is now becoming common knowledge.
There are several steps you can take to avoid this happening at your
site:

\begin{enumerate}
\item Remove CD-ROM drives from systems (impractical)
\item Prevent users getting access to OS installation CD-ROMs (impractical)
\item Set a prom password so that users cannot boot off of a CD-ROM drive
without knowledge of that password.
This can be done on a Sun either on the console or via the {\tt eeprom}
command:

\begin{quote}
\verb+ok> setenv security-mode command+\\
or\\
\verb+# eeprom security-mode command+
\end{quote}

If you forget the prom password, you can reset it via the {\tt eeprom}
program.  If you forget the prom password and cannot get into your
system as root, you are in deep trouble.

\end{enumerate}

\subsection{Preventing root access to documents}

Suppose you have a sensitive document that you don't want anyone else
to read (\emph{e.g.}, the scientific discovery of the century, or more likely
a staff report). You can use {\tt chmod} to ensure that no-one but
yourself and root can read it. If there are other users of the system
with knowledge of the root account, then your document is not secure. 
Under Solaris 2.5 it is possible to easily encrypt your documents so that
a password is required in order to render them readable. One option is
to use {\tt vi -x}. This prompts for a password and the resultant file
(and temporary file in {\tt /tmp}) are encrypted. An alternative is to use
the {\tt crypt} command to encrypt and decrypt ascii files using
a password. See the {\tt vi} and {\tt crypt} man pages for more details.
These facilities are, unfortunately, not available under Digital Unix
nor Linux.

\subsection{C2 security}

The U.S. government's security classification system called Orange Book 
defines a series of divisions for D (least secure) to A (most secure)
and levels within those divisions. Division D (minimal protection)
covers systems such as PCs running DOS. 

Division C, level 1 (discretionary security protection) is where most
Unix systems come in (\emph{i.e.}, they have the concept of users, with users
having control over objects they own).  Division C, level 2 (controlled
access protection) adds auditing and increased validations (and
increased system loading). An example of a C2 implementation would be
Solaris 2.x's SunShield. Note that some C2 implementations are just the
bare minimum to get the government certification and hence they work
but the data provided can come in a very raw form. We do not recommend
the use of C2 on Starlink machines.

\subsection {Anonymous {\tt ftp} and {\tt tftp}}

By default, anonymous {\tt ftp} is disabled. A full description of the setup
and security mechanisms may be found in the {\tt ftpd} manual page and in
Starlink Forum.

{\tt tftp} (trivial file transfer protocol) must be set up with some care.
Starlink sites may wish to use this system for booting X-terminals.  If
so, be sure to set it up as indicated in the X-terminal documentation
to avoid introducing a security loophole into your system. Use NFS
instead if possible.  If you must run the {\tt tftp} daemon (\verb+tftpd+),
configure it to run within a particular directory by specifying the
{\tt -s /tftpboot} in the {\tt /etc/inet/inetd.conf} file (Solaris) or
{\tt /etc/inetd.conf} file (Digital Unix).

%------------------------------------------------------------------------

\section{\label{detecting_security_lapses}\xlabel{detecting_security_lapses}Detecting security lapses}

What might a hacker do once he/she has access to your system? Here are
a some of the tools of the trade:

\begin{description}
\item {\bf Holes} -- 
Just poor security. Some flavours of Unix as supplied by the manufacturer 
are full of holes. 
\item {\bf Trojan Horses} -- 
Where something is placed on your system for activation later -- either at the
request of the person who placed it, or by a user (or you) accidentally falling over it.
\item {\bf Worms} -- 
Where a system gains illegal access to your workstation over the network,
copies itself onto your system, and start trying to gain access to the next
system. 
\item {\bf Viruses} -- 
Where a program `fragment' attaches itself to another program and keeps replicating within further applications as they are run. These are almost unheard of 
on Unix.
\item {\bf Mimics/spoofs} -- 
Programs that appear to be what they are not. Spoofing a login session 
on Unix requires a trivial 13 line script and can be run by any user and
there is very little that you can do about it. Only users who already have
an account on the system can do this but it might appeal to students
wanting to pinch each others passwords.
Use of X-terminals and X-based login sessions prevents this. 
\end{description}

How do you know if your security has been breached? What should you do
if you receive a CERT announcement that many machines have been
hacked via an operating system loophole? Here is a list of things to 
look for on your system that may indicate that you have received a 
visit from a hacker:

\begin{enumerate}
\item Look for files called {\tt ...}, {\tt .\^E}, {\tt" "}
\item Look for a {\tt .exrc} file in {\tt /tmp}
\item Look for extra lines added to the end of {\tt /etc/passwd}
\item Look for shells with the {\tt `s'} bit set (see below)
\item Look for other accounts with UID {\tt `0'} 
\end{enumerate}

This can be done automatically via scripts run from {\tt cron} and the
results mailed back to you. If you really suspect a hacker, you should also
run this at irregular intervals  -- the hacker may have looked at your
{\tt crontab} file and may hide himself when a scan is due.

Here are a selection of commands to help you hunt for hackers.

\subsection{Monitoring files -- times}

All files and directories have three separate times associated with them under
Unix. The commands refer to Solaris 2.x and may be slightly different
on other flavours of Unix.

\begin{itemize}
\item file ACCESS time (\emph{e.g.}, file reads):
\begin{quote} {\tt ls -lu} \end{quote}
\item file CHANGE time (\emph{e.g.}, {\tt chmod, chown, chgrp}):
\begin{quote} {\tt ls -lc} \end{quote}
\item file MODIFICATION time (\emph{e.g.}, file edited):
\begin{quote} {\tt ls -l} \end{quote}
\end{itemize}

The file MODIFICATION time is the most recent time that the contents of
the file were written or altered -- for example by editing. The file
CHANGE time is the most recent time that the file header was altered --
so this will be changed by both editing {\em and} commands like {\tt chmod}.
The distinction is important since it is possible for a user to alter
the MODIFICATION time of a file so that it looks old but the CHANGE
time gives it away.

\subsection{Locating files by criteria -- sweeps}

The excellent {\tt find} command may be used to search all (or part)
of the Unix file tree for files or directories which meet certain
criteria. The full syntax of the command can be found in the man pages
but here are a few useful examples:

\begin{itemize}

\item Search all local disks for setuid files (files with the dreaded
{\tt `s'} bit set):
\begin{quote} {\tt find / -mount  -type f -perm -4000 -ls} \end{quote}
Some system commands are supposed to have this permission (\emph{e.g.},
{\tt passwd} but not, for example, shells (\emph{e.g.}, {\tt csh}).

\item Search from the current directory downwards for files 
which have been changed in the last 3 days:
\begin{quote} {\tt find . -ctime -3 -print} \end{quote}
This command is also useful for finding out who has recently filled up a disk.

\item Search for setuid files which have been modified in the last 3 days
(there should be none!):
\begin{quote} {\tt find / -mount -type f -perm -4000 -mtime -3 -ls} \end{quote} 
This is a good one to have as a {\tt cron} job.

\item Look for files recently altered in {\tt /usr}:
\begin{quote} {\tt find /usr -type f -mtime -s -ls} \end{quote}
This too is a good one to have as a {\tt cron} job.

\item Search for publicly writable files and directories on local
disks:
\begin{quote}
{\verb+find / -mount -perm -2 \( -type f -o -type d \) -print+}
\end{quote}

\item Look for writable block device files and character device files:

\begin{quote} {\tt find / -mount -type b -perm -2 -print} \end{quote}
\begin{quote} {\tt find / -mount -type c -perm -s print | more} \end{quote}

Look for funny looking devices. Why? A hacker with temporary
access to the root account can create a new device with the same major
and minor device numbers as, say, the system disk and then make himself
the owner of that new device. He could then bypass all the Unix file
system security and get at the raw system disk from his own account. 

\end{itemize}

\subsection{Monitoring users}

System administrators are not permitted to monitor users actions
without their knowledge. 

However, you should be aware that Unix has plenty of commands for
monitoring system usage which might be invaluable when chasing a
hacker.

\begin{itemize}
\item {\tt last} --- (to check recent logins)
\item {\tt lastcomm} --- (to check all recently run commands)
\item {\tt view /var/adm/messages} (to check log of some recent system events)
\item {\tt grep su /var/adm/messages} (to check who been using the {\tt su} command)
\end{itemize}

To ``tune'' what sort of messages go into {\tt /var/adm/messages}, use the 
file {\tt /etc/syslog.conf}. You can also divert messages to a
private file. 

%------------------------------------------------------------------------

\section{\label{disaster_recovery}\xlabel{disaster_recovery}Disaster Recovery}

Imagine that you come in one morning and the majority of your hardware
has been stolen (remember that this has already happened at one
Starlink site). What steps would you have had to have taken prior to the
theft to ensure that you can replace the functionality of your system?

\begin{enumerate}

\item Are you insured? Remember that your Starlink site contract (if
applicable) requires you to indemnify Starlink against any loss of
its equipment \emph{i.e.}, it is your institution's responsibility to
replace the equipment. Check with your Site Chairman that you are
properly covered.  You might also like to ask what the insurance
covers:

\begin{enumerate}

\item Cover of specific hardware to a certain value?

\item Cover for the cost of getting everything running again?

\item Cover for consequential loss \emph{i.e.}, loss of data?

\item Cover for hiring equipment temporarily until permanent replacements
are bought? Starlink will of course try to help in any way possible to 
provide stop-gap hardware but in practise, we seldom have unused hardware
available.

\end{enumerate}

\item Is your hardware properly maintained? Starlink maintains most of
the hardware at most sites but there are exceptions, particularly
new sites which arrange their own maintenance via starter grants. 
Even if Starlink is nominally maintaining your hardware, be sure that
you have actually told RAL about your new equipment and its serial numbers
and that all your equipment is actually on a maintenance schedule 
(see \xref{SGP/25}{sgp25}{} for further details). 


\item Is your system regularly backed up and are the backup tapes
safe?  A detailed discussion on backup policies is beyond the scope of
the present document but you absolutely must backup
system and permanent user space. Keep a copy of a recent set of backup
tapes away from the location of your system. Imagine if the whole
building burnt down or, more likely, a thief steals your system with
the backup tape still in the drive. If you prefer to keep your backups
in a fire safe near to the systems, make sure that it is not the sort
that could be easily removed by a thief when he takes your machine away!
It is probably wisest to keep the latest backups locally (for easy access
for file restoration) and an older set off site or in another building
in case of emergency.

\item Do you have detailed records of your current system setup?
If your entire system was stolen/destroyed, do you have sufficient
information to re-construct it? Here is a list of the sort of information 
you would need:

\begin{enumerate}

\item Inventory list with model names and serial numbers for every piece
of equipment. You will need this not only so that you know what you need to 
order in the way of replacements but also because this information will
almost certainly be required for your insurance claim. 

\item Configuration diagram. You will need to know how all these pieces 
of equipment fitted together. Which disks were on which machines?
What were the disks called, or the machines for that matter. 

\item Detailed system-by-system configuration information. For example,
you will need the hostid if you have commercial software tied to that machine
(For example, Sun can give you a new machine with a specified hostid). Also
you will need listings of disk partitions so that you can restore your
backups. 

\begin{enumerate}

\item Machine details (Hostname, model type, \emph{etc.})
\begin{quote} {\tt uname -a} \end{quote}

\item hostid 
\begin{quote} {\tt hostid} \end{quote}

\item eeprom details 
\begin{quote} {\tt /usr/sbin/eeprom} (Solaris 2.x) \end{quote}

\item Processor details 
\begin{quote} {\tt /usr/sbin/psrinfo -v} \end{quote}

\item Memory details 
\begin{quote} {\tt /usr/sbin/prtconf | grep Memory} \end{quote}

\item Disk information  
\begin{quote} {\tt /usr/sbin/prtvtoc} {\it raw disk} (Solaris 2.x) \end{quote}

\item showrev information, including patch level 
\begin{quote} {\tt /usr/bin/showrev -a} (Solaris 2.x) \end{quote}

\end{enumerate}

Scripts to gather most of this information for you for both Solaris and
Digital Unix are available via the \htmladdnormallink{Starlink 
security WWW pages.}{http://star-www.rl.ac.uk/\~{}cac/security}

Example output from such a script showing, the sort of information you
may need for system recovery, can be found in 
\htmlref{Appendix A}{system_recovery}. Make sure that 
you keep a paper copy of this file!

\item Backups of your file system. Make sure that you know which are the last
set of tapes, when they were written and how to restore from them.

\end{enumerate}

\end{enumerate}

%----------------------------------------------------------------------

\section{\label{trusted_host_security}\xlabel{trusted_host_security}Trusted host security}

\subsection {{\tt .rhosts} and {\tt /etc/hosts.equiv}}

This topic is discussed in gory detail in almost every Unix systems
administration or security guide and hence I will not repeat that
information again here (a brief summary is given in 
\htmlref{Appendix B}{app_trusted_host_security}
for your convenience). 
Instead, I will make some general comments and give a few
examples of why these files can be dangerous.

Some people argue that these files should not be used at all due to
their potential security risks: {\tt .rhosts} files are bad once your
system has been hacked because the hacker gets immediate access to 
other systems.

However, I personally don't see a problem using {\tt .rhosts} to allow
transparent access between machines which are tightly-coupled together,
\emph{i.e.},  share the same password information, as long as the
intrusion was via a poor password.

If, however, the intruder gains access to a particular machine via
another method (\emph{e.g.}, user left him/herself logged on, network
loophole, \emph{etc.},) then the {\tt .rhosts} file might grant access
to other machines which the intruder might not otherwise have had.

I think that the biggest worry is that users control their own {\tt .rhosts}
file and can (through error or otherwise) configure it so that
non-accredited users can gain access \emph{i.e.}, without any reference to
you the system manager. For example, one Starlink site
reported that a user trusted hosts from several domains in Spain
with a wildcard in the user field because ``he had a different 
username there''. In practise, this meant that this user had given access
to the system to {\em any} user on those remote Spanish systems.  

I use a {\tt .rhosts} file within our own cluster (\emph{i.e.}, machines
which share the same password information) but would not use it
between, for example, Starlink sites.

The following examples refer to files under the control of the system
administrator. 

\begin{description}

\item [Lethal example 1] -- A remote host has a solitary {\tt +}
associated with it in {\tt /.rhosts}.  Anyone who can gain root access
to ANY workstation (it does not have to be a known one, \emph{i.e.},
one in {\tt /etc/hosts} or equivalent \verb|NIS/NIS+| maps and tables), can
read and write with {\tt rcp}. An intruder could pull your passwd file,
edit it and put it back without leave any trace of his/her presence in
{\tt /var/adm/messages}.  {\tt lastcomm} (if it is running) will only
note that {\tt rcp} ran, and from where but make no note of which file
was involved!

\item [Lethal example 2] -- {\tt /.rhosts} contains ``\verb|+ +|''.
Any user from any machine can log in as root without a password!

\item [Lethal example 3] -- {\tt /etc/hosts.equiv} on machine {\tt littlesun}
contains ``\verb|bigsun +|''. Here {\tt any} user on machine {\tt bigsun} can
log in as {\em any other} user on {\tt littlesun} without giving a password!

\end{description}

If you are going to use trusted hosts, trust should be given to a user
as well as a host. The {\tt /.rhosts}, {\tt /etc/hosts.equiv} and {\tt
.rhosts} file should always contain two entries per line, one entry for
the host and an additional entry for the particular user that is to be
trusted for the host. Hence, if you want to trust user {\tt cac} from
host {\tt bigsun}, the entry should be:

\begin{quote}{\tt bigsun cac} \end{quote}

%------------------------------------------------------------------------

\newpage
\appendix

\section{\label{system_recovery}\xlabel{system_recovery}System recovery information example}

\begin{quote}
\begin{verbatim}

System report. Information for holding with backups
----------------------------------------------------------
machine details
SunOS rlssp0 5.5 Generic sun4m sparc SUNW,SPARCstation-10
Thu May 23 11:52:20 BST 1996
----------------------------------------------------------
eeprom details
tpe-link-test?=true
output-device=screen
input-device=keyboard
sbus-probe-list=f0123
keyboard-click?=false
keymap: data not available.
ttyb-rts-dtr-off=false
ttyb-ignore-cd=true
ttya-rts-dtr-off=false
ttya-ignore-cd=false
ttyb-mode=9600,8,n,1,-
ttya-mode=9600,8,n,1,-
fcode-debug?=false
diag-file: data not available.
diag-device=net
boot-file: data not available.
boot-device=disk
auto-boot?=true
watchdog-reboot?=false
local-mac-address?=false
screen-#columns=80
screen-#rows=34
selftest-#megs=1
scsi-initiator-id=7
use-nvramrc?=false
nvramrc: data not available.
sunmon-compat?=false
security-mode=none
security-password: data not available.
security-#badlogins=0
oem-logo: data not available.
oem-logo?=false
oem-banner: data not available.
oem-banner?=false
hardware-revision: data not available.
last-hardware-update: data not available.
testarea=0
mfg-switch?=false
diag-switch?=false
name=options
----------------------------------------------------------
hostid details
726053b6
----------------------------------------------------------
Processor details
Status of processor 0 as of: 05/23/96 11:52:22
  Processor has been on-line since 05/03/96 16:42:23.
  The sparc processor operates at 50 MHz,
	and has a sparc floating point processor.
----------------------------------------------------------
Showrev information, including patch level
Hostname: rlssp0
Hostid: 726053b6
Release: 5.5
Kernel architecture: sun4m
Application architecture: sparc
Hardware provider: Sun_Microsystems
Domain: ral_project.starlink.uk
Kernel version: SunOS 5.5 Generic November 1995

OpenWindows version: 
OpenWindows Version 3.5 29 September 1995
Patch: 101911-01  Obsoletes:   Packages: SPROf77.2 3.0.1,PATCH=01, 
   SPROlang.2 3.0.1,PATCH=01
----------------------------------------------------------
Disk information
details for disc 1
* /dev/dsk/c0t1d0s2 partition map
*
* Dimensions:
*     512 bytes/sector
*      72 sectors/track
*      14 tracks/cylinder
*    1008 sectors/cylinder
*    2038 cylinders
*    2036 accessible cylinders
*
* Flags:
*   1: unmountable
*  10: read-only
*
*                          First     Sector    Last
* Partition  Tag  Flags    Sector     Count    Sector  Mount Directory
       0      0    00          0   1919232   1919231
       1      0    00    1919232    133056   2052287
       2      0    00          0   2052288   2052287
----------------------------------------------------------
details for disc 2
* /dev/dsk/c0t2d0s2 partition map
*
* Dimensions:
*     512 bytes/sector
*      96 sectors/track
*      15 tracks/cylinder
*    1440 sectors/cylinder
*    2577 cylinders
*    2574 accessible cylinders
*
* Flags:
*   1: unmountable
*  10: read-only
*
*                          First     Sector    Last
* Partition  Tag  Flags    Sector     Count    Sector  Mount Directory
       0      0    00          0     57600     57599
       1      0    00      57600    194400    251999
       2      0    00          0   3706560   3706559   /soft1
       6      0    00     252000   3454560   3706559
----------------------------------------------------------
details for disc 3
* /dev/dsk/c0t3d0s2 partition map
*
* Dimensions:
*     512 bytes/sector
*      80 sectors/track
*       9 tracks/cylinder
*     720 sectors/cylinder
*    2500 cylinders
*    1151 accessible cylinders
*
* Flags:
*   1: unmountable
*  10: read-only
*
*                          First     Sector    Last
* Partition  Tag  Flags    Sector     Count    Sector  Mount Directory
       0      2    00          0     61920     61919   /
       1      3    01      61920    164160    226079
       2      5    00          0    828720    828719
       5      6    00     226080    151920    377999   /opt
       6      4    00     378000    450720    828719   /usr
----------------------------------------------------------

\end{verbatim}
\end{quote}

\newpage
\section{\label{app_trusted_host_security}\xlabel{app_trusted_host_security}Trusted host security}

The method of network communication between Unix machines depends
whether the connection is to a trusted or non-trusted host. The
commands to communicate with a trusted host may differ from those used
to communicate with a non-trusted host. Trusted host commands are more
analogous to commands used on a local machine (\emph{e.g.}, {\tt cp} and
{\tt rcp}) and simplify access by bypassing the password security check
otherwise required.  One can define which remote hosts and which users
on those hosts are trusted.

The file {\tt /etc/hosts.equiv} specifies the names of machines from
which users are allowed to login (\verb+rlogin+), remote copy
(\verb+rcp+) or perform remote shell (\verb+rsh+) functions. {\tt
/etc/hosts.equiv} requires only machine hostnames as entries and allows
a user on a machine registered in the file to login to the local
machine only if an entry for that specific user is found in {\tt
/etc/passwd} on the local machine. If there is an entry for the
specific user in {\tt /etc/passwd} but no entry for the machine of that
user in {\tt /etc/hosts.equiv}, then the user will not be allowed to
use the above commands.  Note that the symbol {\tt +} is a wildcard
\emph{i.e.}, anyone can {\tt rlogin} without a password if there is an
account with the same username on that machine.

The file {\tt .rhosts} may exist in the home directory
of each user. Its purpose is to allow a specific user to access a
particular machine, but not just from any machine. Again, the user must
also have an entry in the {\tt /etc/passwd} file of the machine used.
Be careful if you have a {\tt /.rhosts} file on your systems since users
placed in this file will automatically have access to your system as
root.

Unfortunately, if is necessary to use these mechanisms to access
peripherals despite the potential security risk. For example, it is
necessary to use {\tt /.rhosts} in order to allow a system manager to
dump partitions on a tapeless workstation disk to an DLT on a Server.

\end{document}
