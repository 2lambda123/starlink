\documentclass[11pt]{article}
\pagestyle{myheadings}

% -----------------------------------------------------------------------------
% ? Document identification
\newcommand{\stardoccategory}  {Starlink System Note}
\newcommand{\stardocinitials}  {SSN}
\newcommand{\stardocsource}    {ssn\stardocnumber}
\newcommand{\stardocnumber}    {67.1}
\newcommand{\stardocauthors}   {S C Rehan}
\newcommand{\stardocdate}      {27th November 1996}
\newcommand{\stardoctitle}     {Unix Security Cookbook}
\newcommand{\stardocabstract}  {

This document has been written to help Site Managers secure their
Unix hosts from being compromised by hackers. I have given brief
introductions to the security tools along with downloading, configuring 
and running information. I have also included a section on my 
recommendations for installing these security tools starting from an 
absolute minimum security requirement.
}

% ? End of document identification
% -----------------------------------------------------------------------------

\newcommand{\stardocname}{\stardocinitials /\stardocnumber}
\markright{\stardocname}
\setlength{\textwidth}{160mm}
\setlength{\textheight}{230mm}
\setlength{\topmargin}{-2mm}
\setlength{\oddsidemargin}{0mm}
\setlength{\evensidemargin}{0mm}
\setlength{\parindent}{0mm}
\setlength{\parskip}{\medskipamount}
\setlength{\unitlength}{1mm}

% -----------------------------------------------------------------------------
%  Hypertext definitions.
%  ======================
%  These are used by the LaTeX2HTML translator in conjunction with star2html.

%  Comment.sty: version 2.0, 19 June 1992
%  Selectively in/exclude pieces of text.
%
%  Author
%    Victor Eijkhout                                      <eijkhout@cs.utk.edu>
%    Department of Computer Science
%    University Tennessee at Knoxville
%    104 Ayres Hall
%    Knoxville, TN 37996
%    USA

%  Do not remove the %begin{latexonly} and %end{latexonly} lines (used by 
%  LaTeX2HTML to signify text it shouldn't process).
%begin{latexonly}
\makeatletter
\def\makeinnocent#1{\catcode`#1=12 }
\def\csarg#1#2{\expandafter#1\csname#2\endcsname}

\def\ThrowAwayComment#1{\begingroup
    \def\CurrentComment{#1}%
    \let\do\makeinnocent \dospecials
    \makeinnocent\^^L% and whatever other special cases
    \endlinechar`\^^M \catcode`\^^M=12 \xComment}
{\catcode`\^^M=12 \endlinechar=-1 %
 \gdef\xComment#1^^M{\def\test{#1}
      \csarg\ifx{PlainEnd\CurrentComment Test}\test
          \let\html@next\endgroup
      \else \csarg\ifx{LaLaEnd\CurrentComment Test}\test
            \edef\html@next{\endgroup\noexpand\end{\CurrentComment}}
      \else \let\html@next\xComment
      \fi \fi \html@next}
}
\makeatother

\def\includecomment
 #1{\expandafter\def\csname#1\endcsname{}%
    \expandafter\def\csname end#1\endcsname{}}
\def\excludecomment
 #1{\expandafter\def\csname#1\endcsname{\ThrowAwayComment{#1}}%
    {\escapechar=-1\relax
     \csarg\xdef{PlainEnd#1Test}{\string\\end#1}%
     \csarg\xdef{LaLaEnd#1Test}{\string\\end\string\{#1\string\}}%
    }}

%  Define environments that ignore their contents.
\excludecomment{comment}
\excludecomment{rawhtml}
\excludecomment{htmlonly}

%  Hypertext commands etc. This is a condensed version of the html.sty
%  file supplied with LaTeX2HTML by: Nikos Drakos <nikos@cbl.leeds.ac.uk> &
%  Jelle van Zeijl <jvzeijl@isou17.estec.esa.nl>. The LaTeX2HTML documentation
%  should be consulted about all commands (and the environments defined above)
%  except \xref and \xlabel which are Starlink specific.

\newcommand{\htmladdnormallinkfoot}[2]{#1\footnote{#2}}
\newcommand{\htmladdnormallink}[2]{#1}
\newcommand{\htmladdimg}[1]{}
\newenvironment{latexonly}{}{}
\newcommand{\hyperref}[4]{#2\ref{#4}#3}
\newcommand{\htmlref}[2]{#1}
\newcommand{\htmlimage}[1]{}
\newcommand{\htmladdtonavigation}[1]{}

% Define commands for HTML-only or LaTeX-only text.
\newcommand{\html}[1]{}
\newcommand{\latex}[1]{#1}

% Use latex2html 98.2.
\newcommand{\latexhtml}[2]{#1}

%  Starlink cross-references and labels.
\newcommand{\xref}[3]{#1}
\newcommand{\xlabel}[1]{}

%  LaTeX2HTML symbol.
\newcommand{\latextohtml}{{\bf LaTeX}{2}{\tt{HTML}}}

%  Define command to re-centre underscore for Latex and leave as normal
%  for HTML (severe problems with \_ in tabbing environments and \_\_
%  generally otherwise).
\newcommand{\setunderscore}{\renewcommand{\_}{{\tt\symbol{95}}}}
\latex{\setunderscore}

%  Redefine the \tableofcontents command. This procrastination is necessary 
%  to stop the automatic creation of a second table of contents page
%  by latex2html.
\newcommand{\latexonlytoc}[0]{\tableofcontents}

% -----------------------------------------------------------------------------
%  Debugging.
%  =========
%  Remove % from the following to debug links in the HTML version using Latex.

% \newcommand{\hotlink}[2]{\fbox{\begin{tabular}[t]{@{}c@{}}#1\\\hline{\footnotesize #2}\end{tabular}}}
% \renewcommand{\htmladdnormallinkfoot}[2]{\hotlink{#1}{#2}}
% \renewcommand{\htmladdnormallink}[2]{\hotlink{#1}{#2}}
% \renewcommand{\hyperref}[4]{\hotlink{#1}{\S\ref{#4}}}
% \renewcommand{\htmlref}[2]{\hotlink{#1}{\S\ref{#2}}}
% \renewcommand{\xref}[3]{\hotlink{#1}{#2 -- #3}}
%end{latexonly}
% -----------------------------------------------------------------------------
% ? Document-specific \newcommand or \newenvironment commands.
% ? End of document-specific commands
% -----------------------------------------------------------------------------
%  Title Page.
%  ===========
\renewcommand{\thepage}{\roman{page}}
\begin{document}
\thispagestyle{empty}

%  Latex document header.
%  ======================
\begin{latexonly}
   CCLRC / {\sc Rutherford Appleton Laboratory} \hfill {\bf \stardocname}\\
   {\large Particle Physics \& Astronomy Research Council}\\
   {\large Starlink Project\\}
   {\large \stardoccategory\ \stardocnumber}
   \begin{flushright}
   \stardocauthors\\
   \stardocdate
   \end{flushright}
   \vspace{-4mm}
   \rule{\textwidth}{0.5mm}
   \vspace{5mm}
   \begin{center}
   {\Huge\bf \stardoctitle}
   \end{center}
   \vspace{5mm}

% ? Heading for abstract if used.
   \vspace{10mm}
   \begin{center}
      {\Large\bf Abstract}
   \end{center}
% ? End of heading for abstract.
\end{latexonly}

%  HTML documentation header.
%  ==========================
\begin{htmlonly}
   \xlabel{}
   \begin{rawhtml} <H1> \end{rawhtml}
      \stardoctitle
   \begin{rawhtml} </H1> \end{rawhtml}

% ? Add picture here if required.
% ? End of picture

   \begin{rawhtml} <P> <I> \end{rawhtml}
   \stardoccategory\ \stardocnumber \\
   \stardocauthors \\
   \stardocdate
   \begin{rawhtml} </I> </P> <H3> \end{rawhtml}
      \htmladdnormallink{CCLRC}{http://www.cclrc.ac.uk} /
      \htmladdnormallink{Rutherford Appleton Laboratory}
                        {http://www.cclrc.ac.uk/ral} \\
      \htmladdnormallink{Particle Physics \& Astronomy Research Council}
                        {http://www.pparc.ac.uk} \\
   \begin{rawhtml} </H3> <H2> \end{rawhtml}
      \htmladdnormallink{Starlink Project}{http://www.starlink.ac.uk/}
   \begin{rawhtml} </H2> \end{rawhtml}
   \htmladdnormallink{\htmladdimg{source.gif} Retrieve hardcopy}
      {http://www.starlink.ac.uk/cgi-bin/hcserver?\stardocsource}\\

%  HTML document table of contents. 
%  ================================
%  Add table of contents header and a navigation button to return to this 
%  point in the document (this should always go before the abstract \section). 
  \label{stardoccontents}
  \begin{rawhtml} 
    <HR>
    <H2>Contents</H2>
  \end{rawhtml}
  \htmladdtonavigation{\htmlref{\htmladdimg{contents_motif.gif}}
        {stardoccontents}}

% ? New section for abstract if used.
%  \section{\xlabel{abstract}Abstract}
% ? End of new section for abstract

\end{htmlonly}

% -----------------------------------------------------------------------------
% ? Document Abstract. (if used)
%  ==================
\stardocabstract
% ? End of document abstract
% -----------------------------------------------------------------------------
% ? Latex document Table of Contents (if used).
%  ===========================================
\newpage
\begin{latexonly}
  \begin {center}
    \rule{80mm}{0.5mm} \\ [1ex]
    \vspace{5mm}
     {\Large\bf \stardoctitle} \\ [2.5ex]
%           \stardocversion} \\ [2ex]
    \rule{80mm}{0.5mm}
  \end{center}
  \vspace{30mm}
  \setlength{\parskip}{0mm}
  \tableofcontents
  \setlength{\parskip}{\medskipamount}
  \markright{\stardocname}
\end{latexonly}
% ? End of Latex document table of contents
% -----------------------------------------------------------------------------
\newpage
\renewcommand{\thepage}{\arabic{page}}
\setcounter{page}{1}

\section{\label{introduction}\xlabel{introduction}Introduction} 

This document addresses the process of securing a Unix
box using publicly available software. Most of the software described within
this document is available through Starlink Forum at the following URL:

\begin{quote}{\tt
\htmladdnormallink{http://rlsaxps.bnsc.rl.ac.uk/Forum/Systems-Management/41/}
{http://rlsaxps.bnsc.rl.ac.uk/Forum/Systems-Management/41/}}
\end{quote}

If you do not have access to Starlink Forum then use the following URL:

\begin{quote}{\tt
\htmladdnormallink{ftp://rastro.ukc.ac.uk/pub/security/}
{ftp://rastro.ukc.ac.uk/pub/security/}}
\end{quote}

The tools described here are a small subset of the most useful tools
I have investigated. There are new tools coming out all the time, if
you feel a particular security tool is worth including into this
document then contact the author of this document.

The aim of this paper is to describe what tools are available and what 
they do without introducing too much technical detail. I have also 
included information on which software tools I strongly recommend to be 
installed, what tools are useful to have but not essential and tools for 
the paranoid system manager.

The majority of software packages described are applicable for most
versions of Unix which include Digital UNIX, Solaris 2.x and Linux. If a 
particular OS port is not available this will be indicated.

%[Appendix B contains a lookup table which shows what security tools cater for
%a particular type of security violation.

This document is sectioned into systems security tools which deal mostly 
with the filing system, miscellaneous tools which aid the use of some 
of the described security tools and network security tools.

%------------------------------------------------------------------------

\section{\label{recommendations}\xlabel{Recommendations}Recommendations} 
For site managers who are unfamiliar with most of the security tools
described in this document here are my recommendations:

\begin{itemize}
\item {\bf Minimum}: As an absolute minimum install the {\bf TCP Wrappers} 
which will allow you to log and control access to your hosts. You can add 
{\bf Swatch} to this which will help you to monitor the log files.
\item After installing the minimum required security tools,
I would recommend installing {\bf Crack} and checking user passwords. You can 
setup a cron job to do this.
It is also worth running {\bf TIGER} occasionally to check on the state of the
filesystem. The root areas of the filesystem will not change very 
regularly but the user areas will.  The frequency with which you add
new users and do system upgrades will roughly determine how often you 
use {\bf TIGER}.

If you do install either {\bf Crack} and {\bf TIGER} then I would also 
recommend installing {\bf MERLIN}. This will provide you with a HTML 
front end with which you can execute either program and manage the 
output through hypertext links.

\item Install some network security probing software such
as {\bf ISS} or {\bf SATAN}. If you want to use encrypted passwords over 
the network then it is worth looking at Secure Shell ({\bf SSH}). This 
provides enhanced versions of rlogin, rsh and rcp that provide RSA 
authentication and encryption of communications. This tool is not described 
here but you can fetch it from the above URL in Starlink Forum.

You may also want to consider installing the {\bf Logdaemon} package instead
of the TCP Wrappers. This tool provides replacements for ftp, rlogin,
rexec, rsh daemons and login programs that have added security
features such as login failures and S/Key one-time password support.

If you want to create secure accounts use Operator Shell ({\bf Osh}). This
is a restricted C shell which allows the administrator to control
access to files and directories and to provide logging. It has a built
in restricted vi editor and is not described here but is also available
via Starlink Forum.

\item You may also want to consider installing a firewall.
This is a gateway machine with special security precautions on it, used 
to service network requests. The idea is to protect a cluster of
more loosely administered machines hidden behind it from hackers. This
can make it more difficult for users to login from remote sites if you 
restrict incoming connections to one machine.

\end{itemize}

%------------------------------------------------------------------------
\section{\label{Systems_Security_Tools}\xlabel{Systems_Security_Tools}Systems Security Tools}

This sections deals with controlling physical and logical access to 
hosts on your network.

\subsection{TIGER}
TIGER is used to perform a security audit of Unix systems. 

TIGER includes a set of signatures for various operating systems. These
are secure checksums of system files based on the original release of
the operating system. If you have applied patches since the OS was
released then the patched system binaries will have different
signatures. If you get a signature check error then check to see
if you have applied any patches.

TIGER has one primary goal, which is to report ways ``root'' can be 
compromised. Paths into ``root'' (cron, inetd, setuid executables,
PATH, etc) are all checked to see if anyone other than ``root''
can alter that path. Paths into other accounts are also checked for
these vulnerabilities. The following checks are done:

\begin{itemize}
\item cron entries are checked
\item mail aliases are checked
\item NFS exports are checked
\item inetd entries are checked
\item PATH variables are checked
\item {\it .rhosts} \& {\it .netrc} files are checked
\item Specific file \& directory access permissions are checked
\item File system scans locate unusual files
\item Digital signatures are used to detect alterations to key
binaries (signatures are generated from CD-ROM) and also to 
report binaries for which (updated) security patches exist.
\item Pathnames embedded in any files reported by most of the other
checks are checked.
\end{itemize}

\subsubsection{Installing TIGER}

Download the distribution file via the Unix\_Security Topic (41) 
in the Systems-management conference within Starlink Forum and unpack it. 
Use the following command to extract the files:

\begin{quote}
\verb+% gzcat tiger-2.2.3.tar.gz | tar xvf -+
\end{quote}

\subsubsection{Configuring TIGER}

In the {\it tiger-2.2.3} subdirectory you should create a 
site-\$HOSTNAME file which sets up NIS, hostname aliases, 
crack home directory, etc. There is a {\it site.sample} which 
you should copy onto {\it site-`hostname'} and amend to suit your setup.
If the host you are running the software on is called ``star'', then
do the following:

\begin{quote}
\verb+% cd tiger-2.2.3+\\
\verb+% cp site-sample site-star+\\
\verb+% vi site-star+
\end{quote}
 
The following variables are required to be set in this {\it site-star} file:

\begin{description}
\item {\bf GETFSHOST} is used to determine whether user home directories
are local or remote.
\item {\bf AUTOHOMEMAP} name of file or NIS map containing autmounter
maps for user directories.
\item {\bf HOSTNAMESLIST} list of names the scan host is known by.
\item {\bf CRACK} defines the Crack home directory.
\end{description}

A sample {\it site-star} file is shown in Appendix B.

\subsubsection{Running TIGER}
You can execute the program by changing into the tiger directory 
and running the tiger script as follows:

\begin{quote}
\begin{scriptsize}
\verb+# cd tiger-2.2.3+\\
\verb+# ./tiger+\\
\verb+Configuring...+\\
\verb++\\
\verb+Will try to check using config for 'sun4u' running SunOS 5.5...+\\
\verb+--CONFIG-- [con005c] Using configuration files for SunOS 5.5. Using+\\
\verb+           configuration files for generic SunOS 5.+\\
\verb+           Not all checks may be performed.+\\
\verb+Tiger security scripts *** 2.2.3, 1994.0309.2038 ***+\\
\verb+11:30> Beginning security report for star.+\\
\verb+11:30> Starting file systems scans in background...+\\
\verb+11:30> Checking password files...+\\
\verb+11:30> Checking group files...+\\
\verb+11:30> Checking user accounts...+\\
\verb+11:30> Checking .rhosts files...+\\
\verb+11:30> Checking .netrc files...+\\
\verb+11:30> Checking PATH settings...+\\
\verb+11:31> Checking anonymous ftp setup...+\\
\verb+11:31> Checking mail aliases...+\\
\verb+11:31> Checking cron entries...+\\
\verb+11:31> Checking 'inetd' configuration...+\\
\verb+11:31> Checking NFS export entries...+\\
\verb+11:31> Checking permissions and ownership of system files...+\\
\verb+11:32> Checking for altered or out of date binaries...+\\
\verb+11:32> Checking for indications of breakin...+\\
\verb+11:32> Performing system specific checks...+\\
\verb+./systems/SunOS/5/check: /usr/kvm/eeprom: not found+\\
\verb+11:32> Performing check of embedded pathnames...+\\
\verb+./scripts/sub/check_devs: bad substitution+\\
\verb+11:36> Waiting for filesystems scans to complete...+\\
\verb+11:36> Filesystems scans completed...+\\
\verb+11:36> Security report completed for star.+\\
\verb+Security report is in `./security.report.star.960926-11:30'.+
\end{scriptsize}
\end{quote}

The resulting security report is usually fairly long and is better
interpreted using the Merlin tool described in section 4.1
of this document.


\subsection{Crack}
Crack helps the security administrator identify weak passwords by checking for various weaknesses and attempting to decrypt them.  
System attacks can involve grabbing the password file and running this 
program on it.
Systems employing shadowing password schemes are much harder to break into.

The general procedure used by Crack is to take as its input a series
of password files and source dictionaries. It merges the dictionaries,
turns the password files into a sorted list, and generates lists of
possible passwords from the merged dictionary. Crack works by making
many individual passes over the password entries that are supplied as input.
Each pass generates password guesses based upon a sequence of rules.

\subsubsection{Installing Crack}

Download the distribution file via the Unix\_Security Topic (41) 
in the Systems-management conference within Starlink Forum and unpack it. 
Use the following command to extract the files:

\begin{quote}
\verb+% gzcat crack4.1_***.tar.gz | tar xvf -+
\end{quote}

\begin{description}
\item where *** will be axp, lin or sol depending on the file you have downloaded
\end{description}

\subsubsection{Configuring Crack}
Enter the 
{\it crack} subdirectory and edit the {\it Crack} script. Amend CRACK\_HOME 
to the directory where you installed crack. If your system uses 
shadow passwords then you need to create a {\it passwd } file. Do the 
following as root if you are using Solaris 2.x:

\begin{quote}
\verb+# cd crack+\\
\verb+# Scripts/shadmrg > passwd+
\end{quote}

\subsubsection{Running Crack}
This tool can also be used in conjunction with Merlin. See section 4.1
of this document for more information about Merlin.

Do the following to run Crack:

\begin{quote}
\verb+% cd crack+\\
\verb+% ./Crack [-v] <password file>+
\end{quote}

\begin{tabbing}
This line is used to   set some \=tabs for formatting test \kill
where {\verb+<password file>+} \> is {\it /etc/passwd} for Linux and Digital Unix \\
\> is {\it passwd} for Solaris 2.x                      
\end{tabbing}

\begin{description}
\item {\bf -v } is to verbose crack information into the {\it out.*} files. 
This tends to generate huge files, so I would leave it out unless 
you want to see what the program is doing.
\end{description}

All of the output from Crack is put into some {\it out.*} files in the 
{\it crack} subdirectory. Under Digital Unix the output files are put into 
the {\it crack/done} subdirectory.

Once a password is cracked the {\it out.*} files will contain a line which looks
similar to the one below:

\begin{quote}
\begin{scriptsize}
\verb+ join: Sep 25 15:49:27 Guessed visitor (/bin/tcsh in passwd) [gre=GRE1] vmU4J1w7hZXsQ+
\end{scriptsize}
\end{quote}

Here the visitor account has been cracked. The non-encrypted password is
in square brackets and the encrypted version is at the end.

The {\it Scripts} subdirectory contains some useful utilities:
\begin{description}
\item {\bf Scripts/nastygram} This is the shellscript that is invoked 
by the password cracker  to send mail to users who have guessable 
passwords. Use the -m option when running Crack to enable this option.
Think carefully before choosing to use this option.

\item {\bf Scripts/guess2fbk} This script takes your {\it out* } files  
as  arguments  and reformats  the  `Guessed'  lines  into a slightly 
messy feedback file, suitable for storing with the others.
An occasion where this might be useful is when your Crack dies before
writing out the guess to a feedback file.

\end{description}

Additional dictionaries are available from the Starlink Forum page.

\subsection {ISS}
Internet Security Scanner (ISS) is a system that allows automated scanning of
TCP/IP networked computers. It will interrogate all computers within
a specified IP address range, determining the security posture of
each with respect to several common system vunerabilities.

ISS will scan a domain sequentially looking for connections. When it finds
a host it will try to connect to various ports. Initially it tries the
telnet port, once connected it will log any information that the host
displays.

ISS also does remote procedure call (rpc) checks. It does a 
{\it rpcinfo -p \verb+<hostname>+}. with the information obtained  it looks 
for hosts that are running NIS, rexd, bootparam, whose on the host, 
selection\_svc and NFS. 

If a system shows {\it ypserv}, it is possible
to provide the password file to any remote host asking for it. ISS will
attempt to guess the domainname and that will provide information as
to which machine is the NIS server. 

If a system shows {\it Select\_svr, selection\_svr} is running on 
the machine and there are known holes that let anyone remotely 
read any file on the system, even the password file. 
{\it Selection\_svr} should be disabled.

  When {\it rexd} is running on a remote system, anyone with a 
small C program can emulate the `on' command spoofing any user 
on the remote machine, thus gaining access to the password file 
and adding {\it .rhosts} files. {\it rexd } should be disabled.

  If a machine is running {\it bootparam}, it is likely a server 
to diskless clients.  One problem with {\it bootparam} is that 
if it is running and someone can guess which machines the client 
and servers are, they are able to get the domainname from {\it bootparam},
which goes back to the {\it ypserv} problem.

It also checks for mail aliases, some obvious ones are decode and 
uudecode. With these aliases, you are able to send mail to
decode@hostname with a file that has been uuencoded to overwrite a systems
file, such as {\it .rhosts}. Some of the users it looks for are `bbs',
 `guest', `lp', which are known to have weak or no passwords at all. 
The well known debug and wiz backdoors for sendmail are also checked.

\subsubsection{Installing ISS}

Download the distribution file via the Unix\_Security Topic (41) 
in the Systems-management conference within Starlink Forum and unpack it. 
Use the following command to extract the files:

\begin{quote}
\verb+% gzcat iss1.3_***.tar.gz | tar xvf -+
\end{quote}

\begin{description}
\item where *** will be axp, lin or sol depending on the file you have
downloaded.
\end{description}

\subsubsection{Running ISS}

To carry out a network scan 
from host 129.12.48.129 to 129.12.48.130 do the following:

\begin{quote}
\verb+% cd iss+\\
\verb+% ./iss -o ISS.log 129.12.48.129 129.12.48.130+
\end{quote}

This sends all output information to {\it ISS.log} file. On viewing
this file you will see something like:

\begin{quote}
\begin{scriptsize}
\verb+% more ISS.log+\\
\verb++\\
\verb+       -->    Inet Sec Scanner Log By Christopher Klaus (C) 1995    <--+\\
\verb+              Email: cklaus@iss.net Web: http://iss.net/iss+\\
\verb+       ================================================================+\\
\verb++\\
\verb+Scanning from 129.12.48.129 to 129.12.48.130+\\
\verb+129.12.48.129 kensun2+\\
\verb+> Unix(r) System V Release 4.0 (kensun2) Welcome to Starlink at The University+\\
\verb+SMTP:220 starlink.ukc.ac.uk Sendmail SMI-8.6/SMI-SVR4 ready at Thu, 26 Sep 1996+\\
\verb+550 guest... User unknown+\\
\verb+550 decode... User unknown+\\
\verb+550 bbs... User unknown+\\
\verb+250 0000-lp(0000) <lp@starlink.ukc.ac.uk>+\\
\verb+550 uudecode... User unknown+\\
\verb+500 Command unrecognized+\\
\verb+500 Command unrecognized+\\
\verb+221 starlink.ukc.ac.uk closing connection+\\
\verb++\\
\verb+FTP:220 kensun2 FTP server (Unix(r) System V Release 4.0) ready.+\\
\verb+331 Guest login ok, send ident as password.+\\
\verb+230 Guest login ok, access restrictions apply.+\\
\verb+257 "/" is current directory.+\\
\verb+550 test: Permission denied.+\\
\verb+550 test: No such file or directory.+\\
\verb+221 Goodbye.+\\
\verb+...+\\
\verb+..+\\
\verb+.+
\end{scriptsize}
\end{quote}

The above file shows various tests carried out on the hosts.
The first test on kensun2 connects to port 25 of the
host to check if sendmail is running. When the connection has been establised
it then does test 550 which checks for various mail aliases. 

The next test is for an anonymous ftp server. The connection is first
established and it tries to change to ``/'', make a subdirectory and
change into it. In this case the ``Permission denied'' error message
was generated. The connection is then closed.

%------------------------------------------------------------------------

\section{\label{miscellaneous_tools}\xlabel{miscellaneous_tools}Miscellaneous Tools}

\subsection {MERLIN}

Merlin is a tool for managing other tools. It currently supports
COPS 1.04, Tiger 2.2.3, Crack 4.1, Tripwire 1.2, and SPI 3.2.2.
It can take a powerful but cryptic command-line tool and provide 
it with an easy-to-use graphical interface. It also provides the 
ability to sort reports based on the type of tool used, the 
creation date, or the host where the report is produced.

Merlin does not come with the above packages, you have to install
them yourself and tell Merlin where the home directory is.

\subsubsection{Installing MERLIN}

Download the distribution file via the Unix\_Security Topic (41) 
in the Systems-management conference within Starlink Forum and unpack it. 
Use the following command to extract the files:

\begin{quote}
\verb+% gzcat merlin1.0_***.tar.gz | tar xvf -+
\end{quote}

\begin{description}
\item where *** will be axp,lin or sol depending on the file you have
downloaded.
\end{description}

\subsubsection{Running MERLIN}

MERLIN uses perl5 patch m, which it assumes is installed in 
{\it /usr/local/bin/perl}.
Change into the {\it merlin-1.0} directory and type the following command
to startup Merlin:

\begin{quote}
\verb+# cd merlin-1.0+\\
\verb+# ./merlin+\\
\verb+Merlin is starting up....+
\end{quote}

A Netscape browser will pop up with a Merlin start page. To configure,
click on the {\it Configuration} icon. You will now be presented with
boxes for entering package home directories. Type in the full pathnames
to its top directory (without the ``/'' at the end). You don't need 
to fill them all in, just the tools you have installed or want to run.

When you are done, click the {\it Make changes} box at the bottom of
the page and follow the instructions on the following page.

When you reload the {\it Table of Contents} page you will notice
a hypertext link for the security tool you entered.

Selecting this link will take you to an {\it Options} list which
you click to enable. To initiate the scan click {\it Start the Scan}
at the bottom of the page.

To get a status on the scan run, click the {\it Job Status} icon.

You can view the output by clicking on {\it Report Browser} icon.
You can view output while a job is running. You can sort the reports
out by host or by tool.

I would highly recommend using this tool if you are using Crack or
TIGER. It presents results in a clear and readable manner.

\subsection {Swatch}
Modern Unix systems are capable of logging a variety of information
concerning the health and status of their hardware and operating system
software, but are generally not configured to do so.

Also with a large network a system administrator must often monitor
several log files. Swatch is a utility which will allow a system
manager to log critical system and security related information to 
a dependable, secure, central logging host system. Swatch monitors
log files and acts to filter out unwanted data and take one or more
user specified actions (ring bell, send mail, execute a script, etc.)
based upon patterns in the log.

It's extremely useful when logging to a central host in conjunction
with tcpwrappers to provide extra logging info.

\subsubsection{Installing Swatch}

Download the distribution file via the Unix\_Security Topic (41) 
in the Systems-management conference within Starlink Forum and unpack it. 
Use the following command to extract the files:

\begin{quote}
\verb+% zcat swatch-2.2.tar.Z | tar xvf -+
\end{quote}

\subsubsection{Configuring Swatch}

Each non-comment line in a swatch configuration file consists of four
tab separated fields: a pattern expression, a set of actions to be
done if the expression is matched, an optional time interval, and
the location of a time stamp, if any.

The format of each line is as follows:

\begin{quote}
\begin{scriptsize}
\verb+/pattern/[,/pattern/,...] action[,action,...] [ [ [ HH:] MM:] SS [ start:length  ] ]+
\end{scriptsize}
\end{quote}

\begin{description}
\item The patterns must be regular expressions which Perl will accept. Each
string is compared in order with the expression in the configuration
file and if a match is found the corresponding actions are taken.
\item The time interval can be used to help eliminate redundant 
messages.
\item The time stamp location information is optional and can only be
used when a time interval is specified. Swatch uses it to strip away
the time stamp in order to compare it to other messages which are
stored in its internal history list.
\end{description}

An example follows:

\begin{quote}
\verb+/file system full/ echo,bell  01:00     0:16+
\end{quote}

This will cause the bell to sound when the pattern ``file system
full'' is matched. Also multiple instances of the message will not
be echoed if they appear within a minute of the first one.

Swatch understands the following actions:

\begin{itemize}
\item The {\bf echo} action causes the line to be echoed to swatch's
controlling terminal. The text can normal, bold, underscore, blinking
or inverse mode through the use of an optional argument.
\item The {\bf bell} action sends a bell signal to the controlling
terminal. An optional argument specifies the number of bell signals
to send.
\item The {\bf ignore} actions causes swatch to ignore the current line
of input and proceed to the next one.
\item The {\bf write} and {\bf mail} actions can be used to send a copy
of the line to a user list via the write and mail commands.
\item The {\bf pipe} and {\bf exec} actions were added to provide some 
flexibility. The pipe action allows the user to use matched lines as
input to a particular command on the system. The exec action allows
the user to run a command on the system with the option of using 
selected fields from the matched line as arguments for the command. 
\end{itemize}

For more information about configuring Swatch, see the online manual
page with the following command:

\begin{quote}
\verb+% nroff -man swatch.conf.man | more+
\end{quote}

There are some sample configuration files in the {\it config\_files}
subdirectory inside the main Swatch directory. Copy one of these 
to your home directory as follows:

\begin{quote}
\verb+% cp config_files/swatchrc.personal ~username/.swatchrc+
\end{quote}

\begin{description}
\item where {\bf username} is your login id.
\end{description}

You can amend it as appropriate for your system.

\subsubsection{Running Swatch}

The most useful way of using Swatch is to look at messages as they 
are being added to the syslog file. You can do this by the following
command:

\begin{quote}
\verb+% swatch -t /var/log/syslog+
\end{quote}

This is the default behaviour if swatch is invoked without any arguments.

To process all information in a logfile from beginning to end, use the 
following command:

\begin{quote}
\verb+% swatch -f /var/log/syslog+
\end{quote}

For more information about the command line options, consult the online 
manual page with the following command:

\begin{quote}
\verb+% nroff -man swatch.prog.man | more+
\end{quote}

%------------------------------------------------------------------------

\section{\label{network_security_tools}\xlabel{network_security_tools}Network Security Tools}

The tools in this section relate to the physical network or ``looking
at what is happening on the wire'' and are more monitoring in nature.

\subsection{TCP Wrappers}
The TCP wrapper provides monitoring and control of network services. We
live in world where networking facilities are desirable, forcing you 
to trust external hosts, but all the while having hackers living on
those very same hosts. The tool described here allows you to protect
each network service that is protectable, limit access as much as 
possible, and log all connections to make the job of detecting
and resolving problems easier.

Almost every application of the TCP/IP protocols is based on a 
client-server model. For example, when a user invokes the telnet command to
connect to one of your systems, a telnet server process is executed on
the target host. The telnet server process connects the user to a login
process. A few examples of client and server programs are shown in the
table below:

%\begin{tabbing}
%This line is used to set\=some tabbing \=positions here\= and kill the\kill
%\>client\>server\>application\\
%\>-------------------\>--------------------\>------------------\\
%\>telnet\>telnetd\>remote login\\
%\>ftp\>ftpd\>file transfer\\
%\>finger\>fingerd\>show users\\
%\end{tabbing}

\begin {table}[h]
\begin {center}
\begin {tabular}{||l|l|l||}
\hline
client & server & application \\
\hline
telnet & telnetd   & remote login \\
ftp & ftpd & file transfer \\
finger & fingerd  & show users \\
\hline
\end {tabular}
\caption {Client server processes}
\end {center}
\end {table}

When someone connects to a host, a single daemon (called inetd) waits
for the connection to be establised. It then runs the appropriate 
server program and then goes back to waiting for other connections.

The wrapper programs are setup so that instead of running the server
program, inetd is tricked into running a small wrapper program. The 
wrapper logs the client hostname or address and performs some
additional checks and then executes the appropriate server program.
One of the additional checks include defining access control lists
which are used to allow/deny access to the host.

Note the following:

\begin{itemize}
\item The wrapper programs have no interaction with the client user.
\item The wrappers have no interaction with the server application.
\item Once a wrapper has established a connection between client and
server, the wrapper goes away and there is no overhead on either end.
\end{itemize}

\subsubsection{Installing TCP Wrappers}

Download the distribution file via the Unix\_Security Topic (41) 
in the Systems-management conference within Starlink Forum and unpack it. 
Use the following command to extract the files:

\begin{quote}
\verb+% gzcat tcp_wrappers_7.2-***.tar.gz | tar xvf -+
\end{quote}

\begin{description}
\item where *** will be axp, lin or sol depending on the file you have
downloaded.
\end{description}

The {\it tcpd} executable in the {\it tcp\_wrappers\_7.2} subdirectory 
will monitor telnet, finger, ftp, exec, rsh, rlogin, tftp, talk, 
comsat and other udp or tcp services that have a one-to-one mapping 
onto executable files. As root user, install the files as follows:

\begin{quote}
\verb+# cd tcp_wrappers_7.2+\\
\verb+# cp tcpd /usr/local/sbin+\\
\verb+# cp tcpd.8 /usr/local/man/man8+
\end{quote}

\subsubsection{Configuring TCP Wrappers}

There are a couple of ways that you can install the wrappers. I 
would recommend modifying the {\it /etc/inetd.conf} file. An example
for monitoring ftp services is shown below:

\begin{quote}
\verb+ftp     stream  tcp     nowait  root    /usr/sbin/in.ftpd       in.ftpd+\\
\verb+                                        ^^^^^^^^^^^^^^^^^+\\
\verb+becomes+\\
\verb++\\
\verb+ftp     stream  tcp     nowait  root    /usr/local/sbin/tcpd    in.ftpd+\\
\verb+                                        ^^^^^^^^^^^^^^^^^^^^+
\end{quote}

A few lines from my {\it inetd.conf} are shown is Appendix A. The complete
file is too big to list.

\subsubsection{Running TCP Wrappers}

Once you have finished editing the {\it /etc/inetd.conf} file, you need
to send the inetd daemon a HUP signal. First find the process id
for inetd on your system and then send it a restart signal. For Digital
Unix and Solaris 2.x you can do the following as root user:

\begin{quote}
\begin{verbatim}
# ps -def | fgrep inetd
    root   116     1  0   Sep 07 ?        0:06 /usr/sbin/inetd -s
# kill -HUP 116
\end{verbatim}
\end{quote}

The will cause the inetd daemon to re-read its configuration file
{\it /etc/inetd.conf}.

\subsubsection{Logging information}

By default, all logging information is sent to the {\it /var/log/syslog} 
file. This is logged in the same place as where the sendmail daemon
logs its information. If you want to change this you need to edit
{\it /etc/syslog.conf} file and amend the following line:

\begin{quote}
\verb+mail.debug                      ifdef(`LOGHOST', /var/log/syslog, @loghost)+
\end{quote}

\begin{description}
\item {\bf loghost} is defined in the {\it /etc/hosts}. This is set to
the name of the host you are on. If you would like logging to a central
host then amend {\bf loghost} in the {\it /etc/syslog.conf} file to 
{\bf logmaster}. Edit the {\it /etc/hosts} file or your NIS/NIS+ hosts 
file to set the central host that you would like to have the 
{\bf logmaster} alias. 
An example follows
\end{description}

\begin{quote}
\begin{verbatim}
% more /etc/hosts
#
# Internet host table
#
127.0.0.1       localhost
129.12.48.130   star    loghost
129.12.48.1     brtbio
129.12.48.131   kenterm1
129.12.48.132   kenterm2
129.12.48.133   kenterm3
129.12.12.96    kenterm4
129.12.48.128   kensun1
129.12.48.129   kensun2  logmaster
129.12.48.201   kenets1
129.12.48.32    kenpc1
129.12.48.33    kenpc2
129.12.48.34    pcscr1
\end{verbatim}
\end{quote}

In the above {\it /etc/hosts} file the localhost is called ``star''
and also has the alias name ``loghost''. The remote host ``kensun2''
is used as the central logging machine and has the alias ``logmaster''.

In order to make effective use of the logging information I would 
recommend using a logfile watching utility such as Swatch,
described in Section 3.2. This watches a specfied 
log file and indicates visually when someone starts up a daemon process.

\subsubsection{Access control}

You can also control access to your host using TCP wrappers. You need
to create two files for access control as follows:

\begin{itemize}
\item Access will be granted when a (daemon, remote client) pair matches an
entry in the {\it /etc/hosts.allow} file.
\item Access will be denied when a (daemon, remote client) pair matches an 
entry in the {\it /etc/hosts.deny} file.
\item If neither files exist then daemon access is granted to any host.
\end{itemize}

The format of these files is as follows:

\begin{quote}
\verb+       daemon_list : client_list [ : shell_command ]+
\end{quote}

where
\begin{description}
\item {\bf demon\_list} is a list of one or more daemon process names
such as ftpd, tftpd, telnetd, ... etc or wildcards

\item {\bf client\_list} is a list of one or more host names, host 
addresses, patterns or wildcards.
\end{description}

Here is an example where all service to all hosts is denied, unless
they are permitted access by entries in the allow file:

\begin{quote}
\verb+more /etc/hosts.deny+\\
\verb+ALL: ALL+
\end{quote}

In the above {\it /etc/hosts.deny} file ALL services to ALL hosts is 
denied, unless they are permitted access by entries in the allow file:

\begin{quote}
\verb+more /etc/hosts.allow+\\
\verb+ALL: LOCAL @some_netgroup+\\
\verb+ALL: .foobar.edu EXCEPT terminalserver.foobar.edu+
\end{quote}

In the above {\it /etc/hosts.allow} the first rule allows access from
hosts in the LOCAL domain and from members of the some\_netgroup
netgroup. The second rule permits access from ALL hosts in the 
foobar.edu domain with the exception of terminalserver.foobar.edu.

For further information about access control, consult the online manual
page in the tcp wrappers subdirectory through the following commands:

\begin{quote}
\verb+% cd tcp_wrappers_7.4+\\
\verb+% nroff -man hosts_access.5 | more+
\end{quote}

\subsection{SATAN}
SATAN is the Security Analysis Tool for Auditing Networks. 
It gathers as much information about remote hosts and networks
as possible by examining such network services as finger, NFS, NIS,
ftp and tftp, rexed and other services. It can then either report 
on this data or use a simple rule-based system to investigate any
potential security problems. A great deal of general network
information can be gained when using this tool e.g. network
topology, network services running, types of hardware and 
software being used on the network, etc. It is driven entirely
through a Web-based browser such as Netscape, Mosaic or Lynx and
allows you to examine, query, and analyze the output through this
interface.

SATAN has a few modes of operation. One of the more informative
is exploratory mode which will examine the avenues of trust and 
dependancy and iterate further data collection runs over secondary
hosts. This allows the user to analyze their own network and also 
to examine the real implications inherent in network trust and services
which will help make decisions about the security level of the 
systems involved.

\subsubsection{Installing SATAN}

Download the distribution file via the Unix\_Security Topic (41) 
in the Systems-management conference within Starlink Forum and unpack it. 
Use the following command to extract the files:

\begin{quote}
\verb+% gzcat satan_***.tar.gz | tar xvf -+
\end{quote}

\begin{description}
\item where *** will be axp, lin or sol depending on the file you have
downloaded.
\end{description}


\subsubsection{Configuring SATAN}

After extracting the source code, change into the {\it satan-1.1.1} 
subdirectory. There are a couple of ways to configure SATAN. 
You can either do it via the HTML interface or by editing 
the {\it config/satan.cf} file.

SATAN will support any Level 2 HTML browser such as Mosaic, Lynx or
Netscape versions 1 and 2. It will {\bf NOT} work properly with Netscape
version 3.

To explicitly set the type of browser, edit the {\it config/paths.pl}
and set {\verb+$MOSAIC+} to the location of the executable. 

I would recommend configuring via the HTML interface. See Section 3.2.3 
below for information on starting up the browser.

Configuration options: 

\begin{description}
\item {\bf Attack Level} There are three levels of probes (light,
normal and heavy). Each has a set of programs that they use when
probing a remote system.
\item {\bf satan\_data} specify where the gathered information
should be stored. Defaults to {\it satan-data}.
\item {\bf timeouts} Some network probes will try to continue to 
contact the remote host for a long time. {\it short\_timeout,
med\_timeout and long\_timeout} timeout variables
are available in order to prevent this from slowing down the
overall scan. The default setting is the med\_timeout. 

If a particular probe requires more time then this can be changed 
via the {\it config/satan.cf} file. For example:

\begin{quote}
\begin{verbatim}
%timeouts = (
        'nfs-chk.satan', 120,
        $heavy_tcp_scan, 120,
        $heavy_udp_scan, 120,
        );
\end{verbatim}
\end{quote}

\item {\bf Timeout Signals} When a timeout occurs, a kill signal is 
sent to the process running to stop it. This defaults to 9.

\item {\bf Proximity variables} This variable controls how far out
the attack is done. The value refers to how close the target machine
is. Anything over 0 can affect sites other than your own and it is
recommend that you {\bf NOT} set it above 3. The default setting is
0.

\item {\bf Trusted or Untrusted} SATAN assumes that it is being
run from a host that may appear in other hosts {\it rhosts, 
hosts.equiv or NFS export} files. The default setting is 0 which
means no.

\item {\bf Exceptions} You can specify target machines which do not
require probing. This can be done to stop SATAN scans from going
astray. There are two exception variables as follows:

\begin{itemize}
\item {\it \$only\_attack\_these} is a list of domains and/or networks
that tells SATAN to only attack hosts that match one of those
patterns. For example, to only attack educational site you can say:
\begin{quote}
\verb+$only_attack_these = "edu";+
\end{quote}

\item {\it \$dont\_attack\_these} is a list of domain and/or networks
that SATAN should {\it never} attack. For example, to stay away from
government networks you can say:

\begin{quote}
\verb+$dont_attack_these= "gov";+
\end{quote}

\end{itemize}

\end{description}

\subsubsection{Running SATAN}

Login as ``root'' user and make sure you are in the {\it satan-1.1.1}
subdirectory. Type {\bf ./satan \& } and you will see a browser pop up.

To configure it select SATAN Configuration Management. See the 
Configuring section above for the meaning of the options.

To start a scan, select SATAN Target selection from the start page.
You are presented with the following input boxes:

\begin{description}
\item {\bf Primary target selection} type in the host that you are
running SATAN from if it isn't already in the box.
\item you either {\bf Scan the target  host} or if you have the 
authorization and time you can scan {\bf Scan all hosts in the primary
subnet}.
\item select either a {\bf Light, Normal or Heavy } scan.
\end{description}

Click the {\bf Start the Scan} box.

\subsubsection{Analysing SATAN output}

In the reports if there is a host listed with a red dot next to it,
that means the host has a vulnerability that could compromise it. A
black dot means that no vulnerabilities have been found for that
particular host. Clicking on the hyperlinks will give you more
information on the host.

If you select {\it SATAN Reporting and Analysis} from the start page
you are presented with a list of information.

From here if you select {\it Vulnerabilities} followed by {\it By
Approximate Danger Level} hyperlinks and look at the warning messages, 
this will tell you if SATAN found any problems. You can try scanning
the targets at a higher level.

There is a considerable amount of information in the table of 
contents, too much to go through here in detail. I would
recommend clicking on the hyperlinks to see what they reveal and
going through the online documentation by clicking on {\it SATAN
Documentation} hyperlink from the start page.

\subsection {NFSWatch}
Although not strictly a security tool its main function is to allow
you to monitor NFS packets by host, network or file system. This is 
a useful tool when you want to see where NFS traffic is coming from/
going to and which filesystem it is accessing.

The number and percentage of packets received in each category is
displayed  on the screen in a continuously updated display.

Currently there is no {\bf Linux} version.

\subsubsection{Installing NFSWatch}

Download the distribution file via the Unix\_Security Topic (41) 
in the Systems-management conference within Starlink Forum and unpack it. 
Use the following command to extract the files:

\begin{quote}
\verb+% gzcat nfswatch4.3_***.tar.gz | tar xvf -+
\end{quote}

\begin{description}
\item where *** will be axp or sol depending on the file you have downloaded.
\end{description}

Change into the {\it nfswatch4.3} subdirectory and copy the 
{\it nfswatch} binary to a suitable location along with the man
page {\it nfswatch.8l}.

\subsubsection{Running NFSWatch}

You must be ``root'' user to run nfswatch. 
NFSwatch has many options, too many to describe here in detail but a 
few examples will be described.

To monitor NFS packets between hosts ``star'' and ``kenterm1'', type 
the following command:

\begin{quote}
\verb+# ./nfswatch -dst star -src kenterm1+
\end{quote}

To monitor NFS packets to a fileserver say star, type the following command
and you will observe something similar to the output below:

\newpage

\begin{quote}
\verb+# ./nfswatch -server star -all+
\end{quote}

\begin{quote}
\begin{scriptsize}
\begin{verbatim}
all hosts                   Mon Sep 30 20:27:40 1996   Elapsed time:   00:03:10
Interval packets:      1098 (network)        818 (to host)          0 (dropped)
Total packets:        23069 (network)      14936 (to host)          0 (dropped)
                      Monitoring packets from interface le0
                     int   pct   total                       int   pct   total
ND Read                0    0%        0 TCP Packets          461   56%    13678
ND Write               0    0%        0 UDP Packets          353   43%     1051
NFS Read             160   20%      271 ICMP Packets           0    0%        0
NFS Write              1    0%        1 Routing Control        0    0%       36
NFS Mount              0    0%        7 Address Resolution     2    0%       76
YP/NIS/NIS+            0    0%        0 Reverse Addr Resol     0    0%        0
RPC Authorization    166   20%      323 Ethernet/FDDI Bdcst    4    0%      179
Other RPC Packets      5    1%       56 Other Packets          2    0%      131
                                 3 file systems
     File Sys        int   pct   total       File Sys        int   pct   total
kensun1(32,17)         0    0%       15
star(32,26)          161  100%      252
star(32,3)             0    0%        2
\end{verbatim}
\end{scriptsize}
\end{quote}

\subsubsection{Analysing NFSWatch output}

After the interval which defaults to 10 seconds the screen will clear
to present you with the above display. You will also have an {\it nfswatch}
prompt. Press `h' at this prompt and you will receive brief help for
the available commands.

The first line displays the name of the host being monitored, the current
date and time and the time elapsed since monitoring was started. The 
second line displays the total number of packets received during the
most recent interval, and the third line displays the total number
of packets received since monitoring started. These two lines display
three numbers each: 
\begin{itemize}
\item the total number of packets on the network.
\item the total number of packets received by the destination host.
\item the number of packets dropped by the host.
\end{itemize}

The next part of the screen divides the received packets into 16
categories. Each category is displayed with three numbers:
\begin{itemize}
\item the number of packets received this interval.
\item the percentage  this  represents  of all packets received by 
the host during this interval.
\item the total number  of  packets received  since  monitoring  started.
\end{itemize}

For a complete description of the categories, see the online manual page
by typing the following command:

\begin{quote}
\verb+% man nfswatch+
\end{quote}

The third part of the screen shows the mounted file systems exported
by the file server for mounting through NFS. The numbers in parenthesis
are major and minor device numbers for the file system. If you run
nfswatch on the same host you are monitoring then you will see the
full pathname of the file systems. With each file system, three
numbers are displayed:
\begin{itemize}
\item the number of NFS requests for  this file  system  received  during  the interval
\item the percentage this represents of all NFS requests received  by  the  host
\item the  total  number of NFS requests for this file system received since monitoring started
\end{itemize}

For further help consult the manual page for a full description of all
the commands.

\subsection {Netlog}
This includes a set of programs for network monitoring of all TCP and
UDP connections on a subnet. It can be used for locating suspicious
network traffic. The following programs are included:

\begin{description}
\item {\bf tcplogger} -- Log all TCP connections on a subnet
\item {\bf udplogger} -- Log all UDP sessions on a subnet
\item {\bf extract} -- Process log files created by tcplogger or udplogger
\item {\bf netwatch} -- Realtime network monitor
\end{description}


where 
\begin{itemize}
\item TCP refers to Transmission control protocol. Most daemons use
this protocol for communication mainly because it adds support to
detect errors or lost data and will trigger retransmission until
the data is completely received.
\item UDP refers to User Datagram protocol. UDP neither guarantees delivery nor does it require a connection. As a result it is lightweight and efficient, but all error processing and retransmission must be taken care of by the application program.
\end{itemize}

If you type the following command on a Unix box and look under column 3 
it will tell you what protocol a particular daemon uses:

\begin{quote}
\begin{scriptsize}
\begin{verbatim}
% rpcinfo -p+
   program vers proto   port  service
    100000    4   tcp    111  rpcbind
    100000    3   tcp    111  rpcbind
    100000    2   tcp    111  rpcbind
    100000    4   udp    111  rpcbind
    100000    3   udp    111  rpcbind
    100000    2   udp    111  rpcbind
    100024    1   udp  32772  status
    100024    1   tcp  32771  status
    100232   10   udp  32773  sadmind
    100011    1   udp  32774  rquotad
    100002    2   udp  32775  rusersd
    100002    3   udp  32775  rusersd
    100002    2   tcp  32772  rusersd
    100002    3   tcp  32772  rusersd
    100021    1   udp   4045  nlockmgr
    100021    2   udp   4045  nlockmgr
    100012    1   udp  32776  sprayd
    100008    1   udp  32777  walld
    100001    2   udp  32778  rstatd
    100001    3   udp  32778  rstatd
    100001    4   udp  32778  rstatd
    100021    3   udp   4045  nlockmgr
    100021    4   udp   4045  nlockmgr
    100221    1   tcp  32773
    100068    2   udp  32779
    100068    3   udp  32779
    100068    4   udp  32779
    100068    5   udp  32779
    100083    1   tcp  32774
    100021    1   tcp   4045  nlockmgr
    100021    2   tcp   4045  nlockmgr
    100021    3   tcp   4045  nlockmgr
    100021    4   tcp   4045  nlockmgr
    100003    2   udp   2049  nfs
    100003    3   udp   2049  nfs
    100227    2   udp   2049  nfs_acl
    100227    3   udp   2049  nfs_acl
    100003    2   tcp   2049  nfs
    100003    3   tcp   2049  nfs
    100227    2   tcp   2049  nfs_acl
    100227    3   tcp   2049  nfs_acl
    100005    1   udp  32798  mountd
    100005    2   udp  32798  mountd
    100005    3   udp  32798  mountd
    100005    1   tcp  32778  mountd
    100005    2   tcp  32778  mountd
    100005    3   tcp  32778  mountd
    100068    2   tcp  37135
    100068    3   tcp  37135
    100068    4   tcp  37135
    100068    5   tcp  37135
\end{verbatim}
\end{scriptsize}
\end{quote}


Note that some daemons such as nfsd use both protocols.

\subsubsection{Installing Netlog}

Download the distribution file via the Unix\_Security Topic (41) 
in the Systems-management conference within Starlink Forum and unpack it. 
Use the following command to extract the files:

\begin{quote}
\verb+% gzcat netlog-1.2_sol.tar.gz | tar xvf -+
\end{quote}

Currently this program only compiles under SunOS4 or Solaris 2.x.

\subsubsection{Running Netlog}
Login as ``root'' user and change into the {\it netlog-1.2/bin} 
subdirectory. To log tcp connections to an output file do the
following:

\begin{quote}
\verb+# ./tcplogger -f /tmp/tcplog.out+
\end{quote}

Press CTRL-C to stop the data collection. The output file 
{\it /tmp/tcplog.out} will be in binary format. To convert this to
ascii, type the following command:

\begin{quote}
\begin{scriptsize}
\begin{verbatim}
# ./extract -d /tmp/tcplog.out -o /tmp/tcplog.asc
# more /tmp/tcplog.asc
10/23/96 15:10:44 A1CC70EB gauss.ukc.ac.uk       43595 -> spock.ukc.ac.uk       6000
10/23/96 15:10:44 A1CCE1CC gauss.ukc.ac.uk       43596 -> spock.ukc.ac.uk       6000
10/23/96 15:10:48 1DA9C000 falcon.ukc.ac.uk      1023 -> tsela.ukc.ac.uk       5101
10/23/96 15:10:49  3D57200 raven.ukc.ac.uk       1023 -> tsela.ukc.ac.uk       5101
10/23/96 15:10:51 59283800 gos.ukc.ac.uk         1012 -> tsela.ukc.ac.uk       5101
10/23/96 15:10:52 5B7FD400 swallow.ukc.ac.uk     1023 -> tsela.ukc.ac.uk       5101
\end{verbatim}
\end{scriptsize}
\end{quote}

The first two columns are the date and time. The fourth column is the source
hostname with the source port in the next column. The next column is the 
destination host and port number.

%------------------------------------------------------------------------
\newpage
\appendix

\section{\label{a_sample_inetd.conf_file}\xlabel{a_sample_inetd.conf_file}A sample inetd.conf file}

\begin{quote}
\begin{verbatim}

#ident	"@(#)inetd.conf	1.22	95/07/14 SMI"	/* SVr4.0 1.5	*/
#
#
#
# Configuration file for inetd(1M).  See inetd.conf(4).
#
# To re-configure the running inetd process, edit this file, then
# send the inetd process a SIGHUP.
#
# Syntax for socket-based Internet services:
#  <service_name> <socket_type> <proto> <flags> <user> <server_pathname> <args>
#
# Syntax for TLI-based Internet services:
#
#  <service_name> tli <proto> <flags> <user> <server_pathname> <args>
#
# Ftp and telnet are standard Internet services.
#
ftp	stream	tcp	nowait	root	/usr/local/sbin/tcpd	in.ftpd
telnet	stream	tcp	nowait	root	/usr/local/sbin/tcpd	in.telnetd
#
# Tnamed serves the obsolete IEN-116 name server protocol.
#
name	dgram	udp	wait	root	/usr/sbin/in.tnamed	in.tnamed
#
# Shell, login, exec, comsat and talk are BSD protocols.
#
shell	stream	tcp	nowait	root	/usr/local/sbin/tcpd	in.rshd
login	stream	tcp	nowait	root	/usr/local/sbin/tcpd	in.rlogind
......
.....
....
etc

\end{verbatim}
\end{quote}

\newpage

\section{\label{a_tiger_config.file}\xlabel{a_tiger_config.file}A TIGER config file}

\begin{quote}
\begin{verbatim}

# -*- sh -*-
#
#     tiger - A UN*X security checking system
#     Copyright (C) 1993 Douglas Lee Schales, David K. Hess, David R. Safford
#
#     Please see the file `COPYING' for the complete copyright notice.
#
# site-sample - 04/22/93
#
#-----------------------------------------------------------------------------
#
#
# Sample site configuration file
# 
# Rename this to either "site" or "site-`hostname`"
#
#------------------------------------------------------------------------
#
# How to determine whether user home directory is local or remote?
#
GETFSHOST=getfs-std          # Do everybody... don't check
#GETFSHOST=getfs-nfs	     # Try to guess if it is NFS mount
#GETFSHOST=getfs-automount   # SUN automount tables
#GETFSHOST=getfs-amd         # BSD 4.4 AMD tables
#
# Name of file or NIS map containing automounter maps for user directories
#
AUTOHOMEMAP=auto_home

export GETFSHOST AUTOHOMEMAP
#
# List of '|' separated names this host is known by.
#
# i.e:
#HOSTNAMESLIST="jupiter|jupiter.tamu.edu|sc.tamu.edu|sc"
HOSTNAMESLIST="$HOSTNAME"
#
# Any of the utilities can be replaced by placing assignments here
# Most of the variables are the uppercase version of the command
# name, though there are a few exceptions.
#
#FIND=/usr/local/gnu/bin/find
#AWK=/usr/local/bin/mawk
#
#------------------------------------------------------------------------
#
# Define where Crack is installed (this is the path to the Crack script)
#
CRACK=/usr/local/security/crack

\end{verbatim}
\end{quote}

\end{document}
