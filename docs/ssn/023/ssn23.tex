\documentclass[11pt]{article}
\pagestyle{myheadings}

% -----------------------------------------------------------------------------
% ? Document identification
\newcommand{\stardoccategory}  {Starlink System Note}
\newcommand{\stardocinitials}  {SSN}
\newcommand{\stardocsource}    {ssn\stardocnumber}
\newcommand{\stardocnumber}    {23.3}
\newcommand{\stardocauthors}   {T.M.Gledhill \\
                                C.A.Clayton}
\newcommand{\stardocdate}      {8 Jan 1997}
\newcommand{\stardoctitle}     {Starlink Benchmarking Utility}
\newcommand{\stardocabstract}  {The Starlink Benchmarking Utility provides a
set of tools for investigating the performance of computer systems running
astronomy data reduction software. This manual is intended for Starlink Site Managers and describes how to install and use the package.}
\newcommand{\stardocversion}   {Version \pkgver}
\newcommand{\stardocmanual}    {User's Manual}
% ? End of document identification
% -----------------------------------------------------------------------------

\newcommand{\stardocname}{\stardocinitials /\stardocnumber}
\markright{\stardocname}
\setlength{\textwidth}{160mm}
\setlength{\textheight}{230mm}
\setlength{\topmargin}{-2mm}
\setlength{\oddsidemargin}{0mm}
\setlength{\evensidemargin}{0mm}
\setlength{\parindent}{0mm}
\setlength{\parskip}{\medskipamount}
\setlength{\unitlength}{1mm}

% -----------------------------------------------------------------------------
%  Hypertext definitions.
%  ======================
%  These are used by the LaTeX2HTML translator in conjunction with star2html.

%  Comment.sty: version 2.0, 19 June 1992
%  Selectively in/exclude pieces of text.
%
%  Author
%    Victor Eijkhout                                      <eijkhout@cs.utk.edu>
%    Department of Computer Science
%    University Tennessee at Knoxville
%    104 Ayres Hall
%    Knoxville, TN 37996
%    USA

%  Do not remove the %begin{latexonly} and %end{latexonly} lines (used by 
%  star2html to signify raw TeX that latex2html cannot process).
%begin{latexonly}
\makeatletter
\def\makeinnocent#1{\catcode`#1=12 }
\def\csarg#1#2{\expandafter#1\csname#2\endcsname}

\def\ThrowAwayComment#1{\begingroup
    \def\CurrentComment{#1}%
    \let\do\makeinnocent \dospecials
    \makeinnocent\^^L% and whatever other special cases
    \endlinechar`\^^M \catcode`\^^M=12 \xComment}
{\catcode`\^^M=12 \endlinechar=-1 %
 \gdef\xComment#1^^M{\def\test{#1}
      \csarg\ifx{PlainEnd\CurrentComment Test}\test
          \let\html@next\endgroup
      \else \csarg\ifx{LaLaEnd\CurrentComment Test}\test
            \edef\html@next{\endgroup\noexpand\end{\CurrentComment}}
      \else \let\html@next\xComment
      \fi \fi \html@next}
}
\makeatother

\def\includecomment
 #1{\expandafter\def\csname#1\endcsname{}%
    \expandafter\def\csname end#1\endcsname{}}
\def\excludecomment
 #1{\expandafter\def\csname#1\endcsname{\ThrowAwayComment{#1}}%
    {\escapechar=-1\relax
     \csarg\xdef{PlainEnd#1Test}{\string\\end#1}%
     \csarg\xdef{LaLaEnd#1Test}{\string\\end\string\{#1\string\}}%
    }}

%  Define environments that ignore their contents.
\excludecomment{comment}
\excludecomment{rawhtml}
\excludecomment{htmlonly}

%  Hypertext commands etc. This is a condensed version of the html.sty
%  file supplied with LaTeX2HTML by: Nikos Drakos <nikos@cbl.leeds.ac.uk> &
%  Jelle van Zeijl <jvzeijl@isou17.estec.esa.nl>. The LaTeX2HTML documentation
%  should be consulted about all commands (and the environments defined above)
%  except \xref and \xlabel which are Starlink specific.

\newcommand{\htmladdnormallinkfoot}[2]{#1\footnote{#2}}
\newcommand{\htmladdnormallink}[2]{#1}
\newcommand{\htmladdimg}[1]{}
\newenvironment{latexonly}{}{}
\newcommand{\hyperref}[4]{#2\ref{#4}#3}
\newcommand{\htmlref}[2]{#1}
\newcommand{\htmlimage}[1]{}
\newcommand{\htmladdtonavigation}[1]{}

% Define commands for HTML-only or LaTeX-only text.
\newcommand{\html}[1]{}
\newcommand{\latex}[1]{#1}

% Use latex2html 98.2.
\newcommand{\latexhtml}[2]{#1}

%  Starlink cross-references and labels.
\newcommand{\xref}[3]{#1}
\newcommand{\xlabel}[1]{}

%  LaTeX2HTML symbol.
\newcommand{\latextohtml}{{\bf LaTeX}{2}{\tt{HTML}}}

%  Define command to re-centre underscore for Latex and leave as normal
%  for HTML (severe problems with \_ in tabbing environments and \_\_
%  generally otherwise).
\newcommand{\setunderscore}{\renewcommand{\_}{{\tt\symbol{95}}}}
\latex{\setunderscore}

% -----------------------------------------------------------------------------
%  Debugging.
%  =========
%  Remove % from the following to debug links in the HTML version using Latex.

% \newcommand{\hotlink}[2]{\fbox{\begin{tabular}[t]{@{}c@{}}#1\\\hline{\footnotesize #2}\end{tabular}}}
% \renewcommand{\htmladdnormallinkfoot}[2]{\hotlink{#1}{#2}}
% \renewcommand{\htmladdnormallink}[2]{\hotlink{#1}{#2}}
% \renewcommand{\hyperref}[4]{\hotlink{#1}{\S\ref{#4}}}
% \renewcommand{\htmlref}[2]{\hotlink{#1}{\S\ref{#2}}}
% \renewcommand{\xref}[3]{\hotlink{#1}{#2 -- #3}}
%end{latexonly}
% -----------------------------------------------------------------------------
% ? Document-specific \newcommand or \newenvironment commands.
\renewcommand{\thepage}{\roman{page}}

%  Package Version.
\newcommand{\pkgver}     {v1.0}

% Package FTP directory.
\newcommand{\pkgftpdir}  {/pub/tools}

% Package FTP server.
\newcommand{\pkgftpsrv} {starlink-ftp.rl.ac.uk}

% Package URL.
\newcommand{\pkgurl} {ftp://\pkgftpsrv\pkgftpdir}

% ? End of document-specific commands
% -----------------------------------------------------------------------------
%  Title Page.
%  ===========
\renewcommand{\thepage}{\roman{page}}
\begin{document}
\thispagestyle{empty}

%  Latex document header.
%  ======================
\begin{latexonly}
   CCLRC / {\sc Rutherford Appleton Laboratory} \hfill {\bf \stardocname}\\
   {\large Particle Physics \& Astronomy Research Council}\\
   {\large Starlink Project\\}
   {\large \stardoccategory\ \stardocnumber}
   \begin{flushright}
   \stardocauthors\\
   \stardocdate
   \end{flushright}
   \vspace{-4mm}
   \rule{\textwidth}{0.5mm}
   \vspace{5mm}
%   \begin{center}
%   {\Large\bf \stardoctitle}
%   \end{center}
   \vspace{5mm}
    \begin{center}
      {\Huge\bf  \stardoctitle \\ [2.5ex]}
      {\LARGE\bf \stardocversion \\ [4ex]}
      {\Huge\bf  \stardocmanual}
    \end{center}

% ? Heading for abstract if used.
   \vspace{10mm}
   \begin{center}
      {\Large\bf Abstract}
   \end{center}
% ? End of heading for abstract.

\end{latexonly}

%  HTML documentation header.
%  ==========================
\begin{htmlonly}
   \xlabel{}
   \begin{rawhtml} <H1> \end{rawhtml}
      \stardoctitle\\
      \stardocmanual\\
      \stardocversion
   \begin{rawhtml} </H1> \end{rawhtml}

% ? Add picture here if required.
% ? End of picture

   \begin{rawhtml} <P> <I> \end{rawhtml}
   \stardoccategory\ \stardocnumber \\
   \stardocauthors \\
   \stardocdate
   \begin{rawhtml} </I> </P> <H3> \end{rawhtml}
      \htmladdnormallink{CCLRC}{http://www.cclrc.ac.uk} /
      \htmladdnormallink{Rutherford Appleton Laboratory}
                        {http://www.cclrc.ac.uk/ral} \\
      \htmladdnormallink{Particle Physics \& Astronomy Research Council}
                        {http://www.pparc.ac.uk} \\
   \begin{rawhtml} </H3> <H2> \end{rawhtml}
      \htmladdnormallink{Starlink Project}{http://star-www.rl.ac.uk/}
   \begin{rawhtml} </H2> \end{rawhtml}
   \htmladdnormallink{\htmladdimg{source.gif} Retrieve hardcopy}
      {http://star-www.rl.ac.uk/cgi-bin/hcserver?\stardocsource}\\

%  HTML document table of contents. 
%  ================================
%  Add table of contents header and a navigation button to return to this 
%  point in the document (this should always go before the abstract \section). 
  \label{stardoccontents}
  \begin{rawhtml} 
    <HR>
    <H2>Contents</H2>
  \end{rawhtml}
  \htmladdtonavigation{\htmlref{\htmladdimg{contents_motif.gif}}
        {stardoccontents}}

% ? New section for abstract if used.
  \section{\xlabel{abstract}Abstract}
% ? End of new section for abstract

\end{htmlonly}

% -----------------------------------------------------------------------------
% ? Document Abstract. (if used)
%  ==================
\stardocabstract
% ? End of document abstract
% -----------------------------------------------------------------------------
% ? Latex document Table of Contents (if used).
%  ===========================================
\newpage
\begin{latexonly}
  \begin {center}
    \rule{80mm}{0.5mm} \\ [1ex]
    {\Large\bf \stardoctitle \\ [2.5ex]
            \stardocversion} \\ [2ex]
     \rule{80mm}{0.5mm}
  \end{center}
  \vspace{20mm}
  \setlength{\parskip}{0mm}
  \tableofcontents
  \setlength{\parskip}{\medskipamount}
  \markright{\stardocname}
\end{latexonly}
% ? End of Latex document table of contents
% -----------------------------------------------------------------------------
\newpage
\renewcommand{\thepage}{\arabic{page}}
\setcounter{page}{1}


%------------------------------------------------------------------------------
%  Introduction.
%------------------------------------------------------------------------------
\newpage
\renewcommand{\thepage}{\arabic{page}}
\setcounter{page}{1}
\vspace{30mm}
\section{Introduction}

System benchmarking is important not only in deciding what line to take
in future hardware purchases, but also to check whether current systems are
set up correctly, whether software is executing properly and as a tool to
investigate configuration changes and tuning operations. Although there is
no shortage of available benchmarks, many of them are unfortunately either 
very simplistic, perhaps running a piece of floating point Fortran code in a loop, or test a very specific area of system performance. In both cases the
result is unlikely to give a true indication of how a real piece of software
will perform.

The Starlink Benchmarking Utility is an easy--to--use set of
tools for investigating the performance of computer systems when
running astronomy data reduction software and currently includes
benchmarks from the Starlink Software Collection and IRAF. The
intention is to benchmark the sort of applications that are likely to
be used by astronomers and hopefully to produce a performance estimate
which is of more relevance to astronomers.

The three main aims of the Utility are:

\begin{itemize}
\item To provide an indication of how well a machine performs when
carrying out tasks typical of astronomical data analysis.

\item To check that all Starlink machines show consistent performance
figures and that those figures are within expectations.

\item To act as a performance diagnostic in system tuning operations, allowing
the effects of hardware changes, such as adding extra memory or changing
disk configurations, to be investigated.
\end{itemize}

The principal result produced by the Starlink Benchmarking Utility is
the STARmark rating, which is described in Section~\ref{starmark}. In
addition, as of V1.0, an IRAFmark, based on the IRAF benchmarks, can
also be produced. These ratings should provide convenient benchmarks
for comparing the performance of different host systems. More detailed
benchmarking results can also be produced and this is discussed in
Section~\ref{results}.


%------------------------------------------------------------------------------
%   Installation.
%------------------------------------------------------------------------------

\section{Installation}

The package is currently supported on {\bf Solaris} and {\bf Digital
Unix} platforms at version \pkgver. Compressed tar files for each
platform are available from the Starlink anonymous ftp account.

\vspace{3mm}
\htmladdnormallinkfoot{starbench\_sun4\_Solaris\_\pkgver.tar.Z}{\pkgurl/starbench\_sun4\_Solaris\_\pkgver.tar.Z} ~~~~~~~~~~~~~~~~~(Solaris systems) 

\htmladdnormallinkfoot{starbench\_alpha\_OSF1\_\pkgver.tar.Z}{\pkgurl/starbench\_alpha\_OSF1\_\pkgver.tar.Z} ~~~~~~~~~~~~~ (Digital Unix systems) 


%\> starbench\_sun4\_Solaris\_\pkgver.tar.Z \> (Solaris systems) \\
%\> starbench\_alpha\_OSF1\_\pkgver.tar.Z \> (Digital Unix systems)    \\
%\end{tabbing}

Installation of the package is straightforward and proceeds as follows:

\begin{enumerate}
\item Move to the empty directory in which you wish to install the package.
A suitable place would be in the {\tt /star/local} tree, {\em e.g.} 
{\tt /star/local/starbench}. Space requirements are approximately 2MB for
the compressed tar archive and 5.5MB for the installed package.

\item Obtain the compressed tar file, appropriate to your platform, from the Starlink ftp server either {\em via} the URLs given above or directly {\em via} FTP as shown below:

{\tt    \% ftp \pkgftpsrv            \\
        (login as 'anonymous')                 \\
        ftp$>$ cd \pkgftpdir                   \\
        ftp$>$ bin                             \\
        ftp$>$ get starbench\_***\_\pkgver.tar.Z   \\
        ftp$>$ quit }

\item Set the SYSTEM environment variable to identify your system. 

For Digital Unix systems: \\
{\tt \% setenv SYSTEM alpha\_OSF1}

For Solaris systems: \\
{\tt \% setenv SYSTEM sun4\_Solaris}

\item Unpack the {\tt tar} archive:

{\tt \% zcat starbench\_\$SYSTEM\_\pkgver.tar.Z | tar xvf - }

\item Execute the {\tt mk} script. Note that the package will install into the
current directory.

{\tt \% ./mk install}
\end{enumerate}

%------------------------------------------------------------------------------
% Running the benchmarks.
%------------------------------------------------------------------------------

\section{Running the Benchmarks}

The benchmarks are designed to be set up for automatic execution by the
Unix {\tt cron} daemon. This method of execution is preferred since,
not only does it involve a lesser degree of user interaction, but it
allows the benchmarks to be executed a number of times, at regular
intervals, so as to build up a statistical base of results from which
more reliable conclusions can be drawn.

\subsection{Prior Requirements}

The following setup is required in order to run the benchmarks.

\begin{itemize}
\item To run the benchmarks in the recommended fashion the user must
have permission to run jobs using {\tt cron} and {\tt at}. This may
necessitate placing the username in the files {\tt /etc/cron.d/cron.allow}
and {\tt /etc/cron.d/at.allow}.

\item The benchmarks themselves create large temporary files so that a 
minimum of 11MB of free space is recommended in the working directory
(different from the installation directory).

\item It is assumed that the {\bf ussc} is installed under {\tt /star} and
that the user has access to it. 

If IRAF benchmarks are to be run, it is
assumed that IRAF software is installed under {\tt /iraf/iraf}. However,
the environment variable {\tt iraf} can be set to point to ``non-standard''
locations. For example, if IRAF is installed in {\tt /star/iraf/iraf} then...

{\tt > setenv iraf /star/iraf/iraf}

...before starting up the benchmark package.

\item The {\tt tcsh} is used to time the benchmarks and is assumed to be in
{\tt /usr/local/bin}.

\item The following packages are used by the benchmark suite and must be 
installed for the benchmarks to function properly:

  \begin{itemize}
    \item {\bf USSC: } kappa, pisa, specdre, ccdpack
    \item {\bf IRAF: } ccdred, daophot, images
  \end{itemize}
\end{itemize}

\subsection{Automatic Benchmarking -- Recommended Procedure}

The procedure for cron execution of the benchmarks will now be described, along
with some guidelines intended to promote consistency in the results. 

\begin{enumerate}

\item {\bf Move to a suitable directory}

During execution, the benchmarks create a number of temporary files,
some of which can be quite large. These files are created in the
current directory. The best policy is to move to a directory with a
large amount of free space before issuing any benchmark commands; a
minimum of 11MB is recommended. {\em In any case, the benchmarks
cannot be executed whilst the current directory is the installation
directory}.

The location of the filesystems containing the current working
directory, the benchmarking package, the Starlink software (ussc), and
the iraf installation can all be expected to influence the benchmark
results (this will be discussed later).  It is suggested that, whenever
possible, the current working directory should be on a filesystem which
is {\bf local to the machine being tested}. In addition, the filesystem
should not be close to capacity, since full filesystems often
experience reduced I/O efficiency. For more information on how disk I/O
may affect the benchmark results, see Section~\ref{diskeffects}.


\item {\bf Source the startup script}

To make the package available, source the startup script ({\tt starbench.csh})
in the installation directory. For example, if the package was installed in
{\tt /star/local/starbench} then...

{\tt host\% source /star/local/starbench/starbench.csh}

An alias for this command could be defined in your {\tt .tcshrc} script. This
command sets up various environment variables, checks the software installation
and also checks which packages ({\tt ussc} and {\tt iraf}) are installed
on the system. 
 

\item {\bf Submitting the benchmarks for execution}

The benchmarks are submitted for execution by the cron using a single interactive script, {\tt submit}, which prompts for the necessary information. Here is a transcript of an example session:

\begin{verbatim}
host% submit

 Submit
 ------

 This is an  interactive script to submit benchmarks for execution
 by the cron at regular intervals. In this way a benchmark profile
 of the host machine can be built up.

 The benchmarks will be run every hour. Give the number of minutes
 past the hour to start each job. If you are running benchmarks on
 several machines then it is a good idea to space them out.       

 Number of minutes past the hour [0] : 30

 How many times should the benchmarks be run? The default is to run
 them over a twenty-four hour period.

 Number of times to run benchmarks [24] : 12

 Give the name of a directory to hold the log files generated by
 the benchmarks. The default is to  put them in the current work
 directory.

 Give a directory for the log files [.] : 

 The default is to just run the USSC benchmarks [ussc]. If you wish
 you may run the IRAF  benchmarks instead  [iraf] or both  USSC and
 IRAF benchmarks [both].

 Specify which benchmarks to run [ussc] : both


 ...cron jobs started

warning: commands will be executed using /usr/local/bin/tcsh
job 793688940.a at Sat Feb 25 05:09:00 1995

 ...cron termination procedure queued

host%
\end{verbatim} 

The user is given the choice of when to execute the benchmarks (minutes
past the hour) and how many times to execute them. When making this
choice you should obviously avoid times when other heavy system
processes, such as disk dumps or scratch disk purges, are executing. In
addition, if benchmarks are being executed on more than one host, but
are using the same disk space, then they should be spaced out in time
so that they cannot interfere with one another. A reasonable option
would be to benchmark 3 hosts at a time with executions separated by 20
minutes ({\em i.e.} 0, 20 and 40 minutes past the hour).

The user is given the choice of running the {\tt ussc} benchmarks or the
{\tt iraf} benchmarks, or running both. {\em The default is to just run 
the ussc benchmarks}.

The {\tt submit} script makes an entry in the user's crontab file
causing the benchmarks to be executed every hour at the requested time.
In addition, a script to terminate execution of the benchmarks, after
the required number of runs, and return the user's crontab file to its
previous state is queued for future execution using the Unix {\tt at}
command.

Each time the benchmark suite is executed, the results are recorded in
a log file (filetype {\tt .log}) in the directory specified by the
user. These log files contain the benchmark timings along with
information on the host system characteristics. They are simple ASCII
files and can be examined at any time.


\end{enumerate}

\subsection{Interactive Benchmarking}

Although {\tt submit} is the preferred command for running them,
it is also possible to run the benchmarks on an interactive one-off basis 
using the command {\tt bench}. Simply type {\tt bench}, after sourcing the
package startup script (previous section), and this will run the benchmarks
and print the results out to the screen. 

If the {\tt bench} command line switch {\tt -l} is used and the name of
a log file given then the results will be sent to a log file rather
than to the screen. If this option is used then the procedure could be
backgrounded. This direct method of running the software will only
produce a snapshot of the system and the results may not be as reliable
as the multiple executions generated by {\tt submit}.  However, it is
useful for getting quick results or when troubleshooting.  For
example:

{\tt host\% bench -l quick-bench.log}

Note that log files should have {\tt .log} as their file type so that they
can be recognised as such by the {\tt scan} command.

The packages to benchmark can be selected using the {\tt -p} command line
switch, which takes the value {\tt iraf}, {\tt ussc} or {\tt both}. For
example, to only use the iraf packages:

{\tt host\% bench -l quick-bench.log -p iraf}

The default is to just use the {\tt ussc} packages.

%\newpage
\subsection{Processing the Log Files}

The log files produced by each execution of the benchmarks can all be
processed together using the {\tt scan} command. This command accesses
all of the log files in a given directory and sorts them according to
host. Benchmarks executed on different hosts may log their results to a
common directory and {\tt scan} will sort them out, producing a results
file for each host. For each host, average benchmark timings are
computed and written to a file {\tt <host>.bch}.  These files are
formatted lists of numbers and are largely for internal use within the
package. If a {\tt .bch} file already exists for a host, then {\tt scan}
will rename it and produce a new one.

The {\tt scan} command then reads the {\tt <host.bch>} files (one for
each host) and calculates mean execution times and errors for each
benchmark. These results are then compared with `standard' measurements
and a STARmark rating is calculated. If the IRAF benchmarks have
been run, then an IRAFmark rating is produced. An example transcript
follows:


\begin{verbatim}
host% cd /data/user/results
host% scan

 Scan
 ----

 This utility will scan through the benchmark logfiles created by
 the 'submit' or 'bench -l' commands to extract statistics and to
 calculate a  mean value and error for each benchmark.

 The performance is then compared with 'standard' results and the
 STARmark96  benchmark rating is calculated. If IRAF benchmarking
 was selected, then an IRAFmark96 rating will also be calculated.
             

 Give the directory path for the log files. The default assumes
 that they are in the current directory.                       

 Log files directory [.] : <RETURN>

 Benchmark Results
 =================

 Starlink Benchmark Utilities v1.0

 Hostname           : uhsul1
 User               : tmg
 Opsys              : SunOS 5.5
 Platform           : SUNW,Ultra-1
 Physical Memory    : 96.00Mb
 Online Processors  : 1
 Processor Speed    : 143MHz
 USSC version       : USSC176

 STARmark96 : 2.44 +/- 0.02  (based on 12 measurements)
 IRAFmark96 : 2.89 +/- 0.04  (based on 12 measurements)

host%
\end{verbatim} 

The results produced by the {\tt scan} command can be sent to a log file
{\em as well as} to the screen using the {\tt -l} command line switch:

{\tt host\% scan -l results.lis}

The {\tt scan} command can also produce extended results enabling a
more detailed investigation of a system's performance. This facility is
also very useful for checking that the benchmarks have executed correctly
and that the results are not unduly influenced by any one application.
Using these extended results is described in section~\ref{results}.
  
\newpage
\section{The STARmark and IRAFmark Ratings}
\label{starmark}

The STARmark and IRAFmark ratings are intended as estimates of how well a
machine performs a sequence of astronomical data reduction tasks using
Starlink and IRAF software. However, the applications that go to make up
these two rating are totally different, including different mixes of CPU
activity and disk I/O. It is therefore {\em not possible} to use STARmark
and IRAFmark ratings to compare the performance of Starlink and IRAF
software on a particular machine.  They are, simply, measuring different
things. The IRAF benchmarks have been included to provide another
performance estimate for a particular hardware configuration, which will
be more appropriate for sites running predominantly IRAF software. 

Neither are the STARmark and IRAFmark ratings intended as definitive
performance ratings for a machine. Indeed, they cannot be since they
only test a limited subset of a machine's capability.  Rather than
trying to separate out and measure specific aspects of performance
(such as CPU power, I/O efficiency, caching ability {\em etc.}) the
current philosophy behind this package is to fold all of these factors
into global performance estimates which, hopefully, will be of more
relevance to the astronomer sitting in front of the screen.  The
STARmark and IRAFmark figures simply provide a convenient single-figure
handle on these estimates.

Each benchmark is timed (using the shell {\tt time} command) to give
the ``total CPU execution time'' or {\bf tcpu} for the benchmark. This
is the sum of the user and kernel mode CPU times. In order to calculate
the STARmark and IRAFmark ratings, the {\bf tcpu} figures for each
benchmark must be calibrated against a `standard' result.  In the case
of the current package (\pkgver) the standard is a 64Mb SPARC10 model
51 (RAL machine rlssp0). Comparison of the benchmark tcpu times on the
machine being tested with those on the standard machine gives a {\bf
speed} rating for each benchmark ({\em i.e.} how much faster the
benchmark runs on the machine being tested than on the standard
machine), so that, for any particular benchmark:

\vspace{5mm}
\begin{large}
~~~~~~~~speed = $\frac{{\sf tcpu_{rlssp0}}}{{\sf tcpu_{machine}}}$
\end{large}
\vspace{5mm}

Four benchmarks are currently incorporated in the STARmark rating,
based on the KAPPA, PISA, SPECDRE and CCDPACK Starlink packages. The
{\bf speed} ratings for these benchmarks are averaged to form the 
STARmark rating.

\begin{tabbing}
~~~~~~~~STARmark = 0.25(speed(KAPPA) + speed(PISA) \=+ speed(SPECDRE)  \\
                                                     \>+ speed(CCDPACK))
\end{tabbing}

The error quoted on the STARmark figure is derived from the spread in
the results from multiple executions of the benchmark package using the
{\tt submit} command.  This spread in results is likely to reflect
variations in load and activity on the system throughout the timespan
of the benchmark executions. The quoted error therefore represents a
lower limit on the real uncertainty which will include systematic
effects ({\em e.g.} whether the working directory is on a local or
remote filesystem) which may well dominate.

In a similar way, the IRAFmark rating is based on three IRAF
packages: CCDRED, DAOPHOT and IMAGES.

\begin{tabbing}
~~~~~~~~IRAFmark = (speed(CCDRED) + speed(DAOPHOT) \=+ speed(IMAGES))/3.0  \\
\end{tabbing}


{\bf The current list of expected STARmark and IRAFmark figures for various
machines and configurations is available to Site Managers through the
Starlink Forum System Managers' Conference, \htmladdnormallinkfoot{Note 4.0}{http://rlsaxps.bnsc.rl.ac.uk/Forum/Systems-Management/4}}.


%\newpage
\section{Extended Results}
\label{results}

The {\tt scan} command will print out more detailed results if given
the {\tt -f} command line switch. An example transcript is given below.
 

\begin{verbatim}
host% cd /data/user/results
host% scan -f

 Scan
 ----

 This utility will scan through the benchmark logfiles created by
 the 'submit' or 'bench -l' commands to extract statistics and to
 calculate a  mean value and error for each benchmark.

 The performance is then compared with 'standard' results and the
 STARmark96  benchmark rating is calculated. If IRAF benchmarking
 was selected, then an IRAFmark96 rating will also be calculated.
           

 Give the directory path for the log files. The default assumes
 that they are in the current directory.                      

 Log files directory [.] : 


 Benchmark Results
 =================

 Starlink Benchmark Utilities v1.0

 Hostname           : uhsul1
 User               : tmg
 Opsys              : SunOS 5.5
 Platform           : SUNW,Ultra-1
 Physical Memory    : 96.00Mb
 Online Processors  : 1
 Processor Speed    : 143MHz
 USSC version       : USSC176

 STARmark96 : 2.44 +/- 0.02  (based on 12 measurements)
 IRAFmark96 : 2.89 +/- 0.04  (based on 12 measurements)

 Average Load : 0.065 +/- 0.106

 Benchmark  ucpu    kcpu       tpcu            Speed          Elapsed
 ---------  ----    ----       ----            -----          -------

       FFT  23.37   0.46  23.83 +/- 0.04    3.17 +/- 0.01   25.14 +/- 0.51

       SLA  12.06   0.02  12.08 +/- 0.10    1.47 +/- 0.01   12.42 +/- 0.18

     KAPPA   9.77   3.87  13.64 +/- 0.22    2.68 +/- 0.04   86.64 +/- 6.33

      PISA  10.34   2.40  12.75 +/- 0.23    2.43 +/- 0.04   48.47 +/- 1.12

   SPECDRE   7.87   3.81  11.68 +/- 0.18    2.47 +/- 0.04   35.01 +/- 0.54

   CCDPACK  13.32   5.33  18.66 +/- 0.18    2.17 +/- 0.02   67.20 +/- 1.94
 
    CCDRED  10.12   4.00  14.11 +/- 0.23    2.59 +/- 0.04  101.44 +/- 4.40

   DAOPHOT  16.05   3.91  19.96 +/- 0.17    1.55 +/- 0.01   37.18 +/- 1.49

    IMAGES  11.28   1.60  12.88 +/- 0.43    2.24 +/- 0.07   46.45 +/- 5.92


\end{verbatim}

The full output shows the CPU execution times recorded for individual 
benchmarks, the {\bf speed} ratings and the elapsed times.
The nomenclature is as follows:

\begin{tabbing}
~~~~~~\=~~~~~~~~~~~~~~~~~~~~~~~~~~~~~~\=~~~~~~~\=         \\
\> ucpu   \>:  \> Average CPU time (secs.) in user mode for each benchmark. \\
\>        \>   \> This is the time the CPU spends executing the user's code.\\
\> kcpu   \>:  \> Average CPU time (secs.) in kernel mode for each benchmark. \\
\>        \>   \> This is the time the CPU spends executing system instructions.        \\
\> tcpu   \>:  \> Average Total CPU time (ucpu + kcpu) for each benchmark. \\
\>        \>   \> The error on tcpu (secs.) is the standard deviation. \\
\> speed  \>:  \> The ratio of tcpu on the standard machine to tcpu on the \\ 
\>        \>   \> machine being benchmarked. The errors on the `speed' \\
\>        \>   \> figures are simply the propagated tcpu errors. \\
\> elapsed \>: \> Average elapsed time (secs.) for each benchmark. The error \\
\>        \>   \> is the standard deviation.
\end{tabbing}

Since it is the four `speed' figures for the KAPPA, PISA, SPECDRE and
CCDPACK benchmarks that are averaged to give the STARmark rating, a
full output can be used to check whether the computed STARmark is being
biased by any particular benchmark. The same applies for the IRAFmark.
The benchmark scripts do check for successful completion and flag log
files with problems so that {\tt scan} will not use them. However, it
is also a good idea to check the errors on the {\bf tcpu} and {\bf
speed} figures to make sure that the results have not been influenced
by a spurious result. If the errors appear high then the log files
should be examined.

In addition to the `astronomy' benchmarks, the full output also shows
results for the FFT and SLA benchmarks. These benchmarks are purely CPU
thrashers and do not do any disk I/O (although the FFT benchmark can
page heavily on systems with less than 32MB physical memory). They can
be useful in assessing to what degree the other benchmarks are likely
to be influenced by disk I/O considerations but they are not
incorporated in the STARmark or IRAFmark figures.

The individual benchmarks are described in Appendix A.


\section{Disk I/O Effects}
\label{diskeffects}

The package does not, at present, contain a disk I/O benchmark, although the
inclusion of one has been considered at some length. The effects of disk I/O efficiency on overall performance are obviously important, especially in
astronomy where large data files are routinely accessed. This section 
documents some of the problems inherent in producing a meaningful `disk
benchmark'. 

First of all, one has to decide what to measure. The processes of disk
reading and writing can be split into two parts. The first part, which
we will call the `internal' part, incorporates the calls to system I/O
routines and the transfer of data to or from internal buffers and
caches. The second part, which we will call the `external' part,
incorporates the actual transfer of information to or from a physical
location on the disk. The efficiency of the internal part of disk I/O
depends on the machine and operating system whereas the efficiency of
the external part of disk I/O depends on the disk hardware and bus
technology. These two aspects of disk I/O should really be benchmarked
separately, however, there are too many possible hardware combinations
here to produce a `standard' benchmark result.  Yet it is precisely the
combined internal and external efficiencies which influence the overall
disk I/O performance of the system.

The benchmarks in this package measure the {\bf tcpu} (total CPU)
times. In disk I/O processes it is only the internal (system) part of
the I/O that is clocked by the CPU and hence appears in the {\bf tcpu}
figure. The external part of the I/O operation (physically reading and
writing) only appears in elapsed time. The elapsed time that it takes a
process to execute depends directly on what is going on on the rest of
the system, in other words on the system load.  Without knowing in
advance how elapsed time on a particular machine scales with system
load, the external portion of disk I/O cannot be benchmarked. One
possible solution is to run a disk benchmark on a completely empty
system where load is not a factor. However, even in this case the
effects of system daemons and processes may need to be considered.
Most background system daemon activity can legitimately be considered
as part of the operating system, however sporadic activity caused, for
example, by remote ftp connections or tftp requests from booting
hardware, could influence the benchmarks.

In cases where a disk is not local to the system running the benchmarks
then things are even more complicated, since the load on the machine
serving the disk must be taken into consideration.

Whenever possible the benchmarks should be executed from a disk
which is local to the machine being tested so as to minimise the effects
of load on the file server on the benchmark results. In addition, filesystems
which are close to capacity  or which are heavily fragmented should be avoided
since they may suffer from reduced I/O performance. If it is suspected that
the choice of disk is unduly influencing the benchmark results then the
benchmarks should be run from several different disks and the results
compared.

Finally, disk I/O benchmarks are heavily influenced by the effects of a
disk cache. I/O to often-accessed small files can often be fulfilled
from the cache and thus a benchmark involving such files would really
be testing the speed at which your CPU can read blocks out of the cache
and not the speed of your disk I/O subsystem.  However, if your
application actually does do a lot of reading and writing of small
files, then the inclusion of the effects of the cache in your benchmark
is valid. Also, the record length size plays a major role in
performance measurements. Hence, the most reliable disk benchmark is
your actual application. This is the approach taken in the present
benchmarking suite. Real astronomical applications are used and thus
they naturally include disk I/O elements of the relevant types and in the
appropriate proportions.

\section{Elapsed Time}
The STARmark and IRAFmark ratings (Section~\ref{starmark}) only
consider the performance of a machine in terms of CPU time consumed and
make no use of elapsed time.  This is simply because elapsed time is
highly dependant on machine load and it is not possible to produce
reliable benchmarks even if one measures the machine load and attempts
to account for it. Furthermore, it is usually not possible to measure
elapsed time on an ``unloaded'' system since a networked Unix system is
never idle.


Measurements of CPU usage, on the other hand, are highly repeatable and
thus are suitable as a benchmark. They are also useful for identifying
small changes in performance resulting from changes to system
configuration.
 
Although elapsed time is difficult to interpret, it is of great
interest to the interactive user who experiences elapsed rather than
CPU time. The elapsed time is available in the individual log files
produced by {\tt submit} and {\tt bench -l} and is output by the {\tt
scan -f} command. It can be particularly important when exploring the
effects of disk location. For example, trials have shown that elapsed
time for a benchmark run can increase by up to 500\% when a remote disk
is used rather than a local one whereas the increase in CPU time,
although measurable, was small in comparison.  Bearing in mind the
difficulties inherent in benchmarking disk I/O, it is still often
possible to extract useful information from the elapsed times and we
suggest some guidelines for doing so:

\begin{itemize}
\item As described in Section~\ref{diskeffects} and above, elapsed time is
highly dependent on system load in a way which is hard to quantify. Pick
a period when the system should be relatively quiescent, {\em e.g.} overnight.
Avoid other cron procedures such as disk dumps or scratch disk purges. Beware
of http or ftp connections. When disk space (working directory, benchmark 
software or ussc) are on remote disks then activity on the disk server needs
to be considered also.
\item Run the benchmarks over an extended period of time using the {\tt
submit} command. For example, from midnight to 6am on several days of
the week.  Elapsed times are especially susceptible to sporadic system
activity. Establishing a large base of measurements spread over time
will smooth out these effects.
\item Use the mean elapsed times produced by {\tt scan -f} and check
their standard deviations. If the standard deviation on a measurement
appears large then check the benchmark log
({\tt .log}) files for anomalous results. If you suspect that one
particular measurement is unduly influencing the result then try
removing the log file ({\em e.g.} by renaming the {\tt .log} extension)
and then run {\tt scan -f} again.
\end{itemize}

 

\section{Other Benchmarks}

The benchmarks in this package are specifically intended to evaluate
performance on systems principally running Starlink and IRAF software.
On systems where other packages ({\em e.g.} AIPS) are heavily used then
a benchmark specific to that package may be more appropriate.  Such
benchmarks may have been provided by the suppliers of the software.

A number of benchmarks based on IRAF tasks are included in this package
at V1.0. However, a suite of more specialized IRAF benchmarks, which
test the installation and performance of IRAF software, exists and was
produced by NOAO. These benchmarks are described in the document {\em A
set of Benchmarks for Measuring IRAF System Performance} written by
Doug Tody which is available from the Starlink IRAF mirror as
\htmladdnormallinkfoot{bench.ps.Z}{ftp://star-www.rl.ac.uk/pub/iraf/iraf/docs/bench.ps.Z}.


\subsection{Systems running AIPS}

A suite of benchmarks for systems running AIPS is available (the ``Dirty Dozen
Test'' or DDT) which will produce an AIPSmark rating. 
The DDT suite is a reliable way of determining two things: 
firstly that a given AIPS installation gives accurate results, and 
secondly it is a measure of the performance of AIPS on the given system. 

Full details of the AIPS DDT, including latest results, can be found  \htmladdnormallinkfoot{at this URL}{http://info.cv.nrao.edu/aips/ddt.html}

\section{Miscellaneous Issues}

\subsection{Changes to the ussc}
\label{ussc-changes}

The {\bf ussc} is continually being updated and since the benchmarks use the 
installed copy of the collection, the tests being run will gradually
change over time. Hence, it will be necessary to re-run the benchmarks
every so often to measure the performance of machines with the current
version of the {\bf ussc}. The standard figures embedded in the benchmarking
software will be updated once a year and the year identifier in the
STARmark rating incremented to distinguish results from different 
incarnations of the software.

It might seem odd to use the everchanging installed copy of the 
software for the benchmarks rather than a self contained, static copy.
However, the size of the benchmarking package would be enormous if
it had to carry a private copy of the applications around with it.
Moreover, we are interested in how fast new machines run our current
release of the software, not a frozen copy, and hence despite the
obvious drawback, we have elected to use the installed {\bf ussc} for
benchmarking.

\subsection{New Versions of IRAF}

Similar arguments to those in Section~\ref{ussc-changes} apply with
regard to IRAF software. The current V1.0 release of this package has
been tested with IRAF V2.10.4. As new versions of IRAF appear, modifications
to this package will no doubt be required.

\subsection{Multiprocessor Machines}

The benchmarks in this package do not test the benefit of multiprocessor
machines. Since benchmark tasks all execute sequentially it is unlikely 
that they would take advantage of multiple processors.

\subsection{Manufacturers' SPECmark Ratings}

While the results from this package should be in broad agreement with manufacturers' SPECmark ratings, they will provide a more realistic performance
estimate for Starlink machines. SPECmark ratings tend to indicate the 
{\em potential} that it is possible to realise with a machine rather than
the performance that will actually be returned when running `real' applications.

\section{Acknowledgements}

The production of this IRAF benchmarking software has been undertaken
after discussions with Doug Tody, NOAO, the developer of IRAF.  However,
the benchmarking package has been developed completely independently of
the IRAF team so that any errors in the package or spurious results that
it might produce are wholly the responsibility of Starlink.
 
\newpage
\appendix
\section{Benchmark Descriptions}

This appendix describes each benchmark and the operations it performs.

{\bf FFT:}

A fortran program which runs a Fast Fourier Transform algorithm on a
1024x1024 element double precision array using KAPPA subroutines. This
benchmark should test CPU floating point speed and the ability of the
machine to handle large arrays. A considerable amount of virtual memory
is allocated (approximately 32MB) which may result in substantial
paging activity, depending on installed physical memory and other
system processes. Fortran compiler optimisation flag -O is used.

{\bf SLA:}

The SLA benchmark exercises each of the subroutines in the C implementation of the SLALIB Starlink subroutine package. This benchmark performs a substantial amount of floating point arithmetic with no disk I/O. C compiler flag -O is
used.

{\bf KAPPA:}

This benchmark runs a total of 22 KAPPA tasks sequentially (detail enhancement,
image arithmetic, configuration change, compression and expansion, filtering 
and image statistics). Most of the image processing tasks use a 319x503 pixel
real image. A substantial amount of disk I/O is done.

{\bf PISA:}

This benchmark runs a modified version of the PISA demo script which locates
and classifies objects on a test image and fits them with profiling functions.
Again, a substantial amount of disk I/O is done.

{\bf SPECDRE:}

This benchmark runs a modified version of the SPECDRE demo script which processes spectral data and performs several fits. Some disk I/O.

{\bf CCDPACK:}

This benchmark runs a modified version of the CCDPACK demo which performs
inter-image alignment, normalisation and mosaicing on test data. Disk I/O
is quite intensive.

{\bf CCDRED:}

This benchmark uses the CCDRED package in IRAF to create and process
test CCD images. A substantial amount of disk I/O is done, and at least
10MB of free disk space is required.

{\bf DAOPHOT:}

The IRAF DAOPHOT package is used to perform crowded field photometry of
a simulated galaxy cluster image. 

{\bf IMAGES:}

The IRAF IMAGES package is used to perform image processing tasks on
a test 512x512 CCD image. 

\section{Changes}

\subsection{V0.8}

The package was released at v0.8 in March 1995

\subsection{V0.9}

The following changes were made at v0.9:

\begin{enumerate}
\item All benchmark scripts now check for successful completion. Any anomalies
are reported.
\item The output from the {\tt scan} command has been improved. A new {\tt -l}
command line flag allows results to be logged to a file.
\item Benchmark scripts have been fixed to work with recent changes to the
{\tt ussc}.
\item The FFT benchmark has been re-written to use the PDA library.
\item The startup script now checks for the presence of benchmarked
packages. Any absences are noted.
\item More system information produced as part of the benchmark headers.
\item Package startup is now achieved by sourcing a startup script 
{\tt starbench.csh}.
\item Makefile updated to Starlink standard.
\end{enumerate}

\subsection{V1.0}

The following changes were made at v1.0:

\begin{enumerate}
\item Three IRAF benchmarks based on the CCDRED, DAOPHOT and IMAGES package
are included. These benchmarks use the locally installed version of IRAF
except for modified login.cl and mkiraf scripts, which are included with
this package.
\item The package startup script (starbench.csh) has been enhanced to check
for the presence of {\tt ussc} and {\tt iraf} packages. It now prints out
package availability.
\item A -p command line switch is added to the {\tt bench} command, to
facilitate running {\tt ussc} and {\tt iraf} packages. This is also used
by the {\tt submit} command.
\item The {\tt scan} command now checks the benchmark version numbers in the
log files before using them. It now calculates an IRAFmark where appropriate.
\item The SLALIB C benchmark uses customized comment-stripped code.
\end{enumerate} 


\end{document}
