\documentclass[11pt]{article}
\pagestyle{myheadings}

% -----------------------------------------------------------------------------
% ? Document identification
\newcommand{\stardoccategory}  {Starlink System Note}
\newcommand{\stardocinitials}  {SSN}
\newcommand{\stardocnumber}    {33.1}
\newcommand{\stardocsource}    {ssn\stardocnumber}
\newcommand{\stardocauthors}   {Adrian Fish}
\newcommand{\stardocdate}      {14 December 1995}
\newcommand{\stardoctitle}     {Installing Starlink FORUM \\[1.5ex]
                                A Conferencing Facility on the World Wide Web}
% ? End of document identification
% -----------------------------------------------------------------------------

\newcommand{\stardocname}{\stardocinitials /\stardocnumber}
\markright{\stardocname}
\setlength{\textwidth}{160mm}
\setlength{\textheight}{230mm}
\setlength{\topmargin}{-2mm}
\setlength{\oddsidemargin}{0mm}
\setlength{\evensidemargin}{0mm}
\setlength{\parindent}{0mm}
\setlength{\parskip}{\medskipamount}
\setlength{\unitlength}{1mm}

% -----------------------------------------------------------------------------
%  Hypertext definitions.
%  ======================
%  These are used by the LaTeX2HTML translator in conjunction with star2html.

%  Comment.sty: version 2.0, 19 June 1992
%  Selectively in/exclude pieces of text.
%
%  Author
%    Victor Eijkhout                                      <eijkhout@cs.utk.edu>
%    Department of Computer Science
%    University Tennessee at Knoxville
%    104 Ayres Hall
%    Knoxville, TN 37996
%    USA

%  Do not remove the %begin{latexonly} and %end{latexonly} lines (used by 
%  star2html to signify raw TeX that latex2html cannot process).
%begin{latexonly}
\makeatletter
\def\makeinnocent#1{\catcode`#1=12 }
\def\csarg#1#2{\expandafter#1\csname#2\endcsname}

\def\ThrowAwayComment#1{\begingroup
    \def\CurrentComment{#1}%
    \let\do\makeinnocent \dospecials
    \makeinnocent\^^L% and whatever other special cases
    \endlinechar`\^^M \catcode`\^^M=12 \xComment}
{\catcode`\^^M=12 \endlinechar=-1 %
 \gdef\xComment#1^^M{\def\test{#1}
      \csarg\ifx{PlainEnd\CurrentComment Test}\test
          \let\html@next\endgroup
      \else \csarg\ifx{LaLaEnd\CurrentComment Test}\test
            \edef\html@next{\endgroup\noexpand\end{\CurrentComment}}
      \else \let\html@next\xComment
      \fi \fi \html@next}
}
\makeatother

\def\includecomment
 #1{\expandafter\def\csname#1\endcsname{}%
    \expandafter\def\csname end#1\endcsname{}}
\def\excludecomment
 #1{\expandafter\def\csname#1\endcsname{\ThrowAwayComment{#1}}%
    {\escapechar=-1\relax
     \csarg\xdef{PlainEnd#1Test}{\string\\end#1}%
     \csarg\xdef{LaLaEnd#1Test}{\string\\end\string\{#1\string\}}%
    }}

%  Define environments that ignore their contents.
\excludecomment{comment}
\excludecomment{rawhtml}
\excludecomment{htmlonly}

%  Hypertext commands etc. This is a condensed version of the html.sty
%  file supplied with LaTeX2HTML by: Nikos Drakos <nikos@cbl.leeds.ac.uk> &
%  Jelle van Zeijl <jvzeijl@isou17.estec.esa.nl>. The LaTeX2HTML documentation
%  should be consulted about all commands (and the environments defined above)
%  except \xref and \xlabel which are Starlink specific.

\newcommand{\htmladdnormallinkfoot}[2]{#1\footnote{#2}}
\newcommand{\htmladdnormallink}[2]{#1}
\newcommand{\htmladdimg}[1]{}
\newenvironment{latexonly}{}{}
\newcommand{\hyperref}[4]{#2\ref{#4}#3}
\newcommand{\htmlref}[2]{#1}
\newcommand{\htmlimage}[1]{}
\newcommand{\htmladdtonavigation}[1]{}

% Define commands for HTML-only or LaTeX-only text.
\newcommand{\html}[1]{}
\newcommand{\latex}[1]{#1}

% Use latex2html 98.2.
\newcommand{\latexhtml}[2]{#1}

%  Starlink cross-references and labels.
\newcommand{\xref}[3]{#1}
\newcommand{\xlabel}[1]{}

%  LaTeX2HTML symbol.
\newcommand{\latextohtml}{{\bf LaTeX}{2}{\tt{HTML}}}

%  Define command to re-centre underscore for Latex and leave as normal
%  for HTML (severe problems with \_ in tabbing environments and \_\_
%  generally otherwise).
\newcommand{\setunderscore}{\renewcommand{\_}{{\tt\symbol{95}}}}
\latex{\setunderscore}

% -----------------------------------------------------------------------------
%  Debugging.
%  =========
%  Remove % from the following to debug links in the HTML version using Latex.

% \newcommand{\hotlink}[2]{\fbox{\begin{tabular}[t]{@{}c@{}}#1\\\hline{\footnotesize #2}\end{tabular}}}
% \renewcommand{\htmladdnormallinkfoot}[2]{\hotlink{#1}{#2}}
% \renewcommand{\htmladdnormallink}[2]{\hotlink{#1}{#2}}
% \renewcommand{\hyperref}[4]{\hotlink{#1}{\S\ref{#4}}}
% \renewcommand{\htmlref}[2]{\hotlink{#1}{\S\ref{#2}}}
% \renewcommand{\xref}[3]{\hotlink{#1}{#2 -- #3}}
%end{latexonly}
% -----------------------------------------------------------------------------
% ? Document-specific \newcommand or \newenvironment commands.
% ? End of document-specific commands
% -----------------------------------------------------------------------------
%  Title Page.
%  ===========
\renewcommand{\thepage}{\roman{page}}
\begin{document}
\thispagestyle{empty}

%  Latex document header.
%  ======================
\begin{latexonly}
   CCLRC / {\sc Rutherford Appleton Laboratory} \hfill {\bf \stardocname}\\
   {\large Particle Physics \& Astronomy Research Council}\\
   {\large Starlink Project\\}
   {\large \stardoccategory\ \stardocnumber}
   \begin{flushright}
   \stardocauthors\\
   \stardocdate
   \end{flushright}
   \vspace{-4mm}
   \rule{\textwidth}{0.5mm}
   \vspace{5mm}
   \begin{center}
   {\Large\bf \stardoctitle}
   \end{center}
   \vspace{5mm}

% ? Heading for abstract if used.
%  \vspace{10mm}
%  \begin{center}
%     {\Large\bf Abstract}
%  \end{center}
% ? End of heading for abstract.
\end{latexonly}

%  HTML documentation header.
%  ==========================
\begin{htmlonly}
   \xlabel{}
   \begin{rawhtml} <H1> \end{rawhtml}
      \stardoctitle
   \begin{rawhtml} </H1> \end{rawhtml}

% ? Add picture here if required.
% ? End of picture

   \begin{rawhtml} <P> <I> \end{rawhtml}
   \stardoccategory\ \stardocnumber \\
   \stardocauthors \\
   \stardocdate
   \begin{rawhtml} </I> </P> <H3> \end{rawhtml}
      \htmladdnormallink{CCLRC}{http://www.cclrc.ac.uk} /
      \htmladdnormallink{Rutherford Appleton Laboratory}
                        {http://www.cclrc.ac.uk/ral} \\
      \htmladdnormallink{Particle Physics \& Astronomy Research Council}
                        {http://www.pparc.ac.uk} \\
   \begin{rawhtml} </H3> <H2> \end{rawhtml}
      \htmladdnormallink{Starlink Project}{http://www.starlink.ac.uk/}
   \begin{rawhtml} </H2> \end{rawhtml}
   \htmladdnormallink{\htmladdimg{source.gif} Retrieve hardcopy}
      {http://www.starlink.ac.uk/cgi-bin/hcserver?\stardocsource}\\

%  HTML document table of contents. 
%  ================================
%  Add table of contents header and a navigation button to return to this 
%  point in the document (this should always go before the abstract \section). 
  \label{stardoccontents}
  \begin{rawhtml} 
    <HR>
    <H2>Contents</H2>
  \end{rawhtml}
  \htmladdtonavigation{\htmlref{\htmladdimg{contents_motif.gif}}
        {stardoccontents}}

% ? New section for abstract if used.
% \section{\xlabel{abstract}Abstract}
% ? End of new section for abstract

\end{htmlonly}

% -----------------------------------------------------------------------------
% ? Document Abstract. (if used)
%  ==================
% ? End of document abstract
% -----------------------------------------------------------------------------
% ? Latex document Table of Contents (if used).
%  ===========================================
% \newpage
\begin{latexonly}
   \setlength{\parskip}{0mm}
   \tableofcontents
   \setlength{\parskip}{\medskipamount}
   \markright{\stardocname}
\end{latexonly}
% ? End of Latex document table of contents
% -----------------------------------------------------------------------------
\newpage
\renewcommand{\thepage}{\arabic{page}}
% \setcounter{page}{1}

\section{Introduction}

Starlink FORUM is a conferencing facility accessed via Web Browsers
including Mosiac and Netscape. Starlink FORUM has a look and feel similar
to the VAXnotes product from DEC (now called DECnotes). Starlink FORUM is
run from an HTTP server as a CGI script and is written in Perl. 

\subsection{History}

A version of FORUM was originally written by Richard West at Leicester
University. Starlink became interested in Richard's program when the need
for a VAXnotes replacement under Unix was recognized, but the original
FORUM did not look at all like VAXnotes and was not able to identify
unseen notes (to do this, records of which user has seen which notes must
be maintained).  However it did mimic to some extent the structure of a
VAXnotes conference, although the names of the levels were different to
VAXnotes and there was an additional level (Riposte) that has been
retained in FORUM. 

At the Starlink SMM in June 1995 it was agreed that Adrian Fish and Dave
Rawlinson should develop FORUM further, with the aim of adopting a
VAXnotes look and feel. Adrian Fish took Richard West's FORUM (with a few
DJR mods) and added the VAXnotes look and feel plus the mechanics for
updating and accessing unseen notes. The current version is 0.9-16. 
 
\subsection{Use of FORUM by Starlink}
 
It is intended that FORUM is available to both Starlink staff and Starlink
users.  Access to some Conferences will be restricted to, for example,
Site Managers, while all users will be able to access others.  To deter
abuse, records of all access to the system will be retained.  This and the
unseen notes mechanism requires that an identity checking daemon is
running on client machines. 
 
There is one substantial difference between FORUM and VAXnotes --- whereas
VAXnotes could maintain an index of all Conferences, even when these
Conferences were located on seperate systems at several Starlink sites,
FORUM maintains an index only for each instance of FORUM.  Consequently,
all those Conferences intended for users at all Starlink sites are best
located on a single central machine, at present one of the Project's
Alphas at RAL.  This does not rule out copies of FORUM at Starlink sites
for purely local purposes, but such local use is not expected to be
widespread. Most Starlink Site Managers therefore need read no further!
The FORUM code is available to Starlink sites and to other
astronomy-related groups or organizations besides Starlink, on request. 
 
\section{Pre-requisites}

\subsection{Authentication Daemon}

Starlink FORUM requires an authentication daemon (e.g. {\tt pidentd}) in
order to use the updating and accessing of unseen notes. Otherwise the
user cannot tell where or when new notes have been added and cannot access
private conferences. 

There is a kit for the {\tt pidentd} daemon available via anonymous FTP
from UCL. The details are: 

\begin{verbatim}
   Site:  ftp.star.ucl.ac.uk   
   File:  pub/system/pidentd.tar.Z
\end{verbatim}

Full details on how to install {\tt pidentd} on several common platforms
are contained in this kit.

Install an authentication daemon on all machines at your site that will be
accessing FORUM.

\subsection{HTTP Daemon}

FORUM has so far only been run from an independent HTTP server daemon on a
machine running no other HTTP servers. This is the recommended setup, purely
because we haven't tested any other setups.

Several customisations to the HTTP daemon configuration scripts are required to
run FORUM. These modifications are covered later in this document.

You should fetch and install an {\tt httpd} kit on the machine you have
selected to run FORUM.

\section{Fetching and Unpacking the FORUM Kit}

The current version of the kit is available via anonymous FTP from UCL. The
details are:

\begin{verbatim}
   Site:  ftp.star.ucl.ac.uk   
   File:  pub/system/forum.tar.Z
\end{verbatim}

Log on as {\tt root} and create a directory to hold the kit. Note that this
directory will also contain your site's FORUM Notes directory structure, so 
potentially this could require alot of disk space. Ideally this directory
should be on a disk local to the machine selected to run the HTTP daemon for
FORUM. Create and move to this directory:

\begin{verbatim}
  # mkdir <forumdir>
  # setenv INSTALL <forumdir>
  # cd $INSTALL
\end{verbatim}

Now copy over the kit using anonymous FTP:

\begin{verbatim}
  % ftp ftp.star.ucl.ac.uk
    Name (ftp.star.ucl.ac.uk:afish): anonymous
    Password: <user@yoursite>
    ftp> cd pub/system
    ftp> binary
    ftp> get forum.tar.Z
          .....
    ftp> quit
\end{verbatim}

Unpack the kit:

\begin{verbatim}
    % zcat forum.tar.Z | tar xvf -
\end{verbatim}

\subsection{Kit contents}

Once unpacked, you will find several directories (some of which
contain support files) and some empty files that will be used by FORUM 
for logging access and error messages, plus the FORUM Perl script and this
document. Here's a list:

\begin{verbatim}
    Conferences   --    Directory root for FORUM Notes hierarchy
    misc          --    Directory containing on-line help and information files
    icons         --    Directory containing images used by FORUM
    upload        --    Directory for storing uploaded FORUM notes files

    errlogfile    --    File that FORUM error messages are appended to
    logfile       --    File that FORUM access messages are appended to

    forum.pl      --    FORUM Perl script

    ssnxx.tex     --    LaTeX source of FORUM Installation Guide
\end{verbatim}



\section{FORUM and HTTP}

FORUM takes advantage of the Common Gateway Interface (CGI) capabilities of the 
HTTP server. For further details, see the comments in the {\tt forum.pl}
script.

\section{Customising the HTTP config files}

\subsection{{\tt httpd.conf}}

Authentication checking needs to be turned on for the HTTP server. Add the
following line to {\tt httpd.conf}:

\begin{verbatim}
    IdentityCheck on
\end{verbatim}

\subsection{{\tt access.conf}}

CGI support needs to be enabled for the HTTP server. Add the following lines to
{\tt access.conf}; you should substitute the directory pointed to by {\tt
\$INSTALL} above for the example directory shown here:

\begin{verbatim}
    <Directory /softdev/afish/forum>
    Options ExecCGI FollowSymLinks
    </Directory>
\end{verbatim}


\subsection{{\tt srm.conf}}

The following aliases are required by FORUM. The directory paths shown below
are examples. You should use the directory path specified in {\tt \$INSTALL}
above. Add the appropriate lines to {\tt srm.conf} and also the {\tt AddType}
line so that the CGI system recognises {\tt .pl} scripts.

\begin{verbatim}
    Alias /forum-icons/ /softdev/afish/forum/icons/
    Alias /forum-upload/ /softdev/afish/forum/upload/
    Alias /Forum /softdev/afish/forum/forum.pl

    AddType application/x-httpd-cgi .pl
\end{verbatim}


\section{Things to modify inside {\tt forum.pl}}

Before firing up FORUM for the first time, you must customise certain variables
within the FORUM Perl script to reflect your site. Here is a list of the 
variables that require modification, and Appendix \ref{perlmods} shows an
example.

\begin{verbatim}
    $ForumRootDir      --  Same as $INSTALL (remember to add trailing /)
    $ForumServerRoot   --  e.g. http://<your_FORUM_server_hostname>/
    $AdminName         --  FORUM administrators real name
    $AdminMail         --  E-mail address of FORUM administrator (don't 
                           forget to \-escape the @)
    $LocalDomain       --  Local DNS domain; used internally to identify an
                           individual user
    $LocalSubnet       --  Your local subnet. There is no support for multiple
                           subnets
    $HomePageURL       --  Link back to your site's home page ...
    $HomePageURLstr    --  ... and what you want to call it
\end{verbatim}

There are many other variables you can change, including some of the aliases
defined already. If you intend to modify variables other than those listed
above, then please read this document and the source comments carefully.

\subsection{Adding FORUM managers}

FORUM provides certain management facilities for authorised users. Only a 
manager is able to create new conferences, for instance.

To authorise a user to be a manager, you will need to edit the file {\tt
Conferences/.userlist.manager}. The line(s) should be of the form:

\begin{verbatim}
     <user>:<hostname>
\end{verbatim}

For example, at UCL the {\tt .userlist.manager} file contains the line:

\begin{verbatim}
     afish:*.star.ucl.ac.uk
\end{verbatim}

The {\tt *} wildcard means that the user {\tt afish} connecting to FORUM from
any machine in the DNS domain {\tt star.ucl.ac.uk} will be able to access
FORUM management features.

You can configure more than one user as a manager. For example:

\begin{verbatim}
     afish,jrd:*.star.ucl.ac.uk
\end{verbatim}

And you can be more specific still:

\begin{verbatim}
     afish:zuaxp1.star.ucl.ac.uk
     jrd:zuaxp8.star.ucl.ac.uk
\end{verbatim}

{\large\bf Note:} The format described above for authorising user access is
also used for restricting access to private conferences. When the manager
creates a new conference, there are several choices available for restricting
access. A private conference is the most restricted and requires that a FORUM
manager enter username/hostname pairs (that can include wildcards) for
authorised users. The Add Conference page will show full details.

\section{Start your HTTP server}

You can now start your HTTP server and attempt to access FORUM. Use your
favourite Web Browser and go to your FORUM URL:

\begin{verbatim}
     http://<your_FORUM_server_hostname>/Forum
\end{verbatim}

If you are the manager you can now create new conferences.

\section{Using FORUM}

Experimentation is the key. User familiar with VAXnotes should have no problem
finding buttons to do most VAXnotes functions. Updating of unseen notes is done
on the fly.

{\large\bf N.B.} All users who access FORUM for the first time (and this
includes FORUM managers) should hit the {\bf [User Profile]} button and enter
appropriate details.

\section{To Do}

There are several things that I would like to implement in FORUM, and I may
still get the chance:

\begin{itemize}
\item Improve individual note security. Currently anyone who has access to a
      particular note can delete it (but not Topics or Conferences; only a 
      manager can do that). I need to check that the writer of the note is the
      one that is trying to delete it.
\item The {\tt identd} daemon interaction adds a significant delay to the
      system. An alternative method for passing user/hostname information
      in the first call to the FORUM HTTP server is required.
\item Searching notes based on username and since/before times is a good idea
      (and shouldn't be difficult)
\item I'd really like to re-structure the {\tt forum.pl} file in an OO sort of
      way; this would make it easier to maintain
\end{itemize}


\newpage
\appendix

\section{Example editable section of {\tt forum.pl}}
\label{perlmods}

Here is an example of part of the UCL {\tt forum.pl} script. Use the examples
to decide how to modify the various fields for your site.

\begin{verbatim}
#----------------------------------------------------------------------------
# ++++ EDIT below this line to customise for your site.
#
# Directory where everything's kept
#
$ForumRootDir = "/softdev/afish/forum/";
#
# URL of server
#
$ForumServerRoot = $CONF_ForumServerRoot || "http://zuaxp7.star.ucl.ac.uk/";
#
# Administrator details
#
$AdminName = $CONF_AdminName || "Adrian Fish";
$AdminMail = $CONF_AdminMail || "afish\@star.ucl.ac.uk";
#
# Local Domain Info
#
$LocalDomain = $CONF_LocalDomain || "star.ucl.ac.uk";
$LocalSubnet = $CONF_LocalSubnet || "128.40.1";
#
# Local Home page Info
#
$HomePageURL = $CONF_HomePageURL || "http://www.star.ucl.ac.uk/";
$HomePageURLstr = $CONF_HomePageURLstr || "UCL Starlink Home Page";
#
# Usage information location
#
$UsageURL = "";
#
# Maximum upload file size (in MB)
#
$MaxUploadSiz = $CONF_MaxUploadSiz || 10;
#
# URL of Forum (set up as Alias in httpd's srm.conf)
#
$ForumURLRoot = $ForumServerRoot."Forum";
#
# Name of directory containing conferences
#
$ConferenceRootDir = $ForumRootDir."Conferences/";
#
# Location of Forum icons
#
$icon_rooturl = "/forum-icons/";
#
# URL of Forum upload directory
#
$upload_rooturl = $ForumServerRoot."forum-upload/";
#
# Location of Help page
#
$ForumHelpFile = $ForumRootDir."misc/help.html";
#
# Location of README file
#
$ForumREADMEFile = $ForumRootDir."misc/readme.html";
#
# Name of local Faults conference (case-sensitive)
#
$FaultsConferenceName = "FAULTS";
#
# Default Header/Title Image
#
$HeadImage = "forum.gif";
#
#
# ++++ NOTHING TO EDIT BELOW HERE
#-----------------------------------------------------------------------------
\end{verbatim}



\end{document}
