\documentclass[11pt,twoside,noabs,nolof]{starlink}
%------------------------------------------------------------------------------
\stardoccategory    {Starlink System Note}
\stardocinitials    {SSN}
\stardocnumber      {3.1}
\stardocauthors     {William Lupton}
\stardocdate        {22nd December 1989}
\stardoctitle       {The ECH Echelle Spectrograph Model Package}
\stardocname  {\stardocinitials /\stardocnumber}
%------------------------------------------------------------------------------

\newcommand{\mansection}[2]{\subsection{#1 --- #2}}
\newenvironment{mansectionroutines}{\begin{description}}{\end{description}}

\newcommand{\mansectionitem}[1]{\item[#1:]}

\newcommand{\manroutine}[3]{\ssttitle{#1}{#2}\addcontentsline{toc}{subsection}{\protect\numberline{}#1---#2}}

\newcommand{\manroutineitem}[2]{\subsubsection*{#1}#2\par{}}
\newcommand{\manroutinebreakitem}[2]{\subsubsection*{#1}#2\par{}}

% two column tables
\newcommand{\mantwocolumnentry}[1]{\\ #1 & }
\newenvironment{mantwocolumntable}{%
\begin{center}%
\begin{tabular}{||l|p{80mm}||}%
\hline}%
{\end{tabular}%
\end{center}}

% parameter tables
\newenvironment{manparametertable}{\begin{tabular}{lllp{80mm}}}%
{\end{tabular}}
\newcommand{\manparameterentry}[3]{\\#1 & #2 & #3 & }


% teletype conversion
\newcommand{\mantt}[1]{\texttt{#1}}


\newenvironment{mandescription}{\begin{description}}{\end{description}}

\newcommand{\mandescriptionitem}[1]{\item [#1]}


\newcommand{\coude}{coud\'{e}}

%-------------------------------------------------------------------------

\begin{document}

\scfrontmatter

\section{Introduction}

The ECH package consists of a set of routines which model the behaviour of the
UCL \coude\ echelle spectrograph (UCLES) and of echelle spectrographs of
similar design. Such spectrographs can differ in
\begin{itemize}
\item collimator focal length (which is in fact irrelevant to the model package)
\item cross dispersing prism apex angle
\item echelle ruling frequency
\item echelle blaze angle
\item camera focal length
\end{itemize}

It is not capable of dealing with spectrographs which differ in any more major
ways from UCLES, although it could be extended to do so should the need arise.

\section{Routine Summary}

The following routines are available. Refer to the next section for an overview
of how they are used.

%\input{ech_block.tex_summary}-------------------------------------------------
\begin{mansectionroutines}
\mansectionitem {{\mantt{ECH\_BLOCK}}}
  {\mantt{BLOCK}} {\mantt{DATA}} to initialise spectrograph configuration
                 {\mantt{COMMON}} block
\mansectionitem {{\mantt{ECH\_COLLIMATOR}}}
  Determine which collimator to use at a given wavelength
\mansectionitem {{\mantt{ECH\_DISP}}}
  Determine the dispersion in mm for a given wavelength
                in a given order at a specified configuration
\mansectionitem {{\mantt{ECH\_FORMAT\_CENTRE}}}
  Determine the appropriate spectrograph settings
                         to place a specified wavelength in a specified order
                         at the centre of the field
\mansectionitem {{\mantt{ECH\_FRSPR}}}
  Determine the free spectral range in Angstroms
                 of a specified order at a specified configuration
\mansectionitem {{\mantt{ECH\_INIT}}}
  Initialise {\mantt{COMMON}} blocks for a specified instrument combination
\mansectionitem {{\mantt{ECH\_LOAD}}}
  Load {\mantt{COMMON}} blocks describing spectrograph parameters
\mansectionitem {{\mantt{ECH\_ORDERNUM}}}
  Determine the order number of a specified wavelength
                    at a specified configuration
\mansectionitem {{\mantt{ECH\_PRISMPOS}}}
  Determine the optimal prism rotation and position
                    at a specified wavelength
\mansectionitem {{\mantt{ECH\_REFINDEX}}}
  Determine the refractive index of fused silica
                    at a specified wavelength
\mansectionitem {{\mantt{ECH\_SLITANGLE}}}
  Determine the optimum slit angle
                     at a specified configuration
\mansectionitem {{\mantt{ECH\_WAVECEN}}}
  Determine the central wavelength of the order
                   nearest to the centre of the field at a specified
                   configuration
\mansectionitem {{\mantt{ECH\_WCENTRAL}}}
  Determine the central wavelength in Angstroms
                    in a specified order at a given configuration
\mansectionitem {{\mantt{ECH\_XDISPFS}}}
  Determine the cross-dispersion in mm for a given wavelength
                   at a specified configuration
\end{mansectionroutines}
%------------------------------------------------------------------------------

\section{Overview}

The routine that is called is \texttt{ECH\_LOAD}. This reads and interprets a
spectrograph parameter file which defines the list of available spectrographs
and their echelles and cameras. The use of \texttt{ECH\_LOAD} is in fact optional
--- if it is not called then only UCLES with its 31 and 79 echelle and its long
camera is available.

The next routine that is called is \texttt{ECH\_INIT}. This selects the actual
instrument / echelle / camera combination that is to be modelled. \texttt{ECH\_INIT} allows use of the syntax ``instrument/echelle/camera'' in response
to the prompt for the echelle name. The user response is split up into tokens
and then the first token is checked against the list of defined instruments. If
it matches one of them, attention switches to the next token. If it doesn't,
the first instrument defined in the spectrograph parameter file is assumed and
the same token is regarded as a candidate echelle name. The list of defined
echelles for the current instrument is searched in exactly the same way, as is
the list of defined cameras. Attention only moves to the next token if a match
is found to the current token. If, after all this, any unmatched tokens remain,
an error is reported and the first instrument /  echelle / camera combination
defined in the spectrograph parameter file is assumed.

Once \texttt{ECH\_INIT} has been called, any of the other routines can be called.
Two of the routines are at a higher level than the rest of them. These are \texttt{ECH\_FORMAT\_CENTRE}, which determines the spectrograph settings for a given
central wavelength and order number, and \texttt{ECH\_WAVECEN}, which determines
the central wavelength and order number for a given set of spectrograph
settings.

One of the best ways to work out how to use the routines is to study and
understand examples of their use. The source code for the ECHWIND program is a
good place to start and Francisco Diego's PhD thesis (he is one of the
principal designers of UCLES) provides much valuable background information and
develops most of the algorithms. A future version of this note will perhaps
contain more examples.

\appendix
\section{Routine Descriptions}

This section contains details of each routine. In particular, refer to the
descriptions of the \texttt{ECH\_LOAD} and \texttt{ECH\_INIT} routines for details of
how to create and use spectrograph parameter files, and refer to the
description of the \texttt{ECH\_FORMAT\_CENTRE} routine for details of how the
spectrograph configuration for a specified wavelength is determined.

%\input{ech_block.tex_descr}---------------------------------------------------

\manroutine {{\mantt{ECH\_BLOCK}}}{  {\mantt{BLOCK}} {\mantt{DATA}} to %
initialise spectrograph configuration}


 \manroutineitem {Description }{}
      This refers to the {\mantt{ECH\_PARAMS}} {\mantt{COMMON}} block, which %
 contains the parameters
      of all possible spectrographs, not to the {\mantt{ECH\_COMMON}} {\mantt{%
 COMMON}} block, which
      contains the details only of the current configuration.

      This {\mantt{BLOCK}} {\mantt{DATA}} initialises the {\mantt{COMMON}} %
 block in such a way that a call
      to {\mantt{ECH\_INIT}} without a call to {\mantt{ECH\_LOAD}} to load the %
 parameters of a
      specific spectrograph configuration will use the values which are correct
      for {\mantt{UCLES}}. These values also serve to provide defaults for %
 quantities
      which are not overridden as a result of {\mantt{ECH\_LOAD}}.

\manroutineitem {Bugs }{}
     None known.

\manroutineitem {Authors }{}
     W.F. Lupton  {\mantt{AAO}}  ({\mantt{AAOEPP}}::{\mantt{WFL}})



\manroutine {{\mantt{ECH\_COLLIMATOR}}}{  Determine which collimator to use at %
a given wavelength}

\manroutineitem {Description }{}
     If a central wavelength is given, use the {\mantt{WIDE}} collimator for %
wavelengths
     longer than 4000 Angstroms and the {\mantt{UV}} collimator otherwise. %
Otherwise
     make the decision based on whether gamma is greater or less than 0.1369
     degrees. For {\mantt{UCLES}}, this value of gamma corresponds a central %
wavelength
     of very close to 4000 Angstroms.

\manroutineitem {Invocation }{}
     {\mantt{CALL}} {\mantt{ECH\_COLLIMATOR}} ({\mantt{WAVE}}, {\mantt{GAMMA}},%
 {\mantt{COLOUR}}, {\mantt{STATUS}})

\manroutineitem {Arguments }{}
\begin{manparametertable}
\manparameterentry {{\mantt{READ}}}{{\mantt{WAVE}}    }{{\mantt{REAL}} } %
Central wavelength of detector. If
                                      specified as {\mantt{<=}} 0, is not used
                                      (Angstroms)
\manparameterentry {{\mantt{READ}}}{{\mantt{GAMMA}}   }{{\mantt{REAL}} } %
Echelle gamma angle corresponding to the
                                      current configuration. Is used only if
                                      {\mantt{WAVE}} is {\mantt{<=}} 0
                                      (radians)
\manparameterentry {{\mantt{WRITE}}}{{\mantt{COLOUR}}  }{{\mantt{CHARACTER}} } %
The optimal collimator, either {\mantt{WIDE}} or {\mantt{UV}}
\manparameterentry {{\mantt{READ}}, {\mantt{WRITE}}}{{\mantt{STATUS}}  }{{%
\mantt{INTEGER}} } Global status value
\end{manparametertable}
\manroutineitem {Bugs }{}
     The algorithm is too simplistic. It should take account of the detector
     size and guarantee that the best choice of collimator is made for the
     whole detector rather than just for its centre (although this is rather
     hard without knowledge of the configuration history!).

\manroutineitem {Authors }{}
     C.J. Hirst  {\mantt{UCL}}  ({\mantt{ZUVAD}}::{\mantt{CJH}})


\manroutine {{\mantt{ECH\_DISP}}}{  Determine the dispersion in mm for a given %
wavelength}

\manroutineitem {Description }{}
     Use the grating equation to calculate the echelle beta angle at the given
     wavelength in the given order. Then calculate it at the point where the
     dispersion is by definition zero. Knowing the focal length of the camera,
     calculate the dispersion in mm.

     {\mantt{ECH\_INIT}} must be called before calling this routine.

\manroutineitem {Invocation }{}
     {\mantt{CALL}} {\mantt{ECH\_DISP}} ({\mantt{WAVE}}, M, {\mantt{THETA}}, {%
\mantt{GAMMA}}, {\mantt{DEL}}, {\mantt{STATUS}})

\manroutineitem {Arguments }{}
\begin{manparametertable}
\manparameterentry {{\mantt{READ}}}{{\mantt{WAVE}}    }{{\mantt{REAL}} } %
Wavelength of point at which dispersion
                                      is to be calculated (Angstroms)
\manparameterentry {{\mantt{READ}}}{M       }{{\mantt{INTEGER}} } Order number %
in which this wavelength
                                      lies (it doesn't have to be within the
                                      free spectral range)
\manparameterentry {{\mantt{READ}}}{{\mantt{THETA}}   }{{\mantt{REAL}} } %
Echelle theta angle corresponding to the
                                      current configuration (radians)
\manparameterentry {{\mantt{READ}}}{{\mantt{GAMMA}}   }{{\mantt{REAL}} } %
Echelle gamma angle corresponding to the
                                      current configuration (radians)
\manparameterentry {{\mantt{WRITE}}}{{\mantt{DEL}}     }{{\mantt{REAL}} } %
Dispersion (mm)
\manparameterentry {{\mantt{READ}}, {\mantt{WRITE}}}{{\mantt{STATUS}}  }{{%
\mantt{INTEGER}} } Global status value
\end{manparametertable}
\manroutineitem {Bugs }{}
     None known.

\manroutineitem {Authors }{}
     C.J. Hirst  {\mantt{UCL}}  ({\mantt{ZUVAD}}::{\mantt{CJH}})


\manroutine {{\mantt{ECH\_FORMAT\_CENTRE}}}{  Determine the appropriate %
spectrograph settings}

\manroutineitem {Description }{}
     Firstly, calculate the prism system rotation and position. The required
     echelle gamma angle is not quite the same as the prism rotation angle.
     This is because of the 12 degree separation between the incoming ray to
     to the echelle grating and the outgoing ray to the camera - the prism
     rotation angle has to be multiplied by an empirical correction factor
     of 1.011 (I am not sure of the complete justification for this).

     Having determined the appropriate echelle gamma angle, the cross
     dispersion is now correct and it is necessary to determine the
     appropriate echelle theta angle to give the correct wavelength in the
     centre of the field. If the order number was not explicitly specified,
     assume the order within which the given wavelength lies with the free
     spectral range. Theta is determined simply by stepping along the order
     and, once having stepped past the required wavelength, binary chopping
     until convergence.

     Finally, determine the appropriate collimator to use and the necessary
     slit assembly rotation in order to give vertical slit images on the
     detector.

     {\mantt{ECH\_INIT}} must be called before calling this routine.

\manroutineitem {Invocation }{}
     {\mantt{CALL}} {\mantt{ECH\_FORMAT\_CENTRE}} ({\mantt{WC}}, {\mantt{MCEN}}%
, {\mantt{THETA}}, {\mantt{GAMMA}}, {\mantt{COL\_COLOUR}}, {\mantt{PRISM\_POS}},
                                                        {\mantt{SLIT\_ANGLE}}, %
{\mantt{STATUS}})

\manroutineitem {Arguments }{}
\begin{manparametertable}
\manparameterentry {{\mantt{READ}}}{{\mantt{WC}}          }{{\mantt{REAL}} } %
Desired central wavelength
                                          (Angstroms)
\manparameterentry {{\mantt{READ}}}{{\mantt{MCEN}}        }{{\mantt{INTEGER}} }%
 Desired central order. If {\mantt{<=0}} the
                                          order in which the above wavelength
                                          lies within the free spectral range
                                          is used
\manparameterentry {{\mantt{WRITE}}}{{\mantt{THETA}}       }{{\mantt{REAL}} } %
Echelle theta (radians)
\manparameterentry {{\mantt{WRITE}}}{{\mantt{GAMMA}}       }{{\mantt{REAL}} } %
Echelle gamma (radians)
\manparameterentry {{\mantt{WRITE}}}{{\mantt{COL\_COLOUR}}  }{{\mantt{%
CHARACTER}} } Collimator ({\mantt{UV}} or {\mantt{WIDE}})
\manparameterentry {{\mantt{WRITE}}}{{\mantt{PRISM\_POS}}   }{{\mantt{REAL}} } %
Prism position (mm)
\manparameterentry {{\mantt{WRITE}}}{{\mantt{SLIT\_ANGLE}}  }{{\mantt{REAL}} } %
Slit angle (radians)
\manparameterentry {{\mantt{READ}}, {\mantt{WRITE}}}{{\mantt{STATUS}}      }{{%
\mantt{INTEGER}} } Global status value
\end{manparametertable}
\manroutineitem {Bugs }{}
     None known.

\manroutineitem {Authors }{}
     F. Diego  {\mantt{UCL}}  ({\mantt{ZUVAD}}::{\mantt{FD}})


\manroutine {{\mantt{ECH\_FRSPR}}}{  Determine the free spectral range in %
Angstroms}

\manroutineitem {Description }{}
     Determine the central wavelength of the specified order and then
     use the grating equation directly to calculate the free spectral
     range.

     {\mantt{ECH\_INIT}} must be called before calling this routine.

\manroutineitem {Invocation }{}
     {\mantt{CALL}} {\mantt{ECH\_FRSPR}} (M, {\mantt{THETA}}, {\mantt{GAMMA}}, %
{\mantt{FREE}}, {\mantt{STATUS}})

\manroutineitem {Arguments }{}
\begin{manparametertable}
\manparameterentry {{\mantt{READ}}}{M       }{{\mantt{INTEGER}} } Order number %
for which to calculate the
                                      free spectral range
\manparameterentry {{\mantt{READ}}}{{\mantt{THETA}}   }{{\mantt{REAL}} } %
Echelle theta angle corresponding to the
                                      current configuration (radians)
\manparameterentry {{\mantt{READ}}}{{\mantt{GAMMA}}   }{{\mantt{REAL}} } %
Echelle gamma angle corresponding to the
                                      current configuration (radians)
\manparameterentry {{\mantt{WRITE}}}{{\mantt{FREE}}    }{{\mantt{REAL}} } Free %
spectral range (Angstroms)
\manparameterentry {{\mantt{READ}}, {\mantt{WRITE}}}{{\mantt{STATUS}}  }{{%
\mantt{INTEGER}} } Global status value
\end{manparametertable}
\manroutineitem {Bugs }{}
     None known.

\manroutineitem {Authors }{}
     C.J. Hirst  {\mantt{UCL}}  ({\mantt{ZUVAD}}::{\mantt{CJH}})


\manroutine {{\mantt{ECH\_INIT}}}{  Initialise {\mantt{COMMON}} blocks for a %
specified instrument combination}

\manroutineitem {Description }{}
     If a parameter name has been specified, ask the user which configuration
     to use. Otherwise use the one which was supplied.

     The resulting configuration name is a string which consists of zero or
     more words separated by slahes. They are converted to upper case and
     then an attempt is made to interpret the first as the name of a supported
     instrument. If no match is found, the first instrument defined in the
     spectrograph parameter file read by {\mantt{ECH\_LOAD}} is assumed or, if %
{\mantt{ECH\_LOAD}}
     was not called, the default instrument ({\mantt{UCLES}}) is used. Having %
determined
     the instrument, an attempt is made to interpret the remaining words as
     names of supported echelles and cameras respectively. If no match is
     found, the first defined echelle and camera of the instrument are assumed.

     Having determined the instrument, echelle and camera, copy the relevant
     parameters to the {\mantt{ECH\_COMMON}} {\mantt{COMMON}} block for use by %
the rest of the
     spectrograph model routines.

     The result is that simple input such as ``31'' and ``79'' can continue to
     be accepted as it always has, but that it is easy to refer to other
     instruments, echelles and cameras.

     This routine must be called before any of the other spectrograph model
     routines are called. {\mantt{ECH\_LOAD}} may optionally be called before %
this
     routine. If it is not, the only available instrument is {\mantt{UCLES}} %
with the
     31 and 79 echelles and the {\mantt{LONG}} camera.

\manroutineitem {Invocation }{}
     {\mantt{CALL}} {\mantt{ECH\_INIT}} ({\mantt{PARAM}}, {\mantt{CONFIG}}, {%
\mantt{STATUS}})

\manroutineitem {Arguments }{}
\begin{manparametertable}
\manparameterentry {{\mantt{READ}}}{{\mantt{PARAM}}    }{{\mantt{CHARACTER}} } %
Name of program parameter
                                         corresponding to the configuration
                                         to use. If blank, no parameter is read
\manparameterentry {{\mantt{READ}}, {\mantt{WRITE}}}{{\mantt{CONFIG}}  }{{%
\mantt{CHARACTER}} } Configuration to use. See description
                                         above. Is read if {\mantt{PARAM}} is %
blank. Is
                                         set to echelle name if {\mantt{PARAM}}
                                         is non-blank
\manparameterentry {{\mantt{READ}}, {\mantt{WRITE}}}{{\mantt{STATUS}}   }{{%
\mantt{INTEGER}} } Global status value
\end{manparametertable}
\manroutineitem {Bugs }{}
     None known.

\manroutineitem {Authors }{}
     C.J. Hirst  {\mantt{UCL}}  ({\mantt{ZUVAD}}::{\mantt{CJH}})


\manroutine {{\mantt{ECH\_LOAD}}}{  Load {\mantt{COMMON}} blocks describing %
spectrograph parameters}

\manroutineitem {Description }{}
     If a parameter name has been specified, ask the user which file contains
     the parameters. Otherwise use the one which was supplied.

     This file is a text file. Any line beginning with a ``{\mantt{!}}'' is a %
comment line
     and is ignored. Non-comment lines are split up into space, tab or comma
     separated tokens (quotes can be used to protect delimiters within tokens
     should this be necessary) and these tokens are processed in pairs (the
     number of tokens on a line must be even and will probably usually be 2).
     The first of each pair is a keyword and the second is a value associated
     with that keyword. The keyword determines the expected type of the value.
     All non quoted tokens are converted to upper case.

     As the file is processed, there is always a current instrument, a current
     echelle and a current camera. Initially these are the defaults of
     ``{\mantt{UCLES}}'', ``31'' and ``{\mantt{LONG}}'' respectively. They are %
changed by
     ``{\mantt{INSTRUMENT}} instrument'', ``{\mantt{ECHELLE}} echelle'' and ``{%
\mantt{CAMERA}} camera'' entries in
     the file. Individual spectrograph parameters are each associated with the
     instrument and with either the echelle or the camera and their entries
     always apply to the current instrument, echelle and camera.

     Instrument-related parameters are:

\begin{mandescription}
\mandescriptionitem {FCOL}   collimator focal length ({\mantt{REAL}} mm)
\mandescriptionitem {NPR}    number of prisms in cross-disperser ({\mantt{%
INTEGER}})
\mandescriptionitem {ANGLE}  prism angle ({\mantt{REAL}} degrees)

\end{mandescription}
     Echelle-related parameters are:

\begin{mandescription}
\mandescriptionitem {D}      number of lines per mm ({\mantt{REAL}})
\mandescriptionitem {M0}     central order number ({\mantt{INTEGER}})
\mandescriptionitem {WAVE0}  central wavelength ({\mantt{REAL}} Angstroms)
\mandescriptionitem {THETAB} blaze angle ({\mantt{REAL}} degrees)
\mandescriptionitem {THETA0} central theta ({\mantt{REAL}} degrees)
\mandescriptionitem {GAMMA0} central gamma ({\mantt{REAL}} degrees)

\end{mandescription}
     Camera-related parameters are:

\begin{mandescription}
\mandescriptionitem {FCAM}   camera focal length ({\mantt{REAL}} mm)

\end{mandescription}
     Note that the order of instruments, echelles and cameras within the
     file may be significant. For example, the default instrument, echelle
     or camera may be the first one that was defined in the file (although
     this routine does not need to make any assumptions about this).

     The following example describes {\mantt{UCLES}} and illustrates a %
notional short
     camera of focal length 400mm.
     \begin{terminalv}
*       !+ UCLES.DAT
*       !
*       !  UCLES spectrograph parameters.
*       !
*       INSTRUMENT UCLES
*          FCOL 6000.0
*          NPR 3
*          ANGLE 54.1
*          ECHELLE 31
*             D 31.6046
*             M0 138
*             WAVE0 4119.68
*             THETAB 64.66
*             THETA0 0.0
*             GAMMA0 0.0
*          ECHELLE 79
*             D 79.0115
*             M0 55
*             WAVE0 4097.99
*             THETAB 63.55
*             THETA0 0.0
*             GAMMA0 0.0
*          CAMERA LONG
*             FCAM 700.0
*          CAMERA SHORT
*             FCAM 400.0
*     \end{terminalv}
     This routine should be called before {\mantt{ECH\_INIT}} is called. If it %
is not
     called, {\mantt{ECH\_INIT}} will only be able to support the 31 and 79 %
echelles on
     the {\mantt{UCLES}} {\mantt{LONG}} camera.

\manroutineitem {Invocation }{}
     {\mantt{CALL}} {\mantt{ECH\_LOAD}} ({\mantt{PARAM}}, {\mantt{FILE}}, {%
\mantt{STATUS}})

\manroutineitem {Arguments }{}
\begin{manparametertable}
\manparameterentry {{\mantt{READ}}}{{\mantt{PARAM}}    }{{\mantt{CHARACTER}} } %
Name of program parameter
                                         corresponding to the file to use.
                                         If blank, no parameter is read
\manparameterentry {{\mantt{READ}}, {\mantt{WRITE}}}{{\mantt{FILE}}     }{{%
\mantt{CHARACTER}} } File to use. Default file type is
                                         .{\mantt{DAT}}. Is read if {\mantt{%
PARAM}} is blank. Is
                                         written if {\mantt{PARAM}} is non-blank
\manparameterentry {{\mantt{READ}}, {\mantt{WRITE}}}{{\mantt{STATUS}}   }{{%
\mantt{INTEGER}} } Global status value
\end{manparametertable}
\manroutineitem {Bugs }{}
     None known.

\manroutineitem {Authors }{}
     W.F. Lupton  {\mantt{AAO}}  ({\mantt{AAOEPP}}::{\mantt{WFL}})


\manroutine {{\mantt{ECH\_ORDERNUM}}}{  Determine the order number of a %
specified wavelength}

\manroutineitem {Description }{}
     Use the grating equation to calculate the free spectral range and then
     derive the order number directly from it.

     {\mantt{ECH\_INIT}} must be called before calling this routine.

\manroutineitem {Invocation }{}
     {\mantt{CALL}} {\mantt{ECH\_ORDERNUM}} (W, {\mantt{THETA}}, {\mantt{GAMMA}%
}, M, {\mantt{STATUS}})

\manroutineitem {Arguments }{}
\begin{manparametertable}
\manparameterentry {{\mantt{READ}}}{W       }{{\mantt{REAL}} } Wavelength %
whose order number is to be
                                      determined (ie the order number within
                                      which it lies within the free spectral
                                      range) (Angstroms)
\manparameterentry {{\mantt{READ}}}{{\mantt{THETA}}   }{{\mantt{REAL}} } %
Echelle theta angle corresponding to the
                                      current configuration (radians)
\manparameterentry {{\mantt{READ}}}{{\mantt{GAMMA}}   }{{\mantt{REAL}} } %
Echelle gamma angle corresponding to the
                                      current configuration (radians)
\manparameterentry {{\mantt{WRITE}}}{M       }{{\mantt{INTEGER}} } Order %
number corresponding to wavelength
\manparameterentry {{\mantt{READ}}, {\mantt{WRITE}}}{{\mantt{STATUS}}  }{{%
\mantt{INTEGER}} } Global status value
\end{manparametertable}
\manroutineitem {Bugs }{}
     None known.

\manroutineitem {Authors }{}
     C.J. Hirst  {\mantt{UCL}}  ({\mantt{ZUVAD}}::{\mantt{CJH}})


\manroutine {{\mantt{ECH\_PRISMPOS}}}{  Determine the optimal prism rotation %
and position}

\manroutineitem {Description }{}
     Determine the prism rotation and position by direct calculation. The
     rotation is such that a ray of the specified wavelength will fall on
     the centre of the detector. The position is such that the echelle is
     fully illuminated by rays of the specified wavelength.

     {\mantt{ECH\_INIT}} must be called before calling this routine.

\manroutineitem {Invocation }{}
     {\mantt{CALL}} {\mantt{ECH\_PRISMPOS}} ({\mantt{WC}}, {\mantt{GAMMA}}, {%
\mantt{POS}}, {\mantt{STATUS}})

\manroutineitem {Arguments }{}
\begin{manparametertable}
\manparameterentry {{\mantt{READ}}}{{\mantt{WC}}      }{{\mantt{REAL}} } %
Desired central wavelength (Angstroms)
\manparameterentry {{\mantt{WRITE}}}{{\mantt{GAMMA}}   }{{\mantt{REAL}} } %
Optimal prism rotation (radians)
\manparameterentry {{\mantt{WRITE}}}{{\mantt{POS}}     }{{\mantt{REAL}} } %
Optimal prism position (mm)
\manparameterentry {{\mantt{READ}}, {\mantt{WRITE}}}{{\mantt{STATUS}}  }{{%
\mantt{INTEGER}} } Global status value
\end{manparametertable}
\manroutineitem {Bugs }{}
     None known.

\manroutineitem {Authors }{}
     C.J. Hirst  {\mantt{UCL}}  ({\mantt{ZUVAD}}::{\mantt{CJH}})


\manroutine {{\mantt{ECH\_REFINDEX}}}{  Determine the refractive index of %
fused silica}

\manroutineitem {Description }{}
     Use the empirical expression derived by I.H. Malitson, J.O.S.A 55, 1205,
     Oct 1965. This expression is:
     \begin{terminalv}
*                  0.6961663 . w**2     0.4079426 . w**2      0.8974794 . w**2
*     ref**2 = 1 + ----------------- + ------------------- + ------------------
*                 w**2 - 0.0684043**2  w**2 - 0.0062414**2   w**2 - 9.896161**2
*     \end{terminalv}
     where wavelength is measured in microns. This routine uses wavelength in
     Angstroms and converts to microns internally.

\manroutineitem {Invocation }{}
     {\mantt{CALL}} {\mantt{ECH\_REFINDEX}} ({\mantt{WAVE}}, {\mantt{REF}}, {%
\mantt{STATUS}})

\manroutineitem {Arguments }{}
\begin{manparametertable}
\manparameterentry {{\mantt{READ}}}{{\mantt{WAVE}}    }{{\mantt{REAL}} } %
Wavelength (Angstroms)
\manparameterentry {{\mantt{WRITE}}}{{\mantt{REF}}     }{{\mantt{REAL}} } %
Refractive index
\manparameterentry {{\mantt{READ}}, {\mantt{WRITE}}}{{\mantt{STATUS}}  }{{%
\mantt{INTEGER}} } Global status value
\end{manparametertable}
\manroutineitem {Bugs }{}
     None known.

\manroutineitem {Authors }{}
     W.F. Lupton  {\mantt{AAO}}  ({\mantt{AAOEPP}}::{\mantt{WFL}})


\manroutine {{\mantt{ECH\_SLITANGLE}}}{  Determine the optimum slit angle}

\manroutineitem {Description }{}
     Determine the slit angle by direct calculation.

     {\mantt{ECH\_INIT}} must be called before calling this routine.

\manroutineitem {Invocation }{}
     {\mantt{CALL}} {\mantt{ECH\_SLITANGLE}} ({\mantt{GAMMA}}, {\mantt{ANG}}, {%
\mantt{STATUS}})

\manroutineitem {Arguments }{}
\begin{manparametertable}
\manparameterentry {{\mantt{READ}}}{{\mantt{GAMMA}}   }{{\mantt{REAL}} } %
Echelle gamma angle corresponding to the
                                      current configuration (radians)
\manparameterentry {{\mantt{WRITE}}}{{\mantt{ANG}}     }{{\mantt{REAL}} } %
Optimal slit angle (radians)
\manparameterentry {{\mantt{READ}}, {\mantt{WRITE}}}{{\mantt{STATUS}}  }{{%
\mantt{INTEGER}} } Global status value
\end{manparametertable}
\manroutineitem {Bugs }{}
     None known.

\manroutineitem {Authors }{}
     C.J. Hirst  {\mantt{UCL}}  ({\mantt{ZUVAD}}::{\mantt{CJH}})


\manroutine {{\mantt{ECH\_WAVECEN}}}{  Determine the central wavelength of the %
order}

\manroutineitem {Description }{}
     Use a binary chop search to determine the order which corresponds to
     the given echelle gamma angle. Having determined the order, perform a
     similar search to determine the wavelength within the order which
     corresponds to the given echelle theta angle.

     {\mantt{ECH\_INIT}} must be called before calling this routine.

\manroutineitem {Invocation }{}
     {\mantt{CALL}} {\mantt{ECH\_WAVECEN}} ({\mantt{THETA}}, {\mantt{GAMMA}}, {%
\mantt{WC}}, {\mantt{MC}}, {\mantt{STATUS}})

\manroutineitem {Arguments }{}
\begin{manparametertable}
\manparameterentry {{\mantt{READ}}}{{\mantt{THETA}}   }{{\mantt{REAL}} } %
Echelle theta angle corresponding to the
                                      current configuration (radians)
\manparameterentry {{\mantt{READ}}}{{\mantt{GAMMA}}   }{{\mantt{REAL}} } %
Echelle gamma angle corresponding to the
                                      current configuration (radians)
\manparameterentry {{\mantt{WRITE}}}{{\mantt{WC}}      }{{\mantt{REAL}} } %
Central wavelength corresponding to the
                                      current configuration (ie that wavelength
                                      which lies within order M and is closest
                                      to that in the centre of the field)
\manparameterentry {{\mantt{WRITE}}}{{\mantt{MC}}      }{{\mantt{INTEGER}} } %
Order number corresponding to the current
                                      configuration (the order closest to that
                                      in the centre of the field)
\manparameterentry {{\mantt{READ}}, {\mantt{WRITE}}}{{\mantt{STATUS}}  }{{%
\mantt{INTEGER}} } Global status value
\end{manparametertable}
\manroutineitem {Bugs }{}
     None known.

\manroutineitem {Authors }{}
     C.J. Hirst  {\mantt{UCL}}  ({\mantt{ZUVAD}}::{\mantt{CJH}})


\manroutine {{\mantt{ECH\_WCENTRAL}}}{  Determine the central wavelength in %
Angstroms}

\manroutineitem {Description }{}
     Use the grating equiation directly to determine the central wavelength.

     {\mantt{ECH\_INIT}} must be called before calling this routine.

\manroutineitem {Invocation }{}
     {\mantt{CALL}} {\mantt{ECH\_WCENTRAL}} (M, {\mantt{THETA}}, {\mantt{GAMMA}%
}, {\mantt{WC}}, {\mantt{STATUS}})

\manroutineitem {Arguments }{}
\begin{manparametertable}
\manparameterentry {{\mantt{READ}}}{M          }{{\mantt{INTEGER}} } Order %
number
\manparameterentry {{\mantt{READ}}}{{\mantt{THETA}}      }{{\mantt{REAL}} } %
Echelle theta angle corresponding to the
                                      current configuration (radians)
\manparameterentry {{\mantt{READ}}}{{\mantt{GAMMA}}      }{{\mantt{REAL}} } %
Echelle gamma angle corresponding to the
                                      current configuration (radians)
\manparameterentry {{\mantt{WRITE}}}{{\mantt{WC}}         }{{\mantt{REAL}} } %
Central wavelength of this order
                                      (Angstroms)
\manparameterentry {{\mantt{READ}}, {\mantt{WRITE}}}{{\mantt{STATUS}}  }{{%
\mantt{INTEGER}} } Global status value
\end{manparametertable}
\manroutineitem {Bugs }{}
     None known.

\manroutineitem {Authors }{}
     C.J. Hirst  {\mantt{UCL}}  ({\mantt{ZUVAD}}::{\mantt{CJH}})



\manroutine {{\mantt{ECH\_XDISPFS}}}{  Determine the cross-dispersion in mm %
for a given wavelength}

\manroutineitem {Description }{}
     Calculate the refractive index at the given wavelength. Then calculate it
     at the point where the cross-dispersion is by definition zero. Knowing
     the focal length of the camera, calculate the cross-dispersion in mm on
     the assumption that the light enters and leaves the prisms at the same
     angles of incidence (a first order approximation is used).

     {\mantt{ECH\_INIT}} must be called before calling this routine.

\manroutineitem {Invocation }{}
     {\mantt{CALL}} {\mantt{ECH\_XDISPFS}} ({\mantt{WAVE}}, {\mantt{GAMMA}}, {%
\mantt{DEL}}, {\mantt{STATUS}})

\manroutineitem {Arguments }{}
\begin{manparametertable}
\manparameterentry {{\mantt{READ}}}{{\mantt{WAVE}}    }{{\mantt{REAL}} } %
Wavelength of point at which
                                      cross-dispersion is to be calculated
                                      (Angstroms)
\manparameterentry {{\mantt{READ}}}{{\mantt{GAMMA}}   }{{\mantt{REAL}} } %
Echelle gamma angle corresponding to the
                                      current configuration (radians)
\manparameterentry {{\mantt{WRITE}}}{{\mantt{DEL}}     }{{\mantt{REAL}} } %
Cross-dispersion (mm)
\manparameterentry {{\mantt{READ}}, {\mantt{WRITE}}}{{\mantt{STATUS}}  }{{%
\mantt{INTEGER}} } Global status value
\end{manparametertable}
\manroutineitem {Bugs }{}
     None known.

\manroutineitem {Authors }{}
     C.J. Hirst  {\mantt{UCL}}  ({\mantt{ZUVAD}}::{\mantt{CJH}})


%------------------------------------------------------------------------------

\end{document}
