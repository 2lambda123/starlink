\documentstyle[11pt]{article}
\pagestyle{myheadings}

%------------------------------------------------------------------------------
\newcommand{\stardoccategory}  {Starlink System Note}
\newcommand{\stardocinitials}  {SSN}
\newcommand{\stardocnumber}    {39.6}
\newcommand{\stardocauthors}   {D.\ L.\ Terrett}
\newcommand{\stardocdate}      {18 November 1992}
\newcommand{\stardoctitle}     {GKS --- Installation and Modification}
%------------------------------------------------------------------------------

\newcommand{\stardocname}{\stardocinitials /\stardocnumber}
\renewcommand{\_}{{\tt\char'137}}     % re-centres the underscore
\markright{\stardocname}
\setlength{\textwidth}{160mm}
\setlength{\textheight}{230mm}
\setlength{\topmargin}{-2mm}
\setlength{\oddsidemargin}{0mm}
\setlength{\evensidemargin}{0mm}
\setlength{\parindent}{0mm}
\setlength{\parskip}{\medskipamount}
\setlength{\unitlength}{1mm}

\begin{document}
\thispagestyle{empty}
SCIENCE \& ENGINEERING RESEARCH COUNCIL \hfill \stardocname\\
RUTHERFORD APPLETON LABORATORY\\
{\large\bf Starlink Project\\}
{\large\bf \stardoccategory\ \stardocnumber}
\begin{flushright}
\stardocauthors\\
\stardocdate
\end{flushright}
\vspace{-4mm}
\rule{\textwidth}{0.5mm}
\vspace{5mm}
\begin{center}
{\Large\bf \stardoctitle}
\end{center}
\vspace{5mm}

\section{Introduction}

This note describes how to install GKS 7.2 and how to rebuild the system with a
different set of device handlers.
GKS 7.2 is distributed by RAL Central Computing Division (CCD) but has been
modified by Starlink to suit Starlink's computing environment.
It does not describe how to write a handler; a guide for this is being prepared
by RAL Computing Services Division.

\section{Installing the system}

GKS is provided as part of the Starlink Core software. The directory locations
used throughout this note are relative to {\tt STARDISK:[STARLINK.LIB]} in the
Core.

When the software has been installed the following directories exist:

\begin{itemize}

\item {\tt [.GKS]} :
All the files required to link and run programs with GKS, plus those files
needed to re-build the system which differ from the CCD version.

\item {\tt [.GKS.CCD]} :
All the files needed to rebuild the system as distributed by CCD including the
workstation description tables (WDT) for all the CCD device handlers.
This directory has not been modified by Starlink.

\item {\tt [GKS.DRIVERS]} :
Source code (in {\tt GKS72.TLB}) and WDT files for all the device handlers
provided by or modified by Starlink. 

\end{itemize} 

Once the following logical names have been defined (by {\tt
[STARLINK]STARTUP.COM} on Starlink systems):

\begin{verbatim}
      "GKS_DIR"            = "STARDISK:[STARLINK.LIB.GKS]"
      "GKS_EMF"            = "GKS_DIR:GKSEMF.DAM"
      "GKS_EPAR"           = "GKS_DIR:GKSE.PAR"
      "GKS_FONTS"          = "GKS_DIR:GKSDBS.DAM"
      "GKS_IMAGE"          = "GKS_DIR:GKS_IMAGE.EXE"
      "GKS_IMAGE_ADAM"     = "GKS_DIR:GKS_IMAGE_ADAM.EXE"
      "GKS_IMAGE_ADAM_PAR" = "GKS_DIR:GKS_IMAGE_ADAM_PAR.EXE"
      "GKS_PAR"            = "GKS_DIR:GKS.PAR"
      "GKS_WDT"            = "GKS_DIR:GKSWDT.DAM"
      "GKS_WS_IMAGE"       = "GKS_DIR:GKS_WS_IMAGE.EXE"
      "GKSSHARE"           = "GKS_DIR:GKS_WS_IMAGE.EXE"
\end{verbatim} 

the system is ready to use provided that the workstation handlers used at RAL
are appropriate. The last ({\tt GKSSHARE}) is required in order to support older
applications. 

On non-Starlink systems the root of directory tree ({\tt [.GKS]}) can be
any directory you chose provided that the logical names are modified
appropriately.
If a different directory is used it should replace {\tt [.GKS]} throughout
this document.

The handlers present in the library can be examined by listing the file
{\tt [.GKS]WORKSTATIONS\-.TEMPLATE}. The version at the time of writing is
shown in appendix~\ref{se:temp}.

\section{Changing the workstations}

Changing the set of workstations included in the system is simply a matter of
editing {\tt [.GKS]\-WORKSTATIONS.DAT} to include or exclude workstations as
required and executing the command procedure {\tt [.GKS]GKS\_REBUILD.COM}. This
procedure assumes that {\tt GKS\_DIR} has been defined, and new versions of
{\tt GKS\_WS\_IMAGE.EXE} and {\tt GKSWDT.DAM} will be created in {\tt GKS\_DIR}
and {\tt GKSLIB.OLB} modified. You may wish to make a backup of these files
before proceeding.

The program that creates the binary WDT file generates a considerable amount
of output including many ``caution" messages; these are normal and can be
ignored.

Starlink sites also need to rebuild the shareable image libraries. This is done
by running the command procedures {\tt STAR\_LINK.COM} and {\tt
STAR\_LINK\_ADAM.COM} in {\tt STARDISK:[STARLINK.LIB]}.

The template file {\tt [.GKS]WORKSTATIONS\-.TEMPLATE} may be used as a template
for your own {\tt [.GKS]\-WORKSTATIONS.DAT}. The supplied version of {\tt
[.GKS]\-WORKSTATIONS.DAT} is that for use on the Starlink Project VAXCluster 
at RAL.


\section{Modifying or adding a workstation}

The source code of the Starlink written or modified device handlers is found in
{\tt [.GKS.DRIVERS]\-GKS72.TLB} in modules called {\tt GKntWD} where {\tt t}
indicates the device manufacture and {\tt n} distinguishes between different
models, ({\em e.g.} the Tektronix 4010 handler is in {\tt GK0TWD}). The source
of the workstation description tables is found in {\tt [.GKS.DRIVERS]} in files
called {\tt Gnnnn.WDT} where {\tt nnnn} is the workstation type.

Drivers are compiled by:

\begin{verbatim}
      $ FORTRAN filename+[.GKS.CCD]GKS_INCLUDES/LIBRARY
\end{verbatim}

New versions of the library and binary WDT are created by executing the
procedure {\tt [GKS.CCD]\-GKS\_CONFIGURE.\-COM}. This procedure prompts for the
locations of all the files needed to rebuild the system and by specifying
search lists the procedure can be directed to pick up new and modified files
from your own directories. For example, suppose your current default directory
contains some modified WDT files and some object modules containing modified
drivers.

The prompts from the procedure and your responses would be:
\begin{quote}
Where is your existing version of GKSLIB.OLB?\\
{\tt [.GKS]}

Where are the GKS development files to be found? 
If they are not all in the same directory you can enter a search list\\
{\tt [.GKS],[.GKS.CCD]}

Where do you want to build the new version?
If you specify the same location as the existing version of GKSLIB.OLB the
existing library will be modified rather than a new version being created\\
{\tt []}

Enter the names of any object files you want inserted in the library, separated
by commas.
The names may contain wild cards\\
{\tt GK\%\%WDT.OBJ}

Enter the locations to be searched for the workstation description table files,
separated by commas.
If you enter more than one location they will be searched in the order
specified when searching for each workstation\\
{\tt [],[.GKS.DRIVERS],[.GKS.CCD]}

Enter the name of the file containing the list of workstations that you
want in your configuration\\
{\tt [.GKS]WORKSTATIONS.DAT}
\end{quote}

The new versions of {\tt GKSLIB.OLB}, {\tt GKS\_WS\_IMAGE.EXE} and {\tt 
GKSWDT.DAM} will be created in your current default directory. The answers to
the prompts can be supplied as parameters to the command procedure; an example
of this can be found in {\tt [.GKS]GKS\_REBUILD.COM}.

\section{Test program}

{\tt [.GKS.CCD]TEST.OBJ} is the object module for a test program that will
execute GKS calls typed in at the keyboard or read from a file. The program is
driven by typing the name of a GKS function ({\em e.g.} {\tt GOPKS}) and will
then prompt for all the required parameters. Typing a ``?" will list all the
available functions. This program was written by, but is not supported by, RAL
CCD division.

\section{Changes since previous releases}

If you have written a driver for a previous release you may have to make the
following changes to the code:

\begin{itemize}

\item Change INCLUDE statement from 
\begin{verbatim}
      'INCLUDE "GINCLUDE:filename"'
\end{verbatim}
to
\begin{verbatim}
      'INCLUDE "(filename)"'
\end{verbatim}

\item Replace {\tt GKS\_} by {\tt G} in all subroutine names.

\end{itemize}

\section{Starlink Modifications}

GKS 7.2 as distributed by Starlink differs from the CCD distribution in
the following ways:

\begin{itemize}

\item An escape function to suppress the clearing of the screen on open
workstation has been implemented.

\item The library has been split into two parts. {\tt GKSLIBG.OLB} contains the
user callable routines plus a few others. {\tt GKSLIB.OLB} contains the rest,
including device handlers. This allows the {\tt GKSLIB.OLB} to be built as a
shareable library without preventing applications programs from replacing the
error handling routine.

\item {\tt GKS\_WS\_IMAGE} does not contain the user callable routines.

\end{itemize}

A system that conforms to the CCD system can be created be inserting the
contents of {\tt GKSLIBG\-.OLB} into {\tt GKSLIB.OLB} and rebuilding the
shareable image specifying {\tt [.GKS.CCD]} only as the location of the
``development" files.

\newpage
\appendix
\section{Template of GKS Workstations}
\label{se:temp}
\begin{quote}\small
\begin{verbatim}
!  WORKSTATIONS.TEMPLATE
!
!  This is a template for WORKSTATIONS.DAT which lists all the workstations
!  that are to be included when the GKS library and workstation description
!  file are built
!
!  Workstations that are not required should be either deleted or commented
!  out with a ! character at the beginning of the line.
!
!  The metafile workstations (10 & 50) MUST be included and the total number
!  of workstations included must not exceed 60. So at least 21 workstations
!  must be commented out; this file currently contains 81 workstations, 21 of
!  which are commented out.
!
!  ***  Never remove the next two workstations ***
!
0010	! GKS metafile input (Annex E)
0050	! GKS metafile output (Annex E)
!
0101	! Sigmex R5664, colour, 4 planes
0102	! Sigmex T5674-T1 ,'A' series , greyscale, 1 plane
0103	! Sigmex T5674-T1 ,'B' series , greyscale, 1 plane
0104	! Sigmex T5674-T4 ,'A' series , greyscale, 4 planes
0105	! Sigmex T5674-T4 ,'B' series , greyscale, 4 planes
0106	! Sigmex T5684-T4, colour, 4 planes
0108	! Sigmex T5472-T1, black and white, 1 plane
0109	! Sigmex T5472-T1, black and white, 1 plane
0110	! Sigmex T5484-T4, colour, 4 planes, joystick input
!
0201	! Tektronix 4010 storage tube
0203	! Tektronix 4014 storage tube
!
0700	! Calcomp 81 plotter, A4 sheet feed
0701	! Calcomp 81 plotter, A3 sheet feed
0702	! Calcomp 81 plotter, A4 roll feed
0703	! Calcomp 81 plotter, A3 roll feed
0704	! Calcomp 81 plotter, roll feed to workstation viewport
! 
! Starlink supported workstations
!
0160	! Sigmex ARGS 7000
0161	! Sigmex ARGS 7000 overlay plane
0162	! Sigmex ARGS 7000 + VT terminal
0163	! Sigmex ARGS 7000 overlay plane + VT terminal
!
0801	! Cifer T5
0825	! Pericom 7800
0827	! Pericom Graphpack
0845	! GraphOn 235
!
1000	! Zeta 8
1001	! Zeta 8 long workstation
!
1200	! Printronix P300
1201	! Printronix P300 square workstation
!
!1500	! BBC micro + Acorn termulator chip
!
1740	! VAXstation 2000
1741	! VAXstation 4 plane monochrome
1742    ! VAXstation 8 plane colour
1743	! VAXstation 8 plane monchrome
!
2600	! Canon LPB-8 Laser printer landscape mode
2601	! Canon LPB-8 Laser printer portrait mode
2610    ! Canon for inclusion in Tex documents - landscape
2611    ! Canon for inclusion in Tex documents - portrait
!
2700    ! Apple laserwriter A4 page (portrait)
2701    ! Apple laserwriter  A4 page (landscape)
2702    ! Encapsulated Apple laserwriter  A4 page (portrait)
2703    ! Encapsulated Apple laserwriter  A4 page (landscape)
2704    ! Postscript A4 page (portrait)
2705    ! Postscript A4 page (landscape)
2706    ! Encapsulated Postscript A4 page (portrait)
2707    ! Encapsulated Postscript A4 page (landscape)
2708	! LJ 250 Inkjet (portrait)
2709	! LJ 250 Inkjet (landscape)
!
3200    ! Digisolve Ikon 1024 x 780
3201    ! Digisolve Ikon 1024 x 780 overlay
3202    ! Digisolve Ikon 1024 x 780 + VT terminal 
3203    ! Digisolve Ikon 1024 x 780 overlay + VT terminal
!
3800	! X Windows
3801	! X Windows
3802	! X Windows
3803	! X Windows
!
3805	! X Windows overlay
3806	! X Windows overlay
3807	! X Windows overlay
3808	! X Windows overlay
!
!  AAO supported workstations
!
!0400	! HP 2648A graphics terminal
!2000	! QMS landscape software linetypes
!2001	! QMS portrait     "         "
!2010	! QMS landscape hardware linetypes
!2011	! QMS portrait     "         "
!
!  Obsolete workstations
!
!0830	! Lear-Siegler ADM 3
!0800	! Cifer 2634G
!1100	! Versatec 1200A fan fold
!1101	! Versatec 1200A square workstation
!
!  Unsupported Workstations 
!
!0410	! HP 7221B sheet A4 plotter
!0411	! HP 7221B sheet A3 plotter
!0430	! HPGL plotter A4 paper
!0431	! HPGL plotter A3 paper
!0432	! HPGL plotter A2 paper
!0433	! HPGL plotter A1 paper
!1400	! Canon 1080 (Epson compatible) dot printer
!3700	! LN03 laser printer - low res
!3701	! LN03 laser printer - high res
!3900	! Complot Houston EDP3 pen plotter
\end{verbatim}
\end{quote}

\end{document}
