\documentclass[11pt,twoside]{article}

%%% XXX Comment out before submission!!!
\def\RCS$#1: #2 ${\expandafter\def\csname RCS#1\endcsname{#2}}
\RCS$Revision$

\usepackage{adassconf}

% NB: wierdo conference style has \title and co typeset material,
% rather than store it, so \begin{document} must go here, and there's no
% \maketitle
\begin{document}

\paperID{P2.1.10}
\contact{Norman Gray}
\email{norman@astro.gla.ac.uk}

\title{Porting the Starlink Software Collection to GNU Autoconf}
% XXX Comment out
\marginpar{[v\RCSRevision]}
\titlemark{Starlink/GNU Autoconf}

\author{Norman Gray\altaffilmark{1}}
\affil{Starlink Project,
        Rutherford Appleton Laboratory,
        Chilton,
        Didcot,
        OX11 0QX,
        UK}
\author{Tim Jenness}
\affil{Joint Astronomy Centre,
       660 North A'ohoku Place
       University Park
       Hilo, HI 96720,
       USA}
\author{Alasdair Allan\altaffilmark{1},
        David S. Berry\altaffilmark{1},
        Malcolm J. Currie,
        Peter W. Draper\altaffilmark{1},
        Mark B. Taylor\altaffilmark{1}}
\affil{Starlink}
\author{Brad Cavanagh}
\affil{Joint Astronomy Centre}

\altaffiltext{1}{Alternate affiliations:
NG,
%        Department of Physics and Astronomy,
        University of Glasgow%,
%        Glasgow,
%        G12 8QQ, 
%        UK%
;
AA, University of Exeter,
DSB,
%        Centre for Astrophysics,
        University of Central Lancashire%,
%        Preston,
%        PR1 2HE,
%        UK%
;
PWD,
%        Department of Physics, 
        University of Durham%, 
        %Rochester Building, 
        %Science Laboratories, 
        %South Road, 
%        Durham, DH1 3LE, 
%        UK%
;
MBT,
%        Department of Physics,
        University of Bristol%,
%        %Tyndall Avenue,
%        Bristol,
%        BS8 1TL,
%        UK%
        }

\paindex{Gray, N.}
\aindex{Jenness, T.}
\aindex{Allan, A.}
\aindex{Berry, D. S.}
\aindex{Currie, M. J.}
\aindex{Draper, P. W.}
\aindex{Taylor, M. B.}
\aindex{Cavanagh, B.}

\authormark{Gray, Jenness, Allan, Berry, Currie, Draper, Taylor \& Cavanagh}



\keywords{lkajsfs}

\begin{abstract}

The Classic Starlink Software Collection (aka `USSC', see below)
currently runs on three different Unix platforms and contains
approximately 130 separate software items, totalling millions of lines
of code, in a mix of Fortran, C, C++, Tcl, Perl and others.  We get requests
to port to an increasing range of platforms, while suffering a
decrease in the funding support for the classic collection.  We have
therefore changed the build system from a hand-maintained collection
of makefiles with hard-wired OS configurations to a scheme involving
feature-discovery via GNU Autoconf.

As a result of this work, we have already ported the USSC to Mac OSX
and Cygwin.  This had some unexpected benefits and costs, and
valuable lessons.
\end{abstract}

\section{The Problems}

\begin{itemize}
\item It's Big!  About 130 items, with around 1700 kSLOC of our
  multi-language legacy code, 300 kSLOC of Java (not included in
  this build system), plus another 600 kSLOC of thirdparty code, some
  tweaked, and all built at the same time.

\item According to David A Wheeler's SLOCCount,
  that's \$110M worth, by the way.  And it works out larger than
  everything in the RH7.1 distribution except the kernel, Mozilla and XFree86
  \htmladdURL{http://www.dwheeler.com/sloc/redhat71-v1/redhat71sloc.html}.
  Busy types, aren't we?

\item It (was) weird!  Our legacy build system was idiosyncratic.  It
  worked, but was unusual both in the way the code was distributed and
  built.  This presented a barrier to potential developers and
  maintainers, and an ongoing and unsustainable cost.

\item It's Crusty!  Some of that code dates back quite a long way, and
  some of it comes from community donations.  While the majority of it
  is well-written, there are a few\dots gems in there.

%\item It (was) Scattered!  For historical reasons, the master copies
%  of all our code weren't kept in the one place.  Simply
%  \emph{finding} all our code was slightly harder than we expected.

\item Did I mention funding problems?

\end{itemize}

\section{History}


Starlink was set up originally in the late seventies, as a way of
supplying UK astronomy with hardware (`astronomers will never need
more than six VAXes\dots'), naturally along with the data-analysis
software to go with them, and the system management to make it all
work smoothly.

We switched the hardware to Unix in the early nineties -- ending up
supporting specifically SunOS and Ultrix.  We then shifted to Solaris
and Digital Unix (our first 64-bit OS), and ended up, last year,
working with Solaris, Alpha and RedHat, at which point we had managed
to drop dependence on NAG.  We also ported the software
collection then, renamed to the USSC, the `Unix Starlink Software
Collection'.  That meant that some software was dropped, and the rest
preened quite extensively.  The build system we are now moving away
from was designed in this period.

In the late nineties, Starlink slimmed down, dropped the hardware and
management provision, and focused on maintaining and developing the
large legacy codebase.

Now we're concentrating on new software, we need to make the classic
software as accessible and maintainable as possible, since we expect
the community will have to take a larger role in its curation.


\section{The Outcome}


\begin{itemize}
\item The project now uses `standard' tools for all its source code
  maintainance and distribution.  This makes the code much more
  maintainable by community developers.

\item As of October 2004, we have 2300kSLOC building unattended on
  Linux (RHEL), Solaris, Cygwin (incomplete), and OSX (almost).  Other
  platforms should be easy (but we can only cope with so much pain at
  one time).

\item All available in one place for the first time, via anonymous CVS
  (see `Contacts' box).

\item Unexpected but valuable preening, refactoring, bugfixing.

\item We extended the autoconf support for Fortran, and this will be
  offered back to the autoconf mainline.

\item GPLed wherever possible.

\item The project has always obsessed about coding standards and
  coding discipline.  This concern paid off, with interest, here.

\end{itemize}  


\section{Patching autoconf and automake}

We have a number of project-specific code and installation
conventions, which we wanted to preserve, and a few build tools which
need special makefile rules.  This is what \texttt{automake} is for.

We tried doing everything with autoconf macros, but that quickly
became far too precarious -- this route isn't suitable for more than
light customisation.

Since, for safety, we had to have \texttt{autoconf}, \texttt{automake}
and \texttt{libtool} snapshots in our repository in any case, it was a
natural (though nerve-wracking) move to start using patched versions
of those tools.

We needed more Fortran support (checking for open specifiers,
intrinsics, record lengths and the like) than was 
available in standard \texttt{autoconf}.  This group of modifications
are due to be contributed back to the \texttt{autoconf} maintainers.

The \texttt{automake} source code isn't an easy read, but it's pretty
well written, and reasonably easy to patch.  The result is robust and
portable distributed Makefiles.



\section{Lessons and Warnings}

\begin{itemize}
\item The Fortran support in autoconf is rather slim.

\item The port to OSX was easier, in some ways, than we expected,
  partly because the OSX system compiler is a modified GCC.  We needed
  the Fink/OpenDarwin port of \texttt{g77}, and that causes lots of
  linking problems (the whole \texttt{restFP/saveFP} saga).  Be
  warned: the OSX linker has some very fixed ideas about things.

\item Let's say that again.  The OSX linker will find more duplicated
  symbols than you thought you had symbols.  It doesn't like common
  blocks at all.  And it has very certain opinions about how libraries
  should be put together.  GNU \texttt{libtool} can hide many of the
  details from you, but be prepared to spend time tweaking your library code.

\item We couldn't automatically convert our old build system to the
  new one, because it was mostly hand-maintained.  But that turned out
  not to be a problem in fact: disciplined coding in the past meant
  that packages could
  generally be ported to the new system with little thought, and this
  turned out to be a small part of the effort.

\item The original plan was to autoconf everything with only necessary
  code changes.  However it was impossible to stop ourselves
  refactoring and tidying, sometimes rather extensively.  This is
  both a warning to other projects that they won't be able to stop
  developers doing this, and a benefit, in the sense that a lot of
  code-hygiene tasks which have been too boring, confusing or risky in
  the past, become a lot less so when you're adjusting everything
  anyway.

\item Simply finding all our code turned out to be tricky.  For
  historical reasons, including the fact that CVS wasn't really around
  ten years ago, the master copies of code were held by the
  (distributed) developers.  Collecting all the code into one
  repository was a valuable exercise, making better curation easy.
  This also meant that we separated out generated code and sources, and
  rescued some code from oblivion, as developers retired or left the
  project.

\item We should have bowed to the inevitable, and started tweaking the
  autotools earlier.  Delaying this meant some functionality had to be
  implemented twice.

\item GPLing was harder than we expected.  This is largely because of
  the donations in the past, of code written in gentler times, with
  bizarre `licences'.

\item It tool a \emph{lot} longer than we expected (surprise!).  Around 6
  person-months to get the initial system up and running, and then
  another 6 to put the remainder into the system.

\end{itemize}



\section{Contacts}

\begin{itemize}
\item Norman Gray: \texttt{norman@astro.gla.ac.uk},
  \htmladdURL{http://www.astro.gla.ac.uk/users/norman/}

\item Tim Jenness: \texttt{t.jenness@jach.hawaii.edu},
  \htmladdURL{http://www.jach.hawaii.edu/~timj/}

\item Starlink: \htmladdURL{http://www.starlink.ac.uk} and
  \htmladdURL{http://dev.starlink.ac.uk}

\item AnonCVS: access at
  \texttt{:pserver:anonymous@cvs.starlink.ac.uk:/cvs} 
  with password `starlink'

\item CVS browser: \htmladdURL{http://cvsweb.starlink.ac.uk}
\end{itemize}


\end{document}
