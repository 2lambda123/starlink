\documentclass[11pt,twoside]{article}

%%% XXX Comment out before submission!!!
\def\RCS$#1: #2 ${\expandafter\def\csname RCS#1\endcsname{#2}}
\RCS$Revision$

\usepackage{adassconf}

%%% XXX comment this out, too
\def\urlbreakingslash{\discretionary{/}{}{/}}
{\catcode`\/=\active
  \gdef\beginurlcodes{\makeusletter
    \def~{\raisebox{-0.2\normalbaselineskip}{\char126}}%
    \catcode`\/=\active \let/\urlbreakingslash
}}

% NB: wierdo conference style has \title and co typeset material,
% rather than store it, so \begin{document} must go here, and there's no
% \maketitle
\begin{document}

\paperID{P2.1.10}
\contact{Norman Gray}
\email{norman@astro.gla.ac.uk}

\title{Porting the Starlink Software Collection to GNU Autotools}
% XXX Comment out
\marginpar{[v\RCSRevision]}
\titlemark{Starlink \& GNU Autoconf}

\author{Norman Gray\altaffilmark{1}}
\affil{Starlink Project,
    Rutherford Appleton Laboratory,
    Chilton,
    Didcot,
    OX11 0QX,
    UK.
    \texttt{<}\htmladdURL{http://www.astro.gla.ac.uk/users/norman/}\texttt{>}}
\author{Tim Jenness}
\affil{Joint Astronomy Centre,
    660 North A'ohoku Place,
    University Park,
    Hilo, HI 96720,
    USA.
    \texttt{<}\htmladdURL{http://www.jach.hawaii.edu/~timj/}\texttt{>}}
\author{Alasdair Allan\altaffilmark{1},
    David S. Berry\altaffilmark{1},
    Malcolm J. Currie,
    Peter W. Draper\altaffilmark{1},
    Mark B. Taylor\altaffilmark{1}}
\affil{Starlink}
\author{Brad Cavanagh}
\affil{Joint Astronomy Centre}

\altaffiltext{1}{Alternate affiliations:
NG,
%        Department of Physics and Astronomy,
        University of Glasgow%,
%        Glasgow,
%        G12 8QQ, 
%        UK%
;
AA, University of Exeter;
DSB,
%        Centre for Astrophysics,
        University of Central Lancashire%,
%        Preston,
%        PR1 2HE,
%        UK%
;
PWD,
%        Department of Physics, 
        University of Durham%, 
        %Rochester Building, 
        %Science Laboratories, 
        %South Road, 
%        Durham, DH1 3LE, 
%        UK%
;
MBT,
%        Department of Physics,
        University of Bristol%,
%        %Tyndall Avenue,
%        Bristol,
%        BS8 1TL,
%        UK%
        }

\paindex{Gray, N.}
\aindex{Jenness, T.}
\aindex{Allan, A.}
\aindex{Berry, D. S.}
\aindex{Currie, M. J.}
\aindex{Draper, P. W.}
\aindex{Taylor, M. B.}
\aindex{Cavanagh, B.}

\authormark{Gray, Jenness, Allan, Berry, Currie, Draper, Taylor \& Cavanagh}



\keywords{Starlink, autoconf, automake}

\begin{abstract}
The Starlink software collection currently runs on three different
Unix platforms and contains around 100 separate software items,
totalling 2.3~million lines of code, in a mixture of languages.  We
have changed the build system from a hand-maintained collection of
makefiles with hard-wired OS variants to a scheme involving
feature-discovery via GNU Autoconf.

As a result of this work, we have already ported the collection to
Mac\,OS\,X and Cygwin.  This had some unexpected benefits and costs, and
valuable lessons.
\end{abstract}



The Starlink Software Collection (Giaretta 2004) is a large set of
data-reduction 
applications, written over about two decades in a variety of
languages.  Until recently the source code was managed, and the
products distributed, using a system that worked, but which was
rather ad hoc and opaque, and which was correspondingly rather
expensive to maintain.  For a variety of technical, political, and financial
reasons we made the decision to overhaul this build system, resulting
in an CVS-plus-autoconf setup which is much more familiar to current users and
developers.   In Section~\ref{s:background} we elaborate the
motivations for this and the problems we faced; in
Section~\ref{s:result} we briefly describe the resulting system; and
in Section~\ref{s:lessons} we list some of the lessons we learned, in
the hope that these will be of use to other projects considering the
same important move.


\section{Background}
\label{s:background}

% Ought this section to be shorter -- folk are interested in the
% solutions, not the original problems.

\htmladdnormallinkfoot{Starlink}{http://www.starlink.ac.uk} was set up
originally in the late seventies, as a way of supplying UK astronomy
with hardware (`astronomers will never need more than six
VAXes\dots'), naturally along with the data-analysis software to go
with it, and the system management to make it all work smoothly.

We switched the hardware and software to Unix in the early nineties,
ending up via a sequence of lesser migrations (plus Digital/Compaq
rebrandings) supporting the collection on Solaris, Tru64 and RedHat
Linux.  The port to Unix meant that some software was dropped at this
time, and the rest preened quite extensively.  The build system we are
now moving away from was designed in this period.

Starting in the late nineties, Starlink slimmed down (the diet was not
voluntary), dropped the hardware and management provision, and focused
on maintaining and developing the large legacy codebase -- now renamed
the `Classic' collection, in the best marketing traditions -- and on
developing new software, mostly in Java, in tune with more recent
developments in astronomical software.  Though it is not now actively
developed, the classic software is still depended upon by astronomers
world-wide, as well as being a valuable resource for pipelines
(Cavanagh 2003; Currie 2003) and application engines in the VO (Giaretta 2004).
For this reason, and because we expect the community will have to take
a larger role in its curation, we need to make the classic software as
accessible as possible.

The biggest problem with the software was that there was an awful lot
of it!  Even with ongoing pruning, we have ended up with about 100
components, totalling around 1700\,kSLOC written by the project or
curated by it, in various languages
including Fortran, C, C++, Perl and Tcl/Tk.  We add to this another
600\,kSLOC of thirdparty code, some tweaked, and all built at the same
time.  Not included in this total is about 300\,kSLOC of Java,
built separately.  Just to put this in context, David A
Wheeler's \texttt{SLOCCount} would have us believe that that's worth
\$110M and, according to his analysis of the RH7.1 distribution
(Wheeler 2001), it appears that 1.7\,MSLOC is larger than anything in
that distribution except the kernel, Mozilla and XFree86.  Busy types,
aren't we?

As mentioned above, the build system we had was idiosyncratic, with
code scattered amongst the distributed developers, and a set of source
and compiled build products maintained by cut-and-paste of template
makefiles.  This system did work, and was a reasonable design for the
early nineties, but it required discipline and effort to maintain, and
worked largely because most users obtained their software through
pre-built tape and CD distributions, in many cases installed by system
managers who were also at that time employed by the Starlink Project
(whose complaints could therefore be bought off by donations of beer).
As Starlink's role changes, it is important that this software set
appear more normal, and be maintainable (without medication) by a
broader range of developers.

One of the reasons the build system was odd was because the collection
had a large number of conventions about installation locations, plus
documentation and code-generation tools, all of which had to be
catered for when building software.  In order to avoid major rewrites
of the code base, it was necessary to duplicate these conventions and
tool support in the new build system.

Portability was not expected to be a major problem (and wasn't in
fact), since the software was already being supported on three Unix
platforms; as well, some of the older applications had already been
through a port from VMS to Unix.  Also, the project has benefited from
generally good software practices over the years, so the code is
mostly clean, conservative, and well documented; the project's
long-standing obsession with code-standards has certainly paid off,
with interest.  Known portability issues were isolated, though these
were handled by including platform-specific versions of key routines,
selected by the user at build-time; this limited the target platforms
to those explicitly supported by the project.

% Do we need to mention the age of the software -- crustiness?



\section{The Outcome}
\label{s:result}

With this project now largely complete, we have made a number of
significant improvements to the collection.

All the source code is now conveniently available in a single CVS
repository, with anonymous access
(\texttt{:pserver:anonymous@cvs.starlink.ac.uk:/cvs} with password `starlink').

The code now uses the GNU autotools throughout.  Libtool handles the
mind-bending details of building static and shared libraries, with the
result that the collection now builds and installs the shared
libraries that were too much trouble before.  Automake generates the
makefiles, respecting all of our installation conventions and
generating support for our internal build tools.  And autoconf
generates the \texttt{./configure} scripts which test the capabilities
of the build system and adapt the makefiles and other files
appropriately.  It turned out to be impractical to use standard
autoconf and automake, and so the former was extended with new macros,
and the latter adapted with new logic.  Because we are using automake,
the amount of text we have to put into a \texttt{Makefile.am} is
substantially reduced, with very little cut-and-pasted boilerplate.

As a result of using the autotools, the entire 2.3\,MSLOC collection
now builds successfully on Linux (RHEL is our current test platform), Solaris,
Windows (using Cygwin), and Mac\,OS\,X, even though the latter two
were not explicit goals of this porting project.  It should work on
other platforms also, but we have not yet tested it there.



\section{Lessons and Warnings}
\label{s:lessons}

The Fortran support in autoconf is rather slim.  A large proportion of the
autoconf extensions addressed the rather inadequate support for
Fortran in the standard autoconf distribution, and these new macros -- for
testing for intrinsics, open specifiers, support for \texttt{\%val}
and the like -- will be offered back to the autoconf mainline.

The port to OS\,X was easier, in some ways, than we expected, partly
because the OS\,X system compiler is a modified GCC.  However Apple's
GCC installation does not include \texttt{g77}, so we needed to
install the Fink/OpenDarwin port of that.  This causes a number of
linking problems because of slight differences in the code generated
by the two GCC back-ends.  This is the \texttt{restFP/saveFP} problem:
see (Gray 2004) for a summary.  Be warned: the OS\,X linker has some
very fixed ideas about the organisation of code.

Let's say that again.  The OS\,X linker can in some circumstances
resolve library symbols lazily at runtime, and as a result is particularly
alert to duplications, and will find more duplicated
globals than you thought you had symbols.  It doesn't like common
blocks at all.  And it has very certain opinions about how libraries
should be put together.  GNU \texttt{libtool} can hide many of the
details from you, but be prepared to spend time tweaking your library
code.

We couldn't automatically convert our old build system to the new one,
because it was mostly hand-maintained.  But that turned out not to be
a problem in fact: disciplined coding in the past meant that packages
could generally be ported to the new system with little thought, and
this turned out to be a small part of the effort.

The original plan was to autoconf everything with only necessary code
changes.  However it was impossible to stop ourselves refactoring and
tidying, sometimes rather extensively.  This is both a warning to
other projects that they won't be able to stop developers doing this,
and a benefit, in the sense that a lot of code-hygiene tasks that have
been too boring, confusing or risky in the past, become a lot less so
when you're adjusting everything anyway.  In addition, merely
gathering all our code into a coherent repository turned out to be a
non-trivial but valuable exercise.  For historical reasons, including
the fact that systems like CVS were less common ten years ago, the
master copies of code were held by the (distributed) developers.  As a
consequence of the audit, all the code is now in one place, we have
identified modules of generated code and the sometimes
developer-specific tools used to produce them, and we rescued some
code from backup oblivion, as developers retired or left the project.
This consolidation overlaps with the refactoring work, since the
consequent perception that code is owned in common means that the
developers are more willing to criticise each others' code, and with
appropriate consultation refactor or fix it.

Starlink is a well-run project with disciplined developers, but we
were surprised by the number of individually minor hurdles we had
accumulated over a decade, despite our care, enough that working with
the code would have been daunting for a non-insider.  Given that we
last that long, we should probably plan to have another buildsystem
overhaul in a decade or so.

We should have bowed to the inevitable, and started patching the
autotools earlier, since delaying this meant some functionality had to be
implemented twice.  Our initial expectation was that we could provide
all of our extra boilerplate using autoconf macros (using the now
partly deprecated \texttt{aclocal} extension mechanism), but as soon as
this boilerplate consisted of extra makefile rules, the scripting
involved in supporting this became extremely complicated and obviously
fragile.  Patching automake is not a trivial undertaking, but the
resulting increase in robustness, and the control it gave us over the
build system, made it worthwhile.  Though it is undeniably
complicated, the automake code is well written, and reasonably easy to
patch.  Even if we had not patched it, it
would probably be wise, in a project of this size, to keep a
repository copy of the automake and autoconf distributions and mandate
their use when bootstrapping the source tree; apart from the increase
in long-term security, there are a number of potentially nasty
version-skew problems that this practice evades.  This admittedly adds
an extra step for a developer interested in working on CVS sources,
but since the bootstrapping of the CVS sources is necessarily
elaborate anyway, and since it is only the more sophisticated
developers who would work on CVS rather than distributed sources, this
is probably not too much of a disadvantage in practice.

As a general point, we can recommend the use of automake.  Though
there is a rather steep learning curve, and one does naturally suspect
it of being dangerously clever, this turns out not to be a problem
\emph{as long as} you are willing to build your project the way
automake things it ought to be built.  Since the (GNU) coding
standards it implements are both conventional and reasonably sensible,
this is not the imposition it appears.  However it follows that
adapting a pre-existing makefile to automake can be frustrating and
hard, and it is better in this case to start again, and produce the
much shorter \texttt{Makefile.am} automake source file from scratch.

In parallel with this porting project, we worked through the internal
bureaucracy involved in applying a GPL licence to the code where
possible.  This turned out to be harder than we expected, largely
because of the donations, in the past, of code written in gentler
times, with bizarre licences (such as `public domain, for academic use
only'!).  We ended up conceding defeat on this, as it would clearly be
a huge effort to work through all the contributed code, identifying
first authors then consistent conditions for each module.  We have
adopted the pragmatic policy of stating that everything in the
collection with an explicit CCLRC copyright is GPL; users concerned
about the remainder should get in touch with us and we'll try to work
it out.

The project took a \emph{lot} longer than we expected (surprise!).  It
took around six person-months to get the initial system up and
running, and then another six to adapt the bulk of our code to the new
system.



\begin{references}

\reference  Cavanagh, B., Hirst, P., Jenness, T., Economou, F.,
Currie, M. J., Todd, S., \& Ryder, S. D. 2003, \adassxii, 237

\reference Currie, M.\ J.\ 2003, \adassxiii, 409
% paper ref P4.25

\reference Giaretta, D, et al., 2004, \adassxiv, \paperref{O7.3}
% O7.3: VO-enabling a major astronomical analysis and  reduction software system    David Giaretta, Starlink/RAL , Malcolm Currie,  Starlink/RAL , Stephen Currie, Starlink/RAL ,  Mark Taylor, Starlink/U of Bristol , Norman Gray,  Starlink/U of Glasgow , Peter Draper, Starlink/U  of Durham , David Berry, Starlink/U C lancs ,  Alasdair Allan, Starlink/U of Exeter    

\reference Giaretta, D, et al., 2004, \adassxiv, \paperref{D12}

\reference Gray, Norman, 2004, `OSX and the restFP/saveFP problem'.
Web page:
\htmladdURL{http://www.astro.gla.ac.uk/users/norman/note/2004/restFP/}
(cited November 2004)

%`restFP/saveFP and
%\texttt{-lcc\textunderscore dynamic} -- why?'.  Thread on
%\texttt{fortran-dev@lists.apple.com}, archived at
% let the URL break (no global defs allowed, and presumably no
% \usepackage{url}, or else we could improve the definition of
% \makeURL in adassconf.sty)
%{\def\/{\discretionary{/}{}{/}}%
%  \htmladdnormallink
%    {\makeURL{http:/\/lists.apple.com\/archives\/fortran-dev\/2004\/Jul\/msg00036.html}}
%    {http://lists.apple.com/archives/fortran-dev/2004/Jul/msg00036.html}}
%\htmladdURL{http://lists.apple.com/archives/fortran-dev/2004/Jul/msg00036.html}
%(cited November 2004).

\reference Wheeler, David A., 2001 (updated 2002), `More than a
Gigabuck: Estimating GNU/Linux's Size'.  Web page:
%{\def\/{\discretionary{/}{}{/}}%
%  \htmladdnormallink
%    {\makeURL{http:/\/www.dwheeler.com\/sloc\/redhat71-v1\/redhat71sloc.html}}
%    {http://www.dwheeler.com/sloc/redhat71-v1/redhat71sloc.html}}
\htmladdURL{http://www.dwheeler.com/sloc/redhat71-v1/redhat71sloc.html}
(cited November 2004)

\end{references}

\end{document}
