%APN3_PROCEEDINGS_FORM%%%%%%%%%%%%%%%%%%%%%%%%%%%%%%%%%%%%%%%%%%%%%%%
%
% TEMPLATE.TEX -- APN3 (2003) ASP Conference Proceedings template.
%
% Derived from ADASS VIII (98) ASP Conference Proceedings template
% Updated by N. Manset for ADASS IX (99), F. Primini for ADASS 2000,
% D.Bohlender for ADASS 2001, and H. Payne for ADASS XII and LaTeX2e.
%
% Use this template to create your proceedings paper in LaTeX format
% by following the instructions given below.  Much of the input will
% be enclosed by braces (i.e., { }).  The percent sign, "%", denotes
% the start of a comment; text after it will be ignored by LaTeX.
% You might also notice in some of the examples below the use of "\ "
% after a period; this prevents LaTeX from interpreting the period as
% the end of a sentence and putting extra space after it.
%
% You should check your paper by processing it with LaTeX.  For
% details about how to run LaTeX as well as how to print out the User
% Guide, consult the README file.  You should also consult the sample
% LaTeX papers, sample1.tex and sample2.tex, for examples of including
% figures, html links, special symbols, and other advanced features.
%
% If you do not have access to the LaTeX software or a laser printer
% at your site, you can still prepare your paper following the
% instructions in the User Guide.  In such cases, the editors will
% process the file and make any necessary editorial adjustments.
%
%%%%%%%%%%%%%%%%%%%%%%%%%%%%%%%%%%%%%%%%%%%%%%%%%%%%%%%%%%%%%%%%%%%%%%%%
%
\documentclass[11pt,twoside]{article}  % Leave intact
\usepackage{adassconf}

% If you have the old LaTeX 2.09, and not the current LaTeX2e, comment
% out the \documentclass and \usepackage lines above and uncomment
% the following:

%\documentstyle[11pt,twoside,adassconf]{article}

\begin{document}   % Leave intact

%-----------------------------------------------------------------------
%			    Paper ID Code
%-----------------------------------------------------------------------
% Enter the proper paper identification code.  The ID code for your
% paper is the session number associated with your presentation as
% published in the official conference proceedings.  You can
% find this number locating your abstract in the printed proceedings
% that you received at the meeting or on-line at the conference web
% site; the ID code is the letter/number sequence proceeding the title
% of your presentation.
%
% This will not appear in your paper; however, it allows different
% papers in the proceedings to cross-reference each other.  Note that
% you should only have one \paperID, and it should not include a
% trailing period.
%
% EXAMPLE: \paperID{O4-1}
% EXAMPLE: \paperID{P7-7}
%

\paperID{D-12}

%-----------------------------------------------------------------------
%		            Paper Title
%-----------------------------------------------------------------------
% Enter the title of the paper.
%
% EXAMPLE: \title{A Breakthrough in Astronomical Software Development}
%
% If your title is so long as to fill the page header when you print it,
% then please supply a short form as a \titlemark.
%
% EXAMPLE:
%  \title{Rapid Development for Distributed Computing, with Implications
%         for the Virtual Observatory}
%  \titlemark{Rapid Development for Distributed Computing}
%

\title{Starlink Software Developments}
%\titlemark{ }

%-----------------------------------------------------------------------
%		          Authors of Paper
%-----------------------------------------------------------------------
% Enter the authors followed by their affiliations.  The \author and
% \affil commands may appear multiple times as necessary (see example
% below).  List each author by giving the first name or initials first
% followed by the last name.  Authors with the same affiliations
% should grouped together.
%
% EXAMPLE: \author{Raymond Plante, Doug Roberts,
%                  R.\ M.\ Crutcher\altaffilmark{1}}
%          \affil{National Center for Supercomputing Applications,
%                 University of Illinois Urbana-Champaign, Urbana, IL
%                 61801}
%          \author{Tom Troland}
%          \affil{University of Kentucky}
%
%          \altaffiltext{1}{Astronomy Department, UIUC}
%
% In this example, the first three authors, "Plante", "Roberts", and
% "Crutcher" are affiliated with "NCSA".  "Crutcher" has an alternate
% affiliation with the "Astronomy Department".  The fourth author,
% "Troland", is affiliated with "University of Kentucky"

\author{Peter W. Draper}
\affil{Department of Physics, University of Durham, South Road, Durham,
DH1 3LE, UK}

\author{Alasdair Allan}
\affil{University of Exeter, UK}

\author{David S. Berry}
\affil{University of Central Lancashire, UK}

\author{Malcolm J. Currie, David Giaretta, Steve Rankin}
\affil{Rutherford Appleton Laboratory, UK}

\author{Norman Gray}
\affil{University of Glasgow, UK}

\author{Mark B. Taylor}
\affil{University of Bristol, UK}


%-----------------------------------------------------------------------
%			 Contact Information
%-----------------------------------------------------------------------
% This information will not appear in the paper but will be used by
% the editors in case you need to be contacted concerning your
% submission.  Enter your name as the contact along with your email
% address.
%
% EXAMPLE:  \contact{Dennis Crabtree}
%           \email{crabtree@cfht.hawaii.edu}
%

\contact{Peter Draper}
\email{p.w.draper@durham.ac.uk}

%-----------------------------------------------------------------------
%		      Author Index Specification
%-----------------------------------------------------------------------
% Specify how each author name should appear in the author index.  The
% \paindex{ } should be used to indicate the primary author, and the
% \aindex for all other co-authors.  You MUST use the following
% syntax:
%
% SYNTAX:  \aindex{Lastname, F. M.}
%
% where F is the first initial and M is the second initial (if
% used).  This guarantees that authors that appear in multiple papers
% will appear only once in the author index.
%
% EXAMPLE: \paindex{Crabtree, D.}
%          \aindex{Manset, N.}
%          \aindex{Veillet, C.}
%
% NOTE: this information is also used to build the author list that
% appears in the table of contents.  Authors will be listed in the order
% of the \paindex and \aindex commmands.
%

\paindex{Draper, P. W.}
\aindex{Allan, A.}
\aindex{Berry, D. S.}
\aindex{Currie, M. J.}
\aindex{Giaretta, D.}
\aindex{Rankin, S.}
\aindex{Gray, N.}
\aindex{Taylor, M. B.}

%-----------------------------------------------------------------------
%		      Author list for page header
%-----------------------------------------------------------------------
% Please supply a list of author last names for the page header. in
% one of these formats:
%
% EXAMPLES:
% \authormark{Lastname}
% \authormark{Lastname1 \& Lastname2}
% \authormark{Lastname1, Lastname2, ... \& LastnameN}
% \authormark{Lastname et al.}
%
% Use the "et al." form in the case of seven or more authors, or if
% the preferred form is too long to fit in the header.

\authormark{Draper, Allan, Berry, Currie, Giaretta, Rankin, Gray, Taylor}

%-----------------------------------------------------------------------
%			Subject Index keywords
%-----------------------------------------------------------------------
% Enter a comma separated list of up to 6 keywords describing your
% paper.  These will NOT be printed as part of your paper; however,
% they will be used to generate the subject index for the proceedings.
% There is no standard list; however, you can consult the indices
% for past proceedings (http://adass.org/adass/proceedings/).
%
% EXAMPLE:  \keywords{visualization, astronomy: radio, parallel
%                     computing, AIPS++, Galactic Center}
%
% In this example, the author noticed that "radio astronomy" appeared
% in the ADASS VII Index as "astronomy" being the major keyword and
% "radio" as the minor keyword.  The colon is used to introduce another
% level into the index.

\keywords{Starlink, software: applications, software: data analysis,
software: pipeline, visualization, time-series, catalogs,
image: astrometry, Java, spectroscopy, Virtual Observatory, VOTable, WCS}
%-----------------------------------------------------------------------
%			       Abstract
%-----------------------------------------------------------------------
% Type abstract in the space below.  Consult the User Guide and Latex
% Information file for a list of supported macros (e.g. for typesetting
% special symbols). Do not leave a blank line between \begin{abstract}
% and the start of your text.

\begin{abstract}          % Leave intact
% Place the text of your abstract here - NO BLANK LINES
Various recent changes to the software produced by Starlink were
demonstrated. These cover areas such as tables handling, time-series analysis,
pipeline processing, astrometric calibration, spectral and cube visualisation
and ports to the Mac\,OS\,X and Cygwin environments. Particular emphasis
was given to the applicability to the Virtual Observatory.
\end{abstract}

%-----------------------------------------------------------------------
%			      Main Body
%-----------------------------------------------------------------------
% Place the text for the main body of the paper here.  You should use
% the \section command to label the various sections; use of
% \subsection is optional.  Significant words in section titles should
% be capitalized.  Sections and subsections will be numbered
% automatically.
%
% EXAMPLE:  \section{Introduction}
%           ...
%           \subsection{Our View of the World}
%           ...
%           \section{A New Approach}
%
% It is recommended that you look at the sample papers, sample1.tex
% and sample2.tex, for examples for formatting references, footnotes,
% figures, equations, html links, lists, and other special features.

\section{Changes to the Classic Starlink Software Collection}

The
\htmladdnormallink{Starlink Software Collection}{http://www.starlink.ac.uk}
has very recently been converted to use a build system based on GNU Autotools,
moved to a single CVS repository and re-licensed under the GPL. 
These are seen as key developments in the move to a more open strategy for
future support, and so that the whole system can be more easily ported to
Unix-like operating systems (in particular
\textsc{posix} systems which include an X~server).
The details of this work are described in Gray et al. (2005).  
Figure 1 shows some of our applications running under Mac\,OS\,X (similar screen
shots of Cygwin/Windows, Solaris and various flavours of Linux have been
omitted for space reasons).

\begin{figure}
\epsscale{0.8}\plotone{Draper_D-12_1.eps}
\caption{GAIA and KAPPA running under Mac\,OS\,X, a development recently made
possible by the introduction of a GNU Autotools-based build system.}
\end{figure}

\section{TOPCAT \& STIL, New Features}

Starlink \htmladdnormallink{TOPCAT}{http://www.starlink.ac.uk/topcat} 
is a user-friendly graphical program for viewing, analysing
and editing tables in many different formats. It is based on STIL --
the Starlink Tables Infrastructure Library -- a pure-Java class library for
accessing and manipulating astronomical tables. Recent changes to
TOPCAT include:
\begin{itemize}
\item New `activation actions', such as causing the display of region around
      an object from an image cutout service. These are extensible.
\item Many plotting improvements -- faster, tidier, flippable axes.
\item Graphical selection of arbitrary shaped subsets.
\item Faster and improved matching.
\item Support for more table formats (VOTable 1.1, CSV, FITS-plus).
\end{itemize}

\begin{figure}
\epsscale{0.6}\plotone{Draper_D-12_2.eps}
\caption{TOPCAT shown selecting an arbitrary region to create a subset}
\end{figure}

For more details of TOPCAT and STIL see Taylor (2005).

\section{FROG, New Application}

Starlink \htmladdnormallink{FROG}{http://www.starlink.ac.uk/frog} is a new
time-series analysis and display application written in Java. It has been
designed to provide a simple user interface for astronomers wanting to do time
domain astrophysics, but still have access to powerful features like those
found in PERIOD. A full description of FROG can be found in Allan (2005).

\section{ORAC-DR Pipelines}

\htmladdnormallink{ORAC-DR}{http://www.jach.hawaii.edu/JACpublic/UKIRT/software/oracdr}
is a generic pipeline system written by the Joint Astronomy Centre,
Hawaii (Cavanagh et al. 2003), which uses Starlink applications to
process data.  As part of its support for UK astronomy, Starlink has
been developing new ORAC-DR pipelines and recipes for some
unsupported, or partly supported instruments, most recently SOFI (ESO)
and NIRI (Gemini).  Currie (2005) compares one such recipe for the ISAAC
instrument with the developing ESO equivalent.

\section{GAIA, New Features}

\htmladdnormallinkfoot{GAIA}{http://www.starlink.ac.uk/gaia} is a classic
Starlink application (Draper 2004a), based on the
\htmladdnormallink{ESO Skycat}{http://archive.eso.org/skycat/} tool, for the
visualisation and analysis of images and catalogues. 
It offers many proto-VO features, such as remote catalogue and image
downloads, which are widely used. 
Recent changes (beyond its arrival on Mac\,OS\,X and Cygwin, along with the
other Starlink applications) are the ability to display slices from cubes and
to perform automated astrometric calibrations.

\begin{figure}
\epsscale{0.7}\plotone{Draper_D-12_3.eps}
\caption{GAIA displaying a slice, along the third axis of an NDF cube. It is
also possible to step automatically through a range of slices, and collapse
a range into an image. Displayed images can be analysed like any other image.}
\end{figure}

\section{SPLAT, New Features}
\htmladdnormallink{SPLAT}{http://www.starlink.ac.uk/splat}
is a Java based tool for visualising and analysing spectra, which is
described in Draper (2004b). Some of its more recent enhancements are:
\begin{itemize}
\item Tables access, including VOTable as provided by the STIL library
(Taylor 2005).
\item Graphic annotations and log axes.
\item Interpolated backgrounds using hand-drawn splines.
\item Hierarchical data browsing and access.
\item Spectral line deblending (from the command-line only at present).
\end{itemize}
\begin{figure}
\epsscale{0.7}\plotone{Draper_D-12_4.eps}
\caption{SPLAT reading a FITS binary table and displaying the spectrum formed
from the two selected columns. Note the use of log axes.}
\end{figure}

\section{Virtual Observatory Tie-ins}

Several of the applications described here process VOTables and offer
Web-service access to their functionality.
They are also written in Java, so are ideally placed to quickly adapt and
interact with VO services as they become available (importantly much of this is
also from a desktop point of view). 
This stance has recently been demonstrated by the extension of SPLAT (now
SPLAT-VO) to include a Simple Spectral Access Protocol UI.
The Starlink classic code base is now much more portable, and freely
available, which makes it more attractive for processing data, and it comes
with a well proven pipeline system, as well as also being wrappable in Java so
that it can act as ready supply of web-services.
For more about Starlink's efforts in the VO area see Giaretta et al. (2005).

\section{Finding Out More About Starlink Applications}

Information about Starlink applications and how to download them is available
at \htmladdURL{http://www.starlink.ac.uk}.

%-----------------------------------------------------------------------
%			      References
%-----------------------------------------------------------------------
% List your references below within the reference environment
% (i.e. between the \begin{references} and \end{references} tags).
% Each new reference should begin with a \reference command which sets
% up the proper indentation.  Observe the following order when listing
% bibliographical information for each reference:  author name(s),
% publication year, journal name, volume, and page number for
% articles.  Note that many journal names are available as macros; see
% the User Guide listing "macro-ized" journals.
%
% EXAMPLE:  \reference Hagiwara, K., \& Zeppenfeld, D.\  1986,
%                Nucl.Phys., 274, 1
%           \reference H\'enon, M.\  1961, Ann.d'Ap., 24, 369
%           \reference King, I.\ R.\  1966, \aj, 71, 276
%           \reference King, I.\ R.\  1975, in Dynamics of Stellar
%                Systems, ed.\ A.\ Hayli (Dordrecht: Reidel), 99
%           \reference Tody, D.\  1998, \adassvii, 146
%           \reference Zacharias, N.\ \& Zacharias, M.\ 2003,
%                \adassxii, \paperref{P7.6}
%
% Note the following tricks used in the example above:
%
%   o  \& is used to format an ampersand symbol (&).
%   o  \'e puts an accent agu over the letter e.  See the User Guide
%      and the sample files for details on formatting special
%      characters.
%   o  "\ " after a period prevents LaTeX from interpreting the period
%      as an end of a sentence.
%   o  \aj is a macro that expands to "Astron. J."  See the User Guide
%      for a full list of journal macros
%   o  \adassvii is a macro that expands to the full title, editor,
%      and publishing information for the ADASS VII conference
%      proceedings.  Such macros are defined for ADASS conferences I
%      through XI.
%   o  When referencing a paper in the current volume, use the
%      \adassxii and \paperref macros.  The argument to \paperref is
%      the paper ID code for the paper you are referencing.  See the
%      note in the "Paper ID Code" section above for details on how to
%      determine the paper ID code for the paper you reference.
%
\begin{references}

\reference Allan, A.\ 2005, \adassxiv, \paperref{P1-2-2}
\reference Cavanagh, B., Hirst, P., Jenness, T., Economou, F., Currie, M. J.,
Todd, S., \& Ryder, S.  D.\ \adassxii, 237
\reference Currie, M.\ J.\ 2005, \adassxiv, \paperref{P2-2-2}
\reference Draper, P. W.\  2004a, 
\htmladdnormallink{Starlink User Note 243}{http://www.starlink.ac.uk/cgi-bin/htxserver/sun243.htx/},
Starlink Project CCLRC
\reference Draper, P. W.\  2004b,
\htmladdnormallink{Starlink User Note 214}{http://www.starlink.ac.uk/cgi-bin/htxserver/sun214.htx/},
Starlink Project CCLRC
\reference Giaretta, D., Currie, M.\ J., Rankin, S., Taylor, M., Gray, N.,
Draper, P.\ W., Berry, D.\ S., \& Allan, A.\ \adassxiv, \paperref{O7-3}
\reference Gray, N., Jenness, T., Allan, A., Berry, S.\ B., Currie, M.\ J.,
Draper, P.\ W., Taylor, M.\ B., Cavanagh, B.\ C.\ 2005, \adassxiv, \paperref{P2-1-10}
\reference Taylor, M. B.\ 2005, \adassxiv, \paperref{FM-3}


\end{references}

% Do not place any material after the references section

\end{document}  % Leave intact
