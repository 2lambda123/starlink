%ADASS_PROCEEDINGS_FORM%%%%%%%%%%%%%%%%%%%%%%%%%%%%%%%%%%%%%%%%%%%%%%%
%
% SAMPLE1.TEX -- ADASS XII (2002) ASP Conference Proceedings sample
% paper with minimal markup. Based on the sample from ADASS XI (01).
%
% This is a simple example.  If you want to see a more comprehensive
% sample paper,  take a look at sample2.tex.
%
% Much of the input will be enclosed by braces (i.e., { }).  The
% percent sign, "%", denotes the start of a comment; text after it
% will be ignored by LaTeX.  You might also notice in some of the
% examples below the use of "\ " after a period; this prevents LaTeX
% from interpreting the period as the end of a sentence and putting
% extra space after it.
%
% You should check your paper by processing it with LaTeX.  For
% details about how to run LaTeX as well as how to print out the User
% Guide, consult the README file.
%
% If you do not have access to the LaTeX software or a laser printer
% at your site, you can still prepare your paper following the
% instructions in the User Guide.  In such cases, the editors will
% process the file and make any necessary editorial adjustments.
%
%%%%%%%%%%%%%%%%%%%%%%%%%%%%%%%%%%%%%%%%%%%%%%%%%%%%%%%%%%%%%%%%%%%%%%%%
%
\documentclass[11pt,twoside]{article}  % Leave intact
\usepackage{adassconf}

% If you have the old LaTeX 2.09, and not the current LaTeX2e, comment
% out the \documentclass and \usepackage lines above and uncomment
% the following:

%\documentstyle[11pt,twoside,adassconf]{article}

\begin{document}   % Leave intact

%-----------------------------------------------------------------------
%			    Paper ID Code
%-----------------------------------------------------------------------
% Enter the proper paper identification code.  The ID code for your
% paper is the session number associated with your presentation as
% published in the official conference proceedings.  You can
% find this number locating your abstract in the printed proceedings
% that you received at the meeting or on-line at the conference web
% site; the ID code is the letter/number sequence proceeding the title
% of your presentation.
%
% This will not appear in your paper; however, it allows different
% papers in the proceedings to cross-reference each other.  Note that
% you should only have one \paperID, and it should not include a
% trailing period.
%

\paperID{D1.1}

%-----------------------------------------------------------------------
%		            Paper Title
%-----------------------------------------------------------------------
% Enter the title of the paper.
%
% EXAMPLE: \title{A Breakthrough in Astronomical Software Development}
%
% If your title is so long as to fill the page header when you print it,
% then please supply a short form as a \titlemark.
%
% EXAMPLE:
%  \title{Rapid Development for Distributed Computing, with Implications
%         for the Virtual Observatory}
%  \titlemark{Rapid Development for Distributed Computing}
%

\title{GAIA 3D: Visualising Data Cubes}
%\titlemark{ }

%-----------------------------------------------------------------------
%		          Authors of Paper
%-----------------------------------------------------------------------
% Enter the authors followed by their affiliations.  The \author and
% \affil commands may appear multiple times as necessary.  List each
% author by giving the first name or initials first followed by the
% last name.  Authors with the same affiliations should grouped
% together.
%
% Try to limit the front matter to no more than three \author
% commands.  Group authors with the same affiliations.  Too many
% \author commands fills the first page of the paper with little
% actual text.

\author{Peter W.\ Draper}
\affil{Department of Physics, Durham University, South Road, 
       Durham DH1 3LE, UK}

\author{Malcolm J.\ Currie}
\affil{CCLRC Rutherford Appleton Laboratory, Chilton, Didcot,
       Oxfordshire, OX11 0QX, UK}

\author{Tim Jenness, Jamie Leech \& Frossie Economou}
\affil{Joint Astronomy Centre, 660 North A'ohoku Place, University Park,
       Hilo, HI 96720, USA}

\author{David S.\ Berry}
\affil{Centre for Astrophysics, University of Central Lancashire, Preston, 
       Lancashire, PR1 2HE, UK}

\author{Mark B.\ Taylor}
\affil{H H Wills Physics Laboratory, Bristol University, Tyndall Avenue, 
       Bristol, BS8 1TL, UK}

%-----------------------------------------------------------------------
%			 Contact Information
%-----------------------------------------------------------------------
% This information will not appear in the paper but will be used by
% the editors in case you need to be contacted concerning your
% submission.  Enter your name as the contact along with your email
% address.

\contact{Peter W. Draper}
\email{p.w.draper@durham.ac.uk}

%-----------------------------------------------------------------------
%		      Author Index Specification
%-----------------------------------------------------------------------
% Specify how each author name should appear in the author index.  The
% \paindex{ } should be used to indicate the primary author, and the
% \aindex for all other co-authors.  You MUST use the following
% syntax:
%
% SYNTAX:  \aindex{LASTNAME, F. M.}
%
% where F is the first initial and M is the second initial (if
% used).  This guarantees that authors that appear in multiple papers
% will appear only once in the author index.
%
% EXAMPLE: \paindex{Crabtree, D.}
%          \aindex{Manset, N.}
%          \aindex{Veillet, C.}
%
% NOTE: this information is also used to build the author list that
% appears in the table of contents.  Authors will be listed in the order
% of the \paindex and \aindex commmands.
%

\paindex{Draper, P. W.}
\aindex{Currie, M. J.}
\aindex{Jenness, T.}
\aindex{Leech, J.}
\aindex{Economou, F.}
\aindex{Berry, D. S.}
\aindex{Taylor, M. B.}

%-----------------------------------------------------------------------
%                     Author list for page header
%-----------------------------------------------------------------------
% Please supply a list of author last names for the page header. in
% one of these formats:
%
% EXAMPLES:
% \authormark{LASTNAME}
% \authormark{LASTNAME1 \& LASTNAME2}
% \authormark{LASTNAME1, LASTNAME2, ... \& LASTNAMEn}
% \authormark{LASTNAME et al.}
%
% Use the "et al." form in the case of seven or more authors, or if
% the preferred form is too long to fit in the header.

\authormark{Draper, Currie, Jenness, Leech, Economou, Berry \& Taylor}

%-----------------------------------------------------------------------
%			Subject Index keywords
%-----------------------------------------------------------------------
% Enter up to 6 keywords describing your paper.  These will NOT be
% printed as part of your paper; however, they will be used to
% generate the subject index for the proceedings.  There is no
% standard list; however, you can consult the indices for past ADASS
% proceedings (http://adass.org/adass/proceedings/).

\keywords{Visualization: GAIA, JCMT, Spectra, Applications Software, 
          Data Cubes, PLASTIC}

%-----------------------------------------------------------------------
%			       Abstract
%-----------------------------------------------------------------------
% Type abstract in the space below.  Consult the User Guide and Latex
% Information file for a list of supported macros (e.g. for typesetting
% special symbols). Do not leave a blank line between \begin{abstract}
% and the start of your text.

\begin{abstract}          % Leave intact
The GAIA application has been a popular choice for examining and analysing
astronomical images for several years. As data volumes have increased and
spectral and time data cubes have become more prevalent there has been a
strong demand for cube visualization. This paper will briefly describe the
major enhancements that have been made to GAIA to support data cubes including
real-time spectral extraction (single spectrum or area), calculation of
moments, creation of channel maps and movie displays. It will also say how the
PLASTIC protocol has been used to interoperate with VO-enabled applications.
\end{abstract}

%-----------------------------------------------------------------------
%			      Main Body
%-----------------------------------------------------------------------
% Place the text for the main body of the paper here.  You should use
% the \section command to label the various sections; use of
% \subsection is optional.  Significant words in section titles should
% be capitalized.  Sections and subsections will be numbered
% automatically.

\section{Introduction}

The JAC has recently started producing data cubes using the HARP focal plane
array receiver and its digital autocorrelation spectrometer ACSIS. These
instruments will be used by astronomers of various levels of experience, so
require a straight-forward data visualisation system, but which is also
astronomically focused, that is has full support for celestial and spectral
coordinates, and can read astronomy data formats (FITS and NDF). A review of
what software is currently available was undertaken by Leech and Jenness
(2004), and with those conclusions in mind work has been undertaken to enhance
the GAIA image display tool to work with data cubes from many different
sources and extreme sizes (4Gb+ on 64bit systems).

Interoperating with other environments, like the VO, and co-operating with
specialised tools is clearly desirable, so we have extended GAIA to broadcast
spectra, images and tables and accept images and tables using the PLASTIC
protocol (Taylor et al. 2006).  This allows the sophisticated selections of a
tool like TOPCAT (Taylor, 2005) to be used, without the need to extend GAIA's
table handling.

\section{Displaying Image Slices from Cubes}

The most basic operation when handling cubes is displaying image slices. This
shown in Figure~\ref{D1.1-fig1}. Care has been taken to make sure moving
between slices is a very fast operation.

\begin{figure}
\epsscale{0.8}\plotone{D1.1_1.eps}
\caption{GAIA display a cube and the toolbox with controls for manipulating
the image slice, note that the slice is no.\ 131 along the third axis, which
has a frequency of $2.223439E+10$ Hz. You change the slice by dragging the
slider control}
\label{D1.1-fig1}
\end{figure}

\section{Spectral Extraction}

Extracting a spectrum from a cube is very simple, just click on an image
position and drag around. The spectrum will be interactively updated.

In addition to single point extraction you can also extract spectra which are
averaged over regions, circles, rectangles, ellipses, polygons and lines of
various kinds. You can also mark an extracted spectrum as the reference
spectrum for simple point-to-point and region-to-region comparisons.
The results of doing these operations are shown in Figure~\ref{D1.1-fig2}.

\begin{figure}
\epsscale{0.8}\plotone{D1.1_2.eps}
\caption{GAIA extracting region spectra. The two regions shown have been
averaged and extracted from the cube.}
\label{D1.1-fig2}
\end{figure}


\section{Coordinate Systems}

The \textbf{Coords} menu offers the chance to switch between a preset list of
common coordinate systems. So you can see you extractions in wavelength,
frequency and (if a rest frequency is defined) velocity. Most of the spectral
systems defined in FITS paper III are supported (using the AST library).

\section{Other Operations}

Other operations that are supported on cubes are animating through a range,
collapsing to various integrated measurements, creating channel map images
(where each sub-panel is itself a collapsed range), and interactive baseline
subtraction. Each extracted image can also be operated on by the image
analysis tools of GAIA, so you can get region statistics, aperture photometry,
object detection, contouring, grid overlays etc.

\section{Spectral analysis}

GAIA does not offer any spectral analysis features, concentrating mainly on
visualisation, so you can `send' the extracted spectrum to the SPLAT-VO
(Draper \& Taylor 2006) tool, or any other PLASTIC enabled spectral tool, as
they become available. You can also save spectra to several formats for more
traditional processing.

%\section{Command-line Support}
%
%Many of the GAIA cube facilities (and also its image ones) are available to be
%used from the command-line in the KAPPA (Currie \& Berry, 2006)
%package. Script support is available in the DATACUBE package (Allan \& Currie,
%2006).

\section{Future Developments}

The most obvious missing feature to support datacube visualisation is some
kind of 3D rendering. Clearly there are many sophisticated packages already
available in this field, but they tend to have a steep learning curve and to
not support astronomy concepts very well. With that in mind we expect to
produce a simple interface, that renders isosurfaces and volumes (to an
existing library) and that deals with coordinates and astronomy data formats.

Support for visualisation of CUPID results (HARP and SCUBA-2 data products,
Berry 2006) are also a priority.

\subsection{Obtaining the software}

GAIA is available under the GPL and is part of the `Keoe' JAC binary releases
described at (note this release does not include PLASTIC spectrum support):
\begin{quote}
\makeURL{http://www.jach.hawaii.edu/software/starlink/}
\end{quote}
Currently they are 32/64 bit Linux \& Mac OS X 10.4 builds.

The GAIA source code can be obtained from (with latest developments):
\begin{quote}
\makeURL{http://cvsweb.starlink.ac.uk} \\
\makeURL{http://dev.starlink.ac.uk}
\end{quote}
This known to build under Solaris, Tru64 UNIX and Cygwin, as well.

% You can also add an acknowledgments section as indicated below.
\acknowledgments

This work has been funded by the Particle Physics and Astronomy Research
Council (PPARC) for the Joint Astronomy Centre, Hawaii (JAC), with the support
of the Central Laboratories for the Research Councils (CLRC). GAIA is based on
the Skycat tool, which was developed at ESO.

%-----------------------------------------------------------------------
%			      References
%-----------------------------------------------------------------------
% List your references below within the reference environment
% (i.e. between the \begin{references} and \end{references} tags).
% Each new reference should begin with a \reference command which sets
% up the proper indentation.  Observe the following order when listing
% bibliographical information for each reference:  author name(s),
% publication year, journal name, volume, and page number for
% articles.  Note that many journal names are available as macros; see
% the User Guide for a listing "macro-ized" journals.
%
% Note the following are some of the tricks that can be used:
%
%   o  \& is used to format an ampersand symbol (&).
%   o  \'e puts an accent grave over the letter e.  See the User Guide
%      for details on formatting special characters.
%   o  "\ " after a period prevents LaTeX from interpreting the period
%      as an end of a sentence.
%   o  \aj is a macro that expands to "Astron. J."  See the User Guide
%      for a full list of journal macros
%   o  \adassvii is a macro that expands to the full title, editor,
%      and publishing information for the ADASS VII conference
%      proceedings.  Such macros are defined for ADASS conferences I
%      through IX.
%   o  When referencing a paper in the current volume, use the
%      \adassix and \paperref macros.  The argument to \paperref is
%      the paper ID code for the paper you are referencing.  See the
%      note in the "Paper ID Code" section above for details on how to
%      determine the paper ID code for the paper you reference.
%
\begin{references}
\reference Berry, D. S., Reinhold, K., Jenness, T., Economou, F. 
           \ 2006, \adassxvi, \paperref{P2.12}

\reference Draper, P. W., Taylor, M. B.\  2006, SPLAT-VO, 
           A Spectral Analysis Tool, Starlink User Note 214

\reference Leech, J., \& Jenness, T. \ 2004 \adassxiv

\reference Taylor, J., Boch, T., Comparato, M., Taylor, M., 
           Winstanely, N., Mann, R. \ 2006 \adassxvi, \paperref{O7.4}

\reference Taylor, M. B.\ 2005, \adassxiv

%\reference Warren-Smith, R. F. \& Berry D. B. \ 2004 \adassxvi

\end{references}

% Do not place any material after the references section

\end{document}  % Leave intact
