%ADASS_PROCEEDINGS_FORM%%%%%%%%%%%%%%%%%%%%%%%%%%%%%%%%%%%%%%%%%%%%%%%
%
% TEMPLATE.TEX -- ADASS VIII (98) ASP Conference Proceedings template
% Updated by N. Manset for ADASS IX (99), F. Primini for ADASS 2000,
% and D.Bohlender for ADASS 2001.
%
% Use this template to create your proceedings paper in LaTeX format
% by following the instructions given below.  Much of the input will
% be enclosed by braces (i.e., { }).  The percent sign, "%", denotes
% the start of a comment; text after it will be ignored by LaTeX.  
% You might also notice in some of the examples below the use of "\ "
% after a period; this prevents LaTeX from interpreting the period as
% the end of a sentence and putting extra space after it.  
% 
% You should check your paper by processing it with LaTeX.  For
% details about how to run LaTeX as well as how to print out the User
% Guide, consult the README file.  
% You should also consult the sample LaTeX papers,
% sample1.tex and sample2.tex, for examples of including figures, html
% links, special symbols, and other advanced features.
%
% If you do not have access to the LaTeX software or a laser printer
% at your site, you can still prepare your paper following the
% instructions in the User Guide.  In such cases, the editors will
% process the file and make any necessary editorial adjustments.
% 
%%%%%%%%%%%%%%%%%%%%%%%%%%%%%%%%%%%%%%%%%%%%%%%%%%%%%%%%%%%%%%%%%%%%%%%%
% 
\documentstyle[11pt,adassconf]{article}  % Leave intact

\begin{document}   % Leave intact

%-----------------------------------------------------------------------
%			    Paper ID Code
%-----------------------------------------------------------------------
% Enter the proper paper identification code.  The ID code for your
% paper is the session number associated with your presentation as
% published in the official ADASS 2000 conference proceedings.  You can
% find this number locating your abstract in the printed proceedings
% that you received at the meeting or on-line via 
% http://adass2002.stsci.edu; the ID code is the letter/number sequence 
% proceeding the title of your presentation.  
%
% This will not appear in your paper; however, it allows different
% papers in the proceedings to cross-reference each other.
%
% EXAMPLE: \paperID{O4.1}
% EXAMPLE: \paperID{P7.7}
%
% Note that you should only have one \paperID, and it should not
% include a trailing period.  

\paperID{P8.2}

%-----------------------------------------------------------------------
%		            Paper Title 
%-----------------------------------------------------------------------
% Enter the title of the paper.
%
% EXAMPLE: \title{A Breakthrough in Astronomical Software Development}
% 

\title{Source Code Management and Software Distribution using Open Source Technologies}

%-----------------------------------------------------------------------
%		          Authors of Paper
%-----------------------------------------------------------------------
% Enter the authors followed by their affiliations.  The \author and
% \affil commands may appear multiple times as necessary (see example
% below).  List each author by giving the first name or initials first
% followed by the last name.  Authors with the same affiliations
% should grouped together. 
%
% EXAMPLE: \author{Raymond Plante, Doug Roberts, 
%                  R.\ M.\ Crutcher\altaffilmark{1}}
%          \affil{National Center for Supercomputing Applications, 
%                 University of Illinois Urbana-Champaign, Urbana, IL
%                 61801}
%          \author{Tom Troland}
%          \affil{University of Kentucky}
%
%          \altaffiltext{1}{Astronomy Department, UIUC}
%
% In this example, the first three authors, "Plante", "Roberts", and
% "Crutcher" are affiliated with "NCSA".  "Crutcher" has an alternate 
% affiliation with the "Astronomy Department".  The fourth author,
% "Troland", is affiliated with "University of Kentucky"

\author{Martin Bly}
\affil{Rutherford Appleton Laboratory, Chilton, Didcot, Oxfordshire  OX11 0QX, United Kingdom}
\author{Alasdair Allan}
\affil{School of Physics, University of Exeter, Stocker Road, Exeter  EX4 4QL, United Kingdom }
\author{Tim Jenness}
\affil{Joint Astronomy Centre, 660 N. A`oh\=ok\=u Place, University Park, Hilo, HI 96720}

%-----------------------------------------------------------------------
%			 Contact Information
%-----------------------------------------------------------------------
% This information will not appear in the paper but will be used by
% the editors in case you need to be contacted concerning your
% submission.  Enter your name as the contact along with your email
% address.
% 
% EXAMPLE:  \contact{Dennis Crabtree}
%           \email{crabtree@cfht.hawaii.edu}
%

\contact{Martin Bly}
\email{bly@star.rl.ac.uk}

%-----------------------------------------------------------------------
%		      Author Index Specification
%-----------------------------------------------------------------------
% Specify how each author name should appear in the author index.  The 
% \paindex{ } should be used to indicate the primary author, and the
% \aindex for all other co-authors.  You MUST use the following
% syntax: 
%
% SYNTAX:  \aindex{LASTNAME, F. M.}
% 
% where F is the first initial and M is the second initial (if
% used).  This guarantees that authors that appear in multiple papers
% will appear only once in the author index.  
%
% EXAMPLE: \paindex{Crabtree, D.}
%          \aindex{Manset, N.}        
%          \aindex{Veillet, C.}        

\paindex{Bly, M.J.}
\aindex{Allan, A.}
\aindex{Jenness, T.}

%-----------------------------------------------------------------------
%			Subject Index keywords
%-----------------------------------------------------------------------
% Enter up to 6 keywords describing your paper.  These will NOT be
% printed as part of your paper; however, they will be used to
% generate the subject index for the proceedings.  There is no
% standard list; however, you can consult the indices for past ADASS
% proceedings (http://iraf.noao.edu/ADASS/adass.html). 
%
% EXAMPLE:  \keywords{visualization, astronomy: radio, parallel
%                     computing, AIPS++, Galactic Center}
%
% In this example, the author noticed that "radio astronomy" appeared
% in the ADASS VII Index as "astronomy" being the major keyword and
% "radio" as the minor keyword.

\keywords{management: code, starlink, RPM}

%-----------------------------------------------------------------------
%			       Abstract
%-----------------------------------------------------------------------
% Type abstract in the space below.  Consult the User Guide and Latex
% Information file for a list of supported macros (e.g. for typesetting 
% special symbols). Do not leave a blank line between \begin{abstract} 
% and the start of your text.

\begin{abstract}          % Leave intact

The Starlink Software Collection (USSC) runs on three different
platforms and contains approximately 130 separate software items,
totalling over 6 million lines of code. Distribution of such large
software systems and installation at multiple remote sites has always
been problematic due to the complex web of inter-dependencies such
systems invariably generate.

The rise of the Open Source movement has brought standard tools into
common use to cope with such large and complex tasks.  The RedHat
Package Manager (RPM) software is one such which is available for many
platforms.  We have shown it is possible to automate the distribution
and installation of the Starlink Software using RPM. We anticipate that
this will vastly simplify installation and package management for
Systems Administrators who must support the USSC in production data
processing environments.

\end{abstract}

%-----------------------------------------------------------------------
%			      Main Body
%-----------------------------------------------------------------------
% Place the text for the main body of the paper here.  You should use
% the \section command to label the various sections; use of
% \subsection is optional.  Significant words in section titles should
% be capitalized.  Sections and subsections will be numbered
% automatically. 
%
% EXAMPLE:  \section{Introduction}
%           ...
%           \subsection{Our View of the World}
%           ...
%           \section{A New Approach}
%
% It is recommended that you look at the sample papers, sample1.tex
% and sample2.tex, for examples for formatting references, footnotes,
% figures, equations, html links, lists, and other special features.  

\section{Introduction}

The \htmladdnormallinkfoot{Starlink Software
Collection}{http://www.starlink.ac.uk/} (Bly, Giaretta \& Taylor 2003)
is a large collection of
software packages comprising subroutine libraries, applications
packages and utilities for astronomical data reduction and analysis.
The whole collection is governed by a set of interdependencies which
not only complicates the process of building the software but also
which packages depend on which others at runtime.

Each package has its own {\tt{makefile}} which contains the rules
defining how and in what order the components should be built, and
their dependencies on other packages.  However, none of the packages
are able to trigger the building of another package --- this is
traditionally done by a master {\tt{makefile}} which builds the
packages in the correct order.

The master {\tt{makefile}} is maintained by the Software Manager but
has two disadvantages --- it does not express the package dependencies at
run-time, and contains only a simple ordered-list of packages to build
--- making it difficult to slot a new package into a suitable place in
the build sequence.  A new approach to managing the build and
installation dependencies was needed.  We have investigated `wrapping'
the USSC using the RedHat Package Manager (RPM) system (Bailey, 1997) and have shown
it is possible to automate the building and distribution and simplify
the installation and maintenance of the USSC for Systems Administrators
and users.

\section{Why Choose the RPM System}

The \htmladdnormallinkfoot{RedHat Package Manager}{http://www.rpm.org/}
is becoming ubiquitous in the Linux world, having been adopted by most
of the major distributions.  It is also supported on may other Unix
systems including those supported by Starlink.  The RPM system has
several advantageous features:

\begin{description}

\item[Tracking mechanism] -- it keeps track of installed packages and
package version numbers, and all files associated with each package.

\item[Dependency and dependents checking] -- RPM checks dependencies of
the packages it is processing and warns of conflicts and unfulfilled
dependencies, and checks for packages that are dependent on those
being processed.

\item[Query capabilities] -- it has a full suite of query capabilities
that provide information about the package, its dependencies and status.

\item[Relocation] -- RPM can install packages in locations other than the
one they were intended to go (provided the set is re-locatable).

\item[Adaptability] -- the RPM system works with existing build systems,
and can easily be used to provide a wrapping for existing package build
systems, from the simple to the most complex of systems.

\item[Ease of use] -- installation of patches and updates can be automated.

\item[Open-Source] -- the RPM package is open-source and runs on many
operating systems.

\end{description}


\section{Wrapping the Starlink Packages}

Each Starlink package has its own makefile and documents which conform
to a standard template, although the makefile and documents may have
slight variations from the standard.  This makes them suitable for
processing to provide RPM with the information it requires.

RPM requires a `spec' file for a package, to define the various
dependencies.  These can be generated by hand or automatically by
processing a master dependency list.    A template dependency file
{\tt{depend.ini}} was created by hand listing the software group,
version, and list of SUNs (Starlink User Notes --- the documents) and
then a Perl script is used to extract summary and abstract information
from the package documents.  The {\bf{buildrequires}} and {\bf{fixup}}
lines are then added by hand examination of the {\tt{makefile}} to see
which files are edited and in what way at installation time.
{\bf{Buildrequires}} expresses the additional packages required at
build time and {\bf{fixup}} expresses files that have to be changed at
installation.

\begin{figure}
\begin{verbatim}
[ast]
group=Starlink/Libraries 
version=1.5.8 
suns=sun210,sun211 
requires=htx
buildrequires=sla,ems,chr,sae 
fixup=STARBIN/ast_dev 
summary=AST - A Library for Handling World Coordinate\
Systems in Astronomy 
abstract=The AST library provides a comprehensive range\
of facilities for attaching world coordinate systems to\
astronomical data, for retrieving and interpreting that\
information and for generating graphical output based on it.
\end{verbatim}
\caption{A typical {\tt{depend.ini}} dependency file expressing 
build and installation dependencies.}
\end{figure}

A dependency file may contain details of more than one package --- the
abstract extraction script {\tt{getabst.pl}} processes all the package
entries.  Where the document listed in the {\tt{depend.ini}} file does not have
the appropriate \LaTeX\ keywords, the keys are left blank.  Since this is 
demonstrating the concept, the system depends on an existing Starlink
installation with source files from which it extracts its data.

Once a dependency file is ready, the RPM `spec' file(s) can be
generated. This is the file that controls what RPM does when building
and manipulating the package. A Perl script {\tt{mkspec.pl}} has been
produced to interpret the dependency file and generate `spec' files
based on a template for all packages listed in the dependency file.

Existing Starlink installations do not have the source packaged in a
single tarball though the makefiles can provide them via the 
{\tt{export\_source}} target which generates a compressed tarball of the
appropriate files.  The master {\tt{makefile}} can generate 
tarballs for all the packages.  

The next step is to create a set of links for each package from an
existing USSC installation to the standard location for RPM build
directories.  A Perl script {\tt{mklinks.pl}} does this for all the
packages in the dependency file, creating links from
{\tt{/usr/src/redhat/SOURCES}} to the source files.  Since the default
location is {\tt{/usr}}, one has to be {\tt{root}} for this and the remaining
steps.  Once the links are in place, the `spec' files are copied to the
{\tt{/usr/src/redhat/SPECS}} directory and the {\tt{rpm}} program can
take over to build the RPMs, both source and installation sets.

% \begin{figure}
% \begin{verbatim}
% # cd /usr/src/redhat/SOURCES/ 
% # cp /work/rpm/htx-1.2.3.spec ../SPECS 
% # rpm -ba ../SPECS/htx-1.2.3.spec
% Executing(%prep): /bin/sh -e /var/tmp/rpm-tmp.1413
%  + umask 022
%  + cd /usr/src/redhat/BUILD
%  + cd /usr/src/redhat/BUILD
%  + rm -rf htx-1.2.3
%  + /bin/mkdir -p htx-1.2.3
% ...
%  + rm -rf /var/tmp/starlink/htx-root
%  + exit 0 
% # ls RPMS/i386 SRPMS 
% htx-1.2.3-1.i386.rpm 
% htx-1.2.3-1.src.rpm
% \end{verbatim}
% \caption{Copying the spec file and building the RPMs}
% \end{figure}

\section{Results}

The resulting RPM files are re-locatable so you don't have to install
the packages in the default location ({\tt{/star}}).  Using the
{\tt{--relocate}} command line switch one can instead direct the
package to any chosen path --- since the normal location is
{\tt{/star}}, one has to be careful where there is an existing
(non-RPM) USSC installation. Most packages are easily re-locatable and
do not require special tricks to be detailed in the {\tt{depend.ini}} file.
However, some packages do have complex installation requirements and
these need to be carefully expressed.  The whole collection can be
processed to build RPM sets and installations made based on them.

\subsection{Problems}
RPM itself requires access to the standard RPM database of dependencies
needed to track all the files.  This is owned by {\tt{root}} so general
users cannot create and install packages using the standard RPM
distribution. At the time this work was undertaken, the facilities in
RPM to allow alternative databases didn't work, however a distribution
that can use a database elsewhere should be possible.

This work is proof of concept. The technology for extracting
dependencies and details from the existing build systems is rather
basic, and some Starlink packages are under regular development which
tends to change the dependencies. Stable packages such as libraries are
easier to deal with.  Nevertheless we have demonstrated that RPM can be
adapted to deal with an alien software management system.


%-----------------------------------------------------------------------
%			      References
%-----------------------------------------------------------------------
% List your references below within the reference environment
% (i.e. between the \begin{references} and \end{references} tags).
% Each new reference should begin with a \reference command which sets
% up the proper indentation.  Observe the following order when listing
% bibliographical information for each reference:  author name(s),
% publication year, journal name, volume, and page number for
% articles.  Note that many journal names are available as macros; see
% the User Guide listing "macro-ized" journals.   
%
% EXAMPLE:  \reference Hagiwara, K., \& Zeppenfeld, D.\  1986, 
%                Nucl.Phys., 274, 1
%           \reference H\'enon, M.\  1961, Ann.d'Ap., 24, 369
%           \reference King, I.\ R.\  1966, \aj, 71, 276
%           \reference King, I.\ R.\  1975, in Dynamics of Stellar 
%                Systems, ed.\ A.\ Hayli (Dordrecht: Reidel), 99
%           \reference Tody, D.\  1998, \adassvii, 146
%           \reference Zacharias, N.\ \& Zacharias, M.\ 2003,
%                \adassxii, \paperref{P7.6}
% 
% Note the following tricks used in the example above:
%
%   o  \& is used to format an ampersand symbol (&).
%   o  \'e puts an accent agu over the letter e.  See the User Guide
%      and the sample files for details on formatting special
%      characters.  
%   o  "\ " after a period prevents LaTeX from interpreting the period 
%      as an end of a sentence.
%   o  \aj is a macro that expands to "Astron. J."  See the User Guide
%      for a full list of journal macros
%   o  \adassvii is a macro that expands to the full title, editor,
%      and publishing information for the ADASS VII conference
%      proceedings.  Such macros are defined for ADASS conferences I
%      through XI.
%   o  When referencing a paper in the current volume, use the
%      \adassxii and \paperref macros.  The argument to \paperref is
%      the paper ID code for the paper you are referencing.  See the 
%      note in the "Paper ID Code" section above for details on how to 
%      determine the paper ID code for the paper you reference.  
%
\begin{references}

\reference Bailey, E., 1997, `Maximum RPM', Sams, ISBN: 0672311054
\reference Bly, M.\ J., Giaretta, D.\ L.\ \& Taylor, M.\ B., \adassxii, \paperref{D1}

\end{references}

% Do not place any material after the references section

\end{document}  % Leave intact
