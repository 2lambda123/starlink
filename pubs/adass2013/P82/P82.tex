% This is the aspauthor.tex LaTeX file
% Copyright 2010, Astronomical Society of the Pacific Conference Series

\documentclass[11pt,twoside]{article}
\usepackage{asp2010}

\resetcounters

\bibliographystyle{asp2010}

\markboth{Currie et al.}{Starlink in 2013}

\newcommand{\aspconf}{ASP Conf.\ Ser.}

\begin{document}

\title{Starlink Software in 2013}
\author{Malcolm~J.~Currie$^1$, David~S.~Berry$^1$, Tim~Jenness$^2$,
Andy~G.~Gibb$^3$, Graham~S.~Bell$^1$, and Peter~W.~Draper$^5$
\affil{$^1$Joint Astronomy Centre, 660 N.\ A`oh\=ok\=u Place, Hilo, HI
96720, USA}
\affil{$^2$Department of Astronomy, Cornell University, Ithaca, NY
  14853, USA}
\affil{$^3$Department of Physics and Astronomy, University of British
  Columbia, 6224 Agricultural Road, Vancouver, BC V6T~1Z1, Canada}
\affil{$^4$Department of Physics, University of Durham, South Road,
  Durham, DH1~3LE, UK}
}

\begin{abstract}
  Although the Starlink Project was closed in 2005, Starlink software
  continues to be developed and supported by the Joint Astronomy
  Centre for its data-reduction requirements. This paper summarises
  new features added since the \textit{Kapuahi} release in 2012, with
  emphasis on the reduction of sub-millimetre data, illustrated with
  SCUBA-2 continuum images and ACSIS/HARP spectral cubes of improved
  fidelity.
\end{abstract}

\section{Introduction}

The Starlink Software Collection is an open-source software project
hosted on github\footnote{\url{https://github.com/Starlink}} that has
been in constant development since the early 1980s
\citep{1982QJRAS..23..485D}.

\section{Hikianalia Release}

The \textit{Hikianalia} release was made in April 2013. The key
changes relative to the previous \textit{Kapuahi} release
\citep{P05_adassxxii} are detailed below.

\subsection*{64-bit Integer Support}

Data sets have continued to grow in size and the need for 64-bit
integers has become more important. The \textit{Herschel} telescope
generates data products with 64-bit integers and people attempted to
convert these files into Starlink NDF format \citep[see][for an
overview of NDF]{P91_adassxxiii}, with little success. Historically we
had been reticent regarding the addition of support for 64-bit
integers given varied Fortran compiler support for the
\texttt{INTEGER*8} data type and requiring C99 compilers for a
guaranteed 64-bit data type in C (\texttt{int64\_t}). With modern
compilers this is no longer an issue and type \texttt{\_INT64} was
added to the Starlink file format (HDS) and the application code
dealing with multiple types was consistently modified to use code
automatically generated by the \textsc{Generic} application. This
required that many generic source files were re-generated from the
derived files which had been committed during the switch to a unified
revision control system back in 2005 \citep{2005ASPC..347..119G}.

\subsection*{Application Updates}

Applications such as \textsc{Convert}, \textsc{Ccdpack} and
\textsc{KAPPA} all automatically received the ability to process data
with 64-bit integers.

\altsubsubsection*{GAIA}

A new toolbox was added for overlaying STC-S regions on an image.

\altsubsubsection*{KAPPA}

Commands were added to report values of configuration parameters from
a text file (using the standard configuration format supported by the
\textsc{Grp} library) or NDF file history, and also to expand file lists.

Features were added to the commands \textsc{chanmap},
\textsc{collapse}, \textsc{mstats}, \textsc{normalize},
\textsc{parget}, \textsc{rotate}, \textsc{wcsadd}, and
\textsc{wcsremove}.

\altsubsubsection*{SMURF}

The main focus of the \textit{Hikianalia} release was to support the
reduction of data from SCUBA-2
\citep{2013MNRAS.430.2545C,2013MNRAS.430.2513H}. In particular a new
approach to map-making was developed whereby all data are involved in
each iteration rather than iteratively processing small chunks and
coadding them all at the end.

\altsubsubsection*{POLPACK}

A new command \textsc{polrotref} has been added to rotate the
reference direction of a pair of Q and U images.

\altsubsubsection*{SPLAT}

Much improved support for more complex SSAP queries \citep[see][for a
protocol introduction]{2004SPIE.5493..262D}.

\subsection*{Library Updates}

\altsubsubsection*{AST}

The AST library \citep[see][and references
therein]{2012ASPC..461..825B} can now read FITS headers that use the
SAO convention for repesenting a distorted TAN projection. \textit{is
  there a refernece for this convention?} and can now handle missing
\textsc{cnpix1} and \textsc{cnpix2} keywords in DSS headers. There
were also modifications to flux conservation and normalisation in the
rebinning routines.

\altsubsubsection*{NDF}

The NDF library will now limit the maximum size of an NDF
section. There have also been efficiency improvements in the way that
provenance information is stored.

\bibliography{P82}

\end{document}
