\documentstyle[11pt]{article}
\pagestyle{myheadings}

% -----------------------------------------------------------------------------
% ? Document identification
\newcommand{\stardoccategory}  {Starlink User Note}
\newcommand{\stardocinitials}  {SUN}
\newcommand{\stardocnumber}    {188.4}
\newcommand{\stardocsource}    {sun188.4}
\newcommand{\stardocauthors}   {R.F.~Warren-Smith \& P.W~Draper}
\newcommand{\stardocdate}      {9th May 2006}
\newcommand{\stardoctitle}     {HTX\\[1ex]
                                Hypertext Cross-Reference Utilities}
\newcommand{\stardocversion}   {Version 1.3}
\newcommand{\stardocmanual}    {User's Manual}
% ? End of document identification
% -----------------------------------------------------------------------------

\newcommand{\stardocname}{\stardocinitials /\stardocnumber}
\markright{\stardocname}
\setlength{\textwidth}{160mm}
\setlength{\textheight}{230mm}
\setlength{\topmargin}{-2mm}
\setlength{\oddsidemargin}{0mm}
\setlength{\evensidemargin}{0mm}
\setlength{\parindent}{0mm}
\setlength{\parskip}{\medskipamount}
\setlength{\unitlength}{1mm}

% -----------------------------------------------------------------------------
%  Hypertext definitions.
%  ======================
%  These are used by the LaTeX2HTML translator in conjunction with star2html.

%  Comment.sty: version 2.0, 19 June 1992
%  Selectively in/exclude pieces of text.
%
%  Author
%    Victor Eijkhout                                      <eijkhout@cs.utk.edu>
%    Department of Computer Science
%    University Tennessee at Knoxville
%    104 Ayres Hall
%    Knoxville, TN 37996
%    USA

%  Do not remove the %\begin{rawtex} and %\end{rawtex} lines (used by 
%  star2html to signify raw TeX that latex2html cannot process).
%\begin{rawtex}
\makeatletter
\def\makeinnocent#1{\catcode`#1=12 }
\def\csarg#1#2{\expandafter#1\csname#2\endcsname}

\def\ThrowAwayComment#1{\begingroup
    \def\CurrentComment{#1}%
    \let\do\makeinnocent \dospecials
    \makeinnocent\^^L% and whatever other special cases
    \endlinechar`\^^M \catcode`\^^M=12 \xComment}
{\catcode`\^^M=12 \endlinechar=-1 %
 \gdef\xComment#1^^M{\def\test{#1}
      \csarg\ifx{PlainEnd\CurrentComment Test}\test
          \let\html@next\endgroup
      \else \csarg\ifx{LaLaEnd\CurrentComment Test}\test
            \edef\html@next{\endgroup\noexpand\end{\CurrentComment}}
      \else \let\html@next\xComment
      \fi \fi \html@next}
}
\makeatother

\def\includecomment
 #1{\expandafter\def\csname#1\endcsname{}%
    \expandafter\def\csname end#1\endcsname{}}
\def\excludecomment
 #1{\expandafter\def\csname#1\endcsname{\ThrowAwayComment{#1}}%
    {\escapechar=-1\relax
     \csarg\xdef{PlainEnd#1Test}{\string\\end#1}%
     \csarg\xdef{LaLaEnd#1Test}{\string\\end\string\{#1\string\}}%
    }}

%  Define environments that ignore their contents.
\excludecomment{comment}
\excludecomment{rawhtml}
\excludecomment{htmlonly}
%\end{rawtex}

%  Hypertext commands etc. This is a condensed version of the html.sty
%  file supplied with LaTeX2HTML by: Nikos Drakos <nikos@cbl.leeds.ac.uk> &
%  Jelle van Zeijl <jvzeijl@isou17.estec.esa.nl>. The LaTeX2HTML documentation
%  should be consulted about all commands (and the environments defined above)
%  except \xref and \xlabel which are Starlink specific.

\newcommand{\htmladdnormallinkfoot}[2]{#1\footnote{#2}}
\newcommand{\htmladdnormallink}[2]{#1}
\newcommand{\htmladdimg}[1]{}
\newenvironment{latexonly}{}{}
\newcommand{\hyperref}[4]{#2\ref{#4}#3}
\newcommand{\htmlref}[2]{#1}
\newcommand{\htmlimage}[1]{}
\newcommand{\htmladdtonavigation}[1]{}

%  Starlink cross-references and labels.
\newcommand{\xref}[3]{#1}
\newcommand{\xlabel}[1]{}

%  LaTeX2HTML symbol.
\newcommand{\latextohtml}{{\bf LaTeX}{2}{\tt{HTML}}}

%  Define command to re-centre underscore for Latex and leave as normal
%  for HTML (severe problems with \_ in tabbing environments and \_\_
%  generally otherwise).
\newcommand{\latex}[1]{#1}
\newcommand{\setunderscore}{\renewcommand{\_}{{\tt\symbol{95}}}}
\latex{\setunderscore}

%  Redefine the \tableofcontents command. This procrastination is necessary 
%  to stop the automatic creation of a second table of contents page
%  by latex2html.
\newcommand{\latexonlytoc}[0]{\tableofcontents}

% -----------------------------------------------------------------------------
%  Debugging.
%  =========
%  Remove % on the following to debug links in the HTML version using Latex.

% \newcommand{\hotlink}[2]{\fbox{\begin{tabular}[t]{@{}c@{}}#1\\\hline{\footnotesize #2}\end{tabular}}}
% \renewcommand{\htmladdnormallinkfoot}[2]{\hotlink{#1}{#2}}
% \renewcommand{\htmladdnormallink}[2]{\hotlink{#1}{#2}}
% \renewcommand{\hyperref}[4]{\hotlink{#1}{\S\ref{#4}}}
% \renewcommand{\htmlref}[2]{\hotlink{#1}{\S\ref{#2}}}
% \renewcommand{\xref}[3]{\hotlink{#1}{#2 -- #3}}
% -----------------------------------------------------------------------------
% ? Document specific \newcommand or \newenvironment commands.

\newcommand{\cgiurl}{http://hoohoo.ncsa.uiuc.edu/cgi/overview.html}
\newcommand{\formsurl}{http://www.ncsa.uiuc.edu/SDG/Software/Mosaic/Docs/fill-out-forms/overview.html}
\newcommand{\HTMLurl}{http://www.w3.org/hypertext/WWW/MarkUp/MarkUp.html}
\newcommand{\latextohtmlurl}{http://cbl.leeds.ac.uk/nikos/tex2html/doc/latex2html/latex2html.html}
\newcommand{\mosaicurl}{http://www.ncsa.uiuc.edu/SDG/Software/Mosaic/NCSAMosaicHome.html}
\newcommand{\netscapeurl}{http://home.netscape.com/}
\newcommand{\psurl}{http://www.cs.indiana.edu/docproject/programming/postscript/postscript.html}
\newcommand{\urlurl}{http://www.w3.org/hypertext/WWW/Addressing/URL/Overview.html}
\newcommand{\wwwurl}{http://www.w3.org/hypertext/WWW/}

\newcommand{\cgiref}[1]{\htmladdnormallink{#1}{\cgiurl}}
\newcommand{\HTMLref}[1]{\htmladdnormallink{#1}{\HTMLurl}}
\newcommand{\formsref}[1]{\htmladdnormallink{#1}{\formsurl}}
\newcommand{\latextohtmlref}[1]{\htmladdnormallink{#1}{\latextohtmlurl}}
\newcommand{\mosaicref}[1]{\htmladdnormallink{#1}{\mosaicurl}}
\newcommand{\netscaperef}[1]{\htmladdnormallink{#1}{\netscapeurl}}
\newcommand{\psref}[1]{\htmladdnormallink{#1}{\psurl}}
\newcommand{\urlref}[1]{\htmladdnormallink{#1}{\urlurl}}
\newcommand{\wwwref}[1]{\htmladdnormallink{#1}{\wwwurl}}

\newcommand{\HTXref}[1]{\htmlref{#1}{sect:HTX}}
\newcommand{\latexref}[1]{\xref{#1}{sun9}{}}
\newcommand{\linkerref}[1]{\htmlref{#1}{hlink}}
\newcommand{\indexref}[1]{\htmlref{#1}{sect:indexfiles}}
\newcommand{\docref}[1]{\htmlref{#1}{sect:hypertextdocuments}}
\newcommand{\dirref}[1]{\htmlref{#1}{sect:documentdirectories}}
\newcommand{\libref}[1]{\htmlref{#1}{sect:documentlibraries}}
\newcommand{\labref}[1]{\htmlref{#1}{sect:targetlabels}}
\newcommand{\remref}[1]{\htmlref{#1}{sect:remotedocuments}}
\newcommand{\servref}[1]{\htmlref{#1}{sect:remoteservers}}
\newcommand{\findmeref}[1]{\htmlref{#1}{findme}}
\newcommand{\showmeref}[1]{\htmlref{#1}{showme}}

\newcommand{\hi}[1]{{\tt{#1}}}
\newcommand{\qt}[1]{``#1''}
\newcommand{\st}[1]{{\em{#1}}}
\begin{htmlonly}
   \renewcommand{\qt}[1]{{\tt{"}}#1{\tt{"}}}
%   \renewcommand{\_}{\_}
\end{htmlonly}

% ? End of document specific commands
% -----------------------------------------------------------------------------
%  Title Page.
%  ===========
\renewcommand{\thepage}{\roman{page}}
\begin{document}
\thispagestyle{empty}

%  Latex document header.
%  ======================
\begin{latexonly}
   CCLRC / {\sc Rutherford Appleton Laboratory} \hfill {\bf \stardocname}\\
   {\large Particle Physics \& Astronomy Research Council}\\
   {\large Starlink Project\\}
   {\large \stardoccategory\ \stardocnumber}
   \begin{flushright}
   \stardocauthors\\
   \stardocdate
   \end{flushright}
   \vspace{-4mm}
   \rule{\textwidth}{0.5mm}
   \vspace{5mm}
   \begin{center}
   {\Huge\bf  \stardoctitle \\ [2.5ex]}
   {\LARGE\bf \stardocversion \\ [4ex]}
   {\Huge\bf  \stardocmanual}
   \end{center}
   \vspace{5mm}

% ? Heading for abstract if used.
   \vspace{10mm}
   \begin{center}
      {\Large\bf Abstract}
   \end{center}
% ? End of heading for abstract.
\end{latexonly}

%  HTML documentation header.
%  ==========================
\begin{htmlonly}
   \xlabel{}
   \begin{rawhtml} <H1> \end{rawhtml}
      \stardoctitle\\
      \stardocversion\\
      \stardocmanual
   \begin{rawhtml} </H1> \end{rawhtml}

% ? Add picture here if required.
% ? End of picture

   \begin{rawhtml} <P> <I> \end{rawhtml}
   \stardoccategory \stardocnumber \\
   \stardocauthors \\
   \stardocdate
   \begin{rawhtml} </I> </P> <H3> \end{rawhtml}
      \htmladdnormallink{CCLRC}{http://www.cclrc.ac.uk} /
      \htmladdnormallink{Rutherford Appleton Laboratory}
                        {http://www.cclrc.ac.uk/ral} \\
      \htmladdnormallink{Particle Physics \& Astronomy Research Council}
                        {http://www.pparc.ac.uk} \\
   \begin{rawhtml} </H3> <H2> \end{rawhtml}
      \htmladdnormallink{Starlink Project}{http://star-www.rl.ac.uk/}
   \begin{rawhtml} </H2> \end{rawhtml}
   \htmladdnormallink{\htmladdimg{source.gif} Retrieve hardcopy}
      {http://star-www.rl.ac.uk/cgi-bin/hcserver?\stardocsource}\\

%  HTML document table of contents. 
%  ================================
%  Add table of contents header and a navigation button to return to this 
%  point in the document (this should always go before the abstract \section). 
  \label{stardoccontents}
  \begin{rawhtml} 
    <HR>
    <H2>Contents</H2>
  \end{rawhtml}
  \renewcommand{\latexonlytoc}[0]{}
  \htmladdtonavigation{\htmlref{\htmladdimg{contents_motif.gif}}
        {stardoccontents}}

% ? New section for abstract if used.
  \section{\xlabel{abstract}Abstract}
% ? End of new section for abstract
\end{htmlonly}

% -----------------------------------------------------------------------------
% ? Document Abstract. (if used)
%   ==================
This document describes a set of ``Hypertext Cross-Reference
Utilities'' (HTX) which are designed to help maintain large
documentation sets whose constituent \docref{documents} are written
using the \HTMLref{\st{Hypertext Markup Language} (HTML)}.

The central part of HTX is a \st{hypertext linker},
\linkerref{\hi{hlink}}. This allows hyper-links (or cross-references)
to be established between related documents in such a way that it is
easy to maintain their integrity as individual documents are updated.
Information produced by this linking process is also used by other HTX
utilities to provide \htmlref{document
search}{sect:searchingfordocuments} facilities and the ability to
\htmlref{randomly access}{sect:accessingdocuments} any part of a
documentation set. This latter capability forms a basis for
constructing \htmlref{hypertext help}{sect:hypertexthelp} systems for
use by other software.

The expected readership of this document includes those who read
hypertext documentation, those who write it, and those who maintain
it, especially those who \xref{write and maintain Starlink
documentation}{sgp28}{}. Software developers may also be interested in
the possibilities for hypertext help that HTX provides.


% ? End of document abstract
% -----------------------------------------------------------------------------
% ? Latex document Table of Contents (if used).
%  ===========================================
\newpage
\begin{latexonly}
   \newpage
   \markright{\stardocname}
   \null\vspace{5mm}
   \begin {center}
   \rule{80mm}{0.5mm} \\ [1ex]
   {\Large\bf \stardoctitle \\ [2.5ex]
              \stardocversion} \\ [2ex]
   \rule{80mm}{0.5mm}
   \end{center}
   \setlength{\parskip}{0mm}
   \latexonlytoc
   \setlength{\parskip}{\medskipamount}
   \markright{\stardocname}
\end{latexonly}
% ? End of Latex document table of contents
% -----------------------------------------------------------------------------
%  Introduction page.
%  =================
\newpage
\renewcommand{\thepage}{\arabic{page}}
\setcounter{page}{1}
\begin{latexonly}
   \newpage
   \markright{\stardocname}
   \renewcommand{\thepage}{\arabic{page}}
   \setcounter{page}{1}
   \null\vspace{5mm}
   \begin{center}
   \rule{80mm}{0.5mm}\\[1ex]
   {\Large\bf \stardoctitle\\[2.5ex]
              \stardocversion}\\[2ex]
   \rule{80mm}{0.5mm}
   \end{center}
   \vspace{30mm}
\end{latexonly}

\section{\xlabel{introduction}INTRODUCTION}

\subsection{\label{sect:HTX}\xlabel{what_does_HTX_do}What Does HTX Do?}

HTX is a set of utilities that allows you to maintain and access a
collection of dynamic \docref{multi-page hypertext documents} that
refer to each other. Its main element is a \linkerref{\st{hypertext
linker}} which can be used to establish cross-references between
documents and to re-establish these whenever changes occur to
individual documents.

The information generated by the \htmlref{linking
process}{sect:howlinkingworks} is used by other HTX utilities to
provide access to hypertext documentation, permitting
\htmlref{document searches}{sect:searchingfordocuments} and the rapid
\htmlref{display of selected parts}{sect:accessingdocuments} of
documents.

Other software can also make use of these facilities to obtain random
access to any part of a cross-linked documentation set, forming a
basis for on-line or context-sensitive \htmlref{hypertext
help}{sect:hypertexthelp} systems.

See \S\ref{sect:commanddescriptions} for descriptions of each of the
utilities that HTX provides.

\subsection{\label{sect:hypertextdocuments}\xlabel{what_is_a_hypertext_document}What is a Hypertext Document?}

The documents that \HTXref{HTX} is concerned with are written using
the \htmladdnormallinkfoot{\st{Hypertext Markup Language}
(HTML)}{\HTMLurl} which can be read using \wwwref{World Wide Web (WWW)}
browsers such as \mosaicref{\st{Mosaic}} and \netscaperef{\st{Netscape}}.

Of course, any page of hypertext written using HTML could be
considered a document in its own right. However, most documents of
substance will consist of multiple pages of HTML, joined together
using hypertext links. It is convenient to \dirref{gather these
related pages together} and to consider them as a single multi-page
hypertext document. This is the sense in which the term \st{document}
is used here.

\subsection{\xlabel{why_are_cross_references_a_problem}Why are Cross-References a Problem?}

The hyper-links that join different \HTMLref{HTML} pages {\bf within}
\docref{hypertext documents} are normally defined by the document's
author.  These are an internal matter -- if the document structure
changes, the author can change the internal links accordingly.  If
proper use is made of relative file names in establishing these links,
it should also be possible to transport these documents as a whole to
a different location without damaging the internal links.

\HTXref{HTX} addresses the different problem of maintaining a
collection of dynamic hypertext documents (probably written by many
different authors) that have cross-references established between
them. That is, each document may contain hyper-links to the contents
of any other document in the collection.

In this situation, document locations may often need to change, and
revision of individual documents will also result in the referenced
material moving around within them.  Since the links that implement
HTML cross-references consist, essentially, of file names, they cannot
survive this process and will rapidly end up pointing to the wrong
places unless remedial action is taken.

In a large document collection, re-organising all necessary
cross-references following every document change is not a task to
contemplate doing by hand. Establishing and repairing these links is
therefore a job for the HTX \linkerref{hypertext linker}.

\subsection{\xlabel{mixing_local_and_remote_documents}Mixing Local and Remote Documents}

The inherent fragility of hypertext links is a problem faced by any
project that needs to maintain a collection of related
\docref{documents} in hypertext form. The documentation for software
projects typifies this.  However, software documentation also suffers
from an additional problem because it is usually distributed and
installed along with the particular items of software it
describes. This means that any sub-set of the complete documentation
set must also be viable in its own right.

To give an example, a particular site may only have installed one or
two items out of a complete collection of software, and so may only
have a couple of documents that relate to the installed software
available locally.  This means that many of the cross-references that
these documents make will be to other documents that do not exist on
the local system. Ideally, these links should not simply fail.

The problem can be solved by using a \servref{document server} to
provide access to the missing documents from a central archive, where
a copy of all the documents is maintained. A reader with any of the
documents installed locally will then potentially have easy access to
the entire documentation set, albeit with some time penalty when
accessing non-local documents. One of the functions that \HTXref{HTX}
performs, therefore, is to generate appropriate requests for
\remref{\st{remote documents}} for processing by a \htmlref{central document
server}{sect:remoteservers}\begin{latexonly}
(see \S\ref{sect:remotedocuments} and
\S\ref{sect:remoteservers})\end{latexonly}.

\section{\xlabel{document_organisation}DOCUMENT ORGANISATION}

This section describes how you should organise your hypertext
documentation to make use of \HTXref{HTX} to maintain and use
cross-references between \docref{documents}.

\subsection{\label{sect:documentdirectories}\xlabel{document_directories}Document Directories}

\HTXref{HTX} assumes that each \docref{hypertext document} resides
beneath a top-level directory and it recognises these \st{document
directories} from their extension, which should always be \qt{.htx}.
When it needs to locate the contents of a document, it will search
inside document directories (to any required depth) looking for files
with an extension of \qt{.html} which contain the hypertext itself, in
\HTMLref{HTML} format.

The first step in organising your hypertext documentation should
therefore be to collect all the HTML files comprising each document
together into a directory (or directory tree) and to give the (top
level) directory a distinctive name such as:

\begin{quote}
\begin{verbatim}
design-study.htx
\end{verbatim}
\end{quote}

The \qt{design-study} part of this is the document name by which it
will be known to other authors, and the \qt{.htx} part identifies it to
HTX as a candidate for the various operations it performs. Once you
have made a document available to other people, you should not
subsequently change its name, even if you move it somewhere else,
because other documents may be referring to it and will still know it
by its original name.

\subsection{\label{sect:documentlibraries}\xlabel{document_libraries}Document Libraries}

Because it is concerned with collections of \docref{hypertext
documents}, \HTXref{HTX} uses the concept of a \st{document
library}. There is nothing particularly grand about this.  It is
simply a directory in which you store a collection of documents. You
can call it whatever you like and populate it with the
\dirref{\qt{.htx} directories} for each of the documents in your
collection. It is also a convenient place to keep other files related
to the documentation set, perhaps even versions of the same documents
in other formats.

The key thing about a document library is that it defines a place in
which to look for documents. When you \st{link} a set of documents
together (the term used for establishing the correct
cross-references), HTX identifies the documents involved by searching
in the document libraries you specify, looking for all the \qt{.htx}
directories they contain. By default, it will simply look in your
current directory.

\subsection{\label{sect:searchpath}\xlabel{multiple_libraries_and_the_library_search_path}\xlabel{HTX_PATH_search_path}Multiple Libraries and the Library Search Path}

You may not always want to modify the contents of existing
\libref{libraries}, even although you may be referring to their
\docref{documents}.  For example, you may have an established set of
documentation in one or more libraries and be developing a new
document that will eventually form part of that set. Your new document
will probably need to refer to the others while you are working on it,
but you may not want to (or have permission to) modify any of the
existing documents.

In this case, you can specify a search path, on which \HTXref{HTX}
will look for other documents to which you may be referring, but it
will not attempt to link (\st{i.e.}\ modify) those other
documents. The search path is specified as a colon-separated list of
directories via the \hi{HTX\_PATH} environment variable, for instance:

\begin{quote}
\begin{verbatim}
setenv HTX_PATH $HOME/mydocs:/docman/newdocs:/docman/olddocs
\end{verbatim}
\end{quote}

This search path is used by all HTX commands when they need to find
documents for which no explicit location has been given.  Note that if
two documents with the same name occur at different points on your
\hi{HTX\_PATH}, only the {\bf first} one will be used.

You can set \hi{HTX\_PATH} to search anywhere you like, but if you do not
specify it yourself, it defaults to:

\begin{quote}
\begin{verbatim}
$INSTALL/docs:$INSTALL/help:$STARLINK/docs:$STARLINK/help
\end{verbatim}
\end{quote}

where \hi{INSTALL} in turn defaults to \hi{\$HOME/star} and
\hi{STARLINK} defaults to \hi{/star}. The values of these environment
variables are evaluated {\bf when the HTX software is installed} (not
when you later make use of it). By default, therefore, HTX will search
for the standard set of Starlink documents and on-line hypertext help,
plus any others you may have installed locally under your own user
name (in the location identified by the \hi{INSTALL} environment
variable).

\section{\label{sect:crossreferences}\xlabel{establishing_cross_references}ESTABLISHING CROSS-REFERENCES}

Before you can establish cross-references between \docref{documents}
using \HTXref{HTX}, you must export \labref{labelled targets} for the
cross-references to point at from each of the referenced
documents. You must also enter links to implement the cross-references
themselves. Both of these make use of standard \HTMLref{HTML}
\st{anchors}, but in a special form that makes them recognisable to
HTX.

\subsection{\xlabel{exporting_targets}\label{sect:targetlabels}Exporting Cross-Reference Targets}

If one \docref{document} is to cross-reference another, the referenced
document should export a label that the other one can refer to. The
position of this label identifies the point in the referenced document
to which a reader should be transferred when the cross-reference is
followed.  If you understand \HTMLref{HTML}, you will be familiar with
this idea, as it uses an HTML \st{destination anchor} for this
purpose. For use with \HTXref{HTX}, this should have a form such
as:\footnote{Any other HTML anchor parameters may also be present and
the double quotes are optional if the enclosed characters are all
alphanumeric.}

\begin{quote}
\begin{verbatim}
<A NAME="xref_conclusions">any text</A>
\end{verbatim}
\end{quote}

In plain HTML, this makes \qt{any text} a potential destination for a
hypertext link, to which any other piece of HTML may refer.

What makes this anchor special to HTX is the
\qt{xref\_} prefix given to the value of
the \hi{NAME} parameter. HTX recognises anchors with this form as
potential targets for document cross-references, and this
distinguishes HTX targets from other HTML destination anchors. This is
important, because it gives the document's author control over which
points in the document other authors may reference.

Normally, many HTML destination anchors will be used for internal
navigation within a document. When the document is revised, these may
be subject to change, so they should not normally be referred to by
other documents. However, destination anchors with the
\qt{xref\_} prefix can be chosen carefully (and
named appropriately) to identify stable components of a document, to
which enduring references can be made from other documents. The author
is, in effect, exporting a set of labels that tell other authors which
parts of his document it is safe to refer to.

We will refer to an anchor with the form above as a
\st{cross-reference target} and the characters that follow \qt{xref\_}
as its \st{label} (\qt{conclusions} in the example above). As in
normal HTML, HTX labels are case sensitive. Unlike plain HTML,
however, they must be unique within an entire hypertext document, not
just within a single HTML file.

\subsection{\xlabel{referring_to_targets}\label{sect:referringtotargets}Referring to Cross-Reference Targets}

To establish a cross-reference to another \docref{document} you again
use an \HTMLref{HTML} anchor, this time with the \hi{HREF} parameter,
in a form similar to the following:\footnote{Again, any other HTML
anchor parameters may also be present.}

\begin{quote}
\begin{verbatim}
<A HREF="/star/docs/sun188.htx/node33.html#xref_appendix">any text</A>
\end{verbatim}
\end{quote}

This is, of course, the standard HTML method of forming a hypertext
link to another location identified by the \urlref{\st{Uniform
Resource Locator} (URL)} given between double quotes. If this URL is
correct (\st{i.e.}\ fully identifies the HTML file you want to
reference), then this anchor will work as it stands, without any
intervention from \HTXref{HTX}.

What makes this link special to HTX is the particular form that the
URL takes.  Anchors containing such URLs are recognised by HTX and can
then be repaired when the cross-reference target changes, by
appropriately editing new values into the fields that are no longer
valid. The key components of the URL that HTX recognises are:

\begin{quote}
\begin{verbatim}
/sun188.htx/
\end{verbatim}
\end{quote}

which identifies the name of the document being referenced (in this
case \qt{sun188}, \st{i.e.}\ this document), and:

\begin{quote}
\begin{verbatim}
#xref_appendix
\end{verbatim}
\end{quote}

where \qt{\#xref\_} is a flag that identifies this
as an HTX cross-reference and \qt{appendix} is the
\labref{cross-reference label} in the target document that identifies
which part of that document is being referenced. Note that HTX will
only allow you to refer to cross-reference labels that have been
exported by the target document and will warn you if you use an
invalid label.

Because HTX can repair URL fields that are wrong, it is quite
acceptable to omit everything inessential and to strip the URL down to
just:

\begin{quote}
\begin{verbatim}
<A HREF="/sun109.htx/#xref_conclusions">any text</A>
\end{verbatim}
\end{quote}

However, entering a full URL (or at least a guess at it) is often a
good idea to start with, as it means the cross-reference may still
work even if someone forgets to link your document.

\subsection{\xlabel{blank_labels}\label{sect:entiredocuments}Referring to an Entire Document}

\HTXref{HTX} recognises a \labref{cross-reference target} with the form:

\begin{quote}
\begin{verbatim}
<A NAME="xref_">any text</A>
\end{verbatim}
\end{quote}

(\st{i.e.}\ with a blank label) as a special case, to be used to
identify the ``top'' page of a \docref{hypertext document}. This
should be the point to which a reader is transferred if the entire
document is requested and no particular part of it is specified.

Any other document can refer to this label, using an anchor with the
form:

\begin{quote}
\begin{verbatim}
<A HREF="/any_where/docname.htx/any_file.html#xref_">any text</A>
\end{verbatim}
\end{quote}

but its inclusion as a cross-reference target in the referenced
document is optional. If it is omitted, HTX will assume that the top
\HTMLref{HTML} page in a document is named after the document
itself. That is, it will use a \urlref{URL} such as:

\begin{quote}
\begin{verbatim}
/some_where/docname.htx/docname.html#xref_
\end{verbatim}
\end{quote}

If your document does not follow this convention, then you should
include a cross-reference target with a blank label at the start of
whichever HTML file serves the same purpose.

\subsection{\label{sect:howtolink}\xlabel{how_to_link}Linking your Hypertext Documents}

Once you have set up a collection of \docref{documents} with
cross-references implemented using the special forms of \HTMLref{HTML}
anchor recognised by \HTXref{HTX}, the next step is to link these
documents together.  This consists of editing the \urlref{URLs} in HTX
cross-references so that they identify the current true location of
the referenced files. No other URLs are touched during this process.

Linking hypertext documents is straightforward. Simply go to the
directory containing the document collection (the \libref{document
library}) and use the command:

\begin{quote}
\linkerref{\hi{hlink}}
\end{quote}

{\bf WARNING: this will modify your documents. If you are worried
about this, then make a backup first.}

The \hi{hlink} command will display some informational messages as it
links your documents and will warn you if you have referred to any
\labref{cross-reference labels} that do not exist.  When it completes,
you should be able to read your documents using any \wwwref{WWW}
browser and to follow all the cross-references that they contain.

When you change any of your documents, you can re-link them to
accommodate the changes simply by using the \hi{hlink} command
again. Because no essential information is discarded during linking,
this process can be repeated indefinitely. You do not need a fresh
copy of the documents each time.

If the documents you want to link reside in more than one document
library, then simply list the library directories on the command line,
as follows:

\begin{quote}
\begin{verbatim}
hlink dir1 dir2 dir3
\end{verbatim}
\end{quote}

Remember that any other local documents that you refer to (but do not
actually want to modify) should reside in libraries that appear on
your \hyperref{\hi{HTX\_PATH} search path}{\hi{HTX\_PATH} search path (see
\S}{)}{sect:searchpath}. If they do not, they will be regarded as
\remref{remote documents} and the URLs generated for them will invoke
a \servref{remote document server}\begin{latexonly}
(see \S\ref{sect:remotedocuments} and \S\ref{sect:remoteservers})
\end{latexonly}.
This is usually less efficient than referring directly to local files.

\subsection{\label{sect:convertingfromlatex}\xlabel{using_cross_references_from_latex}Using Cross-References from \LaTeX}

It is quite straightforward to insert \HTXref{HTX} cross-reference
anchors in \HTMLref{HTML} form directly into \docref{documents} when
writing them by hand. However, many substantial documents already
exist in \latexref{\LaTeX} format and can be converted relatively
easily into hypertext by using the
\htmladdnormallinkfoot{\latextohtml}{\latextohtmlurl} converter
(\xref{see SUN/201}{sun201}{}). \LaTeX{} is, in any case, a convenient
format for the primary source of many documents.

A facility is therefore available for generating HTX cross-references
from commands within \LaTeX{} documents.  The \LaTeX{} commands
involved are normally ignored, but are detected by the
\xref{\hi{star2html}}{sun199}{the_star2html_command} command that provides
Starlink additions to the standard \latextohtml. To allow these
commands to work, the appropriate \xref{document
preamble}{sun199}{using_a_template} must also be included. The procedure to
follow is described in \xref{SUN/199}{sun199}{}, but an outline of the
principles involved is also given here.

To export an HTX \labref{cross-reference label} from a \LaTeX{}
document that is to be converted into hypertext, the following would
be used:

\begin{quote}
\begin{verbatim}
\xlabel{label}
\end{verbatim}
\end{quote}

which results in the HTML anchor:

\begin{quote}
\begin{verbatim}
<A NAME=xref_label>?</A>
\end{verbatim}
\end{quote}

where \qt{?} is some invisible character that simply serves to mark
the position to which links should point. Note that the name of the
label chosen should be unique within a \mbox{\LaTeX} document and must
only contain alphanumeric characters (or be \hyperref{blank}{blank --
see \S}{}{sect:entiredocuments}).

To insert a cross-reference to the target above into another document,
you would use:

\begin{quote}
\begin{verbatim}
\xref{any text}{docname}{label}
\end{verbatim}
\end{quote}

which results in the HTML anchor:

\begin{quote}
\begin{verbatim}
<A HREF="http://star-www.rl.ac.uk/cgi-bin/htxserver/docname.html?xref_label">any text</A>
\end{verbatim}
\end{quote}

This contains an initial guess at the required URL (a document stored on the
RAL service), which will be recognised and 
\hyperref{modified as necessary}{modified as necessary (see \S}{)}{sect:referringtotargets} when the
document is linked.

\section{\label{sect:howlinkingworks}\xlabel{how_linking_works}HOW HTX LINKING WORKS}

This section describes the processes that occur when a set of
\docref{hypertext documents} is linked. It is intended to clarify the
messages that appear and to give some background information to help
solve problems and avoid inefficiencies.

\subsection{\label{sect:indexfiles}\xlabel{index_files}Index Files}

When you invoke the \linkerref{\hi{hlink}} command, it first looks to
find all the \docref{documents} that may need to be linked. This it
does by searching in all the \libref{document libraries} specified on
the command line (or just the current directory by default), looking
for \dirref{directories with a \qt{.htx} extension}.

It then ensures that each document has an up to date \st{index file}
(a file called \qt{htx.index} stored within the document directory),
creating one if necessary. Index files are text files that contain,
amongst other things, details of all the potential in-going and
out-going cross-references for a document and list the \HTMLref{HTML}
files in which they occur.  This is a summary of the cross-links the
document might potentially be involved in and serves to reduce the
time needed to re-link it on subsequent occasions. New index files are
normally only re-generated when a document changes.

In addition to cross-reference information, index files also contain
details of the title of each HTML page in the document and an
indication of which is the \hyperref{\qt{top} page}{\qt{top} page (see
\S}{)}{sect:entiredocuments}. This information is used by other HTX
utilities to perform \htmlref{documentation
searches}{sect:searchingfordocuments}, \st{etc.}

\subsection{\label{sect:linkediting}\xlabel{link_editing}Link-Editing}

The \linkerref{linker} identifies those \docref{documents} that have
changed since the \libref{library} they reside in was last linked by
looking for a file called \qt{htx.log} which serves as a date-stamp in
each document library (if this file is absent, all documents in the
library are considered changed). The contents of the \qt{htx.log} file
also list the location of each local document when the library was
last re-linked. Any that have moved to a different location are also
regarded as \qt{changed}.

The linker identifies all other documents which refer to any of the
changed documents. Since all of these are potentially involved in
cross-references that may no longer be valid, they will all need to be
re-linked.

A further search is then made of the libraries that appear on the
\hyperref{\hi{HTX\_PATH} search path}{\hi{HTX\_PATH} search path (see
\S}{)}{sect:searchpath}. If any of the documents that needs re-linking
refers to a document on this path, then the latter document's
\indexref{index file} is searched to identify the \HTMLref{HTML} file
that is being referenced.

Finally, the affected HTML files are edited to insert the new, correct
\urlref{URLs}. After this, any \wwwref{WWW} browser should be able to
follow any of the links between the documents and arrive at the
correct location.

A document set will need linking again whenever a document is added or
removed, or whenever one of its documents changes. However, because
the linker can quickly identify which files are affected and can
utilise existing index file information, subsequent linking operations
are much faster.

\subsection{\xlabel{absolute_and_relative_links}Absolute and Relative Links}

When establishing a link between two local \docref{documents}, the
\HTXref{HTX} \linkerref{linker} has a choice of using an absolute file
name for the target, or a relative one. The rule used is that relative
cross-references are always generated if two documents reside in the
same \libref{document library} (\st{i.e.}\ the same directory),
otherwise an absolute name is used.

This means that if a document library is moved, all cross-links
between documents within it will remain intact. Only those from
documents outside the library will fail and require re-linking.

\subsection{\label{sect:remotedocuments}\xlabel{references_to_remote_documents}References to Remote Documents}

In the interests of efficiency, the \HTXref{HTX} \linkerref{linker} is
designed to generate references to locally-installed documentation
whenever possible. If a referenced \docref{document} cannot be found
locally, however, it will instead generate a reference to a
\servref{remote document server} -- which should have a complete set
of documentation available.  In this way, links to documents that are
not available locally should not fail, and the reader should always
see a complete set of documentation.

The linker regards a document as remote if it is referenced by one of
the documents it is linking, but it cannot find it either in the
\libref{document libraries} being linked or on the \hyperref{\hi{HTX\_PATH}
search path}{\hi{HTX\_PATH} search path (see \S}{)}{sect:searchpath}.

Note that the HTX linker will not look up a document on a remote
machine. All it does is to insert the remote server reference.
References to invalid \labref{labels} in remote documents (or, indeed,
to non-existent documents) cannot therefore be detected at link time
and diagnosing this type of error is the responsibility of the server
itself.

\section{\xlabel{when_to_relink}WHEN TO RE-LINK}

The \HTXref{HTX} \linkerref{linker} is equipped to detect when
\docref{documents} have changed and to decide when re-linking is
necessary. However, it takes a pragmatic approach to this in order to
keep linking time to a minimum. As a result, there are occasions on
which it may report that no documents need to be re-linked when this
is not, in fact, the case.

This section is intended to help you understand when you need to
re-link your hypertext documents and how to tell the
linker to do this for you.

\subsection{\xlabel{accommodating_normal_document_changes}Accommodating Normal Document Changes}

A documentation set may need to be re-linked whenever any file in any
of its \docref{documents} changes, or when any document is added or
removed, or moved to a new location.  The \HTXref{HTX}
\linkerref{linker} detects changed documents by observing the
modification dates of the document \dirref{\qt{.htx} directories}, the
associated \indexref{index files}, and both the modification date and
contents of the \qt{htx.log} date-stamp file in each \libref{document
library}.

This process is effective, except that it is sometimes possible to
modify an \HTMLref{HTML} file within a document without changing the
modification date on the \qt{.htx} directory that contains it. This is
particularly likely if the HTML file resides in a sub-directory within
the document. If such modifications are carried out, a simple remedy
is to use the UNIX \hi{touch} command to update the document's
\qt{.htx} directory and record the change before running the linker.

As an alternative, the \hi{hlink} command may be invoked with the
\hi{-d} flag to specify a \st{deep dependency} test of all document
files. This causes it to check the modification dates of all the HTML
files in all the documents in order to detect changes. This is a
thorough way of detecting changes, but may be rather too slow for
regular use on large document collections.

You should also remember that re-linking can be rendered necessary not
only by changes in document contents, but also by changes in document
location. Thus, if documents or libraries are moved, it is always wise
to re-run \hi{hlink}.

\subsection{\xlabel{forcing_a_relink}Forcing a Re-Link}

There are several ways to force a re-link of a \docref{document} or
\libref{library}.

To re-link a particular document, or set of documents, from scratch,
their \indexref{index files} may simply be removed, using a command
such as:

\begin{quote}
\begin{verbatim}
rm *.htx/htx.index
\end{verbatim}
\end{quote}

The \linkerref{linker} will then re-generate these indices and
consider the affected documents to have changed. They, and all other
documents that refer to them, will then be re-linked as
necessary. Alternatively, to re-link {\bf all} documents from scratch
in this way, the \hi{-a} flag may be given on the \hi{hlink} command
line.

Removing the \htmlref{\qt{htx.log} date-stamp file}{sect:linkediting}
from a document library will also force the documents in that library
to be re-linked but without unnecessarily re-generating any of its
index files.  This can be somewhat faster if the index files are known
to be up to date.

\subsection{\xlabel{suppressing_relinking}\label{sect:suppressing}Suppressing Re-Linking}

It is often convenient to use the \linkerref{\hi{hlink}} command as
part of software-maintenance procedures, such as in makefiles, since
this allows your documentation set to be re-linked automatically
whenever software is installed or de-installed. However, in a large
software installation this can take some time. Rather than re-linking
after each item of software is changed, it can sometimes be more
efficient to wait until a sequence of changes has been completed. A
grand re-link of the entire documentation set can then be performed.

To permit this, you can set the \hi{HTX\_NOLINK} environment variable
(to any value). While this variable is set, the \hi{hlink} command
will cease to function normally. Instead. it will simply produce a
message saying that hypertext linking has been suppressed. You can
then perform your software updates without waiting for re-linking to
occur. When you have finished, you can un-set the \hi{HTX\_NOLINK}
environment variable and use \hi{hlink} to bring the cross-references
in the associated documentation up to date.

\section{\xlabel{remote_document_servers}\label{sect:remoteservers}REMOTE
DOCUMENT SERVERS}

This section explains how a remote document server is specified and
the role it performs.

\subsection{\label{sect:specifyingaserver}\xlabel{HTX_SERVER_remote_server_URL}Specifying a Document Server}

If \HTXref{HTX} detects that a \docref{document} has been referenced
but cannot find it locally (see \S\ref{sect:remotedocuments}), it will
generate a reference to a remote document server in place of a link to
the local document. The \urlref{URL} it generates in this case takes
the form:

\begin{quote}
\begin{verbatim}
$HTX_SERVER/docname.htx/some_file.html?xref_label
\end{verbatim}
\end{quote}

That is, it replaces any directory path at the start of the URL with
the URL of a remote document server, obtained by translating the
\hi{HTX\_SERVER} environment variable. It also replaces
\qt{\#xref\_} with \qt{?xref\_}
so that the server will receive the \labref{label field} of the URL as
a \st{query string} which it can interpret.

You may set the \hi{HTX\_SERVER} environment variable to identify any
document server you like (for instance, one that you write
yourself). However, if you do not specify one explicitly, the default
is to use the Starlink HTX document server at RAL, whose URL is:

\begin{quote}
\begin{verbatim}
http://star-www.rl.ac.uk/cgi-bin/htxserver
\end{verbatim}
\end{quote}

Once a document has been linked, the choice of server is fixed, until
it is linked again.

\subsection{\label{sect:docserversrole}\xlabel{the_document_servers_role}The Document Server's Role}

It is the remote document server's responsibility to interpret the
\urlref{URL} it receives when the cross-reference is followed and,
typically, to return the correct URL (or the document itself) to the
browser.  The server will therefore have to search for the
\docref{document} in whatever form it takes on the remote machine (it
may not necessarily be in \HTMLref{HTML} format).  This may involve
similar steps to those that the \HTXref{HTX} \linkerref{linker} would
have performed had the document been available locally
(\st{e.g.}\ looking the reference up in an \indexref{index file}),
except that they are performed at run-time rather than in advance when
the document is linked. The server should also issue any necessary
diagnostics -- about invalid \labref{labels} or documents that cannot
be found, for example.

\section{\label{sect:accessingdocuments}\xlabel{gaining_access_to_documents}GAINING ACCESS TO DOCUMENTS}

In previous sections we have concentrated on how to write
\htmlref{hypertext documents}{sect:hypertextdocuments} and organise
them into an inter-linked documentation set. In this section, we turn
our attention to accessing those parts of the documentation set that
we want to read.

\subsection{\label{sect:accessingbyname}\xlabel{displaying_documents_by_name}Displaying Documents by Name}

The simplest form of \htmlref{document}{sect:hypertextdocuments}
access is provided by the \showmeref{\hi{showme}} command, which
simply displays the document you name. For instance:

\begin{quote}
\begin{verbatim}
showme sun188
\end{verbatim}
\end{quote}

would display the \htmlref{\qt{top} page}{sect:entiredocuments} of the
document called \xref{\qt{sun188}}{sun188}{} (\st{i.e.}\ the one you
are reading now).  Using this command is normally easier than entering
a full \urlref{URL} for the document.

The \hi{showme} command displays the document \htmlref{using a WWW
browser}{sect:usingwwwbrowsers} and will make use of one you already
have running if possible. You can specify \htmlref{which
browser}{sect:specifyingabrowser} to use, if required\begin{latexonly}
(see \S\ref{sect:specifyingabrowser})\end{latexonly}.

If you do not give any directory information, \hi{showme} will search
for the document using the \hyperref{\hi{HTX\_PATH} search
path}{\hi{HTX\_PATH} search path (see \S}{)}{sect:searchpath}, but you
can specify explicitly where the document is if you prefer, as in:

\begin{quote}
\begin{verbatim}
showme ~/mydocs/galaxy_survey
\end{verbatim}
\end{quote}

If \hi{showme} cannot find the document locally, it will try and fetch
it from the \htmlref{remote document server}{sect:remoteservers}
instead\begin{latexonly}
(see \S\ref{sect:remoteservers})
\end{latexonly}.
You can suppress this behavior using the \hi{-l} switch if you prefer:

\begin{quote}
\begin{verbatim}
showme -l document
\end{verbatim}
\end{quote}

in which case failure to find the document locally will simply result
in an error. You can also specify that a remote copy of the document
is required using the \hi{-r} switch, in which case any local copy
will be ignored and the remote document server will be asked to supply
the document.

\subsection{\xlabel{displaying_parts_of_documents}Displaying Parts of Documents by Label}

The \showmeref{\hi{showme}} command also allows you to read selected
parts of \htmlref{documents}{sect:hypertextdocuments}, specified using
the \htmlref{cross-reference labels}{sect:targetlabels} they
contain\begin{latexonly}
(see \S\ref{sect:targetlabels})
\end{latexonly}.
To do this, you simply add the label as an extra argument. Thus:

\begin{quote}
\begin{verbatim}
showme sun188 hlink
\end{verbatim}
\end{quote}

will display the description of the \htmlref{\hi{hlink}}{hlink}
command in \xref{this document}{sun188}{}.

Using this feature requires that you know the names of the
cross-reference labels present in the document, or at least requires
that they be easily guessable. If this is not the case, you may prefer
to select which parts of documents to view by means of a
\htmlref{document search}{sect:searchingfordocuments}
instead\begin{latexonly}
(see \S\ref{sect:searchingfordocuments})
\end{latexonly}.

\section{\label{sect:searchingfordocuments}\xlabel{searching_for_information}SEARCHING FOR INFORMATION IN DOCUMENTS}

As well as allowing you to access
\htmlref{documents}{sect:hypertextdocuments} directly \htmlref{by
name}{sect:accessingbyname} (which is normally the fastest method if
you know where to find the information you want),
\htmlref{HTX}{sect:HTX} also allows you to search for information by
keyword.

\subsection{\xlabel{performing_keyword_searches}Performing Keyword Searches}

Keyword searching is performed using the \findmeref{\hi{findme}}
command, which is simply illustrated:

\begin{quote}
\begin{verbatim}
findme HTX
\end{verbatim}
\end{quote}

This command will search your documentation set for the string
\qt{HTX} and will then display a list of the documents found using
your \htmlref{WWW browser}{sect:usingwwwbrowsers}, with each entry in
the list being a hyper-link to the document in question. It is then a
simple matter to follow the link to the document you want to read (in
this example there will probably only be one document to choose from
-- \xref{this one}{sun188}{}).

The way in which the \hi{findme} command performs its search is
explained in the \htmlref{next section}{sect:depthofsearch}, but in
essence it attempts to find information about major topics quickly by
searching only the main titles of documents. It then goes on to
consider more detailed (and time consuming) searches only for more
obscure topics that can't be found readily. The progress of the search
is displayed on your terminal, so you can interrupt it if you fail to
find what you want quickly and don't want to wait.

The more level of detail \hi{findme} needs to consider, the more
detailed will be the list of results it generates, with individual
\HTMLref{HTML} pages being listed if appropriate. This strategy of
performing progressively deeper searches can be observed if you ask
for information on something a little more obscure, like:

\begin{quote}
\begin{verbatim}
findme findme
\end{verbatim}
\end{quote}

which makes \hi{findme} search for information on itself, or even
something very obscure, like:

\begin{quote}
\begin{verbatim}
findme HTX_PATH
\end{verbatim}
\end{quote}

which will (probably) only be found in the body of the text of this
document.

\subsection{\label{sect:depthofsearch}\xlabel{controlling_the_depth_of_search}Controlling the Depth of Search}

The \findmeref{\hi{findme}} command allows you to find information at
the level of detail you want by searching any of four categories of
information associated with \htmlref{hypertext
documents}{sect:hypertextdocuments}:

\begin{enumerate}

\item \label{sect:searchname}{\bf The document name (-n
switch)}\mbox{}\\ This is given by the name of the
\htmlref{directory}{sect:documentdirectories} that contains the
document\begin{latexonly}
(see \S\ref{sect:documentdirectories})
\end{latexonly}.
For example, the name of the document you are reading now is
\xref{\qt{sun188}}{sun188}{} because its hypertext version resides in
a directory called \qt{sun188.htx}. The name of a document may
indicate what category of information it contains and is very quick
and easy to search.

\item \label{sect:searchtitle}{\bf The document title (-t
switch)}\mbox{}\\ This is extracted from the \htmlref{\qt{top} HTML
page}{sect:entiredocuments} of a document\begin{latexonly}
(see \S\ref{sect:entiredocuments})
\end{latexonly}
and consists of the text that appears between the \hi{$<$TITLE$>$} and
\hi{$<$/TITLE$>$} tags in the HTML header section for that page. The
title of a document is obviously a good place to search for important
topics and this can be done quite quickly.

\item \label{sect:searchheading}{\bf The document's page headings (-h
switch)}\mbox{}\\ These are extracted in the same way as the document
title (above), but from all the other HTML pages in the document,
excluding the \qt{top} page. If you have
\htmlref{converted}{sect:convertingfromlatex} your document from a
format such as \xref{\LaTeX{}}{sun199}{}\begin{latexonly}
(see \S\ref{sect:convertingfromlatex} and \xref{SUN/199}{sun199}{})
\end{latexonly}, then these will be the section headings
that appear in the printed form of the document. This is normally a
fruitful place to search for slightly more specialised topics and can
be done without a serious time penalty because \htmlref{HTX}{sect:HTX}
caches this information in a document's \htmlref{index
file}{sect:indexfiles}\begin{latexonly}
(see \S\ref{sect:indexfiles})
\end{latexonly}.

\item \label{sect:searchline}{\bf The lines of text in the document (-l switch)}\mbox{}\\
These consist of the contents of all the HTML files in the document
(including all their \HTMLref{HTML} tags, \urlref{URLs},
\st{etc}). This is the ultimate place to search for information, but
this can take quite a while if the documentation set is large.

\end{enumerate}

If none of the switches shown above is used when \hi{findme} is
invoked, its default action is to:

\begin{enumerate}
\item Search the document titles
\item If that fails to find a match, search the page headings
\item If that fails to find a match, search the lines of text
\end{enumerate}

However, if one or more of the \hi{-n}, \hi{-t}, \hi{-h} or \hi{-l}
switches is used, then \st{only} the specified categories of
information will be searched, and this will be done in a single pass
through all the documents. For instance:

\begin{quote}
\begin{verbatim}
findme -n sun
\end{verbatim}
\end{quote}

will cause only the document names to be searched (for the string
\qt{sun}), while:

\begin{quote}
\begin{verbatim}
findme -t sun
\end{verbatim}
\end{quote}

would search only the document titles, and:

\begin{quote}
\begin{verbatim}
findme -t -h sun
\end{verbatim}
\end{quote}

would search both the titles and page headings in a single pass.

\subsection{\xlabel{searching_specific_documents}Searching Specific Documents}

By default, the \findmeref{\hi{findme}} command will search your
entire documentation set, consisting of all the
\htmlref{documents}{sect:hypertextdocuments} found on the
\hyperref{\hi{HTX\_PATH} search path}{\hi{HTX\_PATH} search path (see
\S}{)}{sect:searchpath}.\footnote{It will also include the names and
titles of documents found in \htmlref{catalogue
files}{sect:cataloguefiles}\begin{latexonly} (see
\S\ref{sect:cataloguefiles})
\end{latexonly}.}

However, you can restrict the search to specific documents by listing
them after the keyword you are searching for, thus:

\begin{quote}
\begin{verbatim}
findme targets sun188
\end{verbatim}
\end{quote}

would find information about \qt{targets} in this document. This is
often a useful way of finding reference information once you are
reasonably familiar with a document's contents. Restricting the search
to a specific document will also make it far more rapid.

You may specify as many documents to search as you want. If you do not
give explicit directory information, the \hi{HTX\_PATH} search path
will be used to locate them.

\subsection{\xlabel{other_search_options_available}Other Search Options Available}

The \findmeref{\hi{findme}} command has a number of options that allow
you to fine-tune the \htmlref{search}{sect:searchingfordocuments} that
it performs. These include:

\begin{itemize}
\item Performing case-sensitive searches (by default, keyword matching
is case insensitive)

\item Searching for whole words (by default the keyword you give will
match any string, even if it is only part of a word)

\item Matching patterns in text by including regular expressions in
the keyword string

\item Abbreviating the output list by suppressing information about
individual \HTMLref{HTML} pages and only displaying
\htmlref{document}{sect:hypertextdocuments} names and titles

\item Sorting the output list into order according to the significance
or number of matches found in each document.

\item Including information about the number of matches found.
\end{itemize}

See the \findmeref{\hi{findme} command description}
\begin{latexonly}
in \S\ref{sect:commanddescriptions}
\end{latexonly}
for full details.

\section{\xlabel{using_HTX_from_other_software}USING HTX FACILITIES FROM OTHER SOFTWARE}

\subsection{\label{sect:hypertexthelp}\xlabel{providing_hypertext_help}Providing Hypertext Help}

An important application of the \showmeref{\hi{showme}} command is to
provide hypertext help for other software packages.

In this situation, the help information would be contained in one or
more \htmlref{hypertext documents}{sect:hypertextdocuments} and the
controlling software would invoke the \hi{showme} command to
\htmlref{display the required part}{sect:accessingdocuments} of it on
demand, the information being selected by giving an appropriate
\htmlref{cross-reference label}{sect:targetlabels}\begin{latexonly}
(see \S\ref{sect:targetlabels})
\end{latexonly}. The person
reading the displayed information can then explore any hyper-links
within it to gain further information. These could point at other
documents you have \htmlref{linked}{sect:howtolink} the help
documents against or, indeed, anything else on the \wwwref{WWW}.

When designing a graphical user interface, help information can easily
be made \qt{context sensitive} by setting the cross-reference label
according to the task being performed (\st{e.g.}\ which window is
active) and invoking \hi{showme} when a \qt{help} button is
pressed. In fact, with a dedicated \htmlref{WWW
browser}{sect:usingwwwbrowsers}, one might even consider displaying
the information automatically as the task in hand changed, without
waiting to be asked.

By going via this \htmlref{HTX}{sect:HTX} interface, the controlling
software is insulated from changes in the way the help documentation
is organised and \htmlref{indexed}{sect:indexfiles}. It also need not
concern itself with how to
\htmlref{communicate}{sect:browsercommunication} with the WWW
browser. \label{text:gettingdocumenturls}In some cases, however, you
may want your software to handle the display of help information
yourself. You can do this by using the \hi{-n} switch on the
\hi{showme} command, thus:

\begin{quote}
\begin{verbatim}
showme -n help_document subject_label
\end{verbatim}
\end{quote}

This prevents \hi{showme} from displaying the document. Instead, it
simply writes the \urlref{URL} for the part of the document you
requested to its standard output. Your software can then read this and
handle it in whatever way you choose.

\subsection{\xlabel{using_HTX_to_control_a_WWW_browser}Using HTX to Control a WWW Browser}

If you have used the \showmeref{\hi{showme}} command to obtain a
\urlref{URL} for a \htmlref{document}{sect:hypertextdocuments} (see
\htmlref{above}{text:gettingdocumenturls}), one possible way of using
this might be to save it and pass it back to \hi{showme} later on,
using the \hi{-u} option for displaying a document by URL:

\begin{quote}
\begin{verbatim}
showme -u url
\end{verbatim}
\end{quote}

Of course, the URL you give need not have come from \hi{showme} in the
first place -- any URL that your \htmlref{WWW
browser}{sect:usingwwwbrowsers} can handle would be acceptable.

You can also supply the name of a local file using the \hi{-f} switch,
thus:

\begin{quote}
\begin{verbatim}
showme -f filename
\end{verbatim}
\end{quote}

Because you can give relative file names to \hi{showme}, this is often
an easier way of viewing a file interactively than typing its full
path into the browser.

These examples illustrate how \hi{showme} provides a convenient and
uniform interface for ``remotely controlling'' a WWW browser,
permitting you to display any WWW document for which you have a URL or
file name.

\subsection{\xlabel{performing_document_searches_from_other_software}Performing Document Searches from Other Software}

You can use the \findmeref{\hi{findme}} command from within other
software to implement \htmlref{document
searches}{sect:searchingfordocuments} with a user interface of your
own design. This is done by using the \hi{-html} switch, as follows:

\begin{quote}
\begin{verbatim}
findme -html -q keyword
\end{verbatim}
\end{quote}

This switch prevents \hi{findme} from displaying its list of results
via a \htmlref{WWW browser}{sect:usingwwwbrowsers}, and it instead
writes them to its standard output in \HTMLref{HTML}
format.\footnote{In this case the \hi{-q} switch has also been used to
suppress messages about the progress of the search.} In this mode, a
\qt{naked} output list is produced that lacks the surrounding HTML
document and this allows it to be embedded in any other HTML context
where it may be needed.

An example of this might be a \formsref{WWW forms} interface for
performing local document searches. Typically, a {\em Common Gateway
Interface} (\cgiref{CGI}) script would be written to be invoked by
your \wwwref{WWW} server and perform the search, generating a page of
HTML as its output. Such a script could invoke the \hi{findme} command
in the form above to implement the search and then embed its output in
the page it generates.

Alternatively, a script invoked from the command line or a graphical
user interface might build an HTML document, including search output
from \hi{findme}, in a local file. It could then use the
\showmeref{\hi{showme}} command with its \hi{-f} flag to display the
result.

You can test for the success of a search by examining the return
status from \hi{findme} which is set to the number of documents that
were matched.

\section{\label{sect:cataloguefiles}\xlabel{catalogue_files}CATALOGUE FILES}

Catalogue files provide a way of introducing documents into the
\htmlref{searches}{sect:searchingfordocuments} performed by the
\findmeref{\hi{findme}} command which it would not otherwise be able
to search. This is of particular benefit if some documents are not
stored locally, or are not in \htmlref{hypertext
format}{sect:hypertextdocuments}.

\subsection{\label{sect:cataloguefileformat}\xlabel{catalogue_file_name_and_format}Catalogue File Name and Format}

An \htmlref{HTX}{sect:HTX} catalogue file is a text file with the name
\qt{htx.catalogue} which resides in a \htmlref{document
library}{sect:documentlibraries}. Each line of the file should contain
an entry consisting of three fields separated by white space, in the
form:

\begin{quote}
\begin{verbatim}
docname docfile title_text
\end{verbatim}
\end{quote}

where:

\begin{itemize}
\item {\bf docname} is the name by which the document is to be known
\item {\bf docfile} is the name of the file containing the document, given
relative to the document library directory
\item {\bf title\_text} is the document's title, which may contain
further white space
\end{itemize}

\subsection{\xlabel{a_catalogue_file_example}A Catalogue File Example}

Suppose, for example, that you are converting an existing
documentation set into \htmlref{hypertext
form}{sect:hypertextdocuments}, but still have some documents
available only in \xref{\st{DVI}}{sun9}{} and \psref{\st{postscript}}
format (with file extensions \qt{.dvi} and \qt{.ps}). The
\findmeref{\hi{findme}} command will not be able to search these
\qt{old} documents because it doesn't know how to extract (for
example) their titles from the files provided. To help overcome this,
you would describe these documents in a \htmlref{catalogue
file}{sect:cataloguefiles}, perhaps along the following lines:

\begin{quote}
\begin{verbatim}
review doc1.ps A Review of Documentation Systems
intro doc2.dvi Introduction to Hypertext
writing writing.ps How to Communicate Effectively
...
\end{verbatim}
\end{quote}

Note that the document name and file name need not match.  This file
introduces the listed documents to the \hi{findme} command, tells it
where to find the corresponding document files and allows it to
perform \htmlref{searching}{sect:searchingfordocuments} by document
\htmlref{name and/or title}{sect:depthofsearch} (but not by page
heading or lines of textual content, since it cannot know how to
decode the document format to obtain these).

\subsection{\xlabel{how_catalogue_files_are_searched}How Catalogue Files are Searched}

Documents listed in \htmlref{HTX}{sect:HTX} \htmlref{catalogue
files}{sect:cataloguefiles} are added into the documentation set {\bf
after} all hypertext documents have first been found using the
\hyperref{\hi{HTX\_PATH} search path}{\hi{HTX\_PATH} search path (see
\S}{)}{sect:searchpath}.  If a document is found in hypertext form, it
occludes any subsequence occurrence of a document with the same name
in a catalogue file. This means that if you convert an \qt{old}
document into hypertext form (with a \qt{.htx} file extension), the
new version will automatically be found in preference to the old one
-- there is no need to remove it from the catalogue file.

Catalogue files are also found by following the \hi{HTX\_PATH} search
path after it has been used to find hypertext documents, and are
recognised by the name \qt{htx.catalogue}. If more than one catalogue
file is found, their contents are simply concatenated in the order in
which they are found.

Duplicate entries for a document are permitted in catalogue files (and
can also arise when catalogue files are concatenated). They provide a
mechanism for a document to have alternative titles. This can
sometimes improve the usefulness of document searches if the original
title lacks any useful keywords (you might think of this as combining
both a title index and a subject index into the same file).  If more
than one title entry is matched for a particular document, then the
one that occurs first in the catalogue file(s) is used.

\subsection{\xlabel{providing_a_local_catalogue_of_remote_documents}Providing a Local Catalogue of Remote Documents}

The documents listed in \htmlref{HTX}{sect:HTX} \htmlref{catalogue
files}{sect:cataloguefiles} need not necessarily exist on the local
file system. HTX will check to see if they do, and will generate
hyper-links to them if they appear to be readable. For files that are
not accessible, however, it will generate a reference to the
\htmlref{remote document server}{sect:remoteservers}\begin{latexonly}
(see \S\ref{sect:remoteservers})
\end{latexonly}. This reference will
take the \htmlref{standard
form}{sect:specifyingaserver}\begin{latexonly}
(\S\ref{sect:specifyingaserver})
\end{latexonly}
using the document name -- not the \htmlref{file
name}{sect:cataloguefileformat} used in the the catalogue file.

Catalogue files can therefore be used as a
\htmlref{searchable}{sect:searchingfordocuments} catalogue of
documents that are available remotely. In fact, a \htmlref{document
library}{sect:documentlibraries} containing only a catalogue file
could be searched by the \findmeref{\hi{findme}} command and any
matches would then refer to the remote version of the document, in
whatever form it happens to be stored.

\newpage
\appendix
\section{\xlabel{command_descriptions}\label{sect:commanddescriptions}DESCRIPTIONS OF HTX COMMANDS}
%+
%  Name:
%     SST.TEX

%  Purpose:
%     Define LaTeX commands for laying out Starlink routine descriptions.

%  Language:
%     LaTeX

%  Type of Module:
%     LaTeX data file.

%  Description:
%     This file defines LaTeX commands which allow routine documentation
%     produced by the SST application PROLAT to be processed by LaTeX and
%     by LaTeX2html. The contents of this file should be included in the
%     source prior to any statements that make of the sst commnds.

%  Notes:
%     The commands defined in the style file html.sty provided with LaTeX2html 
%     are used. These should either be made available by using the appropriate
%     sun.tex (with hypertext extensions) or by putting the file html.sty 
%     on your TEXINPUTS path (and including the name as part of the  
%     documentstyle declaration).

%  Authors:
%     RFWS: R.F. Warren-Smith (STARLINK)
%     PDRAPER: P.W. Draper (Starlink - Durham University)

%  History:
%     10-SEP-1990 (RFWS):
%        Original version.
%     10-SEP-1990 (RFWS):
%        Added the implementation status section.
%     12-SEP-1990 (RFWS):
%        Added support for the usage section and adjusted various spacings.
%     8-DEC-1994 (PDRAPER):
%        Added support for simplified formatting using LaTeX2html.
%     {enter_further_changes_here}

%  Bugs:
%     {note_any_bugs_here}

% -

%  Define length variables.
\newlength{\sstbannerlength}
\newlength{\sstcaptionlength}
\newlength{\sstexampleslength}
\newlength{\sstexampleswidth}

%  Define a \tt font of the required size.
\newfont{\ssttt}{cmtt10 scaled 1095}

%  Define a command to produce a routine header, including its name,
%  a purpose description and the rest of the routine's documentation.
\newcommand{\sstroutine}[3]{
   \goodbreak
   \rule{\textwidth}{0.5mm}
   \vspace{-7ex}
   \newline
   \settowidth{\sstbannerlength}{{\Large {\bf #1}}}
   \setlength{\sstcaptionlength}{\textwidth}
   \setlength{\sstexampleslength}{\textwidth}
   \addtolength{\sstbannerlength}{0.5em}
   \addtolength{\sstcaptionlength}{-2.0\sstbannerlength}
   \addtolength{\sstcaptionlength}{-5.0pt}
   \settowidth{\sstexampleswidth}{{\bf Examples:}}
   \addtolength{\sstexampleslength}{-\sstexampleswidth}
   \parbox[t]{\sstbannerlength}{\flushleft{\Large {\bf #1}}}
   \parbox[t]{\sstcaptionlength}{\center{\Large #2}}
   \parbox[t]{\sstbannerlength}{\flushright{\Large {\bf #1}}}
   \begin{description}
      #3
   \end{description}
}

%  Format the description section.
\newcommand{\sstdescription}[1]{\item[Description:] #1}

%  Format the usage section.
\newcommand{\sstusage}[1]{\item[Usage:] \mbox{} \\[1.3ex] {\ssttt #1}}

%  Format the invocation section.
\newcommand{\sstinvocation}[1]{\item[Invocation:]\hspace{0.4em}{\tt #1}}

%  Format the arguments section.
\newcommand{\sstarguments}[1]{
   \item[Arguments:] \mbox{} \\
   \vspace{-3.5ex}
   \begin{description}
      #1
   \end{description}
}

%  Format the returned value section (for a function).
\newcommand{\sstreturnedvalue}[1]{
   \item[Returned Value:] \mbox{} \\
   \vspace{-3.5ex}
   \begin{description}
      #1
   \end{description}
}

%  Format the parameters section (for an application).
\newcommand{\sstparameters}[1]{
   \item[Parameters:] \mbox{} \\
   \vspace{-3.5ex}
   \begin{description}
      #1
   \end{description}
}

%  Format the examples section.
\newcommand{\sstexamples}[1]{
   \item[Examples:] \mbox{} \\
   \vspace{-3.5ex}
   \begin{description}
      #1
   \end{description}
}

%  Define the format of a subsection in a normal section.
\newcommand{\sstsubsection}[1]{ \item[{#1}] \mbox{} \\}

%  Define the format of a subsection in the examples section.
\newcommand{\sstexamplesubsection}[2]{\sloppy
\item[\parbox{\sstexampleslength}{\ssttt #1}] \mbox{} \\ #2 }

%  Format the notes section.
\newcommand{\sstnotes}[1]{\item[Notes:] \mbox{} \\[1.3ex] #1}

%  Provide a general-purpose format for additional (DIY) sections.
\newcommand{\sstdiytopic}[2]{\item[{\hspace{-0.35em}#1\hspace{-0.35em}:}] \mbox{} \\[1.3ex] #2}

%  Format the implementation status section.
\newcommand{\sstimplementationstatus}[1]{
   \item[{Implementation Status:}] \mbox{} \\[1.3ex] #1}

%  Format the bugs section.
\newcommand{\sstbugs}[1]{\item[Bugs:] #1}

%  Format a list of items while in paragraph mode.
\newcommand{\sstitemlist}[1]{
  \mbox{} \\
  \vspace{-3.5ex}
  \begin{itemize}
     #1
  \end{itemize}
}

%  Define the format of an item.
\newcommand{\sstitem}{\item}

%% Now define html equivalents of those already set. These are used by
%  latex2html and are defined in the html.sty files.

\begin{htmlonly}

%  Re-define \ssttt.
   \newcommand{\ssttt}{\tt}

%  sstroutine.
   \renewcommand{\sstroutine}[3]{
      \subsection{#1\xlabel{#1}-\label{#1}#2}
      \begin{description}
         #3
      \end{description}
   }

%  sstdescription
   \renewcommand{\sstdescription}[1]{\item[Description:]
      \begin{description}
         #1
      \end{description}
   }

%  sstusage
   \renewcommand{\sstusage}[1]{\item[Usage:]
      \begin{description}
         {\ssttt #1}
      \end{description}
   }

%  sstinvocation
   \renewcommand{\sstinvocation}[1]{\item[Invocation:]
      \begin{description}
         {\ssttt #1}
      \end{description}
   }

%  sstarguments
   \renewcommand{\sstarguments}[1]{
      \item[Arguments:]
      \begin{description}
         #1
      \end{description}
   }

%  sstreturnedvalue
   \renewcommand{\sstreturnedvalue}[1]{
      \item[Returned Value:]
      \begin{description}
         #1
      \end{description}
   }

%  sstparameters
   \renewcommand{\sstparameters}[1]{
      \item[Parameters:]
      \begin{description}
         #1
      \end{description}
   }

%  sstexamples
   \renewcommand{\sstexamples}[1]{
      \item[Examples:]
      \begin{description}
         #1
      \end{description}
   }

%  sstsubsection
   \renewcommand{\sstsubsection}[1]{\item[{#1}]}

%  sstexamplesubsection
   \renewcommand{\sstexamplesubsection}[2]{\item[{\ssttt #1}] \\ #2}

%  sstnotes
   \renewcommand{\sstnotes}[1]{\item[Notes:]
      \begin{description}
         #1
      \end{description}
   }

%  sstdiytopic
   \renewcommand{\sstdiytopic}[2]{\item[{#1}]
      \begin{description}
         #2
      \end{description}
   }

%  sstimplementationstatus
   \renewcommand{\sstimplementationstatus}[1]{\item[Implementation Status:] 
      \begin{description}
         #1
      \end{description}
   }

%  sstitemlist
   \newcommand{\sstitemlist}[1]{
      \begin{itemize}
         #1
      \end{itemize}
   }
\end{htmlonly}

%  End of "sst.tex" layout definitions.

\sstroutine{
   findme
}{
   Search for documents by keyword and display a list of those found
}{
   \sstdescription{
      This command performs \htmlref{keyword
      searching}{sect:searchingfordocuments} of locally available (or
      locally catalogued) \htmlref{documents}{sect:hypertextdocuments}
      and displays a list of those found using a \htmlref{WWW
      browser}{sect:usingwwwbrowsers}. This list includes hyper-links
      to the parts of each document that were matched.
   }
   \sstinvocation{
      findme [switches] [keyword] [doclist]
   }
   \sstdiytopic{
      Parameters
   }{
      \sstsubsection{
         keyword
      }{
         The string of characters to be searched for. This is treated as a
         single string (not as multiple keywords) and should be quoted if it
         contains special characters or white space. Pattern matching
         characters, as used in \qt{sed} or \qt{grep} regular expressions, may be
         included.

         If this parameter is omitted, then all documents searched will be
         matched. This provides a convenient way of listing all the
         documents available.
      }
      \sstsubsection{
         doclist
      }{
         An optional space-separated list of the documents to be
         searched. If this is omitted, then the complete set of
         hypertext documents will be searched, as found on the
         \hyperref{\hi{HTX\_PATH} search path.}{\hi{HTX\_PATH} search
         path (see \S}{).}{sect:searchpath} Any further documents
         described in \htmlref{catalogue
         files}{sect:cataloguefiles}\begin{latexonly}
         (\S\ref{sect:cataloguefiles})
         \end{latexonly}
         will also be included.

         Any \qt{.htx} extension on document names is ignored.
      }
   }
   \sstdiytopic{
      Switches
   }{
      \sstsubsection{
         -b
      }{
         Requests that a \qt{brief} list be produced of the documents found. This
         means that only document names and titles will appear and references
         to individual pages will be omitted. By default, individual pages will
         be listed if the search has included them and they match the search
         criteria.
      }
      \sstsubsection{
         -c
      }{
         Indicates that case is significant when searching for the keyword. By
         default, differences in case are ignored.
      }
      \sstsubsection{
         -f
      }{
         Indicates that a full search should be performed, involving searching
         document names, titles, page headings and lines of text. This option
         is shorthand for the switches \hi{-n}, \hi{-t}, \hi{-h} and \hi{-l} used together.
      }
      \sstsubsection{
         -h
      }{
         Indicates that a search for the keyword is to be performed on
         all the \htmlref{page
         headings}{sect:searchheading}\begin{latexonly} (see
         \S\ref{sect:depthofsearch})
         \end{latexonly} within each document.This provides a convenient
         compromise between speed of execution and full search
         coverage, and generally produces an acceptable amount of
         output. By default, a search on page headings is performed if
         the keyword cannot be found in any document title. Page
         headings may only be searched in local hypertext (\qt{.htx})
         documents.
      }
      \sstsubsection{
         -html
      }{
         Indicates that the list of documents found by this command should not
         be displayed using a WWW browser. Instead, the results are simply
         written to standard output as a list in \HTMLref{HTML} format (without
         a surrounding HTML document). This
         provides an interface for other software that will display
         the results itself.

         Note that the \urlref{URLs} used for hyper-links in this output list
         will be suitable only for local use by a WWW browser running on the
         same machine as the \hi{findme} command. They may not be suitable,
         for instance, for embedding in an HTML document that will be
         interpreted by a remote WWW browser.
      }
      \sstsubsection{
         -l
      }{
         Indicates that a search for the keyword is to be performed on
         all \htmlref{lines of text}{sect:searchline}
         \begin{latexonly}
         (see \S\ref{sect:depthofsearch}) \end{latexonly} within each
document (note that this will include all HTML tags, URLs,
\st{etc.}). This provides the fullest possible form of keyword search,
but may take some time to complete and could generate a large volume
of output. By default, a search of document lines is only performed if
the keyword cannot be found in any document title or page
heading. Line-oriented searching can only be performed on local
hypertext (\qt{.htx}) documents.
      }
      \sstsubsection{
         -m
      }{
         Indicates that the output list is to contain information on which
         search criteria were matched and how many matches were found. By
         default, this information is omitted.
      }
      \sstsubsection{
         -n
      }{
         Indicates that a search for the keyword is to be performed on
         the \htmlref{name}{sect:searchname}
\begin{latexonly} (see \S\ref{sect:depthofsearch}) \end{latexonly} of each document.
         Searching of document names is not performed by default.
      }
      \sstsubsection{
         -q
      }{
         Indicates that the search should progress in \qt{quiet} mode without
         producing messages about its progress. By default, messages about the
         progress of the search are written to the controlling terminal.
      }
      \sstsubsection{
         -s
      }{
         Indicates that the output list is to be sorted so that the most
         significant matches appear first and the least significant last.
         In assessing this, matches to the document name are given the highest
         significance, then matches to titles, page headings and finally lines
         of textual content (see \S\ref{sect:depthofsearch}). An alphabetical sort on document title and page
         heading is used to resolve any remaining ambiguity over output order.
         By default, a simple alphabetical sort on document title and page
         heading is used alone.
      }
      \sstsubsection{
         -t
      }{
         Indicates that a search for the keyword is to be performed on
         the \htmlref{title}{sect:searchtitle} \begin{latexonly} (see
         \S\ref{sect:depthofsearch}) \end{latexonly} of each document.
         This provides a quick but effective form of search for major
         topics and gives at most one entry in the output list for
         each document matched. By default, title matching is
         performed first, and a search of page headings (and
         eventually lines of textual content) is performed only if
         this initial search fails to match any documents (see
         \S\ref{sect:depthofsearch}). Using \hi{-t} prevents these
         subsequent searches from happening automatically.
      }
      \sstsubsection{
         -w
      }{
         Indicates that the keyword supplied must match an entire word
         (\st{i.e.}\ a
         string delimited by characters which are not underscores or
         alphanumerics, or delimited by the beginning or end of the text, or
         by a newline). By default, the specified string of characters is
         matched wherever it occurs, so long as it does not span multiple
         lines of text.
      }
      \sstsubsection{
         -warn
      }{
         Indicates that any warning messages issued by the WWW browser
         (\st{e.g.}\ when it is started up) are to be suppressed. By default, these
         warnings are written to standard error along with any other warning
         or error messages.
      }
   }
   \sstdiytopic{
      Exit Status
   }{
      The exit status from this command is set equal to the number of
      documents matched by the search. Thus a non-zero exit status
      indicates success, while a zero status indicates failure to find
      any document. Note that this is the reverse of the convention
      normally adopted by UNIX commands.
   }
   \sstdiytopic{
      Notes On Searching
   }{
      \sstitemlist{

         \sstitem
         If none of the switches \hi{-n}, \hi{-t}, \hi{-h} or \hi{-l} is used, the keyword
         given will first be searched for in the title of each document. If this
         fails to produce a match, it will next be searched for in the page
         headings of each document. If this also fails to produce a match, a final
         search of the lines of text within each document will be made.

         \sstitem
         If one or more of the switches \hi{-n}, \hi{-t}, \hi{-h} or \hi{-l} is used, the
         automatic sequence of searches described above will not occur. Instead,
         only those document components specified by these switches (name, title,
         page header and lines of text, respectively) will be searched. This will
         be done in a single pass through all documents.

         \sstitem
         To obtain the fullest possible (but slowest) search, use the \hi{-f}
         option. This is equivalent to using all of the switches
         \hi{-n}, \hi{-t}, \hi{-h} and \hi{-l} together.
      }
   }
   \sstdiytopic{
      Notes on Specifying Documents
   }{
      \sstitemlist{

         \sstitem
         If no documents are specified, then all directories on the
         \hi{HTX\_PATH} search path will be inspected for hypertext
         (\qt{.htx}) documents and all those found will be searched.
         In addition, if any directory on the search path contains a
         catalogue file, the documents it describes will also be
         included in any search of document names or titles (but not
         of page headings or lines of text).

         \sstitem
         If one or more document names are supplied, then the search will be
         restricted to the specified documents only.
         If these document names are supplied without directory information, then
         they will be located by following the \hi{HTX\_PATH} search path
         and then by reading the contents of any catalogue files, if necessary.
         If document names are supplied with explicit directory information,
         then they must refer to local hypertext documents and no
         search will be made to locate them.

         \sstitem
         If documents with the same name are found both locally in hypertext
         form and in one or more catalogue files, then the hypertext version takes
         precedence.
         If a document with the same name appears more than once in the list
         given for the \hi{doclist} parameter, then the first occurrence takes
         precedence, except that the first occurrence of a name with explicit
         directory information always takes precedence over the same document
         specified without directory information.
      }
   }
   \sstexamples{
      \sstexamplesubsection{
         findme
      }{
         Finds all available documents.
      }
      \sstexamplesubsection{
         findme guide
      }{
         Finds all documents with the string \qt{guide} in them. This is done
         by first searching their titles, then (if that fails) by searching
         all their page headings, then (if that also fails) by searching all
         of their lines of text.
      }
      \sstexamplesubsection{
         findme -t guide
      }{
         Finds all documents with the string \qt{guide} in their
         title. Only titles are searched.
      }
      \sstexamplesubsection{
         findme -n sun
      }{
         Finds all documents whose names contain the string \qt{sun}.
      }
      \sstexamplesubsection{
         findme -w star
      }{
         Finds documents that contain \qt{star} as a word on its own.
      }
      \sstexamplesubsection{
         findme -c GnS
      }{
         Finds documents that contain the string \qt{GnS} with the correct
         capitalisation.
      }
      \sstexamplesubsection{
         findme -l -b unix
      }{
         Searches all lines of text in all documents for the string \qt{unix} and
         displays a brief listing of the results, so that only the relevant
         document names and titles are shown.
      }
      \sstexamplesubsection{
         findme -h DAT\_ sun92
      }{
         Finds a document called \qt{sun92} and searches its page headings (only)
         for the string \qt{DAT\_}. Each page which matches is listed.
      }
      \sstexamplesubsection{
         findme -t -h '?\$' docs/$*$.htx
      }{
         Searches the titles and page headings of all hypertext documents
         stored in directory \hi{docs} and lists those which end in a question
         mark (\qt{?\$} is a regular expression specifying that there should be a
         question mark at the end of a line).
      }
      \sstexamplesubsection{
         findme -h -html -q "\$keyword" pkg\_manual >>/tmp/results\$\$
      }{
         This command might be used to provide a command lookup facility for a
         software package. It searches all the page headings in
         the document called \qt{pkg\_manual} for the command stored
         in the \hi{keyword} variable
         and appends the resulting list (in HTML format) to a scratch file in
         which an HTML page of results is being constructed. Messages about
         the progress of the search are suppressed with the \hi{-q} switch.
      }
   }
   \sstdiytopic{
      Environment Variables Used
   }{
      \sstsubsection{
         \htmlref{\hi{HTX\_BROWSER}}{sect:specifyingabrowser}
      }{
         The command which will be used to invoke the \htmlref{WWW
         browser}{sect:usingwwwbrowsers}\begin{latexonly} (see
         \S\ref{sect:specifyingabrowser})\end{latexonly}.
      }
      \sstsubsection{
         \htmlref{\hi{HTX\_PATH}}{sect:searchpath}
      }{
         A colon-separated list of the library directories in which to
         search for hypertext documents\begin{latexonly} (see
         \S\ref{sect:searchpath})\end{latexonly}.
      }
      \sstsubsection{
         \htmlref{\hi{HTX\_SERVER}}{sect:specifyingaserver}
      }{
         The URL of the document server to be used for serving remote
         documents\begin{latexonly} (see
         \S\ref{sect:specifyingaserver})\end{latexonly}.
      }
      \sstsubsection{
         \htmlref{\hi{HTX\_TMP}}{sect:intermachinecommunication}
      }{
         The name of the directory in which to create temporary communication
         files. If this variable is not set or is null, then
         \hi{\$HOME/.htxtmp} is used instead\begin{latexonly} (see
         \S\ref{sect:intermachinecommunication})\end{latexonly}.
      }
   }
}
\newpage
\sstroutine{
   hlink
}{
   Perform cross-linking of hypertext documents
}{
   \sstdescription{
      This command searches for \htmlref{hypertext
      documents}{sect:hypertextdocuments} in a specified list of
      \htmlref{library directories}{sect:documentlibraries} and
      link-edits their \qt{.html} files to insert the correct
      \urlref{URLs} so that
      \htmlref{cross-reference}{sect:crossreferences} links between
      the documents point at the appropriate files.
   }
   \sstinvocation{
      hlink [switches] [dirlist]
   }
   \sstdiytopic{
      Parameters
   }{
      \sstsubsection{
         dirlist
      }{
         A space-separated list of directories in which to search for
         the hypertext documents to be linked (as identified by their
         \qt{.htx} file extension). If this list is not given, the
         default directory \qt{.} is used.
      }
   }
   \sstdiytopic{
      Switches
   }{
      \sstsubsection{
         -a
      }{
         Specifies that all documents in the specified directories should be
         re-linked \qt{from source}. This means that all documents will be
         re-examined for cross-references, regardless of whether they appear to
         have changed since they were last linked, and all the
         resulting cross-references will be resolved.
      }
      \sstsubsection{
         -d
      }{
         Specifies that a \qt{deep} dependency test should be performed to see
         which documents have changed, and therefore need re-linking. This
         involves examining the modification dates of all the \qt{.html} files in
         each document. A deep dependency test is more likely to identify
         changed documents than the default method (which simply examines the
         modification date of the \dirref{document directory} file) but it may take
         considerably longer.
      }
      \sstsubsection{
         -r
      }{
         Specifies that documents that make references only to other \remref{\qt{remote}}
         documents should be re-linked. This option is present for historical
         reasons and it should not normally be necessary to use it. It may
         eventually be removed.
      }
      \sstsubsection{
         -v
      }{
         Specifies \qt{verbose mode}, which results in additional
         information about the linking process being written to
         standard output.
      }
   }
   \sstnotes{
      Documents which are found on the
      \htmlref{\hi{HTX\_PATH} search path}{sect:searchpath} will be
      linked against (\st{i.e.}\ references to them will be resolved),
      but they will not themselves be modified unless they are also
      found in the directory list specified on the command line.
   }
\newpage
   \sstexamples{
      \sstexamplesubsection{
         hlink
      }{
         Re-links all documents in the current directory against
         themselves and all other documents found on the
         \hi{HTX\_PATH} search path. New \htmlref{index
         files}{sect:indexfiles} are first generated for any documents
         that have changed.
      }
      \sstexamplesubsection{
         hlink . mydocs
      }{
         Re-links all documents found in both the current directory
         and the \qt{mydocs} directory against themselves and all
         other documents found on the \hi{HTX\_PATH} search path. Only
         those documents residing in the \qt{.} and \qt{mydocs}
         directories will be modified by this command.
      }
      \sstexamplesubsection{
         hlink -d
      }{
         Re-links all documents in the current directory after first checking
         the modification dates of {\bf all} the \qt{.html} files they contain
         to determine which documents have changed since the last re-link.
      }
      \sstexamplesubsection{
         hlink -a dir1 dir2 dir3
      }{
         Generates new index files and re-links all documents in the specified
         directories regardless of whether they have changed since the last
         re-link.
      }
   }
   \sstdiytopic{
      Environment Variables Used
   }{
      \sstsubsection{
         \htmlref{\hi{HTX\_NOLINK}}{sect:suppressing}
      }{
         Setting this environment variable (to any value) will suppress
         re-linking. A message to this effect will be produced
         instead\begin{latexonly} (see
         \S\ref{sect:suppressing})\end{latexonly}.
      }
      \sstsubsection{
         \htmlref{\hi{HTX\_PATH}}{sect:searchpath}
      }{
         A colon-separated list of the library directories in which to
         search for hypertext documents\begin{latexonly} (see
         \S\ref{sect:searchpath})\end{latexonly}.
      }
      \sstsubsection{
         \htmlref{\hi{HTX\_SERVER}}{sect:specifyingaserver}
      }{
         The URL of the document server to be used for serving remote
         documents\begin{latexonly} (see
         \S\ref{sect:specifyingaserver})\end{latexonly}.
      }
   }
}
\newpage
\sstroutine{
   showme
}{
   Display a specified part of a document using a WWW browser
}{
   \sstdescription{
      This command takes the name of a
      \htmlref{document}{sect:hypertextdocuments} and (optionally) the
      name of a \htmlref{cross-reference label}{sect:targetlabels}
      within it, and displays the requested part of the document using
      a \htmlref{WWW browser}{sect:usingwwwbrowsers}. It may also be
      used to display local files or any other \wwwref{WWW} page for
      which a \urlref{URL} is available.

   }
   \sstinvocation{
      showme [switches] doc [label]
   }
   \sstdiytopic{
      Parameters
   }{
      \sstsubsection{
         doc
      }{
         The name of the document. If no directory information is supplied,
         the document will be located using the
         \hyperref{\hi{HTX\_PATH} search path}{\hi{HTX\_PATH} search
         path (see \S}{)}{sect:searchpath}. If directory information is
         given, no search will be performed. In either case, a
         \qt{.htx} extension to the document name is optional.

         If the \hi{-f} option is used, this parameter is interpreted
         as the name of a local file which is to be displayed. If the
         \hi{-u} option is used, it is interpreted as a general URL.
      }
      \sstsubsection{
         label
      }{
         An optional cross-reference label. If given, this specifies
         which part of the document is required. If omitted, it is
         assumed that the entire document is required and the
         appropriate \htmlref{\qt{top} page}{sect:entiredocuments} is
         displayed\begin{latexonly} (see
         \S\ref{sect:entiredocuments})\end{latexonly}.

         This parameter is ignored if the \hi{-f} or \hi{-u} options are used.
      }
   }
   \sstdiytopic{
      Switches
   }{
      \sstsubsection{
         -f
      }{
         Specifies that the document name supplied is the name of a
         local file (\st{e.g.}\ in \HTMLref{HTML} format) which is to
         be displayed. If this option is used, the \hi{label}
         parameter and the \hi{-l} and \hi{-r} switches are ignored.
      }
      \sstsubsection{
         -l
      }{
         Specifies that a local document should be displayed and that it
         should not be fetched from a remote document server. In this
         case, if the document cannot be found locally an error will result.
      }
      \sstsubsection{
         -n
      }{
         Specifies that the document is not to be displayed. In this case, no
         WWW browser will be used and the URL that would otherwise have been
         passed to it is simply written to standard output. This provides an
         interface for use by other software that will handle
         document display itself.

         Note that the URL generated is suitable only
         for local use by a WWW browser running on the same machine as the
         \hi{showme} command. It may not be suitable, for instance,
         for embedding in an HTML document that will be interpreted by a remote
         WWW browser. 
      }
      \sstsubsection{
         -r
      }{
         Specifies that a \htmlref{remote
         document}{sect:remotedocuments} is required. In this case, a
         reference to a remote document server will be generated and
         passed to the WWW browser, even if a local copy of the
         document exists.
      }
      \sstsubsection{
         -u
      }{
         Specifies that the document name supplied is a URL which is to be
         passed directly to the WWW browser for interpretation. Any form of URL
         which the browser can handle may be given. If this option is used, the
         \hi{label} parameter and the \hi{-l} and \hi{-r} switches are ignored.
      }
      \sstsubsection{
         -warn
      }{
         Indicates that any warning messages issued by the WWW browser
         (\st{e.g.}\ when it is started up) are to be suppressed. By default, these
         warnings are written to standard error along with any other warning
         or error messages.
      }
   }
   \sstnotes{
      \sstitemlist{
         \sstitem
         If the document is found locally, the local copy will be displayed.
         If it is not found locally (subject to any command switches) a reference
         to a remote document server will be passed to the WWW browser instead.
         \sstitem
         If a cross-reference label is specified for a local document but it
         cannot be found in that document, then an error will result.
         There is no checking of the existence of remote documents, nor of
         the cross-reference labels they contain. This checking must be performed
         by the remote document server.
         \sstitem
         The \hi{-l} and \hi{-r} switches are mutually exclusive. If both are given,
         the latter one predominates.
         Similarly, the \hi{-f} and \hi{-u} switches are mutually exclusive. If both
         are given, the latter one predominates.
      }
   }
   \sstexamples{
      \sstexamplesubsection{
         showme sun188
      }{
         Displays the document called
         \xref{\qt{sun188}}{sun188}{}. The local copy is used, if
         available, otherwise it is fetched from the remote document
         server.
      }
      \sstexamplesubsection{
         showme sun188 showme
      }{
         Displays the part of document \qt{sun188} identified by the
         cross-reference label \qt{showme} (\st{i.e.}\ the section you are
         reading now). As before, the local copy of the document is used, if
         possible.
      }
      \sstexamplesubsection{
         showme -l quantum-theory
      }{
         Searches for a local copy of the document \qt{quantum-theory} and
         displays it. If the document cannot be found locally, an error
         results.
      }
      \sstexamplesubsection{
         showme -r quantum-theory speculation
      }{
         Displays the section identified by the \qt{speculation}
         cross-reference label in a remote copy of the document called
         \qt{quantum-theory}. Any local copy is ignored.
      }
      \sstexamplesubsection{
         showme mydocs/help available\_commands
      }{
         Displays the section identified by the
         \qt{available\_commands} label in a local document stored in
         the directory \hi{mydocs/help.htx}.  Because explicit
         directory information is given, the document is not searched
         for using the \hi{HTX\_PATH} search path. If the document
         doesn't exist, a copy will be fetched from the remote
         document server.
      }
      \sstexamplesubsection{
         showme -n -l mydocs/help available\_commands
      }{
         Performs the same function as the previous example, except
         that the document is not actually displayed. Instead, its URL
         is simply written to standard output. The \hi{-l} switch
         specifies that a local document is required, so an error will
         result if it cannot be found.
      }
   }
   \sstdiytopic{
      Environment Variables Used
   }{
      \sstsubsection{
         \htmlref{\hi{HTX\_BROWSER}}{sect:specifyingabrowser}
      }{
         The command which will be used to invoke the \htmlref{WWW
         browser}{sect:usingwwwbrowsers}\begin{latexonly} (see
         \S\ref{sect:specifyingabrowser})\end{latexonly}.
      }
      \sstsubsection{
         \htmlref{\hi{HTX\_PATH}}{sect:searchpath}
      }{
         A colon-separated list of the library directories in which to
         search for hypertext documents\begin{latexonly} (see
         \S\ref{sect:searchpath})\end{latexonly}.
      }
      \sstsubsection{
         \htmlref{\hi{HTX\_SERVER}}{sect:specifyingaserver}
      }{
         The URL of the document server to be used for serving remote
         documents\begin{latexonly} (see
         \S\ref{sect:specifyingaserver})\end{latexonly}.
      }
      \sstsubsection{
         \htmlref{\hi{HTX\_TMP}}{sect:intermachinecommunication}
      }{
         The name of the directory in which to create temporary communication
         files. If this variable is not set or is null, then
         \hi{\$HOME/.htxtmp} is used instead\begin{latexonly} (see
         \S\ref{sect:intermachinecommunication})\end{latexonly}.
      }
   }
}
\newpage
\section{\label{sect:usingwwwbrowsers}\xlabel{using_WWW_browsers}USING WWW BROWSERS WITH HTX}

\subsection{\xlabel{specifying_a_WWW_browser}\xlabel{HTX_BROWSER_browser_command}\label{sect:specifyingabrowser}Specifying a WWW Browser}

\htmlref{HTX}{sect:HTX} commands that need to display
\htmlref{documents}{sect:hypertextdocuments} do so using a WWW browser
and will try to make use of one that you already have running, if
possible. Otherwise a new one will be started. By default, one of the
browsers in the list \st{firefox}, \st{mozilla}, \st{netscape},
and \st{mosaic} will be used (in that priority order). On Mac OS X
the default browser will be used.

You can change the browser that is used by means of the HTX\_BROWSER
environment variable. Thus, if you normally use the command \qt{mozilla} to
invoke your browser (and also have \qt{firefox} installed), you could put the
following into your \hi{.login} file:

\begin{quote}
\begin{verbatim}
setenv HTX_BROWSER mozilla
\end{verbatim}
\end{quote}

Note that this command should not be an alias (\st{e.g.}\ if you
normally use the C~shell).

Currently, HTX can work only with the browsers listed above.

\subsection{\xlabel{defining_your_X_display}Defining Your X~Display}

In order to run a \htmlref{WWW browser}{sect:usingwwwbrowsers} on an
X~windows display, the environment variable \hi{DISPLAY} should first
be defined to identify the display you want to
use. \htmlref{HTX}{sect:HTX} only supports WWW browsers which use
\st{X~displays}.

If you are a regular user of an X~display, you will probably already
have the necessary command in your \hi{.login} file. If not, you may
want to use the Starlink \hi{xdisplay} command (described in
\xref{SUN/129}{sun129}{}) to make the appropriate definition.

If HTX detects that the \hi{DISPLAY} variable has not been defined, it
will attempt to run the \hi{xdisplay} utility itself before starting a
WWW browser. For this to succeed, \hi{xdisplay} must be installed on
your system and the directory \hi{/star/bin}\footnote{Or the
equivalent if your Starlink software is not installed under the
standard \hi{/star} directory.} must be on your UNIX \hi{PATH}. If the
\hi{xdisplay} command cannot be found (and the \hi{DISPLAY}
environment variable is undefined), then an error will result.

Because the \hi{xdisplay} command may take some time to complete, you
will be warned if this must be repeated by each subsequent HTX
command (not all WWW browsers require this), as it will slow down your
access to documents. Defining the \hi{DISPLAY} environment variable
yourself will eliminate this delay.

The DISPLAY variable is not required on Mac OS X. HTX does not support
X-based WWW browsers on OS X.

\subsection{\label{sect:browsercommunication}\xlabel{WWW_browser_communication}WWW Browser Communication}

\htmlref{HTX}{sect:HTX} communicates with your \htmlref{WWW
browser}{sect:usingwwwbrowsers} using a \st{remote control} mechanism
which is different for each browser. 

In general, if you use only one machine at a time and never start up
more than one copy of your browser, you are unlikely to have
problems. However, if you use multiple machines or multiple browsers,
the following describes some of the browser-specific behaviour you may
need to be aware of.

\subsubsection{Netscape, Mozilla and Firefox Behaviour}

\htmlref{HTX}{sect:HTX} communicates with these browsers using
the remote control commands, which work by means of \st{X~resources}, in
essence looking for an existing window on your X~display. If there is more
than one of these, the first one will generally be used. If it cannot find
such a window, HTX will start a new instance.

One problem with this arrangement is that HTX may detect an instance running
on a different machine but displaying on the same X~display. In this case,
attempts to communicate with it will succeed, but the browser may not
necessarily be able to access the local files that HTX asks it to view (but
see~\S\ref{sect:intermachinecommunication}). The best way around this problem
is to ensure that the browser to be used by HTX is started first.

\subsection{\label{sect:intermachinecommunication}\xlabel{inter_machine_browser_communication}Inter-Machine Browser Communication}

Often, your file system will be shared by more than one machine so
that communication between different machines is possible via
files. Such an arrangement provides an opportunity for
\htmlref{HTX}{sect:HTX} to pass information to WWW browsers which are
already running on other machines. With the
\netscaperef{\st{Netscape}} family of browsers, this can make working on
multiple machines considerably easier by reducing the number of browser
invocations you require. For it to work successfully, however, the
files you are accessing must be known by identical names on all the
machines involved.

If your file system is suitably set up, then HTX will normally be able
to take advantage of this without any further action on your part. To
achieve this, the temporary files which HTX creates to communicate
with the browser will be placed in the directory
\hi{\$HOME/.htxtmp}\footnote{This directory is created automatically
when needed.}. Therefore, if you use the same login directory (and
\hi{\$HOME} translates to the same name) on each machine, these
temporary files will be accessible to the browser wherever it is
running.

If your login directory is not the same on each machine, then you can
set the environment variable \hi{HTX\_TMP} to give the name of a
suitable alternative directory to hold these temporary communication
files. The directory you use should be accessible (for both reading
and writing) and be known by identical names on all the machines which
will use it.

\end{document}
