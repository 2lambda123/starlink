\documentstyle[11pt]{article}
\pagestyle{myheadings}

% -----------------------------------------------------------------------------
% ? Document identification
\newcommand{\stardoccategory}  {Starlink User Note}
\newcommand{\stardocinitials}  {SUN}
\newcommand{\stardocsource}    {sun\stardocnumber}
\newcommand{\stardocnumber}    {185.2}
\newcommand{\stardocauthors}   {A J Chipperfield \\
                                B K McIlwrath}
\newcommand{\stardocdate}      {4 November 1994}
\newcommand{\stardoctitle}     {MESSGEN\\ [\latex{2ex}]
                                Starlink Facility Error Message Generation}
\newcommand{\stardocversion}   {Version 1.2}
\newcommand{\stardocmanual}    {}
\newcommand{\stardocabstract}  {This document describes the Starlink 
MESSGEN program which creates files
required for proper operation of the system for associating messages with
Starlink facility status values on UNIX platforms. It should be read by 
authors of Starlink subroutine libraries.

The utility program CREMSG is also described. It can be used to produce
source files for MESSGEN.}

% ? End of document identification

% -----------------------------------------------------------------------------

\newcommand{\stardocname}{\stardocinitials /\stardocnumber}
\markright{\stardocname}
\setlength{\textwidth}{160mm}
\setlength{\textheight}{230mm}
\setlength{\topmargin}{-2mm}
\setlength{\oddsidemargin}{0mm}
\setlength{\evensidemargin}{0mm}
\setlength{\parindent}{0mm}
\setlength{\parskip}{\medskipamount}
\setlength{\unitlength}{1mm}

% -----------------------------------------------------------------------------
%  Hypertext definitions.
%  ======================
%  These are used by the LaTeX2HTML translator in conjunction with star2html.

%  Comment.sty: version 2.0, 19 June 1992
%  Selectively in/exclude pieces of text.
%
%  Author
%    Victor Eijkhout                                      <eijkhout@cs.utk.edu>
%    Department of Computer Science
%    University Tennessee at Knoxville
%    104 Ayres Hall
%    Knoxville, TN 37996
%    USA

%  Do not remove the %begin{latexonly} and %end{latexonly} lines (used by 
%  star2html to signify raw TeX that latex2html cannot process).
%begin{latexonly}
\makeatletter
\def\makeinnocent#1{\catcode`#1=12 }
\def\csarg#1#2{\expandafter#1\csname#2\endcsname}

\def\ThrowAwayComment#1{\begingroup
    \def\CurrentComment{#1}%
    \let\do\makeinnocent \dospecials
    \makeinnocent\^^L% and whatever other special cases
    \endlinechar`\^^M \catcode`\^^M=12 \xComment}
{\catcode`\^^M=12 \endlinechar=-1 %
 \gdef\xComment#1^^M{\def\test{#1}
      \csarg\ifx{PlainEnd\CurrentComment Test}\test
          \let\html@next\endgroup
      \else \csarg\ifx{LaLaEnd\CurrentComment Test}\test
            \edef\html@next{\endgroup\noexpand\end{\CurrentComment}}
      \else \let\html@next\xComment
      \fi \fi \html@next}
}
\makeatother

\def\includecomment
 #1{\expandafter\def\csname#1\endcsname{}%
    \expandafter\def\csname end#1\endcsname{}}
\def\excludecomment
 #1{\expandafter\def\csname#1\endcsname{\ThrowAwayComment{#1}}%
    {\escapechar=-1\relax
     \csarg\xdef{PlainEnd#1Test}{\string\\end#1}%
     \csarg\xdef{LaLaEnd#1Test}{\string\\end\string\{#1\string\}}%
    }}

%  Define environments that ignore their contents.
\excludecomment{comment}
\excludecomment{rawhtml}
\excludecomment{htmlonly}

%  Hypertext commands etc. This is a condensed version of the html.sty
%  file supplied with LaTeX2HTML by: Nikos Drakos <nikos@cbl.leeds.ac.uk> &
%  Jelle van Zeijl <jvzeijl@isou17.estec.esa.nl>. The LaTeX2HTML documentation
%  should be consulted about all commands (and the environments defined above)
%  except \xref and \xlabel which are Starlink specific.

\newcommand{\htmladdnormallinkfoot}[2]{#1\footnote{#2}}
\newcommand{\htmladdnormallink}[2]{#1}
\newcommand{\htmladdimg}[1]{}
\newenvironment{latexonly}{}{}
\newcommand{\hyperref}[4]{#2\ref{#4}#3}
\newcommand{\htmlref}[2]{#1}
\newcommand{\htmlimage}[1]{}
\newcommand{\htmladdtonavigation}[1]{}
\newcommand{\latexhtml}[2]{#1}

%  Starlink cross-references and labels.
\newcommand{\xref}[3]{#1}
\newcommand{\xlabel}[1]{}

%  LaTeX2HTML symbol.
\newcommand{\latextohtml}{{\bf LaTeX}{2}{\tt{HTML}}}

%  Define command to re-centre underscore for Latex and leave as normal
%  for HTML (severe problems with \_ in tabbing environments and \_\_
%  generally otherwise).
\newcommand{\latex}[1]{#1}
\newcommand{\setunderscore}{\renewcommand{\_}{{\tt\symbol{95}}}}
\latex{\setunderscore}

%  Redefine the \tableofcontents command. This procrastination is necessary 
%  to stop the automatic creation of a second table of contents page
%  by latex2html.
\newcommand{\latexonlytoc}[0]{\tableofcontents}

% -----------------------------------------------------------------------------
%  Debugging.
%  =========
%  Remove % on the following to debug links in the HTML version using Latex.

% \newcommand{\hotlink}[2]{\fbox{\begin{tabular}[t]{@{}c@{}}#1\\\hline{\footnotesize #2}\end{tabular}}}
% \renewcommand{\htmladdnormallinkfoot}[2]{\hotlink{#1}{#2}}
% \renewcommand{\htmladdnormallink}[2]{\hotlink{#1}{#2}}
% \renewcommand{\hyperref}[4]{\hotlink{#1}{\S\ref{#4}}}
% \renewcommand{\htmlref}[2]{\hotlink{#1}{\S\ref{#2}}}
% \renewcommand{\xref}[3]{\hotlink{#1}{#2 -- #3}}
%end{latexonly}
% -----------------------------------------------------------------------------
% ? Document specific \newcommand or \newenvironment commands.
% ? End of document specific commands
% -----------------------------------------------------------------------------
%  Title Page.
%  ===========
\renewcommand{\thepage}{\roman{page}}
\begin{document}
\thispagestyle{empty}

%  Latex document header.
%  ======================
\begin{latexonly}
   CCLRC / {\sc Rutherford Appleton Laboratory} \hfill {\bf \stardocname}\\
   {\large Particle Physics \& Astronomy Research Council}\\
   {\large Starlink Project\\}
   {\large \stardoccategory\ \stardocnumber}
   \begin{flushright}
   \stardocauthors\\
   \stardocdate
   \end{flushright}
   \vspace{-4mm}
   \rule{\textwidth}{0.5mm}
   \vspace{5mm}
   \begin{center}
   {\Huge\bf  \stardoctitle \\ [2.5ex]}
   {\LARGE\bf \stardocversion \\ [4ex]}
   {\Huge\bf  \stardocmanual}
   \end{center}
   \vspace{5mm}

% ? Heading for abstract if used.
   \vspace{10mm}
   \begin{center}
      {\Large\bf Abstract}
   \end{center}
% ? End of heading for abstract.
\end{latexonly}

%  HTML documentation header.
%  ==========================
\begin{htmlonly}
   \xlabel{}
   \begin{rawhtml} <H1> \end{rawhtml}
      \stardoctitle\\
      \stardocversion\\
      \stardocmanual
   \begin{rawhtml} </H1> \end{rawhtml}

% ? Add picture here if required.
% ? End of picture

   \begin{rawhtml} <P> <I> \end{rawhtml}
   \stardoccategory \stardocnumber \\
   \stardocauthors \\
   \stardocdate
   \begin{rawhtml} </I> </P> <H3> \end{rawhtml}
      \htmladdnormallink{CCLRC}{http://www.cclrc.ac.uk} /
      \htmladdnormallink{Rutherford Appleton Laboratory}
                        {http://www.cclrc.ac.uk/ral} \\
      \htmladdnormallink{Particle Physics \& Astronomy Research Council}
                        {http://www.pparc.ac.uk} \\
   \begin{rawhtml} </H3> <H2> \end{rawhtml}
      \htmladdnormallink{Starlink Project}{http://star-www.rl.ac.uk/}
   \begin{rawhtml} </H2> \end{rawhtml}
   \htmladdnormallink{\htmladdimg{source.gif} Retrieve hardcopy}
      {http://star-www.rl.ac.uk/cgi-bin/hcserver?\stardocsource}\\

%  HTML document table of contents. 
%  ================================
%  Add table of contents header and a navigation button to return to this 
%  point in the document (this should always go before the abstract \section). 
  \label{stardoccontents}
  \begin{rawhtml} 
    <HR>
    <H2>Contents</H2>
  \end{rawhtml}
  \renewcommand{\latexonlytoc}[0]{}
  \htmladdtonavigation{\htmlref{\htmladdimg{contents_motif.gif}}
        {stardoccontents}}

% ? New section for abstract if used.
  \section{\xlabel{abstract}Abstract}
% ? End of new section for abstract
\end{htmlonly}

% -----------------------------------------------------------------------------
% ? Document Abstract. (if used)
%  ==================
\stardocabstract
% ? End of document abstract
% -----------------------------------------------------------------------------
% ? Latex document Table of Contents (if used).
%  ===========================================
 \newpage
 \begin{latexonly}
   \setlength{\parskip}{0mm}
   \latexonlytoc
   \setlength{\parskip}{\medskipamount}
   \markright{\stardocname}
 \end{latexonly}
% ? End of Latex document table of contents
% -----------------------------------------------------------------------------
\newpage
\renewcommand{\thepage}{\arabic{page}}
\setcounter{page}{1}

\section {\xlabel{introduction}Introduction}
The Starlink MESSGEN facility is a system devised and written by Brian 
McIlwrath to enable error messages to be associated with Starlink STATUS 
values on UNIX systems.
This feature is available with the VAX/VMS error system and has proved useful.

MESSGEN can create a Fortran INCLUDE file and a C header file defining
symbolic names for the status values and a {\em facility message file}
which is used by the
\xref{Starlink Error Message Service (EMS)}{ssn4}{} to associate 
messages with the status values. 

A utility program CREMSG is also provided in the MESSGEN release. It may be 
used to create a MESSGEN source file from an existing, standard-format Fortran
error INCLUDE file.

\section{\xlabel{messgen_source_files}MESSGEN Source Files}
The starting point for the Starlink MESSGEN facility is a source file 
conventionally named {\em facility}{\bf \_err.msg}. The format is very
similar to that for the VAX/VMS MESSAGE facility. This is most easily
described by an example:
\begin{small}
\begin{quote} \begin{verbatim}
.TITLE          PAR             ADAM/SSE parameter routines
.FACILITY       PAR,190/PREFIX=PAR__
.IDENT          'Version 0.5'
!author         REVAD:BDK 18Nov84 (B.D.Kelly)

.SEVERITY       INFORMATIONAL
.BASE 100

NOUSR           <No user available for input>
PARNF           <Parameter does not exist>
EOL             <End of parameter list>
NULL            <Null parameter value>
ABORT           <Parameter request aborted>

.END
\end{verbatim} \end{quote}
\end{small}
Notes:
\begin{enumerate}
\item Lines may be up to 100 characters long but may not be continued to the
next line.
\item ! indicates that the remainder
of the line is comment, except on a .TITLE line.
\item The .TITLE directive is optional but recommended. The given text is 
displayed as the file is read by MESSGEN.
\item The .FACILITY directive specifies the facility name, the facility number
(which is allocated by the Starlink Infrastructure Support Group) and the
prefix to be used for symbolic constants for the status values. 
In the example, the facility name is PAR, the facility number is 190 and the
symbolic constant for the first error value will be PAR\_\_NOUSR.
\item The .IDENT directive is optional and has no effect.
\item The .SEVERITY directive defines the severity of the error. In addition
to INFORMATIONAL, any of the values permitted by the VAX/VMS MESSAGE facility
are allowed.
The severity value has no meaning on UNIX, or in most cases on VAX/VMS, but 
it does affect the error value generated for subsequent errors in the file. 
There may be more than one .SEVERITY directive in the file or it may be omitted,
in which case ERROR is assumed.
\item The .BASE directive specifies a message number to be associated with 
the next message and is optional. 
In the absence of any .BASE directives, message numbering is sequential
(from 1) within each facility.
\item Message lines specify the ident of the message and the text 
(enclosed in \verb!<>!).
\item 
\xref{SUN/104}{sun104}{calculating_globally_unique_error_codes} 
gives a formula for calculating unique error numbers. This formula
corresponds with .SEVERITY ERROR. 
\item The .END directive is optional and has no effect.
\item If the VAX/VMS MESSAGE facility was used to generate facility INCLUDE
files which were subsequently ported to UNIX, the original 
.MSG source files should be used to generate the error files for UNIX. This
will ensure that the same error values are generated.
\end{enumerate}

\section{\xlabel{running_messgen}Running MESSGEN}
To produce the required files from MESSGEN source files, on UNIX, type:
\begin{quote} 
\verb!% messgen -[cfFev]! {\em files}
\end{quote}
where {\em files} may be any number of MESSGEN source files and the options
 may be one or more of:
\begin{quote}
\begin{description}
\item[-c] Generate a C header file ({\em facility}\_err.h).
\item[-f] Generate a Fortran INCLUDE file ({\em facility}\_err).
\item[-F] As for option -f, but with an uppercased output filename
          ({\em FACILITY}\_ERR).
\item[-e] Generate a facility message file (fac\_{\em facnum}\_err).
\item[-v] Output diagnostic information.
\item[-V] Output the names of generated files to {\tt stdout}.
\end{description}
where {\em facility} is the facility name, {\em e.g.}\ PAR, and {\em facnum} is
the facility number, {\em e.g.}\ 190 for PAR.
\end{quote}

\section{\xlabel{running_cremsg}Running CREMSG}
If a MESSGEN source file is not available but there is an existing, 
standard-format Fortran error INCLUDE file, program CREMSG can create a MESSGEN
source file from it. Standard-format means that the error numbers are defined
by statements of the form:
\begin{quote} \begin{verbatim}
      INTEGER PAR__NOUSR
      PARAMETER ( PAR__NOUSR = 146703139 )
*   No user available for input
 
      INTEGER PAR__PARNF
      PARAMETER ( PAR__PARNF = 146703147 )
*   Parameter does not exist

    ... etc ...
\end{verbatim} \end{quote}
{\em i.e.} the format produced by the VAX/VMS MESSAGE/ERRGEN system.

To produce the corresponding MESSGEN source file, type:
\begin{quote}
\verb!% cremsg [-v]! {\em filename}
\end{quote}
where {\em filename} is the name of the INCLUDE file and option \verb!-v! may
be used to obtain diagnostic information.

Notes:
\begin{enumerate}
\item The file produced will be named {\em fac}{\bf \_err.msg} file, where 
{\em fac} is the lower case of the facility name determined from the error 
idents in the INCLUDE file ({\bf par} in the above example). This will not
always give the correct name.
\item The .TITLE directive produced will contain only the facility name. It
may be desirable to add a short description by hand.
\item CREMSG will ignore any initial heading comment lines in the INCLUDE file.
\item CREMSG will accept the message comment line either first or last in each
definition sequence in the INCLUDE file but a blank line must be present 
between each sequence.
\end{enumerate}

\section{\xlabel{using_the_facility_message_files}Using the Facility Message Files}
Normally the facility message file for each Starlink library will be installed
in {\bf /star/help} when the library is installed. It is recommended that the
message file and error include files are included in the source tar file rather
than building them from the MESSGEN source file during the {\bf make build}.
This will avoid an unnecessary dependency.

The internal EMS source file {\bf ems1\_fcerr.c} contains C code for obtaining 
the message given the status value -- this may be copied by other systems 
requiring to do the same job on UNIX platforms but note that the system is not
fully portable and Starlink reserve the right to alter it. 
See 
\xref{SSN/4}{ssn4}{operating_system_specific_routines}
(Appendix on Portability) for a description of how the routine operates on UNIX.

% Should the following not be subsections of a \section{Changes} section?
\section{Changes in Version 1.1}
\begin{itemize}
\item The CREMSG utility is added (Section 4).
\item The first issue of this document gave the wrong .SEVERITY value 
corresponding with the formula in SUN/104 -- this has been corrected
(Section 2 Note 9).
\end{itemize}

\section{Changes in Version 1.1-1}
\begin{itemize}
\item Generated C header files now contains a conditional statement to prevent 
repeat definitions.
\item A hypertext version of this document is released. Apart from the 
addition of this section, there are no changes to the text.
\end{itemize}

\section{Changes in Version 1.2}
Added the -F option to generate Fortran output files with uppercase filenames.
\end {document}
