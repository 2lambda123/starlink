\documentstyle[twoside,11pt]{article}
\pagestyle{myheadings}

% -----------------------------------------------------------------------------
% ? Document identification
\newcommand{\stardoccategory}  {Starlink User Note}
\newcommand{\stardocinitials}  {SUN}
\newcommand{\stardocsource}    {sun\stardocnumber}
\newcommand{\stardocnumber}    {219.1}
\newcommand{\stardocauthors}   {M.\, J.\, Bly}
\newcommand{\stardocdate}      {5 June 1997}
\newcommand{\stardoctitle}     {The GWM Xdisplay Widget}
\newcommand{\stardocversion}   {[software-version]}
\newcommand{\stardocmanual}    {User's Guide}
\newcommand{\stardocabstract}  {%
The GWM Xdisplay widget is a Tcl/Tk-based replacement for the standard GWM
Xdisplay manager.  It provides several facilities not available in the 
standard display manager, including manipulation of the colour table of the
display, capture of the displayed plot or image to JPEG format, and dumping
to print files of various formats. 
}
% ? End of document identification

% -----------------------------------------------------------------------------

\newcommand{\stardocname}{\stardocinitials /\stardocnumber}
\markboth{\stardocname}{\stardocname}
\setlength{\textwidth}{160mm}
\setlength{\textheight}{230mm}
\setlength{\topmargin}{-2mm}
\setlength{\oddsidemargin}{0mm}
\setlength{\evensidemargin}{0mm}
\setlength{\parindent}{0mm}
\setlength{\parskip}{\medskipamount}
\setlength{\unitlength}{1mm}

% -----------------------------------------------------------------------------
%  Hypertext definitions.
%  ======================
%  These are used by the LaTeX2HTML translator in conjunction with star2html.

%  Comment.sty: version 2.0, 19 June 1992
%  Selectively in/exclude pieces of text.
%
%  Author
%    Victor Eijkhout                                      <eijkhout@cs.utk.edu>
%    Department of Computer Science
%    University Tennessee at Knoxville
%    104 Ayres Hall
%    Knoxville, TN 37996
%    USA

%  Do not remove the %begin{latexonly} and %end{latexonly} lines (used by 
%  star2html to signify raw TeX that latex2html cannot process).
%begin{latexonly}
\makeatletter
\def\makeinnocent#1{\catcode`#1=12 }
\def\csarg#1#2{\expandafter#1\csname#2\endcsname}

\def\ThrowAwayComment#1{\begingroup
    \def\CurrentComment{#1}%
    \let\do\makeinnocent \dospecials
    \makeinnocent\^^L% and whatever other special cases
    \endlinechar`\^^M \catcode`\^^M=12 \xComment}
{\catcode`\^^M=12 \endlinechar=-1 %
 \gdef\xComment#1^^M{\def\test{#1}
      \csarg\ifx{PlainEnd\CurrentComment Test}\test
          \let\html@next\endgroup
      \else \csarg\ifx{LaLaEnd\CurrentComment Test}\test
            \edef\html@next{\endgroup\noexpand\end{\CurrentComment}}
      \else \let\html@next\xComment
      \fi \fi \html@next}
}
\makeatother

\def\includecomment
 #1{\expandafter\def\csname#1\endcsname{}%
    \expandafter\def\csname end#1\endcsname{}}
\def\excludecomment
 #1{\expandafter\def\csname#1\endcsname{\ThrowAwayComment{#1}}%
    {\escapechar=-1\relax
     \csarg\xdef{PlainEnd#1Test}{\string\\end#1}%
     \csarg\xdef{LaLaEnd#1Test}{\string\\end\string\{#1\string\}}%
    }}

%  Define environments that ignore their contents.
\excludecomment{comment}
\excludecomment{rawhtml}
\excludecomment{htmlonly}

%  Hypertext commands etc. This is a condensed version of the html.sty
%  file supplied with LaTeX2HTML by: Nikos Drakos <nikos@cbl.leeds.ac.uk> &
%  Jelle van Zeijl <jvzeijl@isou17.estec.esa.nl>. The LaTeX2HTML documentation
%  should be consulted about all commands (and the environments defined above)
%  except \xref and \xlabel which are Starlink specific.

\newcommand{\htmladdnormallinkfoot}[2]{#1\footnote{#2}}
\newcommand{\htmladdnormallink}[2]{#1}
\newcommand{\htmladdimg}[1]{}
\newenvironment{latexonly}{}{}
\newcommand{\hyperref}[4]{#2\ref{#4}#3}
\newcommand{\htmlref}[2]{#1}
\newcommand{\htmlimage}[1]{}
\newcommand{\htmladdtonavigation}[1]{}
\newcommand{\latexhtml}[2]{#1}
\newcommand{\html}[1]{}

%  Starlink cross-references and labels.
\newcommand{\xref}[3]{#1}
\newcommand{\xlabel}[1]{}

%  LaTeX2HTML symbol.
\newcommand{\latextohtml}{{\bf LaTeX}{2}{\tt{HTML}}}

%  Define command to re-centre underscore for Latex and leave as normal
%  for HTML (severe problems with \_ in tabbing environments and \_\_
%  generally otherwise).
\newcommand{\latex}[1]{#1}
\newcommand{\setunderscore}{\renewcommand{\_}{{\tt\symbol{95}}}}
\latex{\setunderscore}

%  Redefine the \tableofcontents command. This procrastination is necessary 
%  to stop the automatic creation of a second table of contents page
%  by latex2html.
\newcommand{\latexonlytoc}[0]{\tableofcontents}

% -----------------------------------------------------------------------------
%  Debugging.
%  =========
%  Remove % on the following to debug links in the HTML version using Latex.

% \newcommand{\hotlink}[2]{\fbox{\begin{tabular}[t]{@{}c@{}}#1\\\hline{\footnotesize #2}\end{tabular}}}
% \renewcommand{\htmladdnormallinkfoot}[2]{\hotlink{#1}{#2}}
% \renewcommand{\htmladdnormallink}[2]{\hotlink{#1}{#2}}
% \renewcommand{\hyperref}[4]{\hotlink{#1}{\S\ref{#4}}}
% \renewcommand{\htmlref}[2]{\hotlink{#1}{\S\ref{#2}}}
% \renewcommand{\xref}[3]{\hotlink{#1}{#2 -- #3}}
%end{latexonly}
% -----------------------------------------------------------------------------
% ? Document specific \newcommand or \newenvironment commands.
% ? End of document specific commands
% -----------------------------------------------------------------------------
%  Title Page.
%  ===========
\renewcommand{\thepage}{\roman{page}}
\begin{document}
\thispagestyle{empty}

%  Latex document header.
%  ======================
\begin{latexonly}
   CCLRC / {\sc Rutherford Appleton Laboratory} \hfill {\bf \stardocname}\\
   {\large Particle Physics \& Astronomy Research Council}\\
   {\large Starlink Project\\}
   {\large \stardoccategory\ \stardocnumber}
   \begin{flushright}
   \stardocauthors\\
   \stardocdate
   \end{flushright}
   \vspace{-4mm}
   \rule{\textwidth}{0.5mm}
   \vspace{5mm}
   \begin{center}
   {\Huge\bf  \stardoctitle \\ [2.5ex]}
%  {\LARGE\bf \stardocversion \\ [4ex]}
   {\Huge\bf  \stardocmanual}
   \end{center}
   \vspace{5mm}

% ? Heading for abstract if used.
   \vspace{10mm}
   \begin{center}
      {\Large\bf Abstract}
   \end{center}
% ? End of heading for abstract.
\end{latexonly}

%  HTML documentation header.
%  ==========================
\begin{htmlonly}
   \xlabel{}
   \begin{rawhtml} <H1> \end{rawhtml}
      \stardoctitle\\
%     \stardocversion\\
      \stardocmanual
   \begin{rawhtml} </H1> \end{rawhtml}

% ? Add picture here if required.
% ? End of picture

   \begin{rawhtml} <P> <I> \end{rawhtml}
   \stardoccategory\ \stardocnumber \\
   \stardocauthors \\
   \stardocdate
   \begin{rawhtml} </I> </P> <H3> \end{rawhtml}
      \htmladdnormallink{CCLRC}{http://www.cclrc.ac.uk} /
      \htmladdnormallink{Rutherford Appleton Laboratory}
                        {http://www.cclrc.ac.uk/ral} \\
      \htmladdnormallink{Particle Physics \& Astronomy Research Council}
                        {http://www.pparc.ac.uk} \\
   \begin{rawhtml} </H3> <H2> \end{rawhtml}
      \htmladdnormallink{Starlink Project}{http://star-www.rl.ac.uk/}
   \begin{rawhtml} </H2> \end{rawhtml}
   \htmladdnormallink{\htmladdimg{source.gif} Retrieve hardcopy}
      {http://star-www.rl.ac.uk/cgi-bin/hcserver?\stardocsource}\\

%  HTML document table of contents. 
%  ================================
%  Add table of contents header and a navigation button to return to this 
%  point in the document (this should always go before the abstract \section). 
  \label{stardoccontents}
  \begin{rawhtml} 
    <HR>
    <H2>Contents</H2>
  \end{rawhtml}
  \newcommand{\latexonlytoc}[0]{}
  \htmladdtonavigation{\htmlref{\htmladdimg{contents_motif.gif}}
        {stardoccontents}}

% ? New section for abstract if used.
  \section{\xlabel{abstract}Abstract}
% ? End of new section for abstract
\end{htmlonly}

% -----------------------------------------------------------------------------
% ? Document Abstract. (if used)
%  ==================
\stardocabstract
% ? End of document abstract
% -----------------------------------------------------------------------------
% ? Latex document Table of Contents (if used).
%  ===========================================
 \newpage
 \begin{latexonly}
   \setlength{\parskip}{0mm}
   \latexonlytoc
   \setlength{\parskip}{\medskipamount}
   \markboth{\stardocname}{\stardocname}
 \end{latexonly}
% ? End of Latex document table of contents
% -----------------------------------------------------------------------------
\cleardoublepage
\renewcommand{\thepage}{\arabic{page}}
\setcounter{page}{1}

\section{\label{introduction}\xlabel{introduction}Introduction}

The TCL/GWM image display and plotting widget\footnote{A widget is the 
name given to this type of application in the TCL system.  Any resemblance
to the widgets found in beer cans is a figment of the imagination.} is a 
alternative Xdisplay manager, to be used in place of the standard
GWM Xdisplay manager.

By using TCL to develop the display manager, quite small amounts of code 
can be used to generate quite powerful systems, and it is relatively easy 
to extend to provide new features.  This version is only a demonstration of 
what can be achieved.  Users may make copies of the relevant code and 
extend it if they so wish.

Full details on the TCL/GWM widget and the standard GWM Xdisplay
manager are give in \xref{SUN/186}{sun186}{} (STARTCL) and
\xref{SUN/130}{sun130}{} (GWM) respectively.

\section{\label{enabling_the_widget}\xlabel{enabling_the_widget}Enabling the Widget}

The TCL/GWM widget is the default display manager in use on Starlink
systems, so just logging in and initialising the Starlink Software should
enable you to use the widget.

It lives in the directory {\tt{/star/bin/startcl}} on most Starlink systems,
whereas the older standard display manager lives in {\tt{/star/bin}}.  If you
have {\tt{/star/bin/startcl}} in your {\tt{PATH}} before {\tt{/star/bin}},
you should get the TCL/GWM widget.  This is setup by the Starlink
initialisation on Starlink systems.

If you want to use the older display manager, you should put {\tt{/star/bin}}
in you {\tt{PATH}} before {\tt{/star/bin/startcl}}, or ensure that
{\tt{/star/bin/startcl}} is NOT in your {\tt{PATH}}.

\section{\label{using_the_widget}\xlabel{using_the_widget}Using the Widget}

If you have the TCL/GWM widget enabled it will be used when any `xwindows' 
output is produced by a Starlink graphics program, whether the requirement
is to display a colour image or a simple line plot.  

The TCL/GWM widget can be invoked by the standard graphics system and can
be generated with an overlay if required, in the same manner as the
standard GWM `xwindows' display.  Those graphics programs that automatically
generate a display `xwindows' can use the TCL/GWM widget if it is available.

A typical TCL/GWM widget might be generated thus:

\begin{quote}
{\tt \% xmake -colours 128 xwindows}
\end{quote}

where {\tt{\%}} is the shell prompt.  This will generate a widget with 
128 colours allocated called {\tt{xwindows}}

If you need an overlay plane ({\em{e.g.}}\/ to plot contour maps over images)
you can use the {\tt{-overlay}} switch:

\begin{quote}
{\tt \% xmake -colours 128 -overlay xwindows}
\end{quote}

A full list of the TCL/GWM widget command switches is given in 
\xref{SUN/186}{sun186}{gwm_widget}.

\subsection{\label{the_main_display_window}\xlabel{the_main_display_window}The  main display window}

The main window of the TCL/GWM widget is by default the standard GWM
`xwindows' size with buttons along the bottom which activate the
facilities available.  Along the bottom and down the right side are the
scroll bars (a widget with an overlay plane has two sets of scroll
bars, the inner set for the overlay and the outer set for the main
plotting surface.)

In order across the bottom, the buttons are:

\begin{description}

\item[{\tt{Exit}}] Press this and the widget will exit.

\item[{\tt{Colours}}] Displays the {\sf{Gwm Colours}} dialog box that
allows you to manipulate the colours on the plotting surface of the
widget, including the foreground, background, overlay and crosshair
cursor.

\item[{\tt{Clear}}] Wipes the plotting surface clean.

\item[{\tt{Clear Overlay}}] Wipes the overlay surface clean.  This button is
only present if you start the TCL/GWM widget with an overlay.

\item[{\tt{Capture}}] Displays the {\sf{Gwm JPEG}} dialog box that
allows you to dump the current plot to a JPEG file.

\item[{\tt{Print}}] Displays the {\sf{Gwm Print}} dialog box that
allows you to dump the current plot to a print file in a number of
formats, including PostScript.

\item[{\tt{Crosshair}}] Toggles the appearance of the crosshair cursor.

\item[{\tt{Help}}] Displays the help text.  The help text is
essentially the same information provided in this document.

\end{description}

\subsection{\label{colour_manipulation}\xlabel{colour_manipulation}Colour manipulation}

The {\sf{Gwm Colours}} dialog box provides a mechanism which allows you to
alter the colours on the plotting surface.  You can change the colours
using the Red, Green and Blue sliders, and you can select which
colour or pixel value to alter by selecting the appropriate button.

\begin{description}

\item[Checkbuttons]~

\begin{description}

\item[{\tt{Background:}}] Colour (pixel value) 0 -- this is the
background colour of the plotting surface.

\item[{\tt{Foreground:}}] Colour (pixel value) 1 -- this is the
foreground colour of the plotting surface.

\item[{\tt{Other:}}] This option toggles the {\tt{Pixel~value}} slider which
enables you to select any of the colours or pixel values to change,
including 0 (background), 1 (foreground) and the overlay (if you have
one enabled).

\item[{\tt{Overlay:}}] If you have generated a TCL/GWM widget with an
overlay for plotting contours on an image, you can change the colour
used for the overlay by selecting the {\tt{Overlay}} button.

\item[{\tt{Crosshair:}}] This button allows you to modify the colour
used to display the crosshair cursor.  You don't have to have the
crosshair cursor enabled to set the colour.

\end{description}

\item[Buttons]~

\begin{description}

\item[{\tt{OK:}}] Closes the {\sf{Gwm Colours}} dialogue box.

\item[{\tt{Help:}}] Displays this help text.

\end{description}

\item[Sliders]~

\begin{description}

\item[{\tt{Pixel~value:}}] When you have the {\tt{Other}} button selected,
this slider selects which colour or pixel value is to be modified.

\item[{\tt{Red, Green, Blue:}}] These control the intensity of the red,
green and blue colour for any given colour (pixel value) previously
selected.

\end{description}

\end{description}

The {\sf{Gwm Colours}} dialog box can be left on screen while plots are being
made to the TCL/GWM widget.

\subsection{\label{capturing_the_display}\xlabel{capturing_the_display}Capturing the display}

The {\sf{Gwm JPEG}} dialog box allows you to dump the plot currently on
the plotting surface to a named JPEG file.  You can also select the
image quality.

\begin{description}

\item[Slider]~

\begin{description}

\item[{\tt{Image quality:}}] Selects the quality of the image produced
in the JPEG compression.  You can vary the quality from 50\% to 95\% with
95\% being the highest quality and therefore largest file.

\end{description}

\item[Checkbutton]~

\begin{description}

\item[{\tt{Progressive:}}] A `progressive' image is an image that, when
loaded by a JPEG viewing program, will be displayed in increasing
quality as the viewer program reads more of the file.  This is the
option to select if the JPEG image is to be used on the World Wide
Web.

\end{description}

\item[Entry]~
 
\begin{description}

\item[{\tt{Filename:}}] You should type in the name of the file you
want to dump into, including the file extension if you want one.  JPEG
files traditionally have the extension {\tt{.jpg}} but this widget does NOT
supply any default extension.

\end{description}

\item[Buttons]~

\begin{description}

\item[{\tt{OK:}}] Start the dump.  When the dump is complete, the
{\sf{Gwm JPEG}} dialog box exits and control is returned to the GWM/TCL
widget.

\item[{\tt{Cancel:}}] Abandons the {\sf{Gwm JPEG}} dialog without
dumping to file.

\item[{\tt{Help:}}] Displays the help text.

\end{description}

\end{description}

The {\tt{Filename}} is the only entry you must fill in, there is no
default filename for the dump.

When you have chosen and entered a filename, select the {\tt{OK}} button to
capture the plot and complete the dump to JPEG file.

\subsection{\label{printing_the_display}\xlabel{printing_the_display}Printing the display}

The {\sf{Gwm Print}} dialog allows you to create a print file in a
specified format with a chosen filename.   The file can be printed
later on an appropriate printer.

\begin{description}

\item[Menu]~

\begin{description}

\item[{\tt{Format:}}] The format menu gives a choice of various print
formats: B/W (greyscale) and colour PostScript,  B/W (greyscale) and colour
Encapsulated PostScript, and HP InkJet.

\end{description}

\item[Checkbuttons]~

\begin{description}

\item[{\tt{As Window: / Colour:}}] The two checkbuttons control the
background colour of the image as printed.  You can chose either the
background colour of the image on the plotting surface, or define the
colour yourself using the `Colour:' checkbutton, and selecting the
required colour in the entry window below.  White is the default
colour.

\end{description}

\item[Entry]~

\begin{description}

\item[{\tt{Filename:}}] You can choose the filename to which the print
image is dumped.  There is no default file extension for any of the
Formats -- you have complete control of the full filename.  The
convention is that standard PostScript files are given a {\tt{.ps}}
extension and Encapsulated PostScript files are given a {\tt{.eps}}
extension.  HP InkJet dump files may be given a {\tt{.dat}} extension.

\end{description}

\item[Buttons]~

\begin{description}

\item[{\tt{OK:}}] Activates the print action and closes the {\sf{Gwm
Print}} dialogue box.

\item[{\tt{Cancel:}}] Exits from the {\sf{Gwm Print}} dialog box
without creating a print file.

\item[{\tt{Help:}}] Displays the help text.

\end{description}

\end{description}

To create a print file, select a format, background colour and filename,
then select the {\tt{OK}} button.  The dump will be started and the control
returned to the TCL/GWM widget.  While the print is being captured, the
{\tt{Capture}} and {\tt{Print}} buttons on the TCL/GWM widget are greyed
out, indicating that the functions are unavailable.  When the print capture
and dump has completed, the buttons return to normal and it is then OK to
send more plotting commands to the widget.

\section{\label{differences}\xlabel{differences}Differences in behaviour}

\subsection{\label{size_of_plotting_surface}\xlabel{size_of_plotting_surface}Size of plotting surface}

The TCL/GWM widget is controlled slightly differently than the standard GWM 
Xdisplay manager when specifying the size of the plotting surface.

The standard GWM Xdisplay manager creates a plotting surface with a boarder
round it either of the default size (780 by 512 pixels), or governed by the 
{\tt{-geometry}} command line switch.  The command:

\begin{quote}
{\tt \% xmake -geometry 900x600 -colours 64 xwindows}
\end{quote}

will generate a 64 colour window 900 by 600 pixels in size called xwindows.

The TCL/GWM widget takes any {\tt{-geometry}} command line switch information 
to use as the size of the entire widget, and packs the buttons and scrollbars
round the largest plotting surface it can create inside the size of widget
defined.

To create a plotting surface 900 by 600 pixels you should use the
{\tt{-height}} and {\tt{-width}} command line switches.  The command:

\begin{quote}
{\tt \% xmake -height 600 -width 900 -colours 64 xwindows}
\end{quote}                                                            

will create a widget with a plotting surface 900 by 600 with the buttons
and scroll bars packed round the plotting surface.

You should note that the standard GWM Xdisplay manager does not understand the
{\tt{-height}} and {\tt{-width}} command line switches and will generate an
error message.

\subsection{\label{x_resources}\xlabel{x_resources}X resources}

The TCL/GWM widget responds to a slightly different form of X-resource
default definitions.  These are typically defined in 
{\verb+${HOME}/.Xdefaults+}.

The following example shows definitions that will create a plotting
surface of 850 by 600 pixels with 64 colours and with foreground and
background colours set to green and black.

\begin{quote}
\begin{verbatim}
!
! GWM 
!   standard set
*Gwm*foreground: green
*Gwm*background: black
*Gwm*colours: 64
*Gwm*geometry: 850x600
!   TCL set
*gwm*Foreground: green
*gwm*Background: black
*gwm*Colours: 64
*Gwm*Height: 600
*Gwm*Width: 850
!
\end{verbatim}
\end{quote}

Defaults set in the {\tt{.Xdefaults}} file will be overridden by any
command line switches or by applications generating their own `xwindows'
displays with specific size, colour and other options.

\end{document}
