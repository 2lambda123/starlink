\documentstyle[twoside,11pt]{article}
\pagestyle{myheadings}

% -----------------------------------------------------------------------------
% ? Document identification
\newcommand{\stardoccategory}  {Starlink User Note}
\newcommand{\stardocinitials}  {SUN}
\newcommand{\stardocsource}    {sun\stardocnumber}
\newcommand{\stardocnumber}    {186.3}
\newcommand{\stardocauthors}   {D.\,L.\,Terrett}
\newcommand{\stardocdate}      {3 MArch 1999}
\newcommand{\stardoctitle}     {STARTCL \\[2ex]Starlink Extensions to Tcl \& Tk}
\newcommand{\stardocversion}   {version 1.3}
\newcommand{\stardocmanual}    {User's Guide}
\newcommand{\stardocabstract}  {%
The STARTCL package is a set of extensions to the \TclTk\ language to
enable \Tcl\ and \Tk\ applications to display Starlink graphics inside widgets
embedded in the application, send and receive ADAM messages and access
ADAM noticeboards.
 
This document will be of interest to programmers writing graphics user
interfaces for Starlink applications in \TclTk . 
}
% ? End of document identification

% -----------------------------------------------------------------------------

\newcommand{\stardocname}{\stardocinitials /\stardocnumber}
\markboth{\stardocname}{\stardocname}
\setlength{\textwidth}{160mm}
\setlength{\textheight}{230mm}
\setlength{\topmargin}{-2mm}
\setlength{\oddsidemargin}{0mm}
\setlength{\evensidemargin}{0mm}
\setlength{\parindent}{0mm}
\setlength{\parskip}{\medskipamount}
\setlength{\unitlength}{1mm}

% -----------------------------------------------------------------------------
%  Hypertext definitions.
%  ======================
%  These are used by the LaTeX2HTML translator in conjunction with star2html.

%  Comment.sty: version 2.0, 19 June 1992
%  Selectively in/exclude pieces of text.
%
%  Author
%    Victor Eijkhout                                      <eijkhout@cs.utk.edu>
%    Department of Computer Science
%    University Tennessee at Knoxville
%    104 Ayres Hall
%    Knoxville, TN 37996
%    USA

%  Do not remove the %\begin{rawtex} and %\end{rawtex} lines (used by 
%  star2html to signify raw TeX that latex2html cannot process).
%\begin{rawtex}
\makeatletter
\def\makeinnocent#1{\catcode`#1=12 }
\def\csarg#1#2{\expandafter#1\csname#2\endcsname}

\def\ThrowAwayComment#1{\begingroup
    \def\CurrentComment{#1}%
    \let\do\makeinnocent \dospecials
    \makeinnocent\^^L% and whatever other special cases
    \endlinechar`\^^M \catcode`\^^M=12 \xComment}
{\catcode`\^^M=12 \endlinechar=-1 %
 \gdef\xComment#1^^M{\def\test{#1}
      \csarg\ifx{PlainEnd\CurrentComment Test}\test
          \let\html@next\endgroup
      \else \csarg\ifx{LaLaEnd\CurrentComment Test}\test
            \edef\html@next{\endgroup\noexpand\end{\CurrentComment}}
      \else \let\html@next\xComment
      \fi \fi \html@next}
}
\makeatother

\def\includecomment
 #1{\expandafter\def\csname#1\endcsname{}%
    \expandafter\def\csname end#1\endcsname{}}
\def\excludecomment
 #1{\expandafter\def\csname#1\endcsname{\ThrowAwayComment{#1}}%
    {\escapechar=-1\relax
     \csarg\xdef{PlainEnd#1Test}{\string\\end#1}%
     \csarg\xdef{LaLaEnd#1Test}{\string\\end\string\{#1\string\}}%
    }}

%  Define environments that ignore their contents.
\excludecomment{comment}
\excludecomment{rawhtml}
\excludecomment{htmlonly}
%\end{rawtex}

%  Hypertext commands etc. This is a condensed version of the html.sty
%  file supplied with LaTeX2HTML by: Nikos Drakos <nikos@cbl.leeds.ac.uk> &
%  Jelle van Zeijl <jvzeijl@isou17.estec.esa.nl>. The LaTeX2HTML documentation
%  should be consulted about all commands (and the environments defined above)
%  except \xref and \xlabel which are Starlink specific.

\newcommand{\htmladdnormallinkfoot}[2]{#1\footnote{#2}}
\newcommand{\htmladdnormallink}[2]{#1}
\newcommand{\htmladdimg}[1]{}
\newenvironment{latexonly}{}{}
\newcommand{\hyperref}[4]{#2\ref{#4}#3}
\newcommand{\htmlref}[2]{#1}
\newcommand{\htmlimage}[1]{}
\newcommand{\htmladdtonavigation}[1]{}

%  Starlink cross-references and labels.
\newcommand{\xref}[3]{#1}
\newcommand{\xlabel}[1]{}

%  LaTeX2HTML symbol.
\newcommand{\latextohtml}{{\bf LaTeX}{2}{\tt{HTML}}}

%  Define command to re-centre underscore for Latex and leave as normal
%  for HTML (severe problems with \_ in tabbing environments and \_\_
%  generally otherwise).
\newcommand{\latex}[1]{#1}
\newcommand{\setunderscore}{\renewcommand{\_}{{\tt\symbol{95}}}}
\latex{\setunderscore}

%  Redefine the \tableofcontents command. This procrastination is necessary 
%  to stop the automatic creation of a second table of contents page
%  by latex2html.
\newcommand{\latexonlytoc}[0]{\tableofcontents}

% -----------------------------------------------------------------------------
%  Debugging.
%  =========
%  Remove % on the following to debug links in the HTML version using Latex.

% \newcommand{\hotlink}[2]{\fbox{\begin{tabular}[t]{@{}c@{}}#1\\\hline{\footnotesize #2}\end{tabular}}}
% \renewcommand{\htmladdnormallinkfoot}[2]{\hotlink{#1}{#2}}
% \renewcommand{\htmladdnormallink}[2]{\hotlink{#1}{#2}}
% \renewcommand{\hyperref}[4]{\hotlink{#1}{\S\ref{#4}}}
% \renewcommand{\htmlref}[2]{\hotlink{#1}{\S\ref{#2}}}
% \renewcommand{\xref}[3]{\hotlink{#1}{#2 -- #3}}
% -----------------------------------------------------------------------------
% ? Document specific \newcommand or \newenvironment commands.

\newcommand{\option}[3]{\par
Name: {\bf{#1}} \\
Class: {\bf{#2}} \\
Command-Line Switch: {\bf{#3}} }

\newenvironment{rquote}{\begin{quote}\raggedright}{\end{quote}}
\newenvironment{tquote}{\begin{quote}\raggedright}{\end{quote}}

\newcommand{\Tcl}{{\sf{Tcl}}}
\newcommand{\Tk}{{\sf{Tk}}}
\newcommand{\TclTk}{{\sf{Tcl/Tk}}}

% ? End of document specific commands
% -----------------------------------------------------------------------------
%  Title Page.
%  ===========
\renewcommand{\thepage}{\roman{page}}
\begin{document}
\thispagestyle{empty}

%  Latex document header.
%  ======================
\begin{latexonly}
   CCLRC / {\sc Rutherford Appleton Laboratory} \hfill {\bf \stardocname}\\
   {\large Particle Physics \& Astronomy Research Council}\\
   {\large Starlink Project\\}
   {\large \stardoccategory\ \stardocnumber}
   \begin{flushright}
   \stardocauthors\\
   \stardocdate
   \end{flushright}
   \vspace{-4mm}
   \rule{\textwidth}{0.5mm}
   \vspace{5mm}
   \begin{center}
   {\Huge\bf  \stardoctitle \\ [2.5ex]}
   {\LARGE\bf \stardocversion \\ [4ex]}
   {\Huge\bf  \stardocmanual}
   \end{center}
   \vspace{5mm}

% ? Heading for abstract if used.
   \vspace{10mm}
   \begin{center}
      {\Large\bf Abstract}
   \end{center}
% ? End of heading for abstract.
\end{latexonly}

%  HTML documentation header.
%  ==========================
\begin{htmlonly}
   \xlabel{}
   \begin{rawhtml} <H1> \end{rawhtml}
      \stardoctitle\\
      \stardocversion\\
      \stardocmanual
   \begin{rawhtml} </H1> \end{rawhtml}

% ? Add picture here if required.
% ? End of picture

   \begin{rawhtml} <P> <I> \end{rawhtml}
   \stardoccategory \stardocnumber \\
   \stardocauthors \\
   \stardocdate
   \begin{rawhtml} </I> </P> <H3> \end{rawhtml}
      \htmladdnormallink{CCLRC}{http://www.cclrc.ac.uk} /
      \htmladdnormallink{Rutherford Appleton Laboratory}
                        {http://www.cclrc.ac.uk/ral} \\
      \htmladdnormallink{Particle Physics \& Astronomy Research Council}
                        {http://www.pparc.ac.uk} \\
   \begin{rawhtml} </H3> <H2> \end{rawhtml}
      \htmladdnormallink{Starlink Project}{http://star-www.rl.ac.uk/}
   \begin{rawhtml} </H2> \end{rawhtml}
   \htmladdnormallink{\htmladdimg{source.gif} Retrieve hardcopy}
      {http://star-www.rl.ac.uk/cgi-bin/hcserver?\stardocsource}\\

%  HTML document table of contents. 
%  ================================
%  Add table of contents header and a navigation button to return to this 
%  point in the document (this should always go before the abstract \section). 
  \label{stardoccontents}
  \begin{rawhtml} 
    <HR>
    <H2>Contents</H2>
  \end{rawhtml}
  \renewcommand{\latexonlytoc}[0]{}
  \htmladdtonavigation{\htmlref{\htmladdimg{contents_motif.gif}}
        {stardoccontents}}

% ? New section for abstract if used.
  \section{\xlabel{abstract}Abstract}
% ? End of new section for abstract
\end{htmlonly}

% -----------------------------------------------------------------------------
% ? Document Abstract. (if used)
%  ==================
\stardocabstract
% ? End of document abstract
% -----------------------------------------------------------------------------
% ? Latex document Table of Contents (if used).
%  ===========================================
 \newpage
 \begin{latexonly}
   \setlength{\parskip}{0mm}
   \latexonlytoc
   \setlength{\parskip}{\medskipamount}
   \markboth{\stardocname}{\stardocname}
 \end{latexonly}
% ? End of Latex document table of contents
% -----------------------------------------------------------------------------
\cleardoublepage
\renewcommand{\thepage}{\arabic{page}}
\setcounter{page}{1}

\section{\label{introduction}\xlabel{introduction}Introduction}

A powerful public domain utility called
\htmladdnormallink{\TclTk}{http://sunscript.sun.com/} (``Tcl and the Tk
Toolkit'', \htmladdnormallink{John K.
Ousterhout}{http://www.sunlabs.com/people/john.ousterhout/},
Addison-Wesley, 1994) is coming into wide use as a command language and
user interface builder.

This note describes three extensions to \Tcl\ and the \Tk\ Toolkit\footnote{
These extensions will only work with \Tk\ Version 4.2 or later} that 
enable \TclTk\ applications to cooperate with Starlink applications. The 
extensions are:

\begin{itemize}
\item A \Tk\ widget that displays Starlink graphics by emulating the ``gwm'' 
graphics window server and a ``gwm'' canvas item.
\item An interface to the Adam message system.
\item An interface to the Adam notice board system.
\end{itemize}

\begin{quote}
{\em The message system and notice board extensions are both based on work 
done by Dennis Kelly of the 
\htmladdnormallink{Royal Observatory Edinburgh}{http://www.roe.ac.uk/}.  }
\end{quote}

A version of the \Tcl\ shell (called {\tt{atclsh}}) and the windowing
shell (called {\tt{awish}}) with all of these extension built in are
provided which are useful for trying them out. However, for production
systems, a bespoke version of the shell with a subset of the available
extensions (both from Starlink and elsewhere) will normally be
required.

The Starlink extensions can be built into a \Tcl\ or \Tk\ application
by including one or more of the following functions in
{\tt{tclAppInit.c}} or {\tt{tkAppInit.c}}.

\[\begin{tabular}{|l|l|l|l|}
\hline
Extension &Function &include file &object module\\
\hline
Adam &\tt Tcladam\_Init(interp) &\tt tclAdam.h &\tt tclAdam.o\\
Gwm &\tt Tkgwm\_Init(interp) &\tt tkGwm.h &\tt tkGwm.o\\
Nbs &\tt Tclnbs\_Init(interp) &\tt tclNbs.h &\tt tclNbs.o\\
\hline
\end{tabular}\]

where {\tt{interp}} is a pointer to a {\tt{Tcl\_Interp}}. All three functions
return a \Tcl\ status value. The application should then be linked with the
specified object module all of which can be found in {\tt{/star/lib}}.

Alternatively, 
the Adam and Gwm extensions can be loaded at run time into
the standard interpreters with {\tt{load}} commands of the form:
\begin{quote}{\tt
load /star/lib/libtclAdam.so tclAdam\\
load /star/lib/libtkGwm.so tkGwm
}\end{quote}

The disadvantages of this method is the need to supply a path name for
the shareable library which makes the script dependent on the location
of the starlink software and that this mechanism may not be available
on all operating systems in the future.

\section{\label{gwm_extension}\xlabel{gwm_extension}The GWM Extension}

A \xref{Graphics Window Manager}{sun130}{} (GWM) widget is a window that 
conventional ({\em{i.e.}}\ non-event driven) programs that use
Starlink graphics packages ({\em{e.g.}}, 
\xref{GKS}{sun83}{}, 
\xref{SGS}{sun85}{}, 
\xref{PGPLOT}{sun15}{}, 
\xref{NCAR}{sun88}{}, 
\xref{IDI}{sun65}{} {\em{etc.}}) can use
to plot in on X windows displays. GWM is described in SUN/130. The gwm widget
makes it possible to integrate GKS and IDI based graphics with user interfaces
written in \Tcl\ and \Tk.  It also has commands for dumping its contents 
in JPEG, PostScript or LJ250 inkjet format, changing the colour lookup table,
clearing the window, scrolling the window  and displaying a cross hair
that covers full width and height of the window.

The GWM canvas item turns an area of a canvas widget into a GWM window;
this provides similar functionality to the GWM widget but also allows
other canvas items (lines, polygons, {\em{etc.}}) to be overlayed on
the window and therefore graphics from external programs and element
created and minipulated by \Tk\ to be mixed.

\subsection{\label{gwm_widget}\xlabel{gwm_widget}The GWM Widget}

\subsubsection{WIDGET COMMAND}

The {\bf{gwm}} command creates a new \Tcl\ command whose name is
{\em{pathName}}.  The command may then be used to perform various
operations on the widget. It has the following general form:

\begin{quote}
{\bf gwm} {\em pathName option ?arg arg...?}
\end{quote}

{\em{Option}} and {\em{args}} determine the exact behaviour of the command.

The following widget commands are possible for gwm widgets:

{\em pathName {\bf cget} option}
\begin{rquote}
Returns the current value of the configuration option given by {\em option}.
{\em option} may have any of the values accepted by the gwm command.
\end{rquote}

{\em pathName {\bf clear}}
\begin{rquote}
Clear the picture in the widget by setting the entire window to pixel
value zero.
\end{rquote}

{\em pathName {\bf configure} ?option? ?value option value...?}
\begin{rquote}
Query or modify the configuration options of the widget.
If no {\em{option}} is specified, returns a list describing all of
the available options for {\em{pathName}} (see {\bf{Tk\_ConfigureInfo}} for
information on the format of this list).  If {\em option} is specified
with no {\em value}, then the command returns a list describing the
one named option (this list will be identical to the corresponding
sublist of the value returned if no {\em option} is specified).  If
one or more {\em option-value} pairs are specified, then the command
modifies the given widget option(s) to have the given value(s);  in
this case the command returns an empty string.
{\em Option} may have any of the values accepted by the {\bf gwm}
command.
\end{rquote}

{\em pathName {\bf get colour} index}
\begin{rquote}
Returns the colour of colour table entry {\em index} as a text string in 
format \hbox{\em \#RRRRGGGGBBBBB}. {\em Index} -1 returns the 
colour of the overlay. 
\end{rquote}

{\em pathName {\bf ovclear}}
\begin{rquote}
Clear the overlay.
\end{rquote}

{\em pathName {\bf print} filename}
\begin{rquote}
Write the contents of the widget to {\em filename} in the format specified
by the {\em printformat} option. This command returns as soon as the printing
has started; the generation of the file continues whenever the \Tcl\ interpreter
is idle. When the printing is complete, the global variable specified by the
{\em printvar} option is set to 1.
\end{rquote}

{\em pathName {\bf set colour} index colour}
\begin{rquote}
Set the colour of colour table index {\em index} to {em colour}. {\em Index}
-1 sets the colour of the overlay plane.
\end{rquote}

{\em pathName {\bf set crosshair} x\_position y\_position}
\begin{rquote}
Sets the XY position of the crosshair.
The values may be in any of the forms acceptable to {\bf Tk\_GetPixels}.
\end{rquote}

\subsubsection{Printing}

The gwm widget has a command for printing the contents of the gwm window in
a number of formats (JPEG, various flavours of PostScript and HP inkjet).
The process of generating the file can be quite time consuming for a large
window so the print command returns control as soon as the window contents
have been captured and the output file created. Actual generation of the
file is then carried out whenever the \Tcl\ interpreter is idle enabling the
application to continue to respond to X events while the printing is in
progress.
Completion of the printing process is signaled by a global \Tcl\ variable
({\bf gwm\_printvar} by default) being set to 1. 

The simplest way to detect completion is to wait for the value of the 
variable to change. {\em{e.g.}}: 

\begin{quote}
\begin{verbatim}
global gwm_printvar
set gwm_printvar 0
if [catch {.gwm print plot.ps}] {
   ...an error
} else {
   tkwait variable gwm_printvar
}
\end{verbatim}
\end{quote}

An alternative approach is to set a trace on the variable so that a 
procedure is invoked when the print completes. {\em{e.g.}}:

\begin{quote}
\begin{verbatim}
proc done {name1 name2 opt} {
    trace vdelete gwm_printvar w done
}

global gwm_printvar
set gwm_printvar 0
if [catch {.gwm print plot.ps}] {
    ...an error
} else {
    trace variable gwm_printvar w done
}
\end{verbatim}
\end{quote}

In this example the trace routine merely deletes itself but any other
processing can be done here.

\subsubsection{WIDGET OPTIONS}

\option{background}{Background}{-background {\rm or} -bg}
\begin{rquote}
Specifies the background colour (colour table entry 0) of the gwm window.
Defaults to Black.
\end{rquote}

\option{colours}{Colours}{-colours}
\begin{rquote}
Specifies the number of colours that the gwm window will attempt to allocate
for its colour table. Setting this option after the widget has been created 
has no effect. Defaults to 64.
\end{rquote}

\option{crosshair}{Crosshair}{-crosshair}
\begin{rquote}
Specifies whether the cross-hair should be displayed.
\end{rquote}

\option{crossColour}{CrossColour}{-crosscolour}
\begin{rquote}
Specifies the colour of the crosshair.
\end{rquote}

\option{cursor}{Cursor}{-cursor}
\begin{rquote}
Specifies the mouse cursor to be used for the widget.
The value may have any of the forms acceptable to {\bf Tk\_GetCursor}.
\end{rquote}

\option{foreground}{Foreground}{-foreground {\rm or} -fg}
\begin{rquote}
Specifies the foreground colour (colour table entry 1) of the gwm window.
Defaults to White.
\end{rquote}

\option{name}{Name}{-name {\rm or} -gwmname}
\begin{rquote}
Specifies the name of the gwm window. Setting
this option after the widget has been created has no effect. The gwm name 
is the name used by graphics programs to connect to the window. Defaults
to ``xwindows''.
\end{rquote}

\option{height}{Height}{-height}
\begin{rquote}
Specifies the width of the gwm widget. The value is in any of the forms
acceptable to {\bf Tk\_GetPixels}. Changing this option after the widget 
has been created has no effect. Defaults to 512 pixels.
\end{rquote}

\option{jpegQuality}{JpegQuality}{-jpegquality}
\begin{rquote}
Specifies the quality of the JPEG image in the range 0-100 where 0 is the
lowest quality/highest compression and 100 is the highest quality/lowest
compression. The normally useful range is 50-95, values above 95 give 
dramatically increased file size with practically no increase in quality
while below 50 the image quality degrades the image quality without much 
additional compression.
\end{rquote}

\option{jpegProgressive}{JpegProgressive}{-jpegprogressive}
\begin{rquote}
Specifies a boolean value that determines whether the JPEG image is
``progressive''. Progressive images are useful for use on world wide web
servers as it allows a browser to display a low quality image quickly and
then improve the quality as more data arrives. However not all browsers
and other image reading software can interpret progressive JPEGs.
\end{rquote}

\option{minColours}{MinColours}{-mincolours}
\begin{rquote}
Specifies the minimum number of colours that the gwm window should allocate 
for its colour table. If fewer than this number of colours are available,
the creation of the widget will fail. Changing this option after the widget
has been created has no effect. Defaults to 2.
\end{rquote}

\option{overlay}{Overlay}{-overlay}
\begin{rquote}
Specifies a boolean value that determines whether the gwm window will have
an overlay plane.  Changing this option after the widget
has been created has no effect. Defaults to ``no''.
\end{rquote}

\option{paperWidth}{PaperWidth}{-paperwidth}
\begin{rquote}
Specifies the maximum width of the plot created by the {\bf print} command. 
The value is in any of the forms acceptable to {\bf Tk\_GetPixels}. The 
plot will be the largest size possible that will fit on the paper (after
rotation through 90 degrees if necessary) with the same aspect ratio
as the gwm window. Defaults to 180mm (a little less than the width of an
A4 page).
\end{rquote}

\option{paperHeight}{PaperHeight}{-paperheight}
\begin{rquote}
Specifies the maximum height of the plot in the same way as {\bf paperWidth}.
Defaults to 250mm (a little less than the height of an A4 page).
\end{rquote}

\option{printBackground}{PrintBackground}{printbackground {\rm or} -printbg}
\begin{rquote}
Specifies the colour used for the background of a plot. Any pixel with
value zero is plotted in the background colour. Defaults to White.
\end{rquote}

\option{printFormat}{PrintFormat}{-printformat}
\begin{rquote}
Specifies the format of the file produced by the {\bf print} command. Valid 
values are:
\begin{description}
\item[ps postscript] Monochrome PostScript.
\item[colour\_ps colour\_postscript] Colour PostScript.
\item[eps bw\_eps] Monochrome encapsulated PostScript.
\item[colour\_eps] Colour encapsulated PostScript.
\item[HPinkjet] LJ250 Inket.
\item[JPEG] JPEG
\end{description}
\end{rquote}

\option{printVariable}{PrintVariable}{-printvariable}
\begin{rquote}
Specifies the name of the global variable that is set when a print command
has completed. The default is {\bf gwm\_printvar}.
\end{rquote}

\option{takeFocus}{TakeFocus}{-takefocus}
\begin{rquote}
See the options man page for a description of this option.
\end{rquote}

\option{width}{Width}{-width}
\begin{rquote}
Specifies the width of the gwm widget. The value is in any of the forms
acceptable to {\bf Tk\_GetPixels}. Changing this option after the widget 
has been created has no effect.
\end{rquote}

\option{xOffset}{XOffset}{-xoffset}
\begin{rquote}
Specifies by how much the picture is offset from the left hand edge of the
window. The value is in any of the forms acceptable to {\bf Tk\_GetPixels}. 
Positive values shift the picture to the right and negative values shift the
picture to the right. The default is zero.
\end{rquote}

\option{xOvOffset}{XOvOffset}{-xovoffset}
\begin{rquote}
Specifies by how much the overlay is offset from the left hand edge of the
window. The value is in any of the forms acceptable to {\bf Tk\_GetPixels}. 
Positive values shift the picture to the right and negative values shift the
picture to the right. The default is zero.
\end{rquote}

\option{yOffset}{YOffset}{-yoffset}
\begin{rquote}
Specifies by how much the picture is offset from the top edge of the
window. The value is in any of the forms acceptable to {\bf Tk\_GetPixels}. 
Positive values shift the picture to down the screen and negative values 
shift the picture to up the screen. The default is zero.
\end{rquote}

\option{yOvOffset}{YOvOffset}{-yovoffset}
\begin{rquote}
Specifies by how much the overlay is offset from the top edge of the
window. The value is in any of the forms acceptable to {\bf Tk\_GetPixels}. 
Positive values shift the picture to down the screen and negative values 
shift the picture to up the screen. The default is zero.
\end{rquote}

\subsubsection{Colourmaps}

A gwm widget always uses the colourmap of its parent window; if there
are insufficient colourmap entries left (less than {\bf minColours})
the creation of the widget will fail. In these circumstances, a gwm
widget can still be created by creating a frame or top level widget
with a new colourmap (with the {\bf -colormap new} option) and using
this as the parent of the gwm widget\footnote{Some existing graphics
programs may attempt to write to the wrong colourmap when used with a
gwm widget packed into a frame with a new colourmap and generate an X
error. Relinking the program with the latest version of the Starlink
software will correct this problem}.

In the case of a frame with a new colourmap it is also necessary to inform
the window manager so that the correct colourmap is installed when the
frame receives the input focus with:
\begin{tquote}{\tt
wm colormapwindows [winfo toplevel \$w] \$w
}\end{tquote}
where {\tt w} has been set to the name of the new frame widget.

The same technique can be used to create a gwm widget with a visual type
other than the default for the screen being used.

\subsubsection{\Tk\ Procedures}

A number of \Tk\ procedures can be found in {\tt{/star/lib/startcl}} that may
be useful when using gwm widgets either directly or as examples that can be
modified to suit particular purposes. To use them directly, append 
{\tt /star/lib/startcl} to the global variable {\bf auto\_path}, {\em{e.g.}}:

\begin{tquote}{\tt
lappend auto\_path /star/lib/startcl
}\end{tquote}

These procedures make use of a global array called {\bf gwm\_priv}.

{\bf gwm\_colourDialog} {\em w gwm exit}
\begin{rquote}
Creates a modeless dialog box with controls for modifying the colours of
a gwm widget.

\begin{description}
\item[{\em w}] A name to be used for the top level widget of the dialog box.
\item[{\em gwm}] The name of the gwm widget to be manipulated.
\item[{\em c}] The control used to pop up the dialog box. This control will
be disabled when the dialog box is popped up and re-enabled when it is
dismissed.

\end{description}
\end{rquote}

{\bf gwm\_gwmWithScroll} {\em w args}
\begin{rquote}
Returns a gwm widget with scroll bars for scrolling the window (and  overlay
plane if there is one).
\begin{description}
\item[{\em w}] A name to be used for the frame widget that contains the gwm
widget and the scroll bars.
\item[{\em args}] Additional arguments to the gwm widget creation command.
\end{description}
\end{rquote}

{\bf gwm\_printDialog} {\em w gwm}
\begin{rquote}
Creates a modal dialog box containing controls for printing a gwm widget.
\begin{description}
\item[{\em w}] A name to be used for the top level widget of the dialog box.
\item[{\em gwm}] The name of the gwm widget to be printed.
\item[{\em c}] The control used to pop up the dialog box. This control will
be disabled when the print starts and re-enabled when it completes.
\end{description}
\end{rquote}

An gwm server that uses these procedures can be found in
{\tt{/star/bin/startcl/gwm}}.  This is the default server for Starlink
systems.  It creates a server with buttons for changing the colours,
printing the gwm window and clearing the picture and overlay planes.
The coordinates of the mouse pointer are also displayed whenever it is
over the gwm window.  A listing of this script can be found in
appendix~\ref{example_gwm_server_script}.

Arguments for the gwm widget creation command can be supplied on the command
line (except {\bf -name} which is interpreted as an {\bf awish} option; use
{\bf -gwmname instead}). 

To use your own gwm server by default whenever a gwm server
is created (either with the xmake command or automatically by a
graphics program), copy {\tt{/star/bin/startcl/gwmXrefresh}} to a
directory in your {\tt{PATH}} that is positioned before
{\tt{/star/bin/startcl}} and edit line 13 so that the variable {\tt
gwmscript} is set to the location of suitable script that creates a gwm
widget.

\subsection{\label{gwm_canvas_item}\xlabel{gwm_canvas_item}GWM Canvas item}

A gwm canvas item is a rectangular area of a canvas widget that has been
turned into a gwm window. 

The canvas item duplicates much of the functionality of the gwm widget
except that:

\begin{description}
\item The canvas item does not support overlays.
\item Printing to JPEG or LJ250 format is not supported.
\item Printing is not done as a background activity so the \Tk\ application
freezes while the canvas is being printed.
\item Only one gwm item can exist on any particular canvas
\item There is no crosshair (although this can easily be implemented with
line canvas items).
widget. 
\end{description}

The important additional functionality that the canvas item offers is the
ability to overlay the gwm picture with other canvas item; since these items
can be manipulated through \Tcl\ procedures it is possible to write programs
that enable the user to interact with graphics in a much wider variety of
ways than is possible though GKS or IDI alone. However, any drawing done
by an external program will overwrite any existing canvas items until the
window is redrawn by Tk; the easiest way to force the window to be redrawn
is to use the canvas {\tt raise} or {\tt lower} commands (for example, to
``lower'' the gwm canvas item so that it is underneath all  other items).

Gwm items are created with widget commands of the following form:

\begin{quote}
{\em pathName} create gwm {\em x y option value option value ...}
\end{quote}

The arguments {\em x} and {\em y} give the coordinates of the item's
top left hand corner. After the coordinates there may be any number of
{\em option}-{\em value} pairs, each of which sets one of the
configuration options of the item. The same {\em option}-{\em value}
pairs may be used in itemconfigure widget commands to change the item's
configuration. The following options are supported for gwm items:

{\bf -background} {\em colour}\\
{\bf -bg} {\em colour}

\begin{rquote}
Specifies the background colour (colour table entry 0) of the gwm window.
Defaults to Black.
\end{rquote}

{\bf -colours} {\em number}
\begin{rquote}
Specifies the number of colours that the gwm window will attempt to allocate
for its colour table. Setting this option after the item has been created 
has no effect. Defaults to 64.
\end{rquote}

{\bf -command} {\em number}
\begin{rquote}
Specifies the name of the command that is created to clear (erase the picture
in) the item and modify the colour table. If no value is specified, no 
command will be created. Changing this option after the item has been created
has no effect.
\begin{quote}{\tt
{\em command} clear
}\end{quote}
clears the item.
\begin{quote}{\tt
{\em command} set colour {\em n} {\em colour}
}\end{quote}
sets colour table entry {\em n} to {\em colour}.
\begin{quote}{\tt
{\em command} get colour {\em n}
}\end{quote}
returns the colour stored in colour table entry {\em n}.
\end{rquote}

{\bf -foreground} {\em colour}\\
{\bf -fg} {\em colour}
\begin{rquote}
Specifies the foreground colour (colour table entry 1) of the gwm window.
Defaults to White.
\end{rquote}

{\bf -gwmname} {\em name}\\
{\bf -name} {\em name}
\begin{rquote}
Specifies the name of the gwm window. Setting
this option after the item has been created has no effect. The gwm name 
is the name used by graphics programs to connect to the window. Defaults
to ``xwindows''.
\end{rquote}

{\bf -height} {\em height}
\begin{rquote}
Specifies the width of the gwm item. The value is in any of the forms
acceptable to {\bf Tk\_GetPixels}. Changing this option after the item 
has been created has no effect. Defaults to 512 pixels.
\end{rquote}

{\bf -mincolours} {\em number}
\begin{rquote}
Specifies the minimum number of colours that the gwm window should allocate 
for its colour table. If fewer than this number of colours are available,
the creation of the item will fail. Changing this option after the item
has been created has no effect. Defaults to 2.
\end{rquote}

% {\bf -ovcolour} {\em colour}
% \begin{rquote}
% Specifies the overlay colour of the gwm window. Ignored if {\bf -overlay}
% is false.  Defaults to White.
% \end{rquote}
%
% {\bf -overlay} {\em boolean}
% \begin{rquote}
% Specifies whether the gwm window will have
% an overlay plane.  Changing this option after the item
% has been created has no effect. Defaults to ``no''.
% \end{rquote}

{\bf -tags} {\em taglist}
\begin{rquote}
Specifies a set of tags to apply to the item. {\em Taglist} consists
of a list of tag names, which replace any existing tags for the item.
{\em Taglist} may be an empty list.
\end{rquote}

{\bf -width} {\em width}
\begin{rquote}
Specifies the width of the gwm item. The value is in any of the forms
acceptable to {\bf Tk\_GetPixels}. Changing this option after the item 
has been created has no effect.
\end{rquote}

% {\bf -xovoffset} {\em offset}
% \begin{rquote}
% Specifies by how much the overlay is offset from the left hand edge of the
% item. The value is in any of the forms acceptable to {\bf Tk\_GetPixels}. 
% Positive values shift the picture to the right and negative values shift the
% picture to the right. The default is zero.
% \end{rquote}
%
%
% {\bf -yovoffset} {\em offset}
% \begin{rquote}
% Specifies by how much the overlay is offset from the top edge of the
% item. The value is in any of the forms acceptable to {\bf Tk\_GetPixels}. 
% Positive values shift the picture to down the screen and negative values 
% shift the picture up the screen. The default is zero.
% \end{rquote}


\section{\label{adam_message_system_extension}\xlabel{adam_message_system_extension}ADAM Message System Extension}

The Adam message system extensions provide access to the message system 
at the level of the ams library and a fairly detailed knowledge 
of the Adam message system and how it is used by Adam tasks is required. 
Therefore, this section describes a set of \Tcl\ procedures that use the 
message system interface to provide a higher level interface that is 
suitable for controlling Adam tasks from a graphical user interface. The 
low-level commands are described in
appendix~\ref{low_level_adam_message_system_commands}.

\begin{quote}
{\em Some of the features of Adam tasks described here appeared in release
3.1 and so you must ensure that any tasks to be used with \Tcl\ have been
linked with Adam 3.1 or later.}
\end{quote}

\subsection{\label{message_system}\xlabel{message_system}The Message System}

Command languages (such as ICL or \TclTk ) control Adam tasks by
sending them Adam messages and responding to any messages from the task
The messages from the task request such things as new values for
parameters or contain text to displayed to the user.  This section
describes what sort of messages tasks respond to and what replies can
be expected; it concentrates on data reduction tasks and ignores some
of the more complicated behavour of instrument control tasks.

\subsubsection{The OBEY message}

An adam task supports one or more commands, normally referred to as
{\em actions}, ({\em{e.g.}} the kappa task supports ``add'', ``cadd'',
``stats'' etc.) and is instructed to execute one of them by sending it
an ``OBEY'' message. The message contains the action name ({\em{e.g.}} ``add'')
and a command line which can contain parameter values and any of the
keywords that modify the behaviour of the parameter system ({\em{e.g.}}
``in1=test accept''). If the task recognises the action
name\footnote{Tasks that contain only one command ignore the action
name and execute the command regardless} it responds with an {\tt
actstart} message; the message name field of the message will be the
action name and the status field will be {\tt DTASK\_\_ACTSTART}. The
task then starts executing the command.

In the simplest possible case, where the action writes no text and does 
not need to prompt for any parameter values, the task completes the action
and sends an {\tt endmsg} message to the command language. The message name 
is again the action name
and the status will be {\tt DTASK\_\_ACTCOMPLETE}. The task is now ready to 
execute another action.

If, as usually happens, the task does want to display some text, it will send
one or more {\tt inform} messages; the text to display will be in the {\em
value} field of the message. Note that the message system makes no distinction
between text and error reports, however, the adam error reporting system 
always prefixes the first line of an error message with two exclamation marks
and subsequent lines with a single exclamation mark. If the task needs
a value for a parameter it will send a {\tt paramreq} message; the {\em value}
field will contain the parameter name, prompt, default
value etc.. The command language must reply with a {\tt paramrep} message 
with the new parameter value in the {\em value} field.

The only other message that a task might send as a result of an ``OBEY'' is
a {\tt sync} message; the purpose of this message is to instruct the command
language to ensure that all output has been actually written to the screen 
for the user to read---for example, instructions on how to perform some
interaction with a graphics device. In the context of a GUI, where updates
to the screen are performed asynchronously whenever the GUI process is idle,
the message probably has no useful function. However, a {\tt synchrep}
message must be sent to acknowledge the {\tt sync} message (the adamtask
library described below does this automatically).

If an error occurs during the execution of a command, the {\em status} field
of the final {\tt endmsg} message will be the final exit status of the action;
ie.\ something other than {\tt DTASK\_\_ACTCOMPLETE}. If the action cannot even
be started (if, for example, the action name isn't recognised) there will be
no {\tt actstart} message and the {\tt actcomplete} message will contain an
error status (eg. {\tt SUBPAR\_\_NOACT}).

\subsubsection{The SET message}

To set a parameter, you send a ``SET'' message with the message {\em
name} set to the parameter name in the form {\em action}{\tt{:}}{\em
parameter} (eg.  stats:ndf) and the value field to the desired
parameter value. The task will respond with a {\tt setresponse} message
with a status field set to {\tt SAI\_\_OK} if the parameter was
sucessfully set.  If not, it will send one of more {\tt inform}
messages containing error messages followed by a {\tt setresponse}
message with a status other than {\tt SAI\_\_OK} (for example, {\tt
SUBPAR\_\_NOPAR} if the parameter didn't exist). When the task executes
the coresponding action it will behave exactly as if the parameter had
been set with a command line argument in the ``OBEY'' message.

\subsubsection{The GET message}

To get a parameter value, send a ``GET'' message with the parameter in the
message {\tt name} field; the task will respond with a {\tt getresponse} 
message with the {\tt name} set to the parameter name and with the
parameter's value in the {\tt value} field. Errors are handled in the same
way as for a ``SET''. The ``GET'' message is only really useful if the task
is running with its type set to ``I'', otherwise all of an actions parameters
are annulled (set to have no value) when an action completes.

\subsubsection{The CONTROL message}

A ``CONTROL'' message instructs a task to perform one of the pre-defined 
actions that all Adam tasks support. There are two control actions recognised:
\begin{description}
\item[{\tt default}] instructs the task to 
change to a new default directory. The value field of the message should 
be set to the new directory. The task will respond with a {\tt
controlresponse} message with the value field set to the new directory. 
If a ``CONTROL'' message with the action set to ``default'' has a blank value
field the value field in the {\tt setresponse} message will be set to the
current default directory.
\item[{\tt par\_reset}] sets all of the tasks parameters to the 
state they were in when the task was first loaded. The task responds with
a {\tt controlresponse} message.
\end{description}

\subsubsection{The CANCEL message}
The final type of message is the ``CANCEL'' message which is similar to an
``OBEY'' message but data reduction tasks do not support any ``CANCEL'' 
actions.

\subsection{\label{tcltk_interface}\xlabel{tcltk_interface}The \TclTk\ Interface}

The interface is loaded by reading the file
{\tt{/star/lib/startcl/adamtask.tcl}} with the tcl command:

\begin{tquote}{\tt
source /star/lib/startcl/adamtask.tcl
}\end{tquote}

This defines a procedure called {\tt adamtask} which starts an Adam 
task. It takes two arguments: a name to be used to refer to the task in later 
\Tcl\ commands, and, optionally, the name of the executable file to run. 
if the second argument is omitted, the name must be the name, as known to the
Adam message system, of a task already loaded.
In addition it 
creates a new \Tcl\ command with the same name as the task which can then be
used to communicate with the task. 

For example:
\begin{tquote}{\tt
adamtask kappa /star/bin/kappa/kappa\_mon
}\end{tquote}
will start the kappa monolith and create a \Tcl\ command called {\bf kappa} 
which can be used to send messages to the task. 
The first argument to this new command is the type of Adam message to 
send to the task and is one of {\tt set}, {\tt get}, {\tt obey}, {\tt 
cancel}, {\tt paramreply} or {\tt syncrep}.

The format of the command to send a task an {\bf obey} message is:
\begin{quote}{\tt
<taskname> obey <action> <command-line>
}\end{quote}
So, for example, we can instruct kappa to execute the add command with:
\begin{tquote}{\tt
kappa obey add "in1=image1 in2=image2 out=result"
}\end{tquote}
However, we don't receive any notification of when the obey has 
completed. 

The action to be taken when a message of a particular type is sent by the 
task is specified by additional arguments consisting of a message type 
with a hyphen prepended followed by a \Tcl\ command to be executed when the 
message is received. The command is executed in global scope in the same 
way as widget and X event bindings. So, for example:
\begin{tquote}{\tt
kappa obey add "in1=image1 in2=image2 out=result" $\backslash$\\
\hspace*{2em}-endmsg \{puts "add finished"\} 
}\end{tquote}
Will print ``add finished'' on the terminal when the obey has completed. 

The various fields in the message can be inserted in the command to be 
executed wherever the percent character appears (similar to the 
substitutions performed when widget bindings are invoked). For example:
\begin{tquote}{\tt
kappa obey stats "ndf=result" -inform \{puts \%V\}
}\end{tquote}
will print the messages output by the stats command on the terminal. 
The following tokens are recognised for all message
types: 
\[\begin{tabular}{|c|l|}\hline
token &replaced by\\
\hline
\tt\%C &Message context\\
\tt\%T &Task name\\
\tt\%N &Message name\\
\tt\%P &Path\\
\tt\%M &Message id\\
\tt\%S &Status\\
\tt\%V &Message value\\
\tt\%R &Reply Token\\
\tt\%\% &\%\\
\hline
\end{tabular}\]
In the case of a {\bf paramreq} message the value is in the form of a 
\Tcl\ list and the following tokens are replaced by the appropriate element 
of the list:
\[\begin{tabular}{|c|l|}\hline
token &replaced by\\
\hline
\tt\%n &parameter name\\
\tt\%p &prompt string\\
\tt\%d &parameter default\\
\tt\%h &help text\\
\tt\%e &error message\\
\hline
\end{tabular}\]

Dealing with a paramreq message is more complicated than any other 
message type because it demands that a reply is sent to the task before it
will continue. The format of the command to send a {\bf paramreply} message is:
\begin{quote}{\tt
<task-name> paramreply <reply-token> <reply>
}\end{quote}
The {\tt <reply-token>} parameter is used by the 
message system to route the message back to the task that made the 
parameter request and can be extracted from the {\bf paramreq} message.

About the simplest example is:
\begin{tquote}{\tt
kappa obey cadd "in=image1 out=result" $\backslash$\\
\hspace*{2em}-paramreq \{kappa paramreply \%R !\} 
}\end{tquote}
This will reply with ``!'' whenever any parameter is prompted for. A 
more realistic example is:

\begin{quote}
\begin{verbatim}
proc prompt {task reptok param string default} {
#+
# Handles a prompt request from a task.
#-

# Create dialog box to display the prompt and receive the user's reply.
    toplevel .${task}_prompt
    wm title .${task}_prompt "Parameter prompt from ${task}"
    wm transient .${task}_prompt .

# Create the dialog box layout.
    label .${task}_prompt.label -bd 5 -text \
        "$task requires a value for $param"
    label .${task}_prompt.prompt -text $string -bd 5
    entry .${task}_prompt.entry -bd 2 -relief sunken
    .${task}_prompt.entry insert end $default
    button .${task}_prompt.ok -text OK -command "
        $task paramreply $reptok \[.${task}_prompt.entry get\]
        destroy .${task}_prompt
    "
    pack .${task}_prompt.label
    pack .${task}_prompt.ok -side bottom -pady 10
    pack .${task}_prompt.prompt .${task}_prompt.entry -side left
}
kappa obey cadd "in=image1 out=result" \
    -paramreq {kappa prompt %T %R %n %p %d} 
\end{verbatim}
\end{quote}

Default handlers are provided for {\bf paramreq}, {\bf inform} and {\bf sync} 
messages. The handler for {\bf paramreq} messages is the one listed above. The 
handler for {\bf inform} messages creates a dialog box (one for each task)
containing a scrolling text widget and inserts the message into it. These
ensure that, by default, important messages are not ignored; they are 
adequate during
development but for production systems bespoke {\bf inform} message handlers
will almost certainly be required. The handler for {\bf sync} messages 
calls the \Tcl\ {\bf update} command and then sends a {\bf syncrep} message to
the task.

If you want to ignore a message type then specify an empty string as
the handler procedure. For example:
\begin{tquote}{\tt
kappa obey stats "ndf=result logfile=stats.log" -inform ""
}\end{tquote}
will run ``stats'' and log the output to a file instead of displaying them.

The format of the remaining command options that send messages to a task are:
\begin{description}
\item[cancel] {\em action command-line}
\item[set] {\em parameter-name parameter-value}
\item[get] {\em parameter-name}
\end{description}

There are some additional command options that don't actually send any messages
to the task:

\begin{description}
\item [{\bf kill}] kills the adam task (provided that the task was loaded
by the {\tt adamtask} procedure) and deletes the task procedure.
\item [{\bf forget}] deletes the task procedure and removes the task from
the list of known tasks so that it doesn't get killed when the \Tcl\ application
exits.
\item [{\bf path}]  returns 1
if a message system path to the task has been established and 0 if it hasn't. 
This can be used to wait for a task to be loaded before sending it any 
messages. For example:

\begin{quote}
\begin{verbatim}
adamtask kappa /star/bin/kappa/kappa\_mon
set count 0
while {[kappa path] == 0} {
    after 100
    incr count
    if {$count > 100} {
       tk_dialog .loadError $title \
         "Timed out waiting for task kappa to start" error 0 OK
       kappa kill
    }
}
\end{verbatim}
\end{quote}

loads kappa and tries 100 times at 100 millisecond intervals to establish 
a message system path.
\end{description}

All the task commands return immediately the appropriate message has been 
sent; they do not wait for the task to acknowledge the receipt of the 
message. The following code illustrates how to wait for an action (in 
this case an obey) to be completed:
\begin{tquote}{\tt
kappa obey stats -endmsg \{set done 1\}\\
tkwait variable done
}\end{tquote}
Note that the set command is executed in global scope so in this case there 
is no need to declare {\bf done} to be global. If {\bf done} were to be 
set in a procedure as in:
\begin{tquote}{\tt
procedure setdone \{\} \{\\
\hspace*{2em}global done\\
\hspace*{2em}set done 1\\
\}\\
kappa obey stats -endmsg setdone
}\end{tquote}
then a global command is required. While the tkwait command is waiting, 
widget and X event bindings can still be triggered and adam messages 
received.

If a large number of adam messages are sent without waiting for an 
acknowledgement there is a risk of filling up the buffers in the message 
system and causing the system to deadlock. In order to avoid this 
happening you should:
\begin{itemize}
\item Call {\bf update idletasks} to give the \Tcl\ application the chance 
to respond to incoming adam messages.
\item Use the mechanism described above to wait for acknowledgement 
messages whenever the task is expected to respond promptly.
\item Use the command associated with the receipt of the acknowledgement to
send the next message. For example, a series of {\bf obey} messages can 
be `chained'
together by send each obey in response to the {\bf endmsg} message from the
previous {\bf obey}.
\end{itemize}

\section{\label{noticeboard_extension}\xlabel{noticeboard_extension}Noticeboard Extension}

The Adam notice board system (\xref{SUN/77}{sun77}{}) stores arrays of 
bytes in shared memory.
Each notice board item has associated with it a data type and dimensions but
the notice board system itself does not attach any meaning to them; it is
up to the applications accessing the noticeboard to agree on the interpretation
of the type and dimension information. The types recognizes by the \TclTk\ 
interface to the noticeboard system and the way it interprets the data 
value are:

\begin{description}
\item[{\tt \_INTEGER}] 4 byte signed integers.
\item[{\tt \_REAL}] 4 byte floating point numbers.
\item[{\tt \_DOUBLE}] 8 byte floating point numbers.
\item[{\tt \_LOGICAL}] 4 byte logical values where zero is FALSE and any 
other value is TRUE.
\item[{\tt \_CHAR}] A character string.
\end{description}

The notice board extension defines one new command {\bf nbs}. 
The {\bf nbs} command can be used to read nbs notice board items with:
\begin{tquote}{\tt
nbs get <nbs\_name>
}\end{tquote}
which returns the values stored in the specified nbs item. If the type of the
item is one of \_INTEGER, \_REAL, \_DOUBLE or \_LOGICAL the values stored in
noticeboard are formatted into a list of integer, float or logical values as
appropriate. If the item is of any other type the entire item is simply
returned as a string with no interpretation.

Notice board items can be written with:
\begin{tquote}{\tt
nbs put <nbs\_name> <value>
}\end{tquote}
where {\tt <nbs\_name>} is a fully qualified nbs item name and {\tt 
<values>} is the values to be written. 
The format of {\tt <values>} depends on the type of the nbs item; if it
is \_INTEGER, \_REAL, \_DOUBLE or \_LOGICAL, it must be a list of values of
that can be converted to the appropriate type by the usual rules and of
exactly the correct length to fill the item. If the type is \_CHAR it must
by a string, no longer than the length of the item; the string will be padded
with blanks to the correct length before being written to the noticeboard. If
the item has any other type {\tt <value>} must be a string of exactly the same
length of the item.

The command:
\begin{tquote}{\tt
nbs info <nbs\_name>
}\end{tquote}
returns a list of information about the nbs item. The first element of the list
which indicates whether the item is a primitive item (1) or a structure item
(0). If it is a structure item the remaining elements of the list are the names
of the children of the item. If it is a primitive item the remaining elements
are:
\begin{description}
\item[1] The type.
\item[2] The length. 
\item[3] A list of the dimensions. 
\end{description}

Noticeboard items can also be ``monitored''; at some specified time 
interval the 
value stored in the item is copied into a \Tcl\ variable, again using the
standard type conversions.
\begin{tquote}{\tt
nbs monitor <nbs\_name> <variable>
}\end{tquote}
adds {\tt <nbs\_name>} to the list of NBS items to be monitored and 
associates it with the global \Tcl\ variable {\tt <variable>}.
\begin{tquote}{\tt
nbs start <interval>
}\end{tquote}
starts monitoring all the NBS items in the list with the interval between 
checks set to {\tt <interval>} milliseconds.

\begin{tquote}{\tt
nbs stop
}\end{tquote}
stops the monitoring process and
\begin{tquote}{\tt
nbs clear
}\end{tquote}
empties the list of nbs items being monitored.

\begin{tquote}{\tt
nbs monitor
}\end{tquote}
returns a list of all the noticeboard items currently being monitored. 
Each element of the list consists of a two element list giving the 
notice board item name and the name of the corresponding \Tcl\ variable name.

Examples are given in appendix~\ref{noticeboard_examples}.

\newpage
\appendix
\section{\label{example_gwm_server_script}\xlabel{example_gwm_server_script}Example gwm server script}

\begin{small}
\begin{quote}
\begin{verbatim}
#!/star/bin/awish
#
# gwm.tcl
#
# This file is an example of using the gwm widget and associated procedures
#
# It creates a gwm server window with scroll bars for scrolling the
# window and buttons for changing colours, printing, clearing and exiting.
# A crosshair is optionally displayed at the cursor position.
#
# Any arguments to the script that are not recognised as wish options
# are used as arguments for the widget creation command.
#

# Add the location of the gwm procedures to the auto load path.
    lappend auto_path /star/lib/startcl

# Create the gwm widget with scroll bars. The arguments to this script
# are concatonated with the command to create the widget and the resulting
# string evaluated so that the script arguments become additional arguments
# to the command.
    set create [concat {gwm_gwmWithScroll .gwm} $argv]
    set gwm [eval $create]

# Create and pack a frame for the control buttons.
    pack [frame .bottom] -side bottom -fill x
    pack [frame .buttons -relief sunken -border 2] -padx 3 -pady 3 \
	-side right -in .bottom

# Pack a label along side the button frame to match the appearance of the
# frame
    pack [label .bottom.fill -relief sunken -border 2 -anchor w] -fill both \
	-padx 3 -pady 3 -side left -expand y

# Create the command buttons.
    button .buttons.exit -text Exit -command exit -padx 10
    button .buttons.colours -text Colours -padx 10 \
	-command "gwm_colourDialog .col $gwm .buttons.colours"
    button .buttons.clear -text Clear -padx 10 -command "$gwm clear"
    button .buttons.ovclear -text "Clear Overlay" -padx 10 \
	-command "$gwm ovclear"
    button .buttons.print -text Print -padx 10 \
	-command "gwm_printDialog .pr $gwm .buttons.print"
    checkbutton .buttons.crosshair -text Crosshair -padx 10 \
	-variable crosshair -command crossHair

# Pack the buttons into the frame.
    pack .buttons.exit .buttons.clear .buttons.print .buttons.crosshair \
	-side left -expand y -padx 5 -pady 5

pack .buttons.colours -after .buttons.exit  -padx 3 -pady 3 \
            -side left -expand y

# Pack the "clear overlay" button if the window has an overlay.
    if [$gwm cget -overlay] {
	pack .buttons.ovclear -after .buttons.clear  -padx 3 -pady 3 \
	    -side left -expand y
    }

# Bind a procedure that displays the current pointer position in a pop-up
# window to pressing mouse button 2.
    bind $gwm <ButtonPress-2> {showPointer %x %y}
    bind $gwm <ButtonRelease-2> {destroy .position}

# Change the cursor so that is doesn't interfere with the pop-up.
    $gwm configure -cursor draft_small

# Pack the gwm widget's frame into the top level widget. This is done last
# so that when the top level is resized it is the gwm widget that gets
# resized to fit rather than the frame containing the buttons.
    pack .gwm -in .

# Allow the window to be resized by the window manager by setting the
# minimum size.
    wm minsize . 1 1

# Map the window and find out how big it is and use this as the maximum 
# size allowed by the window manager. Also constrain the minimum size so
# that the buttons are always visible.
    update idletasks
    wm maxsize . [winfo width .] [winfo height .]
    wm minsize . [winfo reqwidth .buttons ] [winfo reqwidth .buttons ]

# Change the main window title to be the name of gwm window
   wm title . [$gwm cget -gwmname]

proc crossHair {} {
#+
# This procedure is called whenever the "crosshair" checkbutton is toggled
# and either enables the crosshair whenever the pointer is in the gwm 
# widget and binds the crosshair position to the pointer or disables the
# crosshair.
#-
    global crosshair
    global gwm
    if $crosshair {
	bind $gwm <Any-Motion> { %W set crosshair %x %y }
	bind $gwm <Any-Enter> { %W configure -crosshair yes }
	bind $gwm <Any-Leave> { %W configure -crosshair no }
    } {
	bind $gwm <Any-Motion> {}
	bind $gwm <Any-Enter> {}
	bind $gwm <Any-Leave> {}
	$gwm configure -crosshair no
    }
}
	
proc showPointer {x y} {
#+
# This procedure pops up a panel that displays values of the parameters
# x and y corrected for any scrolling of the gwm widget.
#-
    toplevel .position -bd 3 -relief raised
    wm overrideredirect .position 1
    global gwm
    set xpos [expr [winfo rootx $gwm] + $x]
    set ypos [expr [winfo rooty $gwm] + $y]
    wm geometry .position +$xpos+$ypos
    set x [expr $x - [$gwm cget -xoffset]]
    set y [expr $y - [$gwm cget -yoffset]]
    label .position.x -text "X = $x"
    label .position.y -text "Y = $y"
    pack .position.x .position.y
}
# End of script - enter event loop...
\end{verbatim}
\end{quote}
\end{small}

\newpage
\section{\label{noticeboard_examples}\xlabel{noticeboard_examples}Noticeboard Examples}

\subsection{\label{listing_notice_board_contents}\xlabel{listing_notice_board_contents}Listing a noticeboard contents}

\begin{quote}
\begin{verbatim}
proc showNbs {item} {
#+
# Lists the names and values of all the children of the noticeboard
# item $item. If $item is a noticeboard name, the entire noticeboard is
# listed.
#-
    set info [nbs info $item]
    if {[lindex $info 0]} {
        puts "$item: ([lindex $info 1]*[lindex $info 2]) \
            [lindex $info 3] [nbs get $item]"
    } {
        for {set i 1} {$i < [llength $info]} {incr i} {
           showNbs $item.[lindex $info $i]
        }
    }
}
\end{verbatim}
\end{quote}

\newpage
\section{\label{low_level_adam_message_system_commands}\xlabel{low_level_adam_message_system_commands}Low Level ADAM Message System Commands}

The command:

\begin{tquote}{\tt
adam\_start {\em name}
}\end{tquote}

initialises the Adam message system and registers {\em name} as the task
name and creates the following additional commands:

\begin{itemize}

\item {\bf adam\_path \em name}

Returns 1 if a path to {\em name} exists and 0 if it doesn't.

\item {\bf adam\_receive }

Receives an Adam message and returns a list who's
elements are:
\begin{quote}{\em
command task msg\_name path messid msg\_status msg\_value
}\end{quote}
where {\em command} is one of:
\begin{tquote}{\tt
actstart paramreq paramrep inform sync syncrep trigger startmsg 
endmsg getresponse setresponse controlresponse
}\end{tquote}
If {\em command} is {\tt paramreq} then {\tt msg\_value} is a list who's
elements are:
\begin{quote}{\em
parameter\_name prompt\_string default\_value help\_text error\_message
}\end{quote}
Note that if there is no message available {\bf adam\_receive} will wait
until a message arrives.

\item {\bf adam\_send {\em task msg\_name context msg\_value} }

Sends a message to a task starting a new transaction. {\em context}
must be one of:

\begin{tquote}{\tt
get set obey cancel control
}\end{tquote}
{\bf adam\_send} returns a list who's elements are the path and messid of
the new transaction.

\item {\bf adam\_reply {\em path messid msg\_status msg\_name msg\_value}}

Sends a message on the transaction specified by {\em path} and
{\em{messid}}.  Valid values for {\em msg\_status} are:

\begin{tquote}{\tt
actstart actcomplete paramrep paramreq inform sync syncrep trigger
}\end{tquote}

\item {\bf adam\_getreply {\em timeout path messid}}

Waits for an adam message to arrive on the given path and messid and
returns a list whose elements are the same as the list returned by {\bf
adam\_receive}.

\end{itemize}

\end{document}
