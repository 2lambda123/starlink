\PassOptionsToPackage{font=footnotesize}{caption}
\documentclass[twoside,11pt]{starlink}
\usepackage{graphicx}
\usepackage{pdfpages}
\usepackage{float}
\usepackage[labelformat=empty]{subfig}

\stardocauthors     {D.S. Berry }
\stardocdate        {14th April 2022}
\stardoctitle       {POL2MAP internal structure}
\stardocversion     {V1.1}
\stardocabstract    {A description of the internal structure of the
SMURF:POL2MAP script intended for maintainers.}

\begin{document}
\scfrontmatter

\newcommand{\ptmap}{\texttt{pol2map}\ }
\newcommand{\mmap}{\texttt{makemap}\ }
\newcommand{\sloop}{\texttt{skyloop}\ }
\newcommand{\cqu}{\texttt{calcqu}\ }
\newcommand{\cp}[1]{configuration parameter \texttt{#1}\ }
\newcommand{\pp}[1]{the \texttt{pol2map} parameter \texttt{#1}\ }
\newcommand{\Pp}[1]{The \texttt{pol2map} parameter \texttt{#1}\ }

\section{Introduction}
The \ptmap command within the Starlink SMURF package creates a
vector catalogue from one or more POL2 observations.  In addition, it can
also create maps of Q, U and I --- both for individual observations and for
the co-add of all supplied observations. It accepts either raw data or
partially processed data (\emph{i.e.} Q/U/I time-streams or maps for
individual observations) as input, and will apply the required processing
to each input file to create the final catalogue.

\subsection{Other Reading}
Information about \ptmap is available in the
following documents:
\begin{itemize}
\item The reference documentation describing the operation and parameters of
\ptmap is contained within SUN/258 ``SMURF - the Sub-Millimetre User Reduction Facility''.
\item A more extensive description of how to use \ptmap is available
in SC/22 ``The POL-2 Data Reduction Cookbook''.
\item An initial analysis made during POL-2 commissioning of the methods used within
\ptmap is available in file \texttt{smurf/docs/pol2map/pol2map.tex}.
\item A report on POL-2 data reduction methods created towards the end of
commissioning is is available in file \texttt{smurf/docs/pol2-dr/pol2-dr.tex}.
\item The JCMT Software Blog contains entries describing new \ptmap
features and related POL-2 analysis added subsequent to commissioning.
\item More information about POL2 in general and the IP model in
particular is available on the
\htmladdnormallink{{\em POL2 commissioning wiki}}
{https://www.eao.hawaii.edu/pol2/Data\%20Reduction\%20And\%20Analysis}
\latex{ (URL \texttt{https://www.eao.hawaii.edu/pol2/Data\%20Reduction\%20And\%20Analysis})}.
\item The code comments in the script file \texttt{\$SMURF\_DIR/pol2map.py}.
\end{itemize}

\section{The \ptmap Operating Environment}
The \ptmap command is an alias set up by the SMURF
initialisation script, which points to the \texttt{\$SMURF\_DIR/pol2map.py}
script. This is a stand-alone Python script that uses the facilities in
the file \texttt{\$SMURF\_DIR/starutil.py} to provide a Starlink-like
interface for the the \ptmap command. These facilities are used
by various other Python scripts in SMURF - e.g. \sloop, and include
\begin{itemize}
\item Defining and accessing script parameters. These may be provided on the
script's command line or via user prompts. Default values and acceptable
data ranges may be defined. By design these parameters attempt to mimic the
behaviour of ADAM parameters. Note though that many aspects of the ADAM parameter
system have not been replicated, such as the help system, the PROMPT and RESET
keywords and the facility for saving the value of each parameter in an external
file.
\item Running Starlink commands and accessing screen output or ADAM
parameter values written by such commands.
\item Managing groups of NDFs or single NDFs by providing an ``NDG'' class
that wraps the facilities of the Starlink NDG library (see SUN/2). Within the
\ptmap script, each group of one or more NDFs is represented by an
instance of the NDG class.
\item Display and logging of output information.
\item Various other utilities, such as getting the value of a FITS
header from an NDF.
\end{itemize}
The actual data processing required to produce the vector catalogue and
other outputs is performed within various other Starlink tasks - the
python code within the \ptmap script is used purely to glue these
other tasks together. Tasks from the SMURF, KAPPA and POLPACK packages
are used.

\section{Use of pol2map}
The usual way to use the \ptmap script is in a sequence of
three invocations, or ``steps''. This is described in section 3.1, ``The
Data Flow'', in SC/22. The first invocation creates an auto-masked total
intensity map. This total intensity map is passed to the second
invocation using \pp{MASK}, which uses it to create a mask
that outlines the areas containing significant total intensity. The
second invocation proceeds to create a new total intensity map using this
mask to define the AST mask on every invocation (i.e. external masking is
used rather than auto-masking). The third invocation creates Q and U
masks using the same external mask as the second invocation. It also uses
the externally masked total intensity map created by the second
invocation to define the expected level of Instrumental Polarisation (IP)
added into the Q and U data. This IP is removed (within \mmap)
before the Q and U maps are created. Having created the externally-masked
Q and U maps, the third invocation finished by using all three externally
masked maps to create the final vector catalogue.

The second and third invocations can be combined into a single invocation
by assigning values to all three map output parameters (IOUT, QOUT and
UOUT) and using the new total intensity map (i.e. the value assigned to
IOUT) as the IP reference map assigned to \pp{IPREF}.


\section{High Level Structure}

The \ptmap script has the following sections:

\begin{verbatim}
- prologue
- imports and function definitions
- Get values for all script parameter
- Classify the input data files
- Create Stokes time series from raw input data
- For Stokes = ( I, Q, U ):
   - Create maps from all time series for the current Stokes parameter
   - Coadd all maps for the current Stokes parameter
   - Bin the coadd if the vector catalogue is to use a different bin size
- Create the vector catalogue from the binned coadds
- Clean up
\end{verbatim}


The \ptmap script starts with a Starlink-style prologue
containing a description of the command and its parameters. This prologue
is the primary source for the reference documentation included in SUN/258
and should be updated each time any functionality is changed. It is
processed using the Starlink SST package when SMURF is built, to generate
the latex documentation included in SUN/258.

Then follows a bunch of imports and the definition of some global
variables. Several functions are then defined for later use before the
main entry point is reached. The rest of the code is then divided into the
following top-level functional blocks:

\subsection{Get Values for all Script Parameters}
Define the names, types, allowed values, etc, for all script
parameters, and then get the value to use for each parameter - either
from the command line or from a user prompt. Perform additional checks
on the values and derive useful related values from them. The functions
for defining and accessing script parameters are in
\texttt{\$SMURF\_DIR/starutil.py}.


\subsection{Classify the Input Data Files} \Pp{IN} is used to
access a group of input NDFs. Each one is checked to see what sort of
data it holds. Allowed forms are:
\begin{itemize}
\item SCUBA-2 data files containing raw POL-2 analysed intensity time
series.
\item SCUBA-2 data files containing down-sampled Q, U or I time series
(these are derived from the corresponding raw analysed intensity time
series).
\item Two-dimensional maps holding previously generated Q, U or I images
(these are derived from the corresponding down-sampled Q, U or I time
series).
\end{itemize}
The \texttt{smurf:pol2check} command is used to do this.

\subsection{Create Stokes Time Series from Raw Data}
If the list of input data files contained any raw POL-2 analysed intensity
time series, then these are processed using \cqu to create
down-sampled Q, U and I time series files. All raw data files relating to
an observation are processed by \cqu together. Each
observation is processed separately. The resulting down-sampled Q, U and
I time-stream files are placed in the directory specified by the script's
QUDIR parameter. If this directory already contains pre-existing files
with the same names, then they are retained in preference to creating new
files by running \cqu (so long as \pp{REUSE} script
parameter has not been set to FALSE).

\subsection{Create Maps from Stokes Time Series\label{se:maps}}
The next section creates maps from any input I, Q, U time streams - both
those supplied directly via the ``IN'' parameter and those created from
any raw  data files specified via ``IN''. There are two very different
ways in which this can be done:
\begin{itemize}
\item If script parameter ``SKYLOOP'' is FALSE (the default), then
\texttt{smurf:makemap} is used to process the I, Q or U time-series files
for each observation separately. Thus \mmap is run three times
for each observation creating an I map, a Q map and a U map for each
individual observation. Once all observations have been processed, all the
I maps (one for each observation) are combined to form an I co-added.
Likewise all the Q maps are combined to form a Q co-add and all the U
maps are combined to form a U coadd.
\item If script parameter ``SKYLOOP'' is TRUE, then \texttt{smurf:skyloop}
is used three times to create I, Q and U coadds. Each invocation of
\sloop processes all available I, Q or U time-stream files from
all observations. By default, \sloop does not create individual
maps for individual observations - it just creates the coadd of all
observations. However, supplying a value for the skyloop OBSDIR parameter when
running \sloop causes maps for individual observations to be
placed into the directory specified by OBSDIR. These are only usually
needed if the \ptmap MAPVAR parameter is set TRUE, indicating
that the variances in the final I, Q, U coadds and catalogue should be
based on the spread of values seen between the individual observation maps.
\end{itemize}
If MAPVAR=YES, using SKYLOOP=YES produces lower variances than
SKYLOOP=NO. See section 3.7 of the POL-2 cookbook (SC/22). If MAPVAR=NO,
there seems to be little benefit in using \sloop.
\subsection{Create the Coadded Maps}
If SKYLOOP=NO, then all the individual I maps (one for each observation)
are coadded to create an I coadd. Likewise Q and U coadds are formed from
the individual Q or U maps. If MAPVAR=YES then the Variance values at each
pixel in the
three coadds are formed from the spread of values between the individual
observation maps at the same pixel position\footnote{These variances will
only be meaningful is a reasonable number of observations are being
coadded.}. If MAPVAR=NO, the coadd Variances are calculated on the basis
of the Variance values stored in the input maps, as created by
\mmap. The MAPVAR=NO Variances are usually a lot smaller
(unrealistically small, it may be thought), compared to the MAPVAR=YES
variances.

If SKYLOOP=YES, then the I, Q and U coadds already exist, having been created
directly by \sloop. However, if MAPVAR=YES, then the Variances
in these coadds need to be replaced by Variances formed from the differences
between the individual observation maps. To do this, the individual
observation maps formed by \sloop are coadded using
\texttt{kappa:wcsmosaic} with GENVAR=YES. This causes the required Variances
to be stored in the output NDF created by \texttt{wcsmosaic}. Note, the
relationship between the individual observations maps and the coadd
created by \sloop is not straightforward - the coadd is not
literally the co-addition of the individual observation maps. To create
an individual observation map, \sloop adds together the AST
model from the final iteration with the RES (residuals) model for an
individual observation, and then forms a map from the sum. For this
reason, the literally co-addition of the individual observation maps is
less reliable than the actual coadd formed directly by \sloop.
Therefore, the Variances created by \texttt{wcsmosaic} are copied into
the coadd created directly by \sloop, rather than just adopting
the \texttt{wcsmosaic} output NDF as the whole coadd.

\subsection{Create the Maps to Use When Creating the Catalogue}
If an output vector catalogue is being created, we need to ensure that we
have coadded maps, with Variances, that have the required pixel size specified
by script parameter BINSIZE. In the face of some pressure from a few users
to do otherwise, my advice has always been to create the
initial maps using 4 arc-sec pixels - for both 850 and 450 um data - and
then bin these maps up to a larger size prior to forming the catalogue
(as if often desired by users to decrease the noise in the vector
maps)\footnote{The alternative would be to run \mmap or
\sloop with pixel size set to BINSIZE.}.
This is because the behaviour of the map-making algorithm
has been more thoroughly investigated at 4 arc-seconds than at larger
pixel sizes. Also, the effects of pixel size on pixel SNR values, and thus
on mask sizes (which are usually defined by SNR cuts), is not always
understood by users. Increasing the map-maker pixel size causes the SNR
in each pixel to increase, thus causing SNR-based masks to increase in
area, thus muddying the waters when trying to assess the noise-reducing
effects of increasing the pixel size\footnote{Any perceived improvement
may simply be a consequence of increasing the mask size, rather than
increasing the pixel size.}.

However there is a complication associated with binning up the coadded
maps. If you simply use \texttt{kappa:sqorst} (for instance) to bin up
the maps, the variances in the binned up map will be determined assuming
there are no pixel-to-pixel correlations in the noise in the input map -
in other words, it is assumed that the noise will drop as the square root
of the number of input values included in each output pixel. But SCUBA-2
maps \emph{do} have pixel-to-pixel noise correlations, caused by the artificial
large-scale structures introduced by the map-making algorithm. For this
reason, the output Variances created by \texttt{sqorst} would be
artificially low, since \texttt{sqorst} assumes that neighbouring noise
values will tend to cancel out when coadded more than they actually do in
the case of SCUBA-2 maps.

To generate realistic variances in the binned up maps, it is necessary to
bin up the individual observation maps and then create a new coadd with the
larger bin size, generating new variances (in the case of MAPVAR=YES) in
the process by looking at the variance between the binned up values at
each point on the sky. This is done using ``\texttt{wcsmosaic
genvar=yes}''.

\subsection{Create the Vector Catalogue}
The I, Q and U coadds at the required pixel size specified by script
parameter BINSIZE are first trimmed so that they have the same pixel
index bounds. They are then stacked into a cube as required by the
\texttt{polpack:polvec} command, which is used to create the vector
catalogue. Meta-data required by \texttt{polvec} is added to the cube,
including suitable WCS. If required, the cube data is then converted from
units of pW to units of mJy/beam (see script parameter JY). The FCF used
to do this conversion is calculated by \mmap or
\sloop and stored as a FITS header in the resulting maps. These
values are the standard Mairs et al (2021) values, but use a weighted
average if the data in the coadds straddle a date at which the FCF changed.
The catalogue is then created using \texttt{polvec}. An extra column is
then added to the catalogue that defines whether each vector is inside or
outside the AST mask. A similar column is also added for the PCA/FLT/COM
mask.

\subsection{Clean up - and debugging}
The names of any new maps created are recorded in the file specified by
script parameter NEWMAPS, then all temporary NDFs created during the
execution of the script are deleted. These NDFs can be retained for
debugging purposes by setting script parameter RETAIN=YES when running
\ptmap. These NDFs are stored in a subdirectory inside the
directory specified by environment variable STAR\_TEMP, which defaults to
\texttt{/tmp} is not specified. The path to this subdirectory is displayed when
the script terminates. The log file created by \ptmap (see
parameter LOGFILE) contains a record of each starlink command executed by
\ptmap. This, together with the RETAIN parameter, allows for
easy (ish) debugging.

\section{Specific Issues}
\subsection{Masking}

When creating maps from I, Q or U time-series data - using either
\mmap directly or via \sloop - spatial masks are used for
various purposes. One mask is used to mask the AST model within
\mmap / \sloop and another mask is used to mask the COM, FLT and PCA
models (all three of these models are masked using the same mask).

How to control the extent of these masks is described in
\htmladdnormallink{{\em this blog post}}
{https://www.eaobservatory.org/jcmt/2019/07/controlling-the-masks-used-by-pol2map/}
\latex{ (URL \texttt{https://www.eaobservatory.org/jcmt/2019/07/controlling-the-masks-used-by-pol2map/})}.


\subsubsection{Masking the AST model}
The AST mask is used to suppress the growth of large scale gradients across
the map by resetting the areas outside the mask to zero at the end of
each \mmap or \sloop iteration. It should be large enough to enclose all
areas where significant emission is expected, but no larger.

\paragraph{Auto-masking:}
When \pp{MASK} is set to the default value of ``AUTO'' (typically only done
on the ``step 1'' invocation of \ptmap), a new AST mask is determined
automatically at the end of each \mmap or \sloop iteration. The new mask
is created by converting the current estimate of the final map into an SNR
map and performing a cut at the SNR value given by \cp{ast.zero\_snr} (which
defaults to 3) and then extending each ``island'' in the mask down to the
SNR value given by \cp{ast.zero\_snrlo} (which defaults to 2). This continual
changing of the mask after each iteration can sometimes upset the convergence
of the algorithm, for which reason \ptmap uses \cp{ast.zero\_freeze} to freeze
the mask when the relative map-change between iterations drops to 0.2.

\paragraph{External-masking:}
If \pp{MASK} is set to the name of an existing NDF, then the NDF is used in a
manner determined by \pp{MASKTYPE}. By default, it is assumed that the NDF
holds an intensity map of the sky. This map is converted to an SNR map and is
thresholded at a  value of 3.0. Each island of emission is then extended down
to an SNR of 2.0 to create the final mask. This is done using
\texttt{cupid:findclumps} and occurs just before maps are created from the
Stokes time-series data (see section \ref{se:maps}). Note, very bright
sources are problematic because they can produce masks that cover nearly
the whole map. To avoid this, if the source regions in the mask occupy
more than 20\% of the good map pixels, then a new mask is created with a
higher SNR threshold. This process is repeated until the source regions
occupy less than 20\% of the good pixels (the SNR threshold is raised by
a factor of 1.2 on each loop).

The same mask is used on all iterations of \mmap or \sloop.

External masking is usually used on the ``step 2'' and ``step 3''
invocations  of \ptmap) to create the final I, Q and U co-adds. Note, the
image supplied for \pp{MASK} is normally the total-intensity map created
by the ``step 1'' invocation of \ptmap. The same mask is used to create
the externally masked Q and U as well as the externally masked I map.
Often, there is not much evidence for real emission in the Q and U maps,
as the polarised intensity measured by Q and U is typically only a few
percent of the total intensity. The external mask derived from the ``step
1'' total intensity map will thus often enclose areas where there is no
significant Q or U emission. This could potentially allow large scale
artificial structures to grow in these areas (the bane of all SCUBA-2 data
reduction). However, we continue to advise the use of the same mask for Q
and U as for I, because processing all three Stokes parameters in the
same way seems the most likely way to produce comparable results.


\subsubsection{Masking the COM, FLT and PCA models}
The other mask is used to exclude time-series data from the calculation
of the COM, FLT and PCA models. Bright sources particularly can upset the
calculation of these models, for instance producing ringing in the FLT
model or negative bowling in the COM model. Time-series samples that fall
within this mask are excluded from the calculation of each of these
models. This mask should enclose just the brightest areas of the expected
emission. The action of this masking results in these three models being
largely unconstrained within the source areas of the mask. In other words, any
spurious features that develop in the map within the masked regions will be
blanked out by the masking and thus have no effect on the models, thus
allowing the spurious features to remain in the map and possibly grow in
strength. This gives rise to two considerations regarding this mask:

\begin{enumerate}
\item The masking should only be applied for a few iterations at the
start of the map-making algorithm. By default, the masking is applied
only until the relative map-change between iterations drops to 0.5
(specified by \cp{xxx.zero\_niter}, where \texttt{xxx} is \texttt{flt},
\texttt{com} or \texttt{pca}). By that time, most of the astronomical
signal should have been moved out of the time-stream residuals and into
the map, resulting in little need for further masking of the time-streams
when calculating the COM, FLT and PCA models.

\item The mask should be no larger than necessary, thus minimising the
unconstrained areas of the map. How large is ``necessary'' is unclear.
The default size (which, when using external masking, is determined by an
SNR cut-off of 3 extended down to 2) has recently been shown to be
sufficiently large to allow arbitrary large scale structures to appear in
the masked regions. These structures appear to be weak but are
sufficient to produce a noticeable zero-point offset when Q values at 450
um are plotted against Q values at 850 um (restricted to the interior of
the mask). The arbitrary structures that appear at the two wavelengths are
independent of each other, and so will in general cause the values at the
two wavelengths to be offset in different ways from the ``truth'',
resulting in a systematic difference between the two wavelengths in the
masked regions. Exploring the possibility for reducing the size of this
second mask in order to minimise the offset between 450 um and 850 um is
something which could be useful. Ray Furuya has done some work in this
area, having been the first person to spot the 450/850 offset.
\end{enumerate}

\subsection{Use of PCA}
The Common-mode (COM) model describes a single time-stream signal present
in all bolometers. The Principal Component Analysis (PCA) model extends
this concept in order to represent multiple independent time-stream
signals (``principal components'') where each one is present in only a
subset of bolometers. The common-mode signal can be thought of as being
equivalent to the strongest principal component.

Effectively, the PCA model caters for cases where the ``common-mode''
signal varies across the focal plane. Varying common-mode is not a big
problem for normal SCUBA-2 observations, where the bulk of the
common-mode comes from sky emission that varies with time but is more or
less constant across the focal plane at any one moment. However, it can be
a significant feature of POL2 data because there is evidence that the
Instrumental Polarisation varies significantly across the focal plane.
This varying IP, acting on the flat sky emission incident on the telescope,
will produce instrumental Q and U signals that vary across the focal
plane. These form the common-mode signal when producing maps of Q and U.

Of course there are other significant sources of non-common ``common-mode''
present in all SCUBA-2 data. Non-POL2 data can sometimes benefit from the
use of a PCA model, but in most cases the much higher scan speed and
sample rate (compared to POL2 Q and U time-streams) make it untenable
since the PCA model takes a long time to calculate and remove.

In \ptmap, both COM and PCA models are used by default, with COM being
calculated and removed before PCA on each iteration. In principle, it
should not be necessary to use COM in addition to PCA since PCA should
remove the common-mode signal anyway. But COM and PCA calculate the
strongest correlated component in very different ways and so do not
produce the same result. In practice, it seems that flatter maps are
obtained if COM is removed independently before removing PCA.

The most important parameter of the PCA model is the number of correlated
signals (principal components) to remove. This is specified by \cp{pca.pcathresh}.
If it was set to one, the PCA and COM models would be roughly equivalent,
both removing the single strongest correlated signal. In practice, it
defaults to 150 when using external masking and 50 when using
auto-masking\footnote{Note, to specify an absolute number of components
to remove, the supplied value for \cp{pca.pcathresh} must be negative.}.

Increasing the value of \cp{pca.pcathresh} has the following effects:

\begin{enumerate}
\item On each iteration, the PCA model takes longer to calculate and remove
as the number of components removed increases.
\item The final map becomes flatter as the number of components removed
increases. This diminishes artificial large scale structures but can also
remove real astronomical structure if the value is too high.
\item The number of map-maker iterations needed to reach convergence
increases as the number of components removed increases. This slow-down
in convergence is a function of the brightness of the sources within the
data. Since sources within I maps are much stronger than within Q or U
maps, the slow-down is much more noticeable when creating I maps.
\end{enumerate}

This final point can make it impossible to create converged I maps within
a reasonable time in some cases, using the default values for
\cp{pca.pcathresh}. The \ptmap script contains a trap for such cases when
\pp{SKYLOOP} is set to FALSE (i.e. typically on step 1 - the auto-masked
invocation). In such cases, \mmap is run with the ABORTSOON command line
parameter set TRUE. This causes \mmap to abort as soon as it becomes
clear that the convergence criterion (specified by configuration parameter
MAPTOL) will not be reached within the number of iterations allowed by
\cp{NUMITER}\footnote{This is determined at the end of each iteration from
a scatter plot of log(relative map change) against log(iteration number)
for all completed iterations. Such a plot is usually close to a straight
line and so can be extrapolated to determine the likely number of
iterations needed to reach the required relative map change, specified by
\cp{MAPTOL}.}. If \mmap is aborted in this way, \ptmap will re-run \mmap
again using a lower value for \cp{pca.pcathresh}. This continues until a
value for \cp{pca.pcathresh} is found for which a map can be created
within the maximum number of iterations.

Consistency of processing is a key theme in \ptmap. By itself, the above
scheme for modifying the value of \cp{pca.pcathresh} could lead to
different observations being processed with different values. To avoid
this, the above scheme is only used if:

\begin{itemize}
\item No value for \cp{pca.pcathresh} has been supplied by the user. If
the user supplied a value, it is always used, even if doing so means that
convergence is not reached within the allowed number of iterations.
\item No previous maps have been made for the current field. Thus, the
above scheme is only used when producing maps from the first observation.
Subsequent observations use the same value even if that means that
convergence is not reached within the allowed number of iterations. The
value of \cp{pca.pcathresh} that was used to produce a map is stored with
in its NDF History component and is accessed from there on subsequent
invocations of \ptmap. Note, this means that when \ptmap is run with
\pp{SKYLOOP} set to TRUE (typically step 2 or 3), the value to use for
\cp{pca.pcathresh} will be read from the first auto-masked I map created
by the earlier step 1 invocation of \ptmap.
\end{itemize}

Note, the value for \cp{pca.pcathresh} that is used to produce I maps is
also used to produce Q and U maps. In general, the signal is much weaker
in Q and U maps than in I maps and so it would be possible to use a
higher value for \cp{pca.pcathresh}. This is not done however, in order
to retain consistency in the creation of I, Q and U maps.

The calculation and subtraction of the PCA model is performed within \mmap.
See file \texttt{libsmf/smf\_calcmodel\_pca.c} in the SMURF source code
repository.

\subsection{Instrumental Polarisation Correction}

A huge amount of effort has been put into determining an effective IP
model for POL2 - starting with the work of James Kennedy who tried to use
POL2 skydip observations to determine an IP model that included spatial
variation across the focal plane. When the original ``stare \& spin''
observing mode was replaced by the current ``scan \& spin'' mode, a new
simpler IP model that included no spatial variation was derived from
observations of an unpolarised point-like source (Uranus - see SMURF script
\texttt{pol2ip.py}). This model was tweaked at various times, but could
not be made to do an effective correction at 450 um (i.e. variation of Q
and U with elevation remained even after IP correction). This model was
then replaced by a new model determined from a set of observations of
bright extended (possibly polarised) sources - see blog post
https://www.eaobservatory.org/jcmt/2019/08/new-ip-models-for-pol2-data/).

Both the current IP model and the previous Uranus-based IP model
make no attempt to include any model of the physical processes that
create IP within the telescope and instrument. They both just fit
arbitrary polynomials to the observed IP. This is different to the
earlier model implement by James Kennedy, which was based on the expected
behaviour of IP, given the physical processes which create it. Pierre
remains convinced that the IP model could be improved by incorporating
more knowledge of the physics and optics involved in its creation.  He
has produced several attempts at such models but none have produced
results any better than those of the simpler models based on arbitrary
polynomial fits. The usual test for an IP model is to porocess a set of
Uranus observations. Since Uranus is assumed to be intrisically
unpolarised, and significant Q or U emission in the final maps indicate
some failing in the IP model.

The subtraction of the IP is performed within \mmap, as part of the
initial data processing performed before the iterative map-making
algorithm begins. See file \texttt{libsmf/smf\_subip.c} in the SMURF
source code repository.

\subsection{Pointing Corrections}
The first invocation of \ptmap, ``step 1'' - used to create the
auto-masked I map, produces an I map from each observation independently
using \mmap. These are then coadded. Any pointing error within an
individual observation will result in the sources in the I map made from
that observation being slightly displaced from those in the coadd. After
the coadd has been created, each individual observation map is compared
to the coadd using \texttt{kappa:align2d}. This finds the sub-pixel shift
needed to minimize the differences between the coadd and the individual
observation map. These offsets are stored in the FITS extension of the
individual observation I map as headers PNTRQ\_DX and PNTRQ\_DY (arc-sec
parallel to the celestial longitude and latitude axes).

When ``step 2'' is performed, these offsets are read back from the step 1
I maps (stored in the directory specified by \pp{MAPDIR}) and are used to
create a text file which is then passed to \sloop and thus to \mmap
ADAM parameter \texttt{POINTING}. This file contains details of the
pointing corrections to be applied to the tracking information stored in
each time-series data file. It is given as tables of longitude and
latitude shift as function of observation time. Thus the pointing errors
that are determined at the end of step 1 are corrected for and
removed when maps are made in steps 2 and 3.

The pointing corections that were applied by \mmap during the creation of
an observation map are stored in the map's FITS extension in headers
POINT\_DX and POINT\_DY. These differ from headers PNTRQ\_DX and
PNTRQ\_DY, which hold the remaining pointing error still present in the
map, determind by comparing the observation map to the coadd.

\subsection{Units and FCFs}
All maps produced by \ptmap are always in units of pW. This includes I, Q
and U co-adds and all the maps stored in the MAPDIR
directory\footnote{There have been one or two requests for a switch to
allow the maps to be produced in mJy/beam.}. On the
other hand, the I, Q, U and PI (polarised intensity) columns within the
output vector catalogue are in mJy/beam by default\footnote{This can be changed
by running \ptmap with \texttt{jy=no}, in which case the catalogue holds
intensity values in pW.}.

The FCF used to convert pW to mJy/beam can be supplied by the user using
\pp{FCF}. If not supplied, a default is used which depends on the
waveband (450 or 850 um) and the observation dates. These default FCFs
are taken from Mairs et al (2021).

BISTRO fields typically have 20 observations, which are usually all processed
together by \ptmap. So there is a significant possibility that \ptmap
will be supplied with observations that span both sides of a date when
the FCF changed. In such cases the default FCF used to convert pW to
mJy/beam is a weighted mean of the FCFs before and after the change. The
weights are proportional to the amount of usable data on either side of the
change date. How these are calculated depnds on whether \mmap or \sloop
is being used to produce the maps:

\begin{description}
\item[\mmap]:
The nominal FCF for each observation is determined from the observation
start time. The weight for each observation map is equal to the median
exposure time in the map, as stored in the EXP\_TIME keyword in the map's
FITS extension. The FCF for the coadd is then just the weighted mean of
th nominal FCFs for all observation maps.

\item[\sloop]: The simple procedure described above cannot be used with
\sloop since \sloop does not calculate the median exposure time for each
individual observation map. In this case the FCF is calculated within
\mmap, which is invoked on each iteration of \sloop. The FCF is
calculated as a similar weighted mean within \mmap, averaged over all the
supplied chunks of data (i.e. observations). The nominal FCF for each
chunk is determined on the basis of the observation date for the chunk.
The associated wirght is the total exposure time summed over all pixels
in the map. The resulting FCF value is stored in the NOMFCF header in the
output map's FITS extension, from where it is retrieved within \ptmap.
\end{description}

\subsection{The Default \ptmap Configuration}
\Pp{CONFIG} allows the user to specify values for configuration
parameters that control the map-making process. Any user-supplied
configuration is augmented by various values determined within \ptmap and
then passed on to \mmap or \sloop. See the user docs for \pp{CONFIG} for
details. It is usually not necessary for the user to supply a value for
\texttt{CONFIG} since the default configuration provided by \ptmap
is usually sufficient. Most of this default configuration was determined over
an extensive period during POL2 commissioning. The analyses that lead to
some of the settings are described in articles on the POL2 commissioning
wiki, although many are not. In addition, several years of experience
with POL2 data has potentially made some of these earlier settings ready
for review, although doing a complete re-assessment of the optimal
configuration would be a very big job.

The exact content of the default configuration depends on the type of
masking being done - auto-masking or external masking - as determined by
\pp{MASK}.


\end{document}
