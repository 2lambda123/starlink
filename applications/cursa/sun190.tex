\documentclass[twoside,11pt]{starlink}
%
% CURSA Catalogue and Table Manipulation Applications: User's Manual
%
% Copyright 2001 Council for the Central Laboratory of the Research
% Councils.

%------------------------------------------------------------------------------

%
% Set the CURSA version number.

\providecommand{\CURSAversion}{6.4~}

%------------------------------------------------------------------------------

% ? Specify used packages
% ? End of specify used packages


%------------------------------------------------------------------------------


% -----------------------------------------------------------------------------
% ? Document identification
% Fixed part
\stardoccategory    {Starlink User Note}
\stardocinitials    {SUN}
\stardocsource      {sun\stardocnumber}
\stardoccopyright
{Copyright \copyright\ 2001 Council for the Central Laboratory of the Research Councils}

% Variable part - replace [xxx] as appropriate.
\stardocnumber      {190.11}
\stardocauthors     {A.C.~Davenhall}
\stardocdate        {4th November 2001}
\stardoctitle     {CURSA \\ [1ex]
                            Catalogue and Table Manipulation Applications}
\stardocversion     {Version \CURSAversion}
\stardocmanual      {User's Manual}
\stardocabstract
{CURSA is a package of Starlink applications for manipulating astronomical
catalogues and similar tabular datasets. It provides facilities for:
browsing or examining catalogues, selecting subsets from a catalogue,
sorting catalogues, copying catalogues, pairing two catalogues, converting
catalogue coordinates between some celestial coordinate systems, plotting
finding charts and photometric calibration.  Also, subsets can be extracted
from a catalogue in a format suitable for plotting using other Starlink
packages, such as PONGO.  CURSA can access catalogues held in the popular
FITS table format, the Tab-Separated Table (TST) format or the Small Text
List (STL) format.  Catalogues in the STL and TST formats are simple ASCII
text files.  CURSA also includes some facilities for accessing remote
on-line catalogues via the Internet.

This manual describes how to use version \CURSAversion of CURSA.  Its
intended readership is users and potential users of CURSA.}

% ? End of document identification

% -----------------------------------------------------------------------------
% ? Document specific \providecommand or \newenvironment commands.

% Define commands for displaying angles as sexagesimal hours and minutes
% or degrees and minutes.

\providecommand{\tmin}   {\mbox{$^{\rm m}\!\!.$}}
\providecommand{\hm}[3] {$#1^{\rm h}\,#2\tmin#3$}
\providecommand{\dm}[2] {$#1^{\circ}\,#2\raisebox{-0.5ex}{$^{'}$}$}
\providecommand{\arcmin} {\raisebox{-0.5ex}{$^{'}$} }

\providecommand{\arcsec} {$\hspace{-0.05em}\raisebox{-0.5ex}
                     {$^{'\hspace{-0.1em}'}$}
                     \hspace{-0.7em}.\hspace{-0.05em}$}
\providecommand{\tsec}   {\mbox{$^{\rm s}\!\!.$}}
\providecommand{\hms}[4] {$#1^{\rm h}\,#2^{\rm m}\,#3\tsec#4$}
\providecommand{\dms}[4] {$#1^{\circ}\,#2\raisebox{-0.5ex}{$^{'}$}\,#3\arcsec#4$}

% ? End of document specific commands

% -----------------------------------------------------------------------------
%  Title Page.
%  ===========
\begin{document}
\scfrontmatter

\begin{center}
{\Large\textbf{CURSA Quick Reference}}
\end{center}

To set up for using CURSA type: ~~ \texttt{cursa}

\subsubsection*{Applications}

\begin{tabular}{ll}

\texttt{xcatview}     & browse and generate selections from a catalogue
                     (X-windows, easy-to-use), \\
\texttt{catview}      & browse and generate selections from a catalogue
                     (command line), \\
\texttt{catselect}    & select a subset from a catalogue, \\
\texttt{catcoord}     & convert between celestial coordinate systems, \\
\texttt{catchart}     & plot a finding chart, \\
\texttt{catchartrn}   & set up ready for plotting a finding chart, \\
\texttt{catheader}    & list various header information for a catalogue, \\
\texttt{catcopy}      & copy a catalogue, \\
\texttt{catsort}      & sort a catalogue, \\
\texttt{catpair}      & pair two catalogues, \\
\texttt{catgrid}      & bin one, two or three columns into a histogram,
                     image or data cube, \\
\texttt{catphotomfit} & define photometric transformation coefficients, \\
\texttt{catphotomtrn} & apply photometric transformation coefficients to
                     programme objects, \\
\texttt{catphotomlst} & list photometric transformation coefficients, \\
\texttt{catcdsin}     & convert a CDS text catalogue to STL format, \\
\texttt{catgscin}     & convert a region in the HST \textit{Guide Star Catalog}\,
                     to a more convenient format, \\
\texttt{catremote}    & access remote on-line catalogues. \\
\end{tabular}

\subsubsection*{Catalogue formats}

\begin{description}

  \item[FITS tables] (file types: \texttt{.FIT .fit .FITS .fits .GSC .gsc}).
   Binary and ASCII FITS tables.

  \item[TST] (file types: \texttt{.TAB .tab}).
   The Tab-Separated Table format used by GAIA.

  \item[STL] (file types: \texttt{.TXT .txt}).
   The Small Text List format.

% \item[FITS tables] (file types: \texttt{.FIT .fit .FITS .fits .GSC .gsc}).
%  Both binary and ASCII FITS tables can be read; only binary
%  FITS tables can be written. To access tables in extensions other than
%  the first enclose the required extension number in curly brackets after
%  the file name, for example \texttt{catname.FIT\{3\}} for a table in the
%  third extension.

% \item[TST] (file types: \texttt{.TAB .tab}).
%  The Tab-Separated Table format.  To suppress the interpretation of
%  Right Ascension and Declination append \texttt{SIMPLE}' in curly brackets
%  after the file name, for example \texttt{catname.TAB\{SIMPLE\}}.

% \item[STL] (file types: \texttt{.TXT .txt}).
%  The Small Text List format.  To write a KAPPA variant STL append
%  `\texttt{KAPPA}' in curly brackets after the file name, for example
%  \texttt{catname.FIT\{KAPPA\}}. KAPPA variant STLs can be read by simply
%  giving the file name.

\end{description}

\subsubsection*{Expressions}

\textbf{arithmetic operators:} \texttt{+ - * / **}  \\
\textbf{relational operators:}  \texttt{.EQ. .NE. .GE. .GT. .LE.
   .LT. == /= >= > <= <}  \\
\textbf{logical (boolean) operators:} \texttt{.AND. .OR. .NOT. \& $|$ \# }  \\
\textbf{brackets:} use brackets, `\texttt{(}', `\texttt{)}', as appropriate,  \\
\textbf{sexagesimal values:} use a colon (`:') to separate
   hours/degrees, minutes and seconds. Unsigned values are interpreted
   as hours; values in degrees must always have a sign (`\texttt{+}' or
   `\texttt{-}'). Sexagesimal values are converted to radians prior to
   evaluating the expression.  \\
\textbf{great circle distance:}
   \texttt{GREAT(}$\alpha_{1},\delta_{1}$,$\alpha_{2},\delta_{2}$\texttt{)} \\
\textbf{position angle} of point $(\alpha_{2},\delta_{2})$ from point
  $(\alpha_{1},\delta_{1})$:
  \texttt{PANGLE(}$\alpha_{1},\delta_{1}$,$\alpha_{2},\delta_{2}$\texttt{)}
\newpage




\subsection*{CURSA home page and on-line documentation}
\label{HOMEPAGE}

A `home page' giving useful information about CURSA is available via
the World Wide Web. Its URL is:

\begin{quote}
\url{http://www.starlink.ac.uk/cursa/}
\end{quote}

An on-line version of SUN/190 (this manual) is also available via the
World Wide Web.  On Starlink systems type:

\begin{terminalv}
% showme sun190
\end{terminalv}

Otherwise access URL:

\begin{quote}
\url{http://www.starlink.ac.uk/docs/sun190.htx/sun190.html}
\end{quote}


\subsection*{Assistance and further information}

If you are experiencing difficulties using CURSA then in the first
instance you should probably seek advice and assistance from your local
site manager.  Bug reports should be sent to username:

\begin{quote}
\texttt{starlink@jiscmail.ac.uk}
\end{quote}

Bug reports should always be sent to username \texttt{starlink@jiscmail.ac.uk}.
However, you are welcome to contact me directly for advice and assistance.
Suggestions for enhancements and improvements to CURSA are also
welcome.  Details of how to contact me are given below.
\begin{flushright}
Clive Davenhall \\
\end{flushright}

\vspace{3mm}

Postal address: Institute for Astronomy, Royal Observatory, Blackford Hill,
Edinburgh, \\
\begin{tabular}{l}
EH9 3HJ, United Kingdom.  \\
\end{tabular}

\vspace{4mm}

Electronic mail: \texttt{acd@roe.ac.uk}

\vspace{4mm}

Fax: \\
\begin{tabular}{lr}
from within the United Kingdom: &    0131-668-8416 \\
from overseas:                  & +44-131-668-8416 \\
\end{tabular}


\newpage
\subsection*{Acknowledgments}

CURSA is far from being all my own work. Clive~Page, Rodney~Warren-Smith
and Alan~Wood have all been involved in aspects of its development.
Indeed, Clive~Page wrote the expression parser which CURSA uses, and
Appendix~\ref{EXPR} is based on documentation which he supplied.
Malcolm~Currie and Anne Sansom tested an early version of \texttt{xcatview}
and suggested several significant improvements. Numerous other people have
made useful contributions.

I am grateful to everyone who has contributed time and expertise.
\begin{flushright}
Clive Davenhall \\
\raggedright \textit{Department of Physics and Astronomy, University of
Leicester \\
Saint Indract's Day 1995}
\end{flushright}

Various items of external software have been introduced into version 3.1
of CURSA which it is a pleasure to acknowledge.  \texttt{catremote} accesses
remote on-line catalogues using the \textit{catlib}\, library developed by
Allan~Brighton,  Miguel~Albrecht and colleagues at the European Southern
Observatory.  Patrick~Wallace gave useful advice and assistance during the
development of \texttt{catcoord} and this application uses his SLA library to
convert between celestial coordinate systems.  \texttt{catchart} uses
Tim~Pearson's PGPLOT to produce its plots.  Last, but not least, FITS
tables continue to be accessed using Bill~Pence's invaluable FITSIO library.
I am also grateful to the numerous people who have commented on, and
suggested improvements to, CURSA.
\begin{flushright}
Clive Davenhall \\
\raggedright \textit{Institute for Astronomy, University of Edinburgh \\
Saint M\'{e}dard's Day 1997}
\end{flushright}

I am grateful to John Lucey for useful discussions about photometric
calibration and for kindly providing the data used in the example
catalogue of observations of photometric standard stars.  Peter~Draper
gave helpful comments on the section of the manual describing the
photometric calibration.

\begin{flushright}
Clive Davenhall \\
\raggedright \textit{Institute for Astronomy, University of Edinburgh \\
Saint Aed's Day 1997}
\end{flushright}

\newpage
\subsection*{Revision history}

\begin{enumerate}

  \item 8th May 1995: Original draft (ACD).

  \item 10th October 1995: \htmladdnormallink{Version 1}
   {http://www.roe.ac.uk/acdwww/cursa/news/v1_1.lis} (ACD).
%  {http://www.starlink.ac.uk/cursa/news/v1_1.lis} (ACD).

  \item 11th April 1996: Version 2.  Modified so that the Latex source
   could be used to create an HTML as well as a paper version (ACD).

  \item 31st January 1997: Version 3.
   Modified for \htmladdnormallink{release 2.1}
   {http://www.roe.ac.uk/acdwww/cursa/news/v2_1.lis} of the CURSA package.
%  {http://www.starlink.ac.uk/cursa/news/v2_1.lis} of the CURSA package.
   The major changes were the addition of the Small Text List (STL) format
   and the new application \texttt{catselect} (ACD).

  \item 8th June 1997: Version 4.
   Modified for \htmladdnormallink{release 3.1}
   {http://www.roe.ac.uk/acdwww/cursa/news/v3_1.lis} of the CURSA package.
%  {http://www.starlink.ac.uk/cursa/news/v3_1.lis} of the CURSA package.
   The major changes were the new applications \texttt{catcoord}, \texttt{catchart}, \texttt{catchartrn} and \texttt{catremote} (ACD).

  \item 10th November 1997: Version 5.
   Modified for \htmladdnormallink{release 4.1}
   {http://www.roe.ac.uk/acdwww/cursa/news/v4_1.lis} of the CURSA package.
%  {http://www.starlink.ac.uk/cursa/news/v4_1.lis} of the CURSA package.
   The major changes were the new applications for photometric calibration:
   \texttt{catphotomfit}, \texttt{catphotomtrn} and \texttt{catphotomlst} (ACD).

  \item 13th December 1998: Version 6.
   Modified for \htmladdnormallink{release 5.1}
   {http://www.roe.ac.uk/acdwww/cursa/news/v5_1.lis} of the CURSA package.
%  {http://www.starlink.ac.uk/cursa/news/v5_1.lis} of the CURSA package.
   The major changes were the new application \texttt{catcdsin} and additional
   formats for reading sexagesimal angles from fixed-format STL catalogues
   (ACD).

  \item 29th November 1999: Version 7.
   Modified for \htmladdnormallink{release 6.1}
   {http://www.roe.ac.uk/acdwww/cursa/news/v6_1.lis} of the CURSA package.
%  {http://www.starlink.ac.uk/cursa/news/v6_1.lis} of the CURSA package.
   The major changes were the addition of the Tab-Separated Table (TST)
   format, the new application \texttt{catgrid} and options for plotting
   scatter-plots and histograms in \texttt{xcatview} (ACD).

  \item 25th July 2000: Version 8.
   Modified for \htmladdnormallink{release 6.2}
   {http://www.roe.ac.uk/acdwww/cursa/news/v6_2.lis} of the CURSA package.
%  {http://www.starlink.ac.uk/cursa/news/v6_2.lis} of the CURSA package.
   Version 6.2 contains no major enhancements, just some minor improvements
   and bug fixes.  The main changes to the document are the removal of the
   description of the Tab-Separated Table (TST) format, which has been
   moved to \xref{SSN/75}{ssn75}{} and the inclusion of an additional
   appendix in the hyper-text version which gives descriptions of individual
   applications (ACD).

  \item 14th May 2001: Version 9.
   Modified for \htmladdnormallink{release 6.3}
   {http://www.roe.ac.uk/acdwww/cursa/news/v6_3.lis} of the CURSA package.
%  {http://www.starlink.ac.uk/cursa/news/v6_3.lis} of the CURSA package.
   Version 6.3 contains no major changes, but rather a number of
   enhancements and bug fixes.  There are improvements to the applications
   \texttt{catcopy} and \texttt{catchart}.  The facilities to access remote
   catalogues via the Internet have been completely re-worked.  A `quiet
   mode' has been added to most of the applications.  Support for the
   little-used CHI/HDS catalogue format has been removed.

  \item 4th November 2001: Version 10.
   Modified for \htmladdnormallink{release 6.4}
   {http://www.roe.ac.uk/acdwww/cursa/news/v6_4.lis} of the CURSA package.
%  {http://www.starlink.ac.uk/cursa/news/v6_4.lis} of the CURSA package.
   Version 6.4 contains no major enhancements.  Application \texttt{catheader}
   has been re-worked and now offers various options and more convenient
   output.  A bug in \texttt{xcatview} has been fixed.

\end{enumerate}

% Section 1 begins ...
\cleardoublepage
\newpage

\section{\xlabel{INTRO}Introduction}

CURSA\footnote{Cursa is the common name for $\beta$ Eridanus. It is of
Arabic origin and derives from an abbreviation of the name of an
asterism involving $\lambda$, $\beta$ and $\psi$ Eri and $\tau$ Ori:
`the Foremost Footstool of Orion'.  These details come from a \textit{Short Guide to Modern Star Names and Their Derivations} by
P.~Kunitzsch and T.~Smart\cite{KUNIT}.} (Catalogue Utilities for Reporting,
Selecting and Arithmetic) is a package of Starlink applications for
manipulating astronomical catalogues and similar tabular datasets. This
manual describes version \CURSAversion of CURSA.  Though CURSA is
primarily intended for use with astronomical catalogues it can be used
equally well with other tabular data, such as a table of private
astronomical results, or, indeed, data which are entirely non-astronomical,
provided that they are in an appropriate format.

The facilities provided by CURSA include: browsing or examining catalogues,
selecting subsets from catalogues, sorting catalogues, copying catalogues,
pairing two catalogues, converting catalogue coordinates between some
celestial coordinate systems, plotting finding charts and photometric
calibration.  Also, subsets can be extracted from a catalogue in a format
suitable for plotting using other Starlink packages, such as PONGO.

CURSA can access catalogues held in either the popular FITS table format,
the Tab-Separated Table (TST) format or the Small Text List (STL) format.
Both ASCII and binary FITS tables can be read, though only binary FITS
tables can be written.  Catalogues in the STL and TST formats are simple
ASCII text files which can be created with a text editor.  Unlike the other
formats which CURSA can access, the STL format is specific to CURSA.
Nonetheless, the STL format exists in order to allow easy access to both
private tables and versions of standard catalogues held as text files.  It
is usually straightforward to create an STL catalogue from a text file
containing a private list or catalogue.  CURSA includes a facility for
automatically converting text versions of catalogues obtained from the
Centre de Donn\'{e}es astronomiques de Strasbourg (CDS) into the STL format.
CURSA also has some facilities for accessing remote on-line catalogues via
the Internet.

CURSA is available on all the variants of Unix currently supported by
Starlink.  The following section briefly describes some sources of
catalogues in suitable formats.  Subsequent sections introduce
some general information about CURSA and the later sections describe the
individual applications.  Two tutorial examples (`recipes' in the jargon
of cookbooks) of using CURSA are included in \xref{SC/6: \textit{The CCD
Photometric Calibration Cookbook}}{sc6}{}\/\cite{SC6}.


\section{\label{OBTAIN}\xlabel{OBTAIN}Obtaining copies of catalogues}

The FITS table format is a popular and widely used format for
distributing astronomical catalogues. Several CD-ROMs contain catalogues
in this format.  Some of the more generally useful ones are:

\begin{itemize}

  \item \textit{Selected Astronomical Catalogs volumes I, II, III and IV},
   produced by the US Astronomical Data Center (ADC) at the NASA Goddard
   Space Flight Center,

  \item \textit{HST Guide Star Catalog}, produced by the NASA Space Telescope
   Science Institute (see Section~\ref{GSCIN}, below),

  \item several of the \textit{Einstein Observatory}\, CD-ROMs.

\end{itemize}

In particular the four volumes of \textit{Selected Astronomical Catalogs}\,
are an extremely useful collection of widely used catalogues.  Also, the
Centre de Donn\'{e}es astronomiques de Strasbourg (CDS) and the US
Astronomical Data Center now make many of the catalogues  in their
extensive collections available on-line.  Briefly, the CDS and ADC may be
contacted as follows.

\begin{description}

  \item[CDS] URL:
   \url{http://cdsweb.u-strasbg.fr/CDS.html}

   Anonymous ftp site: \texttt{cdsarc.u-strasbg.fr}, directory: \texttt{/pub/cats}

   Electronic mail: \texttt{question@simbad.u-strasbg.fr}

   Postal address: Centre de Donn\'{e}es astronomiques de
   Strasbourg, Observatoire de Strasbourg, 11, rue de l'Universit\'{e},
   67000 Strasbourg, France.

  \item[ADC]  URL: \url{http://adc.gsfc.nasa.gov/}

   Anonymous ftp site: \texttt{adc.gsfc.nasa.gov}, directory: \texttt{/pub/adc/archives}

   Electronic mail: \texttt{request@nssdca.gsfc.nasa.gov}

   Postal address: World Data Center A for Rockets and
   Satellites, NASA, Goddard Space Flight Center, Code 633,
   Greenbelt, Maryland 20771, USA.

\end{description}

In the case of the CDS catalogues can usually be retrieved as either
FITS files or simple text files.  It is usually preferable to retrieve
the catalogues as text files because they can then be automatically
reformatted into CURSA STL format catalogues (see Section~\ref{CDSIN})
which properly interpret their coordinates, thus allowing full use of
CURSA's facilities for manipulating and displaying angles (see
Section~\ref{CELCOORD} and Appendix~\ref{ANGLE}).

Further details of the CD-ROMs and the data centres may be found in the
CURSA
`\htmladdnormallink{home page}{http://www.starlink.ac.uk/cursa/}'
(see page~\pageref{HOMEPAGE} for the URL) and in SUN/162\cite{SUN162},
though the latter is now somewhat out of date.

An additional small collection of catalogues which have had their celestial
coordinates reformatted to take full advantage of CURSA's facilities
for manipulating and displaying angles (see Section~\ref{CELCOORD} and
Appendix~\ref{ANGLE}) is available by anonymous ftp.  The details are
as follows.

\begin{tabular}{ll}
Anonymous ftp to: & \texttt{ftp.roe.ac.uk}        \\
Directory:        & \texttt{/pub/acd/catalogues}  \\
\end{tabular}

Remember to reply \texttt{anonymous} when prompted for a username and
to give your e-mail address as the password.  Retrieve file \texttt{0CONTENTS.LIS} for a list of the catalogues available.  If you encounter
difficulty using ftp then contact your site manager in the first
instance.  The list of catalogues is also available from the CURSA home
page.

CURSA includes a facility which provides some limited access to remote
catalogues held on-line at various astronomical data centres and archives
around the world.  You can select a subset from one of these catalogues and
save it as a CURSA Tab-Separated Table (TST) format catalogue which can
then be input to the other CURSA applications. This facility is available
as part of \texttt{xcatview} (see Section~\ref{XVIEW}) and as application \texttt{catremote} (see Section~\ref{REMACCSS}).

In the past Starlink provided the SCAR (Starlink Catalogue Access and
Reporting) system for manipulating astronomical catalogues on its
VAX/VMS service. SCAR had its own unique format for storing catalogues.
It is possible to convert SCAR catalogues to FITS tables and make them
accessible to CURSA. If you have any SCAR catalogues (either public
catalogues from the standard collection or private catalogues) which you
would like converted then please contact me in the first instance (see
page~\pageref{HOMEPAGE} for details).


\section{\xlabel{START}Getting started}

CURSA is an optional Starlink software item. Before proceeding you should
check with your local site manager whether it is installed at your site,
and if not attempt to persuade him to install it.  If CURSA is installed at
your site the following directory should exist:

\begin{terminalv}
/star/bin/cursa
\end{terminalv}

The procedure to set up for using CURSA is the same on all the variants
of Unix supported by Starlink. Simply type:

\begin{terminalv}
% cursa
\end{terminalv}

The following message should appear:

\begin{terminalv}
CURSA commands are now available -- (Version 6.3)
\end{terminalv}

If it does not, then the probable cause is that CURSA is not installed
correctly at your site; check with your local site manager.

You do not need any special quotas or privileges to use CURSA. However,
obviously, you need enough disk space to accommodate any catalogues
that you might use and any output files that you might create.


\section{\xlabel{COMP}\label{COMP}Terminology}

An astronomical \textbf{catalogue} is basically a \textbf{table} of values,
consisting of the measurements of the same property for a set of
objects, together with the auxiliary information necessary to describe
this table. There are several different terminologies for describing
the elements of such tables. For simplicity, in CURSA a terminology which
corresponds loosely to that used intuitively for the paper versions of
astronomical catalogues is used:

\begin{description}

  \item[row] the values for all the properties associated with some
   particular object,

  \item[column] the value of a single property for all the objects in
   a catalogue,

  \item[field] the value of a single property for a single object
   (that is, the intersection of a row and a column).

\end{description}

Some of the other terminologies are shown for comparison in
Table~\ref{TABLE_NOT}\footnote{This table is adapted from \textit{Database Systems in Science and Engineering}\, by J.R.~Rumble and
F.J.~Smith\cite{RUMBLE}, p158.}.

\begin{table}[htbp]

\begin{center}
\begin{tabular}{lll}
CURSA             &  Fortran           &  Relational Database \\ \hline
table             &  file              &  relation      \\
row               &  record            &  tuple         \\
column            &  field             &  attribute     \\
field             &  data item, field  &  component     \\
format            &  format            &  schema        \\
number of columns &  number of fields  &  arity, degree \\
number of rows    &  number of records &  cardinality   \\
\end{tabular}
\end{center}

\caption{Alternative terminologies for the components of tables
\label{TABLE_NOT} }

\end{table}

In CURSA each \textbf{catalogue} can contain only one \textbf{table} and the two terms can usually be used interchangeably without
introducing any ambiguity. However, where it is necessary to
differentiate between the two sorts of entities, \textbf{table} is used
to denote the simple matrix of rows and columns and \textbf{catalogue} is
used to denote the combination of a table and its associated auxiliary
information. (Note, however, that this usage implies nothing about the
contents of the catalogue; it may contain a published astronomical catalogue,
a set of private astronomical results or, indeed, data which are entirely
non-astronomical.)

A CURSA catalogue which contains celestial coordinates in a restricted
format which CURSA can interpret is called a \textbf{target list}.  The
applications which convert between celestial coordinates and plot
finding charts operate on target lists.  Target lists are described in
Section~\ref{TARGLIST}.

Columns may either be \textbf{scalars} in which case each field comprises
a single datum, or \textbf{vectors}, one-dimensional arrays where each
field comprises a one-dimensional array of values.

Columns have a number of \textbf{attributes}, such as their name, data type
and units. A column's attributes hold all the details which define its
characteristics.  The more important column attributes are described
in Section~\ref{COLS}, below.

Catalogues can also contain auxiliary information which applies to the
entire catalogue. CURSA recognises two types of auxiliary information:
\textbf{parameters} and \textbf{textual information}. A \textbf{parameter} is
a single datum, such as the epoch or equinox of celestial coordinates stored
in a catalogue. CURSA parameters are similar to FITS keywords (in
fact, CURSA interprets named keywords in a FITS table as parameters).
Parameters have attributes similar to columns.

\textbf{Textual information} is information, usually descriptive,
associated with the catalogue and intended to be read by a human. For a
FITS table the textual information is basically the contents of any
`COMMENTS' and `HISTORY' keywords\footnote{This statement is something
of an over-simplification. See Appendix~\ref{FORMAT} for a complete
description of the way that FITS headers are interpreted as textual
information.}.

In the jargon of relational database systems auxiliary information is
often called \textbf{metadata}.  In the context of CURSA the metadata for
a catalogue comprises the details of the columns (name, data type,
units, \emph{etc.}), the parameters and the textual information.


\subsection{\label{COLS}Column attributes}

In order to use CURSA you do not need to know the details of all the
attributes of a column, but there are a few which you will probably
encounter. These attributes are listed in Table~\ref{COLUMN_ATT} and are
described briefly below.

\begin{table}[htbp]

\begin{center}
\begin{tabular}{ll}
Attribute  & Comments                              \\ \hline
NAME       & Name of the column                    \\
DTYPE      & Data type                             \\
DIMS       & Dimensionality: scalar or vector      \\
SIZE       & Size (number of elements) of a vector \\
UNITS      & Units of the column                   \\
EXFMT      & External display format               \\
COMM       & Comments describing the column        \\
\end{tabular}
\end{center}

\caption{Attributes of columns\label{COLUMN_ATT} }

\end{table}

\paragraph{NAME}
The name of the column. The rules for column names are as follows.

\begin{itemize}

  \item The name must be unique within the totality of parameters and
   columns for the catalogue. This condition is necessary in order that
   a component (parameter or column) may be identified unambiguously
   when its name is used in an expression (see Appendix~\ref{EXPR}).

  \item A name may comprise up to fifteen characters. This value is chosen
   for consistency with HDS and is adequate for FITS tables.

  \item The name can contain only: upper or lower case alphabetic
   characters (a-z, A-Z), numeric characters (0-9) and the underscore
   character (`\_'). Note that lower case alphabetic characters must
   be allowed in order to access existing FITS tables. \textit{However,
   corresponding upper and lower case characters are considered to be
   equivalent.} Thus, for example, the names: \texttt{HD\_NUMBER}, \texttt{HD\_Number} and \texttt{hd\_number} would all refer to the same column.

  \item The first character must be a letter.

\end{itemize}

\paragraph{DTYPE}
The data type of values held in the column. CURSA supports the
standard data types of Fortran 77 (apart from the COMPLEX data types)
and also signed one and two byte INTEGERs.

\paragraph{DIMS}
The dimensionality of the column: scalar or a vector.

\paragraph{SIZE}
If the column is a vector this attribute contains the number of
elements in the vector. If the column is a scalar it is set to one.

\paragraph{UNITS}
The units in which values stored in the column are expressed. The UNITS
attribute is used to identify, and control the appearance of, columns
of angles (see Appendix~\ref{ANGLE}). Apart from this exception the units
are treated purely as comments and no attempts are made to automatically
propagate and convert units in calculations and selections.

\paragraph{EXFMT}
The format used to represent a field extracted from a column for
external display by \texttt{xcatview} (see section~\ref{XVIEW}) or \texttt{catview} (see section~\ref{VIEW}). The external format specifier should be
a valid Fortran 77 format specifier for the data type of the column.

\paragraph{COMM}
Explanatory comments describing the column.


\section{\xlabel{NULLS}\label{NULLS}Null values}

CURSA supports null values in catalogues. Null values are used to represent
a field for which no actual value is available. Null values can arise in
several ways. They may be present in a catalogue when it is read by
CURSA. An example might be a catalogue of multi-colour photometry for a
set of stars where measures for some colours were missing for some of the
stars. Null values would be used to represent the missing values.
Alternatively, they might arise where an expression is being used to
compute a new column and evaluation of the expression for the current
row results in a `$\div$ by zero' error. No valid value will be
available for the expression, so a null value will be substituted.

Throughout CURSA null values have the single, simple meaning that `no value
is available for this datum'. It is possible to invent schemes where a set
of null values are supported, each with a subtly different gradation of
meaning. However, CURSA does not support such schemes.

\subsection{Processing null values}

You do not need to know all the details of how CURSA manipulates null
values internally. However, the following points may be useful.

\begin{itemize}

  \item When a value for a new column is computed from an algebraic
   expression in which one of the input fields is null, then the
   result is also null. For example, if the new column was being
   computed for the expression `\texttt{x + y}' then the result would be
   null if either \texttt{x} or \texttt{y} (or both) were null. The generation
   of a null in this fashion is not considered an error and no message
   or warning is reported.

  \item When a relational expression is being used to generate a selection
   then rows for fields with null values occurring in the expression do not
   satisfy the expression. For example, if the selection was defined by
   `\texttt{x > 2.0}' then rows where \texttt{x} is null would not satisfy
   the expression.

  \item The function \texttt{NULL} is available to determine whether a
   field is null or not. For example, to select the rows for which
   column \texttt{x} is not null the following expression would be used:
   `\texttt{.NOT. NULL(x)}'.

\end{itemize}

\subsection{Displaying null values}

When null values are displayed by \texttt{xcatview} (see
Section~\ref{XVIEW}) or \texttt{catview} (see Section~\ref{VIEW}) they
will normally be represented by the string `\texttt{<null>}'. However, if
there is insufficient space to display them in this way they will be
represented by the single character `\texttt{?}'.


\section{\xlabel{CELCOORD}\label{CELCOORD}Celestial coordinates}

Most astronomical catalogues contain columns of celestial
coordinates of some sort: usually Right Ascension and
Declination for some equinox and epoch, or perhaps Galactic or
ecliptic coordinates. The storage, manipulation and presentation
for display of celestial coordinates in the computer-readable
version of astronomical catalogues is something of a vexed topic
which has caused a deal of confusion and difficulty, much of it,
in principle, unnecessary.

For preexisting catalogues, such as those described in
Section~\ref{OBTAIN}, the format of the celestial
coordinates will already be fixed and CURSA will simply display the
columns in whatever way is possible. For example, many catalogues
contain the hours, or degrees, minutes and seconds which
comprise a coordinate as separate columns; a form which is
singularly inconvenient for further processing. However, CURSA has
some special facilities for processing and displaying coordinates, and
catalogues that have been specifically prepared for CURSA can take
advantage of these.

CURSA can store columns of coordinates as radians but automatically
present them as sexagesimal hours or degrees when they are listed by the
browsing applications \texttt{xcatview} (see Section~\ref{XVIEW}) or \texttt{catview} (see Section~\ref{VIEW}). The advantages of this approach are
that internally within CURSA the coordinates remain in radians, which is
the most convenient form for computations, but they are presented to the
user, and he interacts with them, as sexagesimal hours or degrees, which
is the way that he naturally thinks about them.

Also, it is possible within \texttt{xcatview} or \texttt{catview} to
interactively alter the precise way that a coordinate is formatted for
display. These facilities are described in detail in
Appendix~\ref{ANGLE}. Similarly, while displaying coordinates in units of
hours or degrees formatted as sexagesimal values is usually the required
behaviour, occasionally you may want to display angles as simple decimal
numbers expressed in radians (as they are represented internally). Both
\texttt{xcatview} and \texttt{catview} provide this facility.

CURSA application \texttt{catgscin} (see Section~\ref{GSCIN}) reformats
coordinates in regions of the HST \textit{Guide Star Catalog}\, to a
format which is fully compatible with CURSA.  Similarly, \texttt{catremote}
(see Section~\ref{REMACCSS}), the application for extracting subsets
from remote on-line catalogues, returns coordinates which are fully
compatible.  Also, \texttt{catcdsin} (see Section~\ref{CDSIN}) will
usually reformat the text versions of CDS catalogues into STL format
catalogues containing CURSA-compatible coordinates.  Finally, the CURSA
`\htmladdnormallink{home page}{http://www.starlink.ac.uk/cursa/}'
(see page~\pageref{HOMEPAGE} for the URL) contains a list of catalogues
which have been converted to have coordinates which are fully compatible
with CURSA.


\section{\xlabel{TARGLIST}\label{TARGLIST}Target lists}

A \textbf{target list} is a catalogue which contains celestial coordinates
in a restricted format which CURSA can interpret.  The applications for
converting between celestial coordinate systems, \texttt{catcoord} (see
Section~\ref{CONCOORD}) and plotting finding charts, \texttt{catchart} (see
Section~\ref{FCHART}) read target lists.  The applications for
importing a region from the HST \textit{Guide Star Catalog}, \texttt{catgscin}
(see Section~\ref{GSCIN})
and extracting a subset from a remote catalogue, \texttt{catremote} (see
Section~\ref{REMACCSS}) produce target lists.  Similarly, the catalogues
generated with \texttt{catcdsin} (see Section~\ref{CDSIN}) from the text
version of CDS catalogues are usually target lists.  Though a target
list places restrictions on the names and units of the celestial
coordinates, the catalogue itself can be in any of the formats supported
by CURSA: FITS table, TST or STL (see Appendix~\ref{FORMAT}).

A target list must contain columns of Right Ascension and Declination,
for some equinox and epoch, called respectively \texttt{RA} and \texttt{DEC}.
These coordinates must be stored in radians in a format which CURSA can
interpret (see Section~\ref{CELCOORD}, above and Appendix~\ref{ANGLE}).

Additional optional columns allow the proper motion, parallax and radial
velocity to be specified.  These quantities are used for accurate
conversions between celestial coordinate systems.  All the columns which
can be used to specify coordinates in a target list are listed in
Table~\ref{TARGETLIST}.  The columns marked with a bullet (`$\bullet$')
in the `Mandatory' column must be present.  The other columns are
optional. However, if they are present they must be used as described.
The names of the columns are chosen to be consistent with the
recommendations of the CDS (see \textit{Astronomical Catalogues at CDS:
Adopted Standards} by F.~Ochsenbein\cite{CDSTAND}, p14).

\begin{table}[htbp]

\begin{center}
\begin{tabular}{lllc}
Description                      & Name       & Units   & Mandatory? \\ \hline
Right Ascension                         & \texttt{RA}   & Radians & $\bullet$  \\
Declination                             & \texttt{DEC}  & Radians & $\bullet$  \\
Annual proper motion in Right Ascension & \texttt{PMRA} & Radians & \\
Annual Proper motion in Declination     & \texttt{PMDE} & Radians & \\
Parallax                                & \texttt{PLX}  & Radians & \\
Radial velocity                         & \texttt{RV}   & Km/sec  & \\
\end{tabular}

\caption{Columns defining celestial coordinates in a target list
\label{TARGETLIST} }
\end{center}

\end{table}

The proper motions are specified per year rather than per century.  Also
the proper motion in Right Ascension is simply the rate of change of
Right Ascension, $\dot{\alpha}$ (leading to large values for stars close
to the poles), \textit{not}\, the angle on the sky, $\dot{\alpha}\cos\delta$.
The latter quantity is tabulated in some catalogues.  Similarly some
catalogues give the proper motion as a position angle and size.
In both these cases the tabulated values must be converted to the
required form before they can be used in a target list.

The usual astronomical sign convention for radial velocity is used: objects
which are receding should have a positive radial velocity.

A target list can also contain two optional parameters: \texttt{EQUINOX}
and \texttt{EPOCH}.  These parameters respectively contain the equinox
and epoch of the coordinates.  Both parameters are of data type
CHARACTER.

The value of both parameters is a Besselian or Julian
epoch\footnote{In this context an epoch is simply an instant of
time.} expressed in years.  The numeric value may optionally be preceded
by a letter `\texttt{B}' or `\texttt{J}' to indicate a Besselian or Julian
epoch respectively.  If this preceding letter is omitted then values
before 1984.0 are assumed to be Besselian and subsequent values to be
Julian.  This behaviour is consistent with the relevant IAU
recommendations.  Table~\ref{EQEP} lists some examples of valid
equinoxes and epochs.

\begin{table}[htbp]

\begin{center}
\begin{tabular}{ll}
Example        & Notes  \\ \hline
\texttt{B1950}    & often used in older catalogues \\
\texttt{J2000}    & often used in modern catalogues \\
\texttt{B1975}    & \\
\texttt{1992.37}  & interpreted as J1992.37 (because after 1984.0)  \\
\texttt{1943}     & interpreted as B1943.0 (because before 1984.0)  \\
\texttt{B1987}    & `\texttt{B}' is necessary here (because after 1984.0)  \\
\end{tabular}

\caption{Example equinoxes and epochs \label{EQEP} }
\end{center}

\end{table}

An example of a simple target list is available as file \texttt{/star/share/cursa/simple.TXT}.  In this target list the coordinates
simply comprise Right Ascension and Declination.  A more complicated
example where the coordinates include proper motions \emph{etc.} is
available as file \texttt{/star/share/cursa/propmotn.TXT}.  Note that
though CURSA must interpret the columns of proper motions \emph{etc.} as
having units of radians they can be tabulated in an STL format catalogue
in seconds of arc by using the TBLFMT option in the column definitions,
as in this example.  This option will often be convenient when creating
target lists.


\section{\xlabel{ACCESS}Accessing catalogues}

Most of the CURSA applications prompt you to enter the name of at least
one catalogue.  You should reply with the name of the file containing the
catalogue.  The file name of a CURSA catalogue comprises the `catalogue
name' followed by the `file type', for example:

\begin{terminalv}
perseus.FIT
\end{terminalv}

where `\texttt{perseus}' is the catalogue name and `\texttt{.FIT}' the file
type.  The catalogue name is restricted to contain only: upper or lower
case alphabetic characters (a-z, A-Z), numeric characters (0-9) and the
underscore character (`\_').  The file name may optionally be preceded by
a directory specification.

CURSA uses the `file type' of the file name to determine the format of
the catalogue (FITS table, TST or STL) and therefore the file name \textit{must}\, end in the appropriate file type.  The file types for the three
catalogue formats are:

\begin{quote}
\begin{tabular}{ll}
\textbf{FITS table:} & \texttt{.FIT .fit .FITS .fits .GSC .gsc}  \\
                  & \\
\textbf{TST:}        & \texttt{.TAB .tab}  \\
                  & \\
\textbf{STL:}        & \texttt{.TXT .txt}  \\
\end{tabular}
\end{quote}

The \texttt{.GSC} and \texttt{.gsc} file types for FITS tables are provided in
order to allow regions of the HST \textit{Guide Star Catalog}\, to be accessed
easily. Other FITS tables obtained from an external source, such as those
mentioned in Section~\ref{OBTAIN}, may have a different file type. They
must be renamed (with the Unix command \texttt{mv}) to have a recognised file
type before they can be accessed with CURSA.

A few additional details which are specific to the individual catalogue
formats are described below. The peculiarities and limitations of the
three catalogue formats are described in full in Appendix~\ref{FORMAT}.

\subsection{FITS tables}

(File types: \texttt{.FIT .fit .FITS .fits .GSC .gsc}).  Mixed
capitalisations, such as \texttt{.Fit}, are also supported.  To access a FITS
table in the current directory you need only supply the file name. To
access a FITS table in another directory you should precede the file name
with an absolute or relative directory specification\footnote{ Of course,
you can precede the name of a catalogue in the current directory with a
directory specification if you want to, but there is no point in doing
so.}.

Usually the table component of a FITS file occurs in the first FITS
extension to the file. When reading an existing FITS file CURSA will look
for a table in the first extension. In cases where the table is located
in an extension other than the first you can specify the required
extension by giving its number inside curly brackets after the name
of the file. For example, if the table occurred in the third extension
of a FITS file called \texttt{perseus.FIT} you would specify:

\begin{terminalv}
perseus.FIT{3}
\end{terminalv}

The closing curly bracket is optional. When CURSA writes FITS tables
the table is always written to the first extension.

\subsection{TST}

(File types: \texttt{.TAB .tab}). Mixed capitalisations, such as \texttt{.Tab}, are also supported.  To access a TST (Tab-Separated Table) format
catalogue in the current directory you need only supply the file name.
To access a TST catalogue in another directory you should precede the
file name with an absolute or relative directory specification.

\subsection{STL}

(File types: \texttt{.TXT .txt}). Mixed capitalisations, such as \texttt{.Txt}, are also supported.  To access an STL (Small Text List) format
catalogue in the current directory you need only supply the file name.
To access an STL catalogue in another directory you should precede the
file name with an absolute or relative directory specification.

An input STL catalogue may be in either the standard form or the KAPPA
variant form (see Appendix~\ref{STLKAP}).  By default CURSA writes standard
STLs.  It can be made to write a KAPPA variant STL by appending `\texttt{KAPPA}' inside curly brackets after the name of the file.  For example,
to write a KAPPA variant STL called \texttt{perseus.TXT} you would specify:

\begin{terminalv}
perseus.TXT{KAPPA}
\end{terminalv}

`\texttt{KAPPA}' can be abbreviated down to just `\texttt{K}' and can be given
in either case.  Also the closing curly bracket is optional.


\section{\xlabel{PROMPTS}\label{PROMPTS}Answering prompts in CURSA
applications}

Clearly you will usually reply to a prompt from a CURSA application by
entering a suitable value. However, as usual for Starlink applications,
the following special replies may also be entered:

\begin{description}

  \item[ \texttt{?} ] display one line of help information about the
   prompt, and then re-prompt,

  \item[ \texttt{!} ] a value is not available to answer the prompt. The
   application will take appropriate action; in CURSA it will usually
   abort,

  \item[ \texttt{!!} ] abort the application\footnote{Of course the
   application may also be aborted by typing `\texttt{<Control-C>}'.
   Typing `\texttt{<Control-C>}' is less likely to tidy up properly any files
   which are open, though this is unlikely to be important in
   practice.}.

\end{description}

Note that these special replies are not available in \texttt{catcdsin}
(see Section~\ref{CDSIN}) and \texttt{catremote} (see Section~\ref{REMACCSS}).


\section{\xlabel{SUMMARY}Summary of applications}

CURSA contains the following applications.

\begin{description}

  \item[ \texttt{xcatview} ] browse and generate selections from a
   catalogue (easy-to-use X-windows version with a graphical user
   interface; see Section~\ref{XVIEW}),

  \item[ \texttt{catview} ] browse and generate selections from a
   catalogue (command line version; see Section~\ref{VIEW}),

  \item[ \texttt{catheader} ] list various header information for a catalogue
   (see Section~\ref{HEAD}),

  \item[ \texttt{catcopy} ] copy a catalogue (see Section~\ref{COPY}),

  \item[ \texttt{catsort} ] sort a catalogue (see Section~\ref{SORT}),

  \item[ \texttt{catselect} ] select a subset from a catalogue and save it
   as a new catalogue (see Section~\ref{SELECT}),

  \item[ \texttt{catcoord} ] convert catalogue coordinates between celestial
   coordinate systems (see Section~\ref{CONCOORD}),

  \item[ \texttt{catchart} ] plot a basic finding chart from a target
   list (see Section~\ref{FCHART}),

  \item[ \texttt{catchartrn} ] customise a target list for prior to
   plotting it as a finding chart (see Section~\ref{FCHART}),

  \item[ \texttt{catpair} ] pair two catalogues (see Section~\ref{PAIR}),

  \item[ \texttt{catphotomfit} ] define photometric transformation
   coefficients using observations of standard stars
   (see Section~\ref{PHOTCAL}),

  \item[ \texttt{catphotomtrn} ] apply photometric transformation
   coefficients to programme objects (see Section~\ref{PHOTCAL}),

  \item[ \texttt{catphotomlst} ] list photometric transformation
   coefficients (see Section~\ref{PHOTCAL}),

  \item[ \texttt{catgrid} ] bin one, two or three columns from a catalogue
   into, respectively, a histogram, image or data cube
   (see Section~\ref{GRIDS}),

  \item[ \texttt{catcdsin} ] convert the text file version of a CDS
   catalogue to the CURSA STL format (see Section~\ref{CDSIN}),

  \item[ \texttt{catgscin} ] convert a region in the HST \textit{Guide Star
   Catalog}\, to a more convenient format (see Section~\ref{GSCIN}),

  \item[ \texttt{catremote} ] extract a subset from a remote on-line
   catalogue (see Section~\ref{REMACCSS}).

\end{description}

To run any of the applications you simply type its name and answer the
ensuing prompts (or, in the case of \texttt{xcatview} dialogue boxes).

\texttt{xcatview} and \texttt{catview} provide essentially the same
functionality. However, \texttt{xcatview} is \textit{much}\, easier to use
and is strongly recommended over \texttt{catview} for casual,
interactive examination of a catalogue. It does, however, have to be
run from a terminal (or workstation console) capable of supporting
X-windows output. The only circumstances where \texttt{catview} is likely
to be preferable are if you have a terminal which does not support X
output or you are performing repetitive `batch' type operations from a
script.

\subsection{\label{COPYTEXT}Copying textual information}

The applications which create a new catalogue from an existing one
(\texttt{catcopy}, \texttt{catsort},
\newline \texttt{catselect}, \texttt{catcoord}, \texttt{catchartrn},
\texttt{catphotomtrn} and \texttt{catgscin}) all have a uniform option to
control the amount of textual information that they write to the new
catalogue.

By default the textual information for the new catalogue is a copy of
the textual information for the original catalogue (which is usually
what is required).  However, options are available to either copy all
the details of the original catalogue (including the column and
parameter definitions) as textual information for the new catalogue or
to copy no textual information to the new catalogue.

These options are invoked by specifying an extra item on the command
line when the application is invoked.  For example, for \texttt{catcopy}:

\begin{itemize}

  \item to copy just the textual information from the original
   catalogue simply give the command name:

  \begin{terminalv}
catcopy
  \end{terminalv}

  \item to copy the entire description of the input catalogue as
   textual information in the output catalogue:

  \begin{terminalv}
catcopy text=all
  \end{terminalv}

  \item to copy no textual information to the output catalogue:

  \begin{terminalv}
catcopy  text=none
  \end{terminalv}

\end{itemize}

The other applications include exactly the same way option.  There must be
one or more spaces between the application name and the `\texttt{text=}' item.

\subsection{\label{QUIET}Quiet mode}

Most of the applications have a `quiet mode' in which they issue fewer
informational and warning messages.  The exceptions are \texttt{catcdsin}
and \texttt{catremote}, which are Perl scripts rather than conventional
applications.  The quiet mode suppresses only some informational and
warning messages; it does not affect error messages.  All the applications
which support the quiet mode use the same mechanism to control it.
By default the applications are in a `verbose' mode in which they
issue informational and warning messages.  To switch to quiet mode an
additional option is specified when invoking any of the applications which
support it, for example:

\begin{terminalv}
catcopy  quiet=true
\end{terminalv}

The quiet mode will now remain in effect, not just for the one invocation
of \texttt{catcopy}, but for all subsequent invocations of all the applications
that support the quiet mode.  To revert to verbose mode type, for example:

\begin{terminalv}
catchart  quiet=false
\end{terminalv}

The quiet mode can also be set as one of the configuration options
of \texttt{xcatview} (see Section~\ref{XVIEW}).  Finally, I advise you to
use the quiet mode with caution; it is usually better to see the
informational and warning messages.

\subsection{Extra functionality}

CURSA can inter-operate with a number of other packages.  These
packages provide additional functionality which is not available in
CURSA.  Perhaps the most extensive and useful is FTOOLS, which is
briefly described in Section~\ref{FTOOLS}, below.  Another useful
external package is Starbase, which is briefly described in
Section~\ref{STARBASE}, below.

The image display and analysis tool \xref{GAIA}{sun214}{}\cite{SUN214}
reads and writes catalogues in the TST format.  Thus, catalogues in
this format can be exchanged between CURSA and GAIA.  Some limited
inter-operability is possible between CURSA and the image processing
package \xref{KAPPA}{sun95}{}\cite{SUN95} (see Appendix~\ref{STLKAP}) and
the image analysis package \xref{PISA}{sun109}{}\cite{SUN109} (see
Appendix~\ref{PISA}).

Finally, CURSA is augmented by the CAT Fortran 77 subroutine library for
manipulating catalogues and tables.  Using CAT it is straightforward to
write your own programs to perform specialised tasks not covered by the
more general CURSA applications. Programs written with CAT are fully
inter-operable with the standard CURSA applications (in fact the CURSA
applications themselves use CAT). CAT is comprehensively documented in
\xref{SUN/181}{sun181}{}\cite{SUN181}.  A set of simple example programs
are included with the CAT library.

\subsection{\label{FTOOLS}Inter-operability with FTOOLS}

FTOOLS is a package for manipulating FITS files, including FITS tables.
It comprises a collection of utility programs to create, examine and
modify FITS files.  FTOOLS contains many useful functions which
complement CURSA.  It is developed and maintained by the High Energy
Astrophysics Science Archive Research Center (HEASARC) at the NASA
Goddard Space Flight Center (GSFC) and is in widespread use around the
world.

FTOOLS can inter-operate with CURSA.  However, clearly, it can only
access FITS files, not the other formats accessible to CURSA.  If
your CURSA catalogues are in one of the other formats you should use
\texttt{catcopy} to convert them to FITS tables prior to accessing them
with FTOOLS.  Also, in order to interpret the celestial coordinates
in catalogues CURSA uses specific FITS keywords in the FITS header.
Though these keywords are perfectly standard, and FTOOLS will process
catalogues containing them, it attaches no special significance to them
and will not attempt to interpret the celestial coordinates.

There is a `home page' for FTOOLS at the GSFC.  The URL is:

\begin{quote}
\url{http://heasarc.gsfc.nasa.gov/docs/software/ftools/ftools_menu.html}
\end{quote}

An identical copy is maintained at the LEDAS data archive service of
the Department of Physics and Astronomy, University of Leicester.  The
URL is:

\begin{quote}
\url{http://ledas-www.star.le.ac.uk/ftools/ftools_menu.html}
\end{quote}

This copy may be more convenient for users in the UK or Europe.  The
home pages give access to a great deal of useful information about
FTOOLS.  Copies of the software and its user manual can be retrieved.
FTOOLS is available for all the variants of Unix supported by Starlink
(and numerous other systems).

\subsection{\label{STARBASE}Inter-operability with Starbase}

Starbase is a simple relational database management system (RDBMS)
for manipulating astronomical catalogues and tables.  It was developed
by John~Roll of the Smithsonian Astrophysical Observatory.  It comprises
a collection of programs which use standard Unix features and tools.  The
basic facilities of Starbase are similar to the Unix RDBMS /rdb.

Starbase operates on tables in the Tab-Separated Table (TST) format
(see Appendix~\ref{TST}).  It works best on small tables of fewer than
10,000 rows.  Starbase can inter-operate with CURSA, though obviously
only on catalogues in the TST format.  If you wish to use Starbase with
catalogues that are not in the TST format then use \texttt{catcopy} (see
Section~\ref{COPY}) to convert them to this format.

Further information about Starbase is available from its `home page'
at URL:

\begin{quote}
\url{http://cfa-www.harvard.edu/~john/starbase/starbase.html}
\end{quote}

Copies of Starbase can be obtained from this location.  Also, there is a
list of `frequently asked questions' (FAQs) about Starbase at URL:

\begin{quote}
\url{http://www.astro.uiuc.edu/~bima/starbase/}
\end{quote}


\section{\xlabel{XVIEW}\label{XVIEW}Browsing and selecting with an X display}

\texttt{xcatview} is a powerful and flexible catalogue browser. However,
it can only be used from a terminal (or workstation console) capable
of displaying X output. Before starting \texttt{xcatview} you should ensure
that your terminal (or console) is configured to receive X output. Then
simply type:

\begin{terminalv}
xcatview
\end{terminalv}

and follow the ensuing dialogue boxes. Copious on-line help is available
within \texttt{xcatview}. To obtain it simply click on the `\texttt{Help}'
button; every dialogue box in \texttt{xcatview} contains a `\texttt{Help}'
button.

In addition to accessing local catalogues \texttt{xcatview} provides some
limited facilities to access remote catalogues held on-line at various
astronomical data centres and archives around the world.  These facilities
provide the same functionality as the application \texttt{catremote} and
are described in greater detail in Section~\ref{REMACCSS}.  Obviously
they will only be available if the computer on which CURSA is running
has appropriate network connections (which will usually be the case at
a normal Starlink node).

\texttt{xcatview} provides the following facilities:

\begin{itemize}

  \item list columns in a catalogue,

  \item list parameters and textual information from a catalogue,

  \item list new columns computed `on the fly' using an algebraic
   expression defined in terms of existing columns and parameters. For
   example, if the catalogue contained columns \texttt{V} and \texttt{B\_V}
   (corresponding to the $V$\, magnitude and $B-V$\, colour) then the
   $B$\, magnitude  could be listed by specifying the expression `\texttt{V + B\_V}'. The syntax for expressions is described in
   Appendix~\ref{EXPR},

  \item fast creation of a subset within a specified range for a sorted
   column (see Section~\ref{SORT} for details of how to create a
   catalogue sorted on a specified column),

  \item creation of subsets defined by algebraic criteria. For example,
   if the catalogue again contained columns \texttt{V} and \texttt{B\_V} then
   to find the stars in the catalogue fainter than twelfth magnitude
   and with a $B-V$\, of greater than 0.5 the criteria would be `\texttt{V > 12.0 .AND. B\_V > 0.5}'. Again see Appendix~\ref{EXPR} for the
   syntax of expressions,

  \item compute statistics for one or more columns.  The statistics
   are computed from either all the rows in the catalogue or just the
   subset of rows contained in a previously created selection.  The
   statistics computed are described in detail in Section~\ref{STATS}
   below,

  \item plot a simple scatter-plot from two columns.  The scatter-plot
   can show either all the rows in the catalogue or just the subset of
   rows contained in a previously created selection,

  \item plot a histogram from a column.  The histogram can be computed
   from either all the rows in the catalogue or just the subset of rows
   contained in a previously created selection,

  \item subsets extracted from the catalogue can be saved as new
   catalogues. These subsets can include new columns computed from
   expressions as well as columns present in the original catalogue,

  \item subsets extracted from the catalogue can be saved in a text file
   in a form suitable for printing, or in a form suitable for passing
   to other applications (that is, unencumbered with extraneous
   annotation).

\end{itemize}

A tutorial example of using \texttt{xcatview} to select stars which meet
specified criteria from a catalogue (a `recipe' in the jargon of
cookbooks) is included in \xref{SC/6: \textit{The CCD Photometric Calibration
Cookbook}}{sc6}{}\cite{SC6}.


\subsection{\label{STATS}Statistics computed for individual columns}

Statistics can be computed for one or more individual columns.  They can
be computed from either all the rows in the catalogue or just the subset
of rows comprising a selection which has been created previously.  Obviously,
only non-null rows are used in the calculations.  Statistics can be
displayed for columns of any data type, though for CHARACTER and LOGICAL
columns the only quantity which can be determined is the number of
non-null rows.

For each chosen column its name, data type and the number of non-null rows
(that is, the number of rows used in the calculation) are displayed and the
statistics listed in Table~\ref{STATT} are computed.  Though all these
quantities are standard statistics there is a remarkable amount of muddle
and confusion over their definitions, with textbooks giving divers
differing formul\ae.  For completeness, and to avoid any possible
ambiguity, the definitions used in \texttt{xcatview} and \texttt{catview} are
given below.  These formul\ae\ follow the \textit{CRC Standard Mathematical
Tables}\/\cite{CRCMT} except for the definition of skewness which is taken
from Wall\cite{WALL79}.

\begin{table}[htbp]

\begin{center}
\begin{tabular}{l}
Minimum             \\
Maximum             \\
Total range         \\
  \\
First quartile      \\
Third quartile      \\
Interquartile range \\
  \\
Median              \\
Mean                \\
Mode (approximate)  \\
  \\
Standard deviation  \\
Skewness            \\
Kurtosis            \\
\end{tabular}
\end{center}

\caption{Statistics computed for columns
\label{STATT} }

\end{table}

In the following the set of rows for which statistics are computed is
called the `current selection' and it contains $n$\, non-null rows.
$x_{i}$\, is the value of the column for the $i$th non-null row in the
current selection.  The definitions of the various statistics are then as
follows.

\begin{itemize}

  \item The minimum and maximum are (obviously) simply the smallest and
   largest values in the current selection and the total range is simply
   the positive difference between these two values.

  \item If the column is sorted into ascending order then the $j$th quartile,
   $Q_{j}$, is the value of element $j(n + 1)/4$, where $j = 1$, 2 or 3.
   Depending on the value $n$, there may not be an element which
   corresponds exactly to a given quartile.  In this case the value is
   computed by averaging the two nearest elements.

   The interquartile range is simply the positive difference between
   $Q_{1}$and $Q_{3}$.

  \item The median is simply the second quartile ($j = 2$).  The mean has
   its usual definition: the sum of all the values divided by the number
   of values.

   The value computed for the mode is not exact.  Indeed it is not
   obvious that the mode is defined for ungrouped data.  Rather, the value
   given is computed from the empirical relation:

  \begin{equation}
   {\rm mode} = {\rm mean} - 3({\rm mean} - {\rm median})
  \end{equation}

  \item The standard deviation, $s$, is defined as:


  \begin{equation}
   s = \sqrt{ \frac{1}{(n - 1)} \sum_{i=1}^{n} (x_{i} - {\rm mean})^2 }
  \end{equation}

  \item The skewness and kurtosis are defined in terms of moments.  The
   $k$th moment, $u_{k}$, is defined as

  \begin{equation}
   u_{k} = \frac{1}{n} \sum_{i=1}^{n} (x_{i} - {\rm mean})^{k}
  \end{equation}

   then

  \begin{equation}
   {\rm skewness} = u_{3}^{2} / u_{2}^{3}
  \end{equation}

   and

  \begin{equation}
   {\rm kurtosis} = u_{4} / u_{2}^{2}
  \end{equation}

   The expected values for the skewness and kurtosis are:

  \begin{itemize}

    \item skewness = 0 for a symmetrical distribution,

    \item kurtosis = 3 for a normal (or Gaussian) distribution.

  \end{itemize}

\end{itemize}

\subsection{Restarting xcatview after a crash}

Occasionally, due to some misadventure, \texttt{xcatview} might crash.  In
this eventuality some temporary files can be left in existence; these must
be deleted before \texttt{xcatview} can be used again.  The files will be in
subdirectory \texttt{adam} of your top-level directory (unless you have
explicitly assigned this directory to be elsewhere).  The files have names
beginning with \texttt{catview} and \texttt{xcatview}, for example:

\begin{terminalv}
catview_5003
xcatview_5001
\end{terminalv}

Simply delete these files and \texttt{xcatview} can then be started as usual.


\section{\xlabel{VIEW}\label{VIEW}Browsing and selecting from the command line}

\texttt{catview} is available for browsing catalogues and selecting subsets
from the command line. It provides the same functionality as
\texttt{xcatview}\footnote{Technically \texttt{xcatview} is a `front-end'
tcl/tk graphical user interface which manages the dialogue boxes and
forwards input from the user to the \texttt{catview} ADAM A-task, which,
in turn, manipulates the catalogue. Thus, strictly speaking, you are
running the same application in both cases. However, as a user you will
not normally be concerned with these details.}, but is much less easy to
use. Indeed it is not really intended for casual, interactive usage. If at
all possible I recommend that you use \texttt{xcatview} for casual, interactive
browsing of a catalogue. However, if you do not have an X display available
then you will have to use \texttt{catview}. It is also useful for running
prepared scripts which perform routine, standard, `batch' type operations.

In order to run \texttt{catview} type:

\begin{terminalv}
catview
\end{terminalv}

and the following prompt should appear:

\begin{terminalv}
ACTION - Action: /''/ >
\end{terminalv}

Using \texttt{catview} you can create an arbitrary number of selections from
the catalogue, each defined by its own criteria. \texttt{catview} has the
notion of the `current selection', which is the selection that it is
working on currently.  Columns chosen for display to the screen or a
text file, are listed from the current selection and statistics are
computed from the current selection.  Similarly when a new selection is
created it is extracted from the rows in the current selection.  By default
the most recent selection is the current one, though you may choose to make
any of the selections the current one.  If no selections have been made, the
current selection is the entire catalogue.

You issue commands to invoke the various functions supported by \texttt{catview} and reply to the prompts that they issue, as appropriate.
Type \texttt{HELP} for a list of the commands available.  They are as
follows.

\begin{description}

  \item[ \texttt{OPEN} ] Open a catalogue. You will be prompted for the
   name of the catalogue.

  \item[ \texttt{SHOWCOL} ] List a summary of all the columns in the
   catalogue.

  \item[ \texttt{DETCOL} ] List full details of all the columns in the
   catalogue.

  \item[ \texttt{SHOWPAR} ] List a summary of all the parameters in the
   catalogue.

  \item[ \texttt{DETPAR} ] List full details of all the parameters in the
   catalogue.

  \item[ \texttt{SHOWTXT} ] List the textual information associated with
   the catalogue.

  \item[ \texttt{SHOWROWS} ] Display the number of rows in the current
   selection.

  \item[ \texttt{SETCMP} ] Enter the list of columns which are to be
   displayed.  The list defines the columns which are listed by commands
   `\texttt{LIST}' or `\texttt{PREV}'. Both columns in the catalogue and new,
   computed columns may be listed.  Items in the list should be separated
   by a semi-colon (`\texttt{;}').  New columns have the form:

  \begin{terminalv}
name{expression}units
  \end{terminalv}

   where \texttt{name} is the name of the new column, \texttt{expression} is
   the expression which defines it and \texttt{units} are the units.  \texttt{name} and \texttt{expression} are mandatory, but \texttt{units} is
   optional. See Appendix~\ref{EXPR} for the syntax of expressions.

   As an example, to list catalogue columns \texttt{V}, \texttt{B\_V} and a
   computed column \texttt{B} defined by `\texttt{V + B\_V}' you would enter:

  \begin{terminalv}
V;B_V;B{V+B_V}
  \end{terminalv}

   Occasionally you may need to enter a list of columns and expressions
   which is longer than a single line.  Such long lists can be entered
   using a continuation line mechanism.  This mechanism is described in
   Section~\ref{LONGLINE}.

  \item[ \texttt{SHOWSEL} ] List details of all the selections which
   currently exist.

  \item[ \texttt{CHOSEL} ] Choose one of the existing selections to become
   the current selection.

  \item[ \texttt{SETSEL} ] Create a new selection.  You will be prompted to
   enter the expression defining the selection (see Appendix~\ref{EXPR}).
   Occasionally you may need to enter an expression which is longer than a
   single line.  Such long expressions can be entered using a continuation
   line mechanism.  This mechanism is described in Section~\ref{LONGLINE}.

  \item[ \texttt{SHOWRNG} ] List the columns on which a (fast) range
   selection can be created; that is, the columns on which the
   catalogue is sorted. In practice the catalogue is unlikely to be
   simultaneously sorted on more than one column.

  \item[ \texttt{SETRNG} ] Create a new range selection.  Range selections
   can be created essentially instantaneously, irrespective of the
   size of the catalogue.  However they can only be created on sorted
   columns.  You will be prompted for the name of the required column
   and the minimum and maximum values to be included in the range. If
   the column contains a celestial coordinate in a format that CURSA
   can recognise (see Appendix~\ref{ANGLE}) then the minimum and
   maximum values can optionally be entered as sexagesimal values in
   hours or degrees.  The usual rules for interpreting sexagesimal
   values in expressions are followed.  For example any of the following
   three values could be entered and all correspond to the same
   coordinate:

  \begin{center}
  \begin{tabular}{rll}
   \texttt{3:00:00}    & ~~~ & (hours)   \\
   \texttt{+45:00:00}  & ~~~ & (degrees) \\
   \texttt{0.78539816} & ~~~ & (radians) \\
  \end{tabular}
  \end{center}

  \item[ \texttt{SETROW} ] Set the current row number. You will be prompted
   to supply the required row number.

  \item[ \texttt{LIST} ] List the next page of output to the display
   terminal.

  \item[ \texttt{PREV} ] List the previous page of output to the display
   terminal.

  \item[ \texttt{SETSTAT} ] Enter the list of columns for which statistics
   are to be computed.  The list should comprise column names separated
   be semi-colons (`\texttt{;}').  For example, to compute statistics for
   columns \texttt{V}, \texttt{B\_V} and \texttt{U\_B} you would enter:

  \begin{terminalv}
V;B_V;U_B
  \end{terminalv}

   Occasionally you may need to enter a list of columns which is longer
   than a single line.  Such long lists can be entered using a continuation
   line mechanism.  This mechanism is described in Section~\ref{LONGLINE}.

  \item[ \texttt{SETDECPL} ] Set the number of decimal places to which the
   statistics will be displayed; you will be prompted to enter the
   required number.  Note that this option merely controls the number of
   decimal places to which the statistics are displayed.  They are
   always computed as DOUBLE PRECISION numbers in order to ensure the
   maximum possible accuracy.

  \item[ \texttt{STATS} ] Compute statistics for the specified columns from
   the current selection.  Optionally the statistics can be saved as a
   text file.  You will be prompted for the required file name; enter
   `\texttt{none}' if this option is not required.  In either case the
   statistics will be listed on the display terminal.

  \item[ \texttt{SCOPEN} ] Open a new scatter-plot.  You will be prompted
   for the following information.

  \begin{description}

    \item[ \texttt{GRPHDV} ] The name of the graphics device on which
     the plot will be drawn.  See Section~\ref{RCATCHART} and
     Table~\ref{GRPHDV} for details of the graphics devices available.

    \item[ \texttt{TITLE} ] The title of the plot.

    \item[ \texttt{XEXPR} ] The column or expression to be plotted as
     the $x$\/ axis.

    \item[ \texttt{YEXPR} ] The column or expression to be plotted as
     the $y$\/ axis.

  \end{description}

  \item[ \texttt{SCRANGE} ] Set the axis ranges of a scatter-plot.
   You will be prompted to indicate whether the plot is to be
   auto-scaled and also the axis ranges required.  Note that the
   ranges are prompted for (but not used) even if the plot is to be
   auto-scaled.

  \item[ \texttt{SCPLOT} ] Plot a scatter-plot from the current selection.
   You will be prompted for the plotting symbol and symbol colour
   required.

  \item[ \texttt{SCSHRNG} ] Show the range of the current scatter-plot.

  \item[ \texttt{SCLOSE} ] Close the current scatter-plot.

  \item[ \texttt{HSOPEN} ] Open a new histogram.  You will be prompted
   for the following information.

  \begin{description}

    \item[ \texttt{GRPHDV} ] The name of the graphics device on which
     the plot will be drawn.  See Section~\ref{RCATCHART} and
     Table~\ref{GRPHDV} for details of the graphics devices available.

    \item[ \texttt{TITLE} ] The title of the plot.

    \item[ \texttt{XEXPR} ] The column or expression to be plotted as
     the $x$\/ axis.

  \end{description}

  \item[ \texttt{HSRANGE} ] Set the $x$\/ axis range and other details
   of a histogram.  You will be prompted to indicate whether the histogram
   is to be auto-scaled and also the $x$\/ axis range required.  Note that
   the range is prompted for (but not used) even if the plot is to be
   auto-scaled.  Other details required are the specification for each
   histogram bin (the total number of bins or the width of each bin)
   and whether the histogram is to be normalised.

  \item[ \texttt{HSPLOT} ] Plot a histogram from the current selection.
   You will be prompted for the line colour required.

  \item[ \texttt{HSSHRNG} ] Show the range of the current histogram.

  \item[ \texttt{HSCLOSE} ] Close the current histogram.

  \item[ \texttt{FILE} ] List the current selection to a text file.  You
   will be prompted for the first and last rows (within the current
   selection) to be listed and for the output file name.  If you enter
   \texttt{0} for the last row number the last row in the selection will
   be listed (this trick avoids having to find the number of the last
   row). The columns specified by \texttt{SETCMP} are listed.

  \item[ \texttt{SAVECAT} ] Save the current selection as a catalogue.
   You will be prompted for the following information.

  \begin{description}

    \item[ \texttt{CATOUT} ] The name of the new catalogue.

    \item[ \texttt{CFLAG} ] The set of columns to be included in the
     new catalogue. The possible replies are:

    \begin{description}

      \item[ \texttt{TRUE} ] include all the columns in the original
       catalogue,

      \item[ \texttt{FALSE} ] include only the columns currently specified
       by \texttt{SETCMP}.

    \end{description}

    \item[ \texttt{TFLAG} ] Specify whether or not to copy the textual
     information associated with the original catalogue to the new
     catalogue. The possible replies are:

    \begin{description}

      \item[ \texttt{TRUE} ] copy the textual information,

      \item[ \texttt{FALSE} ] do not copy the textual information.

    \end{description}

    \item[ \texttt{COMM} ] Enter a line of text to be added as comments to
     the new catalogue.

  \end{description}

  \item[ \texttt{SHOWFMT} ] Show the data type, units and external display
   format for a column. You will be prompted for the name of the
   required column.

  \item[ \texttt{SETFMT} ] Set a new external display format and units
   for a given column. You will be prompted for the name of the
   column and the new external display format and units. Remember that
   the external display format must be a valid Fortran 77 format for
   the data type of the column.

  \item[ \texttt{SETCONF} ] Set a number of screen configuration options.
   You will be prompted to supply the following options.

  \begin{description}

    \item[ \texttt{SWID} ] The screen height in characters.

    \item[ \texttt{SHT} ] The screen width in characters.

    \item[ \texttt{SEQNO} ] Specify whether each line listed by
     \texttt{LIST} or \texttt{PREV} is started with a sequence number.
     The options are:

    \begin{description}

      \item[ \texttt{TRUE} ] include a sequence number,

      \item[ \texttt{FALSE} ] do not include a sequence number.

    \end{description}

    \item[ \texttt{NLIST} ] The number of lines to be output by a single
     invocation of \texttt{LIST} or \texttt{PREV}.

    \item[ \texttt{ANGRPN} ] Specify the way in which columns recognised by
     CURSA as containing angles (see Section~\ref{CELCOORD} and
     Appendix~\ref{ANGLE}) are to be listed. The options are:

    \begin{description}

      \item[ \texttt{SEXAGESIMAL} ] output as sexagesimal hours or degrees,

      \item[ \texttt{RADIANS} ] output as radians.

    \end{description}

    \item[ \texttt{ANGRF} ] Specify whether UNITS attribute of angles
     is to be reformatted prior to display.  The options are:

    \begin{description}

      \item[ \texttt{TRUE} ] reformat the UNITS attribute,

      \item[ \texttt{FALSE} ] do not reformat the UNITS attribute.

    \end{description}

  \end{description}

  \item[ \texttt{SETFILE} ] Set a number of configuration options for the
   output text file. Most of these options define the items which will
   be included in the text file.   You will be prompted to supply the
   following options.

  \begin{description}

    \item[ \texttt{FPGSZE} ] The number of lines in each page of the output
     file.

    \item[ \texttt{FWID} ] The width of each line of the output file in
     characters.

    \item[ \texttt{FSUMM} ] Specify whether a summary of the catalogue is
     to be included. The options are:

    \begin{description}

      \item[ \texttt{A} ] do not include a summary,

      \item[ \texttt{F} ] include a summary.

    \end{description}

    \item[ \texttt{FCOL} ] Specify whether details of all the columns are
     to be included. The options are:

    \begin{description}

      \item[ \texttt{A} ] do not include any details,

      \item[ \texttt{S} ] include a summary,

      \item[ \texttt{F} ] include full details.

    \end{description}

    \item[ \texttt{FPAR} ] Specify whether details of all the parameters are
     to be included. The options are:

    \begin{description}

      \item[ \texttt{A} ] do not include any details,

      \item[ \texttt{S} ] include a summary,

      \item[ \texttt{F} ] include full details.

    \end{description}

    \item[ \texttt{FTXT} ] Specify whether a copy of the textual information
     is to be included. The options are:

    \begin{description}

      \item[ \texttt{A} ] do not include the textual information,

      \item[ \texttt{F} ] include the textual information.

    \end{description}

    \item[ \texttt{FTABL} ] Specify whether a table of values is to be
     included. The options are:

    \begin{description}

      \item[ \texttt{A} ] do not include the table,

      \item[ \texttt{S} ] include the table, but without any column
       headings,

      \item[ \texttt{F} ] include the table with column headings.

    \end{description}

  \end{description}


  \item[ \texttt{COLNAME} ] List the names of all the columns in the
   catalogue.

  \item[ \texttt{HELP} ] Display a brief list of all the commands
   available.

  \item[ \texttt{EXIT} ] Terminate \texttt{catview}.

\end{description}


\subsection{\label{VIEW_SCRIPT}Running catview from a script}

In order to run \texttt{catview} from a script simply type the commands and
responses that you would have issued interactively into a text file.
They should be typed exactly as you would enter them interactively.

Figure~\ref{CATVIEW_SCRIPT} shows an example of a script for \texttt{catview}. It selects quasars with redshift greater than three and
brighter than nineteenth magnitude from a catalogue\footnote{The
catalogue used in this example is the \textit{Catalogue of Quasars and Active
Galactic Nuclei}\, by M.-P. Veron-Cetty and P. Veron\cite{VERON89}.}
and writes selected columns from the subset to a file in a format suitable
for passing to subsequent applications (that is, without any annotation).
The individual commands are:

\begin{description}

  \item[ \texttt{OPEN} ] Open the catalogue, here called `\texttt{qsover}'.

  \item[ \texttt{SETSEL} ] Select the objects with redshift greater than
   three and brighter than nineteenth magnitude.

  \item[ \texttt{SETCMP} ] Specify the columns to be listed: \texttt{ra},
   \texttt{dec}, \texttt{redshift}, \texttt{v}.

  \item[ \texttt{SETFILE} ] Set the configuration options for the
   information to be included in the text file.  The options given
   correspond to including only the specified columns, without any
   annotation.

  \item[ \texttt{FILE} ] Write the text file.  All the rows in the
   selection are written to file \texttt{qso.lis}.

  \item[ \texttt{EXIT} ] Terminate \texttt{catview}.

\end{description}

\begin{figure}[htbp]
\begin{center}

\begin{tabular}{l}
\texttt{OPEN}                    \\
\texttt{qsover}                  \\
\texttt{SETSEL}                  \\
\texttt{Redshift>3.0 .and. v<19.0} \\
\texttt{SETCMP}                  \\
\texttt{ra;dec;redshift;v}       \\
\texttt{SETFILE}                 \\
\texttt{60}                      \\
\texttt{132}                     \\
\texttt{A}                       \\
\texttt{A}                       \\
\texttt{A}                       \\
\texttt{A}                       \\
\texttt{S}                       \\
\texttt{FILE}                    \\
\texttt{1}                       \\
\texttt{0}                       \\
\texttt{qso.lis}                 \\
\texttt{EXIT}                    \\
\end{tabular}

\caption{Example script for \texttt{catview}\label{CATVIEW_SCRIPT} }

\end{center}
\end{figure}

To run \texttt{catview} from a script simply use Unix's input redirection
mechanism:

\begin{terminalv}
catview < catview_script.lis
\end{terminalv}

where \textit{catview\_script.lis}\/ is the name of the script.

\subsection{\label{LONGLINE}Continuation lines for long lists of
columns and expressions}

Occasionally you might need to enter a long list of columns and
expressions for display (\texttt{catview} option \texttt{SETCMP}) or a long
expression for a selection (\texttt{catview} option \texttt{SETSEL}).  In both
these cases a continuation line mechanism is available which allows lists
and expressions which are longer than a single input line to be entered.
This option is only available in \texttt{catview}, \textit{not}\, in \texttt{xcatview}.  \textit{If you need to specify long lists of columns and
expressions to be displayed, or a long expression defining a selection,
then you must use}\, \texttt{catview}.  In practice this restriction is not
too onerous because long lines usually arise when expressions are being
used to compute a set of new columns, which is often done from the command
line anyway.

In order to extend the list of columns and expressions to be displayed
across several lines simply append an `\texttt{@}' character to the end of
the line to be continued. The prompt:

\begin{terminalv}
CMPLIST - Columns to be listed>
\end{terminalv}

will be repeated and the line can be continued.  The details are as
follows.

\begin{itemize}

  \item An arbitrary number of continuation lines can be entered.

  \item A line which does not end in `\texttt{@}' is assumed to be the last.

  \item The `\texttt{@}' used to indicate continuation lines is quite separate
   from the `\texttt{;}' used to separate columns and expressions; ending a
   line in `\texttt{@}' does \textit{not}\, allow a `\texttt{;}' to be omitted.

  \item The list can be split at any point: in the middle of names,
   expressions \emph{etc.} (though the input is more easily read by eye if
   it is split at a natural break point such as the end of a column name).

\end{itemize}

Though this mechanism allows long lists and expressions to be entered,
there are necessarily still limits on the length of the list of columns and
expressions for display and on expressions defining selections, because
they are represented within \texttt{catview} as Fortran 77 CHARACTER variables.
In version \CURSAversion of CURSA these limits are:

\begin{quote}
\begin{tabular}{lr}
List of columns and expressions for display: &  1000 characters. \\
Expression defining a selection:             &   200 characters. \\
\end{tabular}
\end{quote}

\subsubsection{Examples}

\begin{enumerate}

  \item This example defines two new columns (\texttt{ra2000} and \texttt{dec2000}) and sets some existing columns to be displayed.

  \begin{terminalv}
ra2000{hmsrad(rah,ram,0.0)}radians{hours}; @
dec2000{dmsrad(decsign,decd,decm,0.0)}radians{degrees}; @
pk; v; v_limit; diam; radvel
  \end{terminalv}

  \item This example shows the definition of a single column (\texttt{ra2000}) being split across several lines.

  \begin{terminalv}
ra2000{hmsrad(rah,ram,  @
0.0)}radi  @
ans{hours};
  \end{terminalv}

  \item This example is a complete catview script for computing and
   listing two new columns.

  \begin{terminalv}
open
PLN
setcmp
ra2000{hmsrad(rah,ram,0.0)}radians{hours}; @
dec2000{dmsrad(decsign,decd,decm,0.0)}radians{degrees};
list
exit
  \end{terminalv}

\end{enumerate}

Exactly the same syntax applies when entering expressions to define
selections.


\section{\xlabel{HEAD}\label{HEAD}Listing header details}

To display a brief summary of a catalogue simply type:

\begin{terminalv}
catheader
\end{terminalv}

You then answer the prompt described below.  In this description the
prompt is identified by the corresponding ADAM parameter name, which
appears at the start of the prompt line.

\begin{description}

  \item[ \texttt{CATALOGUE} ] Enter the name of the catalogue.

\end{description}

By default the information displayed is:

\begin{itemize}

  \item the number of rows,

  \item the number of columns,

  \item the number of catalogue parameters,

  \item a list of the names of all the columns.

\end{itemize}

It is possible to specify that various other information is to be
displayed by including parameter \texttt{detail} on the command line.  For
example:

\begin{terminalv}
catheader detail=columns
\end{terminalv}

will list the details of all the columns in the catalogue.
Table~\ref{HEADERDET} shows the options available for \texttt{detail}.
There must be one or more spaces between `\texttt{catheader}' and `\texttt{detail}'.

It is also possible to copy the output from \texttt{catheader} to a text
file as well as displaying it on the terminal screen.  Type:

\begin{terminalv}
catheader  file=true
\end{terminalv}

The output will be written to a text file with the same name as the
catalogue but file type `\texttt{.lis}'.  Thus, if the catalogue was in file
\texttt{perseus.FIT} then the information would be written to file
\texttt{perseus.lis}.

\begin{table}[htbp]

\begin{center}
\begin{tabular}{ll}
Option           & Description \\ \hline
\texttt{summary}    & summary (default) \\
\texttt{columns}    & full details of all the columns\\
\texttt{parameters} & full details of all the parameters\\
\texttt{text}       & list of all textual information \\
\texttt{ast}        & list of any AST information \\
\texttt{full}       & full details (all the above) \\
\end{tabular}
\end{center}

\begin{quote}
\caption[Options available for \texttt{catheader} parameter \texttt{detail}]
{Options available for \texttt{catheader} parameter \texttt{detail}.  Some
catalogues may contain AST information.  AST is a mechanism for describing
the coordinate systems in which catalogue columns are expressed.  It is
documented in \xref{SUN/210}{sun210}{}\cite{SUN210} and
\xref{SUN/211}{sun211}{}\cite{SUN211}.
\label{HEADERDET} }
\end{quote}

\end{table}


\section{\xlabel{COPY}\label{COPY}Copying a catalogue}

To generate a new copy of a catalogue type:

\begin{terminalv}
catcopy
\end{terminalv}

The amount of textual information written to the output catalogue is
controlled using the command line mechanism described in
Section~\ref{COPYTEXT}.  You then answer the two prompts described below.
In these descriptions the prompts are identified by the corresponding
ADAM parameter name, which appears at the start of the prompt line.

\begin{description}

  \item[ \texttt{CATIN} ] Enter the name of the input catalogue.

  \item[ \texttt{CATOUT} ] Enter the name of the output catalogue.

\end{description}

It is possible to use \texttt{catcopy} to generate a copy of a catalogue in
the same format (FITS table, TST or STL) as the original, but there
is little point in doing so; the same result can be achieved using the
Unix command \texttt{cp}, which is much quicker. The real usefulness of
\texttt{catcopy} is in converting a catalogue to a new format; that is,
for example, converting a FITS table to an STL catalogue or vice versa.

\texttt{catcopy} has options to omit some or all of the parameters in the
input catalogue from the output catalogue or to add new parameters to the
output catalogue.  To omit all the parameters from the output catalogue
type:

\begin{terminalv}
catcopy  copypar=none
\end{terminalv}

To omit (or `filter out') selected parameters type:

\begin{terminalv}
catcopy  copypar=filter
\end{terminalv}

After being prompted for the input and output catalogues, \texttt{CATIN} and
\texttt{CATOUT}, as above, you will be prompted for the following parameter:

\begin{description}

  \item[ \texttt{PFILTER} ] Enter a comma-separated list of the parameters to
   filter out (that is, which are not to be copied).

\end{description}

Alternatively, this parameter can be given on the command line, for
example:

\begin{terminalv}
catcopy  copypar=filter  pfilter=\'FSTATION,PLATESCA,TELFOCUS\'
\end{terminalv}

Note that here the list of parameters must be enclosed in quotes, and
each quote must be preceded by a backslash character, as shown, to prevent
the quote being interpreted by the Unix shell.  To add new parameters to
the output catalogue type:

\begin{terminalv}
catcopy  addpar=true
\end{terminalv}

An arbitrary number of new parameters can be added.  After being prompted
for the input and output catalogues, \texttt{CATIN} and \texttt{CATOUT}, you will
be prompted to supply the following details for each parameter.

\begin{description}

  \item[ \texttt{PNAME} ] Name of the parameter.

  \item[ \texttt{PARTYP} ] Data type of the parameter.  The permitted types
   are: \texttt{REAL}, \texttt{DOUBLE}, \texttt{INTEGER}, \texttt{LOGICAL} and \texttt{CHAR}.

  \item[ \texttt{PCSIZE} ] Size of the parameter if it is of data type \texttt{CHAR}.

  \item[ \texttt{PVALUE} ] Value of the parameter.

  \item[ \texttt{PUNITS} ] Units of the parameter.

  \item[ \texttt{PCOMM} ] Comments describing the parameter.

\end{description}


\section{\xlabel{SORT}\label{SORT}Sorting a catalogue}

To sort a catalogue into ascending or descending order on some (numeric)
column type:

\begin{terminalv}
catsort
\end{terminalv}

Note that \texttt{catsort} generates a new sorted catalogue; it does not
overwrite the existing catalogue. The amount of textual information
written to the output catalogue is controlled using the command line
mechanism described in Section~\ref{COPYTEXT}.  You then answer the
series of prompts described below.  In these descriptions the prompts
are identified by the corresponding ADAM parameter name, which appears at
the start of the prompt line.

\begin{description}

  \item[ \texttt{CATIN} ] Enter the name of the input catalogue.

  \item[ \texttt{CATOUT} ] Enter the name of the output catalogue.

  \item[ \texttt{FNAME} ] Enter the name of the column on which the output
   catalogue is to be sorted.  Catalogues can be sorted on columns of any
   of the numeric data types.  They should not be sorted on columns of
   data type CHARACTER or LOGICAL.

  \item[ \texttt{ORDER} ] (default = `\texttt{ASCENDING}') Specify the order
   required for the output    catalogue.  The alternatives available are
   `\texttt{ASCENDING}' or `\texttt{DESCENDING}'.  Abbreviations down to and
   including `\texttt{A}' or `\texttt{D}' are permitted.

\end{description}

If a catalogue is sorted on a column which contains null values then the
rows for which the column is null will appear after all the rows with a
valid value. The order of such rows is unpredictable.



\section{\xlabel{SELECT}\label{SELECT}Selecting subsets from a catalogue}

Subsets can be extracted from a catalogue according to some criteria
using either the catalogue browsers \texttt{xcatview} (see
Section~\ref{XVIEW}) and \texttt{catview} (see Section~\ref{VIEW}) or using
application \texttt{catselect}.  Whereas the selection options in the
catalogue browsers are oriented towards the interactive exploration
and display of catalogues, \texttt{catselect} is oriented towards creating
`one-off' selections from a catalogue and saving them as a new
catalogue.  Also, \texttt{catselect} contains options for types of
selections which are not available in the catalogue browsers.

The remainder of this section describes \texttt{catselect}.  In addition to
saving the selected objects as a new catalogue, \texttt{catselect}
provides an option to save the rejected objects, which did not meet
whatever criteria was specified, to a second catalogue.  The types of
selection available in \texttt{catselect} are listed in Table~\ref{SELTYPT}
and described in Section~\ref{SELTYP}.

\begin{table}[htbp]

\begin{center}
\begin{tabular}{lcc}
Type of selection            & Option   & Browser?    \\ \hline
Arbitrary expression         & \texttt{E}  & $\bullet$   \\
Range within a sorted column & \texttt{R}  & $\bullet$   \\
Rows inside a rectangle      & \texttt{A}  & ($\bullet$) \\
Rows inside a circle         & \texttt{C}  & ($\bullet$) \\
Rows inside a polygon        & \texttt{P}  & \\
Every $n$th row              & \texttt{N}  & \\
\end{tabular}
\end{center}

\begin{quote}
The `Option' column lists the response to prompt \texttt{SETYP} which
corresponds to the type of selection.

The `Browser?' column indicates whether the type of selection is
available in the catalogue browsers \texttt{xcatview} and \texttt{catview}.
A bullet (`$\bullet$') indicates that the type of selection is
available.  A bullet in parenthesis (`($\bullet$)') indicates that the
type of selection is available in the browsers by entering the
appropriate arbitrary expression.
\end{quote}

\caption{Types of selection available in \texttt{catselect}
\label{SELTYPT} }

\end{table}

\subsection{Running catselect}

To run \texttt{catselect} simply type:

\begin{terminalv}
catselect
\end{terminalv}

The amount of textual information written to the output catalogue is
controlled using the command line mechanism described in
Section~\ref{COPYTEXT}.
You must answer a series of prompts in order to generate a catalogue
containing the required selection.  Some of these prompts differ
depending on the type of selection required, but the first few are
always the same.  These first few prompts are listed below together
with a corresponding explanation.  In this list the prompts are
identified by the corresponding ADAM parameter name, which appears at
the start of the prompt line.

\begin{description}

  \item[ \texttt{CATIN} ] Enter the name of the input catalogue from which
   objects are to be selected.

  \item[ \texttt{CATOUT} ] Enter the name of the output catalogue to
   contain the selected objects.  A catalogue with this name must \textit{not}\, already exist.  \texttt{catselect} will automatically create the
   output catalogue \textit{in toto}.

  \item[ \texttt{REJCAT} ] Specify whether a second output catalogue,
   containing the objects which did not satisfy the selection criteria,
   is to be created.  The options are:

  \begin{description}

    \item[ \texttt{TRUE} ] produce a catalogue of rejected objects,

    \item[ \texttt{FALSE} ] (default) do not produce a catalogue of rejected
     objects.

  \end{description}

  \item[ \texttt{CATREJ} ] Enter the name of the output catalogue to
   contain the rejected objects.  Obviously, this prompt is issued only
   if you specified that such a catalogue was required by replying `\texttt{TRUE}' to the \texttt{REJCAT} prompt.  A catalogue with the specified name
   must \textit{not}\, already exist.  \texttt{catselect} will automatically
   create the output catalogue \textit{in toto}.

  \item[ \texttt{SELTYP} ] Specify the type of selection required.  The
   options are:

  \begin{description}

    \item[ \texttt{E} ] arbitrary expression,

    \item[ \texttt{R} ] range within a sorted column,

    \item[ \texttt{A} ] rectangular area,

    \item[ \texttt{C} ] circular area,

    \item[ \texttt{P} ] polygonal area,

    \item[ \texttt{N} ] every \textit{n}th entry,

    \item[ \texttt{H} ] list the options available.

  \end{description}

   See Section~\ref{SELTYP} for a description of the options.  You will
   be re-prompted until a valid option is given.  Similarly, you will be
   reprompted after giving option `\texttt{H}'.

\end{description}

The remaining prompts vary, depending on the type of selection which is
being performed.  They are discussed with the corresponding type of
selection in Section~\ref{SELTYP}.

\subsection{\label{SELTYP}Types of selections}

This section describes the different types of selections available in
\texttt{catselect}.  The various types of selection are listed in
Table~\ref{SELTYPT}.

\begin{description}

  \item[Arbitrary expression] (\texttt{SELTYP} option \texttt{E}). \\
   Select the rows which satisfy an arbitrary mathematical expression.
   You will be prompted to enter the required expression.  Expressions
   are described in Appendix~\ref{EXPR}.  Occasionally you may need
   to enter an expression which is longer than a single line.  Such long
   expressions can be entered using the continuation line mechanism
   described in Section~\ref{LONGLINE}.

  \item[Range within a sorted column] (\texttt{SELTYP} option \texttt{R}). \\
   Select rows within a given range of a sorted column.  Range selections
   can be created essentially instantaneously, irrespective of the
   size of the catalogue.  However they can only be created on sorted
   columns.  You will be prompted for the name of the required column
   and the minimum and maximum values to be included in the range. If
   the column contains a celestial coordinate in a format that CURSA
   can recognise (see Appendix~\ref{ANGLE}) then the minimum and
   maximum values can optionally be entered as sexagesimal values in
   hours or degrees.  The usual rules for interpreting sexagesimal
   values in expressions are followed.  For example any of the following
   three values could be entered and all correspond to the same
   coordinate:

  \begin{center}
  \begin{tabular}{rll}
   \texttt{3:00:00}    & ~~~ & (hours)   \\
   \texttt{+45:00:00}  & ~~~ & (degrees) \\
   \texttt{0.78539816} & ~~~ & (radians) \\
  \end{tabular}
  \end{center}

  \item[Rectangular area] (\texttt{SELTYP} option \texttt{A}). \\
   Select the rows that lie within a given rectangular area.  You will
   be prompted for the name of the column defining the $x$\/ axis of
   the area and then the minimum and maximum $x$\/ coordinates of the
   area.  Corresponding prompts are then issued for the $y$\/ axis.

  \item[ Circular area] (\texttt{SELTYP} option \texttt{C}). \\
   This option, sometimes called a `cone search' (because it finds all
   the objects in a conical volume), finds all the rows within a
   specified radius of a given point.  It is usually used to find all
   the rows that lie within a specified angular distance from a given
   point on the celestial sphere.

   You will first be prompted for the names of the column containing
   the Right Ascension and then the column containing the Declination.
   Next you will be prompted for the Right Ascension of the central
   position followed by the central Declination.  Finally you will be
   prompted for the radius of the circle.

   The Right Ascension should be entered as a sexagesimal value in
   hours, the Declination as a sexagesimal value in degrees and the
   radius as a sexagesimal value in minutes of arc.  For example, to
   specify a search to find objects within twenty-three minutes of arc
   of Right Ascension \hms{10}{30}{00}{0} and Declination
   \dms{+35}{20}{00}{0} the values entered would be:

  \begin{quote}
  \begin{tabular}{lr}
   Central Right Ascension:   &  \texttt{10:30:00}  \\
   Central Declination:       &  \texttt{35:00:00}  \\
   Radius:                    &  \texttt{23:00}  \\
  \end{tabular}
  \end{quote}

   If a search radius of twenty-three seconds of arc was required the
   value entered would be `\texttt{0:23}' (note the leading zero and
   colon).  A decimal point and fractional seconds of arc can be entered
   if required.  For example, twenty-three and a half seconds of arc
   would be entered as `\texttt{0:23.5}'.

  \item[Polygonal area] (\texttt{SELTYP} option \texttt{P}). \\
   This option selects all the rows which lie inside a polygon which you
   specify.  The polygon can be of an arbitrary shape and have an
   arbitrary number of corners.  This option might be used to select
   objects in an irregularly shaped region of sky or to find objects with
   unusual properties in some two-dimensional space.  It could, for
   example, be used to isolate stars in the red giant branch of a
   Hertzsprung-Russell diagram.

   The coordinates of the polygon corners are read from a CURSA
   catalogue which you should prepare before running \texttt{catselect}.
   This polygon catalogue is probably most easily prepared using the
   STL format (see Appendix~\ref{FORMAT}); then it can simply be typed in
   with a text editor.
   All that the catalogue needs to contain are the two columns
   containing the coordinates of the polygon corners.  The names of
   these columns are not fixed; \texttt{catselect} prompts for them.
   Figure~\ref{POLY} shows an example STL format polygon catalogue.
   This example is available as file:

  \begin{terminalv}
/star/share/cursa/polygon.TXT
  \end{terminalv}

  \begin{figure}[htbp]

  \begin{terminalv}
!+
! Example STL format catalogue of polygon corners.
!
! Each row in the catalogue corresponds to a corner of the polygon.
!
! A C Davenhall (Edinburgh) 31/1/97.
!-

C X REAL 1  ! X coordinates of the polygon corners.
C Y REAL 2  ! Y coordinates of the polygon corners.

BEGINTABLE
 35.0   23.0
 68.0  122.0
159.0  143.0
174.0   76.0
105.0   68.0
  \end{terminalv}

  \caption{Example STL format catalogue of polygon corners
  \label{POLY} }

  \end{figure}

   Once the `polygonal area' option has been selected you will be
   prompted to enter the names of the columns holding the $x$\/ and
   $y$\, coordinates in which the polygon is defined in the input
   catalogue, the name of the polygon catalogue and finally the names
   of the columns holding the $x$\/ and $y$\/ coordinates in the polygon
   catalogue (\texttt{X} and \texttt{Y} in Figure~\ref{POLY}).

  \item[Every \textit{n}th entry] (\texttt{SELTYP} option \texttt{N}). \\
   This option selects every \textit{n}th row from the input catalogue; you
   are prompted for the value of \textit{n}.  This simple option is useful
   for producing a smaller, but representative, sample from a larger
   catalogue.  Such a sample might then be investigated interactively using
   \texttt{xcatview} (see Section~\ref{XVIEW}) or \texttt{catview} (see
   Section~\ref{VIEW}) in the case where the original catalogue was too
   large for interactive analysis.

\end{description}


\section{\xlabel{CONCOORD}\label{CONCOORD}Converting between celestial
coordinate systems}

CURSA contains some limited facilities for converting between different
celestial coordinate systems.  Application \texttt{catcoord} can convert
mean equatorial coordinates for a given equinox and epoch to mean
equatorial coordinates for a new equinox and epoch, to Galactic
coordinates\footnote{\texttt{catcoord} calculates `new' (IAU 1958)
Galactic longitude and latitude, conventionally denoted $l,b$.
Previously these coordinates were often denoted $l^{II},b^{II}$\,
in order to differentiate them from `old' (pre-1958) Galactic
coordinates, $l^{I},b^{I}$.  Old Galactic coordinates are now rarely
used.}, or to de Vaucouleurs' supergalactic coordinates.  \texttt{catcoord}
uses the Starlink subroutine library SLA to convert between coordinate
systems.  The manual for this library, \xref{SUN/67}{sun67}{}\cite{SUN67},
contains a brief introduction to the various celestial coordinate systems.
Further details can be found in standard textbooks on spherical astronomy
(see, for example, \textit{Spherical Astronomy}\, by R.M.~Green\cite{GREEN}).

\texttt{catcoord} creates a copy of the catalogue with the new coordinates
added.  It operates on a target list (see Section~\ref{TARGLIST}).
That is, it requires that the input catalogue contains columns of
coordinates which it can interpret.  The input catalogue must contain
columns of Right Ascension and Declination for some equinox and epoch.
Optionally it can also contain columns of proper motion in Right
Ascension and Declination, parallax and radial velocity which permit
more accurate conversions.  It is not necessary that all four additional
columns be present in order to use them.  For example, if only columns
of proper motion are present they can be used in isolation.  These additional
columns are usually only available in catalogues of relatively nearby and
well-observed stars.  In most catalogues the coordinates will simply
comprise a Right Ascension and Declination for some equinox and epoch.

The coordinates computed by \texttt{catcoord} are suitable for plotting,
display, pairing \emph{etc}.  However, for \textit{accurate} work they are \textit{not}\, suitable for further subsequent conversions to another equinox
and epoch.  This limitation arises because only new coordinates are
computed; the proper motions \emph{etc.} are not revised for the new equinox
and epoch.  Thus, in accurate work, new coordinates should always be
computed from the original coordinates in the target list, not from
intermediate coordinates created with \texttt{catcoord}.  However, this
caveat is only important when accurate coordinates are being computed.

\texttt{catcoord} offers only a limited set of conversions (converting
mean equatorial coordinates to a new equinox and epoch, to Galactic
coordinates or to supergalactic coordinates).  Additional conversions,
such as converting mean equatorial coordinates for some equinox and epoch
to apparent coordinates, are available using the Starlink package COCO (see
\xref{SUN/56}{sun56}{}\cite{SUN56}).  To use COCO first use \texttt{xcatview}
(see Section~\ref{XVIEW}) to save the coordinates as a text file in a
suitable format and them import them into COCO.

\subsection{Running catcoord}

To run \texttt{catcoord} in its simplest mode type:

\begin{terminalv}
catcoord
\end{terminalv}

By default \texttt{catcoord} simply reads columns of Right Ascension and
Declination from the input target list and computes equatorial
coordinates for some new equinox and epoch.  To compute more accurate
coordinates using columns of proper motion, parallax \emph{etc.} in the
input catalogue type:

\begin{terminalv}
catcoord full=true
\end{terminalv}

Similarly, to compute Galactic rather than equatorial coordinates type:

\begin{terminalv}
catcoord  coords=galactic
\end{terminalv}

or for supergalactic coordinates:

\begin{terminalv}
catcoord  coords=supergalactic
\end{terminalv}

These options may be combined.  Thus, to compute Galactic coordinates
from accurate input coordinates type:

\begin{terminalv}
catcoord  full=true  coords=galactic
\end{terminalv}

The amount of textual information written to the output catalogue is
controlled using the command line mechanism described in
Section~\ref{COPYTEXT}.

You then answer a series of prompts to define the required conversion.
All the possible prompts are listed below, identified by the
corresponding ADAM parameter name.  All the prompts will not appear in
a given run.  For example, \texttt{catcoord} tries to obtain the equinox
and epoch of the input coordinates from the input target list and will
only prompt you if it cannot find them.

The new coordinates may either be written to the same columns as the
original input coordinates (thus replacing them) or to new columns (in
which case both sets of coordinates will continue to be available).
All the other columns and parameters in the catalogue are simply copied.

\begin{description}

  \item[ \texttt{CATIN} ] Enter the name of the input catalogue (which
   must be a target list, see Section~\ref{TARGLIST}).

  \item[ \texttt{CATOUT} ] Enter the name of the output catalogue to
   contain the new coordinates.

  \item[ \texttt{EPOCHI} ] Specify the epoch of the input catalogue,
   for example, `\texttt{J2000}' or `\texttt{B1950}'.

  \item[ \texttt{EQUINI} ] Specify the equinox of the input catalogue,
   for example, `\texttt{J2000}' or `\texttt{B1950}'.

  \item[ \texttt{RAIN} ] Enter the name of the column containing Right
   Ascension in the input catalogue.

  \item[ \texttt{DECIN} ] Enter the name of the column containing
   Declination in the input catalogue.

  \item[ \texttt{PMRA} ] Enter the name of the column containing the
   proper motion in Right Ascension in the input catalogue.  Enter
   `\texttt{NONE}' if no column is available.

  \item[ \texttt{PMDE} ] Enter the name of the column containing the
   proper motion in Declination in the input catalogue.  Enter
   `\texttt{NONE}' if no column is available.

  \item[ \texttt{PLX} ] Enter the name of the column containing the
   parallax in the input catalogue.  Enter `\texttt{NONE}' if no column is
   available.

  \item[ \texttt{RV} ] Enter the name of the column containing the
   radial velocity in the input catalogue.  Enter `\texttt{NONE}' if no
   column is available.

  \item[ \texttt{EPOCHO} ] Specify the epoch of the output catalogue,
   for example, `\texttt{J2000}' or `\texttt{B1950}'.

  \item[ \texttt{EQUINO} ] Specify the equinox of the output catalogue,
   for example, `\texttt{J2000}' or `\texttt{B1950}'.

  \item[ \texttt{RAOUT} ] Enter the name of the column containing the
   new Right Ascension in the output catalogue.

  \item[ \texttt{DECOUT} ] Enter the name of the column containing the new
   Declination in the output catalogue.

  \item[ \texttt{L} ] Enter the name of the column containing the
   new Galactic longitude in the output catalogue.

  \item[ \texttt{B} ] Enter the name of the column containing the new
   Galactic latitude in the output catalogue.

  \item[ \texttt{SGL} ] Enter the name of the column containing the
   new supergalactic longitude in the output catalogue.

  \item[ \texttt{SGB} ] Enter the name of the column containing the new
   supergalactic latitude in the output catalogue.

\end{description}


\section{\xlabel{FCHART}\label{FCHART}Plotting finding charts}

CURSA contains the application \texttt{catchart} for plotting a basic finding
chart showing objects selected from a catalogue which lie within in a given
region of the sky.  \texttt{catchart} plots equatorial coordinates using
the tangent plane projection conventional in optical astronomy.  This
projection is described in standard textbooks on spherical astronomy (see,
for example, \textit{Spherical Astronomy}\, by R.M.~Green\cite{GREEN}).

\texttt{catcoord} plots target lists (see Section~\ref{TARGLIST}).  It will
plot either a single target list or superimpose several target lists on
a single finding chart.

All that can be guaranteed about a target list is that it contains
columns defining the coordinates of the objects.  Therefore, by default,
\texttt{catcoord} plots all the objects in the list using the same default
plotting symbol (circle, square \emph{etc}\/), drawn to the same size in
the same colour.  However, it is often desirable to plot the objects with
a specified symbol and colour and the size of the symbol varying with some
property of the object (traditionally magnitude).  The application \texttt{catchartrn} allows extra columns to be added to the target list which
prescribe how each object is to be plotted.

Thus, a simple, default finding chart can be produced by running \texttt{catchart} on any target list.  However, the plot can be customised
by running \texttt{catchartrn}, to specify how each object is to be plotted,
prior to running \texttt{catchart}.  The next section (\ref{CHARTCATS})
suggests some catalogues which might be suitable for producing finding
charts, the subsequent one (\ref{RCATCHART}) describes how to run \texttt{catchart} and the following one (\ref{GRT}) how to customise the plot.
The final section (\ref{EXAMPLECHART}) gives a complete worked example.

\subsection{\label{CHARTCATS}Suitable catalogues}

Any target list can be plotted as a finding chart.  However, often you
will want to select and plot stars from one of the large, general-purpose
astrometric catalogues.  Depending on the details of your work you might
either want to simply plot these stars in isolation or to use them as
`background' objects in a plot which also includes your more specialised
programme objects.  Various general-purpose astrometric catalogues which
are suitable for plotting finding charts are available to CURSA, either
by issuing a remote query via the Internet or by obtaining a local
copy of the catalogue.

\subsubsection{Remote query}

Remote catalogues and databases can be queried using \texttt{catremote} or
\texttt{xcatview} (see Section~\ref{REMACCSS}).  The catalogues available
include:

\begin{itemize}

  \item the \htmladdnormallinkfoot{SuperCOSMOS Sky Surveys}
   {http://www-wfau.roe.ac.uk/sss/} (SSS),

  \item the \htmladdnormallinkfoot{USNO PMM}{http://www.nofs.navy.mil/}
   catalogue\cite{PMM},

  \item various traditional standard catalogues, including the \textit{Catalogue of Positions and Proper Motions}\, (PPM)\cite{PPMN, PPMS},
   the \textit{SAO Catalog}, the \textit{Bonner Durchmusterung}\, and objects
   in \htmladdnormallinkfoot{SIMBAD}{http://simbad.u-strasbg.fr/Simbad}.

\end{itemize}

Example graphics translation files (see Section~\ref{GRT}) are available
to customise charts produced using most of these catalogues (see
Section~\ref{RCATCHARTRN}).  Subsets from the SuperCOSMOS catalogues can
be obtained by either using CURSA's remote access facilities or via form
on the SuperCOSMOS Web pages.

\subsubsection{Local copies}

Versions of the following catalogues are available which are fully
compatible with CURSA.  You can obtain a local copy, find the objects in
a given region of sky with \texttt{catselect} (see Section~\ref{SELECT}), and
then plot a finding chart.

\begin{itemize}

  \item The \textit{Catalogue of Positions and Proper Motions}\, (PPM)
   by S.~R\"{o}ser and U.~Bastian\cite{PPMN, PPMS}.  The PPM
   is similar in scale and scope to the SAO catalogue, but more modern
   and accurate.  A CURSA-compatible version is available and can be
   obtained by ftp (see Section~\ref{OBTAIN}).  This version covers both
   hemispheres.

  \item The Hipparcos and Tycho catalogues.  These catalogues, compiled
   from observations made with the Hipparcos astrometric satellite,
   contain extremely accurate coordinates.  They are available as
   ASCII files on CD-ROM.  Programs to reformat the files into
   CURSA-compatible catalogues can be obtained by ftp (see
   Section~\ref{OBTAIN}).

  \item Regions of the Hubble Space Telescope \textit{Guide Star
   Catalog}\, (GSC).  CURSA application \texttt{catgscin} can be used to
   reformat a GSC region into a CURSA-compatible target list (see
   Section~\ref{GSCIN}).

\end{itemize}


\subsection{\label{RCATCHART}Running catchart}

To run \texttt{catchart} simply type:

\begin{terminalv}
catchart
\end{terminalv}

It is possible to supply a title for the finding chart:

\begin{terminalv}
catchart  title=\'NGC 3623\'
\end{terminalv}

% \begin{terminalv}
% \texttt{catchart ~ title=$\texttt{\backslash}$'NGC~3623$\texttt{\backslash}$'}
% \end{terminalv}

Note that the title must be enclosed in quotes and each quote preceded
by a backslash character (as shown) in order to prevent the quotes from
being interpreted by the Unix shell.  By default \texttt{catchart} will plot
a single target list.  To plot several target lists superimposed on a
single finding chart type:

\begin{terminalv}
catchart  multiple=yes
\end{terminalv}

Also by default \texttt{catchart} marks the centre of the chart with a `gun
sight' open cross.  To suppress this cross type:

\begin{terminalv}
catchart  mcentre=no
\end{terminalv}

(think of `mark centre?' to remember `\texttt{mcentre}'.)  These options
can be combined.  For example, to plot several target lists with no
central cross type:

\begin{terminalv}
catchart  multiple=yes  mcentre=no
\end{terminalv}

You should then answer the following prompts.

\begin{description}

  \item[ \texttt{GRPHDV} ] Enter the name of the graphics device on which
   the plot is to be produced.  The names of some common graphics
   devices are listed in Table~\ref{GRPHDV}.  On Starlink systems you
   can run a program to list all the graphics devices which are
   currently available by typing:

  \begin{terminalv}
/star/bin/examples/gnsrun_gks
  \end{terminalv}

   See \xref{SUN/57}{sun57}{}\cite{SUN57} for further details.  Where
   the alternative exists the plots usually look better with with a
   `landscape' rather than `portrait' orientation.

  \begin{table}[htbp]

  \begin{center}
  \begin{tabular}{ll}
   Device                         & Name             \\ \hline
   X-windows.                     & \texttt{xwindows}   \\
                                  & \\
   Postscript A4 landscape        & \texttt{ps\_l}      \\
   Postscript A4 portrait         & \texttt{ps\_p}      \\
   Colour postscript A4 landscape & \texttt{pscol\_l}   \\
   Colour postscript A4 portrait  & \texttt{pscol\_p}   \\
                                  & \\
   Encapsulated postscript (landscape)        & \texttt{epsf\_l}    \\
   Encapsulated postscript (portrait)         & \texttt{epsf\_p}    \\
   Colour encapsulated postscript (landscape) & \texttt{epsfcol\_l} \\
   Colour encapsulated postscript (portrait)  & \texttt{epsfcol\_p} \\
  \end{tabular}
  \end{center}

  \caption{The names of some common graphics devices \label{GRPHDV} }

  \end{table}

  \item[ \texttt{GRPLST} ] Enter the name of the of the required target
   list.  If several are to be superimposed on a single finding chart
   you will be repeatedly prompted to enter the next to be plotted.
   To terminate the sequence enter `\texttt{QUIT}'.

\end{description}

\subsection{\label{GRT}Customising the plot}

By default \texttt{catchart} plots all the objects in a target list using
the same plotting symbol drawn to a constant size in the same colour.
Often this effect will not be what you want.  Traditionally in
astronomical atlases and charts stars are shown as circles whose
size varies with their magnitude.  Also different symbols and colours
may be used to indicate different types of object or different aspects
of the same sort of object.

The target lists which {catchart} might have to plot can come from a wide
variety of sources (for example, \texttt{catremote} allows you to retrieve
target lists from data centres and archives scattered around the
world).  All that can be guaranteed about them is that they
will contain columns of celestial coordinates.  No other assumptions
can be made about the other columns which they may contain or how the
objects in them should be plotted.  It is not even possible to guarantee
that the columns will include a magnitude; many non-optical catalogues
do not and even if they do it may not be appropriate to plot symbols
scaled on the magnitude.

To solve this problem application \texttt{catchartrn} is provided to allow
you to prescribe how the objects in a target list are to be plotted; you
specify the symbol, size and colour of the plotted objects.  These
quantities may be constant for all the objects or may be computed for
each object, based on the value of other columns for the object (the
traditional example is computing the symbol size from the magnitude).
\texttt{catchartrn} adds some extra columns and parameters to the target list
defining how the objects are to be plotted and \texttt{catchart}
automatically uses these.  This technique is very flexible and allows a
great deal of control over the way objects are plotted.  \texttt{catchartrn} itself reads a prescription of how the objects are to be
plotted from a simple pre-existing file, the so-called \textbf{graphics
translation file}.  Example graphics translation files are provided for
most of the catalogues in CURSA's default list of remote on-line
catalogues (see Table~\ref{EXAMPLEGRT} and Section~\ref{REMACCSS}).
You can either use one of these or prepare your own.  Thus, the sequence
for preparing a customised finding chart is:

\begin{enumerate}

  \item obtain or prepare a graphics translation file,

  \item run \texttt{catchartrn} to add the extra columns defining how
   the objects are to be plotted,

  \item run \texttt{catchart} to plot the finding chart.

\end{enumerate}

Often you will use the same graphics translation file for different
finding charts plotted from the same catalogue, or even from different
catalogues.  Usually you will need some knowledge of the columns in
the target list in order to construct a graphics translation file.
For example, you would need to know the name of the column containing
magnitude if you wished to scale the symbols on magnitude.  You can, of
course, examine the target list using \texttt{xcatview} (see
Section~\ref{XVIEW}), \texttt{catview} (see Section~\ref{VIEW}) or \texttt{catheader} (see Section~\ref{HEAD}).

The following sections describe how to run \texttt{catchartrn}, give a brief,
tutorial introduction to the graphics translation file, and finally
document the file format in full.  Creating a graphics translation file is
usually straightforward, particularly if you use one of the examples as a
starting point, and the tutorial will probably give enough information to
allow you to create your own.  You will probably only need to read the
full description if you want to create more complex effects.

\subsubsection{\label{RCATCHARTRN}Running catchartrn}

Once you have prepared a suitable graphics translation file you run
\texttt{catchartrn} to customise a target list by simply typing:

\begin{terminalv}
catchartrn
\end{terminalv}

The amount of textual information written to the new target list is
controlled using the command line mechanism described in
Section~\ref{COPYTEXT}.  You should then answer the following prompts.

\begin{description}

  \item[ \texttt{GTFILE} ] Enter the name of the graphics translation file.

  \item[ \texttt{CATIN} ] Enter the name of the original target list.

  \item[ \texttt{CATOUT} ] Enter the name of the output target list,
   customised for plotting.

\end{description}

Example graphics translation files are available for most of the catalogues
in the default list of remote on-line catalogues supplied with CURSA.
The files available are listed in Table~\ref{EXAMPLEGRT}.  The
SuperCOSMOS graphics translation files both plot all the objects in
the finding chart as ellipses.  In \texttt{scosmosbw.grt} all the objects are
plotted using the default colour (usually black objects on a white
background or vice versa).  In \texttt{scosmoscol.grt} the colour used for each
object varies with the value of the \texttt{CLASS} classification column in
the target list, according to the following scheme: stars are shown in
blue, galaxies in red, unclassifiable objects in green and objects
considered to be noise in yellow.  Most of the other files plot the objects
as symbols which scale with magnitude or flux.

\begin{table}[htbp]

\begin{center}
\begin{tabular}{rl}
Catalogue & File \\ \hline
\textit{Bonner Durchmusterung}                & \texttt{/star/share/cursa/bd.grt}  \\
HST \textit{Guide Star Catalog}               & \texttt{/star/share/cursa/gsc.grt} \\
IRAS \textit{Point Source Catalogue}          & \texttt{/star/share/cursa/iras\_psc.grt} \\
\textit{Positions and Proper Motions}\, (PPM) & \texttt{/star/share/cursa/ppm.grt} \\
\textit{Third Ref. Catalogue of Bright Galaxies} & \texttt{/star/share/cursa/rc3.grt} \\
\textit{SAO Catalog}                          & \texttt{/star/share/cursa/sao.grt} \\
SIMBAD                                     & \texttt{/star/share/cursa/simbad.grt} \\
SuperCOSMOS surveys (black and white plot) & \texttt{/star/share/cursa/scosmosbw.grt}  \\
SuperCOSMOS surveys (colour plot)          & \texttt{/star/share/cursa/scosmoscol.grt} \\
USNO PMM                                   & \texttt{/star/share/cursa/usno.grt} \\
\end{tabular}
\end{center}

\caption{Graphics translation files for remote catalogues \label{EXAMPLEGRT} }

\end{table}

\subsubsection{Tutorial example graphics translation files}

By convention graphics translation files have file type `\texttt{.grt}'.
A \textbf{graphics translation file} is a simple ASCII text file which
can be created and modified with an editor.  Figure~\ref{GRAPHTRAN}
shows a simple graphics translation file.  This example is available as
file:

\begin{terminalv}
/star/share/cursa/simple.grt
\end{terminalv}

It plots all the stars in a target list extracted from the version of
the \textit{Bonner Durchmusterung}\/ available at LEDAS as red filled circles
scaled according to magnitude.  The lines beginning with an exclamation mark
(`\texttt{!}') are comments and are ignored.  Similarly text to the right
of exclamation marks is ignored.  Blank lines are ignored.

\begin{figure}[htbp]

\begin{terminalv}
!+
! Simple graphics translation file.
!
! This file is suitable for use with target lists extracted from
! the version of the Bonner Durchmusterung available on-line at the
! Department of Physics and Astronomy, University of Leicester.
!
! All the stars are plotted as red filled circles scaled according
! to magnitude.
!
! A.C. Davenhall (Edinburgh) 10/6/97.
!-

SYMBOL = filledcircle   ! Plot the stars as filled circles,
COLOUR = red            ! coloured red.

UNITS  = fraction       ! Symbol size expressed as fraction of X range.

!
! Determine the symbol size by scaling the magnitudes between brightest
! and faintest stars in the target list.  VMAG is the magnitude column
! in the Bonner Durchmusterung.  Note how the minimum and maximum
! symbol sizes are flipped to accommodate magnitudes increasing 'the wrong
! way round'.

SIZE1  =  ascale(VMAG, 5.0D-2, 1.0D-2)
\end{terminalv}

\caption{Simple graphics translation file \label{GRAPHTRAN} }

\end{figure}

The plotting symbol is defined by the \texttt{SYMBOL} item.  The various
options are listed in Table~\ref{GRT_SYMBOL}.  Similarly, the colour
is set by item \texttt{COLOUR}.  The permitted colours are given in
Table~\ref{GRT_COLOUR}.  The symbol size is simply a fraction of the
plotting area available, as specified by \texttt{UNITS}.  The alternative
units are listed in Table~\ref{GRT_UNITS}.  The size of the plotting
symbol is defined by parameter \texttt{SIZE1}.  \texttt{SIZE1} can be any
valid CURSA expression (including a constant value, such as `\texttt{SIZE1~=~5.0E-2}', of course).  The additional functions \texttt{scale}
and \texttt{ascale} are provided for scaling quantities for display. They
are described in the following section.

Figure~\ref{GRAPHTRAN2} shows a more complicated graphics translation
file.  This example is available as file:

\begin{terminalv}
/star/share/cursa/complex.grt
\end{terminalv}

Again it plots all the stars in a target list scaled according to
magnitude.  However, here the scaling is between the fixed magnitude
range 7.5 - 10.0 rather than being determined from the brightest and
faintest stars in the list.  Also, the \texttt{IF \ldots ELSE \ldots END
IF} construct is used to vary the plotting symbol with magnitude.  Stars
brighter than magnitude 7.5 are plotted as blue open stars, stars
between magnitude 7.5 and 9.0 as red filled circles and fainter stars as
red open circles.

\begin{figure}[htbp]
\begin{terminalv}
!+
! More complicated graphics translation file.
!
! This file is suitable for use with target lists extracted from
! the version of the Bonner Durchmusterung available on-line at the
! Department of Physics and Astronomy, University of Leicester.
!
! Stars brighter than magnitude 7.5 are plotted as blue open stars.
! Fainter stars are plotted as red circles.  If the star is between
! magnitudes 7.5 and 9.0 the circle is solid, otherwise it is open.
! In all cases the size is scaled according to magnitude between the
! fixed range 7.5 - 10.0.
!
! A.C. Davenhall (Edinburgh) 10/6/97.
!-

IF VMAG < 7.5
  SYMBOL = openstar       ! Open star,
  COLOUR = blue           ! coloured blue.

ELSE IF VMAG >= 7.5  AND  VMAG < 9.0
  SYMBOL = filledcircle   ! filled circle,
  COLOUR = red            ! coloured red.

ELSE
  SYMBOL = opencircle     ! open circle,
  COLOUR = red            ! coloured red.

END IF

UNITS  = fraction         ! Symbol size expressed as fraction of X range.

!
! Determine the symbol size by scaling the magnitudes between the
! fixed range 7.5 - 10.0  VMAG is the magnitude column in the Bonner
! Durchmusterung.  Note how the minimum and maximum symbol sizes are
! flipped to accommodate magnitudes increasing 'the wrong way round'.

SIZE1  = scale(VMAG, 7.5D0, 1.0D1, 5.0D-2, 1.0D-2)
\end{terminalv}

\caption{Star / galaxy graphics translation file \label{GRAPHTRAN2} }

\end{figure}

Both the examples given here have shown the symbol size being scaled
with magnitude.  However, it is important to realise that the
expressions defining both \texttt{SIZE1} and the \texttt{IF \ldots ELSE \ldots
END IF} conditions can be any valid CURSA expressions (see
Appendix~\ref{EXPR}) involving any columns in the target list.  Graphics
translation files are provided for most of the catalogues in the default
list of remote on-line catalogues used by CURSA (see Table~\ref{EXAMPLEGRT})
and these can be used as further examples.

\subsubsection{The graphics translation file}

This section fully documents the \textbf{graphics translation file}.
By convention graphics translation files have file type `\texttt{.grt}'.
A graphics translation file is a simple ASCII text file which
can be created and modified with an editor.  The following general
rules apply to the contents of graphics translation files:

\begin{itemize}

  \item they are free format; there is no requirement that items occur
   at fixed absolute positions in lines,

  \item keywords are case-insensitive (though throughout this manual they
   are shown in upper-case for clarity),

  \item blank lines are ignored; they may be introduced freely to
   improve readability as required.

\end{itemize}

Any text following an exclamation mark (`\texttt{!}') is treated as a
comment and ignored.  The exclamation mark introducing a comment may
be either the first non-blank item in a line (`comment lines') or
may follow other items (`in-line comments').  Comments are terminated
automatically at the end of the line.

The graphics translation file defines how objects in a target list are
to be plotted.  Each symbol plotted is defined by a number of items:
\texttt{SYMBOL}, \texttt{COLOUR}, \texttt{UNITS}, \texttt{SIZEn} and \texttt{LABEL},
They are specified using the syntax:

\begin{center}
\textit{item\_name} \texttt{=} \textit{value}
\end{center}

For example:

\begin{terminalv}
SYMBOL = opencircle
\end{terminalv}

specifies that the objects will be plotted as open circles.  The details of
the individual items are as follows.

\paragraph{\texttt{SYMBOL}} The name of the symbol to be used to plot
the object.  The permitted names are listed in Table~\ref{GRT_SYMBOL}.
If omitted the default is \texttt{undefined}.

\begin{table}[htbp]

\begin{center}
\begin{tabular}{lllc}
Graphics symbol          & Name      & Size and shape specification & $N$ \\ \hline
omit from the plot       & \texttt{omit}           &  none             & 0 \\
undefined (\texttt{catchart} chooses)    & \texttt{undefined}    &  size  & 1 \\
dot                      & \texttt{dot}            &  none             & 0 \\
open circle              & \texttt{opencircle}     &  radius           & 1 \\
filled circle            & \texttt{filledcircle}   &  radius           & 1 \\
open square              & \texttt{opensquare}     &  centre to side   & 1 \\
filled square            & \texttt{filledsquare}   &  centre to side   & 1 \\
open triangle            & \texttt{opentriangle}   &  centre to vertex & 1 \\
filled triangle          & \texttt{filledtriangle} &  centre to vertex & 1 \\
open star (five point)   & \texttt{openstar}   &  centre to vertex     & 1 \\
filled star (five point) & \texttt{filledstar} &  centre to vertex     & 1 \\
plus sign (upright cross) & \texttt{plus}      &  centre to end of arm & 1 \\
multiplication sign (diagonal cross) & \texttt{mult} &  centre to end of arm & 1 \\
asterisk                 & \texttt{asterisk}   &  centre to end of arm & 1 \\
open ellipse             & \texttt{openellipse}   & $a$, $b$, position angle \S & 3 \\
filled ellipse           & \texttt{filledellipse} & $a$, $b$, position angle \S & 3 \\

\end{tabular}
\end{center}

\begin{quote}
\S - $a$\, = semi-major axis, $b$\, = semi-minor axis.
\end{quote}

\begin{quote}
\caption[Plotting symbols]{Plotting symbols.  $N$\, is the number of
\texttt{SIZEn} values needed to specify the symbol \label{GRT_SYMBOL} }
\end{quote}

\end{table}

\paragraph{\texttt{COLOUR}} The name of the colour to be used to plot
each object.  The names of the permitted colours are shown listed in
Table~\ref{GRT_COLOUR}.  If omitted the default is \texttt{default}.

\begin{table}[htbp]

\begin{center}
\begin{tabular}{ll}
Colour   & Name \\ \hline
default  & \texttt{default}  \\
red      & \texttt{red}      \\
green    & \texttt{green}    \\
blue     & \texttt{blue}     \\
cyan     & \texttt{cyan}     \\
magenta  & \texttt{magenta}  \\
yellow   & \texttt{yellow}   \\
\end{tabular}
\end{center}

\begin{quote}
\caption[Plotting colours]{Plotting colours.  The default colour is the
opposite of the plot background.  Usually it will be black or white
\label{GRT_COLOUR} }
\end{quote}

\end{table}

\paragraph{\texttt{UNITS}} The units of the \texttt{SIZEn} columns.  The options
are listed in Table~\ref{GRT_UNITS}.  The size of the symbol may be
specified as an absolute angular size, in which case the units will
usually be one of the angular measures.  Here the size of the symbol
corresponds to the actual size of an extended object on the sky.  For
example, the size of a circle could correspond to a circular isophote for
a nebula.  The `fraction' option is provided for the other case where the
symbol size varies with some property of the object, such as magnitude,
which is not an actual angular extent on the sky.  Here the symbol size is
simply a fraction of the $x$\/ axis range of the plot (expressed on a scale
where the entire range corresponds to 1.0).  If omitted the default is \texttt{fraction}.

\begin{table}[htbp]

\begin{center}
\begin{tabular}{ll}
Description             & Name           \\ \hline
fraction of $x$\/ range & \texttt{fraction} \\
seconds of arc          & \texttt{arcsec}   \\
minutes of arc          & \texttt{arcmin}   \\
degrees                 & \texttt{degrees}  \\
hours                   & \texttt{hours}    \\
radians                 & \texttt{radians}  \\
\end{tabular}

\caption{Plotting \texttt{UNITS} \label{GRT_UNITS} }
\end{center}

\end{table}


\paragraph{\texttt{SIZE1}, \texttt{SIZE2} and \texttt{SIZE3}} Expressions defining the
size of each symbol.  The values are (more or less; see below) normal CURSA
expressions involving columns in the targets list (see Appendix~\ref{EXPR}).
Most symbols (see Table~\ref{GRT_SYMBOL}) require only one size to be
specified.  This simple size is always given by \texttt{SIZE1}.  However, the
\texttt{openellipse} and \texttt{filledellipse} symbols require three values to
define the ellipse.  Here \texttt{SIZE1} is the semi-major axis, \texttt{SIZE2}
the semi-minor axis and \texttt{SIZE3} the position angle.  \texttt{SIZE1} and
\texttt{SIZE2} should evaluate to a value with the units specified by the \texttt{UNITS} item above.  However, the \texttt{SIZE3} should always be a position
angle in degrees, measured eastwards from north, following the usual
convention.

The following two functions were added to assist in calculating sizes for
columns such as magnitudes or flux which are not naturally angular extents
and need to be scaled to produce a symbol size.

\begin{description}

  \item[\texttt{scale(column, colmin, colmax, smin, smax)}] ~
  \\ Performs a simple linear scaling.  \texttt{column} is the name of
   the column to be scaled.  \texttt{colmin} and \texttt{colmax} are the
   minimum and maximum values in the column to be scaled.  If a
   value of the column, $V_{c}$, lies within the range \texttt{colmin}
   to \texttt{colmax}, then the scaled value returned, $V_{s}$, is
   computed using the formula:

  \begin{equation}
   V_{s} = \texttt{smin} + (V_{c} - \texttt{colmin}) .
   \frac{(\texttt{smax} - \texttt{smin} ) }{(\texttt{colmax} - \texttt{colmin} ) }
  \end{equation}

   If $V_{c}$\, is larger than \texttt{colmax} then \texttt{colmax} is
   returned; if it is smaller than \texttt{colmin} then \texttt{colmin} is
   returned.  \texttt{smin} and \texttt{smax} are the largest and smallest
   values of the plotting symbol, expressed in the units of \texttt{UNITS}.
   To accommodate quantities such as magnitudes which increase `the
   wrong way round' simply flip the values for \texttt{smin} and \texttt{smax}.

  \item[\texttt{ascale(column, smin, smax)}] ~
  \\ (auto-scale) is similar to \texttt{scale}, but the scaling is defined by
   the minimum and maximum values of the column.

\end{description}

\paragraph{\texttt{LABEL}} The name of the column to be used to label each object.

\vspace{5mm}

If the graphics translation file simply consists of a set of specifiers
for the above items they will be applied to all the objects in the list.
Often this approach will be adequate.  However, sometimes it will be
desired to plot different objects in different ways (for example with
different symbols or colours), depending on whether or not they meet
some criteria.  This behaviour is achieved by enclosing the definitions
for the graphics attributes within a set of clauses, where each clause
defines some aspect of the symbol to be used for objects which meet the
criteria.  The syntax is:

\begin{quote}
{\tt IF \textit{condition}

~~$\vdots$

ELSE IF \textit{condition}

~~$\vdots$

ELSE IF \textit{condition}

~~$\vdots$

ELSE

~~$\vdots$

END IF}
\end{quote}

\textit{condition} is the condition, expressed in terms of columns in the
targets list, which objects must satisfy to be plotted in the particular
way.  An example might be `\texttt{MAG .LT. 12.0}' to plot objects brighter
than 12th magnitude in a given way.  The following points apply.

\begin{itemize}

  \item An arbitrary number of clauses are permitted; there is no upper
   limit.

  \item The optional \texttt{ELSE} is a special clause which is applied
   to any objects in the targets list which do not satisfy any of the
   other cases.  There is no condition attached to \texttt{ELSE}.  If
   \texttt{ELSE} is omitted then objects which satisfy none of the cases
   are not written to the output target list (and hence are not
   plotted).

  \item Because the conditions defining the set of objects to be included
   in each case are not necessarily mutually exclusive it is technically
   possible for a given object to match more than one case.  In this event
   it will be plotted in the manner prescribed by the first case it
   matches.

  \item Any graphics attributes defined outside an \texttt{IF \ldots END
   IF} apply to all the objects plotted.

  \item An arbitrary number of separate \texttt{IF \ldots END IF} constructs
   can be included in the graphics translation file.  Typically, more than
   one might be used to set up different aspects of the plotting symbol
   (for example, one construct to set the plotting colour, based on the
   photometric colour of the object and a second to set the symbol shape
   based on the object classification).

  \item However, \texttt{IF \ldots END IF} constructs may \textit{not}\, be
   nested.

  \item The two words `\texttt{ELSE IF}' may be separated by zero, one
   or an arbitrary number of spaces; similarly the two words `\texttt{END IF}'
   may be separated by zero, one or an arbitrary number of spaces.

  \item The \texttt{LABEL} item cannot appear inside a clause.  If present it
   must be outside any clauses and refers to the entire target list.

\end{itemize}

\subsection{\label{EXAMPLECHART}Worked example}

This section gives a complete worked example of producing a customised
finding chart.  It starts be searching a remote on-line catalogue with
\texttt{catremote} (see Section~\ref{REMACCSS}) to find the list of objects
which will produce the chart.  The finding chart will show objects in the
USNO PMM catalogue\cite{PMM} within 5 minutes of arc of the radio source
PKS~1417-19.  Proceed as follows.

\begin{enumerate}

  \item You need to know the coordinates of the central object for epoch
   and equinox J2000.  You may already know them, or it might be
   convenient to look them up in a paper catalogue or on-line using
   \htmladdnormallinkfoot{SIMBAD}{http://simbad.u-strasbg.fr/Simbad}.
   Alternatively, it is easy to obtain them using \texttt{catremote}, type:

  \begin{terminalv}
catremote name  simbad_ns@eso  pks1417-19
  \end{terminalv}

   The object name is entered without spaces and may be in either case
   (upper or lower).  \texttt{catremote} will return the coordinates of
   the object for epoch and equinox J2000:

  \begin{terminalv}
Right Ascension: +14:19:50
Declination: -19:28:21
  \end{terminalv}

   The Right Ascension is in sexagesimal hours and the Declination in
   sexagesimal degrees.  If you know the coordinates for some equinox other
   than J2000 then you can use the Starlink utility COCO (see
   SUN/56\cite{SUN56}) to convert them to the required equinox.

  \item To search the USNO PMM catalogue\cite{PMM} for objects within
   5 minutes of arc of this position type:

  \begin{terminalv}
catremote query usno@eso  14:19:50  -19:28:21  5
  \end{terminalv}

   \texttt{catremote} will respond:

  \begin{terminalv}
!(Info.) Catalogue usno_eso_141950_m192821.tab written successfully.
  \end{terminalv}

   and the target list of selected objects will be written to file
   \texttt{usno\_eso\_141950\_m192821.tab} in your current directory.

  \item The target list can be customised for plotting using the graphics
   translation file supplied for the USNO PMM (see Table~\ref{EXAMPLEGRT}).
   Type:

  \begin{terminalv}
catchartrn
  \end{terminalv}

   and answer the prompts (the prompts are shown on the left and replies
   on the right):

  \begin{tabular}{ll}
   \texttt{GTFILE - Graphics translation file:}      & \texttt{/star/share/cursa/usno.grt} \\
   \texttt{CATIN - Input target list:}               & \texttt{usno\_eso\_141950\_m192821.tab} \\
   \texttt{CATOUT - Output graphics attribute list:} & \texttt{usno\_plot.txt} \\
  \end{tabular}

   The customised target list will be written to file \texttt{usno\_plot.txt}.

  \item To plot a finding chart from the customised target list type:

  \begin{terminalv}
catchart ~ title=\'PKS 1417-19\'
  \end{terminalv}

   Note that the title must be enclosed in quotes and the quotes preceded
   by a backslash (as shown) to prevent them being interpreted by the
   Unix shell.  Answer the prompts as follows (again prompts to the left,
   replies to the right):

  \begin{tabular}{ll}
   \texttt{GRPHDV - Graphics device:} & \texttt{ps\_l}          \\
   \texttt{GRPLST - Target list:}     & \texttt{usno\_plot.txt} \\
  \end{tabular}

   The position of PKS 1417-19 will be marked by an open cross.  Here
   the finding chart has been written as a postscript file, called \texttt{gks74.ps}, which may be printed, displayed interactively \emph{etc},
   as desired.

\end{enumerate}


\section{\xlabel{PLOT}\label{PLOT}Plotting with other packages}

CURSA contains only limited facilities for plotting: there are the
applications for generating finding charts described in
Section~\ref{FCHART} and both \texttt{xcatview} (see Section~\ref{XVIEW})
and \texttt{catview} (see Section~\ref{VIEW}) can plot simple scatter-plots
and histograms.  For more sophisticated plots it is necessary to export
the columns to be plotted into specialised plotting packages.  Both \texttt{xcatview} and \texttt{catview} can generate output files suitable for input to
plotting packages.  Usually such files should consist of just the table of
values to be plotted, with no extraneous annotation or formatting. The
example in Section~\ref{VIEW_SCRIPT} and Figure~\ref{CATVIEW_SCRIPT}
shows how to configure \texttt{catview} to produce such an output file.

Several plotting packages are available on Starlink. One such is PONGO,
which is documented in \xref{SUN/137}{sun137}{}\cite{SUN137}.
Figures~\ref{PONGO_SCATTER} and \ref{PONGO_AITOFF} show two example PONGO
scripts for producing plots. Figure~\ref{PONGO_SCATTER} produces a
scatter-plot of redshift against $V$\, magnitude from a table where the
$V$\, magnitude is read from the fifth column and the redshift from the
fourth.  Figure~\ref{PONGO_AITOFF} produces an all-sky plot using a PONGO
Aitoff projection. Here the celestial coordinates are read from the second
and third columns in the table.

\begin{figure}[htbp]

\begin{terminalv}
proc scatter
   PONGO
   BEGPLOT xwindows
   READF xcol=5 ycol=4 all RESET
   DLIMITS
   BOXFRAME
   POINTS 17
   LABEL 'V magnitude' 'Redshift' 'Redshift against V'
   ENDPLOT
endproc
\end{terminalv}

\caption{Procedure to produce a PONGO scatter-plot}
\label{PONGO_SCATTER}

\end{figure}

\begin{figure}[htbp]

\begin{terminalv}
proc aitoff
   PONGO
   BEGPLOT xwindows

   RESETPONGO
   EXPAND 0.7

   READF xcol=2 ycol=3 all RESET

{ Aitoff projection.

   DLIMITS XMIN=-3.3 XMAX=3.3 YMIN=-1.6 YMAX=1.6 PROJECTION=AITOFF ~
     RACENTRE=12 DECCENTRE=0
   WNAD
   MTEXT T 1.0 0.5 0.5 'Aitoff centre \ga=12\uh\d \gd=0\(2729)'

   GRID PROJECTION=AITOFF
   POINTS 17

   VSTAND
   CHANGE RESET

   ENDPLOT
endproc
\end{terminalv}

\caption{Procedure to produce a PONGO Aitoff all-sky plot}
\label{PONGO_AITOFF}

\end{figure}

You can use these examples as a basis for your own scripts. They are
simple text files prepared with an editor; either type them in \textit{ab ovo}\, or paste them from the Latex source for this document, and
modify as appropriate.

To run a PONGO script enter the following commands.

\begin{description}

  \item[ \texttt{icl} ] start ICL; the prompt will change to `\texttt{ICL>}'.

  \item[ \texttt{load scatter} ] load the PONGO procedure (here assumed to
   have file name \texttt{scatter.icl}; substitute the name of your file
   as appropriate).

  \item[ \texttt{scatter} ] run the procedure. Again substitute the name
   of your file as appropriate.

  \item[ \texttt{exit} ] leave ICL.

\end{description}

Alternatively you can use PONGO interactively to assemble the required
plot. Many more options are available than are described here. They
are fully documented in \xref{SUN/137}{sun137}{}.


\section{\xlabel{PAIR}\label{PAIR}Pairing two catalogues}

\texttt{catpair} is provided to identify `corresponding' objects
in two catalogues; objects are considered to correspond if they have
similar positions. An output catalogue is generated from the list of
corresponding objects.

In astronomical catalogues the `corresponding' rows in two catalogues
are usually rows which contain data for the same astronomical object.
Traditionally in relational database systems corresponding rows are
identified by having identical values for some field, such as a name.
For example, two rows might be considered to correspond if a name field
in both catalogues adopted the value `NGC~1305' for both rows. This
operation is usually called \textbf{joining} the two catalogues.

\begin{figure}[htbp]

\begin{terminalv}
          +------------------------------------------------------+
          |         x                    *                 *     |
          |  x                *                *                 |
          |                         x                x*          |
          |           x                  *                  *    |
          |    x          x*       *             *               |
     Dec. |                                           *          |
          |        x*            x         x*                    |
          |                                      *          *    |
          |   x        x*  x         *                           |
          |                                 *       x*           |
          |      x       x      x                          *     |
          |                             *                        |
          +------------------------------------------------------+

                                    R.A.
\end{terminalv}

\begin{center}
\begin{tabular}{lll}
\texttt{x} &  -  &  Object in primary dataset.    \\
$\ast$  &  -  &  Object in secondary dataset.  \\
   &  &  \\
\end{tabular}

Adjacent objects are pairs.
\end{center}

\caption{Two datasets for joining \label{TWO_CAT} }

\end{figure}

In astronomical problems such joining by an exact match is relatively
uncommon. A more common case is where corresponding objects are
identified by similar positions in both catalogues. This situation is
illustrated in Figure~\ref{TWO_CAT}. The important point here is that,
essentially because of measurement errors, the corresponding positions
are merely similar, not an exact match. This circumstance makes
establishing corresponding rows a much more complicated and problematic
process. In practice the positions used are almost always some type of
two-dimensional coordinates; usually celestial coordinates such as Right
Ascension and Declination, or possibly Cartesian coordinates of some
sort. In principle one, three or higher dimensional coordinates could be
used though they are not important in practice. \texttt{catpair} only supports
joining based on two-dimensional coordinates, though the coordinates
may be either Cartesian or spherical-polar.

In CURSA this special sort of join based on an approximate match in
two-dimensional coordinates is called \textbf{pairing}. Thus, in this
usage, pairing is a special case of joining catalogues, albeit one which
is important in astronomical practice.

\texttt{catpair} operates on two input catalogues, known as the \textbf{primary}
and \textbf{secondary} catalogues. To fix ideas, think of the primary as being
a small list of target objects which you have compiled, and the secondary
as being a standard catalogue, such as the SAO star catalogue, one of the
\textit{Durchmusterungen}\, or Dreyer's \textit{New General Catalogue}\, of
non-stellar objects. The final result of the pairing is a new catalogue
containing the paired objects; the \textbf{output} catalogue.

If you wish to pair several catalogues to create a single output
catalogue you should invoke \texttt{catpair} several times, creating
intermediate paired catalogues as appropriate.

Pairing is a relatively complicated process and you must answer several
prompts to fully specify the operations to be performed. The following
two sections, `Requirements' and `Running \texttt{catpair}' respectively
describe the requirements for \texttt{catpair} and how to run it. You should
read at least these two sections. Subsequent sections describe various
aspects of the pairing process in greater detail. While it is not strictly
necessary to read these latter sections, they may help you to understand
what \texttt{catpair} is doing and hence to use it more effectively.


\subsection{Requirements}

Obviously before running \texttt{catpair} you must have a primary and a
secondary catalogue. The secondary catalogue \textit{must}\, be sorted on
the second column that is to be used for the pairing (usually this will
be the $y$ or Declination coordinate). If your secondary is not sorted
in this way then use \texttt{catsort} (see Section~\ref{SORT}, above) to
create a suitably sorted secondary catalogue.

You need to know the names of the columns in both catalogues which
contain the coordinates which are to be used for the pairing (and whether
they are Cartesian or spherical-polar coordinates). If you are in doubt
about the columns in the catalogues use \texttt{catheader} (see
Section~\ref{HEAD}, above) to obtain the details. If the coordinates
are Cartesian then the coordinates in both input catalogues must be in
the same system, with the same units,\footnote{\texttt{catpair} does not
actually check that the units attribute is the same for the various
columns holding the coordinates because in CURSA units are treated
purely as comments.} zero point and orientation. That is, a given value
for the coordinates (say 23.5, 105.7) should correspond to the same
position in both catalogues. If the coordinates are spherical-polar
they must always be in units of radians. The coordinates in the two
catalogues should be of the same type (equatorial, Galactic \emph{etc.}\/)
and if they are equatorial they should have the same system, epoch and
equinox.

Finally you need to specify the critical distance, $D$, which determines
whether two objects, one in each catalogue, are considered pairs or not.
If the actual separation of the two objects is less than or equal to
this distance then they are considered pairs; if it is greater then they
are not. In \texttt{catpair} this critical distance may be either a
constant, a column in the primary (so it varies for different objects in
the primary) or an expression based on columns in the primary. In
practice the value adopted for the critical distance is often derived
from the errors associated with the positions in the catalogues. If
you do not already know the errors on the positions in your catalogues,
you could consult the textual information information associated with
the catalogue, which will often contain these details. Again use \texttt{catheader} (see Section~\ref{HEAD}) to access this information.


\subsection{Running catpair}

To run \texttt{catpair} simply type:

\begin{terminalv}
catpair
\end{terminalv}

By default \texttt{catpair} writes a summary of the pairing options
specified as textual information in the output catalogue.  This
information is useful documentation of the pairing and you will
usually want to retain it.  However, you can specify that it is not to
be written by specifying an extra item on the command line, as follows:

\begin{terminalv}
catpair  text=none
\end{terminalv}

There must be one or more spaces between `\texttt{catpair}' and `\texttt{text=none}'.  \texttt{catpair} has an option to include in the output
catalogue three special columns containing additional details for the
paired objects.  These columns are described in Section~\ref{PAIR_SPCOL},
below.  By default these additional columns are not created.  To include
them in the output catalogue type:

\begin{terminalv}
catpair  spcol=true
\end{terminalv}

You must answer a fairly long series of prompts in order to
specify the behaviour of \texttt{catpair}. These prompts are listed below,
in the order in which they are issued by the program, together with a
corresponding explanation. In this list the prompts are identified
by the corresponding ADAM parameter name, which appears at the start
of the prompt line.

\begin{description}

  \item[ \texttt{PRIMARY} ] Enter the name of the primary input
   catalogue.

  \item[ \texttt{SECOND} ] Enter the name of the secondary input
   catalogue. This catalogue \textit{must}\, be sorted on the second
   column to be used in the pairing (usually the $y$ or Declination
   coordinate).

  \item[ \texttt{OUTPUT} ] Enter the name of the output catalogue to
   contain the set of paired objects. A catalogue with this name must \textit{not}\, already exist. \texttt{catpair} will automatically create the output
   catalogue \textit{in toto}.

  \item[ \texttt{CRDTYP} ] (default = `\texttt{S}') Specify the type of
   coordinates which are to be used for the pairing. The possibilities are
   either Cartesian coordinates (`\texttt{C}') or celestial spherical-polar
   coordinates (`\texttt{S}') such as Right Ascension and Declination.

  \item[ \texttt{PCRD1} ] Enter the name of the column in the primary
   catalogue containing the first column to be used in the pairing. This
   column will usually be an $x$\/ coordinate or a Right Ascension.

  \item[ \texttt{PCRD2} ] Enter the name of the column in the primary
   catalogue containing the second column to be used in the pairing. This
   column will usually be a $y$\/ coordinate or a Declination.

  \item[ \texttt{SCRD1} ] Enter the name of the column in the secondary
   catalogue containing the first column to be used in the pairing. This
   column will usually be an $x$\/ coordinate or a Right Ascension.

  \item[ \texttt{SCRD2} ] Enter the name of the column in the secondary
   catalogue containing the second column to be used in the pairing. This
   column will usually be a $y$ coordinate or a Declination. The secondary
   catalogue \textit{must}\, be sorted on this column.

  \item[ \texttt{PDIST} ] Enter the critical distance determining whether
   two objects, one in each catalogue, are considered pairs or not. If the
   actual separation of the two objects is less than or equal to this
   distance then they are considered pairs; if it is greater then they are
   not.  In the simplest case this critical distance is a simple numeric
   value, such as twenty-three minutes of arc, constant for all the objects
   in the catalogues. However, it may also be a column in the primary
   catalogue (but \textit{not}\, a column in the secondary) or an expression
   involving columns in the primary (see Section~\ref{PAIR_CRIT}, below).

   If the pairing coordinates are Cartesian then a constant critical distance
   would typically be specified as a simple decimal number, for example
   `23.0'. However, if they were celestial coordinates then it could be
   specified as any of the forms in which an angle can be input: a floating
   point number in radians, or a sexagesimal value in hours or degrees. In
   addition a special format is available in \texttt{catpair} in which the
   separation is given as a floating point number expressed in seconds of
   arc, immediately followed by the string `\texttt{arcsec}'. For example, a
   separation of twenty-three minutes of arc could be entered as any of
   the following values:

  \begin{center}
  \begin{tabular}{rll}
   \texttt{+00:23:00}    & ~~~ & (sexagesimal degrees) \\
   \texttt{1380.0arcsec} & ~~~ & (seconds of arc)      \\
   \texttt{00:01:31.99}  & ~~~ & (sexagesimal hours)   \\
   \texttt{6.6904288E-3} & ~~~ & (radians)             \\
  \end{tabular}
  \end{center}

   Note that the sign is necessary in the value in sexagesimal degrees to
   ensure that the value is interpreted as degrees, not hours. The examples
   in sexagesimal hours and radians are not particularly sensible here.

  \item[ \texttt{PRTYP} ] (default = `\texttt{C}') Select the `type of
   pairing' required, that is specify which set of rows from the two input
   catalogues are to be retained in the output catalogue. Briefly, the
   options are:

  \begin{description}

    \item[\texttt{C} ] (COMMON) retain only the common or paired rows in the
     two catalogues,

    \item[\texttt{M} ] (MOSAIC) retain all the rows in the primary and the
     unpaired rows in the secondary,

    \item[\texttt{P} ] (PRIMARY) retain all the rows in the primary (for
     unpaired objects columns copied from the secondary are set to null).

    \item[\texttt{R} ] (PRIMREJ) retain only the unpaired rows in the primary,

    \item[\texttt{A} ] (ALLREJ) retain the unpaired rows in both the primary
     and the secondary.

  \end{description}

   These options are described in greater detail in Section~\ref{PAIR_PRTYP},
   below.

  \item[ \texttt{MULTP} ] (default = `\texttt{yes}') Specify how multiple
   matches in the primary are to be handled. The options are either to
   retain the single closest match or to retain all the matches. The
   treatment of multiple matches is described in detail in
   Section~\ref{PAIR_MULTIPLE}, below.

  \item[ \texttt{MULTS} ] (default = `\texttt{no}') Specify how multiple
   matches in the secondary are to be handled. The options are either to
   retain the single closest match or to retain all the matches. The
   treatment of multiple matches is described in detail in
   Section~\ref{PAIR_MULTIPLE}, below.

  \item[ \texttt{ALLCOL} ] (default = `\texttt{yes}') Specify the set of
   columns to be retained in the output catalogue. The options are to either
   retain all the columns from both input catalogues or to retain specified
   columns from either input catalogue. If you are in doubt you should
   retain all the columns. This alternative is the `safest' and simplest,
   though it may result in the output catalogue containing columns which you
   do not need and consequently using more disk space than is strictly
   necessary.

   If you choose to retain all the columns they are simply copied
   automatically from the input catalogue, without further intervention on
   your part. However, if you choose to specify the columns to retain you
   will subsequently be prompted for the names of the columns to be retained
   (and hence you must be prepared with this information). The details of
   specifying named input columns are described in Section~\ref{PAIR_INP_COL},
   below.

   If you choose to retain all the columns, the columns created in the output
   catalogue will have the same names (and other attributes) as the
   corresponding columns in the input catalogue. However, in the case where
   identically named columns in the primary and secondary catalogues would
   cause the output catalogue to contain two identically named columns, the
   names of the columns in the output catalogue are disambiguated by
   appending `\texttt{\_S}' to the name of the column originating in the
   secondary.

  \item[ \texttt{PRMPAR} ] (default = `\texttt{yes}') Specify whether the
   parameters of the primary are to be copied to the output catalogue.

  \item[ \texttt{SECPAR} ] (default = `\texttt{no}') Specify whether the
   parameters of the secondary are to be copied to the output catalogue.

  \item[ \texttt{PTEXT} ] (default = `\texttt{C}') Specify what textual
   information associated with the primary is to be copied to the output
   catalogue.  The options are: `\texttt{A}' - all,  `\texttt{C}' - comments
   and history only and `\texttt{N}' - none.

  \item[ \texttt{STEXT} ] (default = `\texttt{N}') Specify what textual
   information associated with the secondary is to be copied to the output
   catalogue.  The options are: `\texttt{A}' - all,  `\texttt{C}' - comments
   and history only and `\texttt{N}' - none.

\end{description}

\subsubsection{\label{PAIR_SPCOL}Special columns}

If \texttt{catpair} is invoked with the option \texttt{spcol=true} then three
special columns giving details of the pairing for each object will be
included in the output catalogue.  These columns are:

\begin{description}

  \item[\texttt{SEPN}] the separation between the paired primary and secondary
   objects,

  \item[\texttt{PMULT}] the number of matches in the primary,

  \item[\texttt{SMULT}] the number of matches in the seconary.

\end{description}

Usually fields in columns \texttt{PMULT} and \texttt{SMULT} will have a value
of one for paired objects.  However, in cases where there were multiple
matches for the pair the values will be larger.  See
Section~\ref{PAIR_MULTIPLE}, below for a discussion of the handling of
multiple matches.

\subsubsection{\label{PAIR_INP_COL}Retaining specified columns}

If you choose to retain in the output catalogue only some of the columns
in the two input catalogues you will be prompted to supply the names of
the columns required and hence you must be prepared with this information.
If you are not familiar with the details of the columns in your input
catalogues you can use \texttt{catheader} (see Section~\ref{HEAD}, above) to
obtain the necessary information.

Once you have indicated that you are to retain only specified columns (by
replying `\texttt{NO}' to prompt \texttt{ALLCOL}) you will be prompted to enter
the names of columns to be retained from the primary catalogue. Type the
name of the first column required then hit return. For example to retain
column \texttt{X} simply type:

\begin{terminalv}
X
\end{terminalv}

A corresponding column with the same name and other attributes will be
created in the output catalogue. Columns may also be retained with a name
in the output catalogue which differs from the name of the corresponding
input column. In this case you type: the name of the input column, a right
chevron and the name required for the new output column. For example, if
the column was called \texttt{X} in the input catalogue and \texttt{X\_PRIM}
in the output catalogue you would type:

\begin{terminalv}
X > X_PRIM
\end{terminalv}

An arbitrary number of spaces may appear on either side of the right
chevron. A column with the specified new name will be created in the
output catalogue, and all its other attributes will be the same as those
of the corresponding column in the input catalogue.

Continue in this fashion until you have entered all the columns required
from the primary.  Then type:

\begin{terminalv}
END
\end{terminalv}

Next you will be prompted for the names of the columns required from the
secondary. Proceed exactly as for the primary and again type \texttt{END}
when you have finished.

If you are retaining a large number of columns it is inconvenient (and,
indeed, error-prone) to have to supply all the column names interactively
in response to prompts. In this case it is much more  convenient to run
\texttt{catpair} from a script, and I strongly recommend that you do so. This
option is described in Section~\ref{PAIR_SCRIPT}, below.

The handling of multiple columns with the same name in the output catalogue
is rather different when column names are being specified than when all the
columns are being copied automatically. A single column with the specified
name is created in the output catalogue and values for all the appropriate
columns in the input catalogue are written to the field of this column
for the current row. This behaviour is adopted because there there are
cases, particularly in MOSAIC and ALLREJ pairing where you might want
fields for corresponding columns in the two input catalogues to be
written to a single column in the output catalogue. In the case where
fields are available from both the primary and secondary catalogues it is
always the field from the secondary which is retained.

\subsubsection{\label{PAIR_SCRIPT}Running from a script}

Often it is more convenient to run \texttt{catpair} from a prepared script
rather than answering the prompts interactively. This end is simply
achieved using Unix's input redirection mechanism. Simply enter the
responses to the various prompts into a text file, in the correct order,
using a text editor. Then type:

\begin{terminalv}
catpair  < script_file
\end{terminalv}

where \textit{script\_file}\, is the name of the file you have created.
Figure~\ref{PAIR_SCRIPT_EXAM} shows an annotated example \texttt{catpair}
script for pairing with Cartesian coordinates.  This script is available
as file:

\begin{terminalv}
/star/share/cursa/catpair_cart.script
\end{terminalv}

An example script showing pairing with spherical-polar coordinates is
available as file:

\begin{terminalv}
/star/share/cursa/catpair_sphplr.script
\end{terminalv}

It may be convenient to use these scripts as starting points for
developing your own scripts.

\begin{figure}[htbp]

\begin{center}
\begin{tabular}{ll}
\texttt{prim}        & primary catalogue \\
\texttt{sec}         & secondary catalogue \\
\texttt{out}         & output catalogue \\
\texttt{C}           & the pairing coordinates are Cartesian \\
\texttt{X}           & column with $x$-coordinate for pairing in the primary \\
\texttt{Y}           & column with $y$-coordinate for pairing in the primary \\
\texttt{X}           & column with $x$-coordinate for pairing in the secondary \\
\texttt{Y}           & column with $y$-coordinate for pairing in the secondary \\
\texttt{10.0}        & the critical distance \\
\texttt{C}           & COMMON pairing \\
\texttt{Y}           & include all the primary multiple matches \\
\texttt{Y}           & include all the secondary multiple matches \\
\texttt{N}           & specify the columns to retain \\
\texttt{X}           & \} \\
\texttt{Y}           & \} columns retained from the primary \\
\texttt{ROW}         & \}  \\
\texttt{END}         & end of list of columns from the primary  \\
\texttt{X > X\_SEC}  & \} columns retained from the secondary \\
\texttt{Y > Y\_SEC}  & \} (note the renaming of these columns) \\
\texttt{ROW > ROW\_SEC}    & \}   \\
\texttt{END}         & end of list of columns from the secondary   \\
\texttt{Y}           & include primary parameters \\
\texttt{N}           & exclude secondary parameters \\
\texttt{N}           & exclude primary textual information \\
\texttt{N}           & exclude secondary textual information \\
\end{tabular}
\end{center}

\begin{quote}
The column on the left (in a \texttt{courier} font) shows the entries in a
\texttt{catpair} script file. The column on the right (in a roman font) briefly
describes the corresponding entry.
\end{quote}

\caption{An annotated example \texttt{catpair} script
\label{PAIR_SCRIPT_EXAM} }

\end{figure}


\subsection{\label{PAIR_CRIT}Pairing criteria}

This section discusses the criteria used to determine whether two
objects, one from each of the two input catalogues, `correspond' or
pair. The two objects pair if the difference in their two-dimensional
coordinates is smaller than some specified critical distance, $D$. The
formul\ae\ differ for Cartesian and celestial coordinates.

\subsubsection{Cartesian coordinates}

If the two objects have Cartesian coordinates $x_{1},y_{1}$ and
$x_{2},y_{2}$ then the criterion is simply that $D$ should be less
than or equal to the Pythagorean distance between the two points:

\begin{equation}
D \leq \sqrt{ (x_{1} - x_{2})^{2} + (y_{1} - y_{2})^{2} }
\end{equation}

\subsubsection{Celestial coordinates}

If the two objects have celestial spherical-polar coordinates (in
practice Right Ascension and Declination) $\alpha_{1},\delta_{1}$ and
$\alpha_{2},\delta_{2}$ then the criterion is that $D$ should be less
than or equal to the great circle distance between the two coordinates:

\begin{equation}
D \leq \arccos ( {\rm abs} ( \sin \delta_{1} \sin \delta_{2} \, + \,
   \cos (\alpha_{1} - \alpha_{2} ) \cos \delta_{1} \cos \delta_{2} ) )
\label{GCD}
\end{equation}

Equation~\ref{GCD} is the natural form for the great circle distance,
simply derived by applying spherical trigonometry to the two
coordinates. In practice it has the disadvantage that because of
numerical errors it is inaccurate when the great circle distance is a
small angle. There are algebraically equivalent formulations which
retain numerical accuracy for small angles. In \texttt{catpair} the great
circle distance is calculated using the appropriate SLA routine\footnote{See
\xref{SUN/67}{sun67}{}\cite{SUN67}. The actual routine used is \texttt{SLA\_DSEP}.}, which uses such a formulation in order to ensure accuracy for
small angles.

\subsubsection{\label{CRIT_DIST}Cases for the critical distance}

The following three cases for the value of the critical distance,
$D$, are supported by \texttt{catpair}.

\begin{enumerate}

  \item It is a constant, for example twenty-three minutes of arc.  Any
   objects in the catalogues correspond if their positions differ by
   twenty-three minutes of arc or less.  Of the various cases this is
   the simplest.

  \item It adopts the value of a column in the primary. Typically such a
   column would be an error associated with the position; objects with a
   small error would only pair with a nearby object, but objects with a
   large error would pair with objects further away.

  \item It adopts a value computed from an expression involving columns
   in the primary. This case is a generalisation of the preceding one.

\end{enumerate}

A fourth case in which the critical distance is computed from an
expression involving columns in \textit{both}\, catalogues is \textit{not}\,
supported in \texttt{catpair}. A special instance of this case which sometimes
arises is where both input catalogues have errors in their coordinates which
vary with the objects in the catalogues and thus are stored as columns, one
in each catalogue. Objects are considered to pair when their error circles
overlap. Here the expression for the critical distance, $D$, would involve
columns (containing the errors) from both catalogues and hence this case is
\textit{not}\, supported.


\subsection{\label{PAIR_PRTYP}Rows in the output catalogue}

\begin{figure}[htbp]

\begin{terminalv}
               Primary           Secondary
          row catalogue          catalogue row
           1   xxxxxxx     +----->XXXXXXX   1
           2   XXXXXXX-----+      xxxxxxx   2
           3   xxxxxxx            xxxxxxx   3
           .   XXXXXXX-----+      xxxxxxx   .
           .   xxxxxxx     +----->XXXXXXX   .
               xxxxxxx            xxxxxxx
               XXXXXXX----------->XXXXXXX
               xxxxxxx     +----->XXXXXXX
               xxxxxxx     |      xxxxxxx
               XXXXXXX-----+      xxxxxxx
               XXXXXXX-----+      xxxxxxx
               xxxxxxx     |      xxxxxxx
               xxxxxxx     |      xxxxxxx
                           +----->XXXXXXX
                                  xxxxxxx
                                  xxxxxxx
                                  xxxxxxx
\end{terminalv}

\caption{Rows in paired catalogues \label{PAIR_JOINED_CAT} }

\end{figure}

Figure~\ref{PAIR_JOINED_CAT} illustrates the result of pairing two catalogues,
with a set of corresponding rows in the catalogues identified. There are a
number of options for the set of rows to be included in an output catalogue
generated from such a pairing. The various alternatives available in
\texttt{catpair} are described below.

\begin{description}

  \item[COMMON] (often called the `inner join' in relational database
   terminology). Only the objects common to both catalogues are retained;
   that is, only the paired objects are retained. This option might be
   used when pairing a list of target stars with a standard catalogue.

  \item[PRIMARY] (often called the `outer join' in relational database
   terminology). All the rows in the primary catalogue are retained.
   For paired objects fields corresponding to the secondary will contain
   actual values, for unpaired objects they will contain null values.
   The corresponding case of retaining all the rows in the secondary
   can be realised by regarding the primary as the secondary and vice
   versa. This option might also be used when pairing a list of target
   stars with a standard catalogue.

  \item[MOSAIC] The output catalogue contains a row for every paired row
   in the input catalogues and also a row for every unpaired row in
   either catalogue. This is option useful for constructing a mosaic of a
   larger area of sky from several slightly overlapping catalogues.

  \item[PRIMREJ] Only the unpaired objects in the primary catalogue are
   retained.    The corresponding case of retaining all the unpaired rows in
   the secondary can be realised by regarding the primary as the secondary
   and vice versa. This option might be used in proper motion studies.

  \item[ALLREJ] The output catalogue contains a row for all the unpaired
   objects in either catalogue. This option might also be used in proper
   motion studies.

\end{description}


\subsection{\label{PAIR_MULTIPLE}Multiple matches}

This section describes how multiple matches are handled by \texttt{catpair}. Multiple matches can arise because the pairing techniques are
matching objects with similar rather than identical positions and
an object in one catalogue can pair with several in the other catalogue.
The terminology used in this section is:

\begin{description}

  \item[match] a match is any object which lies within the critical
   distance, $D$, for an object in the other catalogue,

  \item[pair] a pair is any object chosen from amongst the set of matches
   to correspond to an object in the other catalogue.

\end{description}

That is, any match is potentially a pair and the pairing algorithm must
prescribe which matches are considered pairs. There are three cases for
multiple matches:

\begin{enumerate}

  \item a single object in the primary matches several objects in the
   secondary (see Figure~\ref{PAIR_PRIM_MULT}),

  \item a single object in the secondary is matched by several objects
   in the primary (see Figure~\ref{PAIR_SEC_MULT}),

  \item in crowded catalogues more complicated situations can arise,
   as illustrated in Figure~\ref{PAIR_CROWD}. The results of pairing such
   catalogues are, in general, unpredictable.

\end{enumerate}

\texttt{catpair} is unsuitable for handling the third case, and should \textit{not}\, be used with catalogues where it is likely to be important. There are,
however, several options for handling the first two cases:

\begin{enumerate}

  \item only accept the closest of the matches as the pair,

  \item accept all the matches as pairs,

  \item use further information from the catalogues (such as magnitude
   or colour) to disambiguate a single pair from amongst the matches.

\end{enumerate}

The third option is not practical in a general purpose program such as
\texttt{catpair} because it relies on astronomical knowledge about the
catalogues being paired. Either of the first two options may be
appropriate, depending on the details of the pairing being performed.
\texttt{catpair} provides both options separately for multiple matches in
the primary and secondary, and you should choose the alternatives
appropriate for your work.

%-----------------------------------------------
\begin{figure}[htbp]

\begin{terminalv}
                                        Primary           Secondary
                         o              xxxxxxx    +------>XXXXXXX
            o    +---------+            xxxxxxx    |       xxxxxxx
                 |o     o  |            XXXXXXX----+------>XXXXXXX
                 |    *    |  o         xxxxxxx    |       xxxxxxx
             o   |  o     o|                       +------>XXXXXXX
                 +---------+   o                   |       xxxxxxx
                                                   +------>XXXXXXX
                                                           xxxxxxx
\end{terminalv}

\begin{center}
\begin{tabular}{lll}
{$\ast$} & -  & Object in primary.   \\
\texttt{o}  & -  & Object in secondary. \\
    &  &  \\
\end{tabular}

\begin{quote}
For secondary objects to match with the primary object they must fall
inside the square (strictly speaking the square should be a circle with
a radius equal to the critical distance, $D$).
\end{quote}
\end{center}

\caption{A single primary object matches several secondary objects
\label{PAIR_PRIM_MULT} }

\end{figure}

%-----------------------------------------------
\begin{figure}[htbp]

\begin{terminalv}
                                        Primary           Secondary
                               o        xxxxxxx            xxxxxxx
                 +---------+            XXXXXXX----+       xxxxxxx
             o   | +-------|-+          xxxxxxx    +------>XXXXXXX
                 | |  *  o | |          XXXXXXX----+       xxxxxxx
                 | |    *  | |     o                       xxxxxxx
              o  +---------+ |                             xxxxxxx
                   +---------+ o                           xxxxxxx
                                                           xxxxxxx
\end{terminalv}

\begin{center}
See Figure~\ref{PAIR_PRIM_MULT} for details of the symbols.
\end{center}

\caption{A single secondary object is matched by several primary
objects \label{PAIR_SEC_MULT} }

\end{figure}

%-----------------------------------------------
\begin{figure}[htbp]

\begin{terminalv}
                   o           o
                o+---------+
            o    | +o------|-+  o
                 | |  *  o | |o
              o  | |o   * o| |     o
             o   +---------+ |
                   +--o------+ o
                o          o
\end{terminalv}

\begin{center}
See Figure~\ref{PAIR_PRIM_MULT} for details of the symbols.
\end{center}

\caption{A crowded field with multiple matches of both primary and
secondary objects \label{PAIR_CROWD} }

\end{figure}

An example might help to illustrate the difference between multiple
matches in the primary and secondary. Suppose the primary was a private
list of target objects and the secondary was the NGC catalogue.
Table~\ref{NGC_PAIR} shows the equatorial coordinates for the triplet of
galaxies NGC~3623, NGC~3627 and NGC~3628\footnote{These data were taken
from \textit{NGC 2000.0}\, by R.W.~Sinnott\cite{NGC2000}.}. Consider the
following two cases.

\begin{itemize}

  \item If a target object in the primary had coordinates \hm{11}{19}{5},
   \dm{+13}{20} with an error circle of 30\arcmin then all three galaxies
   would be matches. This case is an example of multiple matches in the
   secondary.

  \item Conversely, if there were two target objects with coordinates of
   \hm{11}{18}{8}, \dm{+13}{01} and \hm{11}{18}{9}, \dm{+13}{07} and both
   with an error circle of 10\arcmin then they would both match NGC~3623 and
   neither would match the other members of the triplet. This case is an
   example of multiple matches in the primary.

\end{itemize}



\begin{table}[htbp]

\begin{center}
\begin{tabular}{r @{\hspace{0.5cm} } rr @{\hspace{0.5cm} }rr}
NGC & \multicolumn{2}{c}{$\alpha$} & \multicolumn{2}{c}{$\delta$} \\ \hline
     & h  & m    & $^{\circ}$ & \arcmin  \\
3623 & 11 & 18.9 & +13        & 05 \\
3627 & 11 & 20.2 & +12        & 59 \\
3628 & 11 & 20.3 & +13        & 36 \\
\end{tabular}
\end{center}

\caption{\label{NGC_PAIR}Coordinates for a triplet of galaxies}

\end{table}

\subsection{Pairing algorithm}

\begin{figure}[htbp]

\begin{terminalv}
               Primary           Secondary
          row catalogue          catalogue row
           1   xxxxxxx            xxxxxxx   1
           2   xxxxxxx            xxxxxxx   2
           3   xxxxxxx    +------>xxxxxxx   3
           .   xxxxxxx    |       xxxxxxx   .
           .   xxxxxxx----|       xxxxxxx   .
               xxxxxxx    |       xxxxxxx
               xxxxxxx    +------>xxxxxxx
               xxxxxxx            xxxxxxx
               xxxxxxx            xxxxxxx
               xxxxxxx            xxxxxxx
                                  xxxxxxx
                                  xxxxxxx
                                  xxxxxxx
                                  xxxxxxx
\end{terminalv}

\caption{The index join \label{PAIR_INDEX_JOIN} }

\end{figure}

This section describes the pairing algorithm used by \texttt{catpair}.
Strictly speaking you should not need to know the details of the
algorithm in order to use \texttt{catpair}, but the information is
provided for reference and completeness. \texttt{catpair} uses an index join
technique which is illustrated in Figure~\ref{PAIR_INDEX_JOIN}. The
secondary catalogue is sorted on the second coordinate to be used in
the pairing.\footnote{Spherical-polar coordinates must be sorted on
Declination or latitude in order to avoid problems with the zero --
twenty-four hour boundary.} The algorithm is then as follows. Every
entry in the primary catalogue is examined sequentially and for each
entry the critical distance, $D$, is used to compute the minimum and
maximum values of the sorted coordinate which could pair with the
primary row. The rows in the secondary catalogue corresponding to these
minimum and maximum values are then identified (remember that the
secondary is sorted on this column) to yield a range of rows which might
pair. All of these rows are then examined individually to check if
they do pair.

The advantages of this technique are that it is relatively straightforward
and it does not require the primary catalogue to be sorted (though the
secondary must). The main disadvantage is that the ranges in the secondary
corresponding to subsequent rows in the primary may overlap, thus leading
to multiple reads of rows in the secondary. The technique is most
appropriate where a small primary is being paired with a large secondary;
perhaps a small personal list of target objects is being paired with a
large standard catalogue. However, it will certainly work if
the primary and secondary are of similar size; it will merely take
somewhat longer to execute than is strictly necessary.


\section{\xlabel{PHOTCAL}\label{PHOTCAL}Photometric calibration}

The purpose of the photometric calibration functions in CURSA is to
convert a list of instrumental magnitudes, typically measured for a
set of objects in a series of CCD frames, into calibrated magnitudes
in some standard photometric system.  To fix ideas, think of a group
of programme objects for which instrumental magnitudes have been
determined from a set of CCD frames using an aperture photometry package
such as PHOTOM (see \xref{SUN/45}{sun45}{}\cite{SUN45}).  These
instrumental magnitudes are to be calibrated into standard $R$
magnitudes in the Johnson-Morgan \textit{UBVRI}\, system.

Astronomical photometry is a diverse subject.  There are many different
standard photometric systems, many ways of making photometric
observations and many ways of reducing them.  CURSA provides only some
simple and basic facilities.  Though they will be useful and give
reasonably accurate results in many circumstances they are certainly
not appropriate in all circumstances.  \textit{In particular, they are
not suitable for high precision photometry.}  Whether they are suitable
for you will depend on the details of your programme.

This section is not an introduction to how to calibrate photometric
observations.  Rather, it describes the principles behind the CURSA
photometric calibration functions so that you can decide whether they
are suitable for your purposes and describes how to use them.  For a
more general introduction to calibrating photometric observations
see \xref{SC/6: \textit{The CCD Photometric Calibration
Cookbook}}{sc6}{}\cite{SC6}.  SC/6 also includes a tutorial example (a
`recipe' in the jargon of cookbooks) of using the CURSA photometric
calibration functions.


\subsection{Description}

The CURSA photometric calibration functions, in common with most photometric
calibration methods, use \textbf{standard stars}.  In essence, as well as
observing instrumental magnitudes for the programme objects that you are
studying you also observe instrumental magnitudes for selected standard
stars.  These standard stars have a known brightness in your target
photometric system.  Numerous catalogues of photometric standard stars
are available (see \xref{SC/6}{sc6}{}\cite{SC6} for a brief discussion).
You then define the transformation between the instrumental and standard
system for the standard stars and apply this transformation to
calibrate the instrumental magnitudes of the programme objects into
the standard system.

In addition the observed brightness of a star varies throughout a
night  because of \textbf{atmospheric extinction} or the dimming of
starlight by the terrestrial atmosphere.  The longer the path length
the starlight traverses through the atmosphere the more that it is dimmed.
Thus, a star close to the horizon will be dimmed more than one close
to the zenith.  The path length through the atmosphere is known as the
\textbf{air mass}.  The air mass can be calculated from the zenith
distance.  In order to calibrate photometry air masses must be available
for both the programme and standard stars.

Thus, a basic set of photometric data consists of:

\begin{itemize}

  \item a set of standard stars with measured instrumental magnitudes,
   known `catalogue' magnitudes (so-called because they are obtained
   from catalogues of photometric standards) in the target photometric
   system and air masses,

  \item a set of programme objects with measured instrumental magnitudes
   and air masses.

\end{itemize}

The standards are invariably stars; the programme objects can be any
sort of astronomical object.  Photometric calibration is a two-stage
process:

\begin{enumerate}

  \item define the transformation between the instrumental and catalogue
   magnitudes for the standard stars, typically by some sort of least
   squares fitting,

  \item apply the transformation to convert the instrumental magnitudes
   into calibrated magnitudes for the programme objects.

\end{enumerate}

In CURSA the relation between instrumental and catalogue magnitudes
is assumed to be of the form:

\begin{equation}
m_{\rm catalogue} = m_{\rm inst} - A + Z + \kappa X
\end{equation}

where:

\begin{description}

  \item[$m_{\rm catalogue}$] is the calibrated magnitude,

  \item[$m_{\rm inst}$] is the instrumental magnitude,

  \item[$A$] is an arbitrary constant which is often added to the
   instrumental constants,

  \item[$Z$] is the photometric zero point between the standard and
   instrumental systems,

  \item[$\kappa$] is the atmospheric extinction correction,

  \item[$X$] is the air mass.

\end{description}

See \xref{SC/6}{sc6}{} for further discussion of the arbitrary constant
$A$.  This equation is a particularly simple form for the relation
between instrumental and catalogue magnitudes.  In particular, it omits
any `colour corrections' caused by the instrumental and standard
systems being sensitive to different wavelengths.  \textit{Thus, the
CURSA photometric calibration functions should only be used when the
instrumental photometric system is well-matched to the target
photometric system.}  Though this may seem a serious limitation,
in practice with modern instrumentation the instrumental system is often
a good match to the standard system.  For the same reason the CURSA
applications are not suitable for very high precision work, where even
small discrepancies between the instrumental and standard systems
must be allowed for.

The basic reason why colour corrections are ignored is because by doing
so the functions are much more general.  They do not impose constraints
on the photometric system that you are using (other than that the
instrumental and standard systems should be well-matched) and they do
not require you to make observations in any given colours.

Fitting the instrumental and standard magnitudes for the standard stars
is usually an `iterative', interactive process.  Typically, you will
start by fitting all the standard stars, examine the residuals,
reject the stars with large residuals, fit the remaining stars and
continue until you have a satisfactory solution.  (Aberrant results
for individual stars can be caused by various effects, including
passing clouds.)

For completeness, the subroutine used by the CURSA photometric
calibration applications to fit the instrumental and catalogue magnitudes
for the standard stars is \texttt{PDA\_DBOLS}.  This subroutine is
described in \xref{SUN/194}{sun194}{}\cite{SUN194}.

\subsection{Assembling the input catalogues}

You need to prepare two catalogues: one containing the observations of
the standard stars, the other the observations of the programme objects.
Neither catalogue is likely to contain more than, at most, a few score
entries.  The most convenient way to create these catalogues is to
use the STL format (see Appendices~\ref{STLTUT} and \ref{STLREF}) and
type them in using an editor.  Note that separate sets of catalogues
should usually be prepared for each night that observations were made;
observations from different nights should not normally be combined
prior to calibration.

The instrumental magnitudes will be assembled from the output of other
programs, such as PHOTOM.  The standard or catalogue magnitudes will
ultimately come from the catalogues of standards which you used when
selecting the standard stars to observe.  The air mass (or zenith
distance) will often be included in either your observing logs or the
header information of your CCD frames.  If the air mass is not available
then the CURSA applications can automatically calculate it from the
zenith distance.  Note that it is the \textit{observed}\, zenith distance,
that is as affected by atmospheric refraction, which is required.
If the zenith distance is not available either then you will have to
calculate it from whatever information you have about the celestial
coordinates and times of your observations.  Most standard textbooks on
spherical astronomy give the requisite formul\ae\ (see, for example,
\textit{Spherical Astronomy}\, by R.M.~Green\cite{GREEN}).

The catalogues of standard stars and programme objects are discussed
separately below.

\subsubsection{Standard star catalogue}

Figure~\ref{PHOTOSTDCAT} shows an example catalogue of standard stars.
The observations used in this example were kindly provided by
John~Lucey.  The example is available as file:

\begin{terminalv}
/star/share/cursa/photostandards.TXT
\end{terminalv}

The catalogue must contain columns containing the instrumental magnitude,
the catalogue magnitude and the air mass (or alternatively the observed
zenith distance).  It may optionally contain a column containing a name
for each of the standard stars and a column of `include in the fit'
flags.  All five columns are included in the example.  If supplied, the
star name is listed in the table of residuals produced when the fit is
made.  Often being able to identify each standard star will be useful
to you.  The `include in the fit' flag column is of data type \texttt{LOGICAL} and determines whether each star is included in the fit or
not.  To include or exclude a given star in the fit you simply edit
the STL format catalogue and toggle the value of the flag for the
star to `\texttt{T}' (or `\texttt{TRUE}') or `\texttt{F'} (or `\texttt{FALSE}') to
include or exclude it as appropriate.  This procedure is much less
troublesome and error-prone than deleting and reinserting stars from
the catalogue.  Initially set the flags for all the stars to `\texttt{T}'
(or `\texttt{TRUE}') so that they are all included in the fit.  In the
example all the stars are included in the fit except 99Z367 (the
penultimate one in the list).  This star is excluded as an illustration.
When preparing your own catalogues you will usually initially include all
the stars.

\begin{figure}[htbp]

\begin{terminalv}
!+
! Example catalogue of photometric standards.
!
! These data were observed with the Jacobus Kapteyn Telescope
! (JKT) on La Palma on 16/11/1993.  The catalogue magnitudes are in
! the R band and the instrumental magnitudes approximate to this
! system.  The data are provided courtesy of John Lucey (Durham).
!
! A C Davenhall (Edinburgh) 12/10/97.
!-

C NAME    CHAR*7  1  EXFMT=A7    ! Star name.
C MCAT    DOUBLE  2  EXFMT=F7.3  ! Catalogue magnitude.
C MINST   DOUBLE  3  EXFMT=F7.3  ! Instrumental magnitude.
C AIRMASS DOUBLE  4  EXFMT=F7.3  ! Air mass.
C INCL    LOGICAL 5  EXFMT=L5    ! 'Include in the fit' flags.

BEGINTABLE
113Z475  09.737  16.37  1.16  T
110Z450  11.033  17.74  2.20  T
114Z531  11.672  18.29  1.13  T
113Z475  09.737  16.39  1.41  T
114Z548  10.868  17.50  1.23  T
 94Z251  10.547  17.17  1.14  T
 93Z424  11.067  17.69  1.18  T
 95Z74   10.931  17.55  1.17  T
 96Z737  10.982  17.62  1.26  T
 97Z249  11.369  17.99  1.14  T
 94Z251  10.547  17.21  1.57  T
 95Z301  10.527  17.16  1.32  T
 99Z367  10.618  17.23  1.15  F
 96Z737  10.982  17.67  1.81  T
\end{terminalv}

\caption{Example of a catalogue of photometric standard stars
\label{PHOTOSTDCAT} }

\end{figure}

The zenith distance is an angle and if it is used it must ultimately
be presented to the CURSA applications in radians.  If you wish you
can simply type the values into the STL catalogue in radians.
Alternatively, if it is more convenient, you can define the zenith
distance column as containing a sexagesimal angle, usually in degrees,
and type in the values as sexagesimal degrees.  The example catalogue
of programme objects in Figure~\ref{PHOTOPRGCAT} includes a column of
zenith distances in this form.

Though both the columns of star names and `include in the fit' flags
are optional I recommend that you use them.

The columns do not have to have the names shown in the example.
However, if you use these names you will be able to accept the defaults
from the prompts in the CURSA applications.

Obviously the catalogue can contain additional columns, though these
are not used.  For example, if you are calibrating multi-colour photometry
you could prepare a single catalogue containing the instrumental and
catalogue magnitudes in all the colours observed.  Obviously the
columns for magnitudes in different colours would have to have different
names.  If you did not observe all the stars in all the colours simply
use the STL mechanism for indicating null values (see
Section~\ref{STLNULL}) to represent the missing measurements.

\subsubsection{Programme object catalogue}

Figure~\ref{PHOTOPRGCAT} shows an example catalogue of programme
objects.  This example is available as file:

\begin{terminalv}
/star/share/cursa/photoprog.TXT
\end{terminalv}

As an illustration this catalogue contains columns of both the air mass
and the observed zenith distance.  It does not need to contain both,
but must contain one or the other.  Here the zenith distance has been
entered as sexagesimal degrees and minutes.

\begin{figure}[htbp]

\begin{terminalv}
!+
! Example catalogue of photometric programme objects.
!
! Note that this table contains both the air mass and the observed
! zenith distance.  The zenith distance is given in sexagesimal
! degrees and minutes.
!
! A C Davenhall (Edinburgh) 12/10/97.
!-

C MINST   DOUBLE  1 EXFMT=F7.3    ! Instrumental magnitude.
C AIRMASS DOUBLE  2 EXFMT=F7.3    ! Air mass.
C ZENDIST DOUBLE  3 UNITS='RADIANS{DM}' TBLFMT=DEGREES ! Zenith distance.

BEGINTABLE
 17.38  1.00   1:43
 17.03  1.24  36:06
 17.49  1.11  25:47
 17.87  1.04  15:28
 17.42  1.05  18:20
 17.26  1.91  58:27
\end{terminalv}

\caption{Example of a catalogue of photometric programme objects
\label{PHOTOPRGCAT} }

\end{figure}

The columns do not have to have the names shown in the example.
However, if you use these names you will be able to accept the defaults
from the prompts in the CURSA applications.

The catalogue can contain additional columns; indeed a programme catalogue
will often contain celestial coordinates and/or object names.  Also, if
you are calibrating multi-colour photometry you could prepare a single
catalogue containing the instrumental magnitudes in all the colours
observed.  Obviously the columns for magnitudes in different colours
would have to have different names.  If you did not observe all the
objects in all the colours simply use the STL mechanism for indicating
null values (see Section~\ref{STLNULL}) to represent the missing
measurements.

\subsection{Applications for photometric calibration}

CURSA contains three applications for photometric calibration:

\begin{description}

  \item[\texttt{catphotomfit}] define the transformation coefficients from
   the standard stars,

  \item[\texttt{catphotomtrn}] apply the transformation coefficients to
   determine the calibrated magnitudes for the programme objects,

  \item[\texttt{catphotomlst}] list the contents of a transformation
   coefficients file.

\end{description}

The usual sequence of using these applications is:

\begin{enumerate}

  \item run \texttt{catphotomfit} to determine the transformation
   coefficients.  Examine the residuals, exclude aberrant standard
   stars and re-run.  Repeat this process until you get a satisfactory
   fit,

  \item run \texttt{catphotomtrn} to apply the transformation coefficients
   to the programme objects and determine calibrated magnitudes for
   them.

\end{enumerate}

The transformation coefficients are passed from \texttt{catphotomfit} to
\texttt{catphotomtrn} via a file, the so-called `transformation
coefficients file'.  Normally you do not need to inspect this file.
However, if you wish to do so then \texttt{catphotomlst} is available
for this purpose.

The details of running the individual applications are described below.

\subsection{Running catphotomfit}

To perform a simple fit to a set of standard stars type:

\begin{terminalv}
catphotomfit
\end{terminalv}

Your catalogue of standard stars should contain an air mass for each
star.  \texttt{catphotomfit} will determine the transformation coefficients,
display them together with the residuals and write the coefficients
to a file.  If your catalogue contains a column of observed zenith
distances rather than air masses then type:

\begin{terminalv}
catphotomfit  zenithdist=true
\end{terminalv}

See Section~\ref{AIRMASS} for details of how the air mass is calculated
from the zenith distance.  If some of the transformation coefficients
are fixed (that is, you know them beforehand) type:

\begin{terminalv}
catphotomfit  fixed=true
\end{terminalv}

You will be prompted for details of which coefficients are fixed and
their values.  If all the coefficients are fixed then obviously no fit
is made.  However, the residuals are still computed and listed and a
file of transformation coefficients is written.  To suppress the listing
of residuals type:

\begin{terminalv}
catphotomfit  resid=false
\end{terminalv}

These options can be combined.  Thus, to read a catalogue containing
zenith distances rather than air masses and fix some of the
transformation coefficients type:

\begin{terminalv}
catphotomfit  zenithdist=true  fixed=true
\end{terminalv}

You then answer a series of prompts.  All the possible prompts are
listed below, identified by the corresponding ADAM parameter name.  All
the prompts will not appear in a given run.  For example, none of
the prompts \texttt{FZEROP}, \texttt{ZEROP}, \texttt{FATMOS} or \texttt{ATMOS}
appear if none of the transformation coefficients are fixed.

\begin{description}

  \item[ \texttt{FZEROP} ] Specify whether the zero point is fixed.  The
   possible replies are:

  \begin{description}

    \item[ \texttt{TRUE} ] the zero point is fixed,

    \item[ \texttt{FALSE} ] the zero point is not fixed.

  \end{description}

  \item[ \texttt{ZEROP} ] Enter the value of the fixed zero point.

  \item[ \texttt{FATMOS} ] Specify whether the atmospheric extinction is
   fixed.  The possible replies are:

  \begin{description}

    \item[ \texttt{TRUE} ] the atmospheric extinction is fixed,

    \item[ \texttt{FALSE} ] the atmospheric extinction is not fixed.

  \end{description}

  \item[ \texttt{ATMOS} ] Enter the value of the fixed atmospheric
   extinction.

  \item[ \texttt{INSCON} ] Enter the arbitrary constant previously added
   to the instrumental magnitudes.

  \item[ \texttt{CATALOGUE} ] Enter the name of the catalogue containing
   the standard and catalogue magnitudes.

  \item[ \texttt{NAME} ] Enter the name of the column containing names of
   the standard stars.  The special value `\texttt{NONE}' indicates that a
   column of star names is not required.

  \item[ \texttt{INCLUDE} ] Enter the name of the column of `include in the
   fit' flags for the standard stars.  The special value `\texttt{ALL}'
   indicates that all the stars are to be included in the fit.

  \item[ \texttt{CATMAG} ] Enter the name of the column or expression holding
   the standard or catalogue magnitudes.

  \item[ \texttt{INSMAG} ] Enter the name of the column or expression
   holding the instrumental magnitudes.

  \item[ \texttt{AIRMASS} ] Enter the name of the column or expression
   holding the air masses.

  \item[ \texttt{ZENDST} ] Enter the name of the column or expression
   holding the observed zenith distances.

  \item[ \texttt{FILNME} ] Enter the name of the file which is to contain
   the transformation coefficients.

\end{description}

\begin{figure}[htbp]

\begin{terminalv}

Coefficients determined successfully from fitting 13 stars:

 zero point = 23.474252
 atmospheric extinction = 0.085569

 (minimum residual vector length = 0.018932)

Seq.  Star         Fit Air      Cat.       Instrumental Mag.
no.                    mass     Mag.    calc.  observe residual
  1  113Z475        Y  1.16    9.737    9.745  16.370  -0.008 :**********
  2  110Z450        Y  2.20   11.033   11.026  17.740   0.007 :********
  3  114Z531        Y  1.13   11.672   11.668  18.290   0.004 :*****
  4  113Z475        Y  1.41    9.737    9.744  16.390  -0.007 :********
  5  114Z548        Y  1.23   10.868   10.869  17.500  -0.001 :*
  6  94Z251         Y  1.14   10.547   10.547  17.170   0.000 :
  7  93Z424         Y  1.18   11.067   11.063  17.690   0.004 :****
  8  95Z74          Y  1.17   10.931   10.924  17.550   0.007 :********
  9  96Z737         Y  1.26   10.982   10.986  17.620  -0.004 :*****
 10  97Z249         Y  1.14   11.369   11.367  17.990   0.002 :**
 11  94Z251         Y  1.57   10.547   10.550  17.210  -0.003 :***
 12  95Z301         Y  1.32   10.527   10.521  17.160   0.006 :*******
 13  99Z367            1.15   10.618   10.606  17.230   0.012 :---------->
 14  96Z737         Y  1.81   10.982   10.989  17.670  -0.007 :*********

Standard deviation of the residuals:
   Fitted stars:  0.005       (13 points).
   All stars:     0.006       (14 points).
\end{terminalv}

\caption{Example output from \texttt{catphotomfit} \label{PHOTOFITOUT} }

\end{figure}

Figure~\ref{PHOTOFITOUT} shows the output displayed by \texttt{catphotomfit}.  The transformation coefficients are self-explanatory.
The minimum residual vector length is a measure of the goodness of the
fit.  The table of residuals is also mostly self-explanatory.  The column
of star names will be absent if parameter \texttt{NAME} was specified as
`\texttt{NONE}'.  A `\texttt{Y}' in the `Fit' column indicates that the star
was included in the fit.  The residuals are defined in the sense:

\begin{equation}
m_{\rm catalogue} - m_{\rm calculated}
\end{equation}

The calculated magnitudes and residuals are shown to three places of
decimals.  This format does not imply that the results are this
accurate; the actual accuracy will depend on the data used.  It is
noteworthy, however, that in the example data the largest residual is
only slightly larger than 0.01 magnitude, despite the method ignoring
colour corrections.

The bar to the right of the residuals is a simple graphic
representation of the absolute size of the residual; the length of the
bar is scaled according to the absolute size of the residual for the
star.  The scaling is such that the largest absolute residual amongst
the stars included in the fit is ten asterisks long.  Stars which are
included in the fit are shown as a row of asterisks (`\texttt{*}').  Stars
which are excluded from the fit are shown as a row of dashes (`\texttt{-}').  Because excluded stars will often have larger residuals than the
included stars, for excluded stars with residuals larger than the
largest included residual a right chevron (`\verb->-') is shown beyond
the last dash (thus forming an arrow).

For completeness, and to avoid any possible ambiguity, the formula used
to compute the standard deviation, $s$, is:

\begin{equation}
s = \sqrt{ \frac{1}{(n - 1)}
    \sum_{i=1}^{n} (m(i)_{\rm catalogue} - m(i)_{\rm calculated})^2 }
\end{equation}

where $n$\, is either the number of stars included in the fit or the
total number of stars, as appropriate.

\subsection{Running catphotomtrn}

To convert a catalogue of instrumental magnitudes into calibrated
magnitudes for programme objects type:

\begin{terminalv}
catphotomtrn
\end{terminalv}

A new catalogue will be written which contains the new calibrated
magnitudes as well as all the columns in the original catalogues.  Also
the transformation coefficients are added as parameters to the output
catalogue.  If your original catalogue contains a column of zenith
distances rather than air masses then type:

\begin{terminalv}
catphotomtrn  zenithdist=true
\end{terminalv}

See Section~\ref{AIRMASS} for details of how the air mass is calculated
from the zenith distance.  The amount of textual information written to
the output catalogue is controlled using the command line mechanism
described in Section~\ref{COPYTEXT}.

You then answer a series of prompts.  All the possible prompts are
listed below, identified by the corresponding ADAM parameter name.  In
a given run either \texttt{AIRMASS} or \texttt{ZENDST} will appear, but not
both.

\begin{description}

  \item[ \texttt{FILNME} ] Enter the name of the file containing the
   transformation coefficients.

  \item[ \texttt{INSCON} ] Enter the arbitrary constant previously added
   to the instrumental magnitudes.  The default will be the value
   read from the transformation coefficients file, which corresponds
   to the value added to the instrumental magnitudes for the standard
   stars.  Usually it is good practice to add the same arbitrary
   value to the instrumental magnitudes for both the standard stars
   and programme objects.

  \item[ \texttt{CATIN} ] Enter the name of the input catalogue.

  \item[ \texttt{CATOUT} ] Enter the name of the output catalogue to
   contain the calibrated magnitudes.

  \item[ \texttt{INSMAG} ] Enter the name of the column or expression
   in the input catalogue holding the instrumental magnitudes.

  \item[ \texttt{AIRMASS} ] Enter the name of the column or expression
   in the input catalogue holding the air masses.

  \item[ \texttt{ZENDST} ] Enter the name of the column or expression
   in the input catalogue holding the observed zenith distances.

  \item[ \texttt{CALMAG} ] Enter the name of the column in the output
   catalogue to hold the calibrated magnitudes.

\end{description}

\subsection{Running catphotomlst}

To display the contents of a transformation coefficients file type:

\begin{terminalv}
catphotomlst
\end{terminalv}

By default the transformation coefficients are shown to six places of
decimals.  Usually this precision will be more than adequate given
the accuracy of the photometry and the fitting technique.  However,
you can specify the number of decimal places used.  For example, type:

\begin{terminalv}
catphotomlst  decpl=8
\end{terminalv}

to show the coefficients to eight places of decimals.

\subsection{\label{AIRMASS}Calculating the air mass}

\texttt{catphotomfit} and \texttt{catphotomtrn} can optionally calculate
the air mass from the observed zenith distance.  They use subroutine
\texttt{SLA\_AIRMAS} in the SLA subroutine library (see
\xref{SUN/67}{sun67}{}\cite{SUN67}) for this task.  This routine is
more than sufficiently accurate for the present purposes.  The
following notes are based on the documentation for \texttt{SLA\_AIRMAS}
in SUN/67.

The air mass is calculated using Hardie's\cite{HARDIE62} polynomial fit
to Bemporad's data for the relative air mass, $X$, in units of thickness
at the zenith as tabulated by Schoenberg\cite{SCHOEN29}.  This method
is adequate for all normal needs as it is accurate to better than 0.1\%
up to $X = 6.8$ and better than 1\% up to $X = 10$.  Bemporad's tabulated
values are unlikely to be trustworthy to such accuracy because of
variations in density, pressure and other conditions in the atmosphere
from those that he assumed.  At zenith distances greater than about
$87^{\circ}$ the air mass is held constant to avoid arithmetic overflows.


\section{\xlabel{GRIDS}\label{GRIDS}Binning columns in a catalogue into a
grid}

CURSA includes application \texttt{catgrid} for binning columns in a
catalogue into a grid.  One, two or three columns in a catalogue may
be binned into, respectively, a histogram, two-dimensional `image' or
data cube.  The grid generated might be useful as an aid to visualising
the data.  It is saved as a file which can be displayed and manipulated
with other Starlink software.

You specify the dimensionality required for the grid (one to three)
and the names of the columns corresponding to each axis.  A
regularly-spaced grid is constructed spanning the entire range of the
values occurring in the specified columns.  The value of each element
of the grid is set to the number of points which lie within it.
Optionally the grid may be normalised by dividing by the total number of
points in the catalogue.

The output file is written in the standard Starlink NDF data format
(see \xref{SUN/33}{sun33}{}\cite{SUN33}).  It can be displayed and
manipulated using packages such as
GAIA (see \xref{SUN/214}{sun214}{}\cite{SUN214}),
KAPPA (see \xref{SUN/95}{sun95}{}\cite{SUN95}) and
Figaro (see \xref{SUN/86}{sun86}{}\cite{SUN86}).  Alternatively, the file
can be imported into a visualisation package such DX (see
\xref{SUN/203}{sun203}{}\cite{SUN203} and \xref{SC/2}{sc2}{}\cite{SC2}).
The latter option is likely to be particularly appropriate for data
cubes.  The CONVERT package (see \xref{SUN/55}{sun55}{}\cite{SUN55})
is available for converting an NDF format file into a number of other
common formats, including FITS images and simple ASCII text files.

\subsection{Running catgrid}

To generate a grid from columns in a catalogue type:

\begin{terminalv}
catgrid
\end{terminalv}

By default \texttt{catgrid} generates an un-normalised grid.  To generate
a normalised grid type:

\begin{terminalv}
catgrid  normal=true
\end{terminalv}

There must be one or more spaces between `\texttt{catgrid}' and `\texttt{normal=true}'.  You then answer the prompts described below.  In the
descriptions the prompts are identified by the corresponding ADAM
parameter name, which appears at the start of the prompt line.

\begin{description}

  \item[ \texttt{CATIN} ] Enter the name of the input catalogue.

  \item[ \texttt{NDIM} ] Enter the dimensionality of the output grid.  The
   permitted values are one to three.

  \item[ \texttt{COLX} ] Enter the name of the column to be used for the
   first ($x$\/) axis of the grid.

  \item[ \texttt{XBINS} ] Enter the number of bins required along the first
   ($x$\/) axis of the grid.

\end{description}

If you specified a dimensionality of two or three then prompts
corresponding to \texttt{COLX} and \texttt{XBINS} for the second ($y$\/)
and third ($z$\/) axis are issued, as appropriate.

\begin{description}

  \item[ \texttt{GRID} ] Enter the name of the file to contain the output
   grid.  Note that NDF files have file type `\texttt{.sdf}', but the
   name should be entered without the file type.

\end{description}


\section{\xlabel{CDSIN}\label{CDSIN}Importing CDS catalogues}

A large collection of astronomical catalogues are available on-line
at the Centre de Donn\'{e}es astronomiques de Strasbourg (CDS)
and can be retrieved by anonymous ftp (see Section~\ref{OBTAIN}).
Most of these catalogues are available in two formats: FITS tables
and simple ASCII text files.  The recommended route to access these
catalogues with CURSA is to retrieve the simple text file version.
It is then usually possible to automatically construct an STL description
of the file which correctly interprets the celestial coordinates in
the catalogue.  (If you retrieve the FITS table version the individual
sexagesimal components of the celestial coordinates will be treated as
separate columns, making the coordinates difficult to process.)

Each CDS catalogue usually comprises (at least\footnote{There may
be additional auxiliary files which can be ignored for the purposes
of the present discussion.}) two files: the data file containing the
columns and rows of the catalogue and a `CDS description file'
detailing the contents of the catalogue.  By convention this CDS
description file has the file name `\texttt{ReadMe}' (and consequently
is known as the `\texttt{ReadMe} file').  The description of the catalogue
which it contains is in a standardised form.  CURSA application \texttt{catcdsin} will read a CDS \texttt{ReadMe} file and construct an equivalent
CURSA STL description file from it.  This STL description file will
usually contain a description of the celestial coordinates in the
catalogues which is fully compatible with CURSA.

\texttt{catcdsin} does not copy the CDS catalogue.  It merely constructs
an STL description file from the CDS description file.  Both these
files describe (using a different syntax, of course) the same catalogue
text file.  Because \texttt{catcdsin} does not have to copy the catalogue
it executes quickly, irrespective of the size of the catalogue.

\subsection{Running catcdsin}

Unlike most other CURSA applications \texttt{catcdsin} is not an ADAM
A-task (it is, in fact, a Perl script).  Consequently, it handles
parameters slightly differently to other applications.  However,
it never prompts for any parameters so the differences will not usually
be important to you.

Suppose that you had the text version of a CDS catalogue and its
corresponding \texttt{ReadMe} file in your current directory.  You would
simply type:

\begin{terminalv}
catcdsin
\end{terminalv}

\texttt{catcdsin} generates the corresponding STL description, displays
the name of the STL description file it has created and terminates.
There are various options which can be specified.  By default \texttt{catcdsin} copies the \texttt{ReadMe} file to the description file as
textual comments.  This behaviour can be suppressed by typing:

\begin{terminalv}
catcdsin  text=none
\end{terminalv}

(This option is analogous to the usual mechanism for controlling the
amount of textual information copied, which is described in
Section~\ref{COPYTEXT}.)  The equinox and epoch of the celestial
coordinates cannot be reliably determined automatically from the
\texttt{ReadMe} file.  You will need to read the \texttt{ReadMe} file
yourself and decide what they are.  They can then be specified by
typing, for example:

\begin{terminalv}
catcdsin  equinox=J2000  epoch=J1995.3
\end{terminalv}

Obviously, you substitute values appropriate to your catalogue.  The
equinox and epoch should have their usual CURSA syntax (see
Section~\ref{TARGLIST}).  Either, neither or both can be specified.
If you wish to suppress the automatic interpretation of celestial
coordinates, and instead have the sexagesimal subdivisions of angles
treated as separate columns, type:

\begin{terminalv}
catcdsin  angles=no
\end{terminalv}

The input CDS description file does not have to be called \texttt{ReadMe}.
For example, a file called \texttt{cdsdesc.lis} could be processed by
typing:

\begin{terminalv}
catcdsin infile=cdsdesc.lis
\end{terminalv}

These various options can be combined.  For example, to process a
file called \texttt{cdsdesc.lis}, specifying the equinox as J2000 and
not copying the CDS file as textual information type:

\begin{terminalv}
catcdsin  infile=cdsdesc.lis  equinox=J2000  text=none
\end{terminalv}

CDS \texttt{ReadMe} files can (and often do) contain descriptions of more
than one catalogue or table.  Usually these catalogues or tables will
be closely related; perhaps a main catalogue and a table of notes.
\texttt{catcdsin} creates a separate STL description file for every
catalogue found in the \texttt{ReadMe} file.

A few CDS catalogues do not contain celestial coordinates; spectral
line wavelength lists are the obvious example.  Occasionally the
coordinates may be in a non-standard format which \texttt{catcdsin} does
not interpret properly.  In this case it may be possible to fix-up the
STL description file generated by \texttt{catcdsin} by hand.  See
Appendices~\ref{STLTUT} and \ref{STLREF} for details of the STL format.
Such occurrences seem to be rare.


\section{\xlabel{GSCIN}\label{GSCIN}Importing regions of the HST GSC}

The Hubble Space Telescope \textit{Guide Star Catalog}\, (GSC, see
Section~\ref{OBTAIN}, above) is divided into some 9537 regions and
each region is held as a separate FITS table file. These FITS tables can
be read directly by CURSA. However, there is a CURSA utility to convert
them to a more convenient format. I recommend that you use it to
convert a region before accessing the region with other CURSA applications.
The utility generates a new version of the region with the following
changes:

\begin{itemize}

  \item the units of the Right Ascension and Declination are changed
   from degrees to radians and the UNITS attributes set so that CURSA
   can format the coordinates as sexagesimal values for display,

  \item the region is sorted into ascending Declination order so that
   (fast) range selections can be performed on it.

\end{itemize}

In order to convert a GSC region simply type:

\begin{terminalv}
catgscin
\end{terminalv}

The amount of textual
information written to the output catalogue is controlled using the
command line mechanism described in Section~\ref{COPYTEXT}.
You then answer the following prompt.

\begin{description}

  \item[ \texttt{CATIN} ] Enter the name of the input GSC region.

\end{description}

Here `\texttt{CATIN}' is the name of the corresponding ADAM parameter name,
which appears at the start of the prompt line.  The original version of
the region is not overwritten, but rather a new catalogue containing the
modified version is created.

GSC regions have names of the form \textit{region\_number}\texttt{.gsc} (where
\textit{region\_number}\, is an integer number).  The converted region is
written to a file called \texttt{gsc}\textit{region\_number}\texttt{.FIT}. Thus,
for example, the converted version of region \texttt{5828.gsc} would appear
as file \texttt{gsc5828.FIT}.


\section{\xlabel{REMACCSS}\label{REMACCSS}Accessing remote catalogues}

CURSA provides some limited facilities for accessing remote catalogues
held on-line at various astronomical data centres and archives around
the world.  You can select a subset from one of these catalogues and
save it as a catalogue on your local computer using either \texttt{catremote} or the catalogue browser \texttt{xcatview}\footnote{Technically
\texttt{xcatview} is a `front-end' graphical user interface which invokes
\texttt{catremote} to access the remote catalogue, though as a user you
will not normally be concerned with these details.  Note, however,
that the command line catalogue browser, \texttt{catview}, cannot access
remote catalogues.} (see Section~\ref{XVIEW}).  \texttt{catremote} and
\texttt{xcatview} provide the same functionality for accessing remote
catalogues, though \texttt{xcatview} is slightly more convenient to use.
Currently the only sort of selection which is permitted on remote
catalogues is to select all the objects in the catalogue which lie
within a given angular distance from a given point on the celestial
sphere (thus, the selection corresponds to the `circular area' or
`cone search' option of \texttt{catselect}, see Section~\ref{SELECT}).
The remote catalogues are accessed via the Internet and obviously the
option will only work when CURSA is being run on a computer with suitable
network connections.  If you are using CURSA at a normal Starlink node
then remote access will usually be available.  Conversely (and obviously)
if you are using it on a stand-alone Linux PC without network connections
then remote access will not be possible.

The remainder of this section refers to \texttt{catremote}.  However, all
the material, apart from the specific instructions for running
\texttt{catremote}, is equally applicable to accessing remote catalogues
using \texttt{xcatview}.  Section~\ref{REMRUN} describes how to run
\texttt{catremote} and Section~\ref{REMCATS} how to configure it to specify
the list of remote catalogues which are accessible.  Strictly speaking this
information is all that you need to know in order to use \texttt{catremote}.
However, it is useful if you understand various peculiarities and
shortcomings which are contingent on the way that the remote access
operates; subsequent sections provide the details.

\subsection{Running catremote \label{REMRUN}}

Unlike most other CURSA applications \texttt{catremote} is not an ADAM
A-task (it is, in fact, a Perl script).  Usually this difference will
not be important to you, but it does mean that \texttt{catremote} behaves
slightly differently from the other applications when it is prompting for
input values.  In particular, it has no default replies for prompts and
the special replies described in Section~\ref{PROMPTS} are not available.
Nor does it support either the normal options for copying textual
information (see Section~\ref{COPYTEXT}) or the quiet mode (see
Section~\ref{QUIET}).

\begin{table}[htbp]

\begin{center}
\begin{tabular}{rl}
Mode          &  Description \\ \hline
\texttt{list}    & list the catalogues currently available \\
\texttt{details} & show details of a named catalogue \\
\texttt{query}   & submit a query to a remote catalogue and retrieve the results \\
\texttt{name}    & resolve an object name into coordinates \\
\texttt{help}    & list the modes available \\
\end{tabular}
\end{center}

\caption{The modes of \texttt{catremote}
\label{MODES} }

\end{table}

The basic purpose of \texttt{catremote} is to query remote astronomical
catalogues and archives.  In addition to this basic query function it has
a number of auxiliary functions, and each function corresponds to a mode
of the program.  For completeness all the various modes are listed in
Table~\ref{MODES}, though the modes that you are likely to use are:

\begin{description}

  \item[\texttt{list}] list all the catalogues which are currently available.

  \item[\texttt{query}] submit a query to a named catalogue and retrieve the
   results.  The basic type of query supported is the `cone search' or
   `circular area search' which returns all the objects found in a given
   circular area of sky.  This area is specified by its central Right
   Ascension and Declination and angular radius.  The objects returned are
   formatted as a catalogue and written to a local file.

  \item[\texttt{name}] submit a name of an astronomical object to a remote
   name-resolver.  If the name-resolver finds this name in its database
   then the Right Ascension and Declination of the object are returned and
   displayed.

\end{description}

These modes are described briefly below.  Though incomplete, this
description should be enough to allow you to use \texttt{catremote}.  There
is a comprehensive description in \xref{SSN/76}{ssn76}{}\cite{SSN76}.

All of \texttt{catremote}'s arguments can be specified on the command line,
where they are identified by position.  The first command-line argument of
\texttt{catremote} is always the mode of operation, and is one of the values
listed in Table~\ref{MODES}.  The mode can only be specified on the
command line.  If it is omitted then `\texttt{help}' mode is assumed and the
various modes are listed.  The subsequent arguments required depend on the
mode chosen and are summarised in Table~\ref{ARGS}.  These arguments may
optionally be omitted, starting at the right.  Omitted arguments (other
than the mode) will always be prompted for.

\begin{table}[htbp]

\begin{center}
\begin{tabular}{ll}
\texttt{catremote list}    & \\
\texttt{catremote query}   & \textit{cat-name $\alpha$ $\delta$ radius} \\
\texttt{catremote name}    & \textit{name-resolver object-name} \\
\end{tabular}
\end{center}

\caption{Arguments for the various modes of \texttt{catremote}
\label{ARGS} }

\end{table}

\subsubsection{Listing the accessible catalogues  \label{REMLISTS} }

To obtain a list of all the remote catalogues which are currently
accessible simply type:

\begin{terminalv}
catremote  list
\end{terminalv}

A list of all the catalogues which are accessible will be displayed.  This
list will look something like the extract shown in Table~\ref{REMLIST}.
Each line of the list refers to a different catalogue.  The first item
on each line is the name of the catalogue.  The second item is the type
of the catalogue; the usual value is `\texttt{catalog}', which corresponds
to a simple catalogue, though other alternatives are possible.
The remainder of the line is a brief description of the catalogue.  Thus,
`\texttt{ppm@eso}' is the name of the PPM\footnote{This catalogue is a version
of the \textit{Catalogue of Positions and Proper Motions}\,
(PPM)\cite{PPMN,PPMS}.}
% compiled by S.~R\"{o}ser and U.~Bastian\cite{PPMN},\cite{PPMS}.}
catalogue at ESO.  You will give this name  when you specify the catalogue
to be queried.  The catalogue descriptions are usually quite
brief and often contain acronyms.

\begin{table}[htbp]

\begin{center}
\begin{tabular}{l}
\texttt{~~~$\vdots$ } \\
\texttt{usno@eso  catalog  USNO PMM at ESO} \\
\texttt{gsc@lei  catalog  Guide Star Catalog at LEDAS} \\
\texttt{ppm@lei  catalog  Positions and Proper Motions (PPM) Catalogue at LEDAS} \\
\texttt{sao@cadc  catalog  Smithsonian Astrophysical Observatory (SAO) Catalog at CADC} \\
\texttt{bd@lei  catalog  Bonner Durchmusterung (BD) Catalogue at LEDAS} \\
\texttt{simbad@eso  catalog  CDS SIMBAD object database at ESO} \\
\texttt{~~~$\vdots$ } \\
\end{tabular}
\end{center}

\begin{quote}
\caption{Example extract from list of remote catalogues accessible to
\texttt{catremote} \label{REMLIST} }
\end{quote}

\end{table}

By convention the names have the form \textit{catalogue}\texttt{@}\textit{institution}\, where \textit{catalogue}\, is an abbreviation for the
catalogue and \textit{institution}\, an abbreviation for the institution
where the on-line version is located.  The common values for \textit{institution}\, are listed in Table~\ref{REMINST}.

\begin{table}[htbp]

\begin{center}
\begin{tabular}{cl}
Abbreviation  &  Institution  \\ \hline
\texttt{cadc} & Canadian Astronomy Data Centre, Dominion Astrophysical
    Observatory  \\
\texttt{eso} &  European Southern Observatory, Garching bei M\"{u}nchen \\
\texttt{lei} &  Department of Physics and Astronomy, University of Leicester \\
\texttt{roe} &  Royal Observatory Edinburgh \\
\end{tabular}
\end{center}

\caption{Abbreviations for institutions hosting remotely accessible
catalogues  \label{REMINST} }

\end{table}

\subsubsection{Querying a remote catalogue}

To query a remote catalogue to find the objects which lie within a given
angular distance of a central Right Ascension and Declination simply type:


\begin{verse}
\texttt{catremote ~ query} \textit{cat-name $\alpha$ $\delta$ radius}
\end{verse}

For example:

\begin{terminalv}
catremote query usno@eso  12:15:00  30:30:00  10
\end{terminalv}

The arguments can be omitted from the right and any that are omitted will
be prompted for.  The individual arguments are as follows.

\begin{description}

  \item[\textit{cat-name}\/] Name of the catalogue to be queried.

  \item[$\alpha$\/] Central Right Ascension of the query.  The value
   should be for equinox J2000 and given in sexagesimal hours with a colon
   (`\texttt{:}') as the separator.

  \item[$\delta$\/] Central Declination of the query.  The value should be
   for equinox J2000 and given in sexagesimal degrees with a colon (`\texttt{:}')
   as the separator.  Southern Declinations are negative.

  \item[\textit{radius}\/] Radius of the query in minutes of arc.

\end{description}

This description of the query mode is something of a simplification: for
some catalogues it is possible to apply an additional condition which
the objects satisfy as well as lying within a given circle of sky; see
\xref{SSN/76}{ssn76}{}\cite{SSN76} for details.  If your coordinates for
the central position are for some equinox other than J2000 then you can use
the Starlink utility COCO (see SUN/56\cite{SUN56}) to convert them to the
required equinox.

\texttt{catremote} writes the extracted objects to a catalogue in your
current directory.  This catalogue is formatted as a Tab-Separated Table
(TST)\footnote{Unlike other CURSA applications \texttt{catremote} will only
write catalogues in the TST format.  This restriction is not important
because all the other CURSA applications can read catalogues in this
format.  If you want to convert the catalogue to another format (for
example, in order to input it into some other program) then simply use
\texttt{catcopy}, as described in Section~\ref{COPY}.}.  The name of the
catalogue is generated automatically from the name of the remote
catalogue and the coordinates of the central position.  For example,
if the name of the remote catalogue was `\texttt{usno@eso}' and the central
position was Right Ascension \texttt{10:30:00} and Declination \texttt{20:40:00} then the name of the local catalogue would be:

\begin{terminalv}
usno_eso_103000_204000.tab
\end{terminalv}

Note that the `\texttt{@}' in the remote catalogue name has been replaced
with an underscore (`\texttt{\_}') and the colons (`\texttt{:}') have been
removed from the coordinates.  Also, for a negative Declination the
minus sign is replaced by an `\texttt{m}'.  These substitutions are made
in order to ensure that the catalogue name consists only of alphabetic
characters, digits and underscores.  This restriction is not really
necessary on Unix systems but may be useful if the catalogue is ever
copied to another operating system.

\subsubsection{Finding the coordinates of a named object}

\texttt{catremote} can be used to query a remote `name resolver' to find
the coordinates of a named object.  Type:

\begin{verse}
\texttt{catremote ~ name} \textit{name-resolver object-name}
\end{verse}

For example:

\begin{terminalv}
catremote name simbad_ns@eso ngc6240
catremote name simbad_ns@eso iras20056+1834
catremote name simbad_ns@eso bd+303639
catremote name simbad_ns@eso pks1417-19
catremote name simbad_ns@eso mkn477
\end{terminalv}

If the name is recognised then the Right Ascension and Declination of the
object are displayed (the Right Ascension is in sexagesimal hours, the
Declination is in sexagesimal degrees and both are for epoch and equinox
J2000).  The technique only works if the name is recognised by the name
resolver.  The details of individual arguments are as follows.

\begin{description}

  \item[\textit{name-resolver}\/] The name of the name resolver which is to
   be queried.  In the above examples the SIMBAD name resolver provided by
   ESO using the SIMBAD integrated database maintained by the
   \htmladdnormallink{Centre de Donn\'{e}es astronomiques de Strasbourg}
   {http://cdsweb.u-strasbg.fr/CDS.html} (CDS) is being used.  This name
   resolver is included in the default list of remote on-line catalogues
   provided with CURSA.

  \item[\textit{object-name}] The name of an astronomical object which is to
   be resolved.  It should be entered without embedded spaces.  The case of
   letters (upper or lower) is not usually significant.  That is, case is
   not significant for \texttt{simbad\_ns@eso} and probably will not be
   significant for other name resolvers.

\end{description}

\subsection{Environment variables}

\texttt{catremote} takes some input from Unix shell environment variables
and these variables can be used to control its behaviour.  Some of the
variables are optional, but others are mandatory and must be set before
\texttt{catremote} is invoked.  Default values are set when CURSA is started.
All the environment variables used are listed in Table~\ref{ENVARS}, though
the only ones that you are likely to need to change are \texttt{CATREM\_CONFIG}
and \texttt{CATREM\_MAXOBJ}, which are described briefly below.  For a
complete description see \xref{SSN/76}{ssn76}{}\cite{SSN76}.

\begin{table}[htbp]

\begin{center}
\begin{tabular}{rl}
Environment Variable     & Description \\ \hline
\texttt{CATREM\_URLREADER} & Program to submit query \\
\texttt{CATREM\_CONFIG}    & URL of configuration file \\
\texttt{CATREM\_MAXOBJ}    & Maximum number of objects in results table \\
\texttt{CATREM\_ECHOURL}   & Echo URL sent to remote server? \\
\end{tabular}
\end{center}

\caption{Environment variables used by \texttt{catremote} \label{ENVARS} }

\end{table}

\begin{description}

  \item[\texttt{CATREM\_CONFIG}] specifies the configuration file to be used.
   The configuration file defines the list of catalogues which are
   currently available to \texttt{catremote}.  Specifying the configuration
   file is described in Section~\ref{REMCATS}, below.

  \item[\texttt{CATREM\_MAXOBJ}] is the maximum number of objects which the
   returned table is allowed to contain.  The default is 1000.

\end{description}

\subsection{Specifying the list of remote catalogues  \label{REMCATS} }

The list of all the remote catalogues which are currently accessible
is defined in a so-called \textbf{configuration file}.  This file is not
usually a file on your local computer (though in some cases it can be;
see below), but rather it is located on a remote machine.  The remote
catalogues which you can access are not necessarily on the same remote
machine as the configuration file, though sometimes they will be.  \texttt{catremote} accesses the configuration file via the Hyper-Text Transfer
Protocol (HTTP)  developed for the World Wide Web.  You specify the
configuration file to be used by setting the Unix shell environment
variable \texttt{CATREM\_CONFIG} to the URL (Uniform Resource Locator)
for the file.  This process is exactly analogous to specifying the URL
of a Web page when using a Web browser.  The default configuration file
used by CURSA is:

\begin{terminalv}
$CURSA_DIR/cursa.cfg
\end{terminalv}

To specify a given configuration file you simply set environment
variable \texttt{CATREM\_CONFIG} to the required URL prior to running \texttt{catremote} or \texttt{xcatview}.  For example, to specify a copy of the
original ESO configuration file type:

\begin{terminalv}
% setenv  CATREM_CONFIG  http://archive.eso.org/skycat/skycat2.0.cfg
\end{terminalv}

\subsubsection{Creating your own configuration file}

You can create your own configuration file.  Such a file might contain,
for example, only the catalogues which you use regularly.  However, I
recommend that you only try to create your own configuration file if you
really understand what you are doing.  Configuration files are documented
in \xref{SSN/75}{ssn75}{}\cite{SSN75}.

\subsection{How remote access works  \label{REMWORK} }

This section outlines how the remote access mechanism works.  It is
not strictly necessary to follow it in order to use \texttt{catremote},
though it may help you to appreciate some of the reasons behind some
of \texttt{catremote}'s behaviour.  The configuration file used by \texttt{catremote} is no more than its name implies.  It simply defines a list of
remote catalogues and provides some details for each: its computer network
address, the sorts of query that it will accept, a short description
\emph{etc}.

For every remote catalogue listed in the configuration file there must
be a server running on a remote machine.  This server will accept
queries sent from \texttt{catremote}, interrogate the relevant catalogue
to select the objects which satisfy the query and return the selected
objects to \texttt{catremote}.

There is a standard protocol for both the queries and the returned
results which allows \texttt{catremote} and the various servers to
communicate.  This protocol is a subset of a proposed general format
for exchanging information between remote astronomical information services
which is being developed at the Centre de Donn\'{e}es astronomiques de
Strasbourg (CDS) and elsewhere.  The proposal is described in the
working document \textit{Astronomical Server URL}\, by M.~Albrecht \textit{et
al.}\cite{SERVERURL}.  It is important to realise that this protocol
is general, and allows not just \texttt{catremote}, but also various other
clients, such as \xref{GAIA}{sun214}{}\cite{SUN214} and
\htmladdnormallinkfoot{\textit{SkyCat}\/}{http://archive.eso.org/skycat/},
to communicate with various different servers for differing purposes.
Thus, it is not optimised for \texttt{catremote}, resulting in some
peculiarities in the catalogues of selected objects written by \texttt{catremote} (see Section~\ref{REMPEC}, below).

The \textit{Astronomical Server URL}\, protocol, as its name implies, uses
the Hyper-Text Transfer Protocol (HTTP) developed for the World Wide
Web.  Thus, in order for \texttt{catremote} to work successfully your
local computer must be configured for running Web clients (such as \texttt{netscape}).  Of course, most Starlink nodes (and, indeed, most networked
computers) will be so configured.  One way of thinking of \texttt{catremote} is that it is functioning as a very specialised Web browser.
Similarly, the remote servers are, strictly speaking, `gateways'
using the Common Gateway Interface (CGI).

There are various types of remote servers: catalogues, name servers,
data archives and image servers.  All are `catalogues' in the sense of
returning a table of values.  A catalogue server returns traditional
columns such as position, magnitude, colours, spectral type \emph{etc}.
A name server primarily returns columns containing celestial coordinates
and alternative names for the object.  A data archive will return a normal
catalogue of values but at least one column will list URLs pointing to
images or `bulk data' files for the objects tabulated.  It is important to
realise that though \texttt{catremote} can return these special columns \textit{CURSA contains no facilities for interpreting them}.  When \texttt{catremote}
displays the list of accessible catalogues it includes the type of each
(catalogue, data archive, name server or image server) immediately after
the name and before the description.

\subsection{Peculiarities and shortcomings  \label{REMPEC} }

You may notice the following peculiarities and shortcomings with
selections extracted from remote catalogues.

\begin{enumerate}

  \item The selection does not contain all the columns which you
   expected to be present in the catalogue.  Sometimes the remote
   versions of catalogues contain only some of the columns present
   in the full catalogue (as published, or as available from the CDS).
   Obviously, the decision about which columns to include in the
   on-line catalogue rests entirely with the institution which is
   providing it and is entirely outwith the control of Starlink.
   The catalogues available from Leicester usually seem to contain
   most of the columns available in the corresponding originals.

  \item If the remote catalogue is a `data archive' then usually the
   returned selection will contain at least one `odd' column comprising a
   list of URLs (see Section~\ref{REMWORK}, above).  This column is
   intended to to give access to an image or other `bulk' data for each
   object.  \textit{CURSA contains no facilities to process these columns
   and access the bulk data}.

  \item The protocol used to return the catalogue of selected objects
   is rather deficient in metadata (see Section~\ref{COMP}).  In particular,
   the only information returned for each column is its name; the units,
   data type and external format are not specified.  The Right Ascension
   and Declination are exceptions in that they are returned with known
   units.

\end{enumerate}

\subsection{Local or remote access?}

This section considers whether it is better to access a given catalogue
remotely or to obtain a copy of the complete catalogue (for example,
as described in Section~\ref{OBTAIN}) and to access it on your local
computer.

The advantages of remote access are that it is very quick and easy.
Also the Right Ascension and Declination are automatically returned in
a form which is fully compatible with CURSA.  However, the catalogue
may contain only some of the columns and most of the metadata (see
Section~\ref{COMP}) will be missing.  Finally only `circular area'
selections are possible.

The advantages of local access are that the entire catalogue, including all
its columns and metadata, is available and a variety of different sorts
of selection can be made on it.  The disadvantages are that more effort
is involved in obtaining a copy and creating a version with coordinates
which are fully compatible with CURSA.

As a rough guide, you should probably use remote access if you just want
to make a quick `circular area' selection and simply list or plot the
results.  However, if you are intending to make extensive use of a
catalogue it is probably better to have a local copy.


\newpage
\appendix

\section{\xlabel{EXPR}\label{EXPR}Expression syntax}

Expressions in CURSA are mainly used for two purposes:

\begin{itemize}

  \item computing a new column to appear in a listing or output
   catalogue,

  \item defining a new selection.

\end{itemize}

The rules for expressions are similar in both cases and both usages are
described here.

\subsection{Creating a new column}

Expressions for creating a new column have an algebraic format, and comprise:
columns, vector column elements, parameters and constants linked by
arithmetic operators and mathematical functions. For example, suppose that
a catalogue contained scalar columns called \texttt{x}, \texttt{y} and \texttt{z}
and parameters called \texttt{p} and \texttt{q}.  Some valid expressions are:

\begin{equation}
\begin{array}{l}
\texttt{x}  \\
\texttt{p}  \\
\texttt{x + p}  \\
\texttt{(x + y + 2) / (p + q)}  \\
\texttt{(2.0*x + y + 3.75*p) + 13.0) / (z + 1.8*q)}
\end{array}
\end{equation}

Remember that in CURSA column and parameter names are not case-sensitive.
Thus the following column or parameter names would all be considered
equivalent:

\begin{terminalv}
HD_NUMBER
HD_Number
hd_number
\end{terminalv}

Vector column elements occur in expressions with their usual syntax:
the name of the base column followed by the element number enclosed
in square brackets. The first element in a vector is numbered one. For
example, an expression to add two to the fourth element of vector
\texttt{FLUX} would be `\texttt{FLUX[4] + 2.0}'.

\subsection{Defining a new selection}

Expressions for defining a selection have a similar algebraic format to
those for creating a new column.  However, they must include a
relational operator to define the selection criterion.  All the other
rules are exactly as for defining a new column.  Following the above
example some valid expressions for defining a selection are:

\begin{equation}
\begin{array}{l}
\texttt{x > 3.0}  \\
\texttt{x > y + p + 2.36}  \\
\texttt{y == 3}
\end{array}
\end{equation}

Angles can be included in expressions using sexagesimal notation.  For
example:

\begin{equation}
\texttt{dec > +190:30:00 ~ .AND. ~ dec < +191:30:00}
\end{equation}

(remember that if a sexagesimal number is unsigned it is interpreted as
hours; to be interpreted as degrees it must be signed).

\subsection{Details of expressions}

The arithmetic operators are:

\begin{tabular}{ll}
\texttt{+}  & addition,        \\
\texttt{-}  & subtraction,     \\
\texttt{*}  & multiplication,  \\
\texttt{/}  & division.        \\
\end{tabular}

brackets (`\texttt{(}' and `\texttt{)}') may be used as required.


\subsection{Mathematical functions provided}

Table~\ref{FUNCS} lists the mathematical functions which are provided.
The letters denote data types permitted, coded as follows: B = BYTE,
H = half INTEGER, I = INTEGER, R = REAL, D = DOUBLE PRECISION, C =
CHARACTER, L = LOGICAL. The appearance of N as an argument means that
any numeric type (BHIRD) is permitted, as a result it means that the
type is the widest type of any of the arguments.  R/D means that the
result is REAL unless one or more arguments is of DOUBLE PRECISION
type in which case D is the result.

\begin{table}[htbp]

\begin{center}
\begin{tabular}{ll}
Function            & Notes   \\ \hline
B = BYTE(N)         & convert to BYTE data type  \\
H = HALF(N)         & convert to INTEGER*2 data type  \\
I = INT(N)          & convert to INTEGER data type  \\
R = REAL(N)         & convert to REAL data type  \\
D = DBLE(N)         & convert to DOUBLE PRECISION data type  \\
I = NINT(N)         & convert to nearest INTEGER  \\
N = MIN(N,N)        & the function must have precisely two arguments  \\
N = MAX(N,N)        & the function must have precisely two arguments  \\
N = MOD(N,N)        & remainder  \\
N = ABS(N)          & absolute value  \\
R/D = SQRT(N)       & square root  \\
R/D = LOG(N)        & natural logarithm  \\
R/D = LOG10(N)      & logarithm to the base 10  \\
R/D = EXP(N)        & exponential  \\
R/D = SIN(N)        & sine; argument in radians  \\
R/D = COS(N)        & cosine; argument in radians  \\
R/D = TAN(N)        & tangent; argument in radians  \\
R/D = ASIN(N)       & arc-sine; result in radians  \\
R/D = ACOS(N)       & arc-cosine; result in radians  \\
R/D = ATAN(N)       & arc-tangent; result in radians  \\
R/D = ATAN2(N,N)    & arc-tangent (two arguments) result in radians  \\
I = IAND(I,I)       & bitwise logical AND  \\
I = IOR(I,I)        & bitwise logical OR  \\
I = XOR(I,I)        & bitwise logical exclusive OR  \\
R/D = DTOR(N)       & degrees to radians conversion  \\
R/D = RTOD(N)       & radians to degrees conversion  \\
C = UPCASE(C)       & convert character string to upper case  \\
C = STRIP(C)        & leading and trailing spaces are removed  \\
C = SUBSTR(C,N,N)   & returns characters from positions argument 2 \\
                    & to argument 3 inclusive, with the positions \\
                    & starting at 1  \\
L = NULL(*)         & .TRUE. if argument is NULL  \\
D = HMSRAD(N,N,N)   & converts 3 arguments hours, minutes, seconds to \\
                    & radians  \\
D = DMSRAD(C,N,N,N) & first argument is the sign (`$+$' or `-' ), \\
                    & converts degrees, minutes, seconds to radians  \\
D = GREAT(N,N,N,N)  & great circle distance between two spherical \\
                    & coordinates. All the input arguments and the \\
                    & return argument are in radians. The input arguments \\
      & are in the order: $(\alpha_{1},\delta_{1}$,$\alpha_{2},\delta_{2})$ \\
D = PANGLE(N,N,N,N) & position angle of point $(\alpha_{2},\delta_{2})$
      from point \\
      & $(\alpha_{1},\delta_{1})$.  All the input arguments and the return \\
      & argument are in radians. The input arguments are in \\
      & the order: $(\alpha_{1},\delta_{1}$,$\alpha_{2},\delta_{2})$ \\
\end{tabular}
\end{center}

\caption{Mathematical functions which may be used in expressions
\label{FUNCS} }

\end{table}


\subsection{Rules for expressions}

The expression string can contain constants, column and parameter names,
operators, functions, and parentheses.  In general the usual rules of
algebra and Fortran should be followed, with some minor exceptions
as noted below.

\begin{enumerate}

  \item Spaces are permitted between items, except that a function-name
   must be followed immediately by a left parenthesis.  Spaces are not
   permitted within items such as names and numerical constants, but can be
   used within character strings and date/time values in curly braces.

  \item Lower-case letters are treated everywhere as identical to the
   corresponding upper-case letter.

  \item Column and parameter names can be up to fifteen characters long, and
   may consist of letters, digits, and underscores, except that the first
   character must not be a digit.

  \item Vector elements are supported but with a restricted syntax:
   they may consist of a name followed by an unsigned integer constant
   subscript enclosed in square brackets, for example \texttt{FLUX[4]} or
   \texttt{MAGNITUDE[13]}. The first element of the vector is numbered one.

  \item CHARACTER constants may be enclosed in a pair of single or
   double quotes; embedded quotes of the same type may be denoted by
   doubling up on the quote character within the string, for example
   \verb+'DON''T'+ or \verb+"DON""T"+.

  \item LOGICAL constants may be \texttt{.TRUE.} or \texttt{.FALSE.} but
   abbreviations of these words are allowed down to \texttt{.T.} and
   \texttt{.F.}

  \item Numerical constants may appear in any valid form for Fortran 77
   (except that embedded spaces are not allowed).  Some additional forms
   are also permitted, as shown below.

  \item \%Xstring \%Ostring \%Bstring for hexadecimal, octal and binary
   INTEGER constants respectively.

  \item Angles in sexagesimal notation: colons must be used to separate
   items, for example hours:minutes:seconds (or degrees:minutes:seconds).
   If there is a leading sign then the value will be taken as
   degrees:minutes:seconds, otherwise hours:minutes:seconds.  In either
   case the value is converted to RADIANS.

  \item A date/time value may be given as a string enclosed in curly
   braces; a range of common formats are permitted, with order
   year-month-day or day-month-year, and the month as a number or
   three-character abbreviation.  The time may follow with colons separating
   hours:minutes:seconds.  Examples of some valid dates:
  \begin{terminalv}
1992-JUL-26 12:34:56
92.7.26
26/7/92T3:45
  \end{terminalv}

  \item Relational operators are supported in both Fortran 77 form
   (for example \texttt{.GE. .NE.}) as well as in the Fortran 90 forms (for
   example, $\texttt{>=}$ $\texttt{/=}$ ).

  \item Single-symbol forms for \texttt{.AND. .OR. and .NOT.} are provided as
   an alternative: \texttt{\&} $\texttt{|}$ \texttt{\#} respectively.

  \item The dots may be left off the Fortran 77 forms of the relational
   operators and the logical operators \texttt{.AND.} and \texttt{.OR.} where
   spaces or parentheses separate them from names or constants, but the
   logical constants and the \texttt{.NOT.} operator need the enclosing dots
   to distinguish them from other lexical items in all cases.

  \item INTEGER division does not result in truncation (as in Fortran)
   but produces a floating-point result.  The \texttt{NINT} or \texttt{INT}
   function should be used (as appropriate) if an INTEGER result is
   required.

  \item The functions \texttt{MAX} and \texttt{MIN} must have exactly two
   arguments.

  \item All arithmetic is carried out internally in DOUBLE PRECISION
   (but the compiler works out the effective data type of the result using
   the normal expression rules).

  \item Exponentiation is performed by log/exp functions, with use of
   ABS to avoid taking logs of negative arguments, thus \texttt{-2**3} will
   come out as `+8' not `-8'.

\end{enumerate}

\subsection{Operator precedence}

The operator precedence rules are show in Table~\ref{PREC}. The rules
of Fortran 90 are used as far as possible; in this table the larger
numbers denote higher precedence (tighter binding).

\begin{table}[htbp]

\begin{center}
\begin{tabular}{rl}
Precedence  & Function/operator  \\ \hline
  2 & start/end of expression  \\
  4 & \texttt{(  )}  \\
  6 & \texttt{,}  \\
  8 & \texttt{.EQV.  .NEQV.}  \\
 10 & \texttt{.OR.}   $\texttt{|}$  \\
 12 & \texttt{.AND.  \&} \\
 14 & \texttt{.NOT.  \#}  \\
 16 & \texttt{.EQ. .GE. .GT. .LE. .LT. .NE.} $\texttt{== ~ >= ~ > ~ <= ~ < ~ /=}$ \\
 18 & \texttt{FROM  TO}  \\
 20 & \texttt{//}  \\
 22 & \texttt{+ -} (binary operators)  \\
 24 & \texttt{+ -} (unary operators)  \\
 26 & \texttt{* /}  \\
 28 & \texttt{**}  \\
 30 & all functions  \\
\end{tabular}
\end{center}

\caption{Operator precedence rules\label{PREC} }

\end{table}

Note that all operators except \texttt{**} associate from left to right, but
\texttt{**} and functions associate from right to left.


\section{\xlabel{ANGLE}\label{ANGLE}Storing and representing columns of angles}

CURSA provides special facilities for representing columns which
contain angles. Usually angular columns are used to store celestial
coordinates such as Right Ascension and Declination, position angles
or small angles such as the great circle distance between neighbouring
points or the angular size of extended objects.  However, they
can contain any angular measure. The basis of these facilities is that
columns of angles are stored and manipulated internally in radians, but are
automatically displayed by \texttt{xcatview} (see Section~\ref{XVIEW}) and \texttt{catview} (see Section~\ref{VIEW}) as hours or degrees, minutes or seconds,
optionally formatted as a sexagesimal value.

CURSA recognises a column which is to be treated in this way by a
combination of the DTYPE (data type) and UNITS attributes.

\begin{itemize}

  \item The data type must be DOUBLE PRECISION or
   REAL,\footnote{DOUBLE PRECISION is more common in practice because
   REAL numbers are insufficiently accurate to represent an angle to
   a precision of a second of arc or better.}

  \item The units attribute of the column should be set to
   `\texttt{RADIANS}' followed by an angular format specifier
   enclosed in curly brackets (`\{\}'). The simplest forms of this
   angular format specifier are simply `\texttt{HOURS}' and `\texttt{DEGREES}' for hours and degrees respectively. Thus, examples of
   the UNITS attribute are:

  \begin{description}

    \item[\texttt{RADIANS\{HOURS\}}] to display the column in hours,

    \item[\texttt{RADIANS\{DEGREES\}}] to display the column in degrees.

  \end{description}

   The angular format specifiers are described in full in the following
   section.

\end{itemize}

Incidentally, the external display format attribute of the column, EXFMT,
must be set to a valid Fortran 77 format specifier corresponding to the
data type of the column, because of the way that CURSA represents
catalogues as FITS tables.

It is possible to interactively modify the way that a column of angles
is displayed by \texttt{xcatview} or \texttt{catview}. This alteration is
achieved by supplying a new UNITS attribute for the column. In practice
it is only the angular format specifier which is changed; the UNITS must
still be of the form

\begin{terminalv}
RADIANS{angular format specifier}
\end{terminalv}

The angular format specifier is described in full below.

\subsection{Angular format specifiers}

The angular format specifier forms part of the UNITS attribute for an
angular column. The UNITS attribute for an angular column has the
form:

\begin{terminalv}
RADIANS{angular format specifier}
\end{terminalv}

The simplest angular format specifiers are `\texttt{HOURS}' and
`\texttt{DEGREES}'.

\begin{description}

  \item[\texttt{HOURS}] will cause the angle to be displayed as
   hours, minutes and seconds, with the seconds displayed to one
   place of decimals,

  \item[\texttt{DEGREES}] will cause the angle to be displayed as
   degrees, minutes and seconds, with the seconds displayed as a
   whole number.

\end{description}

If the angular format specifier is omitted altogether and
the UNITS attribute simply set to `\texttt{RADIANS}' or `\texttt{RADIANS\{\}}' then the angle will be interpreted exactly as
though the angular format specifier had been `\texttt{DEGREES}'.
There are additional simple angular format specifiers for displaying
angles as minutes or seconds of arc or time to a specified number of
decimal places:

\begin{description}

  \item[\texttt{ARCMIN}\textit{.n}] minutes of arc,

  \item[\texttt{ARCSEC}\textit{.n}] seconds of arc,

  \item[\texttt{TIMEMIN}\textit{.n}] minutes of time,

  \item[\texttt{TIMESEC}\textit{.n}] seconds of time.


\end{description}

\textit{.n}\, is the number of decimal places required.  If \textit{.n}\, is
omitted then the value will be displayed as an integer number.  Though
these angular specifiers can be used to display any angle, obviously
they are most likely to be useful for small angles.

These simple angular format specifiers will usually be
adequate for representing columns of celestial coordinates.
However, sometimes you might wish to specify a different
representation for an angle. CURSA accepts angular
format specifiers which permit angles to be represented in a
number of different formats. These specifiers are constructed
from a selection from amongst the following elements:

\begin{center}
\texttt{I B L + Z H D M S T} .\textit{n}
\end{center}

The meaning of each of the individual elements is as follows.

\begin{description}

  \item[\texttt{I}] Use the ISO standard separator for expressing
   times, a colon (`:'), to separate hours or degrees, minutes and
   seconds.

  \item[\texttt{B}] Use a single blank space to separate hours or
   degrees, minutes and seconds.

  \item[\texttt{L}] Use a letter (\texttt{h}, \texttt{d}, \texttt{m}, or
   \texttt{s}, as appropriate) to separate hours or degrees, minutes
   and seconds.

  \item[\texttt{+}] Insert a plus sign (`$+$') before positive angles
   (a minus sign is, of course, always inserted before negative
   angles).

  \item[\texttt{Z}] Insert leading zeros before the hours, degrees, minutes
   or seconds. Hours, minutes and seconds are assumed to be two-digit
   numbers and degrees three-digit.

  \item[\texttt{H}] Express the angle in units of hours.

  \item[\texttt{D}] Express the angle in units of degrees.

  \item[\texttt{M}] If an \texttt{M} occurs when either \texttt{H} or \texttt{D}
   is present then it indicates that the hours or degrees are to be
   subdivided into sexagesimal minutes.  If an \texttt{M} occurs when
   neither \texttt{H} nor \texttt{D} is present then it indicates that the
   units are minutes of either arc or time.

  \item[\texttt{S}] If an \texttt{S} occurs when an \texttt{M} is present then
   it indicates that the minutes are to be subdivided into sexagesimal
   seconds (the minutes may be either the actual units or themselves
   a sexagesimal subdivision of hours or degrees; see \texttt{M} above).
   If an \texttt{S} occurs when an \texttt{M} is not present then it indicates
   that the units are seconds of arc or time.

  \item[\texttt{T}] In the case where \texttt{H} and \texttt{D} are both absent
   and either or both of \texttt{M} and \texttt{S} are present then \texttt{T}
   indicates that the units are minutes or seconds of time.  If it is
   omitted in this case then the units are minutes or seconds of arc.
   If either \texttt{H} or \texttt{D} is present then \texttt{T} is ignored.

  \item[\textit{.n}] Display the least significant unit (seconds,
   minutes, degrees or hours, as appropriate) to \textit{n}\, decimal
   places.

\end{description}

Any of the items may be omitted, down to and including a
completely blank specifier.

The items can occur in any order, except that \textit{.n}\, must
occur last. However, for human readability I recommend that the
items occur in the order:

\begin{center}
(any of: \texttt{I}, \texttt{B}, \texttt{L}, \texttt{+} or \texttt{Z}) (\texttt{H}
or \texttt{D}) \texttt{M S T} .\textit{n}
\end{center}

If items are omitted the following defaults apply.

\begin{itemize}

  \item If neither \texttt{I}, \texttt{B} nor \texttt{L} is specified
   then \texttt{I} is assumed.

  \item If \texttt{+} is omitted then positive angles are not
   preceded by a `$+$' sign.

  \item If \texttt{Z} is omitted then leading zeros are omitted in the
   primary units (hours, degrees, minutes or seconds), but leading zeros
   are always included in any sexagesimal subdivisions.

  \item If none of \texttt{H}, \texttt{D}, \texttt{M} or \texttt{S} are specified
   then \texttt{D} is assumed (that is, the default units are degrees).

  \item If \texttt{H} or \texttt{D} are present but \texttt{M} is omitted then
   the hours or degrees are not subdivided into minutes.

  \item If \texttt{M} is present but \texttt{S} is omitted then the minutes
   are not subdivided into seconds.

  \item If \texttt{S} is present in addition to \texttt{H} or \texttt{D} but
   \texttt{M} is absent then \texttt{S} is ignored (this case is technically
   illegal).

  \item If \textit{.n}\, is omitted then the least significant unit
   (seconds, minutes, degrees or hours, as appropriate) is
   displayed as a whole number, without any places of decimals.

\end{itemize}

Table~\ref{SEXAG_EXAM} lists a number of examples of angular format
specifiers which might be used to represent `large' angles, such as
celestial coordinates, together with examples of how they would
represent an angle.  Table~\ref{SMALL_EXAM} lists a number of examples of
angular format specifiers which might be used to represent small angles,
such as the great circle distance between two neighbouring objects or the
angular size of an extended object, together with examples of how they
would represent an angle.

\begin{table}[htbp]

\begin{center}
\begin{tabular}{lll}
Specifier     & Example            & Notes          \\ \hline
\texttt{D}       & \texttt{63}          & Integer degrees \\
\texttt{D.2}     & \texttt{62.86}       & Degrees to two places of decimals \\
\texttt{DM}      & \texttt{62:52}       & Degrees and integer minutes \\
\texttt{DM.2}    & \texttt{62:51.58}    & Degrees and minutes to two places of decimals \\
\texttt{DMS}     & \texttt{62:51:35}    & Degrees, minutes and integer seconds \\
\texttt{DMS.2}   & \texttt{62:51:34.65} & Degrees, minutes and seconds to two places of decimals \\
              &                    & \\
\texttt{H}       & \texttt{4}            & Integer hours \\
\texttt{H.2}     & \texttt{4.19}         & Hours to two places of decimals \\
\texttt{HM}      & \texttt{4:11}         & Hours and integer minutes \\
\texttt{HM.2}    & \texttt{4:11.44}      & Hours and minutes to two places of decimals \\
\texttt{HMS}     & \texttt{4:11:26}      & Hours, minutes and integer seconds \\
\texttt{HMS.2}   & \texttt{4:11:26.31}   & Hours, minutes and seconds to two places of decimals \\
              &                    & \\
\texttt{BHMS.2}  & \texttt{4 11 26.31}   & Space character as separator \\
\texttt{LHMS.2}  & \texttt{4h11m26.31s}  & Letter as separator \\
\texttt{ZHMS.2}  & \texttt{04:11:26.31}  & Leading zeros \\
\texttt{+HMS.2}  & \texttt{+4:11:26.31}  & Signed value \\
              &                    & \\
\texttt{L+ZDM.3} & \texttt{+062d51.577}  & Letter separator, leading zeros and signed \\
\end{tabular}

\begin{quote}
The examples show how the various specifiers would represent an angle of
1.09710742 radians (or \dms{62}{51}{34}{65}).
% (or 62$^{\circ}$ 51' 34.65").
\end{quote}

\caption{\label{SEXAG_EXAM}Examples of sexagesimal format specifiers}
\end{center}

\end{table}

\begin{table}[htbp]

\begin{center}
\begin{tabular}{lll}
Specifier   & Example        & Notes          \\ \hline
\texttt{M}     & \texttt{3}        & Integer minutes of arc \\
\texttt{M.3}   & \texttt{3.227}    & Minutes of arc to three places of decimals \\
\texttt{MS}    & \texttt{3:14}     & Minutes and integer seconds of arc \\
\texttt{MS.3}  & \texttt{3:13.600} & Minutes and seconds of arc to three places of decimals \\
\texttt{S}     & \texttt{194}      & Integer seconds of arc \\
\texttt{S.3}   & \texttt{193.600}  & Seconds of arc to three places of decimals \\
            &                & \\
\texttt{MT}    & \texttt{0}        & Integer minutes of time \\
\texttt{MT.3}  & \texttt{0.215}    & Minutes of time to three places of decimals \\
\texttt{MST}   & \texttt{0:13}     & Minutes and integer seconds of time\\
\texttt{MST.3} & \texttt{0:12.907} & Minutes and seconds of time to three places of decimals \\
\texttt{ST}    & \texttt{13}       & Integer seconds of time \\
\texttt{ST.3}  & \texttt{12.907}   & Seconds of time to three places of decimals \\
            &                & \\
\texttt{BMS}   & \texttt{3 14}     & Space character as separator \\
\texttt{LMS}   & \texttt{3m14s}    & Letter as separator \\
\texttt{ZMS}   & \texttt{03:14}    & Leading zeros \\
\texttt{+MS}   & \texttt{+3:14}    & Signed value \\
            &                & \\
\texttt{L+ZMS} & \texttt{+03m14s}  & Letter separator, leading zeros and signed \\
\end{tabular}

\begin{quote}
These specifiers might typically be used to represent the great circle
distance between neighbouring objects or the angular size of an extended
object.  There is no reason why they should not be used to represent
`large' angles such as celestial coordinates, though the output would
look a bit odd.  The examples show how the various specifiers would
represent an angle of 9.3860x10$^{-4}$ radians (or \dms{0}{3}{13}{66}).
% (or 3' 13.6" of arc).
\end{quote}

\caption{\label{SMALL_EXAM}Examples of angular format specifiers for
small angles}
\end{center}

\end{table}

The simple angular format specifiers, `\texttt{HOURS}', `\texttt{DEGREES}',
`\texttt{ARCMIN}', `\texttt{ARCSEC}', `\texttt{TIMEMIN}' and `\texttt{TIMESEC}'
are just synonyms for particular cases of the
general specifiers. They are listed, together with the
equivalent full specification in Table~\ref{SEXAG_SIMPLE}.

\begin{table}[htbp]

\begin{center}
\begin{tabular}{lllc}
Simple Specifier & Equivalent Full Specifier & Example & Notes \\ \hline
\texttt{HOURS}      & \texttt{IHMS.1} & \texttt{14:11:26.3} & 1  \\
\texttt{DEGREES}    & \texttt{IDMS}   & \texttt{62:51:35}   & 1  \\
                 &              &                  &    \\
\texttt{ARCMIN}     & \texttt{M}      & \texttt{3}          & 2  \\
\texttt{ARCSEC}     & \texttt{S}      & \texttt{194}        & 2  \\
\texttt{TIMEMIN}    & \texttt{MT}     & \texttt{0}          & 2  \\
\texttt{TIMESEC}    & \texttt{ST}     & \texttt{13}         & 2  \\
                 &              &                  &    \\
\texttt{ARCMIN.3}   & \texttt{M.3}    & \texttt{3.227}      & 3  \\
\texttt{ARCSEC.3}   & \texttt{S.3}    & \texttt{193.600}    & 3  \\
\texttt{TIMEMIN.3}  & \texttt{MT.3}   & \texttt{0.215}      & 3  \\
\texttt{TIMESEC.3}  & \texttt{ST.3}   & \texttt{12.907}     & 3  \\
\end{tabular}

\vspace{4mm}

\begin{quote}
\textbf{Notes}

\begin{enumerate}

  \item The number of decimal places is fixed for these specifiers.

  \item The number of decimal places has been omitted so integers
   without any decimal places are assumed.

  \item Three places of decimals were specified.

\end{enumerate}

The example for the first two specifiers is an angle of 1.09710742 radians;
for the remaining specifiers the example is an angle of 9.3860x10$^{-4}$
radians.

\end{quote}

\caption{\label{SEXAG_SIMPLE}The simple angular format specifiers and
their equivalents}
\end{center}

\end{table}


\section{\xlabel{FORMAT}\label{FORMAT}Catalogue formats}

CURSA can access catalogues held in three different formats: FITS tables,
TST and STL.  The restrictions and peculiarities associated with each of
these formats are described below.

CURSA determines the type of a catalogue from the `file type' component
of the name of the file holding the catalogue. The file types for the
various formats are included in the descriptions below. If a file name
is specified without a file type then it is assumed to be a FITS table.


\subsection{FITS}

File types: \texttt{.FIT .fit .FITS .fits .GSC .gsc}

Mixed capitalisations, such as \texttt{.Fit}, are also supported.

The \texttt{.GSC} and \texttt{.gsc} file types tables are provided in order to
allow regions of the HST \textit{Guide Star Catalog}\, to be accessed easily
(see also Section~\ref{GSCIN}).

CURSA can read both binary and formatted FITS tables. It can write only
binary FITS tables. It should be able to handle most components of
FITS tables, with the exception of variable length array columns. If
a variable length array column is encountered a warning message will
be reported and the column will be ignored.

If a column containing no data is encountered a warning message will be
generated and the column will be ignored.

In common with other Starlink software, CURSA does not support the
COMPLEX REAL and COMPLEX DOUBLE PRECISION data types. If it encounters
COMPLEX columns in a FITS table it represents them as follows:

\begin{itemize}

  \item a COMPLEX REAL scalar column is represented as a REAL vector
   column of two elements,

  \item a COMPLEX REAL vector column of $n$ elements is represented as
   a REAL vector column of $2n$ elements,

  \item a COMPLEX DOUBLE PRECISION scalar column is represented as a
   DOUBLE PRECISION vector column of two elements,

  \item a COMPLEX DOUBLE PRECISION vector column of $n$ elements is
   represented as a DOUBLE PRECISION vector column of $2n$ elements.

\end{itemize}

Usually the table component of a FITS file occurs in the first FITS
extension to the file. When reading an existing FITS file CURSA will look
for a table in the first extension. In cases where the table is located
in an extension other than the first you may specify the required
extension by giving its number inside curly brackets after the name
of the file. For example, if the table occurred in the third extension
of a FITS file called \texttt{perseus.FIT} you would specify:

\begin{terminalv}
perseus.FIT{3}
\end{terminalv}

The closing curly bracket is optional. When CURSA writes FITS tables
the table is always written to the first extension.

\subsubsection{Textual information}

The textual information for a FITS table comprises the entire contents
of the primary header and the appropriate table extension header of the
FITS file containing the table. The entire contents of both headers are
returned because this is the best way to present the maximum amount of
information about the catalogue to the user in its full context. For
example, a FITS table COMMENT keyword may be used to annotate other
keywords and if only the COMMENT keywords were returned `out of context'
they would be difficult to understand, and perhaps even misleading.

In addition CURSA invents two additional lines of textual information.
The first precedes the primary header and serves to introduce it. The
second is inserted between the primary header and the table extension
header, and serves to introduce the table extension header.

\subsection{\label{TST}TST}

File types: \texttt{.TAB  .tab}

Mixed capitalisations, such as \texttt{.Tab}, are also supported.

CURSA can read and write catalogues in the TST (Tab-Separated Table)
format.  The TST format is a standard for exchanging catalogue data
and is commonly used to transfer subsets extracted from remote catalogues
or archives across the Internet.  Typically when a client such as \texttt{catremote} (see Section~\ref{REMACCSS}) running on your local computer
queries a remote catalogue or archive the selected objects will be returned
as a tab-separated table.  In addition to CURSA, the TST format is
also used by GAIA (see \xref{SUN/214}{sun214}{}\cite{SUN214}),
\htmladdnormallinkfoot{\textit{SkyCat}\/}{http://archive.eso.org/skycat/}
and Starbase (see Section~\ref{STARBASE}).  It is documented in
\xref{SSN/75}{ssn75}{}\cite{SSN75}.

Compared to the other formats supported by CURSA, the TST format is
somewhat deficient in the amount of metadata that it includes.  In
particular, the details stored for each column do not include its data
type or units.  Consequently, CURSA deduces a data type for each column
by reading the values that it contains.  This procedure usually works
reasonably well, though occasionally it produces bizarre results.
Unfortunately there is no similar simple trick which can replace the
missing units.  If you find that you need to fix-up the column details in
a TST catalogue one approach is to use \texttt{catcopy} (see Section~\ref{COPY})
to convert the catalogue to the STL format (see Appendices~\ref{STLTUT}
and \ref{STLREF}) and then edit the STL column definitions, as appropriate.
When CURSA writes a TST catalogue it saves the column data type,
external format and units.  These details are written in a format which
CURSA can interpret if it subsequently reads the catalogue.  Though this
enhancement is specific to CURSA it is entirely consistent with the TST
format and does not affect the ability of external programs to read the
catalogues.  The format in which the additional information is stored is
documented in SSN/75.

The TST format does not support vector columns.  If a catalogue
containing vector columns is written as a tab-separated table each vector
element is written as a scalar column.

Unsurprisingly, given its provenance as a medium for transporting
subsets extracted from remote catalogues across the Internet, the
tab-separated table format is intended for use with relatively small
catalogues and is unsuitable for very large ones.  Currently CURSA sets no
upper limit to the size of catalogue for which it can be used.  However,
if you attempt to read a catalogue containing more than 15,000 rows a
warning message is issued.  A large TST format catalogue may take a while
to open for reading and CURSA may be unable to access a very large TST
catalogue\footnote{For information, the underlying reason for this
behaviour is that CURSA attempts to memory-map work arrays to hold the
columns of an TST catalogue and then reads the table into these arrays
when an input catalogue is opened.  For a very large catalogue CURSA may
be unable to map the required arrays.}.

\subsubsection{Textual information}

The textual information for a tab-separated table comprises the entire
description of the table.  This approach makes the maximum amount of
information about the catalogue available to the user in its full context.

\subsubsection{Null values}

In a tab-separated table the values for adjacent fields in a given row are
separated by a tab character.  In tab-separated tables written by CURSA
null values are represented by two adjacent tab characters.  That is, no
value is included for the null field.

\subsection{STL}

File types: \texttt{.TXT  .txt}

Mixed capitalisations, such as \texttt{.Txt}, are also supported.

CURSA can read and write catalogues in the STL (Small Text List) format.
Unlike the other formats which CURSA can access the STL format is specific
to CURSA.  Nonetheless the STL format exists in order to allow
easy access to both private tables and versions of standard catalogues
held as text files.  It is usually straightforward to create an STL
catalogue from a text file containing a private list or standard catalogue.

In the STL format both the table of values for the catalogue and the
definitions of its columns, parameters \emph{etc.} are held in simple ASCII
text files.  These files may be created and modified with a text editor.
The information defining the catalogue is called the \textbf{description} of
the catalogue and the file in which it is held is called the \textbf{description file}.

When you specify a small text list you give the name of the description
file.  The table of values comprising the catalogue may either be in the
same file as the description or in a separate file.  If the table of values
occurs in a separate file then the name of this file is specified in the
description file and CURSA places no restrictions on this name other than
those imposed by the host operating system.

Appendix~\ref{STLTUT} is a simple tutorial introduction to STL
descriptions.  The basic format is described in full in
Appendix~\ref{STLREF}.  In addition to the basic STL format there is a
variant which allows STL format files to inter-operate with applications
in the KAPPA image processing package (see
\xref{SUN/95}{sun95}{}\cite{SUN95}).  This variant is described in
Appendix~\ref{STLKAP}.

CURSA can read STL format catalogues with either a free format or a
fixed-format table of values.  However, CURSA can only write STL format
catalogues with a free format table.  The KAPPA variant of the STL may be
both read and written.

As its name implies, the Small Text List format is intended for use
with relatively small catalogues and it is unsuitable for very large
catalogues.  Currently there is no upper limit to the size of catalogue
for which it can be used.  However, if you attempt to read a catalogue
containing more than 15,000 rows a warning message is issued.  A large STL
format catalogue may take a while to open for reading and CURSA may be
unable to access a very large STL catalogue\footnote{For
information, the underlying reason for this behaviour is that CURSA
attempts to memory-map work arrays to hold the columns of an STL
catalogue and then reads the table into these arrays when an input
catalogue is opened.  For a very large catalogue CURSA may be unable to
map the required arrays.}.

\subsubsection{Textual information}

The textual information for an STL catalogue comprises the entire contents
of the description. This approach makes the maximum amount of information
about the catalogue available to the user in its full context.

\subsubsection{\label{STLNULL}Null values}

The STL format provides support for null values (see Section~\ref{NULLS}).
A null value for a field in an STL table is indicated by inserting
the string `\verb-<null>-' at the appropriate place in the input file.
When CURSA reads this string it will interpret it as a null value.
Actually, if CURSA encounters any value for a field which it cannot
interpret given the data type of the column (such as a string containing
alphabetic characters in a field for an INTEGER column) then the field
is interpreted as null.  However, when preparing STL files I recommend
that you indicate nulls using the string `\verb-<null>-'.  This string
is recognised as indicating a null value even for CHARACTER columns.

When CURSA writes an STL catalogue null fields in the table are
represented by the string `\verb-<null>-'.

Null values are not permitted in the KAPPA variant of the STL format
(see Appendix~\ref{STLKAP}).


\section{\xlabel{STLTUT}\label{STLTUT}STL description tutorial}

\subsection{First example}

The easiest way to introduce the STL (Small Text List) description file
format is to explain an example.  Figure~\ref{DESCR_SIMPLE} shows a simple
description file for a small text list.  This example is available as
file:

\begin{terminalv}
/star/share/cursa/simple.TXT
\end{terminalv}

\begin{figure}[htbp]

% Ensure that this figure corresponds to the example data file.

\begin{terminalv}
!+
! Simple STL example; stellar photometry catalogue.
!
! A.C. Davenhall (Edinburgh) 24/1/97.
!-

C RA   DOUBLE  1  UNITS='RADIANS{HOURS}'    TBLFMT=HOURS
C DEC  DOUBLE  2  UNITS='RADIANS{DEGREES}'  TBLFMT=DEGREES
C V    REAL    3  UNITS='MAG'
C B_V  REAL    4  UNITS='MAG'
C U_B  REAL    5  UNITS='MAG'

P EQUINOX  CHAR*10  'J2000.0'
P EPOCH    CHAR*10  'J1996.35'

BEGINTABLE
5:09:08.7   -8:45:15   4.27  -0.19  -0.90
5:07:50.9   -5:05:11   2.79  +0.13  +0.10
5:01:26.3   -7:10:26   4.81  -0.19  -0.74
5:17:36.3   -6:50:40   3.60  -0.11  -0.47
...
\end{terminalv}

\caption{A simple STL description file\label{DESCR_SIMPLE} }

\end{figure}

In a small text list the table of values can be in the same
file as the description (as in Figure~\ref{DESCR_SIMPLE}) or in a
separate file.
% In the case of a direct access binary catalogue the
% table of values must always be in a separate file.
If the table is
included in the description file it must occur after the description,
from which it is separated by a line containing the single word `\texttt{BEGINTABLE}'.

The first five lines of Figure~\ref{DESCR_SIMPLE} are
comments.  They are ignored by CURSA and their only purpose is to
provide information to a user reading the description file.  Comments
are identified by an exclamation mark (`\texttt{!}').  In the example the
comments all occupy their own line.  However, they can be included on
the same line as other elements of the description file; any text to
the right of an exclamation mark is interpreted as a comment.

Blank lines are ignored.  They can be introduced to improve the
readability of a description file as required.

Throughout this manual keywords and directives are shown in upper
case for clarity.  However, they are actually case-insensitive.

The example in Figure~\ref{DESCR_SIMPLE} contains five columns:
\texttt{RA}, \texttt{DEC}, \texttt{V}, \texttt{B\_V} and \texttt{U\_B}.  Each column
must be defined on a separate line (if necessary the definition of a
column can be continued onto another line, though in the example none
are.  However, a single line can only contain the definition of one
column).

The definition of each column starts with the letter `\texttt{C}'
(indicating that a column is being defined), followed by the name, the
data type and the `position' in the column.  Here the `position' is
simply the sequence number of the column in the table (starting counting
at one), with the columns being separated by one or more spaces.  Further
details of the column are specified using an `\texttt{item\_name=value}'
notation; in the example the units are set in this way.  All these items
must be separated by one or more spaces.  The full syntax for defining
columns is described in Section~\ref{COL}.  For columns \texttt{RA} and
\texttt{DEC} the \texttt{UNITS} item is indicating that the columns should be
displayed as sexagesimal angles in hours and degrees respectively.
Similarly, the \texttt{TBLFMT} item is specifying that the columns are
listed as sexagesimal angles in the file.  Note that sexagesimal angles
stored in STL files must contain no embedded spaces and a colon (`:')
must be used to separate the hours or degrees, minutes and seconds.

The example contains two parameters: \texttt{EQUINOX} and \texttt{EPOCH}.
Parameters are defined in a similar way to columns.  Each column
must be defined on a separate line (if necessary the definition of a
column can be continued onto another line, though in the example none
are.  However, a single line can only contain the definition of one
parameter).

The definition of each parameter starts with the letter `\texttt{P}'
(indicating that a parameter is being defined), followed by the name, the
data type and the value of the parameter.  Further details of the
parameter can be specified using an `\texttt{item\_name=value}'
notation, as for columns (though none are set in the example).

The table of values immediately follows the `\texttt{BEGINTABLE}' line,
without any intervening lines.  The table is in `free' format, with
columns being separated by blanks.  The example does not include any
character columns with embedded blanks, but any such columns would be
enclosed in quotes.  The physical order of the columns in the table
simply corresponds to the order specified when each column was defined
in the description.  Thus, in the example the first column is \texttt{RA},
the second \texttt{DEC}, \emph{etc}.

In the example the columns of angles \texttt{RA} and \texttt{DEC} are stored
in the file as sexagesimal hours and degrees respectively.  This format
is convenient and is probably what you will usually use.  However, it is
possible to store columns of angles in a file in radians (and in fact
CURSA does this when it writes an STL catalogue).  There is an example of
such a catalogue in file:

\begin{terminalv}
/star/share/cursa/simple_radians.TXT
\end{terminalv}

\subsection{Second example}

Figure~\ref{DESCR_COMP} is an example of a more complicated description
file for an STL catalogue.  This example is available as file:

\begin{terminalv}
/star/share/cursa/complex.TXT
\end{terminalv}

It is basically similar to the description file for the small text list
in Figure~\ref{DESCR_SIMPLE}, but with the differences listed below.

\begin{figure}[htbp]

% Ensure that this figure corresponds to the example data file.

\begin{terminalv}
!+
! More complicated example of an STL.
!
! A.C. Davenhall (Edinburgh) 24/1/97.
!-

C RA      DOUBLE    1  UNITS='RADIANS{HOURS}'    TBLFMT=HOURS9
C DEC     DOUBLE   12  UNITS='RADIANS{DEGREES}'  TBLFMT=DEGREES8
C NAME    CHAR*10  25  COMMENTS='Star name.'     TBLFMT=A7
C FLUX[4] REAL     31  UNITS='Jansky'            EXFMT=F6.1
:           COMMENTS='Flux at 12, 25, 60 and 100 micron.'
C V       REAL     58  UNITS='MAG'  COMMENTS='V magnitude.' EXFMT=F4.2
C B_V     REAL     64  UNITS='MAG'  COMMENTS='B-V colour.'  EXFMT=F5.2
C U_B     REAL     71  UNITS='MAG'  COMMENTS='U-B colour.'  EXFMT=F5.2

P EQUINOX  CHAR*10  'J2000.0'
P EPOCH    CHAR*10  'J1996.35'

T Catalogue of U,B,V colours and fluxes.
T
T UBV photometry from Mount Pumpkin Observatory,
T IR Fluxes from Sage, Parsley and Thyme (1987).

D FILE='complex.dat'      ! File holding the table.
D POSITION=CHARACTER      ! Table is fixed-format.
\end{terminalv}

\caption{A more complicated STL description file\label{DESCR_COMP} }

\end{figure}

\begin{itemize}

  \item Here the `position' of each
   column (the third item of information given for each column) is
   \textit{not}\, a simple sequence number, but rather is the position
   of the first character associated with the column in each record of the
   table (starting counting at one).
%   This arrangement is necessary to
%   locate columns in a direct access file.  Optionally positions can
%   be specified in this fashion for small lists, if desired (see
%   Section~\ref{DIR}).
   The directive \texttt{POSITION=CHARACTER} indicates that positions are
   specified in this way.  This option is particularly useful for
   reading existing fixed-format files.  If a fixed-format STL is
   being read then a format must be specified for each column using
   either \texttt{TBLFMT} or \texttt{EXFMT}.  Similarly, if a sexagesimal
   angle in hours or degrees is being read from a fixed-format STL
   then the total width of the column must be appended to the units
   of \texttt{HOURS} or \texttt{DEGREES} specified for the \texttt{TBLFMT}
   item.

  \item Column \texttt{FLUX} is a four-element vector.  Vector columns
   are defined using the usual CURSA notation: appending the number
   of elements in the vector, enclosed in square brackets (`\texttt{[]}'),
   after the column name.
%   Note that because \texttt{FLUX} is a four-element \texttt{REAL} vector it
%   occupies sixteen bytes in each record.

  \item The details specified for \texttt{FLUX} are continued on a second
   line.  If the first non-blank character in a line is a colon (`\texttt{:}') then the line continues the definition on the previous line.
   The colon must be followed by at least one space.  An arbitrary
   number of continuation lines are allowed.

  \item The lines beginning with `\texttt{T}' are lines of textual
   information associated with the catalogue.  They are processed by
   CURSA in the order in which they appear.  Note that the `\texttt{T}'
   must be followed by at least one space.

  \item The lines beginning with `\texttt{D}' define additional directives
   associated with the catalogue (see Section~\ref{DIR}).  There
   must be at least one space following the `\texttt{D}'.  In the
   example each directive occurs on its own line.  However, an arbitrary
   number can be included on a single line, if required (though if more
   than one are included on a line they must be separated by one or
   more spaces).  Note also the use of in-line comments in these lines;
   the text to the right of the exclamation marks is a comment and is
   ignored.
%   The three directives shown are mandatory for a direct access
%   binary catalogue.

   \texttt{FILE} is the name and directory specification of the file
   holding the table of values comprising the catalogue.  It is
   expressed using the syntax of the host operating system and there
   are no restrictions on it other than those imposed by the host
   operating system.
%   For a direct access binary catalogue the table
%   must be in a separate file.  For a small text list it may optionally
%   be in a separate file, in which case the file name is specified
%   using the identical \texttt{FILE} mechanism.

%   \texttt{RECORDSIZE} is the size in bytes of each record (or row) in the
%   table file and \texttt{ROWS} is the number of rows (or records) in the
%   table.  Though both these directives are mandatory for direct access
%   binary catalogues neither is needed for small text lists.

\end{itemize}


\section{\xlabel{STLREF}\label{STLREF}STL description reference}

\subsection{Basics}

Description files are text files which can be created and modified
with an editor.  They have the following properties:

\begin{itemize}

  \item they are free format; there is no requirement that items occur
   at fixed absolute positions in lines,

  \item the space character is used as a separator; items in the
   description \textit{must}\, be separated by one or more spaces,

  \item keywords are case-insensitive (though throughout this manual they
   are shown in upper-case for clarity),

  \item blank lines are ignored; they may be introduced freely to
   improve readability as required.

\end{itemize}

A CURSA catalogue comprises: columns, parameters and textual information
(see Section~\ref{COMP}).  All these items are defined in the description
file.  It can also contain \textbf{directives} which provide additional
information.  Each column, parameter, line of textual information and set
of directives occupies one or more contiguous lines of the description
file.  The components can occur in any order.

The first non-blank character of a line determines the type of component
it contains, according to the following scheme:

\begin{tabular}{lcl}
\texttt{C} & -- & column,              \\
\texttt{P} & -- & parameter,           \\
\texttt{T} & -- & textual information, \\
\texttt{D} & -- & directive,           \\
\texttt{:} & -- & continuation of the preceding line.  \\
\end{tabular}

These characters do not have to occur at the start of a line; they can
be preceded by (and only by) an arbitrary number of spaces.  The single
character is all that is required to specify the type of component.
However, it can be part of a word if required for clarity, as long as
the word starts with the correct letter.  For example, `\texttt{COLUMN}'
could be used instead of `\texttt{C}' to introduce a column.

Columns and parameters have the following general format:

\begin{quote}
\texttt{C} or \texttt{P} ~~~ mandatory items ~~~ optional items
\end{quote}

All items must be separated by one or more spaces.  The
mandatory items must be supplied.  They occur in a fixed order and
only the value is given.  The optional items usually
correspond to attributes of the column or parameter.  They may be
supplied if required; if they are omitted defaults are adopted.
Optional items are specified using the notation:

\begin{terminalv}
item_name=value
\end{terminalv}

Spaces are not permitted between the item name, equals sign and value.
The mandatory and optional items for columns and parameters are
described in Sections~\ref{COL} and \ref{PAR} respectively.

Textual information has the following format:

\begin{quote}
\texttt{T} ~~~ line of text
\end{quote}

Note that there must be one or more spaces between the `\texttt{T}' (or
word beginning with `\texttt{T}') and the line of textual information.
CURSA accesses lines of textual information in the order in which
they are entered into the description file.

Sets of directives have the format:

\begin{quote}
\texttt{D} ~~~ directives
\end{quote}

An arbitrary number of directives can be specified on each line.  Each
directive is specified using the notation:

\begin{terminalv}
directive_name=value
\end{terminalv}

Spaces are not permitted between the directive name, equals sign and
value.  The various directives are listed in Section~\ref{DIR}.

\subsubsection{Continuation lines}

A colon (`\texttt{:}') as the first non-blank character of a line indicates
that it is continuing a definition begun on a previous line.  An
arbitrary number of spaces can precede the colon and at least one
must follow it.  An unlimited number of continuation lines are allowed.

\subsubsection{Strings}

Strings which include spaces (for example, perhaps the units or comments
attribute of a column) must be enclosed in single or double quotes.  The
quotes must be `matching': a string started with a single quote must be
ended with a single quote and similarly for a double quote.  A double quote
can be included within a string terminated with single quotes and vice
versa.  Strings which do not include embedded spaces may optionally be
enclosed in quotes, but they do not need to be.

\subsubsection{Comments}

Any text following an exclamation mark (`\texttt{!}') is treated as a
comment and ignored.  The exclamation mark introducing a comment may
be either the first non-blank item in a line (`comment lines') or
may follow other items (`in-line comments').  Comments are terminated
at the end of the line.

An exclamation mark within a string terminated with quotes is not
interpreted as a comment, but is considered part of the string.


\subsection{Columns \label{COL}}

Columns have the following format:

\begin{quote}
\texttt{C} ~~~ name ~~~ data type ~~~ position ~~~ (optional items)
\end{quote}

Values for the name, data type and position are mandatory and values
must be given in the order indicated.  An arbitrary number of optional
items may be specified using the notation `\texttt{item\_name=value}'.
All items must be separated by one or more spaces.  The individual
items are described below.

\subsubsection{Mandatory items}

\paragraph{Name}  The name of the column must conform to the usual
CURSA rules for column names.  Vector columns are indicated by using
the usual notation: the number of elements is given in square brackets
after the name.  For example \texttt{FLUX[4]} indicates a four element
vector.

\paragraph{Data type}  The permitted data types are listed in
Table~\ref{DATA_TYPES}.  Note that for character columns the size of
the character string is indicated by the number at the end of the string
(following the usual Fortran syntax).

\begin{table}[htbp]

\begin{center}
\begin{tabular}{llc}
CURSA Data Type & Description      & Standard \\
                &                  & Fortran 77? \\ \hline
BYTE            & Signed byte      & No   \\
WORD            & Signed word      & No   \\
INTEGER         & Signed integer   & Yes  \\
REAL            & Single precision & Yes  \\
DOUBLE          & Double precision & Yes  \\
LOGICAL         & Logical          & Yes  \\
CHAR[$*n$]      & Character string & Yes  \\
\end{tabular}

\begin{quote}
$n$ is the number of elements in the character string; it is a positive
integer.
\end{quote}

\caption{Permitted data types \label{DATA_TYPES} }
\end{center}

\end{table}

\paragraph{Position}  The position in the table where the column is
located.  For small text lists positions may be specified as either the
sequence number of the column in the table or the sequence number of the
first character corresponding to the column in each row, depending
on the setting of directive \texttt{POSITION} (see Section~\ref{DIR}).
In both cases counting starts at one.
% For
% a direct access binary catalogue it must be the position of the first
% byte corresponding to the column in each input record.

\subsubsection{\label{COLOPT}Optional items}

The optional items are listed in Table~\ref{COL_OPT} and are described
below.  Most optional items correspond directly to an attribute of the
column (see Section~\ref{COLS} and Table~\ref{COLUMN_ATT}).  If they are not
specified a default is adopted.

\begin{table}[htbp]

\begin{center}
\begin{tabular}{ll}
Item           & Description                         \\ \hline
\texttt{ORDER}    & order of the column                 \\
\texttt{UNITS}    & units                               \\
\texttt{EXFMT}    & external format for display         \\
\texttt{PREFDISP} & preferential display flag           \\
\texttt{COMMENTS} & descriptive comments                \\
\texttt{SCALEF}   & scale factor                        \\
\texttt{ZEROP}    & zero point                          \\
\texttt{TBLFMT}   & format in the table                 \\
% \textsf{NULL}     & type of null (to be added later)    \\
% \textsf{EXCEPT}   & exception value (to be added later) \\
\end{tabular}

\caption{Optional items for columns \label{COL_OPT} }

\end{center}
\end{table}

\paragraph{\texttt{ORDER}} The order of the column.  The permitted values
are: \texttt{ASCENDING}, \texttt{DESCENDING} and \texttt{UNORDERED}.  The default
is \texttt{UNORDERED}.

\paragraph{\texttt{UNITS}} The units of the column.  The default is a
blank string (corresponding to no units).  Columns of angles may be
represented internally within a CURSA application as radians and displayed
as hours, degrees, minutes or seconds with optional sexagesimal
subdivision using the notation described in Appendix~\ref{ANGLE}.

\paragraph{\texttt{EXFMT}}  The external format of the column; a Fortran 77
format specifier valid for the data type of the column which will be
used to display it.  The default depends on the data type.

\paragraph{\texttt{PREFDISP}}  The preferential display flag, indicating
whether the column will be displayed by default.
% by applications such as \texttt{xcatview}.
The permitted values are \texttt{TRUE} and \texttt{FALSE}.
The default is \texttt{TRUE}.

\paragraph{\texttt{COMMENTS}}  Comments describing the column.  The default
is a blank string (corresponding to no comments).

\paragraph{\texttt{SCALEF}} The scale factor used to calculate the actual
value of a scaled column.  For a scaled column the actual value of each
field is computed by applying a scale factor and zero point to the value
stored in the table, according to the formula:

\begin{equation}
{\rm actual~value} =
                ( \texttt{SCALEF} \times {\rm stored~value} ) + \texttt{ZEROP}
\end{equation}

An example of an STL catalogue containing scaled columns is available in
file:

\begin{terminalv}
/star/share/cursa/scale.TXT
\end{terminalv}

\paragraph{\texttt{ZEROP}} The zero point used to calculate the actual value
of a scaled column from the scaled value stored. See above for the
formula used.

\paragraph{\texttt{TBLFMT}} This item is not an attribute of the column,
rather it is the format to be used to read the column from the table
in small text lists.  It will usually be a Fortran 77 format specifier
valid for the data type of the column, though some special forms are
provided for reading sexagesimal angles.  These special forms are
described in Section~\ref{TBLFMT_ANG}, below.  If \texttt{TBLFMT} is
omitted then it defaults to the value of \texttt{EXFMT} for the column.

\subsubsection{\label{TBLFMT_ANG}Storing sexagesimal angles}

Columns of angles may be stored formatted as sexagesimal hours or degrees
or as minutes or seconds of arc or time in an STL catalogue.  These
options both make the catalogues much easier to read by eye and allow
STL descriptions to be prepared for many existing catalogues which are
held as text files.

The \texttt{TBLFMT} item for a column in a catalogue is usually the
Fortran 77 format specifier to read the column (see
Section~\ref{COLOPT} above).  However, it has some special values to
describe sexagesimal angles.  These special values divide into two
categories, one suitable for simple angles and the other covering more
complex cases.  In a simple angle a colon (`\texttt{:}') is used to separate
the sexagesimal components.  For complex angles the separator can be a
space, or any other character, or indeed there may be no separator
at all.  The facilities for complex angles can handle most of the
formats used in practice to represent angles in astronomical catalogues
formatted as text files.  The simple and complex options are described
separately below.

\paragraph{Simple sexagesimal angles}

\begin{table}[htbp]

\begin{center}
\begin{tabular}{llrl}
\texttt{TBLFMT}   & Units           & Example         & Corresponding \\
specifier      &                 &                 & angle         \\ \hline
\texttt{DEGREES}  & degrees         & \texttt{30:00:00}  & 30$^{\circ}$  \\
\texttt{HOURS}    & hours           & \texttt{2:00:00}   & 30$^{\circ}$ or $2^{{\rm h}}$  \\
\texttt{ANGLE}    & varies \dag     & \texttt{+30:00:00} & 30$^{\circ}$  \\
               &                 &                 &           \\
\texttt{ARCMIN}   & minutes of arc  & \texttt{30:00}     & 30\arcmin \\
\texttt{ARCSEC}   & seconds of arc  & \texttt{30.0}      & 30\arcsec \\
\texttt{TIMEMIN}  & minutes of time & \texttt{30}        & $30^{{\rm m}}$ \\
\texttt{TIMESEC}  & seconds of time & \texttt{30.0}      & $30^{{\rm s}}$ \\
\end{tabular}

\vspace{7mm}

\begin{tabular}{l}
\dag - signed values interpreted as degrees, unsigned values as hours.
\end{tabular}
\end{center}

\caption{\label{STLTBL_ANG}Values of column item \texttt{TBLFMT} for
sexagesimal angles in tables}

\end{table}

For simple sexagesimal angles the \texttt{TBLFMT} item has the form:

\begin{quote}
\texttt{TBLFMT=}\textit{units}
\end{quote}

where \textit{units}\/ is the units of the angle.  The permitted values are
shown in Table~\ref{STLTBL_ANG}.  For example, an angle tabulated in
units of degrees would be represented as:

\begin{terminalv}
TBLFMT=DEGREES
\end{terminalv}

The angles tabulated must use a colon as a sexagesimal separator (as shown
in Table~\ref{STLTBL_ANG}).  Columns of angles stored in this form must
obey the following constraints:

\begin{itemize}

  \item they should be in units of hours, degrees, minutes of arc or
   time or seconds of arc or time,

  \item they should contain no embedded spaces,

  \item a colon (`:') should be used to separate the hours or degrees,
   minutes and seconds,

  \item if the units are hours or degrees then optionally either the
   seconds or the minutes and seconds may be omitted,

  \item small angles expressed in minutes of arc or time may optionally
   be subdivided into either seconds (with a colon as a separator) or
   decimal minutes,

  \item small angles expressed in seconds of arc or time may be
   represented either with or without a decimal fraction.

\end{itemize}

These simple sexagesimal formats are suitable for use in both
free-format and fixed-format STLs.  Indeed, they are the only way of
representing sexagesimal angles in free-format STLs.

If a fixed-format STL is being read then the total width of the column
(in characters) must be appended to the specifier.
Figure~\ref{DESCR_COMP} shows an example of this option; here column
\texttt{RA} has a width of nine characters and column \texttt{DEC} a width
of eight.  The following files contain examples of the use of these
options:

\begin{terminalv}
/star/share/cursa/simple.TXT
/star/share/cursa/complex.TXT
/star/share/cursa/propmotn.TXT
\end{terminalv}

\paragraph{Complex sexagesimal angles}

The \texttt{TBLFMT} item has additional options for reading more complex
sexagesimal angles from STL catalogues.  These options cover most of the
formats used in practice to represent angles in astronomical catalogues
held as text files.  They should \textit{only}\/ be used in fixed-format
STLs; if they are used in free-format STLs the results are
unpredictable.  For complex sexagesimal angles the \texttt{TBLFMT} item
has the form:

\begin{quote}
\texttt{TBLFMT=}\textit{units}\/\texttt{\{}\textit{element\_descriptors}\/\texttt{\}}
\end{quote}

\textit{units}\/ is the units of the angle, as for simple sexagesimal
angles.  Again the permitted values are as listed in
Table~\ref{STLTBL_ANG}.  \textit{element\_descriptors}\/ is a series of
Fortran-like descriptors for the individual components of the
sexagesimal angle.  A sexagesimal angle is allowed to comprise up
to four components:

\begin{itemize}

  \item an optional separate sign,

  \item the `main component'; the integer part of the angle in the
   specified units.  Here this component is called the \textbf{quotient}
   (following more-or-less the usual usage of the term).  This
   component is mandatory,

  \item either one or two sexagesimal subdivisions.  These components
   are optional; the format for a given tabulated angle may contain
   zero, one or two sexagesimal subdivisions.

\end{itemize}

The descriptors used to read these components are very similar to
the descriptors used in Fortran FORMAT statements.  In the following
$n$\/ is the total number of characters occupied by the item and
$m$\/ the number of decimal places.  Both $n$\/ and $m$\/ are
positive integers.  The following rules apply.

\begin{itemize}

  \item The descriptor for a separate sign is of the form
   \texttt{A}\textit{n}.

  \item All the numeric components (the quotient and any sexagesimal
   subdivisions) have a descriptor of the form \texttt{I}\textit{n}\/ for
   INTEGER values or \texttt{F}\textit{n.m}\/ for REAL ones.

  \item Spaces, or any other separator characters, can be skipped by
   descriptors of the form \textit{n}\/\texttt{X}.

  \item All components are separated by a comma (`\texttt{,}'), again
   \textit{cf}\/ Fortran FORMAT statements.

\end{itemize}

You simply assemble an appropriate set of descriptors to describe a
given angular format.  The only real restriction is that the quotient
and any sexagesimal subdivisions must occur in order of decreasing size
(that is, quotient first, least significant subdivision last).  However,
it is very unusual for sexagesimal angles to be tabulated in any other
order.  The following additional points apply to the optional separate
sign.

\begin{itemize}

  \item It can occur anywhere in the format; it does not have to be
   the first component.

  \item Positive angles are indicated by any of the following
   characters: \texttt{+}, \texttt{n} or \texttt{N} (\texttt{N} for north).
   Negative angles are indicated by any of the following characters:
   \texttt{-}, \texttt{s} or \texttt{S} (\texttt{S} for south).

  \item The separate sign is optional.  Negative values can also be
   indicated by including a minus sign (`\texttt{-}') as the first
   character of the quotient (\textit{cf}\/ the usual rules for reading
   numbers in Fortran).

\end{itemize}

Figure~\ref{STLCOMPANG} shows an example of an STL format catalogue
containing several columns of complex sexagesimal angles.  This
catalogue is available as file:

\begin{terminalv}
/star/share/cursa/angles.TXT
\end{terminalv}

\begin{figure}[htbp]

% Ensure that this figure corresponds to the example catalogue.

\begin{terminalv}
!+
! STL catalogue showing examples of complex sexagesimal angle formats.
!
! A.C. Davenhall (Edinburgh) 4/8/98.
!-

C ANGLE1  DOUBLE   3  UNITS='RADIANS{DEGREES}'
:  TBLFMT=DEGREES{A1,I2,1X,I2,1X,I2}

C ANGLE2  DOUBLE  15  UNITS='RADIANS{DEGREES}'
:  TBLFMT=DEGREES{A1,I2,I2,I2}

C ANGLE3  DOUBLE  25  UNITS='RADIANS{BDMS.2}'
:  TBLFMT=DEGREES{A1,I2,1X,I2,1X,F5.2}

C ANGLE4  DOUBLE  40  UNITS='RADIANS{HM.1}'
:  TBLFMT=HOURS{I2,1X,F4.1}

C ANGLE5  DOUBLE  50  UNITS='RADIANS{D.2}'
:  TBLFMT=DEGREES{F6.2,2X,A1}

C ANGLE6  DOUBLE  61  UNITS='RADIANS{ARCMIN.1}'
:  TBLFMT=ARCMIN{F6.1}

D POSITION=CHARACTER  ! Table is fixed format.

! Notes.
! (1) The complex sexagesimal angle-formats can only be used in
!     fixed-format STL tables.
! (2) The last two rows of the table show various illegal cases
!     which CURSA interprets as null values.

!    ANGLE1    ANGLE2         ANGLE3    ANGLE4      ANGLE5  ANGLE6
!        10        20        30        40        50        60
! 3456789 123456789 123456789 123456789 123456789 123456789 123456789
BEGINTABLE
   30 30 30    303030    30 30 30.12    6 34.5    30.12  N    23.1
  N30:25  0   N3025 0   N30 25  0.34    8 56.7   178.34       17.5
  n 6 23,45   n 62345   n 6 23 45.45   14 02.0    45.45  +   -45.6
  + 3  3  0   + 3 3 0   + 3  3  0.56    4 23.6    56.56      +23.4
  -30 00 00   -300000   -30 00 00.67    5 45.2    40.67  -  -123.4
  S25a57 00   S255700   S25 57 00.78   17 42.1    73.78  S    55.6
  s40 00q37   s400037   s40 00 37.90   18 19.5   123.90  s    34.7
  S25 67 00    256700    25 67 00.01    4 60.1   <null>        bad
  S25 00 60    250060    25 00 60.12    1 60.0   <null>       55.x
\end{terminalv}

\begin{quote}
\caption[Example STL format catalogue with complex sexagesimal angles]
{An example STL format catalogue containing columns of complex sexagesimal
angles\label{STLCOMPANG} }
\end{quote}

\end{figure}

The interpretation of the \texttt{TBLFMT} items for these angles is quite
straightforward.  For example, column \texttt{ANGLE1} starts in the third
character of each record and has units of degrees.  It has a separate
sign as its first character.  The quotient degrees, minutes and
seconds are all two-character INTEGER values and are separated by
one space (or rather by any single character).

\subsection{Parameters \label{PAR}}

Parameters have the following format:

\begin{quote}
\texttt{P} ~~~ name ~~~ data type ~~~ value ~~~ (optional items)
\end{quote}

Values for the name, data type and value are mandatory and values
must be given in the order indicated.  An arbitrary number of optional
items may be specified using the notation `\texttt{item\_name=value}'.
All items must be separated by one or more spaces.  The individual
items are described below.

\subsubsection{Mandatory items}

The name and data type are the name and data type of the parameter
respectively.  They are specified as in exactly the same way as the
corresponding items for columns (see Section~\ref{COL}).  Value is the
value of the parameter.  If it is a character string containing spaces
it must be enclosed in quotes.

\subsubsection{Optional items}

The optional items are listed in Table~\ref{PAR_OPT}. Their details
are exactly the same as the corresponding optional items for columns
(see Section~\ref{COL}).

\begin{table}[htbp]

\begin{center}
\begin{tabular}{ll}
Item           & Description                   \\ \hline
\texttt{UNITS}    & units                         \\
\texttt{EXFMT}    & external format for display   \\
\texttt{PREFDISP} & preferential display flag     \\
\texttt{COMMENTS} & descriptive comments          \\
\end{tabular}

\caption{Optional items for parameters \label{PAR_OPT} }

\end{center}
\end{table}

\subsection{Directives \label{DIR}}

Sets of directives have the following format:

\begin{quote}
\texttt{D} ~~~ directives
\end{quote}

An arbitrary number of directives can be included on each line.  They
must be separated by one or more spaces.  Each directive is specified
using the notation `\texttt{directive\_name=value}'.  The individual
directives are listed in Table~\ref{DIRECT} and described below.

\begin{table}[htbp]

\begin{center}
\begin{tabular}{llc}
Directive  & Description               & Default \\ \hline
\texttt{FILE} & name of the file containing the table       & \S           \\
\texttt{POSITION}   & method of specifying column positions & \texttt{COLUMN} \\
\texttt{SKIP} & number of header records to skip            & 0            \\
\end{tabular}

\vspace{1cm}

\begin{tabular}{rcl}
 \S        & - & either specify \texttt{FILE} or include the table in the description file \\
\end{tabular}

\caption{Directives \label{DIRECT} }
\end{center}

\end{table}

% \begin{table}[htbp]
%
% \begin{center}
% \begin{tabular}{llcc}
% Directive  & Description  & \multicolumn{2}{c}{Default} \\
%            &                                  & STL\dag & DABC\ddag \\  \hline
% \texttt{FILE} & name of the file containing the table & ($\bullet$)\S & $\bullet$* \\
% \texttt{POSITION}   & method of specifying column positions & \texttt{COLUMN} & -- \\
% \texttt{RECORDSIZE} & Size of each record in bytes    & -- & $\bullet$ \\
% \texttt{ROWS} & number of rows in the catalogue       & -- & $\bullet$ \\
% \texttt{SKIP} & number of header records to skip      & 0  & 0         \\
% \end{tabular}
%
% \vspace{1cm}
%
% \begin{tabular}{rcl}
%  $\bullet$ & - & mandatory                      \\
%  --        & - & not needed and ignored         \\
%  \dag      & - & small text list                \\
%  \ddag     & - & direct access binary catalogue \\
%  \S        & - & either specify \texttt{FILE} or include the table in the description file \\
%  ~*        & - & but see footnote~\ref{RECORDLENGTH} on including the table in the description file \\
% \end{tabular}
%
% \caption{Directives \label{DIRECT} }
% \end{center}
%
% \end{table}

\paragraph{\texttt{FILE}} The name of the file holding the catalogue table
in the case where it is not held in the same file as the description.
The file name may optionally be preceded by a directory specification.
The assemblage of the file name and directory specification are
expressed in the syntax of the host operating system.  To be pedantic
this specification means that description files are not portable
across operating systems (and, indeed, across different machines
running the same operating system).  However, this restriction is
unlikely to be important in practice.  If a file name is supplied
without a directory specification it is assumed to reside in the same
directory as the description file.

\paragraph{\texttt{POSITION}} The way in which the location of columns in
the table are specified in a small text list.  The options are:

\begin{description}

  \item[\texttt{~~COLUMN}] Each column is identified by a sequence number
   (starting at one).  This method is suitable for free format small
   text lists.

  \item[\texttt{~~CHARACTER}] Each column is identified by the
   sequence number of the first character (starting at one)
   corresponding to it in each line, record or row of the table.  This
   option is suitable for
%   both
   fixed format small text files.
%   and direct access binary catalogues.
   Characters in the input lines are counted starting at one.

\end{description}

% \paragraph{\texttt{RECORDSIZE}} The size of each record, in bytes, of a
% direct access binary catalogue.

% \paragraph{\texttt{ROWS}} The number of rows in a direct access binary
% catalogue.

\paragraph{\texttt{SKIP}} The number of lines or records to skip at the
start of the table.  It is intended for skipping over `header' records.
The default is zero.


\section{\xlabel{STLKAP}\label{STLKAP}KAPPA format STL}

CURSA also supports a variant of the STL format which allows STL
catalogues to inter-operate with applications in the KAPPA image
processing package (see \xref{SUN/95}{sun95}{}\cite{SUN95}).

The KAPPA variant of the STL format is very similar to the standard
form but a `\texttt{\#}' character is used instead of an `\texttt{!}' to
introduce comments and the lines defining columns, parameters \emph{etc.}
begin with `\texttt{\#C~}', `\texttt{\#P~}' \emph{etc.} respectively.  The
KAPPA variant versions are listed in full in Table~\ref{KAP_ITEM}.

\begin{table}[htbp]

\begin{center}
\begin{tabular}{ccl}
KAPPA variant  &  Standard form & Description \\ \hline
\texttt{\#}        & \texttt{!}        & Comment \\
\texttt{\#C}       & \texttt{C}        & Column \\
\texttt{\#P}       & \texttt{P}        & Parameter \\
\texttt{\#T}       & \texttt{T}        & Textual information \\
\texttt{\#D}       & \texttt{D}        & Directive \\
\texttt{\#:}       & \texttt{:}        & Continuation \\
\texttt{\#BEGINTABLE} & \texttt{BEGINTABLE} & Start of table \\

\end{tabular}
\end{center}

\caption{\label{KAP_ITEM}Items in a KAPPA format STL}

\end{table}

A `\texttt{\#}' used to introduce a comment \textit{must}\, be followed by at
least one blank space.  Currently null values are \textit{not}\, permitted
in the table of values for a KAPPA format STL.  In all other respects a
KAPPA format STL behaves like a standard one.  Blank lines are permitted
in a KAPPA format STL description.

KAPPA format STLs have the same file types as standard ones: \texttt{.TXT}
or \texttt{.txt}.  In fact the standard and KAPPA forms can be mixed freely
in an input STL catalogue, though I do not recommend that you do this
because the result looks rather untidy.

By default CURSA writes standard STLs.  It can be made to write a KAPPA
format STL by appending `\texttt{KAPPA}' inside curly brackets after the
name of the file\footnote{This convention is just the usual CURSA
syntax for specifying extra information about a catalogue; \textit{cf}\/
reading FITS tables.}.  For example, to write a KAPPA format STL called
\texttt{perseus.TXT} you would specify:

\begin{terminalv}
perseus.TXT{KAPPA}
\end{terminalv}

`\texttt{KAPPA}' may be abbreviated down to just `\texttt{K}' and may be given
in either case.  Also the closing curly bracket is optional.  An example
KAPPA format STL is available in file:

\begin{terminalv}
/star/share/cursa/kappa.TXT
\end{terminalv}

\subsection{Inter-operability with KAPPA}

Catalogues written in the KAPPA variant STL format permit a limited
degree of inter-operability between CURSA and KAPPA.  Currently the
KAPPA applications which access tables read them as ASCII text files.
Typically these files can contain header comments beginning with a
`\texttt{\#}' character.  This format is consistent with the KAPPA variant
STL, but KAPPA does not `know' that it is reading STL format files.

A table written by a KAPPA application typically consists of just the
table of values, with one row per line and the fields separated by one
or more spaces.  Before such a table can be accessed with CURSA you must
create a description for it.  Either the description can be edited into
the start of the file (the example in file \texttt{/star/share/cursa/kappa.TXT} was created in this way) or the
description can be in a separate file, as described in
Appendices~\ref{STLTUT} and \ref{STLREF}.

If an STL catalogue is to be written by CURSA and subsequently accessed
with KAPPA then it \textit{must}\, be written in the STL variant format.
Also, it must not contain any null values because the KAPPA applications
are not able to interpret them.

It is possible that future versions of KAPPA may use the full STL format
in which case a greater degree of inter-operability will be possible.


\section{\xlabel{PISA}\label{PISA}Inter-operability with PISA}

PISA (Position, Intensity and Shape Analysis) is a Starlink package for
detecting objects in two-dimensional image frames and determining
the properties which characterise them (position, intensity, ellipticity,
orientation \emph{etc}\/).  It is documented in
\xref{SUN/109}{sun109}{}\cite{SUN109}.  PISA lists the details of the
objects which it has detected in simple text files.  A limited degree of
inter-operability with CURSA is possible by preparing a suitable description
of these files so that they can be accessed as STL format catalogues (see
Appendix~\ref{FORMAT}).

PISA generates two principal output files:

\begin{itemize}

  \item the results file, containing details of the objects detected
   in the image frame (position, intensity \emph{etc}\/).  The default name
   of this file is \texttt{pisafind.dat},

  \item the sizes file, containing a set of areal profiles (that is,
   the number of pixels detected within each of a set of intensity
   thresholds) for the objects detected in the image frame.  The default
   name of this file is \texttt{pisasize.dat}.

\end{itemize}

Template STL description files for these two types of file are
available respectively as files:

\begin{terminalv}
/star/share/cursa/pisaresults.TXT
/star/share/cursa/pisasizes.TXT
\end{terminalv}

To access given PISA files simply take copies of the templates.  If
you have used the default file names for the PISA files then you will
be able to simply use the templates without modification.  However,
if you specified your own file names you will need to edit your copies
of the templates to refer to your files.  Comments in the templates
describe the changes, which are trivial; all that is needed is to
change the \texttt{FILE} directive to specify your files.

Note that the columns in the PISA files do not change, so you should
not need to alter the column definitions in the templates.  If an
object does not contain any pixels at a given intensity threshold
(because it is too faint) then the corresponding field in the sizes
file is set to -1.0.



\section{Detailed description of applications}

This appendix gives detailed descriptions of all the CURSA applications.

% -- Application descriptions ------------------------------------------
% \input{sun190help.tex}
\newpage
\sstroutine{
   CATCDSIN
}{
   Convert a CDS ReadMe file into a CURSA STL description file
}{
   \sstdescription{
      This application converts a CDS ReadMe file into a CURSA STL
      description file.

      The text versions of catalogues obtained from the CDS are usually
      accompanied by a description file which documents their contents.
      This description file is usually called ReadMe and contains a
      description in a standard form.  catcdsin interprets the contents
      of a CDS ReadMe file and uses them to construct a CURSA STL
      description file for the catalogue.  catcdsin does not alter the
      catalogue data file itself: both the ReadMe file and the STL
      description file constructed from it refer to the same catalogue
      file.

      CDS ReadMe files can (and often do) contain descriptions of more
      than one catalogue (usually these will be closely related
      catalogues or tables; perhaps a main catalogue and a table of
      notes).  catcdsin creates a separate STL description file for
      every catalogue found in the ReadMe file.

      The names of the catalogue files are included in the ReadMe
      file and STL description file names are constructed from them;
      the user cannot specify the description file names.  However,
      there are several options which can be specified.  By default
      catcdsin reads a file called ReadMe, though a different name can
      be given.  By default catcdsin attempts to interpret columns
      of angles in the ReadMe file and construct valid STL angular
      column descriptions from them, though this behaviour can be
      suppressed.  Optionally, parameters specifying the equinox and
      epoch of the coordinates can be added to the STL description
      files.
   }
   \sstusage{
      catcdsin
   }
   \sstparameters{
      \sstsubsection{
         INFILE  =  CHARACTER  (read)
      }{
         The name of the CDS ReadMe (or description) file which is
         to be processed (default = ReadMe).
      }
      \sstsubsection{
         ANGLES  =  CHARACTER  (read)
      }{
         Determines whether STL angular column descriptions (which
         CURSA can interpret) are to be constructed from angular
         columns in the CDS description.  Typically the latter will
         comprise separate columns for the sexagesimal hours, degrees,
         minutes and seconds.  The options are:
         Y  -  yes; construct STL angular column descriptions (default),
         N  -  no; copy the columns unaltered from the CDS ReadMe file.
      }
      \sstsubsection{
         EQUINOX  =  CHARACTER  (read)
      }{
         Equinox of the catalogue coordinates.  If specified, an EQUINOX
         parameter with the given value is written to the STL description.
         If omitted no EQUINOX parameter is written.
      }
      \sstsubsection{
         EPOCH  =  CHARACTER  (read)
      }{
         Epoch of the catalogue coordinates.  If specified, an EPOCH
         parameter with the given value is written to the STL description.
         If omitted no EPOCH parameter is written.
      }
      \sstsubsection{
         TEXT  =  CHARACTER  (read)
      }{
         Flag indicating whether the entire CDS ReadMe file is to be
         copied to the STL description as textual information.  The
         valid responses are:
         A or C -  all or comments; copy the ReadMe file (default),
         N  -  none; do not copy the ReadMe file.
      }
   }
   \sstexamples{
      \sstexamplesubsection{
         catcdsin
      }{
         A CDS ReadMe file called ReadMe will be processed.  catcdsin
         will attempt to construct STL angular column descriptions
         from any columns of angles in the file.
      }
      \sstexamplesubsection{
         catcdsin  infile=cdsdesc.lis
      }{
         A CDS ReadMe file called cdsdesc.lis will be processed.
      }
      \sstexamplesubsection{
         catcdsin  angles=no
      }{
         File ReadMe will be processed, but the column descriptions will
         be copied unaltered, with no attempt to interpret columns of
         angles.
      }
      \sstexamplesubsection{
         catcdsin  equinox=J2000  epoch=J1995.3
      }{
         File ReadMe will be processed.  Parameters corresponding to
         the given equinox and epoch will be written to the STL
         description file.  Note that either, both or neither of the
         equinox and epoch can be specified.
      }
      \sstexamplesubsection{
         catcdsin  text=none
      }{
         File ReadMe will be processed.  However, the ReadMe file will
         not be copied to the STL description as textual information.
      }
   }
}
\newpage
\sstroutine{
   CATCHART
}{
   Plot a one or more target lists as a finding chart
}{
   \sstdescription{
      Plot a one or more target lists as a finding chart.  The lists
      are plotted using a tangent plane projection.  If several lists
      are plotted they are superimposed on a single finding chart.  In
      this case the coordinates for all the lists must be for the same
      equinox and epoch.
   }
   \sstusage{
      catchart
   }
   \sstparameters{
      \sstsubsection{
         GRPHDV  =  CHARACTER (read)
      }{
         The name of the graphics device on which the plot will be
         produced.
      }
      \sstsubsection{
         MCENTRE  =  LOGICAL (read)
      }{
         A flag indicating whether the centre of the plot will be marked
         with a {\tt '}gun sight{\tt '} cross.  It is coded as follows:
         .TRUE.  -  mark with a cross,
         .FALSE. -  do not mark with a cross.
      }
      \sstsubsection{
         MULTIPLE  =  LOGICAL (read)
      }{
         A flag indicating whether more than target list is to be
         plotted.  It is coded as follows:
         .TRUE.  -  plot several target lists,
         .FALSE. -  plot a single target list.
      }
      \sstsubsection{
         GRPLST  =  CHARACTER (read)
      }{
         The name of the target list to be plotted.
      }
      \sstsubsection{
         TITLE  =  CHARACTER (read)
      }{
         Title for the plot.
      }
      \sstsubsection{
         QUIET  =  LOGICAL (read)
      }{
         Operate in quiet mode where warnings are suppressed.  The
         permitted values are:
         TRUE  - quiet mode,
         FALSE - verbose mode.
      }
   }
   \sstexamples{
      \sstexamplesubsection{
         catchart
      }{
         A graphics device and target list will be prompted for and then
         the target list will be plotted.  Most of the other parameters
         will only be prompted for if they cannot be read from the
         target list.  The centre of the plot will be marked with a
         {\tt '}gun sight{\tt '} cross.
      }
      \sstexamplesubsection{
         catchart  multiple=yes
      }{
         Plot several target lists superimposed as a single finding
         chart.  You will be prompted in sequence for the required
         target lists.  When you have entered all the required lists
         reply QUIT.
      }
      \sstexamplesubsection{
         catchart  mcentre=no
      }{
         A graphics device and target list will be prompted for and then
         the target list will be plotted without a central cross.
      }
      \sstexamplesubsection{
         catchart  multiple=yes  mcentre=no
      }{
         Plot several target lists superimposed on a single finding
         chart with no central cross.
      }
   }
}
\newpage
\sstroutine{
   CATCHARTRN
}{
   Translate a target list into a graphics attribute list
}{
   \sstdescription{
      Translate a target list into a graphics attribute list.  The
      program computes some extra columns and parameters which define
      how the objects in a target list are to be plotted (that is, the
      symbol, column and size used to draw each object).  The details
      required are read from a pre-prepared file, the so-called
      {\tt '}graphics translation file{\tt '}.
   }
   \sstusage{
      catchartrn
   }
   \sstparameters{
      \sstsubsection{
         GTFILE  =  CHARACTER (read)
      }{
         Name of the graphics translation file.
      }
      \sstsubsection{
         CATIN  =  CHARACTER (read)
      }{
         Name of the input target list.
      }
      \sstsubsection{
         CATOUT  =  CHARACTER (read)
      }{
         Name of the output graphics attribute list.
      }
      \sstsubsection{
         TEXT  =  CHARACTER (read)
      }{
         Flag indicating the textual header information to be copied.
         The valid responses are:
         A - all; the output catalogue will contain a complete copy
             of the header information for the input catalogue,
             duplicated as comments,
         C - (default) copy only the comments from the input catalogue.
             In the case of a FITS table the COMMENTS and HISTORY
             keywords will be copied.
         N - none; no textual header information is copied.
      }
      \sstsubsection{
         QUIET  =  LOGICAL (read)
      }{
         Operate in quiet mode where warnings are suppressed.  The
         permitted values are:
         TRUE  - quiet mode,
         FALSE - verbose mode.
      }
   }
   \sstexamples{
      \sstexamplesubsection{
         catchartrn
      }{
         The graphics translation file, input and output catalogues will
         be prompted for.  Then the output catalogue will be written
         containing a copy of the original catalogue and new columns
         defining how the objects are to be plotted.  Any comments in the
         input catalogue will be copied.
      }
      \sstexamplesubsection{
         catchartrn  text=all
      }{
         The graphics translation file, input and output catalogues will
         be prompted for.  Then the output catalogue will be written
         containing a copy of the original catalogue and new columns
         defining how the objects are to be plotted.  All the header
         information in the input catalogue will be duplicated as comments
         in the output catalogue.
      }
      \sstexamplesubsection{
         catchartrn  text=none
      }{
         The graphics translation file, input and output catalogues will
         be prompted for.  Then the output catalogue will be written
         containing a copy of the original catalogue and new columns
         defining how the objects are to be plotted.  Any comments in the
         input catalogue will not be copied.
      }
      \sstexamplesubsection{
         catchartrn  graphics-translation-file  input-catalogue  output-catalogue
      }{
         Here the graphics translation file, input and output catalogues
         have been specified on the command line.  Because no value was
         specified for parameter TEXT the default will be adopted and any
         comments in the input catalogue will be copied.
      }
   }
}
\newpage
\sstroutine{
   CATCOORD
}{
   Convert to a new celestial coordinate system
}{
   \sstdescription{
      Convert to a new celestial coordinate system.

      The application will convert mean equatorial coordinates to
      mean equatorial coordinates for another equinox and epoch,
      to Galactic coordinates or to de Vaucoulerurs supergalactic
      coordinates.  The new coordinates may be computed simply from an
      existing Right Ascension and Declination.  Alternatively, more
      accurate values may be computed using columns of proper motion and
      parallax if these are available in the input catalogue.

      A copy of the catalogue containing the new coordinates is created.
      The new coordinates may either replace coordinates in existing
      columns or be written to new columns.
   }
   \sstusage{
      catcoord
   }
   \sstparameters{
      \sstsubsection{
         CATIN  =  CHARACTER (read)
      }{
         Name of the input catalogue.
      }
      \sstsubsection{
         CATOUT  =  CHARACTER (read)
      }{
         Name of the output catalogue.
      }
      \sstsubsection{
         EPOCHI  =  CHARACTER (read)
      }{
         The epoch of the input coordinates, eg: J2000 or B1950.
      }
      \sstsubsection{
         EQUINI  =  CHARACTER (read)
      }{
         The equinox of the input coordinates, eg: J2000 or B1950.
      }
      \sstsubsection{
         RAIN  =  CHARACTER (read)
      }{
         The name of the column containing Right Ascension in the input
         catalogue.
      }
      \sstsubsection{
         DECIN  =  CHARACTER (read)
      }{
         The name of the column containing Declination in the input
         catalogue.
      }
      \sstsubsection{
         FULL  =  LOGICAL (read)
      }{
         A flag indicating whether full input coordinates (including
         proper motions and parallax) are to be used or not.  It is
         coded as follows:
         TRUE  -  use full input coordinates,
         FALSE \_  simply use input Right Ascension and Declination.
      }
      \sstsubsection{
         PMRA  =  CHARACTER (read)
      }{
         The name of the column containing the proper motion in Right
         Ascension in the input catalogue.
      }
      \sstsubsection{
         PMDE  =  CHARACTER (read)
      }{
         The name of the column containing the proper motion in
         Declination in the input catalogue.
      }
      \sstsubsection{
         PLX  =  CHARACTER (read)
      }{
         The name of the column containing the parallax in the input
         catalogue.
      }
      \sstsubsection{
         RV  =  CHARACTER (read)
      }{
         The name of the column containing the radial velocity in the
         input catalogue.
      }
      \sstsubsection{
         COORDS  =  CHARACTER (read)
      }{
         The type of output coordinates to be computed.  The options
         are:
         EQUATORIAL    - equatorial coordinates,
         GALACTIC      - Galactic coordinates,
         SUPERGALACTIC - de Vaucoulerurs supergalactic coordinates.
      }
      \sstsubsection{
         EPOCHO  =  CHARACTER (read)
      }{
         The epoch of the output coordinates, eg: J2000 or B1950.
      }
      \sstsubsection{
         EQUINO  =  CHARACTER (read)
      }{
         The equinox of the output coordinates, eg: J2000 or B1950.
      }
      \sstsubsection{
         RAOUT  =  CHARACTER (read)
      }{
         The name of the column to contain the Right Ascensions computed
         for the new equinox and epoch.
      }
      \sstsubsection{
         DECOUT  =  CHARACTER (read)
      }{
         The name of the column to contain the Declinations computed
         for the new equinox and epoch.
      }
      \sstsubsection{
         L  =  CHARACTER (read)
      }{
         The name of the column to contain the computed Galactic
         longitude.
      }
      \sstsubsection{
         B  =  CHARACTER (read)
      }{
         The name of the column to contain the computed Galactic
         latitude.
      }
      \sstsubsection{
         SGL  =  CHARACTER (read)
      }{
         The name of the column to contain the computed supergalactic
         longitude.
      }
      \sstsubsection{
         SGB  =  CHARACTER (read)
      }{
         The name of the column to contain the computed supergalactic
         latitude.
      }
      \sstsubsection{
         TEXT  =  CHARACTER (read)
      }{
         Flag indicating the textual header information to be copied.
         The valid responses are:
         A - all; the output catalogue will contain a complete copy
             of the header information for the input catalogue,
             duplicated as comments,
         C - (default) copy only the comments from the input catalogue.
             In the case of a FITS table the COMMENTS and HISTORY
             keywords will be copied.
         N - none; no textual header information is copied.
      }
      \sstsubsection{
         QUIET  =  LOGICAL (read)
      }{
         Operate in quiet mode where warnings are suppressed.  The
         permitted values are:
         TRUE  - quiet mode,
         FALSE - verbose mode.
      }
   }
   \sstexamples{
      \sstexamplesubsection{
         catcoord
      }{
         The input and output catalogues and various other details will
         be prompted for.  A new catalogue containing the revised
         coordinates will be written.  By default the new equatorial
         coordinates will be computed only from the Right Ascension and
         Declination in the input catalogue.  Any comments in the input
         catalogue will be copied.
      }
      \sstexamplesubsection{
         catcoord  full=true
      }{
         The input and output catalogues and various other details will
         be prompted for.  A new catalogue containing the revised
         coordinates will be written.  The new equatorial coordinates will
         be computed from the {\tt '}full{\tt '} coordinates in the input catalogue
         (that is, columns of proper motion and parallax will be used).
      }
      \sstexamplesubsection{
         catcoord  coords=galactic
      }{
         The input and output catalogues and various other details will
         be prompted for.  A new catalogue containing Galactic
         coordinates will be written.  The Galactic coordinates will
         be computed only from the Right Ascension and Declination in the
         input catalogue.
      }
      \sstexamplesubsection{
         catcoord  full=true  galactic=true
      }{
         The input and output catalogues and various other details will
         be prompted for.  A new catalogue containing Galactic
         coordinates will be written.  The Galactic coordinates will
         be computed from the {\tt '}full{\tt '} coordinates in the input catalogue
         (that is, columns of proper motion and parallax will be used).
      }
      \sstexamplesubsection{
         catcoord  text=all
      }{
         The input and output catalogues and various other details will
         be prompted for.  A new catalogue containing the revised
         coordinates will be written.  All the header information in the
         input catalogue will be duplicated as comments in the output
         catalogue.
      }
      \sstexamplesubsection{
         catcoord  text=none
      }{
         The input and output catalogues and various other details will
         be prompted for.  A new catalogue containing the revised
         coordinates will be written.  Any comments in the input catalogue
         will not be copied.
      }
   }
}
\newpage
\sstroutine{
   CATCOPY
}{
   Generate a new copy of a CAT catalogue
}{
   \sstdescription{
      Generate a new copy of a CAT catalogue.  By default all the
      columns, parameters and textual information in the input catalogue
      are copied.

      Optionally some or all of the parameters in the input catalogue
      can be omitted from the output catalogue and new parameters can
      be added to the output catalogue.  Also any textual information
      associated with the input catalogue can be omitted from the output
      catalogue.

      It is possible to use catcopy to generate a copy of a catalogue
      in the same format (FITS table or whatever) as the original, but
      there is little point in doing so; the same result can be achieved
      using the Unix command {\tt '}cp{\tt '}, which is much quicker. The real
      usefulness of catcopy is in converting a catalogue to a new format,
      for example, converting a FITS table to an STL (small text list)
      format catalogue.
   }
   \sstusage{
      catcopy
   }
   \sstparameters{
      \sstsubsection{
         CATIN  =  CHARACTER (read)
      }{
         Name of the input catalogue.
      }
      \sstsubsection{
         CATOUT  =  CHARACTER (read)
      }{
         Name of the output catalogue.
      }
      \sstsubsection{
         COPYPAR  =  CHARACTER (read)
      }{
         Flag indicating which parameters are to be copied.
         The valid responses are:
         A - all; (default) copy all the parameters,
         F - filter; omit (that is, filter out) selected parameters,
         N - none; omit all the parameters.
      }
      \sstsubsection{
         PFILTER  =  CHARACTER (read)
      }{
         A comma-separated list of the parameters to filter out (that is,
         to omit).
      }
      \sstsubsection{
         ADDPAR  =  LOGICAL (read)
      }{
         Flag indicating whether any new parameters are to be added to
         the output catalogue.  The permitted values are:
         TRUE  - add new parameters,
         FALSE - (default) do not add new parameters.
      }
      \sstsubsection{
         PNAME  =  CHARACTER (read)
      }{
         Name of the current new parameter.
      }
      \sstsubsection{
         PARTYP  =  CHARACTER (read)
      }{
         Data type of the current new parameter.  The permitted types are:
         REAL, DOUBLE, INTEGER, LOGICAL and CHAR.
      }
      \sstsubsection{
         PCSIZE  =  INTEGER (read)
      }{
         Size of the current new parameter if it is of type CHAR.
      }
      \sstsubsection{
         PVALUE  =  CHARACTER (read)
      }{
         Value of the current new parameter.
      }
      \sstsubsection{
         PUNITS  =  CHARACTER (read)
      }{
         Units of the current new parameter.
      }
      \sstsubsection{
         PCOMM  =  CHARACTER (read)
      }{
         Comments describing the current new parameter.
      }
      \sstsubsection{
         TEXT  =  CHARACTER (read)
      }{
         Flag indicating the textual header information to be copied.
         The valid responses are:
         A - all; the output catalogue will contain a complete copy
             of the header information for the input catalogue,
             duplicated as comments,
         C - (default) copy only the comments from the input catalogue.
             In the case of a FITS table the COMMENTS and HISTORY
             keywords will be copied.
         N - none; no textual header information is copied.
      }
      \sstsubsection{
         QUIET  =  LOGICAL (read)
      }{
         Operate in quiet mode where warnings are suppressed.  The
         permitted values are:
         TRUE  - quiet mode,
         FALSE - verbose mode.
      }
   }
   \sstexamples{
      \sstexamplesubsection{
         catcopy
      }{
         The input and output catalogues will be prompted for and then
         copying proceeds.  Any parameters and comments in the input
         catalogue will be copied.
      }
      \sstexamplesubsection{
         catcopy  input-catalogue  output-catalogue
      }{
         Here the input and output catalogues have been specified on
         the command line.  Copying proceeds and any parameters and
         comments in the input catalogue will be copied.
      }
      \sstexamplesubsection{
         catcopy  copypar=none
      }{
         The input and output catalogues will be prompted for and then
         copying proceeds.  None of the parameters in the input catalogue
         will be written to the output catalogue.
      }
      \sstexamplesubsection{
         catcopy  copypar=filter
      }{
         The input catalogue, output catalogue and a list of parameters
         will be prompted for.  Copying then proceeds.  All the parameters
         specified in the list will not be written to the output catalogue.
      }
      \sstexamplesubsection{
         catcopy  copypar=filter  pfilter=$\backslash${\tt '}FSTATION,PLATESCA,TELFOCUS$\backslash${\tt '}
      }{
         The input and output catalogues will be prompted for and then
         copying proceeds.  The parameters given in the list (FSTATION,
         PLATESCA and TELFOCUS) will not be written to the output catalogue
         (that is, they will be {\tt '}filtered out{\tt '}).  The items in the list
         must be separated by commas.  When the list is specified on the
         command line, as here, it must be enclosed in quotes and each
         quote must be preceded by a backslash character (as shown) to
         prevent it being interpreted by the Unix shell.
      }
      \sstexamplesubsection{
         catcopy  addpar=true
      }{
         The input and output catalogues will be prompted for.  The
         details of additional parameters to be added to the output
         catalogue will then be prompted for.  The details which must
         be supplied for each parameter are: name, data type (and size
         if of type CHAR), value, units and comments.  An arbitrary
         number of comments can be added.  Once all the parameters required
         have been specified then copying proceeds.
      }
      \sstexamplesubsection{
         catcopy  text=all
      }{
         The input and output catalogues will be prompted for and then
         copying proceeds.  All the header information in the input
         catalogue will be duplicated as comments in the output
         catalogue.
      }
      \sstexamplesubsection{
         catcopy  text=none
      }{
         The input and output catalogues will be prompted for and then
         copying proceeds.  Any comments in the input catalogue will not
         be copied.
      }
   }
}
\newpage
\sstroutine{
   CATGRID
}{
   Generate an NDF grid from up to three columns in a catalogue
}{
   \sstdescription{
      Generate a grid, formatted as a Starlink NDF file, from up to
      three columns in a catalogue.  If one column is specified the
      output grid corresponds to a histogram, two columns correspond
      to a two-dimensional {\tt '}image{\tt '} and three columns to a data cube.

      The dimensionality of the grid (1, 2 or 3) is specified.  Then,
      for each axis of the grid, the name of the corresponding column
      in the catalogue and the number of elements in the grid along the
      axis are given.  The limits of the grid along each axis correspond
      to the range of values of the corresponding catalogue column.  The
      value of each element in grid is set to the number of points which
      lie within it.  Optionally the grid may be normalised by dividing
      by the total number of points in the catalogue.

      The grids generated can be displayed and manipulated using Starlink
      software such as GAIA (SUN/214), KAPPA (SUN/95) and Figaro (SUN/86)
      or visualisation packages such as DX (SUN/203 and SC/2).
   }
   \sstusage{
      catgrid
   }
   \sstparameters{
      \sstsubsection{
         CATIN  =  CHARACTER (read)
      }{
         Name of the input catalogue.
      }
      \sstsubsection{
         NDIM  =  INTEGER (read)
      }{
         Number of dimensions in the output grid.  The permitted values
         are 1 - 3.
      }
      \sstsubsection{
         COLX  =  CHARACTER (read)
      }{
         Name of the column to be used for the X-axis of the grid.
      }
      \sstsubsection{
         XBINS  =  INTEGER (read)
      }{
         Number of bins in the grid along the X-axis.
      }
      \sstsubsection{
         COLY  =  CHARACTER (read)
      }{
         Name of the column to be used for the Y-axis of the grid.
      }
      \sstsubsection{
         YBINS  =  INTEGER (read)
      }{
         Number of bins in the grid along the Y-axis.
      }
      \sstsubsection{
         COLZ  =  CHARACTER (read)
      }{
         Name of the column to be used for the Z-axis of the grid.
      }
      \sstsubsection{
         ZBINS  =  INTEGER (read)
      }{
         Number of bins in the grid along the Z-axis.
      }
      \sstsubsection{
         GRID  =  NDF (Write)
      }{
         The name of the output data grid or histogram.
      }
      \sstsubsection{
         NORMAL  =  LOGICAL (read)
      }{
         Flag indicating whether the grid of values is to be normalised
         or not.  If NORMAL is set to TRUE the grid will be normalised
         (that is, the value of each grid element will be the number of
         points occupying the element divided by the total number of
         points in the catalogue); if it is set to FALSE it will not.
         The default is FALSE.
      }
      \sstsubsection{
         QUIET  =  LOGICAL (read)
      }{
         Operate in quiet mode where warnings are suppressed.  The
         permitted values are:
         TRUE  - quiet mode,
         FALSE - verbose mode.
      }
   }
   \sstexamples{
      \sstexamplesubsection{
         catgrid
      }{
         The input catalogue will be prompted for, followed by the
         dimensionality (1, 2 or 3) of the output grid.  For each axis
         the name of the corresponding catalogue column and the number of
         elements along the axis are prompted for.  Finally, the name of
         the NDF file to hold will be prompted for.  An un-normalised
         grid will be generated.
      }
      \sstexamplesubsection{
         catgrid  normal=true
      }{
         The input catalogue, dimensionality, details of each axis and
         output NDF file are all prompted for.  A normalised grid will
         be generated.
      }
   }
}
\newpage
\sstroutine{
   CATGSCIN
}{
   Convert a GSC region to the preferred CURSA format
}{
   \sstdescription{
      Convert a FITS table containing a region of the HST Guide Star
      Catalogue (GSC) into a FITS table with the preferred CURSA format.
      This format has the Right Ascension and Declination formatted as
      CURSA angular columns and is sorted on Declination.  Though CURSA
      can access the GSC regions directly it is more convenient to
      process them with catgscin first.

      The name of the output FITS table is generated automatically from
      the name of the input GSC region.  GSC regions have names of the
      form region-number.gsc (where region-number is an integer number).
      The name of the output catalogue is {\tt '}gsc{\tt '} followed by the region
      number.  Thus, for example, if region 5828.gsc was imported the
      converted catalogue would be written to FITS table gsc5828.FIT.
   }
   \sstusage{
      catgscin
   }
   \sstparameters{
      \sstsubsection{
         CATIN  =  CHARACTER (read)
      }{
         Name of the input GSC region.
      }
      \sstsubsection{
         TEXT  =  CHARACTER (read)
      }{
         Flag indicating the textual header information to be copied.
         The valid responses are:
         A - all; the output catalogue will contain a complete copy
             of the header information for the input GSC region,
             duplicated as comments,
         C - (default) copy only the comments from the input GSC region.
         N - none; no textual header information is copied.
      }
      \sstsubsection{
         QUIET  =  LOGICAL (read)
      }{
         Operate in quiet mode where warnings are suppressed.  The
         permitted values are:
         TRUE  - quiet mode,
         FALSE - verbose mode.
      }
   }
   \sstexamples{
      \sstexamplesubsection{
         catgscin
      }{
         The input GSC region will be prompted for and then conversion
         proceeds.  Comments in the input region will be copied.
      }
      \sstexamplesubsection{
         catgscin  text=all
      }{
         The input GSC region will be prompted for and then conversion
         proceeds.  All the header information in the input region
         will be duplicated as comments in the output catalogue.
      }
      \sstexamplesubsection{
         catgscin  text=none
      }{
         The input GSC region will be prompted for and then conversion
         proceeds.  Comments in the input catalogue will not be copied.
      }
      \sstexamplesubsection{
         catgscin  input-region
      }{
         Here the input region has been specified on the command line.
         Because no value was specified for parameter TEXT the default
         will be adopted and comments in the region catalogue will be
         copied.
      }
   }
}
\newpage
\sstroutine{
   CATHEADER
}{
   List various header information for a catalogue
}{
   \sstdescription{
      List various header information for a catalogue.  By default the
      information listed is: the number of rows, the number of columns,
      the number of catalogue parameters and a list of the names of all
      the columns.  Parameter DETAILS can be used to specify that various
      alternative details are to be listed.

      The output is directed to the standard output and optionally may
      also be copied to a text file.  If the name of the catalogue is
      CNAME, then this output file will be called CNAME.lis.

      Application parameters ROWS, COLS, PARS and NAMES are written only
      if DETAILS=SUMMARY or FULL.
   }
   \sstusage{
      catheader
   }
   \sstparameters{
      \sstsubsection{
         CATALOGUE  =  CHARACTER (read)
      }{
         Name of the catalogue.
      }
      \sstsubsection{
         FILE  =  LOGICAL (read)
      }{
         Flag indicating whether or not an output file is to be written.
         It is coded as follows:
         .TRUE.  - write the output file,
         .FALSE. - do not write the output file.
      }
      \sstsubsection{
         DETAIL  =  CHARACTER (read)
      }{
         Flag specifying the details which catheader is to display.
         The options are:
         SUMMARY    - summary (default),
         COLUMNS    - full details of all the columns,
         PARAMETERS - full details of all the parameters,
         TEXT       - textual information,
         AST        - details of any AST information,
         FULL       - full information (all the above).
      }
      \sstsubsection{
         QUIET  =  LOGICAL (read)
      }{
         Operate in quiet mode where warnings are suppressed.  The
         permitted values are:
         TRUE  - quiet mode,
         FALSE - verbose mode.
      }
      \sstsubsection{
         ROWS  =  INTEGER (write)
      }{
         The number of rows in the catalogue.
      }
      \sstsubsection{
         COLS  =  INTEGER (write)
      }{
         The number of columns in the catalogue.
      }
      \sstsubsection{
         PARS  =  INTEGER (write)
      }{
         The number of parameters in the catalogue.
      }
      \sstsubsection{
         NAMES  =  CHARACTER (write)
      }{
         A list of the names of all the columns in the catalogue.
      }
   }
   \sstexamples{
      \sstexamplesubsection{
         catheader
      }{
         The catalogue name will be prompted for, then the default
         details will be displayed.
      }
      \sstexamplesubsection{
         catheader  input-catalogue
      }{
         Here the input catalogue has been specified on the command
         line.  The default details will be displayed.
      }
      \sstexamplesubsection{
         catheader  details=columns
      }{
         The catalogue name will be prompted for, then details of all
         the columns in the catalogue will be displayed.
      }
      \sstexamplesubsection{
         catheader  file=true
      }{
         The catalogue name will be prompted for, then the default
         details will be both displayed and written to a text file
         called input-catalogue.lis.
      }
   }
}
\newpage
\sstroutine{
   CATPAIR
}{
   Pair two catalogues
}{
   \sstdescription{
      Pair two catalogues to create a new output catalogue.  The input
      catalogues are paired on the basis of similar two-dimensional
      coordinates.  The coordinates may be either celestial spherical-
      polar or Cartesian.  An index join method is used.

      catpair is a powerful and flexible application.  See SUN/190 for
      a full description.
   }
   \sstusage{
      catpair
   }
   \sstparameters{
      \sstsubsection{
         PRIMARY  =  CHARACTER (read)
      }{
         The name of the primary input catalogue.
      }
      \sstsubsection{
         SECOND  =  CHARACTER (read)
      }{
         The name of the secondary input catalogue.  This catalogue
         must be sorted on the second column to be used in the pairing.
         Usually this column will be the Declination or Y coordinate.
      }
      \sstsubsection{
         OUTPUT  =  CHARACTER (read)
      }{
         The name of the output paired catalogue.  A catalogue with
         this name must not already exist.
      }
      \sstsubsection{
         CRDTYP  =  CHARACTER (read)
      }{
         The type of coordinates to be paired.  The possibilities are
         either Cartesian coordinates (`C{\tt '}) or celestial spherical-polar
         coordinates (`S{\tt '}) such as Right Ascension and Declination.
      }
      \sstsubsection{
         PCRD1  =  CHARACTER (read)
      }{
         The name of the column in the primary catalogue containing the
         first column to be used in the pairing. This column will usually
         be an X coordinate or a Right Ascension.
      }
      \sstsubsection{
         PCRD2  =  CHARACTER (read)
      }{
         The name of the column in the primary catalogue containing the
         second column to be used in the pairing. This column will usually
         be a Y coordinate or a Declination.
      }
      \sstsubsection{
         SCRD1  =  CHARACTER (read)
      }{
         The name of the column in the secondary catalogue containing the
         first column to be used in the pairing. This column will usually
         be an X coordinate or a Right Ascension.
      }
      \sstsubsection{
         SCRD2  =  CHARACTER (read)
      }{
         The name of the column in the secondary catalogue containing the
         second column to be used in the pairing. This column will usually
         be a Y coordinate or a Declination. The secondary catalogue must
         be sorted on this column.
      }
      \sstsubsection{
         PDIST  =  CHARACTER (read)
      }{
         The {\tt '}critical distance{\tt '}; the maximum separation for two objects
         to be considered pairs.  It may be either a constant, the
         name of a column in the primary catalogue or an expression
         involving columns in the primary catalogue.
      }
      \sstsubsection{
         PRTYP  =  CHARACTER (read)
      }{
         The `type of pairing{\tt '} required, that is the set of rows from the
         two input catalogues are to be retained in the output catalogue.
         Briefly, the options are: C - common, P - primary, M - mosaic,
         R - primrej or A - allrej.  See SUN/190 for more details.
      }
      \sstsubsection{
         MULTP  =  LOGICAL (read)
      }{
         Specify how multiple matches in the primary are to be handled.
         The options are either to retain the single closest match or to
         retain all the matches.
      }
      \sstsubsection{
         MULTS  =  LOGICAL (read)
      }{
         Specify how multiple matches in the secondary are to be handled.
         The options are either to retain the single closest match or to
         retain all the matches.
      }
      \sstsubsection{
         ALLCOL  =  LOGICAL (read)
      }{
         Specify the set of columns to be retained in the output catalogue.
         The options are to either retain all the columns from both input
         catalogues or to retain specified columns from either input
         catalogue.  If you are in doubt you should retain all the columns.
      }
      \sstsubsection{
         SPCOL  =  LOGICAL (read)
      }{
         Flag indicating whether special columns giving details of the
         paired objects are to be included in the output catalogue.
         If SPCOL is set to TRUE the following columns are included:
         SEPN, the separation of the paired primary and secondary objects;
         PMULT, the number of matches in the primary;
         SMULT, the number of matches in the seconary.
      }
      \sstsubsection{
         PRMPAR  =  LOGICAL (read)
      }{
         Specify whether the parameters of the primary are to be copied to
         the output catalogue.
      }
      \sstsubsection{
         SECPAR  =  LOGICAL (read)
      }{
         Specify whether the parameters of the secondary are to be copied
         to the output catalogue.
      }
      \sstsubsection{
         PTEXT  =  CHARACTER (read)
      }{
         Specify whether any textual information associated with the
         primary is to be copied to the output catalogue.  The options
         are: A - all (create a duplicate of the primary header as
         comments), C - just copy comments (and history) or N - none.
      }
      \sstsubsection{
         STEXT  =  CHARACTER (read)
      }{
         Specify whether any textual information associated with the
         secondary is to be copied to the output catalogue.  The options
         are: A - all (create a duplicate of the secondary header as
         comments), C - just copy comments (and history) or N - none.
      }
      \sstsubsection{
         TEXT  =  CHARACTER (read)
      }{
         Specify whether a set of comments describing the specification
         of the pairing pairing is written to the output catalogue.  The
         options are: Y - write comments (default), N - do not write
         comments.
      }
      \sstsubsection{
         COLBUF  =  CHARACTER (read)
      }{
         Name for the individual columns to be included in the output
         catalogue.  Enter {\tt '}END{\tt '} to finish.
      }
      \sstsubsection{
         QUIET  =  LOGICAL (read)
      }{
         Operate in quiet mode where warnings are suppressed.  The
         permitted values are:
         TRUE  - quiet mode,
         FALSE - verbose mode.
      }
   }
   \sstexamples{
      \sstexamplesubsection{
         catpair
      }{
         Answer the numerous prompts and pair two catalogues.
      }
      \sstexamplesubsection{
         catpair spcol=true
      }{
         Answer the numerous prompts and pair two catalogues.  The output
         catalogue of paired objects will contain three additional
         columns containing details for the paired objects.
      }
      \sstexamplesubsection{
         catpair text=n
      }{
         Answer the numerous prompts and pair two catalogues, but
         specify that a summary of the pairing specification is not to
         be written as comments to the output catalogue.
      }
   }
   \sstnotes{
      catpair is intended for the case where the primary catalogue is
      a relatively small list of target objects which is being paired
      with a larger secondary catalogue.  It will still work if the
      primary is a large catalogue, but it is not optimised for this
      case and will take some time.  Conversely, the size of the
      secondary catalogue is largely immaterial.
   }
   \sstdiytopic{
      Pitfalls
   }{
      Ensure that the secondary catalogue is sorted on the second
      pairing column.  Usually this column will be the Declination or Y
      coordinate.  If the secondary is not suitably sorted then use
      application catsort to sort it.
   }
   \sstdiytopic{
      Prior Requirements
   }{
      The secondary catalogue must be sorted on the second pairing
      coordinate.  Usually this coordinate will be the Declination or
      Y coordinate.  If the secondary is not suitably sorted then use
      application catsort to sort it.
   }
}
\newpage
\sstroutine{
   CATPHOTOMFIT
}{
   Fit instrumental to standard magnitudes
}{
   \sstdescription{
      This application fits instrumental magnitudes (typically
      measured from images in a CCD frame) to standard magnitudes
      in some photometric system for a set of photometric standard
      stars.  The instrumental and standard magnitudes, and other
      quantities, are read from a catalogue.

      The transformation coefficients determined by the fit are
      written to a file and can subsequently be used to calibrate
      the instrumental magnitudes of a set of programme objects.
      The equation used to relate the instrumental and standard
      magnitudes is:

        Mstd = Minst - arb $+$ zero $+$ (atmos $*$ airmass)

      where:
        Mstd    - standard or calibrated magnitude,
        Minst   - instrumental magnitude,
        arb     - arbitrary constant added to the instrumental magnitudes,
        zero    - zero point,
        atmos   - atmospheric extinction,
        airmass - air mass through which the standard star was observed.

      Note that this relation is a particularly simple way of relating
      standard and instrumental magnitudes.  In particular no correction
      is made for any colour correction between the standard and
      instrumental systems.

      The application has a number of options, including the following.

      $*$ Either or both of the zero point and atmospheric extinction
        can be supplied rather than fitted.  If both are supplied then
        no fit is necessary and the file of transformation coefficients
        is simply written.

      $*$ A table showing the residuals between the standard magnitudes
        and the calibrated magnitudes computed from the instrumental
        magnitudes can be listed.

      $*$ The table of residuals may optionally include a column showing
        a name for each of the standard stars.

      $*$ Optionally a column containing the observed zenith distance
        may be supplied instead of a column containing the air mass.
        In this case the air mass is automatically calculated from the
        observed zenith distance.

      $*$ By default all the stars in the input catalogue are included in
        the fit.  However, optionally a column of `include in fit{\tt '}
        flags may be supplied and a star will only be included if it
        has the flag set to TRUE.  This mechanism provides an easy way
        to exclude stars which give a poor fit.
   }
   \sstusage{
      catphotomfit
   }
   \sstparameters{
      \sstsubsection{
         FIXED  =  LOGICAL (read)
      }{
         Flag; are any of the coefficients fixed or are they all
         determined by the fit.  It is coded as follows:
         TRUE  - some or all of the coefficients are fixed,
         FALSE - all the coefficients are determined from the fit.
      }
      \sstsubsection{
         ZENITHDIST  =  LOGICAL (read)
      }{
         Flag; is the air mass read directly from a column or is it
         computed from the observed zenith distance?  It is coded as
         follows:
         TRUE  - computed from the observed zenith distance,
         FALSE - read directly from a column.
      }
      \sstsubsection{
         FZEROP  =  LOGICAL (read)
      }{
         Flag; is the zero point fixed.  It is coded as follows:
         TRUE  - the zero point is fixed,
         FALSE - the zero point is determined from the fit.
      }
      \sstsubsection{
         ZEROP  =  DOUBLE PRECISION (read)
      }{
         Value of the fixed zero point.
      }
      \sstsubsection{
         FATMOS  =  LOGICAL (read)
      }{
         Flag; is the atmospheric extinction fixed.  It is coded as
         follows:
         TRUE  - the atmospheric extinction is fixed,
         FALSE - the atmospheric extinction is determined from the fit.
      }
      \sstsubsection{
         ATMOS  =  DOUBLE PRECISION (read)
      }{
         Value of the fixed atmospheric extinction.
      }
      \sstsubsection{
         RESID  =  LOGICAL (read)
      }{
         Flag; are the residuals to be listed?  It is coded as follows:
         TRUE  - list the residuals,
         FALSE - do not list the residuals.
      }
      \sstsubsection{
         CATALOGUE  =  CHARACTER (read)
      }{
         Name of the catalogue containing the standard and instrumental
         magnitudes.
      }
      \sstsubsection{
         NAME  =  CHARACTER (read)
      }{
         Name of a column containing names of the standard stars.
         The special value NONE indicates that a column of star names
         is not required.
      }
      \sstsubsection{
         INCLUDE  =  CHARACTER (read)
      }{
         Name of a column of `include in fit{\tt '} flags for the standard
         stars.  The special value ALL indicates that all the stars are
         to be included in the fit.
      }
      \sstsubsection{
         CATMAG  =  CHARACTER (read)
      }{
         Name of the column or expression holding the standard or
         catalogue magnitudes.
      }
      \sstsubsection{
         INSMAG  =  CHARACTER (read)
      }{
         Name of the column or expression holding the instrumental
         magnitudes.
      }
      \sstsubsection{
         AIRMASS  =  CHARACTER (read)
      }{
         Name of the column or expression holding the air mass.
      }
      \sstsubsection{
         ZENDST  =  CHARACTER (read)
      }{
         Name of the column or expression holding the observed zenith
         distance.
      }
      \sstsubsection{
         INSCON  =  DOUBLE PRECISION (read)
      }{
         Arbitrary constant previously added to the instrumental
         magnitudes.
      }
      \sstsubsection{
         FILNME  =  CHARACTER (read)
      }{
         The name of the file which is to contain the transformation
         coefficients.
      }
      \sstsubsection{
         QUIET  =  LOGICAL (read)
      }{
         Operate in quiet mode where warnings are suppressed.  The
         permitted values are:
         TRUE  - quiet mode,
         FALSE - verbose mode.
      }
   }
   \sstexamples{
      \sstexamplesubsection{
         catphotomfit
      }{
         The input catalogue and various other details will be prompted
         for.  The transformation coefficients and a table of residuals
         will be displayed.  The transformation coefficients will be
         written to a file.
      }
      \sstexamplesubsection{
         catphotomfit  zenithdist=true
      }{
         You should supply a column containing the observed zenith
         distance rather than one containing the air mass.  This
         column will be used to calculate the air mass automatically.
      }
      \sstexamplesubsection{
         catphotomfit  fixed=true
      }{
         You will supply either the zero point, the atmospheric
         extinction or both, rather than allowing them to be fitted.
         You will be prompted for the appropriate details.
      }
      \sstexamplesubsection{
         catphotomfit  resid=false
      }{
         A table of residuals will not be listed.  However, the
         transformation coefficients will still be displayed and
         written to a file.
      }
   }
}
\newpage
\sstroutine{
   CATPHOTOMLST
}{
   List a file of photometric transformation constants
}{
   \sstdescription{
      List the contents of a file of transformation coefficients for
      converting instrumental magnitudes into calibrated or standard
      magnitudes in some photometric system.  Such files are created
      by application catphotomfit.
   }
   \sstusage{
      catphotomlst
   }
   \sstparameters{
      \sstsubsection{
         FILNME  =  CHARACTER (read)
      }{
         The name of the file which contains the transformation
         coefficients.
      }
      \sstsubsection{
         DECPL  =  INTEGER (read)
      }{
         The number of decimal places for displaying the transformation
         coefficients.  Note that this quantity controls only the
         precision with which the coefficients are displayed; they
         are stored in the file as DOUBLE PRECISION numbers.
      }
   }
   \sstexamples{
      \sstexamplesubsection{
         catphotomlst
      }{
         The file of transformation coefficients will be prompted for
         and listed.
      }
      \sstexamplesubsection{
         catphotomlst  decpl=8
      }{
         The file of transformation coefficients will be prompted for
         and listed.  The coefficients will be displayed to a
         precision of eight places of decimals.
      }
   }
}
\newpage
\sstroutine{
   CATPHOTOMTRN
}{
   Transform instrumental to calibrated mags. for programme stars
}{
   \sstdescription{
      This application transforms instrumental magnitudes into
      calibrated magnitudes in some photometric system for a catalogue
      of programme objects.  A new catalogue is written which contains
      the calibrated magnitudes as well as all the columns in the
      original catalogue.

      The transformation coefficients used to convert the instrumental
      magnitudes are read from a file.  Application catphotomfit
      can be used to prepare a suitable file.  See the documentation
      for this application for details of the transformation used.

      The transformation includes a term for the air mass through
      which the object was observed.  By default a column containing
      the air mass is read from the input catalogue.  However,
      optionally, a column containing the observed zenith distance
      of the object may be read instead and used to automatically
      calculate the air mass.
   }
   \sstusage{
      catphotomtrn
   }
   \sstparameters{
      \sstsubsection{
         ZENITHDIST  =  LOGICAL (read)
      }{
         Flag; is the air mass read directly from a column or is it
         computed from the observed zenith distance?  It is coded as
         follows:
         TRUE  - computed from the observed zenith distance,
         FALSE - read directly from a column.
      }
      \sstsubsection{
         FILNME  =  CHARACTER (read)
      }{
         The name of the file which contains the transformation
         coefficients.
      }
      \sstsubsection{
         INSCON  =  DOUBLE PRECISION (read)
      }{
         Arbitrary constant previously added to the instrumental
         magnitudes.
      }
      \sstsubsection{
         CATIN  =  CHARACTER (read)
      }{
         Name of the input catalogue.
      }
      \sstsubsection{
         CATOUT  =  CHARACTER (read)
      }{
         Name of the output catalogue.
      }
      \sstsubsection{
         INSMAG  =  CHARACTER (read)
      }{
         Name of the column or expression in the input catalogue
         holding the instrumental magnitudes.
      }
      \sstsubsection{
         AIRMASS  =  CHARACTER (read)
      }{
         Name of the column or expression in the input catalogue
         holding the air mass.
      }
      \sstsubsection{
         ZENDST  =  CHARACTER (read)
      }{
         Name of the column or expression in the input catalogue
         holding the observed zenith distance.
      }
      \sstsubsection{
         CALMAG  =  CHARACTER (read)
      }{
         Name of the column in the output catalogue to hold the
         calibrated magnitudes.
      }
      \sstsubsection{
         TEXT  =  CHARACTER (read)
      }{
         Flag indicating the textual header information to be copied.
         The valid responses are:
         A - all; the output catalogue will contain a complete copy
             of the header information for the input catalogue,
             duplicated as comments,
         C - (default) copy only the comments from the input catalogue.
             In the case of a FITS table the COMMENTS and HISTORY
             keywords will be copied.
         N - none; no textual header information is copied.
      }
      \sstsubsection{
         QUIET  =  LOGICAL (read)
      }{
         Operate in quiet mode where warnings are suppressed.  The
         permitted values are:
         TRUE  - quiet mode,
         FALSE - verbose mode.
      }
   }
   \sstexamples{
      \sstexamplesubsection{
         catphotomtrn
      }{
         The input and output catalogues and various other details will
         be prompted for.  A new catalogue containing the calibrated
         magnitudes will be written.  All the header information in the
         input catalogue will be duplicated as comments in the output
         catalogue.
      }
      \sstexamplesubsection{
         catphotomtrn  zenithdist=true
      }{
         The input and output catalogues and various other details will
         be prompted for.  You should supply a column containing the
         observed zenith distance rather than one containing the air
         mass.  This column will be used to calculate the air mass
         automatically.
      }
      \sstexamplesubsection{
         catphotomtrn  text=all
      }{
         The input and output catalogues and various other details will
         be prompted for.  A new catalogue containing the calibrated
         magnitudes will be written.  All the header information in the
         input catalogue will be duplicated as comments in the output
         catalogue.
      }
      \sstexamplesubsection{
         catphotomtrn  text=none
      }{
         The input and output catalogues and various other details will
         be prompted for.  A new catalogue containing the calibrated
         magnitudes will be written.  Any comments in the input catalogue
         will not be copied.
      }
   }
}
\newpage
\sstroutine{
   CATREMOTE
}{
   A simple script to query remote catalogues
}{
   \sstdescription{
      catremote is a tool for querying remote astronomical catalogues,
      databases and archives via the Internet.  It allows remote
      catalogues to be queried and the resulting table saved as a local
      file written in the Tab-Separated Table (TST) format.  It also
      provides a number of related auxiliary functions.

      catremote has several different modes of usage, each providing a
      different function.  The modes are:

      list    - list the catalogues currently available,

      details - show details of a named catalogue,

      query   - submit a query to a remote catalogue and retrieve the results,

      name    - resolve an object name into coordinates,

      help    - list the modes available.

      There is an introduction to using catremote in SUN/190 and it is
      comprehensively documented in SSN/76.
   }
   \sstusage{
      Arguments for catremote can be specified on the command line.
      If arguments other than the first are omitted then they will usually
      be prompted for.  The first argument is the mode of operation and
      its value determines the other arguments which are required.  The
      arguments for the various modes are:

       catremote list    server-type

       catremote details db-name

       catremote query   db-name alpha delta radius additional-condition

       catremote name    db-name object-name

       catremote help

      The individual arguments are described in the `Arguments{\tt '} section.
      If the mode is omitted then  {\tt '}help{\tt '} mode is assumed.

      In addition to the command-line arguments, catremote takes some
      input from Unix shell environment variables and these variables can
      be used to control its behaviour.
   }
   \sstexamples{
      \sstexamplesubsection{
         catremote
      }{
      }
      \sstexamplesubsection{
         catremote help
      }{
         List the various modes in which catremote may be used.
      }
      \sstexamplesubsection{
         catremote list
      }{
         List all the catalogues and databases in the current configuration
         file.
      }
      \sstexamplesubsection{
         catremote list namesvr
      }{
         List all the name servers (that is, databases of server type
         {\tt '}namesvr{\tt '}) in the current configuration file.
      }
      \sstexamplesubsection{
         catremote details usno@eso
      }{
         Show details of the USNO PMM astrometric catalogue (whose name
         is {\tt '}usno@eso{\tt '}).
      }
      \sstexamplesubsection{
         catremote query usno@eso 12:15:00 30:30:00 10
      }{
         Find all the objects in the USNO PMM which lie within ten minutes
         of arc of Right Ascension 12:15:00.0 (sexagesimal hours) and
         Declination 30:30:00.0 (sexagesimal degrees, both J2000).  The
         objects selected will be saved as a catalogue called
         usno\_eso\_121500\_303000.tab created in your current directory.
         This catalogue will be written in the Tab-Separated Table (TST)
         format.
      }
      \sstexamplesubsection{
         catremote query usno@eso 12:15:00 30:30:00 10 14,16
      }{
         Find all the objects in the USNO PMM which lie within ten minutes
         of arc of Right Ascension 12:15:00.0 (sexagesimal hours) and
         Declination 30:30:00.0 (sexagesimal degrees, both J2000) which
         also lie in the magnitude range 14 to 16.
      }
      \sstexamplesubsection{
         catremote name simbad\_ns@eso ngc3379
      }{
         Find the equatorial coordinates of the galaxy NGC 3379.  The
         coordinates returned are for equinox J2000.
      }
   }
   \sstdiytopic{
      Environment Variables
   }{
      CATREM\_URLREADER (read)
         catremote uses a separate program to submit the URL constituting
         a query to the server and return the table of results.  This
         environment variable specifies the program to be used.  See
         SSN/76 for further details.  (Mandatory.)

      CATREM\_CONFIG (read)
         This environment variable specifies the configuration file to be
         used.  It should be set to either the URL (for a remote file) or
         the local file name, including a directory specification (for a
         local file).  Configuration files mediate the interaction between
         catremote and the remote catalogue; see SSN/76 for further
         details.  (Mandatory.)

      CATREM\_MAXOBJ (read)
         The maximum number of objects which the returned table is allowed
         to contain.

      CATREM\_ECHOURL (read)
         This environment controls whether the URL representing the query
         submitted to the remote catalogue is also displayed to the user.
         The default is {\tt '}no{\tt '}; to see the URL set CATREM\_ECHOURL to {\tt '}yes{\tt '}.
         Seeing the URL is potentially useful when debugging configuration
         files and remote catalogue servers but is not usually required
         for normal operation.
   }
}
\newpage
\sstroutine{
   CATSELECT
}{
   Generate a selection from a catalogue
}{
   \sstdescription{
      Generate a selction from a catalogue using one of a number of a
      different type of selections.  Optionally the rejected objects
      may be written to a separate catalogue.  The following types
      of selections are available.

      Arbitrary expression: objects which satisfy an algebraic
        expression which you supply.

      Range within a sorted column: objects which lie within a given
        range for a specified column.  This option only works on
        sorted columns.  However, because it is not necessary to read
        the entire column it works essentially instantaneously,
        irrespective of the number of rows in the catalogue.

      Rectangular area: objects which lie within a given rectangle.
        (If the columns are spherical-polar coordinates, such as
        Right Ascension and Declination, rather than Cartesian
        coordinates then the sides of the rectangle become parallels
        and great circles.)

      Circular area: objects which lie within a given angular distance
        from a specified point.  This type of selection is only likely
        to be used on columns of celestial coordinates.

      Polygonal area: objects which lie inside (or outside) a given
        polygon.

      Every Nth entry: every Nth object from the catalogue.  This option
        is useful for producing a smaller, but representative, sample of
        a large catalogue.  Such a sample might then be investigated
        interactively in cases where the original catalogue was too
        large to be studied interactively.
   }
   \sstusage{
      catselect
   }
   \sstparameters{
      \sstsubsection{
         CATIN  = CHARACTER (read)
      }{
         Give the name of the input catalogue.
      }
      \sstsubsection{
         CATOUT  =  CHARACTER (read)
      }{
         Give the name of the output catalogue of selected objects.
      }
      \sstsubsection{
         CATREJ  =  CHARACTER (read)
      }{
         Give the name of the output catalogue of rejected objects.
      }
      \sstsubsection{
         SELTYP  =  CHARACTER (read)
      }{
         Enter the required type of selection ({\tt "}H{\tt "} for a list).
      }
      \sstsubsection{
         TRNFRM  =  LOGICAL (read)
      }{
         Transform criteria to catalogue system before selection?
      }
      \sstsubsection{
         TARGET  =  LOGICAL (read)
      }{
         Output the selection as a target list?
      }
      \sstsubsection{
         REJCAT  =  LOGICAL (read)
      }{
         Produce a second output catalogue containing the rejected objects?
      }
      \sstsubsection{
         EXPR  =  CHARACTER (read)
      }{
         Enter an expression defining the required selection.
      }
      \sstsubsection{
         PNAME  =  CHARACTER (read)
      }{
         Enter the name of column or parameter.
      }
      \sstsubsection{
         MINRNG  =  CHARACTER (read)
      }{
         Enter minimum value of the required range.

         If the column within which the range is being specified is
         not an angle then simply enter the required value.

         If the column is an angle then the value can be entered as
         either a decimal value in radians or a sexagesimal value in
         hours or degrees, minutes and seconds.  If a sexagesimal value
         is specified then the hours or degrees, minutes and seconds
         should be separated by a colon ({\tt '}:{\tt '}).  Optionally fractional
         seconds can be specified by including a decimal point and the
         required number of places of decimals.  An unsigned value is
         assumed to be in hours and a signed value in degrees (a
         negative angle cannot be specified in hours).  That is,
         a positive angle in degrees must be preceded by a plus sign.

         Examples: any of the following values could be entered to
         to specify an angle of 30 degrees:

             2:00:00.0   hours (decimal point included in seconds)
             2:00:00     hours (integer number of seconds)

           $+$30:00:00.0   degrees (decimal point included in seconds)
           $+$30:00:00     degrees (integer number of seconds)

             0.5235988   radians
      }
      \sstsubsection{
         MAXRNG  =  CHARACTER (read)
      }{
         Enter maximum value of the required range.

         If the column within which the range is being specified is
         not an angle then simply enter the required value.

         If the column is an angle then the value can be entered as
         either a decimal value in radians or a sexagesimal value in
         hours or degrees, minutes and seconds.  If a sexagesimal value
         is specified then the hours or degrees, minutes and seconds
         should be separated by a colon ({\tt '}:{\tt '}).  Optionally fractional
         seconds can be specified by including a decimal point and the
         required number of places of decimals.  An unsigned value is
         assumed to be in hours and a signed value in degrees (a
         negative angle cannot be specified in hours).  That is,
         a positive angle in degrees must be preceded by a plus sign.

         Examples: any of the following values could be entered to
         to specify an angle of 30 degrees:

             2:00:00.0   hours (decimal point included in seconds)
             2:00:00     hours (integer number of seconds)

           $+$30:00:00.0   degrees (decimal point included in seconds)
           $+$30:00:00     degrees (integer number of seconds)

             0.5235988   radians
      }
      \sstsubsection{
         FREQ  =  INTEGER (read)
      }{
         Every FREQth object will be selected.
      }
      \sstsubsection{
         XCOL  =  CHARACTER (read)
      }{
         Enter X coordinate column from input catalogue.
      }
      \sstsubsection{
         YCOL  =  CHARACTER (read)
      }{
         Enter Y coordinate column from input catalogue.
      }
      \sstsubsection{
         CATPOLY  =  CHARACTER (read)
      }{
         Give the name of the catalogue containing the polygon.
      }
      \sstsubsection{
         XPLCOL  =  CHARACTER (read)
      }{
         Enter X coordinate column from polygon catalogue.
      }
      \sstsubsection{
         YPLCOL  =  CHARACTER (read)
      }{
         Enter Y coordinate column from polygon catalogue.
      }
      \sstsubsection{
         INSIDE  =  LOGICAL (read)
      }{
         The objects either inside or outside the polygon may be selected.
      }
      \sstsubsection{
         XMIN  =  DOUBLE (read)
      }{
         Minimum X value for the required rectangle.

         If the X column within which the minimum is being specified is
         not an angle then simply enter the required value.

         If the column is an angle then the value can be entered as
         either a decimal value in radians or a sexagesimal value in
         hours or degrees, minutes and seconds.  If a sexagesimal value
         is specified then the hours or degrees, minutes and seconds
         should be separated by a colon ({\tt '}:{\tt '}).  Optionally fractional
         seconds can be specified by including a decimal point and the
         required number of places of decimals.  An unsigned value is
         assumed to be in hours and a signed value in degrees (a
         negative angle cannot be specified in hours).  That is,
         a positive angle in degrees must be preceded by a plus sign.

         Examples: any of the following values could be entered to
         to specify an angle of 30 degrees:

             2:00:00.0   hours (decimal point included in seconds)
             2:00:00     hours (integer number of seconds)

           $+$30:00:00.0   degrees (decimal point included in seconds)
           $+$30:00:00     degrees (integer number of seconds)

             0.5235988   radians
      }
      \sstsubsection{
         XMAX  =  DOUBLE (read)
      }{
         Maximum X value for the required rectangle.

         If the X column within which the maximum is being specified is
         not an angle then simply enter the required value.

         If the column is an angle then the value can be entered as
         either a decimal value in radians or a sexagesimal value in
         hours or degrees, minutes and seconds.  If a sexagesimal value
         is specified then the hours or degrees, minutes and seconds
         should be separated by a colon ({\tt '}:{\tt '}).  Optionally fractional
         seconds can be specified by including a decimal point and the
         required number of places of decimals.  An unsigned value is
         assumed to be in hours and a signed value in degrees (a
         negative angle cannot be specified in hours).  That is,
         a positive angle in degrees must be preceded by a plus sign.

         Examples: any of the following values could be entered to
         to specify an angle of 30 degrees:

             2:00:00.0   hours (decimal point included in seconds)
             2:00:00     hours (integer number of seconds)

           $+$30:00:00.0   degrees (decimal point included in seconds)
           $+$30:00:00     degrees (integer number of seconds)

             0.5235988   radians
      }
      \sstsubsection{
         YMIN  =  DOUBLE (read)
      }{
         Minimum Y value for the required rectangle.

         If the Y column within which the minimum is being specified is
         not an angle then simply enter the required value.

         If the column is an angle then the value can be entered as
         either a decimal value in radians or a sexagesimal value in
         hours or degrees, minutes and seconds.  If a sexagesimal value
         is specified then the hours or degrees, minutes and seconds
         should be separated by a colon ({\tt '}:{\tt '}).  Optionally fractional
         seconds can be specified by including a decimal point and the
         required number of places of decimals.  An unsigned value is
         assumed to be in hours and a signed value in degrees (a
         negative angle cannot be specified in hours).  That is,
         a positive angle in degrees must be preceded by a plus sign.

         Examples: any of the following values could be entered to
         to specify an angle of 30 degrees:

             2:00:00.0   hours (decimal point included in seconds)
             2:00:00     hours (integer number of seconds)

           $+$30:00:00.0   degrees (decimal point included in seconds)
           $+$30:00:00     degrees (integer number of seconds)

             0.5235988   radians
      }
      \sstsubsection{
         YMAX  =  DOUBLE (read)
      }{
         Maximum Y value for the required rectangle.

         If the Y column within which the minimum is being specified is
         not an angle then simply enter the required value.

         If the column is an angle then the value can be entered as
         either a decimal value in radians or a sexagesimal value in
         hours or degrees, minutes and seconds.  If a sexagesimal value
         is specified then the hours or degrees, minutes and seconds
         should be separated by a colon ({\tt '}:{\tt '}).  Optionally fractional
         seconds can be specified by including a decimal point and the
         required number of places of decimals.  An unsigned value is
         assumed to be in hours and a signed value in degrees (a
         negative angle cannot be specified in hours).  That is,
         a positive angle in degrees must be preceded by a plus sign.

         Examples: any of the following values could be entered to
         to specify an angle of 30 degrees:

             2:00:00.0   hours (decimal point included in seconds)
             2:00:00     hours (integer number of seconds)

           $+$30:00:00.0   degrees (decimal point included in seconds)
           $+$30:00:00     degrees (integer number of seconds)

             0.5235988   radians
      }
      \sstsubsection{
         RACOL  =  CHARACTER (read)
      }{
         Enter Right Ascension column from input catalogue.
      }
      \sstsubsection{
         DCCOL  =  CHARACTER (read)
      }{
         Enter Declination column from input catalogue.
      }
      \sstsubsection{
         RACEN  =  CHARACTER (read)
      }{
         The central Right Ascension.

         The value may be specified as either a sexagesimal value in
         hours or a decimal value in radians.  If the value is supplied
         as sexagesimal hours then the hours, minutes and seconds must
         be separated by a colon ({\tt '}:{\tt '}).  Optionally fractional
         seconds can be specified by including a decimal point and the
         required number of places of decimals.  A negative angle may
         be specified by preceding the value by a minus sign.

         Examples: an angle of 10 hours, 30 minutes and 15.3 seconds
         may be specified by entering either of the following two
         values:

             10:30:15.3     sexagesimal hours
             2.7500062      radians
      }
      \sstsubsection{
         DCCEN  =  CHARACTER (read)
      }{
         The central Declination.

         The value may be specified as either a sexagesimal value in
         degrees or a decimal value in radians.  If the value is supplied
         as sexagesimal degrees then the degrees, minutes and seconds must
         be separated by a colon ({\tt '}:{\tt '}).  Optionally fractional
         seconds can be specified by including a decimal point and the
         required number of places of decimals.  A negative angle may
         be specified by preceding the value by a minus sign.

         Examples: a negative angle of 33 degrees, 30 minutes and 15.2
         seconds may be specified by entering either of the following two
         values:

         \sstitemlist{

            \sstitem
               33:30:15.2     sexagesimal degrees

            \sstitem
               0.584759       radians
         }
      }
      \sstsubsection{
         RADIUS  =  CHARACTER (read)
      }{
         The selection radius.

         The value may be specified as either a sexagesimal value in
         minutes and seconds of arc or a decimal value in radians.  If
         a sexagesimal value is supplied then the minutes and seconds
         of arc must be separated by a colon ({\tt '}:{\tt '}).  Note that a colon
         must be present if the value is to interpretted as minutes of
         arc; if no colon is present it will be interpretted as radians.
         Optionally fractional seconds can be specified by including a
         decimal point and the required number of places of decimals.  A
         negative angle may be specified by preceding the value by a
         minus sign.

         Examples: a radius of two minutes of arc may be specified by
         entering either of the following two values:

             2:0            sexagesimal minutes of arc
             5.8178E-4      radians
      }
      \sstsubsection{
         EQUINX  =  CHARACTER (read)
      }{
         The equinox of the catalogue coordinates.

         The equinox is specified as a time system followed by the
         value in that system in years.  A single alphabetical character
         is used to identify each of the two time systems supported:
         B for Bessellian and J for Julian.  Optionally decimal
         fractions of a year may be specified by including a decimal
         point followed by the required fraction.

         Examples:
             J2000          Julian equinox 2000.
             B1950          Bessellian equinox 1950.
      }
      \sstsubsection{
         EPOCH  =  CHARACTER (read)
      }{
         The epoch of the catalogue coordinates.

         The epoch is specified as a time system followed by the
         value in that system in years.  A single alphabetical character
         is used to identify each of the two time systems supported:
         B for Bessellian and J for Julian.  Optionally decimal
         fractions of a year may be specified by including a decimal
         point followed by the required fraction.

         Examples:
             J1996.894      Julian epoch of 1996.894.
             B1955.439      Bessellian epoch of 1955.439.
      }
      \sstsubsection{
         RAC  =  CHARACTER (read)
      }{
         Enter the name of the Right Ascension column.
      }
      \sstsubsection{
         DECC  =  CHARACTER (read)
      }{
         Enter the name of the Declination column.
      }
      \sstsubsection{
         PMRAC  =  CHARACTER (read)
      }{
         Name of the proper motion in Right Ascension column (radians/year).
      }
      \sstsubsection{
         PMDEC  =  CHARACTER (read)
      }{
         Name of the proper motion in Declination column (radians/year).
      }
      \sstsubsection{
         PLXC  =  CHARACTER (read)
      }{
         Name of the parallax column (radians).
      }
      \sstsubsection{
         RVC  =  CHARACTER (read)
      }{
         Name of the radial velocity column (Km/sec).

         A positive value corresponds to an object which is red-shifted
         or receding and a negative value to one which is blue-shifted
         or approaching.
      }
      \sstsubsection{
         LABELC  =  CHARACTER (read)
      }{
         Name of the column used to label objects on plots.
      }
      \sstsubsection{
         NOROWS  =  LOGICAL (read)
      }{
         Flag indicating whether a selection which contains no rows is
         to be considered an error or not.  Coded as follows:
         .TRUE.  -  consider an error,
         .FALSE. -  do not consider an error (default).
         If NOROWS is set to .TRUE. then a selection with no rows will
         raise the status SAI\_\_WARN.
      }
      \sstsubsection{
         TEXT  =  CHARACTER (read)
      }{
         Flag indicating the textual header information to be copied to
         the output catalogues.  The valid responses are:
         A - all; the output catalogue will contain a complete copy
             of the header information for the input catalogue,
             duplicated as comments,
         C - (default) copy only the comments from the input catalogue.
             In the case of a FITS table the COMMENTS and HISTORY
             keywords will be copied.
         N - none; no textual header information is copied.
      }
      \sstsubsection{
         QUIET  =  LOGICAL (read)
      }{
         Operate in quiet mode where warnings are suppressed.  The
         permitted values are:
         TRUE  - quiet mode,
         FALSE - verbose mode.
      }
      \sstsubsection{
         NUMSEL  =  INTEGER (write)
      }{
         Number of rows selected.
      }
   }
   \sstexamples{
      \sstexamplesubsection{
         catselect
      }{
         The input and output catalogues and the type of selection
         required will be promted for.  Additional prompts specifiy
         the details of the selection.  Any comments in the input
         catalogue will be copied.
      }
      \sstexamplesubsection{
         catselect  text=all
      }{
         The input and output catalogues and the type of selection
         required will be promted for.  Additional prompts specifiy
         the details of the selection.  All the header information in
         the input catalogue will be duplicated as comments in the
         output catalogue.
      }
      \sstexamplesubsection{
         catselect  text=none
      }{
         The input and output catalogues and the type of selection
         required will be promted for.  Additional prompts specifiy
         the details of the selection.  Any comments in the input
         catalogue will not be copied.
      }
   }
}
\newpage
\sstroutine{
   CATSORT
}{
   Create a copy of a catalogue sorted on a specified column
}{
   \sstdescription{
      Create a copy of a catalogue sorted on a specified column.  Note
      that catsort generates a new sorted catalogue; it does not overwrite
      the original catalogue.  The new catalogue can be sorted into either
      ascending or descending order.  All the columns and parameters in the
      input catalogue are copied.  Optionally any textual information
      associated with the input catalogue can also be copied.

      Catalogues can be sorted on columns of any of the numeric data
      types.  They should not be sorted on columns of data type CHARACTER
      or LOGICAL.  If a catalogue is sorted on a column which contains null
      values then the rows for which the column is null will appear after
      all the rows with a valid value. The order of such rows is
      unpredictable.
   }
   \sstusage{
      catsort
   }
   \sstparameters{
      \sstsubsection{
         CATIN  =  CHARACTER (read)
      }{
         Name of the input catalogue.
      }
      \sstsubsection{
         CATOUT  =  CHARACTER (read)
      }{
         Name of the output catalogue, sorted on the specified column.
      }
      \sstsubsection{
         FNAME  =  CHARACTER (read)
      }{
         The name of the column the output catalogue is to be sorted on.
      }
      \sstsubsection{
         ORDER  =  CHARACTER (read)
      }{
         Order into which the catalogue is to be sorted: ascending or
         descending.
      }
      \sstsubsection{
         TEXT  =  CHARACTER (read)
      }{
         Flag indicating the textual header information to be copied.
         The valid responses are:
         A - all; the output catalogue will contain a complete copy
             of the header information for the input catalogue,
             duplicated as comments,
         C - (default) copy only the comments from the input catalogue.
             In the case of a FITS table the COMMENTS and HISTORY
             keywords will be copied.
         N - none; no textual header information is copied.
      }
      \sstsubsection{
         QUIET  =  LOGICAL (read)
      }{
         Operate in quiet mode where warnings are suppressed.  The
         permitted values are:
         TRUE  - quiet mode,
         FALSE - verbose mode.
      }
   }
   \sstexamples{
      \sstexamplesubsection{
         catsort
      }{
         The following parameters will be prompted for: input catalogue,
         output catalogue, name of the column to be sorted on and the
         order required (ascending or descending).  The sorted catalogue
         will then be created.  Any comments in the input catalogue will
         be copied.
      }
      \sstexamplesubsection{
         catsort  text=all
      }{
         The following parameters will be prompted for: input catalogue,
         output catalogue, name of the column to be sorted on and the
         order required (ascending or descending).  The sorted catalogue
         will then be created.  All the header information in the input
         catalogue will be duplicated as comments in the output
         catalogue.
      }
      \sstexamplesubsection{
         catsort  text=none
      }{
         The following parameters will be prompted for: input catalogue,
         output catalogue, name of the column to be sorted on and the
         order required (ascending or descending).  The sorted catalogue
         will then be created.  Any comments in the input catalogue will not
         be copied.
      }
      \sstexamplesubsection{
         catsort  input-catalogue  output-catalogue  column-name  ascending
      }{
         Here the all the required parameters have been specified on the
         command line.  Because no value was specified for parameter
         TEXT the default will be adopted and any comments in the input
         catalogue will be copied.
      }
   }
   \sstdiytopic{
      Pitfalls
   }{
      Catalogues should not be sorted on columns of data type CHARACTER
      or LOGICAL.
   }
}
\newpage
\sstroutine{
   CATVIEW
}{
   Application to browse and generate selections from a catalogue
}{
   \sstdescription{
      catview is an application for browsing catalogues and selecting
      subsets from the command line.  It provides facilities to:

      $*$ list the columns in a catalogue,

      $*$ list the parameters and textual information from a catalogue,

      $*$ list new columns computed on-the-fly using an algebraic
        expression defined in terms of existing columns and parameters.
        For example, if the catalogue contained columns V and B\_V
        (corresponding to the V magnitude and B-V colour) then the B
        magnitude  could be listed by specifying the expression
        V $+$ B\_V.

      $*$ fast creation of a subset within a specified range for a sorted
        column,

      $*$ creation of subsets defined by algebraic criteria. For example,
        if the catalogue again contained columns V and B\_V then
        to find the stars in the catalogue fainter than twelfth magnitude
        and with a B-V of greater than 0.5 the criteria would be
        V $>$ 12.0 .AND. B\_V $>$ 0.5,

      $*$ subsets extracted from the catalogue can be saved as new
        catalogues. These subsets can include new columns computed from
        expressions as well as columns present in the original catalogue,

      $*$ subsets extracted from the catalogue can be saved in a text file
        in a form suitable for printing, or in a form suitable for passing
        to other applications (that is, unencumbered with extraneous
        annotation).
   }
   \sstusage{
      catview
   }
   \sstparameters{
      \sstsubsection{
         CNAME  =  CHARACTER (read)
      }{
         Give the name of the catalogue to be reported.
      }
      \sstsubsection{
         ACTION  =  CHARACTER (read)
      }{
         Enter required action; HELP for a list of options.
      }
      \sstsubsection{
         CMPLST  =  CHARACTER (read)
      }{
         Enter list of columns and expressions, separated by semi-colons.
      }
      \sstsubsection{
         SELNO  =  INTEGER (read)
      }{
         Enter the number of the required selection.
      }
      \sstsubsection{
         EXPR  =  CHARACTER (read)
      }{
         Enter an expression defining the required selection.
      }
      \sstsubsection{
         MINRNG  =  CHARACTER (read)
      }{
         Enter minimum value of the required range.

         If the column within which the range is being specified is
         not an angle then simply enter the required value.

         If the column is an angle then the value can be entered as
         either a decimal value in radians or a sexagesimal value in
         hours or degrees, minutes and seconds.  If a sexagesimal value
         is specified then the hours or degrees, minutes and seconds
         should be separated by a colon (:).  Optionally fractional
         seconds can be specified by including a decimal point and the
         required number of places of decimals.  An unsigned value is
         assumed to be in hours and a signed value in degrees (a
         negative angle cannot be specified in hours).  That is,
         a positive angle in degrees must be preceded by a plus sign.

         Examples: any of the following values could be entered to
         to specify an angle of 30 degrees:

             2:00:00.0   hours (decimal point included in seconds)
             2:00:00     hours (integer number of seconds)

           $+$30:00:00.0   degrees (decimal point included in seconds)
           $+$30:00:00     degrees (integer number of seconds)

             0.5235988   radians
      }
      \sstsubsection{
         MAXRNG  =  CHARACTER (read)
      }{
         Enter maximum value of the required range.

         If the column within which the range is being specified is
         not an angle then simply enter the required value.

         If the column is an angle then the value can be entered as
         either a decimal value in radians or a sexagesimal value in
         hours or degrees, minutes and seconds.  If a sexagesimal value
         is specified then the hours or degrees, minutes and seconds
         should be separated by a colon (:).  Optionally fractional
         seconds can be specified by including a decimal point and the
         required number of places of decimals.  An unsigned value is
         assumed to be in hours and a signed value in degrees (a
         negative angle cannot be specified in hours).  That is,
         a positive angle in degrees must be preceded by a plus sign.

         Examples: any of the following values could be entered to
         to specify an angle of 30 degrees:

             2:00:00.0   hours (decimal point included in seconds)
             2:00:00     hours (integer number of seconds)

           $+$30:00:00.0   degrees (decimal point included in seconds)
           $+$30:00:00     degrees (integer number of seconds)

             0.5235988   radians
      }
      \sstsubsection{
         ROWNO  =  INTEGER (read)
      }{
         Enter the required row number in the current selection.
      }
      \sstsubsection{
         FIRSTR  =  INTEGER (read)
      }{
         Enter the first row to be listed in the current selection.
      }
      \sstsubsection{
         LASTR  =  INTEGER (read)
      }{
         Enter the last row to be listed (0 = last in the current selection).
      }
      \sstsubsection{
         FLNAME  =  CHARACTER (read)
      }{
         Enter the name of the output text file.
      }
      \sstsubsection{
         CATOUT  =  CHARACTER (read)
      }{
         Enter the name of the output catalogue.
      }
      \sstsubsection{
         CFLAG  =  LOGICAL (read)
      }{
         Columns to be saved:
         true - all columns;  false - only currently chosen.
      }
      \sstsubsection{
         TFLAG  =  LOGICAL (read)
      }{
         Save header text from base catalogue?  The permitted
         responses are:  true - save header;  false - do not save text.
      }
      \sstsubsection{
         COMM  =  CHARACTER (read)
      }{
         Enter comments to annotate the new catalogue.
      }
      \sstsubsection{
         PNAME  =  CHARACTER (read)
      }{
         Enter the name of column or parameter.
      }
      \sstsubsection{
         UNITS  =  CHARACTER (read)
      }{
         Enter the new units for the column or parameter.
      }
      \sstsubsection{
         EXFMT  =  CHARACTER (read)
      }{
         Enter the new external format for the column or parameter.
      }
      \sstsubsection{
         SWID  =  INTEGER (read)
      }{
         Enter the screen width in characters.
      }
      \sstsubsection{
         SHT  =  INTEGER (read)
      }{
         Enter the screen height in number of lines.
      }
      \sstsubsection{
         SEQNO  =  LOGICAL (read)
      }{
         Should a sequence number be listed with each row?
      }
      \sstsubsection{
         NLIST  =  INTEGER (read)
      }{
         Enter the number of lines for LIST to output; -1 for them all
      }
      \sstsubsection{
         ANGRPN  =  CHARACTER (read)
      }{
         Control the way in which angles are displayed.  The permitted
         responses are:  SEXAGESIMAL - sexagesimal hours or degrees,
         RADIANS - radians.
      }
      \sstsubsection{
         ANGRF  =  LOGICAL (read)
      }{
         Reformat the UNITS attribute for angles?
      }
      \sstsubsection{
         GUI  =  LOGICAL (read)
      }{
         Is the application being run from a GUI?
      }
      \sstsubsection{
         FPRINT  =  LOGICAL (read)
      }{
         Flag; is output file a print file or a data file, coded as follows:
         .TRUE.  -  print file,
         .FALSE. -  data file.
      }
      \sstsubsection{
         FPGSZE  =  INTEGER (read)
      }{
         Enter the number of lines in a page of output.
      }
      \sstsubsection{
         FWID  =  INTEGER (read)
      }{
         Enter the width of line in the output file, in characters.
      }
      \sstsubsection{
         FSUMM  =  CHARACTER (read)
      }{
         Include summary in text file?  The permitted responses are:
         A = absent, F = include summary.
      }
      \sstsubsection{
         FCOL  =  CHARACTER (read)
      }{
         Include column details in text file?  The permitted responses
         are:  A = absent, S = summary only, F = full details.
      }
      \sstsubsection{
         FPAR  =  CHARACTER (read)
      }{
         Include parameter details in text file?  The permitted responses
         are:  A = absent, S = summary only, F = full details.
      }
      \sstsubsection{
         FTXT  =  CHARACTER (read)
      }{
         Include header text in text file?  The permitted responses are:
         A = absent, F = include full text.
      }
      \sstsubsection{
         FTABL  =  CHARACTER (read)
      }{
         Include data table in text file?  The permitted responses are:
         A = absent, S = columns only, F = Columns and headings.
      }
      \sstsubsection{
         CMPSTT  =  CHARACTER (read)
      }{
         Enter list of columns separated by semi-colons.
      }
      \sstsubsection{
         DECPL  =  INTEGER (read)
      }{
         Enter the number of decimal places for displaying statistics.
         Note that this quantity controls only the precision with
         which the statistics are displayed, not the precision with
         which they are computed; they are computed as DOUBLE PRECISION
         numbers.
      }
      \sstsubsection{
         SFNAME  =  CHARACTER (read)
      }{
         Enter the name of the file to hold the column statistics.
      }
      \sstsubsection{
         GRPHDV  =  CHARACTER (read)
      }{
         Give the name of the graphics device.
      }
      \sstsubsection{
         TITLE  =  CHARACTER (read)
      }{
         Enter the title to be displayed on the plot.
      }
      \sstsubsection{
         XEXPR  =  CHARACTER (read)
      }{
         Enter column or expression defining the plot X-axis.
      }
      \sstsubsection{
         YEXPR  =  CHARACTER (read)
      }{
         Enter column or expression defining the plot Y-axis.
      }
      \sstsubsection{
         AUTOSCL  =  LOGICAL (read)
      }{
         Flag; is the scatter-plot to be auto-scaled?
      }
      \sstsubsection{
         CXMIN  =  CHARACTER (read)
      }{
         Minimum value to be plotted on X axis.
      }
      \sstsubsection{
         CXMAX  =  CHARACTER (read)
      }{
         Maximum value to be plotted on X axis.
      }
      \sstsubsection{
         CYMIN  =  CHARACTER (read)
      }{
         Minimum value to be plotted on Y axis.
      }
      \sstsubsection{
         CYMAX  =  CHARACTER (read)
      }{
         Maximum value to be plotted on Y axis.
      }
      \sstsubsection{
         PLTSYM  =  CHARACTER (read)
      }{
         Plotting symbol to be used in scatter-plot.
      }
      \sstsubsection{
         COLOUR  =  CHARACTER (read)
      }{
         Colour of the plotting symbols to be used in scatter-plot.
      }
      \sstsubsection{
         BINSP  =  LOGICAL (read)
      }{
         Histogram bin specification:
         TRUE  -  the bins are specified by their width,
         FALSE -  the total number of bins is specified.
      }
      \sstsubsection{
         BINDET  =  REAL (read)
      }{
         The details of the histogram bins.  If BINSP is TRUE then
         BINDET is the width of each bin.  If BINSP is FALSE then it
         is the total number of bins.
      }
      \sstsubsection{
         NORML  =  LOGICAL (read)
      }{
         Flag; is the histogram to be normalised?
      }
      \sstsubsection{
         QUIET  =  LOGICAL (read)
      }{
         Operate in quiet mode where warnings are suppressed.  The
         permitted values are:
         TRUE  - quiet mode,
         FALSE - verbose mode.
      }
   }
   \sstexamples{
      \sstexamplesubsection{
         catselect
      }{
         You will be placed in a command prompt where you enter commands
         to examine the catalogue and generate subsets of it.  Type
         HELP to see a list of commands.
      }
   }
   \sstdiytopic{
      Pitfalls
   }{
      catview is not really intended to be used interactively and is
      somewhat terse and inconvenient.  If possible you should use the
      GUI-based catalogue browser xcatview instead.  However, xcatview
      requires an X display and catview may be useful if you do not
      have one.  It may also be useful for running prepared scripts which
      perform routine, standard, batch type operations.
   }
}

% \section{References}

% -- References --------------------------------------------------------

% \input{refs.tex}
\newpage
\addcontentsline{toc}{section}{References}
\begin{thebibliography}{99}

  \bibitem{SERVERURL} M.~Albrecht, M.~Barylak, D.~Durand, P.~Fernique,
   A.~Micol, F.~Ochsenbein, F.~Pasian, B.~Pirenne, D.~Ponz and
   M.~Wenger, 19 September 1996, \textit{Astronomical Server URL}\,
   (Version 1.0).  See URL: \texttt{http://vizier.u-strasbg.fr/doc/asu.html}

  \bibitem{PPMS} U.~Bastian, S.~R\"{o}ser, V.V.~Nesterov, D.D.~Polozhentsev,
   Kh.I.~Potter, R.~Wielen, L.I.~Yagudin and Ya.S.~Yatskiv, 1991,
   \textit{Astron. Astrophys. Suppl}, \textbf{87}, pp159-162.

  \bibitem{SUN203} D.S.~Berry, G.J.~Privett and A.C.~Davenhall,
   15 September 1997, \xref{SUN/203.3}{sun203}{}: \textit{SX \& DX ---
   IBM Data Explorer for Data Visualisation}, Starlink.

  \bibitem{CRCMT} W.H. Beyer (editor), 1974, \textit{CRC Standard
   Mathematical Tables}, twenty-fourth edition (CRC Press: Cleveland,
   Ohio).

  \bibitem{SUN95} M.J.~Currie and D.S.~Berry, 20 October 2000,
   \xref{SUN/95.16}{sun95}{}: \textit{KAPPA -- Kernel Application Package},
   Starlink.

  \bibitem{SUN55} M.J.~Currie, G.J.Privett, A.J.Chipperfield, D.S.~Berry
   and A.C.~Davenhall, 21 September 2000, \xref{SUN/55.14}{sun55}{}: \textit{CONVERT --- A Format-conversion Package}, Starlink.

  \bibitem{SUN162} A.C.~Davenhall, 18 March 1993, \xref{SUN/162.1}{sun162}{}:
   \textit{A Guide to Astronomical Catalogues, Databases and Archives available
   through Starlink}, Starlink.

  \bibitem{SC2} A.C.~Davenhall, 1 October 1997, \xref{SC/2.3}{sc2}{}:
   \textit{The DX Cookbook}, Starlink.

  \bibitem{SSN75} A.C.~Davenhall, 26 July 2000,
   \xref{SSN/75.1}{ssn75}{}: \textit{Writing Catalogue and Image Servers for
   GAIA and CURSA}, Starlink.

  \bibitem{SUN181} A.C.~Davenhall, 4 April 2001,
   \xref{SUN/181.10}{sun181}{}: \textit{CAT --- Catalogue and Table Manipulation
   Library: Programmer's Manual}, Starlink.

  \bibitem{SSN76} A.C.~Davenhall, 24 May 2001,
   \xref{SSN/76.1}{ssn76}{}: \textit{CATREMOTE --- a Tool for Querying Remote
   Catalogues}, Starlink.

  \bibitem{SUN214} P.W.~Draper and N.~Gray, 16 October 2000,
   \xref{SUN/214.8}{sun214}{}: \textit{GAIA --- Graphical Astronomy and
   Image Analysis Tool}, Starlink.

  \bibitem{SUN109} P.W.~Draper and N.~Eaton, 24 May 1999,
   \xref{SUN/109.10}{sun109}{}: \textit{PISA -- Position Intensity and Shape
   Analysis}, Starlink.

  \bibitem{SUN45} N.~Eaton, P.W.~Draper and A.~Allan, 15 November 1999,
   \xref{SUN/45.10}{sun45}{}: \textit{PHOTOM -- A Photometry Package},
   Starlink.

  \bibitem{GREEN} R.M.~Green, 1985, \textit{Spherical Astronomy}\,
   (Cambridge University Press: Cambridge).

  \bibitem{HARDIE62} R.H.~Hardie, 1962, \textit{Photoelectric Reductions},
   Chapter 8 of \textit{Astronomical Techniques}, ed. W.A.~Hiltner, \textit{Stars and Stellar Systems}, \textbf{II}\, (University of Chicago Press:
   Chicago), pp178-208.  See especially p180.

  \bibitem{SUN137} P.~Harrison, P.~Rees and P.~Draper, 12 November 1997,
   \xref{SUN137.6}{sun137}{}: \textit{PONGO -- A Set of Applications for
   Interactive Data Plotting}, Starlink.

  \bibitem{KUNIT} P.~Kunitzsch and T.~Smart, 1986, \textit{Short Guide to
   Modern Star Names and Their Derivations}\, (Otto Harrassowitz:
   Wiesbaden).

  \bibitem{SUN194} H.~Meyerdierks, D.S.~Berry, P.W.~Draper, G.J.~Privett
   and M.J.~Currie, 14 February 1997, \xref{SUN/194.2}{sun194}{}: \textit{PDA
   --- Public Domain Algorithms}, Starlink.

  \bibitem{PMM} D.~Monet, A.~Bird, B.~Canzian, H.~Harris, N.~Reid,
   A.~Rhodes, S.~Sell, H.~Ables, C.~Dahn, H.~Guetter, A.~Henden,
   S.~Leggett, H.~Levison, C.~Luginbuhl, J.~Martini, A.~Monet, J.~Pier,
   B.~Riepe, R.~Stone, F.~Vrba and R.~Walker,
   1996, \textit{USNO-SA1.0}, (U.S. Naval Observatory: Washington DC).
   See also URL: \htmladdnormallink{
   \texttt{http://www.nofs.navy.mil/}}{http://www.nofs.navy.mil/}

  \bibitem{CDSTAND} F.~Ochsenbein, 12 September 1994, \textit{Astronomical Catalogues at CDS: Adopted Standards}, version 1.4, p14.
   Available on-line from the CDS (see Section~\ref{OBTAIN}).

  \bibitem{SC6} J.~Palmer and A.C.~Davenhall, 31 August 2001,
   \xref{SC/6.4}{sc6}{}: \textit{The CCD Photometric Calibration Cookbook},
   Starlink.

  \bibitem{PPMN} S.~R\"{o}ser and U.~Bastian, 1988, \textit{Astron.
   Astrophys. Suppl}, \textbf{74}, pp444-451.

  \bibitem{RUMBLE} J.R.~Rumble and F.J.~Smith, 1990, \textit{Database
   Systems in Science and Engineering}\, (Adam Hilger: Bristol).

  \bibitem{SCHOEN29} E.~Schoenberg, 1929, \textit{Hdb. d. Ap}, \textbf{2},
   (Julius Springer: Berlin), p268.

  \bibitem{SUN86} K.T.~Shortridge, H.~Meyerdierks, M.J.~Currie,
   M.J.~Clayton, J.~Lockley, A.C.~Charles, A.C.~Davenhall, M.B.~Taylor,
   T.~Ash, T.~Wilkins, D.~Axon, J.~Palmer, A.~Holloway and V.~Graffagnino,
   31 October 2001, \xref{SUN/86.19}{sun86}{}: \textit{FIGARO --- A General
   Data Reduction System}, Starlink.

  \bibitem{NGC2000} R.W.~Sinnott, 1988, \textit{NGC 2000.0}\, (Cambridge
   University Press: Cambridge and Sky Publishing Corporation: Cambridge,
   Massachusetts).

  \bibitem{SUN57} D.L.~Terrett and N.~Eaton, 12 July 1995,
   \xref{SUN/57.8}{sun57}{}: \textit{GNS -- Graphics Workstation Name Service},
   Starlink.

  \bibitem{VERON89} M.-P. Veron-Cetty and P. Veron, 1989, \textit{Catalogue
   of Quasars and Active Galactic Nuclei}, fourth edition (ESO Sci. Rep. 7).

  \bibitem{WALL79} J.V. Wall, 1979, `Practical Statistics for
   Astronomers', \textit{Q. J. R. Astron. Soc}, \textbf{20},
   pp138-152.

  \bibitem{SUN56} P.T.~Wallace, 21 June 1995, \xref{SUN/56.10}{sun56}{}:
   \textit{COCO --- Conversion of Celestial Coordinates}, Starlink.

  \bibitem{SUN67} P.T.~Wallace, 17 October 2000, \xref{SUN/67.51}{sun67}{}:
   \textit{SLALIB --- Positional Astronomy Library}, Starlink.

  \bibitem{SUN33} R.F.~Warren-Smith, 11 January 2000,
   \xref{SUN/33.7}{sun33}{}: \textit{NDF --- Routines for Accessing the
   Extensible N-Dimensional Data Format}, Starlink.

  \bibitem{SUN210} R.F.~Warren-Smith and D.S.~Berry, 23 May 2000,
   \xref{SUN/210.7}{sun210}{}: \textit{AST --- A Library for Handling World
   Coordinate Systems in Astronomy}\/ (Fortran Version), Starlink.

  \bibitem{SUN211} R.F.~Warren-Smith and D.S.~Berry, 23 May 2000,
   \xref{SUN/211.7}{sun211}{}: \textit{AST --- A Library for Handling World
   Coordinate Systems in Astronomy}\/ (C Version), Starlink.

\end{thebibliography}

% ----------------------------------------------------------------------

\typeout{  }
\typeout{*****************************************************}
\typeout{  }
\typeout{Reminder: run this document through Latex three times}
\typeout{to resolve the references.}
\typeout{  }
\typeout{*****************************************************}
\typeout{  }

\end{document}
