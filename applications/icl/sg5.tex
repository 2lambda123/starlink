\documentclass[twoside,11pt]{report}

% ? Specify used packages
% \usepackage{graphicx}        %  Use this one for final production.
% \usepackage[draft]{graphicx} %  Use this one for drafting.
% ? End of specify used packages

\pagestyle{myheadings}

% -----------------------------------------------------------------------------
% ? Document identification
% Fixed part
\newcommand{\stardoccategory}  {Starlink Guide}
\newcommand{\stardocinitials}  {SG}
\newcommand{\stardocsource}    {sg\stardocnumber}

% Variable part - replace [xxx] as appropriate.
\newcommand{\stardocnumber}    {5.2}
\newcommand{\stardocauthors}   {J A Bailey\footnote{Anglo-Australian
Observatory}\\A J Chipperfield}
\newcommand{\stardocdate}      {9th June 1998}
\newcommand{\stardoctitle}     {ICL\\[2.5ex]
                                The Interactive Command Language\\
                                for ADAM}
\newcommand{\stardocversion}   {Version 1.5-6}
\newcommand{\stardocmanual}    {User's Guide}

\newcommand{\stardocabstract}  {ICL is a language designed to provide
a programmable user interface to an astronomical data reduction or data
acquisition system. 
It is the primary user interface for the ADAM software environment.
\par
This document is a re-formatted version of SG/5.1 -- the text has not changed.
For information on ICL for Unix, ICL help is more reliable.}
% ? End of document identification
% -----------------------------------------------------------------------------

% +
%  Name:
%     sg.tex
%
%  Purpose:
%     Template for Starlink Guide (SG) documents.
%     Refer to SUN/199
%
%  Authors:
%     AJC: A.J.Chipperfield (Starlink, RAL)
%     BLY: M.J.Bly (Starlink, RAL)
%
%  History:
%     17-JAN-1996 (AJC):
%        Original with hypertext macros, based on MDL plain originals.
%     16-JUN-1997 (BLY):
%        Adapted for LaTeX2e.
%        Added picture commands.
%     {Add further history here}
%
% -

\newcommand{\stardocname}{\stardocinitials /\stardocnumber}
\markboth{\stardocname}{\stardocname}
\setlength{\textwidth}{160mm}
\setlength{\textheight}{230mm}
\setlength{\topmargin}{-2mm}
\setlength{\oddsidemargin}{0mm}
\setlength{\evensidemargin}{0mm}
\setlength{\parindent}{0mm}
\setlength{\parskip}{\medskipamount}
\setlength{\unitlength}{1mm}

% -----------------------------------------------------------------------------
%  Hypertext definitions.
%  ======================
%  These are used by the LaTeX2HTML translator in conjunction with star2html.

%  Comment.sty: version 2.0, 19 June 1992
%  Selectively in/exclude pieces of text.
%
%  Author
%    Victor Eijkhout                                      <eijkhout@cs.utk.edu>
%    Department of Computer Science
%    University Tennessee at Knoxville
%    104 Ayres Hall
%    Knoxville, TN 37996
%    USA

%  Do not remove the %begin{latexonly} and %end{latexonly} lines (used by 
%  star2html to signify raw TeX that latex2html cannot process).
%begin{latexonly}
\makeatletter
\def\makeinnocent#1{\catcode`#1=12 }
\def\csarg#1#2{\expandafter#1\csname#2\endcsname}

\def\ThrowAwayComment#1{\begingroup
    \def\CurrentComment{#1}%
    \let\do\makeinnocent \dospecials
    \makeinnocent\^^L% and whatever other special cases
    \endlinechar`\^^M \catcode`\^^M=12 \xComment}
{\catcode`\^^M=12 \endlinechar=-1 %
 \gdef\xComment#1^^M{\def\test{#1}
      \csarg\ifx{PlainEnd\CurrentComment Test}\test
          \let\html@next\endgroup
      \else \csarg\ifx{LaLaEnd\CurrentComment Test}\test
            \edef\html@next{\endgroup\noexpand\end{\CurrentComment}}
      \else \let\html@next\xComment
      \fi \fi \html@next}
}
\makeatother

\def\includecomment
 #1{\expandafter\def\csname#1\endcsname{}%
    \expandafter\def\csname end#1\endcsname{}}
\def\excludecomment
 #1{\expandafter\def\csname#1\endcsname{\ThrowAwayComment{#1}}%
    {\escapechar=-1\relax
     \csarg\xdef{PlainEnd#1Test}{\string\\end#1}%
     \csarg\xdef{LaLaEnd#1Test}{\string\\end\string\{#1\string\}}%
    }}

%  Define environments that ignore their contents.
\excludecomment{comment}
\excludecomment{rawhtml}
\excludecomment{htmlonly}

%  Hypertext commands etc. This is a condensed version of the html.sty
%  file supplied with LaTeX2HTML by: Nikos Drakos <nikos@cbl.leeds.ac.uk> &
%  Jelle van Zeijl <jvzeijl@isou17.estec.esa.nl>. The LaTeX2HTML documentation
%  should be consulted about all commands (and the environments defined above)
%  except \xref and \xlabel which are Starlink specific.

\newcommand{\htmladdnormallinkfoot}[2]{#1\footnote{#2}}
\newcommand{\htmladdnormallink}[2]{#1}
\newcommand{\htmladdimg}[1]{}
\newenvironment{latexonly}{}{}
\newcommand{\hyperref}[4]{#2\ref{#4}#3}
\newcommand{\htmlref}[2]{#1}
\newcommand{\htmlimage}[1]{}
\newcommand{\htmladdtonavigation}[1]{}

%  Starlink cross-references and labels.
\newcommand{\xref}[3]{#1}
\newcommand{\xlabel}[1]{}

%  LaTeX2HTML symbol.
\newcommand{\latextohtml}{{\bf LaTeX}{2}{\tt{HTML}}}

%  Define command to re-centre underscore for Latex and leave as normal
%  for HTML (severe problems with \_ in tabbing environments and \_\_
%  generally otherwise).
\newcommand{\latex}[1]{#1}
\newcommand{\setunderscore}{\renewcommand{\_}{{\tt\symbol{95}}}}
\latex{\setunderscore}

%  Redefine the \tableofcontents command. This procrastination is necessary 
%  to stop the automatic creation of a second table of contents page
%  by latex2html.
\newcommand{\latexonlytoc}[0]{\tableofcontents}

% -----------------------------------------------------------------------------
%  Debugging.
%  =========
%  Remove % on the following to debug links in the HTML version using Latex.

% \newcommand{\hotlink}[2]{\fbox{\begin{tabular}[t]{@{}c@{}}#1\\\hline{\footnotesize #2}\end{tabular}}}
% \renewcommand{\htmladdnormallinkfoot}[2]{\hotlink{#1}{#2}}
% \renewcommand{\htmladdnormallink}[2]{\hotlink{#1}{#2}}
% \renewcommand{\hyperref}[4]{\hotlink{#1}{\S\ref{#4}}}
% \renewcommand{\htmlref}[2]{\hotlink{#1}{\S\ref{#2}}}
% \renewcommand{\xref}[3]{\hotlink{#1}{#2 -- #3}}
%end{latexonly}
% -----------------------------------------------------------------------------
% ? Document specific \newcommand or \newenvironment commands.
% ? End of document specific commands
% -----------------------------------------------------------------------------
%  Title Page.
%  ===========
\renewcommand{\thepage}{\roman{page}}
\begin{document}
\thispagestyle{empty}

%  Latex document header.
%  ======================
\begin{latexonly}
   CCLRC / {\sc Rutherford Appleton Laboratory} \hfill {\bf \stardocname}\\
   {\large Particle Physics \& Astronomy Research Council}\\
   {\large Starlink Project\\}
   {\large \stardoccategory\ \stardocnumber}
   \begin{flushright}
   \stardocauthors\\
   \stardocdate
   \end{flushright}
   \vspace{-4mm}
   \rule{\textwidth}{0.5mm}
   \vspace{5mm}
   \begin{center}
   {\Huge\bf  \stardoctitle \\ [2.5ex]}
   {\LARGE\bf \stardocversion \\ [4ex]}
   {\Huge\bf  \stardocmanual}
   \end{center}
   \vspace{5mm}

% ? Add picture here if required for the LaTeX version.
%   e.g. \includegraphics[scale=0.3]{filename.ps}
% ? End of picture

% ? Heading for abstract if used.
   \vspace{10mm}
   \begin{center}
      {\Large\bf Abstract}
   \end{center}
% ? End of heading for abstract.
\end{latexonly}

%  HTML documentation header.
%  ==========================
\begin{htmlonly}
   \xlabel{}
   \begin{rawhtml} <H1> \end{rawhtml}
      \stardoctitle\\
      \stardocversion\\
      \stardocmanual
   \begin{rawhtml} </H1> \end{rawhtml}

% ? Add picture here if required for the hypertext version.
%   e.g. \includegraphics[scale=0.7]{filename.ps}
% ? End of picture

   \begin{rawhtml} <P> <I> \end{rawhtml}
   \stardoccategory \stardocnumber \\
   \stardocauthors \\
   \stardocdate
   \begin{rawhtml} </I> </P> <H3> \end{rawhtml}
      \htmladdnormallink{CCLRC}{http://www.cclrc.ac.uk} /
      \htmladdnormallink{Rutherford Appleton Laboratory}
                        {http://www.cclrc.ac.uk/ral} \\
      \htmladdnormallink{Particle Physics \& Astronomy Research Council}
                        {http://www.pparc.ac.uk} \\
   \begin{rawhtml} </H3> <H2> \end{rawhtml}
      \htmladdnormallink{Starlink Project}{http://star-www.rl.ac.uk/}
   \begin{rawhtml} </H2> \end{rawhtml}
   \htmladdnormallink{\htmladdimg{source.gif} Retrieve hardcopy}
      {http://star-www.rl.ac.uk/cgi-bin/hcserver?\stardocsource}\\

%  HTML document table of contents. 
%  ================================
%  Add table of contents header and a navigation button to return to this 
%  point in the document (this should always go before the abstract \section). 
  \label{stardoccontents}
  \begin{rawhtml} 
    <HR>
    <H2>Contents</H2>
  \end{rawhtml}
  \renewcommand{\latexonlytoc}[0]{}
  \htmladdtonavigation{\htmlref{\htmladdimg{contents_motif.gif}}
        {stardoccontents}}

% ? New section for abstract if used.
  \section{\xlabel{abstract}Abstract}
% ? End of new section for abstract
\end{htmlonly}

% -----------------------------------------------------------------------------
% ? Document Abstract. (if used)
%  ==================
\stardocabstract
% ? End of document abstract
% -----------------------------------------------------------------------------
% ? Latex document Table of Contents (if used).
%  ===========================================
  \newpage
  \thispagestyle{empty}
  \cleardoublepage
  \setcounter{page}{1}
  \begin{latexonly}
    \setlength{\parskip}{0mm}
    \latexonlytoc
    \setlength{\parskip}{\medskipamount}
    \markboth{\stardocname}{\stardocname}
  \end{latexonly}
  \newpage
  \thispagestyle{empty}
% ? End of Latex document table of contents
% -----------------------------------------------------------------------------

\cleardoublepage
\renewcommand{\thepage}{\arabic{page}}
\markboth{\stardocname}{\stardocname}
\setcounter{page}{1}

   \part{Introduction to ICL}
   \markboth{\stardocname}{\stardocname}
   \chapter{\xlabel{introduction}Introduction}
   \section{\xlabel{what_is_icl}What is ICL?}
The Interactive Command Language (ICL) is a language designed to provide
a programmable user interface to an astronomical data reduction or data
acquisition system. 
It is the primary user interface for the ADAM software environment and its use
with ADAM is described in Part II.

ICL is in some ways similar to a high level programming
language such as Fortran or Pascal, but it has some important differences.
   \begin{itemize}
   \item It is a {\em command} language. One of its main uses is to enable
the typing of commands with few restrictions on the possible command format
For example ICL can be used to run the FIGARO data reduction system and
it is possible to type FIGARO commands in
exactly the same format as was previously used from DCL.
   \item It is an {\em interactive} language. ICL provides a complete 
environment for entering, editing and debugging programs, rather than
relying on external editors, linkers {\em etc.} 
   \item ICL can be used as a programming language, but it is
intended for writing relatively simple and straightforward programs.
It's requirements are different from those of most modern programming languages,
which are designed for the needs of big software projects such as writing
operating systems or controlling missiles. ICL is designed to make simple
programs easy to write.
   \end{itemize}
   \section{\xlabel{icl_documentation}ICL Documentation}
This users' guide provides
an introduction to ICL for beginners, and should provide the information
needed to get started with it. The various appendices give further details
on many aspects of the language.

ICL also provides an on-line help system. Simply type HELP and a
list of topics will be displayed (it works in exactly the same way as
the DCL help system).

\section{\xlabel{icl_and_fortran}ICL and FORTRAN}
The reader of this manual is assumed to be familiar with programming in
FORTRAN, and the manual will compare ICL features with the corresponding 
features of FORTRAN where appropriate. 

\section{\xlabel{history_and_support}History and Support}
ICL was written by Jeremy Bailey, working at the Anglo-Australian Observatory
and subsequently at the Joint Astronomy Centre, Hilo.
Responsibility for support has now passed to the ADAM Support Group, part of 
the Science and Engineering Research Council's Starlink Project at the 
Rutherford Appleton Laboratory.

\chapter{\xlabel{getting_started}Getting Started}
\section{\xlabel{direct_mode}Direct Mode}
In this chapter we describe how to run ICL in direct mode and use its
variables, expressions {\em etc.} which enable us to use the VAX as a very
expensive electronic calculator. In direct mode we type in commands
and they are executed immediately. The alternative to direct mode is the
use of procedures, in which the commands are entered into a procedure,
and subsequently executed by running the procedure. 

ICL is started up by using the command ICL. After a short delay ICL will
respond by displaying any startup messages which have been set, followed by
the prompt \verb+ICL>+, meaning it is now ready to accept a command.
\begin{verbatim}
    $ ICL

      Interactive Command Language   -   Version 1.5-6

      - Type HELP [command] for help on ICL and its commands

    ICL>
\end{verbatim}
\section{\xlabel{entering_commands}Entering Commands}
When entering commands in ICL all the normal command line editing facilities
available in DCL may be used. The Up arrow or CTRL/B key may be used to recall
previous commands and the Down arrow key will step to the next command. Any
command back to the start of the ICL session may be recalled.

A command may take more than one line. A tilde \verb+~+ symbol at the end
of a line is used to indicate that the command continues on the next line.

\section{\xlabel{the_immediate_statement}The Immediate Statement}
The first ICL statement we will introduce is the immediate statement. This
is used to make ICL do simple calculations. It simply consists of an
equals sign (\verb+=+) followed by an expression, and causes the value of
the expression to be printed on the terminal.
\begin{verbatim}
    ICL> = 1 + 2 + 3
             6
    ICL> = SQRT(2)
    1.414214
    ICL>
\end{verbatim}

ICL arithmetic expressions are very similar to expressions in FORTRAN, and
will be discussed in more detail later.
\section{\xlabel{the_assignment_statement}The Assignment Statement}
The other commonly used statement is the assignment statement, which is
exactly the same as the FORTRAN assignment statement, and is used to
assign a value to a variable.
\begin{verbatim}
    ICL> PI = 3.1415926    
    ICL> = 2 * pi           
    6.283185
    ICL>
\end{verbatim}
Note that the case of letters doesn't matter. PI and pi are the same variable.
\section{\xlabel{expressions}Expressions}
Expressions are built up by operating on {\em values} using {\em operators}.
The values can be represented by either {\em constants} or {\em variables} or
can be the result of a {\em function} operating on another expression.
\subsection{\xlabel{values}Values}
ICL operates on values of four different types:
\begin{itemize}
\item Integer
\item Real
\item Logical
\item String
\end{itemize}
The integer, real and logical types are the same as their FORTRAN equivalents
(the ICL real type is strictly equivalent to FORTRAN's double precision being
stored with about 16 decimal digits of precision. The String type represents
strings of characters and is similar to the FORTRAN CHARACTER*n type.
\subsection{\xlabel{constants}Constants}
Values of all four types can be represented by appropriate constants.
Integer and real constants are represented by numbers written in the
same formats that are accepted in FORTRAN. Integer constants may also
be entered in binary, octal or hexadecimal format by preceding the
value with \verb+%B+, \verb+%O+ or \verb+%X+.

Logical constants are typed as \verb+TRUE+ or \verb+FALSE+. Note that they
are not delimited by decimal points as in FORTRAN.

String constants consist of any sequence of characters enclosed in
either single or double quotes. Two consecutive quote symbols in a
string are used to represent a single quote. String constants are the
one place in ICL where the case of letters is significant.

Some examples of constants:

\begin{verbatim}
    Real          1.234E-5       3.14159 
 
    Integer       123      %B100110     %O377      %Xffff

    Logical       TRUE     FALSE

    String        'This is a string'     "So is this"      ''
\end{verbatim}

The last example defines a string of zero length (valid in ICL but not in
FORTRAN).
\subsection{\xlabel{variables}Variables}
Variables in ICL are represented by names composed of characters which
may be letters, digits or the underscore character ( \verb+_+ ). The first
character must be a letter. The first 15 characters of a variable name
are significant ({\em i.e.} two variable names which are the same in the
first 15 characters, but differ in subsequent characters refer to the
same variable).

An important difference between ICL and FORTRAN is in the handling
of variable {\em types}. In FORTRAN each variable name has a unique type
associated with it, which is either derived implicitly from the first
letter of the name, or is explicitly specified in a declaration.

In ICL names do not have types, only values have types. A variables gains
a type when it is assigned a value. This type can change when a new value
is assigned to it. Thus we can have the following sequence of assignments
making the variable X an integer, real, logical and string in sequence.
\begin{verbatim}
    ICL> X = 123      
    ICL> X = 123.456    
    ICL> X = TRUE         
    ICL> X = 'String'   
\end{verbatim}
This approach to variable types means we do not have to declare the 
variables we use which helps to keep programs 
simple.\footnote{The disadvantage is that ICL cannot usually spot cases
where we accidentally mistype the name of a variable, as can languages which 
enforce declaration of variables (such as FORTRAN with the IMPLICIT NONE
directive). This is not thought to be a serious problem for the relatively
simple programs for which ICL is intended.}
\subsection{\xlabel{operators}Operators}
The operators which are used to build up expressions are listed in the following
table.
\begin{verbatim}
Priority

   1 (highest)              **
   2                        *   /
   3                        +   -
   4                        =   >   <   >=  <=  <>  :
   5 (lowest)               NOT   AND   OR   &
\end{verbatim}
The order of evaluation of expressions is determined by the priority
of the operators. The rules are the same as those in FORTRAN\footnote
{But different from those in Pascal} with arithmetic operators having
the highest priority, and logical operators the lowest. This means that
a condition such as \( 0 < X+Y \leq 1 \) can be expressed in ICL as follows
\begin{verbatim}
    X+Y > 0  AND  X+Y <= 1
\end{verbatim}
without requiring any parentheses.
In general however, it is good practice
to use parentheses to clarify any situation in which the order of evaluation
is in doubt.

In evaluating expressions ICL will freely apply type conversion to its
operands in order to make sense of them. This means not only that integers
will be converted to reals when required, but also that strings will be
converted to numbers when possible. A string can be converted to a numeric value
if the value of the string is itself a valid ICL expression. For example the
string \verb+'1.2345'+ has a numeric value. The string \verb-'X+1'- has a numeric
value if X is currently a numeric variable (or if X is a string which has
a numeric value).

The operator \& performs string concatenation -- operands with numeric values
will be converted to strings. The : operator is used for
formatting numbers into character strings as described below.
\subsection{\xlabel{functions}Functions}
ICL provides a variety of standard functions. Functions are written in
exactly the same way in ICL as in FORTRAN, and all the standard FORTRAN
77 generic functions which are relevant to the ICL data types are
provided in ICL with the
 same names as in FORTRAN. Thus SIN, COS, TAN,
ASIN, ACOS, ATAN, ATAN2, LOG, LOG10, EXP, SQRT and ABS are all valid
functions. A more complete list of functions is given in Appendix A.
\section{\xlabel{formatting_numbers_for_output}Formatting Numbers For Output}
While FORTRAN regards formatting of numbers for output as part of an
output operation, ICL performs formatting using an operator (:), which
produces a string result from a numeric operand. Thus if I is an integer
variable the expression \verb+I:5+ has as its value the string which is 
produced by converting I with a field width of 5 characters. Thus it
is equivalent to an I5 format in FORTRAN. Similarly if X is a real
variable the expression \verb+X:10:4+ produces the value of X formatted
in a FORTRAN F10.4 format ({\em i.e.} a field width of 10 characters, and
four decimal places). The ICL formatting is not precisely equivalent
to the FORTRAN form because ICL will extend the field width if a number
is too large to fit in the requested width.
\begin{verbatim}
    ICL> = 1.234567:5:2
     1.23
    ICL> = 12.34567:5:2
    12.35
    ICL> = 123.4567:5:2
    123.46
    ICL> = 123456.7:5:2
    123456.70
\end{verbatim}
Integers can also be formatted in binary, octal, decimal or hexadecimal formats
using the functions BIN, OCT, DEC and HEX. These have the form 
\verb+HEX(X,n,m)+ which would return a string of n characters containing
the number X with m significant digits. n and m may be omitted in which case
they default to the number of digits needed to represent a full 32 bit word.
Using these forms together with constants in various bases, ICL can be used to 
perform conversions between various bases.
\begin{verbatim}
    ICL> = %Xffff
          65535
    ICL> = hex(65535)
     0000FFFF
    ICL> = OCT(%XFF,5,5)
    00377
\end{verbatim}
\section{\xlabel{commands}Commands}
We have now met two of the three statement types available in direct mode,
the immediate and assignment statements. We now introduce the third and
most important one, the command. Commands are used for three purposes. 
\begin{itemize}
\item To provide features of the ICL system itself --- These are
commands such as PRINT, EDIT, LIST {\em etc.}
\item To call procedures written in ICL.
\item To provide the commands of the system ICL is being used
to run --- In this case ADAM.
\end{itemize}
\subsection{\xlabel{command_format}Command Format}
An ICL command consists of a command name,
which is formed using exactly the same rules as we described earlier for
variable names, followed by optionally, one or more parameters.
Parameters can take three forms:
\begin{itemize}
\item An expression or variable name enclosed in parentheses.
\item A string enclosed in quotes.
\item Any sequence of characters not including a space, quote, comma or
left parenthesis.
\end{itemize}
The parameters may be separated by commas, or by one or more spaces.

The first form of the parameter is used when we want to pass the value
of an expression to the command, or we want to give the command a variable
into which it will return a value. The other two forms both pass a string.

We can illustrate these various cases by using the PRINT command, which 
prints its parameters on the terminal.
\begin{verbatim}
    ICL> X=1.234
    ICL> PRINT (X)
    1.23400
    ICL> PRINT 'HELLO'
    HELLO
    ICL> PRINT HELLO
    HELLO
\end{verbatim}
In most cases therefore we do not need to use the quoted form of string parameters
because the simpler form will work. We need the quoted form of strings for those
cases in which we need to include a left parenthesis, or spaces in the string. 
Here is an example with several parameters.
\begin{verbatim}
    ICL> X=2
    ICL> PRINT The Square Root of (X) is (SQRT(X))
    The Square Root of          2 is 1.414214    
\end{verbatim}
What is happening here is that, since spaces are parameter separators, 
\verb+'The'+, \verb+'Square'+, \verb+'Root'+ and \verb+'of'+ are all
received by PRINT as independent parameters. However PRINT simply concatenates
all its parameters, with a space between each pair, and thus the result is the string
just as we typed it. Many other ICL commands which accept strings work in this
way. This means that strings with {\em single} spaces do not usually need quotes
when used as command parameters.
\subsection{\xlabel{summary_of_rules_for_command_parameters}Summary of Rules for Command Parameters}
The rules for specifying command parameters can be summarized as follows.
\begin{itemize}
\item To pass the value of an expression (which might be just a single
number), or the name of a variable, the expression or variable should be placed
in parentheses.
\item Anything not in parentheses is passed as a string, or as a sequence of
strings if it contains spaces or commas.
\item Any string which does not fit these restrictions, can be passed by
placing it in quotes.
\end{itemize}
FORTRAN programmers may find it useful to note that the parenthesized form
of the command parameters is exactly equivalent to the normal FORTRAN method
of passing parameters to a subroutine. Thus the ICL command
\begin{verbatim}
    COMMAND (A) (B) (C)
\end{verbatim}
is equivalent to the FORTRAN
\begin{verbatim}
    CALL COMMAND (A, B, C)
\end{verbatim}
{\em Any} parameter can always be passed in this way. The other forms of parameter
simply provide a convenient way of handling the common case of passing
a string constant.

These rules may appear somewhat confusing specified in this way, but what 
they achieve is to allow us to type many familiar commands in the way
we are used to. Thus the following are valid ICL commands:
\begin{verbatim}
    ICL> $DIR/SIZE/DATE/SINCE=TODAY

    ICL> SPLOT MYSPECT LOW=10 HIGH=100 \
\end{verbatim}


\chapter{\xlabel{icl_procedures}ICL Procedures}
An ICL Procedure is essentially the equivalent of a Subroutine in FORTRAN.
It allows us to write a sequence of ICL statements which can be run with
a single command. The procedure may have a number of parameters which are
used to pass values to the procedure, and return values from it.
\section{\xlabel{direct_entry_of_procedures}Direct Entry of Procedures}
To enter an ICL procedure we type a PROC command which specifies the name
of the procedure, and the names of any of its parameters. ICL then returns
a new prompt using the name of the procedure rather than \verb+ICL>+ to show
that we are in procedure entry mode. The statements that make up the procedure 
are then entered, followed by an \verb+END PROC+ or \verb+ENDPROC+ to mark the
end of the procedure.
\begin{verbatim}
    ICL> PROC SQUARE_ROOT X
    SQUARE_ROOT> {  An ICL procedure to print the square root of a number  }
    SQUARE_ROOT> PRINT The Square Root of (X) is (SQRT(X))
    SQUARE_ROOT> END PROC
    ICL>
\end{verbatim}
The line beginning with \{ is a comment which will be ignored by ICL when
executing the procedure. The closing \} is not necessary but looks neater.
\section{\xlabel{running_a_procedure}Running a Procedure}
To run the procedure we have entered we use the command format described
earlier, using the procedure name as the command, and adding any parameters
required.
\begin{verbatim}
    ICL> SQUARE_ROOT (2)
    The Square Root of         2 is 1.414214
\end{verbatim}
But we can also specify the parameter without parentheses:
\begin{verbatim}
    ICL> SQUARE_ROOT 2
    The Square Root of 2 is  1.414214
\end{verbatim}
What has happened here is that instead of the numeric value 2, we have
passed the string '2' as the parameter. However, the SQRT function requires
a numeric argument, so converts this string to a number. Thus, the free
approach to type conversion, means that in many cases the rules for command
parameters described in the previous chapter can be relaxed.
\begin{verbatim}
    ICL> Y=3
    ICL> SQUARE_ROOT Y
    The Square Root of Y is  1.732051
\end{verbatim}
In this case the parameter passed was the string 'Y', but once again this
was converted to a numeric value for the SQRT function.
\section{\xlabel{listing_procedures}Listing Procedures}
The LIST command can be used to list a procedure on the terminal. Just type
LIST followed by the name of the procedure you want to list.
\begin{verbatim}
    ICL> LIST SQUARE_ROOT


        PROC SQUARE_ROOT X
        {  An ICL procedure to print the square root of a number  }
        PRINT The Square Root of (X) is (SQRT(X))
        END PROC

    ICL>
\end{verbatim}
To find the names of all the current procedures use the command PROCS.
\begin{verbatim}
    ICL> PROCS

    SQUARE_ROOT

    ICL> 
\end{verbatim}
\section{\xlabel{editing_procedures}Editing Procedures}
Entering procedures directly is fine for very simple procedures, but for
anything more complex it is likely that some mistakes will be made in
entering the procedures. When this happens it will be necessary to 
{\em edit} the procedure. Editing is accomplished from within ICL using
standard editors. 
For example the command
\begin{verbatim}
    ICL> EDIT SQUARE_ROOT
\end{verbatim}
would be used to edit the SQUARE\_ROOT procedure. By default the TPU editor
is used. It is also possible to select the EDT or LSE editors using the 
SET EDITOR command. All these editors are described in DEC's documentation.

When editing a procedure there are two possible options.
\begin{itemize}
\item We can leave the name of the procedure unchanged, but edit the code. In
this case we create a new version of the procedure, which replaces the old
one when we exit from the editing session.
\item We can change the name of the procedure by editing the PROC statement
at the start of the procedure. This creates a new procedure with the
new name, and leaves the old procedure unchanged.
\end{itemize}

It is possible to enter procedures completely using the editors. However
it is recommended that procedures be entered originally using direct entry.
The advantage is that during direct entry, any errors will be detected immediately.
Thus if we mistype the PRINT line in the above example we get the following
error message:
\begin{verbatim}
    SQUARE_ROOT> PRINT The Square Root of (X is (SQRT(X))
    PRINT The Square Root of (X is (SQRT(X))
                                ^
    Right parenthesis expected
\end{verbatim}
The error message consists of the line in which the error was detected. A
pointer which indicates where in the line ICL had got to when it found
something was wrong, and a message indicating what was wrong. In this case
it found the \verb+'is'+ string when a right parenthesis was expected.

Following such an error message we can use the command line editing facility
to correct the line and reenter it. If the same error occurred during
procedure entry using an editor, the error message would only be generated
at the time of exit from the editing session, and it would be necessary to 
EDIT the procedure again to correct it.
\section{\xlabel{direct_commands_during_procedure_entry}Direct Commands During Procedure Entry}
It is sometimes useful to have a command directly executed while entering
a procedure. When using the direct entry method this can be done by
prefixing the command with a \% character. For example:
\begin{verbatim}
    ICL> PROC SQUARE_ROOT X
    SQUARE_ROOT> %HELP
\end{verbatim}
would give us on-line help information we might need to complete the
procedure. If the \% was omitted the HELP command would be included as
part of the procedure.
\section{\xlabel{saving_and_loading_procedures}Saving and Loading Procedures}
Procedures created in the above way will only exist for the duration of
an ICL session. If we need to keep them longer they need to be saved in
a disk file. This is achieved by means of the SAVE command. To save our
SQUARE\_ROOT procedure on disk we would use:
\begin{verbatim}
    ICL> SAVE SQUARE_ROOT
\end{verbatim}
To load it again, probably in a subsequent ICL session we would use
the LOAD command.
\begin{verbatim}
    ICL> LOAD SQUARE_ROOT
\end{verbatim}
The SAVE command causes the procedure to be saved in a file with
name SQUARE\_ROOT.ICL in the current default directory. LOAD will load
the procedure from the same file, also in the default directory. With
LOAD however it is possible to specify an alternative directory if
required:
\begin{verbatim}
    ICL> LOAD DISK$USER:[ABC]SQUARE_ROOT
\end{verbatim}
If you have many procedures you may not want to save and load them all
individually. It is possible to save all the current procedures using
the command:
\begin{verbatim}
    ICL> SAVE ALL
\end{verbatim}
This saves all the current procedures in a single file with the name SAVE.ICL.
These procedures may then be reloaded by the command:
\begin{verbatim}
    ICL> LOAD SAVE
\end{verbatim}
The SAVE ALL command is rarely used however, because this command is
executed automatically whenever you exit from ICL. This ensures that ICL
procedures do not get accidentally lost because you forget to save them.
\section{\xlabel{control_structures}Control Structures}
The ICL control structures provide a means of controlling the flow of
execution within a procedure. Unlike the statements we have met so far
these can only be used within a procedure and are not accepted in direct mode.
There are two types of control structure:
\begin{itemize}
\item The IF or conditional structure.
\item The LOOP structure.
\end{itemize}
\subsection{\xlabel{the_if_structure}The IF structure}
The IF structure is essentially the same as the block IF of FORTRAN. It
has the following general form:
\begin{verbatim}
    IF expression
        statements
    ELSE IF expression
        statements
    ELSE IF expression
        statements
    ELSE
        statements
    END IF
\end{verbatim}
The expressions (which must give logical values) are evaluated in turn until
one is found to be true, and the following statements are then executed.
If none of the expressions
are true the statements following ELSE are executed.

Every IF structure must begin with IF and end with END IF (or ENDIF). The
ELSE IF (ELSEIF) and ELSE clauses are optional, so the simplest IF structure
would have the form:
\begin{verbatim}
    IF expression
        statements
    ENDIF
\end{verbatim}
The following example illustrates the use of the IF structure, and shows how
one IF structure may be nested within another.
\begin{verbatim}
    PROC QUADRATIC A,B,C
    
    {  A Procedure to find the roots of the quadratic equation  }
    {  A * X**2 + B * X + C = 0                                 }
    
      IF A=0 AND B=0
        PRINT The equation is degenerate
      ELSE IF A=0
        PRINT Single Root is (-C/B)
      ELSE IF C=0
        PRINT The roots are (-B/A) and 0
      ELSE
        RE = -B/(2*A)
        DISCRIMINANT = B*B - 4*A*C
        IM = SQRT(ABS(DISCRIMINANT)) / (2*A)
        IF DISCRIMINANT >= 0
          PRINT The Roots are (RE + IM) and (RE - IM)
        ELSE
          PRINT The Roots are complex
          PRINT (RE) +I* (IM) and
          PRINT (RE) -I* (IM)
        ENDIF
      ENDIF
    END PROC
\end{verbatim}
\subsection{\xlabel{the_loop_structure}The LOOP Structure}
The LOOP structure is used to repeatedly execute a group of statements. It has
three different forms, the simplest being as follows:
\begin{verbatim}
    LOOP
        statements
    END LOOP
\end{verbatim}
This form sets up an infinite loop, but an additional statement, BREAK, may
be used to terminate the loop. BREAK would, of course, normally have to
be inside an IF structure.
\begin{verbatim}
    PROC COUNT
        {  A procedure to print the numbers from 1 to 10  }
        I = 1
        LOOP
            PRINT (I)
            I = I+1
            IF I > 10
                BREAK
            ENDIF
        ENDLOOP
    ENDPROC
\end{verbatim}
As it is frequently required to loop over a sequential range of numbers in
this way, a special form of the LOOP statement is provided for this purpose.
It has the following form:
\begin{verbatim}
    LOOP FOR variable = expression1 TO expression2 [ STEP expression3 ]
        statements
    END LOOP
\end{verbatim}
This form is essentially equivalent to the DO loop in FORTRAN. The expressions
specifying the range of values for the control variable are rounded to the
nearest integer so that the variable always has integer values. Using this 
form of the LOOP statement we can simplify the previous example as 
follows:
\begin{verbatim}
    PROC COUNT
        {  A procedure to print the numbers from 1 to 10  }
        LOOP FOR I = 1 TO 10
            PRINT (I)
        ENDLOOP
    ENDPROC
\end{verbatim}
Note that there is an optional STEP clause in the LOOP FOR statement. If
this is not specified a STEP of 1 is assumed. The STEP clause can be used
to specify a different value. A step of -1 must be specified to get a loop
which counts down from a high value to a lower value. For example:
\begin{verbatim}
    LOOP FOR I = 10 TO 1 STEP -1
\end{verbatim}
will count down from 10 to 1. 

The third form of the LOOP structure allows the setting up of loops which
terminate on any general condition. It has the form:
\begin{verbatim}
    LOOP WHILE expression
        statements                                              
    END LOOP
\end{verbatim}
The expression is evaluated each time round the loop, and if it has the logical
value TRUE the statements which form the body of the loop are executed. If
it has the value FALSE execution continues with the statement following
END LOOP.

Using this form we can write yet another version of the COUNT procedure:
\begin{verbatim}
    PROC COUNT
        {  A procedure to print the numbers from 1 to 10  }
        I = 1
        LOOP WHILE I <= 10
            PRINT (I)
            I = I+1
        ENDLOOP
    ENDPROC
\end{verbatim}
In the above case the LOOP WHILE form is more complicated than the LOOP FOR
form. However LOOP WHILE can be used to express more general forms of loop
where the termination condition is something derived inside the loop. An
example is a program which prompts the user for an answer to a question
({\em e.g.} yes or no) and has to keep repeating the prompt until a valid
answer is received.
\begin{verbatim}
    FINISHED = FALSE
    LOOP WHILE NOT FINISHED
        INPUT Enter YES or NO:  (ANSWER)
        FINISHED = ANSWER = 'YES' OR ANSWER = 'NO'
    END LOOP
\end{verbatim}    

\section{\xlabel{prompting_for_procedure_parameters}Prompting for Procedure Parameters}
Normally a procedure will not be obeyed unless the required number of
parameters is provided -- the TOOFEWPARS exception is signalled.
However, this check can be switched off using the NOCHECKPARS command. Then
the procedure will be executed with the parameter being undefined.
This fact can be used to write a procedure which will prompt for an undefined
parameter, {\em e.g.}:
\begin{quote} \begin{verbatim}
PROC SQUARE_ROOT X
{ An ICL procedure to print the square root of a number,   }
{ prompting if the number is not given on the command line }
  IF UNDEFINED(X)
    INPUT 'Give the value of X: ' (X)
  ENDIF
  PRINT The Square Root of (X) is (SQRT(X))
END PROC
\end{verbatim} \end{quote}
{\em Note that only trailing parameters may be omitted}

\section{\xlabel{variables_in_procedures}Variables in Procedures}
Any variable used within a procedure is completely distinct from a variable of
the same name used outside the procedure, or within a different procedure, as
can be seen in the following example:

\begin{verbatim}
    ICL> X = 1
    ICL> PROC FRED
    FRED> X = 1.2345
    FRED> =X
    FRED> END PROC
    ICL> FRED
    1.2345
    ICL> =X
            1
    ICL>
\end{verbatim}
When we run the procedure FRED we get the value of the variable X in the
procedure. Then typing =X gives the value of X outside the procedure which
has remained unchanged during execution of the procedure. This feature has the 
consequence that we can use procedures freely without having to worry about
any possible side effects of the procedure on variables outside it.

The situation is exactly the same as that in FORTRAN where variables in a
subroutine are local to the subroutine in which they are used. In FORTRAN
the COMMON statement is provided for use in cases where it is required to
extend the scope of a variable over more than one routine. ICL does not have
a COMMON facility but does provide an alternative mechanism for accessing
variables outside their scope using the command VARS and the function
VARIABLE.

The command VARS is used to list all the variables of a procedure.
It has one parameter, which is the name of the procedure. If the parameter
is omitted, then the outer level variables, {\em i.e.} those that are not part
of any procedure are listed. Thus in the previous example:
\begin{verbatim}
    ICL> VARS FRED
                     X  REAL     1.23450E+00
    ICL> VARS       
                     X  INTEGER          1
    ICL>
\end{verbatim}
VARIABLE is a function whose result is the value of a given variable in
a given procedure:
\begin{verbatim}
    ICL> = VARIABLE(FRED,X)
    1.234500
    ICL>
\end{verbatim}
and thus allows a variable belonging to a procedure to be accessed outside
that procedure.

Note that the variables belonging to a procedure continue to exist after
a procedure finishes execution, and if the procedure is executed a second time,
they will retain their values from the first time through the procedure.

\section{\xlabel{tracing_procedure_execution}Tracing Procedure Execution}
The commands SET TRACE and SET NOTRACE switch ICL in and out of trace mode.
When in trace mode each statement executed will be listed on the terminal.
Trace mode is very useful for debugging procedures. The commands can either be
issued from direct mode to turn on tracing for the entire execution of a
procedure, or inserted in the procedure itself, making it possible to trace
just part of its execution.
\section{\xlabel{running_icl_as_a_batch_job}Running ICL as a Batch Job}
It is sometimes useful to run one or more ICL procedures as a batch job.
It is quite easy to set this up using a parameter with the ICL command to
specify a file from which commands will be taken.
\begin{verbatim}
    $ ICL filename
\end{verbatim}
This form of the command is equivalent to typing ICL and then typing
\begin{verbatim}
    ICL> LOAD filename
\end{verbatim} 
Note, that as mentioned earlier, a LOAD file may include direct commands
as well as procedures. In order to create a Batch job we must set up a file
which contains all the procedures we want, a command (or commands) to run them
and an EXIT command to terminate the job. Here is the file for a simple 
Batch job to print a table of Square roots using our earlier example 
procedure. 
\begin{verbatim}

    PROC SQUARE_ROOT X
    {  An ICL procedure to print the square root of a number   }
    PRINT The Square Root of (X) is (SQRT(X))
    END PROC
    
    PROC TABLE
    
    { A procedure to print a table of square roots of numbers  }
    { from 1 to 100                                            }
    
      LOOP FOR I=1 TO 100
        SQUARE_ROOT (I)
      END LOOP
    END PROC

    {  Next the command to run this procedure  }

    TABLE

    {  And then an EXIT command to terminate the job  }

    EXIT

\end{verbatim}
This file can be generated using the EDIT command from DCL. If the
procedures have already been tested from ICL, it is convenient to use
a SAVE ALL command (or exit from ICL) to save them, and then edit the
SAVE.ICL file adding the additional direct commands.
Supposing this file is called TABLE.ICL. To create a batch job we also
need a command file, which we could call TABLE.COM which would contain
the following:
\begin{verbatim}

    $ ICL TABLE
    $ EXIT

\end{verbatim}
It might also need to contain a SET DEF command to set the appropriate
directory, or a directory specification on the TABLE file name if it is
not in the top level directory.

To submit the job to the batch queue the following command is used
\begin{verbatim}

    $ SUBMIT/KEEP TABLE

\end{verbatim}
The /KEEP qualifier specifies that the output file for the batch job is
to be kept. This file will appear as TABLE.LOG in the top level directory
and will contain the output from the batch job. A /OUTPUT qualifier
can be used to specify a different file name or directory for it.

\chapter{\xlabel{inputoutput}Input/Output}

\section{\xlabel{terminal_io}Terminal I/O}

We have already met the PRINT command for output to the terminal. The analogous
commands for terminal input are called INPUT, INPUTR, INPUTI and INPUTL.
INPUT reads a line of text from the terminal into a string variable and
has the form:
\begin{verbatim}

   ICL> INPUT Enter your name>  (name)

\end{verbatim}
The last parameter must specify a variable in which the input will be returned
and must therefore be in parentheses. The earlier parameters form a prompt
string.

INPUTR, INPUTI and INPUTL are used to input real, integer or logical values.
A single line of text may be used to supply values for more than one variables
so these commands have the form:
\begin{verbatim}

   ICL> INPUTR  Prompt  (X)  (Y)  (Z)

\end{verbatim}
Only the first parameter is used to provide the prompt string, so if it
has spaces in it the string must be enclosed in quotes.

INPUTL will accept values of TRUE, FALSE, YES and NO in either upper or
lower case, as well as abbreviations.

\section{\xlabel{text_file_io}Text File I/O}

ICL can also read and write text files. In order to access files they must
first be opened using one of the commands CREATE, OPEN or APPEND. 
\begin{verbatim}

    CREATE MYFILE

\end{verbatim}
will create a file called MYFILE and open it for output. MYFILE is the name
used within ICL for the file. The file will appear in your default VMS
directory as MYFILE.DAT.

The VMS file name may be specified explicitly by adding a second parameter
to the CREATE, OPEN or APPEND command.
\begin{verbatim}

    OPEN  INFILE  DISK$DATA:[ABC]FOR008.DAT

\end{verbatim}
opens an existing file for input and the file is known internally as INFILE.

The APPEND command opens an existing file for output. Anything written to
the file is appended to the existing contents.

A line of text is written to a text file with the WRITE command. WRITE
is similar to PRINT, the only difference being that its first parameter
specifies the internal name of the file to which the data will be written.

Text is read from files with the commands READ, READR, READI and READL.
These are analogous to the INPUT commands for terminal input. The first
parameter specifies the internal name of the file. READ reads a line of
text into a single string variable. The other commands read one or more
real, integer or logical values.

When a file is no longer required it may be closed using the CLOSE command
which has a single parameter, the internal name of the file.

The following example uses these procedures to read an input file containing
three real numbers in free format, and output the same numbers as a formatted
table.                                          

\begin{verbatim}
    PROC REFORMAT
      
    {  Open input file and create output file  }

        OPEN INFILE
        CREATE OUTFILE

    {  Loop copying lines from input to output  }
                                                      
        LOOP
           READR INFILE (R1) (R2) (R3)
           WRITE OUTFILE (R1:10:2) (R2:10:2) (R3:10:2)
        END LOOP

    END PROC

\end{verbatim}

Note that no specific test for completion of the loop is included. When an end
of file condition is detected on the input file the procedure will exit and
return to the ICL$>$ prompt with an appropriate message. \footnote{A tidier
exit can be arranged by using an exception handler for the EOF exception.}

\section{\xlabel{screen_mode}Screen Mode}

ICL's screen mode allows more control over the terminal screen than is possible
in the normal mode. It also allows more control over the use of the keyboard.
Screen mode is implemented using the DEC screen management (SMG\$) routines
of the run time library, and will work on any terminal compatible with these
routines.

Screen mode is selected by the SET SCREEN command. In screen mode the terminal
screen is divided into an upper fixed region and a lower scrolling region.
The size of the scrolling region may be specified by an optional parameter
to SET SCREEN.
\begin{verbatim}

    ICL> SET SCREEN 10

\end{verbatim}
will select 10 lines of scrolling region. The size of the scrolling region
can be changed by further SET SCREEN commands. SET NOSCREEN is used to leave
screen mode and return to normal mode.

Standard terminal I/O operations work exactly as normal in the scrolling
region of the screen. However, an additional facility is the ability to
examine text which has scrolled off the top of the scrolling region. The
Next Screen and Prev Screen keys on a VT200 (or CTRL/N, CTRL/P on other
terminals) may be used to move through the text. Any output since screen
mode was started is available in this way.

To write to the fixed part of the screen the command LOCATE is used.
\begin{verbatim}

    LOCATE 6 10   This text will be written starting at Row 6 Column 10
            
\end{verbatim}
The first two parameters specify the row and column at which the text will
start. The remaining parameters form the text to be written.

The SET ATTRIBUTES command provides further control over text written with
the LOCATE command. This command has a parameter string composed of any
combination of the letters R (Reverse Video), B (Bold), U (Underlined),
F (Flashing) and D (Double Size). These attributes apply to all LOCATE commands
until the next SET ATTRIBUTES command. SET ATTRIBUTES with no parameter
gives normal text.

The CLEAR command is used to clear all or part of the fixed region of the
screen. For example:
\begin{verbatim}
    CLEAR 6 10
\end{verbatim}
clears lines 6 to 10 of the screen.

\section{\xlabel{keyboard_facilities_in_screen_mode}Keyboard Facilities in Screen Mode} 

The KEY command may be used to define an equivalence string for any key
on the keyboard. For example:
\begin{verbatim}
    KEY  PF1  PROCS#
\end{verbatim}
defines the PF1 key so as to issue the PROCS command. The \# character is
used to indicate a Return character in the equivalence string.

The name of a main keyboard key may be specified as a single character or
as an integer representing the ASCII code for the key. Keypad of Function
keys are specified by names as follows:

Keypad keys --- PF1, PF2, PF3, PF4, KP0, KP1, KP2, KP3, KP4, KP5, KP6, KP7,
KP8, KP9, ENTER, MINUS, COMMA, PERIOD.

Function Keys (VT200) --- F6, F7, F8, F9, F10, F11, F12, F13, F14, HELP,
DO, F17, F18, F19, F20.

Editing Keypad (VT200) --- FIND, INSERT\_HERE, REMOVE, SELECT, PREV\_SCREEN,
NEXT\_SCREEN.

Cursor Keys --- UP, DOWN, LEFT, RIGHT.

KEYTRAP and INKEY allow an ICL procedure to test for keyboard input during
its execution, without having to issue an INPUT command and thus wait for
input to complete. KEYTRAP specifies the name of a key to be trapped. INKEY
is an integer function which returns zero if no key has been pressed or
the key value if a key has been pressed since the last call. The key value
is the ASCII value for ASCII characters, or a number between 256 and 511
for keypad and function keys. The KEYVAL function may be used to obtain
the value from the key name.

\begin{verbatim}
                 
{  Trap ENTER and LEFT and RIGHT arrow keys  }

    KEYTRAP ENTER
    KEYTRAP LEFT
    KEYTRAP RIGHT
                                              
    LOOP

      K = INKEY()
      IF K = KEYVAL('ENTER') THEN
      .
      ELSE IF K = KEYVAL('LEFT') THEN
      .
      ELSE IF K = KEYVAL('RIGHT') THEN
      .
      ELSE
      .
      ENDIF

    END LOOP
       
\end{verbatim}

The command
\begin{verbatim}

 KEYOFF keyname

\end{verbatim}
 may be used to turn off trapping of a key.
    

\chapter{\xlabel{access_to_dcl}Access to DCL}
When working from ICL it is frequently useful to be able to access features of
Digital's command language DCL. 
Typical operations we may want to do include listing directories, copying
files, allocating tape drives and mounting tapes.

\section{\xlabel{the_command}The \$\ Command}
The command \$ allows any DCL command to be issued from inside ICL. It's
form is simply:
\begin{verbatim}

    $ dcl_command

\end{verbatim}
where dcl\_command is any command we could issue from the DCL \$ prompt.
For example:

\begin{verbatim}

    $ COPY *.DST DATADIR:*.DST
    $ RUN MYPROGRAM

\end{verbatim}
There is one restriction --- we must use a complete DCL command. We couldn't
for example, just type \$ COPY and let DCL prompt us for the two file
specifications as we could from the DCL \$ prompt. Apart from this any
command acceptable to DCL can be issued in this way.

`DCL' may be used as an alternative to the \$ command. Thus the
above example could also have been written as:

\begin{verbatim}

    DCL COPY *.DST DATADIR:*.DST
    DCL RUN MYPROGRAM

\end{verbatim}

\section{\xlabel{the_spawn_command}The SPAWN Command}
There is also a way round the restriction mentioned above. This is to 
use the command SPAWN rather than the \$ command. For example:

\begin{verbatim}

    ICL> SPAWN COPY
    _From: *.DST
    _To: DATADIR:*.DST

\end{verbatim}
in which case we get the From: and To: prompts just as we do in normal DCL.
The disadvantage of SPAWN is that it is normally much slower. This is because
SPAWN creates a new subprocess to issue each command, whereas DCL creates
a permanent subprocess in which all commands are issued.

SPAWN has another use --- by just typing SPAWN we can get a DCL \$ prompt
from which a series of DCL commands can be executed. LOGOUT is then used
to return control to ICL.

\section{\xlabel{changing_the_default_directory}Changing the Default Directory}
It might seem that the above facilities provide all we need. Unfortunately
things are not that simple. The problem is that VMS only provides the facility
to issue a DCL command in a subprocess, not in the process we are actually
running ICL in. Thus although we can issue any DCL command we cannot issue
DCL commands in the process we are running ICL in. In many cases this does not
matter, the command will have the same effect whatever process it is issued
from.

However, this is not always the case. One example is changing the default
directory --- this can be done using the ICL command \$ SET DEFAULT. This 
will change the default directory of the DCL subprocess, but not of the 
process running ICL.

Thus an additional ICL command \verb+DEFAULT+ (which may be abbreviated
to \verb+DEF+) has been provided. This changes the default directory of
both the process running ICL and the DCL subprocess (if one exists). The
format for specifying the directory is exactly the same as that accepted
by the DCL SET DEFAULT command.

\section{\xlabel{allocating_and_mounting_tape_drives}Allocating and Mounting Tape Drives}
Similar problems occur when allocating and mounting tape drives. 
\$ ALLOCATE will allocate the device to the DCL subprocess. This {\em may}
be what you want, for example, if you are going to use another DCL
command (such as BACKUP) to read or write the tape. However if the tape
is to be processed using a FIGARO command it must be allocated to the
process running ICL.

A set of commands has been provided for this purpose as follows:

\begin{center}
\begin{tabular}{lll}
\\
command & abbreviation & function\\
\\
ALLOC dev & ALL & allocate a device\\
MOUNT dev & MOU & mount a device\\
DISMOUNT dev & DISMOU & dismount a device\\
DEALLOC dev & DEALL & deallocate a device\\
\\
\end{tabular}
\end{center}
Mount performs a MOUNT/FOREIGN at the tapes initialized density. It does
not provide the many qualifiers of the DCL command. There are several
additional optional parameters for some of these commands. ALLOC may
specify a generic name, and the name of the device actually allocated
will be returned in the optional second parameter.
\begin{verbatim}

    ICL> ALL MT
    _MTA0: Allocated
    ICL> ALL MT (DEVICE)
    _MTA1: Allocated
    ICL> =DEVICE
    _MTA1:

\end{verbatim}
DISMOUNT has an optional parameter which is used to specify that the
tape be dismounted without unloading.
\begin{verbatim}

    ICL> DISMOU MTA1 NOUNLOAD

\end{verbatim}

\chapter{\xlabel{errors_and_exceptions}Errors and Exceptions}

\section{\xlabel{icl_exceptions}ICL Exceptions}

Error conditions and other unexpected events are referred to as Exceptions
in ICL. When such a condition is detected in direct mode a message is 
output. For example, if we enter a statement which results in an error
\begin{verbatim}
    ICL> =SQRT(-1)
    SQUROONEG   Square Root of Negative Number
    ICL>
\end{verbatim}

We get a message consisting of the name of the exception (SQUROONEG) and
a description of the nature of the exception. A full list of ICL
exceptions is given in appendix C.

If the error occurs within a procedure the message contains a little
more information. As an example, if we use our square root procedure of
chapter 3 with an invalid value we get the following messages:
\begin{verbatim}
    ICL> SQUARE_ROOT (-1)
    SQUROONEG   Square Root of Negative Number
    In Procedure: SQUARE_ROOT
    At Statement: PRINT  The Square Root of (X) is (SQRT(X))
    ICL>
\end{verbatim}

If one procedure is called by another, the second procedure will also
be listed in the error message. If we run the following procedure

\begin{verbatim}
    PROC TABLE
    {  Print a table of Square roots from 5 down to -5  }
       LOOP FOR I = 5 TO -5 STEP -1
         SQUARE_ROOT (I)
       END LOOP
    END PROC
\end{verbatim}

we get

\begin{verbatim}
    ICL> TABLE
    The Square Root of 5 is 2.236068
    The Square Root of 4 is 2
    The Square Root of 3 is 1.732051
    The Square Root of 2 is 1.414214
    The Square Root of 1 is 1
    The Square Root of 0 is 0
    SQUROONEG   Square Root of Negative Number
    In Procedure: SQUARE_ROOT
    At Statement: PRINT  The Square Root of (X) is (SQRT(X))
    Called by: TABLE
    ICL>
\end{verbatim}

\section{\xlabel{exception_handlers}Exception Handlers}

It is often useful to be able to modify the default behaviour on an
error condition. We may not want to output an error message and return
to the ICL$>$ prompt, but rather to handle the condition in some other way.
This can be done by writing an {\em exception handler}. Here is an
example of an exception handler in the SQUARE\_ROOT procedure.

\begin{verbatim}
    PROC SQUARE_ROOT X
    {  An ICL procedure to print the square root of a number   }
      PRINT The Square Root of (X) is (SQRT(X))

      EXCEPTION SQUROONEG
        {  Handle the imaginary case  }
        SQ = SQRT(ABS(X))
        PRINT The Square Root of (X) is (SQ&'i')
      END EXCEPTION

    END PROC
\end{verbatim}

Now running the TABLE procedure gives

\begin{verbatim}
    ICL> TABLE
    The Square Root of 5 is 2.236068
    The Square Root of 4 is 2
    The Square Root of 3 is 1.732051
    The Square Root of 2 is 1.414214
    The Square Root of 1 is 1
    The Square Root of 0 is 0
    The Square Root of -1 is 1i
    The Square Root of -2 is 1.414214i
    The Square Root of -3 is 1.732051i
    The Square Root of -4 is 2i
    The Square Root of -5 is 2.236068i
    ICL>
\end{verbatim}
The Exception handler has two effects. First the code contained in the 
exception handler is executed when the exception occurs. Second, the procedure
exits normally to its caller (in this case TABLE) rather than aborting
execution completely and returning to the ICL$>$ prompt.

Exception handlers are included in a procedure following the normal code
of the procedure but before the END PROC statement. There may be any number
of exception handlers in a procedure, each for a different exception.
The exception handler begins with an EXCEPTION statement specifying the
exception name, and finishes with an END EXCEPTION statement. Between
these may be any ICL statements, including calls to other procedures.

An exception handler does not have to be in the same procedure in which the
exception occurred, but could be in a procedure further up in the chain of 
calls. In our example we could put an exception handler for SQUROONEG in
TABLE rather than in SQUARE\_ROOT.

\begin{verbatim}
    PROC TABLE
    {  Print a table of Square roots from 5 down to -5  }
       LOOP FOR I = 5 TO -5 STEP -1
         SQUARE_ROOT (I)
       END LOOP

       EXCEPTION SQUROONEG
          PRINT 'Can''t handle negative numbers - TABLE Aborting'
       END EXCEPTION
    END PROC
\end{verbatim}

giving:

\begin{verbatim}
    ICL> TABLE
    The Square Root of 5 is 2.236068
    The Square Root of 4 is 2
    The Square Root of 3 is 1.732051
    The Square Root of 2 is 1.414214
    The Square Root of 1 is 1
    The Square Root of 0 is 0
    Can't handle negative numbers - TABLE aborting
    ICL>
\end{verbatim}

Below is an example of a pair of procedures which use an exception handler
for floating point overflow in order to locate the largest floating point 
number allowed on the system. Starting with a value of 1 this is multiplied
by 10 repeatedly until floating point overflow occurs. The highest value
found in this way is then multiplied by 1.1 repeatedly until overflow occurs.
Then by 1.01 {\em etc.}

\begin{verbatim}
    PROC LARGE  START, FAC, L

    { Return in L the largest floating point number before        }
    { overflow occurs when START is repeatedly multiplied by FAC  }

      L = START
      LOOP
        L = L * FAC
      END LOOP
   
      EXCEPTION FLTOVF
      { This exception handler doesn't have any code - it just  }
      { causes the procedure to exit normally on overflow       }
      END EXCEPTION
    END PROC

    PROC LARGEST

    {  A Procedure to find the largest allowed floating point   }
    {  number on the system                                     }

      FAC = 10.0
      LARGE  1.0,(FAC),(L)
      LOOP WHILE FAC > 0.00000001
         LARGE (L),(1.0+FAC),(L)
         FAC = FAC/10.0
      END LOOP

      PRINT  The largest floating point number allowed is (L)
    END PROC
\end{verbatim}          



\section{\xlabel{keyboard_aborts}Keyboard Aborts}

One exception which is commonly encountered is that which results when
a Control-C is entered on the terminal. This results in the exception
CTRLC and may therefore be used to abort execution of a procedure and 
return ICL to direct mode. However, an exception handler for CTRLC may
be added to a procedure to modify the behaviour when a control-C is
typed.

\section{\xlabel{signal_command}SIGNAL command}

The exceptions described up to now have all been generated internally by the 
ICL system, or in the case of CTRLC are initiated by the user. It is also
possible for ICL procedures to generate exceptions, which may be used to
indicate error conditions. This is done by using the SIGNAL command. This
has the form
\begin{verbatim}
    SIGNAL  name  text
\end{verbatim}
where name is the name of the exception, and text is the message text 
associated with the exception. The exception name may be any valid ICL
identifier. Exceptions generated by SIGNAL work in exactly the same way
as the standard exceptions listed in appendix C. An exception handler
will be executed if one exists, otherwise an error message will be output
and ICL will return to direct mode.

One use of the SIGNAL command is as a means of escaping from deeply nested
loops. The BREAK statement can be used to exit from a single loop but is
not applicable if two or more loops are nested. In these cases the following
structure could be used
\begin{verbatim}
    LOOP
      LOOP
        LOOP
          .
          IF FINISHED
            SIGNAL ESCAPE
          END IF
          .
        END LOOP
      END LOOP
    END LOOP

    EXCEPTION ESCAPE
    END EXCEPTION
\end{verbatim}

where the exception handler again contains no statements, but simply exists
to cause normal procedure exit, rather than an error message when the 
exception is signalled.

\chapter{\xlabel{extending_icl}Extending ICL}

If you use ICL frequently you will find it convenient to define your own
commands. We have already met one way of defining new commands. This is
to write an ICL procedure as described in Chapter 3. There are several other
ways of defining new commands which are described in this chapter. If you
have many commands which you want to use frequently it is convenient to
put these into a login file which will be automatically loaded each time
ICL is started up.

\section{\xlabel{login_files}Login files}

An ICL login file works in exactly the same way as your DCL LOGIN.COM file. ICL
uses the logical name ICL\_LOGIN to locate your login file so if you create a
file called LOGIN.ICL in your top level directory you can use use the following
DCL DEFINE command (which you should put in your DCL LOGIN.COM file).
\begin{verbatim}
   $ DEFINE ICL_LOGIN DISK$USER:[ABC]LOGIN.ICL
\end{verbatim}
where DISK\$USER:$[$ABC$]$ needs to be replaced by the actual directory used.

This file will then be loaded automatically whenever you start up ICL and
can include procedures, definitions of commands, or indeed, any valid ICL
commands. Below is an example of a login file which illustrates some of
the facilities which may be used.
\begin{verbatim}

    {  ICL  Login File  }

    {  Define TYPE command  }

    DEFSTRING  T(YPE)  $ TYPE

    {  Define EDIT command  }

    HIDDEN PROC EDIT name
       IF INDEX(name,'.') = 0
          #EDIT (name)
       ELSE
          DCL EDIT (name)
       ENDIF
    END PROC

    {  Login Message  }

    PRINT 
    PRINT   Starting ICL at (TIME()) on (DATE())
    PRINT

\end{verbatim}
   
\section{\xlabel{the_defstring_command}The DEFSTRING Command}

The first entry in the file defines the DCL TYPE command so that it is
accessible directly from ICL without having to do enter DCL TYPE. This is
done using the DEFSTRING command which defines an equivalence string for
a command. In this case the command is TYPE and the equivalence string is
DCL TYPE. The notation T(YPE) specifies possible abbreviations for the
TYPE command, indicating that we could actually use T, TY or TYP instead
of TYPE in full.

\section{\xlabel{hidden_procedures}Hidden Procedures}

The definition of the EDIT command is done in a different way. Since EDIT
is used within ICL to edit procedures, if we just used DEFSTRING to define
EDIT as \$ EDIT we would lose the ability to edit ICL procedures. The EDIT
command would always edit VMS files.

The procedure used to redefine EDIT gets round this by testing for the
existence of a dot in the name to be edited using the INDEX function. If
a dot is present it assumes that a VMS file is being edited and issues the
command \$ EDIT (name). If no dot is present it is assumed that an ICL
procedure is being edited, and the command \#EDIT (name) is issued. The \#     
character forces the internal definition of EDIT to be used, rather than
the definition currently being defined.

The procedure is written as a {\em hidden} procedure, indicated by the word
HIDDEN preceding PROC. A hidden procedure works in exactly the same way
as a normal procedure, but it does not appear in the listing of procedures
produced by a PROCS statement, nor can it be edited, deleted or saved. It
is convenient to make all procedures in your login file hidden procedures
so that they do not clutter your directory of procedures, and cannot be
accidentally deleted.

\section{\xlabel{the_defproc_command}The DEFPROC Command}

If you have many procedures of this type you may not wish to include them
in full in your login file, because this will require them all to be compiled
when ICL starts up and may therefore slow down the start up process. The
DEFPROC command allows you to define commands which run ICL procedures,
but with the procedures only being compiled when they are required. For
example if the EDIT procedure described above was put in the source file
EDIT.ICL we could put the following DEFPROC command in the login file.

\begin{verbatim}
    DEFPROC ED(IT) EDIT.ICL
\end{verbatim}

This command specifies that the command EDIT (with minimum abbreviation
ED) is to run the procedure EDIT in source file EDIT.ICL. 
                   
The procedure will not be loaded and compiled until the first time the
EDIT command is issued.
    
\section{\xlabel{defining_additional_help_topics}Defining Additional Help Topics}

ICL includes a HELP command which provides on line documentation on ICL
itself. Using the DEFHELP command it is possible to extend the facility
to access information on the commands you have added. In order to do this
you need to create a help library in the normal format used by the VMS
help system. This is described in the VAX/VMS documentation for the librarian
utility. You can then specify topics from this library which will be
available using the ICL HELP command using a command of the form
\begin{verbatim}
    DEFHELP  EDIT  LIBRARY.HLB
\end{verbatim}
This will cause a
\begin{verbatim}
    HELP EDIT
\end{verbatim}
command to return the information on EDIT in help library LIBRARY.HLB rather
than in the standard ICL library.

\section{\xlabel{other_ways_of_defining_commands}Other Ways of Defining Commands}

The DEFUSER command allows an ICL command to be defined to call a FORTRAN
subroutine. In order to make this work you have to link the subroutine into
a shareable image (The VAX/VMS linker manual explains how to do this). The
subroutine should have a single character string parameter in which it will
receive the command line parameters of the original ICL command (after
substitution of any bracketed expressions).

An easier way of achieving a similar result is to write your FORTRAN
subroutine as an ADAM A-task, and use the DEFINE command as described in
the next section.
                    
\cleardoublepage
\part{ICL and ADAM}
   \markboth{\stardocname}{\stardocname}

\chapter{\xlabel{introduction_to_adam}Introduction to ADAM}

\section{\xlabel{what_is_adam}What is ADAM?}

ADAM (Astronomical Data Acquisition Monitor) consists of a number of facilities
which can be combined in a toolkit approach to support a range of software,
from simple applications to sophisticated, multi-tasking, observing and data
analysis systems.
A typical ADAM system consists of a number of tasks which communicate with each
other following well defined protocols.
(On VMS, each task is a separate VMS process.)
Tasks are written using a standard set of subroutine libraries which provide
the ADAM facilities.
ADAM is often termed a {\em software environment} as, in a completely ADAM
system, it is what the user's application code `sees' around itself.

ADAM was originally developed by the Royal Greenwich Observatory to run on the
Perkin-Elmer computers of the INT and JKT on La Palma, for instrument
control on these telescopes. The Royal Observatory
Edinburgh adapted ADAM to run on VAX/VMS systems to provide the
instrument control environment for UKIRT. In doing so they incorporated
most features of the Starlink Software Environment (SSE) so that SSE programs
were equivalent to ADAM A-tasks. VAX ADAM has now been adopted as the
standard instrument control environment for the AAT, the WHT at La Palma
and the JCMT on Mauna Kea as well as UKIRT. ADAM has also been adopted
by Starlink as its standard environment for data reduction. 
Responsibility for support of ADAM now rests with the ADAM Support Group, 
which is part of the Starlink project.

There are several references in the following chapters to Starlink User Notes
(SUNs), Starlink System Notes (SSNs) and Starlink Guides (SGs).
On Starlink systems they will be found in a directory with logical name 
DOCSDIR. Your site manager should be able to provide hardcopies.

\section{\xlabel{the_role_of_the_command_language}The Role of the Command Language}
The command language was originally conceived as playing a key role
in the ADAM system by providing the only user interface to the system.
However, other user interfaces to ADAM have been developed, so users of
ADAM systems will not necessarily find themselves using ICL (or its predecessor
ADAMCL) when working with an ADAM system.

When using ADAM for data reduction the command language will probably be
the standard means of running ADAM. When ADAM is used for on line instrument
control the command language can be used, but this is generally done only
in the testing phase of the instrument. In fully developed systems the 
instrument will probably be controlled through a user interface
which makes more sophisticated use of the terminal.
There are two such systems in current use.

At UKIRT, the instrument control software is using the screen management
system (SMS). With SMS the user is presented with menus from which
selections are made using the cursor keys. The SMS menu selections actually
result in command language code being executed, but the user does not
normally interact with the command language directly.

At the AAT, the user interface for instrument control is currently by
means of ADAM tasks known as U-tasks. The user of a U-task sees a screen
divided into a fixed region in which status information on the instrument
or observing process is displayed, and scrolling regions for message
output and command input. The command language is not normally involved,
though it is always possible to use the command language to control
a U-task, which is equivalent in this respect to any other ADAM task.       

\chapter{\xlabel{using_adam_for_data_reduction}Using ADAM for Data Reduction}
                                                                  
\section{\xlabel{atasks}A-tasks}

When ADAM is used for data reduction, ICL is used to communicate with one
or more A-tasks or application tasks which contain the data reduction programs.
Each of these tasks can be associated with an ICL command by means of the
DEFINE command. When the associated ICL command is issued the A-task will
be loaded and executed. If the same ICL command is issued a second time
the task will not normally need to be reloaded so that execution will be
faster. However, if too many tasks are loaded one of these tasks may have
to be killed before another can be loaded. The maximum number of tasks
allowed to be loaded at one time has a default value of three, but may
be adjusted, on a system basis, by the system logical name ADAM\_CACHLIMIT. 
The least recently used task is always selected
for killing when the limit is exceeded. This process is known as task caching
and such tasks are referred to as cached tasks.
                 
\section{\xlabel{monoliths}Monoliths}

To avoid frequent loading and killing of tasks, large data reduction packages
are normally organized as {\em Monoliths} or M-tasks. A monolith is a single
task which contains a large number of independent programs which would
otherwise be separate A-tasks. Thus a monolith will have many commands
associated with it, whereas an A-task only has a single command. The
monolith will be loaded when the first command is issued and then any
of the other monolith commands may be issued without requiring the task
to be reloaded.

\section{\xlabel{running_kappa}Running KAPPA}

An example of an ADAM monolith is the Starlink Kernel Applications Package
(KAPPA). KAPPA is described in more detail in the Starlink document SUN/95.
To run KAPPA type:

\begin{verbatim}
    $ ADAM KAPPA
\end{verbatim}

This is actually a convenient way of combining the commands\footnote{
In fact, any ICL command may be specified in place of KAPPA in the ADAM
command.}:

\begin{verbatim}
    $ ADAMSTART
    $ ICL
    ICL> KAPPA
\end{verbatim}

To run any ADAM task from ICL it is necessary to have obeyed the command 
ADAMSTART in DCL before running ICL.
If you use ADAM tasks frequently, you may include this command in your 
LOGIN.COM file. The ADAM procedure will not obey ADAMSTART again if it has
already been obeyed, although it wouldn't matter if it did.

When the command KAPPA is obeyed in ICL, you will see         

\begin{verbatim}
     Help key KAPPA redefined
     
     --    Initialized for KAPPA   --
     --   Version 0.8, 1991 August --

        Type HELP KAPPA or KAPHELP for KAPPA help
    ICL>
\end{verbatim}

The KAPPA command causes all the individual commands of the KAPPA monolith
to be defined and then outputs its initialization message. It does not cause
the KAPPA monolith itself to be loaded. This occurs when the first KAPPA
command is issued. 

The simplest way of using KAPPA is to type just the command names. You will
then be prompted for all the required parameters. For example, to use the
ADD command which adds two images just type ADD and the following dialogue
will result

\begin{verbatim}
    ICL> ADD
    Loading KAPPA_DIR:KAPPA into 0591KAPPA
    IN1 - First input NDF> IM1
    IN2 - Second input NDF> IM2
    OUT - Output NDF> SUM
    ICL>
\end{verbatim}

The above command causes the images contained in HDS data files IM1.SDF
and IM2.SDF to be added and the result placed in a new file called SUM.SDF.

Since this is the first KAPPA command the monolith needs to be loaded, and a
loading message\footnote{These loading messages can be turned off by the SET
NOMESSAGES command, if desired} is output which tells us the name of the task
being created, in this case 0591KAPPA\footnote{The name is prefixed with a
number ({\em e.g.} 0591) to make a unique process name, so that there is no conflict
between your version of KAPPA, and another version run at the same time by
another user}. 

Each parameter prompt consists of three parts, the name of the
parameter ({\em e.g.} IN1), a brief description (First input NDF\footnote{ 
`NDF' stands for {\em Extensible N-Dimensional Data Format}. For more
information see Section \ref{accdat}.}), and in some
cases a {\em suggested} value enclosed between / characters.

When responding to a parameter prompt the user has several options.
\begin{itemize}

\item Just hit the return key. The suggested value will be used.

\item Type in a new value.

\item Hit the TAB key. This causes the suggested value to be placed in the
input buffer. It can then be edited by the normal command line editing keys.
                                                                   
\item Enter ! --- the NULL value. The effect of doing this will depend upon
the application but frequently causes termination.

\item Enter !! --- by convention, this causes the program to be aborted.

\item Enter ? or ?? --- this causes the help system to be entered,
giving further information on the parameter and/or the application
(assuming that such text has been provided by the application 
implementor\footnote{For a detailed description of the parameter help system,
see SUN/115.}).
\end{itemize}                                                     

If the parameters of a command are known they can be placed on the command
line itself. Thus the above example would become:

\begin{verbatim}
    ICL> ADD IM1 IM2 SUM
\end{verbatim}

In this example it is important to get the parameters in the right order.
If we are not sure about the order of the parameters, but know their names
we can specify the parameters by name on the command line.

\begin{verbatim}
    ICL> ADD OUT=SUM IN1=IM1 IN2=IM2 TITLE='Sum of 2 Images'
\end{verbatim}

TITLE is an example of a parameter which can only be specified using the 
`{\em name}\/=' syntax as it has not been allocated a position in the order of
parameters.
If TITLE is not specified, a default value will be used.

\section{\xlabel{adam_parameter_format}ADAM Parameter Format}
                 
ADAM task parameters can be of a number of different types as follows:

\begin{itemize}

\item Numbers --- These can be Integer, Real or Double Precision and are
entered in the usual format ({\em e.g.} as in FORTRAN)

\item Strings --- These are represented in the usual ICL formats. Quotation
marks may be omitted where there is no ambiguity. In the above example they
were necessary on `Sum of 2 images' when used on the command line as the spaces
would otherwise make it appear to be four different parameters. However, they
would not be needed for the same string in response to a single parameter 
prompt.

\item Logical values --- These can be represented by the words TRUE, FALSE,
YES, NO, T, F, Y or N, regardless of case.

\item Arrays --- arrays of any of the above items may be represented by
a list of values enclosed in square brackets, {\em e.g.} [1, 2, 3]. 
Two dimensional arrays may be represented as [ [1, 2], [3, 4] ] {\em etc.}
The outer brackets may be omitted when responding to a prompt.

\item Names --- These are the names of HDS (hierarchical data system) objects
or files, or the names of devices (graphics devices, tape drives {\em etc.}).
Normally these names can be typed directly as MTA0, ARGS etc, but there
are a few cases of ambiguity.
These cases can be resolved by prefixing the name with an @ character.
For example, if it is required to specify an HDS container file with a name
different from that of the object it contains, or with a specific generation
number, the file specification can be enclosed in quotes
({\em e.g.} \verb+"data.sdf;3"+). 
However, this would be interpreted as a character string unless prefixed with
@.
Suggested values in prompts are always displayed prefixed with @.

\end{itemize}

\section{\xlabel{kappa_from_procedures}KAPPA from Procedures}

ICL becomes particularly useful when a series of KAPPA operations are to
be performed on a sequence of files. We can then use an ICL procedure for
this purpose. In such cases we will probably want to use an ICL variable
or expression to specify at least one of the parameter names. This is
perfectly acceptable provided the expression is placed in parentheses so
that it will be evaluated and not treated as a string. A frequent occurrence
is that we want to process a sequence of files which have names such as
RUN1, RUN2, RUN3 {\em etc.} ICL therefore provides a function SNAME to generate
such sequential names
\footnote{Note that for reasons explained in Section \ref{retval}, strings
defining names to be used as ADAM task parameters are generally prefixed with
@.}.
It has the form:
\begin{verbatim}
    SNAME(string,n,m)
\end{verbatim}

and produces a name which is the concatenation of the string with the integer
n. An optional third parameter m specifies a minimum number of digits for
the numeric part of the name, leading zeros are inserted if necessary to
produce at least m digits.
\begin{verbatim}
    SNAME('@RUN',3)      has the value   '@RUN3'                        
    SNAME('@IPCS',17,3)  has the value   '@IPCS017'
\end{verbatim}
The latter example being the format of name produced directly by the IPCS
observing software at the AAT. Using this way of specifying the name it
is easy to write an ICL procedure to add a whole series of images together
using the KAPPA ADD command.

\begin{verbatim}
    PROC KADD
       
    {   Add images RUN1 to RUN20 to form SUM  }

       ADD RUN1 RUN2 SUM TITLE='Sum of 2 images'
       LOOP FOR I=3 TO 20
         TITLE = '''Sum of ' & I:2 & ' images'''
         ADD  SUM  (SNAME('RUN',I))  SUM  TITLE=(TITLE)
       END LOOP

    END PROC
\end{verbatim}

\section{\xlabel{error_reports}Error Reports}
In the event of an error occurring in the task, error reports will be
displayed. For example:

\begin{verbatim}
    ICL> ADD
    Loading KAPPA_DIR:KAPPA into 0591KAPPA
    IN1 - First input NDF > !
    !! Null NDF structure specified for the IN1 parameter
    !  ADD: Error adding two NDF data structures
    ADAMERR   %PAR, Null parameter value
\end{verbatim}

The first two messages are issued by the task. Such messages will usually
indicate which ADAM subsystem or routine generated them by a prefix (ADD:
in the example).

The message starting `ADAMERR' is issued by ICL on receiving a termination
message containing a `bad' status.
ADAMERR is the name of an ICL exception -- the \%PAR following ADAMERR tells 
us that the PAR subsystem within ADAM generated the bad status value.


\chapter{\xlabel{writing_adam_tasks}Writing ADAM tasks}

\section{\xlabel{introduction}Introduction}

It is easy to write your own ADAM A-tasks which can be run from ICL in the
same way as the KAPPA programs. This is the easiest way of allowing ICL
to run your own FORTRAN programs\footnote{The alternative way is by using
the DEFUSER command}.
A detailed discussion of writing A-tasks can be found in SG/4 and SUN/101.

Here is a simple example of an ADAM A-task. Its source code consists of
the FORTRAN source file MYTASK.FOR containing the following:

\begin{verbatim}                   
    SUBROUTINE MYTASK(STATUS)
    IMPLICIT NONE
    INTEGER STATUS

    CALL MSG_OUT(' ','Hello',STATUS)

    END
\end{verbatim}
The example task consisted essentially of just one statement --- the call
of MSG\_OUT. This routine is part of a subroutine library which is provided
for outputting information to the terminal. The MSG\_ routines should always
be used for terminal output from ADAM tasks. FORTRAN PRINT and WRITE statements
must not be used for this purpose.

Every ADAM task also has an interface file. The interface file contains
information on the parameters of the task. Our task doesn't have any parameters
so its interface file is fairly simple. It consists of the following in
the file MYTASK.IFL.

\begin{verbatim}
    INTERFACE MYTASK
    ENDINTERFACE
\end{verbatim}      

\section{\xlabel{compiling_and_linking}Compiling and Linking}

Before compiling and linking ADAM tasks we need to run the command ADAM\_DEV
which sets up appropriate definitions. This in turn requires
that the command ADAMSTART has been run.

\begin{verbatim}
    $ ADAMSTART
    $ ADAM_DEV
\end{verbatim}

The task can then be compiled and linked using the following commands:

\begin{verbatim}
    $ FOR MYTASK
    $ ALINK MYTASK
\end{verbatim}

To run the task start up ICL and define a command to run the task using
the ICL DEFINE command.

\begin{verbatim}
    ICL> DEFINE  TEST  MYTASK
    ICL> TEST
    Loading MYTASK into 03BCMYTASK
    Hello
    ICL> TEST
    Hello
    ICL>
\end{verbatim}
The DEFINE command defines the command TEST to run the A-task MYTASK. For
this to work, MYTASK has to be in the default directory. If it were somewhere
else a directory specification could be included on MYTASK in the DEFINE
command.

TEST then causes the task to be loaded and executed. Typing TEST again causes
it to be executed a second time, but this time it doesn't have to be loaded.
                                                                        
\section{\xlabel{tasks_with_parameters}Tasks with Parameters}

The following is an example of a task with a parameter. This task calculates
the square of a number and outputs its value on the terminal.

\begin{verbatim}
      SUBROUTINE SQUARE(STATUS)
      INTEGER STATUS
      REAL R,RR
      CALL PAR_GET0R('VALUE',R,STATUS)
      RR = R*R           
      CALL MSG_SETR('RVAL',R)
      CALL MSG_SETR('RSQUARED',RR)
      CALL MSG_OUT(' ','The Square of ^RVAL is ^RSQUARED',STATUS)
      END
\end{verbatim}                                               

This program uses the subroutine PAR\_GET0R to get the value of the parameter
VALUE (there are similar routines for other types). Output of the numbers
is done using the routine MSG\_SETR to give values to the tokens RVAL and
RSQUARED which are then inserted into the MSG\_OUT output string using
the \verb$^RVAL$ notation. The interface file for this example is given
below.        

\begin{verbatim}
  INTERFACE SQUARE
 
    PARAMETER VALUE
      TYPE _REAL
      POSITION 1
      VPATH PROMPT
      PPATH CURRENT
      PROMPT 'Number to be squared'
    ENDPARAMETER

  ENDINTERFACE
\end{verbatim}               

The interface file has an entry for the parameter VALUE. The TYPE field
specifies the type of the parameter. The underscore prefix on '\_REAL'
identifies it as a {\em primitive} type ({\em i.e.} a simple number or string,
rather than an HDS structure or Device name). The position field specifies
the position that the parameter is expected in if it appears on the command
line.

The VPATH entry specifies how the parameter value is to be obtained if it
not found on the command line. In this case it is to be prompted for. The
PPATH entry specifies how the default value that appears in the parameter
prompt is to be obtained. In this case the CURRENT value ({\em i.e.} the value
the parameter had at the end of the last execution of the command) is used.
The PROMPT field gives the prompt string to be used.

Interface files are described in more detail in SUN/115.
                                                      
To run this example we would compile and link it as described above and
then use the following ICL commands:

\begin{verbatim}
    ICL> DEFINE SQUARE SQUARE
    ICL> SQUARE 12
    Loading SQUARE into 03BCSQUARE
    The Square of 12 is 144
    ICL> SQUARE (SQRT(3))
    The Square of 1.73205 is 3
    ICL> SQUARE
    VALUE - Number to be squared /0.173205E+01/ > 7
    The Square of 7 is 49
    ICL>         
\end{verbatim}

\section{\xlabel{status_and_error_handling}STATUS and error handling}

Most ADAM subroutines have an integer parameter called STATUS. STATUS has a
success value (SAI\_\_OK) which each routine will return if it completes
successfully. If the routine fails for some reason it will return an error code
indicating the nature of the error. An ADAM A-task routine (such as SQUARE)
will be called with a STATUS of SAI\_\_OK. If it returns a bad status value to
its caller this will result in an appropriate message being output.

There is a further important feature of the status convention. If an
ADAM routine is called with its STATUS argument having an error value
on input, then the routine will do nothing and will return immediately.
This feature means that it is usually not necessary to check STATUS
after each routine is called. A series of ADAM routines can be called
with the STATUS being passed from one to the next. If an error occurs in
one of them, the subsequent routines will do nothing and the final status
will indicate the error code from the routine that failed. If this STATUS
value is then returned by the A-task main routine to its caller an error
message will result. Thus the error will be correctly processed with no
special code being added to check for errors.

It is important, however, to take care that code that does not consist of
calls to ADAM routines does not get executed after an error has occurred.
For this reason our SQUARE example would be better written as:

\begin{verbatim}
      SUBROUTINE SQUARE(STATUS)
      INTEGER STATUS
      INCLUDE 'SAE_PAR'  ! This provides the ADAM status codes
      REAL R,RR
      CALL PAR_GET0R('VALUE',R,STATUS)                        
      IF (STATUS .EQ. SAI__OK) THEN
         RR = R*R           
         CALL MSG_SETR('RVAL',R)
         CALL MSG_SETR('RSQUARED',RR)
         CALL MSG_OUT(' ','The Square of ^RVAL is ^RSQUARED',STATUS)
      ENDIF
      END
\end{verbatim}                                               
         
This ensures that if the STATUS from PAR\_GET0R is bad the rest of the routine
is not executed with an undefined value of R. It is actually not necessary
to include MSG\_OUT in the IF block as this would not execute if STATUS was
bad.                  

More sophisticated error handling can be provided by using routines in the
ERR\_ package. These facilities are fully described in SUN/104.

\section{\xlabel{returning_values_to_icl}Returning values to ICL}
\label{retval}
We have already seen how ICL can be used to supply values for ADAM task
parameters. It is also possible for ADAM tasks to return values to ICL.
The following modified version of SQUARE does not output its result on the
terminal, but returns it to the parameter VALUE using a call to the routine
PAR\_PUT0R which is analogous to PAR\_GET0R.

\begin{verbatim}
      SUBROUTINE SQUARE(STATUS)
      INTEGER STATUS
      REAL R,RR
      CALL PAR_GET0R('VALUE',R,STATUS)
      IF (STATUS .EQ. SAI__OK) THEN      
         RR = R*R           
         CALL PAR_PUT0R('VALUE',RR,STATUS)
      ENDIF
      END
\end{verbatim}                            

We could run this from ICL as follows (having done a DEFINE SQUARE SQUARE)
to define the command:

\begin{verbatim}
    ICL> X=5
    ICL> SQUARE (X)
    ICL> =X
    25
    ICL>
\end{verbatim}

In order for the ADAM task to return a value to ICL we must use a variable
for the parameter and place it on the command line. The variable name must
be placed in parentheses, then the name of a temporary HDS object is 
substituted by ICL.

A modification of this scheme is needed with character variables to allow the
case where the contents of the character variable is itself a device, file or 
object name. 
In such cases, the supplied name cannot be replaced by some other name so, 
to indicate that they may not be replaced, name values in variables must be 
preceded by @.

\section{\xlabel{graphics_with_adam_}Graphics with ADAM}                                                

All graphics in ADAM is based on the use of the GKS graphics system. Most
users, however, will not use the GKS routines directly but will use a
higher level package such as SGS, NCAR or PGPLOT. The following example
uses SGS to draw a circle on a selected graphics device.
(SGS is described in SUN/85, and its use within ADAM in SUN/113.) 
\begin{verbatim}
      SUBROUTINE CIRCLE(STATUS)
      IMPLICIT NONE
      INTEGER STATUS
      REAL RADIUS          ! Radius of circle               
      INTEGER ZONE         ! SGS Zone
      INCLUDE 'SAE_PAR'    ! Adam Constants
                         
*   Get the radius of the circle
      CALL PAR_GET0R('RADIUS',RADIUS,STATUS)

*   Get the graphics device, and open SGS
      CALL SGS_ASSOC('DEVICE','WRITE',ZONE,STATUS)

*   Draw the circle
      IF (STATUS .EQ. SAI__OK) THEN
          CALL SGS_CIRCL(0.5,0.5,RADIUS)
      ENDIF

*   Close the graphics workstation 
*   and cancel the parameter
      CALL SGS_CANCL(ZONE,STATUS)

*   Close down SGS
      CALL SGS_DEACT(STATUS)

      END                  
\end{verbatim}
           
The interface file for this example is as follows:

\begin{verbatim}
INTERFACE CIRCLE

   PARAMETER RADIUS
      TYPE _REAL
      POSITION 1
      VPATH PROMPT
      PPATH CURRENT
      PROMPT 'Radius of Circle'
   ENDPARAMETER

   PARAMETER DEVICE
      PTYPE DEVICE
      POSITION 2
      VPATH PROMPT
      PPATH CURRENT
      PROMPT 'Graphics Device'
   ENDPARAMETER

ENDINTERFACE
\end{verbatim}                

When SGS is used from ADAM no calls to SGS\_OPEN and SGS\_CLOSE are made.
Instead the routines SGS\_ASSOC and SGS\_ANNUL are used. SGS\_ASSOC makes
the association between an ADAM parameter (DEVICE) and an SGS zone whose
zone identifier is returned to the program in the ZONE parameter. When
SGS plotting is finished SGS\_ANNUL is called and given the same ZONE
value. SGS\_ASSOC has an additional parameter, the access mode, which has
possible values 'READ', 'WRITE' and 'UPDATE'.

There are a number of similar pairs of \_ASSOC and \_CANCL routines in ADAM
which work in similar ways. MAG\_ASSOC and MAG\_CANCL are used to handle
magnetic tape devices, FIO\_ASSOC and FIO\_CANCL to handle file I/O {\em etc.}
Many of them also have an \_ANNUL subroutine which frees the associated
resource but does not cancel the associated ADAM parameter.

To run CIRCLE from ICL we could either use:

\begin{verbatim}
    ICL> CIRCLE  0.3  ARGS1
\end{verbatim}

or let it prompt for the parameters:

\begin{verbatim}
    ICL> CIRCLE
    RADIUS - Radius of Circle /0.300000E+00/ > 0.2
    DEVICE - Graphics Device /@ARGS1/ > PRINTRONIX
    ICL> $ PRINT/NOFEED PRINTRONIX.BIT
    ICL>                               
\end{verbatim}

In the latter case the default values of the parameters are the values from the
previous time (a result of using PPATH CURRENT). With a hard copy graphics
device such as PRINTRONIX, a file is created which must then be sent to
the device with a DCL PRINT command.
                                                                  
\section{\xlabel{accessing_data}Accessing Data}
\label{accdat}
The basic means of storing and accessing data for ADAM is the Hierarchical
Data System (HDS). SUN/92 describes this system
and the DAT\_ package of routines that are used to access data in this form.
In HDS a data file contains a number of named components which can
either be primitive items (numbers, character strings or arrays) or can
themselves be structures containing further components.

To simplify the exchange of data between different applications packages,
Starlink have released a set of standards for representing data within HDS.
This is the {\em Extensible N-Dimensional Data Format} (NDF) -- it is
described in SGP/38. A library of subroutines (the NDF library, described in 
SUN/33) is provided for accessing these standard structures and will generally
be used when writing applications.

Below is a simple example that calculates the mean value of the data in
an NDF. For such data files the data will be found in a component
of the file, called .DATA.

\begin{verbatim}
      SUBROUTINE MEAN(STATUS)
      IMPLICIT NONE
      INTEGER STATUS
      INCLUDE 'SAE_PAR'
      INTEGER NELM                       ! Number of Data elements
      CHARACTER*(DAT__SZLOC) LOC         ! HDS locator
      INTEGER PNTR                       ! Pointer to Data
      REAL MN                            ! Mean value of data
               
* Start an NDF context
      CALL NDF_BEGIN

* Get locator to parameter
      CALL NDF_ASSOC('INPUT','READ',LOC,STATUS)

* Map the data array
      CALL NDF_MAP(LOC,'DATA','_REAL','READ',PNTR,NELM,STATUS)  

* If everything OK calculate mean value of data array and output it  
      IF (STATUS .EQ. SAI__OK) THEN
          CALL MEAN_SUB(NELM,%VAL(PNTR),MN)
          CALL MSG_SETR('MEAN',MN)
          CALL MSG_OUT(' ','Mean Value of Array is ^MEAN',STATUS)
      ENDIF                                                        

* End the NDF context
      CALL NDF_END(STATUS)

      END

                                         
      SUBROUTINE MEAN_SUB(NELM,ARRAY,MEAN)

*  Subroutine to calculate the mean value of the array

      IMPLICIT NONE
      INTEGER NELM
      REAL ARRAY(NELM)
      REAL MEAN
      INTEGER I
                
      MEAN = 0.0
      DO I=1,NELM
         MEAN = MEAN + ARRAY(I)
      ENDDO
      MEAN = MEAN / NELM
      END

\end{verbatim}                

There are a number of points to note about this example:
\begin{itemize}

\item The routine NDF\_ASSOC is analogous to SGS\_ASSOC as used in the previous
example. It returns an NDF identifier associated with an ADAM parameter
(INPUT).

\item The routine NDF\_MAP returns a pointer (PNTR) to the data array.
PNTR is an integer variable which contains the memory address of the data.
Mapping is used in applications of this type as an alternative to reading
the data into an array. The advantage of mapping is that we don't have to
make a guess at the maximum size of array we expect, in order to know how
big an array to declare. NDF\_MAP will return a pointer to a real array.
If the original data is of a different type it will be converted to real.

\item The pointer concept is not supported in standard FORTRAN, but can
be made use of in VAX FORTRAN by passing the pointer to a subroutine using
the \%VAL(PNTR) construct. In the subroutine (MEAN\_SUB in this case) the
corresponding parameter can then be treated as if we had passed the actual
array.                           

\item The routine NDF\_END is called to end the current NDF context and
unmap the mapped data array. For more information on NDF contexts see SUN/33.

\end{itemize} 
The interface file for this example could be as follows:             

\begin{verbatim}
INTERFACE MEAN

   PARAMETER INPUT
      TYPE NDF
      POSITION 1
      VPATH PROMPT
      PPATH CURRENT
      PROMPT 'NDF to calculate Mean Value from'
   ENDPARAMETER

ENDINTERFACE
\end{verbatim}

The above example shows how to handle the case where an NDF file is used
for input. Where output to an NDF file is involved it is usually necessary
to create a new HDS file when the application runs. This can be achieved
using routine NDF\_CREAT or NDF\_CREP. For details of their usage, see SUN/33.

\chapter{\xlabel{adam_as_a_data_aquisition_environment}ADAM as a Data Acquisition Environment}

\section{\xlabel{instrumentation_tasks}Instrumentation Tasks}
When used for data acquisition, ICL is used to control one or more 
`instrumentation tasks' (I-tasks).
A I-task is used to control an individual instrument or other hardware
component of the system.
An I-task can respond to a number of different commands
(referred to as {\em Actions}) rather than the single command of an A-task;
they also have the ability to perform two or more actions concurrently.

Multiple I-tasks may be involved when several instruments
are used in combination and sometimes it is convenient to have another I-task
controlling the individual instrument I-tasks.
In this case, ICL would be used to control the controlling task which would
relay to ICL any messages or prompts from the subsidiary I-tasks intended for 
the operator.

Instrumentation tasks are fully described in SUN/134.

\section{\xlabel{dtasks_and_ctasks}D-tasks and C-tasks}
Instrumentation tasks combine the functions of Device (D) task and Control
(C or CD) tasks which were used with ADAM V1.
These old-style tasks are now considered `unfashionable' but they continue to
work with ICL and it will be some time before they are replaced in all systems.

\section{\xlabel{task_loading}Task Loading}

In a data acquisition situation the task caching scheme described in the
previous section is usually inappropriate. We don't want tasks to be killed
when the caching limit is exceeded, and we may not want a task to be killed
when ICL exits (as happens with cached tasks). Also the unique process names
created for cached tasks are undesirable as many different tasks may want
to communicate with a given task, and therefore need to know its name.

Therefore I-tasks are normally loaded as uncached tasks, and this is achieved
by explicitly loading them using one of the three ICL load commands.
\begin{description}

\item[ALOAD] --- Loads a task into a subprocess without waiting for completion.
                                             
\item[LOADW] --- Loads a task into a subprocess and waits for completion.     

\item[LOADD] --- Loads a task into a detached process and waits for
completion.      

\end{description}

\section{\xlabel{killing_tasks}Killing Tasks}

Uncached tasks remain loaded until explicitly killed, or until the creating
process is logged out. They remain loaded when ICL exits. Thus it is possible
to use ICL to load a task, then exit from ICL, and subsequently communicate
with the task from a second invocation of ICL (which might be started on
a different terminal).

Tasks are killed using the KILL or KILLW commands.

\begin{description}

\item[KILL] --- Kills a task without waiting for completion

\item[KILLW] --- Kills a task and waits for completion                    
                                                     
\end{description}

\section{\xlabel{the_adam_message_system}The ADAM message system}

Communication between ADAM tasks, and between ICL and tasks, makes use of
the ADAM message system. The message system is involved in the communication
between ICL and A-tasks described in the last chapter, but in this case
its details are largely hidden from the user. In the data acquisition case
the use of the message system by the command language is usually more explicit.

ICL can send four types of messages to tasks, which are distinguished by
a {\em context} which is one of GET, SET, OBEY or CANCEL. 

The ICL command to send a message to a task is SEND.

\section{\xlabel{an_example}An Example}

As an example we will consider a I-task called PHOTOM which controls a simple
optical photometer. The photometer includes a filter wheel with five positions.

\subsection{\xlabel{the_itask_interface_file}The I-task Interface File}

The I-task has a parameter FILTER\_DEMAND which specifies the required filter
and an action FILTER which moves the filter wheel to the required position.
These are specified in the interface file for the task as follows:

\begin{verbatim}
    PARAMETER FILTER_DEMAND
      TYPE '_CHAR'
      VPATH 'INTERNAL'
      IN 'U','B','V','R','I'
    ENDPARAMETER

    ACTION FILTER
      OBEY
        NEEDS FILTER_DEMAND
      ENDOBEY
      CANCEL
      ENDCANCEL
    ENDACTION
\end{verbatim}

Many I-task parameters have VPATH specified as 'INTERNAL' --
this means that the parameter value is stored in memory for speed of access.
The IN field
specifies the valid values for the parameter (it is also possible to use
a RANGE specification to restrict the values of a parameter).

The action entry specifies that the action FILTER may be obeyed and
cancelled, and that they OBEY action needs the parameter FILTER\_DEMAND which
may be specified as the first (only) parameter on the command line.

                                      
\subsection{\xlabel{running_photom}Running PHOTOM}

The I-task could be used as follows:

\begin{verbatim}
    ICL> LOADW PHOTOM
    Loading PHOTOM into PHOTOM
    ICL> SEND PHOTOM SET FILTER_DEMAND V
    ICL> SEND PHOTOM GET FILTER_DEMAND
    V
    ICL> SEND PHOTOM OBEY FILTER
    ICL>
\end{verbatim}

The last command starts the FILTER action but does not wait for it to complete.
ICL can accept and execute other commands while the filter wheel is moving
to its position. When the action finally completes the I-task returns a
completion message to ICL which will be output on the terminal.

\begin{verbatim}
    ICL>
    "obey" FILTER - action complete
\end{verbatim}
                               
\subsection{\xlabel{supplying_parameters_in_the_obey_message}Supplying Parameters in the Obey Message}

A parameter in the NEEDS list for an action may be supplied along with the
OBEY message rather than set independently with a SET message. Thus we
can use:

\begin{verbatim}
    ICL> SEND PHOTOM OBEY FILTER B
\end{verbatim}
                           
\subsection{\xlabel{cancelling_actions}Cancelling Actions}

The CANCEL context enables us to cancel the FILTER action before it has
completed by issuing the following command:

\begin{verbatim}
    ICL> SEND PHOTOM CANCEL FILTER
\end{verbatim}
                               
\subsection{\xlabel{missing_parameters}Missing Parameters}

If we try to start the FILTER action without specifying a value for the 
FILTER\_DEMAND parameter (either by SEND SET or by specifying it on the 
command line), the parameter system will attempt to get a value in the
same way that it does for A-tasks ({\em but note that `INTERNAL' parameters
will not be prompted for -- a pseudo VPATH of `DYNAMIC,DEFAULT' is used and
there is no implied PROMPT at the end.})
Therefore, as there is no DEFAULT value specified for FILTER\_DEMAND, and
assuming that no dynamic default value is specified by the program, the
null response is assumed.

\begin{verbatim}
    ICL> SEND PHOTOM OBEY FILTER
    Bad status returned from task, meaning is :-
    ICL>
    %PAR-I-NULL, Null parameter value
    ICL>
\end{verbatim}

Note that old-style D-tasks always had parameters defined as 'INTERNAL'
and could not prompt for missing parameters anyway.

\subsection{\xlabel{the_get_command}The GET command}

A SEND PHOTOM GET command causes the parameter value to be output on the
terminal. It is often more useful to put the value into an ICL variable.
This can be accomplished using the GET command.

\begin{verbatim}
    ICL> GET PHOTOM FILTER_DEMAND (F)
    ICL> =F
    B
    ICL>
\end{verbatim}

\subsection{\xlabel{the_obeyw_command}The OBEYW command}

It is often more convenient to let ICL wait for an action to complete rather than
have it proceed concurrently with other ICL commands. The OBEYW command
is used for this purpose.

\begin{verbatim}
    ICL> OBEYW PHOTOM FILTER I
    ICL>
\end{verbatim}

The OBEYW command waits until the action has completed before the ICL$>$
prompt reappears and more commands may be entered.

\section{\xlabel{multiple_concurrent_actions}Multiple Concurrent Actions}
                    
I-tasks have the ability to perform more than one action concurrently.
ICL includes two commands which enable actions to be performed concurrently
while waiting for all of the actions to complete. Suppose our PHOTOM
I-task has another action APERTURE which moves the aperture wheel to
a required position. The following is a procedure which would allow the
filter and aperture wheels to be moved simultaneously.

\begin{verbatim}
    PROC SETUP  F  A

{  move filter to position F and aperture to position A  }
      
{  Start both actions  }

      STARTOBEY (P1) (M1) PHOTOM FILTER (F)
      STARTOBEY (P2) (M2) PHOTOM APERTURE (A)
      
{  Wait for the actions to complete  }
                                        
      ENDOBEY (P1) (M1)
      ENDOBEY (P2) (M2)

    END PROC
\end{verbatim}          

STARTOBEY starts an action and returns the {\em path} and {\em message
identifier}, which together uniquely identify a given invocation of an action,
into two ICL variables. ENDOBEY can then be used to wait for the completion
of such an action. The above procedure will therefore exit only when both
actions are complete.

\appendix
\chapter{\xlabel{icl_functions}ICL Functions}  

Where the type is given as `Real or Integer' the type of the result is the
same as the type of the argument.
\begin{center}
\begin{tabular}{llll}

Name & Type & Definition \\ 
\\
ABS(X) & Real or Integer & $\mid x \mid$\\
\\
ASIN(X) & Real & $\sin^{-1} x \; where -1 \leq x \leq 1 $\\
   &  &  $ -\pi/2 \leq result \leq \pi/2$\\
\\
ASIND(X) & Real & $\sin^{-1} x \; where -1 \leq x \leq 1 $\\
   &  &  $ -90 \leq result \leq 90$\\
\\
ACOS(X) & Real & $\cos^{-1} x \; where -1 \leq x \leq 1 $\\
   &  &  $ 0 \leq result \leq \pi$\\
\\
ACOSD(X) & Real & $\cos^{-1} x \; where -1 \leq x \leq 1 $\\
   &  &  $ 0 \leq result \leq 180$\\
\\
ATAN(X) & Real & $\tan^{-1} x $ \\
   &  &  $ -\pi/2 \leq result \leq \pi/2$\\
\\
ATAND(X) & Real & $\tan^{-1} x $ \\
   &  &  $ -90 \leq result \leq 90$\\
\\
ATAN2(X1,X2) & Real & $\tan^{-1} (x_{1}/x_{2}) $\\
   & & $ -\pi < result \leq \pi $ according to quadrant \\
   & & in which point ($x_{2},x_{1}$) lies\\
\\
\end{tabular}
\end{center}
\newpage
\begin{center}
\begin{tabular}{llll}
Name & Type & Definition \\
\\
ATAN2D(X1,X2) & Real & $\tan^{-1} (x_{1}/x_{2}) $\\
   & & $ -180 < result \leq 180 $ according to quadrant \\
   & & in which point ($x_{2},x_{1}$) lies\\
\\
BIN(I,n,m) & String & integer I formatted in binary \\
 & & into an n char string with m significant digits \\
\\
CHAR(I) & String & Character whose ASCII value is I \\
\\
COS(X) & Real & $\cos x$ (x in radians) \\
\\
COSD(X) & Real & $\cos x$ (x in degrees) \\
\\
COSH(X) & Real & $\cosh x$ \\
\\
DATE() & String & The Current Date \\
\\
DEC(I,n,m) & String & integer I formatted in decimal \\
 & & into an n char string with m significant digits \\
\\
DECL(S) & Real & A Declination in Radians \\
  & & from a string in degrees, minutes, seconds \\
\\
DEC2S(R,NDP,SEP) & String & String in DMS from \\
  & & Dec in radians. NDP decimal places on \\
  & & seconds. Character SEP separates fields \\
\\
DIM(X1,X2) & Real or Integer & Positive difference \\
   & & $x_{1} - min(x_{1},x_{2})$ \\
\\ 
ELEMENT(I,DELIM,S) & String & Ith element of delimited string, I $\geq 0$\\
\\
EXP(X) & Real & $e^{x}$ \\
\\
FILE\_EXISTS(S) & Logical & TRUE if file S exists \\
\\
FLOAT(I) & Real & Integer I converted to real \\
\\
GETNBS(S) & Any & Value of noticeboard item \\
\\
GET\_SYMBOL(S) & String & Get Value of DCL Symbol \\
\\
\end{tabular}
\end{center}
\newpage
\begin{center}
\begin{tabular}{llll}
Name & Type & Definition \\
\\
HEX(I,n,m) & String & integer I formatted in hexadecimal \\
 & & into an n char string with m significant digits \\
\\
IAND(I1,I2) & Integer & Bitwise AND of two integers \\
\\                                                 
ICHAR(S) & Integer & ASCII value of first character of String \\
\\
IEOR(I1,I2) & Integer & Bitwise exclusive OR of two integers \\
\\
IFIX(X) & Integer & Real to Integer by Truncation \\
 & & Equivalent to INT \\
\\
INDEX(S1,S2) & Integer & Position of first occurrence \\
  &  & of Pattern S2 in string S1 (Zero if not found) \\
\\       
INKEY() & Integer & Key value of last trapped key pressed \\
  & & Only works in screen mode \\
\\
INOT(I) & Integer & Bitwise Complement of Integer \\
\\
INT(X) & Integer & Real to Integer by Truncation \\
\\
INTEGER(X) & Integer & The value of X converted to an integer \\
    & & Equivalent to NINT \\
\\
IOR(I1,I2) & Integer & Bitwise OR of two integers \\
\\                                
KEYVAL(S) & Integer & Value of key with name S \\
\\
LEN(S) & Integer & Length of String S \\
\\
LGE(S1,S2) & Logical & True if S1 $\geq$ S2 (ASCII collating sequence) \\
\\
LGT(S1,S2) & Logical & True if S1 $>$ S2 (ASCII collating sequence) \\
\\
LLE(S1,S2) & Logical & True if S1 $\leq$ S2 (ASCII collating sequence) \\
\\
LLT(S1,S2) & Logical & True if S1 $<$ S2 (ASCII collating sequence) \\
\\
LOG(X) & Real & $\ln x $\\
\\
\end{tabular}
\end{center}
\newpage
\begin{center}
\begin{tabular}{llll}
Name & Type & Definition \\
\\
LOG10(X) & Real & $\log_{10} x $\\
\\
LOGICAL(X) & Logical & The value of X converted to logical \\
\\
MAX(X1,X2, ...) & Real or Integer & Maximum of two or more arguments \\
\\
MIN(X1,X2, ...) & Real or Integer & Minimum of two or more arguments \\
\\
MOD(X1,X2) & Real or Integer & $x_{1} - int(x_{1}/x_{2}) \times x_{2}$ \\
\\
NINT(X) & Integer & Nearest integer to X \\
\\
OCT(I,n,m) & String & integer I formatted in octal \\
 & & into an n char string with m significant digits \\
\\
OK(stat) & Logical & True if VMS Status OK \\
\\
RA(S) & Real & Right Ascension in Radians \\
  & & from a string in hours, minutes, seconds \\
\\                                                       
RA2S(R,NDP,SEP) & String & String in HMS from \\
  & & RA in radians. NDP decimal places on \\
  & & seconds. Character SEP separates fields \\
\\
RANDOM(I) & Real & Random number between 0 and 1 from seed I \\
     & &  I must be a variable \\
\\
REAL(X) & Real & The value of X converted to real\\
\\
SIGN(X1,X2) & Real or Integer & $x_{1}$ with sign of $x_{2}$ \\
\\
SIN(X) & Real & $\sin x$ (x in radians) \\
\\
SIND(X) & Real & $\sin x$ (x in degrees) \\
\\
SINH(X) & Real & $\sinh x$ \\
\\
SNAME(S,n[,m]) & String & String formed by concatenating string S with integer\\
 & &          n formatted into m characters with leading zeros.\\
 & &          If m is omitted, SNAME uses the required number\\
 & &          of characters\\
\\
\end{tabular}
\end{center}
\newpage
\begin{center}
\begin{tabular}{llll}
Name & Type & Definition \\
\\
STRING(X) & Logical & The value of X converted to a string \\
\\
SQRT(X) & Real & $\sqrt{x}$\\
\\
SUBSTR(S,n,m) & String & Substring of S beginning at n ($\geq$ 1) of length m \\
\\
TAN(X) & Real & $\tan x$ (x in radians) \\
\\
TAND(X) & Real & $\tan x$ (x in degrees) \\
\\
TANH(X) & Real & $\tanh x$  \\
\\                                   
TIME() & String & The current time \\
\\
TYPE(X) & String & The Type of Variable X \\
 & & `REAL', `INTEGER', `LOGICAL', \\
 & & `STRING' or `UNDEFINED' \\        
\\
UNDEFINED(X) & Logical & TRUE if X is undefined \\
\\
UPCASE(S) & String & String S converted to Upper case \\
\\
VARIABLE(proc,X) & Any & Returns value of variable X of procedure proc\\
\\
\end{tabular}
\end{center}


\chapter{\xlabel{icl_commands}ICL Commands}

\section{\xlabel{alloc}ALLOC}



    ALLOC \hspace{.5cm} dev \hspace{.5cm} [(actdev)] \hspace{.5cm} [(status)]

    Allocate a device:
\begin{description}

\item[dev] - The name of the device to be allocated
            This may be a generic name in which case the first
            available device will be allocated.

\item[actdev] - (optional) A variable in which the name of the
            device actually allocated will be returned.

\item[status] - (optional) A variable in which the status of
            the allocate request will be returned. The success of
            the request may be tested using the function OK(status).

\end{description}

  ALLOC should be used in preference to the \$ ALLOCATE command when
  the device is to be used by an ICL procedure. For example to allocate
  a tape drive used by a FIGARO input command.

  ALLOC may be abbreviated to ALL

{\em e.g. }
\begin{verbatim}
    ALLOC MT (DEVICE) (STATUS) 
\end{verbatim}   

\section{\xlabel{ALOAD}ALOAD\label{ALOAD}}

 
    ALOAD \hspace{.5cm} exename \hspace{.5cm} [taskname]

    Load an ADAM task into a subprocess:
\begin{description}

\item[exename] - The name of the executable image file to
            be used for the task.

\item[taskname] - (optional) The name of the task to be created.
            If omitted it defaults to the file name part of
            exename.

\end{description}

\section{\xlabel{APPEND}APPEND\label{APPEND}}
                                       
   APPEND \hspace{.5cm} intname \hspace{.5cm} [filename]

 Open an existing text file for output. Output will be appended to the
 existing text.

\begin{description}

\item[intname]  -  The internal name by which the file will be known
                  within ICL.

\item[filename]  -  The VMS file name of the file. If omitted the
                  file name will be intname.DAT.

\end{description}



\section{\xlabel{CHECKTASK}CHECKTASK\label{CHECKTASK}}
                                       
   CHECKTASK \hspace{.5cm} taskname \hspace{.5cm} (loaded)

 Check whether an ADAM task is currently loaded.

\begin{description}

\item[taskname]  -  The name of the task.

\item[loaded]  -  An ICL variable which will be set to the logical value
                  TRUE if the task is loaded, FALSE otherwise.

\end{description}

\section{\xlabel{CLEAR}CLEAR\label{CLEAR}}

   CLEAR \hspace{.5cm} first \hspace{.5cm} last

In screen mode the range of lines on the screen between {\it first} and {\it
last} 
are cleared.

\section{\xlabel{CLOSE}CLOSE\label{CLOSE}}
                                       
   CLOSE \hspace{.5cm} intname

 Close a text file previously opened with CREATE, OPEN or APPEND.

\begin{description}

\item[intname]  -  The internal name of the file.

\end{description}


\section{\xlabel{CREATE}CREATE\label{CREATE}}
                                       
   CREATE \hspace{.5cm} intname \hspace{.5cm} [filename]

 Create a text file and open it for output.

\begin{description}

\item[intname]  -  The internal name by which the file will be known
                  within ICL.

\item[filename]  -  The VMS file name of the file. If omitted the
                  file name will be intname.DAT.

\end{description}

\section{\xlabel{CREATEGLOBAL}CREATEGLOBAL\label{CREATEGLOBAL}}

   CREATEGLOBAL \hspace{.5cm} parname \hspace{.5cm} type

 Create an ADAM global parameter.

\begin{description}

\item[parname]  -  The name of the global parameter to be created. If
                  a parameter of this name already exists it will be deleted.

\item[type]     -  The type of the parameter to be created. Must be
                  one of \_REAL, \_INTEGER, \_DOUBLE, \_LOGICAL, or
                  \_CHAR*n (where n is an integer).

\end{description}

\section{\xlabel{DCL}\$\ (DCL)\label{DCL}}

   \$ command

 The parameters of the \$ command are concatenated to form a command which is
 executed by the underlying operating system's command language. 
 `DCL' may be used as an alternative to `\$'.

 If the underlying command language is DCL, the specified command is obeyed 
 within a permanent DCL subprocess which is created on the first call to DCL.

 Because the commands are executed in the DCL subprocess rather than in the
 process running ICL, some commands may not have the expected effect. 
 {\em e.g.} \$ SET DEF will set the default directory for the subprocess but not
 for the process running ICL. Use the ICL DEFAULT (DEF) command instead.
 Furthermore, \$ ALLOCATE will allocate a device to the subprocess and not to
 the process running ICL. Use the ICL command ALLOC for this purpose.

 Control-C may be used to abort a DCL command running in the subprocess.

\section{\xlabel{DEALLOC}DEALLOC\label{DEALLOC}}

   DEALLOC \hspace{.5cm} dev 

 Deallocate a device previously allocated using the ICL ALLOC command.

 DEALLOC may be abbreviated to DEALL.

\section{\xlabel{DEFAULT}DEFAULT\label{DEFAULT}}

   DEFAULT \hspace{.5cm} directory

 Set the default directory for both the process running ICL, and for the
 DCL subprocess. The directory may be specified in any of the forms
 accepted by the DCL  SET DEF command. DEFAULT with no parameter
 displays the current default directory.

 DEFAULT should be used in preference to DCL SET DEF which will only
 set the default directory for the DCL subprocess.

 DEFAULT may be abbreviated to DEF.

{\em e.g.}
\begin{verbatim}

    DEF [-]
    DEF DISK$DATA:[ABC.DATA]

\end{verbatim}

\section{\xlabel{DEFHELP}DEFHELP\label{DEFHELP}}

   DEFHELP \hspace{.5cm} name \hspace{.5cm} help\_library \hspace{.5cm} 
[topic...]

 Allows the ICL HELP command to be able to access information in other 
 help libraries.

\begin{description}

 \item[name] - The Help name to be defined.

 \item[help\_library] - The Help library to be used.

 \item[topic...] - An optional list of keywords specifying the path within
{\em help\_library} to the required information. If omitted, {\em topic} 
defaults to {\em name}.

\end{description}

After {\em name} has been defined in this way
\begin{verbatim}

  ICL> HELP name

\end{verbatim}
will return the {\em topic}  information in {\em help\_library}.

\section{\xlabel{DEFINE}DEFINE\label{DEFINE}}

   DEFINE \hspace{.5cm} command \hspace{.5cm} taskname \hspace{.5cm} [action]

 Define a command which issues an OBEYW to an ADAM task

\begin{description}

\item[command] - The command to be defined. An abbreviation
              may be specified in the form COM(MAND) where
              COM is the minimum acceptable abbreviation.

\item[taskname] - The task which will execute the OBEYW.

\item[action] - (optional)  The action to be executed. If
              omitted it defaults to command.

\end{description}
Once a command has been defined then
\begin{verbatim}
    ICL> command . . .
\end{verbatim}
is equivalent to typing 
\begin{verbatim}
    ICL> OBEYW taskname action . . .
\end{verbatim}
with any parameters of command being appended to the OBEYW as the VALUE
string sent to the task.

\section{\xlabel{DEFPROC}DEFPROC\label{DEFPROC}}

   DEFPROC \hspace{.5cm} command \hspace{.5cm} file \hspace{.5cm} [procedure]

 Define a command which runs a procedure from a source file.

\begin{description}

\item[command] - The command to be defined. An abbreviation
              may be specified in the form COM(MAND) where
              COM is the minimum acceptable abbreviation.

\item[file]  - The source file containing the procedure.

\item[procedure] - (optional)  The procedure to be executed. If
              omitted it defaults to command.

\end{description}

When {\em command} is issued, file.ICL is loaded and compiled (if it is
not already loaded) and the procedure is called. Any parameters specified
with the original command are passed to the procedure.

\section{\xlabel{DEFSHARE}DEFSHARE\label{DEFSHARE}}

   DEFSHARE \hspace{.5cm} command \hspace{.5cm} image \hspace{.5cm} [action]

 Define a command which issues an OBEYW to an ADAM shareable image monolith

\begin{description}

\item[command] - The command to be defined. An abbreviation
              may be specified in the form COM(MAND) where
              COM is the minimum acceptable abbreviation.

\item[image] - The logical name of the shareable image monolith
               which will execute the OBEYW.

\item[action] - (optional)  The action to be executed. If
              omitted it defaults to command.

\end{description}
Once a command has been defined with DEFSHARE then
\begin{verbatim}
    ICL> command . . .
\end{verbatim}
is equivalent to typing 
\begin{verbatim}
    ICL> OBEYW taskname action . . .
\end{verbatim}
with any parameters of command being appended to the OBEYW as the VALUE
string sent to the task. The task must be linked into a shareable image
monolith rather than a normal executable task.


\section{\xlabel{DEFSTRING}DEFSTRING\label{DEFSTRING}}

   DEFSTRING \hspace{.5cm} command \hspace{.5cm} equivalence\_string

 Associate a command with an equivalence string.

\begin{description}

\item[command] - The command to be defined. An abbreviation
              may be specified in the form COM(MAND) where
              COM is the minimum acceptable abbreviation.

\item[equivalence\_string] - The equivalence string for the command.

\end{description}
Issuing the command is equivalent to typing the equivalence string. Any
parameters of the command are appended to the equivalence string.

\section{\xlabel{DEFTASK}DEFTASK\label{DEFTASK}}

   DEFTASK \hspace{.5cm} command \hspace{.5cm} [taskname]

 Define a command which issues a SEND to an ADAM task

\begin{description}

\item[command] - The command to be defined. An abbreviation
              may be specified in the form COM(MAND) where
              COM is the minimum acceptable abbreviation.

\item[taskname] - (optional) The task which will receive the SEND.
              If omitted it defaults to command.

\end{description}
Once a command has been defined using DEFTASK then
\begin{verbatim}
    ICL> command . . .
\end{verbatim}
is equivalent to typing 
\begin{verbatim}
    ICL> SEND action . . .
\end{verbatim}
the parameters of command must supply the CONTEXT (GET, SET, OBEY, CANCEL)
and the parameter or action name to be sent to the task.


\section{\xlabel{DEFUSER}DEFUSER\label{DEFUSER}}

   DEFUSER \hspace{.5cm} command \hspace{.5cm} image \hspace{.5cm} [routine]

 Define a command which calls a user written subroutine in a shareable image.

\begin{description}

\item[command] - The command to be defined. An abbreviation
              may be specified in the form COM(MAND) where
              COM is the minimum acceptable abbreviation.         

\item[image] - The name of the shareable image containing the routine.
               This must be a logical name if the image is not in SYS\$SHARE.  

\item[routine] - (optional)  The name of the subroutine to be called. This must
               be a universal symbol of the shareable image. If
              omitted it defaults to command.

\end{description}

 The subroutine should have a single character string parameter in which it
 will receive the ICL command line parameter string. The string is received as  
 typed, except for the substitution of bracketed expressions.

 Typing {\em command} causes the shareable image to be activated dynamically
 (if it is not already loaded), and the subroutine to be called. The 
 subroutine can return an error message to ICL by signalling a VMS condition.
 This will result in the ICL exception USERERR. The exception text will
 contain the message associated with the VMS status returned.

\section{\xlabel{DELETE}DELETE\label{DELETE}}

   DELETE \hspace{.5cm} proc

 Delete procedure proc from the procedure table.

\section{\xlabel{DISMOUNT}DISMOUNT\label{DISMOUNT}}

   DISMOUNT \hspace{.5cm} dev \hspace{.5cm} [ NOUNLOAD ]

 Dismount a tape previously mounted using the ICL MOUNT command. The
 NOUNLOAD parameter if specified means that the tape will be dismounted
 without unloading.

 DISMOUNT may be abbreviated to DISMOU

\section{\xlabel{DUMPTASK}DUMPTASK\label{DUMPTASK}}

   DUMPTASK \hspace{.5cm}  taskname

This command causes the ADAM task of name {\em taskname} to output a
stack dump on the terminal from which it was loaded. For this command
to be effective, the task must include a call to DTASK\_SETDUMP(STATUS).

\section{\xlabel{EDIT}EDIT\label{EDIT}}

   EDIT \hspace{.5cm} proc

 Edit procedure `proc' using the selected editor. If the name of the
 procedure is unchanged during the editing session the new version
 replaces the old one. If the name is changed a new procedure is
 created and the old version of `proc' remains unchanged. By default 
 the TPU editor is used. The editor may be changed to EDT or LSE
 using the SET EDITOR command.

 To edit a {\em file} use the command:

    DCL EDIT filename

\section{\xlabel{ENDOBEY}ENDOBEY\label{ENDOBEY}}

   ENDOBEY \hspace{.5cm} (path) \hspace{.5cm} (messid)

 Wait for completion of an ADAM I-task action initiated by STARTOBEY

\begin{description}

\item[path]  -  The path associated with the action.

\item[messid]  -  The message identifier associated with the action.

\end{description}

\section{\xlabel{EXIT}EXIT\label{EXIT}}

   EXIT

 EXIT from ICL and return control to DCL. On exiting from ICL a
 copy of all the current procedures is saved in the file SAVE.ICL

\section{\xlabel{GET}GET\label{GET}}

   GET \hspace{.5cm} taskname \hspace{.5cm} parameter \hspace{.5cm} (variable)

 Get a parameter of an ADAM I-task and put the value into an ICL variable.

\begin{description}

\item[taskname] - The name of the I-task.

\item[parameter] - The name of the parameter to be got.

\item[variable] - The ICL variable into which the value will be put.

\end{description}


\section{\xlabel{GETGLOBAL}GETGLOBAL\label{GETGLOBAL}}

   GETGLOBAL \hspace{.5cm}  parameter \hspace{.5cm} (variable)

Get the value of a ADAM global parameter

\begin{description}

\item[parameter] - The name of the global parameter to be got.

\item[variable] - An ICL variable to receive the result.

\end{description}

\section{\xlabel{GETNBS}GETNBS\label{GETNBS}}

   GETNBS \hspace{.5cm}  nbsname \hspace{.5cm} (variable)

Get the value of a noticeboard item

\begin{description}

\item[nbsname] - The name of the noticeboard item to be got.

\item[variable] - An ICL variable to receive the result.

\end{description}                                

The noticeboard name must correspond to a primitive item with one of the
standard HDS types ({\em e.g.} \_INTEGER).
A dot notation is used to specify structure
components ({\em e.g.} NOTICEBOARD.ITEM1.ITEM2). If the primitive item is an array
only the first component is accessed.


\section{\xlabel{GETPAR}GETPAR\label{GETPAR}}

   GETPAR \hspace{.5cm} command \hspace{.5cm} parameter \hspace{.5cm} (variable)

Get the value of a parameter from the task's parameter file.
This will only be relevant for parameters which are not {\em internal}.

\begin{description}

\item[command] - A command name associated with an ADAM task by means
                 of a DEFINE command. The task will usually
                 be an A-task or monolith.

\item[parameter] - The name of the parameter to be got.

\item[variable] - An ICL variable to receive the result.

\end{description}
                                                            
\section{\xlabel{HELP}HELP\label{HELP}}

   HELP \hspace{.5cm} [topic...]

 Provide on line documentation on some aspect of ICL, or on some other
 topic which has been made available using the DEFHELP command.

\begin{description}

\item[topic...]  - An optional list of keywords specifying the path within the
help library to the required information. 
The first keyword may be the {\em name} specified in a DEFHELP command. 

\end{description}

\section{\xlabel{INPUT}INPUT\label{INPUT}}

   INPUT \hspace{.5cm} [prompt] \hspace{.5cm}  (string)

 Input a string from the terminal.
\begin{description}

\item[prompt]  -  (optional) A prompt string to be output on the terminal

\item[string]  -  A variable into which the input string will be read.

\end{description}


\section{\xlabel{INPUTI}INPUTI\label{INPUTI}}


    INPUTI \hspace{.5cm} prompt \hspace{.5cm} (i) \hspace{.5cm} (j)  ....

 Input integers from the terminal

\begin{description}

\item[prompt]  -  A prompt string to be output on the terminal.

\item[i,\/j etc] -  The variables into which integers will be read.

\end{description}

\section{\xlabel{INPUTL}INPUTL\label{INPUTL}}


    INPUTL \hspace{.5cm} prompt \hspace{.5cm} (l) \hspace{.5cm} (m)  ....

 Input logical values from the terminal

\begin{description}

\item[prompt]  -  A prompt string to be output on the terminal.

\item[l,m etc] -  The variables into which logical values will be read.

\end{description}
The logical values may be input as YES, NO, TRUE, FALSE or any
abbreviation of these, in either upper or lower case.


\section{\xlabel{INPUTR}INPUTR\label{INPUTR}}

    INPUTR \hspace{.5cm} prompt \hspace{.5cm} (x) \hspace{.5cm} (y)  ....

 Input real numbers from the terminal

\begin{description}

\item[prompt]  -  A prompt string to be output on the terminal.

\item[x,y etc] -  The variables into which the numbers will be read.

\end{description}

\section{\xlabel{KEY}KEY\label{KEY}}

    KEY  \hspace{.5cm} keyname \hspace{.5cm} equivalence\_string
         
 Define an equivalence string for a key

\begin{description}

\item[keyname] -  The name of the key 

\item[equivalence\_string]  -  The equivalence string for the key. A \#
              character may be used to indicate a RETURN character
              terminating the string.

\end{description}
                                     
The KEY command may be used at any time but the definition is only
effective when screen mode is in use (however, the first SET SCREEN call
in a session clears all key definitions).

\section{\xlabel{KEYTRAP}KEYTRAP\label{KEYTRAP}}

    KEYTRAP  \hspace{.5cm} keyname 
         
 Specify trapping of a key

\begin{description}

\item[keyname] -  The name of the key 

\end{description}
                                     
The KEYTRAP command may be used at any time but the definition is only
effective when screen mode is in use (however, the first SET SCREEN call
in a session clears all key definitions). The INKEY function is used to test
for the pressing of a trapped key.

\section{\xlabel{KEYUSER}KEYUSER\label{KEYUSER}}

    KEYUSER  \hspace{.5cm} keyname  \hspace{.5cm} image \hspace{.5cm} routine
         
 Associate key with a user written subroutine in a shareable image

\begin{description}

\item[keyname] -  The name of the key

\item[image] - logical name of the shareable image 
                                     
\item[routine] - name of the subroutine to be called. This must be a universal
symbol of the shareable image.

\end{description}
                                     
The KEYUSER command may be used at any time but the definition is only
effective when screen mode is in use (however, the first SET SCREEN call
in a session clears all key definitions). KEYUSER causes the specified routine
to be called at AST level immediately the key is pressed. If necessary the
shareable image is activated dynamically

\section{\xlabel{KILL}KILL\label{KILL}}

    KILL \hspace{.5cm} taskname

 Kill an ADAM task

\begin{description}

\item[taskname] -  The name of the task to be killed.

\end{description}

\section{\xlabel{KILLDCL}KILLDCL\label{KILLDCL}}

    KILLDCL

 Kill the DCL subprocess

\section{\xlabel{KILLW}KILLW\label{KILLW}}

    KILLW \hspace{.5cm} taskname

 Kill an ADAM task and wait for it to die.

\begin{description}

\item[taskname] -  The name of the task to be killed.

\end{description}

\section{\xlabel{LIST}LIST\label{LIST}}

    LIST \hspace{.5cm} proc

 List procedure `proc' on the terminal

\section{\xlabel{LOAD}LOAD\label{LOAD}}

    LOAD  file

 Accept commands from `file.ICL' rather than from the terminal.

 `file' may specify a procedure previously saved using the SAVE command.

\section{\xlabel{LOADD}LOADD\label{LOADD}}


    LOADD \hspace{.5cm} exename \hspace{.5cm} [taskname] \hspace{.5cm} [priority]

    Load an ADAM task into a detached process and wait for
    loading to complete.
\begin{description}

\item[exename] - The name of the executable image file to
            be used for the task.

\item[taskname] - (optional) The name of the task to be created.
            If omitted it defaults to the file name part of
            exename.

\item[priority] - (optional) The priority for the created task.

\end{description}

\section{\xlabel{LOADW}LOADW\label{LOADW}}


    LOADW \hspace{.5cm} exename \hspace{.5cm} [taskname]

    Load an ADAM task into a subprocess and wait for loading to complete:
\begin{description}

\item[exename] - The name of the executable image file to
            be used for the task.

\item[taskname] - (optional) The name of the task to be created.
            If omitted it defaults to the file name part of
            exename.

\end{description}

\section{\xlabel{MOUNT}MOUNT\label{MOUNT}}

    MOUNT \hspace{.5cm} dev \hspace{.5cm} [ (status) ]

 Mount a tape - equivalent to the DCL MOUNT/FOREIGN command. The tape
 is mounted as a foreign tape at its initialized density.

 The optional parameter status is a variable in which the status of
 the mount operation will be returned. The success of the operation
 may be tested using the function OK(status).

 MOUNT may be abbreviated to MOU.

\section{\xlabel{NOREP}NOREP\label{NOREP}}

   NOREP

Turn off reporting. Reporting is turned on by the REPORT command.
                 


\section{\xlabel{OBEYW}OBEYW\label{OBEYW}}

    OBEYW \hspace{.5cm} taskname \hspace{.5cm} action  . . . 

 Send an OBEY message to an ADAM task and wait for it to complete. Load the
 task as as a cached task if necessary.

\begin{description}

 \item[taskname] - The name of the task to which the obey message will be sent.

 \item[action] - The action name to be obeyed.

 \item[. . .] - Any parameters for the action.

\end{description}

\section{\xlabel{OPEN}OPEN\label{OPEN}}
                                       
   OPEN \hspace{.5cm} intname \hspace{.5cm} [filename]

 Open an existing text file for input. 

\begin{description}

\item[intname]  -  The internal name by which the file will be known
                  within ICL.

\item[filename]  -  The VMS file name of the file. If omitted the
                  file name will be intname.DAT.

\end{description}

\section{\xlabel{PRINT}PRINT\label{PRINT}}

    PRINT  \hspace{.5cm}  p1 \hspace{.5cm} p2  ....

 The parameters of PRINT are concatenated and printed on the terminal.

\section{\xlabel{PROCS}PROCS\label{PROCS}}

    PROCS

 List the names of all current procedures.

\section{\xlabel{PUTNBS}PUTNBS\label{PUTNBS}}

   PUTNBS \hspace{.5cm}  nbsname \hspace{.5cm} value

Write the value of a noticeboard item

\begin{description}

\item[nbsname] - The name of the noticeboard item to be written.

\item[value] - The new value for the item.

\end{description}                                

The noticeboard name must correspond to a primitive item with one of the
standard HDS types ({\em e.g.} \_INTEGER). 
A dot notation is used to specify structure
components ({\em e.g.} NOTICEBOARD.ITEM1.ITEM2). 
If the primitive item is an array, only the first component is accessed.


\section{\xlabel{READ}READ\label{READ}}

   READ \hspace{.5cm} intname \hspace{.5cm}  (string)

 Read a line from a text file.
\begin{description}

\item[intname]  -  The internal name of the text file. The file must have
                   been opened with the OPEN command.

\item[string]  -  A variable into which the input string will be read.

\end{description}


\section{\xlabel{READI}READI\label{READI}}


    READI \hspace{.5cm} intname \hspace{.5cm} (i) \hspace{.5cm} (j)  ....

 Read a line of integers from a text file.

\begin{description}

\item[intname]  -  The internal name of the text file. The file must have
                   been opened with the OPEN command.

\item[i,j etc] -  The variables into which integers will be read.

\end{description}

\section{\xlabel{READL}READL\label{READL}}


    READL \hspace{.5cm} prompt \hspace{.5cm} (l) \hspace{.5cm} (m)  ....

 Read a line of logical values from a text file.

\begin{description}

\item[intname]  -  The internal name of the text file. The file must have
                   been opened with the OPEN command.

\item[l,m etc] -  The variables into which logical values will be read.

\end{description}
The logical values may be input as YES, NO, TRUE, FALSE or any
abbreviation of these, in either upper or lower case.


\section{\xlabel{READR}READR\label{READR}}

    READR \hspace{.5cm} intname \hspace{.5cm} (x) \hspace{.5cm} (y)  ....

 Read a line of real numbers from a text file

\begin{description}

\item[intname]  -  The internal name of the text file. The file must have
                   been opened with the OPEN command.

\item[x,y etc] -  The variables into which the numbers will be read.

\end{description}

                                 
\section{\xlabel{REPFILE}REPFILE\label{REPFILE}}

    REPFILE \hspace{.5cm} [logfile] \hspace{.5cm} [DTNS]

Examine a logfile

\begin{description}

\item[logfile] -  Name of the logfile to be examined - defaults to
ADAM\_LOGFILE

\item[DTNS]  -  Four character string specifying which items in each log 
record are displayed.
D = Date, T = Time, N = Name, S = String.
Replace the item letter by anything else to stop the display of that item.
DTNS is case independent.

\end{description}
        
\section{\xlabel{REPORT}REPORT\label{REPORT}}

    REPORT \hspace{.5cm} [logfile]

Turn on reporting. After reporting has been turned on, a log of input/output and
ADAM message system traffic is written to the file specified by {\it logfile}.
If the parameter is omitted the logical name ADAM\_LOGFILE is used. 
If the file already exists, the log is appended to it.

The logfile may be examined using the REPFILE command or by the ADAM task,
LISTLOG. LISTLOG is found along the ADAM\_EXE search path and provides a means
of obtaining readable hardcopy of the log. A DCL symbol LISTLOG is defined
by ADAMSTART to run this task from DCL.

The command NOREP turns off reporting.


\section{\xlabel{SAVE}SAVE\label{SAVE}}

   SAVE \hspace{.5cm} proc

 Save procedure `proc' in the file `proc.ICL'. 

   SAVE  ALL

 Save all procedures in file `SAVE.ICL'.

 Procedures save in this way may be reloaded using LOAD.

\section{\xlabel{SAVEINPUT}SAVEINPUT\label{SAVEINPUT}}

   SAVEINPUT \hspace{.5cm} lines \hspace{.5cm} filename

 Save previous input lines in a text file. `lines' specifies the number of
lines to be saved, and `filename' is the file in which they are saved. If
`filename' is omitted it defaults to SAVEINPUT.ICL in the default directory.
If `lines' is omitted the entire input buffer is saved.

\section{\xlabel{SEND}SEND\label{SEND}}

    SEND \hspace{.5cm} taskname \hspace{.5cm} context \hspace{.5cm} . . . 

 Send a GET, SET, OBEY or CANCEL message to an ADAM task. Load the
 task as as a cached task if necessary.

\begin{description}

 \item[taskname] - The name of the task to which the message will be sent.
It must correspond with taskname in the ALOAD, LOADW or DEFINE command and
include any directory specification given there.

 \item[context] - The context (GET, SET, OBEY or CANCEL).

 \item[. . .] - The remainder of the message. This will depend upon context.
\begin{description}
\item[SET] parameter value
\begin{description}
\item[parameter] - The name of the parameter whose value is to be set.
\item[value] - The value to be set. Type conversion will be done where
necessary.
\end{description}
\item[GET] parameter
\begin{description}
\item[parameter] - The name of the parameter. Its value will be displayed.
\end{description}
\item[OBEY] action parameters
\begin{description}
\item[action] - The name of the action to be obeyed.
\item[parameters] - the parameter string for action.
\end{description}
\item[CANCEL] action parameters
\begin{description}
\item[action] - The name of the action to be cancelled.
\item[parameters] - the parameter string for action.
\end{description}
\end{description}
OBEY, SET and GET contexts may be used with A-task monoliths. 
In that case, {\em action} is replaced by {\em atask} and {\em parameter} 
by {\em atask:parameter} where {\em atask} is the name of
the individual A-task within the monolith.
\end{description}


\section{\xlabel{SET}SET\label{SET}}

The SET command is used to control various features of the state of ICL.

\subsection{\xlabel{SET_ATTRIBUTES}SET ATTRIBUTES\label{SET_ATTRIBUTES}}

   SET ATTRIBUTES \hspace{.5cm} attributes

Sets the attributes for text written with the LOCATE command. {\it attributes}
is a string containing any combination of the letters D (double size),
B (bold), R (Reverse video), U (underlined) and F (flashing).

\subsection{\xlabel{SET_AUTOLOAD}SET AUTOLOAD, SET NOAUTOLOAD\label{SET_AUTOLOAD}}

   SET AUTOLOAD

   SET NOAUTOLOAD

Switches on/off the automatic loading of tasks in response to an invocation.
ICL's default is AUTOLOAD.
If AUTOLOAD is not selected, tasks must be loaded explicitly with command
ALOAD or LOADW.

\subsection{\xlabel{SET_CHECKPARS}SET CHECKPARS, SET NOCHECKPARS\label{SET_CHECKPARS}}

   SET CHECKPARS

   SET NOCHECKPARS

Switches on/off the checking of parameters supplied to ICL procedures.
ICL's default is CHECKPARS -- in that case a procedure will not be executed
unless the correct number of parameters is supplied. NOCHECKPARS allows
the procedure to use the UNDEFINED function to test if the parameter was
supplied and prompt if not. 
{\em Note that only trailing parameters may be omitted.}

\subsection{\xlabel{SET_EDITOR}SET EDITOR\label{SET_EDITOR}}

   SET EDITOR \hspace{.5cm} name

Sets the editor to be used for editing of ICL procedures. {\em name}
must be one of TPU, EDT or LSE. (LSE may not be available on all systems).
The default editor is TPU.

\subsection{\xlabel{SET_HELPFILE}SET HELPFILE\label{SET_HELPFILE}}

   SET HELPFILE library

Specifies a help library to be used by the ICL HELP system for topics other
than those specified by a DEFHELP command. The library specification 
{\em library} should include a directory specification.

The default help library is ICLDIR:ICLHELP.


\subsection{\xlabel{SET_MESSAGES}SET MESSAGES, SET NOMESSAGES\label{SET_MESSAGES}}

   SET MESSAGES

   SET NOMESSAGES

These commands control whether or not loading messages are output when
ADAM tasks are loaded. By default loading messages are output.

\subsection{\xlabel{SET_PRECISION}SET PRECISION\label{SET_PRECISION}}

   SET PRECISION \hspace{.5cm} digits

Set the number of decimal digits precision for unformatted conversions
of real values to strings. The default value is 6 and may be set to any 
value between 1 and 16.

\subsection{\xlabel{SET_PROMPT}SET PROMPT\label{SET_PROMPT}}

   SET PROMPT \hspace{.5cm} string

This command can be used to specify a prompt string to replace the
default ICL$>$ prompt.
                                                             
\subsection{\xlabel{SET_SAVE}SET SAVE, SET NOSAVE\label{SET_SAVE}}

   SET SAVE

   SET NOSAVE

This command controls whether ICL procedures are saved in a SAVE.ICL file
when ICL exits. By default a SAVE.ICL file is created if there are any
procedures to save. 

\subsection{\xlabel{SET_SCREEN}SET SCREEN, SET NOSCREEN\label{SET_SCREEN}}

   SET SCREEN n

   SET NOSCREEN

These commands select screen I/O mode or normal I/O mode. On the SET SCREEN
mode the parameter n specifies the number of lines in the scrolling region.
If n is omitted eight lines of scrolling region are allocated.

\subsection{\xlabel{SET_TRACE}SET TRACE, SET NOTRACE\label{SET_TRACE}}

   SET TRACE

   SET NOTRACE

These commands turn on and off the tracing of procedure execution. By 
default tracing is off. When tracing is on, each line of the procedure is 
output on the terminal before execution.



\section{\xlabel{SETGLOBAL}SETGLOBAL\label{SETGLOBAL}}

   SETGLOBAL \hspace{.5cm}  parameter \hspace{.5cm} value

Set the value of a ADAM global parameter

\begin{description}

\item[parameter] - The name of the global parameter to be set.

\item[value] - The value for the parameter. It should not be enclosed in quotes
unless the quotes are required as part of the global parameter value.

\end{description}
If the global parameter does not exist, one of type \_CHAR*132 is created.
This is usually OK but could give problems with \_LOGICAL parameters.
The CREATEGLOBAL command can be used to create global parameter storage of a 
specific type.

\section{\xlabel{SETPAR}SETPAR\label{SETPAR}}

   SETPAR \hspace{.5cm} command \hspace{.5cm} parameter \hspace{.5cm} value

Set the value of a parameter into the task's parameter file.
This will only be relevant for parameters which are not {\em internal}.

\begin{description}

\item[command] - A command name associated with an ADAM task by means
                 of a DEFINE command. The task will usually be an A-task
                 or monolith.

\item[parameter] - The name of the parameter to be set.

\item[value] - The value for the parameter.

\end{description}

If the parameter value is an HDS object name or device name it must be 
prefixed by @. SETPAR sets the current value of the parameter. Thus the
VPATH or PPATH must include CURRENT for SETPAR to have any effect on
subsequent execution of the command.
Note that if the task is loaded, it will probably have its parameter file open
and locked against ICL writing into it.

\section{\xlabel{SIGNAL}SIGNAL\label{SIGNAL}}

   SIGNAL \hspace{.5cm} name  \hspace{.5cm} [text]

 Signal an ICL exception

\begin{description}

 \item[name] - The exception name --- any valid ICL identifier.

 \item[text] - A message text associated with the exception.

\end{description}

Following SIGNAL an exception handler will be executed if one exists
for the exception. Otherwise a message will be output, and control will
return to direct mode.


\section{\xlabel{SPAWN}SPAWN\label{SPAWN}}

    SPAWN \hspace{.5cm}  [ dcl\_command ]

 SPAWN with parameters concatenates the parameters to form a DCL command.
 creates a subprocess, and executes the DCL command in the subprocess.

 SPAWN with no parameters creates a subprocess in which a series of DCL
 commands may be executed. Use LOGOUT to return to ICL.

 In most cases DCL is a faster alternative to SPAWN as the subprocess does
 not have to be created for each new command. However SPAWN will do a few
 things which are not possible under DCL. SPAWNed commands will prompt for
 their parameters, for example, whereas commands issued using DCL will not.


\section{\xlabel{STARTOBEY}STARTOBEY\label{STARTOBEY}}

   STARTOBEY \hspace{.5cm} (path) \hspace{.5cm} (messid) \hspace{.5cm} task 
\hspace{.5cm} action \hspace{.5cm} value

 Send an OBEY message to an ADAM task and return the path and message-id.
 Used in conjunction with ENDOBEY to set up multiple concurrent actions in
 a I-task.

\begin{description}

\item[path] - An ICL variable to receive the path associated with the action.

\item[messid] - An ICL variable to receive the message identifier associated 
               with the action.

\item[task] - The task to which the OBEY message will be sent.

\item[action] - The action to be executed.

\item[value] - Any parameters associated with the action.

\end{description}

\section{\xlabel{TASKS}TASKS\label{TASKS}}

   TASKS

 List all loaded ADAM tasks.

\section{\xlabel{VARS}VARS\label{VARS}}

   VARS \hspace{.5cm} [ proc ]

 Lists the variables of procedure `proc' with their current types and
 values. VARS with no parameters lists the variables at direct mode.

\section{\xlabel{WAIT}WAIT\label{WAIT}}

    WAIT  interval

Wait for a specified {\it interval} expressed in seconds.

\section{\xlabel{WRITE}WRITE\label{WRITE}}

    WRITE  intname \hspace{.5cm}  p2 \hspace{.5cm} p3  ....

 The parameters of WRITE are concatenated and written to the text file
of name {\it intname}. This file must have been previously opened with an
APPEND or CREATE command.


\chapter{\xlabel{icl_exceptions}ICL Exceptions}
\begin{center}
\begin{tabular}{ll}
\\
name & description \\
\\
ADAMERR  &  An Error has occurred in an ADAM task. \\
   &  An associated message will give further information. \\
\\
ASSNOTVAR  &  An Assignment has been made to a procedure formal parameter \\
   &  which does not correspond to a variable in the procedure call. \\
\\
CLOSEERR  &  Error closing text file. \\
\\           
CONVERR  &  Error converting RA or Dec to string or vice versa \\
\\
CTRLC  &  A Control-C has been entered on the terminal.\\
\\
DEVERR  &  Error allocating or mounting device.\\
\\
EDITERR  &  Attempt to use the LSE editor on a system on which LSE is \\
  &  not available, or an attempt to use TPU when the TPU shareable \\
  &  image is not accessible. \\
\\           
EOF & End of file encountered on text file operation. \\
\\
FIGERR  &  Error in a FIGARO program, or an attempt to use a FIGARO \\
  &  command when the FIGARO shareable image is not accessible. \\
\\
FLTDIV  &  Floating point division by zero. \\
\\
FLTOVF  &  Floating point overflow. \\
\\
\end{tabular}
\end{center}
\newpage
\begin{center}
\begin{tabular}{ll}
\\
name & description \\
\\
IFERR  &  The expression in an IF or ELSE IF statement does not  \\
   & evaluate to a logical value.  \\                              
\\
INTOVF  &  Integer Overflow. \\
\\
INVARGMAT  &  Invalid argument to a mathematical function.  \\
\\
INVSET  &  Invalid SET command.  \\
\\
LOGZERNEG  &  Logarithm of zero or negative number.  \\
\\
NBSERR  &  Error in GETNBS or PUTNBS. \\
\\                                        
OPENERR  &  Error opening text file. \\
\\
OPNOTLOG  &  Operands of a logical operator (AND, OR {\em etc.}) are \\
   &  not logical values.  \\
\\
OPNOTNUM  &  Operands of a formatting operation (:) are not  \\
  &  numeric.  \\
\\
PROCERR  &  Unrecognized procedure or command name.  \\
\\                                        
READERR &  Error reading from text file. \\
\\
RECCALL  &  Attempt to make a recursive call of a procedure. \\
\\
SCREENERR  &  Error in screen mode I/O. \\
\\
SQUROONEG  &  Square root of negative number. \\
\\
STKOVFLOW  &  ICL's stack has overflowed. \\
\\
STKUNDFLOW  &  ICL stack underflow --- If this occurs it indicates \\
    & an internal error in ICL  ---  Please report the circumstances.  \\
\\
TOOFEWPARS  &  Not enough parameters for a function or command. \\
\\
\end{tabular}
\end{center}
\newpage
\begin{center}
\begin{tabular}{ll}
\\
name & description \\
\\
TOOMANYPARS  &  Too many parameters for a procedure or command. \\
\\
UNDEFVAR  &  Attempt to use an undefined variable --- {\em i.e.} one \\
  & that has not yet had a value assigned.  \\
\\
UNDEXP  &  Undefined Exponentiation.  \\
\\
USERERR  &  Error accessing a routine defined using DEFUSER, or \\
  &  error during such a routine.  \\
\\
WHILEERR & The expression in a WHILE statement does not \\
  &  evaluate to a logical value. \\                    
\\
WRITERR  &  Error writing to text file. \\
\\
\end{tabular}
\end{center}

\chapter{\xlabel{icl_syntax}ICL Syntax}
The following gives a formal definition of the syntax of ICL.
Note that there are in effect two levels to the ICL Syntax. One
level describes how statements (strictly simple\_statements as defined
below) are built up, while a second level
describes how statements (normally one per line) are combined to form
control structures and procedures. The definitions of {\em if\_block},
{\em loop\_block} and {\em procedure} below form the second level syntax,
and in these cases items on different lines in the definition, must
appear on separate lines. The notation is as follows.
\begin{itemize}

\item Lower case words denote non terminal symbols of the grammar.
{\em i.e.} symbols which are defined in terms of other symbols.

\item Upper case words or other characters are terminal symbols.
{\em i.e.} the basic lexical elements out of which the language is built.

\item The construction  \verb+a | b+ denotes a choice between two options.

\item The construction  \verb+[ a ]+ implies that a is optional.

\item The construction  \verb+{ a }+ implies that the symbol a may be repeated
zero or more times.

\end{itemize}
Both spaces and layout are significant. Spaces may not appear within
identifiers, or numbers. Spaces may be used as separators in parameter
lists. layout is currently restricted to one statement per line. Thus the
end of line character is effectively a statement separator.

\begin{verbatim}

letter  =   any of the letters A to Z or a to z

digit   =   0 | 1 | 2 | 3 | 4 | 5 | 6 | 7 | 8 | 9

binary_digit  =  0 | 1

octal digit  =  0 | 1 | 2 | 3 | 4 | 5 | 6 | 7

hex_digit  =  digit | A | B | C | D | E | F | a | b | c | d | e | f

comment  =  ; anything   |
            { anything

unquoted_string  =  any sequence of characters not including a space,
                    comma or left parenthesis

string_delimiter  =  ' | "

open_string  =   any sequence of characters not including a string delimiter

string  =   string_delimiter  open_string  string_delimiter  
            { string_delimiter  open_string  string_delimiter  }

identifier  =  letter  {  letter | digit | _  }

     ( certain identifiers have a special meaning within the language
       and are not available for general use

       these are  AND, OR, NOT, LOOP, WHILE, FOR, IF, ELSE, END, PROC,
                  TRUE, FALSE, BREAK, ENDIF, ELSEIF, ENDPROC, ENDLOOP,
                  EXCEPTION, ENDEXCEPTION. 

       In identifiers a letter in upper or lower case is considered
       to be the same  )

integer  =  digit  {  digit  }

binary_integer  =  %B binary_digit  {  binary_digit  }

octal_integer  =  %O octal_digit  {  octal_digit  }

hex_integer  =  %X  hex_digit  {  hex_digit  }

scale_factor  =  E  integer  |
                 E  +  integer  |
                 E  -  integer

real  =  integer  .  {  digit  }  |
         integer  scale_factor  |
	 integer  .  {  digit  }  scale_factor

number  =  real | integer | binary_integer | octal_integer | hex_integer

multiplication_operator  =  * | /

addition_operator  =  + | -

relational_operator  =   = | < | > | >= | <= | <>

logical_operator  =  AND | OR

function_call =  identifier  (  [ expression { , expression } ] )

primary  =  identifier  |  number  |  string  | function_call
            TRUE  |  FALSE  |  (  expression  )

factor  =  primary  {  **  primary  }

term    =  factor  {  multiplication_operator  factor  }

simple_expression  =  [  addition operator ]  term  
			{  addition_operator  term  }

relation  =  simple_expression  |
             simple_expression  relational_operator  simple_expression  |
             simple_expression : simple_expression [ : simple_expression ]

expression  =  relation  |
               NOT  relation  |
               relation  {  logical_operator  relation  }  |
               relation  {  &  relation  }

parameter  =  unquoted_string  |  string  |  (  expression  )

simple_statement  =   =  expression  |
                      identifier  =  expression  |
                      comment  |
                      simple_statement  comment

command   =   identifier  [  parameter  { [,]  parameter  }  ]  


statement  =  simple_statement  |  command  |  if_block  |  
              loop_block  |  BREAK

if_block  =   IF  expression
                { statement }
              [ ELSE IF  expression
                { statement } ]
              [ ELSE
                { statement } ]
              END IF

loop_clause =  WHILE expression  |  
               FOR identifier = expression TO expression [ STEP expression ]

loop_block =  LOOP [loop_clause]
              { statement }
              END LOOP

procedure  =  PROC identifier [  identifier  { [,]  identifier }  ]
              { statement }
              { EXCEPTION identifier  
                { statement }
                END EXCEPTION }
              END PROC

\end{verbatim}
 
\end{document}
