\documentclass[11pt]{article}
\pagestyle{myheadings}

% -----------------------------------------------------------------------------
% ? Document identification
\newcommand{\stardoccategory}  {Starlink User Note}
\newcommand{\stardocinitials}  {SUN}
\newcommand{\stardocsource}    {sun\stardocnumber}
\newcommand{\stardocnumber}    {53.3}
\newcommand{\stardocauthors}   {Marion Fisher\\
                                William Lupton\\
                                Anne Charles}
\newcommand{\stardocdate}      {01 October 1996}
\newcommand{\stardoctitle}     {ECHWIND --- UCL Echelle Spectrograph\\
                                Observation Planning Program}
\newcommand{\stardocversion}   {Version 3.0}
\newcommand{\stardocmanual}    {User Guide}
%\newcommand{\stardocabstract}  {[Text of abstract]}
% ? End of document identification

% -----------------------------------------------------------------------------

\newcommand{\stardocname}{\stardocinitials /\stardocnumber}
\markright{\stardocname}
\setlength{\textwidth}{160mm}
\setlength{\textheight}{230mm}
\setlength{\topmargin}{-2mm}
\setlength{\oddsidemargin}{0mm}
\setlength{\evensidemargin}{0mm}
\setlength{\parindent}{0mm}
\setlength{\parskip}{\medskipamount}
\setlength{\unitlength}{1mm}

% -----------------------------------------------------------------------------
%  Hypertext definitions.
%  ======================
%  These are used by the LaTeX2HTML translator in conjunction with star2html.

%  Comment.sty: version 2.0, 19 June 1992
%  Selectively in/exclude pieces of text.
%
%  Author
%    Victor Eijkhout                                      <eijkhout@cs.utk.edu>
%    Department of Computer Science
%    University Tennessee at Knoxville
%    104 Ayres Hall
%    Knoxville, TN 37996
%    USA

%  Do not remove the %begin{latexonly} and %end{latexonly} lines (used by 
%  star2html to signify raw TeX that latex2html cannot process).
%begin{latexonly}
\makeatletter
\def\makeinnocent#1{\catcode`#1=12 }
\def\csarg#1#2{\expandafter#1\csname#2\endcsname}

\def\ThrowAwayComment#1{\begingroup
    \def\CurrentComment{#1}%
    \let\do\makeinnocent \dospecials
    \makeinnocent\^^L% and whatever other special cases
    \endlinechar`\^^M \catcode`\^^M=12 \xComment}
{\catcode`\^^M=12 \endlinechar=-1 %
 \gdef\xComment#1^^M{\def\test{#1}
      \csarg\ifx{PlainEnd\CurrentComment Test}\test
          \let\html@next\endgroup
      \else \csarg\ifx{LaLaEnd\CurrentComment Test}\test
            \edef\html@next{\endgroup\noexpand\end{\CurrentComment}}
      \else \let\html@next\xComment
      \fi \fi \html@next}
}
\makeatother

\def\includecomment
 #1{\expandafter\def\csname#1\endcsname{}%
    \expandafter\def\csname end#1\endcsname{}}
\def\excludecomment
 #1{\expandafter\def\csname#1\endcsname{\ThrowAwayComment{#1}}%
    {\escapechar=-1\relax
     \csarg\xdef{PlainEnd#1Test}{\string\\end#1}%
     \csarg\xdef{LaLaEnd#1Test}{\string\\end\string\{#1\string\}}%
    }}

%  Define environments that ignore their contents.
\excludecomment{comment}
\excludecomment{rawhtml}
\excludecomment{htmlonly}

%  Hypertext commands etc. This is a condensed version of the html.sty
%  file supplied with LaTeX2HTML by: Nikos Drakos <nikos@cbl.leeds.ac.uk> &
%  Jelle van Zeijl <jvzeijl@isou17.estec.esa.nl>. The LaTeX2HTML documentation
%  should be consulted about all commands (and the environments defined above)
%  except \xref and \xlabel which are Starlink specific.

\newcommand{\htmladdnormallinkfoot}[2]{#1\footnote{#2}}
\newcommand{\htmladdnormallink}[2]{#1}
\newcommand{\htmladdimg}[1]{}
\newenvironment{latexonly}{}{}
\newcommand{\hyperref}[4]{#2\ref{#4}#3}
\newcommand{\htmlref}[2]{#1}
\newcommand{\htmlimage}[1]{}
\newcommand{\htmladdtonavigation}[1]{}

%  Starlink cross-references and labels.
\newcommand{\xref}[3]{#1}
\newcommand{\xlabel}[1]{}

%  LaTeX2HTML symbol.
\newcommand{\latextohtml}{{\bf LaTeX}{2}{\tt{HTML}}}

%  Define command to re-centre underscore for Latex and leave as normal
%  for HTML (severe problems with \_ in tabbing environments and \_\_
%  generally otherwise).
\newcommand{\latex}[1]{#1}
\newcommand{\setunderscore}{\renewcommand{\_}{{\tt\symbol{95}}}}
\latex{\setunderscore}

%  Redefine the \tableofcontents command. This procrastination is necessary 
%  to stop the automatic creation of a second table of contents page
%  by latex2html.
\newcommand{\latexonlytoc}[0]{\tableofcontents}

% -----------------------------------------------------------------------------
%  Debugging.
%  =========
%  Remove % on the following to debug links in the HTML version using Latex.

% \newcommand{\hotlink}[2]{\fbox{\begin{tabular}[t]{@{}c@{}}#1\\\hline{\footnotesize #2}\end{tabular}}}
% \renewcommand{\htmladdnormallinkfoot}[2]{\hotlink{#1}{#2}}
% \renewcommand{\htmladdnormallink}[2]{\hotlink{#1}{#2}}
% \renewcommand{\hyperref}[4]{\hotlink{#1}{\S\ref{#4}}}
% \renewcommand{\htmlref}[2]{\hotlink{#1}{\S\ref{#2}}}
% \renewcommand{\xref}[3]{\hotlink{#1}{#2 -- #3}}
%end{latexonly}
% -----------------------------------------------------------------------------
% ? Document specific \newcommand or \newenvironment commands.
% ? End of document specific commands
% -----------------------------------------------------------------------------
%  Title Page.
%  ===========
\renewcommand{\thepage}{\roman{page}}
\begin{document}
\thispagestyle{empty}

%  Latex document header.
%  ======================
\begin{latexonly}
   CCLRC / {\sc Rutherford Appleton Laboratory} \hfill {\bf \stardocname}\\
   {\large Particle Physics \& Astronomy Research Council}\\
   {\large Starlink Project\\}
   {\large \stardoccategory\ \stardocnumber}
   \begin{flushright}
   \stardocauthors\\
   \stardocdate
   \end{flushright}
   \vspace{-4mm}
   \rule{\textwidth}{0.5mm}
   \vspace{5mm}
   \begin{center}
   {\Huge\bf  \stardoctitle \\ [2.5ex]}
   {\LARGE\bf \stardocversion \\ [4ex]}
   {\Huge\bf  \stardocmanual}
   \end{center}
   \vspace{5mm}

% ? Heading for abstract if used.
%  \vspace{10mm}
%  \begin{center}
%     {\Large\bf Abstract}
%  \end{center}
% ? End of heading for abstract.
\end{latexonly}

%  HTML documentation header.
%  ==========================
\begin{htmlonly}
   \xlabel{}
   \begin{rawhtml} <H1> \end{rawhtml}
      \stardoctitle\\
      \stardocversion\\
      \stardocmanual
   \begin{rawhtml} </H1> \end{rawhtml}

% ? Add picture here if required.
% ? End of picture

   \begin{rawhtml} <P> <I> \end{rawhtml}
   \stardoccategory \stardocnumber \\
   \stardocauthors \\
   \stardocdate
   \begin{rawhtml} </I> </P> <H3> \end{rawhtml}
      \htmladdnormallink{CCLRC}{http://www.cclrc.ac.uk} /
      \htmladdnormallink{Rutherford Appleton Laboratory}
                        {http://www.cclrc.ac.uk/ral} \\
      \htmladdnormallink{Particle Physics \& Astronomy Research Council}
                        {http://www.pparc.ac.uk} \\
   \begin{rawhtml} </H3> <H2> \end{rawhtml}
      \htmladdnormallink{Starlink Project}{http://star-www.rl.ac.uk/}
   \begin{rawhtml} </H2> \end{rawhtml}
   \htmladdnormallink{\htmladdimg{source.gif} Retrieve hardcopy}
      {http://star-www.rl.ac.uk/cgi-bin/hcserver?\stardocsource}\\

%  HTML document table of contents. 
%  ================================
%  Add table of contents header and a navigation button to return to this 
%  point in the document (this should always go before the abstract \section). 
  \label{stardoccontents}
  \begin{rawhtml} 
    <HR>
    <H2>Contents</H2>
  \end{rawhtml}
  \renewcommand{\latexonlytoc}[0]{}
  \htmladdtonavigation{\htmlref{\htmladdimg{contents_motif.gif}}
        {stardoccontents}}

% ? New section for abstract if used.
% \section{\xlabel{abstract}Abstract}
% ? End of new section for abstract
\end{htmlonly}

% -----------------------------------------------------------------------------
% ? Document Abstract. (if used)
%  ==================
%\stardocabstract
% ? End of document abstract
% -----------------------------------------------------------------------------
% ? Latex document Table of Contents (if used).
%  ===========================================
 \newpage
 \begin{latexonly}
   \setlength{\parskip}{0mm}
   \latexonlytoc
   \setlength{\parskip}{\medskipamount}
   \markright{\stardocname}
 \end{latexonly}
% ? End of Latex document table of contents
% -----------------------------------------------------------------------------
\newpage
\renewcommand{\thepage}{\arabic{page}}
\setcounter{page}{1}
%------------------------------------------------------------------------------
% Add any \newcommand or \newenvironment commands here
%------------------------------------------------------------------------------
\newcommand{\coude}{coud\'{e}}

\section{Introduction}

ECHWIND is used to assist in the planning of observations with either the
Utrecht Echelle Spectrograph (UES) or the UCL \coude\ echelle spectrograph
(UCLES). It can be run at your home institution, when preparing an observing
proposal, or before or during an observing run.

The program allows the user to view interactively that part of the spectrum
which will fall on the detector for a given central wavelength. This is
achieved by drawing a complete echellogram in an x-window along with a box
representing the exact size and shape of the detector. The positions of
individual spectral lines can also be marked on the display. The user can
then move the detector window around by moving a crosshair cursor to the
desired centre of the detector and then clicking the left-hand mouse button
to set the detector position, or else by explicitly specifying the desired
central wavelength. At each position the screen displays the minimum,
central and maximum wavelengths and order numbers falling on the detector
as well as an estimate of the length of the detector in Angstroms.  The
position of the box can be marked on the screen so as to ease preparation
of observations where multiple overlapping wavelength ranges are required,
and facilities are available for saving the current detector position in a
text file. For further details, see the detailed description of how to run
the program in section~\ref{Detailed Description}.

\section{Getting Going}
\subsection{Startup}

To run echwind, type:

\begin{verbatim}
      % echwind
\end{verbatim}

This sets up any necessary environment variables, and runs the program. A
PGPLOT window will pop up, and you will be prompted for the following
items:

\begin{verbatim}
      CONFIG - Instrument configuration?  /UES/31/LONG/ >
      DETLIST - File containing user detector list?  >
      DETSYSTEM - File containing system detector list?  /detectors.dat/ >
      Detector (TEK, EEV6, EEV9, EEV-THIN, LORAL or xx,yy in mm)? > tek
      ANGWAVE - Slit angle offset or wavelength?  >
      WAVE - Central wavelength?  >
      LINES - File containing line list?  > ${ECHWIND_HOME}balmer.dat
      Graphics device/type (? to see list, default /NULL): /xwin
\end{verbatim}

The above example shows echwind set up for UES with the 31 echelle and the
TEK chip, with Balmer lines plotted on the echellogram. If you press 
\verb+<CR>+ after the lines prompt, no lines will be plotted.

For the 'Graphic device/type' answer {\tt /xwin} or whatever is appropriate 
for your terminal. Normally you should choose a terminal device type. Device 
types which are output only, such as {\tt /ps}, will result in a plot file being
created for the parameters which you have chosen, 
then the program will exit immediately.

Full details of all parameters are given in section~\ref{Detailed Description}.

\subsection{Menu}

Various menu options are available for zooming, saving configurations,
marking detector boxes, positioning to specified wavelengths and so on. To
obtain the menu, press the middle or righthand mouse button.
Now move the cursor to the item of your choice and then press
any mouse button. The menu will disappear and whatever is necessary will
happen --- this may involve having to respond to a prompt in the original 
window.

To exit from the program, select the EXIT menu option.

Full details of all menu options are given in 
section~\ref{Detailed Description}.

\section{Detailed Description}
\label{Detailed Description}

This section contains a detailed description of all of the facilities offered
by ECHWIND. 

\subsection{Selecting the Echelle Grating and Instrument: CONFIG, New ECHELLE}

A spectrograph parameter file is read on startup and this
defines the available and default instruments, echelles and cameras.
The instruments currently supported are UES and UCLES, the default being
UES with the 31 echelle. Configurations are specified at the CONFIG prompt
as follows:

\begin{verbatim}
      UES/31/LONG
      UES/79/LONG
      UCLES/31/LONG
      UCLES/79/LONG
\end{verbatim}

You can change the echelle grating from within the program by selecting the
``New ECHELLE'' menu option. This simply switches between the 31 lines/mm and
the 79 lines/mm gratings, the selected instrument and camera are unchanged.
To change instruments, you have to EXIT from echwind using the menu, and
rerun it specifying the required configuration.

\subsection{Selecting the Detector: Detector, New DETECTOR, DETLIST, DETSYSTEM}

You are prompted for the detector name on startup. As there is no default
value, if no detector name is entered the detector will have zero size on
startup.

The only information pertaining to the detector that is actually needed is its
size and for unsupported detectors you can specify the size in mm in the form
``x,y'' (the quotes are needed if specifying it at the prompt) rather
than giving the name.

The prompt for the detector name includes the list of supported detectors
(provided that it is not too long). Your system manager can set up a
system-wide list in the {\tt \$ECHWIND\_HOME} directory.
The default file for the system-wide detector list is {\tt detectors.dat}.
In addition you can specify your own private list on
startup by answering the DETLIST prompt. 

The list is read from a text file with a default file extension of {\tt .dat}.
The file format is very simple: each line
consists of the detector name in single quotes (like a Fortran CHARACTER
constant) followed by the detector X and Y sizes in mm, separated by a
comma. For example, this is the default system-wide detector list for UES:
\begin{verbatim}
      'TEK'           24.5,24.5
      'EEV6'          27.9,25.9
      'EEV9'          46.1,23.0
      'EEV-THIN'      55.3,27.6
      'LORAL'         30.7,30.7
\end{verbatim}
If a detector name appears both in the system-wide list and in your
user-supplied list then your definition of it is used rather than the
system-wide one.

You can change the detector from within the program by selecting the ``New
DETECTOR'' menu option. You will then be prompted for a new detector name;
you can either give the the name of a known detector or enter the x and y
sizes in mm, separated by a comma.

\subsection{Selecting a Camera Rotation Angle: ANGWAVE, New ANGLE}

It is common observing practice  to align the detector with the orders.
This has the dual advantages that it makes it easy to extract orders
without having to track them, and that you may get an extra complete order
to fit onto the detector. 

You can select a new camera rotation angle from within the program by selecting
the ``New ANGLE'' menu option. This prompts you for a new rotation
angle or corresponding wavelength. If you specify the wavelength (in
Angstroms) at which you plan to align the orders with the detector, the
display of the orders will be rotated accordingly and the corresponding
slit angle is reported.

Alternatively, if you specify a value that is less than or equal to 1000,
it is assumed that you are providing a value in degrees for the slit angle
offset. {\em Note that there is currently a problem here and the reported
slit angle offsets do not correspond to actual behaviour. Consequently, you
should specify only wavelengths at the ANGWAVE prompt or at the ``New
ANGLE'' menu option.}


\subsection{Selecting a Central Wavelength: WAVE, New WAVELENGTH}

By default the detector will initially be placed at the origin of the
dispersion and cross dispersion axes (about 4020\AA\, for the UES 31
echelle, about 4120\AA\, for the UCLES 31 echelle). 

You can select a new central wavelength from within the program by selecting
the ``New WAVELENGTH'' menu option. This prompts you for a new central
wavelength.

\subsection{Selecting a Line List: LINES, Mark LINES}

Often you will be interested in specific spectral lines or wavelength
ranges. You can prepare a text file containing a list of wavelengths and
text strings with which to label those wavelengths and can enter the
filename at the LINES prompt. The lines will be plotted in the correct
positions on the echellogram, in white where they appear within the free
spectral range and in grey where they do not. Each line has the specified
text string plotted near to it. By default this file has an extension of
{\tt .dat}.

The file format is very simple: each line
consists of the wavelength in Angstroms followed by the associated text in
single quotes (like a Fortran CHARACTER constant), as in
\begin{verbatim}
      6562.80 'A'
      4861.32 'B'
      4340.46 'C'
      4101.73 'D'
\end{verbatim}

You can select a new line list from within the program by selecting the ``Mark 
LINES'' menu option, which prompts you for a new line list. These lines will
be plotted, and any existing lines from previous operations will remain
plotted. The only way to delete plotted lines is to restart echwind.

The line list file will be looked for in the local directory or in the explicit 
directory given. Alternatively, you may prefix the file name with
\verb+${ECHWIND_HOME}+. The release provides two lines list files in the 
\verb+${ECHWIND_HOME}+ directory. They are {\tt balmer.dat} and 
{\tt lines.dat} which are specified by 
\verb+${ECHWIND_HOME}balmer.dat+
and \verb+${ECHWIND_HOME}lines.dat+ respectively. 

The contents of \verb+${ECHWIND_HOME}balmer.dat+ are as follows:

\begin{verbatim}
      6562.80 'A'
      4861.32 'B'
      4340.46 'C'
      4101.73 'D'
      3970.07 'E'
      3889.05 'F'
      3835.38 'G'
      3797.90 'H'
      3770.63 'I'
      3750.15 'J'
\end{verbatim}

The contents of \verb+${ECHWIND_HOME}lines.dat+ are as follows:

\begin{verbatim}
      4861.3198 'Hb'
      4340.4678 'Hg'
      5889.9512 'D2'
      5895.9243 'D1'
      3302.368  'U2'
      3302.978  'U1'
      4226.728  'CaI'
      3933.663  'CaII K'
      3968.468  'CaII H'
\end{verbatim}

\subsection{Marking Detector Positions: Mark BOX, Delete BOX, Clear BOXES}

You can select the ``Mark BOX'' menu option to mark the current position of the
detector in green (``SAVE'' does an implicit ``Mark BOX'' as well). Selecting
``Delete BOX'' will delete the most recently marked box. Selecting ``Clear
BOXES'' will delete all boxes.

\subsection{Selecting a Zoom Factor}

By default the echellogram will initially be displayed with a zoom factor of 1.

You can zoom and unzoom from within the program. Select one of the `$
\times  1.3$', `$ \times 2.0$' and `$ \times 3.0$' menu options to zoom in;
select one of the `$ \times 0.77$', `$ \times 0.5$' and `$ \times 0.33$'
menu options to zoom out. The detector is always placed at the centre of
the display after zooming. Thus the ``CENTRE'' menu option is simply a pan
operation (it does what `$ \times 1.0$' would do).

\subsection{Saving Configurations: SAVE}

ECHWIND supports the saving of spectrograph configurations in files for
later reference. To save the current spectrograph configuration (or
detector position, however you like to think of it), select the ``SAVE''
menu option. This will mark the current position of the detector in green
(like ``Mark BOX'') and will issue an appropriate prompt for the file name. 
Text files (which are ASCII files with a default file extension of
.TXT) are created only for the first SAVE operation and are appended to for
subsequent operations.

The text file is of the form
\begin{verbatim}
 ECHWIND V3.0
  
   Instrument: UES                 
   Echelle: 31                  
   Camera: LONG                
   No. of prisms:            3
   Detector length:    40.67613    
   Maximum wavelength:    4545.881      in order no:          126
   Minimum wavelength:    3634.631      in order no:          156
   Central Wavelength:    4013.619      in order no:          142
        Theta        Gamma    prism (mm)   slit angle
       1.490        0.007       -0.40        0.050
\end{verbatim}

\subsection{Doing Nothing}

If you have selected the menu but then decide that you made a mistake in so
doing, you can move the cursor to below the ``EXIT'' menu option and then
press any mouse button to return without taking any action.

\subsection{Exiting: EXIT}

To exit from the program, select the ``EXIT'' menu option.

\subsection{Acknowledgements}

C.J. Hirst of UCL wrote the original UCLES VAX version of ECHWIND in 1988.
Stephen Meatheringham of Mt Stromlo ported the original version to Unix.
Marion Fisher added support for UES and tidied up the graphics in 1996.
Various others assisted with testing.

\end{document}
