\documentclass[twoside,11pt]{article}

% ? Specify used packages
% \usepackage{graphicx}        %  Use this one for final production.
% \usepackage[draft]{graphicx} %  Use this one for drafting.
% ? End of specify used packages

\pagestyle{myheadings}

% -----------------------------------------------------------------------------
% ? Document identification
% Fixed part
\newcommand{\stardoccategory}  {Starlink System Note}
\newcommand{\stardocinitials}  {SSN}
\newcommand{\stardocsource}    {ssn\stardocnumber}

% Variable part - replace [xxx] as appropriate.
\newcommand{\stardocnumber}    {73.1}
\newcommand{\stardocauthors}   {Mark Taylor}
\newcommand{\stardocdate}      {17 January 2000}
\newcommand{\stardoctitle}     {EXTREME --- 
                                Utilities for handling extreme data sets}
\newcommand{\stardocabstract}  {
This package provides some utilities and background documentation 
for adapting the USSC to handle very large data sets.
The principal focus of this is to move to use of 64 bits of address
space on 64-bit operating systems.
}
% ? End of document identification
% -----------------------------------------------------------------------------

% +
%  Name:
%     ssn.tex
%
%  Purpose:
%     Template for Starlink System Note (SSN) documents.
%     Refer to SUN/199
%
%  Authors:
%     AJC: A.J.Chipperfield (Starlink, RAL)
%     BLY: M.J.Bly (Starlink, RAL)
%     PWD: Peter W. Draper (Starlink, Durham University)
%     MBT: Mark B. Taylor (Starlink)
%
%  History:
%     17-JAN-1996 (AJC):
%        Original with hypertext macros, based on MDL plain originals.
%     16-JUN-1997 (BLY):
%        Adapted for LaTeX2e.
%     13-AUG-1998 (PWD):
%        Converted for use with LaTeX2HTML version 98.2 and
%        Star2HTML version 1.3.
%     17-JAN-1999 (MBT):
%        Instantiated as SSN/73.
%     {Add further history here}
%
% -

\newcommand{\stardocname}{\stardocinitials /\stardocnumber}
\markboth{\stardocname}{\stardocname}
\setlength{\textwidth}{160mm}
\setlength{\textheight}{230mm}
\setlength{\topmargin}{-2mm}
\setlength{\oddsidemargin}{0mm}
\setlength{\evensidemargin}{0mm}
\setlength{\parindent}{0mm}
\setlength{\parskip}{\medskipamount}
\setlength{\unitlength}{1mm}

% -----------------------------------------------------------------------------
%  Hypertext definitions.
%  ======================
%  These are used by the LaTeX2HTML translator in conjunction with star2html.

%  Comment.sty: version 2.0, 19 June 1992
%  Selectively in/exclude pieces of text.
%
%  Author
%    Victor Eijkhout                                      <eijkhout@cs.utk.edu>
%    Department of Computer Science
%    University Tennessee at Knoxville
%    104 Ayres Hall
%    Knoxville, TN 37996
%    USA

%  Do not remove the %begin{latexonly} and %end{latexonly} lines (used by 
%  LaTeX2HTML to signify text it shouldn't process).
%begin{latexonly}
\makeatletter
\def\makeinnocent#1{\catcode`#1=12 }
\def\csarg#1#2{\expandafter#1\csname#2\endcsname}

\def\ThrowAwayComment#1{\begingroup
    \def\CurrentComment{#1}%
    \let\do\makeinnocent \dospecials
    \makeinnocent\^^L% and whatever other special cases
    \endlinechar`\^^M \catcode`\^^M=12 \xComment}
{\catcode`\^^M=12 \endlinechar=-1 %
 \gdef\xComment#1^^M{\def\test{#1}
      \csarg\ifx{PlainEnd\CurrentComment Test}\test
          \let\html@next\endgroup
      \else \csarg\ifx{LaLaEnd\CurrentComment Test}\test
            \edef\html@next{\endgroup\noexpand\end{\CurrentComment}}
      \else \let\html@next\xComment
      \fi \fi \html@next}
}
\makeatother

\def\includecomment
 #1{\expandafter\def\csname#1\endcsname{}%
    \expandafter\def\csname end#1\endcsname{}}
\def\excludecomment
 #1{\expandafter\def\csname#1\endcsname{\ThrowAwayComment{#1}}%
    {\escapechar=-1\relax
     \csarg\xdef{PlainEnd#1Test}{\string\\end#1}%
     \csarg\xdef{LaLaEnd#1Test}{\string\\end\string\{#1\string\}}%
    }}

%  Define environments that ignore their contents.
\excludecomment{comment}
\excludecomment{rawhtml}
\excludecomment{htmlonly}

%  Hypertext commands etc. This is a condensed version of the html.sty
%  file supplied with LaTeX2HTML by: Nikos Drakos <nikos@cbl.leeds.ac.uk> &
%  Jelle van Zeijl <jvzeijl@isou17.estec.esa.nl>. The LaTeX2HTML documentation
%  should be consulted about all commands (and the environments defined above)
%  except \xref and \xlabel which are Starlink specific.

\newcommand{\htmladdnormallinkfoot}[2]{#1\footnote{#2}}
\newcommand{\htmladdnormallink}[2]{#1}
\newcommand{\htmladdimg}[1]{}
\newcommand{\hyperref}[4]{#2\ref{#4}#3}
\newcommand{\htmlref}[2]{#1}
\newcommand{\htmlimage}[1]{}
\newcommand{\htmladdtonavigation}[1]{}

\newenvironment{latexonly}{}{}
\newcommand{\latex}[1]{#1}
\newcommand{\html}[1]{}
\newcommand{\latexhtml}[2]{#1}
\newcommand{\HTMLcode}[2][]{}

%  Starlink cross-references and labels.
\newcommand{\xref}[3]{#1}
\newcommand{\xlabel}[1]{}

%  LaTeX2HTML symbol.
\newcommand{\latextohtml}{\LaTeX2\texttt{HTML}}

%  Define command to re-centre underscore for Latex and leave as normal
%  for HTML (severe problems with \_ in tabbing environments and \_\_
%  generally otherwise).
\renewcommand{\_}{\texttt{\symbol{95}}}

% -----------------------------------------------------------------------------
%  Debugging.
%  =========
%  Remove % on the following to debug links in the HTML version using Latex.

% \newcommand{\hotlink}[2]{\fbox{\begin{tabular}[t]{@{}c@{}}#1\\\hline{\footnotesize #2}\end{tabular}}}
% \renewcommand{\htmladdnormallinkfoot}[2]{\hotlink{#1}{#2}}
% \renewcommand{\htmladdnormallink}[2]{\hotlink{#1}{#2}}
% \renewcommand{\hyperref}[4]{\hotlink{#1}{\S\ref{#4}}}
% \renewcommand{\htmlref}[2]{\hotlink{#1}{\S\ref{#2}}}
% \renewcommand{\xref}[3]{\hotlink{#1}{#2 -- #3}}
%end{latexonly}
% -----------------------------------------------------------------------------
% ? Document specific \newcommand or \newenvironment commands.
% ? End of document specific commands
% -----------------------------------------------------------------------------
%  Title Page.
%  ===========
\renewcommand{\thepage}{\roman{page}}
\begin{document}
\thispagestyle{empty}

%  Latex document header.
%  ======================
\begin{latexonly}
   CCLRC / \textsc{Rutherford Appleton Laboratory} \hfill \textbf{\stardocname}\\
   {\large Particle Physics \& Astronomy Research Council}\\
   {\large Starlink Project\\}
   {\large \stardoccategory\ \stardocnumber}
   \begin{flushright}
   \stardocauthors\\
   \stardocdate
   \end{flushright}
   \vspace{-4mm}
   \rule{\textwidth}{0.5mm}
   \vspace{5mm}
   \begin{center}
   {\Large\textbf{\stardoctitle}}
   \end{center}
   \vspace{5mm}

% ? Heading for abstract if used.
   \vspace{10mm}
   \begin{center}
      {\Large\textbf{Abstract}}
   \end{center}
% ? End of heading for abstract.
\end{latexonly}

%  HTML documentation header.
%  ==========================
\begin{htmlonly}
   \xlabel{}
   \begin{rawhtml} <H1> \end{rawhtml}
      \stardoctitle
   \begin{rawhtml} </H1> <HR> \end{rawhtml}

   \begin{rawhtml} <P> <I> \end{rawhtml}
   \stardoccategory\ \stardocnumber \\
   \stardocauthors \\
   \stardocdate
   \begin{rawhtml} </I> </P> <H3> \end{rawhtml}
      \htmladdnormallink{CCLRC / Rutherford Appleton Laboratory}
                        {http://www.cclrc.ac.uk} \\
      \htmladdnormallink{Particle Physics \& Astronomy Research Council}
                        {http://www.pparc.ac.uk} \\
   \begin{rawhtml} </H3> <H2> \end{rawhtml}
      \htmladdnormallink{Starlink Project}{http://www.starlink.rl.ac.uk/}
   \begin{rawhtml} </H2> \end{rawhtml}
   \htmladdnormallink{\htmladdimg{source.gif} Retrieve hardcopy}
      {http://www.starlink.rl.ac.uk/cgi-bin/hcserver?\stardocsource}\\

%  HTML document table of contents. 
%  ================================
%  Add table of contents header and a navigation button to return to this 
%  point in the document (this should always go before the abstract \section). 
  \label{stardoccontents}
  \begin{rawhtml} 
    <HR>
    <H2>Contents</H2>
  \end{rawhtml}
  \htmladdtonavigation{\htmlref{\htmladdimg{contents_motif.gif}}
        {stardoccontents}}

% ? New section for abstract if used.
  \section{\xlabel{abstract}Abstract}
% ? End of new section for abstract

\end{htmlonly}

% -----------------------------------------------------------------------------
% ? Document Abstract. (if used)
%  ==================
\stardocabstract
% ? End of document abstract
% -----------------------------------------------------------------------------
% ? Latex document Table of Contents (if used).
%  ===========================================
  \newpage
  \begin{latexonly}
    \setlength{\parskip}{0mm}
    \tableofcontents
    \setlength{\parskip}{\medskipamount}
    \markboth{\stardocname}{\stardocname}
  \end{latexonly}
% ? End of Latex document table of contents
% -----------------------------------------------------------------------------
\cleardoublepage
\renewcommand{\thepage}{\arabic{page}}
\setcounter{page}{1}

% ? Main text

\section{Introduction}

The extreme dataset project is intended to allow use of ``unusually''
large data sets, although the sizes for which special measures
are required will become less and less unusual as time goes on.
The principal underlying problem is that as images get larger
32 bits are no longer enough to index into an image.
The largest integer that can be stored in 32 bits 
is approximately $4 \times 10^9$ (unsigned) or $2 \times 10^9$ (signed).
If the operating system itself uses unsigned 32 bit pointers to 
address bytes
in memory, this means that it is impossible to map an image of more
than 4Gbyte or, say, two images of half that size. 
This could correspond to, for instance, an input and an output image
simultaneously mapped each with an HDS type of \_DOUBLE and 
dimensions of 16k $\times$ 16k pixels.  
For this sort of work therefore an operating system with 64-bit 
pointers is required.

For the systems supported by
Starlink this currently means that Compaq Tru64 Unix can be used,
as can Solaris running in 64-bit mode.  
On appropriate hardware the Solaris kernel 
may be compiled for 32 bit or 64 bit mode;
but almost\footnote{
   There can be trouble with applications which use 
   {\tt libkvm} or access {\tt /proc}.}
all binaries which run on the 32-bit version
will also run on the 64-bit version, 
so that reconfiguring a system from 32-bit to 64-bit should be
fairly painless.
You can tell if your Solaris kernel is 64-bit 
by using the {\tt isainfo -v} command; on a 64-bit system the
following response will be given
\begin{quote}
\begin{verbatim}
% isainfo -v
64-bit sparcv9 applications
32-bit sparc applications
\end{verbatim}
\end{quote}

User code will also run up against these problems.
It is often necessary to count the pixels, or the bytes,
in an image, and this is typically done using a Fortran {\tt INTEGER}
or a C {\tt int}.  These are normally signed 32-bit values, with
a maximum value of about $2 \times 10^9$; the pixel count of a 
47k $\times$ 47k image, or the byte count of a 16k $\times$ 16k 
\_DOUBLE image, will overflow this limit.

Another common requirement is holding a pointer, which has ultimately been 
acquired from a C routine such as {\tt malloc(3)}, 
in a variable.
In C this will be taken care of automatically because the compiler
will ensure that a pointer type is long enough to hold memory addresses.
In Fortran however there is no pointer type an INTEGERs, which are
normally 32 bits, have been used.
The solution to this, explained in \xref{SUN/209}{sun209}{pointers}
is to use the \xref{CNF\_PVAL}{sun209}{CNF\_PVAL} function.

This package provides some tools and instructions for 
package maintainers to use 
in modifying their source code to work in a 64-bit environment.

\section{Changeover to 64-bit systems}

All of the changes discussed in this document
are intended to be backwardly compatible, 
so that packages thus modified should, eventually, 
continue to work unchanged in 32-bit environments.
Because of the difficulties of handling the changeover of all 
Starlink packages simultaneously however, conversion of a package
to work properly in 64 bits should be considered effectively as
a port to a new platform.  
When the entire USSC is properly in 64 bits it will be possible
to discontinue support of the 32-bit platforms as a separate SYSTEM
type in the {\tt mk} file, but packages built for the the new 
64-bit platforms should run properly on 32-bit systems.
Two new values of the {\tt mk}/{\tt makefile} environment variable 
SYSTEM will therefore be introduced: 
``sun4\_Solaris\_64'' and ``alpha\_OSF1\_64''.


\section{Tools}

This section documents utilities distributed with this package
to modify source code for use in 64-bit environments.

\subsection{{\tt inscnf} -- Insertion of CNF\_PVAL}

As described in \xref{SUN/209}{sun209}{pointers} the argument of
any invocation of the \%VAL directive in Fortran code 
should be wrapped in a call to 
the \xref{CNF\_PVAL}{sun209}{CNF\_PVAL} function.

The program {\tt inscnf} is provided to add such calls automatically
to Fortran source code.  It is invoked as a filter, so may take zero,
one or two arguments, using standard input and standard output for 
unsupplied filenames as appropriate.  It makes an attempt to abide by
conventions of case usage and spacing in the source code, and to
break lines where necessary at reasonable places.  For files,
and for source lines, which does not contain any invocations of \%VAL, 
the output will be identical to the input.  Hence a pipeline like
\begin{quote}
\begin{verbatim}
% inscnf foo.f | diff foo.f - 
\end{verbatim}
\end{quote}
will show up any differences the utility has made.

The utility should be pretty robust, but it may be possible to
confuse it with some illegal/deprecated things like 
`\verb+\r+' or `\verb+\t+' 
characters, using columns 73--80 for comments, or splitting the string
`\%VAL' between source lines.  If it believes
that it might have got into serious trouble (that is, that it may
have corrupted the meaning of the source code), then {\tt inscnf}
will write a warning on standard error, or terminate with error
status, or both.

A shell script, {\tt do\_inscnf}, gives an example of applying the filter
to all the Fortran files in a directory.  This can be used as is,
or taken as an example.  It defines the location of the input and
output files, writes a log file of changes made and writes messages 
to standard error for files which may have generated problems.
Only files which have been modified are written to the output directory.
It also performs some crude checks of its own on the input and output
files to see if it looks like erroneous changes have been made:
\begin{itemize}
\item 
It checks to see if an input file apparently contains any invocations
of {\tt \%VAL} and {\tt CNF\_PVAL}, and guesses whether {\tt inscnf} 
should have made any changes.
\item
Where {\tt inscnf} has made changes it performs a crude comparison of
the input and output files to see if they look the same apart from
insertion of {\tt CNF\_PVAL} invocations.
\end{itemize}
If either of these checks produces a surprising result it writes a 
message to standard error; this does not mean that {\tt inscnf} has
made an error, but it is probably worth checking this by hand.







% ? End of main text
\end{document}

% $Id$
