\documentstyle[11pt]{article}
\pagestyle{myheadings}

% -----------------------------------------------------------------------------
% ? Document identification
%------------------------------------------------------------------------------
\newcommand{\stardoccategory}  {}
\newcommand{\stardocinitials}  {}
\newcommand{\stardocsource}    {}
\newcommand{\stardocnumber}    {}
\newcommand{\stardocauthors}   {fred}
\newcommand{\stardocdate}      {now}
\newcommand{\stardoctitle}     {POLMAP \\ [1ex]
                                Display of Polarisation Maps}
\newcommand{\stardocversion}   {Version 1.0}
\newcommand{\stardocmanual}    {On-line Manual}
% ? End of document identification
% -----------------------------------------------------------------------------

\newcommand{\stardocname}{\stardocinitials /\stardocnumber}
\markright{\stardocname}
\setlength{\textwidth}{160mm}
\setlength{\textheight}{230mm}
\setlength{\topmargin}{-2mm}
\setlength{\oddsidemargin}{0mm}
\setlength{\evensidemargin}{0mm}
\setlength{\parindent}{0mm}
\setlength{\parskip}{\medskipamount}
\setlength{\unitlength}{1mm}

% -----------------------------------------------------------------------------
%  Hypertext definitions.
%  ======================
%  These are used by the LaTeX2HTML translator in conjunction with star2html.

%  Comment.sty: version 2.0, 19 June 1992
%  Selectively in/exclude pieces of text.
%
%  Author
%    Victor Eijkhout                                      <eijkhout@cs.utk.edu>
%    Department of Computer Science
%    University Tennessee at Knoxville
%    104 Ayres Hall
%    Knoxville, TN 37996
%    USA

%  Do not remove the %\begin{rawtex} and %\end{rawtex} lines (used by
%  star2html to signify raw TeX that latex2html cannot process).
%\begin{rawtex}
\makeatletter
\def\makeinnocent#1{\catcode`#1=12 }
\def\csarg#1#2{\expandafter#1\csname#2\endcsname}

\def\ThrowAwayComment#1{\begingroup
    \def\CurrentComment{#1}%
    \let\do\makeinnocent \dospecials
    \makeinnocent\^^L% and whatever other special cases
    \endlinechar`\^^M \catcode`\^^M=12 \xComment}
{\catcode`\^^M=12 \endlinechar=-1 %
 \gdef\xComment#1^^M{\def\test{#1}
      \csarg\ifx{PlainEnd\CurrentComment Test}\test
          \let\html@next\endgroup
      \else \csarg\ifx{LaLaEnd\CurrentComment Test}\test
            \edef\html@next{\endgroup\noexpand\end{\CurrentComment}}
      \else \let\html@next\xComment
      \fi \fi \html@next}
}
\makeatother

\def\includecomment
 #1{\expandafter\def\csname#1\endcsname{}%
    \expandafter\def\csname end#1\endcsname{}}
\def\excludecomment
 #1{\expandafter\def\csname#1\endcsname{\ThrowAwayComment{#1}}%
    {\escapechar=-1\relax
     \csarg\xdef{PlainEnd#1Test}{\string\\end#1}%
     \csarg\xdef{LaLaEnd#1Test}{\string\\end\string\{#1\string\}}%
    }}

%  Define environments that ignore their contents.
\excludecomment{comment}
\excludecomment{rawhtml}
\excludecomment{htmlonly}
%\end{rawtex}

%  Hypertext commands etc. This is a condensed version of the html.sty
%  file supplied with LaTeX2HTML by: Nikos Drakos <nikos@cbl.leeds.ac.uk> &
%  Jelle van Zeijl <jvzeijl@isou17.estec.esa.nl>. The LaTeX2HTML documentation
%  should be consulted about all commands (and the environments defined above)
%  except \xref and \xlabel which are Starlink specific.

\newcommand{\htmladdnormallinkfoot}[2]{#1\footnote{#2}}
\newcommand{\htmladdnormallink}[2]{#1}
\newcommand{\htmladdimg}[1]{}
\newenvironment{latexonly}{}{}
\newcommand{\hyperref}[4]{#2\ref{#4}#3}
\newcommand{\htmlref}[2]{#1}
\newcommand{\htmlimage}[1]{}
\newcommand{\htmladdtonavigation}[1]{}

%  Starlink cross-references and labels.
\newcommand{\xref}[3]{#1}
\newcommand{\xlabel}[1]{}

%  LaTeX2HTML symbol.
\newcommand{\latextohtml}{{\bf LaTeX}{2}{\tt{HTML}}}

%  Define command to re-centre underscore for Latex and leave as normal
%  for HTML (severe problems with \_ in tabbing environments and \_\_
%  generally otherwise).
\newcommand{\latex}[1]{#1}
\newcommand{\setunderscore}{\renewcommand{\_}{{\tt\symbol{95}}}}
\latex{\setunderscore}

%  Redefine the \tableofcontents command. This procrastination is necessary
%  to stop the automatic creation of a second table of contents page
%  by latex2html.
\newcommand{\latexonlytoc}[0]{\tableofcontents}

% -----------------------------------------------------------------------------
%  Debugging.
%  =========
%  Remove % on the following to debug links in the HTML version using Latex.

% \newcommand{\hotlink}[2]{\fbox{\begin{tabular}[t]{@{}c@{}}#1\\\hline{\footnotesize #2}\end{tabular}}}
% \renewcommand{\htmladdnormallinkfoot}[2]{\hotlink{#1}{#2}}
% \renewcommand{\htmladdnormallink}[2]{\hotlink{#1}{#2}}
% \renewcommand{\hyperref}[4]{\hotlink{#1}{\S\ref{#4}}}
% \renewcommand{\htmlref}[2]{\hotlink{#1}{\S\ref{#2}}}
% \renewcommand{\xref}[3]{\hotlink{#1}{#2 -- #3}}
% -----------------------------------------------------------------------------
% ? Document specific \newcommand or \newenvironment commands.
\newcommand{\mylabel}[1] {\xlabel{#1}\label{#1}}

% ? End of document specific commands
% -----------------------------------------------------------------------------
%  Title Page.
%  ===========
\renewcommand{\thepage}{\roman{page}}
\begin{document}
\thispagestyle{empty}

%  Latex document header.
%  ======================
\begin{latexonly}
   CCLRC / {\sc Rutherford Appleton Laboratory} \hfill {\bf \stardocname}\\
   {\large Particle Physics \& Astronomy Research Council}\\
   {\large Starlink Project\\}
   {\large \stardoccategory\ \stardocnumber}
   \begin{flushright}
   \stardocauthors\\
   \stardocdate
   \end{flushright}
   \vspace{-4mm}
   \rule{\textwidth}{0.5mm}
   \vspace{5mm}
   \begin{center}
   {\Huge\bf  \stardoctitle \\ [2.5ex]}
   {\LARGE\bf \stardocversion \\ [4ex]}
   {\Huge\bf  \stardocmanual}
   \end{center}
   \vspace{5mm}

% ? Heading for abstract if used.
   \vspace{10mm}
   \begin{center}
      {\Large\bf Abstract}
   \end{center}
% ? End of heading for abstract.
\end{latexonly}

%  HTML documentation header.
%  ==========================
\begin{htmlonly}
   \xlabel{}
   \begin{rawhtml} <H1> \end{rawhtml}
      \stardoctitle\\
      \stardocversion\\
      \stardocmanual
   \begin{rawhtml} </H1> \end{rawhtml}

% ? Add picture here if required.
% ? End of picture

   \begin{rawhtml} <P> <I> \end{rawhtml}
   \stardoccategory \stardocnumber \\
   \stardocauthors \\
   \stardocdate
   \begin{rawhtml} </I> </P> <H3> \end{rawhtml}
      \htmladdnormallink{CCLRC}{http://www.cclrc.ac.uk} /
      \htmladdnormallink{Rutherford Appleton Laboratory}
                        {http://www.cclrc.ac.uk/ral} \\
      \htmladdnormallink{Particle Physics \& Astronomy Research Council}
                        {http://www.pparc.ac.uk} \\
   \begin{rawhtml} </H3> <H2> \end{rawhtml}
      \htmladdnormallink{Starlink Project}{http://star-www.rl.ac.uk/}
   \begin{rawhtml} </H2> \end{rawhtml}
   \htmladdnormallink{\htmladdimg{source.gif} Retrieve hardcopy}
      {http://star-www.rl.ac.uk/cgi-bin/hcserver?\stardocsource}\\

%  HTML document table of contents.
%  ================================
%  Add table of contents header and a navigation button to return to this
%  point in the document (this should always go before the abstract \section).
  \label{stardoccontents}
  \begin{rawhtml}
    <HR>
    <H2>Contents</H2>
  \end{rawhtml}
  \renewcommand{\latexonlytoc}[0]{}
  \htmladdtonavigation{\htmlref{\htmladdimg{contents_motif.gif}}
        {stardoccontents}}

% ? New section for abstract if used.
  \section{\xlabel{abstract}Abstract}


% ? End of new section for abstract
\end{htmlonly}

% -----------------------------------------------------------------------------
% ? Document Abstract. (if used)
%   ==================
% ? End of document
% -----------------------------------------------------------------------------
% ? Latex document Table of Contents (if used).
%  ===========================================
 \newpage
 \begin{latexonly}
   \setlength{\parskip}{0mm}
   \latexonlytoc
   \setlength{\parskip}{\medskipamount}
   \markright{\stardocname}
 \end{latexonly}
% ? End of Latex document table of contents
% -----------------------------------------------------------------------------
\newpage
\renewcommand{\thepage}{\arabic{page}}
\setcounter{page}{1}

\section {\mylabel{POLMAP_OVERVIEW}Introduction}
PolMap is an interactive tool for ...

\section {\mylabel{POLMAP_TUTORIAL}Tutorial}
This is a click-by-click introduction to the use of PolMap, which aims to
get you going with the minimum delay. It does not cover all possibilities,
and assumes you have a relatively simple problem to solve. 

\section {\mylabel{POLMAP_CONTROLS}A Tour of the Graphical User Interface}
This section gives detailed descriptions of each control in the PolMap 
Graphical User Interface. 

\subsection {\mylabel{POLMAP_FILE_MENU}The {\em File} Menu}
The {\em File} menu provides access to commands which control
termination of PolMap, and the creation of the output files. They are:

\subsubsection {\mylabel{POLMAP_SAVE}{\em Save}} 

\subsubsection {\mylabel{POLMAP_EXIT}{\em Exit}} 

\subsubsection {\mylabel{POLMAP_QUIT}{\em Quit}} 

\subsection {\mylabel{POLMAP_EDIT_MENU}The {\em Edit} Menu}
The {\em Edit} menu provides access to commands which relate to
the changing of information stored within PolMap. They are:

\subsection {\mylabel{POLMAP_OPTIONS_MENU}The {\em Options} Menu}
The {\em Options} menu provides access to commands which control the
various options which are available to customise the operation of PolMap.
These options may also be set on the command line used to activate 
{\tt polmap}. The menu items are:

\subsubsection {\mylabel{POLMAP_COLOURS}{\em Colours}} 
This allows control of the colours used to represent various objects in
the PolMap GUI. Upon selection, a menu of objects appear including:

\begin{description}

\item [{\em Cross-hair}] - This sets the colour of the cross-hair which
is used within the \htmlref{image display area}{POLMAP_IMAGE_DISPLAY} if
the \htmlref{{\em Use Cross-hair}}{POLMAP_USE_CROSS_HAIR} option is
selected. It defaults to yellow.

\item [{\em Missing Pixels}] -  This sets the colour used to represent
missing data in the \htmlref{image display area}{POLMAP_IMAGE_DISPLAY}.
It defaults to cyan.

\item [{\em Polygon Outlines}] - This sets the colour of any polygons drawn
over the image. It defaults to red.

\item [{\em Selection Box}] - This sets the colour of the box
which indicates an area of the image to be zoomed, or from which objects
are to be deleted. It defaults to red. \hyperref{Go here}{See
section }{}{POLMAP_AREA_SELECTION} for instructions on selecting areas.

\end{description}

When one of these items is selected, a menu of colours appears from which
a selection may be made. The available colours are red, blue, green,
cyan, magenta, yellow, black.

\subsubsection {\mylabel{POLMAP_STATUS_ITEMS}{\em Status Items}} This is used to
customise the contents of the \htmlref{status area}{POLMAP_STATUS_AREA}
shown below the image display area. Selecting this item will result in the 
\htmlref{{\em Select status items}}{POLMAP_STATUS_ITEMS_DIALOG} dialog
box appearing.

\subsubsection {\mylabel{POLMAP_USE_CROSS_HAIR}{\em Use Cross-hair}}
If this button is selected then a cross-hair will be used instead of a
pointer within the \htmlref{image display area}{POLMAP_IMAGE_DISPLAY}.
The default is to use a cross-hair.

\subsubsection {\mylabel{POLMAP_DISPLAY_HELP_AREA}{\em Display Help Area}} 
If this button is selected, then the PolMap GUI will include a
\htmlref{help area}{POLMAP_HELP_AREA} in which dynamic help information
on the control under the pointer is displayed. Otherwise, the help area
is not displayed.

\subsubsection {\mylabel{POLMAP_DISPLAY_STATUS_AREA}{\em Display Status Area}} 
If this button is selected, then the PolMap GUI will include a
\htmlref{status area}{POLMAP_HELP_AREA} in which information
describing the current state (such as image names, options values, 
pointer co-ordinates, etc) is displayed. Otherwise, the
status area is not displayed. The contents of this area may be customised
using the \htmlref{{\em Status Items}}{POLMAP_STATUS_ITEMS} entry in the
\htmlref{{\em Options}}{POLMAP_OPTIONS_MENU} menu.

\subsubsection {\mylabel{POLMAP_SAVE_OPTIONS}{\em Save Options}} 
If this menu item is selected, the current option values (together with
the values shown in the \htmlref{{\em \% not white}}{POLMAP_NOT_WHITE}
and \htmlref{{\em \% black}}{POLMAP_BLACK} controls) will be saved so that
they become the default values for future invocations of
PolMap.\footnote{The current values are passed back to the {\tt polmap} A-task 
upon completion, and they are then stored in the A-tasks parameter file.}

\subsection {\mylabel{POLMAP_HELP_MENU}The {\em Help} Menu}
This menu provides access to the PolMap hyper-text help information. 
\hyperref{Go here}{See section }{}{POLMAP_USING_HELP} for more information
on selecting which browser is to be used. This menu provides several preset
access points for the help information. The {\em Pointer} item allows the 
required help to be specified by pointing at a control on the GUI using the 
mouse, and then pressing the left mouse button. A similar effect can be
produced without using this option by positioning the pointer and then
pressing the {\tt F1} button on the keyboard.

\subsection {\mylabel{POLMAP_IMAGE_DISPLAY}The Image Display Area}
The image is displayed in a square area to the right of the
screen using a monochrome colour table. Various aspects of the display
such as the image scale, the data values corresponding to black and
white, the pixel coordinates at the centre of the display, etc, can be
adjusted using the controls in the \htmlref{{\em Display
Controls}}{POLMAP_DISPLAY_CONTROLS} box, to the left of the image
display. 

The user can interact with the image in various ways using the mouse. The
details of these interactions depend on the current state of the other
controls, and fall into one of several different ``modes''. A description of 
the current mode is displayed above the image, and the dynamic help
information shown in the \htmlref{{\em Help Area}}{POLMAP_HELP_AREA}
describes the interactions available in the current mode. The available
modes are:

\begin{description}

\item [\mylabel{POLMAP_MODE_0} ????] - This is the default
mode in which the user... The available interactions in
this mode are:

\begin{itemize}

\item Click and drag to select an area of the image. \hyperref{Go here}{See
section }{}{POLMAP_AREA_SELECTION} for instructions on selecting areas.

\end{itemize}

\item [\mylabel{POLMAP_MODE_1} Edit or create a polygon] - In this mode
the user either modifies an existing mask polygon or creates a new
polygon. The available interactions in this mode are:

\begin{itemize}

\item To move a vertex of an existing polygon, position the pointer over
the vertex, press and hold the left mouse button, and then move the
pointer. The vertex and connecting edges will be moved with the pointer
until the mouse button is released.

\item To add another vertex to an existing polygon, position the pointer 
over the edge where the new vertex is required, and click the left mouse
button. 

\item To commence the definition of a new polygon, position the pointer
away from any existing vertices or edges, and click the left mouse button.
The first vertex of the new polygon will be placed at the selected
position, and the display area will then enter ``\htmlref{Complete a
Polygon}{POLMAP_MODE_2}'' mode.

\item Click anywhere else and drag to select an area. \hyperref{Go here}{See
section }{}{POLMAP_AREA_SELECTION} for instructions on selecting areas.

\end{itemize}

\item [\mylabel{POLMAP_MODE_2} Complete a polygon] - In this mode, the
user completes the definition of a polygon started in ``\htmlref{Edit or
create a polygon}{POLMAP_MODE_1}'' mode. Adjacent vertices around the
polygon are supplied by positioning the pointer and clicking the left
mouse button. The polygon is closed when a vertex is given close to the
first vertex, and the display area then reverts to ``\htmlref{Edit or
create a polygon}{POLMAP_MODE_1}'' mode. The available interactions in 
this mode are:

\begin{itemize}
\item Click on the first vertex to close the polygon.
\item Click anywhere else to add another vertex to the polygon.
\item Click anywhere else and drag to select an area. \hyperref{Go here}{See
section }{}{POLMAP_AREA_SELECTION} for instructions on selecting areas.
\end{itemize}

\item [\mylabel{POLMAP_MODE_3} ??????????] - This mode is not yet used.

\item [\mylabel{POLMAP_MODE_4} Identify a new centre] - This mode is
entered when the \htmlref{{\em Centre}}{POLMAP_CENTRE} button is pressed.
The cursor changes to a circle to indicate the change of mode. The cursor
should then be positioned over the image, and the left mouse button
pressed. The image will then be re-displayed with the same scale factor,
but with the selected image position in the centre of the display window.

The available interactions in this mode are:
\begin{itemize}
\item Click to re-display the image centred on the pointer position.
\end{itemize}

\end{description}

\subsection {\mylabel{POLMAP_DISPLAY_CONTROLS}The ``{\em Display Controls}'' Box}
This box is situated to the left of the \htmlref{image display
area}{POLMAP_IMAGE_DISPLAY} and contains controls related to the image
display.

\subsubsection {\mylabel{POLMAP_ZOOM}The {\em Zoom} Button}
This button is enabled when an area of the image is selected by clicking
and dragging over the image. Pressing the {\em Zoom} button will then
result in the selected area of the image being re-displayed so that it
fills the display window in at least one dimension. The button will then
be disabled until another image area is selected. Several zoom operations
may be applied in sequence, and may be undone in reverse sequence by
repeated use of the \htmlref{{\em Unzoom}}{POLMAP_UNZOOM} button. 

\subsubsection {\mylabel{POLMAP_UNZOOM}The {\em Unzoom} Button}
This button is enabled when one or more zoom or centre operations have 
been applied to the image. Single clicking on this button will undo the
most recent zoom or centre operation, and double clicking will undo
all zoom and centre operations.

\subsubsection {\mylabel{POLMAP_CENTRE}The {\em Centre} Button}
This button is used to re-display the current image putting a specified
position at the centre of the display window. When the button is pressed 
the pointer or cross-hair will change to a circle within the image display 
area. Position this circular cursor at the position which is to be
located at the centre of the display window, and press the left mouse
button. The image will be re-displayed, and the circular cursor will
revert to its previous form.

\subsubsection {\mylabel{POLMAP_DELETE}The {\em Delete} Button}
This button is used to delete polygon vertices. First
select an area by clicking and dragging over the image, and then press
the {\em Delete} button. Any polygon vertices within the box will be deleted.

\subsubsection {\mylabel{POLMAP_CANCEL}The {\em Cancel} Button}
Pressing this button will cancel the most recent operation on the display
in the following order:

\begin{itemize}
\item If the \htmlref{{\em Centre}}{POLMAP_CENTRE} button has been
pressed, then the re-centring operation will be canceled.

\item Otherwise, if an area has been selected by clicking and dragging
over the image, then the area selection is canceled.

\item Otherwise, if the user is in the process of 
\htmlref{supplying a mask polygon}{POLMAP_MODE_2} then the incomplete
polygon is deleted, and the interaction mode within the display window
reverts to ``\htmlref{Edit or create a polygon}{POLMAP_MODE_1}''.
\end{itemize}

\subsubsection {\mylabel{POLMAP_NOT_WHITE}The ``{\em \% not white}'' Button}
This control specifies the percentage of the pixels within the displayed
part of the image which should be shown as black or grey. Thus, the
difference between $100.0$ and the displayed figure gives the percentage
of the image pixels which are shown as pure white. To enter a new value,
position the pointer over the entry box and type the new value.
Alternatively, the ``up'' and ``down'' arrows at either end of the data
entry box can be pressed, in which case the displayed value will
automatically be incremented or decremented until the arrow is released.
If the {\tt return} key on the keyboard is pressed while the pointer is
over the data entry box, then the image will be re-displayed immediately
with the new value. Otherwise, the image will be re-displayed 2.5 seconds
after the value has changed (unless a new value is entered for the ``{\em
\% not white}'' or ``\htmlref{{\em \% black}}{POLMAP_BLACK}'' control in
the mean-time). This delay allows a new value to be entered in the ``{\em
\% black}'' control before the image is re-displayed.

The current value of this control is saved along with the current options
values when the \htmlref{{\em Save Options}}{POLMAP_SAVE_OPTIONS} item in
the \htmlref{{\em Options}}{POLMAP_OPTIONS_MENU} menu is selected.

\subsubsection {\mylabel{POLMAP_BLACK}The ``{\em \% black}'' Button}
This control specifies the percentage of the pixels within the displayed
part of the image which should be shown as pure black. To enter a new value,
position the pointer over the entry box and type the new value.
Alternatively, the ``up'' and ``down'' arrows at either end of the data
entry box can be pressed, in which case the displayed value will
automatically be incremented or decremented until the arrow is released.
If the {\tt return} key on the keyboard is pressed while the pointer is
over the data entry box, then the image will be re-displayed immediately
with the new value. Otherwise, the image will be re-displayed 2.5 seconds
after the value has changed (unless a new value is entered for the ``{\em
\% not white}'' or ``\htmlref{{\em \% black}}{POLMAP_BLACK}'' control in
the mean-time). This delay allows a new value to be entered in the ``{\em
\% not white}'' control before the image is re-displayed.

The current value of this control is saved along with the current options
values when the \htmlref{{\em Save Options}}{POLMAP_SAVE_OPTIONS} item in
the \htmlref{{\em Options}}{POLMAP_OPTIONS_MENU} menu is selected.

\subsubsection {\mylabel{POLMAP_LOCK_SCALING}The {\em Lock Scaling} Button}
The pixel values shown as pure black and pure white in the displayed
image are specified indirectly as percentage points in the histogram of
displayed pixel values (see the ``\htmlref{{\em \% not
white}}{POLMAP_WHITE}'' and ``\htmlref{{\em \% black}}{POLMAP_BLACK}''
controls). This means that the grey scale will, in general, change when a
different section of the image is displayed (for instance, when using the
\htmlref{{\em Zoom}}{POLMAP_ZOOM} button). To prevent this happening, the
{\em Lock Scaling} button can be selected. The current data values
implied by the ``\htmlref{{\em \% not white}}{POLMAP_WHITE}'' and
``\htmlref{{\em \% black}}{POLMAP_BLACK}'' values will then be used
directly for all subsequent image displays until the {\em Lock Scaling}
button is de-selected. Any changes made to the ``\htmlref{{\em \% not
white}}{POLMAP_WHITE}'' and ``\htmlref{{\em \% black}}{POLMAP_BLACK}''
values will be ignored until the {\em Lock Scaling} button is
de-selected.

\subsection {\mylabel{POLMAP_STATUS_AREA}The Status Area}
This is the area underneath the displayed image and is used to display
various items of information about the current state of the program. The
exact contents of this area can be customised using the \htmlref{{\em
Status Items}}{POLMAP_STATUS_ITEMS} entry in the \htmlref{{\em
Options}}{POLMAP_OPTIONS_MENU} menu. It can also be removed entirely from
the GUI by de-selecting the \htmlref{{\em Display Status
Area}}{POLMAP_DISPLAY_STATUS_AREA} button in the \htmlref{{\em
Options}}{POLMAP_OPTIONS_MENU} menu.

\subsection {\mylabel{POLMAP_HELP_AREA}The Help Area}
This is an area at the bottom of the GUI in which is displayed brief help
information on the control currently under the pointer. It is updated
dynamically as the pointer is moved around the GUI. More detailed help
can be obtained using the \htmlref{{\em Help}}{POLMAP_HELP_MENU} menu.
This area can be removed entirely from the GUI by de-selecting the
\htmlref{{\em Display Help Area}}{POLMAP_DISPLAY_HELP_AREA} button in the
\htmlref{{\em Options}}{POLMAP_OPTIONS_MENU} menu.

\subsection {\mylabel{POLMAP_STATUS_ITEMS_DIALOG}The ``{\em Select status items}'' Dialog Box}
To be written.

\section {\mylabel{POLMAP_HOW_DO_I}How do I...}
This section provides instructions for performing some common tasks
within the PolMap Graphical User Interface.

\subsection {\mylabel{POLMAP_USING_HELP}Using the Help System}
Information on the use and state of the PolMap GUI are available on-line
in various ways:

\begin{itemize}

\item An indication of what is happening, or what the user should
probably be doing, is displayed above the displayed image. This is a very
broad indication, primarily intended to indicate what is going on during
any long pauses (for instance, while the output images are being
created).

\item The \htmlref{status area}{POLMAP_STATUS_AREA} situated below
the image display, displays various items of information describing the
current option values, pointer position, etc. The
contents of this area can be customised using the \htmlref{{\em Status
Items}}{POLMAP_STATUS_ITEMS} entry in the \htmlref{{\em
Options}}{POLMAP_OPTIONS_MENU} menu.

\item A brief description of the control, or controls, currently under the
pointer is displayed in the \htmlref{help area}{POLMAP_HELP_AREA} at the
bottom of the GUI (under the status area). This information is
dynamically updated as the pointer is moved around the screen.

\item More detailed hyper-text help information is also available, and a
separate hyper-text browser will automatically be created to display it
when required. Either {\tt netscape} or {\tt Mosaic} can be used by
assigning the name of the required browser to the {\tt HTX\_BROWSER}
environment variable. If {\tt HTX\_BROWSER} is not defined, then ``{\tt
netscape}'' is used.

This hyper-text help information may be accessed in several ways:

\begin{enumerate}

\item The \htmlref{{\em Help}}{POLMAP_HELP_MENU} menu, situated at the
far right of the menu bar can be used to access various preset points in
the help text.

\item A description of a specific control in the GUI can be obtained by 
positioning the pointer over the control, and pressing the {\tt F1}
button on the keyboard.

\item A description of a specific control may also be obtained by clicking 
on the {\em Pointer...} entry in the \htmlref{{\em
Help}}{POLMAP_HELP_MENU} menu, positioning the cursor over the control,
and clicking the left mouse button. 

\end{enumerate}

\end{itemize}

\subsection {\mylabel{POLMAP_IMAGE_SCALING}Selecting the Displayed Data Range}
To be written.

\subsection {\mylabel{POLMAP_AREA_SELECTION}Selecting an Area of the Image}
To be written.

\end{document}
