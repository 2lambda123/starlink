\documentclass[twoside,11pt]{article}

% ? Specify used packages
% \usepackage{graphicx}        %  Use this one for final production.
\usepackage[draft]{graphicx} %  Use this one for drafting.
% ? End of specify used packages

\pagestyle{myheadings}

% -----------------------------------------------------------------------------
% ? Document identification
% Fixed part
\newcommand{\stardoccategory}  {Starlink User Note}
\newcommand{\stardocinitials}  {SUN}
\newcommand{\stardocsource}    {sun\stardocnumber}

% Variable part - replace [xxx] as appropriate.
\newcommand{\stardocnumber}    {5.18}
\newcommand{\stardocauthors}   {P.\,T.\,Wallace, Norman Gray}
\newcommand{\stardocdate}      {2nd May 2003}
\newcommand{\stardoctitle}     {ASTROM \\[2ex] Basic astrometry program}
\newcommand{\stardocversion}   {v3.7}
\newcommand{\stardocmanual}    {User's Guide}
\newcommand{\stardocabstract}  {%
ASTROM performs ``plate reductions''.  You supply star positions from a
catalogue and the \xy\ ccoordinates of the corresponding star images.
ASTROM uses this information to establish the relationship between
\xy\ and \radec\ enabling the coordinates of ``unknown stars'' to be
determined.}

% ? End of document identification
% -----------------------------------------------------------------------------

% +
%  Name:
%     sun.tex
%
%  Purpose:
%     Template for Starlink User Note (SUN) documents.
%     Refer to SUN/199
%
%  Authors:
%     AJC: A.J.Chipperfield (Starlink, RAL)
%     BLY: M.J.Bly (Starlink, RAL)
%     PWD: Peter W. Draper (Starlink, Durham University)
%
%  History:
%     17-JAN-1996 (AJC):
%        Original with hypertext macros, based on MDL plain originals.
%     16-JUN-1997 (BLY):
%        Adapted for LaTeX2e.
%        Added picture commands.
%     13-AUG-1998 (PWD):
%        Converted for use with LaTeX2HTML version 98.2 and
%        Star2HTML version 1.3.
%     {Add further history here}
%
% -

\newcommand{\stardocname}{\stardocinitials /\stardocnumber}
\markboth{\stardocname}{\stardocname}
\setlength{\textwidth}{160mm}
\setlength{\textheight}{230mm}
\setlength{\topmargin}{-2mm}
\setlength{\oddsidemargin}{0mm}
\setlength{\evensidemargin}{0mm}
\setlength{\parindent}{0mm}
\setlength{\parskip}{\medskipamount}
\setlength{\unitlength}{1mm}

% -----------------------------------------------------------------------------
%  Hypertext definitions.
%  ======================
%  These are used by the LaTeX2HTML translator in conjunction with star2html.

%  Comment.sty: version 2.0, 19 June 1992
%  Selectively in/exclude pieces of text.
%
%  Author
%    Victor Eijkhout                                      <eijkhout@cs.utk.edu>
%    Department of Computer Science
%    University Tennessee at Knoxville
%    104 Ayres Hall
%    Knoxville, TN 37996
%    USA

%  Do not remove the %begin{latexonly} and %end{latexonly} lines (used by
%  LaTeX2HTML to signify text it shouldn't process).
%begin{latexonly}
\makeatletter
\def\makeinnocent#1{\catcode`#1=12 }
\def\csarg#1#2{\expandafter#1\csname#2\endcsname}

\def\ThrowAwayComment#1{\begingroup
    \def\CurrentComment{#1}%
    \let\do\makeinnocent \dospecials
    \makeinnocent\^^L% and whatever other special cases
    \endlinechar`\^^M \catcode`\^^M=12 \xComment}
{\catcode`\^^M=12 \endlinechar=-1 %
 \gdef\xComment#1^^M{\def\test{#1}
      \csarg\ifx{PlainEnd\CurrentComment Test}\test
          \let\html@next\endgroup
      \else \csarg\ifx{LaLaEnd\CurrentComment Test}\test
            \edef\html@next{\endgroup\noexpand\end{\CurrentComment}}
      \else \let\html@next\xComment
      \fi \fi \html@next}
}
\makeatother

\def\includecomment
 #1{\expandafter\def\csname#1\endcsname{}%
    \expandafter\def\csname end#1\endcsname{}}
\def\excludecomment
 #1{\expandafter\def\csname#1\endcsname{\ThrowAwayComment{#1}}%
    {\escapechar=-1\relax
     \csarg\xdef{PlainEnd#1Test}{\string\\end#1}%
     \csarg\xdef{LaLaEnd#1Test}{\string\\end\string\{#1\string\}}%
    }}

%  Define environments that ignore their contents.
\excludecomment{comment}
\excludecomment{rawhtml}
\excludecomment{htmlonly}

%  Hypertext commands etc. This is a condensed version of the html.sty
%  file supplied with LaTeX2HTML by: Nikos Drakos <nikos@cbl.leeds.ac.uk> &
%  Jelle van Zeijl <jvzeijl@isou17.estec.esa.nl>. The LaTeX2HTML documentation
%  should be consulted about all commands (and the environments defined above)
%  except \xref and \xlabel which are Starlink specific.

\newcommand{\htmladdnormallinkfoot}[2]{#1\footnote{#2}}
\newcommand{\htmladdnormallink}[2]{#1}
\newcommand{\htmladdimg}[1]{}
\newcommand{\hyperref}[4]{#2\ref{#4}#3}
\newcommand{\htmlref}[2]{#1}
\newcommand{\htmlimage}[1]{}
\newcommand{\htmladdtonavigation}[1]{}

\newenvironment{latexonly}{}{}
\newcommand{\latex}[1]{#1}
\newcommand{\html}[1]{}
\newcommand{\latexhtml}[2]{#1}
\newcommand{\HTMLcode}[2][]{}

%  Starlink cross-references and labels.
\newcommand{\xref}[3]{#1}
\newcommand{\xlabel}[1]{}

%  LaTeX2HTML symbol.
\newcommand{\latextohtml}{\LaTeX2\texttt{HTML}}

%  Define command to re-centre underscore for Latex and leave as normal
%  for HTML (severe problems with \_ in tabbing environments and \_\_
%  generally otherwise).
\renewcommand{\_}{\texttt{\symbol{95}}}

% -----------------------------------------------------------------------------
%  Debugging.
%  =========
%  Remove % on the following to debug links in the HTML version using Latex.

% \newcommand{\hotlink}[2]{\fbox{\begin{tabular}[t]{@{}c@{}}#1\\\hline{\footnotesize #2}\end{tabular}}}
% \renewcommand{\htmladdnormallinkfoot}[2]{\hotlink{#1}{#2}}
% \renewcommand{\htmladdnormallink}[2]{\hotlink{#1}{#2}}
% \renewcommand{\hyperref}[4]{\hotlink{#1}{\S\ref{#4}}}
% \renewcommand{\htmlref}[2]{\hotlink{#1}{\S\ref{#2}}}
% \renewcommand{\xref}[3]{\hotlink{#1}{#2 -- #3}}
%end{latexonly}
% -----------------------------------------------------------------------------
% ? Document specific \newcommand or \newenvironment commands.
\newcommand{\radec}     {$[\alpha,\delta\,]$}
\newcommand{\xy}        {$[x,y\,]$}
\newcommand{\xieta}     {$[\xi,\eta\,]$}
% ? End of document specific commands
% -----------------------------------------------------------------------------
%  Title Page.
%  ===========
\renewcommand{\thepage}{\roman{page}}
\begin{document}
\thispagestyle{empty}

%  Latex document header.
%  ======================
\begin{latexonly}
   CCLRC / {\textsc Rutherford Appleton Laboratory} \hfill {\textbf \stardocname}\\
   {\large Particle Physics \& Astronomy Research Council}\\
   {\large Starlink Project\\}
   {\large \stardoccategory\ \stardocnumber}
   \begin{flushright}
   \stardocauthors\\
   \stardocdate
   \end{flushright}
   \vspace{-4mm}
   \rule{\textwidth}{0.5mm}
   \vspace{5mm}
   \begin{center}
   {\Huge\textbf  \stardoctitle \\ [2.5ex]}
   {\LARGE\textbf \stardocversion \\ [4ex]}
   {\Huge\textbf  \stardocmanual}
   \end{center}
   \vspace{5mm}

% ? Add picture here if required for the LaTeX version.
%   e.g. \includegraphics[scale=0.3]{filename.ps}
% ? End of picture

% ? Heading for abstract if used.
   \vspace{10mm}
   \begin{center}
      {\Large\textbf Abstract}
   \end{center}
% ? End of heading for abstract.
\end{latexonly}

%  HTML documentation header.
%  ==========================
\begin{htmlonly}
   \xlabel{}
   \begin{rawhtml} <H1 ALIGN=CENTER> \end{rawhtml}
      \stardoctitle\\
      \stardocversion\\
      \stardocmanual
   \begin{rawhtml} </H1> <HR> \end{rawhtml}

% ? Add picture here if required for the hypertext version.
%   e.g. \includegraphics[scale=0.7]{filename.ps}
% ? End of picture

   \begin{rawhtml} <P> <I> \end{rawhtml}
   \stardoccategory\ \stardocnumber \\
   \stardocauthors \\
   \stardocdate
   \begin{rawhtml} </I> </P> <H3> \end{rawhtml}
      \htmladdnormallink{CCLRC}{http://www.cclrc.ac.uk} /
      \htmladdnormallink{Rutherford Appleton Laboratory}
                        {http://www.cclrc.ac.uk/ral} \\
      \htmladdnormallink{Particle Physics \& Astronomy Research Council}
                        {http://www.pparc.ac.uk} \\
   \begin{rawhtml} </H3> <H2> \end{rawhtml}
      \htmladdnormallink{Starlink Project}{http://www.starlink.ac.uk/}
   \begin{rawhtml} </H2> \end{rawhtml}
   \htmladdnormallink{\htmladdimg{source.gif} Retrieve hardcopy}
      {http://www.starlink.ac.uk/cgi-bin/hcserver?\stardocsource}\\

%  HTML document table of contents.
%  ================================
%  Add table of contents header and a navigation button to return to this
%  point in the document (this should always go before the abstract \section).
  \label{stardoccontents}
  \begin{rawhtml}
    <HR>
    <H2>Contents</H2>
  \end{rawhtml}
  \htmladdtonavigation{\htmlref{\htmladdimg{contents_motif.gif}}
        {stardoccontents}}

% ? New section for abstract if used.
  \section{\xlabel{abstract}Abstract}
% ? End of new section for abstract
\end{htmlonly}

% -----------------------------------------------------------------------------
% ? Document Abstract. (if used)
%  ==================
\stardocabstract
% ? End of document abstract
% -----------------------------------------------------------------------------
% ? Latex document Table of Contents (if used).
%  ===========================================
  \newpage
  \begin{latexonly}
    \setlength{\parskip}{0mm}
    \tableofcontents
    \setlength{\parskip}{\medskipamount}
    \markboth{\stardocname}{\stardocname}
  \end{latexonly}
% ? End of Latex document table of contents
% -----------------------------------------------------------------------------
\cleardoublepage
\renewcommand{\thepage}{\arabic{page}}
\setcounter{page}{1}

% ? Main text

\section{\xlabel{introduction}Introduction}
\label{introduction}

ASTROM is a simple plate\footnote{For ``plate'' read CCD exposure
\textit{etc}.} reduction utility, designed to allow the non-specialist
user to get good results with a minimum of trouble and esoteric
knowledge.  The user supplies a text file containing information about
the exposure and the positions of reference and unknown stars;  ASTROM
performs the various coordinate transformation and fitting operations
required, displays a synopsis report on the command terminal, and
prepares a detailed report for later printing.

\section{\xlabel{operating_instructions}Operating Instructions}
\label{operating_instructions}

ASTROM is run by means of the following command
% (C-shell on Unix, DCL on VAX/VMS, MS-DOS on PC):
\begin{quote}
\texttt{astrom} [\texttt{input=}\textit{input}]
    [\texttt{report=}\textit{report}]
    [\texttt{summary=}\textit{summary}] \\\relax
    [\texttt{log=}\textit{log}]
    [\texttt{fits=}\textit{fits}]
    [\texttt{wcsstyle=}\textit{wcsstyle}]
\end{quote}

All the parameters are optional, and may appear in any order.  The
parameters \textit{input} and \textit{report} default to
\texttt{astrom.dat} and \texttt{astrom.lis}; the \textit{summary}
defaults to the terminal.  If \textit{log} or \textit{fits} is not
specified, no corresponding file is generated.  A parameter can be
defaulted either by leaving it out or by using a pair of
double-quotes.  The specifiers \texttt{input=} and \texttt{report=}
are optional; if they are omitted, they are assigned to the first and
second unqualified filenames.  Thus
\begin{quote}
\texttt{astrom astrom.input}
\end{quote}
would read the input file \texttt{astrom.input}, and write a report on
the default \texttt{astrom.lis}.
\begin{quote}
\texttt{astrom report=astrom.report fits=wcsout input=astrom.input}
\end{quote}
reads \texttt{astrom.input}, and produces its report on
\texttt{astrom.report}, plus a number of FITS files with names
beginning \texttt{wcsout} (see Section~\ref{output} below).  The
specifiers \texttt{input=-} and \texttt{report=-} cause the input to
be read from, and the report to be written to, the standard input and
output respectively.

A summary report normally appears on the command terminal, but can be
sent to a file by giving the file's name as the \texttt{summary=}
parameter.  This should be monitored and any reference stars with
large residuals noted.  The input file can then be edited as necessary
and the job rerun.  Finally, the report file can be printed in the
normal way.  An example report file is reproduced in
Appendix~\ref{example_report}.  For further details on the log and
FITS files, see Section~\ref{output}.

\section{\xlabel{the_input_file}The Input File}
\label{the_input_file}

The input file is an ordinary text file and is terminated either
by end-of-file or by a record beginning \texttt{E}.  Uppercase and
lowercase are both acceptable throughout the file and may be mixed freely;
leading spaces are ignored.  A comment, beginning \texttt{*}, can be
appended to any record.  Completely blank records (and any beginning
with \texttt{*}) are ignored and can be used to improve layout and
provide commentary.

Most of the records consist of (or contain) various numbers of numeric
fields, separated by spaces (or commas).  In many instances it is
simply the number of fields present which enables ASTROM to determine
which sort of record has been read.  Free-format number decoding
is used throughout; spaces can be freely inserted between fields,
and many other freedoms are permitted (see the documentation for the
\xref{SLALIB}{sun67}{} routines \xref{SLA\_DFLTIN}{sun67}{SLA_DFLTIN}
and \xref{SLA\_DBJIN}{sun67}{SLA_DBJIN} in SUN/67).

Each file typically specifies a single plate reduction;  however,
multiple \textit{sequences} of records, each specifying a complete
and separate plate reduction, can be used, each sequence being separated
from the next by a record beginning \texttt{/}.

% This feature is used by the CHART utility
% (see SUN/32), which is a convenient way of generating
% ASTROM input without having to copy out star catalogue
% entries by hand.

\goodbreak

The overall layout of each sequence is as follows:

\begin{center}
\begin{tabular}{|l|c|l|}
\hline
\textit{Group} & \textit{Records} & \textit{Mandatory?} \\
\hline
results equinox  &   1           & no \\
telescope type   &   1           & no \\
plate data       &   1           & yes \\
observation data &   1-3         & no \\
reference stars  &  2-3 per star & at least 2 stars \\
unknown stars    &  1-2 per star & no \\
\hline
\end{tabular}
\end{center}

Several sorts of record involve celestial positions.  Although ASTROM
can be made to work internally in \textit{observed} coordinates
(\emph{i.e.}\ as affected by refraction \emph{etc.}), all input data
and reports are in terms of various sorts of \textit{mean} \radec.
In any particular instance, the mean coordinate system is specified
by quoting an \textit{equinox}.  An \textit{epoch} is also required,
for the calculation of proper motion;  in the case of catalogue stars
this is frequently the same as the equinox.

Both the old pre~IAU~1976 (loosely FK4) system and the new post~IAU~1976
(loosely FK5) system are supported, and data in the two systems can
be mixed freely.  ASTROM follows the established convention of using
the equinox to distinguish between the two systems.  If the equinox
is prefixed by \texttt{B} (which stands for \textit{Besselian})
then the position is an old FK4 one;  if a prefix of \texttt{J} is
used (standing for \textit{Julian}), the position is a new FK5 one.
If no prefix is used, pre 1984.0 equinoxes indicate the old FK4
system, and equinoxes of 1984.0 or later indicate the new FK5 system.
The \texttt{B} or \texttt{J} prefix may also be used with epochs
although the distinction is unlikely to be significant.  The two most
common equinoxes are B1950 and J2000.  All FK4 positions include E-terms
of aberration, consistent with published catalogues.

When using the old system (\emph{e.g.}\ B1950) to specify the position
of an object whose proper motion is presumed to be negligible, it is
necessary to specify an epoch (as well as the equinox).  This is because
the old system is not an inertial frame; failure to recognize galactic
rotation at the time the system was first established means that the FK4
frame is rotating, and that even extragalactic objects have fictitious
proper motions which need to be taken into account in precise work.
ASTROM accepts an optional format of reference star data for such cases,
where the proper motions are omitted and an epoch is mandatory.

For most ASTROM applications, star positions in the frame of the
Hipparcos catalogue can be assumed to be equivalent to FK5 J2000.
However, it is unwise to mix Hipparcos and pre-Hipparcos positions.  If
only Hipparcos-compatible stars are used, and if for ASTROM purposes
they are regarded as FK5 J2000, the results of the ASTROM fit will be
in the Hipparcos frame to a high degree of accuracy.

Appendix~\ref{appendix_the_input_file} contains a detailed specification
of the syntax of the ASTROM input file, together with a comprehensive
example.  As an introduction, we will look at a simple but typical
example of such a file:

\begin{quote}
\begin{small}
\begin{tabular}{|l|}
\hline
\\
\verb|B1950                                     * Results in FK4| \\
\verb|SCHM                                      * Schmidt geometry| \\
\verb|19 04 00.0  -65 00 00  B1950.0  1974.5    * Plate centre, and epoch| \\
\verb|18 56 39.426  -63 25 13.23  -0.0002  -0.036  B1950.0  * Ref 1| \\
\verb|44.791   85.643| \\
\verb|19 11 53.909  -63 17 57.57   0.0058  -0.044  1950.0   * Ref 2| \\
\verb|-46.266   92.337| \\
\verb|19 01 13.606  -63 49 14.84   0.0020  -0.026  1950.0   * Ref 3| \\
\verb|17.246   64.945| \\
\verb|19 08 29.088  -63 57 42.79   0.0016   0.018  1950.0   * Ref 4| \\
\verb|-25.314   57.456| \\
\verb|19 02 10.088  -63 29 16.73   0.0012  -0.019  1950.0   * Ref 5| \\
\verb|11.890   82.766| \\
\verb|-5.103    58.868                      *  Candidate| \\
\verb|19 09 46.2  -63 51 27  J2000.0        *  Radio pos| \\
\verb|END| \\
\\
\hline
\end{tabular}
\end{small}
\end{quote}

% \goodbreak
Taking each record (or group of records) in turn:

\begin{quote}
\begin{tabular}{|l|}
\hline
\verb|B1950                                     * Results in FK4| \\
\hline
\end{tabular}
\end{quote}

This is the \textbf{results equinox} record.  It specifies the mean
equator and equinox for the coordinate system of the report.  If this
record is omitted, the results will be in J2000.

\goodbreak
\begin{quote}
\begin{tabular}{|l|}
\hline
\verb|SCHM                                      * Schmidt geometry| \\
\hline
\end{tabular}
\end{quote}

This is the \textbf{telescope type} record, which describes the projection
geometry.  The telescope type is given by the first four characters of
the record;  there are currently six options.  \texttt{SCHM}, as used
here, is for Schmidt telescopes.  \texttt{ASTR} selects the tangent
plane or \textit{gnomonic} projection, produced by conventional
astrographic telescopes (and by pinhole cameras).  Then there are
three special AAT options: \texttt{AAT2} and \texttt{AAT3} for the
Prime Focus doublet and triplet correctors, and \texttt{AAT8} for the
\textit{f}/8 Ritchey-Chr\'etien focus (using the vacuum plateholder).
The option \texttt{JKT8} models the field distortion of the JKT
(\textit{f}/8 Harmer-Wynne focus).  Finally, the option \texttt{GENE}
specifies generalized pincushion/barrel distortion, the magnitude of
which is given by a single numeric parameter $q$ following the telescope
type string; further details are given in Section~\ref{method}.

\goodbreak
\begin{quote}
\begin{tabular}{|l|}
\hline
\verb|19 04 00.0  -65 00 00  B1950.0  1974.5    * Plate centre, and epoch|
\\
\hline
\end{tabular}
\end{quote}

This is the mandatory \textbf{plate data} record, specifying the
point on the sky corresponding to the pole of the projection
geometry (which is usually, but not necessarily\footnote{JKT
\textit{f}/8 plates, for example, are mounted eccentrically.
The plate data record must specify the celestial coordinates
of the optical axis, preferably to within a millimetre or so.
Section~\ref{fitting_plate_centre_and_radial_distortion} gives details
of how the plate centre (and the radial distortion) can be determined
automatically.}, at the geometrical centre of the plate) and the
date on which the exposure occurred.  The \radec\ is expressed as
h~m~s~$^\circ$~$'$~$''$.  The hours, minutes, degrees and arcminutes
fields must all be integers.  The sign of the $\delta$ precedes the
degrees.  The \radec\ must be followed by an equinox.  The epoch specifies
when the picture was taken, needed for the proper motion calculation.
The epoch can be omitted if more precise information is to be supplied
later via the optional observation data records.

\goodbreak
\begin{quote}
\begin{tabular}{|l|}
\hline
\verb|18 56 39.426  -63 25 13.23  -0.0002  -0.036  B1950.0  * Ref 1| \\
\verb|44.791   85.643| \\
\hline
\end{tabular}
\end{quote}

This is the first of several record pairs describing the \textbf{reference
stars}.  At least two such pairs are needed in order to run ASTROM,
and three if both sorts of linear fit are to be done.  At least 10 stars
are required if fitting of the radial distortion and/or plate centre is
to be attempted.  A typical number for an ordinary linear fit would be
about a dozen stars; a thorough job covering a large area of a plate
and with automatic determination of the radial distortion and plate
centre selected would require perhaps 50.  The current limit is 2000.
The first record contains \radec, proper motions in seconds per year and
arcseconds per year respectively (\textit{n.b.}\ not centuries as in some
catalogues), and equinox, followed optionally by epoch and/or parallax.
If the epoch is omitted (as in the above example), it is assumed to be the
same as the equinox.  Parallax, which is in arcseconds, defaults to zero.
For reference stars whose positions are given in the old FK4 system,
and whose proper motions are presumed to be zero in an inertial sense,
an alternative format is available, with the proper motions omitted and
the epoch mandatory:

\begin{quote}
\begin{tabular}{|l|}
\hline
\verb|18 56 39.422  -63 25 14.00  B1950.0  1971.3           * Ref 1| \\
\hline
\end{tabular}
\end{quote}

(Because the FK4 system is not inertial, using the format described
earlier and simply putting zero for the proper motions would \textbf{not}
give the same effect.)  The first 10 characters of any comment are
picked up and appear on the reports as `name'.  The second record of the
reference star pair is the measured \xy.  For the 4-coefficient model to
work properly, $x$ and $y$ must be in the same units.  The reports will
look best if the units are millimetres or thereabouts and the offsets from
zero are reasonably small.  Orientation and handedness are immaterial;
x=east and y=north is the recommended convention as it matches the run
of $\alpha$ and $\delta$.

\goodbreak
\begin{quote}
\begin{tabular}{|l|}
\hline
\verb|-5.103    58.868                      *  Candidate| \\
\verb|19 09 46.2  -63 51 27  J2000.0        *  Radio pos| \\
\hline
\end{tabular}
\end{quote}

These two records both specify \textit{unknown stars}.  The first is \xy,
from which \radec\ will be determined.  The second is \radec\ and equinox,
from which \xy\ will be determined.  It is not necessary to include any
unknown star records at all, if the intention is simply to determine a
plate scale or to check a set of plate measurements.

The above example gives a position for the first unknown star of
\texttt{19 05 01.794 -63 56 16.70}.  The equinox was specified in the
results equinox record, and the epoch in the plate data record.  One way
to express this information in a publication might be as follows:

\begin{center}
19~05~01.79~~-63~56~16.7~~B1950~~epoch~1974.5
\end{center}

The example does not include the optional \textbf{observation data}
and \textbf{colour} records, which enable ASTROM to reconstruct the
precise appearance of the field rather than allowing various predictable
rotations and distortions to be absorbed into the fit.  Also omitted from
the example are requests to include the radial distortion and plate centre
in the fit.  Details of these refinements are given in
Sections~\ref{reduction_in_observed_place}
and~\ref{fitting_plate_centre_and_radial_distortion}.

\section{\xlabel{output}Output}
\label{output}

The report file (named by the \texttt{report=} parameter, and defaulting to
\texttt{astrom.lis}) is a human-readable report of ASTROM's actions
and its results.  The summary file, which normally appears on the
terminal, contains observations and warnings about ASTROM's progress,
and should be monitored.  Neither of these files, however, is intended
to be easily machine-parseable, and so if ASTROM is used within a
larger system, the other output files will be useful.  The log file is
a report on ASTROM's actions and results which is easy to parse, and
easy to extract information from.  The WCS files contain ASTROM's
results in the form of a FITS header, containing FITS WCS keywords.
Both these files are described in the subsections below.

\subsection{\xlabel{output_log}Log output}
\label{output_log}

The log file is a report on ASTROM's actions and results which is easy
to parse, and thus easy for another program, acting as a harness for
ASTROM, to find out about ASTROM's progress.

The log file consists of a number of statements describing the fits
ASTROM has performed, and its success.  The statements describing each
fit are bracketed in a pair of statements `\texttt{FIT }$\langle
n\rangle$' and `\texttt{ENDFIT}', for some fit number~$\langle n\rangle$.
The statements thus bracketed are in table~\ref{t:logstatements}.
\begin{table}
\begin{center}
\begin{tabular}{|l|l|}
\hline
Result & \texttt{RESULT} \textit{keyword value} \\
Status & \texttt{STATUS} $[\texttt{OK} | \texttt{BAD}]$ \\
Residuals & \texttt{RESIDUAL} \textit{source-number dx dy dr} \\
Information & \texttt{INFO} \textit{code details\dots} \\
Warnings & \texttt{WARNING} \textit{code comment} \\
Errors & \texttt{ERROR} \textit{code comment} \\
\hline
\end{tabular}
\end{center}
\caption{\label{t:logstatements}Log file statements}
\end{table}

The \texttt{INFO}, \texttt{WARNING} and \texttt{ERROR} statements
share a common set of codes, and are described in
Appendix~\ref{error_and_warning_messages}.

The \texttt{RESIDUAL} message reports the size of the residuals for
the given source number in the input file.  The residuals are reported
in the~$x$ and~$y$ directions, plus $\delta r=\sqrt{\delta x^2+\delta
y^2}$.

The \texttt{STATUS} message is \texttt{OK} if the fit succeeded, and
\texttt{BAD} otherwise.

The \texttt{RESULT} statements give feedback about the results of the
fit.  Much of the information is also in the FITS WCS headers, if they
are generated.  The keywords are listed in table~\ref{t:result-table}.
\begin{table}
\begin{center}
\begin{tabular}{|l|l|}
\hline
\textbf{keyword} & \textbf{meaning} \\
\hline
\texttt{nstars} & number of ref stars \\
\texttt{xrms} & RMS errors in fitted $x$, arcsec \\
\texttt{yrms} & \dots in $y$ \\
\texttt{rrms} & \dots in $r$ \\
\texttt{plate} & (mean) plate scale, arcsec \\
\texttt{prms} & \texttt{rrms} in pixels \\
\texttt{nterms} & number of terms in fit \\
\texttt{rarad} & projection pole RA, radians \\
\texttt{decrad} & projection pole Dec, radians \\
\texttt{rasex} & projection pole RA, sexagesimal \\
\texttt{decsex} & projection pole Dec, sexagesimal \\
\texttt{wcs} & name of FITS WCS header file \\
\hline
\end{tabular}
\end{center}
\caption{\label{t:result-table}Information returned in log-file
\texttt{RESULT} statement}
\end{table}

\subsection{\xlabel{output_wcs}FITS WCS output}
\label{output_wcs}

If requested (by the presence of the \texttt{fits=} specifier on the
command line), ASTROM will write out the plate solution in a series of
FITS files, containing headers conforming (largely) to the FITS WCS
standards (known as `Paper~I' and
`Paper~II')~\cite{fitswcs1,fitswcs2}.  ASTROM will generally attempt
more than one fit.  The file names will start with the string given in
the \texttt{fits=} parameter.

There is more than one way to encode the required WCS information, and
which way is used depends on the value of the parameter
\texttt{wcsstyle=} on the command line.  The allowed values of this
are \texttt{qtan} and \texttt{xtan}, and these are discussed now.

There is a standard for specifying world coordinate systems in FITS
files~\cite{fitswcs2}, and ASTROM conforms to this.  At present (May
2003) there is only an early \emph{draft} standard for representing
distortions, \emph{Representation of distortions in FITS world
coordinate systems}, Calabretta et~al.\ (also known as `Paper~IV'),
available at Mark Calabretta's web pages~\cite{fitswcsurl}.  Part of
ASTROM's function is to determine and report such distortions, but
since there is not yet any standardised way to do this, we have
something of a problem.

This program does not attempt to produce output using the distortion
model described in Paper~IV; that seems premature.  Instead, it
describes distortions using the model described as `distorted
gnomonic' (\texttt{TAN}) in the late \emph{draft} versions of
Paper~II.  If you specify \texttt{wcsstyle=xtan} (not recommended),
then ASTROM emits FITS headers which fully conform to these drafts; if
you specify \texttt{wcsstyle=qtan}, the FITS headers are essentially
the same, but with the draft paper's \verb|PVj_m| headers
replaced by non-standard \verb|QVj_m|.  In this latter
case, the headers are conformant with the final Paper~II, but the
distortion information is available to software which knows how to use
it.  You are strongly advised \emph{not} to produce new FITS files
using option
\texttt{wcsstyle=xtan}, unless you are obliged to by old versions of
software.

The `distorted gnomonic' \texttt{TAN} projection is not documented
here (deliberately), and the drafts describing it are no longer
readily available on the web.  However, should you need to, you would
probably be able to find a copy through Mark Calabretta's WCS web
pages~\cite{fitswcsurl}.

This is an interim solution.  It's anyone's guess how long it will
take for the FITS community to agree on a final version of Paper~IV.
Once that is finalised, however, it's quite possible that ASTROM's
support for the above header styles will be removed.

\section{\xlabel{method}Method}
\label{method}

For each input sequence, up to three astrometric solutions are
reported. The first is a four coefficient linear model (zero points,
scale and orientation), requiring at least two reference stars.
The second, computed in addition to the 4-coefficient model if there
are at least three reference stars, is a six coefficient linear model
(zero points, scales in $x$ and $y$, orientation and nonperpendicularity).
The third solution, which is performed on request and providing at least
10 reference stars have been supplied, has 7-9 coefficients and includes
in the model the radial distortion coefficient and/or the plate centre,
along with the six linear terms.

The 4-coefficient model is useful (1)~for rough and ready astrometry,
\textit{e.g.}\ from a print using a ruler or graph paper, and (2)~for
identifying an erroneous reference star, the higher order fits tending
to disguise the error.  On most occasions, the 6-coefficient solution
will be the most useful.

Internally, the modelling is done in idealized ``plate coordinates'',
and the various \radec\ and \xy\ data input to or output from ASTROM are
converted to and from this internal standard as required.  The conversion
from \radec\ to plate coordinates consists of the following steps:

\begin{enumerate}

\item Appropriate operations to transform the supplied \radec\ into either
observed coordinates (if the optional observation data have been provided)
or mean coordinates at the plate epoch (if not).

\item Conventional gnomonic projection, using the given plate centre
\radec, to obtain tangential coordinates \xieta.

\item A small adjustment to allow for departures from tangent-plane
geometry.

\end{enumerate}

The distortion model in step 3 is the usual ``cubic'' one, where the
vector from the plate centre to the star image is lengthened by an amount
proportional to the cube of the length of this vector.  The adjustment is
carried out by multiplying each of $\xi$ and $\eta$ by the factor $(1 +
q (\xi^{2}+\eta^{2}))$, the coefficient $q$ depending on the telescope
type specified.  The values for each telescope type are given in the
following table:

\goodbreak
\begin{center}
\begin{tabular}{|c|l|c|}
\hline
\textit{telescope type} & \textit{description} & $q$ \\
\hline
\texttt{ASTR} & astrograph & zero \\
\texttt{SCHM} & Schmidt & $-1/3$ \\
\texttt{AAT2} & AAT PF doublet & +147.1 \\
\texttt{AAT3} & AAT PF triplet & +178.6 \\
\texttt{AAT8} & AAT $f/8$ & +21.2 \\
\texttt{JKT8} & JKT $f/8$ & +14.7 \\
\texttt{GENE} & general & specified \\
\hline
\end{tabular}
\end{center}

\goodbreak
Notes:
\begin{itemize}

\item Positive $q$ values correspond to pincushion distortion,
   negative to barrel distortion.

\item In the case of telescope type \texttt{GENE}
   (generalized pincushion/barrel distortion), $q$ is specified directly
   as a numeric parameter, and therefore can be used for any telescope
   or camera which is adequately described by the distortion model.

\item The difference between the Schmidt and tangent-plane
   projections is conventionally assumed to be that between $r$ and $\tan
   r$; ASTROM's $q = -1/3$ is equivalent to making the approximation
   $\tan r \simeq r + r^3/3$.

\item The coefficient $q$ can, optionally, be determined automatically.

\end{itemize}

For the 4- and 6-coefficient linear models, the fitting process
consists of finding a set of coefficients which transform
the measured reference star \xy\ data into plate coordinates
which approximate those calculated from the \radec\ data.
For the 7-9 coefficient solutions, revised estimates of
the plate centre \radec\ and/or radial distortion
coefficient $q$ are made as well.

The models relate the following three types of coordinate:

\begin{itemize}

\item The \textit{estimated} coordinates $[x_{e},y_{e}]$, derived
   from the reference star \radec\ by gnomonic projection and the
   application of radial distortion, using the current estimates of the
   plate centre and radial distortion coefficient.

\item The \textit{measured} coordinates $[x_{m},y_{m}]$, as supplied.

\item The \textit{predicted} coordinates $[x_{p},y_{p}]$, derived by the
   application of the current linear model to the measured coordinates.

\end{itemize}

Two varieties of \textbf{4-coefficient linear model} are tried, one
the mirror-image of the other.  The \textit{standard} model is:

\begin{eqnarray*}
x_{e} & \simeq & a_{1} + a_{2} x_{m} + a_{3} y_{m} \\
y_{e} & \simeq & b_{1} - a_{3} x_{m} + a_{2} y_{m}
\end{eqnarray*}

The \textit{laterally inverted} model is:

\begin{eqnarray*}
x_{e} & \simeq & a_{1} + a_{2} x_{m} + a_{3} y_{m} \\
y_{e} & \simeq & b_{1} + a_{3} x_{m} - a_{2} y_{m}
\end{eqnarray*}

The one delivering the smallest RMS error is selected.  If only
two reference stars have been supplied, the standard model is
used.

The \textbf{6-coefficient linear model} is as follows:

\begin{eqnarray*}
x_{e} & \simeq & a_{1} + a_{2} x_{m} + a_{3} y_{m} \\
y_{e} & \simeq & b_{1} + b_{2} x_{m} + b_{3} y_{m}
\end{eqnarray*}

Instead of the coefficients $a_{n},b_{n}$ being found directly, the fits
are, in fact, implemented in terms of corrections $\Delta a_{n},\Delta
b_{n}$ to assumed approximate values of $a_{n},b_{n}$.  For example,
the 6-coefficient model is fitted as:

\begin{eqnarray*}
x_{e} - x_{p} & \simeq & \Delta a_{1}
          + \Delta a_{2} x_{m} + \Delta a_{3} y_{m} \\
y_{e} - y_{p} &\simeq & \Delta b_{1}
          + \Delta b_{2} x_{m} + \Delta b_{3} y_{m}
\end{eqnarray*}

When determining the \textbf{plate centre}, the following extra non-linear
terms are added to the basic 6-coefficient linear model:

\begin{eqnarray*}
x_{e} - x_{p} & \simeq & \cdots + p_{1} (x_{p}^{2} + q (3 x_{p}^{2} + y_{p}^{2}))
                    + p_{2} (x_{p}y_{p} + q (2 x_{p}y_{p})) \\
y_{e} - y_{p} & \simeq & \cdots + p_{1} (x_{p}y_{p} + q (2 x_{p}y_{p}))
                    + p_{2} (y_{p}^{2} + q (x_{p}^{2} + 3 y_{p}^{2}))
\end{eqnarray*}

The coefficients $p_{1}$ and $p_{2}$ estimate the offset between the
pole of projection and the current $[x_{p},y_{p}]$ origin.  This offset
is used to improve the plate centre \radec\ (and to correct the zero
point $[a_{1},b_{1}]$) prior to recomputing $[x_{p},y_{p}]$ for each
reference star.

When determining the \textbf{radial distortion coefficient}, the
following extra terms are added:

\begin{eqnarray*}
x_{e} - x_{p} & \simeq & \cdots - \Delta q (x_{p}^{2} + y_{p}^{2}) x_{p} \\
y_{e} - y_{p} & \simeq & \cdots - \Delta q (x_{p}^{2} + y_{p}^{2}) y_{p}
\end{eqnarray*}

The $\Delta q$ obtained from the fit is added to the current $q$ to
provide a better estimate.

The above expressions are similar to those derived by Murray in sections
8.3.1\textit{ff} of \textit{Vectorial Astrometry} (Adam~Hilger,
1983).  The main difference is that in ASTROM the centres of the gnomonic
projection and cubic distortion are assumed to be coincident.

All three types of solution are found by the iterative application
of a least-squares algorithm based on \textit{singular value
decomposition} of the \textit{design matrix}.  (See sections 2.9 and
14.3 of \textit{Numerical Recipes}, Press \textit{et al.}, Cambridge
University Press, 1986.)  This algorithm gives identical results to
the traditional \textit{normal equations} approach, but copes better
with the ill-conditioned character of the 7-9 coefficient model.
The fit minimizes $\Sigma ((x_{e}-x_{p})^{2}+(y_{e}-y_{p})^{2})$.
Each reference star thus produces two rows of design matrix -- one for $x$
and one for $y$.  Internally, the measured coordinates $[x_{m},y_{m}]$
are scaled to unit RMS to reduce the risk of numerical problems during
the fitting process.

In the case of the 4- and 6-coefficient linear models, a single iteration
is, in principle, all that is needed, whatever the starting values for
the coefficients.  However, a second iteration is performed in order to
minimize rounding errors.

The 7-9 coefficient models are highly nonlinear, with adjustments of
plate centre and -- especially -- radial distortion producing large
changes in the scales and zero points which depend on the distribution
of reference stars.  To ensure convergence, given reasonable starting
values for the plate centre and radial distortion coefficient, the
following strategy is used:

\begin{itemize}

\item ASTROM insists that at least 10 reference stars be available
   if a non-linear fit is to be attempted.

\item A fixed and ample number of iterations is used -- currently 20.

\item Iterations which fit the plate centre and/or the distortion
   alternate with ones containing the six linear terms alone.  The final
   iteration is the linear model.

\item Where fitting of both the plate centre and the distortion has been
   requested, for the first few iterations only one or other of these
   is included in the fit, in alternation.  Once reliable estimates of
   each have been obtained the full model is fitted at once.

\end{itemize}

\section{\xlabel{limitation}Limitations}
\label{limitations}

ASTROM aims to deliver results better than 1~arcsec from typical Schmidt
plate measurements, and better than 0.1~arcsec from carefully measured
JKT and AAT plates \emph{etc}.  Astrometric specialists will, nonetheless,
be aware of a number of shortcomings, including the following:

\begin{itemize}

   \item The fit is limited to a 6-coefficient linear model plus cubic
      distortion and plate tilt.  Colour effects -- arising for example
      from chromatic aberrations in the camera optics -- are not allowed
      for, no magnitude or image shape terms are included in the model,
      and the refraction cannot be adjusted automatically.

   \item The zonal distortions of the reference catalogues are neglected.

   \item There is no provision for the simultaneous fitting of more
      than one plate.  This prevents an extended area being modelled
      via overlapping plates, and the determination of proper motion
      and parallax from plates taken at different epochs.

   \item Only rudimentary error information is produced.

\end{itemize}

Despite these limitations, which stem mainly from the need for simplicity
of use, the accuracy of the result tends in practice to be dominated by
the quality of the input data rather than by ASTROM itself.

\section{\xlabel{reduction_in_observed_place}Reduction in Observed Place}
\label{reduction_in_observed_place}

Normally, the ASTROM reduction is carried out (internally) using
\textit{mean} places for the epoch of the plate -- positions
corrected for precession, but not for nutation, aberration, deflection,
and refraction, the effects of which are simply absorbed into the fit.
This approach keeps the input file simple, and delivers perfectly adequate
results for most practical purposes.  However, there are some occasions
on which a more precise reduction may be worthwhile.

Although the nutation, aberration and deflection are always relatively
innocuous -- the nutation produces a small and harmless rotation, the
aberration varies very slowly across the sky, and the deflection is
tiny except close to the Sun -- the effects of atmospheric refraction
can be quite important.  As far as ASTROM is concerned, the refraction
has two aspects:

\begin{itemize}

   \item \textit{Differential refraction} causes the picture to be
      distorted.  The distortion is in the form of a non-linear scale
      reduction in the vertical direction, the reduction being larger
      near the bottom;  this cannot be fully corrected by ASTROM's
      linear model.

   \item \textit{Atmospheric dispersion}, important for detectors of wide
      spectral coverage, causes the images of stars of different colour
      to appear shifted vertically from their nominal positions.

\end{itemize}

Both these effects can be eliminated if the optional time, observatory,
meteorological and colour records are included in the input file.
The advantages of bothering to do this are as follows:

\begin{itemize}

   \item A more accurate result.  The improvement is likely to be modest
      in most instances, but may be significant for low elevations and
      wide fields.

   \item More nearly equal scales will be reported in x and y, and will be
      constant all over the sky.  Apart from providing additional
      reassurance that the fit is good, accurate knowledge of the scales
      is clearly vital if the optical parameters of the telescope are
      being measured for later use in predictions for guide stars or
      fibre feeds \textit{etc}.

   \item The effects of colour may be important and can at least be
      quantified.

\end{itemize}

The optional records all begin with an explicit identifier, of
which only the first character (T, O, M or C) is significant.
They must immediately follow the plate data record, but
can be in any order.   Here is an example:

\begin{quote}
\begin{tabular}{|l|}
\hline
\verb|Time 1984 01 20  16 00| \\
\verb|Obs 149 04.0  -31 16.6  1164| \\
\verb|Met 288 899| \\
\verb|Col 600| \\
\hline
\end{tabular}
\end{quote}

If the time, observatory and meteorological records are all omitted,
any colour records subsequently encountered will be ignored.  In the
absence of full information, ASTROM makes plausible guesses to make good
the deficiencies.  If insufficient information for the observed place
predictions is available, warnings are issued and the astrometry is done
using mean place.  If any of the three observation data records appears
twice, the new information supplants the old, and no error is reported.

The TIME record specifies the time for mid-exposure; this can be
given as a UT date and time (as five numbers, year, month, day, hour,
minutes), or a local sidereal time (hour and minutes) or a Julian epoch
(a single float).  If the time record specifies
a UT, and an epoch is specified on the plate data record, the latter is
ignored.  If the ST option is used, the epoch on the plate data record
must be specified (and should be accurate to a day or two if the annual
aberration and solar deflection are to be correctly computed).  In the
absence of a UT, it is reasonable to guess that the exposure occurred
near upper culmination, which simply requires the ST to be set equal to
the plate centre $\alpha$.  For example, to perform an observed place
reduction on a plate of a field at $\alpha=19^{h}13^{m}$, the following
TIME record might be used:

\begin{quote}
\begin{tabular}{|l|}
\hline
\verb|Time 19 13   * Estimated LST| \\
\hline
\end{tabular}
\end{quote}

The OBSERVATORY record can either specify one of the observatory
identifiers recognized by the \xref{SLALIB}{sun67}{} routine
\xref{SLA\_OBS}{sun67}{SLA_OBS} (see SUN/67):

\begin{quote}
\begin{tabular}{|l|}
\hline
\verb|Obs AAT| \\
\hline
\end{tabular}
\end{quote}

or the observatory position can be given explicitly as in the example
given earlier.  If the TIME record specifies sidereal time, the
observatory longitude may optionally be omitted.  The height (metres above
sea level) is of limited importance unless the meteorological record is
absent, in which case the height is used to estimate the pressure.

The METEOROLOGICAL record specifies the temperature and pressure at the
telescope, in $^\circ$K and mB respectively.  The temperature defaults
to $278^\circ$K; the default pressure is computed from the observatory
height.

COLOUR records can appear anywhere after the time, observatory and
meteorological records, except between a pair of reference star records.
Here is an example:

\begin{quote}
\begin{tabular}{|l|}
\hline
\verb|Colour 550| \\
\hline
\end{tabular}
\end{quote}

The effective wavelength specified by such a record applies to all
stars from that point onwards.  Should two colour records follow
consecutively, the second supplants the first, and no error is reported.
Prior to the first colour record, a default of 500nm is assumed.
Appendix~\ref{effective_wavelengths} contains rough estimates of the
effective wavelength for sources of different colour temperature and
detectors of different passband.  For the photographic case, the following
table (compiled with the help of D.\,Malin) suggests effective wavelengths
for some common combinations of emulsion, filter and star colour;
the \textit{blue} and \textit{red} columns refer to very blue and
very red (thermal) sources respectively (the effects may, of course, be
more extreme for emission-line objects and other non-blackbody sources).
For a star of spectral type G0, the effective wavelength will lie about
halfway between the \textit{blue} and \textit{red} figures.

\begin{center}
\begin{tabular}{|c|c|c||c|c|}
\hline
\textit{band} & \textit{emulsion} & \textit{filter} & \textit{blue} & \textit{red} \\
\hline \hline
U & O & UG\,1 & 365 & 365 \\
  & J &       &     &     \\
\hline
B & IIa\,O & GG\,385 & 410 & 420 \\
  &        & GG\,395 &     &     \\
\cline{2-5}
  & IIa\,J & GG\,385 & 410 & 480 \\
  &        & GG\,395 &     &     \\
\hline
V & IIa\,D & GG\,495 & 550 & 600 \\
\hline
R & IIIa\,F & RG\,610 & 675 & 675 \\
  & 103a\,E & RG\,630 &     &     \\
\cline{3-5}
  & 098-04 & GG\,495 & 600 & 675 \\
\hline
I & IV-N & GG\,695 & 800 & 800 \\
\hline
\end{tabular}
\end{center}

It must again be pointed out that there may be other important
colour effects, apart from atmospheric dispersion, notably where
refracting optics have been used.  There is no attempt in ASTROM
to model such phenomena.

\section{\xlabel{fitting_plate_centre_and_radial_distortion}%
Fitting Plate Centre and Radial Distortion}
\label{fitting_plate_centre_and_radial_distortion}

Given a sufficient number of reference stars, measured to high accuracy
and evenly distributed over the whole plate, it is possible to supplement
the normal 4- and 6-coefficient linear solutions with one in which the
plate centre and/or the radial distortion are determined automatically.

The option of fitting the plate centre (\emph{i.e.}\ the \radec\ of the
centre of projection, which \textit{a priori} may well not be known to
adequate accuracy) is selected simply by beginning the plate centre record
with the tilde character \verb|'~'| (meaning ``approximately''):

\goodbreak
\begin{quote}
\begin{tabular}{|l|}
\hline
\verb|~ 12 53 00.0  -42 00 00  B1950.0  1974.5  * Approx plate centre| \\
\hline
\end{tabular}
\end{quote}

Even though the plate centre is to be adjusted, it is advisable to start
off with the best available estimate.  The difference between this and
the actual centre of the projection pattern is what textbooks refer
to as \textit{tilt}.  Determination of the tilt is most secure where
the radial distortion is pronounced.  Schmidt astrometry is relatively
insensitive to tilt, and attempting to fit the plate centre may be unwise
unless the reference stars are numerous and well-distributed.

A further option (intended for investigating the properties of previously
unmodelled telescopes rather than for routine use), is to fit the radial
distortion coefficient.  This is selected by prefixing the telescope
type record with the tilde character \verb|'~'|:

\goodbreak
\begin{quote}
\begin{tabular}{|l|}
\hline
\verb|~ AAT3                * Guess| \\
\hline
\end{tabular}
\end{quote}

Tilt and distortion may be fitted simultaneously.  Neither adjustment will
be attempted unless at least 10 reference stars are supplied.  If the
fit proves to be unacceptably ill-conditioned, or if the adjustments
are unrealistically large, the fit is rejected.

Though no check is made, it is clearly unwise to request that the tilt
and distortion be included in the model if the reduction is not taking
place in \textit{observed} coordinates (see the previous section).

\section{\xlabel{section_parallax}Parallax}
\label{parallax}

As described in Section~\ref{the_input_file}, reference star celestial
positions may be expressed in either of two formats.  The most usual
format includes proper motions and has an optional epoch which defaults
to that of the equinox.  The other common format, used for reference
stars whose proper motions are assumed to be zero in an inertial frame,
has no proper motions and must have an epoch as well as an equinox.

The first format (\emph{i.e.}\ with proper motions) has the
supplementary option of allowing the annual parallax to be specified,
following or instead of the epoch.  Here are two fictitious
reference star \radec\ records each of which includes parallax:

\goodbreak
\begin{quote}
\begin{small}
\begin{tabular}{|l|}
\hline
\verb|14 39 36.087  -60 50 07.14  -0.49486 +0.6960  J2000.0  0.752 * Ref 1| \\
\verb|09 16 19.03 -10 52 23.2 -0.0401 -0.006 B1950.0 1978.9 0.032 * Ref 2| \\
\hline
\end{tabular}
\end{small}
\end{quote}

In the case where the parallax is supplied without an epoch, which of
the two is meant is deduced from the size of the number given.

In the case where an epoch is supplied as well as well as a parallax,
it is assumed that the parallax has yet to be applied.  In other words,
the option to have the parallax removed from a reference star at the
given catalogue epoch and then put back in for the epoch of the plate
is \textbf{not} provided.  The parallax is only taken into account
(except for second-order effects on the proper motion) when a reduction
in observed place has been requested (by supplying observatory, time and
refraction information -- see Section~\ref{reduction_in_observed_place}).
Note that no provision is made to specify the radial velocity of a
reference star.  This would only matter in cases where the plate epoch
was very distant from the reference star epoch and where both radial
velocity and parallax were large.

No provision exists in ASTROM for specifying the parallax (or proper motion)
of the unknown stars.

%\begin{raggedright}
%\bibliographystyle{unsrt}
%\bibliography{sun5}
%\end{raggedright}
% .bbl file hacked into sun5.tex to keep star2html happy
\begin{thebibliography}{1}

\bibitem{fitswcs1}
E.~W. Greisen and M.~R. Calabretta.
\newblock Representations of world coordinates in {FITS}.
\newblock {\em Astronomy and Astrophysics}, 395(3):1061--1076, December 2002.
\newblock `Paper~I', also available at \cite{fitswcsurl}.

\bibitem{fitswcs2}
M.~R. Calabretta and E.~W. Greisen.
\newblock Representations of celestial coordinates in {FITS}.
\newblock {\em Astronomy and Astrophysics}, 395(3):1077--1122, December 2002.
\newblock `Paper~II', also available at \cite{fitswcsurl}.

\bibitem{fitswcsurl}
M.~R. Calabretta et~al.
\newblock {FITS WCS} pages.
\newblock Web page [cited May 2003].
\newblock \texttt{http://www.atnf.csiro.au/people/mcalabre/WCS/}.

\end{thebibliography}


\newpage
\appendix
\section{\xlabel{appendix_the_input_file}The Input File}
\label{appendix_the_input_file}

This appendix gives a more formal and complete specification of
the input file than is given in the main text, and concludes
with a more comprehensive example.

\begin{description}

\item[\xlabel{INPUT_FILE}INPUT FILE]\mbox{}

 SEQUENCE \\
 \texttt{/} \\
 SEQUENCE \\
 \texttt{/} \\
 $\vdots$ \\
 SEQUENCE \\
 \texttt{END} \\
 Blank records are ignored.
 Any record may end in a comment, which begins with an asterisk.
 Where records contain numbers these are free-format, decoded by the
 \xref{SLA\_DBJIN}{sun67}{SLA_DBJIN} and \xref{SLA\_DFLTIN}{sun67}{SLA_DFLTIN}
 routines in \xref{SLALIB}{sun67}{} (see SUN/67), separated by spaces or a
 single comma.  Lowercase and uppercase can be freely mixed.
 Any number of sequences is permitted.

\goodbreak
\item[\xlabel{SEQUENCE}SEQUENCE]\mbox{}

 RESULTS EQUINOX RECORD, optional \\
 TELESCOPE TYPE RECORD, optional \\
 PLATE DATA RECORD \\
 OBSERVATION DATA, 0-3 records \\
 REFERENCE STARS, 2-3 per star for 2-2000 stars \\
 UNKNOWN STARS, 1-2 per star for any number of stars

\goodbreak
\item[\xlabel{OBSERVATION_DATA}OBSERVATION DATA]\mbox{}

 TIME RECORD \\
 OBSERVATORY RECORD \\
 METEOROLOGICAL RECORD \\
 (Any order; any selection; repeats harmless.)

\goodbreak
\item[\xlabel{REFERENCE_STAR}REFERENCE STAR]\mbox{}

 COLOUR RECORD (optional; repeats harmless) \\
 REFERENCE STAR RA,DEC RECORD \\
 REFERENCE STAR X,Y RECORD \\
 A colour record applies to all subsequent stars,
 both reference and unknown.

\goodbreak
\item[\xlabel{UNKNOWN_STAR}UNKNOWN STAR]\mbox{}

 COLOUR RECORD (optional; repeats harmless) \\
 \textit{and/or:} \\
 UNKNOWN STAR X,Y RECORD \\
 \textit{or:} \\
 UNKNOWN STAR RA,DEC RECORD

\goodbreak
\item[\xlabel{RESULTS_EQUINOX_RECORD}RESULTS EQUINOX RECORD]\mbox{}

 EQUINOX \\
 In the absence of this record, J2000.0 is used.

\goodbreak
\item[\xlabel{TELESCOPE_TYPE_RECORD}TELESCOPE TYPE RECORD]\mbox{}

 [APPROX] PROJECTION \\
 In the absence of this record, \texttt{SCHM} is assumed.

\goodbreak
\item[\xlabel{PLATE_DATA_RECORD}PLATE DATA RECORD]\mbox{}

 [APPROX] RA DEC EQUINOX [EPOCH] \\
 The position specified is that of the plate centre.
 The epoch is optional only if the information is supplied
 later in a time record (see next item).

\goodbreak
\item[\xlabel{TIME_RECORD}TIME RECORD]\mbox{}

 \textit{Either:} \\
 \texttt{T\ldots} UT \\
 \textit{or:} \\
 \texttt{T\ldots} ST \\
 The first of these forms allows the epoch to be omitted from
 the plate data record.

\goodbreak
\item[\xlabel{OBSERVATORY_RECORD}OBSERVATORY RECORD]\mbox{}

 \textit{Either:} \\
 \texttt{O\ldots} STATION \\
 \textit{or:} \\
 \texttt{O\ldots} [LONGITUDE] LATITUDE [HEIGHT] \\
 The longitude may be omitted only if the sidereal
 time has been or will be specified via a time record.
 The height defaults to an estimate based on the air pressure.

\goodbreak
\item[\xlabel{METEOROLOGICAL_RECORD}METEOROLOGICAL RECORD]\mbox{}

 \texttt{M\ldots} TEMPERATURE [PRESSURE] \\
 The temperature defaults to $278^\circ$K.
 The pressure defaults to an estimate based on the height.

\goodbreak
\item[\xlabel{COLOUR_RECORD}COLOUR RECORD]\mbox{}

 \texttt{C\ldots} WAVELENGTH \\
 A wavelength, once specified, applies to all stars from then on.
 The starting default is 500~nm.

\goodbreak
\item[\xlabel{REFERENCE_STAR_RA_DEC_RECORD}REFERENCE STAR RA,DEC
RECORD]\mbox{}

 \textit{Either:} \\
 RA DEC PMR PMD EQUINOX [EPOCH] [PARALLAX] [NAME] \\
 \textit{or:} \\
 RA DEC EQUINOX EPOCH [NAME] \\
 The second format implies inertially zero proper motion.

\goodbreak
\item[\xlabel{REFERENCE_STAR_X_Y_RECORD}REFERENCE STAR X,Y RECORD]\mbox{}

 X Y

\goodbreak
\item[\xlabel{UNKNOWN_STAR_X_Y_RECORD}UNKNOWN STAR X,Y RECORD]\mbox{}

 X Y [NAME]

\goodbreak
\item[\xlabel{UNKNOWN_STAR_RA_DEC_RECORD}UNKNOWN STAR RA,DEC RECORD]\mbox{}

 RA DEC EQUINOX [NAME]

\goodbreak
\item[\xlabel{EQUINOX}EQUINOX]\mbox{}

The epoch of the equator and equinox of a mean \radec\
coordinate system.  A Besselian epoch implies the pre~IAU~1976
system (as used in the FK4 catalogue) and a Julian epoch implies
the post~IAU~1976 system (as used in the FK5 catalogue).

\goodbreak
\item[\xlabel{APPROX}APPROX]\mbox{}

 \verb|~| (tilde) \\
At the start of the telescope type and plate data
records, this specifies that the radial
distortion and plate centre, respectively, are to be fitted.

\goodbreak
\item[\xlabel{PROJECTION}PROJECTION]\mbox{}

 \texttt{ASTR\ldots} = astrograph \\
 \textit{or} \\
 \texttt{SCHM\ldots} = Schmidt (default) \\
 \textit{or} \\
 \texttt{AAT2\ldots} = AAT prime focus doublet \\
 \textit{or} \\
 \texttt{AAT3\ldots} = AAT prime focus triplet \\
 \textit{or} \\
 \texttt{AAT8\ldots} = AAT $f/8$ with vacuum plateholder \\
 \textit{or} \\
 \texttt{JKT8\ldots} = JKT $f/8$ Harmer-Wynne \\
 \textit{or} \\
 \texttt{GENE\ldots} DISTORTION = generalized pincushion/barrel distortion

\goodbreak
\item[\xlabel{PROJECTION}DISTORTION]\mbox{}

A number, the parameter $q$ in the pincushion/barrel distortion expression
$\mbox{\boldmath $r'$}=(1+q\mid\mbox{\boldmath $r$}\mid^2)\mbox{\boldmath $r$}$,
where \mbox{\boldmath $r'$} is the radial vector to the star image
from the intersection of the optical axis and the plate,
and \mbox{\boldmath $r$} is the same vector but assuming tangent-plane
geometry.  The vectors are in units of one focal length.

\goodbreak
\item[\xlabel{EPOCH}EPOCH]\mbox{}

A Besselian or Julian epoch:  a single number resembling years AD,
optionally preceded by \texttt{B} (for \textit{Besselian}) or
\texttt{J} (for \textit{Julian}).  In the absence of a prefix, epochs
before 1984.0 are assumed to be in the Besselian timescale, and epochs
from 1984.0 onwards are assumed to be in the Julian timescale.

\goodbreak
\item[\xlabel{RA_DEC}RA DEC]\mbox{}

A mean \radec\ expressed as six numbers: hours, minutes, seconds,
degrees, arcminutes, arcseconds.  The seconds and arcseconds can be
given to any reasonable precision;  the others must be integers.  All
the numbers except the degrees must be positive; southern $\delta$ is
indicated by minus degrees (even if zero).

\goodbreak
\item[\xlabel{UT}UT]\mbox{}

The UT epoch of observation expressed as six numbers: years AD, month,
day, hours, minutes.  All but the minutes must be integers.  The year,
month and day must form a valid date in the Gregorian calendar.

\goodbreak
\item[\xlabel{ST}ST]\mbox{}

The local (apparent) sidereal time of observation expressed as two
numbers: hours, minutes.  The hours must be an integer.

\goodbreak
\item[\xlabel{STATION}STATION]\mbox{}

A character string specifying one of the observatories supported by
the \xref{SLA\_OBS}{sun67}{SLA_OBS} routine in \xref{SLALIB}{sun67}{}
(see SUN/67).

\goodbreak
\item[\xlabel{LONGITUDE}LONGITUDE]\mbox{}

The east longitude, expressed as two numbers: degrees (which must be an
integer) and arcminutes.  West longitudes may be indicated either by
minus degrees (even if zero) or by east longitude $> 180^\circ$.

\goodbreak
\item[\xlabel{LATITUDE}LATITUDE]\mbox{}

The (geodetic) latitude, expressed as two numbers: degrees (which must
be an integer) and arcminutes.  South latitude is indicated by minus
degrees (even if zero).

\goodbreak
\item[\xlabel{HEIGHT}HEIGHT]\mbox{}

 A single number, the height above sea level in metres.

\goodbreak
\item[\xlabel{TEMPERATURE}TEMPERATURE]\mbox{}

A single number, the ambient temperature in $^\circ$K.

\goodbreak
\item[\xlabel{PRESSURE}PRESSURE]\mbox{}

A single number, the pressure in mB.

\goodbreak
\item[\xlabel{WAVELENGTH}WAVELENGTH]\mbox{}

A single number, the effective wavelength in nm.

\goodbreak
\item[\xlabel{PMR_PMD}PMR PMD]\mbox{}

Two numbers, the proper motions in seconds and arcseconds per year
respectively.

\goodbreak
\item[\xlabel{PARALLAX}PARALLAX]\mbox{}

A single number, the annual parallax in arcseconds.

\goodbreak
\item[\xlabel{NAME}NAME]\mbox{}

The name field, which is always optional, is simply the first 10
characters of the comment, excluding the \texttt{*} and any leading
spaces.

\goodbreak
\item[\xlabel{X_Y}X Y]\mbox{}

Two numbers, the Cartesian coordinates on the plate; the units of $x$
and $y$ should preferably be the same and in the $\mu$m to m range
(\emph{e.g.}\ mm), and the zero points should not be too far from the
region of measurement.

\end{description}

The example input file which appears on the next page includes two
sequences.  The first is for a typical run using measurements from a
Schmidt plate, while the second is for precise reduction, in observed
place, of AAT~$f/8$ measurements, including automatic determination of
the radial distortion and plate centre.  For brevity, neither sequence
includes as many reference stars as would normally be advisable.

\newpage
\begin{center}
\begin{scriptsize}
\begin{tabular}{|l|}
\hline
\verb|B1950                                     * Results in FK4| \\
\verb|SCHM                                      * Schmidt geometry| \\
\verb|19 04 00.0  -65 00 00  B1950.0  1974.5    * Plate centre, and epoch| \\
\verb|18 56 39.426  -63 25 13.23  -0.0002  -0.036  B1950.0  * Ref 1| \\
\verb|44.791   85.643| \\
\\
\verb|19 11 53.909  -63 17 57.57   0.0058  -0.044  1950.0   * Ref 2| \\
\verb|-46.266   92.337| \\
\\
\verb|19 01 13.606  -63 49 14.84   0.0020  -0.026  1950.0   * Ref 3| \\
\verb|17.246   64.945| \\
\\
\verb|19 08 29.088  -63 57 42.79   0.0016   0.018  1950.0   * Ref 4| \\
\verb|-25.314   57.456| \\
\\
\verb|19 02 10.088  -63 29 16.73   0.0012  -0.019  1950.0   * Ref 5| \\
\verb|11.890   82.766| \\
\\
\verb|-5.103    58.868                      *  Candidate| \\
\verb|19 09 46.2  -63 51 27  J2000.0        *  Radio pos| \\
\\
\verb|/                                     *  End of first sequence| \\
\\
\verb|* AAT plate 2266 (f/8 RC)  NGC 3114| \\
\verb|~ AAT8    * To be fitted| \\
\verb|~ 10 01 00.0 -59 53 01 B1950   * To be fitted| \\
\verb|Time 1984 01 20  16 00| \\
\verb|Obs AAT| \\
\verb|Met 288 899| \\
\verb|Colour 450    *  Default colour for reference stars| \\
\verb|10 01 21.203 -59 52 14.05 B1950 J1984.1| \\
\verb|9.0353 18.4211 *130| \\
\verb|10 00 16.401 -59 52 52.16 B1950 J1984.1| \\
\verb|1.7304 17.9282 *70| \\
\verb|10 00 18.516 -59 53 10.20 B1950 J1984.1| \\
\verb|1.9669 17.6566 *73| \\
\verb|10 00 19.620 -59 49 01.62 B1950 J1984.1| \\
\verb|2.1223 21.3760 *74| \\
\verb|10 00 20.525 -59 52 01.09 B1950 J1984.1| \\
\verb|2.2025 18.6888 *75| \\
\verb|10 00 21.416 -59 51 30.27 B1950 J1984.1| \\
\verb|2.3067 19.1501 *76| \\
\verb|10 00 22.896 -59 53 49.60 B1950 J1984.1| \\
\verb|2.4544 17.0626 *80| \\
\verb|10 01 26.159 -59 50 38.50 B1950 J1984.1| \\
\verb|9.6143 19.8435 *134| \\
\verb|10 01 28.328 -59 51 16.86 B1950 J1984.1| \\
\verb|9.8509 19.2653 *138| \\
\verb|10 01 54.446 -59 54 39.28 B1950 J1984.1| \\
\verb|12.7495 16.1963 *156| \\
\verb|10 01 54.523 -59 50 01.72 B1950 J1984.1| \\
\verb|12.8193 20.3493 *157| \\
\verb|10 01 57.438 -59 51 26.29 B1950 J1984.1| \\
\verb|13.1292 19.0793 *161| \\
\verb|10 00 12.385 -60 08 08.51 B1950 J1984.1| \\
\verb|1.1637 4.2188 *65| \\
\verb|C 500| \\
\verb|5.8265 12.7252 *104 red| \\
\verb|C 400| \\
\verb|5.8265 12.7252 *104 blue| \\
\\
\verb|END| \\
\hline
\end{tabular}
\end{scriptsize}
\end{center}

\newpage
\section{\xlabel{example_report}Example Report}
\label{example_report}

Here is an example of the report produced by ASTROM.  It is the result
of a run using the specimen input file presented in
Section~\ref{the_input_file}.  The first part of the report lists the
raw data.  The second part gives the results of the 4-coefficient
solution.  The third part (next page) gives the results of the
6-coefficient solution.  The fourth part gives the predictions for the
unknown stars.

It should be noted that the report file contains Fortran printer codes.

\vfill
\begin{tiny}
\begin{quote}
\begin{verbatim}
1* * * * * * * * * * * * *
 *  A S T R O M E T R Y  *
 * * * * * * * * * * * * *



 Equinox for mean coordinates of results:  B1950.0

 Projection geometry:  Schmidt

 Plate centre:  19 04 00.0   -65 00 00     Equinox B1950.0     Epoch B1974.500


 Reference stars:

                n              RA            Dec         pmR    pmD   Equinox   Epoch      px         Xmeas        Ymeas

                1         18 56 39.426  -63 25 13.23  -0.0002 -0.036  B1950.0  B1950.000  0.000      +44.791      +85.643
                2         19 11 53.909  -63 17 57.57  +0.0058 -0.044  B1950.0  B1950.000  0.000      -46.266      +92.337
                3         19 01 13.606  -63 49 14.84  +0.0020 -0.026  B1950.0  B1950.000  0.000      +17.246      +64.945
                4         19 08 29.088  -63 57 42.79  +0.0016 +0.018  B1950.0  B1950.000  0.000      -25.314      +57.456
                5         19 02 10.088  -63 29 16.73  +0.0012 -0.019  B1950.0  B1950.000  0.000      +11.890      +82.766
\end{verbatim}
\end{quote}

\vfill

\begin{quote}
\begin{verbatim}
1Plate solution: 4-coefficient
 -----------------------------

     X,Y = expected plate coordinates (radians)

     X =  +0.2328072E-03                            Y =  -0.7392285E-03
          -0.3274547E-03 * Xmeas                         -0.5543943E-06 * Xmeas
          -0.5543943E-06 * Ymeas                         +0.3274547E-03 * Ymeas


 Xmeas =  +0.7071360                            Ymeas =   +2.258696
           -3053.849     * X                              -5.170293     * X
           -5.170293     * Y                              +3053.849     * Y


      Plate scale (in measuring units):           67.5425 arcsec
                           Orientation:           +0.097  deg and laterally inverted


 Reference stars:
                             Mean RA,Dec         Equinox B1950.0      Epoch B1974.500           Residuals (arcsec)
                n                    catalogue                     calculated                 dX        dY        dR

                1            18 56 39.421  -63 25 14.11    18 56 39.468  -63 25 14.27      +0.319    -0.145     0.350
                2            19 11 54.051  -63 17 58.65    19 11 54.082  -63 17 58.00      +0.226    +0.638     0.677
                3            19 01 13.655  -63 49 15.48    19 01 13.580  -63 49 15.88      -0.489    -0.407     0.636
                4            19 08 29.127  -63 57 42.35    19 08 29.016  -63 57 42.72      -0.738    -0.365     0.824
                5            19 02 10.117  -63 29 17.20    19 02 10.220  -63 29 16.92      +0.682    +0.280     0.738

                                                                                  RMS :     0.530     0.401     0.665
\end{verbatim}
\end{quote}

\newpage

\begin{quote}
\begin{verbatim}
1Plate solution: 6-coefficient
 -----------------------------

     X,Y = expected plate coordinates (radians)

     X =  +0.2457761E-03                            Y =  -0.7296778E-03
          -0.3274742E-03 * Xmeas                         -0.5253092E-06 * Xmeas
          -0.7235166E-06 * Ymeas                         +0.3273299E-03 * Ymeas


 Xmeas =  +0.7455928                            Ymeas =   +2.230379
           -3053.665     * X                              -4.900618     * X
           -6.749697     * Y                              +3055.011     * Y


     Plate scales (in measuring units):      X    67.5465 arcsec
                                             Y    67.5168 arcsec
                                          mean    67.5316 arcsec
                   Nonperpendicularity:           +0.035  deg
                           Orientation:           +0.109  deg and laterally inverted


 Reference stars:
                             Mean RA,Dec         Equinox B1950.0      Epoch B1974.500           Residuals (arcsec)
                n                    catalogue                     calculated                 dX        dY        dR

                1            18 56 39.421  -63 25 14.11    18 56 39.395  -63 25 14.22      -0.174    -0.112     0.207
                2            19 11 54.051  -63 17 58.65    19 11 54.031  -63 17 58.70      -0.134    -0.047     0.142
                3            19 01 13.655  -63 49 15.48    19 01 13.633  -63 49 15.48      -0.149    -0.006     0.149
                4            19 08 29.127  -63 57 42.35    19 08 29.132  -63 57 42.37      +0.034    -0.026     0.043
                5            19 02 10.117  -63 29 17.20    19 02 10.181  -63 29 17.01      +0.422    +0.191     0.463

                                                                                  RMS :     0.224     0.102     0.246
\end{verbatim}
\end{quote}

\vspace*{20mm}

\begin{quote}
\begin{verbatim}
1Unknown stars:    B1950.0 mean places for epoch B1974.500

             :               4-coefficient model               :                6-coefficient model                :
             :   Xmeas    Ymeas          RA            Dec     :   Xmeas    Ymeas          RA             Dec      :
             :                                                 :                                                   :
             :   -5.103  +58.868 -> 19 05 01.697  -63 56 17.13 :   -5.103  +58.868 -> 19 05 01.7942  -63 56 16.702 :
             :   -5.120  +58.862 <- 19 05 01.874  -63 56 17.54 :   -5.111  +58.856 <- 19 05 01.8741  -63 56 17.539 :
\end{verbatim}
\end{quote}
\end{tiny}

\vspace*{20mm}


The residuals \texttt{dX} and \texttt{dY} are, loosely, in
\textit{standard coordinates}; except at the poles, positive
\texttt{dX} means that the measured position of the image lies to the
east of the expected position, and positive \texttt{dY} means that the
measured position lies to the north of the expected position.

If a 7-9 coefficient fit is requested, and providing there are at
least ten reference stars, an extra section of report precedes the
unknowns.  If the fit is successful, the two solutions used to process
the unknown stars are the 6 and 7-9 coefficient ones, and the
4-coefficient model is not used.

\newpage
\section{\xlabel{effective_wavelengths}Effective Wavelengths}
\label{effective_wavelengths}

Most ASTROM reductions work in mean \radec\ of date, absorbing
various small rotations and distortions of the field into the
fit.  This normally delivers results of adequate accuracy.
However, for slightly improved accuracy, especially at low
elevations and with wide fields, ASTROM can optionally be supplied with
additional information (an accurate date and time,
observatory location, \textit{etc.}) to enable reduction in
\textit{observed} place.  The further possibility is then
available of correcting for atmospheric dispersion, by
supplying effective colours for the reference and unknown stars;
this can sometimes be important.

The following tables estimate the effective wavelength for different
spectral types and detector/filter passbands.  The figures given are the
median wavelength of the appropriate blackbody spectrum within the
specified band, and are thus only a very rough guide.  Some detectors
peak strongly; some filters leak outside their nominal
passband; both may roll off gradually at the edges of the passband;
some stellar spectra have pronounced absorption or emission
features.

The tables cover blue cutoffs from 320~nm and red cutoffs up to 800~nm,
in steps of 20~nm and with a minimum passband of 100~nm, for 13
colour temperatures.  B$-$V and U$-$V colours and
the nearest main-sequence spectral types are also given, as listed
in Table~99 of \textit{Astrophysical Quantities}.

For any given exposure, only one line in the tables will be
needed, corresponding to the spectral response of the detector plus
filter used.  The appropriate effective wavelength for any star in
the exposure can then be found by looking in the appropriate
column.

\clearpage

%%%%%%%%%%%%%%%%%%%%%%%%%%%%%%%%%%%%%%%%%%%%
% LaTeX source generated by program EWLLTX %
%  P.T.Wallace   Starlink   17 March 1989  %
%%%%%%%%%%%%%%%%%%%%%%%%%%%%%%%%%%%%%%%%%%%%

\begin{tiny}

\noindent
\begin{center}
\begin{tabular}{|c|c
@{\hspace{2ex}}c
@{\hspace{2ex}}c
@{\hspace{2ex}}c
@{\hspace{2ex}}c
@{\hspace{2ex}}c
@{\hspace{2ex}}c
@{\hspace{2ex}}c
@{\hspace{2ex}}c
@{\hspace{2ex}}c
@{\hspace{2ex}}c
@{\hspace{2ex}}c
@{\hspace{2ex}}c|l}
\cline{1-14}
Band
&  3000
&  3800
&  4500
&  5400
&  6000
&  6700
&  7600
&  9000
& 11100
& 15400
& 23000
& 38000
& 70000
& T$_c$ ($^\circ$K) \\
\cline{1-14}
320-420 & 392 & 386 & 382 & 377 & 375 & 373 & 371 & 368 & 365 & 362 & 360 & 359 & 358 & \\
320-440 & 408 & 400 & 395 & 389 & 386 & 383 & 380 & 377 & 373 & 369 & 366 & 364 & 363 & \\
320-460 & 424 & 415 & 408 & 401 & 397 & 394 & 390 & 385 & 381 & 375 & 372 & 369 & 367 & \\
320-480 & 440 & 430 & 422 & 413 & 408 & 404 & 399 & 393 & 387 & 381 & 377 & 373 & 371 & \\
320-500 & 456 & 444 & 435 & 424 & 419 & 413 & 407 & 401 & 394 & 387 & 381 & 377 & 375 & \\
320-520 & 472 & 458 & 448 & 436 & 429 & 423 & 416 & 408 & 400 & 391 & 385 & 381 & 378 & \\
320-540 & 488 & 472 & 460 & 447 & 439 & 432 & 424 & 415 & 406 & 396 & 389 & 384 & 381 & \\
320-560 & 503 & 486 & 473 & 457 & 449 & 441 & 432 & 421 & 411 & 400 & 392 & 387 & 384 & \\
320-580 & 519 & 500 & 485 & 468 & 458 & 449 & 439 & 428 & 416 & 404 & 395 & 389 & 386 & \\
320-600 & 534 & 513 & 497 & 478 & 468 & 457 & 446 & 434 & 421 & 408 & 398 & 392 & 388 & \\
320-620 & 549 & 527 & 508 & 488 & 476 & 465 & 453 & 439 & 426 & 411 & 401 & 394 & 390 & \\
320-640 & 563 & 540 & 520 & 498 & 485 & 473 & 459 & 444 & 430 & 414 & 403 & 396 & 391 & \\
320-660 & 578 & 552 & 531 & 507 & 493 & 480 & 466 & 449 & 434 & 417 & 405 & 397 & 393 & \\
320-680 & 592 & 565 & 542 & 516 & 501 & 487 & 472 & 454 & 437 & 419 & 407 & 399 & 394 & \\
320-700 & 607 & 577 & 552 & 525 & 509 & 493 & 477 & 459 & 441 & 422 & 409 & 400 & 395 & \\
320-720 & 620 & 589 & 563 & 533 & 516 & 500 & 482 & 463 & 444 & 424 & 410 & 402 & 396 & \\
320-740 & 634 & 601 & 573 & 541 & 523 & 506 & 488 & 467 & 447 & 426 & 412 & 403 & 397 & \\
320-760 & 648 & 612 & 582 & 549 & 530 & 512 & 492 & 470 & 450 & 428 & 413 & 404 & 398 & \\
320-780 & 661 & 623 & 592 & 557 & 537 & 517 & 497 & 474 & 452 & 430 & 414 & 405 & 399 & \\
320-800 & 674 & 634 & 601 & 564 & 543 & 522 & 501 & 799 & 799 & 799 & 799 & 799 & 799 & \\
340-440 & 410 & 404 & 400 & 396 & 394 & 392 & 390 & 387 & 385 & 382 & 380 & 379 & 378 & \\
340-460 & 426 & 418 & 413 & 407 & 405 & 402 & 399 & 396 & 393 & 389 & 387 & 385 & 383 & \\
340-480 & 441 & 432 & 426 & 419 & 415 & 412 & 408 & 404 & 400 & 396 & 392 & 390 & 388 & \\
340-500 & 457 & 446 & 438 & 430 & 426 & 421 & 417 & 412 & 407 & 401 & 397 & 394 & 393 & \\
340-520 & 473 & 460 & 451 & 441 & 436 & 431 & 426 & 419 & 413 & 407 & 402 & 398 & 396 & \\
340-540 & 488 & 474 & 464 & 452 & 446 & 440 & 434 & 427 & 420 & 412 & 406 & 402 & 400 & \\
340-560 & 504 & 488 & 476 & 463 & 456 & 449 & 442 & 433 & 425 & 416 & 410 & 405 & 403 & \\
340-580 & 519 & 502 & 488 & 473 & 465 & 457 & 449 & 440 & 431 & 421 & 413 & 408 & 406 & \\
340-600 & 534 & 515 & 500 & 483 & 474 & 465 & 456 & 446 & 436 & 425 & 417 & 411 & 408 & \\
340-620 & 549 & 528 & 511 & 493 & 483 & 473 & 463 & 452 & 440 & 428 & 419 & 414 & 410 & \\
340-640 & 564 & 541 & 523 & 503 & 492 & 481 & 470 & 457 & 445 & 432 & 422 & 416 & 412 & \\
340-660 & 578 & 554 & 534 & 512 & 500 & 488 & 476 & 462 & 449 & 435 & 425 & 418 & 414 & \\
340-680 & 593 & 566 & 545 & 521 & 508 & 495 & 482 & 467 & 453 & 438 & 427 & 420 & 416 & \\
340-700 & 607 & 578 & 555 & 530 & 516 & 502 & 488 & 472 & 456 & 440 & 429 & 421 & 417 & \\
340-720 & 621 & 590 & 565 & 538 & 523 & 508 & 493 & 476 & 460 & 443 & 431 & 423 & 418 & \\
340-740 & 634 & 602 & 575 & 546 & 530 & 515 & 498 & 480 & 463 & 445 & 433 & 424 & 420 & \\
340-760 & 648 & 613 & 585 & 554 & 537 & 520 & 503 & 484 & 466 & 447 & 434 & 426 & 421 & \\
340-780 & 661 & 624 & 595 & 562 & 544 & 526 & 508 & 488 & 469 & 449 & 436 & 427 & 422 & \\
340-800 & 674 & 635 & 603 & 569 & 550 & 531 & 512 & 799 & 799 & 799 & 799 & 799 & 799 & \\
360-460 & 428 & 422 & 418 & 415 & 413 & 411 & 409 & 407 & 405 & 402 & 401 & 399 & 399 & \\
360-480 & 443 & 436 & 431 & 426 & 423 & 421 & 418 & 415 & 412 & 409 & 407 & 405 & 404 & \\
360-500 & 459 & 450 & 443 & 437 & 434 & 430 & 427 & 423 & 420 & 416 & 413 & 410 & 409 & \\
360-520 & 474 & 463 & 456 & 448 & 444 & 440 & 436 & 431 & 427 & 422 & 418 & 415 & 414 & \\
360-540 & 490 & 477 & 468 & 459 & 454 & 449 & 444 & 438 & 433 & 427 & 423 & 419 & 418 & \\
360-560 & 505 & 491 & 480 & 469 & 463 & 458 & 452 & 445 & 439 & 432 & 427 & 423 & 421 & \\
360-580 & 520 & 504 & 492 & 479 & 473 & 466 & 460 & 452 & 445 & 437 & 431 & 427 & 424 & \\
360-600 & 535 & 517 & 504 & 489 & 482 & 475 & 467 & 458 & 450 & 441 & 435 & 430 & 427 & \\
360-620 & 550 & 530 & 515 & 499 & 491 & 482 & 474 & 464 & 455 & 445 & 438 & 433 & 430 & \\
360-640 & 564 & 543 & 526 & 509 & 499 & 490 & 481 & 470 & 460 & 449 & 441 & 436 & 432 & \\
360-660 & 579 & 555 & 537 & 518 & 508 & 497 & 487 & 476 & 464 & 452 & 444 & 438 & 434 & \\
360-680 & 593 & 568 & 548 & 527 & 516 & 505 & 493 & 481 & 469 & 456 & 446 & 440 & 436 & \\
360-700 & 607 & 580 & 558 & 536 & 523 & 511 & 499 & 486 & 472 & 459 & 449 & 442 & 438 & \\
360-720 & 621 & 592 & 569 & 544 & 531 & 518 & 505 & 490 & 476 & 461 & 451 & 444 & 440 & \\
360-740 & 635 & 603 & 579 & 552 & 538 & 524 & 510 & 494 & 480 & 464 & 453 & 446 & 441 & \\
360-760 & 648 & 615 & 588 & 560 & 545 & 530 & 515 & 499 & 483 & 466 & 455 & 447 & 443 & \\
360-780 & 662 & 626 & 598 & 568 & 551 & 536 & 520 & 502 & 486 & 468 & 456 & 448 & 444 & \\
360-800 & 674 & 636 & 607 & 575 & 558 & 541 & 524 & 799 & 799 & 799 & 799 & 799 & 799 & \\
380-480 & 446 & 441 & 437 & 434 & 432 & 430 & 428 & 427 & 425 & 423 & 421 & 420 & 419 & \\
380-500 & 461 & 454 & 449 & 444 & 442 & 440 & 437 & 435 & 432 & 429 & 427 & 426 & 425 & \\
380-520 & 476 & 467 & 461 & 455 & 452 & 449 & 446 & 443 & 440 & 436 & 433 & 431 & 430 & \\
380-540 & 491 & 481 & 473 & 466 & 462 & 458 & 455 & 450 & 446 & 442 & 438 & 436 & 435 & \\
380-560 & 506 & 494 & 485 & 476 & 472 & 467 & 463 & 458 & 453 & 447 & 443 & 440 & 439 & \\
380-580 & 521 & 507 & 497 & 486 & 481 & 476 & 471 & 465 & 459 & 452 & 448 & 444 & 443 & \\
380-600 & 536 & 520 & 508 & 496 & 490 & 484 & 478 & 471 & 465 & 457 & 452 & 448 & 446 & \\
380-620 & 551 & 533 & 520 & 506 & 499 & 492 & 485 & 477 & 470 & 462 & 456 & 452 & 449 & \\
380-640 & 565 & 545 & 531 & 516 & 508 & 500 & 492 & 483 & 475 & 466 & 459 & 455 & 452 & \\
380-660 & 580 & 558 & 542 & 525 & 516 & 507 & 499 & 489 & 480 & 470 & 462 & 457 & 454 & \\
380-680 & 594 & 570 & 552 & 534 & 524 & 515 & 505 & 494 & 484 & 473 & 465 & 460 & 457 & \\
380-700 & 608 & 582 & 563 & 542 & 532 & 521 & 511 & 499 & 488 & 476 & 468 & 462 & 459 & \\
380-720 & 622 & 594 & 573 & 551 & 539 & 528 & 517 & 504 & 492 & 480 & 470 & 464 & 461 & \\
380-740 & 636 & 606 & 583 & 559 & 546 & 534 & 522 & 509 & 496 & 482 & 473 & 466 & 462 & \\
380-760 & 649 & 617 & 592 & 567 & 553 & 540 & 528 & 513 & 499 & 485 & 475 & 468 & 464 & \\
380-780 & 662 & 628 & 602 & 575 & 560 & 546 & 533 & 517 & 503 & 488 & 477 & 470 & 465 & \\
380-800 & 675 & 638 & 611 & 582 & 566 & 552 & 537 & 799 & 799 & 799 & 799 & 799 & 799 & \\
400-500 & 464 & 459 & 456 & 453 & 451 & 450 & 448 & 446 & 445 & 443 & 441 & 440 & 440 & \\
400-520 & 479 & 472 & 468 & 463 & 461 & 459 & 457 & 455 & 452 & 450 & 448 & 446 & 445 & \\
400-540 & 494 & 485 & 479 & 474 & 471 & 468 & 465 & 462 & 459 & 456 & 454 & 452 & 451 & \\
400-560 & 508 & 498 & 491 & 484 & 480 & 477 & 474 & 470 & 466 & 462 & 459 & 457 & 455 & \\
400-580 & 523 & 511 & 502 & 494 & 490 & 486 & 482 & 477 & 473 & 468 & 464 & 461 & 460 & \\
400-600 & 538 & 524 & 514 & 504 & 499 & 494 & 489 & 484 & 479 & 473 & 469 & 466 & 464 & \\
400-620 & 552 & 536 & 525 & 514 & 508 & 502 & 497 & 490 & 484 & 478 & 473 & 469 & 467 & \\
400-640 & 567 & 549 & 536 & 523 & 517 & 510 & 504 & 497 & 490 & 482 & 477 & 473 & 471 & \\
400-660 & 581 & 561 & 547 & 532 & 525 & 518 & 511 & 503 & 495 & 486 & 480 & 476 & 474 & \\
400-680 & 595 & 573 & 557 & 541 & 533 & 525 & 517 & 508 & 500 & 490 & 484 & 479 & 476 & \\
400-700 & 609 & 585 & 568 & 550 & 541 & 532 & 523 & 513 & 504 & 494 & 487 & 482 & 479 & \\
\cline{1-14}
nm & M5
& M0
& K5
& K0
& G5
& G0
& F5
& F0
& A5
& A0
& B5
& B0
& O5 & \\
& $+1.61$
& $+1.39$
& $+1.11$
& $+0.84$
& $+0.70$
& $+0.57$
& $+0.45$
& $+0.30$
& $+0.16$
& $0.00$
& $-0.17$
& $-0.31$
& $-0.45$ & B$-$V \\
& $+1.19$
& $+1.24$
& $+1.06$
& $+0.46$
& $+0.20$
& $+0.04$
& $-0.01$
& $+0.02$
& $+0.09$
& $0.00$
& $-0.56$
& $-1.07$
& $-1.20$ & U$-$B \\
\cline{1-14}
\end{tabular}
\end{center}
\clearpage

\noindent
\begin{center}
\begin{tabular}{|c|c
@{\hspace{2ex}}c
@{\hspace{2ex}}c
@{\hspace{2ex}}c
@{\hspace{2ex}}c
@{\hspace{2ex}}c
@{\hspace{2ex}}c
@{\hspace{2ex}}c
@{\hspace{2ex}}c
@{\hspace{2ex}}c
@{\hspace{2ex}}c
@{\hspace{2ex}}c
@{\hspace{2ex}}c|l}
\cline{1-14}
Band
&  3000
&  3800
&  4500
&  5400
&  6000
&  6700
&  7600
&  9000
& 11100
& 15400
& 23000
& 38000
& 70000
& T$_c$ ($^\circ$K) \\
\cline{1-14}
400-720 & 623 & 597 & 578 & 558 & 548 & 539 & 529 & 519 & 508 & 497 & 490 & 484 & 481 & \\
400-740 & 637 & 608 & 588 & 567 & 556 & 545 & 535 & 523 & 512 & 501 & 492 & 486 & 483 & \\
400-760 & 650 & 620 & 597 & 575 & 563 & 551 & 540 & 528 & 516 & 504 & 495 & 489 & 485 & \\
400-780 & 663 & 631 & 607 & 582 & 569 & 557 & 545 & 532 & 520 & 506 & 497 & 490 & 487 & \\
400-800 & 676 & 641 & 615 & 589 & 576 & 563 & 550 & 799 & 799 & 799 & 799 & 799 & 799 & \\
420-520 & 483 & 478 & 475 & 472 & 470 & 469 & 468 & 466 & 464 & 463 & 461 & 461 & 460 & \\
420-540 & 497 & 490 & 486 & 482 & 480 & 478 & 476 & 474 & 472 & 470 & 468 & 467 & 466 & \\
420-560 & 511 & 503 & 498 & 492 & 490 & 487 & 485 & 482 & 479 & 476 & 474 & 472 & 471 & \\
420-580 & 526 & 516 & 509 & 502 & 499 & 496 & 493 & 490 & 486 & 482 & 480 & 478 & 476 & \\
420-600 & 540 & 528 & 520 & 512 & 508 & 505 & 501 & 497 & 493 & 488 & 485 & 482 & 481 & \\
420-620 & 555 & 541 & 531 & 522 & 517 & 513 & 508 & 503 & 499 & 493 & 489 & 487 & 485 & \\
420-640 & 569 & 553 & 542 & 531 & 526 & 521 & 516 & 510 & 504 & 498 & 494 & 491 & 489 & \\
420-660 & 583 & 565 & 553 & 541 & 534 & 529 & 523 & 516 & 510 & 503 & 498 & 494 & 492 & \\
420-680 & 597 & 577 & 563 & 550 & 543 & 536 & 529 & 522 & 515 & 507 & 502 & 498 & 495 & \\
420-700 & 611 & 589 & 573 & 558 & 550 & 543 & 536 & 528 & 520 & 511 & 505 & 501 & 498 & \\
420-720 & 624 & 601 & 584 & 567 & 558 & 550 & 542 & 533 & 524 & 515 & 508 & 504 & 501 & \\
420-740 & 638 & 612 & 593 & 575 & 565 & 557 & 548 & 538 & 528 & 518 & 511 & 506 & 503 & \\
420-760 & 651 & 623 & 603 & 583 & 573 & 563 & 553 & 543 & 532 & 522 & 514 & 509 & 505 & \\
420-780 & 664 & 634 & 612 & 591 & 579 & 569 & 559 & 547 & 536 & 525 & 516 & 511 & 507 & \\
420-800 & 677 & 644 & 621 & 598 & 586 & 575 & 563 & 799 & 799 & 799 & 799 & 799 & 799 & \\
440-540 & 502 & 497 & 494 & 491 & 490 & 489 & 487 & 486 & 484 & 483 & 482 & 481 & 480 & \\
440-560 & 515 & 509 & 505 & 501 & 499 & 498 & 496 & 494 & 492 & 490 & 488 & 487 & 486 & \\
440-580 & 529 & 521 & 516 & 511 & 509 & 507 & 504 & 502 & 499 & 497 & 494 & 493 & 492 & \\
440-600 & 543 & 534 & 527 & 521 & 518 & 515 & 512 & 509 & 506 & 503 & 500 & 498 & 497 & \\
440-620 & 557 & 546 & 538 & 531 & 527 & 524 & 520 & 516 & 513 & 508 & 505 & 503 & 502 & \\
440-640 & 571 & 558 & 549 & 540 & 536 & 532 & 528 & 523 & 519 & 514 & 510 & 508 & 506 & \\
440-660 & 585 & 570 & 559 & 549 & 544 & 540 & 535 & 530 & 524 & 519 & 515 & 512 & 510 & \\
440-680 & 599 & 582 & 570 & 558 & 553 & 547 & 542 & 536 & 530 & 524 & 519 & 516 & 514 & \\
440-700 & 613 & 593 & 580 & 567 & 561 & 554 & 548 & 542 & 535 & 528 & 523 & 519 & 517 & \\
440-720 & 626 & 605 & 590 & 576 & 568 & 561 & 555 & 547 & 540 & 532 & 526 & 522 & 520 & \\
440-740 & 640 & 616 & 600 & 584 & 576 & 568 & 561 & 552 & 544 & 536 & 530 & 525 & 523 & \\
440-760 & 653 & 627 & 609 & 592 & 583 & 575 & 566 & 557 & 549 & 539 & 533 & 528 & 525 & \\
440-780 & 666 & 638 & 619 & 600 & 590 & 581 & 572 & 562 & 553 & 543 & 535 & 530 & 527 & \\
440-800 & 678 & 648 & 627 & 607 & 596 & 587 & 577 & 799 & 799 & 799 & 799 & 799 & 799 & \\
460-560 & 520 & 516 & 513 & 511 & 509 & 508 & 507 & 506 & 504 & 503 & 502 & 501 & 501 & \\
460-580 & 534 & 528 & 524 & 521 & 519 & 517 & 516 & 514 & 512 & 510 & 509 & 508 & 507 & \\
460-600 & 547 & 540 & 535 & 530 & 528 & 526 & 524 & 522 & 519 & 517 & 515 & 513 & 513 & \\
460-620 & 561 & 552 & 546 & 540 & 537 & 535 & 532 & 529 & 526 & 523 & 521 & 519 & 518 & \\
460-640 & 575 & 564 & 556 & 549 & 546 & 543 & 540 & 536 & 533 & 529 & 526 & 524 & 523 & \\
460-660 & 588 & 575 & 567 & 559 & 555 & 551 & 547 & 543 & 539 & 534 & 531 & 529 & 527 & \\
460-680 & 602 & 587 & 577 & 568 & 563 & 559 & 554 & 549 & 545 & 539 & 536 & 533 & 531 & \\
460-700 & 616 & 599 & 587 & 576 & 571 & 566 & 561 & 555 & 550 & 544 & 540 & 537 & 535 & \\
460-720 & 629 & 610 & 597 & 585 & 579 & 573 & 567 & 561 & 555 & 549 & 544 & 540 & 538 & \\
460-740 & 642 & 621 & 607 & 593 & 586 & 580 & 574 & 567 & 560 & 553 & 547 & 544 & 541 & \\
460-760 & 655 & 632 & 616 & 601 & 594 & 587 & 580 & 572 & 565 & 557 & 551 & 547 & 544 & \\
460-780 & 668 & 643 & 626 & 609 & 601 & 593 & 585 & 577 & 569 & 560 & 554 & 550 & 547 & \\
460-800 & 681 & 653 & 634 & 616 & 607 & 599 & 591 & 799 & 799 & 799 & 799 & 799 & 799 & \\
480-580 & 539 & 535 & 533 & 530 & 529 & 528 & 527 & 526 & 524 & 523 & 522 & 521 & 521 & \\
480-600 & 552 & 547 & 543 & 540 & 538 & 537 & 535 & 534 & 532 & 530 & 529 & 528 & 527 & \\
480-620 & 566 & 558 & 554 & 550 & 547 & 546 & 544 & 541 & 539 & 537 & 535 & 534 & 533 & \\
480-640 & 579 & 570 & 564 & 559 & 556 & 554 & 552 & 549 & 546 & 543 & 541 & 540 & 539 & \\
480-660 & 592 & 582 & 575 & 568 & 565 & 562 & 559 & 556 & 553 & 549 & 547 & 545 & 544 & \\
480-680 & 606 & 593 & 585 & 577 & 574 & 570 & 567 & 563 & 559 & 555 & 552 & 549 & 548 & \\
480-700 & 619 & 604 & 595 & 586 & 582 & 578 & 574 & 569 & 565 & 560 & 556 & 554 & 552 & \\
480-720 & 632 & 616 & 605 & 595 & 590 & 585 & 580 & 575 & 570 & 565 & 561 & 558 & 556 & \\
480-740 & 645 & 627 & 615 & 603 & 597 & 592 & 587 & 581 & 575 & 569 & 565 & 562 & 560 & \\
480-760 & 658 & 638 & 624 & 611 & 605 & 599 & 593 & 587 & 580 & 574 & 569 & 565 & 563 & \\
480-780 & 671 & 648 & 633 & 619 & 612 & 605 & 599 & 592 & 585 & 578 & 572 & 568 & 566 & \\
480-800 & 683 & 658 & 642 & 626 & 619 & 612 & 604 & 799 & 799 & 799 & 799 & 799 & 799 & \\
500-600 & 558 & 554 & 552 & 550 & 549 & 548 & 547 & 546 & 544 & 543 & 542 & 542 & 541 & \\
500-620 & 571 & 566 & 562 & 559 & 558 & 557 & 555 & 554 & 552 & 550 & 549 & 548 & 548 & \\
500-640 & 584 & 577 & 573 & 569 & 567 & 565 & 563 & 561 & 559 & 557 & 556 & 554 & 554 & \\
500-660 & 597 & 588 & 583 & 578 & 576 & 573 & 571 & 569 & 566 & 564 & 562 & 560 & 559 & \\
500-680 & 610 & 600 & 593 & 587 & 584 & 582 & 579 & 576 & 573 & 570 & 567 & 565 & 564 & \\
500-700 & 623 & 611 & 603 & 596 & 593 & 589 & 586 & 583 & 579 & 575 & 572 & 570 & 569 & \\
500-720 & 636 & 622 & 613 & 605 & 601 & 597 & 593 & 589 & 585 & 581 & 577 & 575 & 573 & \\
500-740 & 649 & 633 & 623 & 613 & 609 & 604 & 600 & 595 & 590 & 585 & 582 & 579 & 577 & \\
500-760 & 662 & 644 & 632 & 621 & 616 & 611 & 606 & 601 & 596 & 590 & 586 & 583 & 581 & \\
500-780 & 675 & 655 & 642 & 629 & 623 & 618 & 613 & 606 & 601 & 595 & 590 & 587 & 585 & \\
500-800 & 687 & 665 & 650 & 637 & 630 & 624 & 618 & 799 & 799 & 799 & 799 & 799 & 799 & \\
520-620 & 577 & 574 & 571 & 569 & 568 & 567 & 567 & 566 & 565 & 563 & 563 & 562 & 562 & \\
520-640 & 590 & 585 & 582 & 579 & 578 & 576 & 575 & 574 & 572 & 571 & 570 & 569 & 568 & \\
520-660 & 603 & 596 & 592 & 588 & 586 & 585 & 583 & 581 & 580 & 578 & 576 & 575 & 574 & \\
520-680 & 615 & 607 & 602 & 597 & 595 & 593 & 591 & 589 & 586 & 584 & 582 & 581 & 580 & \\
520-700 & 628 & 618 & 612 & 606 & 604 & 601 & 599 & 596 & 593 & 590 & 588 & 586 & 585 & \\
520-720 & 641 & 629 & 622 & 615 & 612 & 609 & 606 & 602 & 599 & 596 & 593 & 591 & 590 & \\
520-740 & 653 & 640 & 631 & 624 & 620 & 616 & 613 & 609 & 605 & 601 & 598 & 596 & 594 & \\
520-760 & 666 & 651 & 641 & 632 & 628 & 624 & 619 & 615 & 611 & 606 & 603 & 600 & 599 & \\
520-780 & 679 & 661 & 650 & 640 & 635 & 630 & 626 & 621 & 616 & 611 & 607 & 604 & 602 & \\
520-800 & 691 & 671 & 659 & 648 & 642 & 637 & 632 & 799 & 799 & 799 & 799 & 799 & 799 & \\
540-640 & 597 & 593 & 591 & 589 & 588 & 587 & 586 & 586 & 585 & 584 & 583 & 582 & 582 & \\
540-660 & 609 & 604 & 601 & 599 & 597 & 596 & 595 & 594 & 592 & 591 & 590 & 589 & 589 & \\
540-680 & 621 & 615 & 611 & 608 & 606 & 605 & 603 & 601 & 600 & 598 & 596 & 595 & 595 & \\
540-700 & 634 & 626 & 621 & 617 & 615 & 613 & 611 & 609 & 607 & 604 & 603 & 601 & 600 & \\
540-720 & 646 & 637 & 631 & 626 & 623 & 621 & 618 & 616 & 613 & 610 & 608 & 607 & 606 & \\
\cline{1-14}
nm & M5
& M0
& K5
& K0
& G5
& G0
& F5
& F0
& A5
& A0
& B5
& B0
& O5 & \\
& $+1.61$
& $+1.39$
& $+1.11$
& $+0.84$
& $+0.70$
& $+0.57$
& $+0.45$
& $+0.30$
& $+0.16$
& $0.00$
& $-0.17$
& $-0.31$
& $-0.45$ & B$-$V \\
& $+1.19$
& $+1.24$
& $+1.06$
& $+0.46$
& $+0.20$
& $+0.04$
& $-0.01$
& $+0.02$
& $+0.09$
& $0.00$
& $-0.56$
& $-1.07$
& $-1.20$ & U$-$B \\
\cline{1-14}
\end{tabular}
\end{center}
\clearpage

\noindent
\begin{center}
\begin{tabular}{|c|c
@{\hspace{2ex}}c
@{\hspace{2ex}}c
@{\hspace{2ex}}c
@{\hspace{2ex}}c
@{\hspace{2ex}}c
@{\hspace{2ex}}c
@{\hspace{2ex}}c
@{\hspace{2ex}}c
@{\hspace{2ex}}c
@{\hspace{2ex}}c
@{\hspace{2ex}}c
@{\hspace{2ex}}c|l}
\cline{1-14}
Band
&  3000
&  3800
&  4500
&  5400
&  6000
&  6700
&  7600
&  9000
& 11100
& 15400
& 23000
& 38000
& 70000
& T$_c$ ($^\circ$K) \\
\cline{1-14}
540-740 & 659 & 647 & 641 & 634 & 631 & 628 & 626 & 622 & 619 & 616 & 614 & 612 & 611 & \\
540-760 & 671 & 658 & 650 & 643 & 639 & 636 & 633 & 629 & 625 & 622 & 619 & 617 & 615 & \\
540-780 & 683 & 669 & 659 & 651 & 647 & 643 & 639 & 635 & 631 & 627 & 623 & 621 & 620 & \\
540-800 & 695 & 678 & 668 & 658 & 654 & 650 & 645 & 799 & 799 & 799 & 799 & 799 & 799 & \\
560-660 & 616 & 613 & 611 & 609 & 608 & 607 & 606 & 605 & 605 & 604 & 603 & 602 & 602 & \\
560-680 & 628 & 623 & 621 & 618 & 617 & 616 & 615 & 614 & 612 & 611 & 610 & 609 & 609 & \\
560-700 & 640 & 634 & 630 & 627 & 626 & 624 & 623 & 621 & 620 & 618 & 617 & 616 & 615 & \\
560-720 & 652 & 645 & 640 & 636 & 634 & 632 & 631 & 629 & 627 & 625 & 623 & 622 & 621 & \\
560-740 & 664 & 655 & 650 & 645 & 643 & 640 & 638 & 636 & 633 & 631 & 629 & 627 & 626 & \\
560-760 & 677 & 666 & 659 & 654 & 651 & 648 & 645 & 642 & 640 & 637 & 634 & 633 & 632 & \\
560-780 & 689 & 676 & 669 & 662 & 658 & 655 & 652 & 649 & 646 & 642 & 639 & 638 & 636 & \\
560-800 & 700 & 686 & 677 & 670 & 666 & 662 & 659 & 799 & 799 & 799 & 799 & 799 & 799 & \\
580-680 & 635 & 632 & 630 & 629 & 628 & 627 & 626 & 625 & 625 & 624 & 623 & 623 & 622 & \\
580-700 & 647 & 643 & 640 & 638 & 637 & 636 & 635 & 634 & 632 & 631 & 630 & 630 & 629 & \\
580-720 & 659 & 653 & 650 & 647 & 645 & 644 & 643 & 641 & 640 & 638 & 637 & 636 & 636 & \\
580-740 & 671 & 664 & 660 & 656 & 654 & 652 & 651 & 649 & 647 & 645 & 643 & 642 & 642 & \\
580-760 & 683 & 674 & 669 & 664 & 662 & 660 & 658 & 656 & 654 & 651 & 649 & 648 & 647 & \\
580-780 & 695 & 685 & 678 & 673 & 670 & 668 & 665 & 663 & 660 & 657 & 655 & 653 & 652 & \\
580-800 & 706 & 694 & 687 & 681 & 678 & 675 & 672 & 799 & 799 & 799 & 799 & 799 & 799 & \\
600-700 & 655 & 652 & 650 & 648 & 648 & 647 & 646 & 645 & 645 & 644 & 643 & 643 & 643 & \\
600-720 & 666 & 662 & 660 & 658 & 657 & 656 & 655 & 654 & 653 & 651 & 651 & 650 & 649 & \\
600-740 & 678 & 673 & 669 & 667 & 665 & 664 & 663 & 661 & 660 & 658 & 657 & 657 & 656 & \\
600-760 & 690 & 683 & 679 & 675 & 674 & 672 & 670 & 669 & 667 & 665 & 664 & 663 & 662 & \\
600-780 & 701 & 693 & 688 & 684 & 682 & 680 & 678 & 676 & 674 & 672 & 670 & 669 & 668 & \\
600-800 & 712 & 703 & 697 & 692 & 689 & 687 & 685 & 799 & 799 & 799 & 799 & 799 & 799 & \\
620-720 & 674 & 671 & 670 & 668 & 668 & 667 & 666 & 666 & 665 & 664 & 663 & 663 & 663 & \\
620-740 & 685 & 682 & 679 & 677 & 676 & 675 & 675 & 674 & 673 & 672 & 671 & 670 & 670 & \\
620-760 & 697 & 692 & 689 & 686 & 685 & 684 & 683 & 681 & 680 & 679 & 678 & 677 & 676 & \\
620-780 & 708 & 702 & 698 & 695 & 693 & 692 & 690 & 689 & 687 & 685 & 684 & 683 & 683 & \\
620-800 & 719 & 712 & 707 & 703 & 701 & 699 & 698 & 799 & 799 & 799 & 799 & 799 & 799 & \\
640-740 & 694 & 691 & 689 & 688 & 687 & 687 & 686 & 686 & 685 & 684 & 684 & 683 & 683 & \\
640-760 & 705 & 701 & 699 & 697 & 696 & 695 & 695 & 694 & 693 & 692 & 691 & 690 & 690 & \\
640-780 & 716 & 711 & 708 & 706 & 705 & 704 & 703 & 701 & 700 & 699 & 698 & 697 & 697 & \\
640-800 & 727 & 721 & 717 & 714 & 713 & 711 & 710 & 799 & 799 & 799 & 799 & 799 & 799 & \\
660-760 & 713 & 711 & 709 & 708 & 707 & 707 & 706 & 706 & 705 & 704 & 704 & 703 & 703 & \\
660-780 & 724 & 721 & 719 & 717 & 716 & 715 & 715 & 714 & 713 & 712 & 711 & 711 & 710 & \\
660-800 & 735 & 730 & 728 & 725 & 724 & 723 & 722 & 799 & 799 & 799 & 799 & 799 & 799 & \\
680-780 & 733 & 730 & 729 & 728 & 727 & 727 & 726 & 726 & 725 & 724 & 724 & 724 & 723 & \\
680-800 & 743 & 740 & 738 & 736 & 736 & 735 & 734 & 799 & 799 & 799 & 799 & 799 & 799 & \\
700-800 & 752 & 750 & 748 & 747 & 747 & 746 & 746 & 799 & 799 & 799 & 799 & 799 & 799 & \\
\cline{1-14}
nm & M5
& M0
& K5
& K0
& G5
& G0
& F5
& F0
& A5
& A0
& B5
& B0
& O5 & \\
& $+1.61$
& $+1.39$
& $+1.11$
& $+0.84$
& $+0.70$
& $+0.57$
& $+0.45$
& $+0.30$
& $+0.16$
& $0.00$
& $-0.17$
& $-0.31$
& $-0.45$ & B$-$V \\
& $+1.19$
& $+1.24$
& $+1.06$
& $+0.46$
& $+0.20$
& $+0.04$
& $-0.01$
& $+0.02$
& $+0.09$
& $0.00$
& $-0.56$
& $-1.07$
& $-1.20$ & U$-$B \\
\cline{1-14}
\end{tabular}
\end{center}
\clearpage
\end{tiny}
%%%%%%%%%%%%%%%%%%%%%%%%%%%%%%%%%%%%%%%%%%%%

\newpage
\section{\xlabel{error_and_warning_messages}Error and Warning Messages}
\label{error_and_warning_messages}

Except where otherwise stated, the following messages appear on both
the command device (typically the terminal) and the report device
(typically the file to be printed).  The `Log:' information shows the
severity and message number which appear in the log file (if one was
specified; see Section~\ref{output_log}).

\begin{quote}
\texttt{Can't open file} \textit{nn}
\end{quote}

The above message may appear when ASTROM is started and
usually means either that the specified data file doesn't exist,
or that file protection problems prevent the report file from
being written.


\begin{quote}
\texttt{Please run ASTROM from the correct script!}\\
\texttt{ASTROM improperly invoked!} \\
\texttt{GETARG error!}
\end{quote}

These are examples of other messages which appear when ASTROM is
started and which depend on the particular type of computer
being used.  They indicate either software errors or some
other gross error.  Attempting to run the ASTROM
executable image directly rather than through the correct
command procedure is one possible cause.

\begin{quote}
\verb|^^^^^^  POSSIBLE DATA ERROR?  ^^^^^^|\\
(Log: error 1)
\end{quote}
This message is output to the command device only.  It means that
at least one
of the fields in the plate centre \radec\ was suspect --
unexpectedly negative, fractional when it should have been
integer, or outside the conventional range.

\begin{quote}
\texttt{Observation data were incomplete and will be ignored}\\
(Log: error 14)
\end{quote}
Many of the observation data may be omitted and have sensible
defaults, but certain combinations of omission make it impossible
for ASTROM to carry out a reduction in observed place.  When this
happens, the above message is output and a mean place reduction
is carried out.

\begin{quote}
\texttt{For the given observation data, the plate centre ZD is}
\textit{xxx.x}\ \texttt{degrees!}\\
\texttt{Reduction will be in MEAN place.}\\
(Log: warning 2)
\end{quote}
This indicates that the
zenith distance of the field centre, if the observation data are to
be believed, is greater than $90^{\circ}$.  The most likely causes
are an incorrectly specified time or observatory site.

\begin{quote}
\texttt{sla\_DS2TP status }\textit{n}\\
(Log: error 9)
\end{quote}
The above message
indicates something badly wrong with the reference star or plate
centre positions.

\goodbreak
\begin{quote}
\texttt{Impossible sla\_SVD error!}\\
(Log: error 9)
\end{quote}
If this message appears please ask your Starlink Site Manager
to contact the Starlink User Support group.

\begin{quote}
\texttt{sla\_SVD warning }\textit{n}\\
(Log: warning 3)
\end{quote}
Certain very rare combinations of (perfectly valid)
data can provoke the above message.
If it appears,
please check the rest of the output to see if it looks
reasonable.  If you
prefer, you can almost certainly make the condition disappear
by changing one of the data values \textit{very} slightly.
If the results look suspect, or if changing the
data fails to eliminate the problem, please
ask your Starlink Site Manager
to contact the Starlink User Support group.

\begin{quote}
\texttt{Fit was ill-conditioned!}\\
(Log: warning 4)
\end{quote}
This warns that the input data failed to
define all the fitted parameters adequately.  The cure is to
use more reference stars or not to fit the plate centre
and/or the distortion.

\begin{quote}
\texttt{Radial distortion coefficient cannot reliably be determined!}\\
(Log: warning 6)
\end{quote}
This message, which appears only on the report device, warns that
the data were of insufficient quality to allow the distortion
to be determined.  This can happen, for example, if the
distortion is very small or if there aren't enough reference stars.

\begin{quote}
\texttt{Plate centre cannot reliably be determined!}\\
(Log: warning 8)
\end{quote}
The above message, which appears only on the report device, warns that
the data were of insufficient quality to allow the plate
centre to be determined.  This can happen, for example, if the
radial distortion is very small or if there aren't enough reference stars.

\begin{quote}
\texttt{*************** INVALID DATA ****************}\\
(Log: error 12)
\end{quote}
This indicates that there is something fatally wrong with the
most recent input record.

\begin{quote}
\texttt{------------ PREMATURE END OF DATA ------------}\\
(Log: error 16)
\end{quote}
This indicates that the input file has ended, or that an
end-of-sequence record or an end-of-file record has been read at a time
when more input records are needed before a plate solution can be
attempted.

\begin{quote}
\texttt{****** PLATE EPOCH WAS NOT SPECIFIED ******}\\
(Log: error 17)
\end{quote}
This happens if no plate epoch is available.  This
information must be supplied, either as part of the plate data record or
in a time record.

\begin{quote}
\texttt{****** TOO MANY REFERENCE STARS; MAX }\textit{n}\ \texttt{******}\\
(Log: error 18)
\end{quote}
ASTROM internal workspace limits have been reached;  reduce the
number of reference stars.  If you have a legitimate requirement
for a larger number of reference stars than is currently allowed,
please ask your Starlink Site Manager to contact the Starlink
User Support Group.

\begin{quote}
\texttt{------------ FIT }\textit{n}\ \texttt{ABORTED ------------}
\end{quote}
This message always follows one of those described earlier,
which will indicate the source of the problem.

\begin{quote}
\texttt{---------- UNKNOWNS BUT NO SOLUTION ------------}\\
(Log: error 19)
\end{quote}
An unknown star record has been encountered before the records defining
a plate solution have been input.  Check the order of records in the
input file.

\begin{quote}
\texttt{FITS Filename <}\textit{filename}\texttt{> too long.  FITS writing abandoned}\\
(Log: error 10)
\end{quote}
The requested FITS header file name is too long (the maximum is
around~100 characters).

\begin{quote}
\texttt{FITS filename blank!?  FITS writing abandoned}\\
(Log: error 11)
\end{quote}
The requested FITS headef file name appears to be blank.

\begin{quote}
\texttt{Unrecognised FITS WCS style, xxx.  FITS writing abandoned}\\
(Log: error 15)
\end{quote}
The header style requested is not one of the supported ones.  No FITS
file will be written.

\begin{quote}
\texttt{FITS error detected: }\dots\\
(Log: error 13)
\end{quote}
The FITS system reported some error (reproduced in the message).  FITS
writing will be abandoned.

\end{document}


% ? End of main text
\end{document}
