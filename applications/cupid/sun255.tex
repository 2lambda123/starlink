\documentclass[twoside,11pt]{article}

% ? Specify used packages
\usepackage{graphicx}        %  Use this one for final production.
\usepackage[english]{babel}
% \usepackage[draft]{graphicx} %  Use this one for drafting.
% ? End of specify used packages

\pagestyle{myheadings}

% -----------------------------------------------------------------------------
% ? Document identification
\newcommand{\stardoccategory}  {Starlink User Note}
\newcommand{\stardocinitials}  {SUN}
\newcommand{\stardocsource}    {sun\stardocnumber}
\newcommand{\stardocnumber}    {255.1}
\newcommand{\stardocauthors}   {D.S. Berry}
\newcommand{\stardocdate}      {19th January 2006}
\newcommand{\stardoctitle}     {CUPID}
\newcommand{\stardoconeline}   {A Clump Identification and Analysis Package}
\newcommand{\stardocversion}   {Version 0.0-1}
\newcommand{\stardocmanual}    {Users' Manual}
\newcommand{\stardocabstract}  {CUPID is a package of the identification
and analysis of clumps of emission within 2- and 3- dimensional 
data arrays.}

% ? End of document identification
% -----------------------------------------------------------------------------

% +
%  Name:
%     sun.tex
%
%  Purpose:
%     Template for Starlink User Note (SUN) documents.
%     Refer to SUN/199
%
%  Authors:
%     AJC: A.J.Chipperfield (Starlink, RAL)
%     BLY: M.J.Bly (Starlink, RAL)
%     PWD: Peter W. Draper (Starlink, Durham University)
%
%  History:
%     17-JAN-1996 (AJC):
%        Original with hypertext macros, based on MDL plain originals.
%     16-JUN-1997 (BLY):
%        Adapted for LaTeX2e.
%        Added picture commands.
%     13-AUG-1998 (PWD):
%        Converted for use with LaTeX2HTML version 98.2 and
%        Star2HTML version 1.3.
%     {Add further history here}
%
% -

\newcommand{\stardocname}{\stardocinitials /\stardocnumber}
\markboth{\stardocname}{\stardocname}
\setlength{\textwidth}{160mm}
\setlength{\textheight}{230mm}
\setlength{\topmargin}{-2mm}
\setlength{\oddsidemargin}{0mm}
\setlength{\evensidemargin}{0mm}
\setlength{\parindent}{0mm}
\setlength{\parskip}{\medskipamount}
\setlength{\unitlength}{1mm}

% -----------------------------------------------------------------------------
%  Hypertext definitions.
%  ======================
%  These are used by the LaTeX2HTML translator in conjunction with star2html.

%  Comment.sty: version 2.0, 19 June 1992
%  Selectively in/exclude pieces of text.
%
%  Author
%    Victor Eijkhout                                      <eijkhout@cs.utk.edu>
%    Department of Computer Science
%    University Tennessee at Knoxville
%    104 Ayres Hall
%    Knoxville, TN 37996
%    USA

%  Do not remove the %begin{latexonly} and %end{latexonly} lines (used by 
%  LaTeX2HTML to signify text it shouldn't process).
%begin{latexonly}
\makeatletter
\def\makeinnocent#1{\catcode`#1=12 }
\def\csarg#1#2{\expandafter#1\csname#2\endcsname}

\def\ThrowAwayComment#1{\begingroup
    \def\CurrentComment{#1}%
    \let\do\makeinnocent \dospecials
    \makeinnocent\^^L% and whatever other special cases
    \endlinechar`\^^M \catcode`\^^M=12 \xComment}
{\catcode`\^^M=12 \endlinechar=-1 %
 \gdef\xComment#1^^M{\def\test{#1}
      \csarg\ifx{PlainEnd\CurrentComment Test}\test
          \let\html@next\endgroup
      \else \csarg\ifx{LaLaEnd\CurrentComment Test}\test
            \edef\html@next{\endgroup\noexpand\end{\CurrentComment}}
      \else \let\html@next\xComment
      \fi \fi \html@next}
}
\makeatother

\def\includecomment
 #1{\expandafter\def\csname#1\endcsname{}%
    \expandafter\def\csname end#1\endcsname{}}
\def\excludecomment
 #1{\expandafter\def\csname#1\endcsname{\ThrowAwayComment{#1}}%
    {\escapechar=-1\relax
     \csarg\xdef{PlainEnd#1Test}{\string\\end#1}%
     \csarg\xdef{LaLaEnd#1Test}{\string\\end\string\{#1\string\}}%
    }}

%  Define environments that ignore their contents.
\excludecomment{comment}
\excludecomment{rawhtml}
\excludecomment{htmlonly}

%  Hypertext commands etc. This is a condensed version of the html.sty
%  file supplied with LaTeX2HTML by: Nikos Drakos <nikos@cbl.leeds.ac.uk> &
%  Jelle van Zeijl <jvzeijl@isou17.estec.esa.nl>. The LaTeX2HTML documentation
%  should be consulted about all commands (and the environments defined above)
%  except \xref and \xlabel which are Starlink specific.

\newcommand{\htmladdnormallinkfoot}[2]{#1\footnote{#2}}
\newcommand{\htmladdnormallink}[2]{#1}
\newcommand{\htmladdimg}[1]{}
\newcommand{\hyperref}[4]{#2\ref{#4}#3}
\newcommand{\htmlref}[2]{#1}
\newcommand{\htmlimage}[1]{}
\newcommand{\htmladdtonavigation}[1]{}

\newenvironment{latexonly}{}{}
\newcommand{\latex}[1]{#1}
\newcommand{\html}[1]{}
\newcommand{\latexhtml}[2]{#1}
\newcommand{\HTMLcode}[2][]{}

%  Starlink cross-references and labels.
\newcommand{\xref}[3]{#1}
\newcommand{\xlabel}[1]{}

%  LaTeX2HTML symbol.
\newcommand{\latextohtml}{\LaTeX2\texttt{HTML}}

%  Define command to re-centre underscore for Latex and leave as normal
%  for HTML (severe problems with \_ in tabbing environments and \_\_
%  generally otherwise).
\renewcommand{\_}{\texttt{\symbol{95}}}

% -----------------------------------------------------------------------------
%  Debugging.
%  =========
%  Remove % on the following to debug links in the HTML version using Latex.

% \newcommand{\hotlink}[2]{\fbox{\begin{tabular}[t]{@{}c@{}}#1\\\hline{\footnotesize #2}\end{tabular}}}
% \renewcommand{\htmladdnormallinkfoot}[2]{\hotlink{#1}{#2}}
% \renewcommand{\htmladdnormallink}[2]{\hotlink{#1}{#2}}
% \renewcommand{\hyperref}[4]{\hotlink{#1}{\S\ref{#4}}}
% \renewcommand{\htmlref}[2]{\hotlink{#1}{\S\ref{#2}}}
% \renewcommand{\xref}[3]{\hotlink{#1}{#2 -- #3}}
%end{latexonly}
% -----------------------------------------------------------------------------
% ? Document specific \newcommand or \newenvironment commands.


% Includes a gif version of a figure, centred within a table, with a caption.
\newcommand{\htmlfig}[3]{
   \label{#1}
   \begin{rawhtml} <CENTER><TABLE NOSAVE > \end{rawhtml}
   \begin{rawhtml} <TR ALIGN=CENTER VALIGN=CENTER NOSAVE> \end{rawhtml}
   \begin{rawhtml} <TD NOSAVE><DT><IMG SRC=" \end{rawhtml}
   #2
   \begin{rawhtml} " NOSAVE ></DT></TD></TR> \end{rawhtml}
   \begin{rawhtml} <CAPTION ALIGN=BOTTOM><FONT SIZE=+1><BR><BR><B> \end{rawhtml}
   #3 
   \begin{rawhtml} </B><BR><BR><BR><BR></FONT></CAPTION></TABLE></CENTER> \end{rawhtml}
}

% centre an asterisk
\newcommand{\lsk}{\raisebox{-0.4ex}{\rm *}}
\begin{htmlonly}
\renewcommand{\lsk}{*} 
\end{htmlonly}

% Environment for indenting and using a small font.
\newenvironment{myquote}{\begin{quote}\begin{small}}{\end{small}\end{quote}}

% In-line typed text, buttons and menu items.
\newcommand{\butt}[1]{{\small \bf \tt #1}}
\newcommand{\menu}[1]{{\small \bf \em #1}}
\newcommand{\wlab}[1]{{\small \bf #1}}
\newcommand{\text}[1]{{\small \tt #1}}

% Quick routine descriptions
\newcommand{\quickdes}[3]{
                         \parbox{1.1in}{\bf #1}
                         \parbox{4.4in}{\raggedright #2 \dotfill}
                         \parbox{0.6in}{\pageref{#3}}
                         \vspace*{0.2in}}

% Quotes for SST.
\newcommand{\qt}[1]{{\tt "}#1{\tt "}}
\newcommand{\qs}[1]{{\tt '}#1{\tt '}}
\begin{htmlonly}
   \renewcommand{\qt}[1]{ {\tt{"}}#1{\tt{"}} }
   \renewcommand{\qs}[1]{ {\tt{'}}#1{\tt{'}} }
\end{htmlonly}

% Routines with descriptions in the appendix.
\newcommand{\routine}[1]{{\sc #1}}
\newcommand{\xroutine}[1]{\htmlref{{\sc #1}}{#1}}

% Latex only sections, subsections etc. Surround these with a latexonly
% environment.
\newcommand{\latexonlysection}[1]{\section{#1}}
\newcommand{\latexonlysubsection}[1]{\subsection{#1}}
\newcommand{\latexonlysubsubsection}[1]{\subsubsection{#1}}
\begin{htmlonly}
   \newcommand{\latexonlysection}[1]{#1}
   \newcommand{\latexonlysubsection}[1]{#1}
   \newcommand{\latexonlysubsubsection}[1]{#1}
\end{htmlonly}

% degrees symbol
\newcommand{\dgs}{\hbox{$^\circ$}} 
\begin{htmlonly}
\renewcommand{\dgs}{ degrees} 
\end{htmlonly}

\newcommand{\dash}{--}
\begin{htmlonly}
\renewcommand{\dash}{-}
\end{htmlonly}
\newcommand{\STARURL}{http://www.starlink.rl.ac.uk}
\newcommand{\TCLURL}{http://www.scriptics.com/}
\newcommand{\IRAFURL}{http://www.starlink.rl.ac.uk/iraf/web/iraf-homepage.html}
\newcommand{\STORE}{http://www.starlink.rl.ac.uk/cgi-bin/storetop}

%+
%  Name:
%     SST.TEX

%  Purpose:
%     Define LaTeX commands for laying out Starlink routine descriptions.

%  Language:
%     LaTeX

%  Type of Module:
%     LaTeX data file.

%  Description:
%     This file defines LaTeX commands which allow routine documentation
%     produced by the SST application PROLAT to be processed by LaTeX and
%     by LaTeX2html. The contents of this file should be included in the
%     source prior to any statements that make of the sst commnds.

%  Notes:
%     The style file html.sty provided with LaTeX2html needs to be used.
%     This must be before this file.

%  Authors:
%     RFWS: R.F. Warren-Smith (STARLINK)
%     PDRAPER: P.W. Draper (Starlink - Durham University)

%  History:
%     10-SEP-1990 (RFWS):
%        Original version.
%     10-SEP-1990 (RFWS):
%        Added the implementation status section.
%     12-SEP-1990 (RFWS):
%        Added support for the usage section and adjusted various spacings.
%     8-DEC-1994 (PDRAPER):
%        Added support for simplified formatting using LaTeX2html.
%     {enter_further_changes_here}

%  Bugs:
%     {note_any_bugs_here}

%-

%  Define length variables.
\newlength{\sstbannerlength}
\newlength{\sstcaptionlength}
\newlength{\sstexampleslength}
\newlength{\sstexampleswidth}

%  Define a \tt font of the required size.
\latex{\newfont{\ssttt}{cmtt10 scaled 1095}}
\html{\newcommand{\ssttt}{\tt}}

%  Define a command to produce a routine header, including its name,
%  a purpose description and the rest of the routine's documentation.
\newcommand{\sstroutine}[3]{
   \newpage
   \goodbreak
   \label{#1}
   \markboth{{\stardocname}~ --- #1}{{\stardocname}~ --- #1}
   \rule{\textwidth}{0.5mm}
   \vspace{-7ex}
   \newline
   \settowidth{\sstbannerlength}{{\Large {\bf #1}}}
   \setlength{\sstcaptionlength}{\textwidth}
   \setlength{\sstexampleslength}{\textwidth}
   \addtolength{\sstbannerlength}{0.5em}
   \addtolength{\sstcaptionlength}{-2.0\sstbannerlength}
   \addtolength{\sstcaptionlength}{-5.0pt}
   \settowidth{\sstexampleswidth}{{\bf Examples:}}
   \addtolength{\sstexampleslength}{-\sstexampleswidth}
   \parbox[t]{\sstbannerlength}{\flushleft{\Large {\bf #1}}}
   \parbox[t]{\sstcaptionlength}{\center{\Large #2}}
   \parbox[t]{\sstbannerlength}{\flushright{\Large {\bf #1}}}
   \begin{description}
      #3
   \end{description}
}

%  Format the description section.
\newcommand{\sstdescription}[1]{\item[Description:] #1}

%  Format the usage section.
\newcommand{\sstusage}[1]{\item[Usage:] \mbox{}
\\[1.3ex]{\raggedright \ssttt #1}}

%  Format the invocation section.
\newcommand{\sstinvocation}[1]{\item[Invocation:]\hspace{0.4em}{\tt #1}}

%  Format the arguments section.
\newcommand{\sstarguments}[1]{
   \item[Arguments:] \mbox{} \\
   \vspace{-3.5ex}
   \begin{description}
      #1
   \end{description}
}

%  Format the returned value section (for a function).
\newcommand{\sstreturnedvalue}[1]{
   \item[Returned Value:] \mbox{} \\
   \vspace{-3.5ex}
   \begin{description}
      #1
   \end{description}
}

%  Format the parameters section (for an application).
\newcommand{\sstparameters}[1]{
   \item[Parameters:] \mbox{} \\
   \vspace{-3.5ex}
   \begin{description}
      #1
   \end{description}
}

%  Format the examples section.
\newcommand{\sstexamples}[1]{
   \item[Examples:] \mbox{} \\
   \vspace{-3.5ex}
   \begin{description}
      #1
   \end{description}
}

%  Define the format of a subsection in a normal section.
\newcommand{\sstsubsection}[1]{ \item[{#1}] \mbox{} \\}

%  Define the format of a subsection in the examples section.
\newcommand{\sstexamplesubsection}[2]{\sloppy
\item[\parbox{\sstexampleslength}{\ssttt #1}] \mbox{} \vspace{1.0ex}
\\ #2 }

%  Format the notes section.
\newcommand{\sstnotes}[1]{\item[Notes:] \mbox{} \\[1.3ex] #1}

%  Provide a general-purpose format for additional (DIY) sections.
\newcommand{\sstdiytopic}[2]{\item[{\hspace{-0.35em}#1\hspace{-0.35em}:}]
\mbox{} \\[1.3ex] #2}

%  Format the implementation status section.
\newcommand{\sstimplementationstatus}[1]{
   \item[{Implementation Status:}] \mbox{} \\[1.3ex] #1}

%  Format the bugs section.
\newcommand{\sstbugs}[1]{\item[Bugs:] #1}

%  Format a list of items while in paragraph mode.
\newcommand{\sstitemlist}[1]{
  \mbox{} \\
  \vspace{-3.5ex}
  \begin{itemize}
     #1
  \end{itemize}
}

%  Define the format of an item.
\newcommand{\sstitem}{\item}

%% Now define html equivalents of those already set. These are used by
%  latex2html and are defined in the html.sty files.
\begin{htmlonly}

%  sstroutine.
   \newcommand{\sstroutine}[3]{
      \subsection{#1\xlabel{#1}-\label{#1}#2}
      \begin{description}
         #3
      \end{description}
   }

%  sstdescription
   \newcommand{\sstdescription}[1]{\item[Description:]
      \begin{description}
         #1
      \end{description}
      \\
   }

%  sstusage
   \newcommand{\sstusage}[1]{\item[Usage:]
      \begin{description}
         {\ssttt #1}
      \end{description}
      \\
   }

%  sstinvocation
   \newcommand{\sstinvocation}[1]{\item[Invocation:]
      \begin{description}
         {\ssttt #1}
      \end{description}
      \\
   }

%  sstarguments
   \newcommand{\sstarguments}[1]{
      \item[Arguments:] \\
      \begin{description}
         #1
      \end{description}
      \\
   }

%  sstreturnedvalue
   \newcommand{\sstreturnedvalue}[1]{
      \item[Returned Value:] \\
      \begin{description}
         #1
      \end{description}
      \\
   }

%  sstparameters
   \newcommand{\sstparameters}[1]{
      \item[Parameters:] \\
      \begin{description}
         #1
      \end{description}
      \\
   }

%  sstexamples
   \newcommand{\sstexamples}[1]{
      \item[Examples:] \\
      \begin{description}
         #1
      \end{description}
      \\
   }

%  sstsubsection
   \newcommand{\sstsubsection}[1]{\item[{#1}]}

%  sstexamplesubsection
   \newcommand{\sstexamplesubsection}[2]{\item[{\ssttt #1}] #2}

%  sstnotes
   \newcommand{\sstnotes}[1]{\item[Notes:] #1 }

%  sstdiytopic
   \newcommand{\sstdiytopic}[2]{\item[{#1}] #2 }

%  sstimplementationstatus
   \newcommand{\sstimplementationstatus}[1]{
      \item[Implementation Status:] #1
   }

%  sstitemlist
   \newcommand{\sstitemlist}[1]{
      \begin{itemize}
         #1
      \end{itemize}
      \\
   }
%  sstitem
   \newcommand{\sstitem}{\item}

\end{htmlonly}

%  End of "sst.tex" layout definitions.
%.



% ? End of document specific commands
% -----------------------------------------------------------------------------
%  Title Page.
%  ===========
\renewcommand{\thepage}{\roman{page}}
\begin{document}
\thispagestyle{empty}

%  Latex document header.
%  ======================
\begin{latexonly}
   CCLRC / \textsc{Rutherford Appleton Laboratory} \hfill \textbf{\stardocname}\\
   {\large Particle Physics \& Astronomy Research Council}\\
   {\large Starlink Project\\}
   {\large \stardoccategory\ \stardocnumber}
   \begin{flushright}
   \stardocauthors\\
   \stardocdate
   \end{flushright}
   \vspace{-4mm}
   \rule{\textwidth}{0.5mm}
   \vspace{5mm}
   \begin{center}
   {\Huge\textbf{\stardoctitle \\ [2.5ex]}}
   {\LARGE\textbf{\stardocversion \\ [4ex]}}
   {\Huge\textbf{\stardocmanual}}
   \end{center}
   \vspace{5mm}

% ? Add picture here if required for the LaTeX version.
% ? End of picture

% ? Heading for abstract if used.
   \vspace{5mm}
   \begin{center}
      {\Large\textbf{Abstract}}
   \end{center}
% ? End of heading for abstract.
\end{latexonly}

%  HTML documentation header.
%  ==========================
\begin{htmlonly}
   \xlabel{}
   \begin{rawhtml} <H1 ALIGN=CENTER> \end{rawhtml}
      \stardoctitle\\
      \stardocversion\\
      \stardocmanual
   \begin{rawhtml} </H1> <HR> \end{rawhtml}

   \begin{rawhtml} <P> <I> \end{rawhtml}
   \stardoccategory\ \stardocnumber \\
   \stardocauthors \\
   \stardocdate
   \begin{rawhtml} </I> </P> <H3> \end{rawhtml}
      \htmladdnormallink{CCLRC}{http://www.cclrc.ac.uk} /
      \htmladdnormallink{Rutherford Appleton Laboratory}
                        {http://www.cclrc.ac.uk/ral} \\
      \htmladdnormallink{Particle Physics \& Astronomy Research Council}
                        {http://www.pparc.ac.uk} \\
   \begin{rawhtml} </H3> <H2> \end{rawhtml}
      \htmladdnormallink{Starlink Project}{http://www.starlink.rl.ac.uk/}
   \begin{rawhtml} </H2> \end{rawhtml}
   \htmladdnormallink{\htmladdimg{source.gif} Retrieve hardcopy}
      {http://www.starlink.rl.ac.uk/cgi-bin/hcserver?\stardocsource}\\

%  HTML document table of contents. 
%  ================================
%  Add table of contents header and a navigation button to return to this 
%  point in the document (this should always go before the abstract \section). 
  \label{stardoccontents}
  \begin{rawhtml} 
    <HR>
    <H2>Contents</H2>
  \end{rawhtml}
  \htmladdtonavigation{\htmlref{\htmladdimg{contents_motif.gif}}
        {stardoccontents}}

% ? New section for abstract if used.
  \section{\xlabel{abstract}Abstract}
% ? End of new section for abstract
\end{htmlonly}

% -----------------------------------------------------------------------------
% ? Document Abstract. (if used)
%  ==================
\stardocabstract

% ? End of document abstract
% -----------------------------------------------------------------------------
% ? Latex document Table of Contents (if used).
%  ===========================================
  \newpage
  \begin{latexonly}
    \setlength{\parskip}{0mm}
    \tableofcontents
    \setlength{\parskip}{\medskipamount}
    \markboth{\stardocname}{\stardocname}
  \end{latexonly}
% ? End of Latex document table of contents
% -----------------------------------------------------------------------------
\cleardoublepage
\renewcommand{\thepage}{\arabic{page}}
\setcounter{page}{1}

\section{\xlabel{introduction}Introduction}


\newpage
\appendix
\section{ \label{APP:DESCRIPTION}Description of the CUPID applications}
\begin{latexonly}
\latexonlysubsection{Alphabetic list of CUPID routines.}
%
% set up a mini table of contents for this section pointing into next section.
%
\quickdes{CLUMPS}{Identify clumps of emission within a 1, 2 or 3 dimensional NDF.}{ CLUMPS }
\quickdes{CUPIDHELP}{Display information about CUPID.}{ CUPIDHELP }
\quickdes{MAKECLUMPS}{Create simulated data containing clumps and noise.}{ MAKECLUMPS }

\end{latexonly}

\subsection{Complete routine descriptions \label{descriptions}}

The CUPID routine descriptions are contained in the following pages.
These descriptions follow the style used in \xref{SUN/95}{sun95}{ap_full}
for NDF applications.

\newpage
\sstroutine{
   CLUMPS
}{
   Identify clumps of emission within a 1, 2 or 3 dimensional NDF
}{
   \sstdescription{
      This application identifies clumps of emission within a 1, 2 or 3
      dimensional NDF. Information about the clumps is returned in
      several different ways:

      \sstitemlist{

         \sstitem
         A pixel mask identifying pixels as background pixels or clump
         pixels can be written to the Quality array of the input NDF (see
         parameter MASK).

         \sstitem
         An output catalogue containing clump parameters can be created (see
         parameter OUTCAT).

         \sstitem
         Information about each clump, including a minimal cut-out image
         of the clump and the clump parameters, is written to the CUPID
         extension of the input NDF (see the section {\tt "}Use of CUPID Extension{\tt "}
         below).

      }
      The algorithm used to identify the clumps (GaussCLumps, ClumpFind,
      etc) can be specified (see parameter METHOD).
   }
   \sstusage{
      clumps in outcat method mask
   }
   \sstparameters{
      \sstsubsection{
         CONFIG = GROUP (Read)
      }{
         Specifies values for the configuration parameters used by the
         clump finding algorithms. If the string {\tt "}def{\tt "} (case-insensitive)
         or a null (!) value is supplied, a set of default configuration
         parameter values will be used.

         A comma-separated list of strings should be given in which each
         string is either a {\tt "}keyword=value{\tt "} setting, or the name of a text
         file preceded by an up-arrow character {\tt "}$\wedge${\tt "}. Such text files should
         contain further comma-separated lists which will be read and
         interpreted in the same manner (any blank lines or lines beginning
         with {\tt "}\#{\tt "}). Settings are applied in the order in which they occur
         within the list, with later settings over-riding any earlier
         settings given for the same keyword.

         Each individual setting should be of the form:

            $<$keyword$>$=$<$value$>$

         where $<$keyword$>$ has the form {\tt "}algorithm.param{\tt "}; that is, the name
         of the algorithm, followed by a dot, followed by the name of the
         parameter to be set. If the algorithm name is omitted, the current
         algorithm given by parameter METHOD is assumed. The parameters
         available for each algorithm are listed in the {\tt "}Configuration
         Parameters{\tt "} sections below. Default values will be used for any
         unspecified parameters. Unrecognised options are ignored (that is,
         no error is reported). [current value]
      }
      \sstsubsection{
         ILEVEL = \_INTEGER (Read)
      }{
         Controls the amount of diagnostic information reported. It
         should be in the range 0 to 6. A value of zero will suppress all
         screen output. Larger values give more information (the precise
         information displayed depends on the algorithm being used). [1]
      }
      \sstsubsection{
         IN = NDF (Update)
      }{
         The 1, 2 or 3 dimensional NDF to be analysed. Information about
         the identified clumps and the configuration parameters used will
         be stored in the CUPID extension of the supplied NDF, and so the
         NDF must not be write protected. See {\tt "}Use of CUPID Extension{\tt "}
         below for further details about the information stored in the CUPID
         extension. Other applications within the CUPID package can be used
         to display this information in various ways.
      }
      \sstsubsection{
         MASK = \_LOGICAL (Read)
      }{
         If true, then a Quality component is added to the supplied NDF
         (replacing any existing Quality component) indicating if each pixel
         is inside or outside a clump. Two quality bits will be used; one is
         set if and only if the pixel is contained within one or more clumps,
         the other is set if and only if the pixel is not contained within
         any clump. These two quality bits have names associated with
         them which can be used with the KAPPA applications SETQUAL,
         QUALTOBAD, REMQUAL, SHOWQUAL. The names used are {\tt "}CLUMP{\tt "} and
         {\tt "}BACKGROUND{\tt "}. [current value]
      }
      \sstsubsection{
         METHOD = LITERAL (Read)
      }{
         The algorithm to use. Each algorithm is described in more detail
         in the {\tt "}Algorithms:{\tt "} section below. Can be one of:

         \sstitemlist{

            \sstitem
            GaussClumps

            \sstitem
            ClumpFind

            \sstitem
            Kimberley

         }
         Each algorithm has a collection of extra tuning values which are
         set via the CONFIG parameter.   [current value]
      }
      \sstsubsection{
         OUT = NDF (Write)
      }{
         An optional output NDF which has the same shape and size as the
         input NDF. The information written to this NDF depends on the value
         of the METHOD parameter. If METHOD is GaussClumps, the output NDF
         receives the sum of all the fitted Gaussian clump models including
         a global background level chosen to make the mean output value
         equal to the mean input value. If METHOD is ClumpFind or Kimberley,
         each pixel in
         the output is the integer index of clump to which the pixel has been
         assigned. Bad values are stored for pixels which are not part of
         any clump. No output NDF will be produced if a null (!) value is
         supplied. Otherwise, the output NDF will inherit the AXIS, WCS and
         QUALITY components (plus any extensions) from the input NDF. [!]
      }
      \sstsubsection{
         OUTCAT = FILENAME (Write)
      }{
         An optional output catalogue in which to store the clump parameters.
         No catalogue will be produced if a null (!) value is supplied.
         The central positons of all clumps in the catalogue fred.FIT can be
         overlayed on a displayed image of the input NDF using the command
         {\tt "}listmake fred plot=mark{\tt "}. The following columns are stored in
         the catalogue:

         \sstitemlist{

            \sstitem
            PeakX: The PIXEL X coordinates of the clump peak value.

            \sstitem
            PeakY: The PIXEL Y coordinates of the clump peak value.

            \sstitem
            PeakZ: The PIXEL Z coordinates of the clump peak value.

            \sstitem
            CenX: The PIXEL X coordinates of the clump centroid.

            \sstitem
            CenY: The PIXEL Y coordinates of the clump centroid.

            \sstitem
            CenZ: The PIXEL Z coordinates of the clump centroid.

            \sstitem
            SizeX: The size of the clump along the X axis, in pixels.

            \sstitem
            SizeY: The size of the clump along the Y axis, in pixels.

            \sstitem
            SizeZ: The size of the clump along the Z axis, in pixels.

            \sstitem
            Sum: The total data sum in the clump.

            \sstitem
            Peak: The peak value in the clump.

            \sstitem
            Area: The total number of pixels falling within the clump.

         }
         If the data has less than 3 pixel axes, then the columns
         describing the missing axes will not be present in the catalogue.

         The {\tt "}size{\tt "} of the clump on an axis is the RMS deviation of each
         pixel centre from the clump centroid, where each pixel is
         weighted by the correspinding pixel data value.

         For the GaussClump algorithm, the Sum and Area values refer
         to the part of the Gaussian within the level defined by the
         GaussClump.ModelLim configuration parameter. [!]
      }
      \sstsubsection{
         RMS = \_DOUBLE (Read)
      }{
         Specifies a value to use as the global RMS noise level in the
         supplied data array. The suggested defaukt value is the square root
         of the mean of the values in the input NDF{\tt '}s Variance component is
         used. If the NDF has no Variance component, the suggested default
         is based on the differences between neighbouring pixel values. Any
         pixel-to-pixel correlation in the noise can result in this estimate
         being too low.
      }
   }
   \sstdiytopic{
      Use of CUPID Extension
   }{
      This application will create an NDF extension called {\tt "}CUPID{\tt "} in the
      input NDF (unless there is already one there), and add the following
      components to it, erasing any of the same name which already exist:

      \sstitemlist{

         \sstitem
         CLUMPS: This a an array of CLUMP structures, one for each clump
         identified by the selected algorithm. Each such structure contains
         the same clump parameters that are written to the catalogue via
         parameter OUTCAT. It also contains a component called MODEL which
         is an NDF containing a section of the main input NDF which is just
         large enough to encompass the clump. Any pixels within this section
         which are not contained within the clump are set bad. So for instance,
         if the input array {\tt "}fred.sdf{\tt "} is 2-dimensional, and an image of it has
         been displayed using KAPPA:DISPLAY, then the outline of clump number 9
         (say) can be overlayed on the image by doing

      }
      contour noclear {\tt "}fred.more.cupid.clumps(9).model{\tt "} mode=good

      \sstitemlist{

         \sstitem
         CONFIG: Lists the algorithm configuration parameters used to
         identify the clumps (see parameter CONFIG).

         \sstitem
         QUALITY\_NAMES: Defines the textual names used to identify background
         and clump pixels within the Quality mask. See parameter MASK.
      }
   }
   \sstdiytopic{
      Algorithms
   }{
      \sstitemlist{

         \sstitem
         GaussClumps: Based on the algorithm described by Stutski \& Gusten
         (1990, ApJ 356, 513). This algorithm proceeds by fitting a Gaussian
         profile to the brightest peak in the data. It then subtracts the fit
         from the data and iterates, fitting a new ellipse to the brightest peak
         in the residuals. This continues until a series of consecutive fits
         are made which have peak values below a given multiple of the noise
         level. Each fitted ellipse is taken to be a single clump and is added
         to the output catalogue. In this algorithm, clumps may overlap. Any
         input variance component is used to scale the weight associated with
         each pixel value when performing the Gaussian fit. The most significant
         configuration parameters for this algorithm are: GaussClumps.FwhmBeam
         and GaussClumps.VeloRes which determine the minimum clump size, and
         GaussClumps.Thresh which (together with the ADAM paramater RMS)
         determine the termination criterion.

         \sstitem
         ClumpFind: Described by Williams et al (1994, ApJ 428, 693). This
         algorithm works by first contouring the data at a multiple of the
         noise, then searches for peaks of emission which locate the clumps,
         and then follows them down to lower intensities. No a priori clump
         profile is assumed. In this algorithm, clumps never overlap. Clumps
         which touch an edge of the data array are not included in the final
         list of clumps.

         \sstitem
         Kimberley: Based on an algorithm developed by Kim Reinhold at JAC.
      }
   }
   \sstdiytopic{
      GaussClumps Configuration Parameters
   }{
      The GaussClumps algorithm uses the following configuration parameters.
      Values for these parameters can be specified using the CONFIG parameter.
      Default values are shown in square brackets:

      \sstitemlist{

         \sstitem
         GaussClumps.FwhmBeam: The FWHM of the instrument beam, in
         pixels. The fitted Gaussians are not allowed to be smaller than the
         instrument beam. This prevents noise spikes being fitted. [3.0]

         \sstitem
         GaussClumps.FwhmStart: An initial guess at the ratio of the typical
         observed clump size to the instrument beam width. This is used to
         determine the starting point for the algorithm which finds the best
         fitting Gaussian for each clump. If no value is supplied, the
         initial guess at the clump size is based on the local profile
         around the pixel with peak value. []

         \sstitem
         GaussClumps.MaxClumps: Specifies a termination criterion for
         the GaussClumps algorithm. The algorithm will terminate when
         {\tt "}MaxClumps{\tt "} clumps have been identified, or when one of the other
         termination criteria is met. [unlimited]

         \sstitem
         GaussClumps.MaxNF: The maximum number of evaluations of the
         objective function allowed when fitting an individual clump. Here,
         the objective function evaluates the chi-squared between the
         current gaussian model and the data being fitted. [100]

         \sstitem
         GaussClumps.MaxSkip: The maximum number of consecutive failures
         which are allowed when fitting Guassians. If more than {\tt "}MaxSkip{\tt "}
         consecutive clumps cannot be fitted, the iterative fitting
         process is terminated. [10]

         \sstitem
         GaussClumps.ModelLim: Determines the value at which each Gaussian
         model is truncated to zero. Model values below ModelLim times the RMS
         noise are treated as zero. [3.0]

         \sstitem
         GaussClumps.NPad: Specifies a termination criterion for the
         GaussClumps algorithm. The algorithm will terminate when {\tt "}Npad{\tt "}
         consecutive clumps have been fitted all of which have peak values less
         than the threshold value specified by the {\tt "}Thresh{\tt "} parameter. [10]

         \sstitem
         GaussClumps.S0: The Chi-square stiffness parameter {\tt "}S0{\tt "} which
         encourages the peak amplitude of each fitted gaussian to be below
         the corresponding maximum value in the observed data (see the Stutski
         \& Gusten paper). [1.0]

         \sstitem
         GaussClumps.Sa: The Chi-square stiffness parameter {\tt "}Sa{\tt "} which
         encourages the peak amplitude of each fitted gaussian to be close to
         the corresponding maximum value in the observed data (see the Stutski
         \& Gusten paper). [1.0]

         \sstitem
         GaussClumps.Sb: An additional Chi-square stiffness parameter which
         encourages the background value to stay close to its initial value.
         This stiffness is not present in the Stutzki \& Gusten paper but is
         added because the background value is usually determined by data
         points which have very low weight and is thus poorly constrained. It
         would thus be possibly to get erroneous background values without
         this extra stiffness. [0.1]

         \sstitem
         GaussClumps.Sc: The Chi-square stiffness parameter {\tt "}Sc{\tt "} which
         encourages the peak position of each fitted gaussian to be close to
         the corresponding peak position in the observed data (see the Stutski
         \& Gusten paper). [1.0]

         \sstitem
         GaussClumps.Thresh: Gives the minimum peak amplitude of clumps to
         be fitted by the GaussClumps algorithm (see also GaussClumps.NPad).
         The value should be supplied as a multiple of the RMS noise level.
         (see ADAM parameter RMS). [20.0]

         \sstitem
         GaussClumps.VeloRes: The velocity resolution of the instrument, in
         channels. The velocity FWHM of each clump is not allowed to be
         smaller than this value. Only used for 3D data. [3.0]

         \sstitem
         GaussClumps.VeloStart: An initial guess at the ratio of the typical
         observed clump velocity width to the velocity resolution. This is used to
         determine the starting point for the algorithm which finds the best
         fitting Gaussian for each clump. If no value is supplied, the
         initial guess at the clump velocity width is based on the local profile
         around the pixel with peak value. Only used for 3D data. []

         \sstitem
         GaussClumps.Wmin: This parameter, together with GaussClumps.Wwidth,
         determines which input data values are used when fitting a Gaussian to
         a given peak in the data array. It specifies the minimum weight
         which is to be used (normalised to a maximum weight value of 1.0).
         Pixels with weight smaller than this value are not included in the
         fitting process. [0.05]

         \sstitem
         GaussClumps.Wwidth: This parameter, together with GaussClumps.Wmin,
         determines which input data values are used when fitting a Gaussian to
         a given peak in the data array. It is the ratio of the width of the
         Gaussian weighting function (used to weight the data around each clump
         during the fitting process), to the width of the initial guess Guassian
         used as the starting point for the Gaussian fitting process. The
         Gaussian weighting function has the same centre as the initial guess
         Gaussian. [2.0]
      }
   }
   \sstdiytopic{
      ClumpFind Configuration Parameters
   }{
      The ClumpFind algorithm uses the following configuration parameters.
      Values for these parameters can be specified using the CONFIG parameter.
      Default values are shown in square brackets:

      \sstitemlist{

         \sstitem
         ClumpFind.DeltaT: The gap between the contour levels. Only accessed
         if no value is supplied for {\tt "}Level1{\tt "}, in which case the contour levels
         are linearly spaced, starting at a lowest level given by {\tt "}Tlow{\tt "} and
         spaced by {\tt "}DeltaT{\tt "}. Note, small values of DeltaT can result in noise
         spikes being interpreted as real peaks, whilst large values can result
         in some real peaks being missed and merged in with neighbouring peaks.
         The default value of two times the RMS noise level (as specified by
         the ADAM parameter RMS) is usually considered to be optimal,
         although this obviously depends on the RMS noise level being correct. []

         \sstitem
         ClumpFind.Level$<$n$>$: The n{\tt '}th data value at which to contour the
         data array (where $<$n$>$ is an integer). Values should be given for
         {\tt "}Level1{\tt "}, {\tt "}Level2{\tt "}, {\tt "}Level3{\tt "}, etc. Any number of contours can be
         supplied, but there must be no gaps in the progression of values for
         $<$n$>$. The values will be sorted into descending order before being
         used. If {\tt "}Level1{\tt "} is not supplied, the contour levels are instead
         determined automatically using parameters {\tt "}Tlow{\tt "} and {\tt "}DeltaT{\tt "}. Note
         clumps found at higher contour levels are traced down to the lowest
         supplied contour level, but any new clumps which are initially found
         at the lowest contour level are ignored. That is, clumps must have
         peaks which exceed the second lowest contour level to be included in
         the returned catalogue. []

         \sstitem
         ClumpFind.MinPix: The lowest number of pixel which a clump can
         contain. If a candidate clump has fewer than this number of pixels,
         it will be ignored. This prevents noise spikes from being interpreted
         as real clumps. [4]

         \sstitem
         ClumpFind.Naxis: Controls the way in which contiguous areas of
         pixels are located when contouring the data. When a pixel is found
         to be at or above a contour level, the adjacent pixels are also checked.
         {\tt "}Naxis{\tt "} defines what is meant by an {\tt "}adjacent{\tt "} pixel in this sense.
         The supplied value must be at least 1 and must not exceed the number
         of pixel axes in the data. The default value equals the number of
         pixel axes in the data. If the data is 3-dimensional, any given pixel
         can be considered to be at the centre of a cube of neighbouring pixels.
         If {\tt "}Naxis{\tt "} is 1 only those pixels which are at the centres of the cube
         faces are considered to be adjacent to the central pixel. If {\tt "}Naxis{\tt "}
         is 2, pixels which are at the centre of any edge of the cube are also
         considered to be adjacent to the central pixel. If {\tt "}Naxis{\tt "} is 3, pixels
         which are at the corners of the cube are also considered to be adjacent
         to the central pixel. If the data is 2-dimensional, any given pixel can
         be considered to be at the centre of a square of neighbouring pixels.
         If {\tt "}Naxis{\tt "} is 1 only those pixels which are at the centres of the
         square edges are considered to be adjacent to the central pixel. If
         {\tt "}Naxis{\tt "} is 2, pixels which are at square corners are also considered
         to be adjacent to the central pixel. For one dimensional data, a
         value of 1 is always used for {\tt "}Naxis{\tt "}, and each pixel simply has 2
         adjacent pixels, one on either side. []

         \sstitem
         ClumpFind.Tlow: The lowest level at which to contour the data
         array. Only accessed if no value is supplied for {\tt "}Level1{\tt "}. See {\tt "}DeltaT{\tt "}.
         The default value is the minimum input data value plus four times the
         RMS noise level. []
      }
   }
   \sstdiytopic{
      Kimberley Configuration Parameters
   }{
      The Kimberley algorithm uses the following configuration parameters.
      Values for these parameters can be specified using the CONFIG parameter.
      Default values are shown in square brackets:
   }
}
\sstroutine{
   CUPIDHELP
}{
   Display information about CUPID
}{
   \sstdescription{
      This application displays information about CUPID. This includes
      general topics common to all applications, as well as detailed
      descriptions of each application. The information is organised in a
      hierarchical structure of topics, subtopics, sub-subtopics, etc.
      Each entry ends with a list of related sub-topics. You can then
      examine any of these sub-topics or return to the previous level.
   }
   \sstusage{
      cupidhelp [topic] [subtopic] [subsubtopic] [subsubsubtopic]
   }
   \sstparameters{
      \sstsubsection{
         TOPIC = LITERAL (Read)
      }{
         Topic for which help is to be given. [{\tt "} {\tt "}]
      }
      \sstsubsection{
         SUBTOPIC = LITERAL (Read)
      }{
         Subtopic for which help is to be given. [{\tt "} {\tt "}]
      }
      \sstsubsection{
         SUBSUBTOPIC = LITERAL (Read)
      }{
         Subsubtopic for which help is to be given. [{\tt "} {\tt "}]
      }
      \sstsubsection{
         SUBSUBSUBTOPIC = LITERAL (Read)
      }{
         Subsubsubtopic for which help is to be given. [{\tt "} {\tt "}]
      }
   }
   \sstdiytopic{
      Navigating the Help Tree
   }{
      The text for each topic is displayed in screen-fulls. A prompt is issued
      at the end of each topic at which you may:

      \sstitemlist{

         \sstitem
            enter a topic and/or subtopic name(s) to display the help for
               that topic or subtopic, so for example, {\tt "}polka parameters dpi{\tt "}
               gives help on DPI, which is a subtopic of Parameters, which
               in turn is a subtopic of CLUMPS;

         \sstitem
            press the RETURN key to see more text at a {\tt "}Press RETURN to
               continue ...{\tt "} request;

         \sstitem
            press the RETURN key at topic and subtopic prompts to move up
               one level in the hierarchy, and if you are at the top level it
               will terminate the help session;

         \sstitem
            enter CTRL/D (i.e. press the CTRL and D keys simultaneously) in
               response to any prompt will terminate the help session;

         \sstitem
            enter a question mark {\tt "}?{\tt "} to redisplay the text for the current
               topic, including the list of topic or subtopic names; or

         \sstitem
            enter an ellipsis {\tt "}...{\tt "} to display all the text below the
               current point in the hierarchy.  For example, {\tt "}CLUMPS...{\tt "}
               displays information on the CLUMPS topic as well as
               information on all the subtopics under CLUMPS.

      }
      You can abbreviate any topic or subtopic using the following
      rules.

      \sstitemlist{

         \sstitem
            Just give the first few characters, e.g. {\tt "}PARA{\tt "} for
               Parameters.

         \sstitem
            Some topics are composed of several words separated by
               underscores.  Each word of the keyword may be abbreviated,
               e.g. {\tt "}Colour\_Set{\tt "} can be shortened to {\tt "}C\_S{\tt "}.

         \sstitem
            The characters {\tt "}\%{\tt "} and {\tt "}$*${\tt "} act as wild-cards, where the
               percent sign matches any single character, and asterisk
               matches any sequence of characters.  Thus to display
               information on all available topics, type an asterisk in
               reply to a prompt.

         \sstitem
            If a word contains, but does end with an asterisk wild-card,
               it must not be truncated.

         \sstitem
            The entered string must not contain leading or embedded
               spaces.

      }
      Ambiguous abbreviations result in all matches being displayed.
   }
}
\sstroutine{
   MAKECLUMPS
}{
   Create simulated data containing clumps and noise
}{
   \sstdescription{
      This application creates a new 1-, 2- or 3-dimensional NDF containing
      a collection of clumps with background noise. It also creates a
      catalogue containing the clump parameters.

      The clumps profiles are Gaussian, with elliptical isophotes. The
      values of each parameter defining the clump shape can be either
      fixed at a constant value or selected from a given probability
      distribution.
   }
   \sstusage{
      makeclumps out outcat
   }
   \sstparameters{
      \sstsubsection{
         ANGLE( 2 ) = \_REAL (Read)
      }{
         Defines the distribution from which the spatial position angle of
         the major axis of the elliptical clump is chosen. Values should
         be supplied in units of degrees. See parameter  PARDIST for
         additional information. Note, angles are always taken from a
         uniform distribution, irrespective of the setting of PARDIST.
         [current value]
      }
      \sstsubsection{
         BACK = \_REAL (Read)
      }{
         The constant background level. [current value]
      }
      \sstsubsection{
         BEAMFWHM = \_REAL (Read)
      }{
         The spatial FHWM (Full Width at Half Max) of the instrumental beam,
         in pixels. The generated clumps are smoothed with a Gaussian beam
         of this FWHM, before noise is added. No spatial smoothing is
         performed if BEAMFWHM is zero. [current value]
      }
      \sstsubsection{
         FWHM1( 2 ) = \_REAL (Read)
      }{
         Defines the distribution from which the FWHM (Full Width at Half
         Max) for pixel axis 1 of each clump is chosen. Values should be
         supplied in units of pixel. See parameter PARDIST for additional
         information. [current value]
      }
      \sstsubsection{
         FWHM2( 2 ) = \_REAL (Read)
      }{
         Defines the distribution from which the FWHM (Full Width at Half
         Max) for pixel axis 2 of each clump is chosen. Values should be
         supplied in units of pixel. See parameter PARDIST for additional
         information. [current value]
      }
      \sstsubsection{
         FWHM3( 2 ) = \_REAL (Read)
      }{
         Defines the distribution from which the FWHM (Full Width at Half
         Max) for pixel axis 3 of each clump is chosen. Values should be
         supplied in units of pixel. See parameter PARDIST for additional
         information. [current value]
      }
      \sstsubsection{
         LBND() = \_INTEGER (Read)
      }{
         The lower pixel bounds of the output NDF. The number of values
         supplied (1, 2 or 3) defines the number of pixel axes in the output
         NDF (an error is reported if the number of values supplied for LBND
         and UBND differ).
      }
      \sstsubsection{
         NCLUMP = \_INTEGER (Read)
      }{
         The number fo clumps to create.
      }
      \sstsubsection{
         OUT = NDF (Write)
      }{
         The NDF to receive the simulated data, including instrumental
         blurring and noise.
      }
      \sstsubsection{
         MODEL = NDF (Write)
      }{
         The NDF to receive the simulated data, excluding noise. A CUPID
         extension is added to this NDF, containing information about each
         clump in the same format as produced by the CLUMPS command. This
         includes an NDF holding an of the individual clump.
      }
      \sstsubsection{
         OUTCAT = FILENAME (Write)
      }{
         The output catalogue in which to store the clump parameters.
         There will be one row per clump, and a subset of the following
         columns will be included, depending on the number of pixel axes
         in the output array:

         \sstitemlist{

            \sstitem
               Peak intensity of the clump, above the background level.
               See parameter PEAK.

            \sstitem
               The pixel coordinate of the clump centre on axis 1.

            \sstitem
               The intrinsic FWHM of the clump on axis 1, measured in
               pixels. This excludes any instrumental blurring specified by
               BEAMFWHM or VELFWHM. See parameter FWHM1.

            \sstitem
               The pixel coordinate of the clump centre on axis 2.

            \sstitem
               The intrinsic FWHM of the clump on axis 2, measured in
               pixels. This excludes any instrumental blurring specified by
               BEAMFWHM. See parameter FWHM2.

            \sstitem
               The spatial position angle of the major axis of the clump,
               measured in degrees, positive from the first pixel axis to the
               second pixel axis. See parameter ANGLE.

            \sstitem
               The pixel coordinate of the clump centre on axis 3.

            \sstitem
               The intrinsic FWHM of the clump on axis 3, measured in
               pixels. This excludes any instrumental blurring specified by
               VELFWHM. See parameter FWHM3.

            \sstitem
               The projection of the internal velocity gradient vector
               onto pixel axis 1, measured in dimensionless units of
               {\tt "}velocity pixels per spatial pixel{\tt "}. See parameter VGRAD1.

            \sstitem
               The projection of the internal velocity gradient vector
               onto pixel axis 2, measured in dimensionless units of
               {\tt "}velocity pixels per spatial pixel{\tt "}. See parameter VGRAD2.

         }
         The output catalogue will contain the first three of these columns
         if the output array is 1-dimensional, the first six if it is
         2-dimensional, and all of them if it is 3-dimensional.
      }
      \sstsubsection{
         PARDIST = LITERAL (Read)
      }{
         The shape of the distribution from which clump parameter values are
         chosen. Can be {\tt "}Normal{\tt "} or {\tt "}Uniform{\tt "}. The distribution for each
         clump parameter is specified by itw own ADAM parameter containing
         two values; the mean and the width of the distribution. If PARDIST
         is {\tt "}Normal{\tt "}, the width is the standard deviation. If PARDIST is
         {\tt "}Uniform{\tt "}, the width is half the range between the maximum and
         minimum parameter values. In either case, if a width of zero is
         supplied, the relevant parameter is given a constant value equal
         to the specified mean. [current value]
      }
      \sstsubsection{
         PEAK( 2 ) = \_REAL (Read)
      }{
         Defines the distribution from which the peak value (above the
         local background) of each clump is chosen. See parameter PARDIST
         for additional information. [current value]
      }
      \sstsubsection{
         RMS = \_REAL (Read)
      }{
         The RMS (Gaussian) noise to be added to the output data. [current value]
      }
      \sstsubsection{
         TRUNC = \_REAL (Read)
      }{
         The level (above the local background) at which clumps should be
         truncated to zero, given as a fraction of the noise level specified
         by RMS. [current value]
      }
      \sstsubsection{
         UBND() = \_INTEGER (Read)
      }{
         The upper pixel bounds of the output NDF. The number of values
         supplied (1, 2 or 3) defines the number of pixel axes in the output
         NDF (an error is reported if the number of values supplied for LBND
         and UBND differ).
      }
      \sstsubsection{
         VELFWHM = \_REAL (Read)
      }{
         The FWHM of the Gaussian velocity resolution of the instrument, in
         pixels. The generated clumps are smoothed on the velocity axis with
         a Gaussian beam of this FWHM, before noise is added. No velocity
         smoothing is performed if VELFWHM is zero. [current value]
      }
      \sstsubsection{
         VGRAD1( 2 ) = \_REAL (Read)
      }{
         Defines the distribution from which the projection of the internal
         velocity gradient vector onto pixel axis 1 of each clump is chosen.
         Values should be supplied in dimensionless units of {\tt "}velocity
         pixels per spatial pixel{\tt "}. See parameter PARDIST for additional
         information. [current value]
      }
      \sstsubsection{
         VGRAD2( 2 ) = \_REAL (Read)
      }{
         Defines the distribution from which the projection of the internal
         velocity gradient vector onto pixel axis 2 of each clump is chosen.
         Values should be supplied in dimensionless units of {\tt "}velocity
         pixels per spatial pixel{\tt "}. See parameter PARDIST for additional
         information. [current value]
      }
   }
   \sstnotes{
      \sstitemlist{

         \sstitem
         If 3D data is created, pixel axes 1 and 2 are the spatial axes,
         and pixel axis 3 is the velocity axis.

         \sstitem
         The positions of the clumps are chosen from a uniform
         distribution on each axis.
      }
   }
}

\newpage
\markboth{\stardocname}{\stardocname}

\section{\label{APP:HISTORY}History}
This section describes the major changes introduced at each release of
CUPID.


\newpage

\end{document}
