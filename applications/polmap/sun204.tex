\documentstyle[11pt]{article}
\pagestyle{myheadings}

% -----------------------------------------------------------------------------
% ? Document identification
\newcommand{\stardoccategory}  {Starlink User Note}
\newcommand{\stardocinitials}  {SUN}
\newcommand{\stardocsource}    {sun204.1}
\newcommand{\stardocnumber}    {204.1}
\newcommand{\stardocauthors}   {T.\,J.\,Harries}
\newcommand{\stardocdate}      {15th January 1996}
\newcommand{\stardoctitle}     {POLMAP \\ [1.5ex]
                                An interactive data analysis package \\[1.0ex]
                                for linear spectropolarimetry}
%  \newcommand{\stardocversion}   {<software_version>}
%  \newcommand{\stardocmanual}    {<manual_type>}
% ? End of document identification
% -----------------------------------------------------------------------------

\newcommand{\stardocname}{\stardocinitials /\stardocnumber}
\markright{\stardocname}
\setlength{\textwidth}{160mm}
\setlength{\textheight}{230mm}
\setlength{\topmargin}{-2mm}
\setlength{\oddsidemargin}{0mm}
\setlength{\evensidemargin}{0mm}
\setlength{\parindent}{0mm}
\setlength{\parskip}{\medskipamount}
\setlength{\unitlength}{1mm}

% -----------------------------------------------------------------------------
%  Hypertext definitions.
%  ======================
%  These are used by the LaTeX2HTML translator in conjunction with star2html.

%  Comment.sty: version 2.0, 19 June 1992
%  Selectively in/exclude pieces of text.
%
%  Author
%    Victor Eijkhout                                      <eijkhout@cs.utk.edu>
%    Department of Computer Science
%    University Tennessee at Knoxville
%    104 Ayres Hall
%    Knoxville, TN 37996
%    USA

%  Do not remove the %\begin{rawtex} and %\end{rawtex} lines (used by
%  star2html to signify raw TeX that latex2html cannot process).
%\begin{rawtex}
\makeatletter
\def\makeinnocent#1{\catcode`#1=12 }
\def\csarg#1#2{\expandafter#1\csname#2\endcsname}

\def\ThrowAwayComment#1{\begingroup
    \def\CurrentComment{#1}%
    \let\do\makeinnocent \dospecials
    \makeinnocent\^^L% and whatever other special cases
    \endlinechar`\^^M \catcode`\^^M=12 \xComment}
{\catcode`\^^M=12 \endlinechar=-1 %
 \gdef\xComment#1^^M{\def\test{#1}
      \csarg\ifx{PlainEnd\CurrentComment Test}\test
          \let\html@next\endgroup
      \else \csarg\ifx{LaLaEnd\CurrentComment Test}\test
            \edef\html@next{\endgroup\noexpand\end{\CurrentComment}}
      \else \let\html@next\xComment
      \fi \fi \html@next}
}
\makeatother

\def\includecomment
 #1{\expandafter\def\csname#1\endcsname{}%
    \expandafter\def\csname end#1\endcsname{}}
\def\excludecomment
 #1{\expandafter\def\csname#1\endcsname{\ThrowAwayComment{#1}}%
    {\escapechar=-1\relax
     \csarg\xdef{PlainEnd#1Test}{\string\\end#1}%
     \csarg\xdef{LaLaEnd#1Test}{\string\\end\string\{#1\string\}}%
    }}

%  Define environments that ignore their contents.
\excludecomment{comment}
\excludecomment{rawhtml}
\excludecomment{htmlonly}
%\end{rawtex}

%  Hypertext commands etc. This is a condensed version of the html.sty
%  file supplied with LaTeX2HTML by: Nikos Drakos <nikos@cbl.leeds.ac.uk> &
%  Jelle van Zeijl <jvzeijl@isou17.estec.esa.nl>. The LaTeX2HTML documentation
%  should be consulted about all commands (and the environments defined above)
%  except \xref and \xlabel which are Starlink specific.

\newcommand{\htmladdnormallinkfoot}[2]{#1\footnote{#2}}
\newcommand{\htmladdnormallink}[2]{#1}
\newcommand{\htmladdimg}[1]{}
\newenvironment{latexonly}{}{}
\newcommand{\hyperref}[4]{#2\ref{#4}#3}
\newcommand{\htmlref}[2]{#1}
\newcommand{\htmlimage}[1]{}
\newcommand{\htmladdtonavigation}[1]{}

%  Starlink cross-references and labels.
\newcommand{\xref}[3]{#1}
\newcommand{\xlabel}[1]{}

%  LaTeX2HTML symbol.
\newcommand{\latextohtml}{{\bf LaTeX}{2}{\tt{HTML}}}

%  Define command to re-centre underscore for Latex and leave as normal
%  for HTML (severe problems with \_ in tabbing environments and \_\_
%  generally otherwise).
\newcommand{\latex}[1]{#1}
\newcommand{\setunderscore}{\renewcommand{\_}{{\tt\symbol{95}}}}
\latex{\setunderscore}

%  Redefine the \tableofcontents command. This procrastination is necessary
%  to stop the automatic creation of a second table of contents page
%  by latex2html.
\newcommand{\latexonlytoc}[0]{\tableofcontents}

% -----------------------------------------------------------------------------
%  Debugging.
%  =========
%  Remove % on the following to debug links in the HTML version using Latex.

% \newcommand{\hotlink}[2]{\fbox{\begin{tabular}[t]{@{}c@{}}#1\\\hline{\footnotesize #2}\end{tabular}}}
% \renewcommand{\htmladdnormallinkfoot}[2]{\hotlink{#1}{#2}}
% \renewcommand{\htmladdnormallink}[2]{\hotlink{#1}{#2}}
% \renewcommand{\hyperref}[4]{\hotlink{#1}{\S\ref{#4}}}
% \renewcommand{\htmlref}[2]{\hotlink{#1}{\S\ref{#2}}}
% \renewcommand{\xref}[3]{\hotlink{#1}{#2 -- #3}}
% -----------------------------------------------------------------------------
% ? Document specific \newcommand or \newenvironment commands.
% ? End of document specific commands
% -----------------------------------------------------------------------------
%  Title Page.
%  ===========
\renewcommand{\thepage}{\roman{page}}
\begin{document}
\thispagestyle{empty}

%  Latex document header.
%  ======================
\begin{latexonly}
   CCLRC / {\sc Rutherford Appleton Laboratory} \hfill {\bf \stardocname}\\
   {\large Particle Physics \& Astronomy Research Council}\\
   {\large Starlink Project\\}
   {\large \stardoccategory\ \stardocnumber}
   \begin{flushright}
   \stardocauthors\\
   \stardocdate
   \end{flushright}
   \vspace{-4mm}
   \rule{\textwidth}{0.5mm}
   \vspace{5mm}
   \begin{center}
   {\Huge\bf  \stardoctitle}
%  {\Huge\bf  \stardoctitle \\ [2.5ex]}
%  {\LARGE\bf \stardocversion \\ [4ex]}
%  {\Huge\bf  \stardocmanual}
   \end{center}
   \vspace{5mm}

% ? Heading for abstract if used.
%  \vspace{10mm}
%  \begin{center}
%     {\Large\bf Abstract}
%  \end{center}
% ? End of heading for abstract.
\end{latexonly}

%  HTML documentation header.
%  ==========================
\begin{htmlonly}
   \xlabel{}
   \begin{rawhtml} <H1> \end{rawhtml}
      \stardoctitle
%     \stardoctitle\\
%     \stardocversion\\
%     \stardocmanual
   \begin{rawhtml} </H1> \end{rawhtml}

% ? Add picture here if required.
% ? End of picture

   \begin{rawhtml} <P> <I> \end{rawhtml}
   \stardoccategory \stardocnumber \\
   \stardocauthors \\
   \stardocdate
   \begin{rawhtml} </I> </P> <H3> \end{rawhtml}
      \htmladdnormallink{CCLRC}{http://www.cclrc.ac.uk} /
      \htmladdnormallink{Rutherford Appleton Laboratory}
                        {http://www.cclrc.ac.uk/ral} \\
      \htmladdnormallink{Particle Physics \& Astronomy Research Council}
                        {http://www.pparc.ac.uk} \\
   \begin{rawhtml} </H3> <H2> \end{rawhtml}
      \htmladdnormallink{Starlink Project}{http://www.starlink.ac.uk/}
   \begin{rawhtml} </H2> \end{rawhtml}
   \htmladdnormallink{\htmladdimg{source.gif} Retrieve hardcopy}
      {http://www.starlink.ac.uk/cgi-bin/hcserver?\stardocsource}\\

%  HTML document table of contents.
%  ================================
%  Add table of contents header and a navigation button to return to this
%  point in the document (this should always go before the abstract \section).
  \label{stardoccontents}
  \begin{rawhtml}
    <HR>
    <H2>Contents</H2>
  \end{rawhtml}
  \renewcommand{\latexonlytoc}[0]{}
  \htmladdtonavigation{\htmlref{\htmladdimg{contents_motif.gif}}
        {stardoccontents}}

% ? New section for abstract if used.
% \section{\xlabel{abstract}Abstract}
% ? End of new section for abstract
\end{htmlonly}

% -----------------------------------------------------------------------------
% ? Document Abstract. (if used)
%   ==================
% ? End of document abstract
% -----------------------------------------------------------------------------
% ? Latex document Table of Contents (if used).
%  ===========================================
% \newpage
 \begin{latexonly}
   \setlength{\parskip}{0mm}
   \latexonlytoc
   \setlength{\parskip}{\medskipamount}
   \markright{\stardocname}
 \end{latexonly}
% ? End of Latex document table of contents
% -----------------------------------------------------------------------------
\newpage
\renewcommand{\thepage}{\arabic{page}}
% \setcounter{page}{1}

\section{Introduction}

A linear polarization spectrum is a set of Stokes vectors
$(I_{\lambda},Q_{\lambda},U_{\lambda})$. Hence linear
spectropolarimetric data is four-dimensional (six-dimensional if the
variance arrays of the Stokes parameters are included) and cannot be
manipulated using the standard spectral analysis packages ({\em e.g.},
{\sc dipso}; Howarth \& Murray 1991).

{\sc polmap} owes much to {\sc dipso} as far as the style of user
interface is concerned, but no attempt was made to provide the wealth
of spectral analysis routines available in {\sc dipso}. The routines
provided by {\sc polmap} are specifically designed for the
spectropolarimetrist. The manual includes a simple step-by-step guide
to the program and some example data analysis recipes.

\subsection{{\sc polmap} and TSP }

The TSP package (Bailey 1990) runs under the ADAM environment and
provides many routines for handling time-series and polarimetric data.
The TSP routines are biased towards data reduction, and facilities are
provided to reduce data from several different instruments. {\sc
polmap} does not contain any data reduction routines, but was designed
with data analysis in mind.

{\sc polmap} also provides routines for displaying data that are not
available in TSP. Command are provided to read and write TSP
polarization spectrum format files from within {\sc polmap}.  Hence the
packages should be seen as complimentary, in a similar manner to {\sc figaro}
and {\sc dipso}.

\subsection{The command interface}

Commands may be entered at the prompt in either upper- or lower-case.
Numerical parameters must follow the command in the order specified and
must be separated by spaces. If any command parameters are omitted on
the command line they are usually prompted for. Several commands may be
entered on the same command line provided they are separated by commas;
the only exception is the {\bf COMFILE} command, which must appear on
its own command line.

If a character string parameter (such as a filename) contains
upper-case characters it must be placed in double quotes in order to
protect it from the command parser. If a range of integers is required
as a parameter to a command a greater-than sign may be employed. For
example instead of {\tt DELETE 4 5 6 7} the form {\tt DELETE 4 > 7} may
be used. {\bf HELP COMMANDS} will give a list of the available commands
and you can also obtain help on a specific command using {\bf HELP}
{\it command}.

\subsection{Data storage}

The data storage system is very similar to {\sc dipso}. The polarization
spectra are stored in a stack and spectra may be put onto the top
(numerically largest) stack entry or pulled from the middle of the stack.
Routines that alter or measure data generally work on the spectrum in the
current arrays and results to any operation are generally stored in these
arrays as well. The entire stack may be saved in NDF format using the {\bf
WRSTK} command.

\subsection{Getting started}

In order to run {\sc polmap} you must source the {\tt polmap.csh} script.
On Starlink systems, the command alias {\tt polmapsetup} will do this
for you if you have initialised the Starlink Software for use.

The {\tt polmap.csh} script sets the environment variable {\tt POLMAP\_DIR}
to point to the location of the {\sc polmap} files, and defines the
command alias {\tt polmap} to run the {\sc polmap} exectuable.

Then copy the file {\tt sn1987a.sdf} from the {\sc polmap} directory
into an empty working directory:

\begin{quote}
{\tt cp~\$POLMAP\_DIR/sn1987a.sdf~.}
\end{quote}

Typing {\tt polmap} at your normal shell prompt will start the
program.  A message-of-the-day will appear with any updates or changes
to the program and the arrow prompt notifies you that the program is
now ready for input.

Data may be read into the program from ascii files ({\bf RDALAS}), {\it
TSP} files ({\bf RDTSP}) or from stack files that were created using
{\sc polmap} ({\bf RDSTK}).   Type {\tt rdtsp sn1987a} to read in the
file demonstration spectrum from your working directory. The case of
the input string in unimportant and if parameters are omitted they are
normally prompted for. Place character string parameters in quotes to
protect them from the command parsing routine.  The polarization
spectrum is now in the current arrays. The current arrays are the
arrays on which most of the {\sc polmap} commands act and are the
arrays to which results of commands are written. The polarization
spectrum can be put onto a {\it stack} of spectra using the {\bf PUT}
command. Try it now. The program should respond with a short message
naming the stack entry in which the spectrum has been copied. It is
important to note that although the spectrum has been put onto the
stack it is also still in the current arrays.  Using {\bf PUT} again
will put another copy of the current spectrum onto the stack. The {\bf
LIST} command will list the stack contents along with some useful
information such as the spectral range and the number of data points.
Typing {\tt LIST 0} will display the contents of the current arrays.

In order to plot the data we must first initialize a graphics
device. This is done using the {\bf DEV} command. The command requires
as a parameter a standard GKS device name (eg. xwindows) along with
the number of subdivisions in $x$ and $y$ into which the plotting
surface must be divided. If $x$ and $y$ are not specified then the
defaults ($x=1$,$y=1$) are employed. Once you have initialised your
graphics device try typing {\tt TRIPLOT 0.03 1}. This command will
produce a triplot of the data in stack entry 1 (which should be the
demonstration spectrum). The top section of the triplot shows the
position angle of the polarization (in degrees), the middle sections
shows the percentage polarization and the final section is the Stokes
$I$ component. Note that the top two sections appear to have a varying
bin size. This is because the polarization spectrum is binned to a
constant percentage polarization error (of 0.03 percent). This means
that regions of the spectrum with smaller errors on Stokes $Q$ and $U$
have a higher spectral resolution. The range over which the spectrum
is plotted can be adjusted using {\bf TRANGE},{\bf PRANGE},{\bf
IRANGE} and {\bf WRANGE} to limit the position angle, percentage
polarization,intensity and wavelength ranges respectively. {\bf
TAUTO},{\bf PAUTO},{\bf IAUTO} and {\bf WAUTO} switch off the ranges
and let the program autolimit them. The wavelength range can also be
set interactively using the {\bf CWRANGE} command, which uses the
cursor.

\subsection{A step-by-step analysis example}

You should now have two copies of the demonstration data on the
stack. Type {\tt title demo2} to change the title of the current
spectrum and then put it on the stack. Listing the stack you should
now find three entries of the same spectrum, so let's delete
some. Typing {\tt DEL 1 2} will delete the first and second stack
entries. Listing the stack again will show that there is now only one
entry, entitled {\tt demo2}. Triplot this entry and type {\tt
CONTDEF}.  This command allows the user to define multiple regions of
continuum. Start at the shortward end of the spectrum and define four
line-free continuum bins.  Define a bin with the end shortward of the
start in order to quit the continuum definition. Typing {\tt CONTFIT 1
2} will fit a second order polynomial to the three Stokes components
of the first stack entry. The fit has been mapped onto the wavelength
grid of the current array, so title the fit and put it onto the
stack. You can plot the two spectra on the same axes using {\tt
TRIPLOT 0.03 1 2}. Note that the line styles are the same for both the
spectra. The {\bf LROT} command will make the line style change for
each plot. In order to rectify the spectrum type {\tt IDIV 1}. The
Stokes $I$ spectrum of the current arrays has been divided into the
$I$ array of stack entry number one. The results are now in the
current arrays. {\tt TRIPLOT 0.03} will plot the current polarization
(hopefully rectified) spectrum. Put this onto the stack and then use
{\tt wrstk junk} to write out the entire stack onto disk. The file
{\tt junk.sdf} will be in our current directory. Typing {\tt QUIT}
will leave the program.

\subsection{Command files}

The {\bf COMFILE} command may be used to read input from a text file
rather than from the keyboard. The {\bf COMFILE} must be given on its
own command line. The text file is simply a list of commands as you
would have typed them in. The default extension of a command file is
{\tt .cmd}. A command file may contain a {\bf COMFILE} command but the
nest must be no deeper than twenty files (more than enough hopefully).
If you have a set of commands that you regularly issue after starting
the program then these may be placed in the file {\tt .polmap} in your
home directory. If found, this command file is executed on start-up.
It might contain a {\bf DEV} command to initialize your specific
device type, or a command to set your plotting styles.

\subsection{The command history}

The {\bf COMMS} command lists the twenty previous commands. Any of
these commands may be accessed using the pling (!) symbol. For example
{\tt !203} will execute the command number 203 from the command
history, in a similar fashion to the UNIX operating system. It should
be noted that if commas are used to separate commands then each will
be given a separate command history number.

\section{The TRIPLOT command}
\subsection{Introduction}

The {\bf TRIPLOT} command provides a flexible but
straightfoward method for displaying polarization spectra. As
mentioned in the previous section a triplot consists of three separate
graphs. The top panel displays the position angle of the polarization
in degrees, the middle section shows the percentage polarization and
the bottom section shows the intensity spectrum. The first command
parameter is the percentage error in polarization with which to bin
the spectrum. If no further parameters are given the polarization
spectrum in the current arrays will be plotted. Additional parameters
refer to stack entries. If only one stack entry is specified then that
polarization spectrum is plotted. If more than one stack entry are
specified then they are all plotted, but on the axis limits of the
first polarization spectrum specified. If don't want the plotting
surface erased before each triplot command is executed then use the
{\bf NOBOX} command. The {\bf BOX} command negates this.

A polarized flux section may be included in the triplot using the {\bf
PFLUX} command. The {\bf NOPFLUX} command negates this. The polarized
flux section appears between the percentage flux and the position
angle section.

In some cases, the polarization structure is more clearly seen in
normalized Stokes $Q$ and $U$. The {\bf QUPANEL} will tell the triplot
command to produce a plot showing Stokes $I$ in the bottom panel and
Stokes $Q$ and Stokes $U$ in place of the usual position angle and
percentage polarization.

\subsection{The plotting ranges}

The plotting limits may be altered by the user. For example, the
wavelength range to be plotted is by default the first and last
wavelength points of the polarization spectrum. The {\bf WRANGE}
command allows the user to specify a range and the {\bf CWRANGE}
allows the user to select a range from a previous triplot using the
cursor. {\bf WAUTO} switches back to the default wavelength range.
{\bf IRANGE} and {\bf IAUTO} can be used to set the plotting range for
the intensity section and {\bf TRANGE} and {\bf TAUTO} refer to the
position angle plot. {\bf PRANGE} and {\bf PAUTO} change the
percentage polarization plotting ranges. {\bf PFRANGE} and {\bf
PFAUTO} perform similar functions on the polarized flux panel.

It is an truism of spectropolarimetry that the most interesting objects
have a polarization with a position angle close to 0$^\circ$ or
180$^\circ$, resulting in a position angle plot that oscillates madly
between the extremes of the plot because of wrap-around. In order to
prevent this the triplot can be forced to plot position angles either
higher than 180$^\circ$ ({\bf THI}) or lower than 0$^\circ$ ({\bf
TLOW}). Simply set the range to (0$^\circ$,200$^\circ$) for
example using {\bf TRANGE} and invoke {\bf THI}. The effects of {\bf
THI} or {\bf TLOW} can be lifted using {\bf TFREE}.

\subsection{Plotting styles}

The type of line used to plot the spectrum can be changed for each
plot using {\bf LROT} and the colour of the line (for those of you
lucky enough to have a colour terminal) can be rotated using {\bf
CROT}. You can select a specific line style or colour using the {\bf
SETLINE} or {\bf SETCOLOUR} commands. If you wish to plot a
join-the-dots spectrum then use the {\bf POLY} command. The {\bf
HIST} command returns to histogram style plots. The data may be
plotted as points with error bars using the {\bf MARK} command.

The axis labels on the plot can be changed. {\bf XLABEL} alters the
(guess) label on the $x$-axis and {\bf ILABEL} changes the label on
the intensity spectrum. The font used to annotate the axes may be
changed using the {\bf FONT} command.

\section{The QUPLOT command}

A polarization spectrum may also be plotted in the $QU$ plane. The
{\bf QUPLOT} commmand requires the bin error a parameter. This plots
the current polarization spectrum in the $QU$ plane. The axis limits
may be alted using {\bf QRANGE} and {\bf URANGE} or autoscaled using
{\bf QAUTO} and {\bf UAUTO}. The {\bf BOX} and {\bf NOBOX} commands
work as described above. The entire current polarization spectrum is
plotted unless a wavelength range is specified using {\bf WRANGE}. The
plotting symbol style my be change with the {\bf SETSYMB} command. For
a list of the various plotting symbols available see the {\sc pgplot}
manual.

If you want the $QU$ points to be joined up in wavelength order (to
emphasize a $QU$ loop for example) then use the commannd {\bf QUJOIN}
before the {\bf QUPLOT} call. {\bf NOQUJOIN} switches off the point
drawing. Arrowheads may be employed using the {\bf QUARROW} command.
The arrowhead style can be altered using the {\bf ARROWSTYLE} command.

\section{Interstellar polarization}

A raw polarization spectrum is a vector sum of the interstellar polarization
spectrum (introduced by interstellar dust grains aligned in the galactic
magnetic field) and the intrinsic polarization spectrum. In order to analyse
the intrinsic polarization spectrum it is therefore necessary to subtract the
interstellar polarization vector. Obtaining a reliable estimate of the
interstellar polarization is difficult, but it is crucial to the
subsequent analysis since the magnitude of this vector is often significantly
larger than that of the intrinsic vector. The wavelength dependence of the
interstellar polarization can be described empirically using the Serkowski law
(Serkowski 1973):

\begin{equation}
\label{serk}
p(\lambda)=p_{max} \exp [-k\ln^2 (\lambda_{max} / \lambda)]
\end{equation}

where $p(\lambda)$ is the percentage polarization at wavelength $\lambda$,
$p_{max}$ is the maximum polarization which occurs at wavelength
$\lambda_{max}$ and $k$ is a curve width parameter. Originally $k=1.15$ but it
has been suggested that allowing $k$ to vary may improve the fit
(Condina-Landaberry \& Magalh\~{a}es 1976 and Whittet {\it et al} 1992).

The command {\bf ISFIT} uses a non-linear least-squares method to fit
a Serkowski law (equation \ref{serk}) to the data in the current
arrays. The four parameters to this command are $q$, the maximum
Stokes $Q$ polarization, $u$, the maximum Stokes $U$ polarization,
$w$, the wavelength of the maximum polarization and $k$ the curve
width parameter. A free parameter is initialised with the $:$ symbol
and a parameter to be held fixed is defined using the $=$ symbol. A
typical command line may look like:

\begin{quote}
{\tt  > isfit q:1.5 u:0.1 w:5500 k=1.15}
\end{quote}

This command holds $k$ at $1.15$ but $q$, $u$  and $w$ are free
parameters. These parameters are used to create a polarization spectrum
at the wavelength bins defined by the spectrum in the current arrays.
This means that the spectrum in the current arrays is lost.  Errors are
reported if the number of free parameters exceeds the number of
available polarization datum.

The interstellar polarization vector can be removed from the target
spectrum using the {\bf SUBTRACT} command. The polarization spectrum
in the current arrays is subtracted from the specified entry. The
subtraction is performed over the overlapping wavelength region. The
Stokes $I$ spectrum is obtained from the stack polarization spectrum
and linear interpolation is used to regrid the stack polarization
spectrum onto the current wavelength array. The resulting polarization
spectrum is placed in the current arrays.

Other useful commands for adding and subtracting polarization vectors
include {\bf STATADD} and {\bf STATSUB}, which add/subtract a
constant vector found using the {\bf PTHETA} command. The {\bf ICADD}
command adds a constant value to the $I$ Stokes spectrum only, thereby
diluting the polarization.

\section{Merging spectra}

The simplest way to merge two spectra is the {\bf CONTADD} command,
which is described above. The {\bf MERGE} command merges two stack
polarization spectra (that may or may not overlap) and uses a weighted
mean on data with $x$-values common to both spectra. The {\bf WMERGE}
command works on spectra that have the same $x$-array (spectra can be
remapped using  the {\bf GRID} and {\bf CONTADD} commands) and
produces a weighted mean of the two spectra.

\section{PA calibration}

Raw polarization spectra often have a PA that is a slow function of
wavelength due to the slight chromicity of the waveplate. Two commands
are provided in order to calibrate this effect ({\bf FITPA} and {\bf
PACALIB}). The {\bf FITPA} command fits a polynomial to the difference
spectrum of the observed PA spectrum (usually a spectrum taken through
a 100\% polarizing filter) and the actual PA.  Care must be taken
that the observed PA spectrum does not wrap-around (i.e. flip from
$\approx0^\circ$ to $\approx180^\circ$) as this will spoil the PA
fit. This is {\em not} checked for by {\sc polmap}, but may be fixed with a
call to {\bf ROTPA} prior to the fit. The {\bf PACALIB}
command applies this fit to the current arrays.

\section{References}

Bailey J. A., 1990, SERC Starlink User Note, No. 66.2

Codina-Landaberry S., Magalh\~{a}es A. M., 1976, AA, 49 407

Howarth I. D., Murray J., 1991, SERC Starlink User Note, No. 50.13

Serkowski K., 1973, in Greenburg J. M., Hayes D. S., eds, IAU Symp. 52,
Interstellar Dust and Related Topics. Reidel, p. 145

Whittet D. C. B., Martin P. G., Hough J. H., Rouse M. F., Bailey J. A.,
Axon D. J., 1992, ApJ, 386, 562

\appendix
\newpage
\section{At-a-glance guide}

The following table summarizes the most frequently used {\sc polmap}
commands. More detailed information is given by the online help
command {\bf HELP}.

\begin{center}
\begin{tabular}{lll}
\hline \hline
Command & Arguments & Description \\ \hline \hline \\
\multicolumn{3}{c}{\bf Plotting} \\ \hline
DEV     & {\it device [nx] [ny]} & Initializes GKS graphics device \\
TRIPLOT     & {\it error [entry] ... }& Plots pol. spectrum in triplot form \\
QUPLOT      & {\it error}        & Plots pol. spectrum in $QU$ space \\
LROT        &                    & Changes the line style for each plot \\
FONT        & {\it type}         & Changes the font style \\
\\
\multicolumn{3}{c}{\bf Reading and writing} \\ \hline
RDALAS,WRALAS & {\it filename}   & Reads/writes current pol. spectrum (ascii format) \\
RDTSP,WRTSP & {\it filename}     & Reads/writes current pol. spectrum (TSP format) \\
RDSTK,WRSTK & {\it filename}     & Reads/writes stack in NDF format \\
\\
\multicolumn{3}{c}{\bf Stack} \\ \hline
PUT         &                    & Puts current pol. spectrum onto stack\\
GET         &     {\it entry}    & Gets stack number {\it entry} \\
LIST        &                    & List stack entries            \\
DEL         & {\it entry [entry] ...}  & Deletes stack entry           \\
\\
\multicolumn{3}{c}{\bf Rectification} \\ \hline
CONTDEF     & {\it [wstart] [wend]} & Defines continuum bins  \\
CONTFIT     & {\it order entry}   & Fits a polynomial to a pol. spectrum \\
IDIV        & {\it entry}         & Divides current Stokes $I$ array into {\it entry} \\
\\
\multicolumn{3}{c}{\bf PA calibration} \\ \hline
ROTPA       & {\it angle}         & Rotates PA of polarization spectrum \\
FITPA       & {\it pa ncoeff}     & Fits polynomial to PA \\
PACALIB     &                     & Applies polynomial fit \\
\\
\multicolumn{3}{c}{\bf Miscellaneous} \\ \hline
MERGE       & {\it entry}         & Merges current pol. spectrum with stack entry. \\
HELP    & {\it command}          & Gives online help for {\it command} \\
PTHETA      &                     & Measures polarization in continuum bins \\
TOV     &  {\it wavelength}      & Converts to velocity space \\
TOW     &  {\it wavelength}      & Converts to wavelength space \\

\end{tabular}
\end{center}

\newpage
\section{List of Commands}

{\bf ! \it n} \\
Runs command number n from the command history. See help on COMMS command.

{\bf ADD \it entry } \\
Adds the current polarization spectrum to the specified stack entry. See notes
on the  SUBTRACT command for more details.

{\bf ARROWSTYLE \it fs angle vent} \\
Sets the style of the arrowheads to be drawn by QUPLOT when QUARROW is
set.  fs=1 (filled arrowhead). fs=2 (outline). The angle denoted the
acute angle of the arrow point in degrees. Vent is the fraction of the
triangular arrowhead that is cut away from the back.

{\bf BIN \it entry error} \\
Bins the specified stack entry so that each wavelength bin has a
constant error in percentage polarization.

{\bf BOX} \\
Flags the graphics screen to clear before each triplot.

{\bf CADD \it q u} \\
Adds the constant polarization vector (q,u) to the currect arrays. q and u
should be in \% polarization.

{\bf CHOPW \it min max} \\
Reduces the wavelength range of the current polarization spectrum to run from
min to  max.

{\bf CIRANGE} \\
Sets the intensity plotting range interactively.

{\bf COMFILE \it filename} \\
This command tells  polmap to use a command file as input instead of the
terminal.

{\bf CPFRANGE} \\
Sets the polarized flux plotting range interactively.

{\bf CPRANGE} \\
Sets the percentage polarization plotting range interactively.

{\bf COMMS} \\
List the twenty previous commands issued.

{\bf CONTADD \it entry} \\
Adds the current (I,Q,U) spectrum to the specified stack entry.

{\bf CONTDEF \it  [min max] [min max] } \\
Defines continuum regions of the polarization spectrum. Multiple continuum bins
may be defined on one command line. If no parameters are given then the
continuum bins are defined using the cursor. A triplot must have previously
been plotted in order to interactively define windows. In order to quit contdef
define a bin with  max less than  min.

{\bf CONTFIT \it entry order } \\
Make a polynomial fit to the continuum of a polarization spectrum. The
continuum is defined using bins set up by  CONTDEF. A polynomial fit is
made to the Stokes vector (I,Q,U). The resulting fit is stored in the
current arrays.

{\bf CONTSUB \it entry  } \\
Subtracts the current (I,Q,U) spectrum from the specified stack entry. This
command can be used to remove polarized continuum counts from a polarization
spectrum.

{\bf COUNTS} \\
Finds the mean value of the I stokes parameter using all the bins in the
current polarization spectrum.

{\bf CREMOVE} \\
Uses the cursor to set two (wavelength,intensity) points. The parts of
the current spectrum with wavelengths outside the range defined are
lost. An intensity i (found by linear interpolation between the two
points) is subtracted from each of the remaining intensities. This
command is useful for removing the effects of a depolaring continuum
from an emission line, for example.


{\bf CROT} \\
Sets the plotting colour for the line to change after each plot

{\bf CSUB \it q u} \\
Subtracts the constant polarization vector (q,u) from the currect
arrays. q and u should be \% polarization.

{\bf CTRANGE} \\
Sets the PA plotting range interactively.

{\bf CURSOR} \\
Use the graphics cursor to measure points on the current plot.

{\bf CWRANGE } \\
Define the plotting wavelength range using the cursor. A triplot must
have previously been made.

{\bf DEL \it  entry [entry] } \\
Deletes the specified entries from the stack.

{\bf DEV \it  device [nx] [ny]} \\
Initialize the plotting device. The device name must be a GKS standard. Use a
question mark as the parameter in order to list the possible devices. The
plotting surface is divided into nx subdivisions in x and ny subdivisions in
y. If nx and ny are not specified then the default values are used (nx=1 and
ny=1).

{\bf DRAWLINE \it x1 y1 x2 y2} \\
This command draws a line in the current line style from point (x1,y1)
to (x2,y2), where the distances are given in world coordinates.

{\bf EDIT \it [wmin] [wmax]} \\
Removes the data points from the current spectrum that lie between
wmin and wmax.If wmin and wmax are not specified the cursor is used to
specify the wavelength range.

{\bf FITPA \it pa ncoeffs} \\
Given the actual pa of the current spectrum a polynomial fit of ncoeff
coefficients is made to the difference between the actual and observed
position angles as a function of wavelength. This fit can then be
applied to another observation using PACALIB.

{\bf FONT \it type} \\
Sets the character font for plotting. There are four fonts:
1. Simple single-stroke (default)
2. Roman font
3. Italic font
4. Script font

{\bf GET \it entry } \\
Takes the polarization spectrum from the stack and places it in the current
arrays.

{\bf HELP \it command} \\
Gives online help to polmap commands. HELP COMMANDS will give a list of
all the possible commands. HELP <command> will provide detailed online
help for the selected command. Parameters shown in brackets are optional.

{\bf HIST} \\
Flags triplot to produce histogram style output.


{\bf IAUTO} \\
Automatically set the Stokes I range.

{\bf ICADD \it constant} \\
Adds a constant to the Stokes I array of the current polarization spectrum.

{\bf ILABEL} \\
Sets the label for the intensity on a triplot.

{\bf INTEG} \\
 Integrates the Stokes I parameter of the current array
using trapezoidal integration.

{\bf IPLOT } \\
Plots the Stokes I parameter of the current spectrum.

{\bf IQUADD \it i q u} \\
Adds the constant polarization vector (i,q,u) to the current arrays.

{\bf IRANGE \it  min max } \\
Limits the Stokes I  plotting range.

{\bf ISFIT} \\
Fits a Serkowski law to the data held in the current arrays.
Non-linear least-squares minimization is used.
The resulting IS law is mapped onto the X array of the current
polarization spectrum.

{\bf LIST \it  [mode] } \\
List the stack contents. If  mode is 0 then the contents of the current
arrays are listed.

{\bf LROT} \\
Flags the line style to change after every plot with nobox set.

{\bf LS} \\
List the polarization spectra titles that are on the current stack in
'page' format, useful for large stacks.

{\bf MARK} \\
Flags triplot to mark bins with points and error bars rather than in
histogram style (HIST) or join-the-dots style (POLY).

{\bf MAXPOL} \\
This command finds the maximum polarization datum within each of the
continuum bins.

{\bf MERGE \it entry} \\
Merges the current polarization spectrum with the spectrum in the specified
stack entry. The resulting spectrum is sorted into wavelength order. Weighted
means for the polarization vector are calculated for bins of identical X value.

{\bf MOTD} \\
Displays the message of the day

{\bf NOBOX} \\
Stops automatic clearing of the plotting surface before a triplot.

{\bf NOCROT} \\
Sets the line ploot colour to the forground colour.

{\bf NOLROT} \\
Stops the line style changing.

{\bf NOPFLUX} \\
Don't include polarized flux box in triplot.

{\bf NOQUARROW} \\
Stops the QUPLOT command joining the dots with arrows.

{\bf NOQUJOIN} \\
Data are plotted as points in the QU plane.

{\bf NOQUPANEL} \\
Switches TRIPLOT back to its default mode of plotting.

{\bf PACALIB} \\
Applies a position angle fit (previously made using FITPA) to the current
spectrum. This command is usually used to remove the rotation of pa with
wavelength associated with achromatic half-wave plates.

{\bf PAGE} \\
Clears the graphics screen and moves on to a new page.

{\bf PAPER \it width height} \\
Sets the physical size of the plotting windows. Width and height must be
in cms and a call to DEV must have been made immediately before.

{\bf PFLUX} \\
Incude polarized flux box in triplot.

{\bf PFRANGE \it  min max } \\
Limits the polarized flux plotting range.

{\bf POLY} \\
Forces join-the-dots plotting rather than histogram plotting.

{\bf POP \it  entry} \\
Same as GET (qv).

{\bf PPLOT} \\
Plots the polarization spectrum of the current arrays.

{\bf PRANGE \it min max } \\
Limits the percentage polarization plotting range.

{\bf PTHETA} \\
Sums the Stokes parameters over each of the pre-defined continuum bins and
display the results.

{\bf PUSH} \\
Same as PUT (qv).

{\bf PUT} \\
Places the current polarization spectrum on to the top of the stack. If the
stack is too large a warning message is given.

{\bf PVAL \it number value} \\
Sets the value of a specified IS parameter to the value

{\bf QAUTO} \\
Automatically set the Stokes Q range.

{\bf QRANGE \it min max } \\
Limits the Stokes Q  plotting range.

{\bf QSM \it width} \\
Convolves the current spectrum with a Gaussian profile of the given
width. Intensities beyond the ends of the arrays are assumed to be zero.

{\bf QUARROW} \\
Forces QUPLOT to join-the-dots using arrows. The style of the
arrowhead may be altered using ARROWSTYLE.

{\bf QUIT} \\
Quits the  polmap program.

{\bf QUJOIN} \\
Forces points to be joined by a dotted line in the QU plane.

{\bf QUPANEL} \\
Flag the TRIPLOT command to plot Stokes I, Q and U rather than the
usual I, P and PA.

{\bf QUPLOT \it error } \\
Plots the data in the current arrays in the QU plane. The data is
binned to the constant error given. The wavelength range of the data
to be plotted can set using WRANGE. The axes of the plot can be
adjusted using QRANGE and URANGE. The data are plotted with error bars
and consecutive points can be joined by setting the QUJOIN flag.

{\bf RDALAS \it filename } \\
Reads in a polarization spectrum from the file filename. The file must
be in ascii format. The first column must be the wavelength followed
by the Stokes Q parameter and its variance and the Stokes U parameter
and its variance.  The polarization spectrum is placed in the current
arrays.

{\bf RDSTK \it filename } \\
Reads a stack into  polmap.

{\bf RDTSP \it filename } \\
Reads in a polarization spectrum from a  TSP (Bailey 1992) format file.
Details of the N-dimensional data format files can be found in the  TSP
user guide. The polarization spectrum is placed in the current arrays.

{\bf REGRID \it binsize} \\
Rebins the current polarization spectrum. Each bin in the results spectrum
corresponds to binsize bins in the old spectrum.

{\bf RETITLE \it entry} \\
Retitle the specified stack entry. The retitle command will prompt for the
new title.

{\bf ROTPA \it angle } \\
Rotates the position angle of the current polarization spectrum through
angle where the angle is given in degrees.

{\bf RVEL \it vel} \\
Applies the given radial velocity correction to the wavelength arrays
of the current polarization spectrum. The standard sign convention is
used in which a positive velocity indicates the source is moving away
from the observer.

{\bf SERKTHRU \it q u lambda k lammax} \\
This command maps a Serkowski law onto the current arrays. The vector
(q,u) at wavelength lambda is given along with the values ok k and
lammax. The Serkowski law that passes through this point is mapped on
the current arrays.

{\bf SETCOLOUR \it colour} \\
Sets the line colour for a triplot.

{\bf SETHEIGHT \it height} \\
Sets the height of the characters that are plotted. Standard size is height=1.

{\bf SETLINE \it linestyle} \\
Sets the line style of a triplot. The parameter sould be in the range 1-5.

{\bf SETSYMB \it  style} \\
Sets the symbol style for a QU plot. The style parameter should be in the
range 0 to 31.

{\bf SMOOTH  \it n} \\
Applies a box smooth to the current polarization spectrum (Bevington
routine). The smooth is repeated n times.

{\bf STATADD} \\
Adds the mean polarization vector obtained from the results of the most recent
PTHETA to the currect arrays.

{\bf STATSUB} \\
Subtracts the mean polarization vector obtained from the results of
the most recent PTHETA to the currect arrays.

{\bf SUBTRACT \it  entry } \\
Subtract the current polarization from a specified stack entry. The Stokes Q
and U parameters of both spectra are normalised and the current polarization
vector is subtracted from the specified stack entry. The Stokes I parameter
is obtained from the stack polarization spectrum. Linear interpolation is used
to regrid the stack polarization spectrum onto the current wavelength grid.

{\bf SWAP \it  entry} \\
Swaps the current polarization spectrum with that of a stack entry.

{\bf TAPPEND \it string} \\
Appends the character string string to the end to the current polarization
spectrum title.

{\bf TAUTO} \\
Switchs the position angle plotting range to (0,180).

{\bf TEXT \it  [x] [y]} \\
This command can be used to annotate diagrams. The x and y axis positions for
the lefthand end of the text may be given. The angle at which the text is
to be written is then prompted for. Finally the text string is prompted for.
The text is written in the current font style. If x and y are not given on
the command line the cursor is used to locate the position of the text.

{\bf TFREE} \\
Releases PA plotting restrictions set using THI and TLOW.

{\bf THI} \\
PA data in the range 0 to (Tmax-180) is plotted above 180. This avoids messy
plots that can be obtained if the PA of the polarization spectrum is close
to 0 or 180. eg. >TRANGE 0 190,THI will plot all PAs between 0 and 10 between
180 and 190.

{\bf TITLE \it title } \\
Titles the current polarization spectrum with the character string  title.
Quotes are usually required to protect the case and numerics.

{\bf TLOW} \\
PA data in the range (180+Tmin) to 180is plotted below 0. This avoids messy
plots that can be obtained if the PA of the polarization spectrum is close to 0
or 180. eg TRANGE -10 180,TLOW will plot all PAs between 170 and 180 between
-10 and 0.

{\bf TOV \it wavelength } \\
Transforms the x-axis of the current polarization spectrum to velocity
space. The wavelength given is the rest wavelength for the transformation and
the velocity is in kilometers per second.

{\bf TOW \it wavelength } \\
Transforms the x-axis of the current polarization spectrum to wavelength
space. The parameter is the wavelength of the rest velocity for the
transformation and should be given in Angstroms.

{\bf TRANGE \it min max } \\
Limits the position angle plotting range.

{\bf TRIPLOT \it error [entry] [entry] ...} \\
Plot the specified polarization spectra in triplot format. This format has the
position angle in the top section, the percentage polarization in the middle
section and the bottom section displays the I Stokes parameter. The
polarization is binned to a constant error. The  DEV command must have been
called to initialize the plotting device. If no  entry is specified the
contents of the current arrays are plotted.

{\bf UAUTO} \\
Automatically set the Stokes U range.

{\bf URANGE  \it min max } \\
Limits the Stokes U  plotting range.

{\bf WAUTO} \\
Automatically set the wavelength range.

{\bf WMERGE \it spec1 spec2 weight1 weight2} \\
Merges two polarization spectra with identical x-grids. The spectra
are merged according to the given weights and the result is placed in
the current arrays.

{\bf WRALAS \it filename} \\
Writes out the polarization spectrum from the current arrays into the
file filename. The file is written in ascii format with wavelength as
the first column followed by the Q Stokes parameter and its variance
and the U Stokes parameter and its variance.

{\bf WRANGE \it min max } \\
Limits the wavelength plotting range.

{\bf WRTSP \it filename } \\
Writes out the polarization spectrum from the current arrays into a  TSP
format file. Details of the N-dimensional data format files can
be found in the  TSP user guide.

{\bf WRSTK \it filename } \\
Writes out the stack into a save file.

{\bf XADD \it constant} \\
Adds a constant value to the x array.

{\bf XGRID \it xstart xend npts} \\
Creates an evenly spaced wavelength grid on the current arrays. The Stokes
parameters are zeroed and the current spectrum is lost. The grid has npts and
runs from xstart to xend.

{\bf XLABEL \it label} \\
Sets the x-axis label for the triplot.

\end{document}

