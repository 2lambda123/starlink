\documentclass[twoside,11pt]{article}

% ? Specify used packages
% \usepackage{graphicx}        %  Use this one for final production.
% \usepackage[draft]{graphicx} %  Use this one for drafting.
% ? End of specify used packages

\pagestyle{myheadings}

% -----------------------------------------------------------------------------
% ? Document identification
% Fixed part
\newcommand{\stardoccategory}  {Starlink System Note}
\newcommand{\stardocinitials}  {SSN}
\newcommand{\stardocsource}    {ssn\stardocnumber}

% Variable part - replace [xxx] as appropriate.
\newcommand{\stardocnumber}    {[number].[version]}
\newcommand{\stardocauthors}   {M. B. Taylor}
\newcommand{\stardocdate}      {19 October 1998}
\newcommand{\stardoctitle}     {SCB - Starlink Source Code browser}
\newcommand{\stardocabstract}  {
This document describes the installation and use of SCB, 
the Starlink Source Code Browser.  
The package consists of two parts, an indexer program and an
extractor program.
The indexer need be run infrequently (when the source code collection
changes), and the extractor can be run to extract a source file
by file- or function-name, either in plain text to the command line,
or in HTML including hyperlinks to called routines to be viewed
using a WWW browser.  In the latter case, the extractor program 
should be installed as a CGI script.
}
% ? End of document identification
% -----------------------------------------------------------------------------

% +
%  Name:
%     ssn.tex
%
%  Purpose:
%     Template for Starlink System Note (SSN) documents.
%     Refer to SUN/199
%
%  Authors:
%     AJC: A.J.Chipperfield (Starlink, RAL)
%     BLY: M.J.Bly (Starlink, RAL)
%     PWD: Peter W. Draper (Starlink, Durham University)
%     MBT: Mark Taylor (Starlink, IoA)
%
%  History:
%     17-JAN-1996 (AJC):
%        Original with hypertext macros, based on MDL plain originals.
%     16-JUN-1997 (BLY):
%        Adapted for LaTeX2e.
%     13-AUG-1998 (PWD):
%        Converted for use with LaTeX2HTML version 98.2 and
%        Star2HTML version 1.3.
%     19-OCT-1998 (MBT):
%        Text added for SCB.
%     {Add further history here}
%
% -

\newcommand{\stardocname}{\stardocinitials /\stardocnumber}
\markboth{\stardocname}{\stardocname}
\setlength{\textwidth}{160mm}
\setlength{\textheight}{230mm}
\setlength{\topmargin}{-2mm}
\setlength{\oddsidemargin}{0mm}
\setlength{\evensidemargin}{0mm}
\setlength{\parindent}{0mm}
\setlength{\parskip}{\medskipamount}
\setlength{\unitlength}{1mm}

% -----------------------------------------------------------------------------
%  Hypertext definitions.
%  ======================
%  These are used by the LaTeX2HTML translator in conjunction with star2html.

%  Comment.sty: version 2.0, 19 June 1992
%  Selectively in/exclude pieces of text.
%
%  Author
%    Victor Eijkhout                                      <eijkhout@cs.utk.edu>
%    Department of Computer Science
%    University Tennessee at Knoxville
%    104 Ayres Hall
%    Knoxville, TN 37996
%    USA

%  Do not remove the %begin{latexonly} and %end{latexonly} lines (used by 
%  LaTeX2HTML to signify text it shouldn't process).
%begin{latexonly}
\makeatletter
\def\makeinnocent#1{\catcode`#1=12 }
\def\csarg#1#2{\expandafter#1\csname#2\endcsname}

\def\ThrowAwayComment#1{\begingroup
    \def\CurrentComment{#1}%
    \let\do\makeinnocent \dospecials
    \makeinnocent\^^L% and whatever other special cases
    \endlinechar`\^^M \catcode`\^^M=12 \xComment}
{\catcode`\^^M=12 \endlinechar=-1 %
 \gdef\xComment#1^^M{\def\test{#1}
      \csarg\ifx{PlainEnd\CurrentComment Test}\test
          \let\html@next\endgroup
      \else \csarg\ifx{LaLaEnd\CurrentComment Test}\test
            \edef\html@next{\endgroup\noexpand\end{\CurrentComment}}
      \else \let\html@next\xComment
      \fi \fi \html@next}
}
\makeatother

\def\includecomment
 #1{\expandafter\def\csname#1\endcsname{}%
    \expandafter\def\csname end#1\endcsname{}}
\def\excludecomment
 #1{\expandafter\def\csname#1\endcsname{\ThrowAwayComment{#1}}%
    {\escapechar=-1\relax
     \csarg\xdef{PlainEnd#1Test}{\string\\end#1}%
     \csarg\xdef{LaLaEnd#1Test}{\string\\end\string\{#1\string\}}%
    }}

%  Define environments that ignore their contents.
\excludecomment{comment}
\excludecomment{rawhtml}
\excludecomment{htmlonly}

%  Hypertext commands etc. This is a condensed version of the html.sty
%  file supplied with LaTeX2HTML by: Nikos Drakos <nikos@cbl.leeds.ac.uk> &
%  Jelle van Zeijl <jvzeijl@isou17.estec.esa.nl>. The LaTeX2HTML documentation
%  should be consulted about all commands (and the environments defined above)
%  except \xref and \xlabel which are Starlink specific.

\newcommand{\htmladdnormallinkfoot}[2]{#1\footnote{#2}}
\newcommand{\htmladdnormallink}[2]{#1}
\newcommand{\htmladdimg}[1]{}
\newcommand{\hyperref}[4]{#2\ref{#4}#3}
\newcommand{\htmlref}[2]{#1}
\newcommand{\htmlimage}[1]{}
\newcommand{\htmladdtonavigation}[1]{}

\newenvironment{latexonly}{}{}
\newcommand{\latex}[1]{#1}
\newcommand{\html}[1]{}
\newcommand{\latexhtml}[2]{#1}
\newcommand{\HTMLcode}[2][]{}

%  Starlink cross-references and labels.
\newcommand{\xref}[3]{#1}
\newcommand{\xlabel}[1]{}

%  LaTeX2HTML symbol.
\newcommand{\latextohtml}{\LaTeX2\texttt{HTML}}

%  Define command to re-centre underscore for Latex and leave as normal
%  for HTML (severe problems with \_ in tabbing environments and \_\_
%  generally otherwise).
\renewcommand{\_}{\texttt{\symbol{95}}}

% -----------------------------------------------------------------------------
%  Debugging.
%  =========
%  Remove % on the following to debug links in the HTML version using Latex.

% \newcommand{\hotlink}[2]{\fbox{\begin{tabular}[t]{@{}c@{}}#1\\\hline{\footnotesize #2}\end{tabular}}}
% \renewcommand{\htmladdnormallinkfoot}[2]{\hotlink{#1}{#2}}
% \renewcommand{\htmladdnormallink}[2]{\hotlink{#1}{#2}}
% \renewcommand{\hyperref}[4]{\hotlink{#1}{\S\ref{#4}}}
% \renewcommand{\htmlref}[2]{\hotlink{#1}{\S\ref{#2}}}
% \renewcommand{\xref}[3]{\hotlink{#1}{#2 -- #3}}
%end{latexonly}
% -----------------------------------------------------------------------------
% ? Document specific \newcommand or \newenvironment commands.
% ? End of document specific commands
% -----------------------------------------------------------------------------
%  Title Page.
%  ===========
\renewcommand{\thepage}{\roman{page}}
\begin{document}
\thispagestyle{empty}

%  Latex document header.
%  ======================
\begin{latexonly}
   CCLRC / {\textsc Rutherford Appleton Laboratory} \hfill {\textbf \stardocname}\\
   {\large Particle Physics \& Astronomy Research Council}\\
   {\large Starlink Project\\}
   {\large \stardoccategory\ \stardocnumber}
   \begin{flushright}
   \stardocauthors\\
   \stardocdate
   \end{flushright}
   \vspace{-4mm}
   \rule{\textwidth}{0.5mm}
   \vspace{5mm}
   \begin{center}
   {\Large\textbf \stardoctitle}
   \end{center}
   \vspace{5mm}

% ? Heading for abstract if used.
   \vspace{10mm}
   \begin{center}
      {\Large\textbf Abstract}
   \end{center}
% ? End of heading for abstract.
\end{latexonly}

%  HTML documentation header.
%  ==========================
\begin{htmlonly}
   \xlabel{}
   \begin{rawhtml} <H1 ALIGN=CENTER> \end{rawhtml}
      \stardoctitle
   \begin{rawhtml} </H1> <HR> \end{rawhtml}

   \begin{rawhtml} <P> <I> \end{rawhtml}
   \stardoccategory\ \stardocnumber \\
   \stardocauthors \\
   \stardocdate
   \begin{rawhtml} </I> </P> <H3> \end{rawhtml}
      \htmladdnormallink{CCLRC}{http://www.cclrc.ac.uk} /
      \htmladdnormallink{Rutherford Appleton Laboratory}
                        {http://www.cclrc.ac.uk/ral} \\
      \htmladdnormallink{Particle Physics \& Astronomy Research Council}
                        {http://www.pparc.ac.uk} \\
   \begin{rawhtml} </H3> <H2> \end{rawhtml}
      \htmladdnormallink{Starlink Project}{http://star-www.rl.ac.uk/}
   \begin{rawhtml} </H2> \end{rawhtml}
   \htmladdnormallink{\htmladdimg{source.gif} Retrieve hardcopy}
      {http://star-www.rl.ac.uk/cgi-bin/hcserver?\stardocsource}\\

%  HTML document table of contents. 
%  ================================
%  Add table of contents header and a navigation button to return to this 
%  point in the document (this should always go before the abstract \section). 
  \label{stardoccontents}
  \begin{rawhtml} 
    <HR>
    <H2>Contents</H2>
  \end{rawhtml}
  \htmladdtonavigation{\htmlref{\htmladdimg{contents_motif.gif}}
        {stardoccontents}}

% ? New section for abstract if used.
  \section{\xlabel{abstract}Abstract}
% ? End of new section for abstract

\end{htmlonly}

% -----------------------------------------------------------------------------
% ? Document Abstract. (if used)
%  ==================
\stardocabstract
% ? End of document abstract
% -----------------------------------------------------------------------------
% ? Latex document Table of Contents (if used).
%  ===========================================
  \newpage
  \begin{latexonly}
    \setlength{\parskip}{0mm}
    \tableofcontents
    \setlength{\parskip}{\medskipamount}
    \markboth{\stardocname}{\stardocname}
  \end{latexonly}
% ? End of Latex document table of contents
% -----------------------------------------------------------------------------
\cleardoublepage
\renewcommand{\thepage}{\arabic{page}}
\setcounter{page}{1}

% ? Main text

\section{Introduction}

The USSC consists of a few million lines of source code, mainly in 
Fortran 77 and C. 
Finding the source code of interest given the name of a
user task, a function, or even a source file can be non-trivial;
the name of the file containing a routine definition,
the package/directory/tar archive in which the file resides,
and the position in the file at which the routine is defined
may all be difficult pieces of information to obtain if one is
unfamiliar with the package in question.

This package seeks to enable fast navigation around the USSC 
source code in two ways.  In the first place it can locate a routine
or file based on an explicit user query (giving the exact or
approximate name of the routine or file and optionally the Starlink 
package in which it resides).  In the second place, it can present
the source code as HTML, suitable for viewing with any WWW browser,
with references to external routines presented as hyperlinks so
that call chains can be followed conveniently.
This second mode of use is clearly much more powerful than the first.

There are two principle components of the system: an indexing program
which is run once to locate and store all the files and routine 
definitions, and subsequently only when the source code is changed,
and an extractor program which interrogates the index to locate 
and output a requested file.

Typically the index
will be generated by the system manager and stored centrally,
and the extractor installed
as a CGI script to serve HTML over the network (if security is 
a concern it is possible to restrict access to local users in the
usual ways).  Simultanously, local users will be able to locate
or extract source files from the command line without using the 
WWW interface.  
There are other possibilities however: the extractor need not be
installed as a CGI script at all so that only command-line use
is available, and users may have their own private indices 
which combine some or all of the main source code collection 
with their own development versions of packages.

The remainder of this document describes installation and use of
the package, and some details of its internal operation.




\section{Installation and setup}

\subsection{Installing the software}

The files comprising SCB should be obtained and installed into
the Starlink tree in more or less the usual way.
However, there are some environment variables which affect the
installation, determining for instance the location of the 
indices, the Perl binary and temporary directories.  
All default to sensible values, but the comments in the
{\tt mk} script should be read and modifications made,
or environment variables set, accordingly.

Some of the environment variables can be used in two ways:
when set before building the package (the {\tt mk build} step)
they modify the default behaviour of the programs any time they are run.  
If the environment variables are set when the programs are being
run however, this value overrides the defaults set at build time.
These variables are:
\begin{description}
\item[SCB_SOURCES]
The directory containing the Starlink source code.  
This directory must contain nothing but Starlink source packages,
each of which must be {\em either\/} in a directory named after the package
{\em or\/} in tar file named {\tt package.tar}, optionally compressed
using {\tt compress} or {\tt gzip} (and named with a suffix .Z or .gz 
respectively).
The former of these is the usual format if the sources are in
their usual location, {\tt /star/sources}.
If undefined at build time and run time, SCB_SOURCES 
defaults to {\tt /star/sources}.
\item[SCB_INDEX]
The directory containing the index files (around 5\,Mbyte for the full
Starlink source collection).
If undefined at run time and build time, SCB_INDEX defaults to
{\tt /star/etc/scb}.
\item[SCB_BROWSER_TMP]
The directory used by the browser program to store temporary files. 
If undefined at run time and build time, SCB_BROWSER_TMP defaults to
{\tt /usr/tmp/scb}.
\item[SCB_INDEXER_TMP]
The directory used by the indexer program to store temporary files. 
If undefined at run time and build time, SCB_INDEXER_TMP defaults to
{\tt /usr/tmp/scb}.
\end{description}

An example site installation might go as follows:
\begin{quote}
\begin{verbatim}
# zcat scb.tar.Z | tar xf -
# setenv PERL5 /usr/bin/perl
# setenv SCB_BROWSER_TMP /scratch/scb/browser
# setenv SCB_INDEXER_TMP /scratch/scb/indexer
# setenv INSTALL /star
# mk build
    ...
# mk install
    ...
# mk test
    ...
*** Installation test for the SCB package has been run
\end{verbatim}
\end{quote}


\subsection{Generating the index}

In order for the browser to work, indexes of the source files must be
generated.  This is done by the script {\tt scbindex.pl}, installed
by default in {\tt /star/bin}.

To index the whole of the source collection, simply invoke the 
indexer program as follows:
\begin{verbatim}
% scbindex.pl
\end{verbatim}
This will 
read the source files from the directory specified in SCB_SOURCES, 
write the index files to the directory specified in SCB_INDEX,
and print to the screen each index entry as it is encountered.
If the entire Starlink source code collection is installed, 
this will take a matter of hours.  Some indication of how far the indexing
has progressed can be seen by looking at the package currently
being indexed (the name is given before the `#' sign on the 
printed lines, e.g. lines which look like
\begin{verbatim}
cmp_activ.f          =>  hds#hds_source.tar>cmp_activ.f
cmp_blk.f            =>  hds#hds_source.tar>cmp_blk.f
cmp_cct              =>  hds#hds_source.tar>cmp_cct
\end{verbatim}
indicate that the HDS package is being indexed).
The packages are dealt with in alphabetical order, so by seeing how 
far the package currently being examined is through the list of files
in the directory SCB_SOURCES you can get a rough idea of how much remains
to be done.

If only one or a few packages have been changed
since the last indexing,
then the index can be updated rather than being regenerated from scratch.
In this case only the changed packages are named on the command line,
e.g.
\begin{verbatim}
% scbindex.pl ast polpack ndf
\end{verbatim}
The packages so named will be taken from the SCB_SOURCES directory in
the same way as if the whole source tree is being indexed.
However it is possible to index source code from elsewhere than the
main Starlink source tree, by giving the full pathname of the directory
or tarfile containing the source.

don't forget:  oh, i forgot.


Since the indexing can take a long time, it may be preferable to 
run it in the background, possibly {\tt renice}d, with the output
redirected elsewhere, e.g.:
\begin{verbatim}
% nice +4 scbindex.pl >& scbindex.log &
\end{verbatim}

Occasionally some non-fatal error messages will be generated in the 
course of the indexing, for instance:
\begin{verbatim}
Use of uninitialized value at FortranTag.pm line 456, <SOURCE> chunk 42.
\end{verbatim}
These indicate that an error has occurred during the indexing process,
but unless they appear very many times can be safely ignored.
If the program stops with an error however, it should be investigated.
The error message in this case should give an indication of what has
gone wrong and how to fix it.



% ? End of main text
\end{document}
