\documentclass[11pt,nolof]{starlink}


% -----------------------------------------------------------------------------
% ? Document identification
\stardoccategory    {Starlink User Note}
\stardocinitials    {SUN}
\stardocsource      {sun\stardocnumber}
\stardocnumber      {38.8}
\stardocauthors     {M.\, J.\, Bly \& P.\, M.\, Allan}
\stardocdate        {26 November 1996}
\stardoctitle       {DOCFIND --- Starlink document index searching}
\stardocversion     {1.3}
\stardocmanual      {Users' Manual}
\stardocabstract    {This document explains how to search the Starlink documentation
using the \texttt{docfind} command.}
\stardoccopyright{Copyright \copyright\ 1996 Particle Physics and Astronomy Research Council}
% ? End of document identification

% -----------------------------------------------------------------------------

%  Title Page.
%  ===========
\begin{document}
\scfrontmatter


%  ==========================

\section{Introduction}

One of the recurrent problems with Starlink is that as the volume of
software grows, so does the number of documents describing it.  These
include project wide documents and documents which are local to other
sites.  It is often extremely difficult to find out which document
should be consulted about a particular topic.

Starlink maintains a list of currently valid project wide documents
called `STARLINK DOCUMENTATION' in the file {\texttt{/star/docs/docs\_lis}}.
This may be printed if required.  It is maintained at RAL by the
Starlink Software Librarian and updated at other sites by the Starlink
software update process.  The current date is stated at the beginning of
the file so you can see if your site is up to date by comparing your
site's file with the RAL file available via the WWW.

An alternative approach is to consult file {\texttt{/star/docs/subject\_lis}}.
This is a Key-Word index to Starlink documentation which does not rely
only on document titles.  Once again, it is maintained up to date at RAL
but other sites may lag behind.  The specified current date will tell all.
{\texttt{/star/docs/analysis\_lis}} is also centrally maintained.  This is a
list of the Starlink documents listed by which Software item is the main
subject of each document.

The {\texttt{docfind}} program has been developed to help users search this
mass of information.

\section{Use}

The {\texttt{docfind}} procedure searches the file {\texttt{/star/docs/subject\_lis}}
for a given key word and then searches the file {\texttt{/star/docs/docs\_lis}}
for the same key word in the titles of the Starlink documents.  This
technique is used since not all key words will appear in the titles of the
documents.

To invoke {\texttt{docfind}} simply type:
\begin{terminalv}
% docfind word
\end{terminalv}

A search is executed for occurrences of {\texttt{word}} in the files, and the
results of the search stored in a temporary file.  This is then displayed
using the Unix {\texttt{more}} command.  The temporary file is deleted one the
{\texttt{more}} output is finished.

\section{\xlabel{alternative_searching_utilities}\label{alternative_searching_utilities}Alternative searching utilities}

The {\texttt{docfind}} command is rather limited in what it can search,
particularly with the advent of hypertext versions of documentation and
the use of the WWW and WWW browsers to provide cross-referenced
information about software.

The Starlink hypertext utilities package \xref{HTX}{sun188}{} has two
powerful search and display `engines' to make searching for and
displaying information easier.  Both use hypertext browsers for
display.

The first tool is the \xref{{\texttt{findme}}}{sun188}{performing_keyword_searches}
command, which may be used to search for information within documents.

The second is the \xref{{\texttt{showme}}}{sun188}{displaying_documents_by_name}
command, which may be used to display a specific document.

Full details of the HTX utilities can be found in \xref{SUN/188}{sun188}{}.
\end{document}
