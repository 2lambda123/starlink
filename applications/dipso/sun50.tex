\documentclass[twoside,11pt]{article}

% ? Specify used packages
% \usepackage{graphicx}        %  Use this one for final production.
% \usepackage[draft]{graphicx} %  Use this one for drafting.
% ? End of specify used packages

\pagestyle{myheadings}

% -----------------------------------------------------------------------------
% ? Document identification
% Fixed part
\newcommand{\stardoccategory}  {Starlink User Note}
\newcommand{\stardocinitials}  {SUN}
\newcommand{\stardocsource}    {sun\stardocnumber}
\newcommand{\stardoccopyright} 
{Copyright \copyright\ 2000 Council for the Central Laboratory of the Research Councils}

% Variable part - replace [xxx] as appropriate.
\newcommand{\stardocnumber} {50.24}
\newcommand{\stardocauthors} {I D Howarth, J Murray, D Mills \& D S Berry}
\newcommand{\stardocdate} {5th March 2004}
\newcommand{\stardoctitle} {DIPSO --- A friendly spectrum analysis program}
\newcommand{\stardocversion}   {V3.6-4}
\newcommand{\stardocmanual}    {User Guide}
\newcommand{\stardocabstract}  {}

% ? End of document identification
% -----------------------------------------------------------------------------

% +
%  Name:
%     sun.tex
%
%  Purpose:
%     Template for Starlink User Note (SUN) documents.
%     Refer to SUN/199
%
%  Authors:
%     AJC: A.J.Chipperfield (Starlink, RAL)
%     BLY: M.J.Bly (Starlink, RAL)
%     PWD: Peter W. Draper (Starlink, Durham University)
%
%  History:
%     17-JAN-1996 (AJC):
%        Original with hypertext macros, based on MDL plain originals.
%     16-JUN-1997 (BLY):
%        Adapted for LaTeX2e.
%        Added picture commands.
%     13-AUG-1998 (PWD):
%        Converted for use with LaTeX2HTML version 98.2 and
%        Star2HTML version 1.3.
%      1-FEB-2000 (AJC):
%        Add Copyright statement in LaTeX
%     {Add further history here}
%
% -

\newcommand{\stardocname}{\stardocinitials /\stardocnumber}
\markboth{\stardocname}{\stardocname}
\setlength{\textwidth}{160mm}
\setlength{\textheight}{230mm}
\setlength{\topmargin}{-2mm}
\setlength{\oddsidemargin}{0mm}
\setlength{\evensidemargin}{0mm}
\setlength{\parindent}{0mm}
\setlength{\parskip}{\medskipamount}
\setlength{\unitlength}{1mm}

% -----------------------------------------------------------------------------
%  Hypertext definitions.
%  ======================
%  These are used by the LaTeX2HTML translator in conjunction with star2html.

%  Comment.sty: version 2.0, 19 June 1992
%  Selectively in/exclude pieces of text.
%
%  Author
%    Victor Eijkhout                                      <eijkhout@cs.utk.edu>
%    Department of Computer Science
%    University Tennessee at Knoxville
%    104 Ayres Hall
%    Knoxville, TN 37996
%    USA

%  Do not remove the %begin{latexonly} and %end{latexonly} lines (used by 
%  LaTeX2HTML to signify text it shouldn't process).
%begin{latexonly}
\makeatletter
\def\makeinnocent#1{\catcode`#1=12 }
\def\csarg#1#2{\expandafter#1\csname#2\endcsname}

\def\ThrowAwayComment#1{\begingroup
    \def\CurrentComment{#1}%
    \let\do\makeinnocent \dospecials
    \makeinnocent\^^L% and whatever other special cases
    \endlinechar`\^^M \catcode`\^^M=12 \xComment}
{\catcode`\^^M=12 \endlinechar=-1 %
 \gdef\xComment#1^^M{\def\test{#1}
      \csarg\ifx{PlainEnd\CurrentComment Test}\test
          \let\html@next\endgroup
      \else \csarg\ifx{LaLaEnd\CurrentComment Test}\test
            \edef\html@next{\endgroup\noexpand\end{\CurrentComment}}
      \else \let\html@next\xComment
      \fi \fi \html@next}
}
\makeatother

\def\includecomment
 #1{\expandafter\def\csname#1\endcsname{}%
    \expandafter\def\csname end#1\endcsname{}}
\def\excludecomment
 #1{\expandafter\def\csname#1\endcsname{\ThrowAwayComment{#1}}%
    {\escapechar=-1\relax
     \csarg\xdef{PlainEnd#1Test}{\string\\end#1}%
     \csarg\xdef{LaLaEnd#1Test}{\string\\end\string\{#1\string\}}%
    }}

%  Define environments that ignore their contents.
\excludecomment{comment}
\excludecomment{rawhtml}
\excludecomment{htmlonly}

%  Hypertext commands etc. This is a condensed version of the html.sty
%  file supplied with LaTeX2HTML by: Nikos Drakos <nikos@cbl.leeds.ac.uk> &
%  Jelle van Zeijl <jvzeijl@isou17.estec.esa.nl>. The LaTeX2HTML documentation
%  should be consulted about all commands (and the environments defined above)
%  except \xref and \xlabel which are Starlink specific.

\newcommand{\htmladdnormallinkfoot}[2]{#1\footnote{#2}}
\newcommand{\htmladdnormallink}[2]{#1}
\newcommand{\htmladdimg}[1]{}
\newcommand{\hyperref}[4]{#2\ref{#4}#3}
\newcommand{\htmlref}[2]{#1}
\newcommand{\htmlimage}[1]{}
\newcommand{\htmladdtonavigation}[1]{}

\newenvironment{latexonly}{}{}
\newcommand{\latex}[1]{#1}
\newcommand{\html}[1]{}
\newcommand{\latexhtml}[2]{#1}
\newcommand{\HTMLcode}[2][]{}

%  Starlink cross-references and labels.
\newcommand{\xref}[3]{#1}
\newcommand{\xlabel}[1]{}

%  LaTeX2HTML symbol.
\newcommand{\latextohtml}{\LaTeX2\texttt{HTML}}

%  Define command to re-centre underscore for Latex and leave as normal
%  for HTML (severe problems with \_ in tabbing environments and \_\_
%  generally otherwise).
\renewcommand{\_}{\texttt{\symbol{95}}}

% -----------------------------------------------------------------------------
%  Debugging.
%  =========
%  Remove % on the following to debug links in the HTML version using Latex.

% \newcommand{\hotlink}[2]{\fbox{\begin{tabular}[t]{@{}c@{}}#1\\\hline{\footnotesize #2}\end{tabular}}}
% \renewcommand{\htmladdnormallinkfoot}[2]{\hotlink{#1}{#2}}
% \renewcommand{\htmladdnormallink}[2]{\hotlink{#1}{#2}}
% \renewcommand{\hyperref}[4]{\hotlink{#1}{\S\ref{#4}}}
% \renewcommand{\htmlref}[2]{\hotlink{#1}{\S\ref{#2}}}
% \renewcommand{\xref}[3]{\hotlink{#1}{#2 -- #3}}
%end{latexonly}
% -----------------------------------------------------------------------------
% ? Document specific \newcommand or \newenvironment commands.

\newcommand{\dipcom}[3] { \item [{#1}] {#2} \par }
\newenvironment{dipdesc}{\begin{description}}{\end{description}}
\newcommand{\dipitem}[2]{ \item[{#1}] {#2} }

\begin{htmlonly}

\newcommand{\dipcom}[3] { \end{description}
                            \subsection{\xlabel{#1}{#1} - {#3}}
                            \label{COM:#1}
                            \begin{description}
                            \item [Syntax:] {\tt{#1} {#2}}
                            \par
                            \item [Description:]}

\newenvironment{dipdesc}{\begin{itemize}}{\end{itemize}}

\newcommand{\dipitem}[2]{ \item {\bf{#1}} {#2} }

\end{htmlonly}


% ? End of document specific commands
% -----------------------------------------------------------------------------
%  Title Page.
%  ===========
\renewcommand{\thepage}{\roman{page}}
\begin{document}
\thispagestyle{empty}

%  Latex document header.
%  ======================
\begin{latexonly}
   CCLRC / \textsc{Rutherford Appleton Laboratory} \hfill \textbf{\stardocname}\\
   {\large Particle Physics \& Astronomy Research Council}\\
   {\large Starlink Project\\}
   {\large \stardoccategory\ \stardocnumber}
   \begin{flushright}
   \stardocauthors\\
   \stardocdate
   \end{flushright}
   \vspace{-4mm}
   \rule{\textwidth}{0.5mm}
   \vspace{5mm}
   \begin{center}
   {\Huge\textbf{\stardoctitle \\ [2.5ex]}}
   {\LARGE\textbf{\stardocversion \\ [4ex]}}
   {\Huge\textbf{\stardocmanual}}
   \end{center}
   \vspace{5mm}

% ? Add picture here if required for the LaTeX version.
%   e.g. \includegraphics[scale=0.3]{filename.ps}
% ? End of picture

% ? Heading for abstract if used.
%   \vspace{10mm}
%   \begin{center}
%      {\Large\textbf{Abstract}}
%   \end{center}
% ? End of heading for abstract.
\end{latexonly}

%  HTML documentation header.
%  ==========================
\begin{htmlonly}
   \xlabel{}
   \begin{rawhtml} <H1> \end{rawhtml}
      \stardoctitle\\
      \stardocversion\\
      \stardocmanual
   \begin{rawhtml} </H1> <HR> \end{rawhtml}

% ? Add picture here if required for the hypertext version.
%   e.g. \includegraphics[scale=0.7]{filename.ps}
% ? End of picture

   \begin{rawhtml} <P> <I> \end{rawhtml}
   \stardoccategory\ \stardocnumber \\
   \stardocauthors \\
   \stardocdate
   \begin{rawhtml} </I> </P> <H3> \end{rawhtml}
      \htmladdnormallink{CCLRC / Rutherford Appleton Laboratory}
                        {http://www.cclrc.ac.uk} \\
      \htmladdnormallink{Particle Physics \& Astronomy Research Council}
                        {http://www.pparc.ac.uk} \\
   \begin{rawhtml} </H3> <H2> \end{rawhtml}
      \htmladdnormallink{Starlink Project}{http://www.starlink.rl.ac.uk/}
   \begin{rawhtml} </H2> \end{rawhtml}
   \htmladdnormallink{\htmladdimg{source.gif} Retrieve hardcopy}
      {http://www.starlink.rl.ac.uk/cgi-bin/hcserver?\stardocsource}\\

%  HTML document table of contents. 
%  ================================
%  Add table of contents header and a navigation button to return to this 
%  point in the document (this should always go before the abstract \section). 
  \label{stardoccontents}
  \begin{rawhtml} 
    <HR>
    <H2>Contents</H2>
  \end{rawhtml}
  \htmladdtonavigation{\htmlref{\htmladdimg{contents_motif.gif}}
        {stardoccontents}}

% ? New section for abstract if used.
%  \section{\xlabel{abstract}Abstract}
% ? End of new section for abstract
\end{htmlonly}

% -----------------------------------------------------------------------------
% ? Document Abstract. (if used)
%  ==================
% \stardocabstract
% ? End of document abstract

% -----------------------------------------------------------------------------
% ? Latex Copyright Statement
%  =========================
\begin{latexonly}
\newpage
\vspace*{\fill}
\stardoccopyright
\end{latexonly}
% ? End of Latex copyright statement

% -----------------------------------------------------------------------------
% ? Latex document Table of Contents (if used).
%  ===========================================
  \newpage
  \begin{latexonly}
    \setlength{\parskip}{0mm}
    \tableofcontents
    \setlength{\parskip}{\medskipamount}
    \markboth{\stardocname}{\stardocname}
  \end{latexonly}
% ? End of Latex document table of contents
% -----------------------------------------------------------------------------

\cleardoublepage
\renewcommand{\thepage}{\arabic{page}}
\setcounter{page}{1}

\section {Introduction}

DIPSO is, historically, a simple plotting package incorporating some
basic astronomical applications. If you just want to read in some
data, plot them, and measure some equivalent widths or fluxes, you can
do that without much effort. First-time users with this type of modest
goal can skim through the documentation to get a feel for what's going
on, then check the command reference section to find the commands
required. You could even go straight to the terminal, and type {\tt{DIPSO}}; 
there's no substitute for hands-on experience. However, DO read the
documentation fully at some time; DIPSO can do a lot of things, some
of which you might not know that you needed until you read about
them....

While it is intended that simple things should be simple, an effort
has been made to make complicated things possible. To this end, a
number of rather rudimentary functions and free parameters are
provided (with reasonable defaults set). A macro facility allows
convenient execution of regularly used sequences of commands, and a
simple FORTRAN interface permits ``personal'' software to be very
simply integrated. The existence of this interface has encouraged the
accretion of several codes for carrying out relatively elaborate
numerical or astrophysical calculations ({\em e.g.} profile fitting,
Fourier analysis, nebular continuum modelling). Because it has a
monolith structure, DIPSO still runs fast, but there is,
unfortunately, quite a lot of documentation to wade through to find
the command you need. Still, you should persevere; somewhere, somehow,
it is quite likely that DIPSO can indeed do what you want. (However,
if you want to display images, or handle errors in a general way
without doing a bit of coding, look elsewhere. You'll probably have to
come back to DIPSO eventually in the latter case, though --- and do a
bit of coding!)

New features in this release, and some history related to earlier releases
are \hyperref{described here}{contained in appendix }{}{APP:HISTORY}.


\section {Getting Started}

\subsection {Absolute beginners}
Sit down at a Starlink terminal, and type:
\begin{verbatim}
   % dipsosetup
   % dipso
\end{verbatim}

You'll get a little ``hello'' message, and a new prompt:
\begin{verbatim}
   >
\end{verbatim}
Type
\begin{verbatim}
   g9.z?
\end{verbatim}
hit the return key. You will get an error message. Ignore this and 
read on. Type
\begin{verbatim}
   Help, Q
\end{verbatim}

and hit return. You have just completed your first DIPSO session,
discovering on the way that DIPSO accepts more than one command on a
line (each command being separated by commas), that upper- and
lower-case inputs are accepted, and that DIPSO knows when you make
mistakes (or at least, some kinds of mistake).

\subsection {Doing something}

You didn't do much, though; you'll need to know a few more commands. A
\htmlref{full reference list of commands}{APP:FULL} (ordered more or less
alphabetically) is appended, but here we'll mention a few basic ones to
get you going. (You should check the command descriptions for details of
how they should be used.)
Once in the program, you can use the \htmlref{{\tt{COMM}}}{COM:COMMANDS} 
command to get lists of commands classified by function together with
brief descriptions (for instance ``\verb+COMMANDS g+'' will list all the
graphics-related commands), or use \htmlref{{\tt{HELP}}}{COM:HELP}  for
more detailed information on individual commands.
Data can be read in using the \htmlref{{\tt{READ}}}{COM:READ}  command, or
in special cases one of the following commands: \htmlref{{\tt
ALASRD}}{COM:ALASRD}, \htmlref{{\tt{SCREENRD}}}{COM:SCREENRD}, 
\htmlref{{\tt{SP0RD}}}{COM:SP0RD},  \htmlref{{\tt{SP1RD}}}{COM:SP1RD}, 
\htmlref{{\tt{SP2RD}}}{COM:SP2RD},  \htmlref{{\tt{ATLASRD}}}{COM:ATLASRD}, 
or \htmlref{{\tt{RESTORE}}}{COM:RESTORE}. 

For ``historical reasons'' many people use the ``Spectrum 0'' format
for input and output of data (\htmlref{{\tt{SP0RD}}}{COM:SP0RD},  \htmlref{{\tt{SP0WR}}}{COM:SP0WR}).  However, the
recommended file i/o commands are \htmlref{{\tt{READ}}}{COM:READ}  and \htmlref{{\tt{WRITE}}}{COM:WRITE}  (or
\htmlref{{\tt{SAVE}}}{COM:SAVE}  and \htmlref{{\tt{RESTORE}}}{COM:RESTORE}),  which preserve all the information which
DIPSO associates with a data set.

To get a plotting surface, use the \htmlref{{\tt{DEV}}}{COM:DEV}  command. Plotting is
usually done with \htmlref{{\tt{PM}}}{COM:PM};  unless you've provided X and Y ranges
(with \htmlref{{\tt{XR}}}{COM:XR}  and \htmlref{{\tt{YR}}}{COM:YR},  or some combination of \htmlref{{\tt{XMAX}}}{COM:XMAX}, 
\htmlref{{\tt{XMIN}}}{COM:XMIN},  \htmlref{{\tt{YMAX}}}{COM:YMAX},  and \htmlref{{\tt{YMIN}}}{COM:YMIN})  the plot is auto-scaled to the
minimum and maximum values in the arrays.

Once you have managed to read in some data, and plot them, you will
soon want to carry out measurements, change the style of the plots,
and so on. To find out how to proceed, you should read the
descriptions of commands like (\htmlref{{\tt{HIST}}}{COM:HIST},  \htmlref{{\tt{POLY}}}{COM:POLY},  \htmlref{{\tt{MARK}}}{COM:MARK}); 
(\htmlref{{\tt{XV}}}{COM:XV},  \htmlref{{\tt{YV}}}{COM:YV},  \htmlref{{\tt{XYV}}}{COM:XYV});  (\htmlref{{\tt{CSET}}}{COM:CSET},  \htmlref{{\tt{CROT}}}{COM:CROT}); 
and (\htmlref{{\tt{TPORT}}}{COM:TPORT},  \htmlref{{\tt{TZONE}}}{COM:TZONE}). 

Type {\tt{q}}  to leave the program. If in the middle of something long and
tedious you despair, you can type control-C; this stops execution of the current
command, and returns you to the DIPSO command prompt.

If you decide in the middle of a DIPSO session that you need to issue some
operating system commands, then simply push the DIPSO task into the background
by typing control-Z,
issue your operating system commands, and then re-enter DIPSO by typing {\tt{fg}}. 

\section {Data Storage}

\subsection{Internal Data Storage}

On being read in, data are stored in the `current' X,Y arrays, which have
space reserved for up to 200,000 pairs of points. By default --- or, in
some cases, by compulsion --- most operations ({\em e.g.} plotting) are
carried out on data stored in these arrays. Data can be saved for later
use by `\htmlref{{\tt{PUSH}}}{COM:PUSH}'ing  them onto a `STACK', which can
be thought of as a series of X,Y arrays. The STACK contents can be
inspected using \htmlref{{\tt{SL}}}{COM:SL}  (`Stack List'), deleted using
\htmlref{{\tt{DEL}}}{COM:DEL},  and brought into the `current' arrays using
\htmlref{{\tt{POP}}}{COM:POP}.  Up to 200 stack entries, or 800,000 points,
are allowed.

\subsection{Data Storage on Disk}

The contents of the current array can be written to a disk file using the
\htmlref{{\tt{WRITE}}}{COM:WRITE}  command, and the contents of the stack
can be written using the \htmlref{{\tt{SAVE}}}{COM:SAVE}  command. These
files can either be standard Starlink ``NDF'' structures (see
\xref{SUN/33}{sun33}{}), or alternatively files containing unformatted data
in the original DIPSO format used prior to version 3.00.

The \htmlref{{\tt{USENDF}}}{COM:USENDF}  command allows the user to select
which format to use. All commands which read or write data to or from the
current arrays or stack are influenced by the \htmlref{{\tt
USENDF}}{COM:USENDF} command setting unless the \hyperref{description of the
command}{description of the command in appendix}{}{APP:FULL} says otherwise.

Use of NDF structures enables data files created by DIPSO to be used by
other Starlink packages (and vice-versa). It also enables data files to
be transferred freely from one operating system to another without
needing to do a format conversion for each one. NDF structures are
contained within disk files which have the file extension ``{\tt{.sdf}}''. 
When referring to an NDF, {\em do not include the file type or an error
will result}. Think of it this way; an NDF is an object contained {\em
within} a disk file. The NDF and the disk file in which it is contained
are separate entities and can in principle have different names. When
asked for an NDF you give the name of the NDF, {\em not the name of the
disk file}. It just so happens that at the moment an NDF named
``{\tt{my\_data}}''  will be contained in a disk file called
``{\tt{my\_data.sdf}}''  but this may not always be the case.

In addition, columns of values can be read from FITS binary or ascii
tables into the current arrays.

\section {Command Input}

DIPSO is basically command driven, although for some of the more
complex algorithms the program prompts on a step-by-step basis. Many
commands can be input on a single line (in upper or lower case), each
command (with its associated parameters) being separated from its
neighbours by a comma. Parameters associated with a particular command
follow it on the command line, separated by spaces. Parameter values
which include commas and/or spaces need to be enclosed in double quotes
when given on the command line, otherwise the commas and spaces will be
interpreted as delimiters. Any mandatory
parameters not specified with a particular command are prompted for,
and failure to complete a command will generally result in any
remaining commands on the line being ignored. {\em e.g.}, the line:

\begin{verbatim}
   READ TEST, DRED, PM, PUSH
\end{verbatim}

will read in the NDF {\tt{test}}  from the disk file {\tt{test.sdf}}.  If DIPSO
fails to read the NDF successfully, you get an error message and the remainder
of the line is ignored. Otherwise, it will attempt to de-redden the data using a
`standard' extinction law (\htmlref{{\tt{DRED}}}{COM:DRED}).  Since a value of E(B-V) is mandatory for
this command, but has not been provided, it is prompted for:

\begin{verbatim}
   DRED: E(B-V)?
\end{verbatim}

(Similarly, if \htmlref{{\tt{READ}}}{COM:READ}  hadn't been told which NDF to read, this
parameter would have been prompted for.) On provision of the
appropriate number, the data are de-reddened, plotted (\htmlref{{\tt{PM}}}{COM:PM})  on the
(previously assigned) plotting device, and then \htmlref{{\tt
PUSH}}{COM:PUSH}ed onto the STACK.
(If a plotting device were not previously assigned, DIPSO would again
report an error and terminate execution of the command line.)

If you're letting DIPSO prompt you for mandatory parameters, and decide
that you want to abort the command line, you can respond to the parameter
prompt with one or two exclamation marks (e.g. {\tt{!}}  or {\tt{!!}}).  In the
first case, the current command will abort, the remainder of the command
line will be rejected, and you will be returned to the DIPSO command
prompt. In the second case, the current command will abort, and DIPSO
will also abort (saving the stack to EXIT\_STK.sdf or EXIT.STK in the
process), returning you to the operating system.

Some commands have optional parameters in addition to any mandatory
ones. For example, \htmlref{{\tt{DRED}}}{COM:DRED}  has three associated parameters:

\begin{verbatim}
   DRED E(B-V) R MODE
\end{verbatim}

of which E(B-V) is the only mandatory one, and therefore the only one
prompted for if not supplied. The other two parameters are R (=
A(V)/E(B-V)) and MODE, a switch which allows an LMC-type extinction
law to be invoked. Optional parameters have defaults supplied; in this
case, R=3.1 and MODE=0 (Galactic law). If you want an LMC-type law
with R=3.1 you must provide all parameters:

\begin{verbatim}
   DRED 0.5 3.1 1
\end{verbatim}

but if you want a Galactic law with R not equal to 3.1 you only need
type ({\em e.g.})

\begin{verbatim}
   DRED 0.5 2.0
\end{verbatim}

If you provide too many parameters, the command will use those it can,
issuing a warning about those it can't; {\em e.g.}

\begin{verbatim}
   DRED 0.5 3.1 0 99.99
\end{verbatim}

will provoke a warning that redundant parameters have been provided.

\subsection{Command Line Recall and Editing}
Command line recall and editing is available in DIPSO. Use the up/down
arrows or the \htmlref{{\tt{RECA}}}{COM:RECA}  command to recall commands, and the left/right
arrows and the delete/backspace key to edit them. The following control
characters are recognised:

\begin{dipdesc}
\dipitem {Control-A :}{ Jump to the start of the input buffer }
\dipitem {Control-E :}{ Jump to the end of the buffer }
\dipitem {Control-N :}{ Toggle overstrike/insert mode }
\dipitem {Control-U :}{ Empty the input buffer}
\end{dipdesc}

Two separate lists of text strings are kept; one for the command lines
given in response to the DIPSO command prompt, and one for the strings
given in response to the prompts issued by each command. The up and down
arrows, and the \htmlref{{\tt{RECA}}}{COM:RECA}  command operate within each list, independently
of the other list.

\section {Command Procedures}

Commands can be input from macros (script or command files). This can
be particularly useful if you frequently carry out a fixed sequence of
operations. The command file can be in the directory from which you
are running DIPSO, or in a default directory ({\tt{OWNERDIR}})  of your own
assignment. Thus if DIPSO is requested to execute a command file
(without a full directory specification being given) it first looks in
the current directory; if it doesn't find it, it looks in a directory
assigned the environment variable {\tt{OWNERDIR}};  and if it
still doesn't find it, you get an error message. All your frequently
used command files can therefore be kept in one place. For example, a
file called {\tt{TEST.CMD}}  may contain the instructions:

\begin{verbatim}
   READ,DRED
   LOGY,YMULT -2.5,XMULT 1.0E-04,XINV
   PUSH,SL
\end{verbatim}

The commands in this file would be executed by typing:

\begin{verbatim}
   @TEST
\end{verbatim}

The unspecified mandatory parameters for \htmlref{{\tt{READ}}}{COM:READ} 
and \htmlref{{\tt{DRED}}}{COM:DRED}  would be prompted for, and input, at
the terminal. The Y values in the `current' array would be replaced by
Log(10) Y (\htmlref{{\tt{LOGY}}}{COM:LOGY}),  then multiplied by -2.5
(\htmlref{{\tt{YMULT}}}{COM:YMULT}  {\tt{-2.5}});  the X values would be multiplied
by $10^{-4}$ (\htmlref{{\tt{XMULT}}}{COM:XMULT}  {\tt{+1.0E-04}}),  then replaced
by $1/X$ values (\htmlref{{\tt{XINV}}}{COM:XINV}).  The final data
would then be \htmlref{{\tt{PUSH}}}{COM:PUSH}ed  onto the stack, the
contents of which would be displayed at the terminal (\htmlref{{\tt
SL}}{COM:SL}).

On completion of the commands in the file, control returns to the terminal.

On startup, DIPSO looks for a command file called {\tt{startup.cmd}} 
in a directory which has been
assigned the environment variable {\tt{OWNERDIR}}.  So, if
you regularly want to change any default settings from those normally
set, just create such a file, containing commands which will set your
customised options ({\em e.g.} you may not like the standard X and Y
labels, or you might want always to use native DIPSO data files rather
than NDF data files, {\em etc.}).

\section {Batch Processing}

To run a dipso job in the background place all the commands you wish
to run in the startup.cmd command file and then type:

\begin{verbatim}
   % dipso &
\end{verbatim}

at the shell prompt. If you wish to direct the output to a log file
then type:

\begin{verbatim}
   % dipso >mylog &
\end{verbatim}

In addition you can use the UNIX input/output redirection operators
\(>\) and \(<\) to direct commands into DIPSO from other files or even
from other programs (see UNIX reference manuals for details).

\section {Plotting}

The plotting commands sit on top of the GKS/SGS/AUTOGRAPH packages (see
\xref{SUN/83}{sun83}{}, \xref{SUN/90}{sun90}{}, {\em etc.}; sometimes the
command names don't relate in an obvious way to the name of the graphics
routine which is called, because DIPSO was originally written using a
different graphics package). Although DIPSO grew with simple data sets in
mind ({\em i.e.} monotonically changing X values) it will plot some more
complex arrays. (The example program for the user interface, described
below, generates a circle.)

\subsection {Plotting options}

Plotting can be done with a variety of symbols (\htmlref{{\tt{MARK}}}{COM:MARK},  \htmlref{{\tt{MROT}}}{COM:MROT}), 
or line types (\htmlref{{\tt{TLINE}}}{COM:TLINE},  \htmlref{{\tt{TROT}}}{COM:TROT})  in \htmlref{{\tt{POLY}}}{COM:POLY}  ({\em i.e}
join-the-dots) or \htmlref{{\tt{HIST}}}{COM:HIST}  (histogram) mode. If you have access to
appropriate hardware, colour plotting is also possible (\htmlref{{\tt{CSET}}}{COM:CSET}, 
\htmlref{{\tt{CROT}}}{COM:CROT}).  Device changes can be made at any time, so that you can, for
example, switch between an Ikon, Pericom, and laser printer at will.
Alternatively, you can stick to a single device and display data in
different zones of the plotting surface. A set of useful sub-zones is
provided automatically (see the \htmlref{{\tt{TZONE}}}{COM:TZONE}  command).

\subsection {Cursor commands}

An important aspect of any plotting package is making measurements
from, or marking points on, a plotting surface using a cursor (where
available). In DIPSO, the cursor will respond to any alphanumeric key.
If the functionality of the command requires only one cursor hit ({\em
e.g.} \htmlref{{\tt{XV}}}{COM:XV}  to measure X values), then the command is exited by making
two cursor hits at the same point. This method of exiting generalises
to other cursor-driven commands which require multiple inputs ({\em
e.g.} \htmlref{{\tt{CREGS}}}{COM:CREGS}). 

\subsection {Default plotting (and other) options}

The default options (all of which can be changed at will) are:

\htmlref{{\tt{DEV}}}{COM:DEV}ice  {\tt{0}}  - (null device) \newline
\htmlref{{\tt{HIST}}}{COM:HIST}  - ("histogram" plotting style) \newline
\htmlref{{\tt{TZONE}}}{COM:TZONE}  {\tt{0}}  - (use the entire plotting surface) \newline
\htmlref{{\tt{NXY}}}{COM:NXY}  - (auto-scaling on X and Y axes) \newline
\htmlref{{\tt{CSET}}}{COM:CSET}  {\tt{1}}  - (plot in white on the Ikon) \newline
\htmlref{{\tt{TLINE}}}{COM:TLINE}  {\tt{1}}  - (continuous lines) \newline
\htmlref{{\tt{BOX}}}{COM:BOX}  - (clears frame between plots) \newline
\htmlref{{\tt{NOFILL}}}{COM:NOFILL}  - (MARK symbols open) \newline
\htmlref{{\tt{TICKS}}}{COM:TICKS}  {\tt{<null>}}  - (Tick marks on axes calculated automatically) \newline
\htmlref{{\tt{FONT}}}{COM:FONT}  {\tt{0}}  - (Hardware character set) \newline
\htmlref{{\tt{XJ}}}{COM:XJ},  \htmlref{{\tt{YT}}}{COM:YT}  - (Plot has "justified" X axis and "trimmed" Y axis) \newline
\htmlref{{\tt{LABON}}}{COM:LABON}  - (Full labelling of axes) \newline
\htmlref{{\tt{GRIDSTYLE}}}{COM:GRIDSTYLE}  {\tt{1}}  - (Four sides to the plot box drawn in) \newline
\htmlref{{\tt{PPROMPT}}}{COM:PPROMPT}  {\tt{F}}  - (PM without arguments plots current arrays) \newline
\htmlref{{\tt{XLAB}}}{COM:XLAB}  {\tt{"Wavelength"}}  \newline
\htmlref{{\tt{YLAB}}}{COM:YLAB}  {\tt{"Flux"}}  \newline

These defaults are chosen as a compromise between aesthetic elegance and
speed of plotting. For an ugly but fast plot, choose \htmlref{{\tt
POLY}}{COM:POLY} and \htmlref{{\tt{GRIDSTYLE}}}{COM:GRIDSTYLE}{\tt  5}; for
truth and beauty, choose \htmlref{{\tt{FONT}}}{COM:FONT}  {\tt{2}}. 

This is as good a point as any to note some other default settings for
DIPSO:

\htmlref{{\tt{ECHO}}}{COM:ECHO}  {\tt{-1}}  - (Commands file inputs not echoed at the terminal) \newline
\htmlref{{\tt{BEEP}}}{COM:BEEP}  - (input errors induce a beep) \newline
\htmlref{{\tt{HANDLER}}}{COM:HANDLER}  {\tt{1}}  - (robust error handling) \newline
\htmlref{{\tt{TPROMPT}}}{COM:TPROMPT}  {\tt{F}}  - (doesn't insist on a string with TITLE) \newline
\htmlref{{\tt{USENDF}}}{COM:USENDF}  {\tt{T}}  - (NDF structures are used to store data on disk) \newline
\htmlref{{\tt{USEHTX}}}{COM:USEHTX}  {\tt{F}}  - (Help information is displayed in  plain text format) \newline

\subsection {Getting hardcopy plots}

DIPSO doesn't ``remember'' what is on the plotting surface in any way.
Thus you can't get an ``instant'' copy of a plot on your terminal
(unless you have some special hardware which will do it for you).
Instead, you must change devices and execute the appropriate series
of commands to do the plot for you.

Plotting on a hardcopy unit (laser printer, line printer, {\em etc.})
will normally leave a (frequently large!) file in your working
directory, and this file will need to be printed on the appropriate
device before you actually get a plot out. Check the GKS documentation
(or your node manager!) for details; and remember to tidy up your
directory afterwards, or you will quickly run out of disk quota!

\section {The User Code Interface}

If DIPSO can't or won't do something reasonably straightforward ({\em
e.g.} a simple functional operation on a single spectrum) --- or even,
if you're ambitious, something quite complicated --- that you
require of it, then you can avail yourself of the user interface. This
consists of a logical function, {\tt{USER}},  which gives simple and
straightforward access to the contents of the `current' arrays; and
two user-callable subroutines, {\tt{UPUSH}}  and {\tt{GETSTK}},  which
allow you to respectively get data from and put data onto the stack.
(These two routines are described at the end of this section.) You are
also free to use any GKS, or SGS routines you may need.

If you want to do something that requires opening new input streams,
it is recommended that you use streams 23-29 inclusive. DIPSO closes
most i/o streams as soon as it has finished with them, but it always
has stream 22 open, and you are strongly discouraged from using
streams 5 and 6 for anything other than standard input and output ({\em
i.e.} the designated device; normally the terminal when DIPSO is used
interactively). (The reason that stream 22 is always open is that GKS
sends its error messages to this stream. Such messages are more likely
to annoy than enlighten DIPSO users!

\subsection {The ``LOGICAL USER'' Function}

To use the interface you will need to write a subprogram called
{\tt{USER}}  which should follow the example given in {\tt{\$DIPSODIR/user.f}}. 

The example subroutine contains almost all the additional
documentation needed to understand how to `do your own thing'. This
documentation may seem a bit technical, but don't be put off by that
(consult one of your local computer devotees if in doubt); the
interface is really {\em very} easy to use. As a bare minimum, you
could copy the example subprogram and just add in your own IF block:

\begin{verbatim}
   ELSE IF( CMD .EQ. '<your command>' ) THEN
      <carry out operations>
\end{verbatim}

The {\tt{USER}}  function delivers lots of variables for you to play
with, but the essential contents of the argument list are:

\begin{itemize}
\item the current X and Y arrays
\item the current command name
\item a string, PARAMS, containing the parameters associated with the command.
\end{itemize}

Sometimes you will want to treat {\tt{PARAMS}}  as a character string (for
example, it might be a filename); but more often you will want to
read numbers from it. You can do this in exactly the same way as DIPSO
does by calling the routine {\tt{DECODE}}: 

\begin{verbatim}
      CALL DECODE( CMD, PARAMS, NP1, NP2, VALUES, PROMPTS, OK )
\end{verbatim}

where:

\begin{quote}
\begin{description}
\item [CMD - character] The command name, passed to {\tt{USER}}  by DIPSO.
\item [PARAMS - character]  Associated parameters, passed as a string by
DIPSO.
\item [NP1 - integer] The minimum number of parameters (between 0 and 10).
\item [NP2 - integer] The maximum number (greater than or equal NP1).
\item [VALUES - real] The array into which real values, decoded from PARAMS,
are passed.
\item [PROMPTS - character] The NP1 prompts for mandatory parameters.
\item [OK - logical] A success/failure switch.
\end{description}
\end{quote}

The {\tt{`PROMPTS'}}  {\em must} be left justified, separated by blanks,
and terminate with a blank. (Have a look at the
{\tt{\$DIPSODIR/user.f}}  code for examples of how to use {\tt{DECODE}}. 

\subsection {\label{SEC:MY_DIPSO}Building your own binary}

Having written your {\tt{USER}}  code, you can build your own {\tt{dipso}} 
binary.

You must first set the environment variable {\tt{SYSTEM}}  appropriately
for your system: {\tt{alpha\_OSF1}}  for AXP OSF/1, or {\tt{sun4\_Solaris}}  for
Sparc Solaris~2.x. 

Then, you can compile and link a new binary by typing:

\begin{verbatim}
   % dipsosetup
   % my_dipso
\end{verbatim}

If you wish to link in subroutines contained in object modules {\tt{sub1.o,} 
sub2.o} etc, you should define an environment variable {\tt{MY\_OBJECTS}} 
specifying these object files:

\begin{verbatim}
   % setenv MY_OBJECTS "sub1.o sub2.o sub3.o"
   % dipsosetup
   % my_dipso
\end{verbatim}

This leaves you with a personal copy of {\tt{dipso}}  in the directory you're
working in. You should then set up an alias to run this version in preference
to the system version:

\begin{verbatim}
   % alias dipso <where ever>/dipso
\end{verbatim}

\subsection {Debugging your code}

(This section can be skipped by people who never make programming
errors!)

DIPSO is equipped with a condition handler to prevent crashes. DIPSO
shouldn't give a crash in the normal run of things (if you get one,
please report it, giving fullest details possible --- preferably a macro
file which always results in the crash), but it may well do so in
user-supplied code. In this case, you will normally want to disable the
condition handler in order to get the system handler, which may tell you
where the crash occurred. To turn the handler off, the \htmlref{{\tt{HANDLER}}}{COM:HANDLER} 
command can be invoked (use {\tt{HANDLER}  0}). On some flavours of UNIX,
this may result in the program appearing to freeze after a crash. If this
happens, pressing control-C should return you normally to a system
prompt.

\subsection {\label{SEC:LOC}Local documentation}

If your site has a `user-enhanced' version of DIPSO that is used by
several people, then it might be convenient to put the executable into a
local public directory. In this case you should persuade your node
manager to create such a directory, with appropriate protections, and
give it the environment variable {\tt{LDIPSODIR}}.  The local version can
then be run, of course, as {\tt{\$LDIPSODIR/dipso}}. 

In response to the \htmlref{{\tt{COMM}}}{COM:COMMANDS}  command, DIPSO {\em
first} looks for a file {\tt{\$LDIPSO\-DIR/command\-.hlp}},  which will be
used if found. It will then go on to look for a file {\tt
\$DIPSODIR/comand\-.hlp} which it will also use if found.

So it's possible to put an appropriately modified version of
{\tt{command.hlp}}  into {\tt{LDIPSODIR}}  to keep users informed of local
additions to available commands. The local version of {\tt{command.lis}} 
should contain only the extra local commands, not the standard commands,
and should be formatted like the standard version.

{\em Note}, in previous versions of DIPSO, command information was stored
in a file called {\tt{command.lis}}  which had a different format to
{\tt{command.hlp}}  files and was processed differently. It is recommended that
any local {\tt{command.lis}}  files still existing be converted into
{\tt{command.hlp}}  format (see {\tt{\$DIPSODIR/comand\-.hlp}}  for a description
of the format). If DIPSO finds a local old-style {\tt{command.lis}}  file
it will simply display its contents (as in previous versions of DIPSO) in
preference to using any new-style {\tt{command.hlp}}  files. Note, however,
that the extended functionality of the \htmlref{{\tt{COMM}}}{COM:COMMANDS} 
command (i.e. word searches, command descriptions and classification) will not be available.

It's possible to give \htmlref{{\tt{HELP}}}{COM:HELP} for local commands,
too, although it can only be accessed as plain text (see
\htmlref{{\tt{USEHTX}}}{COM:USEHTX}). The {\tt{HELP}} command in plain
text mode first
runs through {\tt{\$DIPSO\-DIR/help\-.lis}}  but if it doesn't find a
command name there, it will try to look for an {\tt{\$LDIPSODIR/lhelp.lis}}, 
and search that for help information. This file should match the format
of {\tt{\$DIPSODIR/help.lis}},  but need contain information only on local
commands.

The first thing that DIPSO does is look for a file called
{\tt{\$LDIPSODIR/updates.lis}},  and print out anything it contains. So if
you've made changes, you can announce them to your local community
through this mechanism.

\subsection {Data access}

If you want to do complex operations involving several data sets, you
may well want to access data on the DIPSO stack. Well, you can; but
first, you'll need to understand a bit more about how DIPSO stores
data.

\subsubsection {\label{SEC:WRNDF}More on data storage}


A DIPSO data set contains a variety of information. First of all,
there is a brief header string [CHARACTER*80 TITLE]. Then, of course,
there are the X and Y data arrays [X(MAXPT), Y(MAXPT), MAXPT=64000],
which contain the NPOINT pairs of data points. Now, in order to know
where in the data any gaps occur, DIPSO maintains a separate `break'
array [BREAKS(MAXBRK), MAXBRK=1000] which contains the NBRK break
points associated with the data set. A break point is the index, in the
X and Y arrays, of the last point before a gap in the data set. Thus
if there are 200 points in the data set, and there are breaks between
the 7th and 8th, and 123rd and 124th, data points, then BREAKS(1)=7,
BREAKS(2)=123, and BREAKS(NBRK)=200, where NBRK is 3. Note that the
last point in a data set is always a break point, so that NBRK is
always 1 or greater.

To allow compatibility with some other programs (notably IUEDR), DIPSO
assumes that a specific Y value (zero by default) actually flags a gap
in the data, for some i/o commands ({\em e.g.} {\tt{SP0RD/WR}}).  This
will often be invisible to the user, but you ought to keep it in mind.
Note also that if DIPSO reads in a data set where a gap is padded out
with a whole string of zeros ({\em e.g.} from IUEDR), then it throws
away all but a couple of them, to save space. (Try \htmlref{{\tt{SP0RD}}}{COM:SP0RD}ing  a
hi-res IUEDR spectrum, then \htmlref{{\tt{SP0WR}}}{COM:SP0WR}ing  it; the output is much
smaller than the input). This behaviour does not apply to the \htmlref{{\tt{READ}}}{COM:READ}, 
\htmlref{{\tt{WRITE}}}{COM:WRITE},  \htmlref{{\tt{SAVE}}}{COM:SAVE}  and \htmlref{{\tt{RESTORE}}}{COM:RESTORE}. 

When using NDFs, there is another Y value which is also used (by {\em
all} commands which access NDFs) to flag gaps in the data. This value is
the standard Starlink ``bad'' value which is used to flag invalid or
missing data in many other Starlink packages. Its value is -1.7014117E+38
(software generates these ``bad'' values automatically... you're not
expected to type them in!).

Finally, although DIPSO will plot general X,Y arrays, several of the
applications commands expect and require data that have Angstrom or
km/s as the X unit ({\em e.g.} \htmlref{{\tt{EW}}}{COM:EW}).  A variable,
WORV (which means
``Wavelength OR Velocity''), is used to flag data in which the ``X''
unit is km/s; if this is the case, then WORV=$\lambda/c$, where
$\lambda$ is the rest wavelength to which the velocities are
referenced (in Angstroms) and c is the speed of light (km/s).
Otherwise, WORV=1.0. (You'll just have to think carefully about what
you're doing if your data are in frequency units, I'm afraid ---
WORV=1.0 will generally be associated with your data.)

\subsubsection{Creating NDFs in your own programs}

To output data from other programs in a form suitable for inputting to DIPSO
with the (recommended) \htmlref{{\tt{READ}}}{COM:READ}  command requires the following minimal code (with
appropriate values and names for all variables):

\begin{small}
\begin{verbatim}

*  Global Constants:
        INCLUDE 'SAE_PAR'         ! Include standard starlink constants
                                  ! such as SAI__OK.

*  Local Variables:
        CHARACTER  COMM           ! This should be WRITE or SP0WR and
                                  ! causes the corresponding command to
                                  ! be simulated.
        CHARACTER  NDFNAM         ! Name of output NDF structure. NB, don't
                                  ! include a file type!!
        INTEGER    NPOINT         ! Number of points in XV and YV.
        REAL       XV( NPOINT )   ! X axis values.
        REAL       YV( NPOINT )   ! Y axis values.
        CHARACTER  XLAB           ! X axis label.
        CHARACTER  YLAB           ! Y axis label.
        CHARACTER  TITLE          ! NDF title.
        INTEGER    NBRK           ! Number of points in `breaks` array.
        INTEGER    BREAKS( NBRK ) ! Breaks array.
        REAL       WORV           ! Wavelength of velocity parameter.
        INTEGER    STATUS         ! Should be SAI__OK on entry. SAI__OK on 
                                  ! exit if successful.

*  Create the output NDF.	
        CALL WRITE_NDF ( COMM, NDFNAM, NPOINT , XV , YV , XLAB, YLAB,
     :                   TITLE , NBRK , BREAKS , WORV , STATUS )


\end{verbatim}
\end{small}

Such programs then need to be linked with the DIPSO object libraries, and
the NDF subroutine library (see \xref{SUN/33}{sun33}{}). To create a
program called {\tt fred} which uses {\tt WRITE\_NDF} to create an output
NDF, do the following:

\begin{verbatim}
   % star_dev
   % dipsosetup
   % f77 -o fred fred.f -L$DIPSODIR -ldipsot -L/star/lib `ndf_link`
   % star_dev remove
\end{verbatim}

As well as being readable by DIPSO such output data sets will also be
automatically readable by all the standard STARLINK packages on any
of the supported operating systems.

\subsubsection {Getting data from the stack}

So, now you know what's in a DIPSO data set; and thus, you have a good
idea of the information on the stack. DIPSO lets you take copies of
stack data using calls to the subroutine GETSTK:

\begin{verbatim}
      CALL GETSTK(INDEX, NPOINT, XV, YV, NBRK, BREAKS, TITLE,
     :            WORV, OK)
\end{verbatim}

where:

\begin{quote}
\begin{description}

\item [INDEX - integer] The stack entry you want to access.

\item [NPOINT - integer] On calling, the size of the arrays into which the 
{\tt{XV}}  and {\tt{YV}}  data are to be loaded; and on exit is the
number of elements of the arrays which are occupied ({\em i.e.} the
number of points).

\item [XV - real] A user-supplied array, which contains the X values
of the STACK entry on return. (It is your responsibility to ensure
that the array is big enough to hold all the data from the STACK
entry.)

\item [YV - real] A user-supplied array to hold Y values on return.

\item [NBRK - integer] On entry, NBRK is the size of the BREAKS
array. On return, NBRKS contains the number of 'break points' in the
data set. (Again, you must ensure that enough space is available.)

\item [BREAKS - integer] A user-supplied array of length NBRK, to
hold the indexes of `break points' in the {\tt{XV}}  array.

\item [TITLE - character] The title associated with the data set.

\item [WORV - real] Wavelength {\em or} Velocity.

\item [OK - logical] Success/failure flag. OK = .FALSE. if the
call to GETSTK is identified as unsuccessful.

\end{description}
\end{quote}

\subsubsection {Pushing data onto the stack}

You can also push data onto the stack:

\begin{verbatim}
      CALL UPUSH( ASZE, XV, YV, NPOINT, BSZE, BREAKS, NBRK, TITLE, 
     :            WORV, OK)
\end{verbatim}

The arguments are the same as for GETSTK, except that ASZE (integer)
is the size of your arrays holding the X and Y values, and BSZE
(integer) is the size of your BREAKS array.

To encourage you to look on DIPSO as a tool with which you can interface
your own software, it is worth noting that the {\tt{ELF}}  package
(described below), all the Fourier analysis software, the {\tt{IS}} 
routines, and the \htmlref{{\tt{NEBCONT}}}{COM:NEBCONT}  facility were added to DIPSO with very
little more than the basic interface described above.

If you do write some software that you think may be of general interest, please
contact Ian Howarth ({\tt{idh@star.ucl.ac.uk}});  it may be
possible to incorporate it into the public version of DIPSO.

\section {Emission Line Fitting (ELF)}

DIPSO has access to a suite of subroutines which are designed to fit a
variety of line profiles to observed data. The commands really need a
bit more explanation than can readily be put into the alphabetical
reference list which follows; so here's a bit more explanation...

The primary purpose of the {\tt{ELF}}  routines is to separate blends by
fitting multiple profiles. Gaussian profiles are the most commonly
used, but other analytic forms are possible, as are `numerical'
profiles. Facilities are provided for constraining line centre
positions, widths, and relative fluxes, so that known atomic data
(such as relative wavelengths or intensities within multiplets) can be
utilised. The option of relative flux constraint is particularly
useful when analysing optically thin emission lines (hence Emission
Line Fitting), but the package is entirely happy with absorption lines
(which it treats as emission lines with negative fluxes), or any other
form of data that can be reasonably approximated with the available
profile forms.

Fits are made to the spectrum data stored in the DIPSO `current'
arrays. The continuum level may be set to zero by manipulations within
DIPSO, or a polynomial fit to the continuum can be made simultaneously
with the profile fitting. (The former option is {\em strongly
recommended}.) Specification of constraints and starting values for
the fit parameters is done in a command language (invoked by the DIPSO
command \htmlref{{\tt{ELFINP}}}{COM:ELFINP})  which is described below. After optimisation of
parameters (DIPSO command \htmlref{{\tt{ELFOPT}}}{COM:ELFOPT}),  the full specification of the
fit and the results may be stored. The resulting fit, in spectrum
form, can be pushed on the DIPSO stack (\htmlref{{\tt{ELFPUSH}}}{COM:ELFPUSH}). 

\subsection {ELF commands}

The complete set of ELF commands are described in detail
\hyperref{below}{in appendix}{}{APP:FULL}. They are:

\begin{verbatim}
      ELFINP   ELFOPT   ELFNEWC
      ELFPUSH  ELFLFIX
      ELFPUSHC ELFPOPC  ELFDELC ELFCSL ELFVUC
      ELFSAVEC ELFRESTC ELFWRC
      ELFPIN   ELFPL
\end{verbatim}

As you can see, they are all of the form {\tt{ELF...}},  so that they can
easily be found in the documentation.

\subsection {ELF data storage}

The {\tt{ELF}}  package takes a COPY of the DIPSO `current' spectrum.
The maximum space available is 1000 datum points.
This number is deliberately rather smaller than the space available in DIPSO
itself, since fitting to large numbers of datum points is prohibitively time
consuming.

Fit data are maintained in three storage areas:

\begin{itemize}

\item A `current' area (not to be confused with the DIPSO current
arrays) contains the specification of the fit in progress. If an
optimisation has been carried out, the results (in the sense of
optimised coefficients) are also kept in this `current' array.

\item A stack of fit coefficients. Data may be interchanged between
this stack (again, not to be confused with the main DIPSO stack) and
the ELF `current' area. Space is provided for a maximum of 20 lines in
each fit.

\item A stack of input numerical profiles. These {\em must} be
spectrum data without internal breaks (gaps), stored in VELOCITY
space.

\end{itemize}

The stack of profile types has space for up to ten entries. The first
five of these are reserved for analytically specified profiles, and
the last five for numerical profiles. Profile definition is as
follows:

\begin{quote}
\begin{dipdesc}
\dipitem {Profile 1:} {Gaussian. (C=centre, W=width(FWHM), I=peak flux)}
\dipitem {Profile 2:} {Triangular. (C=centre, W=width(FWHM), I=peak flux)}
\dipitem {Profile 3-5:} {Unused at present.}
\dipitem {Profile 6-10:} {Available for numerical profiles.}
\end{dipdesc}
\end{quote}

\subsection {ELF general procedures}

Although the command descriptions given later describe functionality
in detail, it is probably worth just summarising how to do a simple
Gaussian fit, for illustration. The steps would typically be:

\begin{itemize}

\item \htmlref{{\tt{POP}}}{COM:POP}  the data of interest into the current arrays. Use ({\em
e.g.}) \htmlref{{\tt{RXR}}}{COM:RXR}  to restrict the number of data points to the minimum
consistent with adequately defining the line(s) of interest and a
small amount of continuum.

\item Although you can represent the continuum by a polynomial with
free parameters, convergence is enormously improved if you first
subtract a continuum (using \htmlref{{\tt{PF}}}{COM:PF},  or \htmlref{{\tt{CDRAW}}}{COM:CDRAW},  for example,
together with \htmlref{{\tt{ASUB}}}{COM:ASUB}).  If you really must incorporate a background
polynomial in the fit, keep the degree as low as possible ({\em e.g.}
zero) to avoid indeterminacy.

\item Type \htmlref{{\tt{ELFINP}}}{COM:ELFINP}  to invoke {\tt{ELF}}s  special command language,
and input your first guesses at the values of the parameters to be
optimised. Make life easier for yourself by keeping the number of
non-linear parameters (line centre positions and widths) as small as
possible, and try to make your guesses good ones. Type {\tt{QELF}}  to
leave the command language processor.

\item Type \htmlref{{\tt{ELFOPT}}}{COM:ELFOPT}  to start the {\tt{ELF}}  Fit Coefficient
OPTimisation. If you have many free parameters, and the machine is
being heavily used, go for a cup of tea, after you've checked the
first iteration of the optimisation to make sure everything is as you
expect.

\item Later\ldots you can push a copy of the `best fit' model onto the
DIPSO stack using \htmlref{{\tt{ELFPUSH}}}{COM:ELFPUSH}.  You can also save the fit
coefficients on the separate fit coefficient stack, for later use or
reference, with \htmlref{{\tt{ELFPUSHC}}}{COM:ELFPUSHC}. 

\end{itemize}

A serious program crash while optimising, such as divide by zero or
overflow, may occur occasionally. Such problems are usually caused by
overspecified fits, or starting values that are grossly in error.

\section {Words of Warning!}

Of course, DIPSO generally does nothing that you don't ask it to do.
So, if `nothing' happens, it is probably because you've defined X and
Y ranges that exclude the data, or left the \htmlref{{\tt{BOX}}}{COM:BOX}  switched off
(\htmlref{{\tt{NB}}}{COM:NB}),  or you're not plotting on the device that you think you
are. Something else to watch out for is plotting a STACK entry, and
then trying to do some operation on the plotted data instead of the
data in the `current' arrays. Be careful!

DIPSO carries no `memory' of what's on a particular plotting surface.
This means that if you want a laser-printer plot of what's on the
screen you need to change device and actually do the plot again (don't
forget to change back to your graphics terminal when the plot is
finished!).

\section {Acknowledgements}

Several of the more elaborate computational functions in this version
of DIPSO have been grafted on as a result of people exploiting the user
interface, with their code eventually being adopted for the release
version. The biggest single contribution is the ELF package, which was
developed by Pete Storey ({\tt{pjs@star.ucl.ac.uk}}).  Pete also had a
hand in \htmlref{{\tt{NEBCONT}}}{COM:NEBCONT},  which uses code primarily written by Pat Harrington
(U.\ of Maryland). Stephen Boyle ({\tt{sjb@star.ucl.ac.uk}})  donated the
Fourier, periodogram, and cross-correlation routines; the interstellar
line profile code has a long and chequered history, but was first
brought forth in the good old days of punched cards by Clive
Davenhall. DIPSO's basic structure and command interface owes much to
Dane Maslen. Other contributors include Jack Giddings, Des Middlemass,
David Monk, and Starlink Management.

\newpage
\appendix

\section {\xlabel{HTXAPPFULL}\label{APP:FULL}Command Specifications}


This section contains a roughly alphabetical reference list of commands,
with a description of the the actions invoked.
(Strict alphabetic ordering has been sacrificed in one or two places in
order to group together the texts for closely related commands.) Each
command has its associated parameters listed with it, in the order in
which they must be supplied. Optional parameters (for which defaults are
provided) are given in [brackets].

\begin {description}

\dipcom{@ (AT)}{filename[.typ] p1 p2 p3 ..... p9}{Reads commands from a command file}
Reads commands from a command file. Any
prompts for unspecified mandatory parameters are given at the
terminal. Command files can include blank lines and `comment cards';
the latter must have an exclamation mark [!] or asterisk [*] as the
first character. (Comments may not be flagged with a first-column
``C'', because this could easily be the first character of a command.)

On successful completion (or on failure to execute a command) control
returns to the terminal.

{\em IMPORTANT}: Command files may not contain references to other
command files (nor to themselves), for fairly obvious reasons.

DIPSO first searches for a file of the given name in the current directory (or
whatever directory is given in the file specification). If it fails to find it,
it then looks for a file {\verb+$OWNERDIR/<filename>+}. Thus you can keep a set
of frequently used command files in the directory assigned the environment
variable {\tt{OWNERDIR}}  (this assignment would normally be carried
out in your login scripts).

The default file type ([.typ]) is `.cmd'.

\dipcom{AADD}{n}{Adds the Y values in two arrays}
Adds the contents of the `current' Y array to the values in STACK
entry `n', leaving the result in the `current' array. Both data sets
must have monotonic X arrays for sensible results to emerge. To
perform the arithmetic, the data in the `current' arrays are mapped
onto the STACK X grid. The addition is only performed at X values
where there are valid Y values in both data sets. In conjunction with
\htmlref{{\tt{GRID}}}{COM:GRID},  the \htmlref{{\tt{AADD}}}{COM:AADD}  command can be used to remap data.

\dipcom{ADIV}{n}{Divides the Y values in two arrays}
Divides the Y values in STACK entry `n' by the Y values in the
`current' arrays. In order to do this, the `current' data are mapped
onto the X grid of the STACK data (both X grids must be monotonically
changing). The output data are left stored in the `current' arrays.

If both data sets are recognized by DIPSO as having X units in velocity
space (WORV not equal 1; see the \htmlref{{\tt{TOV}}}{COM:TOV}  command for details), then the
output data are corrected by the ratio of the WORVs (i.e. by the ratio
of the central wavelengths) to give true flux ratio as a function of
velocity. If only one data set is recognized as having X units of
velocity, the division is carried out and an error is reported.

\dipcom{AMAX}{n}{Takes the maximum of the Y values in two arrays}
At each X point, puts the larger of the STACK (entry `n') and `current
array' Y values into the current array. To do this, the data in the
current arrays are mapped onto the STACK X grid. The function returns
Y values only at those X values where valid data occur in both
data sets. \htmlref{{\tt{AMAX}}}{COM:AMAX}  may be useful, when used with \htmlref{{\tt{AMIN}}}{COM:AMIN},  for displaying the
envelope to a set of spectra of a given object.

\dipcom{AMIN}{n}{Takes the minimum of the Y values in two arrays}
At each X point, puts the smaller of the STACK (entry `n') and
`current array' Y values into the current array. To do this the data
in the current arrays are mapped onto the STACK X grid. The function
returns Y values only at those X values where valid data occur in both
data sets. (See also \htmlref{{\tt{AMAX}}}{COM:AMAX}). 

\dipcom{AMULT}{n}{Multiplies the Y values in two arrays}
Multiply the Y values in STACK entry `n' by the values in the
`current' arrays. The data in the `current' arrays are mapped onto the
X grid of the STACK data in order to carry out this operation. (The X
grids are both required to be monotonic). Results are left in the
`current' arrays.

\dipcom{ASUB}{n}{Subtracts the Y values in two arrays}
Subtracts the `current' Y array from STACK entry `n'. In order to
carry out this operation, the `current' data are mapped onto the X
grid of the STACK data. Subtraction is only carried out at X values
where there are valid Y data in both STACK and current arrays, and
both X arrays must be monotonic. Results are left stored in the
`current' arrays.

\dipcom{ASWAP}{(no parameters)}{Swaps the Y values in two arrays}
Swaps the `top' ({\em i.e.} numerically largest) STACK entry with the
contents of the `current' arrays.

\dipcom{ALASCHK}{(no parameters)}{Checks the lines and columns to be used by {\tt{ALASRD}}} 
Checks the current defaults for the lines and columns to be read in
using \htmlref{{\tt{ALASRD}}}{COM:ALASRD}. 

\dipcom{ALASCOLS}{xcol ycol}{Sets the columns to be read by {\tt{ALASRD}}} 
Tells \htmlref{{\tt{ALASRD}}}{COM:ALASRD}  to read X and Y data from the specified columns of a
file. (A ``column'' is a string of alphanumeric characters separated
from other columns by spaces; of course, the columns which \htmlref{{\tt{ALASRD}}}{COM:ALASRD} 
actually acquires must contain exclusively numeric values.) When DIPSO
begins, these are set to 1 and 2 by default; they are {\em not} reset
to these values after execution of \htmlref{{\tt{ALASRD}}}{COM:ALASRD}. 

\dipcom{ALASLINS}{line1 line2}{Sets the lines to be read by {\tt{ALASRD}}} 
Tells \htmlref{{\tt{ALASRD}}}{COM:ALASRD}  to read only lines line1 to line2 (inclusive) of an input
file. If line2 is specified as zero, this is interpreted to mean
end-of-file.

\dipcom{ALASRD}{filename[.typ] [brkval]}{Reads in data from a text file}
Reads data from a formatted file. The simplest structure which \htmlref{{\tt{ALASRD}}}{COM:ALASRD} 
(and DIPSO) can read is one pair of X, Y values per record; this is
the default file structure expected by \htmlref{{\tt{ALASRD}}}{COM:ALASRD}.  More elaborate files
can be read in through prior use of the \htmlref{{\tt{ALASLINS}}}{COM:ALASLINS}  and \htmlref{{\tt{ALASCOLS}}}{COM:ALASCOLS}  commands
(q.v.).

Gaps in the data are assumed to be flagged by Y values of zero, unless
a different ``brkval'' is specified on the command line.

The default file type ([.typ]) is `.DAT'.

\dipcom{ALASWR}{filename[.typ] [brkval]}{Writes the contents of the current arrays to a text file}
Writes the contents of the `current' arrays into a formatted file. The
X data are output in column 1, and the Y data in column 2; gaps in the
data are flagged with a Y value of zero, unless a different value for
``brkval'' is specified on the command line.

The default file type ([.typ]) is `.DAT'.

\dipcom{ANGLE}{Theta}{Sets orientation of text and symbols plotted by {\tt{PWRITE}}  and {\tt{MARK}} }
Changes the angle at which \htmlref{{\tt{PWRITE}}}{COM:PWRITE}  strings and \htmlref{{\tt{MARK}}}{COM:MARK}  symbols are
plotted. Theta is measured in degrees, anticlockwise from the
horizontal, and is initially set to zero.

\dipcom{ATLASRD}{Teff LogG [MODE]}{Reads in Kurucz model atmosphere fluxes}
Reads in Kurucz model atmosphere fluxes, from the data base kept in
the directory with environment variable SPECDAT. The
data are stored in the database in the form of astrophysical fluxes;
however, \htmlref{{\tt{ATLASRD}}}{COM:ATLASRD}  multiplies them up by a factor $\pi$ to produce
`actual' fluxes. The x unit is Angstroms, and the y unit erg/cm2/s/A.

If MODE=0 (the default value) solar abundance models are acquired;
otherwise the `low metal abundance' models (1/30 solar) are read in. A
summary of the available solar abundance models can be obtained using
\htmlref{{\tt{ATLIST}}}{COM:ATLIST},  and the models normalised to cursor-selectable X,Y values
using \htmlref{{\tt{ATNORM}}}{COM:ATNORM}. 

Other files in the SPECDAT database, including extended atmosphere and
Non-LTE models (see {\tt{\$SPECDAT/info.lis}}  for details), can be read in
using SP2RD {\tt{\$SPECDAT/file\-name.typ.}}  The KHMEXT, MLTE and MNLT
models can all be ATNORMed if they are first subjected to \htmlref{{\tt{TENY}}}{COM:TENY},  but
the normalisation constant will be meaningless.

If you cannot access the model atmospheres, it may be because you are not
currently using NDF data format (see command \htmlref{{\tt{USENDF}}}{COM:USENDF}).  Most of the data in
SPECDAT is only available in NDF format. Of course, your node may not have the
SPECDAT database installed, in which case complain to your node manager.

\dipcom{ATLIST}{(no parameters)}{Lists the available Karucz model atmosphere data }
Lists the T(eff) and Log(g) values used to specify the Kurucz solar
abundance model atmosphere fluxes that are accessible to \htmlref{{\tt{ATLASRD}}}{COM:ATLASRD}. 

\dipcom{ATNORM}{[mode]}{Normalises model atmosphere fluxes}
Normalises model atmosphere fluxes stored in the `current' arrays to
the cursor position. The angular diameter implied by the normalising
constant is printed at the terminal. The normalised fluxes are left in
the `current' arrays, and are plotted if mode=1 (the default). No plot
is produced if mode=0.

If the cursor Y value is negative, the plot is assumed to be of X v
Log10(FLUX). In this case, the Y values plotted (and left in the
`current' arrays) are logs of the normalised model atmosphere fluxes;
however, the unnormalised atmosphere data in the `current' arrays MUST
be linear in Y to start off with ({\em i.e.} in the form resulting
from an \htmlref{{\tt{ATLASRD}}}{COM:ATLASRD}).  \htmlref{{\tt{ATNORM}}}{COM:ATNORM}  can also be used to normalise black-body
fluxes generated with \htmlref{{\tt{BBODY}}}{COM:BBODY}. 

\dipcom{BBODY}{temp}{Calculates black-body fluxes at the current X values}
Calculates black-body fluxes [pi times B(nu)] at the x values of the
grid in the `current' arrays, overwriting the y values therein. (GRID
can be used to initialise appropriate x values.) The fluxes can be
normalised to observed data using the \htmlref{{\tt{ATNORM}}}{COM:ATNORM}  command. The unit of
`temp' is Kelvin, the `x' values are assumed to be in Angstroms, and
B(nu) is in erg/cm-2/s/A.

\dipcom{BEEP}{(no parameters)}{Causes a beep after every error}
Turns on the \htmlref{{\tt{BEEP}}}{COM:BEEP}  following \htmlref{{\tt{NOBEEP}}}{COM:NOBEEP}  (also tests your terminal's beeper).

\dipcom{BIN}{X1 DX}{Bins the contents of the current arrays}
Bins the contents of the `current' arrays, which are expected to be in
monotonically increasing X order. X1 is the start wavelength of the
input data, and DX the X range over which binning is to take place;
thus the first X value in the output data set will normally be
approximately:

\begin{verbatim}
   X1 + 0.5*DX.
\end{verbatim}

If you choose a value for DX which is less than the typical separation
of the input X points you will probably get an unsatisfactory result.
The binned data, which are the unweighted averages of the input X and
Y values in each bin, are left in the `current' arrays.

\dipcom{BOX}{(no parameters)}{Causes the plotting surface to be cleared between plots}
Automatic clearing of the plotting surface between plots. This is the
default option on starting up. The inverse function is \htmlref{{\tt{NB}}}{COM:NB}. 

\dipcom{CDRAW}{[filename or mode]}{Draws a 'continuum' using the cursor or by fitting data in a file}
Allows you to draw a `continuum' using the cursor; the input X values
must be in increasing order (CDRAW is terminated with an
`X(N+1).LE.X(N)' type test). The result is stored in the `current'
arrays (so you should first \htmlref{{\tt{PUSH}}}{COM:PUSH}  your spectrum so as not to lose it);
subsequent rectification of data can be carried out using \htmlref{{\tt{ADIV}}}{COM:ADIV}.  (See
also \htmlref{{\tt{CREGD}}}{COM:CREGD},  \htmlref{{\tt{CREGS}}}{COM:CREGS}  and \htmlref{{\tt{PF}}}{COM:PF}). 

If MODE=0 (the default value), the data stored are just those input
with the cursor, giving a `join-the-dots' spectrum. If MODE=1, a
spline fit to the data points is carried out, using the subroutine
INTEP described by Hill (Publ DAO). A smooth curve (which is supposed
to be like one you might draw by hand through all the cursor points)
is calculated on the grid of `x' points of the arrays in the top ({\em
i.e.} numerically largest) stack entry.

It is possible to input data to this routine from a file, rather than at the
terminal. The data are expected to be in the form (NDF or native DIPSO) produced
by the \htmlref{{\tt{WRITE}}}{COM:WRITE}  command, and MODE is assumed to be 1 (MODE=0 would have the
same effect as a simple \htmlref{{\tt{READ}}}{COM:READ},  and is therefore redundant). Thus, using this
option makes \htmlref{{\tt{CDRAW}}}{COM:CDRAW}  act essentially as a straightforward spline interpolation
routine.

\dipcom{CLEAR}{(no parameters)}{Clears the command window}
Clears the text surface. (Simply a PRINT *, ' ' loop; it will not,
therefore, work on a 4010-type device, for which you should use
\htmlref{{\tt{ERASE}}}{COM:ERASE}). 

\dipcom{CLRBRK}{(no parameters)}{Removes breaks from an array}
Clears all breaks ({\em i.e.} gaps in the data) from the current
arrays. (The end-of-data is preserved, internally, as a break point,
however.)

\dipcom{COMMANDS}{[clist/-word] [clist] [clist]}{Display classified lists of commands with brief descriptions}
If no parameters are supplied, a listing of all available commands is
given. If a minus sign is given as the only parameter, a one line
description of each command is also displayed. If the minus sign is
followed by a word, then only those commands which contain the given
word in their descriptions (case insensitive) are displayed. As an example:

\begin{verbatim}
   > COMMANDS -CONTINUUM
\end{verbatim}

would display all commands which contain the string ``continuum'' in their
descriptions.

Alternatively, commands for display can be selected by their function.
All commands are grouped into one or more classes which are identified
by single lower-case letters as follows:

\begin{verbatim}
   a - Analysis,measurement.
   c - Setting of control variables, etc.
   d - Modelling.
   e - Data editing,selection,rejection.
   f - Fitting.
   g - Graphics.
   h - Help information.
   i - Reading, writing, moving and deleting data sets (including 
       disk I/O).
   l - Filtering.
   m - Manipulation (arithmetic,calibration,corrections,etc).
   q - Inquiry (control variables, etc).
   r - Re-gridding.
   t - Title manipulation.
   y - System control tasks.
\end{verbatim}

Each parameter must be a word made up from a selection of these class
identifiers. Commands must belong to all the classes in at least one of
these words to be displayed. Classes can be negated (i.e. explicitly
excluded) by specifying an upper case identifier (i.e. "A" means ``not
in class "a"''). As an example:

\begin{verbatim}
   > COMMANDS gC t+
\end{verbatim}

would list all commands which are to do with graphics but which are not
control commands, and also all commands which are to do with title
manipulation.

Note, if a file called {\tt{\$LDIPSODIR/comand\-.lis}}  exists then the
contents of the file are displayed without processing (all parameters are
ignored). It is recommended that such files be re-formatted to use the
new system \hyperref{as described earlier}{(see section}{)}{SEC:LOC}.

\dipcom{CRASH}{n}{Induce an exception}
Performs various illegal instructions with the object of trying to
crash the program! It can be used to test the behaviour of the signal/condition
handler (see command \htmlref{{\tt{HANDLER}}}{COM:HANDLER}).  The following values of `n' causes the
corresponding condition:

\begin{dipdesc}
\dipitem{0}{ - Divide by zero (an example of a floating point exception).}
\dipitem{1}{ - Access violation.}
\dipitem{2}{ - Overflow.}
\dipitem{3}{ - Underflow.}
\end{dipdesc}

\dipcom{CREGD}{[h]}{plots the regions selected for 'continuum' fitting using {\tt{CREGS}}} 
`Continuum REGion Display': plots the regions selected for `continuum'
fitting using \htmlref{{\tt{CREGS}}}{COM:CREGS}. 

The continuum windows are indicated by horizontal bars, which are
drawn a fraction `h' of the distance from the bottom to the top of the
plot (0$<$h$<$1, default 0.8).

\dipcom{CREGL}{(no parameters)}{Lists the current 'continuum' window selected with {\tt{CREGS}}} 
`Continuum REGion List': lists the current continuum windows selected
with \htmlref{{\tt{CREGS}}}{COM:CREGS}. 

\dipcom{CREGS}{[X1 X2 X3 X4 X5 X6 ... X49 X50]}{Selects 'continuum' regions}
`Continuum REGion Select': select `continuum' regions. These regions
can be input as a parameter list; if no values are provided, the
cursor is activated for interactive selection of continuum windows.
When using the cursor, the input `X' values {\em must} be in
increasing order.

The selected regions can be checked using \htmlref{{\tt{CREGD}}}{COM:CREGD}.  This command will
normally be used in conjunction with \htmlref{{\tt{PF}}}{COM:PF}.  (See also \htmlref{{\tt{CDRAW}}}{COM:CDRAW}). 

\dipcom{CROT}{(no parameters)}{Causes automatic rotation of plotting colours}
Implements automatic rotation of colours (when used with appropriate
hardware). Each plot begins with the colour specified by the last call
to \htmlref{{\tt{CSET}}}{COM:CSET}  (or colour 1, if \htmlref{{\tt{CSET}}}{COM:CSET}  hasn't been called).

To cancel, use \htmlref{{\tt{NCROT}}}{COM:NCROT}. 

\dipcom{CSET}{n}{Sets the plotting colour}
Set colour for Ikon plotting; n = 1-12 (1 = white). (See also \htmlref{{\tt{CROT}}}{COM:CROT}). 

\dipcom{CXR}{(no parameters)}{Uses the cursor to define the X range to be plotted}
`Cursor X range': define the X range for plotting using the cursor.

\dipcom{CXYR}{(no parameters)}{Uses the cursor to define the X and Y ranges to be plotted}
`Cursor X \& Y range': define the X and Y ranges for plotting using
the cursor.

\dipcom{CYR}{(no parameters)}{Uses the cursor to define the Y range to be plotted}
`Cursor Y range': define the Y range for plotting using the cursor.

\dipcom{DEL}{n1 [n2 n3 n4 ....... n48 n49 n50]}{Deletes stack entries}
Deletes STACK entries. At least 1, and up to 50, stack entries may be
deleted; after deletion, the remaining STACK entries are renumbered in
sequence. Ranges of entries can be specified using the ``-'' operator;
{\em e.g.} \htmlref{{\tt{DEL}}}{COM:DEL}  2 4-6 8 will result in the deletion of entries 2, 4, 5,
6 and 8. \htmlref{{\tt{DEL}}}{COM:DEL}  1-50 will clear the stack.

You are recommended to develop the habit of typing \htmlref{{\tt{SL}}}{COM:SL}  immediately after
\htmlref{{\tt{DEL}}}{COM:DEL},  to check what has happened.

\dipcom{DEV}{workstation}{Selects a graphics device}
Opens a GKS workstation (plotting device). If the argument isn't
provided then a table of legitimate workstations is listed, followed
by a prompt. A full list of available workstations is given in
\xref{SUN/83}{sun83}{}.

\dipcom{DRED}{E(B-V) [R MODE]}{De-reddens data}
Dereddens data stored in the `current' arrays. Negative points are
multiplied by -1, dereddened, and the dereddened values again
multiplied by -1.

The default value of R (=A(V)/E(B-V)) is 3.1.

The default value of MODE is 0, which results in a `standard'
Galactic-type law being used (Seaton 1979 for the UV, Howarth 1983 for
optical-IR); any non-zero value results in an LMC-type law being used
(Howarth 1983).

\dipcom{DRLINE}{x1 y1 x2 y2}{Draws a line between two points}
Draws a poly line between two points on the current graph.

\dipcom{ECHO}{[mode]}{Controls echoing of commands within command files}
Controls echoing, at the terminal, of commands (and comments) read in
from command files. If mode=-1 (the default) there is no echoing, but
you get a ``Command sequence completed'' message when the macro file is
closed; if mode=0 this message is suppressed. Mode=1 results in
commands being echoed; mode=2 echoes comment lines ({\em i.e.} those
beginning with ``!'' or ``*''); and mode=3 echoes commands and
comments. All non-zero modes give a ``Command sequence completedi''
on completion.

\dipcom{ELF}{(no parameters)}{A suite of programs for Emission Line Fitting}
This is {\em not a DIPSO command}, but a suite of programs accessed for
Emission Line Fitting.

The available DIPSO/ELF commands are:

\begin{verbatim}
   ELFINP   ELFOPT   ELFNEWC
   ELFPUSH  ELFLFIX
   ELFPUSHC ELFPOPC  ELFDELC ELFCSL ELFVUC
   ELFSAVEC ELFRESTC ELFWRC
   ELFPIN   ELFPL
\end{verbatim}

The most important `core' commands are:

\begin{verbatim}
   ELFINP ELFOPT ELFPUSH
\end{verbatim}

which are sufficient to give you a fit to some data.

\dipcom{ELFDELC}{[n1 n2 n3 n4 n5 n6 n7 n8 n9 n10]}{Deletes entries in the ELF fit co-efficient stack}
``ELF DELete Coefficients''. Deletes entries from the stack of fit
coefficients. Remaining coefficient stack entries are renumbered, so
use of \htmlref{{\tt{ELFCSL}}}{COM:ELFCSL}  immediately following \htmlref{{\tt{ELFDELC}}}{COM:ELFDELC}  is recommended.

\dipcom{ELFNEWC}{(no parameters)}{Clears the ELF fit co-efficient stack}
``ELF NEW Coefficients''. Clears the current array of fit coefficients
(N.B.\ {\em Not} the DIPSO current arrays!). The same result can be
obtained using \htmlref{{\tt{CLEAR}}}{COM:CLEAR}  inside the \htmlref{{\tt{ELFINP}}}{COM:ELFINP}  editor.

\dipcom{ELFINP}{[batch]}{Sets starting values and constraints for fitting emission lines}

``ELF INPut''. Allows you to Specify starting values and constraints
prior to optimising the line fits with ELFOPT. The ``batch'' parameter
controls what happens if the ELFINP command is invoked from within a
command file. If ``batch'' is zero (the default), the user is prompted for
the strings specifying the ELF constraints, starting values, etc (you are
alerted to this by a change of prompt). If ``batch'' is non-zero, the
strings are read from the script file. In either case, the last string
read should be ``QELF''.

The ELF command language allows starting values and constraints to be
entered prior to optimising line fits. The language is similar to
FORTRAN in the logical construction of commands; the following
operators are recognised:

\begin{verbatim}
   : = + - / *
\end{verbatim}

The last four have their conventional arithmetic meanings.

Five variables are recognised:

\begin{verbatim}
   C W I P D
\end{verbatim}

These refer to the line centre position, line width (FWHM), peak
intensity, profile type, and degree of background polynomial. (If no
value is specified for the profile type, it is assumed to be a
Gaussian; {\em i.e.} P=1). To refer to a particular line, the variable
must be followed IMMEDIATELY by the appropriate index ({\em i.e.} C1,
W5, I2 {\em etc}). Numerical constants can be input as integer or decimal
numbers, but exponents are not accepted. Blanks are permitted, but not
within variable names or numerical constants.

\begin{description}

\item [The ``:'' operator:]

Used to specify a starting value for a variable quantity, {\em e.g.}:

\begin{verbatim}
   C1:5000
   w1: 2.5
\end{verbatim}

Starting values for peak fluxes are not required.

\item [The ``='' operator:]

Used to set a fixed quantity, or relationship between two quantities; for
example,

\begin{verbatim}
   C2=5000
   w2=w1
   I3 = 2.486 * I2
   C3 = c1 - 21.6
   p1 = 6
   d=0
\end{verbatim}

The command W2=W1 constrains the width of line 2 to be the same as
that of line 1; P1=6 defines the profile type --- in this case, to the
first `numerical' profile; D=0 fixes the background polynomial to have
degree zero. (Note that the `=' operator is the only one allowed with
the D and P variables.) If D is undefined no background polynomial is
incorporated. (Note that least-squares polynomials can be fitted to
data by defining no lines and an appropriate value of D. For reasons
of numerical stability, the polynomial coefficients are computed using
an x scale centred on the mean x value of the data set; you are
informed of the value of this offset zero-point.) Other examples
should be self-explanatory.

\item [The ``+'' and ``-'' operators:]

These may ONLY be used with the variables C and W.

\item [The ``*'' and ``/'' operators:]

These may {\em only} be used with the variable I.

\end{description}

The other commands available are:

\begin{verbatim}
   HELP - gives a summary of options
   CLEAR - clears the current model specification
   QELF - return to DIPSO input
   L - list the fit specification
\end{verbatim}

There is no command for deleting the specifications for a given line
from the complete coefficient specification. If you need to `remove' a
line, it is necessary to define (using the = operator) the line
intensity to zero, and the line width and position to suitable
arbitrary values.

\dipcom{ELFOPT}{[prmpt]}{Fits a set of emission lines}
``ELF OPTimisation''. Initiates optimisation of fit coefficients (in
the sense of minimising the sum of the squares of the deviations of
the fit from the spectrum data in the DIPSO `current' arrays). In the
case of an analytical profile fit, the data may have velocity or
wavelength as the unit of the X axis. Provided WORV is set correctly,
the fitted flux will be in sensible units (ie those of the Y axis). In
the case of a `numerical' profile, spectrum data {\em must} be in
velocity units. Note that line width is {\em not} permitted as a free
parameter when fitting `numerical' profiles.

\htmlref{{\tt{ELFOPT}}}{COM:ELFOPT}  may not be fast on a busy machine if there are many free
parameters and/or datum points. Also, for complex fits, \htmlref{{\tt{ELFOPT}}}{COM:ELFOPT}  may not
converge on the correct solution. You may therefore want to monitor
the progress of a fit on an iteration by iteration basis. To do this,
you should specify the optional parameter ``prmpt" to have a value of
1 or greater; you will then be prompted after the first (and,
optionally, subsequent) iteration to give you the opportunity to exit
\htmlref{{\tt{ELFOPT}}}{COM:ELFOPT}  cleanly.

The output from a completed fit consists of the optimised parameters
and their errors (calculated in the linear approximation, from the
error matrix). The results may be stored for later inspection, or
output, using \htmlref{{\tt{ELFPUSHC}}}{COM:ELFPUSHC}. 

\dipcom{ELFPUSHC}{(no parameters)}{Pushes the current ELF co-efficients onto the co-efficient stack}
``ELFPUSH Coefficients''. Pushes the coefficients of the current ELF
fit onto the stack of fit coefficients. These results may be
inspected, printed out, or recovered to the status of current fit
coefficients using \htmlref{{\tt{ELFCSL}}}{COM:ELFCSL},  \htmlref{{\tt{ELFVUC}}}{COM:ELFVUC},  \htmlref{{\tt{ELFWRC}}}{COM:ELFWRC},  and \htmlref{{\tt{ELFPOPC}}}{COM:ELFPOPC}. 

\dipcom{ELFPOPC}{n}{Pops an entry from the ELF fit co-efficient stack}
``ELF \htmlref{{\tt{POP}}}{COM:POP}  Coefficients''. Pops a set of fit coefficients from the ELF
fit coefficient stack into the `current' coefficient arrays (where
they can be modified with \htmlref{{\tt{ELFINP}}}{COM:ELFINP}). 

\dipcom{ELFRESTC}{[filename[.typ]]}{Restores an ELF fit co-efficient stack from a binary file }
``ELF RESTore Coefficient stack''. Restores an ELF fit coefficient
stack previously saved using \htmlref{{\tt{ELFSAVEC}}}{COM:ELFSAVEC}.  The SAVEd stack of numerical
profiles is restored {\em only} if there are {\em no} numerical
profiles on the profile stack at the time of the restore. This is to
avoid ambiguity in definition of profile indices. The conditions and
features of the restore are the same as those for the DIPSO stack
restore (`RESTORE'), except that the default filename and type are
[ELFSAVE.ESTK].

\dipcom{ELFSAVEC}{[filename[.typ]]}{Saves an ELF fit co-efficient stack to disk in a binary file}
``ELF \htmlref{{\tt{SAVE}}}{COM:SAVE}  Coefficient stack''. Saves the entire fit coefficient stack
as an unformatted file that can be subsequently recovered using
\htmlref{{\tt{ELFRESTC}}}{COM:ELFRESTC}.  The stack of stored numerical profiles (if any) is also
saved. The default filename and type are [ELFSAVE.ESTK]. The data file
created is {\em not} an NDF and may therefore not be readable on operating
systems other than the one on which it was created.

\dipcom{ELFCSL}{(no parameters)}{Summarises the entries in the ELF fit co-efficient stack}
``ELF Coefficient Stack List''. Gives a summary of entries in the fit
coefficient stack. Individual entries can be inspected in more detail
using \htmlref{{\tt{ELFVUC}}}{COM:ELFVUC}. 

\dipcom{ELFVUC}{n}{Lists a single entry of the ELF fit co-efficient stack}
``ELF View Coefficients''. Gives a listing, at the terminal, of entry
'n' of the ELF fit coefficient stack. An overview of the stack can be
obtained using \htmlref{{\tt{ELFCSL}}}{COM:ELFCSL}. 

\dipcom{ELFWRC}{filename[.typ]}{Writes an ELF fit co-efficient stack to disk in a text file}
``ELF WRite Coefficients''. Writes the contents of the fit coefficient
stack to a file (default type is .DAT). The information given includes
the starting specifications for the fit, the results (if any), and the
line fluxes in units corresponding to the data stored in the DIPSO
stack.

\dipcom{ELFLFIX}{n}{Fits an emission line manually using the cursor}
``ELF Line FIX''. Allows use of the cursor to define a `fixed' ({\em
i.e.} non-optimisable) line in the data. Its principal use is to (in
effect) take out features in the far wings of lines that are being
optimised; these features might otherwise adversely affect the final
fit.

Typing \htmlref{{\tt{ELFLFIX}}}{COM:ELFLFIX}  brings up the cursor; two hits are then required. The
first locates the centre and peak flux of the feature. (These data are
added to the `current' fit coefficients in the same way as using `='
in \htmlref{{\tt{ELFINP}}}{COM:ELFINP}.)  The second locates the half intensity point, on either
side of the feature, to fix the half-width at half maximum (from which
the FWHM follows). (Hitting the same point twice leaves the width
undefined.)

\dipcom{ELFPIN}{n}{Moves data from the DIPSO stack to the ELF profile stack}
``ELF Profile INput''. Transfers a numerical profile from DIPSO stack
entry `n' to storage in the profile stack. The data {\em must} have
units of velocity along the X axis; spectra to which numerical
profiles are fitted must also be in velocity space.

\dipcom{ELFPL}{(no parameters)}{Lists the contents of the ELF profile stack}
``ELF Profile List''. Lists the contents of the profile stack (by
giving the title of the original DIPSO stack entry).

\dipcom{ELFPUSH}{[n1 n2]}{Pushes a data array representing an ELF fit onto the DIPSO stack}
``ELFPUSH fit''. Pushes the result of an ELF fit onto the DIPSO
stack, as a continuous ({\em i.e.} no breaks) spectrum. If no
arguments are specified, the complete fit is pushed, into the next
available DIPSO stack position. If `n1' is specified, only the fit for
line n1 is pushed. If `n2' is also specified, the fits for lines n1 to
n2 inclusive are pushed. If n1 is specified, the background polynomial
(if any exists) is also pushed, before the line(s).

\dipcom{ENV}{name}{Displays the value being used by DIPSO for a specified
environment variable} 
The string value being used for the specified environment variable is
displayed. An error is reported if the no value is available.

DIPSO uses environment variables to specify various directories (eg
\verb+DIPSODIR+), and also the prompt string (\verb+DIPSOPROMPT+).
Default values for these environment variables can be supplied
on the DIPSO command line in a comma separated list of ``name=value''
pairs. These default values will be used if the corresponding environment
variables are not defined. 

\dipcom{ERASE}{[n]}{Erases a plotting zone created by {\tt{TZONE}}} 
Erases plotting zone ``n'' ({\it c.f} \htmlref{{\tt{TZONE}}}{COM:TZONE}).  This is done rather slowly
unless n=0 (the default).

\dipcom{EW}{[mode]}{Measures equivalent widths}
Measures equivalent widths. The cursor is used to define two pairs of
(X,Y)points, between which the equivalent width is measured with
respect to a linear `continuum'; if you need a more complex continuum,
the data can be preprocessed using \htmlref{{\tt{CREGS}}}{COM:CREGS}  together with \htmlref{{\tt{PF}}}{COM:PF},  or using
\htmlref{{\tt{CDRAW}}}{COM:CDRAW}.  To terminate the equivalent width measuring session, define
X2$<$X1. Note that \htmlref{{\tt{EW}}}{COM:EW}  expects to measure spectra with monotonically
increasing X values.

Errors on the measured equivalent widths are calculated using the
prescriptions given by Howarth and Phillips (MNRAS 222, 809, 1986);
these errors are likely to make most sense for interstellar line
measurements. The assumed nature of the errors is controlled by the
value of MODE; the default for MODE is 0. In this case, if the data
have been processed with \htmlref{{\tt{PF}}}{COM:PF}  immediately before using \htmlref{{\tt{EW}}}{COM:EW},  an estimate
of the continuum signal-to-noise is available. This results in
automatic generation of `statistical' errors on the equivalent width
measurements, under the (conservative) assumption that noise (rather
than signal-to-noise) is constant. Such an assumption is likely to be
reasonable for IUE data. Note that on termination of the \htmlref{{\tt{EW}}}{COM:EW}  command
the `statistical' error is LOST. This is a feature, not a bug, and is
intended to stop you making mistakes through oversight. If, after
terminating \htmlref{{\tt{EW}}}{COM:EW},  you find that you want to do further measurements on
the same data, then the sequence:

\begin{verbatim}
   PUSH,PF 0,ADIV m,PM
\end{verbatim}

(where `m' is the STACK entry into which the data are PUSHed) will
normally recover the error estimates. (The \htmlref{{\tt{CREGS}}}{COM:CREGS}  continuum regions are
remembered, and \htmlref{{\tt{PF}}}{COM:PF}  0 will fit a horizontal line to these regions in
the already rectified data, recomputing the error estimate. The Y
coordinate of this line should be 1.0).

If MODE is given a non-zero value, `statistical' errors are calculated
under the assumption of Poisson statistics. Such an assumption may be
considered by some (though not necessarily the author of this
document) to be appropriate to IPCS data. Since the uncertainties on
individual points are not stored, but are calculated (on a square-root
basis) during the \htmlref{{\tt{EW}}}{COM:EW}  measurement, it is essential that {\em
unrectified} data are interrogated in this mode. It may, of course,
prove convenient to overplot a `continuum' to assist interpretation.

In addition to the `statistical' ({\em i.e.} random) errors calculated
by \htmlref{{\tt{EW}}}{COM:EW},  allowance for systematic errors in setting the zero and
continuum levels can be made, using the \htmlref{{\tt{EWERR}}}{COM:EWERR}  command.

WARNING: \htmlref{{\tt{EW}}}{COM:EW}  operates on the data stored in the `current' arrays. Be
careful not to make the mistake of plotting STACK data, and then
trying to measure equivalent widths off of the data on the screen. The
calculated value of \htmlref{{\tt{EW}}}{COM:EW}  is multiplied by the factor WORV.

\dipcom{EWERR}{ErrC Err0}{Estimates the systematic errors in continuum and zero-level values}
Provides the program with estimates of the {\em systematic} errors in
continuum and zero-level placement (expressed as percentages of the
continuum level). These errors are then incorporated into subsequent
error analysis when determining equivalent widths with \htmlref{{\tt{EW}}}{COM:EW}. 

These systematic errors are assumed to propagate quadratically (see
MNRAS 222, 809, 1986).

\dipcom{EXIT}{}{Exits DIPSO, saving the stack in a binary file}
Exits DIPSO, saving the stack in a file called EXIT.STK, or EXIT\_STK.sdf.

\dipcom{EXPAND}{Factor [clist]}{Changes the sizes of individual components of a plot}
Expands components of a plot by a factor ``Factor" with respect to the default
size on a given device. (Factor is absolute, not relative; {\em e.g.} the
sequence \htmlref{{\tt{EXPAND}}}{COM:EXPAND}  2, \htmlref{{\tt{EXPAND}}}{COM:EXPAND}  3 gives 3x enlargement, not 6x.)

If no value is supplied for {\tt{clist}},  then the supplied expansion factor
applies to all the components of the plot listed below. If a value is supplied
for {\tt{clist}},  it must be a string containing some sub-set of the characters
A, I, N, T, M and P. Each character refers to a different component of the plot
as follows:

\begin{dipdesc}
\dipitem{A}{ - M(A)jor tick marks.}
\dipitem{I}{ - M(I)nor tick marks.}
\dipitem{N}{ - (N)umerical axis labels.}
\dipitem{T}{ - (T)extual axis labels.}
\dipitem{M}{ - (M)arkers.}
\dipitem{P}{ - Text created using the (P)WRITE command.}
\end{dipdesc}

The expansion factor is only applied to those components of the plot for which
the corresponding letters are included in the {\tt{clist}}  parameter. The
expansion factor for all other components is left unchanged.

\dipcom{FILL}{(no parameters)}{Causes symbols created by {\tt{MARK}} to be filled in}
Results in filled-in \htmlref{{\tt{MARK}}}{COM:MARK}  symbols ({\it c.f} \htmlref{{\tt{NOFILL}}}{COM:NOFILL}). 

\dipcom{FLUX}{[filename[.typ]] }{Measures fluxes with respect to a linear 'continuum'}
Measures fluxes with respect to (=above) a linear `continuum' defined
using pairs of cursor hits. More complex continua must be rectified
out, using \htmlref{{\tt{CDRAW}}}{COM:CDRAW},  or \htmlref{{\tt{CREGS}}}{COM:CREGS}  together with \htmlref{{\tt{PF}}}{COM:PF}.  To end a \htmlref{{\tt{FLUX}}}{COM:FLUX}  measuring
session, input X2$<$X1. Note that \htmlref{{\tt{FLUX}}}{COM:FLUX}  expects to measure spectra with
monotonically increasing X values.

Fluxes are obtained by trapezoidal integration between X,Y points.
Linear interpolation is used across `breaks' in the data (and a
warning given).

If a file name is supplied (the default file type is .DAT) then the
x1, x2 and flux values are output to the file for future reference.
The file remains open until (i) you exit from the program; (ii) a new
file name is provided; or (iii) \htmlref{{\tt{FLUX}}}{COM:FLUX}  0 is typed in. (The last option
closes any file that is open [but does not open a file called 0.DAT].)
This means that, for example, \htmlref{{\tt{FLUX}}}{COM:FLUX}  can be exited, the X and Y ranges
changed and a new plot generated, then \htmlref{{\tt{FLUX}}}{COM:FLUX}  re-entered, and data will
continue to be output to the previously opened file.

WARNING: \htmlref{{\tt{FLUX}}}{COM:FLUX}  operates on the data stored in the `current' arrays. Be
careful not to make the mistake of plotting STACK data, and then
trying to measure fluxes off the data on the screen. The integrated
flux is multiplied by the factor WORV.

\dipcom{FONT}{n}{Selects font quality}
Selects font quality. Permitted values of `n' are 0 (hardware
characters), 1 (SGS characters), and 2 (NCAR characters). These are in
increasing order of elegance, and execution time! The resulting fonts
are used for all text on the plot ({\em i.e.} apply to \htmlref{{\tt{XLAB}}}{COM:XLAB},  \htmlref{{\tt{YLAB}}}{COM:YLAB}, 
\htmlref{{\tt{PWRITE}}}{COM:PWRITE},  and \htmlref{{\tt{TITLE}}}{COM:TITLE}  when the title is plotted).

\htmlref{{\tt{FONT}}}{COM:FONT}  0 (the default) is not always elegant, and because of problems
with text rotation some devices will write the Y axis label ``upside
down''. If this really worries you you can get round it with \htmlref{{\tt{YLAB}}}{COM:YLAB} 
({\em e.g.} use \htmlref{{\tt{YLAB}}}{COM:YLAB}  xulF).

The elaborate \htmlref{{\tt{FONT}}}{COM:FONT}  2 style gives access to a wide range of special
characters (Greek, italics, mathematical symbols, {\em etc.}) These
cannot be reproduced in this document, but are given in full in
\xref{SUN/90}{sun90}{}
(which documents the routines which DIPSO utilises). The following
formats serve to illustrate the possibilities:

\begin{dipdesc}
\dipitem {'PGU'}{ gives uppercase Greek}
\dipitem {'PGL'}{ gives lowercase Greek}
\dipitem {'B'}{ gives subscripts ("B"elow)}
\dipitem {'S'}{ gives superscripts}
\dipitem {'N'}{ gives normal (ie not sub- or super-script)}
\dipitem {'PRU'}{ gives Roman characters}
\end{dipdesc}

Thus to give the formula for the area of a circle as a title:

\begin{verbatim}
   TITLE "Area = 'PGL'P'PRU'r'S'2"
\end{verbatim}

Note that the single quotes surrounding the style definition strings
are mandatory, as are the uppercase specifications. Note, too, that
the correspondence between Greek and Roman characters is not usually
as obvious as P=``pi''; it is important to check a printed (not
line-printer!) copy of \xref{SUN/90}{sun90}{} for details.

An angstrom symbol ({\AA}) is provided as a special character: `.A'
(in `PRU' style). Other complex characters can be constructed (see
\xref{SUN/90}{sun90}{}); for example, a mass-loss rate symbol (M surmounted by a dot)
is obtained using {\tt{"'PRU'M'H:-85V105PRU'.'H:85V-105'"}}. 

Because the codes for special characters are flagged by apostrophes,
life gets complicated if you want to include a normal apostrophe in a
tring. But not too complicated: you just give it twice. For example,
if you want a title that says:

\begin{center}
Howarth's Fudged Data
\end{center}

you must specify:

\begin{verbatim}
   TITLE "Howarth''s Fudged Data"
\end{verbatim}

to get it (in font~2).

\dipcom{FORMWR}{filename[.typ]}{Writes the contents of the current arrays to a text file}
Results in a formatted write of the contents of the `current' arrays.
The resulting data are in a form suitable for output on a line
printer. The default file type ([.typ]) is '.DAT'.

\dipcom{FRAME}{xsize ysize [locator index]}{Sets up a plotting area of a given absolute size}
Defines a plotting area in absolute (device-independent) terms. The
xsize and ysize parameters are in cm; the locator index follows the
numeric keypad, i.e. 1=bottom left, 9=top right {\em etc.} The
locator index defaults to 5 (centre of plotting zone). To return to
(device-dependent) plotting zones use \htmlref{{\tt{TZONE}}}{COM:TZONE}. 

Subzones within the specified \htmlref{{\tt{FRAME}}}{COM:FRAME}  can be set using \htmlref{{\tt{FRZONE}}}{COM:FRZONE}. 

\dipcom{FRZONE}{x1 x2 y1 y2}{Divides a plotting area (set up by {\tt{FRAME}})  into several sub-frames}
Defines subzones within a plotting area specified by \htmlref{{\tt{FRAME}}}{COM:FRAME}.  The
parameters x1 {\em etc.} are in the range 0-1, and define the
fractional position of the subzone within the frame. An example of the
use of \htmlref{{\tt{FRAME}}}{COM:FRAME}  and \htmlref{{\tt{FRZONE}}}{COM:FRZONE}  can be found in {\tt{DIPSODIR:DEMO2.CMD}} 
({\tt{\$DIPSODIR/demo2.cmd}}. 

\dipcom{FTFILTER}{n1 [f1 f2 n2]}{Filters high frequency components from a Fourier transform}
Filters high-frequency components (which you may identify with noise)
from the real and imaginary parts of a Fourier Transform (see \htmlref{{\tt{FTRANS}}}{COM:FTRANS}). 
\htmlref{{\tt{FTFILTER}}}{COM:FTFILTER}  expects the real part of the FT to be in stack entry ``n1",
and the imaginary part to be in entry ``n2" (which defaults to n1+1).

\htmlref{{\tt{FTFILTER}}}{COM:FTFILTER}  operates by constructing an ``exponent-squared edge
function", and multiplying the FT by that filter (see Bracewell, ``The
Fourier Transform and its Applications", for details). The filter has
a value of unity up to frequency f1, then drops like exp(D[nu]**2),
where D[nu] is the change in frequency. The e-folding scale is
determined by specifying f2, the frequency at which the filter drops
to a value of 0.01.

The filtered real and imaginary parts are pushed onto the stack,
together with the filter used.

The frequencies f1 and f2 will depend critically on your application;
defaults are provided merely to provide an illustration for first-time
users. These defaults are (Nu1+Nu2)/2 for f1, and MIN(Nu2,1.1*f1) for
f2, where Nu1 and Nu2 are the first and last frequencies in the
data set to be filtered.

\dipcom{FTINV}{n1 [n2]}{Takes the inverse Fourier transform of two stack entries}
Computes the inverse fourier transform from the real and imaginary
parts of the FT, where the real part is in stack entry n1, and the
imaginary part is in stack entry n2 (which defaults to n1+1). The
result is pushed onto the stack (and WORV=1 assumed).

\dipcom{FTRANS}{n [p]}{Takes the forward Fourier transform of a stack entry}
Computes the Fourier Transform of stack entry ``n'', pushing the
resulting real and imaginary parts onto the stack, together with the
power spectrum. A fraction ``p'' of the data set is endmasked (at each
end) with a cosine bell; ``p'' defaults to 0.05. For best results you
should subtract a continuum (or at the very least the mean value) from
the data, in order to maximise the effectiveness of the endmasking.

\htmlref{{\tt{FTRANS}}}{COM:FTRANS}  does not use the Fast Fourier Transform algorithm. It does not,
therefore, require the number of datum points to be an integer power
of two (at the price of being slow for large data sets). However, it is
necessary for the data to be uniformly sampled; if \htmlref{{\tt{FTRANS}}}{COM:FTRANS}  determines
that your data set does not contain regularly sampled X values it will
automatically perform a 4-point Laguerre interpolation in order to
simulate such data. For applications where the data sampling rate is
very irregular, or where there are substantial gaps ({\em e.g.} in
light-curves), it is recommended that \htmlref{{\tt{PDGRAM}}}{COM:PDGRAM}  (and \htmlref{{\tt{PDGWINDOW}}}{COM:PDGWINDOW})  be used
for Fourier analysis applications.

The X values in any data set must be monotonically increasing.

It is recommended that you familiarise yourself with ({\em e.g.})
Brault \& White, A\&A 13, 169 (1971) before using \htmlref{{\tt{FTRANS}}}{COM:FTRANS}. 

\dipcom{GRID}{X1 X2 DX}{Creates an array of uniform wavelength values with zero Y values}
Creates a grid of uniform wavelength points, starting at
Lambda(1)=MIN(X1,X2). The Y values are set to zero. This command can
be used with \htmlref{{\tt{AADD}}}{COM:AADD}  to remap data, or to set up a wavelength grid for
({\em e.g.}) \htmlref{{\tt{NEBCONT}}}{COM:NEBCONT}.  Data are left in the `current' arrays.

\dipcom{GRIDSTYLE}{mode}{Controls the axes, grids and labels used to produce a plot}
Controls the design of plots.

\begin{itemize}
\item Mode 1 is the ``ordinary" (4 sides and labels) design;
\item Mode 2 results in a grid drawn over the plot (like course graph paper);
\item Mode 3 gives just the bottom and left-hand axes;
\item Mode 4 gives no box at all, and no labels;
\item mode 5 gives bottom and left-hand axes but no labels.
\end{itemize}

\dipcom{HANDLER}{level}{Switches the DIPSO error handler on and off}
The DIPSO error handler makes the program uncrashable (in principle!). Setting
level=0 results in the system error handler being invoked; fatal errors will
kill the program. Any level greater than zero (such as the default value of 1)
causes the
program to issue a warning message when an error occurs (including control-C
interupts). The current command is aborted and the user is returned to the main
DIPSO command prompt.

To familiarise yourself with the condition handler you can type `CRASH'.


\dipcom{HC}{N}{Creates a theoretical profile for the Hydrogen Lyman-alpha line}
Calculates a theoretical (fully damped) profile for the Hydrogen
Lyman-alpha line (1216 A), with a column density `N'. The profile is
calculated at the X grid in the current arrays, with a continuum level
of unity. If the parameter N is less than 30 it is assumed to
represent log(N).

The most convenient way of using this function is to calculate a
theoretical profile, use \htmlref{{\tt{ADIV}}}{COM:ADIV}  to divide observed data (in the STACK)
by it, then plot the resulting data. The `best' value of the
interstellar H I column is that which gives the `flattest' continuum
around 1216. Of course, interpretation is your responsibility, and the
intrinsic stellar Lyman-alpha profile should always be considered,
together with the possibility that the interstellar line is not fully
damped.

\dipcom{HELP}{[string]}{Displays full help on specified commands}
If no value is supplied for `string' then general help information for
the DIPSO package is displayed. More verbose assistance on a particular
command can be obtained by typing `{\tt{HELP} $<$command name$>$}'. The
information is displayed in one of two formats: plain text, or hypertext.
The hypertext information is taken from SUN/50 and is displayed using a
World-Wide-Web browser. By default the {\em Mosaic} browser is used, but
this can be changed by assigning the required browser command to the
environment variable {\tt HTX\_BROWSER} before entering DIPSO. For
example, to use {\em Netscape} issue the following command before
starting DIPSO:

\begin{verbatim}
   % setenv HTX_BROWSER netscape
\end{verbatim}

If a browser of the requested variety has already been fired-up prior to
starting DIPSO, it is ``hi-jacked'' to show the requested help
information. Otherwise, a new browser is fired-up.

If this command fails, saying that the ``showme'' command cannot be found, 
try defining the showme command explicitly using
\htmlref{{\tt{USEHTX}}}{COM:USEHTX}.

In plain text mode, the information is taken from
{\tt{\$DIPSODIR/dipso.hlp}} and {\tt{\$DIPSODIR/help.lis}}, and is
displayed in the DIPSO command window.

By default, the plain text format is used, but this can be changed using the
\htmlref{{\tt{USEHTX}}}{COM:USEHTX} command. 

Typing an interrogative (``?'' or ``?$<$command name$>$'') has the same effect 
as typing {\tt{HELP}}.

\dipcom{HIST}{(no parameters)}{Causes plots to be done in histogram-style}
Plots to be done histogram-style (as opposed to \htmlref{{\tt{POLY}}}{COM:POLY}  or \htmlref{{\tt{MARK}}}{COM:MARK}). 


\dipcom{HPROT}{[mode]}{Rotates between {\tt{HIST}} and {\tt{POLY}} plots on a given diagram}
Rotates between Hist and Poly plots on a given diagram, starting with
whichever style is currently in force (selected with \htmlref{{\tt{HIST}}}{COM:HIST}  or \htmlref{{\tt{POLY}}}{COM:POLY} 
command); can be useful when comparing ({\em e.g.}) observations and
models. If mode is 1 (default is 0) then the first data set is plotted
in the current style ({\em i.e.} histogram if \htmlref{{\tt{HIST}}}{COM:HIST}  is in force), and
all subsequent data sets plotted on the given diagram will appear in
the alternative style. The cycle can be re-initialised at any time
with \htmlref{{\tt{HIST}}}{COM:HIST}  or \htmlref{{\tt{POLY}}}{COM:POLY}. 

Negated with \htmlref{{\tt{NHPROT}}}{COM:NHPROT}. 

\dipcom{INTEGRATE}{(no parameters)}{Estimates the area under the current arrays}
Estimates the area under the current arrays using simple trapezoidal
integration (and will therefore only work successfully for data sets
with monotonically increasing X values).

\dipcom{INTERP}{n}{Re-grid current arrays onto a regular grid}
Applies Laguerre interpolation to data in the current arrays, to give
a regularly sampled data set with the same number of points as the
original. The parameter ``n" controls the order of the interpolation.

\htmlref{{\tt{INTERP}}}{COM:INTERP}  does (n-1)th order interpolation, where `n' {\it must}\/ be an
even number; thus only odd orders greater than zero are handled.
For example, n=2 gives 1st order ({\em i.e.} linear) interpolation;
n=4 gives 3rd order (i.e. quartic) interpolation. (INTERP cannot
supply quadratic interpolation, since this would require n=3 and odd
values of `n' are not allowed.)

The data must have monotonically increasing X values.

Note that spline interpolation can also be executed, using the \htmlref{{\tt{CDRAW}}}{COM:CDRAW} 
command.

\dipcom{ISATM}{[wavelength]}{Reads atomic data from a disk file for use with {\tt{ISCALC}} and {\tt{ISCOG}}} 
Supplies atomic data for use with \htmlref{{\tt{ISCALC}}}{COM:ISCALC}  and \htmlref{{\tt{ISCOG}}}{COM:ISCOG}.  If a wavelength (in
Angstroms) is provided, then DIPSO searches for a file called ATOMIC.DAT in the
current directory; if it fails, it looks for one in a directory with the
environment variable OWNERDIR; and if it again fails, it opens
{\tt{\$DIPSODIR/ATOMIC.DAT}}.  Once a
file has successfully been opened, DIPSO searches for a line whose wavelength
falls within 0.1A of the argument wavelength to \htmlref{{\tt{ISATM}}}{COM:ISATM}.  If it finds it, it
uploads the associated atomic data.

The file {\tt{\$DIPSODIR/ATOMIC.DAT}}  uses data based on MNRAS 222, 809 (1986)
and references therein, and provides a model for the required format
if you want to set up your own file.
``Comment cards" prefixed by ``*'' or ``!'' are permitted.

If a wavelength is not provided, or if \htmlref{{\tt{ISATM}}}{COM:ISATM}  fails to find a data set
corresponding to a specified wavelength, then a ``data edit'' mode is
entered. This allows you to change or input atomic data, or to carry
out further searches through an {\tt{ATOMIC.DAT}}  file; typing H in
this mode gives more information.

(Incidentally, in case you're wondering, {\tt{ATOMIC.DAT}}  requires
statistical weights in order to permit calculation of damping
constants.)

\dipcom{ISCALC}{(no parameters)}{Creates a theoretical absorption profile for an interstellar cloud}
Calculates a theoretical absorption profile for an interstellar cloud
(or other plane-parallel slab of absorbing material with negligible
forward scattering and a Gaussian line-of-sight velocity distribution
for the absorbers), and pushes the result onto the stack.

Before this command can successfully be invoked, the atomic data must
already have been loaded (with \htmlref{{\tt{ISATM}}}{COM:ISATM}),  a cloud model defined (with
\htmlref{{\tt{ISINP}}}{COM:ISINP}),  and a set of cloud `options' defined (with \htmlref{{\tt{ISOPT}}}{COM:ISOPT}). 

\dipcom{ISCOG}{(no parameters)}{Calculates a Curve-Of-Growth for an interstellar cloud model}
Calculates a Curve-Of-Growth for a cloud model defined with \htmlref{{\tt{ISINP}}}{COM:ISINP}, 
using atomic data input via \htmlref{{\tt{ISATM}}}{COM:ISATM}.  The COG is pushed onto the stack,
with X values (N.f.lambda) in units of (cm-2.dimensionless.Angstroms)
and Y values (W[lambda]/lambda) assuming equivalent width and
wavelength to be measured in the same unit.

\dipcom{ISINP}{(no parameters)}{Sets up an interstellar cloud model}
Invokes a cloud ``editor'', which permits an interstellar cloud model
to be defined prior to calculating theoretical profiles with \htmlref{{\tt{ISCALC}}}{COM:ISCALC}. 
Up to 18 clouds are permitted; each is specified by a velocity
dispersion parameter, ``b'' (see, {\em e.g.}, MNRAS 222, 809 [1986]
for a definition), a central velocity, V, and a column density, N.

Typing \htmlref{{\tt{HELP}}}{COM:HELP}  while in the cloud editor provides more information.

\dipcom{ISOPT}{[filename[.typ]]}{Sets up options required by {\tt{ISCALC}}} 
Loads a variety of options required before \htmlref{{\tt{ISCALC}}}{COM:ISCALC}  can be successfully
executed. These include specifications for (optional) convolution with
an instrumental resolution function, blending with other lines, {\em
etc.} \htmlref{{\tt{ISOPT}}}{COM:ISOPT}  invokes an option ``editor''; typing H in this edit mode
gives more information. Alternatively, if a filename is specified on
the command line, an attempt is made to read options from that file.
Such a file should contain information matching the \htmlref{{\tt{ISOPT}}}{COM:ISOPT}  editor
input; for example, if the file contains:

\begin{verbatim}
   V1=-100
   V2=+100
\end{verbatim}

line profiles will be calculated over (at least) the range -100 km/s
to +100 km/s; default values for all other options will be assumed.
(The default file name is \htmlref{{\tt{ISOPT}}}{COM:ISOPT},  and the default extension is .DAT.)

\dipcom{IUECOR}{camera year day [aperture]}{Applies 'aging' corrections for the IUE cameras}
Applies the `aging' corrections for the IUE cameras described by
Bohlin and Grillmair ({\it Ap.~J.~Sup.,} {\bf 66,} 209, 1988; SWP)
and by Clavel {\it et al.} (ESA IUE Newsletter No.~26, p.~65, 1986;
LWR). The `camera' parameter is 2 for LWR and 3 for SWP; the `day'
parameter is day number in the year. The `aperture' parameter is
relevant only to SWP data, and is 1 for trailed spectra, 2 for small
aperture spectra, and 3 (the default) for large aperture point source
spectra. \htmlref{{\tt{IUECOR}}}{COM:IUECOR}  works on data in the current arrays, replacing
fluxes therein by their corrected values.

\dipcom{LABON}{(no parameters)}{Turns axis labelling back on}
Turns axis labelling back on (after use of \htmlref{{\tt{NLAB}}}{COM:NLAB}). 

\dipcom{LOGAXX}{t/f}{Causes X axes to be plotted on a log10 scale}
If set `True', the X axis will be plotted on a log10 scale.

\dipcom{LOGAXY}{t/f}{Causes Y axes to be plotted on a log10 scale}
If set `True', the Y axis will be plotted on a log10 scale.

\dipcom{LOGX}{(no parameters)}{Takes the base 10 logarithms of the current X values}
Replaces the X values in the current arrays by their base 10 logarithms.

\dipcom{LOGY}{(no parameters)}{Takes the base 10 logarithms of the current Y values}
Replaces Y values in the current arrays by their base 10 logarithms.

\dipcom{LWEIGHT}{weight}{Sets the weight (heaviness) of all plotted lines}
Alters the weight ({\em i.e.} `heaviness') of all plotted lines, on
devices which support this feature. The weight must be 1-5 (initial
setting is 1). See also \htmlref{{\tt{TWEIGHT}}}{COM:TWEIGHT}. 

\dipcom{MARK}{(no parameters)}{Causes plots to be done using symbols}
Plots to be done using symbols, as opposed to \htmlref{{\tt{HIST}}}{COM:HIST}  or \htmlref{{\tt{POLY}}}{COM:POLY}.  The
symbols may be `designed' using \htmlref{{\tt{MSET}}}{COM:MSET},  and expanded and rotated using
\htmlref{{\tt{EXPAND}}}{COM:EXPAND}  and \htmlref{{\tt{ANGLE}}}{COM:ANGLE}. 

\dipcom{MEAN}{(no parameters)}{Find the mean and standard deviation of the current Y values}
Calculates the mean and standard deviation of the Y data stored in the
`current' arrays.

\dipcom{MERGE}{S1 S2 WT1 WT2 [MODE]}{Merges two arrays }
Merges STACK entries S1 and S2, with weights WT1 and WT2 respectively.
This function is an alternative to the \htmlref{{\tt{AADD}}}{COM:AADD}  and \htmlref{{\tt{YMULT}}}{COM:YMULT}  commands,
differing in the respect that if there is a gap in one (but not both)
data sets being manipulated, then the MERGEd data set will not have a
gap. The data sets in the STACK are both required to have monotonically
increasing X values.

If there is any region of overlap between the two data sets, then the Y
values of the data set associated with the larger value of X(1) are
mapped onto the X grid of the other data set. This remapping is done
using a triangular filter, so that some slight smoothing can result if
the sampling rates in the two data sets are very different. Remapping
is carried out ONLY in any overlap region.

This function is intended for merging data sets with (normally)
overlapping, but not necessarily identical, X ranges. In particular,
it can be used for merging IUE SWP and LWR flux calibrated spectra,
and UV and optical data ({\em e.g.} UBV fluxes; see \htmlref{{\tt{UBVRD}}}{COM:UBVRD}).  Because
remapping is done linearly, it is also suitable for averaging similar
spectra.

If MODE=zero (the default value) and the data look like IUE SWP and
LWR spectra (on the grounds of the X ranges), a wavelength-dependent
weighting is used which reflects the inverse sensitivity function of
IUE. If you ARE dealing with IUE data, then inputting the exposure
times as the WTs and using MODE=0 will result in (essentially) a \htmlref{{\tt{MERGE}}}{COM:MERGE} 
at the 'FN' level, which is probably more correct than one at the
`absolute flux' level. If MODE is positive, the additional IUE
weighting is switched off. (Note that if you are NOT working with IUE
data MODE=0 and MODE=1 will give the same result).

If MODE=-1, a straight addition of data in the overlap region is
carried out. This option may be useful for (eg) combining emission
line models with a zero background level. If MODE=-2 (or less), the
two data sets are multiplied in the overlap region. This option may be
useful for ({\em e.g.}) combining absorption line models with a
background level of unity. If MODE is negative, `dummy' values of Wt1
and Wt2 MUST be supplied.

You should only merge data which you expect to have similar Y values
over any range of overlap. Merging dissimilar data sets ({\em e.g.} SWP
and LWR IUE spectra which have not been flux calibrated) will give
unsatisfactory results.

\dipcom{MONGOWR}{filename [badval]}{Writes a file which can be read by MONGO}
Writes a file which can subsequently be read into MONGO. It is
possible to make MONGO leave gaps in plots, corresponding to datum
points with a notifiable `bad' y value; the \htmlref{{\tt{MONGOWR}}}{COM:MONGOWR}  command inserts a
`bad' point with y value `badval' (default 0) into breaks in the
data set to facilitate use of this option.

Most of the more important MONGO functions are reproduced in DIPSO -
you shouldn't have to use this command!

\dipcom{MROT}{(no parameters)}{Causes automatic rotation of plotting symbols}
Implements automatic rotation of plotting symbols. Each plot begins
with the symbol type specified by the last use of \htmlref{{\tt{MARK}}}{COM:MARK}  (or symbol 1,
if \htmlref{{\tt{MARK}}}{COM:MARK}  hasn't been called).

NOT IMPLEMENTED AT PRESENT.

Turned off with \htmlref{{\tt{NMROT}}}{COM:NMROT}. 

\dipcom{MSET}{style nvert}{Selects the symbol to be used if {\tt{MARK}} plots are being created}
Selects the \htmlref{{\tt{MARK}}}{COM:MARK}  plotting symbol. `NVERT' is the number of vertices
the symbol has; `STYLE' takes on values 1-4. Style=1 is a polygon,
Style=2 a `star' design, style=3 is an asterix, Style=4 gives an arrow
symbol, for plotting lower or (with \htmlref{{\tt{ANGLE}}}{COM:ANGLE})  upper limits.

\dipcom{NB}{(no parameters)}{Stops automatic clearing of the plotting frame between plots}
`No Box': switches off automatic clearing off the plotting frame
between plots. (Inverse function is \htmlref{{\tt{BOX}}}{COM:BOX}).  On a device or zone change,
the `box' is switched `on' for the first plot, regardless of any \htmlref{{\tt{NB}}}{COM:NB} 
call.

\dipcom{NCROT}{(no parameters)}{Stops automatic rotation of the colour table}
Stops automatic rotation of the colour table when Ikon plotting and
resets the colour index to 1 (white). (See \htmlref{{\tt{CROT}}}{COM:CROT}  and \htmlref{{\tt{CSET}}}{COM:CSET}). 

\dipcom{NEBCONT}{filename[.typ] [mode1 mode2 mode3]}{Calculates a theoretical nebular re-combination continuum}
Calculates a theoretical nebular recombination continuum. The
requisite data are comparatively numerous, and so are accessed from a
file. The contents of the file must be as follows:

\begin{verbatim}
   Line 1: Te(1-4)
   Line 2: Ne
   Line 3: Log10 H(beta) flux [and Log10 He(1640) flux]
   Line 4: C
   Line 5: A[He(1+)] A[He(2+)]
   Line 6: A[N(1+)] A[N(2+)] A[N(3+)] A[N(4+)]
   Line 7: A[C(1+)] A[C(2+)] A[C(3+)] A[C(4+)]
   Line 8: A[O(1+)] A[O(2+)] A[O(3+)] A[O(4+)]
   Line 9: A[Ne(1+)] A[Ne(2+)] A[Ne(3+)] A[Ne(4+)]
\end{verbatim}

where:

\begin{quote}
\begin{description}

\item [Te(i)] is the electron temperature appropriate to ions of
charge i+, in units of $10^{4}$K;

\item [Ne] is the electron density in cm-3;

\item [H(beta)] flux is the observed value, in cgs units; if the flux
is unknown, enter a value of 0.0 and follow it (on the same line) with
the log of the He II (1640) flux, in the same units;

\item [C] is the logarithmic extinction at H(beta) [C=1.44*E(B-V) for
a standard extinction law];

\item [A(X)] is the ionic abundance, in the form 1000*N(X)/N(H+).

\end{description}
\end{quote}

A specimen file is given in {\tt{\$DIPSODIR/NEBCONT.DAT}}. 
If \htmlref{{\tt{NEBCONT}}}{COM:NEBCONT}  fails to find the
file in the current directory, it tries to look in a directory
assigned the environment variable {\tt{OWNERDIR}}  (which
you can define in your {\tt{.login}}  script or your {\tt{LOGIN.COM}}  procedure).

All data in the file are reproduced at the terminal. By default,
(mode1=mode2=mode3=0), new values are prompted for (simply hit
`return' to get the file values). By setting mode1=1, abundances are
listed but not prompted for (fluxes, C, temperatures and densities
still prompted). Mode2=1 does the same for the H(beta) flux and `C';
and mode3=1, the same for temperatures and density.

The results are calculated at the X co-ordinates of data in the
`current' arrays, and pushed onto the stack. If you are not modelling
real data, \htmlref{{\tt{GRID}}}{COM:GRID}  can be used to set up an X array.

The default file type is {\tt{.DAT}}. 

\dipcom{NECHO}{(no parameters)}{Stops the echoing of commands read from a command file}
Has the same effect as \htmlref{{\tt{ECHO}}}{COM:ECHO}  -1.

\dipcom{NLAB}{[mode]}{Turns off axis labelling}
Selectively turns off axis labelling (modes not yet implemented;
\htmlref{{\tt{NLAB}}}{COM:NLAB}  currently turns off X and Y axis labels, and title).

\dipcom{NHPROT}{(no parameters)}{Stops rotating plots styles between {\tt{HIST}} and {\tt{POLY}}}
Stops rotating plot style between Hist and Poly (after \htmlref{{\tt{HPROT}}}{COM:HPROT}). 

\dipcom{NMROT}{(no parameters)}{Stops automatic rotation of {\tt{MARK}} plotting symbols}
Stops automatic rotation of plotting symbols and sets the \htmlref{{\tt{MARK}}}{COM:MARK}  index to 1.
(See \htmlref{{\tt{MROT}}}{COM:MROT}  and \htmlref{{\tt{MARK}}}{COM:MARK}). 

\dipcom{NOBEEP}{(no parameters)}{Stops the terminal beeping after every error}
Stops the terminal beeping every time you make a boo-boo.

\dipcom{NOFILL}{(no parameters)}{Causes {\tt{MARK}} symbols to be plotted open (unfilled)}
Results in \htmlref{{\tt{MARK}}}{COM:MARK}  symbols being plotted as open (unfilled) figures;
{\it c.f.} \htmlref{{\tt{FILL}}}{COM:FILL}. 

\dipcom{NTROT}{(no parameters)}{Stops automatic rotation of line attributes}
Stops automatic rotation of line attributes and resets the \htmlref{{\tt{TLINE}}}{COM:TLINE}  index
to 1 (continuous lines). (See \htmlref{{\tt{TROT}}}{COM:TROT}  and \htmlref{{\tt{TLINE}}}{COM:TLINE}). 

\dipcom{NROT}{(no parameters)}{Stops rotating everything}
Stops rotating everything, resetting the colour, \htmlref{{\tt{MARK}}}{COM:MARK}  and \htmlref{{\tt{TLINE}}}{COM:TLINE} 
indexes to 1 (i.e. has the same effect as NCROT,NMROT,NTROT).

\dipcom{NX}{(no parameters)}{Returns to auto-scaling of the X axis}
Return to autoscaling of the X axis (after using \htmlref{{\tt{XR}}}{COM:XR},  \htmlref{{\tt{XMIN}}}{COM:XMIN},  \htmlref{{\tt{XMAX}}}{COM:XMAX},  \htmlref{{\tt{CXR}}}{COM:CXR} 
and/or \htmlref{{\tt{CXYR}}}{COM:CXYR}). 

\dipcom{NXY}{(no parameters)}{Returns to auto-scaling of the X and Y axis}
Return to autoscaling of the X \& Y axes (after using \htmlref{{\tt{XR}}}{COM:XR},  \htmlref{{\tt{YR}}}{COM:YR},  \htmlref{{\tt{XMIN}}}{COM:XMIN}, 
\htmlref{{\tt{XMAX}}}{COM:XMAX},  \htmlref{{\tt{YMIN}}}{COM:YMIN},  \htmlref{{\tt{YMAX}}}{COM:YMAX},  \htmlref{{\tt{CXR}}}{COM:CXR},  \htmlref{{\tt{CYR}}}{COM:CYR},  and/or \htmlref{{\tt{CXYR}}}{COM:CXYR}). 

\dipcom{NY}{(no parameters)}{Returns to auto-scaling of the Y axis}
Return to autoscaling of the Y axis (after using \htmlref{{\tt{YR}}}{COM:YR},  \htmlref{{\tt{CYR}}}{COM:CYR}  and/or \htmlref{{\tt{CXYR}}}{COM:CXYR}). 

\dipcom{PAUSE}{(no parameters)}{Pauses until the <RETURN> key is hit}
Gives a \htmlref{{\tt{BEEP}}}{COM:BEEP}  (unless you've specified \htmlref{{\tt{NOBEEP}}}{COM:NOBEEP}),  and causes nothing to
happen until you hit the return key. (You might find a use for this
command in command files.)

\dipcom{PF}{n}{Fits a polynomial through the data in the windows defined by {\tt{CREGS}}} 
`Poly fit': fits a polynomial of degree `n' through data points in the
HIGHEST ({\em i.e.} numerically largest) STACK entry, using continuum
windows defined using \htmlref{{\tt{CREGS}}}{COM:CREGS}.  \htmlref{{\tt{PF}}}{COM:PF}  expects these data to be in order of
increasing X value. Y values, obtained from the parameters of the
`best fit', are calculated at the grid of X points of the data in the
highest STACK entry, between the first and last `continuum window' X
values. The resulting curve is left in the `current' arrays
(overwriting anything there previously).

Because the `continuum window' data are stored internally, it is
possible to carry out a sequence of trial fits in an attempt to obtain
a satisfactory polynomial representation of the continuum. Thus a
typical sequence of commands to rectify data might be as follows:

\begin{verbatim}
    POP n, PUSH       ! Put the  data  of  interest  into  the
                      ! 'current'  arrays and onto the top  of
                      ! the stack.
    XR x1 x2,PM       ! Set the X range and plot the data.
    CREGS,CREGD       ! Select & display  continuum windows.
    PF 1              ! Fit  a  straight line to the  'window'
                      ! data.
    NB,PM             ! Plot the fit through the data.
    PF 3              ! PF 1 unsatisfactory; try again.
    PM                ! Plot the fit; it looks O.K.
    ADIV m            ! Divide data in the top of the STACK by
                      ! by the polynomial approximation to the
                      ! continuum.
    TITLE new title   ! Change the title appropriately.
    BOX,PM            ! Plot the rectified data.
    PUSH              ! Save the rectified data for later use.
    POP j,PUSH        ! Prepare the next data set of interest.
    PM,NB             ! Plot the data.
    PF 3,PM           ! Fit & plot a polynomial; there is  no
                      ! need  to  redefine  continuum  windows
                      ! unless you want to change them.
\end{verbatim}

(Then \htmlref{{\tt{ADIV}}}{COM:ADIV},  \htmlref{{\tt{TITLE}}}{COM:TITLE},  and so on).

Note that a slightly different approach to polynomial fitting is
possible using \htmlref{{\tt{ELFINP}}}{COM:ELFINP}  and \htmlref{{\tt{ELFOPT}}}{COM:ELFOPT}  (which can give you the coefficients
of a least-squares fit).

\dipcom{PDGPEAK}{(no parameters)}{Locates the peak of a data set}
Locates the peak of the data set in the current arrays (notionally, but
not necessarily, produced with \htmlref{{\tt{PDGRAM}}}{COM:PDGRAM}).  \htmlref{{\tt{PDGPEAK}}}{COM:PDGPEAK}  does this by locating
the largest Y values, then fitting a parabola to the point and the
adjacent ones on either side. The peak and central values listed are
those calculated from this parabola.

\dipcom{PDGRAM}{[fl fh df p]}{Takes the "unevenly spaced data periodogram" of the current arrays}
Replaces the data in the current arrays (which should have
monotonically increasing X values) with their ``unevenly spaced data
periodogram'', as defined by Scargle (ApJ 263, 835, 1982).

The parameter fl is the lowest frequency at which the periodogram is
to be evaluated, and defaults to zero; fh, the highest frequency,
defaults to 0.5*(n-1)/(x[n]-x[1]), where there are ``n'' points in the
current arrays; df, the frequency interval, defaults to
1.0/(x[n]-x[1]); and p, the proportion of the data set endmasked at
each end (using a cosine bell), defaults to 0.05.

N.B.\ For large data sets the default frequency parameters may lead to
very large numbers of points in the periodogram, which will
accordingly take a substantial time to evaluate.

The periodogram has X units which are the inverse of the original X
units (ie will normally be in units of spatial or temporal frequency).

\dipcom{PDGWINDOW}{[fl fh df]}{Takes the "window function" of the current arrays}
Replaces the data in the current arrays (which should have
monotonically increasing X values) with their window function, as
defined by Scargle (ApJ 263, 835, 1982). The parameters fl, fh, and df
have the same meanings and default values as for the \htmlref{{\tt{PDGRAM}}}{COM:PDGRAM}  function,
and the caution given for that function regarding lengthy evaluation
times applies also to \htmlref{{\tt{PDGWINDOW}}}{COM:PDGWINDOW}. 

One would normally calculate a data set's window function in addition
to its periodogram (PDGRAM) to ensure that any peaks in the latter are
not an artefact of the sampling frequency.

\dipcom{PLOTINV}{(no parameters)}{Causes the Y axis of subsequent plots to be inverted}
Invert the Y axis on plotting ({\em i.e.} the `bottom' Y value becomes
the `top' value, and vice versa).

\dipcom{PLOTREV}{(no parameters)}{Causes the X axis of subsequent plots to be reversed}
Reverse the X axis (ie the X value at the left-hand edge of the plot
becomes that at the right-hand edge, and vice versa).

\dipcom{PM}{[n1 n2 n3 ..... n48 n49 n50]}{Plots a data array}
Plot data (`PM' comes about `for historical reasons'). If you haven't
used the \htmlref{{\tt{PPROMPT}}}{COM:PPROMPT}  command, then \htmlref{{\tt{PM}}}{COM:PM}  without any arguments will plot the
data stored in the `current' arrays; otherwise up to 50 STACK entries
can be plotted. If you have set \htmlref{{\tt{PPROMPT}}}{COM:PPROMPT}  to TRUE, then if you didn't
provide any arguments for \htmlref{{\tt{PM}}}{COM:PM}  on the command line DIPSO will prompt you
for some when it comes to do the plot. In order to be able to plot
both `current' and `STACK' data, the `current' arrays are awarded the
`honorary' STACK entry number 0 for this command {\em only}. Ranges of
stack entries may be specified using the ``-'' operator; {\em e.g} \htmlref{{\tt{PM}}}{COM:PM}  1 3-5
will result in entries 1, 3, 4 and 5 being plotted.

Even if the \htmlref{{\tt{BOX}}}{COM:BOX}  is switched `on', all the data associated with a single \htmlref{{\tt{PM}}}{COM:PM} 
command are plotted in a single frame; {\em i.e.} a command sequence like:

\begin{verbatim}
   BOX,PM 1 2 0 4
\end{verbatim}

will have a different result from:

\begin{verbatim}
   BOX,PM 1,PM 2,PM 0,PM 4
\end{verbatim}

WARNINGS:

\begin{itemize}

\item If you try to plot more than 50 spectra, numbers 51 et seq will
{\em not} be plotted; moreover {\em no} error message will be given.
You will have to attempt to plot the `current' arrays and the entire
contents of a completely full STACK (or multiply plot the same
data set) for this problem to arise.

\item If autoscaling is in effect, the plotting frame is autoscaled to
the X,Y data in the {\em first} stack entry specified. Thus a command
like:

\begin{verbatim}
   PM 1 2 3 4 5
\end{verbatim}

may produce a different plot to:

\begin{verbatim}
   PM 5 4 3 2 1
\end{verbatim}

\end{itemize}

\dipcom{POLY}{(no parameters)}{Causes plots to be done 'join-the-dots' style}
Plotting to be done `join-the-dots' style (as opposed to \htmlref{{\tt{MARK}}}{COM:MARK}  or \htmlref{{\tt{HIST}}}{COM:HIST}). 

\dipcom{POP}{n}{Pops a stack entry into the current arrays}
Pop STACK entry `n' into the `current' arrays.

\dipcom{PPROMPT}{switch}{Controls which prompts are issued by {\tt{PM}}} 
\htmlref{{\tt{PPROMPT}}}{COM:PPROMPT}  governs the default action of \htmlref{{\tt{PM}}}{COM:PM}.  The value of the `switch'
argument is T(rue) or F(alse), and on startup is set to F. If you set
it to T, then the \htmlref{{\tt{PM}}}{COM:PM}  command will prompt for an argument list if none
is provided on the command line. This could be useful if, for example,
you want to include \htmlref{{\tt{PM}}}{COM:PM}  in command files where you may not know in
advance which stack entries you want to plot. If the switch is set to
F, then typing \htmlref{{\tt{PM}}}{COM:PM}  without any argument results in the data in the
current arrays being plotted.

\dipcom{PS}{X1 X2 [DX] [n1 n2 n3...n50]}{Plots long stretches from a single spectrum}
`Plot Spectrum': this command, which acts on the contents of the
`current' arrays, is intended specifically for plotting long stretches
of a single spectrum ({\em e.g.} high resolution IUE data) in
sequential frames. The parameters are:

\begin{quote}
\begin{dipdesc}
\dipitem {X1:}{ the X value at the left-hand edge of the first frame;}
\dipitem {X2:}{ the X value at the right-hand edge of the first frame;}
\dipitem {DX:}{ the amount by which X1 and X2 are incremented in
successive frames. DX defaults to (X2-X1).}
\end{dipdesc}
\end{quote}

Plots are alternated in zones 5 and 6. After each frame is plotted,
you are prompted as to whether or not you want to continue plotting.

WARNING: On completion of this command the X range and plotting zone
in force will be those used for the last frame; you will probably want
to change them.

\dipcom{PUSH}{(no parameters)}{Pushes the current arrays onto the stack}
`Pushes' the contents of the `current' arrays onto the top of the STACK.

\dipcom{PWRITE}{x y i string}{Writes a character string at given co-ordinates}
Writes a character string `string' at co-ordinates x,y. If \htmlref{{\tt{FONT}}}{COM:FONT}  2 is
active, the ``locator index'', `i', determines the location of the
string with respect to the x,y coords: if i=1, the string is written
to the lower left ({\em i.e.} the top right corner of an imaginary
box containing the string ends at x,y); if i=2 the 'box' is
horizontally centred on, and vertically below, x,y; i=3 has the top
left corner of the 'box' at x,y; i=4,5,6 correspond to the `box'
vertically centred and to the left, centre, right of x,y,
respectively; i=7,8,9 are similar, but above x,y. These locator
indexes are chosen so that the numeric keypad available on most
keyboards acts as a mnemonic.

If the string contains commas, it {\em must} be enclosed in double
quotes ({\tt{"}}).  It is recommended that you develop the habit of
using such quotes in any case.

The \htmlref{{\tt{ANGLE}}}{COM:ANGLE},  \htmlref{{\tt{EXPAND}}}{COM:EXPAND},  and \htmlref{{\tt{FONT}}}{COM:FONT}  commands can be used to modify the
appearance of the string on output.

\dipcom{QAREA}{Zn}{Gives information about a plotting zone}
Reports the (relative) sizes of the grid and graph windows for zone
``Zn'', together with the absolute area of the plotting surface (if
available). See the \htmlref{{\tt{TPORT}}}{COM:TPORT}  command for details.

\dipcom{QSM}{Sigma}{Applies a 'quick' Gaussian smoothing to an array}
Applies a `quick' Gaussian smoothing (FWHM of filter = 2.354*Sigma) to
Y values in the `current' arrays. (Defines gaussian weights at a grid
of delta(x) values, then linearly interpolates when smoothing). Much
faster, and scarcely less accurate, than \htmlref{{\tt{SM}}}{COM:SM}. 

\dipcom{QUIT}{(no parameters)}{Quit}
Quit DIPSO, without saving the stack. Can be abbreviated to ``Q''.

\dipcom{RDCAT}{file xcol ycol}{Reads X and Y values from a catalogue into 
the current arrays}
Reads the values from two specified catalogue columns in the the current
arrays. The file containing the catalogue is given by ``file''. This must
be a full path name (no shell meta-characters such as \verb+~+, \verb+$+,
etc can be used here). The file can be a FITS table (binary or formatted), 
or an STL table (see \xref{SUN/190}{sun190}{} ). The names of the columns 
holding the X and Y values are given by ``xcol'' and ``ycol''. A full list 
of available column names is displayed if an unknown column name is supplied.

\dipcom{READ}{file [sys] [unit]}{Reads data from a disk file to the current arrays (see {\tt{WRITE}})} 
Reads a disk file which has been written with 
\htmlref{{\tt{WRITE}}}{COM:WRITE}.  This and \htmlref{{\tt{RESTORE}}}{COM:RESTORE}  
are the
only input modes that preserve ALL the information associated with a DIPSO
data set, and are the recommended options. The disk data can be in either NDF or
``native DIPSO'' format, depending on the flag established by the \htmlref{{\tt{USENDF}}}{COM:USENDF} 
command. When specifying an NDF, do not include any file type.

If the specified NDF was not created by DIPSO, then an attempt will be
made to create appropriate X values based on the WCS component in the
NDF (or the FITS header cards in the FITS extension if there is no WCS
component). The second and third command parameters, ``sys'' and ``unit'',
may be used to indicate the spectral system and units in use within the NDF
(these are only used if the system and units are not clearly specified in
the NDF). The system value supplied must be a standard FITS-WCS value such as 
``FREQ'', ``WAVE'', ``VRAD'', ``VOPT'', etc. Likewise, the unit should be
a standard FITS-WCS such as ``m'', ``nm'', ``Angstrom'', ``Hz'', ``GHz'',
``m/s'', ``km/s'', etc. The spectral axis values within the NDF will be
converted automatically from their original system to either wavelength in 
Angstroms or optical velocity in km/s for storage in the DIPSO X array. The 
original system and unit adopted for the NDF during this conversion are 
reported. If the user does not specify a system and/or unit, then a
default of ``wavelength in metres'' will be adopted unless the contents
of the NDF suggest some other default.

\dipcom{RECA}{[/all] [string or integer]}{Recalls a previous command or prompt response}
Recalls a previous command line or prompt response, putting it into the
current keyboard input buffer for acceptance (by pressing return) or
further editing. The \htmlref{{\tt{RECA}}}{COM:RECA}  statement can be issued {\em when-ever} dipso
prompts the user for input. If no parameters or qualifiers are given, the
string which the user last typed in is recalled. If a non-numeric
parameter value is given, the most recent string to be given by the user
which starts with the supplied string parameter is recalled. If a numeric
parameter is given, the corresponding string is recalled where ``1'' is
the last (i.e. previous) string, ``2'' is the last-but-one, etc. If the
{\tt{/all}}  qualifier is given (i.e. ``{\tt{RECA/ALL}}''  ), a list of the
last 20 strings given by the user is displayed, and the user is then
re-prompted.

\dipcom{RECALL}{(no parameters)}{Displays a previously RECORDed string (does not execute it)}
Lists a \htmlref{{\tt{RECORD}}}{COM:RECORD}ed string at the terminal (as a 
reminder!)

\dipcom{RECORD}{``string''}{Records a string of commands for subsequent {\tt{REPLAY}}} 
Records a string of commands for subsequent \htmlref{{\tt{REPLAY}}}{COM:REPLAY}.  The string must be
enclose in double quotes; it can itself contain double quotes, but
must then be delimited by pairs of double quotes. The string may not
incorporate \htmlref{{\tt{RECORD}}}{COM:RECORD}  or \htmlref{{\tt{REPLAY}}}{COM:REPLAY}  commands, but the syntax, and general
validity, of the string is otherwise not checked until it is REPLAYed.

To cancel a current RECORDing, use a null string.

\dipcom{REPLAY}{(no parameters)}{Substitutes a previously RECORDed string into the command buffer }
If \htmlref{{\tt{REPLAY}}}{COM:REPLAY}  is included in a command line, then it is replaced in that
command line by the contents of any RECORDed command string.

\dipcom{REPORTING}{level}{Controls the number of error messages displayed }
This command can be used to reduce the number of error messages displayed if one
of the NDF accessing commands fail. The command takes a single integer
parameter. Level = 2 is the initial value and results in all error messages
being delivered to the screen. Level = 1 replaces all errors deriving from a
single failed command with a single (less detailed) error message. Level = 0
suppresses all error reports.

\dipcom{RESTORE}{[file name or NDF name]}{Restores stack data previously saved to disk using {\tt{SAVE}}} 
Restores STACK data previously dumped using \htmlref{{\tt{SAVE}}}{COM:SAVE}.  \htmlref{{\tt{RESTORE}}}{COM:RESTORE}  does not
require the current stack to be empty; however, if it is not then
retrieval may not be complete, depending on the available stack space.
(Notification is given of incomplete retrieval, of course). The data can
be in either NDF or ``native DIPSO'' format, depending on the flag
established by the \htmlref{{\tt{USENDF}}}{COM:USENDF}  command. When specifying an NDF, do not
include any file type. The default NDF name is `SAVE\_STK'. If not using
NDFs, the default file name is `SAVE.STK'.


\dipcom{RETITLE}{n string}{Changes the title of a stack entry}
Changes the title of stack entry ``n''. The string (which may be null)
must be enclosed in double quotes if it contains commas. (The use of
double quotes is in any case recommended.) Special characters may be
incorporated (see the \htmlref{{\tt{FONT}}}{COM:FONT}  command).

\dipcom{RXR}{X1 X2}{Throws away data outside a given X range }
Restrict the X range of data in the `current' arrays ({\em i.e.} throw
away data outside the range X1 to X2). Only works properly on data sets
with monotonically increasing or decreasing X values.

\dipcom{RYR}{Y1 Y2 [X1 X2]}{Throws away data outside a given Y range }
`Restrict Y range'; throws away datum points in the current arrays
which have Y values outside the limits Y1 and Y2. These limits apply
to the entire arrays unless X1 and X2 are supplied, in which case the
operation is carried out only within that X range.

\dipcom{SAVE}{[file name or NDF name [n1-n2]]}{Saves stack data in a binary disk file}
Dumps the contents of the STACK in a form suitable for subsequent
reacquisition using \htmlref{{\tt{RESTORE}}}{COM:RESTORE}.  The data can be in either NDF or ``native
DIPSO'' format, depending on the flag established by the \htmlref{{\tt{USENDF}}}{COM:USENDF} 
command. When specifying an NDF, do not include any file type. The
default NDF name is `SAVE\_STK'. If not using NDFs, the default file name
is `SAVE.STK'.

It is possible to specify subsets of the stack for saving, but then a file or
NDF name is mandatory. The entire contents of the stack are saved by default.

\dipcom{SCREENRD}{[brkval]}{Reads data typed at the terminal into the current arrays}
Input data from the terminal to the current arrays; terminate with
``\verb+\+''. Each input line should contain a pair of X and Y
values but nothing else. If a value of "brkval" is specified on the
command line, then Y values matching it are assumed to flag gaps in
the data.

\dipcom{SCROLLVT}{[n1 n2]}{Causes text to be scrolled between specified lines}
When a VT emulant is being used, this command result in text being
scrolled between lines n1 and n2 only. Thus you can scroll text to,
say, the bottom half of the screen
(\htmlref{{\tt{SCROLLVT}}}{COM:SCROLLVT} {\tt 14 25}) while plotting on
the top half (\htmlref{{\tt{TZONE}}}{COM:TZONE} {\tt 5}). If specified, both n1 and n2 must be in the
range 1 to 25, an at least two lines must be allowed for. Full screen
scrolling results if no line range is given.

\dipcom{SHELL}{command}{Executes a shell command}
The supplied command is executed within a Bourne shell running in a child
process which terminates when the command is completed. Note, the command
is assumed to have failed if the status value returned by the command is
non-zero. An alternative to using the {\tt SHELL} command is to suspend
the process running DIPSO by pressing control-Z, issue the required
command, and then re-awaken the dipso process by typing ``{\tt fg}''. The
{\tt SHELL} command is provided for use in DIPSO command files.

\dipcom{SL}{[N1 [N2]]}{Lists the contents of the stack and current arrays}
Stack list: gives the entry number, number of points, first and last X
values, and the first characters of the title associated with each
STACK entry. By default, the entire stack is listed (N1 = 1, N2 = no
of stack entries), otherwise the first (N1) and last (N2) entries to
be displayed can be specified. If N1=0, the contents of the `current'
arrays are also summarised. The ``-" operator can be used to separate
N1 and N2.

\dipcom{SLWR}{[N1 [N2]]}{As {\tt{SL}},  but writes the information to a text file}
Stack List WRite. Operates identically to \htmlref{{\tt{SL}}}{COM:SL},  but outputs results to a
file STACK.LIS in the current directory, instead of to the terminal.

\dipcom{SM}{Sigma}{Applies Gaussian smoothing to an array}
Smooths Y data in the `current' arrays with a Gaussian filter
(FWHM=2.354*Sigma). \htmlref{{\tt{QSM}}}{COM:QSM}  gives a much faster, and scarcely less
accurate, result.

\dipcom{SNIP}{[X1 X2 X3 X4 X5 X6 ... X49 X50]}{Cuts out data from the current arrays}
Cuts out data from the `current' arrays. Pairs of X values can be
provided as optional parameters; if no parameters are given, the
cursor is used to define regions to be SNIPped. Each pair of X values,
whether input at the terminal or defined by cursor, must be in order
of increasing X (though this constraint does not apply to successive
pairs of values). To get out of \htmlref{{\tt{SNIP}}}{COM:SNIP}  mode when using the cursor,
define X2$<$X1.

WARNING: \htmlref{{\tt{SNIP}}}{COM:SNIP}  {\em only} works successfully on data sets that have
monotonically increasing X values. Remember, `snipping' is done on
data held in the `current' arrays. Be careful not to plot STACK data
and mistakenly think that it is the plotted data that are being
edited. Furthermore, it is not possible to recall snipped-out data
points, so it is wise to maintain an untouched data set on the STACK.

\dipcom{SP0RD}{file name or NDF name [logical]}{Reads an IUE SPECTRUM format 0 file into the current arrays}
Reads a SPECTRUM format 0 file into the `current' arrays. This is the
format that data output from IUEDR are normally in (see
\xref{SUN/37}{sun37}{} for
details); for other purposes, \htmlref{{\tt{READ}}}{COM:READ}  (and \htmlref{{\tt{WRITE}}}{COM:WRITE})  are recommended.
Maximum number of points allowed is 20,000. WORV is assumed to be 1.0.

The data can be in either NDF or ``native DIPSO'' format, depending
on the flag established by the \htmlref{{\tt{USENDF}}}{COM:USENDF}  command. When specifying an NDF, do
not include any file type.

If YES (the default) is supplied for the second (optional) logical
parameter any zeros in the data file are treated as gaps in the data. If
NO is supplied, any zeros are treated as good data values. This replaces
the undocumented PANICRD command.


\dipcom{SP0WR}{file name or NDF name}{Writes the current arrays to an IUE SPECTRUM format 0 file}
Outputs a SPECTRUM format 0 file (see the IUEDR documentation
\xref{SUN/37}{sun37}{}
for a description of the SPECTRUM formats). But you should be using
\htmlref{{\tt{WRITE}}}{COM:WRITE}!  The data output are those stored in the `current' arrays.

The data can be in either NDF or ``native DIPSO'' format, depending on the flag
established by the \htmlref{{\tt{USENDF}}}{COM:USENDF}  command. When specifying an NDF, do not include
any file type.

\dipcom{SP1RD}{filename[.typ]}{Reads an IUE SPECTRUM format 1 file into the current arrays}
Reads a SPECTRUM format 1 file (see \xref{SUN/37}{sun37}{} for details) into the
`current' arrays. Maximum number of points allowed is 20,000. WORV is
assumed to be 1.0. The default file type ([.typ]) is `.DAT'. This command
cannot read NDFs.

\dipcom{SP2RD}{filename[.typ]}{Reads an IUE SPECTRUM format 2 file into the current arrays}
Reads a SPECTRUM format 2 file (see \xref{SUN/37}{sun37}{} for details) into the
`current' arrays. Maximum number of points allowed is 20,000. WORV is
assumed to be 1.0. The default file type ([.typ]) is `.DAT'. This command
cannot read NDFs.

\dipcom{SQRTX}{(no parameters)}{Takes the square root of the X values in the current arrays}
Replaces the X values in the current arrays by their square root
(throwing away any negative values).

\dipcom{SQRTY}{(no parameters)}{Takes the square root of the Y values in the current arrays}
Replaces the Y values in the current arrays by their square root
(throwing away any negative values).

\dipcom{STATUS}{(no parameters)}{Displays the current device, X and Y ranges, etc}
Returns information on the current status (device number, X and Y
ranges, {\em etc.}).

\dipcom{TADD}{[string]}{Adds a string to the end of the current title}
Adds a string to the end of the current title.

\dipcom{TENX}{(no parameters)}{Replaces the current array's X values by 10**X}
Replaces the X values in the current arrays by 10.0**X.

\dipcom{TENY}{(no parameters)}{Replaces the current array's Y values by 10**Y}
Replaces Y values in the current arrays by 10**Y

\dipcom{TICKS}{[dx [dy [nx [ny]]]]}{Controls the appearance of tick marks}
Controls the appearance of tickmarks on plots. The arguments dx and dy
control the spacing between major (labelled) tickmarks on the x and y
axes, respectively; nx and ny control the number of spaces between
major ticks ({\em i.e.} no. of minor ticks = nx/y minus 1).

A zero for any argument results in the relevant aspect of the plot
design returning to automatic control; just typing \htmlref{{\tt{TICKS}}}{COM:TICKS}  (no
arguments) returns to fully automatic tickmark design. Negative
arguments result in no tickmarking.

If you supply grossly inappropriate arguments ({\em e.g.} trying to
squeeze too many numbered ticks onto an axis) then you will certainly
find the resulting plot not to your satisfaction. Beware, in
particular, of forgetting to scale by appropriate powers of ten (which
may be subsumed into the axis label). Unless you are very familiar
with the range of the data being plotted, it is wise to do initial
plots with autodesign in force.

\dipcom{TITLE}{[string]}{Changes the title stored with the current arrays}
Change the title associated with data in the `current' arrays. Null
strings are accepted as such unless \htmlref{{\tt{TPROMPT}}}{COM:TPROMPT}  has previously been set
TRUE (q.v.\ \htmlref{{\tt{TPROMPT}}}{COM:TPROMPT}).  The string must NOT contain any control
characters (which would probably cause a crash in the graphics library
routines). Since a comma is normally interpreted as the end of a
string, it is necessary to enclose strings containing commas in double
quotes; {\em e.g.}

\begin{verbatim}
   >SP0RD TEST,TITLE "Commas can, I think, be useful",PUSH
   >SP0RD TEST,TITLE I have no commas, PUSH
\end{verbatim}

gives:

\begin{verbatim}
   Commas can, I think, be useful
\end{verbatim}

and

\begin{verbatim}
   I have no commas
\end{verbatim}

respectively. It's a good idea to develop the habit of using double
quotes regularly, even if you don't use commas often in strings.

The price paid for being able to include commas in strings is that
other usage of quotes in \htmlref{{\tt{TITLE}}}{COM:TITLE}  is forbidden. (Actually, it's not, but
the rules are so complex as to effectively forbid use of quotes.)

\dipcom{TPROMPT}{logical}{Determines if the {\tt{TITLE}} command should prompt for a string}
If ``logical'' is T or Y, then the command \htmlref{{\tt{TITLE}}}{COM:TITLE}  will prompt for an
input string if none is provided; if F or N (the default), it won't.

\dipcom{TLINE}{line}{Changes the style of all plotted lines}
Change line attributes (continuous, dot-dash {\em etc}). The index `line'
must be in the range 1-5 (1 = continuous lines). (See also \htmlref{{\tt{TROT}}}{COM:TROT}). 

\dipcom{TOFLAMBDA}{(no parameters)}{Converts data to Angstroms and erg/cm2/s/A}
If the X and Y data in the current arrays are in units of Hz and
\newline erg/cm2/s/Hz, \htmlref{{\tt{TOFLAMBDA}}}{COM:TOFLAMBDA}  will convert to Angstroms and
erg/cm2/s/A.

\dipcom{TOFNU}{(no parameters)}{Converts data to Hz and erg/cm2/s/Hz}
If the X and Y data in the current arrays are in units of Angstroms
and erg/cm2/s/A, \htmlref{{\tt{TOFNU}}}{COM:TOFNU}  will convert to Hz and erg/cm2/s/Hz.

\dipcom{TOV}{wav0}{Converts X values from wavelength to velocity}
Convert X values from wavelength to velocity (the inverse of \htmlref{{\tt{TOW}}}{COM:TOW}), 
where Wav0 is the appropriate rest wavelength in Angstroms.

{\em No} changes are made to the Y array. This would normally mean
that subsequent measurements made with \htmlref{{\tt{EW}}}{COM:EW}  or \htmlref{{\tt{FLUX}}}{COM:FLUX}  would be in rather
strange units ({\em e.g.} km/s, or erg/cm2/s/[km/s]). To avoid this
anomaly, an internal variable WORV (for Wavelength OR Velocity) is
associated with each data set. This has the value 1.0 by default, but
is reset to Wav0/c (where c is the speed of light in km/s) on using
\htmlref{{\tt{TOV}}}{COM:TOV}. 

(If you are reading in data sets with velocity as the X co-ordinate it
is usually safest to convert to wavelength [TOW] then back to velocity
[TOV] in order to obtain an appropriate WORV value. This is
unnecessary if you use READ/WRITE or SAVE/RESTORE to move data in and
out of DIPSO.)

WARNING: If you have set the X range, you'll probably need to change
it to get anything on the plotting surface after using \htmlref{{\tt{TOV}}}{COM:TOV}  or \htmlref{{\tt{TOW}}}{COM:TOW}! 

\dipcom{TOW}{wav0}{Converts X values from velocity to wavelength }
Convert X values from velocity to wavelength (see \htmlref{{\tt{TOV}}}{COM:TOV}),  and resets
WORV to 1.0.

\dipcom{TROT}{(no parameters)}{Switches on automatic rotation of line style (i.e. attributes)}
Implements automatic rotation of line attributes. Each plot begins
with the line style defined by the last use of \htmlref{{\tt{TLINE}}}{COM:TLINE}  (or style 1, if
\htmlref{{\tt{TLINE}}}{COM:TLINE}  hasn't been called).

Switched off with \htmlref{{\tt{NTROT}}}{COM:NTROT}. 

\dipcom{TPORT}{Zn Xmin Xmax Ymin Ymax [WXmin WXmax WYmin WYmax]}{Defines a plotting sub-zone}
Defines a plotting subzone (no.\ ``Zn'', where 100$>$Zn$>$8). The
dimensions of the subzone (Xmin...Ymax) are normalised such that the
total dimensions available (regardless of device) are 0 to 1 in both
axes. (The corresponding physical dimensions can be discovered using
\htmlref{{\tt{QAREA}}}{COM:QAREA}.) 

In general, a plot consists of axes, axis labels, and a header, as
well as the data. Technically, the plotting subzone is a ``graph
window''; within this graph window a ``grid window" is present. The
grid window is exactly filled by the axes, and so will normally be
smaller than the graph window (to leave room for the labelling). DIPSO
will normally work out default grid window dimensions, but you can
define your own (WXmin...WYmax). This might be useful if you want
contiguous axes in different plots, for example.

Just as the graph window is defined in terms of fractions of the
available plotting area, so the grid window is defined in terms of
fractions of the available graph window. Thus parameters WXmin {\em
etc.} must also be in the range 0 to 1 (regardless of the graph window
dimensions).

To access a given subzone, use \htmlref{{\tt{TZONE}}}{COM:TZONE}. 

\dipcom{TSTRIP}{(no parameters)}{Removes leading blanks from the current array's title}
Removes leading blanks from the title associated with the `current' arrays.

\dipcom{TSWAP}{n}{Copies the title from a stack entry to the current arrays}
Copies the title from STACK entry `n' to the `current' arrays.

\dipcom{TWEIGHT}{weight}{Sets the weight (heaviness) of the data curve only}
Alters the weight ({\em i.e.} `heaviness') of data curves plotted with
\htmlref{{\tt{PM}}}{COM:PM},  on devices which support this feature. No other lines (axes, {\em
etc}) are effected. The weight must be 1-5 (initial setting is 1). See
also \htmlref{{\tt{LWEIGHT}}}{COM:LWEIGHT}. 

\dipcom{TZONE}{zn}{Selects a particular zone for plotting}
Selects zone ``zn'' for plotting. The zone numbers are:

\begin{quote}
\begin{dipdesc}
\dipitem {0:}{ entire surface}
\dipitem {1:}{ top left quarter}
\dipitem {2:}{ top right quarter}
\dipitem {3:}{ bottom left quarter}
\dipitem {4:}{ bottom right quarter}
\dipitem {5:}{ top half}
\dipitem {6:}{ bottom half}
\dipitem {7:}{ left half}
\dipitem {8:}{ right half}
\end{dipdesc}
\end{quote}

Additional zones can be user-defined if required (see \htmlref{{\tt{TPORT}}}{COM:TPORT}).  When
using such additional zones special care is needed to avoid
overplotting previous zones (use \htmlref{{\tt{ERASE}}}{COM:ERASE}  to get a ``page throw'' on hard
copy devices); this is taken care of automatically with zones 0-8.

\dipcom{UBVRD}{u b v [dx]}{Converts {\tt{UBVRD}} magnitudes to fluxes}
Converts UBV magnitudes to fluxes, and stores the results in the
`current' arrays. If the U, B and/or V magnitude is unknown, 0 should
be entered. (Should you want to actually input a value of 0, I'm
afraid that you'll have to use something like 0.0001, or put in a
value of ({\em e.g.}) 5 and then carry out some arithmetic using \htmlref{{\tt{TENY}}}{COM:TENY}, 
\htmlref{{\tt{YMULT}}}{COM:YMULT}  and \htmlref{{\tt{LOGY}}}{COM:LOGY}). 

The data are plotted at assumed wavelengths of 3600, 4400 and 5500
Angstroms using lines 2*dx wide (default value of dx: 50 Angstroms).
Conversion from magnitudes to fluxes is carried out using

\begin{verbatim}
   Mag = -C - 2.5*Log10(Flux)
\end{verbatim}

where C is 20.94, 20.51 and 21.12 for U, B and V respectively. (These
values are from the absolute flux measurements of Vega made by Tug et
al, Oke \& Schild, and Hayes \& Latham, for V; and a normalised Kurucz
Vega model for U and B).

\dipcom{USEHTX}{[logical] [showme]}{Selects hypertext or plain text format for help information}
If `logical' is Y, YES, T or TRUE, the \htmlref{{\tt{HELP}}}{COM:HELP} 
command will display help information in hypertext format, using a
World-Wide-Web browser. 
If N, NO, F or FALSE is supplied for `logical', help information will be
displayed in plain text format in the DIPSO command window. If no value 
is supplied for `logical', YES is assumed.

If supplied, `showme' should contain a string giving the unix command
used to run the Starlink {\tt showme} utility. The supplied command
string will be used to initiate hypertext help sessions until a new value
is supplied. The command used initially is ``\verb+<dir>/showme+'', where 
\verb+<dir>+ is the path to the directory containing Starlink executable
files which was used when DIPSO was installed (typically {\tt /star/bin}).

\dipcom{USENDF}{[logical]}{Selects NDF or 'native DIPSO' binary file format}
If `logical' is Y, YES, T or TRUE, data accessing commands such as 
\htmlref{{\tt{READ}}}{COM:READ}, \htmlref{{\tt{WRITE}}}{COM:WRITE},  
\htmlref{{\tt{SAVE}}}{COM:SAVE},  \htmlref{{\tt{RESTORE}}}{COM:RESTORE},  etc,
will use NDF structures (see \xref{SUN/33}{sun33}{}) to store data in. If
N, NO, F or FALSE is supplied, they will use ``DIPSO native'' files as
used by DIPSO prior to version 3.0 (but note, that these files will NOT
be portable from one operating system to another). If no value is
supplied for `logical', YES is assumed.

\dipcom{USSPRD}{filename[.typ] [epsmin]}{Reads an IUE ULDA spectrum as produced by USSP }
Reads data from the IUE `Uniform' Low Dispersion Atlas, ULDA, which have been
output using the USSP (see \xref{SUN/20}{sun20}{}). The filename extension defaults to `.ULD'.
This command cannot read NDFs. Each point in the USSP spectrum has an associated
error index, called epsilon, with the following meanings:

\begin{dipdesc}
\dipitem{  100 : } {No special conditions}
\dipitem{ -200 : } {Extrapolated at upper end of ITF}
\dipitem{ -220 : } {Microphonic noise}
\dipitem{ -250 : } {Filtered bright spot}
\dipitem{ -300 : } {Unfiltered bright spot}
\dipitem{ -800 : } {Reseau in extracted spectral region}
\dipitem{-1600 : } {Saturated}
\dipitem{-3200 : } {Not photometrically corrected}
\end{dipdesc}

The epsilons are not necessarily reliable indicators of data quality.
DIPSO rejects points on input if they are flagged with an epsilon less
than or equal to `epsmin' (which defaults to -251), leaving the
spectrum in the current arrays. It is forbidden to set epsmin less
than or equal -1600, since this would result in totally unflagged,
certainly bad, data being acquired.

If epsmin is given a value greater than zero, then {\em all}
datapoints are read into the next available stack entry, and the
epsilon array into the subsequent stack entry. More subtle doctoring
of the data is then possible, using \htmlref{{\tt{USSPCLIP}}}{COM:USSPCLIP}  (q.v.). However, it is
recommended that the default epsmin be accepted unless you really know
what you're doing, and have good reasons to choose a different value.

\dipcom{USSPCLIP}{epsmin n1 [n2 w1 w2]}{Removes points from an IUE USSP spectrum based on 'epsilon' values}
Clips points out of IUE USSP spectra which have `epsilons' less than
or equal to epsmin. The data are expected to have been previously read
into the stack using the \htmlref{{\tt{USSPRD}}}{COM:USSPRD}  command (with its epsmin parameter
given a positive value); `n1' is the stack entry of the flux data, and
`n2' (which defaults to n1+1) that of the epsilon array. The clipping
is done over the wavelength range w1$<$w$<$w2 (default: full
wavelength range).

\dipcom{VCORR}{vel [mode]}{Applies or removes a velocity shift to a set of X values}
If mode=1 (the default), \htmlref{{\tt{VCORR}}}{COM:VCORR}  `unshifts' X values back to a
zero-velocity reference frame by replacing the values with

\begin{verbatim}
   X2 = X1 / (1.0 + vel/C)
\end{verbatim}

where C is the velocity of light in km/s.

If mode=2, \htmlref{{\tt{VCORR}}}{COM:VCORR}  applies a velocity shift to the data by changing the
X values:

\begin{verbatim}
   X2 = X1 * (1.0 + vel/C).
\end{verbatim}

If the mode=1 or 2 the Y values are not changed. If mode=-1 or -2 the
Y values are also adjusted such that $f_{X}dX$ is constant -- {\it
e.g.} if mode=-1 then

\begin{verbatim}
   Y2 = Y1 * (1.0 + vel/C)
\end{verbatim}

and similarly, {\it mutatis mutandis,} for mode=-2.

\dipcom{WRITE}{output file name or NDF name [model NDF name]}{Write data to a disk file from the current arrays (see {\tt{READ}})} 
Writes the contents of the `current' arrays into a disk file suitable for
subsequent re-reading using \htmlref{{\tt{READ}}}{COM:READ}.  The data can be in either NDF or ``native
DIPSO'' format, depending on the flag established by the \htmlref{{\tt{USENDF}}}{COM:USENDF}  command.
When specifying an NDF, do not include any file type.

If the name of an existing NDF is given for the second (optional) parameter
then the specified NDF is used as a ``model'' for the output NDF. The
output NDF is initialised to hold a copy of the model NDF (including all
extensions). The data values in the current array are then inserted (if
possible) into the output DATA array at their correct wavelength
positions. This provides a mecahanism for creating NDFs which look like
they have been created by other packages. For instance, if you want to
use DIPSO to process the NDF ``{\tt{my\_dat}}''  originally created by the
JCMTDR package, then you would use ``{\tt{read}  my\_dat}'' to read it in
as normal, and then you could use ``{\tt{write}  new\_dat my\_dat}'' to
write the processed data to NDF ``{\tt{new\_dat}}'',  copying all the
JCMTDR extension information (etc) from the original ``{\tt{my\_dat}}'' 
NDF. The resulting NDF could then be put straight back into JCMTDR.

\dipcom{XABS}{(no parameters)}{Replaces X values in the current arrays with their absolute value}
Replaces the X values in the current arrays with ABS(X).

\dipcom{XADD}{c}{Adds a constant to the X values in the current arrays}
Adds a constant, ``c", to the X values in the current arrays.

\dipcom{XDEC}{}{Replaces X values in the current arrays by X-INT(X)}
Replaces the X values in the current arrays by [X$-$INT(X)].

\dipcom{XDIV}{c}{Divides the X values in the current arrays by a constant}
Divides the X values in the current arrays by a constant, ``c".

\dipcom{XINT}{(no parameters)}{Replaces X values in the current arrays by INT(X)}
Replaces values in the current X array with INT(X).

\dipcom{XINV}{(no parameters)}{Replaces X values in the current arrays by 1/X}
Replaces the X values in the current arrays by their inverse.

\dipcom{XMULT}{c}{Multiplies the X values in the current arrays by a constant}
Multiplies the X values in the current arrays by a constant, ``c".

\dipcom{XNINT}{(no parameters)}{Replaces X values in the current arrays by the nearest integers}
Replaces X values in the current arrays by NINT(X).

\dipcom{XSUB}{c}{Subtracts a constant from the X values in the current arrays}
Subtracts a constant, ``c" from the X values in the current arrays.

\dipcom{XCORR}{n1 n2 [lolag hilag p]}{Cross-correlates stack entries}
Cross-correlates the data set in stack entry n1 with the data set in
stack entry n2. (Autocorrelation functions can be calculated by
defining n1=n2.) Entry n1 contains the ``stationary'' data.

The range over which the cross-correlation function is evaluated is
controlled by the parameters lolag and hilag, which must be in the
same units as the stack entries. The default lag is given by:

\begin{verbatim}
   lag = MIN{ (x[n]-x[1]), 100*(x[n]-x[1])/n }
\end{verbatim}

where there are `n'' datum points in stack entry n1, with X values from
x[1] to x[n]. Then lolag defaults to -lag, and hilag to +lag. The
parameter p is the fraction of each data set endmasked (at each end,
with a cosine bell), and defaults to 0.05.

The cross-correlation is carried out in the units of the X arrays
(which have to be monotonically increasing, in the same units for both
data sets, and at least partially overlapping if you want sensible
results). This is for generality. However, a typical application would
be to find velocity shifts between data sets in wavelength space ---
but a velocity shift is a function of wavelength in wavelength space.
The way to get round this is to take logs of both data sets 
(\htmlref{{\tt{LOGX}}}{COM:LOGX}),
then evaluate the correlation function. Then:

\begin{verbatim}
   Delta(V) = [10**{Delta(LogLambda)} - 1] * C
\end{verbatim}

where C is the speed of light (making sure that your units all match
up). The arithmetic can all be done in DIPSO
(\htmlref{{\tt{TENX}}}{COM:TENX}, \htmlref{{\tt{XSUB}}}{COM:XSUB},  \htmlref{{\tt{XMULT}}}{COM:XMULT}). 

IMPORTANT: XCORR just does the cross-correlation; it is not a
`black box'. I can't think of a situation in which you shouldn't first
rectify your data (with \htmlref{{\tt{CREGS}}}{COM:CREGS}  and \htmlref{{\tt{PF}}}{COM:PF},  or \htmlref{{\tt{CDRAW}}}{COM:CDRAW},  followed by \htmlref{{\tt{ADIV}}}{COM:ADIV})  then
subtract the continuum (\htmlref{{\tt{YSUB}}}{COM:YSUB} 1), for both stack entries, in order to
minimise edge effects.

\dipcom{XJ}{(no parameters)}{Sets X limits to match exactly the range of the data being plotted}
``Justifies'' the X range --- {\em i.e.} sets X limits to exactly
match the range of the data being plotted (c.p.\ \htmlref{{\tt{XT}}}{COM:XT}). 

\dipcom{XLAB}{[string]}{Sets the label for the X axis}
The label for the X axis; the default is `Wavelength'. If the string
contains commas it {\em must} be enclosed in double quotes ({\it c.f}
\htmlref{{\tt{TITLE}}}{COM:TITLE}). 

\dipcom{XMAX}{x}{Sets the maximum X value to be plotted}
Set the maximum X value for plotting to `x'.
Negated with \htmlref{{\tt{NX}}}{COM:NX}. 

\dipcom{XMIN}{x}{Sets the minimum X value to be plotted}
Set the minimum X value for plotting to `x'.
Negated with \htmlref{{\tt{NX}}}{COM:NX}. 

\dipcom{XR}{x1 x2}{Sets the minimum and maximum X value to be plotted}
Set the X range; x1 is the left-hand value for the plot, x2 the
right-hand value. Negated using \htmlref{{\tt{NX}}}{COM:NX}. 

\dipcom{XREV}{(no parameters)}{Reverses the ordering of data in the current arrays}
Reverses the ordering of the data in the current X arrays, maintaining
X-Y pairing

\dipcom{XSORT}{(no parameters)}{Sorts the X values in the current arrays into ascending order}
Sorts the X values in the current arrays into increasing values,
maintaining X-Y pairing. Breaks in the data are lost.

\dipcom{XT}{(no parameters)}{Trims the X range to the nearest tick marks}
``Trims'' the X range of a plot --- {\em i.e.} sets X limits which are
some integer multiple of the distance between tickmarks (c.p.\ \htmlref{{\tt{XJ}}}{COM:XJ}). 

\dipcom{XV}{(no parameters)}{Obtains X values using the cursor}
Obtain X values using the cursor. Do a cursor hit at the same place
twice to get out of \htmlref{{\tt{XV}}}{COM:XV}  mode.

\dipcom{XYSWAP}{(no parameters)}{Swaps the contents of the X and Y arrays}
Swaps the contents of the X and Y arrays, maintaining the break arrays
unchanged. (Use \htmlref{{\tt{CLRBRK}}}{COM:CLRBRK}  if this leads to unwanted results.)

\dipcom{XYV}{(no parameters)}{Obtains X and Y values using the cursor}
Obtain X and Y values using the cursor. Hitting the same place twice
exits \htmlref{{\tt{XYV}}}{COM:XYV}  mode.

\dipcom{YABS}{(no parameters)}{Replaces Y values in the current arrays with their absolute value}
Replaces the Y values in the current arrays with ABS(Y).

\dipcom{YADD}{c}{Adds a constant to the Y values in the current arrays}
Adds a constant, `c', to the Y values stored in the `current' arrays.

\dipcom{YDEC}{}{Replaces Y values in the current arrays by X-INT(X)}
Replaces the Y values in the current arrays by [Y$-$INT(Y)].

\dipcom{YDIV}{c}{Divides the Y values in the current array by a constant}
Divides the Y values in the `current' arrays by a constant, `c'.

\dipcom{YINT}{(no parameters)}{Replaces Y values in the current arrays by INT(X)}
Replaces Y values in the current arrays with INT(Y).

\dipcom{YINV}{(no parameters)}{Replaces Y values in the current arrays by 1/X}
Replaces the Y values in the current arrays by their inverse.

\dipcom{YMULT}{c}{Multiplies the Y values in the current arrays by a constant}
Multiplies the Y values in the `current' arrays by a constant, `c'.

\dipcom{YNINT}{(no parameters)}{Replaces Y values in the current arrays by the nearest integers}
Replaces Y values in the `current' arrays with NINT(Y).

\dipcom{YSUB}{c}{Subtracts a constant from the Y values in the current arrays}
Subtract a constant, `c', from the Y values stored in the `current' arrays.

\dipcom{YJ}{(no parameters)}{Sets Y limits to match exactly the range of the data being plotted}
``Justifies'' the Y range --- {\em i.e.} sets the Y limits to exactly
match the range of the data being plotted (c.p.\ \htmlref{{\tt{YT}}}{COM:YT}). 

\dipcom{YLAB}{[string]}{Sets the label for the Y axis}
Replaces the label for the Y axis; the default is `Flux'. If the
string contains commas it {\em must} be enclosed in double quotes
({\it c.f} \htmlref{{\tt{TITLE}}}{COM:TITLE}). 

\dipcom{YMAX}{y}{Sets the maximum Y value to be plotted}
Sets the maximum Y value for plotting; negated with \htmlref{{\tt{NY}}}{COM:NY}. 

\dipcom{YMIN}{y}{Sets the minimum Y value to be plotted}
Sets the minimum Y value for plotting; negated with \htmlref{{\tt{NY}}}{COM:NY}. 

\dipcom{YR}{y1 y2}{Sets the minimum and maximum Y value to be plotted}
Sets the Y range for plotting; y1 is the lower value for the plotting
frame, y2 the upper. Return to autoscaling with \htmlref{{\tt{NY}}}{COM:NY}  or \htmlref{{\tt{NXY}}}{COM:NXY}. 

\dipcom{YT}{(no parameters)}{Trims the Y range to the nearest tick marks}
``Trims'' the Y range of a plot --- {\em i.e.} sets Y limits which are
some integer multiple of the distance between tickmarks (c.p.\ \htmlref{{\tt{YJ}}}{COM:YJ}). 

\dipcom{YV}{[Xvalue]}{Obtains Y values using the cursor}
Obtain Y value. If an X value is supplied, then the corresponding Y
value is obtained (by linear interpolation) from the data in the
current arrays. Otherwise, the graphics cursor is brought up to permit
Y values to be measured from the terminal; hit the same place twice to
get out of this mode.

\dipcom{YXN}{power}{Replaces Y values in the current arrays by Y*(X**power)}
Replaces values in the `current' Y arrays by Y*(X**power). (Some
people like to plot data as F*[Lambda**4], for example.)

\dipcom{ZANSTRA}{line F(obs) T(neb) [E(B-V)]}{Calculates a Zanstra temperature given a re-combination line flux}
Calculates a (black-body) Zanstra temperature, using the observed
flux, F(obs) \newline (in erg/cm2/s), of a recombination line. The
lines for which this calculation can be performed are (H I) 4861; (He
I) 4471, 5876; and (He II) 1640, 4686. The `line' parameter is the
wavelength, in Angstroms, of the selected line.

The calculation requires the location of a `continuum' point longwards
of the ionization edge of the ion concerned. This is obtained from the
cursor; thus a plot, in erg/cm2/s/A vs Angstroms (or log10 thereof) is
mandatory. Moreover, for a valid result the cursor hit has to
correspond to the dereddened stellar flux; judicious prior use of
\htmlref{{\tt{NEBCONT}}}{COM:NEBCONT},  \htmlref{{\tt{ASUB}}}{COM:ASUB}  and \htmlref{{\tt{DRED}}}{COM:DRED}  may therefore be useful.

T(neb) is the electron temperature of the ionized nebula, and may be
input in units of K or 10,000K. This parameter is required because of
the (fairly weak) temperature dependence of the ratio, R(line), of the
effective recombination coefficients to the ion and line concerned
(for a discussion of the physics in the Zanstra method, see {\em e.g.}
Osterbrock, `Astrophysics of Gaseous Nebulae'). The adopted
temperature dependences of R(line) are:

\begin{verbatim}
   R(1640) = 2.00 x (t**0.10)
   R(4471) = 19.61 x (t**0.27)
   R(4686) = 4.37 x (t**0.29)
   R(4861) = 8.49 x (t**0.06)
   R(5876) = 5.39 x (t**0.39)
\end{verbatim}

where the parameter t = T(neb)/(10,000K).

E(B-V) is used to deredden the line flux ({\em only} --- {\em i.e.}
{\em not} the continuum flux); a galactic reddening law with R=3.1 is
adopted. If this is inappropriate to your data, deredden F(obs) to
your own prescription and use E(B-V)=0 (which is the default value).

The output from this command consists of the Zanstra temperature and a
normalising constant, C(norm.). The latter quantity is the number by
which a black-body spectrum calculated using \htmlref{{\tt{BBODY}}}{COM:BBODY}  must be multiplied
(YMULT) to make it pass through the (X,Y) co-ordinates selected using
the cursor.

\end{description}

\subsection{Finally...}
Congratulations on reading this far (unless you've cheated, and skipped
straight to the end...). Features, bugs, complaints and comments should
be addressed to Ian Howarth ({\tt{idh@star.ucl.ac.uk}}),  or the Starlink
Software Librarian ({\tt{ussc@star.rl.ac.uk}});  but please check the
documentation first!

\newpage

\section{\label{APP:HISTORY}History}


This section records the changes introduced with each new version of DIPSO.
{\em NOTE,} earlier changes may be over-ridden by later changes.

\subsection{Changes introduced by DIPSO V3.6-5}
\begin{itemize}
\item The ELFINP command now has a parameter that can be set non-zero to
allow input to be read from the currently running script file rather than
from the keyboard.
\end{itemize}

\subsection{Changes introduced by DIPSO V3.6-4}
\begin{itemize}
\item The dipso\_link command has been modified in order to avoid
problems on Solaris.
\end{itemize}

\subsection{Changes introduced by DIPSO V3.6}
\begin{itemize}
\item The {\tt READ} command has been modified so that it will attempt to
read spectral axis information from the WCS component of an NDF if the NDF
was not created by DIPSO. If the NDF has no WCS component, the spectral
axis information will be read from the NDF AXIS structures, or the FITS header 
cards in the NDF FITS extension (if any). Details of the spectral axis are 
displayed.
\item The {\tt WRITE} command has been modified so that it will add a WCS
component to the output NDF describing the spectral axis. This allows it
to be used by other Starlink applications such as SPLAT, KAPPA, etc.
\end{itemize}

\subsection{Changes introduced by DIPSO V3.5-6}
\begin{itemize}
\item A bug has been fixed which caused the ELFRESTC command to be unable to
open files which contain any lower case letters in their file name.
\end{itemize}

\subsection{Changes introduced by DIPSO V3.5-5}
\begin{itemize}
\item A bug has been fixed which caused the ISATM command to be unable to
find entries in the ATOMIC.DAT file when being run under Linux.
\end{itemize}

\subsection{Changes introduced by DIPSO V3.5}
\begin{itemize}
\item The new command {tt RDCAT} has been added to enable data held in FITS 
table format to be read.
\item The maximum number of clouds which can be specified using ISINP has
been increased from 9 to 18.
\item A bug has been fixed which prevented ELFOPT working correctly.
\end{itemize}

\subsection{Changes introduced by DIPSO V3.4}
\begin{itemize}

\item The {\tt USEHTX} command has been modified to allow the path to the
Starlink {\tt showme} utility to be given explicitly. This need only be done 
if the {\tt HELP} command fails to find the {\tt showme} utility (for
instance, if is not on the user's current PATH, or if it is
executed by means of an alias). 

\item The contents of restored stacks held in NDF structures are now
listed as they are restored.

\item The makefile has been modified so that ``my\_dipso'' works as
described in section \ref{SEC:MY_DIPSO} (i.e. you no longer need to
set the INSTALL environment variable prior to using ``my\_dipso'').

\item The new command {\tt ENV} has been added, which displays the value
being used by DIPSO for a named environment variable.

\item The URL for the DIPSO WWW home page has been corrected.

\item Default values for environment variables such as \verb+DIPSODIR+ can
now be supplied on the DIPSO command line. They should take the form of a
comma separated list of ``name=value'' pairs. These values are only used
if the corresponding environment variables are not defined.

\item The command prompt used within DIPSO can now be specified using the 
environment variable \verb+DIPSOPROMPT+.

\item Various changes to the Fortran code have been made to allow DIPSO
to be compiled using the \verb+g77+ compiler under Linux (which requires
closer adherance to the ANSI Fortran-77 standard than other copilers).
None of these changes should produce any noticable change in behaviour.

\end{itemize}

\subsection{Changes introduced by DIPSO V3.3}
\begin{itemize}

\item The internal array sizes have been increased. The current arrays can 
now hold 200,000 points, and the stack arrays can now hold 800,000 points in 
up to 200 stack entries.

\item Hypertext documentation and help have been included through the
{\tt HELP} and {\tt USEHTX} commands, and a hypertext version of SUN/50.

\item The {\tt SHELL} command has been re-instated, which allows system 
commands to be run from within DIPSO without the need to press control-Z.

\item The functionality of the {\tt COMMAND} command has been enhanced to 
provide faciltities for listing commands which do particular jobs, and
which contain specified keywords in their descriptions.

\item NDF accessing is now done using latest NDF facilities to allow
transparent access to foreign data formats (IRAF, FITS, etc, see SUN/55 and SSN/20).

\item All uses of NAG routines have been replaced by equivalent public
domain algorithms (in fact the only command which used NAG was {\tt PF}).

\item The {\tt help.lis} file containing plain text descriptions of all
DIPSO commands has been updated to be consistent with SUN/50. This file
is used by the {\tt HELP} command.

\item  Re-direction of GKS error messages removed. They should now go
to the screen.

\item Bug corrected which caused error messages to be displayed when
{\tt PUSH}ing the current array if the last point in the data array was bad.

\item {\tt ELFIT} changed to avoid floating exception if the variances come
out negative (a warning is issued and zeros are used instead).

\item {\tt ELFSAVEC} now offers the user the chance to overwrite an
existing file if one exists. Also, it no longer converts the
supplied file name to upper case.

\item {\tt SM} bug corrected which caused extension of smoothing domain
beyond the upper wavelength limit, and could potentially cause
completely spurious results if the upper wavelength limit was
at the end of the array.

\item Organisation of source files and object files changed to use a
sigle library.

\end{itemize}

\subsection{Changes introduced by DIPSO V3.2}
\begin{itemize}
\item A new command \htmlref{{\tt{TWEIGHT}}}{COM:TWEIGHT}  has been introduced to allow control of the
weight used to draw data curves independantly of the weight of other
lines.

\item Bug corrected which caused the \htmlref{{\tt{PUSH}}}{COM:PUSH}  command to fail with a message
like:

\begin{verbatim}
   PUSH: breaks/data mismatch
   number of breaks:125
   index of last break:22792
   number of points:22791
   Command line aborted
\end{verbatim}

after reading in a spectrum using \htmlref{{\tt{SP0RD}}}{COM:SP0RD}. This
only occurred if the first and/or last data point in the spectrum
contained the value zero.

\item Bug corrected which caused the \htmlref{{\tt{NEBCONT}}}{COM:NEBCONT}  command to re-issue a prompt
for a value, rather than accepting the default value as read from the
specified file, if <RETURN> is pressed (with mode1,2,3=0).

\item Error messages generated by GKS are now displayed as they occur,
rather than being stored up and displayed altogether when DIPSO is
exited.

\item Bug fixed which caused cursor operations (eg \htmlref{{\tt{XV}}}{COM:XV},  \htmlref{{\tt{SNIP}}}{COM:SNIP},  etc ) to
fail after an invocation of the condition handler caused by floating
point exceptions, control-C's etc.

\item Shell meta-characters can now be included in responses given to
prompts for NDFs.

\end{itemize}

\subsection{Changes introduced by DIPSO V3.1}
\begin{itemize}

\item Command line recall and editing is now available.

\item Condition/signal handling is now available on both VMS and UNIX.
The \htmlref{{\tt{HANDLER}}}{COM:HANDLER}  and \htmlref{{\tt{CRASH}}}{COM:CRASH}  commands have been re-instated.

\item The NDF accessing layer has been completely re-written to
eliminate the many bugs related to the reading and writing of NDFs in
the previous version DIPSO. The description of how to write a program
which can create NDFs suitable for use with DIPSO (see 
\hyperref{earlier}{section}{}{SEC:WRNDF}) has been re-written.

\item Various other bugs have been cleared up.

\item The user can now decide whether to use NDF data files or ``native
DIPSO'' data files. This is accomplished using the \htmlref{{\tt{USENDF}}}{COM:USENDF}  command.
Commands \htmlref{{\tt{READ}}}{COM:READ},  \htmlref{{\tt{WRITE}}}{COM:WRITE},  \htmlref{{\tt{SAVE}}}{COM:SAVE},  {\em etc.}, will now use the selected data
format, and so the commands OREAD, OWRITE, OSAVE, {\em etc.}, are no
longer needed and have been withdrawn.

\item New commands added: \htmlref{{\tt{USENDF}}}{COM:USENDF},  \htmlref{{\tt{REPORTING}}}{COM:REPORTING},  \htmlref{{\tt{TSWAP}}}{COM:TSWAP},  \htmlref{{\tt{TSTRIP}}}{COM:TSTRIP} 

\item Commands re-instated from previous versions: \htmlref{{\tt{RECORD}}}{COM:RECORD},  \htmlref{{\tt{REPLAY}}}{COM:REPLAY}, 
\htmlref{{\tt{RECALL}}}{COM:RECALL},  SHELL (only on VMS),

\item The \htmlref{{\tt{EXPAND}}}{COM:EXPAND}  command now takes an optional second string parameter
which specifies which components of the plot are to be expanded.

\item The behaviour of prompts has been unified (to some extent).
Giving ``!'' in response to any prompt will cause the current command to
abort and return you to the main DIPSO command prompt. Giving ``!!'' will
abort the command and also abort DIPSO (after saving the stack to
EXIT\_STK.sdf or EXIT.STK). Some prompts now include a suggested default
within the prompt string which will be used if a null value is supplied
by the user.

\item \htmlref{{\tt{READ}}}{COM:READ}  command (when using NDFs) will now read NDFs not written by
DIPSO ({\em i.e.} ones which don't have a DIPSO\_EXTRA extension).

\item \htmlref{{\tt{WRITE}}}{COM:WRITE}  command (when using NDFs) now has an optional second string
parameter which specifies the name of a ``model'' NDF on which to base
the output NDF created by the \htmlref{{\tt{WRITE}}}{COM:WRITE}  command.

\item SP0RD command (when using NDFs) now has a second (optional)
YES/NO parameter which determines if zeros in the data file are treated
as gaps in the data.

\item \htmlref{{\tt{SCREENRD}}}{COM:SCREENRD}  command, ``\verb+\+'' is now used to end input on VMS
and UNIX.

\item The VMS version is no longer compiled ``/DEBUG/NOOPT'' so
hopefully it could be a bit (!) faster?

\item VMS and UNIX versions of source files have been unified.

\end{itemize}

\subsection{ Changes introduced by DIPSO V3.00}

The default file format for `unformatted' DIPSO files (SP0) has
changed from native unformatted format to STARLINK NDF format. This
allows the transport of files between different machines without
translation being necessary. It also means that files generated by
DIPSO can be automatically input to all standard STARLINK packages.

This version of DIPSO is the first multi-platform release. It has been
tested on VAX/VMS, DECstation, and Sun Sparcstation machines.
Due to the different way system variables are provided by the two
operating systems DIPSO has been enhanced to support both the VMS
logical name syntax, and the UNIX environment variable syntax. Thus
the following two filename specifications are equivalent and accepted
by DIPSO on all platforms.

\begin{itemize}
\item{VAX logical name form --- {\tt{OWNERDIR:my\_data\_file}}} 
\item{UNIX environment form --- {\tt{\$OWNERDIR/my\_data\_file}}} 
\end{itemize}

UNIX users should note that filenames are ALWAYS CaSe sensitive.

\subsection{ Changes introduced by DIPSO V2.00}

(If you didn't use DIPSO before 1987, pass over this section. If you
are an old hand, then: the more experienced you are, the more
important it is that you should {\em read this section carefully!})

This release of DIPSO is a fairly extensive revision of earlier
versions. In particular, the graphics have been converted to the GKS
standard (mainly by JM), interstellar line analysis has been included
({\tt{IS...}}  commands), and various aspects of Fourier analysis are
now possible. Because interstellar line profiles are now most easily
computed from within DIPSO, the old {\tt{BACHRD}}  and {\tt{BACHWR}} 
commands are no longer documented, and {\tt{DIPSODIR:ATOMIC.DAT}}  has
been extensively revised ({\tt{\$DIPSODIR/ATOMIC.DAT}}  on UNIX
machines). {\tt{ALASRD/ALASWR}}  have been preserved, however, as the
simplest way of getting data in and out of DIPSO. (\htmlref{{\tt{ALASRD}}}{COM:ALASRD}  has
actually been updated to allow more general inputs.)

In an attempt to (partially) rationalise the command names, and make it
easier to locate groups of related commands in the reference section,
some command names have been changed. In particular, the old {\tt{ELF}} 
commands are now all prefixed by {\tt{``ELF''}}  (surprise!). (The minus
side is that you'll have to learn the new command names; but the plus
side is that {\tt{ELF}}  now carries out an error analysis for you.) The
two-spectrum arithmetic functions are now renamed {\tt{ADIV,}  \htmlref{{\tt{AADD}}}{COM:AADD}}  {\em
etc.} (from {\tt{DIV,}  ADD} {\em etc.} to avoid the trap of typing, say,
{\tt{ADD~3}}  in the expectation of adding 3 to the current arrays (the
``A'' prefix stands for ``array''.) Some one-spectrum arithmetic
functions have also had their names changed to a more uniform scheme;
{\em e.g.} {\tt{XSH}}  has become \htmlref{{\tt{XADD}}}{COM:XADD},  {\tt{CMULT}}  has become
\htmlref{{\tt{YMULT}}}{COM:YMULT},  {\em etc.} The old command names will still work in some cases,
but are not recommended. You are urged to read right through the new
documentation for individual commands, as many other minor modifications
have been made, and new functions added.

The most important changes: you will probably find out quite quickly
that the default file extension for command files has changed to
{\tt{.CMD}}. 

The following commands have {\em changed default functionality}, and
you should therefore check them especially carefully: \htmlref{{\tt{ALASRD}}}{COM:ALASRD}, 
\htmlref{{\tt{ALASWR}}}{COM:ALASWR},  \htmlref{{\tt{DRED}}}{COM:DRED}  \htmlref{{\tt{PWRITE}}}{COM:PWRITE},  \htmlref{{\tt{TPORT}}}{COM:TPORT},  \htmlref{{\tt{READ}}}{COM:READ}, 
\htmlref{{\tt{WRITE}}}{COM:WRITE},  \htmlref{{\tt{SAVE}}}{COM:SAVE},  \htmlref{{\tt{RESTORE}}}{COM:RESTORE}. 

\end{document}

