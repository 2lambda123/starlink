\documentclass[twoside,11pt]{article}

%+
%  Name:
%     sun17.tex

%  Purpose:
%     Starlink User Note 17

%  Authors:
%     Richard M. Prestage (JAC, Hawaii)
%     Horst Meyerdierks (ROE)
%     John F. Lightfoot (ROE)
%     Tim Jenness (JAC, Hawaii)
%     Remo Tilanus (JAC, Hawaii)

%  History:
%     13 Jan 1995 (RMP):
%        First UNIX release of SPECX
%     9  Jan 2000 (TIMJ):
%        Try to update document - remove VMS references

%-

% ? Specify used packages
% \usepackage{graphicx}        %  Use this one for final production.
% \usepackage[draft]{graphicx} %  Use this one for drafting.
% ? End of specify used packages

\pagestyle{myheadings}

% -----------------------------------------------------------------------------
% ? Document identification
% Fixed part
\newcommand{\stardoccategory}  {Starlink User Note}
\newcommand{\stardocinitials}  {SUN}
\newcommand{\stardocsource}    {sun\stardocnumber}

% Variable part - replace [xxx] as appropriate.
\newcommand{\stardocnumber}    {17.8}
\newcommand{\stardocauthors}   {R.\,M.\,Prestage, H.\,Meyerdierks,
J.\,F.\,Lightfoot, \\ T.\,Jenness, R.\,P.\,J.\,Tilanus, R.\,Padman}
\newcommand{\stardocdate}      {11 July 2000}
\newcommand{\stardoctitle}     {SPECX --- A Millimetre Wave Spectral Reduction
Package}
\newcommand{\stardocversion}   {6.7-7}
\newcommand{\stardocmanual}    {Users' Manual}
\newcommand{\stardocabstract}  {SPECX is a general purpose mm and submm
spectral line data reduction system. This document provides an overview
of the JAC/Starlink version of SPECX. See MUD/070 for a full manual to
SPECX.}
% ? End of document identification
% -----------------------------------------------------------------------------

% +
%  Name:
%     sun.tex
%
%  Purpose:
%     Template for Starlink User Note (SUN) documents.
%     Refer to SUN/199
%
%  Authors:
%     AJC: A.J.Chipperfield (Starlink, RAL)
%     BLY: M.J.Bly (Starlink, RAL)
%     PWD: Peter W. Draper (Starlink, Durham University)
%
%  History:
%     17-JAN-1996 (AJC):
%        Original with hypertext macros, based on MDL plain originals.
%     16-JUN-1997 (BLY):
%        Adapted for LaTeX2e.
%        Added picture commands.
%     13-AUG-1998 (PWD):
%        Converted for use with LaTeX2HTML version 98.2 and
%        Star2HTML version 1.3.
%     {Add further history here}
%
% -

\newcommand{\stardocname}{\stardocinitials /\stardocnumber}
\markboth{\stardocname}{\stardocname}
\setlength{\textwidth}{160mm}
\setlength{\textheight}{230mm}
\setlength{\topmargin}{-2mm}
\setlength{\oddsidemargin}{0mm}
\setlength{\evensidemargin}{0mm}
\setlength{\parindent}{0mm}
\setlength{\parskip}{\medskipamount}
\setlength{\unitlength}{1mm}

% -----------------------------------------------------------------------------
%  Hypertext definitions.
%  ======================
%  These are used by the LaTeX2HTML translator in conjunction with star2html.

%  Comment.sty: version 2.0, 19 June 1992
%  Selectively in/exclude pieces of text.
%
%  Author
%    Victor Eijkhout                                      <eijkhout@cs.utk.edu>
%    Department of Computer Science
%    University Tennessee at Knoxville
%    104 Ayres Hall
%    Knoxville, TN 37996
%    USA

%  Do not remove the %begin{latexonly} and %end{latexonly} lines (used by
%  LaTeX2HTML to signify text it shouldn't process).
%begin{latexonly}
\makeatletter
\def\makeinnocent#1{\catcode`#1=12 }
\def\csarg#1#2{\expandafter#1\csname#2\endcsname}

\def\ThrowAwayComment#1{\begingroup
    \def\CurrentComment{#1}%
    \let\do\makeinnocent \dospecials
    \makeinnocent\^^L% and whatever other special cases
    \endlinechar`\^^M \catcode`\^^M=12 \xComment}
{\catcode`\^^M=12 \endlinechar=-1 %
 \gdef\xComment#1^^M{\def\test{#1}
      \csarg\ifx{PlainEnd\CurrentComment Test}\test
          \let\html@next\endgroup
      \else \csarg\ifx{LaLaEnd\CurrentComment Test}\test
            \edef\html@next{\endgroup\noexpand\end{\CurrentComment}}
      \else \let\html@next\xComment
      \fi \fi \html@next}
}
\makeatother

\def\includecomment
 #1{\expandafter\def\csname#1\endcsname{}%
    \expandafter\def\csname end#1\endcsname{}}
\def\excludecomment
 #1{\expandafter\def\csname#1\endcsname{\ThrowAwayComment{#1}}%
    {\escapechar=-1\relax
     \csarg\xdef{PlainEnd#1Test}{\string\\end#1}%
     \csarg\xdef{LaLaEnd#1Test}{\string\\end\string\{#1\string\}}%
    }}

%  Define environments that ignore their contents.
\excludecomment{comment}
\excludecomment{rawhtml}
\excludecomment{htmlonly}

%  Hypertext commands etc. This is a condensed version of the html.sty
%  file supplied with LaTeX2HTML by: Nikos Drakos <nikos@cbl.leeds.ac.uk> &
%  Jelle van Zeijl <jvzeijl@isou17.estec.esa.nl>. The LaTeX2HTML documentation
%  should be consulted about all commands (and the environments defined above)
%  except \xref and \xlabel which are Starlink specific.

\newcommand{\htmladdnormallinkfoot}[2]{#1\footnote{#2}}
\newcommand{\htmladdnormallink}[2]{#1}
\newcommand{\htmladdimg}[1]{}
\newcommand{\hyperref}[4]{#2\ref{#4}#3}
\newcommand{\htmlref}[2]{#1}
\newcommand{\htmlimage}[1]{}
\newcommand{\htmladdtonavigation}[1]{}

\newenvironment{latexonly}{}{}
\newcommand{\latex}[1]{#1}
\newcommand{\html}[1]{}
\newcommand{\latexhtml}[2]{#1}
\newcommand{\HTMLcode}[2][]{}

%  Starlink cross-references and labels.
\newcommand{\xref}[3]{#1}
\newcommand{\xlabel}[1]{}

%  LaTeX2HTML symbol.
\newcommand{\latextohtml}{\LaTeX2\texttt{HTML}}

%  Define command to re-centre underscore for Latex and leave as normal
%  for HTML (severe problems with \_ in tabbing environments and \_\_
%  generally otherwise).
\renewcommand{\_}{\texttt{\symbol{95}}}

% -----------------------------------------------------------------------------
%  Debugging.
%  =========
%  Remove % on the following to debug links in the HTML version using Latex.

% \newcommand{\hotlink}[2]{\fbox{\begin{tabular}[t]{@{}c@{}}#1\\\hline{\footnotesize #2}\end{tabular}}}
% \renewcommand{\htmladdnormallinkfoot}[2]{\hotlink{#1}{#2}}
% \renewcommand{\htmladdnormallink}[2]{\hotlink{#1}{#2}}
% \renewcommand{\hyperref}[4]{\hotlink{#1}{\S\ref{#4}}}
% \renewcommand{\htmlref}[2]{\hotlink{#1}{\S\ref{#2}}}
% \renewcommand{\xref}[3]{\hotlink{#1}{#2 -- #3}}
%end{latexonly}
% -----------------------------------------------------------------------------
% ? Document specific \newcommand or \newenvironment commands.

% Environment for indenting and using a small font.
\newenvironment{myquote}{\begin{quote}\begin{small}}{\end{small}\end{quote}}

\newcommand{\text}[1]{{\small \tt #1}}


% ? End of document specific commands
% -----------------------------------------------------------------------------
%  Title Page.
%  ===========
\renewcommand{\thepage}{\roman{page}}
\begin{document}
\thispagestyle{empty}

%  Latex document header.
%  ======================
\begin{latexonly}
   CCLRC / \textsc{Rutherford Appleton Laboratory} \hfill \textbf{\stardocname}\\
   {\large Particle Physics \& Astronomy Research Council}\\
   {\large Starlink Project\\}
   {\large \stardoccategory\ \stardocnumber}
   \begin{flushright}
   \stardocauthors\\
   \stardocdate
   \end{flushright}
   \vspace{-4mm}
   \rule{\textwidth}{0.5mm}
   \vspace{5mm}
   \begin{center}
   {\Huge\textbf{\stardoctitle \\ [2.5ex]}}
   {\LARGE\textbf{\stardocversion \\ [4ex]}}
   {\Huge\textbf{\stardocmanual}}
   \end{center}
   \vspace{5mm}

% ? Add picture here if required for the LaTeX version.
%   e.g. \includegraphics[scale=0.3]{filename.ps}
% ? End of picture

% ? Heading for abstract if used.
   \vspace{10mm}
   \begin{center}
      {\Large\textbf{Abstract}}
   \end{center}
% ? End of heading for abstract.
\end{latexonly}

%  HTML documentation header.
%  ==========================
\begin{htmlonly}
   \xlabel{}
   \begin{rawhtml} <H1> \end{rawhtml}
      \stardoctitle\\
      \stardocversion\\
      \stardocmanual
   \begin{rawhtml} </H1> <HR> \end{rawhtml}

% ? Add picture here if required for the hypertext version.
%   e.g. \includegraphics[scale=0.7]{filename.ps}
% ? End of picture

   \begin{rawhtml} <P> <I> \end{rawhtml}
   \stardoccategory\ \stardocnumber \\
   \stardocauthors \\
   \stardocdate
   \begin{rawhtml} </I> </P> <H3> \end{rawhtml}
      \htmladdnormallink{CCLRC}{http://www.cclrc.ac.uk} /
      \htmladdnormallink{Rutherford Appleton Laboratory}
                        {http://www.cclrc.ac.uk/ral} \\
      \htmladdnormallink{Particle Physics \& Astronomy Research Council}
                        {http://www.pparc.ac.uk} \\
   \begin{rawhtml} </H3> <H2> \end{rawhtml}
      \htmladdnormallink{Starlink Project}{http://www.starlink.ac.uk/}
   \begin{rawhtml} </H2> \end{rawhtml}
   \htmladdnormallink{\htmladdimg{source.gif} Retrieve hardcopy}
      {http://www.starlink.ac.uk/cgi-bin/hcserver?\stardocsource}\\

%  HTML document table of contents.
%  ================================
%  Add table of contents header and a navigation button to return to this
%  point in the document (this should always go before the abstract \section).
  \label{stardoccontents}
  \begin{rawhtml}
    <HR>
    <H2>Contents</H2>
  \end{rawhtml}
  \htmladdtonavigation{\htmlref{\htmladdimg{contents_motif.gif}}
        {stardoccontents}}

% ? New section for abstract if used.
  \section{\xlabel{abstract}Abstract}
% ? End of new section for abstract
\end{htmlonly}

% -----------------------------------------------------------------------------
% ? Document Abstract. (if used)
%  ==================
\stardocabstract
% ? End of document abstract
% -----------------------------------------------------------------------------
% ? Latex document Table of Contents (if used).
%  ===========================================
  \newpage
  \begin{latexonly}
    \setlength{\parskip}{0mm}
    \tableofcontents
    \setlength{\parskip}{\medskipamount}
    \markboth{\stardocname}{\stardocname}
  \end{latexonly}
% ? End of Latex document table of contents
% -----------------------------------------------------------------------------
\cleardoublepage
\renewcommand{\thepage}{\arabic{page}}
\setcounter{page}{1}

\section {Introduction}

SPECX is a general mm and sub-mm wavelength data reduction system written by
Rachael Padman at \htmladdnormallink{MRAO}{http://www.mrao.cam.ac.uk}. At the
time of writing the latest formally released version on the VAX is 6.2.
However, pre-release copies of Version 6.3 are widely used. Branching off from
the 6.3 VMS version, SPECX was ported to UNIX by Horst Meyerdierks of the
\htmladdnormallink{Starlink Project}{http://www.starlink.ac.uk} and Rachael
Padman. Remo Tilanus at JACH supplied the core routines to read GSD files and
Tim Jenness at the JAC took over development for Version 6.7. The software
version count on UNIX increased rather rapidly. 6.4 was the initial UNIX port
of 6.3, but it existed only in beta-test versions. The first released version
was 6.5, but it lacked the commands to read \xref{GSD}{sun229}{} files.
Version 6.6 added support for \xref{GSD}{sun229}{} files. The current version
on UNIX is 6.7 which added support for a new \xref{HDS}{sun92}{} map format
(V4.2) and routines to aid the reduction of raster maps. Version 6.7 is
currently supported by the Joint Astronomy Centre and distributed by Starlink.

Although SPECX may be used to process spectra obtained from many different
instruments, Starlink users will find it particularly useful for the reduction
of data obtained with the James Clerk Maxwell Telescope.  SPECX has it's own
command line interpreter, and manipulates spectra on a pop-down stack.  The
package is designed to work on a variety of graphics terminals but is mainly
used from within an X windows environment. Hardcopy may be output on a number
of printers for which \xref{GKS}{sun83}{} drivers are provided by Starlink. In
the UNIX version these are mainly different PostScript printers.

Some of the major features of SPECX are:
\begin{itemize}
\item
The ability to process up to eight spectra (quadrants) simultaneously:
these may have the same or different centre frequencies, resolutions,
\emph{etc.} This allows users to operate on different filter-banks, or
dual polarisation data for example, in parallel.
\item
The ability to automatically save the current status of the system
after each command is executed. This means that if SPECX is
stopped (unintentionally or otherwise), when restarted it will
`remember' the previous state of the program.
\item
Commands for the listing and display of spectra on a graphics
terminal, with hardcopy on a variety of printers.
\item
Single and multiple scan arithmetic, scan averaging, \emph{etc.}
\item
Commands to store and retrieve intermediate spectra in storage registers.
\item
Commands to perform the fitting and removal of polynomial,
harmonic and Gaussian baselines.
\item
Commands for filtering and editing spectra.
\item
Commands to determine important line parameters (peak intensity, width,
\emph{etc.}).
\item
The ability to perform Fourier transform and power spectrum calculations.
\item
Procedures for the calibration of data.
\item
The ability to assemble a number of reduced individual spectra into a
map file, and contour or greyscale any plane or planes of the resulting cube.
\item
The ability to write indirect command files.
\end{itemize}

A thorough description of the package is given in the ``SPECX Users' Manual''
which has been distributed as a Starlink Miscellaneous User Document
(MUD/70). At the time of writing this MUD is available from Starlink for
Version 6.3. No document has been prepared yet for the UNIX
version, and the document on 6.3 is the closest approximation for UNIX
users.

SPECX is intended primarily for the analysis of JCMT data and can read such
data directly from the \xref{GSD}{sun229}{} files produced at the telescope.  In the VMS
version there are also facilities for reading data from a range of other
important mm-wave telescopes, {\em e.g.} Onsala, FCRAO, UMASS CO survey,
IRAM-FITS, JCMT RxG, NRAO 8-beam (see the SPECX user manual for details).
Spectra and maps can be output as FITS files, Starlink NDFs, or ASCII files
for reading into other packages.

This document concerns itself with the unix version of SPECX and will
no longer discuss the VMS Version 6.2.

\section {Getting started}

SPECX is made available by typing:
\begin{myquote}
\begin{verbatim}
% specxstart
\end{verbatim}
\end{myquote}
and started by typing:
\begin{myquote}
\begin{verbatim}
% specx
\end{verbatim}
\end{myquote}
The program will display the current version number and
a brief informational message, then the SPECX prompt ({\tt>>}) will appear.

\begin{myquote}
\begin{verbatim}
            --------------------------------------------
              Specx v6.7-6 (UNIX) ---- 15 December 1999
                   Copyright (C)  R.Padman & PPARC
             Acknowledgements to Horst Meyerdierks (PPARC)
                  Remo Tilanus & Tim Jenness (JACH)
            --------------------------------------------
        .
        .
        .
>>
\end{verbatim}
\end{myquote}

Commands may then be entered to read and process data.
SPECX commands consist of one or more keywords, separated
by hyphens. Minimum matching is applied to each keyword
individually. A full list of all valid SPECX commands
may be found by entering {\tt show-commands}, with a carriage return
in response to the resulting prompt. On-line help is available
using the command {\tt help}. However, the on-line help is mainly a copy
from the VMS VERSION 6.3. There are some added topics on the UNIX version.

A shell command may be executed by preceding it with a {\tt \$}, {\em e.g.}

\begin{myquote}
\begin{verbatim}
>> $ls
\end{verbatim}
\end{myquote}
% $

The tab and blank characters, `,', `\verb+\+' and `;' are SPECX delimiters;
to pass these to the shell the entire string must be enclosed in single
inverted commas, {\em e.g.}

\begin{myquote}
\begin{verbatim}
>> '$ls -l scan*.dat'
\end{verbatim}
\end{myquote}
% $

The {\tt exit} or \texttt{quit} commands are the normal method of returning to
the shell.  SPECX can also be suspended in the usual way with Control-Z and
later resumed with the shell command {\tt fg}.

In some situations SPECX has to be given an end-of-file character. On
VMS this was Control-Z. On UNIX it is Control-D. As of V6.7-1 Control-C
can be used to abort a command in progress.


\section {SPECX files and directories}

SPECX uses and produces a variety of files. For graphics,
intermediate plot files (called {\tt PLOT.}nnn) are produced in the
current working directory.

SPECX uses Starlink PGPLOT for graphics and thus will produce data files for
plots sent to some graphics devices ({\em e.g.} {\tt gks74.ps}.N for a
PostScript printer). These are created in the current default directory. They
are not submitted to any printer and are not deleted. SPECX does not even
report the exact name of the file when it is created. The user has to keep
score of these files and send them to printers as desired.

There is an option (defaulted true) in SPECX which allows dumping of the
current status of the program after each command is executed. This is saved in
the file {\tt specx.dmp} in the current default directory. If this file does
not exist when SPECX is started for the first time, it will be created. If it
is subsequently deleted (or SPECX is started from a different directory), the
package will restart with default initialisation; otherwise SPECX will be
re-started will all flags as previously set, data files opened, and so on.


\section {Examples}

New users should read the introductory chapters of the SPECX manual,
which give explanations and examples of basic techniques. There is a
test procedure in {\tt \$SYS\_SPECX/test.spx} and the data files
necessary are in the same directory. However, this procedure is not
commented at all. User are advised to compare it with {\tt demo.spx},
which is a copy from the VMS version and will probably not work, but has
comments as to the meaning of the commands.
{\tt vms\_demo.spx} demonstrates the reading of JCMT data,
the fitting of baselines and Gaussian to spectra, and shows how to make
and display a data cube.


\section{Linking user-supplied subroutines}

User-defined commands can be added to SPECX by making use of the
subroutines EXTRNL1 to EXTRNL10. There is currently little support for
this and the user is left to hack her way around the SPECX directory
tree.

It is relatively easy to modify the distributed source code. This is
mainly an option for a local but site-wide version with EXTRNLn
routines. Assuming that you are a Starlink site manager you would
proceed as follows

\begin{myquote}
\begin{verbatim}
% cd /star/sources/specx                       # 1
% setenv INSTALL /star
% ./mk deinstall

% cd /star                                     # 2
% mv specx /star/specx /star/local/specx

% cd /star/local/specx/external                # 3
{Make your modifications in this directory}

% cd /star/local/specx                         # 4
% setenv INSTALL /star/local
% ./mk build
% ./mk clean
% ./mk install

% cd /star/local/etc                           # 5
{Modify files login and cshrc to reflect that SPECX is in /star/local}
\end{verbatim}
\end{myquote}

\begin{itemize}
\item[1] De-install the existing package from {\tt /star/bin/specx} and
   {\tt /star/help/specx}. After this step the existing package will
   still be in its built form in {\tt /star/specx}.
\item[2] Move (or copy with {\tt cp -pr}) the built system to the local
   Starlink tree. You can use any directory.
\item[3] Generate your own version of the ``external'' library. You have
   to carefully inspect the {\tt makefile} and amend it. Also check {\tt
   makefile}s in sister directories like {\tt ../fitting} to see how
   include files are handled.
\item[4] Re-build and install SPECX. Installation goes into {\tt
   /star/local/bin/specx} and {\tt /star/\-local/help/specx}. You can use
   something else instead of {\tt /star/local}.
\item[5] Amend the Starlink login files to reflect the different place
   for the SPECX installation.
\end{itemize}

In order to re-build SPECX you need a number of Starlink libraries and
include files in their usual place. Non-Starlink sites may or may not
have been issued with these libraries. As indicators, you can check
whether you have {\tt /star/lib/libhds*}, {\tt /star/include/dat\_par}
and {\tt /star/figaro/lib/libfit.a}.


\section{\xlabel{data_formats_in_and_data_migration_to_the_unix_version}Data
formats in, and data migration to, the UNIX version}

UNIX SPECX introduced yet another version of data formats (4.1 and 4.2) for
spectra and for maps. It can read and write disk-FITS. From VERSION 6.6
onwards it can also read \xref{GSD}{sun229}{} files.

Version 4.1 spectral format is based on Starlink's Hierarchical Data
Structures (HDS), a binary format that is portable between different
machine architectures. Thus you can take VERSION 4.1 spectra from a Sun
workstation to an Alpha workstation and back, without noticing it.

Versions 4.1 and 4.2 map formats, too, are based on HDS. Thus maps are
portable as well.

Data currently held on a VMS file system can be carried across to UNIX
as detailed below. Note that the way back may not be possible, or at
least difficult.
\begin{itemize}
\item {\bf GSD files:}
    Data in \xref{GSD}{sun229}{} format can be copied binary. One way is the
    UNIX {\tt cp} command if both disks are mounted by the UNIX machine.
    Another way is {\tt ftp} in binary mode. Then use {\tt
    READ-GSD-DATA} in the usual way.

    For going back from UNIX to VMS similar problems may arise as with
    disk-FITS files (see below).

\item {\bf Disk-FITS files:}
    These must be written with IBM byte order (swap order
    on Digital machines). They can then be copied binary. One way is the
    UNIX {\tt cp} command if both disks are mounted by the UNIX machine.
    Another way is {\tt ftp} in binary mode.

    This works between UNIX machines and from VMS to UNIX. If disk-FITS
    files are moved from UNIX to VMS, it may be necessary to make them
    files of fixed record length 2880 byte. {\tt cp} is thus unsuitable.
    {\tt ftp} in binary mode will create the right sort of file with the
    wrong record length of 512 byte. You can either rectify this with
    some VMS command, or use a more tolerant disk-FITS reader under VMS.

\item {\bf Spectral data VERSION 2 and 3:}
    These are VMS binary formats. Take a
    binary copy to the destination machine. The VMS Versions SPECX 6.2
    and 6.3 can use it straight away, of course. For SPECX 6.6 the file
    has to be imported (converted into a VERSION 4.1 file) with the
    command {\tt CONVERT-VAX-FILE}. This keeps the existing file (say
    {\tt file.dat}) and creates an new file ({\tt file.sdf}).

\item {\bf Spectral data Version 4.0 and 4.1:}
    These are portable binary formats
    based on HDS. Just take a binary copy to the destination machine
    (UNIX {\tt cp} or {\tt ftp} in binary mode). The file can be used straight
    away. Optionally you can convert it to the native flavour of the HDS
    format on the destination machine. Use the command {\tt native} in the
    Kappa package. This should speed up data access somewhat.

    Version 4.0 or 4.1 spectral data cannot be imported back into SPECX
    6.2 or 6.3.

\item {\bf Map files Version 2 and 3:}
    These are VMS binary formats. Take a binary
    copy to the destination machine. The VMS Versions SPECX 6.2 and 6.3
    can use it straight away, of course. For SPECX 6.6 the file has to
    be imported (converted into a Version-4.1 file) with the command
    {\tt CONVERT-VAX-MAP}. This keeps the existing file (say {\tt
    file.map}) and creates an new file ({\tt file\_map.sdf}).

\item {\bf Map files Version 4.0:}
    These are ``local'' binary formats written only by
    the beta-test version of SPECX 6.4. ``Local'' means they can only be
    read by the same type of machine that wrote them, either a Sun4
    workstation or an Alpha workstation. These files can be converted to
    format Version 4.1 with {\tt CONVERT-VAX-MAP}. In spite of the name of the
    command, the file will be assumed to have been written by the same
    UNIX machine that the command is run on. The existing file (say
    {\tt file.map}) is kept and a new file ({\tt file\_map.sdf}) created.

    Version-4.0 map files cannot be imported back into SPECX 6.2 or 6.3.

\item {\bf Map files Version 4.1:}
    This is a portable binary format based on HDS.
    Just take a binary copy to the destination machine (UNIX {\tt cp} or
    {\tt ftp} in binary mode). The file can be used straight away.
    Optionally you can convert it to the native flavour of the HDS
    format on the destination machine. Use the command {\tt native} in the
    Kappa package. This should speed up data access somewhat.

    Version 4.1 map files cannot be imported back into SPECX 6.2 or 6.3.

\item {\bf Map files Version 4.2:}
    This is also a portable binary format based on HDS. The structures
have been rearranged somewhat from V4.1 to improve storage efficiency
and to try to prevent corruption of the file if specx crashes. Version-4.1
map files are automatically converted to V4.2.

\end{itemize}

Dump files and the {\tt mapplane.tmp} files are not portable. Care has to
be taken when the same file space is used with different machines,
because SPECX may not run properly on one machine while there is a SPECX
dump in the working directory that was written by another machine.


\section{Acknowledgements}

Starlink would like to acknowledge Rachael Padman for making
the SPECX package available for use, assisting in its installation,
and providing the example data.

\appendix

\section{Release Notes}

This section includes the release notes for the more recent versions
of SPECX. This is especially important for the unix versions since
the main Specx user manual does not include any of the new features
that were not part of SPECX V6.3.

\subsection{V6.7-7}

Minor patches:

\begin{itemize}
\item Add wideband support to DAS-MERGE command
\item FITS DATE header is now Y2K compliant
\item Fix problem translating some dates from the GSD header value due to a loss of  precision.
\item Fix problem reading old GSD files that did not know about rastering.
\end{itemize}

\subsection{V6.7-6}

This release is primarily a Y2K bug fix release and also includes an
updated form of SUN/17.

\begin{description}

\item New commands:
\begin{itemize}
\item Add commands to simplify reading of wideband spectra (READ-DUAL-GSD and
READ-WIDE-SPECTRUM)
\end{itemize}
\item Bug fixes:
\begin{itemize}
\item Fix problems with year 2000
\item Allow byte-swapping during FITS reading (was an option that
    did nothing previously)
\item Read and write Y2K form of DATE-OBS FITS keyword
\end{itemize}
\end{description}

\subsection{V6.7-5}

This is a maintenance update to modify the build system to continue to use
Starlink-PGPLOT.  There is no change to functionality.

\subsection{V6.7-4}

At this update the graphics devices not available to GKS/PGPLOT have been
removed from the graphics devices file (various terminal types and gif files).
There is also a fix to prevent an error when using colour PostScript devices.

\subsection{V6.7-3}

This is a minor update to correct the graphics device names to use the
GKS/PGPLOT device names, and to fix a bug in the plotting system that caused
problems for PostScript devices.

\subsection{V6.7-2}

This release includes a fix to the calculation of integration times from
RASTER maps and some bug fixes.

The changes include:

\begin{itemize}
\item Correct calculation of integration time for Raster data.

     This version correctly calculates the equivalent "on+off"
     integration time for on-the-fly spectra (e.g. a raster time of 5
     seconds in a 55 by y raster will show up in SPECX as an
     integration time of about 17 seconds). This change means that you
     now can safely average raster data with regular "on+off" spectra
     or different size raster maps.

     Also, note that if you want to combine new rasters with raster
     data in existing SPECX files, you'll either have to read the old
     GSD raster files again or you'll have to change the integration
     time in the headers of the data in the old SPECX files:

\[
         INT\_TIME = INT\_TIME * \frac{4}{1+\sqrt{55}}
\]

     for a raster with 55 points per row.

\item Now correctly reads CLASS format FITS data via READ-FITS-SPECTRUM

\item A new variable has been added which gives access to the max and min
     values used for greyscale plots. The new symbol is MAPLIMITS(2), e.g.,
\begin{myquote}
\begin{verbatim}
>> print maplimits(1)
\end{verbatim}
\end{myquote}

   would print the minimum value used for the last greyscale.
     This is sometimes useful from scripts.

\item GET-SPECTRUM-FROM-MAP now sets the scan number stored in the header
     to the position of the spectrum in the map. This was to get over
     a problem encountered when writing spectra to data files that had
     scan numbers of 0.

\end{itemize}

Minor technical changes:

\begin{itemize}

\item GSD-PREFIX can now be 32 characters instead of 16.

\item The title used for dual channel data (eg RxB3) is now constructed
      correctly.

\item System calls are now performed with the SH command on Solaris
      (this is 10 times faster than the equivalent SYSTEM command
      since a \texttt{.cshrc} file is not sourced).  This means that different
      system source files must be used on Alpha and Solaris systems.

\end{itemize}

\subsection{V6.7-1}

This is the first full Starlink/JAC release of SPECX for  and includes
significant new functionality added at the JAC.

Changes to include functionality similar to that found in the VMS version:
\begin{itemize}
\item command line editing
\item exception handling (CTRL C)
\end{itemize}

Other changes:
\begin{itemize}
\item NEW-PLOT and OVERLAY now accept a color index (1-15) as second
        argument. If the color index is negative, each subband (quadrant)
        of a spectrum is plotted with a different color. The default
        is now retained in the dump.

\item Auto Y-axis scaling for NEW-PLOT, R-L-B etc now looks only at points
        in current X-axis range. Still no autoscaling within interactive
        graphics however.

\item READ-GSD-DATA now also checks DATADIR for gsd files after current
        directory (setenv DATADIR ... before starting Specx). Has also
        been speeded up significantly. Reads all subscans at once.
        Drawback: won't re-read the disk file (even when altered) until
        another GSD file has been read.
\item DAS-MERGE default for DC adjust changed to 'N'. DAS-MERGE has been
        changed from a script to a FORTRAN coded command.

\item SET-INTERACTIVE now 'N' upon entry to SPECX.

\item REGRID has been rewritten to be more robust. ** It has new questions **
        The original functionality is retained, but it also allows you specify
        the final number of channels and the centre (in current x-units) of
        the regridded spectrum.

\item PLOT-LINE-PARS: the bug that cause a crash when not plotting a linear
        sequence of parameter maps (e.g 1, 2, 3) has now been fixed.

\item INDEX-FILE, COMPRESS-FILE, LIST-MAP allow limits on scan nr and
        offset positions if more than 25 scans.

\item New NDF map format - much faster.
        Version 4.1 maps WILL BE COPIED to the new format.

\end{itemize}

New commands:
\begin{itemize}
\item DAS-MERGE is now a FORTRAN subroutine i.s.o. a Specx Macro.

\item WRITE-FITS-CUBE (under test).

\item INFO-FILE  (update NDF spectral file header information: for
         use with SPECX macros. Ensures SPECX variables refer to
         indicated file.)

\item READ-GSD-RASTER (under test: directly reads all subscans
         from GSD file into a SPECX file with an optional DAS-MERGE
         along the way. Much faster than SPECX do-loop). Use the
         new MERGE-FILES command to average overlapping spectra
         read with this command.

\end{itemize}

\subsection{V6.6}

Added support for \xref{GSD}{sun229}{} files.

\subsection{V6.5}

First released version. No \xref{GSD}{sun229}{} support.

\subsection{V6.4}

First unix beta release. A port of the VMS SPECX version 6.3.

\subsection{V6.1 -- VMS}

This was a VAX only release.

\begin{itemize}

\item  expression\_evaluation 23-12-90  (see NEWS...)

\item   new\_functions 29-12-90 (see NEWS...)

\item   fixed bug in parser : can now do >> <symbol> arg1....

\item   added N\_FSSP, FSS\_INTS, RLB\_INTS, changed name to BLF\_INTS

\item   fixed bug in default intervals for F-S-S, R-P-B etc
\item   fixed bug in X-V plots (sometimes came out negative)

\item   fixed bug in GEN\_SPRINT, can now format strings

\item   discovered bug in map\_cells .ne. integers

\item   fixed bug in R-G-MAP error return

\item  size of header characters on CONTOUR - not affected by s-p-par?
\item  size of header characters in P-L-P   - reset to expand=1.0 after first plot

\item   fixed bug in OPEN-MAP; left flag for map open set when crashed on entering
  from dump (led to subsequent crash on add-to-map).

\item   fixed GRID-SPECTRUM for grids of one row or column only

\item   added frequency polynomial correction for AOSC etc, modified MKHIS etc
  to plot such scans correctly. Modified XTRANS and XSNART so that MOST
  data reduction routines now work correctly even with polynomial data.

\item   fixed a but in GETPTS in that it didn't prompt with first set of defaults.

\item   added parameters NC etc for CENTRD, STATS to list of built-in parameters

\item   fixed bug in BINDAT to trap too few binning points.

\end{itemize}

\section{SPECX articles}

This section contains articles written by Rachael Padman for the JCMT
Newsletter and are included here since they may be difficult to obtain
from other sources.

\subsection{Frequency and velocity scales in SPECX -- June 1992}

\subsubsection{Introduction}

Traditionally, radio spectral-line analysis programs have displayed velocities
with respect to the Local Standard of Rest (LSR), using the radio definition
of velocity. It is now possible to observe at JCMT using Heliocentric,
Barycentric and Telluric velocity frames, and optical and relativistic
velocity definitions, and SPECX V6.2 (now released to Starlink) contains the
appropriate facilities for dealing with such data.

Before describing what SPECX does, it is necessary to review, briefly, what is
required (see also Gordon 1976). We all know that any relative velocity
between source and observer gives rise to a corresponding Doppler shift in the
received frequency. For \emph{small} velocities, then, radioastronomers (who
think in frequencies) write

\begin{equation}
\frac{\nu}{\nu_0} = \frac{c-v}{c} = 1-\frac{v}{c}
\label{eqn:radio}
\end{equation}

where $v$ is the velocity \emph{away} from the observer. Likewise, optical
astronomers write a similar formula in wavelengths:

\begin{equation}
\frac{\lambda_0}{\lambda} = \frac{c}{c+v} = 1-\frac{v}{c+v}
\label{eqn:optical}
\end{equation}


Finally, it seems that extragalactic observers believe in special relativity,
which of course tells us that:

\begin{equation}
\frac{\nu}{\nu_0} = \sqrt{\frac{c-v}{c+v}}
\label{eqn:special}
\end{equation}

(ignoring the smaller doppler shift due to any transverse component of
velocity). By a simple application of the binomial theorem, we can see that
these are all equivalent for $v << c$. A problem arises because these
equations are used in reverse to \emph{define} a velocity, and this definition
is often used where the approximations in (\ref{eqn:radio}) or
(\ref{eqn:optical}) are inadmissible. Even the (special) relativistic formula
applies only in \emph{local} frames, which are not the same thing at all as
distant cosmological frames. So none of these formula are ``right''.

Normally we reduce our velocities to one of a few commonly agreed standards:
Radio astronomers prefer the Local Standard of Rest (LSR); extragalactic
astronomers apparently prefer a Heliocentric system, whilst for some purposes,
such as instrument or atmospheric diagnostics it may be the Telluric frame that
is of most interest. Or indeed we may have some favourite other frame which is
defined with respect to one of these "standard" frames. One that is used quite
often is the source frame -- for example, if we want the frequency scale to be
that appropriate to a non-moving source we have to transform all telluric
(observed) frequencies to a frame comoving with the source. In the case of
Orion for example this is 9 km/s wrt the LSR, and few of us know the
heliocentric velocity.

Thus our velocity scale is specified by four parameters:
\begin{enumerate}
\item      The velocity transformation law.
\item      The rest frequency or wavelength used in the law.
\item      The standard frame, and
\item      The reference velocity expressed in that frame.
\end{enumerate}
Here I use the general term "frame" to imply this full specification.

\subsubsection{Observation and display of line spectra}

There are two separate issues. First, we need to \emph{observe} such that the
line of interest actually occurs within the passband of the instrument
(normally, but not always, at the centre of the passband). For a heterodyne
receiver we do this by adjusting the local oscillator frequency by an amount
equal to the \emph{total} doppler frequency shift expected for the
line. Second, we then need to \emph{display} the spectrum. Unfortunately, the
spectrometer is back on Earth, at the telescope, so we need to map the
I.F. frequencies of individual spectrometer channels onto velocity (or
frequency) space in some other chosen frame. This `display' frame need not
necessarily be the same as the `observation' frame.  For example, whatever the
observation frame, we may wish to display the output in one of several
standard frames:

\begin{enumerate}
\item    In the telluric frame, where the frequencies of spurious responses in
       the I.F. passband (for example) can be measured, and with any luck,
       disposed of. Or we may wish to measure absolute frequency in this frame
       to identify telluric (atmospheric) emission and/or absorption features.
\item  In the frame of the source (e.g., at a velocity of $V_{lsr}$ offset
       with respect to the LSR frame itself), when we wish to measure the
       absolute frequency of some spectral feature, in case it is a spectral
       line of some species other than the one we desired.
\item  In the LSR frame, if we want to determine a velocity to use for distance
       measurements within the galaxy.
\item  In a heliocentric frame for velocity determinations of external galaxies
       (I am told that this is the normal way of doing this.)
\end{enumerate}

The solution adopted in SPECX is to view all calculations of velocity and/or
frequency as a two stage process. In the first, the spectral header information
is used to calculate the telluric centre frequency of the observation. That is,
we deduce the true frequency as measured at the telescope of a signal appearing
in the centre channel of the spectrometer. SPECX then produces an `X-array'
which contains, for each spectrometer channel, the telluric frequency
corresponding to that channel. Finally, this array is transformed back to the
display frame. By default the display frame is the same as the observation
frame, which is encoded in the SPECX and GSD scan headers, but it may be
changed to any other frame if you like.

To summarize: SPECX will normally display the data using the velocity frame
in which it was observed; however you can use SET-VELOCITY-FRAME to select
another frame, and optionally you can use SET-LINE-REST-FREQ to choose another
reference frequency for the velocity transformation. Current options for
frames include LSR, Geocentric, Heliocentric and Telluric, and you have a
choice of Radio, Optical and Relativistic velocity laws. Because of the
bewildering number of combinations of these variables, the header on the
X-axis of the plot has been modified to give a full specification.

\subsubsection{COMPLICATIONS. 1: Displaying the other sideband.}

The command CHANGE-SIDEBAND does the following: First, it calculates the
telluric centre frequency. Then it adds or subtracts twice the I.F. (depending
on whether the observations are in LSB or USB respectively). Finally, it
transforms this telluric frequency to the nominal frame of the observation,
files it as \texttt{F\_CEN} in the scan header, and changes the sense of the
frequency step in the spectrometer (\texttt{F\_INC}). GSD spectra stored with
storage task V6 or earlier do not have the local oscillator frequency stored,
so SPECX instead asks you for the current sideband and I.F.; for V7 data or
later SPECX can deduce these from the data in the GSD scan header.

Let us take a specific example. We have a RxB3i spectrum which is centred on CS
7-6 (rest freq. 342.883 GHz) in the lower sideband.  The nominal I.F. is 1.5
GHz. In order to look at absolute "rest frame" frequencies, we set the display
frame to be that of the source itself:

\begin{myquote}
\begin{verbatim}
>> set-velocity-frame
Output in different vel frame? (Y/N) [Y]
Velocity frame? (TELLuric, LSR, HELIocentric, GEOcentric) [TELL] LSR
Velocity law definition? (OPTical, RADio, RELativistic) [RAD] RAD
Velocity in new frame? (km/s) [  20.0] vlsr
\end{verbatim}
\end{myquote}
(see Fig 1). (There is a predefined variable "vlsr" equated to the actual value
for the current spectrum, so we can quote this in response to the last query
rather than looking up the actual value.)

For an AOSC spectrum, we would now regrid to a linear scale (see next section
for more details on this). Then we go ahead and change the sideband:
\begin{myquote}
\begin{verbatim}
>> change-sideband
Local oscillator frequency not defined...
Current sideband? (U/L) [L] L
First i.f.? (GHz) [ 1.500000] 1.5

 --- Header entries changed to other sideband ---
   Don't forget to SET-LINE-REST-FREQ to set the
   frequency of the line you wanted to look at -
   the old line will appear at large velocities!
\end{verbatim}
\end{myquote}

As suggested, change the reference frequency to CO 3-2:
\begin{myquote}
\begin{verbatim}
>> set-line-rest-freq
Receiver # 1  Line rest frequency? (GHz) [  0.000000] 345.795989
\end{verbatim}
\end{myquote}

\sloppypar{If we have done all this correctly we now get the absolute frequency plot
shown below (Fig 2). A macro has been provided to do all this -- just type
IMAGE-FREQ (this calls \texttt{specx\_command:image.spx}).}

\subsubsection{COMPLICATIONS. 2: AOSC}

It is well known that the frequency scales of Acousto-Optical spectrometers
tend to be non-linear. To this end, the SPECX frequency calculation stuff
contains a facility for correcting the frequency using a polynomial fit to the
frequency error. For AOSC, the cubic term dominates, and the maximum error is
about 0.6 MHz, but for somewhat complicated reasons, there is an
\emph{additional} zero-order (d.c.) term of some 11 or 12 channels, or about 3
MHz. The implication of this is that a line you might expect to come out in
the centre channel (1024.5) of the spectrum will actually emerge some 3 MHz
away, and to compensate for this the GSD header files correct the
reference-frame observed frequency, \texttt{F\_CEN}, appropriately. This makes
the display work fine, but has implications for when we want to look at image
frequencies (as will be shown in a moment).

To correct the frequency scale for AOSC data in particular, I have written a
small macro, FRQFIX.SPX, which is kept in the "standard" command-files
directory. It is invoked by the symbol LINEARIZE-AOSC-FREQ:
\begin{myquote}
\begin{verbatim}
>> linearize-aosc-freq
FRQFIX> Linearization turned off - reset it? (Y/N) [Y] Y
OK, setting freq coefficients
FRQFIX> Regrid to uniform sampling? (Y/N) [Y] Y
 -- AOS frequency scale linearization applied --
First and last useful channels in input:           1        2038
Linearization has now been turned off!
Reset if another spectrum needs correcting
\end{verbatim}
\end{myquote}

This macro in fact \emph{also} removes the 3-MHz offset from \texttt{F\_CEN},
instead adding it into the zero-order term of the correction polynomial, which
simplifies calculation of the I.F. LINEARIZE also does various checks and
tidies up the program flags. If you choose to REGRID the data at this point
you transform the data to a truly linear scale (in current units). Otherwise
the data will still display correctly, as long as you have the linearization
turned on (set \texttt{fcorrect=true} or use SET-X).

If you now want to look at the other sideband of AOSC data, remember that the
effective I.F. at the centre of the passband is not what you might think it
is, but is actually 1.503 GHz (or 3.943 GHz for RxA1 or RxB2). If on the other
hand you apply the LINEARIZE-AOSC macro the data will be regridded onto a
linear scale \emph{and shifted to the nominal channel}. In this case the
I.F. is the standard 1.5 or 3.94 GHz. There is one Awful Warning: you cannot
apply linearization to data \emph{after} you have done a CHANGE-SIDE. That
flips the spectrometer frequency scale, and so will apply the correction 'the
wrong way round'. So after you have done a CHANGE-SIDE, make sure that
linearization is turned off, either through SET-X or by typing
"\texttt{fcorrect=false}".

I thank Per Friberg, Goeran Sandell and Chris Mayer for their help in
untangling this mess, Louis Noreau for pointing out the need for other
velocity frames and scaling laws, and Paul Feldman for bringing other
infelicities in the velocity and frequency scaling to my attention. The system
in SPECX V6.2 has been written from scratch, and so does not have 10 years of
testing behind it; please bring any demonstrable faults to my attention. (I do
not have Donald Knuth's confidence, and do not offer a steadily increasing
reward for each bug found in my code.)

\subsubsection{Efficient map-making}

Finally, a brief note about map-making. People are making some really quite
big maps. SPECX tries to hold the entire cube in virtual memory at one time,
and in fact with various permutations of INTERPOLATE and ROTATE, may want to
have up to 3 copies of the cube resident. If excessive paging is to be
avoided, then it is desirable that all these fit into the available physical
memory of the machine (the working set). The usual consequence if they do not
is that the whole machine grinds to a halt, without any apparent error...

You can minimize the amount of virtual memory required by only mapping those
spectral channels of interest. Use TRUNCATE (or DROP-CHANNELS) to dispense
with spectral channels lying far away from the line; use BIN-SPECTRUM where
you have higher resolution than you need. Even if you have plenty of memory to
spare, you can greatly speed up the map-making process by using as few
channels as possible. To reinforce this point, when you do an OPEN-MAP you are
now asked explicitly for the number of spectral channels in the cube. You can
also eliminate one cube from memory altogether by using
`interpolation-on-demand' (an option in INTERPOLATE-MAP), albeit at the cost
of slightly increased time to make any particular map.

As noted in the manual, SPECX maps are really channel maps, and no information
is stored in the header about individual spectra. Thus if you want to map data
taken in the other sideband from that of the map header, or taken with a
different spectrometer etc, you first need to INVERT, SHIFT, TRUNCATE etc to
get your spectrum into the right form. There is a new macro that does this
automatically: it can be invoked by the command symbol CONVERT-TO-MAP-FORMAT
this macro is stored along with all the other standard macros in the directory
with logical name "specx\_command".) Just use this command before you
ADD-TO-MAP to convert your current spectrum to the same frequency scaling etc
as the map header.

\subsubsection{References}

M.A. Gordon. Chap 6.1 in "Methods of Experimental Physics", vol 12-C,
    (Astrophysics -- Radio Observations). Academic Press, NY, 1976


\subsection{More on SPECX velocity scales -- January 1993}

Following publication of the last issue of the Newsletter, Pat Wallace (author
of the \xref{SLALIB}{sun67}{} package) wrote to point out that my use of the
phrase ``the LSR'' may have been confusing. The ``local standard of rest'' is
strictly a defined frame, but is meant approximately to represent the basic
solar motion with respect to the neighbouring stars. This motion is
established by measuring the radial velocities of the stars in the solar
neighbourhood. The number you get depends on the depth of the sample (how far
out you go), and the spectral classes of the stars you use. Thus there are
several determinations of the \emph{kinematic} solar motion, which differ by a
few kilometres per second, and by a few degrees in direction. Spectral line
radio astronomers however traditionally use the defined ``LSR'', which has a
velocity of 20km/s in the direction of 18h,+30d(B1900), and differs slightly
from the most modern determinations of the solar motion.

Pat has recently altered the SLALIB routine \texttt{SLA\_RVLSR} to provide a
better estimate of the {\em dynamical} solar motion --- i.e., the motion of
the Sun with respect to the appropriate circular orbit around the galactic
centre.  This also makes \texttt{SLA\_RVLSR} consistent with the routine
\texttt{SLA\_RVGALC}, but unfortunately there is now a velocity difference of
up to $\pm 3$km/s between the velocities calculated by \texttt{SLA\_RVLSR} and
those used by radio astronomers (and SPECX). I propose {\em not} to implement
this new definition in SPECX, as I suspect it will only cause confusion if
Orion starts to come out with "an lsr velocity" of 6km/s. Pat Wallace's
program \xref{RV}{sun78}{}, which is included in the JCMT utilities, {\em is}
based on the SLALIB routines, but the older version is being retained to
prevent undue confusion arising as a result of the change in the SLALIB
routines.

One further note: SPECX and the JCMT control system have both in fact been
using a velocity of 20km/s towards an apex of 18h,+30d(B1950), as used in the
original Bonn software. For consistency I will change the epoch used in SPECX
to B1900 in the next release, but the difference in velocities will be very
small (<<0.1km/s), so shouldn't be observable for AOSC data. The apex used in
the telescope control software will also be changed to B1900 as soon as
practicable.

My thanks to Pat Wallace, Chris Mayer and Per Friberg for helping me to sort
out what was actually going on here.

\subsubsection{References}

\begin{enumerate}
\item J.D.Kraus, 1966. "Radio Astronomy" p47 McGraw-Hill, NY (first edition).
\item D.A.MacRae and G.Westerhout: "Table for the reduction of velocities to the
    Local Standard of Rest", The Observatory, Lund, Sweden 1956. (`The Lund
    Tables'). Uses 20km/s toward 18h,+30d(B1900)
\item M.A.Gordon, 1976, in "Methods of Experimental Physics", vol 12-C, Chap 6.1
    (Astrophysics -- Radio Observations), Academic Press, NY.
\end{enumerate}

\subsubsection{Bugs}

One serious bug has surfaced in SPECX6.2, which means that some maps are
``unopenable''. The program starts to open them, and then falls over. A kluge
which fixes this in some cases is to open another map first, then open the
offending map afterwards. However the only real fix is to replace the
incorrect routine. Any site which has problems can obtain a copy of the
routine direct from me, or from John Lightfoot at ROE (REVAD::JFL).

\subsubsection{JCMT Updates}

A slightly revised version of SPECX (V6.2-A) is running at JCMT. This fixes
the bug described above, and one or two other very minor problems. Changes
include now being able to open up to 8 SPECX data files (was 3), and being
able to change the parameters controlling the ``MRAO colour spiral'' (colour 5
for colour-greyscale mapping). There is also a tentative implementation of
logarithmic greyscales -- just hit the `0' key to toggle from log to linear
and back.

I have added a command definition UTILS, which prints a file that describes
the ``system macros'' -- \texttt{.spx} files written to accomplish certain
oft-needed functions, including those such as `map-average' and `fetch' that
have been described in previous versions of the newsletter.


\subsection{SPECX V6.3 for VMS and a  release -- January 1994}

The good news (for those for whom that sort of news is good) is that, thanks
largely to the efforts of Horst Meyerdierks at The University of Edinburgh, and
following an intensive burst of work over Christmas on my part, SPECX should
shortly be available on Starlink  platforms. Large parts of the port are
complete but it is not quite clear when there will be enough of a system to be
worth releasing it to Starlink. At the date of writing (4th January) neither
GSD nor FITS i/o has been implemented, so there is still no way to import data
other than by direct conversion of VAX files in SPECX internal format. Horst
has written a standard SPECX command to do this - for the present anyway it
produces .SDF files, which use Starlink NDF and HDS libraries. Remo Tilanus has
been working on a GSD reader, and converting the FITS stuff should not be too
difficult (although tape i/o will probably not be supported under ), so
with any luck it will not be too difficult to finish off V6.4.

Unfortunately a number of the nicest features of the VAX version - such as
CTRL C
handling and proper support for dual-screen alpha graphics terminals - are
unlikely to be available in the first  version (6.4), and users may notice
some "debug" type statements which are unfortunately necessary to circumvent
bugs in the f77 compiler, but for most purposes SPECX V6.4 should satisfy
the need. With any luck it will be released to Starlink during February,
and if not, then soon after. Whether the documentation will be
available by then is another matter.

Horst is aware of the need to be able to export SPECX to non-Starlink
sites without them needing to take on the whole of the Starlink environment,
and I am assured that this will be possible. However in the first instance
it is probable that  software will be available only through ROE, or
later through the Starlink Librarian --- in any case, probably
not from me directly.

Meanwhile, for those who do have access to a VAX still, and while
you are at the telescope, the improvements in V6.3 should make life
easier. In fact the release of this version has been prompted mainly
by the successful commissioning of the DAS, which in turn has required
major changes in the internals of SPECX to accommodate extra header
information required to define the frequency scales accurately.
Several new commands have been added for dealing with DAS data, while
others have been modified slightly.

Most importantly, the SPECX internal datafile and map format has changed.
As always, SPECX 6.3 is able to read any existing files and maps you may have,
but it is no longer able to write to them. Map files (with the .MAP
extension) are converted to V3 the first time they are opened by the
new program --- if you intend also to use earlier versions of SPECX then
make a copy of the map before you use V6.3 for the first time. This is
not true of ordinary datafiles; you just can't write spectra to them
any more, but it is a simple matter to open a new file and then copy the
spectra across one by one or in a DO loop.


The reason these file formats have had to change is to accommodate the
LO and IF frequencies in the header, along with the velocity of the
source in various frames. These were not previously available in the
GSD file, but were derived within SPECX from the specified source-frame
centre frequency of each sub-band. This is not however a natural way to
do things, and proved prone to error with increasingly complex data
such as those produced by the DAS. Hence the change. Since things \emph{were}
changing I have taken the opportunity to tidy up some other relics of
bygone computer systems --- RA and DEC are now stored internally as R*8
degrees, and the map offsets are stored as reals. This has meant
changes to some of the macros supplied with SPECX, and if you use these
variables in your own macros then you will need to modify them
accordingly.

\subsubsection{New and modified commands}

\begin{itemize}
\item CLIP-SPECTRUM       --- Sets all values in a spectrum outside of the specified
                        range to a specified value.

\item CONCATENATE-SPECTRA --- Combines all sub-bands of the spectra in X and Y stack
                        positions into a new spectrum (the individual subbands
                        are retained).

\item WRITE               --- A version of PRINT modified to write into an SCL
                        character variable which can then be used in response
                        to a request of character input (a filename say)

\item READ-FITS-SPECTRUM  --- Reads a SPECX or CLASS FITS spectrum from a disk-FITS
                        file into the X stack position.

\item OPEN-FITS-FILE      --- Replaces OPEN-FITS-OUTPUT-FILE, and has new questions.

\item CLOSE-FITS-FILE     --- Replaces CLOSE-FITS-OUTPUT-FILE

\item SET-PLOT-SCALE      --- Now lets you specify auto-scaling for the X and Y
                        plot axes separately.

\item SET-MAP-PARAMETERS  --- Now lets you specify that the map axes are to be
                        scaled in sexagesimal form - i.e., RA and Dec in HMS.

\item SHOW-VARIABLES      --- Accepts VMS-type wildcards, so that, for example, the
                        command ``\verb+>> show-var f*+'' will return all SCL variables
                        that begin with the letter f.

\item PLOT-LINE-PARAMETERS--- As well as the existing set of parameters (Tmax, Vmax,
                        Integrated intensity and Delta(v)), now also offers you
                        the 1st and 2nd moments of the line profile. The 1st
                        moment is the Centroid; the 2nd moment approximates the
                        line width for lines of good Signal/Noise ratio.
\end{itemize}

\subsubsection{Bad channel/Magic value handling}

Bad-channel handling has been available for some time in the various
mapping functions, although it tends to go unseen (except when you
forget to do an INTERPOLATE for sparsely sampled data). A version of
this has now been introduced for single-spectrum data. Channels in
spectra set to this value are not displayed on plots, or used in
computing quantities such as maximum intensity or line width. Channels
which end up being unspecified, as the result of a SHIFT operation say,
are now set to the bad channel value, rather than zero as before. Bad
channels can be interpolated over with smoothing commands such as HANN,
SMOOTH, CONVOLVE, BIN etc.

You can determine the current setting of the bad channel value by doing:
\begin{myquote}
\begin{verbatim}
>> print badpix_value
\end{verbatim}
\end{myquote}
and if you don't like it, you can change it by doing:
\begin{myquote}
\begin{verbatim}
>> badpix_value = -100
\end{verbatim}
\end{myquote}
for example. The one place where bad channels in spectra are not handled
well is in spectra in a map cube  --- fewest problems will result if you
smooth over such bad channels before ADDing to the map, or choose a value
of \texttt{badpix\_value} close to zero (I tend to use 1.e-5 by default). Although
we did have some teething problems with this feature, it seems to be working
pretty well now --- It is particularly useful for dealing with DAS data, where
the end channels of each subband tend to be set to $10^4$ or some such similar
value, and totally destroy the auto-scaling of plots.

Note that if you change the value of \texttt{badpix\_value} as shown above, then
previously ``hidden'' channels in your data will now be displayed, and/or
used in reduction operations.


\subsubsection{Map display}

A few mostly cosmetic changes have been introduced here. Displays involving
greyscales will mostly now have a scale-bar displayed wherever there seems
to be most room. This is not foolproof however, and for some unfortunately
sized maps the scale-bar may vanish off the edge of the plot area -- just
do SET-MAP-SIZE if this is the case. Note that the scale bar will \emph{not}
appear in the case of PLOT-LINE-PARAMETERS when more than one map is made ---
the various maps in this case have different scales, and it did not seem
sensible to try to generate scale-bars for each independent panel.

Also, you can now display RA and Dec coordinates (only) in all forms of
map display (GRID, CHANNEL-MAP, GREY, CONTOUR and PLOT-LINE-PAR) in
standard sexagesimal notation. That is, absolute positions are displayed,
in HMS and DMS respectively. You can however still display things in the
old way if you want -- the map centre RA and DEC are now indicated in the
axis labels. Use SET-MAP-PARAMETERS to change the default axis scaling.


\subsubsection{FITS i/o}

There have been ongoing problems with exchanging single spectra in FITS
format between SPECX and CLASS. We (Remo Tilanus and I) believe that
this has now been cured, following minor changes to both programs (with
the cooperation of the CLASS authors at IRAM). Certainly SPECX now
seems to produce valid CLASS spectra. I have also resurrected John
Richer's FITS reader, and with a great deal of hacking turned it into a
standard command that reads disk-FITS spectra (SIMPLE=.TRUE. only) into
the X position of the stack. This command is READ-FITS-SPECTRUM   ---
open the FITS file with OPEN-FITS-FILE and close it afterwards with
CLOSE-FITS-FILE as before (note that the names of these commands have
been changed from OPEN-FITS-OUTPUT-FILE and CLOSE-FITS-OUTPUT-FILE
respectively).

FITS maps are more of a problem. Claire Chandler has recently modified
WRITE-FITS-MAP to make it compatible with AIPS (it always used to be, so I
assume that AIPS has changed too...), but this is probably \emph{not}
compatible with CLASS. We may eventually need separate commands to write FITS
maps for different targets.


\subsubsection{Frequency and Velocity scaling of Plots}

It is now possible to produce plots which have the image sideband
frequency along the top axis. To get this, all you have to do is to
choose the absolute frequency options in SET-X-SCALE. Note however
that it is unlikely to be correct unless the display frame is the one
in which the source is at rest. Since you *may* have observed with
(say) the LSR velocity not equal to that of the source itself, I can't
protect you from this. But in most cases it will be sufficient to do
\begin{myquote}
\begin{verbatim}
>> SET-VELOCITY-FRAME yes lsr radio vlsr
\end{verbatim}
\end{myquote}
That is, choose a velocity reference frame which is defined as having
velocity offset with respect to the standard of rest the same as that
of the source itself (modify as appropriate for heliocentric,
geocentric or telluric frames, and for different velocities or velocity
definitions).

HOWEVER... You must realize that most plotting packages, PGPLOT
amongst them, accept REAL*4 numbers only. So if you display things in
absolute frequency, it is not possible to get velocities accurate to
better than a couple of MHz in three hundred odd GHz, no matter how
accurately they are calculated internally in SPECX. For more accurate
displays, choose a convenient reference frequency in the absolute
frequency option in SET-X. For  example, if you have a set of lines
in the range 338 -- 339 GHz, then choose a reference frequency of
338 GHz, and remember to add this back on to any frequency you determine
with FIT-GAUSSIAN, or using the cursor.

My thanks to Remo Tilanus in particular for his help,
provision of DAS data, and rapid turn around of queries while we
were solving the problems associated with the DAS. My apologies to
anyone who may have been inconvenienced by problems in the Beta-test
version --- I hope you appreciate that it was the only way to make
DAS data available at all.


\end{document}
