\documentclass[twoside,11pt]{article}

% ? Specify used packages
% \usepackage{graphicx}        %  Use this one for final production.
% \usepackage[draft]{graphicx} %  Use this one for drafting.
% ? End of specify used packages

\pagestyle{myheadings}

% -----------------------------------------------------------------------------
% ? Document identification
% Fixed part
\newcommand{\stardoccategory}  {Starlink User Note}
\newcommand{\stardocinitials}  {SUN}
\newcommand{\stardocsource}    {sun\stardocnumber}
\newcommand{\stardoccopyright} 
{Copyright \copyright\ 2000 Council for the Central Laboratory of the Research Councils}

% Variable part - replace [xxx] as appropriate.
\newcommand{\stardocnumber}    {110.2}
\newcommand{\stardocauthors}   {R.F. Warren-Smith}
\newcommand{\stardocdate}      {5th October 1990}
\newcommand{\stardoctitle}     {SST - A Simple Software Tools Package}
\newcommand{\stardocversion}   {Version 1.0}
\newcommand{\stardocmanual}    {User's Manual}
\newcommand{\stardocabstract}  {
The Simple Software Tools package (SST) is a package of applications
designed to help with the production of software and documentation, with
particular emphasis on ADAM programming using Fortran~77.
}
% ? End of document identification
% -----------------------------------------------------------------------------

% +
%  Name:
%     sun.tex
%
%  Purpose:
%     Template for Starlink User Note (SUN) documents.
%     Refer to SUN/199
%
%  Authors:
%     AJC: A.J.Chipperfield (Starlink, RAL)
%     BLY: M.J.Bly (Starlink, RAL)
%     PWD: Peter W. Draper (Starlink, Durham University)
%
%  History:
%     17-JAN-1996 (AJC):
%        Original with hypertext macros, based on MDL plain originals.
%     16-JUN-1997 (BLY):
%        Adapted for LaTeX2e.
%        Added picture commands.
%     13-AUG-1998 (PWD):
%        Converted for use with LaTeX2HTML version 98.2 and
%        Star2HTML version 1.3.
%      1-FEB-2000 (AJC):
%        Add Copyright statement in LaTeX
%     {Add further history here}
%
% -

\newcommand{\stardocname}{\stardocinitials /\stardocnumber}
\markboth{\stardocname}{\stardocname}
\setlength{\textwidth}{160mm}
\setlength{\textheight}{230mm}
\setlength{\topmargin}{-2mm}
\setlength{\oddsidemargin}{0mm}
\setlength{\evensidemargin}{0mm}
\setlength{\parindent}{0mm}
\setlength{\parskip}{\medskipamount}
\setlength{\unitlength}{1mm}

% -----------------------------------------------------------------------------
%  Hypertext definitions.
%  ======================
%  These are used by the LaTeX2HTML translator in conjunction with star2html.

%  Comment.sty: version 2.0, 19 June 1992
%  Selectively in/exclude pieces of text.
%
%  Author
%    Victor Eijkhout                                      <eijkhout@cs.utk.edu>
%    Department of Computer Science
%    University Tennessee at Knoxville
%    104 Ayres Hall
%    Knoxville, TN 37996
%    USA

%  Do not remove the %begin{latexonly} and %end{latexonly} lines (used by 
%  LaTeX2HTML to signify text it shouldn't process).
%begin{latexonly}
\makeatletter
\def\makeinnocent#1{\catcode`#1=12 }
\def\csarg#1#2{\expandafter#1\csname#2\endcsname}

\def\ThrowAwayComment#1{\begingroup
    \def\CurrentComment{#1}%
    \let\do\makeinnocent \dospecials
    \makeinnocent\^^L% and whatever other special cases
    \endlinechar`\^^M \catcode`\^^M=12 \xComment}
{\catcode`\^^M=12 \endlinechar=-1 %
 \gdef\xComment#1^^M{\def\test{#1}
      \csarg\ifx{PlainEnd\CurrentComment Test}\test
          \let\html@next\endgroup
      \else \csarg\ifx{LaLaEnd\CurrentComment Test}\test
            \edef\html@next{\endgroup\noexpand\end{\CurrentComment}}
      \else \let\html@next\xComment
      \fi \fi \html@next}
}
\makeatother

\def\includecomment
 #1{\expandafter\def\csname#1\endcsname{}%
    \expandafter\def\csname end#1\endcsname{}}
\def\excludecomment
 #1{\expandafter\def\csname#1\endcsname{\ThrowAwayComment{#1}}%
    {\escapechar=-1\relax
     \csarg\xdef{PlainEnd#1Test}{\string\\end#1}%
     \csarg\xdef{LaLaEnd#1Test}{\string\\end\string\{#1\string\}}%
    }}

%  Define environments that ignore their contents.
\excludecomment{comment}
\excludecomment{rawhtml}
\excludecomment{htmlonly}

%  Hypertext commands etc. This is a condensed version of the html.sty
%  file supplied with LaTeX2HTML by: Nikos Drakos <nikos@cbl.leeds.ac.uk> &
%  Jelle van Zeijl <jvzeijl@isou17.estec.esa.nl>. The LaTeX2HTML documentation
%  should be consulted about all commands (and the environments defined above)
%  except \xref and \xlabel which are Starlink specific.

\newcommand{\htmladdnormallinkfoot}[2]{#1\footnote{#2}}
\newcommand{\htmladdnormallink}[2]{#1}
\newcommand{\htmladdimg}[1]{}
\newcommand{\hyperref}[4]{#2\ref{#4}#3}
\newcommand{\htmlref}[2]{#1}
\newcommand{\htmlimage}[1]{}
\newcommand{\htmladdtonavigation}[1]{}

\newenvironment{latexonly}{}{}
\newcommand{\latex}[1]{#1}
\newcommand{\html}[1]{}
\newcommand{\latexhtml}[2]{#1}
\newcommand{\HTMLcode}[2][]{}

%  Starlink cross-references and labels.
\newcommand{\xref}[3]{#1}
\newcommand{\xlabel}[1]{}

%  LaTeX2HTML symbol.
\newcommand{\latextohtml}{\LaTeX2\texttt{HTML}}

%  Define command to re-centre underscore for Latex and leave as normal
%  for HTML (severe problems with \_ in tabbing environments and \_\_
%  generally otherwise).
\renewcommand{\_}{\texttt{\symbol{95}}}

% -----------------------------------------------------------------------------
%  Debugging.
%  =========
%  Remove % on the following to debug links in the HTML version using Latex.

% \newcommand{\hotlink}[2]{\fbox{\begin{tabular}[t]{@{}c@{}}#1\\\hline{\footnotesize #2}\end{tabular}}}
% \renewcommand{\htmladdnormallinkfoot}[2]{\hotlink{#1}{#2}}
% \renewcommand{\htmladdnormallink}[2]{\hotlink{#1}{#2}}
% \renewcommand{\hyperref}[4]{\hotlink{#1}{\S\ref{#4}}}
% \renewcommand{\htmlref}[2]{\hotlink{#1}{\S\ref{#2}}}
% \renewcommand{\xref}[3]{\hotlink{#1}{#2 -- #3}}
%end{latexonly}
% -----------------------------------------------------------------------------
% ? Document specific \newcommand or \newenvironment commands.

%+
%  Name:
%     SST.TEX

%  Purpose:
%     Define LaTeX commands for laying out Starlink routine descriptions.

%  Language:
%     LaTeX

%  Type of Module:
%     LaTeX data file.

%  Description:
%     This file defines LaTeX commands which allow routine documentation
%     produced by the SST application PROLAT to be processed by LaTeX and
%     by LaTeX2html. The contents of this file should be included in the
%     source prior to any statements that make of the sst commnds.

%  Notes:
%     The style file html.sty provided with LaTeX2html needs to be used.
%     This must be before this file.

%  Authors:
%     RFWS: R.F. Warren-Smith (STARLINK)
%     PDRAPER: P.W. Draper (Starlink - Durham University)

%  History:
%     10-SEP-1990 (RFWS):
%        Original version.
%     10-SEP-1990 (RFWS):
%        Added the implementation status section.
%     12-SEP-1990 (RFWS):
%        Added support for the usage section and adjusted various spacings.
%     8-DEC-1994 (PDRAPER):
%        Added support for simplified formatting using LaTeX2html.
%     {enter_further_changes_here}

%  Bugs:
%     {note_any_bugs_here}

%-

%  Define length variables.
\newlength{\sstbannerlength}
\newlength{\sstcaptionlength}
\newlength{\sstexampleslength}
\newlength{\sstexampleswidth}

%  Define a \tt font of the required size.
\latex{\newfont{\ssttt}{cmtt10 scaled 1095}}
\html{\newcommand{\ssttt}{\tt}}

%  Define a command to produce a routine header, including its name,
%  a purpose description and the rest of the routine's documentation.
\newcommand{\sstroutine}[3]{
   \goodbreak
   \rule{\textwidth}{0.5mm}
   \vspace{-7ex}
   \newline
   \settowidth{\sstbannerlength}{{\Large {\bf #1}}}
   \setlength{\sstcaptionlength}{\textwidth}
   \setlength{\sstexampleslength}{\textwidth}
   \addtolength{\sstbannerlength}{0.5em}
   \addtolength{\sstcaptionlength}{-2.0\sstbannerlength}
   \addtolength{\sstcaptionlength}{-5.0pt}
   \settowidth{\sstexampleswidth}{{\bf Examples:}}
   \addtolength{\sstexampleslength}{-\sstexampleswidth}
   \parbox[t]{\sstbannerlength}{\flushleft{\Large {\bf #1}}}
   \parbox[t]{\sstcaptionlength}{\center{\Large #2}}
   \parbox[t]{\sstbannerlength}{\flushright{\Large {\bf #1}}}
   \begin{description}
      #3
   \end{description}
}

%  Format the description section.
\newcommand{\sstdescription}[1]{\item[Description:] #1}

%  Format the usage section.
\newcommand{\sstusage}[1]{\item[Usage:] \mbox{}
\\[1.3ex]{\raggedright \ssttt #1}}

%  Format the invocation section.
\newcommand{\sstinvocation}[1]{\item[Invocation:]\hspace{0.4em}{\tt #1}}

%  Format the arguments section.
\newcommand{\sstarguments}[1]{
   \item[Arguments:] \mbox{} \\
   \vspace{-3.5ex}
   \begin{description}
      #1
   \end{description}
}

%  Format the returned value section (for a function).
\newcommand{\sstreturnedvalue}[1]{
   \item[Returned Value:] \mbox{} \\
   \vspace{-3.5ex}
   \begin{description}
      #1
   \end{description}
}

%  Format the parameters section (for an application).
\newcommand{\sstparameters}[1]{
   \item[Parameters:] \mbox{} \\
   \vspace{-3.5ex}
   \begin{description}
      #1
   \end{description}
}

%  Format the examples section.
\newcommand{\sstexamples}[1]{
   \item[Examples:] \mbox{} \\
   \vspace{-3.5ex}
   \begin{description}
      #1
   \end{description}
}

%  Define the format of a subsection in a normal section.
\newcommand{\sstsubsection}[1]{ \item[{#1}] \mbox{} \\}

%  Define the format of a subsection in the examples section.
\newcommand{\sstexamplesubsection}[2]{\sloppy
\item[\parbox{\sstexampleslength}{\ssttt #1}] \mbox{} \vspace{1.0ex}
\\ #2 }

%  Format the notes section.
\newcommand{\sstnotes}[1]{\item[Notes:] \mbox{} \\[1.3ex] #1}

%  Provide a general-purpose format for additional (DIY) sections.
\newcommand{\sstdiytopic}[2]{\item[{\hspace{-0.35em}#1\hspace{-0.35em}:}]
\mbox{} \\[1.3ex] #2}

%  Format the implementation status section.
\newcommand{\sstimplementationstatus}[1]{
   \item[{Implementation Status:}] \mbox{} \\[1.3ex] #1}

%  Format the bugs section.
\newcommand{\sstbugs}[1]{\item[Bugs:] #1}

%  Format a list of items while in paragraph mode.
\newcommand{\sstitemlist}[1]{
  \mbox{} \\
  \vspace{-3.5ex}
  \begin{itemize}
     #1
  \end{itemize}
}

%  Define the format of an item.
\newcommand{\sstitem}{\item}

%% Now define html equivalents of those already set. These are used by
%  latex2html and are defined in the html.sty files.
\begin{htmlonly}

%  sstroutine.
   \newcommand{\sstroutine}[3]{
      \subsection{#1\xlabel{#1}-\label{#1}#2}
      \begin{description}
         #3
      \end{description}
   }

%  sstdescription
   \newcommand{\sstdescription}[1]{\item[Description:]
      \begin{description}
         #1
      \end{description}
      \\
   }

%  sstusage
   \newcommand{\sstusage}[1]{\item[Usage:]
      \begin{description}
         {\ssttt #1}
      \end{description}
      \\
   }

%  sstinvocation
   \newcommand{\sstinvocation}[1]{\item[Invocation:]
      \begin{description}
         {\ssttt #1}
      \end{description}
      \\
   }

%  sstarguments
   \newcommand{\sstarguments}[1]{
      \item[Arguments:] \\
      \begin{description}
         #1
      \end{description}
      \\
   }

%  sstreturnedvalue
   \newcommand{\sstreturnedvalue}[1]{
      \item[Returned Value:] \\
      \begin{description}
         #1
      \end{description}
      \\
   }

%  sstparameters
   \newcommand{\sstparameters}[1]{
      \item[Parameters:] \\
      \begin{description}
         #1
      \end{description}
      \\
   }

%  sstexamples
   \newcommand{\sstexamples}[1]{
      \item[Examples:] \\
      \begin{description}
         #1
      \end{description}
      \\
   }

%  sstsubsection
   \newcommand{\sstsubsection}[1]{\item[{#1}]}

%  sstexamplesubsection
   \newcommand{\sstexamplesubsection}[2]{\item[{\ssttt #1}] #2}

%  sstnotes
   \newcommand{\sstnotes}[1]{\item[Notes:] #1 }

%  sstdiytopic
   \newcommand{\sstdiytopic}[2]{\item[{#1}] #2 }

%  sstimplementationstatus
   \newcommand{\sstimplementationstatus}[1]{
      \item[Implementation Status:] #1
   }

%  sstitemlist
   \newcommand{\sstitemlist}[1]{
      \begin{itemize}
         #1
      \end{itemize}
      \\
   }
%  sstitem
   \newcommand{\sstitem}{\item}

\end{htmlonly}

%  End of "sst.tex" layout definitions.
%.



% ? End of document specific commands
% -----------------------------------------------------------------------------
%  Title Page.
%  ===========
\renewcommand{\thepage}{\roman{page}}
\begin{document}
\thispagestyle{empty}

%  Latex document header.
%  ======================
\begin{latexonly}
   CCLRC / \textsc{Rutherford Appleton Laboratory} \hfill \textbf{\stardocname}\\
   {\large Particle Physics \& Astronomy Research Council}\\
   {\large Starlink Project\\}
   {\large \stardoccategory\ \stardocnumber}
   \begin{flushright}
   \stardocauthors\\
   \stardocdate
   \end{flushright}
   \vspace{-4mm}
   \rule{\textwidth}{0.5mm}
   \vspace{5mm}
   \begin{center}
   {\Huge\textbf{\stardoctitle \\ [2.5ex]}}
   {\LARGE\textbf{\stardocversion \\ [4ex]}}
   {\Huge\textbf{\stardocmanual}}
   \end{center}
   \vspace{5mm}

% ? Add picture here if required for the LaTeX version.
%   e.g. \includegraphics[scale=0.3]{filename.ps}
% ? End of picture

% ? Heading for abstract if used.
   \vspace{10mm}
   \begin{center}
      {\Large\textbf{Abstract}}
   \end{center}
% ? End of heading for abstract.
\end{latexonly}

%  HTML documentation header.
%  ==========================
\begin{htmlonly}
   \xlabel{}
   \begin{rawhtml} <H1> \end{rawhtml}
      \stardoctitle\\
      \stardocversion\\
      \stardocmanual
   \begin{rawhtml} </H1> <HR> \end{rawhtml}

% ? Add picture here if required for the hypertext version.
%   e.g. \includegraphics[scale=0.7]{filename.ps}
% ? End of picture

   \begin{rawhtml} <P> <I> \end{rawhtml}
   \stardoccategory\ \stardocnumber \\
   \stardocauthors \\
   \stardocdate
   \begin{rawhtml} </I> </P> <H3> \end{rawhtml}
      \htmladdnormallink{CCLRC / Rutherford Appleton Laboratory}
                        {http://www.cclrc.ac.uk} \\
      \htmladdnormallink{Particle Physics \& Astronomy Research Council}
                        {http://www.pparc.ac.uk} \\
   \begin{rawhtml} </H3> <H2> \end{rawhtml}
      \htmladdnormallink{Starlink Project}{http://www.starlink.rl.ac.uk/}
   \begin{rawhtml} </H2> \end{rawhtml}
   \htmladdnormallink{\htmladdimg{source.gif} Retrieve hardcopy}
      {http://www.starlink.rl.ac.uk/cgi-bin/hcserver?\stardocsource}\\

%  HTML document table of contents. 
%  ================================
%  Add table of contents header and a navigation button to return to this 
%  point in the document (this should always go before the abstract \section). 
  \label{stardoccontents}
  \begin{rawhtml} 
    <HR>
    <H2>Contents</H2>
  \end{rawhtml}
  \htmladdtonavigation{\htmlref{\htmladdimg{contents_motif.gif}}
        {stardoccontents}}

% ? New section for abstract if used.
  \section{\xlabel{abstract}Abstract}
% ? End of new section for abstract
\end{htmlonly}

% -----------------------------------------------------------------------------
% ? Document Abstract. (if used)
%  ==================
\stardocabstract
% ? End of document abstract

% -----------------------------------------------------------------------------
% ? Latex Copyright Statement
%  =========================
%\begin{latexonly}
%\newpage
%\vspace*{\fill}
%\stardoccopyright
%\end{latexonly}
% ? End of Latex copyright statement

% -----------------------------------------------------------------------------
% ? Latex document Table of Contents (if used).
%  ===========================================
  \newpage
  \begin{latexonly}
    \setlength{\parskip}{0mm}
    \tableofcontents
    \setlength{\parskip}{\medskipamount}
    \markboth{\stardocname}{\stardocname}
  \end{latexonly}
% ? End of Latex document table of contents
% -----------------------------------------------------------------------------

\cleardoublepage
\renewcommand{\thepage}{\arabic{page}}
\setcounter{page}{1}

\section{INTRODUCTION}

The Simple Software Tools package (SST) is a package of applications
designed to help with the production of software and documentation, with
particular emphasis on ADAM programming using Fortran~77.

As its name suggests, SST is intended for performing fairly simple
manipulation of software, but it aims to tackle some of the
commonly encountered problems which are not catered for in more
sophisticated commercial software tools (such as FORCHECK and VAXset) which
are available on Starlink.
In future, SST may also duplicate a few of the simpler facilities which
commercial products offer in order to avoid the cost (and generally much
higher overheads, such as disk space) which prevent the commercial products
being freely available on all Starlink machines.

This document describes the first version of the SST package, whose main
initial purpose is to provide tools to extract information from subroutine
``prologues'' and to format this information in a variety of ways to produce
different forms of user documentation.
A simple source-code and comment statistics tool is also included.


\section{GETTING STARTED}

All the applications in the SST package run under ADAM, so before you start
you should execute the command:

\begin{verbatim}
   $ ADAMSTART
\end{verbatim}

This is a good opportunity to add this command to your LOGIN.COM file if you
have not already done so.

The commands within the SST package itself can be used either from the ADAM
command language ICL, or from DCL.
These commands are initially made available by typing `SST' at the command
prompt, {\em e.g.}\ from ICL:

\begin{verbatim}
   $ ICL
   ICL> SST
\end{verbatim}

or from DCL:

\begin{verbatim}
   $ SST
\end{verbatim}

In either case, the following message will be displayed, indicating that the
package is ready for use:

\begin{quote}
{\tt Simple Software Tools (SST) commands are now available.}
\end{quote}

You can then invoke any of the SST applications by typing the appropriate
command (see Appendix~\ref{appendix:descriptions} for a detailed description
of the applications available).

When working from ICL you can obtain help about SST by using the HELP command,
thus:

\begin{verbatim}
   ICL> HELP SST
\end{verbatim}

You can also obtain help on each individual application by giving its name,
{\em e.g:}

\begin{verbatim}
   ICL> HELP FORSTATS
\end{verbatim}

will give help on the SST application FORSTATS.
``In-line'' help on an application's parameters is also available, in the
normal ADAM way, by typing `?' or `??' in response to any prompt requesting a
value for a parameter.


\section{OVERVIEW OF SST APPLICATIONS}

\subsection{Producing \LaTeX\ Documentation (PROLAT)}

The SST application PROLAT is provided to facilitate the production of
documentation for packages of applications and subroutine libraries.
It works by extracting the necessary information from ``prologues'' ({\em
i.e.}\ header comments) within the source-code files which comprise the
software, and formatting this information into a \LaTeX\ document.
The documentation in Appendix~\ref{appendix:descriptions}, for example, has
been produced by this means.

For PROLAT to work correctly, the prologue information must use the layout
generated by the extended Language Sensitive Editor STARLSE (SUN/105), with
which a number of the SST applications are designed to integrate.
It is also possible to convert some older ADAM/SSE prologue layouts into the
required form automatically if required (see \S\ref{sect:procvt} for
details).

To use PROLAT, you should supply a file containing prologue information in
the correct format as input, and the resulting output will then be a \LaTeX\
document describing the software in the input file (by default this is
assumed to be an ADAM application program, otherwise known as an A-task).
The document will be written to a file (PROLAT.TEX by default) which can
then be processed directly with the LATEX command.

For example, if an application with suitable prologue information resides in
the file PROG.FOR, then the following sequence of commands would extract this
information and send the derived document to a Canon laser printer:

\begin{verbatim}
   ICL> PROLAT PROG.FOR
   ICL> SPAWN LATEX PROLAT
   ICL> $DVICAN PROLAT
   ICL> $PRCN PROLAT.DVI-CAN
\end{verbatim}

When working from ICL, notice how `SPAWN' must be used in front of the LATEX
command.
This is to overcome problems with the way LATEX writes to the terminal, which
can cause ICL to ``hang'' if a \$LATEX command is issued directly.
The last three commands are, of course, simply the standard processing
sequence for sending a \LaTeX\ source file to a Starlink laser printer (see
SUN/9, SUN/12 \& SUN/34 for further details of \LaTeX).
The entire operation could also be performed from DCL directly.

As with most SST applications, a ``wild card'' input file may be specified
in order to combine a number of program descriptions into a single document.
Thus, the following command could be used to document an entire software
system if the necessary files resided in the current directory:

\begin{verbatim}
   ICL> PROLAT *.FOR
\end{verbatim}

PROLAT has a number of options to tailor its behaviour.
For instance, you can specify that you are processing information about a
subroutine library (rather than application programs) so as to produce a
slightly different documentation format.
You can also specify that you want only the descriptive part of the document
without any \LaTeX\ declarations.
This latter option is useful if you are adding the information to a larger
document which already contains all the necessary \LaTeX\ commands ({\em
e.g.}\ as an appendix, such as in the present document).

Note that PROLAT does not attempt to provide a sophisticated text-processing
language and does not, therefore, exploit all the layout facilities which
\LaTeX\ has to offer.
Instead, it is simply intended for achieving a close approximation to the
final form of a document as quickly and easily as possible.
If you want to go further and use fancy \LaTeX\ fonts or an extravagant layout
in your final document, then you should be prepared to edit the output file
from PROLAT to achieve this.
By adopting this approach, the prologue layout can be kept relatively
simple, so that it can function as the source for other types of
documentation (see below), as well as for possible future documentation
formats yet to be invented.

\subsection{Producing Help Libraries (PROHLP)}

The SST application PROHLP functions in a rather similar way to PROLAT
(above), except that its output format is not \LaTeX, but a text file
designed for insertion into a help library.
By using PROHLP you can therefore make program and subroutine documentation
available on-line, as well as simply on paper.

The help file format produced by PROHLP is designed to integrate both with
the ADAM help system and also with the STARLSE editor.
This means that it is easy to make prologue information available to the
person running a program, and also to make subroutine information available
to someone using STARLSE (see \S\ref{sect:propak} for details of the
latter).
As before, the initial prologue information must be in the form generated by
STARLSE.

To give an example, suppose you have an ADAM application program (an A-task)
in the file PROG.FOR, along with suitable prologue information, and you want
to make this information available in a help library so that it can be
accessed when the program is run.
The first step would be to create the library (if you have not already done
so), as follows:

\begin{verbatim}
   ICL> $LIBRARY/CREATE/HELP MYLIB.HLB
\end{verbatim}

The prologue information is then extracted using PROHLP which, by default,
writes it to the file PROHLP.HLP:

\begin{verbatim}
   ICL> PROHLP PROG.FOR
\end{verbatim}

The extracted information can then be inserted into the library and the
intermediate help file deleted:

\begin{verbatim}
   ICL> $LIBRARY/HELP MYLIB.HLB PROHLP.HLP
   ICL> $DELETE PROHLP.HLP;*
\end{verbatim}

You can check that the help library contains the correct information by using
the DCL command \$HELP with the /LIBRARY qualifier.

This help information can be made available to the application program via
the ADAM help system.
To do this, the interface (.IFL) file for the application should contain
references to the appropriate entries in the help library.
Thus, at the start of the interface file, before any parameter definitions,
there should be a line such as:

\begin{verbatim}
   helplib 'MYDISK::[ME.PROGS]MYLIB'
\end{verbatim}

which identifies the help library to be used.
Note that a full directory specification is needed (or an equivalent logical
name).
Each of the application's parameters for which help information is available
should also have a line in its interface file entry such as:

\begin{verbatim}
   helpkey *
\end{verbatim}

where `$*$' specifies that a default set of keywords should be used to
access the parameter information in the help library.
The library structure which results from using PROHLP is compatible with
this set of defaults.


\subsection{Producing STARLSE Package Definitions (PROPAK)}
\label{sect:propak}

If you have developed a subroutine library using STARLSE, with prologue
information in the correct form, then the SST application PROPAK can be used
to generate an LSE {\em package definition} to facilitate future use of the
library from within the STARLSE editor.
This package definition will define LSE templates for each of the routines
in the library.
It can then be used to make these templates available within the editor, in
the same way that the standard ADAM-subroutine templates are provided within
STARLSE.

A package definition can be generated by providing the appropriate source
code files (containing prologue information) as input to PROPAK, for
instance:

\begin{verbatim}
   ICL> PROPAK *.FOR PACK=XYZ
\end{verbatim}

Note that a value for the PACK parameter (the name of the package to be
produced) must also be given.
It will be prompted for if not supplied on the command line.

By default, the output package definition is written to the file PROPAK.LSE,
which can then be used to extend your STARLSE editing environment.
To do this, the file should be read into an editing buffer and the LSE
command `DO' executed to compile the definition.
If you then move to a buffer with a file type of `.FOR' or `.GEN', the
routine templates you have just defined will become available for use.
Note that LSE allows you to save an extended editing context for future use
so that you do not have to repeat this compilation process each time
(although the resulting files can be rather large).
For details of how to do this the VAXset documentation for LSE should be
consulted.

\paragraph{Accessing help libraries.}
An important feature of PROPAK is its ability to make a package definition
refer to a matching help library produced from the same source files using
PROHLP.
To do so, the HELP parameter should be used with the PROPAK command to
specify the (full) name of the help library to which the package should
refer.
Help library references will then be inserted into the package definition
and the associated help information will become available from within
STARLSE when the definition is compiled.
(To access help on a particular routine, the cursor is placed on the routine
name in the editing buffer and the help key GOLD PF2 is used.)

Note that the ATASK parameter should be set to FALSE when using PROHLP for
this purpose in order to obtain a help library suitable for a subroutine
library, rather than a package of applications programs.


\subsection{Converting ``Old-Style'' ADAM/SSE Prologues (PROCVT)}
\label{sect:procvt}

A considerable amount of ADAM software already in existence uses an
``old-style'' prologue layout based on templates held in the ADAM\_PRO
directory.
Since there has never been a fully satisfactory method of converting such
prologue information into documentation, it is doubtful whether many of
these older prologues are of a high enough quality to be used for this
purpose now.
Nevertheless, in a spirit of optimism, the SST application PROCVT has been
provided to permit the conversion of these old-style prologues into the new
(STARLSE) form required by other SST applications.
PROCVT can also be used as a useful first step in upgrading an old-style
prologue, even if the information it contains also needs attention before it
becomes suitable for documentation purposes.

Normally, PROCVT will be used with both an input and an output file
specified, for instance:

\begin{verbatim}
   ICL> PROCVT OLD.FOR NEW.FOR
\end{verbatim}

This command will convert the old-style prologue held in the file OLD.FOR
into STARLSE format and will write it (together with all the associated
source code) to the file NEW.FOR.
Be careful not to use PROCVT on a file which already has a suitable prologue
layout -- you will not be pleased with the result!

Note that perfect results cannot always be expected with PROCVT unless the
original old-style prologue template has been followed very carefully.
Consequently, it may sometimes be necessary to use STARLSE afterwards to
make fine adjustments before adequate documentation can be produced using
other SST commands.
To facilitate this, PROCVT will insert appropriate STARLSE placeholders into
the output file wherever information is missing.


\subsection{Producing Source-Code Statistics (FORSTATS)}

The SST application FORSTATS is provided as a ``quick look'' method of
inspecting large Fortran~77 software systems, with the aim of assessing the
volume of code present and its likely quality.
These two factors (code volume and quality) are usually the most important
indicators of the amount of work needed to maintain, or change, an existing
software package, especially if you are not initially familiar with it.

Counting lines of code is obviously a simple method of measuring software
volume, but the measurement of quality is more difficult and subjective.
The approach used by FORSTATS depends on the observation that if corners are
cut during code development, then the first and most obvious sign (at least
on Starlink) is usually the omission of in-code comments and prologue
information.
Past experience with Starlink software has shown that there is an important
correlation between the ease of understanding and supporting a large
software package and the overall level of commenting which its original
author has provided.

FORSTATS attempts to quantify this by producing a number of simple
statistics.
These are based on line and character counts derived from the code and
comments in each program unit of a Fortran~77 software system.
It compares these statistics with a set of expected values, which have been
derived experimentally from existing Starlink software, and flags values
which lie outside the expected range.

When developing software, FORSTATS can be used to highlight those areas
where more attention to in-code commentary would be useful to assist future
support programmers.
Conversely, when inheriting an existing software package, FORSTATS can be
used as an aid to identifying possible problem areas before planning
changes, {\em etc.}
In a large software system, it can also be invaluable simply for listing the
program units present and indicating their size.

To run FORSTATS on a software system contained in a set of `.FOR' files, you
might use the command:

\begin{verbatim}
   ICL> FORSTATS *.FOR
\end{verbatim}

This command will analyse the specified files and write a statistics file
(FORSTATS.LIS by default), which can then be printed.
When it is first run, FORSTATS will append an explanatory key to the
statistics file to help you interpret its contents.
You can suppress this key in future by specifying `NOKEY' on the command
line.

\newpage
\appendix
\section{DESCRIPTIONS OF INDIVIDUAL APPLICATIONS}
\label{appendix:descriptions}
\sstroutine{
   FORSTATS
}{
   Produces statistics on Fortran 77 source code and comments
}{
   \sstdescription{
      This application analyses a sequence of Fortran 77 source code
      files, divides their contents into program units, and produces
      statistics about the number and distribution of code and comment
      lines in each unit. These statistics are compared with typical
      values expected for well-crafted Fortran 77 code and obvious
      deviations from the expected values are flagged.

      It is intended as a quick-look tool for assessing the size and
      likely quality of a Fortran 77 software system and is based on
      the observation that adequate commenting is one of the first
      things to be sacrificed if corners are cut during code
      development.  A relatively simple analysis of comment and code
      lines can therefore reveal if this has occurred and can highlight
      potential trouble spots for visual inspection.
   }
   \sstusage{
      FORSTATS IN [OUT]
   }
   \sstparameters{
      \sstsubsection{
         IN() = LITERAL (Read)
      }{
         A list of (optionally wild-carded) file specifications which
         identify the Fortran 77 source code files to be used for
         input.  Up to 10 values may be given, but only a single
         specification such as {\tt '}$*$.FOR{\tt '} is normally required. There is
         no limit to the number of program units which may be held in
         each input file.
      }
      \sstsubsection{
         KEY = \_LOGICAL (Read)
      }{
         If KEY is set to TRUE, then a explanatory key will be appended
         to the output statistics file to describe how to interpret the
         values it contains. If KEY is set to FALSE, no such key will
         be appended. This parameter behaves as a latch and remains set
         to the value previously used until a new value is specified on
         the command line. [TRUE]
      }
      \sstsubsection{
         OUT = FILE (Write)
      }{
         The file to which the statistics derived from the input source
         code will be written. This file is intended for printing.
         [FORSTATS.LIS]
      }
   }
   \sstexamples{
      \sstexamplesubsection{
         FORSTATS $*$.FOR
      }{
         Analyses the source code held in the files $*$.FOR and writes a
         summary of the derived statistics to the default output file
         FORSTATS.LIS. This file may then be printed.
      }
      \sstexamplesubsection{
         FORSTATS SOFTWARE.FOR SOFTWARE.LIS NOKEY
      }{
         Analyses the source code held in the file SOFTWARE.FOR and
         writes the derived statistics to the file SOFTWARE.LIS. No
         explanatory key is written to the output file on this or
         subsequent invocations of FORSTATS until a TRUE value is given
         for the KEY parameter on the command line.
      }
      \sstexamplesubsection{
         FORSTATS IN=[{\tt "}A$*$.FOR{\tt "},{\tt "}B$*$.FOR{\tt "}] OUT=ANALYSIS.LIS
      }{
         In this example, a sequence of input file specifications is
         given and each will be analysed in turn. The combined
         statistics will be written to the file ANALYSIS.LIS.
      }
   }
   \sstnotes{
      This is a general-purpose tool and may be used on any Fortran 77
      software system, since it does not depend on any particular
      layout or format. However, the analysis performed does include a
      test for the existence of prologue information. For this purpose,
      prologues are taken to be delimited by {\tt '}$+${\tt '} or {\tt '}-{\tt '} characters
      appearing in the second column of a comment line. Note that
      unlike some other SST applications, both these characters are
      considered as equivalent by FORSTATS; i.e. the first occurrence
      begins a prologue and the second occurrence ends it, regardless
      of which character is used.
   }
   \sstdiytopic{
      Timing
   }{
      The execution time is approximately proportional to the total
      number of source code lines supplied for analysis. The time will
      be increased somewhat if the code resides in a large number of
      separate files, due to the need to open and close each file.
   }
}
\newpage
\sstroutine{
   PROCVT
}{
   Converts {\tt "}old-style{\tt "} ADAM/SSE prologues into STARLSE format
}{
   \sstdescription{
      This application converts {\tt "}old-style{\tt "} ADAM/SSE routine prologues
      into the format generated by the extended VAX Language Sensitive
      Editor STARLSE (SUN/105). The converted prologue information may
      then be used as input to other SST applications which generate
      documentation, help libraries, etc.
   }
   \sstusage{
      PROCVT IN [OUT]
   }
   \sstparameters{
      \sstsubsection{
         ATASK = \_LOGICAL (Read)
      }{
         If ATASK is set to TRUE, then the style of output prologue
         will be suitable for application programs (i.e.  ADAM
         A-tasks).  If it is set to FALSE, then prologues suitable for
         ordinary subroutines or functions, such as a subroutine
         library, are produced. [TRUE]
      }
      \sstsubsection{
         IN() = LITERAL (Read)
      }{
         A list of (optionally wild-carded) file specifications which
         identify the Fortran 77 source code files with {\tt "}old-style{\tt "}
         ADAM prologues which are to be used for input. Up to 10 values
         may be given, but only a single specification such as {\tt '}$*$.FOR{\tt '}
         is normally required.  There is no limit on the number of
         program units which may be held in each input file.
      }
      \sstsubsection{
         LANG = LITERAL (Read)
      }{
         This value specifies the programming language (i.e. the
         particular dialect of Fortran 77) in which the source code is
         written. It does not affect the processing performed, but will
         be inserted into the output prologues under the {\tt "}Language{\tt "}
         section heading. If a null (!) or blank value is given, then
         an appropriate STARLSE placeholder will be used instead. The
         required entry can then be made later by hand. [{\tt '}VAX Fortran{\tt '}]
      }
      \sstsubsection{
         OUT = FILE (Write)
      }{
         The output file to which the source code will be copied with
         its prologue format converted. If multiple input files are
         specified, then the converted output will be concatenated into
         a single file. [PROCVT.FOR]
      }
   }
   \sstexamples{
      \sstexamplesubsection{
         PROCVT MYPROG\_OLD.FOR MYPROG\_NEW.FOR
      }{
         Reads the application program source code held in the file
         MYPROG\_OLD.FOR and converts its prologue information into
         STARLSE format. The converted source code is written to the
         file MYPROG\_NEW.FOR.
      }
      \sstexamplesubsection{
         PROCVT $*$.FOR OUT=SUBS.FOR ATASK=FALSE LANG={\tt "}Starlink Fortran 77{\tt "}
      }{
         Reads the source code for a subroutine library, held in the
         files $*$.FOR, and converts its prologue information to STARLSE
         format. The converted source code is written to a single
         output file SUBS.FOR. The language specified in the converted
         prologues will be {\tt "}Starlink Fortran 77{\tt "}.
      }
      \sstexamplesubsection{
         PROCVT IN=[{\tt "}$*$.FOR{\tt "},{\tt "}$*$.GEN{\tt "}] NOATASK
      }{
         In this example, a sequence of input file specifications is
         given. Each will be processed in turn, converting its prologue
         information into STARLSE format. Output is written to the
         single default output file PROCVT.FOR.
      }
   }
   \sstnotes{
      Due to the variety of layouts found in existing {\tt "}old-style{\tt "}
      ADAM/SSE prologues, this application is not guaranteed always to
      produce a perfect result. Particular problem areas are the
      inconsistent use of indentation, and the occasional use of
      section headings and placeholders (of the form {\tt '}$<$...$>${\tt '}) which do
      not quite match those given in the original ADAM/SSE prologue
      templates.  Nevertheless, in most cases, PROCVT will produce a
      result which can be edited fairly easily using STARLSE to conform
      with the requirements of other SST applications. Appropriate
      STARLSE placeholders will be inserted into output prologues to
      facilitate the entry of any essential information which cannot be
      found in the corresponding input prologue.
   }
   \sstdiytopic{
      Timing
   }{
      The execution time is approximately proportional to the total
      number of source code lines supplied for conversion. The time
      will be increased somewhat if the code resides in a large number
      of separate files, due to the need to open and close each file.
   }
}
\newpage
\sstroutine{
   PROHLP
}{
   Converts routine prologue information into a help library input
   file
}{
   \sstdescription{
      This application reads a series of Fortran 77 source code files
      containing prologue information formatted using STARLSE (SUN/105)
      and produces an output file containing user documentation for
      each routine in a format suitable for insertion into a help
      library. The help library format may be chosen to suit either a
      package of application programs (i.e. ADAM A-tasks) or a set of
      ordinary subroutines or functions, such as a subroutine library.
   }
   \sstusage{
      PROHLP IN [OUT]
   }
   \sstparameters{
      \sstsubsection{
         ATASK = \_LOGICAL (Read)
      }{
         If ATASK is set to TRUE, then a help library format suitable
         for a package of application programs (i.e. ADAM A-tasks) is
         produced. If it is set to FALSE, then the help library format
         produced is suitable for a subroutine library.  [TRUE]
      }
      \sstsubsection{
         IN() = LITERAL (Read)
      }{
         A list of (optionally wild-carded) file specifications which
         identify the Fortran 77 source code files to be used for
         input. Up to 10 values may be given, but only a single
         specification such as {\tt '}$*$.FOR{\tt '} is normally required.

         If the SINGLE parameter is set to TRUE (the default), then
         only a single prologue will be expected in each input file. If
         it is set to FALSE, then there is no limit to the number of
         prologues which may be held in each input file.
      }
      \sstsubsection{
         OUT = FILE (Write)
      }{
         The output file to which the help information will be written.
         The information in this file will need to be inserted into a
         help library for use.  [PROHLP.HLP]
      }
      \sstsubsection{
         SINGLE = \_LOGICAL (Read)
      }{
         If SINGLE is set to TRUE, then only a single prologue will be
         expected at the start of each input file; if the file contains
         more than one prologue, then the remaining ones will be
         ignored. If SINGLE is set to FALSE, then each input file will
         be searched for all the prologues it contains and each will be
         processed in turn. When appropriate, the former option (the
         default) will result in faster execution since only the
         initial prologue information must then be read, rather than
         the entire contents of each input file.  [TRUE]
      }
   }
   \sstexamples{
      \sstexamplesubsection{
         PROHLP PROG.FOR PROG.HLP
      }{
         Extracts prologue information from the application program
         source code held in the file PROG.FOR and produces a help
         library entry for it. The program{\tt '}s name is used as the
         level-1 help keyword. The output is written to the file
         PROG.HLP.
      }
      \sstexamplesubsection{
         PROHLP $*$.FOR OUT=SUBS.HLP ATASK=FALSE
      }{
         Extracts prologue information for a subroutine library, whose
         source code resides in the files $*$.FOR, one routine per file.
         Help library entries describing the routines are written to
         the file SUBS.HLP, with each routine{\tt '}s name being used as a
         level-1 help keyword.
      }
      \sstexamplesubsection{
         PROHLP IN=SOURCE.FOR NOATASK NOSINGLE
      }{
         Extracts prologue information from a sequence of subroutines
         or functions which are all held in the file SOURCE.FOR and
         produces help library entries for them in the default output
         file PROHLP.HLP. Each routine name is used to form a separate
         level-1 help keyword.
      }
      \sstexamplesubsection{
         PROHLP IN=[{\tt "}A$*$.FOR{\tt "},{\tt "}B$*$.FOR{\tt "}] NOATASK
      }{
         In this example, a sequence of input file specifications is
         given. Each will be processed in turn to generate a help file
         entry from the first prologue encountered in each file.
         Output is written to the file PROHLP.HLP. Once again, each
         routine name generates a separate level-1 help keyword.
      }
   }
   \sstnotes{
      Care must be taken to ensure that begin-prologue and end-prologue
      lines (starting {\tt '}$*$$+${\tt '} and {\tt '}$*$-{\tt '} respectively) appear before and
      after each prologue and that {\tt '}$+${\tt '} and {\tt '}-{\tt '} symbols are not used in
      the second column elsewhere in the file, otherwise prologues may
      not be correctly identified.
   }
   \sstdiytopic{
      Timing
   }{
      The execution time is approximately proportional to the amount of
      information to be read from the input files. In addition, the
      time will be increased somewhat if the input code resides in a
      large number of separate files, due to the need to open and close
      each file. If there is only one prologue in each input file, then
      execution time will be minimised if SINGLE is set to TRUE, since
      only the initial prologue information need then be read, rather
      than the entire contents of each file.
   }
}
\newpage
\sstroutine{
   PROLAT
}{
   Converts routine prologue information into Latex documentation
}{
   \sstdescription{
      This application reads a series of Fortran 77 source code files
      containing prologue information formatted using STARLSE (SUN/105)
      and produces an output file containing Latex user documentation
      for each routine. The documentation format may be chosen to suit
      either a package of application programs (i.e.  ADAM A-tasks) or
      a set of ordinary subroutines or functions, such as a subroutine
      library.
   }
   \sstusage{
      PROLAT IN [OUT]
   }
   \sstparameters{
      \sstsubsection{
         ATASK = \_LOGICAL (Read)
      }{
         If ATASK is set to TRUE, then a style of documentation
         suitable for a package of application programs (i.e. ADAM
         A-tasks) is produced. If it is set to FALSE, then the
         documentation produced is suitable for a subroutine library.
         [TRUE]
      }
      \sstsubsection{
         DOCUMENT = \_LOGICAL (Read)
      }{
         If DOCUMENT is set to TRUE, then the output file will be a
         complete Latex document and will contain all necessary layout
         definitions, etc. Such a file may be passed directly to Latex
         for processing. If DOCUMENT is set to FALSE, then the output
         file will contain only the information extracted from the
         input files. In this case, additional Latex commands must be
         added to produce a complete document (see the Latex
         Definitions section for details). This latter option would
         typically be used when the output is to be included in a
         larger document which already contains the necessary
         definitions. [TRUE]
      }
      \sstsubsection{
         IN() = LITERAL (Read)
      }{
         A list of (optionally wild-carded) file specifications which
         identify the Fortran 77 source code files to be used for
         input. Up to 10 values may be given, but only a single
         specification such as {\tt '}$*$.FOR{\tt '} is normally required.

         If the SINGLE parameter is set to TRUE (the default), then
         only a single prologue will be expected in each input file. If
         it is set to FALSE, then there is no limit to the number of
         prologues which may be held in each input file.
      }
      \sstsubsection{
         OUT = FILE (Write)
      }{
         The output file to which the Latex documentation will be
         written. [PROLAT.TEX]
      }
      \sstsubsection{
         PAGE = \_LOGICAL (Read)
      }{
         If PAGE is set to TRUE, then a new output page will be started
         to hold the information extracted from each input prologue.
         If PAGE is set to FALSE, then this will not occur.  The
         default behaviour is to start each prologue description on a
         new page for applications packages (ADAM A-tasks), but for
         routine descriptions in subroutine libraries to follow
         end-to-end. []
      }
      \sstsubsection{
         SINGLE = \_LOGICAL (Read)
      }{
         If SINGLE is set to TRUE, then only a single prologue will be
         expected at the start of each input file; if the file contains
         more than one prologue, then the remaining ones will be
         ignored. If SINGLE is set to FALSE, then each input file will
         be searched for all the prologues it contains and each will be
         processed in turn. When appropriate, the former option (the
         default) will result in faster execution since only the
         initial prologue information must then be read, rather than
         the entire contents of each input file.  [TRUE]
      }
   }
   \sstexamples{
      \sstexamplesubsection{
         PROLAT MYPROG.FOR MYPROG.TEX
      }{
         Extracts prologue information from the application program
         source code held in the file MYPROG.FOR and produces a Latex
         user document for it.  The Latex output is written to the file
         MYPROG.TEX.
      }
      \sstexamplesubsection{
         PROLAT $*$.FOR OUT=SUBS.TEX ATASK=FALSE
      }{
         Extracts prologue information for a subroutine library, whose
         source code resides in the files $*$.FOR, one routine per file.
         A Latex document describing the routines is written to the
         file SUBS.TEX.
      }
      \sstexamplesubsection{
         PROLAT IN=SOURCE.FOR NOATASK NODOCUMENT NOSINGLE
      }{
         Extracts prologue information from a sequence of subroutines
         or functions which are all held in the file SOURCE.FOR and
         produces Latex output in the default output file PROLAT.TEX.
         This output file contains only the routine descriptions (no
         Latex definitions) and is therefore suitable for inclusion in
         a larger document.
      }
      \sstexamplesubsection{
         PROLAT IN=[{\tt "}$*$.FOR{\tt "},{\tt "}$*$.GEN{\tt "}] NOATASK PAGE
      }{
         In this example, a sequence of input file specifications is
         given. Each will be processed in turn to generate Latex
         documentation from the first prologue encountered in each
         file.  Output is written to the file PROLAT.TEX with the
         description of each routine starting on a new page.
      }
   }
   \sstnotes{
      Care must be taken to ensure that begin-prologue and end-prologue
      lines (starting {\tt '}$*$$+${\tt '} and {\tt '}$*$-{\tt '} respectively) appear before and
      after each prologue and that {\tt '}$+${\tt '} and {\tt '}-{\tt '} symbols are not used in
      the second column elsewhere in the file, otherwise prologues may
      not be correctly identified.
   }
   \sstdiytopic{
      Timing
   }{
      The execution time is approximately proportional to the amount of
      information to be read from the input files. In addition, the
      time will be increased somewhat if the input code resides in a
      large number of separate files, due to the need to open and close
      each file. If there is only one prologue in each input file, then
      execution time will be minimised if SINGLE is set to TRUE, since
      only the initial prologue information need then be read, rather
      than the entire contents of each file.
   }
   \sstdiytopic{
      Latex Definitions
   }{
      If the DOCUMENT=FALSE option is chosen, then the output file will
      contain none of the Latex command definitions needed to produce
      the final document. To define these commands, the contents of the
      file \$SST\_DIR/sst.tex must be included in the Latex input file
      in front of the output from PROLAT (the usual Latex preamble and
      $\backslash$begin\{document\} ... $\backslash$end\{document\} commands will also be needed,
      of course). The layout definitions in this file are designed to
      operate correctly within the Latex environment normally used in a
      Starlink User Note (see the file DOCSDIR:SUN.TEX for example).

      It is recommended that you include the contents of the file
      \$SST\_DIR/sst.tex in your final document explicitly rather than
      by using the Latex $\backslash$include directive, otherwise it may not be
      possible to process the document in future if changes have to be
      made to the Latex definitions in this file.
   }
}
\newpage
\sstroutine{
   PROPAK
}{
   Converts routine prologue information into a STARLSE package
   definition
}{
   \sstdescription{
      This application reads a series of Fortran 77 source code files
      containing prologue information formatted using STARLSE (SUN/105)
      and produces an output file containing an LSE package definition
      suitable for use with STARLSE. Help library references may be
      included if required, allowing the package definition to access
      help information extracted using the PROHLP application.
   }
   \sstusage{
      PROPAK IN [OUT] PACK [HELP]
   }
   \sstparameters{
      \sstsubsection{
         HELP = LITERAL (Read)
      }{
         Name of the help file to which the package definition should
         refer for on-line help information (suitable help libraries
         may be produced using the PROHLP application).  The full file
         name, including a directory name, should be given.  Logical
         names may also be used. The help file need not actually exist
         at the time PROPAK is run.

         If a null (!) or blank value is given for this parameter, then
         no help library references will be included in the package
         definition. [!]
      }
      \sstsubsection{
         IN() = LITERAL (Read)
      }{
         A list of (optionally wild-carded) file specifications which
         identify the Fortran 77 source code files to be used for
         input. Up to 10 values may be given, but only a single
         specification such as {\tt '}$*$.FOR{\tt '} is normally required.

         If the SINGLE parameter is set to TRUE (the default), then
         only a single prologue will be expected in each input file. If
         it is set to FALSE, then there is no limit to the number of
         prologues which may be held in each input file.
      }
      \sstsubsection{
         OUT = FILE (Write)
      }{
         The output file to which the STARLSE package definition will
         be written.  The VAXset documentation on LSE should be
         consulted for details of how to use this file to extend the
         STARLSE editing environment. [PROPAK.LSE]
      }
      \sstsubsection{
         PACK = LITERAL (Read)
      }{
         This value is the name of the LSE package to be generated. To
         avoid possible clashes with existing STARLSE packages, the
         prefix characters used on the routines themselves are
         recommended for use as the package name. For example, if
         routines in the package have names such as XYZ\_ROUTN, then
         {\tt "}XYZ{\tt "} should be used as the package name.
      }
      \sstsubsection{
         SINGLE = \_LOGICAL (Read)
      }{
         If SINGLE is set to TRUE, then only a single prologue will be
         expected at the start of each input file; if the file contains
         more than one prologue, then the remaining ones will be
         ignored. If SINGLE is set to FALSE, then each input file will
         be searched for all the prologues it contains and each will be
         processed in turn. When appropriate, the former option (the
         default) will result in faster execution since only the
         initial prologue information must then be read, rather than
         the entire contents of each input file.  [TRUE]
      }
   }
   \sstexamples{
      \sstexamplesubsection{
         PROPAK MPK\_$*$.FOR MYPACK.LSE MPK
      }{
         Extracts prologue information from the routines held in the
         files MPK\_$*$.FOR, one per file, and produces a STARLSE package
         definition for them. The package is called {\tt "}MPK{\tt "} and its
         definition is written to the file MYPACK.LSE.
      }
      \sstexamplesubsection{
         PROPAK $*$.FOR PACK=TEST HELP=TEST\_DIR:TESTHELP
      }{
         Extracts prologue information for a subroutine library, whose
         source code resides in the files $*$.FOR, one routine per file.
         The resulting definition, of a STARLSE package called {\tt "}TEST{\tt "},
         is written to the default output file PROPAK.LSE. The package
         contains references to the help library TEST\_DIR:TESTHELP,
         from which on-line help may be obtained when using the package
         within STARLSE.
      }
      \sstexamplesubsection{
         PROPAK IN=SOURCE.FOR PACK=MYLIB NOSINGLE
      }{
         Extracts prologue information from a sequence of subroutines
         or functions, all of which are held in the file SOURCE.FOR,
         and defines a STARLSE package called {\tt "}MYLIB{\tt "} in the default
         output file PROPAK.TEX.
      }
      \sstexamplesubsection{
         PROPAK IN=[{\tt "}A$*$.FOR{\tt "},{\tt "}B$*$.FOR{\tt "}] HELP=PACK\_HELP
      }{
         In this example, a sequence of input file specifications is
         given. Each will be processed in turn to generate a combined
         package definition which refers to a help library with the
         logical name PACK\_HELP. The package name will be prompted for.
      }
   }
   \sstnotes{
      Care must be taken to ensure that begin-prologue and end-prologue
      lines (starting {\tt '}$*$$+${\tt '} and {\tt '}$*$-{\tt '} respectively) appear before and
      after each prologue and that {\tt '}$+${\tt '} and {\tt '}-{\tt '} symbols are not used in
      the second column elsewhere in the file, otherwise prologues may
      not be correctly identified.
   }
   \sstdiytopic{
      Timing
   }{
      The execution time is approximately proportional to the amount of
      information to be read from the input files. In addition, the
      time will be increased somewhat if the input code resides in a
      large number of separate files, due to the need to open and close
      each file. If there is only one prologue in each input file, then
      execution time will be minimised if SINGLE is set to TRUE, since
      only the initial prologue information need then be read, rather
      than the entire contents of each file.
   }
}
\end{document}
