\documentstyle[11pt]{article}
\pagestyle{myheadings}

% -----------------------------------------------------------------------------
% ? Document identification
\newcommand{\stardoccategory}  {Starlink System Note}
\newcommand{\stardocinitials}  {SSN}
\newcommand{\stardocsource}    {ssn\stardocnumber}
\newcommand{\stardocnumber}    {21.2}
\newcommand{\stardocauthors}   {M J Bly}
\newcommand{\stardocdate}      {9 April 1996}
\newcommand{\stardoctitle}     {STARX --- X Library linking}
% \newcommand{\stardocabstract}  {[Text of abstract]}
% ? End of document identification
% -----------------------------------------------------------------------------

\newcommand{\stardocname}{\stardocinitials /\stardocnumber}
\markright{\stardocname}
\setlength{\textwidth}{160mm}
\setlength{\textheight}{230mm}
\setlength{\topmargin}{-2mm}
\setlength{\oddsidemargin}{0mm}
\setlength{\evensidemargin}{0mm}
\setlength{\parindent}{0mm}
\setlength{\parskip}{\medskipamount}
\setlength{\unitlength}{1mm}

% -----------------------------------------------------------------------------
%  Hypertext definitions.
%  ======================
%  These are used by the LaTeX2HTML translator in conjunction with star2html.

%  Comment.sty: version 2.0, 19 June 1992
%  Selectively in/exclude pieces of text.
%
%  Author
%    Victor Eijkhout                                      <eijkhout@cs.utk.edu>
%    Department of Computer Science
%    University Tennessee at Knoxville
%    104 Ayres Hall
%    Knoxville, TN 37996
%    USA

%  Do not remove the %begin{latexonly} and %end{latexonly} lines (used by 
%  star2html to signify raw TeX that latex2html cannot process).
%begin{latexonly}
\makeatletter
\def\makeinnocent#1{\catcode`#1=12 }
\def\csarg#1#2{\expandafter#1\csname#2\endcsname}

\def\ThrowAwayComment#1{\begingroup
    \def\CurrentComment{#1}%
    \let\do\makeinnocent \dospecials
    \makeinnocent\^^L% and whatever other special cases
    \endlinechar`\^^M \catcode`\^^M=12 \xComment}
{\catcode`\^^M=12 \endlinechar=-1 %
 \gdef\xComment#1^^M{\def\test{#1}
      \csarg\ifx{PlainEnd\CurrentComment Test}\test
          \let\html@next\endgroup
      \else \csarg\ifx{LaLaEnd\CurrentComment Test}\test
            \edef\html@next{\endgroup\noexpand\end{\CurrentComment}}
      \else \let\html@next\xComment
      \fi \fi \html@next}
}
\makeatother

\def\includecomment
 #1{\expandafter\def\csname#1\endcsname{}%
    \expandafter\def\csname end#1\endcsname{}}
\def\excludecomment
 #1{\expandafter\def\csname#1\endcsname{\ThrowAwayComment{#1}}%
    {\escapechar=-1\relax
     \csarg\xdef{PlainEnd#1Test}{\string\\end#1}%
     \csarg\xdef{LaLaEnd#1Test}{\string\\end\string\{#1\string\}}%
    }}

%  Define environments that ignore their contents.
\excludecomment{comment}
\excludecomment{rawhtml}
\excludecomment{htmlonly}

%  Hypertext commands etc. This is a condensed version of the html.sty
%  file supplied with LaTeX2HTML by: Nikos Drakos <nikos@cbl.leeds.ac.uk> &
%  Jelle van Zeijl <jvzeijl@isou17.estec.esa.nl>. The LaTeX2HTML documentation
%  should be consulted about all commands (and the environments defined above)
%  except \xref and \xlabel which are Starlink specific.

\newcommand{\htmladdnormallinkfoot}[2]{#1\footnote{#2}}
\newcommand{\htmladdnormallink}[2]{#1}
\newcommand{\htmladdimg}[1]{}
\newenvironment{latexonly}{}{}
\newcommand{\hyperref}[4]{#2\ref{#4}#3}
\newcommand{\htmlref}[2]{#1}
\newcommand{\htmlimage}[1]{}
\newcommand{\htmladdtonavigation}[1]{}
\newcommand{\latexhtml}[2]{#1}
\newcommand{\html}[1]{}

%  Starlink cross-references and labels.
\newcommand{\xref}[3]{#1}
\newcommand{\xlabel}[1]{}

%  LaTeX2HTML symbol.
\newcommand{\latextohtml}{{\bf LaTeX}{2}{\tt{HTML}}}

%  Define command to re-centre underscore for Latex and leave as normal
%  for HTML (severe problems with \_ in tabbing environments and \_\_
%  generally otherwise).
\newcommand{\latex}[1]{#1}
\newcommand{\setunderscore}{\renewcommand{\_}{{\tt\symbol{95}}}}
\latex{\setunderscore}

%  Redefine the \tableofcontents command. This procrastination is necessary 
%  to stop the automatic creation of a second table of contents page
%  by latex2html.
\newcommand{\latexonlytoc}[0]{\tableofcontents}

% -----------------------------------------------------------------------------
%  Debugging.
%  =========
%  Remove % on the following to debug links in the HTML version using Latex.

% \newcommand{\hotlink}[2]{\fbox{\begin{tabular}[t]{@{}c@{}}#1\\\hline{\footnotesize #2}\end{tabular}}}
% \renewcommand{\htmladdnormallinkfoot}[2]{\hotlink{#1}{#2}}
% \renewcommand{\htmladdnormallink}[2]{\hotlink{#1}{#2}}
% \renewcommand{\hyperref}[4]{\hotlink{#1}{\S\ref{#4}}}
% \renewcommand{\htmlref}[2]{\hotlink{#1}{\S\ref{#2}}}
% \renewcommand{\xref}[3]{\hotlink{#1}{#2 -- #3}}
%end{latexonly}
% -----------------------------------------------------------------------------
% ? Document specific \newcommand or \newenvironment commands.
% ? End of document specific commands
% -----------------------------------------------------------------------------
%  Title Page.
%  ===========
% \renewcommand{\thepage}{\roman{page}}
\begin{document}
\thispagestyle{empty}

%  Latex document header.
%  ======================
\begin{latexonly}
   CCLRC / {\sc Rutherford Appleton Laboratory} \hfill {\bf \stardocname}\\
   {\large Particle Physics \& Astronomy Research Council}\\
   {\large Starlink Project\\}
   {\large \stardoccategory\ \stardocnumber}
   \begin{flushright}
   \stardocauthors\\
   \stardocdate
   \end{flushright}
   \vspace{-4mm}
   \rule{\textwidth}{0.5mm}
   \vspace{5mm}
   \begin{center}
   {\Large\bf \stardoctitle}
   \end{center}
   \vspace{5mm}

% ? Heading for abstract if used.
%  \vspace{10mm}
%  \begin{center}
%     {\Large\bf Abstract}
%  \end{center}
% ? End of heading for abstract.
\end{latexonly}

%  HTML documentation header.
%  ==========================
\begin{htmlonly}
   \xlabel{}
   \begin{rawhtml} <H1> \end{rawhtml}
      \stardoctitle
   \begin{rawhtml} </H1> \end{rawhtml}

% ? Add picture here if required.
% ? End of picture

   \begin{rawhtml} <P> <I> \end{rawhtml}
   \stardoccategory\ \stardocnumber \\
   \stardocauthors \\
   \stardocdate
   \begin{rawhtml} </I> </P> <H3> \end{rawhtml}
      \htmladdnormallink{CCLRC}{http://www.cclrc.ac.uk} /
      \htmladdnormallink{Rutherford Appleton Laboratory}
                        {http://www.cclrc.ac.uk/ral} \\
      \htmladdnormallink{Particle Physics \& Astronomy Research Council}
                        {http://www.pparc.ac.uk} \\
   \begin{rawhtml} </H3> <H2> \end{rawhtml}
      \htmladdnormallink{Starlink Project}{http://star-www.rl.ac.uk/}
   \begin{rawhtml} </H2> \end{rawhtml}
   \htmladdnormallink{\htmladdimg{source.gif} Retrieve hardcopy}
      {http://star-www.rl.ac.uk/cgi-bin/hcserver?\stardocsource}\\

%  HTML document table of contents. 
%  ================================
%  Add table of contents header and a navigation button to return to this 
%  point in the document (this should always go before the abstract \section). 
  \label{stardoccontents}
  \begin{rawhtml} 
    <HR>
    <H2>Contents</H2>
  \end{rawhtml}
  \newcommand{\latexonlytoc}[0]{}
  \htmladdtonavigation{\htmlref{\htmladdimg{contents_motif.gif}}
        {stardoccontents}}

% ? New section for abstract if used.
% \section{\xlabel{abstract}Abstract}
% ? End of new section for abstract

\end{htmlonly}

% -----------------------------------------------------------------------------
% ? Document Abstract. (if used)
%  ==================
% \stardocabstract
% ? End of document abstract
% -----------------------------------------------------------------------------
% ? Latex document Table of Contents (if used).
%  ===========================================
% \newpage
%  \begin{latexonly}
%  \setlength{\parskip}{0mm}
%  \latexonlytoc
%  \setlength{\parskip}{\medskipamount}
%  \markright{\stardocname}
% \end{latexonly}
% ? End of Latex document table of contents
% -----------------------------------------------------------------------------
% \newpage
% \renewcommand{\thepage}{\arabic{page}}
% \setcounter{page}{1}

\section{Introduction}

The STARX Starlink Environment package provides a system independent
method of linking with the X libraries, to relieve the other
subroutines libraries and applications of having to worry about the
various complications involved.

STARX allows the X Library linking strategy to be changed easily, without 
changing any of the other Starlink Environment components.

The system described has been tested for DEC AXP/Digital Unix, Sun
Sparc/SunOS4, Sun Sparc/Solaris (version 2.3, 2.4 and 2.5), and Intel
PC/Linux.  It does not appear to hinder dynamic linking where
available.

\section{Why have STARX?}

\subsection{Solaris 2}

When we came to build the USSC under Solaris 2.4, it quickly became
apparent that the simple strategy for linking with the X libraries
employed under Solaris 2.3 and on other platforms didn't work.  Changes
under Solaris 2.4 meant that various of the X libraries are
interdependent on each other.  

This is fine using the default dynamic linking --- the linker is
capable of resolving such dependencies.  However, for static linking,
the norm for Starlink applications, the linker cannot successfully
resolve such dependencies. 

One way round this problem is to specify the libraries concerned twice
each in the chain of libraries for linking. However, because of the
way the Starlink link scripts work, it is not possible to do this --- 
multiple occurrences of library names are removed to shorten the number
of linker arguments.

STARX resolves the problem by using alias names for the additional 
occurrences for the libraries concerned.

\subsection{PCs with Linux}

The port of the USSC to Linux on Intel PCs required another `fix' to 
accommodate X Library linking.  

Under Linux systems, the X Libraries are in a different place than on 
most Unix systems, and are not searched automatically by the linker.

The Linux version of STARX provides the path for the linker to find the 
X libraries, in addition to the name of the library required.

\section{How it works}

There are two components to STARX: the link script, and where
necessary, library aliases.

\subsection{Link Script}

The link script {\tt starX\_link} simply takes the form of an {\tt echo}
command with a string of required libraries.  The default version is
this:

\begin{quote}{\tt
echo -lX11}
\end{quote}

which is all that is necessary for DEC Alpha/Digital Unix and Sun Sparc/SunOS4 
systems. 

For Solaris 2.5, the script is this:

\begin{quote}{\tt
echo -lXext -lX11 -lstarXext -lstarX11}
\end{quote}

The last two, {\tt  -lstarXext} and {\tt -lstarX11} are the alias names for
the two X libraries that need to appear twice, {\tt -lXext} and {\tt -lX11}.

For Linux on Intel ix86 systems, the script is this:

\begin{quote}{\tt
echo -L/usr/X11/lib -lX11}
\end{quote}

\subsection{Library Aliases}

The alias names for the libraries are provided by soft links to the 
actual libraries.  These are created in the {\tt /star/lib} directory
(static library) and {\tt /star/share} directory (shared libraries).

For Solaris 2.4, the links created are:

\begin{verbatim}
    /star/lib/libstarX11.a -> /usr/openwin/lib/libX11.a
    /star/lib/libstarXext.a -> /usr/openwin/lib/libXext.a
    /star/share/libstarX11.so -> /usr/openwin/lib/libX11.so
    /star/share/libstarXext.so -> /usr/openwin/lib/libXext.so
\end{verbatim}

The default is to soft link to the Openwindows libraries, though other X
libraries may be used (if installed) by changing the definition of 
{\tt XLIBS\_PATH}.  This should be done by setting the environment variable
in the {\tt mk} script and NOT by changing the default in the {\tt makefile}.

There are no library alias requirements for the other USSC supported systems.

\subsection{Linking}

Linking with the X libraries is typically done in conjunction with linking 
with one or more of the Starlink Environment graphics packages.  Where this
is the case, users will not need to take any action to use the new
linking system, since the graphics systems have been modified to call
the new {\tt starX\_link} script where appropriate.

Where a program needs to link with just the X libraries, a command of
this form should be used:

\begin{quote}{\tt
cc myprog.c -o myprog -L/star/lib `starX\_link` ....}
\end{quote}

The {\tt -L/star/lib `starX\_link`} section pulls in the X libraries in
the correct manner for the system.

\end{document}
