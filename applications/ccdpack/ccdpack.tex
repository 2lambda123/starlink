\documentclass[11pt]{article}
\pagestyle{myheadings}

% -----------------------------------------------------------------------------
\newcommand{\stardoccategory}  {Hypertext routine help}
\newcommand{\stardocinitials}  {HRH}
\newcommand{\stardocsource}    {ccdpack}
\newcommand{\stardocnumber}    {1.0}
\newcommand{\stardocauthors}   {P.W. Draper, M.B. Taylor}
\newcommand{\stardocdate}      {July 2001}
\newcommand{\stardoctitle}     {Hypertext help}
% -----------------------------------------------------------------------------

\newcommand{\stardocname}{\stardocinitials /\stardocnumber}
\markright{\stardocname}
\setlength{\textwidth}{160mm}
\setlength{\textheight}{230mm}
\setlength{\topmargin}{-2mm}
\setlength{\oddsidemargin}{0mm}
\setlength{\evensidemargin}{0mm}
\setlength{\parindent}{0mm}
\setlength{\parskip}{\medskipamount}
\setlength{\unitlength}{1mm}

% -----------------------------------------------------------------------------
%  Hypertext definitions.
%  ======================
%  These are used by the LaTeX2HTML translator in conjuction with star2html.

%  Comment.sty: version 2.0, 19 June 1992
%  Selectively in/exclude pieces of text.
%
%  Author
%    Victor Eijkhout                                      <eijkhout@cs.utk.edu>
%    Department of Computer Science
%    University Tennessee at Knoxville
%    104 Ayres Hall
%    Knoxville, TN 37996
%    USA

%  Do not remove the %\begin{rawtex} and %\end{rawtex} lines (used by
%  star2html to signify raw TeX that latex2html cannot process).
%\begin{rawtex}
\makeatletter
\def\makeinnocent#1{\catcode`#1=12 }
\def\csarg#1#2{\expandafter#1\csname#2\endcsname}

\def\ThrowAwayComment#1{\begingroup
    \def\CurrentComment{#1}%
    \let\do\makeinnocent \dospecials
    \makeinnocent\^^L% and whatever other special cases
    \endlinechar`\^^M \catcode`\^^M=12 \xComment}
{\catcode`\^^M=12 \endlinechar=-1 %
 \gdef\xComment#1^^M{\def\test{#1}
      \csarg\ifx{PlainEnd\CurrentComment Test}\test
          \let\html@next\endgroup
      \else \csarg\ifx{LaLaEnd\CurrentComment Test}\test
            \edef\html@next{\endgroup\noexpand\end{\CurrentComment}}
      \else \let\html@next\xComment
      \fi \fi \html@next}
}
\makeatother

\def\includecomment
 #1{\expandafter\def\csname#1\endcsname{}%
    \expandafter\def\csname end#1\endcsname{}}
\def\excludecomment
 #1{\expandafter\def\csname#1\endcsname{\ThrowAwayComment{#1}}%
    {\escapechar=-1\relax
     \csarg\xdef{PlainEnd#1Test}{\string\\end#1}%
     \csarg\xdef{LaLaEnd#1Test}{\string\\end\string\{#1\string\}}%
    }}

%  Define environments that ignore their contents.
\excludecomment{comment}
\excludecomment{rawhtml}
\excludecomment{htmlonly}
%\end{rawtex}

%  Hypertext commands etc. This is a condensed version of the html.sty
%  file supplied with LaTeX2HTML by: Nikos Drakos <nikos@cbl.leeds.ac.uk> &
%  Jelle van Zeijl <jvzeijl@isou17.estec.esa.nl>. The LaTeX2HTML documentation
%  should be consulted about all commands (and the environments defined above)
%  except \xref and \xlabel which are Starlink specific.

\newcommand{\htmladdnormallinkfoot}[2]{#1\footnote{#2}}
\newcommand{\htmladdnormallink}[2]{#1}
\newcommand{\htmladdimg}[1]{}
\newenvironment{latexonly}{}{}
\newcommand{\hyperref}[4]{#2\ref{#4}#3}
\newcommand{\htmlref}[2]{#1}
\newcommand{\htmlimage}[1]{}
\newcommand{\htmladdtonavigation}[1]{}

%  Starlink cross-references and labels.
\newcommand{\xref}[3]{#1}
\newcommand{\xlabel}[1]{}

%  LaTeX2HTML symbol.
\newcommand{\latextohtml}{{\bf LaTeX}{2}{\tt{HTML}}}

%  Define command to recentre underscore for Latex and leave as normal
%  for HTML (severe problems with \_ in tabbing environments and \_\_
%  generally otherwise).
\newcommand{\latex}[1]{#1}
\newcommand{\setunderscore}{\renewcommand{\_}{{\tt\symbol{95}}}}
\latex{\setunderscore}

%  Redefine the \tableofcontents command. This procrastination is necessary
%  to stop the automatic creation of a second table of contents page
%  by latex2html.
\newcommand{\latexonlytoc}[0]{\tableofcontents}

% -----------------------------------------------------------------------------
%  Debugging.
%  =========
%  Un-comment the following to debug links in the HTML version using Latex.

% \newcommand{\hotlink}[2]{\fbox{\begin{tabular}[t]{@{}c@{}}#1\\\hline{\footnotesize #2}\end{tabular}}}
% \renewcommand{\htmladdnormallinkfoot}[2]{\hotlink{#1}{#2}}
% \renewcommand{\htmladdnormallink}[2]{\hotlink{#1}{#2}}
% \renewcommand{\hyperref}[4]{\hotlink{#1}{\S\ref{#4}}}
% \renewcommand{\htmlref}[2]{\hotlink{#1}{\S\ref{#2}}}
% \renewcommand{\xref}[3]{\hotlink{#1}{#2 -- #3}}
% -----------------------------------------------------------------------------
%  Add any document specific \newcommand or \newenvironment commands here

\newcommand{\butt}[1]{{\bf \tt #1}}
\newcommand{\menu}[1]{{\bf \em #1}}
\newcommand{\wlab}[1]{{\bf #1}}
\newcommand{\text}[1]{{\tt #1}}

%  Don't use the fbox options as this breaks up the text line too much.
%\newcommand{\butt}[1]{\fbox{\tt #1}}
%\newcommand{\menu}[1]{\fbox{\bf \em #1}}
%\newcommand{\wlab}[1]{\fbox{\bf #1}}

%  Don't use the colour option as this looks dreadful in netscape.
%\definecolor{lightgrey}{gray}{.82}
%\newcommand{\butt}[1]{\colorbox{lightgrey}{#1}}
%\newcommand{\menu}[1]{\colorbox{lightgrey}{\em #1}}
% -----------------------------------------------------------------------------
%  Title Page.
%  ===========
\begin{document}
\thispagestyle{empty}

%  Latex document header.
%  ======================
\begin{latexonly}
   CCLRC / {\sc Rutherford Appleton Laboratory} \hfill {\bf \stardocname}\\
   {\large Particle Physics \& Astronomy Research Council}\\
   {\large Starlink Project\\}
   {\large \stardoccategory\ \stardocnumber}
   \begin{flushright}
   \stardocauthors\\
   \stardocdate
   \end{flushright}
   \vspace{-4mm}
   \rule{\textwidth}{0.5mm}
   \vspace{5mm}
   \begin{center}
   {\Large\bf \stardoctitle}
   \end{center}
   \vspace{5mm}

%  Add heading for abstract if used.
   \vspace{10mm}
   \begin{center}
      {\Large\bf Description}
   \end{center}
\end{latexonly}

%  HTML documentation header.
%  ==========================
\begin{htmlonly}
   \xlabel{}
   \begin{rawhtml} <H1> \end{rawhtml}
      \stardoctitle
   \begin{rawhtml} </H1> \end{rawhtml}

%  Add picture here if required.

   \begin{rawhtml} <P> <I> \end{rawhtml}
   \stardoccategory \stardocnumber \\
   \stardocauthors \\
   \stardocdate
   \begin{rawhtml} </I> </P> <H3> \end{rawhtml}
      \htmladdnormallink{CCLRC}{http://www.cclrc.ac.uk} /
      \htmladdnormallink{Rutherford Appleton Laboratory}
                        {http://www.cclrc.ac.uk/ral} \\
      Particle Physics \& Astronomy Research Council \\
   \begin{rawhtml} </H3> <H2> \end{rawhtml}
      \htmladdnormallink{Starlink Project}{http://star-www.rl.ac.uk/}
   \begin{rawhtml} </H2> \end{rawhtml}

%  HTML document table of contents.
%  ================================
%  Add table of contents header and a navigation button to return to this
%  point in the document (this should always go before the abstract \section).
  \label{stardoccontents}
  \begin{rawhtml}
    <HR>
    <H2>Contents</H2>
  \end{rawhtml}
  \renewcommand{\latexonlytoc}[0]{}
  \htmladdtonavigation{\htmlref{\htmladdimg{contents_motif.gif}}
        {stardoccontents}}

%  Start new section for abstract if used.
  \section{\xlabel{abstract}Abstract}

\end{htmlonly}

% -----------------------------------------------------------------------------
%  Document Abstract. (if used)
%  ==================
This document provides any additional hypertext help required by
\xref{CCDPACK}{sun139}{} beyond that available in SUN/139.

Primarily this consists of the window and context sensitive help
for the \xref{XREDUCE}{sun139}{XREDUCE} application.

The XREDUCE GUI is specifically designed to help the inexperienced or
occasional reducer of CCD data (although others will also find it of
use). These aims are met by providing an easy to use, X based,
graphical interface that features contextual help (this document) and
that limits options to those of immediate relevance. It concentrates
on data organization and the definition of any CCD characteristics
rather than on the nature and control of the core CCDPACK reduction
programs.

The reduction of the actual data is separate to the GUI and uses the
automated scheduling facilities of CCDPACK.


% -----------------------------------------------------------------------------
%  Latex document Table of Contents (if used).
%  ===========================================
\begin{latexonly}
  \setlength{\parskip}{0mm}
  \latexonlytoc
  \setlength{\parskip}{\medskipamount}
  \markright{\stardocname}
\end{latexonly}
% -----------------------------------------------------------------------------
\section{This package \xlabel{OnVersion}}
\begin{center}
\begin{latexonly}
{\Huge \bf  CCDPACK \\
\LARGE Data reduction package \\
\Large \em Version 2.0 \\
}
\end{latexonly}
\begin{htmlonly}
   \htmladdimg{ccdv2.gif} \\
\end{htmlonly}
\stardocdate
\end{center}

\section{How to get context sensitive help \xlabel{OnContext}}
Context sensitive help is obtained by putting the mouse pointer over
the part of the window that you're interested in and pressing either
the F1,F2 or Help keys. F2 may be your only option on some keyboards
(OpenWindows).

\section{Help Menu \xlabel{OnHelp}}
Use the \menu{Help} menu to get advice on using this window.

\section{File Menu \xlabel{OnFile}}
The \menu{File} menu can be used to \menu{Exit} this program and
usually also contains items for closing the window (same effect as a
\butt{Cancel} button) or accept the window's contents and act
upon it (same as a \butt{OK} button).

\section{Starting a CCDPACK reduction \xlabel{CCDMainWindow}}
This window displays buttons and menus that allow you to get
ready to perform an automated reduction. The sequence of operations
that you need to perform depends on the type of data you have.

\begin{rawhtml}
<P>
<IMG SRC="xreduce.gif">
<P>
\end{rawhtml}

Under the \menu{Options} menu there is a \menu{Set Detector...}
item. This will list the known CCD detectors. If you're in luck the
one you've obtained your data with will be listed here. If so select it.

The next action that you must perform is to choose one or both of the
buttons \butt{General Options} and \butt{CCD Characteristics}. If you
have been able to choose a known detector with an import table (this
is shown by the text \text{(table)} after its name) then you probably
only need to visit the \butt{General Options} window. Otherwise you
should visit the \butt{CCD Characteristics} window as well.

After completing the \wlab{Configuration} section you now need to
proceed to \wlab{Data Import}. In this section you identify all the
data frames to be processed.

If you have been able to choose an import table then visit the
\butt{Using FITS Headers} window. If not (unless you have an import
table of your own) you must visit the \butt{Manual Organization}
window.

The final section is \wlab{Reduction}. Just press the \butt{Setup and
Run} button to visit the final window.

If you're new to CCD reduction you might want to learn a little more 
about the \xref{basic principles}{sun139}{CCDprinciples} before 
proceeding.

\subsection{Configuration \xlabel{CCDMainConfiguration}}
\begin{itemize}
\item \butt{General Options}: the window activated by this button
allows you to do a general configuration. Things like the name of the
log file used by any programs and whether to generate errors for your
data.
\item \butt{CCD Characteristics}: this window is used to define things
about the CCD your data was taken with. Things like the positions of
any bias strips and the useful CCD area.
\item \butt{Bypass Stage}: this button allows you to proceed to
\wlab{Data Import}. This is probably only of use if you have
\htmlref{restored the state}{CCDMainMenu}
 of the interface from a previous time.
\end{itemize}

\subsection{Data Import \xlabel{CCDMainDataImport}}
After configuration you now need to identify your data.
\begin{itemize}
\item \butt{Using FITS Headers}:
if you have been able to choose an existing import table (or you have
one of your own) then you can just name all your data frames in the
window created by this button.
\item \butt{Manual Organization}:
otherwise you will need to organize your data into its different types
(target, bias, flatfield etc.) by hand in this window.
\item \butt{Bypass Stage}: this button allows you to proceed to
\wlab{Reduction} stage.
This is probably only of use if you have
\htmlref{restored the state}{CCDMainMenu}
of the interface from a previous time (but note that you'll also need to be
using the same data, if not you must not bypass this stage).
\end{itemize}

\subsection{Reduction \xlabel{CCDMainReduction}}
After \wlab{Configuration} and \wlab{Data Import} have been completed
successfully you're now in a position to reduce your data. Press
\butt{Setup and Run} to see the final options and start a reduction
(there is a brief pause at this point while the range of debiassing
options available to you are assessed).

\subsection{Known Data Files \xlabel{CCDMainKnownData}}
The \wlab{Known Data Files} section displays the list of currently
known data files. Due to a limitation in the XREDUCE program it may be
necessary to press the \butt{Update} button to see this list.

\begin{itemize}
\item \butt{Exit}: conditionally exits the program closing the GUI.
\item \butt{Update}: updates the list of known data files.
\end{itemize}

\subsection{Menus \xlabel{CCDMainMenu} \label{CCDMainMenu}}
The \menu{Options} menu provides the following items.

\begin{itemize}
\item \menu{Set detector...}:
this option allows you to choose from a list of known detectors.

\item \menu{Save current state...} \& \menu{Restore state...}:
these options allow you to save and restore the `state' of the GUI.
The `state' is the current internal configuration of the GUI (it
specifies such things as which data frames have been imported, fonts,
colours etc.). This means that you can stop this application at any
time and re-start it where you left off. Or that you can re-configure
it for different telescopes/detectors etc. trivially.  The general
interface configuration options are described in the main help for
\xref{XREDUCE}{sun139}{XREDUCE}.

\item \menu{Monitor output from tasks}:
this allows you to toggle the monitoring of any output from CCDPACK
programs that are controlled by the GUI (unless things are going wrong
you shouldn't normally be interested in this).
\end{itemize}

Use the \menu{Help} menu to get advice on using this window.

The \menu{File} menu allows you to exit from this program.


\section{Selecting a known detector \xlabel{CCDSetDetectorWindow}
         \label{CCDSetDetectorWindow}}
This window displays a list of files that contain ``setups'' for
all the different CCDs that are known, together with one-line
descriptions. Select a line by pressing mouse button 1 over it
and selecting \butt{OK}, or by double clicking over it.

There are two different ``setup'' types available. These are
distinguished by \text{(setup)} and \text{(table)} at the end of their
descriptions.

The most complete setup type are import tables. They contain a
description of how to interpret the FITS header information stored in
your data. Tables have the capability to work out the frame type
(target,bias,flat) of each of your files, so you will only need to
supply the names of the files to process (in \butt{Using FITS Headers}
in the main window). A complete description of the
\xref{table format}{sun139}{TableFormat} is available elsewhere.

Ordinary setups are less comprehensive and contain only the physical
descriptions of the CCD (things like where the bias strips are, the
useful CCD area and the CCD noise characteristics). If you choose one
of these then you will need to indicate which sort of frame each of
your files is at a later stage (in \butt{Manual Organization} in the
main window).

\subsection{Menus \xlabel{CCDSetDetectorMenu}}

The \menu{Options} menu contains one entry \menu{View file
contents...}. Viewing the file contents may allow you to better
determine if a setup or table is correct for your data.

Use the \menu{Help} menu to get advice on using this window.

The \menu{File} menu allows you to exit from this program and close or
accept the window.

\section{Selecting an existing file \xlabel{CCDGetFileNameWindow}}
This window allows you to select a file from a list in a directory.

The \wlab{Directory:} entry shows the current directory. You can edit this
and then move to a new directory by pressing Return.

The \wlab{File Filter:} entry shows the pattern used to filter the
filenames in the current directory (patterns are similar to those used
by \text{ls}). You can apply the filter by either pressing return in
this entry or by pressing the \butt{Filter} button.

The list on the left-hand side shows all the directories that are in
the current directory. You move into a new directory by double
clicking on its name. \text{"../"} moves up one directory.

The list on the right-hand side shows all the files that match the filter
pattern in the current directory. Select a file by clicking its name and
pressing \butt{OK}, or by double clicking.

The \wlab{Name of selected file:} entry shows the name of the chosen file.
You could just type the filename here and accept it by pressing Return or
\butt{OK}.

\subsection{Menus \xlabel{CCDGetFileNameMenu}}

The \menu{Options} menu contains a list of the directories that you
have visited (as well as the current and your home directory) and can
be used as a shortcut.

Use the \menu{Help} menu to get advice on using this window.

The \menu{File} menu allows you to exit from this program and close or
accept the window.

\section{Creating a new file \xlabel{CCDNewFileNameWindow}}
This window allows you to give the name of a file, that may or may not
already exist. Just type in the name of the new file (if not already
shown) and press Return or \butt{OK} to accept.

\subsection{Menus \xlabel{CCDNewFileNameMenu}}
The \menu{Options} contains a single item that allows you to choose
from a list of files that already exist.

Use the \menu{Help} menu to get advice on using this window.

The \menu{File} menu allows you to exit from this program and close or
accept the window.

\section{Setting\xlabel{CCDSetGenGlobalsWindow}the general reduction
         parameters}
In this window you define general features of the reduction. Briefly
the meaning of the sections in this window are.

\begin{itemize}
\item \wlab{Keep application log:} \& \wlab{Name of logfile:}.
The programs that are used by XREDUCE and those that do the actual
reduction can write their output to a CCDPACK specific log file. If
you want them to do this then select the \text{true} button and give
a name for the logfile.

\item \wlab{Look for saturated pixels:}. If your data could have pixels
that are saturated (and the level at which saturation occurs is known)
then set this value \text{true}.

\item \wlab{Flag saturated pixels using BAD value:}. There is more than
one way that CCDPACK can record the fact that a pixel is saturated.
Just leave this option at \text{true} unless you know that this method
is a problem for you.  A \xref{complete description}{sun139}{CCDsaturate}
of the various methods available is given in the main documentation.


\item \wlab{Generate data errors:}. This option tells CCDPACK to
generate errors for your data (it derives these from a consideration
of the readout noise and the Poissonian nature of the electron count).
These errors are stored as variances (the error squared) and can be
propagated during processing.  The down side of this is that storing
errors requires twice the disk space you'd normally require for your
data.

\item \wlab{Preserve data types:}. If your data are stored in a
non-floating point format then CCDPACK can either convert them to
floating point or do its processing in their data type. It may be
occasionally useful to convert the data type to a more precise one, if
you suspect that this will be the case for your data then select
\text{false}.

\end{itemize}

Press \butt{OK} when you're happy with the values.

\subsection{Menus \xlabel{CCDSetGenGlobalsMenu}}

The \menu{Options} menu provides the following items.
\begin{itemize}
\item \menu{Restore setup...} \& \menu{Save setup...}. The values
shown in this window and the \butt{CCD Characteristics} window can be
stored in a file and used again. This allows you to define setups for
commonly used CCDs (some of these are available as part of CCDPACK,
see the
\htmlref{\menu{Set detector...}}{CCDSetDetectorWindow}
item in the \menu{Options} menu in the main window).

\item \menu{Select logfile from existing files...}. This allows you to
browse files that already exist.

\end{itemize}

Use the \menu{Help} menu to get advice on using this window.

The \menu{File} menu allows you to exit from this program or
accept the window.

\section{Setting the CCD characteristics \xlabel{CCDSetCCDGlobalsWindow}
         \label{CCDSetCCDGlobalsWindow}}

In this window you need to define the CCD characteristics necessary
for the type of reduction to be performed. If you're unsure of what
this might entail then see the description of the
\xref{principles of CCD reduction}{sun139}{CCDprinciples}
and a section describing a little more about
\xref{XREDUCE}{sun139}{usingxreduce},
or alternatively push on.

\begin{itemize}
\item \wlab{Extent of useful detector area:}, \wlab{Readout direction:}
      \& \wlab{Bounds of bias strips (1 or 2 pairs)}:.
These values define the part of the CCD data that is actually used to
record data, where the bias strips of your data (if any) are and the
direction between the bias strips.

If you've decided to push on or have no idea if your data have bias
strips and what the ``readout'' direction and useful detector area are
then check under \menu{Options} and select the item
\htmlref{\menu{Display/define CCD geometry...}}{definegeometry}. This
option can also be used to check any values that you have.

\item \wlab{Analogue-to-digital conversion factor:}. This parameter is
\xref{described}{sun139}{CCDADC} in the main document. It is used to change
the output data values to electrons and is required if you want to
determine data errors.

\item \wlab{Readout noise (ADUs):}. This is the
\xref{readout noise}{sun139}{CCDrnoise} of your CCD in ADUs. It is only used
if generating data errors.

\item \wlab{Defect mask:}. If your data have fixed
defective parts then you may want to remove them. The creation of a
file to do this is \xref{discussed}{sun139}{datamasks} in the main
document. Leave this entry blank otherwise.

\item \wlab{Deferred charge:}. If you don't know what this parameter
\xref{means}{sun139}{CCDdeferred} then do not set a value.

\item \wlab{Saturated pixel value (ADUs):}. If you selected to detect
saturated pixels then you need to supply the lowest value that a pixel
must have to be saturated.

\item \wlab{Bias level (ADUs):}. If you do not have any bias frames, or
bias strips then it is possible to debias your data by subtracting a
constant. If your data is already debiassed then you can get it
reprocessed by supplying a zero (0) here (you may want to do this
if you want to generate data errors, and/or apply a defect mask).

\end{itemize}

If you're trying to \xref{reduce}{sun139}{IRreduction} some infra-red
(IR) data then it will also need to be debiassed, even if this has
already been done for you at the telescope. If this is the case then
enter 0 for the \wlab{Bias level:} value. Even if your data has been
debiassed you may well want to generate data errors or remove the
glitched pixels, if so you'll need to enter values for the
\wlab{Analogue-to-digital conversion factor:},
\wlab{Readout noise:} and \wlab{Defect mask:} parts.


\subsection{Menus \xlabel{CCDSetCCDGlobalsMenu}}

The \menu{Options} menu provides the following items.

\begin{itemize}
\item \menu{Restore setup...} \& \menu{Save setup...}\,. The values
shown in this window and the \butt{General Options} window can be
stored in a file and used again. This allows you to define setups for
commonly used CCDs (some of these are available as part of CCDPACK,
see the
\htmlref{\menu{Set detector...}}{CCDSetDetectorWindow}
item in the \menu{Options} menu in the main window).

\item \menu{Display/define CCD geometry...}\,. \label{definegeometry}
This option allows you to interactively define where the bias strips
(if any) and useful area of your data are. First you'll need to select
a raw data frame (not a bias frame) for display.  This option can also
be used to check that the current values for the bias strip positions
and useful area are correct.

\item \menu{Select MASK from existing files...}. This allows you to
browse any existing files to select a mask file.
\end{itemize}

Use the \menu{Help} menu to get advice on using this window.

The \menu{File} menu allows you to exit from this program or
accept the window.

\section{Examining or setting the CCD geometries \xlabel{CCDGeometryWindow}}
This window allows you to examine or define the geometrical
characteristics of your CCD data. The first thing that you need to do
before proceeding is select a data frame to display. This should be a
one where the various regions are clearly defined, so flatfields and
target frames are the more appropriate.

When an image is displayed you can stroke out the positions of the
bias strips and the useful part of the CCD. To define a region select
a button from the \wlab{Define:} section (such as \text{Bias strip 1})
and set the upper left hand point of this region by pressing mouse
button 1, now drag to the right without releasing the button.

To delete a region press mouse button 3 over it, or select its named
button in the \wlab{Define:} list and press the \butt{Remove Region}
button. If it's useful the image can be coloured using the
\butt{Colour} button (and transformed back to greyscale using \butt{Grey}).

The bias strips should be seen as dark regions down the side of the
image, these usually either run down the sides (or tops) of the image
and sometimes run along the sides and either the top or bottom, i.e.

\begin{rawhtml}
<IMG ALIGN=CENTER SRC="geom1.gif">
<P>
or
<P>
<IMG ALIGN=CENTER SRC="geom2.gif">
<P>
\end{rawhtml}

It is often recommended that you only use the strip on the left-hand
side as the overscan (the right-hand strip) may be contaminated with
residual charge. If you decide to use the right-hand strip check that
it is not contaminated first (this should be visible as a gradient in
the intensity level, use the \menu{Options} menu item
\menu{Change display range...} to choose a suitable intensity range
to explore this effect).

\subsection{Selecting an image for display \xlabel{CCDGeometryNDF}}
The displayed image can be changed either by typing a new name into
the \wlab{Current image:} entry or by browsing from a list of files in a
directory using the \menu{Option} menu item \menu{Change image...}\,.

\subsection{Controlling the image \xlabel{CCDGeometryGwm}}
The displayed image can be controlled in several different ways. The
most obvious is by increasing or decreasing its size using the
\butt{Enlarge} and \butt{Shrink} buttons. The image can then be
scrolled using the scrollbars or by pressing and holding down mouse
button 2 over the image and then moving the mouse. The image can be
made to fit the size of the window by selecting the \menu{Resize image
when window changes} item in the \menu{Options} menu and then
changing the window size to that required.

\subsection{Menu \xlabel{CCDGeometryMenu}}
The \menu{Options} menu contains the following items.
\begin{itemize}
\item \menu{Set outline colour...} \& \menu{Set stipple colour...}\,.
These items create a window in which you can select a new colour for
the region outline and stipple (the fill colour). The window is fairly
self-explanatory.
\item \menu{Resize image when window changes}.
If you select this item then the image will be redrawn to fit the window
if it is resized (note this cancels any size changes made using the
\butt{Enlarge} and \butt{Shrink} buttons).
\item \menu{Change image...}\,.
This creates a new window in which you can browse any existing files
for a new image to display.
\item \menu{Change display range...}\,.
This item creates a new window in which you can change the percentile
display range of the image. Use this if the image doesn't show the
regions you want to define sufficiently well.
\end{itemize}

Use the \menu{Help} menu to get advice on using this window.

\section{Manual organization \xlabel{CCDNDFOrganizeWindow}
         \label{CCDNDFOrganizeWindow}}

In this window you need to describe what type of data you have. The
various elements are fairly obvious.

\begin{itemize}

\item \wlab{Frame types present:} just indicate all the types of data
that you have. \text{targets} are the astronomy data so you should
definitely have some of these, you should probably also have some
\text{flatfields} (unless you're reducing IR data), the rest are more
or less optional depending on your data (i.e. if you have no bias
frames then you should probably have some bias strips).

In this region you also have the opportunity to choose calibration
frames that were not produced by CCDPACK (spectral flatfields are one
possibility) or to re-use any that you've already produced.

\item \wlab{Data have same filter type:} if your data consist of
frames taken with more than one filter then select \text{false} here.

\item \wlab{Filter names:} enter the names of any filters here. The
names should be short (like B,V,R) and be separated by commas (the
names may not include spaces).

\item \wlab{Use targets as possible flatfields:} if you do not have
any flatfields (this is typical for IR data) then you may want to use
your target frames to produce flatfields (this is done by median
stacking, and assumes that in your data the objects are moved about the
frame so that sky is the typical contribution at any point).

\item \wlab{Master bias has a mean of zero:} if for some reason you
want to use a master bias calibration frame that was produced by
another package, or by a previous reduction, then you'll need to
indicate whether the mean of the data values in the frame are zero or
not (this effects the way that debiassing is done).

\item \wlab{Dark times are all the same:} if you have any dark
current calibration frames then select \text{false} here only if they
do not have the same exposure time as your target data (usually dark
frames do have the same exposure). If they do not have the same
exposure times you'll have to supply these later.

\item \wlab{Flash times are all the same:} if you have any pre-flash
calibration frames (very unlikely) then select \text{false} here only
if they do not have the same exposure time as your target data.

\end{itemize}

Press \butt{OK} when you're finished with this window and go on to
supply the names of the different types of data.

\section{Organizing frames into types \xlabel{CCDNDFDoImportWindow}}
This window allows you to enter files into their different categories.
The categories that you have selected are shown at the top of the
window in the \wlab{Data type:} section, this looks something like.
\begin{rawhtml}
<P>
<IMG SRC="datatype.gif">
<P>
\end{rawhtml}

To organize your files you select one of the options from
\wlab{Data type:} (in the case above \text{TARGET} is selected), and
then choose the files that you require from the \wlab{Files in
directory} list and enter these into the \wlab{Files selected:}
list. You can select files by clicking them, dragging the pointer over
a range (without releasing the mouse button), or using a combination
of shift and control with mouse clicks (control for disjoint
selections) and then transfer the files using the \butt{Add} button,
or you can just double click on a file to transfer it.

After you've selected all the files in the current category, select
another from the \wlab{Data type:} radio buttons and transfer these to
\wlab{Files selected:}.

You can remove files from \wlab{Files selected:} by using the buttons
\butt{Remove} and \butt{Remove all}.

If you have dark or flash count data that do {\bf not} have the exact
exposure times necessary ({\em i.e.} have the same dark exposure time
as the target data), then the \wlab{Files selected:} region will contain
extra fields (these appear after you've entered a file name) in which
you will need to enter the exposure times.

The other parts of this window allow you to select files from other
directories.

The \xlabel{CCDNDFDoImportDirectory}\wlab{Directory:} entry shows the
current directory. You can edit this and then move to a new directory
by pressing Return.

The \xlabel{CCDNDFDoImportFilter}\wlab{File Filter:} entry shows the
pattern used to filter the filenames in the current directory
(patterns are similar to those used by \text{ls}). You can apply the
filter by either pressing return in this entry or by pressing the
\butt{Filter} button.

The \xlabel{CCDNDFDoImportDirectories}\wlab{Directories:} list on the
left-hand side shows all the directories that are in the current
directory. You move into a new directory double clicking on its
name. \text{"../"} moves up one directory.

When \xlabel{CCDNDFDoImportChoice}you have organized all your frames
into the appropriate groups then exit by pressing \butt{OK}. This
enters information into your data about what type they are (and any
information you gave in the \butt{CCD Characteristics} window) and
might take some time. \butt{Cancel} exits without modifying your data.

\subsection{Menus  \xlabel{CCDNDFDoImportMenu}}
The \menu{Options} menu contains a list of the directories that you have
visited (as well as the current and your home directory) and can be
used as a shortcut to visit them.

Use the \menu{Help} menu to get advice on using this window.

The \menu{File} menu allows you to exit from this program and close or
accept the window.

\section{Importing data with FITS headers \xlabel{CCDFITSImportWindow}}
This window allows you to specify a FITS import control table and
the names of the data frames that you want to process.

The import control table should already be shown if you've chosen from the
list of known detectors. Otherwise you should type in the name of the table
into the \wlab{Import Control Table:} entry (you can also browse existing
files by using the \menu{Select import control table from existing files...}
item in the \menu{Options} menu). Import control tables are
\xref{described}{sun139}{TableFormat} in the main documentation and can be
created or inspected using the \menu{Create/inspect import control table...}
item in the \menu{Options} menu.

To select a file for processing you need to transfer it from the
\wlab{Files in directory:} list to the \wlab{Files selected:} list.
You can select files by clicking them, dragging the pointer over
a range (without releasing the mouse button), or using a combination
of shift and control with mouse clicks (control for picking out a set of
frames that are not adjacent) and then transfer the files using the
\butt{Add} button, or you can just double click on a file to transfer it.

You can remove files from \wlab{Files selected:} by selecting the
them as described above (but in the \wlab{Files selected:} list) and
pressing the \butt{Remove} button. \butt{Remove all} removes them all.

The other parts of this window allow you to select files from other
directories.


The\xlabel{CCDFITSImportDirectory} \wlab{Directory:} entry shows the
current directory. You can edit this and then move to a new directory
by pressing Return.


The\xlabel{CCDFITSImportFilter} \wlab{File Filter:} entry shows the
pattern used to filter the filenames in the current directory
(patterns are similar to those used by \text{ls}). You can apply the
filter by either pressing return in this entry or by pressing the
\butt{Filter} button.


The\xlabel{CCDFITSImportDirectories} \wlab{Directories:} list on the
left-hand side shows all the directories that are in the current
directory. You move into a new directory by double clicking on its
name. \text{"../"} moves up one directory.


When\xlabel{CCDFITSImportChoice} you have selected all your frames and
chosen an import table then exit (and enter the translation of the
FITS items into your data) by pressing \butt{OK} (this might take some
time).
\butt{Cancel} exits without modifying your data.

\subsection{Menus  \xlabel{CCDFITSImportMenu}}
The \menu{Options} menu contains a list of the directories that you have
visited (as well as the current and your home directory) and can be
used as a shortcut to visit them. It also contains.
\begin{itemize}
\item \menu{Select import control table from existing files...}, this just
allows you to browse around for the import control table.
\item \menu{Create/inspect import control table...}, this is a complex
option for inspecting the contents of your import control table, best
avoided by novices. It also helps in the creation of new import tables.
\end{itemize}

Use the \menu{Help} menu to get advice on using this window.

The \menu{File} menu allows you to exit from this program and close or
accept the window.

\section{Creating an import table \xlabel{CCDCreateImportTableWindow}}
This window is designed to help in the creation of import control tables.
These tables describe how to interpret FITS headers. The
\xref{format}{sun139}{TableFormat} is described elsewhere (in the main
document) and should be understood before attempting to use this screen.
An understanding of the information that CCDPACK requires from FITS headers
is also necessary. These are also \xref{listed}{sun139}{reductionitems}
in the main document.

Having a complete set of FITS header items to translate isn't necessary,
but if any are missing then you will need to supply these by using the
\htmlref{\butt{CCD characteristics}}{CCDSetCCDGlobalsWindow}
window or you may need to visit
\htmlref{\butt{Manual Organization}}{CCDNDFOrganizeWindow}
to sort your data into its different types.

\subsection{Using a reference frame \xlabel{CCDCreateImportTableReferenceNDF}
            \label{refndf}}
This part of the window helps you to find out what FITS headers are
available in your data.
\begin{rawhtml}
<P>
<IMG SRC="fitsextract.gif">
<P>
\end{rawhtml}
It does this by extracting the names of any headers in a data frame
and listing them. Choose a reference frame either by typing its name
in the \wlab{Reference frame:} entry or by choosing the
\menu{Select reference frame...} item in the \menu{Options} menu.
Clicking on the name of a header item enters it into the \wlab{FITS
item:} entry on the right.

\subsection{Typing FITS headers \xlabel{CCDCreateImportTableFITStypes}}
In this part of the window you should name all the FITS headers that
you are going to use in the table.
\begin{rawhtml}
<P>
<IMG SRC="fitstype.gif">
<P>
\end{rawhtml}
The data type of the headers are also needed (these are used when
extracting the header values). The data types are called
\wlab{HDS type} as these are the native forms that CCDPACK understands. If you
click in the \wlab{HDS type:} entry a list of the known HDS types will
appear. The meanings of the names are fairly obvious (\text{\_INTEGER}
for integers, \text{\_CHAR} for character strings etc.).

To enter an item and its type, type its name into the \wlab{FITS
item:} entry (or select one from the list extracted from your
\htmlref{reference}{refndf} frame), and then press the \butt{Add}
button. To remove an item from the list select it by clicking on its
name and then press the \butt{Remove} button. The \butt{Unique} button
removes any duplicate entries in the list.

It is also possible to just directly edit any entries in the list of
names and types, but take care to spell these correctly.

\subsection{Mapping FITS functions to extension items
            \xlabel{CCDCreateImportTableFunctions}}
This part of the window is used to specify how FITS header item values
should be converted into CCDPACK extension items (the extension is a
part of your data for storing package specific information).
\begin{rawhtml}
<P>
<IMG SRC="fitsfunction.gif">
<P>
\end{rawhtml}

You should understand the
\xref{structure}{sun139}{TableFormat}
of an import table before attempting to use this part. The known items
are listed for convenience down the left hand side and can be selected
and inserted in the \wlab{Extension item:} entry. The transformation
of the FITS values that results in the required CCDPACK extension item
should be typed into the \wlab{FITS function} entry, pressing
\butt{Add} will then insert this into the list.

To remove an entry from the list just select it and press
\butt{Remove}. \butt{Unique} removes any duplicate extension items
from the list.


When\xlabel{CCDCreateImportTableOK} you've finished creating the
import table press the \butt{OK} button to exit. \butt{Cancel} exits
without checking that the work you've done is saved.


\subsection{Menus \xlabel{CCDCreateImportTableMenu}}
The \menu{Options} menu contains the following items.
\begin{itemize}
\item \menu{Select reference frame...}, this allows you to browse files
to select a data frame. The FITS headers in this frame are extracted
and displayed in the \wlab{Reference frame:} entry.
\item \menu{Read an import table file...}, allows you to browse for an
import table. This is then read in and displayed in the main window.
\item \menu{Save import table in file...}, this writes the current
window contents into a file.
\item \menu{Keep later unique values}, this toggle button indicates
whether you want the \butt{Unique} buttons to keep the any versions from
further down the lists, rather than the from nearer the top. This is
useful as the \butt{Add} buttons insert at the end of the lists.
\end{itemize}

Use the \menu{Help} menu to get advice on using this window.

The \menu{File} menu allows you to exit from this program and close or
accept the window.

\section{Starting the reduction \xlabel{CCDReduceWindow}}
This window allows you to set the final configuration options before
scheduling a reduction and running it in the background.
\begin{rawhtml}
<P>
<IMG SRC="ccdreduce.gif">
<P>
\end{rawhtml}


The\xlabel{CCDReduceDebias} range of possible reduction options
available to you are shown in the \wlab{Debias using:} section (in the
image above all the known options are shown, normally your window will
only show a subset of these). You should accept the default method
unless you have good reason not to.

The various options you might have available are.
\begin{itemize}
\item \text{zeroed master bias, offsetting to bias strips}

In this option all your bias frames are combined together (to reduce
the data noise) into a `master bias'. The mean value of this master is
zero (because the bias frames themselves are zeroed before being
combined), and the bias strips are used to estimate an offset, so that
the master bias zero point is shifted to that of the data.

\item \text{unzeroed master bias}

In this option your bias frames are combined without adjusting their
zero points. The resulting master is subtracted from your other data.

\item \text{interpolate using bias strips}

If you have no bias frames then the bias contribution can estimated
from the data in the bias strips. The estimation method is defined in
the \wlab{Interpolation method:} section.

\item \text{single constant}

If you have no bias frame or bias strips then you must have a bias
level estimate (this is either imported from your FITS headers or
defined in the \htmlref{\wlab{CCD characteristics}}
{CCDSetCCDGlobalsWindow} window).

\end{itemize}


If\xlabel{CCDReduceInterpolation} you want to debias using the bias
strips then you should select
\text{interpolate using bias strips} from the \wlab{Debias using:}
section. The ways in which you can debias your data depend on how many
bias strips you have. If you have two then all four of the methods
described below are available. If you have only one then only two are
available. Only those options that are available to you will be shown.

\begin{itemize}
\item \text{constant for each line}

In this option each line running between your bias strips has the mean
value in the bias strips subtracted from it (normally there is some
smoothing applied to the bias strips to reduce the inter-line noise).

\item \text{single constant for whole frame}

In this option the mean of the data in the bias strips in subtracted
from the whole frame.

\item \text{linear fit for each line}

In this option each line is fitted by a least squares linear fit and
this fit is subtracted from the data (normally there is some smoothing
applied to the bias strips to reduce the inter-line noise).

\item \text{plane across frame}

In this option all the data in the bias strips are fitted by a
plane and the fit to the plane is subtracted from the data.
\end{itemize}

\xlabel{CCDReduceSavespace}

During the reduction many intermediary frames are produced, these are
not normally of any use and can be deleted by choosing an option in
the \wlab{Save how much disk space:} section. Choosing any option
other than \text{none} will reduce the amount of disk space that
is required for the reduction.

\begin{itemize}
\item \text{some}

Choosing this option deletes all intermediary frames as soon as they
are no longer needed.

\item \text{lots}

Choosing this option is the same as choosing \text{some} except that
the original raw data is also deleted as soon as they are no longer
needed.

\end{itemize}

\xlabel{CCDReduceOK}

When you're happy with the settings in this window press \butt{OK} to
schedule the reduction and have it started in the background (at nice
priority). \butt{Cancel} exits without doing this. 

If you decide to schedule a reduction a {\em separate} window will 
appear that monitors the output from the background process. This 
is independent of XREDUCE which can now be closed.

\subsection{Menus \xlabel{CCDReduceMenu}}
The \menu{Options} menu contains the following items.
\begin{itemize}
\item \menu{View frames...}, this allows you to inspect a list of the
frames that will be processed.
\item \menu{Set additional options...}, this creates a new window in
which you can set further options controlling the reduction (such as
the names of the output data frames).
\end{itemize}

Use the \menu{Help} menu to get advice on using this window.

The \menu{File} menu allows you to exit from this program and close or
accept the window.

\section{Setting the extra reduction options \xlabel{CCDReduceExtrasWindow}}
This window allows you to set some additional options that control the
reduction. The options you can set are mostly related to the
names of the calibration frame masters, intermediary products and the
final reduced data.

The way that new names for your reduced data are made is by appending
extensions to the existing names. So for instance (using the default
settings) if you had a frame \text{DATA1.sdf} this would be called
\text{DATA1\_db.sdf} when it was debiassed and \text{DATA1\_db\_fl.sdf} when
it was flatfielded. If it was also dark count corrected (for some
reason) its final name would be \text{DATA1\_db\_dk\_fl.sdf}.

Since flatfields can be colour dependent more than one master flat can
be produced. So the names of these frames have the filter name appended
to them. So for instance if you use the default master flatfield
prefix and have filters \text{B,V,R}, then you will get three master
flats, \text{MASTER\_FLATB.sdf}, \text{MASTER\_FLATV.sdf} and
\text{MASTER\_FLATR.sdf}.

\subsection{Menus \xlabel{CCDReduceExtrasMenu}}
The \menu{Options} menu contains just one item, \menu{View
frames...}. This lists all the frames that are to be reduced.

Use the \menu{Help} menu to get advice on using this window.

The \menu{File} menu allows you to exit from this program or
accept the window.

\section{Background process monitor \xlabel{CCDFileMonitorMain}}
This window displays any output from your background process. The
output is updated as the job proceeds. This window may safely be
disposed of {\em without} affecting the progress of the job. It is
independent of XREDUCE which may also be closed.


\end{document}
