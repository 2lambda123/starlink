\documentclass[twoside,11pt,nolof]{starlink}

% ? Specify used packages
%% \usepackage        %  Use this one for final production.
%% \usepackage %  Use this one for drafting.
% ? End of specify used packages



% -----------------------------------------------------------------------------
% ? Document identification
% Fixed part
\stardoccategory    {Starlink User Note}
\stardocinitials    {SUN}
\stardocsource      {sun\stardocnumber}

% Variable part - replace [xxx] as appropriate.
\stardocnumber      {164.2}
\stardocauthors     {D L Terrett}
\stardocdate        {24 April 1998}
\stardoctitle    {PSMERGE \\
                                Encapsulated PostScript Handling Utility}
\stardocversion     {1.0}
\stardocmanual      {Users' manual}
\stardocabstract  {PSMERGE is a simple tool for creating
PostScript images by combining one or more Encapsulated Postscript images.
The images to be combined may be scaled, rotated and shifted
independently to form the composite.}

% ? End of document identification
% -----------------------------------------------------------------------------


% -----------------------------------------------------------------------------
% ? Document specific \providecommand or \newenvironment commands.
% ? End of document specific commands
% -----------------------------------------------------------------------------
%  Title Page.
%  ===========
\begin{document}
\scfrontmatter

% ? Main text

\section{\xlabel{introduction}Introduction}
\label{introduction}

An Encapsulated PostScript File (EPSF) is a PostScript file structured
so that it can be incorporated or included into another PostScript file
(so that for example a diagram created with a graphics application can
be inserted into a text document created with a word processor).

PSMERGE is a utility program for merging one or more Encapsulated
PostScript Files into a single PostScript file. The input files can be
individually rotated, scaled and shifted. The output file can either be
Encapsulated PostScript or ``normal'' PostScript suitable for sending
to a printer. This allows:

\begin{itemize}
\item Pictures created by different applications to be overlayed.
\item Multiple pictures to be pasted onto a single page.
\item Pictures scaled before insertion into a \LaTeX\ document.
\end{itemize}

\section{\xlabel{using_psmerge}Using PSMERGE}
\label{using_psmerge}

The \texttt{psmerge} program writes its output to the standard output channel
(\texttt{stdout}).  This can be redirected to a file or piped straight to a
printer queue; e.g.:

\begin{terminalv}
% psmerge picture.eps > printme.ps
\end{terminalv}
or
\begin{terminalv}
% psmerge picture.eps | lpr -Pps_printer
\end{terminalv}

The examples above simply converts an EPS file to a normal PostScript
file, the only difference being that the normal PostScript file shifts
the picture origin to the bottom left corner of the printable area on
the printer.  This is useful because if an EPS file is printed without
conversion, the picture origin will be at the bottom left corner of the
paper and on most printers this means that the bottom and left edges
will lie outside the printable area and will be invisible.

\subsection{\xlabel{combining_images}Combining Images}
\label{combining_images}

The principle use of PSMERGE is to combine more than one EPS file into a
single picture.  For example:

\begin{terminalv}
% psmerge pic1.eps pic2.eps > printme.ps
\end{terminalv}

will overlay the two pictures with their origins at the same place on
the paper.

For line graphics and text the order of the files does not make any
difference but if either picture contains filled areas or grey scale
images it does.  Filled areas and grey scales are opaque and will
obliterate anything already plotted in the same region of the page.
Therefore, for example, to overlay a contour map on a grey scale image,
the file containing the contour map must appear after the one
containing the image.

\subsection{\xlabel{scaling_rotating_and_shifting}%
Scaling, rotating and shifting images}
\label{scaling_rotating_and_shifting}

Pictures can be scaled, translated (shifted) and rotated. For example:

\begin{terminalv}
% psmerge -s0.5x0.5 -t72x144 -r10 pic1.eps > printme.ps
\end{terminalv}

will shrink the picture to half its normal size, move the origin 1 inch to the
right and 2 inches upwards and rotate it 10 degrees anti-clockwise.

Translations are in units of 1/72 of an inch\footnote{1/72 of an inch
is sometimes referred to as a PostScript point and is the default unit
for PostScript commands.} and rotations are in degrees anti-clockwise.

\subsection{\xlabel{complex_combinations}Complex combinations}

PSMERGE applies transformations are applied in the order that they
appear on the command line and are cumulative, so that, for example,
\texttt{-r45~-r45} is equivalent to \texttt{-r90} but
\texttt{-t72x72~-s2x2} does not have the same effect as
\texttt{-s2x2~-t72x72}.

After each EPS file is processed the scaling factors are reset to unity
and the translation and rotations reset to zero.  The following example
takes two A4 landscape pictures and pastes them one above the other on
a single A4 page in portrait orientation:

\begin{terminalv}
% psmerge -s0.707x0.707 -r90 -t550x396 pic1.eps \
       -s0.707x0.707 -r90 -t550x0 pic2.eps > printme.ps
\end{terminalv}

If \texttt{-e} appears on the command line anywhere, the output file
will be another EPS file rather than normal PostScript.  This allows
files to be processed by PSMERGE more than once in a pipeline using the
special file name \texttt{-} (hyphen) which means ``use the standard
input as the input file''.  Therefore, the same effect as the previous
example can be achieved with the following command:

\begin{terminalv}
% psmerge -e -t0x560 pic1.eps pic2.eps | \
       psmerge -s0.707x0.707 -r90 -t550x0 - > printme.ps
\end{terminalv}

Instead of scaling each picture and positioning it on the page
independently, the two pictures are combined into one and then this
single picture rotated and scaled.

The operation could be performed in two steps with the output from
the first stored in a temporary file:

\begin{terminalv}
% psmerge -e -t0x560 pic1.eps pic2.eps > temp.eps
% psmerge -s0.707x0.707 -r90 -t550x0 temp.eps > printme.ps
\end{terminalv}

\subsection{\xlabel{summary_of_psmerge_switches}Summary of PSMERGE switches}
\label{summary_of_switches}

The PSMERGE program uses the following command switches:

\begin{description}

\item[\texttt{-e}]: produce Encapsulated PostScript rather than PostScript.

\item[\texttt{-r}\textit{N}]: rotate \textit{N} degrees anti-clockwise.

\item[\texttt{-s}\textit{M}\texttt{x}\textit{N}]: scale image by factor
\textit{M} in X direction and \textit{N} in Y direction.

\item[\texttt{-t}\textit{M}\texttt{x}\textit{N}]: translate (shift)
image \textit{M} points in X direction and\textit{N} points in Y
direction.

\end{description}

% ? End of main text
\end{document}
