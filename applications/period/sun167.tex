\documentclass[twoside,11pt,noabs,nolof]{starlink}

% -----------------------------------------------------------------------------
% ? Document identification
\stardoccategory    {Starlink User Note}
\stardocinitials    {SUN}
\stardocsource      {sun\stardocnumber}
\stardocnumber      {167.6}
\stardocauthors     {V S Dhillon\\ G J Privett\\ K P Duffey}
\stardocdate        {12th December 2001}
\stardoctitle       {PERIOD \\ A Time-Series Analysis Package}
\stardocversion     {Version 5.0}
\stardocmanual      {User's Manual}
% ? End of document identification
% -----------------------------------------------------------------------------
% ? Document specific \providecommand or \newenvironment commands.
\providecommand{\rf}{\par\noindent\hangindent 15pt{}}


% ? End of document specific commands
% -----------------------------------------------------------------------------
%  Title Page.
%  ===========
\begin{document}
\scfrontmatter

\section{\xlabel{introduction}Introduction}

This document describes how to use \texttt{PERIOD} (version 5.0), a
software package designed to search for periodicities in data.
Version 5.0 is the successor to 4.2, of which
it is a Double Precision implementation, also re-configured to utilise
dynamic array allocation/deallocation for the majority of its input,
output and work arrays.

This document is intended for use with the UNIX software (version
5.0). If you have previously used a VMS version, you should be aware
that the UNIX \emph{is} case-sensitive with regard to directory
structure and file names.  For this reason the instructions below are
in lower case and may appear odd to those familiar with VMS.

\section{\xlabel{initializing_and_running}Initializing and Running}

To run \texttt{PERIOD} under UNIX, you need only type the word \texttt{period} at the shell prompt.

\section{\xlabel{dataformatandstorage}Data Format and Storage}
\label{slots}

\subsection{\xlabel{readingascii}Reading ASCII files as input}
\label{slots1}

The simplest form of input for \texttt{PERIOD} is an ASCII file.  The
files may contain any number of rows and columns. It is possible to
specify which column refers to the $x$-axis, which column refers to
the $y$-axis and which column refers to the $y$-axis errors.
The $y$-axis errors are optional and if included are used (or handled)
by all operations in the main \texttt{PERIOD} menu (see
section~\ref{menu}).  However, the only periodicity-finding option in
the \texttt{period\_period} sub-menu which actually uses the errors on
data points is \texttt{CHISQ} (see section~\ref{menu}). The remaining five
techniques (\texttt{CLEAN}, \texttt{FT}, \texttt{PDM}, \texttt{SCARGLE}, \texttt{STRING}) ignore errorbars if they are present. Note also that the input
file must contain $x$-axis values which are in ascending order,
otherwise the program will report a warning and either sort the data
(if requested to do so) or abort the input.

Data is stored and processed within \texttt{PERIOD} using a \emph{slot} system.
A data slot is simply an array holding one dataset. The maximum number
of data slots which can be handled at any one time in \texttt{PERIOD} is 40;
hence \texttt{PERIOD} has the capability of analysing a large amount of data
simultaneously.

The first command that is usually run in \texttt{PERIOD} is \texttt{INPUT}
(see section~\ref{menu}), which loads datasets into data slots:

\begin{terminalv}
Enter first and last slots for input (0,0 to quit) :
\end{terminalv}

In order to load the first slot with a single dataset, you should reply
\texttt{1,1} to the above prompt. Similarly, if you want to load slots
4 through to 9 with 6 datasets, one should reply \texttt{4,9} to the above
prompt. It is important to note that slots \emph{can} be overwritten.
Typing \texttt{0,0} will return the you to the menu.

Most \texttt{PERIOD} commands prompt not only for an input slot, but also for
an output slot:

\begin{terminalv}
Enter first and last slots for input  (0,0 to quit) :
Enter first and last slots for output (0,0 to quit) :
\end{terminalv}

The input should contain the dataset (or datasets) and the output will
contain the result of the operation on the dataset (or datasets). For
example, given a set of 5 time-series (which have previously been
loaded into slots 1 to 5 using \texttt{INPUT}) which need to be fitted
with a sine curve (using the command \texttt{FIT}, see
section~\ref{menu}), you would type \texttt{1,5} in reply to the first
prompt and \texttt{6,10} in reply to the second. \texttt{PERIOD} will then
fit sine curves to the data files loaded in slots 1 to 5 and put the
fits in slots 6 to 10. Clearly, the number of output slots must be
equal to the number of input slots. In addition, if any selected input
slot is empty, \texttt{PERIOD} will abort the operation and return you to
the main menu. It should be noted that in order to save on storage
space you could have typed \texttt{1,5} in reply to the second prompt and
the original data-files would have been overwritten by the resulting
sine curves.

\subsection{\xlabel{readingfitsi}Reading \texttt{OGIP FITS} files as input}
\label{slots2}

Because of the increasing use of \texttt{OGIP} standard \texttt{FITS} files
within the X-ray community, \texttt{PERIOD} has been extended to allow
\texttt{FITS} file table extensions where the \texttt{FITS} keyword \texttt{HDUCLASS} has the value \texttt{OGIP}.

As with ASCII data, data files may contain any number of rows. There is
also no restricition on the number of \texttt{FITS} file extensions within
the file to be read.  Again, you must specify which columns of
the table refers to the $x$-axis, which column refers to the $y$-axis
and which column refers to the $y$-axis errors. The $y$-axis errors are
optional and if included are used (or handled) by all operations in the
main \texttt{PERIOD} menu (see section~\ref{menu}).

After you have entered the name of the file to be read, the file is
opened and the keywords relating to telescope type etc are
read. These are displayed so that you can make sure you are looking at
the right file. The output will look something like this.

\begin{terminalv}
 PERIOD> ogip

 Enter name of OGIP FITS file (<CR> to quit) : test.fits

 File details are...

 Telescope:   ROSAT
 Instrument:  PSPCC
 Detector:    Unknown
 Object:      XRT/PSPC PSF AR LA

 Total number of extensions found:   4
 Number of OGIP extensions found:    4

 Extension   1
 A binary table.
 Size   3 by   2
 EXTNNAME is:  STDGTI
 HDUCLAS1 is:  GTI
 HDUCLAS2 is:  STANDARD

 Extension   2
 A binary table.
 Size   8 by   29380
 EXTNNAME is:  STDEVT
 HDUCLAS1 is:  EVENTS
 HDUCLAS2 is:  ACCEPTED

 Extension   3
 A binary table.
 Size   8 by   1246
 EXTNNAME is:  REJEVT
 HDUCLAS1 is:  EVENTS
 HDUCLAS2 is:  REJECTED
\end{terminalv}

You are then asked to select the extension to be examined and are shown
the header descriptions for each table column. From this information
you can select which column should be in $x$-axis, $y$-axis or the
$y$-axis errors part of your chosen data slot.

\begin{terminalv}
 Enter the extension number to be considered (0 to quit) : 3
 Column        Contents
   1             X
   2             Y
   3             PHA
   4             PI
   5             TIME
   6             DETX
   7             DETY
   8             STATUS
 Enter the TTYPE column number to be read as X (0 to quit) : 1
 Enter the TTYPE column number to be read as Y (0 to quit) : 2
 Enter the TTYPE column number to be read as Y errors (0 to ignore) : 0

 Enter the slot number into which the data should be read (0 to quit) : 1

 Reading data into slot   1

 Read   1246 out of   1246

 ** OK: Filled Slot=  1
\end{terminalv}

After this you are given the option to reread the same document and fill a
different slot.

While slightly more complicated to read than normal ASCII the
program is still very easy to use.

\section{\xlabel{using_this_document}Using this Document}

There are two main sections in this document.  Section~\ref{menu}
should be consulted by experienced users who require detailed
information on individual \texttt{PERIOD} commands while
Section~\ref{recipe} is intended for inexperienced users who require an
introductory recipe detailing the steps that should follow when
searching for periodicities in data.

\emph{Please note that \texttt{PERIOD} also has an extensive on-line help
facility.  This provides information at a level of detail similar to
that found in section~\ref{menu}. However, it also provides a detailed
description of the individual prompts, something which cannot be found
in this document.}

\section{\xlabel{menu_options}Menu Options}
\label{menu}

\texttt{PERIOD} is a menu-driven package. On entering \texttt{PERIOD}, you will be
confronted with the following menu options, which are described in greater
detail below.

\begin{terminalv}
|**************************************************|
|**| PERIOD  :>  A time-series analysis package |**|
|**| Version :>  5.0 for UNIX                   |**|
|**| Date    :>  12 December 2001               |**|
|**************************************************|

Options.
--------

INPUT    --  Input ASCII file data.
OGIP     --  Input OGIP FITS table data
FAKE     --  Create fake data.
NOISE    --  Add noise to data.
DETREND  --  Detrend the data.
WINDOW   --  Set data points to unity.
OPEN     --  Open a log file.
CLOSE    --  Close the log file.
PERIOD   --  Find periodicities.
FIT      --  Fit sine curve to folded data.
FOLD     --  Fold data on given period.
SINE     --  +, -, / or * sine curves.
PLT      --  Call PLT.
STATUS   --  Information on stored data.
OUTPUT   --  Output data.
HELP     --  On-line help.
QUIT     --  Quit PERIOD.

PERIOD>
\end{terminalv}

Any one of these commands can be entered by typing anything from the shortest
unambiguous string up to the full command name. Therefore, \texttt{P} would be
ambiguous, but \texttt{PE} would not.

\subsection*{\texttt{INPUT}}

As described in section~\ref{slots1}, this option allows you to input
ASCII data into \texttt{PERIOD}. The routine determines the number of
columns in the input files and then prompts the user for which columns
refer to the $x$-axis, $y$-axis and $y$-axis errors (if desired, see
section~\ref{slots}). For example, if the user is inputting radial
velocity data, the $x$-axis would most probably be HJD's, the $y$-axis
the heliocentric radial velocities and there would most likely be
errors associated with each radial velocity value. Note that the
$x$-axis values must be in ascending order, otherwise \texttt{INPUT} will
report a warning and either sort the data (if requested to do so) or
abort.  Note also that the $y$-axis errors are used by all options in
the main \texttt{PERIOD} menu, but by only the \texttt{CHISQ}
periodicity-finding option in the \texttt{period\_period} sub-menu.

\subsection*{\texttt{OGIP}}

As described in section~\ref{slots2}, this option allows you to input data
from an \texttt{OGIP FITS} table
into \texttt{PERIOD}. The routine displays some information about
the file requested and allows you to choose which of its available
tables is to be examined. You then select which of the columns in the file
refers to the $x$-axis,
$y$-axis and $y$-axis errors (if desired, see section~\ref{slots}).

\subsection*{\texttt{FAKE}}

Allows you to create fake data with which to test or experiment with \texttt{PERIOD}. Two options are catered for: periodic data or chaotic data. The
periodic data are created by summing a user-specified number of sine curves of
the form:

\begin{terminalv}
Y = GAMMA + (AMPLITUDE * SIN( ((2.0*PI)/PERIOD) * (X - ZEROPT)))
\end{terminalv}

The chaotic data are created using a simple logistic equation of the form:

\begin{terminalv}
Xn+1 = LAMBDA * Xn * (1-Xn)
\end{terminalv}

(see, for example, Scargle 1990ab).

\subsection*{\texttt{NOISE}}

Using this option, it is possible to add noise to data or randomize data. The
latter operation is carried out by specifying the \texttt{[N]}ew dataset option,
which will construct an artificial dataset of the same mean value and the same
standard deviation as the original. Selecting the \texttt{[O]}ld dataset allows
you to apply noise to data, create errorbars on the data points, and/or add
noise to the data sampling (so that, for instance, an evenly sampled dataset
becomes unevenly sampled). This routine is useful, not only in creating
realistic artificial datasets (in conjunction with \texttt{FAKE}), but also in
investigating the effects of noise on a period detection.

\subsection*{\texttt{DETREND}}

This option removes the D.C. bias from data, which if not removed gives rise to
significant power at 0 Hz. There are two options: If the data show no long term
trends, it is best to simply subtract the mean and divide by the standard
deviation (the \texttt{[M]} option). This gives a dataset with a mean of zero and
a standard deviation of one. Otherwise, it is best to subtract a low-order
polynomial fit to the data (the \texttt{[P]} option), since if these are not
removed, a Fourier transform will inject a significant amount of power at the
frequency of the long term variations.

\subsection*{\texttt{WINDOW}}

One of the main problems with the \emph{classical}
periodogram\footnote{Throughout the \texttt{PERIOD} package and this document, the
terms \emph{power spectrum} and \emph{periodogram} are used interchangeably,
although strictly speaking the power spectrum is a theoretical quantity defined
as an integral over continuous time, of which the periodogram is merely an
estimate based on a finite amount of discrete data (Scargle 1982).}  (see
Scargle 1982 for a definition), is \emph{spectral leakage}, of which there are
several forms. Leakage to nearby frequencies (sidelobes) is due to the finite
total interval over which the data is sampled. Leakage to distant frequencies
is due to the finite size of the interval between samples. The \texttt{WINDOW}
option sets all the $y$-axis data points to unity. A discrete Fourier transform
of the resulting data (using, for example, the \texttt{FT} option, see below)
yields the window function (or spectrum), which shows the effects of spectral
leakage.

\subsection*{\texttt{OPEN}}

It is possible to store the fits calculated by \texttt{SINE} and \texttt{PEAKS} in a
log file. This option opens a new log file (if it does not already exist), or
else re-opens an old log file and skips over the existing entries.

\subsection*{\texttt{CLOSE}}

This option closes the currently open log file.

\subsection*{\texttt{PERIOD}}

This is where all the work is done. You will be confronted by the following
sub-menu:

\begin{terminalv}
Options.
--------

SELECT   --  Select data slots.
FREQ     --  Set/show frequency search limits.
CHISQ    --  Chi-squared of sine fit vs frequency.
CLEAN    --  CLEANed power spectrum.
FT       --  Discrete Fourier power spectrum.
PDM      --  Phase dispersion minimization.
SCARGLE  --  Lomb-Scargle normalized periodogram.
STRING   --  String-length vs frequency.
PEAKS    --  Calculate period from periodogram.
SIG      --  Enable/disable significance calc.
HELP     --  On-line help.
QUIT     --  Quit PERIOD_PERIOD.

PERIOD_PERIOD>
\end{terminalv}

\begin{itemize}

\item \texttt{SELECT} -- Selects input and output slots for processing, as
described in section~\ref{slots}. The input slots should contain the
time-series, the output slots will contain, for example, the power spectra.
\texttt{SELECT} must be run \emph{every} time a periodicity-finding option is
about to be executed; although tedious, this prevents one from accidentally
overwriting slots.

\item \texttt{FREQ} -- Sets the frequency search parameters. The minimum frequency,
maximum frequency and frequency interval can be selected by you. Generally,
there is no restriction on the number of frequencies to be stepped-though in
the processing. Alternatively, by entering 0's, default values can be
accepted. Note that the default values are set on entering the \texttt{PERIOD}
package and thus the \texttt{FREQ} option need not be run if default frequencies
are required. The default values are calculated as follows: minimum frequency =
0 (ie. infinite period), maximum frequency = 1 / (2 $\times$ Smallest Data
Interval) (ie. Nyquist), frequency interval = 1 / (4 $\times$ Total Time
Interval).

\item \texttt{CHISQ} -- This is a straight-forward technique where the input data
is folded on a series of trial periods. At each trial period, the data is
fitted with a sine curve. The resulting reduced-$\chi^2$ values are plotted as
a function of trial frequency and the minima in the plot suggest the most
likely periods. See Horne, Wade and Szkody (1986) for an example of the use of
this method, which is ideally suited to the study of radial velocity data or
any other sinusoidal variations. Note that windowed data cannot be processed
by this option since no sine fit is possible.

\item \texttt{CLEAN} -- The \texttt{CLEAN} algorithm was originally developed for use
in aperture synthesis and was later applied to one-dimensional data by Roberts,
Leh\'{a}r and Dreher (1987). An adapted version of Leh\'{a}r's code is used here,
and is particularly useful for unequally spaced data. The algorithm basically
deconvolves the spectral window from the discrete Fourier power spectrum (or
dirty spectrum). This produces a \texttt{CLEAN} spectrum, which is largely free of
the many effects of spectral leakage. In order to prevent small errors from
destabilizing the \texttt{CLEAN} procedure, the user is prompted for two
parameters -- the loop gain and the number of iterations. Briefly, with each
iteration, some fraction (governed by the loop gain) of the window function is
removed from the dirty spectrum. For convergence, the loop gain must lie
between 0 and 2, typical values being between 0.1 and 1. Values at the bottom
of this range require more iterations, but should provide more stability.
Hence, the number of iterations should be large if the loop gain is small,
typical values lying between 1 and 100. Note that an increase in the number of
cleans produces a less noisy spectrum but, in general, the amplitude of the
peaks is decreased, sometimes by a substantial amount. See Roberts, Leh\'{a}r
and Dreher (1987) for further details on choosing these parameters.

\item \texttt{FT} -- This option performs a classical discrete Fourier transform
on the data and sums the mean-square-amplitudes of the result to form a power
spectrum (see, for example, Deeming 1975). This discrete Fourier transform is
defined for arbitrary data spacing and is equal to the convolution of the
true Fourier transform with a spectral window. Hence, the effects of data
spacing, such as aliasing, are all contained in the spectral window, which
can be generated using the \texttt{WINDOW} option (see above). This spectral
window should be analysed in conjunction with the discrete Fourier transform
generated here in order to estimate the effects of aliasing.

\item \texttt{PDM} -- The phase dispersion minimization (PDM) technique is simply
an automated version of the classical method of distinguishing between possible
periods, in which the period producing the least observational scatter about
the mean light curve (or, for example, radial velocity curve) is chosen. This
technique (which is described in detail by Stellingwerf 1978) is well suited to
cases in which only a few observations are available over a limited period of
time, especially if the light curve is highly non-sinusoidal. The data is first
folded on a series of trial frequencies. For each trial frequency, the full
phase interval (0,1) is divided into a user-specified number of bins. The width
of each bin is specified by the user, such that a point need not be picked (if
a bin width narrower than the bin spacing is selected) or a point can belong to
more than one bin (if a bin width wider than the bin spacing is selected). The
variance of each of these bins (or samples) is then calculated. This gives a
measure of the scatter around the mean light curve defined by the means of the
data in each sample. The PDM statistic can then be calculated by dividing the
overall variance of all the samples by the variance of the original (unbinned)
dataset. This process is then repeated for the next trial frequency. Note that
windowed data cannot be passed to this option since its variance is zero. If
the trial period is not a true period, then the overall sample variance will be
approximately equal to the variance of the original dataset (ie. the PDM
statistic will be approximately equal to 1). If the trial period is a correct
period, the PDM statistic will reach a local minimum compared with neighbouring
periods, hopefully near zero.

\item \texttt{SCARGLE} -- By redefining the classical periodogram (ie. the
discrete Fourier periodogram) in such a manner as to make it invariant to a
shift of the origin of time, Lomb (1976) and Scargle (1982) developed a novel
type of periodogram analysis, quite powerful for finding, and testing the
significance of, weak periodic signals in otherwise random, \emph{unevenly
sampled} data. Horne and Baliunas (1986) have elaborated on the method and
Press and Rybicki (1989) present a fast implementation of the algorithm, a
modified version of which is used here. This implementation uses FFTs to
increase the speed of computation (although it is in no way equivalent to
conventional FFT periodogram analysis). Note that windowed data cannot be
passed to this option since it needs to calculate the variance (which is zero)
to normalize the power of the periodogram.

\item \texttt{STRING} -- The string-length method is an intuitively simple method,
described in detail by Dworetsky (1983) and Friend \textit{et al.} (1990). The
data is folded on a series of trial periods and at each period the sum of the
lengths of line segments joining successive points (the string-length) is
calculated. Minima in a plot of string-length versus trial frequency indicate
possible periods. The string-length method is especially useful in the limit
of a very small number (about 20 or more) of randomly spaced observations
of periodic phenomena. Note that windowed data cannot be passed to this
option due to the $y$-data scaling process (see Dworetsky 1983).

\item \texttt{PEAKS} -- This option should be run once a periodogram has been
obtained. It finds the highest peak in the periodogram (or lowest trough if it
is a string-length, PDM or reduced-$\chi^2$ plot) between user-specified
frequencies. The resulting period is calculated, along with an error. Errors on
period detections are notoriously difficult to estimate. The estimate used in
the previous version of \texttt{PERIOD} (v3.0) employed a formula derived by
Kovacs (1981). The derivation assumed a single signal, Gaussian noise and even
data spacing. This is clearly not the case with most astronomical datasets and
the formula is hence of little use (see Horne and Baliunas 1986).
Schwarzenberg-Czerny (1991) presents a detailed account of the accuracy of
period determinations and advises a \emph{post-mortem} analysis by measuring the
width and heights of peaks in a periodogram. Although virtually impossible to
automate, it is possible to do this manually from within \texttt{PERIOD} using the
fitting routines of \texttt{QDP/PLT} (see above). Therefore, for the sake of
generality and to avoid uncertainties, version 4.0 of \texttt{PERIOD} now only
outputs an error derived by calculating the half-size of a single frequency
bin, centred on the peak (or trough) in a periodogram, and then converting to
period units. This error gives an indication of the accuracy to which a peak
can be located in a periodogram (due to the frequency sampling). Clearly, with
a larger frequency search interval it is more difficult to locate a peak
precisely and this is reflected in the error estimate. However, this error
estimate does not take into account the fact that the peak (or trough) may not
represent the true period (which can be shifted due to a number of effects) and
it should therefore be regarded as a \emph{minimum} error and not a formal
error.

If the significance calculation is enabled (with the \texttt{SIG} command, see
below), two false alarm probabilities are quoted alongside the period. The
first (\texttt{FAP1}) is the probability that, given the frequency search
parameters, there is no periodic component present in the data with this
period. The second (\texttt{FAP2}) is the probability that the period is not
actually equal to the quoted value but is equal to some other value. Note that
FAP1 is only output if the whole frequency range is specified to be analysed in
\texttt{PEAKS} (see below). One sigma errors on both significance values are also
given. If the significance values are zero, these errors are displayed as --1,
implying that the false alarm probabilities lie between 0.00 and 0.01 with 95\%
confidence. Clearly, the lower a significance value and its error, the more
likely the quoted period is a correct one. If both the significances and errors
are displayed as --1, this means that the input periodogram has not been
subjected to a significance calculation (ie. the significance calculation has
been disabled). Note that the results can be written to a log file if one is
open. For more information on the \texttt{SIG} option, see below. For useful
discussions on errors and significances of period determinations, see
Schwarzenberg-Czerny (1991) and Nemec and Nemec (1985).

\item \texttt{SIG} -- This option works as a switch, either turning on or turning
off the significance calculation. The default on entering \texttt{PERIOD} is for
the significance calculation to be disabled. This means that no significance
values are calculated or attached to period determinations. By typing \texttt{SIG}, the significance calculation is enabled. You are first prompted for the
number of permutations in the sample. To ensure reliable significance values,
the minimum number of permutations is set to 100. You are then prompted for a
seed for the random number generator -- this number determines the starting
point in a number series of infinite period. Therefore, entering the same seed
on two calls to \texttt{SIG} will result in the same sequence of random numbers.
If \texttt{SIG} is already enabled, one can disable the significance calculation
by typing \texttt{SIG} again.

With the significance calculation enabled, every time a period-finding option
is run (\texttt{CHISQ, FT, SCARGLE, CLEAN, STRING, PDM}) a Fisher randomization
test is performed (see, for example, Nemec and Nemec 1985). This consists of
calculating the periodogram as usual and loading the specified output slot. The
$y$-axis data is then shuffled to form a new, randomized time-series. The
periodogram of this dataset is then calculated (but not stored in the output
slot, which will always contain the periodogram of the real time-series). This
randomization and periodogram calculation loop is then performed for the number
of permutations specified by the user. This can take a considerable amount of
time, depending on the number of data points in the time-series, the frequency
search parameters and the number of permutations.

Once the loop is complete, you should enter the \texttt{PEAKS} option to view the
resulting significances. Two significance estimates are given in \texttt{PEAKS}.
The first, denoted \texttt{FAP1}, represents the proportion of permutations (ie.
shuffled time-series) that contained a trough lower than (in the case of the
\texttt{CHISQ, STRING} and \texttt{PDM} options) or a peak higher than (in the case
of the \texttt{FT, SCARGLE} and \texttt{CLEAN} options) that of the periodogram of
the unrandomized dataset \emph{at any frequency}. This therefore represents the
probability that, given the frequency search parameters, no periodic component
is present in the data with this period and it is only output in \texttt{PEAKS} if
the whole frequency range is specified to be analysed. The second significance,
denoted \texttt{FAP2}, represents the proportion of permutations that, \emph{at the
frequency given by the period output by} \texttt{PEAKS}, contained troughs lower
than (or peaks higher than) the peak or trough in the periodogram of the real
dataset. This therefore represents the probability that the period is not
actually equal to the quoted value but is equal to some other value, and is
quoted for any frequency range specified in \texttt{PEAKS}. Standard errors on both
of these false alarm probabilities are also given (see Nemec and Nemec 1985).

It is perhaps worth mentioning here that significance estimates of period
detections are notoriously unreliable. The methods used in the previous version
of \texttt{PERIOD} (v3.0) suffered from a number of problems. For example, the
F-test used with the \texttt{PDM} method (Stellingwerf 1978) has been proved to be
incorrect (see, for example, Heck, Manfroid and Mersch 1985). Similarly, the
theoretical minimum string-lengths quoted by Dworetsky (1983) are misleading,
since they are based on evenly-spaced functions and it is possible to obtain
values below this even for pure noise data with certain data spacings. The
well-known \texttt{SCARGLE} false alarm probabilities are also incorrect, since
the Horne and Baliunas (1986) equation for the number of independent
frequencies has been shown to be incorrect (Christian Knigge (Oxford), private
communication). Even if correct, the Horne and Baliunas formula would be
incorrect to apply in a general way since it is a poor approximation to small
datasets. The only reliable method of estimating significances from such
non-parametric tests is by some sort of Monte Carlo or randomization method.
As described above, one such method (Fisher randomization) has been
implemented in this version of \texttt{PERIOD} (v4.0) following the prescription
described by Nemec and Nemec (1985).

\item \texttt{HELP} -- This command provides on-line help for \texttt{PERIOD}.
Detailed information about individual commands can be obtained by typing
\texttt{HELP 'COMMAND'} (eg. \texttt{HELP PEAKS}).

\item \texttt{QUIT} (or \texttt{EXIT}) -- This quits the \verb@PERIOD_PERIOD@
sub-menu and returns the user to the main \texttt{PERIOD} menu.

\end{itemize}

\clearpage
Returning to the main \texttt{PERIOD} menu:

\subsection*{\texttt{FIT}}

Folds the data on a given period and zero point and then fits the data with a
sine curve. The sine curve has the form: \texttt{Y = GAMMA + (AMPLITUDE * SIN(
((2.0*PI)/PERIOD) * (X - ZEROPT) ))}. Outputs the fit parameters (which can be
written to a log file) and the resulting sine curve.

\subsection*{\texttt{FOLD}}

Folds the data on a given period and zero point. Hence, this option transforms
the data onto a phase scale, where one phase unit is equal to one period and
phase zero is defined by the zero point. If the zero point is not known, the
data can be folded by taking the first data point as the zero point. This
option is useful for checking whether derived periods actually give sensible
results when applied to the data. In addition to normal folding, it is also
possible to phase bin the data, which folds the data and then averages all the
data points falling into each bin.

\subsection*{\texttt{SINE}}

Adds, subtracts, multiplies or divides a sine curve from data. The sine curve
has the form: \texttt{Y = GAMMA + (AMPLITUDE * SIN( ((2.0*PI)/PERIOD) * (X -
ZEROPT) ))}. This option is useful for removing or adding known periods from/to
data, thus enabling or testing the detection of other periods.

\subsection*{\texttt{PLT}}

This routine calls PGPLOT routines to display the graphs of the slots
requested. The layout of the displays is fixed but output file
types such as landscape postscript files can be created. This
represents slightly less functionality than the original XANADU based
QDP \texttt{PLT} routine, but no QDP \texttt{PLT} routine is currently available for
LINUX.

In order
to receive on-line help, simply type \texttt{HELP} at the \texttt{PERIOD-PLT}
prompt. To exit \texttt{PERIOD-PLT} and return to the \texttt{PERIOD} menu,
type \texttt{EXIT}.

\subsection*{\texttt{STATUS}}

Returns information on the data slots or on the stored fits in the log file.
This command is useful in order to check which slots contain which datasets
and also as a means of obtaining some elementary statistics on the stored data.
You can also use this option to check the fits from the \texttt{SINE} and
\texttt{PEAKS} options stored in the log file without having to exit the package
and read the log file.

\subsection*{\texttt{OUTPUT}}

Writes any selected slot to an ASCII file on disk. This is the only way of
saving data created by \texttt{PERIOD} (it does not write to \texttt{FITS} files),
and should therefore be run before \texttt{QUIT}ing in order to store, say,
a power spectrum.

\subsection*{\texttt{HELP}}

This command provides on-line help for \texttt{PERIOD}. Detailed information about
individual commands can be obtained by typing \texttt{HELP 'COMMAND'} (eg. \texttt{HELP PERIOD}).

\subsection*{\texttt{QUIT} (or \texttt{EXIT})}

This option quits a \texttt{PERIOD} session. However, it does provide a last
chance to stay in the package. This is essential to prevent accidental exit,
since any data files created using \texttt{PERIOD} will be lost on exit from the
package \emph{unless} one \texttt{OUTPUT}s the data first.

\section{\xlabel{simplerecipe}A Simple Recipe}
\label{recipe}

A simple guide designed to introduce inexperienced users to the steps involved
in detecting periodicities is outlined below. Detailed descriptions of the
individual \texttt{PERIOD} commands can be found in section~\ref{menu} and in
the on-line help facility (which also gives help on the individual prompts
one is confronted with).

\begin{enumerate}
\item Create an ASCII data file containing the time-series.
\item Read ASCII data into \texttt{PERIOD} using \texttt{INPUT}.
\item or read \texttt{OGIP FITS} files into \texttt{PERIOD} using \texttt{OGIP}.
\item Detrend the data using the \texttt{[M]} option in \texttt{DETREND} (if the
data show long term variations, use the \texttt{[P]} option instead).
\item Open a log file for the fits using \texttt{OPEN}.
\item Enter the \verb@PERIOD_PERIOD@ menu by typing \texttt{PERIOD}.
\item Select the slots which contain the time-series data and specify the output
slots for the periodograms using \texttt{SELECT}.
\item Set the frequency search limits using \texttt{FREQ}. If you have no idea
what the period is, accept the default values by typing 0 (or alternatively,
by not typing \texttt{FREQ} in the first place).
\item Enable the significance calculation by typing \texttt{SIG} and specifying,
say, 200 permutations.
\item Calculate the Lomb-Scargle periodogram by typing \texttt{SCARGLE}.
\item Now run \texttt{PEAKS} on the resulting periodogram, specifying the
frequency range which contains the peak you wish to measure (you may enter 0,0
if you wish to process the entire range). Write the results to the log file.
\item Now reselect the time-series slots and different output slots for a new
periodogram using \texttt{SELECT}.
\item Type \texttt{CLEAN} with 5 iterations and a loop gain of 0.2, for example.
(Before doing this, you may wish to disable the significance calculation by
typing \texttt{SIG} again, since the \texttt{CLEAN} algorithm can take a
considerable amount of processing time).
\item Run \texttt{PEAKS} on the resulting periodogram and store the results.
\item Now quit the \verb@PERIOD_PERIOD@ sub-menu by typing \texttt{QUIT}.
\item Plot the periodograms using \texttt{PLT}. Check to see the validity of the
highest peak selected.
\item Check the results in the log file using \texttt{STATUS}. In particular, look
closely at the false alarm probabilities.
\item Fold the \emph{original} data on the most likely period using \texttt{FOLD}.
\item Plot the folded data using \texttt{PLT}. If this looks sensible, the period
may well be correct. Make a postscript file by typing \texttt{epsf\_l}. To
see the other options type '?' instead.
\item Output the periodograms and folded data to an ASCII file on disk using
\texttt{OUTPUT}.
\item Exit \texttt{PERIOD} by typing \texttt{QUIT}.
\end{enumerate}

The above description is intended only to be a very brief guide. Clearly, a
great deal more experimentation is required before it can definitely be said that
a period has been detected. For example, you should investigate other large
peaks in the periodogram, try smaller or larger frequency ranges, or try one of
the other periodicity-finding options (a useful comparison of a number of
different techniques is given by Carbonell, Oliver and Ballester 1992 and Heck,
Manfroid and Mersch 1985). Other analysis techniques might also be attempted,
such as subtracting a sine curve from the data in order to investigate its
effects on the harmonics and enable the detection of less-dominant periods.

\section{\xlabel{future_improvements}Future Improvements}

Include more periodicity-finding options, such as AoV
(Schwarzenberg-Czerny 1989), maximum entropy method (Fahlman and Ulrych 1982),
an FFT, and chaos analysis (Canizzo and Goodings 1988, Scargle 1990ab and
Lehto, Czerny and McHardy 1993).

\section{\xlabel{acknowledgments}Acknowledgments}

The author gratefully acknowledges the use of subroutines written by
Joseph Leh\'{a}r (Cambridge), Keith Horne (STScI) and Tom Marsh
(Oxford), as well as those written by Press and Rybicki (1989).
I also  thank
the bug-finding guinea-pigs Martin Still (Sussex), Mark O'Dell
(Sussex), Erik Kuulkers (Amsterdam) and Thomas Augusteijn (Amsterdam)
for their many useful comments and suggestions. Finally, I am indebted
to Christian Knigge (Oxford) for his help with bug fixes and his many
suggestions regarding significance estimates.

\section{\xlabel{references}References}

\noindent \rf{Canizzo, J. \&\ Goodings, D., 1986. \emph{Astrophys. J. Lett.}, \textbf{334}, L31.}
\rf{Carbonell, M., Oliver, R. \&\ Ballester, J.~L., 1992. \emph{Astron. Astrophys.}, \textbf{264}, 350.}
\rf{Deeming, T.~J., 1975. \emph{Astrophys. Space Sci.}, \textbf{36}, 137.}
\rf{Dworetsky, M.~M., 1983. \emph{Mon. Not. R. astr. Soc.}, \textbf{203}, 917.}
\rf{Fahlman, G.~G. \&\ Ulrych, T.~J., 1982. \emph{Mon. Not. R. astr. Soc.}, \textbf{199}, 53.}
\rf{Friend, M.~T., Martin, J.~S., Smith, R.~C. \&\ Jones, D.~H.~P., 1990. \emph{Mon. Not. R. astr. Soc.}, \textbf{246}, 637.}
\rf{Heck, A., Manfroid, J. \&\ Mersch, G., 1985. \emph{Astron. Astrophys. Suppl. Ser.}, \textbf{59}, 63.}
\rf{Horne, J.~H. \&\ Baliunas, S.~L., 1986. \emph{Astrophys. J.}, \textbf{302}, 757.}
\rf{Horne, K., Wade, R.~A. \&\ Szkody, P., 1986. \emph{Mon. Not. R. astr. Soc.}, \textbf{219}, 791.}
\rf{Kovacs, G., 1981. \emph{Astrophys. Space Sci.}, \textbf{78}, 175.}
\rf{Lehto, H.~J., Czerny, B. \&\ McHardy, I.~M., 1993. \emph{Mon. Not. R. astr. Soc.}, \textbf{261}, 125.}
\rf{Lomb, N.~R., 1976. \emph{Astrophys. Space Sci.}, \textbf{39}, 447.}
\rf{Nemec, A.~F~L. \&\ Nemec, J.~M., 1985. \emph{Astron. J.}, \textbf{90}, 2317.}
\rf{Press, W.~H. \&\ Rybicki, G.~B., 1989. \emph{Astrophys. J.}, \textbf{338}, 277.}
\rf{Roberts, D.~H., Leh\'{a}r, J. \&\ Dreher, J.~W., 1987. \emph{Astron. J.}, \textbf{93}, 968.}
\rf{Scargle, J.~D., 1982. \emph{Astrophys. J.}, \textbf{263}, 835.}
\rf{Scargle, J.~D., 1990a. \emph{Astrophys. J.}, \textbf{359}, 469.}
\rf{Scargle, J.~D., 1990b. In \emph{Errors, Bias \&\ Uncertainties in Astronomy}, Eds. Jaschek, C. \&\ Murtagh, F., Cambridge University Press, Cambridge.}
\rf{Schwarzenberg-Czerny, A., 1989. \emph{Mon. Not. R. astr. Soc.}, \textbf{241}, 153.}
\rf{Schwarzenberg-Czerny, A., 1991. \emph{Mon. Not. R. astr. Soc.}, \textbf{253}, 198.}
\rf{Stellingwerf, R.~F., 1978. \emph{Astrophys. J.}, \textbf{224}, 953.}

\clearpage
\appendix


\section{History}

{\noindent This appendix lists all of the changes made to the PERIOD
package and the release dates of\newline the various versions.}
\vspace{10mm}


\begin{small}
\begin{terminalv}

   DATE       Author            Comments

1992 July 10   VSD       VERSION 1.0 released to Robert Smith (Sussex),
                         Mark O'Dell (Sussex) and Erik Kuulkers (Amsterdam).

                         *** VERSION 1.0 ***

1992 July 21   VSD       This history file added to the package.
1992 July 21   VSD       PERIOD_SCARGLE modified to output EFFM instead of
                         PROB and JMAX. This enables PEAKS to calculate te
                         false alarm probability of any peak in the
                         periodogram.
1992 July 21   VSD       PERIOD_PERIOD modified to cope with changes in
                         PERIOD_SCARGLE and calculate the false alarm
                         probability of any peak selected in PEAKS.
                         Also included a trap for invalid values of the
                         loop gain in CLEAN (greater than 0 and less than 2).
1992 July 21   VSD       PERIOD_INPUT modified to prevent crashes if the
                         input file contains two contiguous x-axis points
                         with the same or decreasing values.
1992 July 21   VSD       PERIOD_DOC extended by adding a recipe.

\end{terminalv}
\end{small}


\newpage

\begin{small}
\begin{terminalv}

   DATE       Author            Comments

1992 July 21   VSD       PERIOD_MAIN, _CLEAN, _DETREND, _FIT, _FOLD, _FT,
                         _INPUT, _NOISE, _PERIOD, _PHASE, _POLYFIT and
                         _STATUS modified in order to handle 10000 data or
                         frequency points and 40 data files containing 20
                         columns. This ensures a page file quota of no more
                         than 34000 is required to run PERIOD.
1992 July 21   VSD       PERIOD_LOG, _READFREE, _STATUS modified by setting
                         large loop variables in data or parameter
                         declarations.
1992 July 21   VSD       Included a MAXCOL parameter in PERIOD_MAIN and
                         PERIOD_INPUT in order to cut down array sizes.
                         Also set size of MAXANT to MXROW in order to
                         ensure that only the maximum number of permitted
                         rows are loaded.
1992 July 21   VSD       PERIOD_COM:PERIOD_MAIL.COM added in order to make
                         e-mailing new versions of PERIOD easier.
1992 July 21   VSD       VERSION 1.1 of PERIOD released to incorporate the
                         above changes. Copy given to Mark O'Dell (Sussex).

                         *** VERSION 1.1 ***

1992 July 29   VSD       PERIOD_SCARGLE modified to remove calculation of
                         false alarm probability from subroutine.
1992 July 29   VSD       PEAKS in PERIOD_PERIOD modified to include a more
                         rigorous calculation of the SCARGLE false alarm
                         probability using an analytical expression for the
                         number of independent frequencies.
1992 July 30   VSD       NWK parameter, specifying sizes of the workspace
                         arrays in PERIOD_PERIOD and PERIOD_SCARGLE, reduced
                         from MXROW*100 to MXROW*16 in order to cut down on
                         memory usage.
1992 July 30   VSD       PERIOD_MOMENT modified to remove redundant skew and
                         kurtosis calculations.
1992 Aug 6     VSD       PERIOD_DOC.TEX split into two documents -- a user
                         document (PERIOD_USER.TEX -- LUN 2.1) and a system
                         document (PERIOD_SYSTEM.TEX -- LSN 1.1).
1992 Aug 7     VSD       PERIOD_INTRO modified to give my address as RGO
                         La Palma and my e-mail account as CAVAD::VSD.
1992 Aug 17    VSD       PERIOD_USER.TEX largely rewritten in order to make
                         document clearer and more comprehensive.
1992 Aug 17    VSD       Sampling interval in MEM option of PERIOD_PERIOD
                         now output in PEAKS instead of in MEM.
\end{terminalv}
\end{small}

\newpage

\begin{small}
\begin{terminalv}
   DATE       Author            Comments

1992 Aug 17    VSD       Dimension of NWK in PERIOD_PERIOD changed from 16
                         to 27. This is because the maximum possible workspace
                         usage (assuming 10000 data points and 10000 frequency
                         steps) in PERIOD_SCARGLE is 262144.
1992 Aug 17    VSD       VERSION 1.2 of PERIOD released to incorporate the
                         above changes. Installed on SUSTAR as local
                         STARLINK software. Copy sent to Erik Kuulkers
                         (Amsterdam).

                         *** VERSION 1.2 ***

1992 Sept 24  VSD       Annoying information written to screen when running
                        PLT removed from PERIOD_PLT.
1993 Mar 3    VSD       Now at ING, not Sussex! PERIOD_PDM and PERIOD_FTEST
                        subroutines added for the new Phase Dispersion
                        Minimization (PDM) option.
1993 Mar 3    VSD       PERIOD_PERIOD extensively modified to cope with the
                        new PDM option. The two major changes are the
                        inclusion of the code to call PERIOD_PDM (which
                        calculates the PDM statistic) and PEAKS, which now
                        does an F-test to determine the significance of a
                        PDM minimum by calling PERIOD_FTEST.
1993 Mar 3    VSD       Error check on zero period added to PERIOD_FAKE,
                        PERIOD_FIT, PERIOD_PHASE and PERIOD_SINE.
1993 Mar 4    VSD       Reduced the size of the INFILEARRAY string in
                        PERIOD_PERIOD by shortening 'Frequency' to 'Freq'.
1993 Mar 4    VSD       Adapted PERIOD_SYSTEM.TEX (LSN 1.2) and
                        PERIOD_USER.TEX (LUN 2.2) to incorporate
                        new PDM option and other changes to
                        bring the documentation into line with the
                        STARLINK release.
1993 Mar 4    VSD       New e-mail address, version number and date
                        added to PERIOD_INTRO.
1993 Mar 4    VSD       PERIOD_COMPILE, PERIOD_LIBRARY, PERIOD_MAIL
                        all modified to include the new PERIOD_PDM and
                        PERIOD_FTEST subroutines.
1993 Mar 4    VSD       VERSION 2.0 of PERIOD released to incorporate
                        the above changes. Installed on ING Vax and
                        copies sent to STARLINK (RLVAD::STAR), Sussex
                        (SUSTAR::SYSTEM) and Amsterdam
                        (PSI\%IBERPAC.DWINGELOO::UVAA01::ERIK)

                         *** VERSION 2.0 ***
\end{terminalv}
\end{small}

\newpage

\begin{small}
\begin{terminalv}
   DATE       Author            Comments

1993 Mar 16    VSD       Bug in FOLD option found by Martin Still (Sussex).
                         Traced to PROMPT_ZEROPT parameter not setting
                         ZEROPT parameter correctly. Modified PERIOD_FIT and
                         PERIOD_PHASE subroutines accordingly (by removing
                         the PROMPT_ZEROPT parameter completely).
1993 Mar 17    VSD       Included an on-line help option. A new subroutine,
                         PERIOD_HELP, reads a set of help files contained
                         in the new PERIOD subdirectory PERIOD_HLP. The help
                         files are named according to the convention:
                         PERIOD_'COMMAND'.HLP. PERIOD_MAIN and
                         PERIOD_PERIOD have been modified accordingly,
                         as have the various command procedures and
                         documents (LSN 1.3 and LUN 2.3).
1993 Mar 17    VSD       Minor changes made to PERIOD_INPUT to reduce
                         references made to RUBY.
1993 Mar 17    VSD       TYPE replaced by WRITE in PERIOD_INPUT and
                         PERIOD_STATUS to conform to standard F77.
1993 Mar 17    VSD       Bug in PERIOD_LOG which prevented the opening
                         of an old log file fixed.
1993 Mar 17    VSD       VERSION 3.0 of PERIOD released to incorporate
                         the above changes. Installed on ING Vax and
                         copies sent to STARLINK (RLVAD::STAR), Sussex
                         (SUSTAR::SYSTEM) and Amsterdam
                         (PSI\%IBERPAC.DWINGELOO::UVAA01::ERIK)
                         This was the first version to be released on STARLINK.

                         *** VERSION 3.0 ***

1993 Jul 5    VSD        Included line number as diagnostic of corrupted
                         x-axis data in PERIOD_INPUT (in response to request
                         from Margaret Penston, RGO).
1993 Aug 21   VSD        Bug in PERIOD_PHASE found by Martin Still (Sussex).
                         When folding data using binning on a number of files
                         the folded data is all placed into a single
                         (incorrect) slot. Problem traced to the use of
                         the COUNTER variable for two different tasks
                         (slot number and data array elements). Changed
                         the data array variable to BINCOUNT. Checked all
                         PERIOD subroutines for a similar error and nothing
                         found.
1993 Aug 2    VSD        Bug in PERIOD_CLEAN found (and corrected) by
                         Christian Knigge (Oxford). When processing a number
                         of files using CLEAN the first periodogram is
                         calculated correctly, but subsequent files are
\end{terminalv}
\end{small}

\newpage

\begin{small}
\begin{terminalv}
   DATE       Author            Comments

                         wrongly processed, as are files processed if CLEAN
                         is run twice without exiting PERIOD. Problem traced
                         to the C array which is now initialised on each
                         call to PERIOD_CLEAN. Checked all PERIOD_PERIOD
                         subroutines for a similar error and nothing
                         found.
1993 Aug 22    VSD       PERIOD_CLEAN made more robust by checking the
                         HWHM calculated by PERIOD_HWHM is not zero before
                         calling PERIOD_FILLB -- this used to cause a crash.
1993 Aug 22    VSD       PEAKS in PERIOD_PERIOD now checks that different
                         numbers are specified for the periodogram and
                         time-series. Also now quits if a 0 is entered for
                         the slot number (this did not work properly before).
1993 Aug 22    VSD       Bug in CHAOS option of PERIOD_FAKE did not reset
                         the initial value parameter if processing more
                         than one slot. Problem remedied by introducing
                         a new parameter -- INITVAL.
1993 Aug 22    VSD       PERIOD_DETREND made more robust by removing the
                         PERIOD_POLYFIT subroutine and calling PERIOD_LSQUAR
                         directly from PERIOD_DETREND. Program now outputs
                         the chi-squared value of the polynomial fit. Also
                         rewrote PERIOD_POLY to handle polynomials of order
                         up to 20 (the previous limit was 12) and removed
                         the redundant subroutine PERIOD_DPOLY. Updated the
                         command procedures in PERIOD_COM by removing any
                         reference to PERIOD_POLYFIT, which has now been
                         deleted from the PERIOD_FOR subdirectory.
1993 Aug 22    VSD       Bug found in DETREND -- the polynomial fit was
                         divided and not subtracted from the time-series.
                         Have modified PERIOD_DETREND accordingly.
1993 Aug 22    VSD       Added my INTERNET e-mail address to PERIOD_INTRO
                         and incremented the date and version number (3.1).
1993 Aug 27    VSD       Removed string length significance limits from
                         PEAKS in PERIOD_PERIOD since misleading (the
                         theoretical minimum only applied to an
                         evenly-spaced sinusoid). Followed comment by
                         Christian Knigge (Oxford).
1993 Aug 27    VSD       Corrected reduced chi-squared values in the CHISQ
                         option of PERIOD_PERIOD and PERIOD_FIT by dividing
                         by the number of points minus the number of free
                         parameters (=3 in PERIOD_SINFIT). Previously, the
                         chi-squared value was only divided by the number
                         of points. Bug spotted by Christian Knigge (Oxford).
\end{terminalv}
\end{small}

\newpage

\begin{small}
\begin{terminalv}
   DATE       Author            Comments

1993 Aug 27    VSD       Removed chi-squared value of polynomial fit from
                         PERIOD_DETREND and replaced with simple RMS
                         calculation. VERSION 3.1 never released.

                         *** VERSION 3.1 ***

1993 Nov 13   VSD       Removed CHAOS option from PERIOD_PERIOD since it is
                        of no use in its present form. Modified PERIOD_HELP
                        and various .COM files and deleted the .HLP file to
                        reflect this change.
1993 Nov 13   VSD       Removed MEM option from PERIOD_PERIOD since it is
                        of no use in its present form. Modified PERIOD_HELP
                        and various .COM files and deleted the .HLP file to
                        reflect this change. Removed the now redundant
                        subroutines PERIOD_EVLMEM, PERIOD_MEMCOF and
                        PERIOD_MDIAN1 from the PERIOD_FOR directory and
                        placed them in the PERIOD_DEV directory.
1993 Nov 13   VSD       Minor bug in PERIOD_NOISE discovered. Removed the
                        option which allows one to add Poissonian noise
                        to the data since it incorrectly just added the
                        SQRT of the data value.
1993 Nov 13   VSD       Removed PERIOD_FTEST from the PERIOD_FOR directory
                        and placed in PERIOD_DEV directory. This routine
                        is now redundant since significances are calculated
                        in PERIOD_PERIOD using a Fisher test (see below).
1993 Nov 14   VSD       Major changes to PERIOD_PERIOD in order to include
                        significance estimate for each periodogram. Removed
                        the estimates from PEAKS (which were all incorrect),
                        added an extra option SIG and modified each of the
                        6 remaining PERIOD finding options in order to
                        perform a Fisher randomisation test. Also included
                        new INFO parameter in PERIOD_FT, PERIOD_SCARGLE
                        and PERIOD_CLEAN.
1993 Nov 14   VSD       Modified PERIOD_INTRO to reflect new version, date
                        and removed request for acknowledgments.
1993 Nov 14   VSD       Modified PERIOD_DETREND and PERIOD_POLY by setting
                        maximum polynomial order from 20 to 10 (since this
                        was causing underflow errors).
1993 Nov 14   VSD       Changed variable type of SEED in PERIOD_NOISE from
                        REAL to INTEGER, since this caused a crash.
1993 Nov 14   VSD       Changed check of minimum number of points in
                        PERIOD_SINFIT from .LT. 3 to .LE. 3 since this
                        caused a crash.
1993 Nov 18   VSD       Updated all the .HLP files to reflect the above
                        changes and included a new file for the SIG option.
\end{terminalv}
\end{small}

\newpage

\begin{small}
\begin{terminalv}
   DATE       Author            Comments


1993 Nov 18    VSD       Restructured PERIOD subdirectories. Now only 4
                         subdirectories -- .FOR, .HLP, .DOC and .SYS. The
                         first three directories contain the fortran source
                         code, help files and user documents (as before). The
                         .SYS directory contains the executable image and the
                         command procedures for initialising and making the
                         PERIOD package.
1993 Nov 18    VSD       Restructured the command procedures. There are now
                         only two: PERIOD_START.COM and PERIOD_MAKE.COM
                         (which calls PERIOD_MAKE.OPT). This is to bring it
                         more into line with STARLINK standard.
                         PERIOD_START.COM now defines the new directories
                         and defines a new logical PERIOD_DISK.
                         PERIOD_MAKE.COM compiles, creates and fills the
                         object library, links and then deletes the object
                         files and object library.
1993 Nov 18    VSD       To bring into line with STARLINK, the PERIOD
                         initialisation command is now PERIODSTART (and not
                         PERIOD_START).
1994 Jan 15    VSD       Included a new subroutine PERIOD_CASE which converts
                         strings from lower to upper case. Replaces calls
                         to the VAX/VMS specific STR\$UPCASE.
1994 Jan 24    VSD       Modified PERIOD_STRING and PERIOD_SINFIT so as not
                         to crash when presented with windowed data. All
                         PERIOD_PERIOD options can now either cope with
                         windowed data or else report an error.
1994 Jan 25    VSD       Changes made to PERIOD_DOC files. This file
                         converted to \LaTeX\ and renamed PERIOD_HISTORY.TEX.
                         PERIOD_SYSTEM.TEX renamed to SSN167.TEX and
                         rewritten to reflect above changes. SUN167.TEX also
                         rewritten to reflect above changes. PERIOD_USER.TEX
                         (the old Sussex LUN) deleted.
1994 Jan 26    VSD       Changes made to PERIOD_INPUT so that if it finds
                         x-axis data which is not in ascending order, it
                         prompts users for whether they would like to sort
                         it or not. If they do, it sorts the data using
                         PERIOD_SHELLSORT. If not, it aborts as before.
                         Follows suggestions by Ed Zuiderwijk (RGO).
1994 Feb 1     VSD       Following extensive testing, VERSION 4.0 of PERIOD
                         released to incorporate the above changes.
                         Installed on ING Vax and copies sent to STARLINK
                         (RLVAD::STAR) and Amsterdam (ERIK@astro.uva.nl).

                         *** VERSION 4.0 ***
\end{terminalv}
\end{small}

\newpage

\begin{small}
\begin{terminalv}
   DATE       Author            Comments

1995 June 15  GJP       PERIOD Version 4.0 ported to UNIX.
                        Functionality remains the same as version 4.0
                        Now requires an environmental variable to
                        find the help files. SUN167 and SSN25 re-written
                        dropping references to VMS.
1995 Sept 26  GJP       Added the OGIP reading option and modified
                        the documents accordingly. SSN167 becomes SSN25.
1995 Oct  12  GJP       Modified SUN167, SSN25 and PERIOD_HISTORY
                        to be compatible with star2html so that WWW
                        versions can be created by latex2html. Also,
                        removed conversion of input files names to upper
                        case. Submitted to STARLINK.
1995 Dec  15  BLY       Removed dependencies on XANADU versions on PLT and
                        FITSIO by a) providing xanlib.a and plt.dhf for
                        supported systems, b) using Starlink version of
                        FITSIO.  Makefile and documents modified accordingly.
                        Hypertext versions of documents created.
                        Starlink Release.

                        *** VERSION 4.1 ***

1997 Mar  10  GJP       PERIOD 4.1 ported to LINUX.
                        QDP PLT routine replaced by PGPLOT with some
                        small loss of interactive plotting capability.
                        Slightly clearer error messages in some routines,
                        modified help files and a minor VAXism in
                        period_output was fixed. A number of routines have
                        been changed to ensure that the variables used
                        are preinitialised to zero.

1997 May  07  BLY       Modified period_poly.f to use integer powers not
                        double precision powers when raising X to higher
                        powers since X can be negative.  Old method caused
                        core dump on Digital Unix.

                        *** VERSION 4.2 ***

                        *** VERSION 5.0 ***

2001 Dec  12  KPD       PERIOD implemented in Double Precision.
                        Dynamic array allocation/deallocation introduced
                        for the majority of the input, output and work
                        arrays.
\end{terminalv}
\end{small}

\end{document}
