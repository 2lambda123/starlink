\documentstyle[11pt]{article}
\pagestyle{myheadings}

%------------------------------------------------------------------------------
\newcommand{\stardoccategory}  {Starlink User Note}
\newcommand{\stardocinitials}  {SUN}
\newcommand{\stardocnumber}    {102.3}
\newcommand{\stardocauthors}   {Malcolm J. Currie}
\newcommand{\stardocdate}      {1993 November 5}
\newcommand{\stardoctitle}     {HDSTRACE --- Listing HDS Data Files}
%------------------------------------------------------------------------------

\newcommand{\stardocname}{\stardocinitials /\stardocnumber}
\renewcommand{\_}{{\tt\char'137}}     % re-centres the underscore
\markright{\stardocname}
\setlength{\textwidth}{160mm}
\setlength{\textheight}{230mm}
\setlength{\topmargin}{-2mm}
\setlength{\oddsidemargin}{0mm}
\setlength{\evensidemargin}{0mm}
\setlength{\parindent}{0mm}
\setlength{\parskip}{\medskipamount}
\setlength{\unitlength}{1mm}

% Fake document compatibility.
\newcommand{\latexhtml}[2]{#1}

%------------------------------------------------------------------------------
% Add any \newcommand or \newenvironment commands here

%  Define SST verbs.
%+
%  Name:
%     LAYOUT.TEX

%  Purpose:
%     Define Latex commands for laying out documentation produced by PROLAT.

%  Language:
%     Latex

%  Type of Module:
%     Data file for use by the PROLAT application.

%  Description:
%     This file defines Latex commands which allow routine documentation
%     produced by the SST application PROLAT to be processed by Latex. The
%     contents of this file should be included in the source presented to
%     Latex in front of any output from PROLAT. By default, this is done
%     automatically by PROLAT.

%  Notes:
%     The definitions in this file should be included explicitly in any file
%     which requires them. The \include directive should not be used, as it
%     may not then be possible to process the resulting document with Latex
%     at a later date if changes to this definitions file become necessary.

%  Authors:
%     RFWS: R.F. Warren-Smith (STARLINK)

%  History:
%     10-SEP-1990 (RFWS):
%        Original version.
%     10-SEP-1990 (RFWS):
%        Added the implementation status section.
%     12-SEP-1990 (RFWS):
%        Added support for the usage section and adjusted various spacings.
%     {enter_further_changes_here}

%  Bugs:
%     {note_any_bugs_here}

%-

%  Define length variables.
\newlength{\sstbannerlength}
\newlength{\sstcaptionlength}

%  Define a \tt font of the required size.
\font\ssttt=cmtt10 scaled 1095

%  Define a command to produce a routine header, including its name,
%  a purpose description and the rest of the routine's documentation.
\newcommand{\sstroutine}[3]{
   \goodbreak
   \rule{\textwidth}{0.5mm}
   \vspace{-7ex}
   \newline
   \settowidth{\sstbannerlength}{{\Large {\bf #1}}}
   \setlength{\sstcaptionlength}{\textwidth}
   \addtolength{\sstbannerlength}{0.5em}
   \addtolength{\sstcaptionlength}{-2.0\sstbannerlength}
   \addtolength{\sstcaptionlength}{-4.45pt}
   \parbox[t]{\sstbannerlength}{\flushleft{\Large {\bf #1}}}
   \parbox[t]{\sstcaptionlength}{\center{\Large #2}}
   \parbox[t]{\sstbannerlength}{\flushright{\Large {\bf #1}}}
   \begin{description}
      #3
   \end{description}
}

%  Format the description section.
\newcommand{\sstdescription}[1]{\item[Description:] #1}

%  Format the usage section.
\newcommand{\sstusage}[1]{\item[Usage:] \mbox{} \\[1.3ex] {\ssttt #1}}

%  Format the invocation section.
\newcommand{\sstinvocation}[1]{\item[Invocation:]\hspace{0.4em}{\tt #1}}

%  Format the arguments section.
\newcommand{\sstarguments}[1]{
   \item[Arguments:] \mbox{} \\
   \vspace{-3.5ex}
   \begin{description}
      #1
   \end{description}
}

%  Format the returned value section (for a function).
\newcommand{\sstreturnedvalue}[1]{
   \item[Returned Value:] \mbox{} \\
   \vspace{-3.5ex}
   \begin{description}
      #1
   \end{description}
}

%  Format the parameters section (for an application).
\newcommand{\sstparameters}[1]{
   \item[Parameters:] \mbox{} \\
   \vspace{-3.5ex}
   \begin{description}
      #1
   \end{description}
}

%  Format the examples section.
\newcommand{\sstexamples}[1]{
   \item[Examples:] \mbox{} \\
   \vspace{-3.5ex}
   \begin{description}
      #1
   \end{description}
}

%  Define the format of a subsection in a normal section.
\newcommand{\sstsubsection}[1]{\item[{#1}] \mbox{} \\}

%  Define the format of a subsection in the examples section.
\newcommand{\sstexamplesubsection}[1]{\item[{\ssttt #1}] \mbox{} \\}

%  Format the notes section.
\newcommand{\sstnotes}[1]{\item[Notes:] \mbox{} \\[1.3ex] #1}

%  Provide a general-purpose format for additional (DIY) sections.
\newcommand{\sstdiytopic}[2]{\item[{\hspace{-0.35em}#1\hspace{-0.35em}:}] \mbox{} \\[1.3ex] #2}

%  Format the implementation status section.
\newcommand{\sstimplementationstatus}[1]{
   \item[{Implementation Status:}] \mbox{} \\[1.3ex] #1}

%  Format the bugs section.
\newcommand{\sstbugs}[1]{\item[Bugs:] #1}

%  Format a list of items while in paragraph mode.
\newcommand{\sstitemlist}[1]{
  \mbox{} \\
  \vspace{-3.5ex}
  \begin{itemize}
     #1
  \end{itemize}
}

%  Define the format of an item.
\newcommand{\sstitem}{\item}

%  End of LAYOUT.TEX layout definitions.
%.
%------------------------------------------------------------------------------

\begin{document}
\thispagestyle{empty}
SCIENCE \& ENGINEERING RESEARCH COUNCIL \hfill \stardocname\\
RUTHERFORD APPLETON LABORATORY\\
{\large\bf Starlink Project\\}
{\large\bf \stardoccategory\ \stardocnumber}
\begin{flushright}
\stardocauthors\\
\stardocdate
\end{flushright}
\vspace{-4mm}
\rule{\textwidth}{0.5mm}
\vspace{5mm}
\begin{center}
{\Large\bf \stardoctitle}
\end{center}
\vspace{5mm}

\section{Introduction}
One of the most popular features of the ADAM environment is the
Hierarchical Data System (HDS) (described in SUN/92 and SG/4).   HDS
enables associated data items to be stored together in a single file and
in a structured fashion.  HDS is also highly flexible, allowing a myriad
of ways to organise your data.  So HDS is jolly useful, but it does have
one drawback---you cannot just type or print an HDS file to see its
contents---you must run a programme.  Obviously, you (or your friendly
programmer) could write some code to list a particular structure, but
this would be inefficient given the number of structures that are
already in use, let alone the ones to come.  What is required is a
single utility that lists the contents of an HDS data structure to your
terminal and, optionally, to a file for printing or documentation. This
is where {\footnotesize HDSTRACE} comes in handy.

{\footnotesize HDSTRACE} lists recursively the name, data type, and
values of an HDS structure or object.  The type is bracketed by $<>$ to
distinguish it from the name ({\it c.f.}\ SGP/38).  Indentation
delineates structures.  The format of the output is flexible; and more
importantly, you can now always see the data values---something that was
often impossible with original version.  You control the formatting by
specifying optional parameters on the command line, otherwise they have
sensible defaults.

Until this release {\footnotesize HDSTRACE} was called {\footnotesize
TRACE}. However, because of a clash with the UNIX command {\bf trace} it
has been renamed, though on VMS the command {\tt TRACE} will continue to
invoke this utility.


\section{Running HDSTRACE}
This document adheres to the normal style, where your input follows
the \$, \% or $>$ prompts.

\subsection{VMS}
{\footnotesize HDSTRACE} runs as an {\footnotesize ADAM} task.
{\footnotesize ADAM} must
therefore be initiated directly or indirectly using the {\tt ADAMSTART}
or {\tt ADAM} command before {\footnotesize HDSTRACE} can be run.

{\footnotesize HDSTRACE} may be run directly from DCL with the following
command.
\small
\begin{verbatim}
    $ HDSTRACE
\end{verbatim}
\normalsize
You will be prompted for the HDS object or structure to be traced.

If you wish to trace several files or objects
it is more efficient to run {\footnotesize HDSTRACE} from ICL.
If you are already using ICL just enter

\small
\begin{verbatim}
    ICL> HDSTRACE
\end{verbatim}
If you are in DCL, the following will start both ICL and {\footnotesize
TRACE}.
\begin{verbatim}
    $ ADAM HDSTRACE
\end{verbatim}
\normalsize
The command can be abbreviated to HDST and the old command {\tt TRACE}
will also work from both VMS and ICL.

\subsection{UNIX}
{\footnotesize HDSTRACE} may be also run directly from a UNIX shell,
\small
\begin{verbatim}
    % hdstrace
\end{verbatim}
\normalsize
There is no initialisation procedure to run first, but the command may
not be abbreviated.

\subsection{Getting Help}
The following command gives help on HDSTRACE.
\small
\begin{verbatim}
    ICL> HELP HDSTRACE
\end{verbatim}
\normalsize
Notice that this only available for ICL since there is no global help
system for Starlink applications.  However, it is possible to access
the full help on {\footnotesize HDSTRACE} from DCL or the UNIX shell by
entering {\tt ??} in response to a prompt for a parameter.

\small
\begin{verbatim}
    % hdstrace
    OBJECT - Object to be examined > ??
\end{verbatim}
\normalsize
\subsection{Parameters}
All but one of the parameters used by {\footnotesize HDSTRACE} are
normally defaulted.  If you wish to be prompted for all or some of the
defaulted values give the {\tt PROMPT} keyword on the command line; you
can terminate the prompting by entering a {\tt $\backslash$} at any
prompt and each remaining parameter will take its default value.
Normally, this should not be necessary as most of the time the defaults
will be satisfactory or just one or two need changing by specifying them
on the command line. Prompting does have one advantage in that a {\tt ?}
may be entered to elicit help on that parameter.

Parameters may be given by keyword, {\it e.g.}\
\small
\begin{verbatim}
    $ HDSTRACE OBJECT=MYFILE NLINES=3
\end{verbatim}
\normalsize
or by position, {\it e.g.}\

\small
\begin{verbatim}
    $ HDSTRACE MYFILE NO 3
\end{verbatim}
\normalsize
where the NO is for the FULL parameter, or by a combination of both
provided the positional parameters come before any keywords, {\it e.g.}\

\small
\begin{verbatim}
    $ HDSTRACE MYFILE NLINES=3
\end{verbatim}
\normalsize
Full details of the parameters, their defaults and their command-line
positions are given in the Appendix.

\subsubsection{Tailoring HDSTRACE}

Beginners can skip over this section.

If you want to alter some of the defaults, or always make
{\footnotesize HDSTRACE} prompt for a normally defaulted parameter,
you can have your own version of the interface file.
To achieve this enter the following on VMS.

\small
\begin{verbatim}
    $ SET DEFAULT disk:[abc.xyz]     ! Substitute the actual disk and directory
                                     ! where you keep private versions of IFLs
    $ COPY TRACE_DIR:TRACE.IFL *
    $ EDIT TRACE.IFL                 ! Make the changes you wish.
    $ DEFINE/GROUP ADAM_IFL disk:[abc.xyz] ! The /GROUP may be removed if you only
                                     ! intend to run TRACE from DCL.
\end{verbatim}
\normalsize
SUN/115 tells you how to interpret the interface file; it describes
the meanings and available options of the various keywords, and the use
of the ADAM\_IFL logical name.  The most likely things that you would
wish to alter are the values of {\em default\/} and {\em vpath}.

On UNIX you would do something like the following.
\small
\begin{verbatim}
    % cd ~abc/xyz                    # Substitute the actual disk and directory
                                     # where you keep private versions of IFLs
    % cp $HDSTRACE_DIR/hdstrace.ifl .
    % vi trace.ifl                   # Make the changes you wish.
    % setenv ADAM_IFL ~abc/xyz:$ADAM_IFL # Append the directory to the
                                     # interface-file search path.
\end{verbatim}
\normalsize
SUN/144 has more details of the ADAM\_IFL environmental variable.
\bigskip

The best way to describe the parameters is to show some examples.

\section{Examples}
\label{se:example}
In the following examples a a UNIX shell is used, but the commands would
also work from ICL or DCL except where noted.

\subsection{Getting Started}
The first HDS file is quite simple.  It is an IMAGE-format file may be
used in {\footnotesize KAPPA}.  Note that on UNIX the HDS file name
is case sensitive.

\small
\begin{verbatim}
    % hdstrace beq

    BEQ  <IMAGE>
       TITLE          <_CHAR*72>      'KAPPA - Glitch'
       DATA_ARRAY(109,64)  <_REAL>    124,92,98,156,171,166,171,181,190,180,
                                      ... 127.32,127.56,128.36,129.4,130.08,130.44
       DATA_MIN       <_REAL>         12
       DATA_MAX       <_REAL>         254

    End of Trace.
\end{verbatim}
\normalsize
The first line tells us that the structure's {\em name}\ is BEQ and its
{\em type\/} is IMAGE.  (Note that the structure's name is not
necessarily the file name.)  The {\tt <>} delimiters are just a
convention to differentiate name from type; they are not part of the
type itself.  The next few lines are indented.  {\footnotesize HDSTRACE}
uses indentation to indicate the position within the hierarchy.  Here
TITLE, DATA\_ARRAY {\it etc.}\ are at the top level of the hierarchy
within the {\em container file}.  (If they are at the top, why is BEQ
negatively indented?  BEQ applies to the file as a whole---a kind of
zeroth level. An analogy is your login directory.  It is at the top of
your hierarchy of files in directories.  However, it is a file itself
stored on disc.) The second line shows that the TITLE is a character
object of length 72, and has a {\em value\/} of {\tt 'KAPPA - Glitch'}.
The third line shows an array.  The dimensions follow the name thus
DATA\_ARRAY has 109 columns and 64 lines.  Since there is insufficient
room to list the values of 6976 elements, {\footnotesize HDSTRACE} lists
the first few and last few values separated by an ellipsis to indicate
that there are missing values.

We can look at a specific object or structure.  For example to look
at just the data array in beq enter:

\small
\begin{verbatim}
    % hdstrace beq.data_array

    BEQ.DATA_ARRAY  <_REAL>
       DATA_ARRAY(109,64)  124,92,98,156,171,166,171,181,190,180,196,205,217,
                           ... 126.6,127.32,127.56,128.36,129.4,130.08,130.44

    End of Trace.
\end{verbatim}
\normalsize
The case of the object name you supply does not matter since in HDS all
object names are in uppercase.

We can look at more of the DATA\_ARRAY as follows.

\small
\begin{verbatim}
    % hdstrace beq.data_array nlines=3

    BEQ  <IMAGE>
       TITLE          <_CHAR*72>      'KAPPA - Glitch'
       DATA_ARRAY(109,64)  <_REAL>    124,92,98,156,171,166,171,181,190,180,
        196,205,217,211,204,210,202,197,201,196,194,192,189,190,191,187,196,
        202,202,206,204,202,191,172,150,*,131,176,192,204,206,182,170,164,161,
        ... 127.16,126.2,126.36,126.6,127.32,127.56,128.36,129.4,130.08,130.44
       DATA_MIN       <_REAL>         12
       DATA_MAX       <_REAL>         254

    End of Trace.
\end{verbatim}
\normalsize
The {\tt nlines=3} specifies that up to a maximum of three lines,
excluding any extra line that shows the final few values, can be
used to present the values.  In order to present more values the
continuation lines are indented one column right of the object's name.
Notice {\tt nlines=3} has no effect on the other objects in the
structure because their values each fit onto a line; had there been
other arrays these too may have up to three lines to list their
values.  The NLINES parameter normally defaults to one, hence we usually
just see the initial values on a single line.

The observant reader will have noticed an asterisk replacing a numerical
value in the third line.  This is shorthand for a bad value, and it
has the added advantage that it is more distinctive.

For inspection of two-dimensional arrays and subsets {\footnotesize KAPPA}
offers a number of applications such as INSPECT and LOOK which may be
more convenient.

If the array was much smaller we could display all the values on a single
line.

\small
\begin{verbatim}
    % hdstrace SMALL.DATA_ARRAY

    SMALL  <IMAGE>
       TITLE          <_CHAR*72>      'KAPPA - Pick2d'
       DATA_ARRAY(3,2)  <_REAL>       124,92,98,156,156,149

    End of Trace.
\end{verbatim}
\normalsize
\subsection{Formatting}

If you wish to use the output for some documentation you might like
to align the trace into neat columns.  In the above examples the
type of the DATA\_ARRAY is displaced because of the dimensions.  The
indentation of the type with respect to the start of the name, and the
values with respect to the start of the type may be controlled by
parameters TYPIND and VALIND respectively.  For example,

\small
\begin{verbatim}
    % hdstrace beq typind=20

    BEQ  <IMAGE>
       TITLE               <_CHAR*72>      'KAPPA - Glitch'
       DATA_ARRAY(109,64)  <_REAL>         124,92,98,156,171,166,171,181,190,
                                           ... 127.56,128.36,129.4,130.08,130.44
       DATA_MIN            <_REAL>         12
       DATA_MAX            <_REAL>         254

    End of Trace.
\end{verbatim}
\normalsize
indents the type five characters more than the default.

There is another way to format the output, and that it to place the
values on a new line rather than appending them to the name and type.
This can help when dealing with moderate-sized arrays.

\small
\begin{verbatim}
    % hdstrace beq newline

    BEQ  <IMAGE>
       TITLE          <_CHAR*72>
        'KAPPA - Glitch'
       DATA_ARRAY(109,64)  <_REAL>
        124,92,98,156,171,166,171,181,190,180,196,205,217,211,204,210,202,197,
        ... 127.16,126.2,126.36,126.6,127.32,127.56,128.36,129.4,130.08,130.44
       DATA_MIN       <_REAL>
        12
       DATA_MAX       <_REAL>
        254

    End of Trace.
\end{verbatim}
\normalsize
A record of the trace may be stored in an text file given by the
parameter LOGFILE.  The width of the output may be altered to 132
characters instead of 80 via parameter WIDEPAGE.

\subsection{Structures}

Let us move on to a different example HDS file to demonstrate some of
the other parameters of {\footnotesize HDSTRACE}.

\small
\begin{verbatim}
    % hdstrace $ADAM_USER/global valind=18

    GLOBAL  <STRUC>
       LUT            <ADAM_PARNAME>     {structure}
          NAMEPTR        <_CHAR*132>        'xmascomet_lut'

       IMAGE_OVERLAY  <ADAM_PARNAME>     {structure}
          NAMEPTR        <_CHAR*132>        'ikonov'

       GRAPHICS_DEVICE  <ADAM_PARNAME>   {structure}
          NAMEPTR        <_CHAR*132>        'ikon'

       IMAGE_DISPLAY  <ADAM_PARNAME>     {structure}
          NAMEPTR        <_CHAR*132>        'ikon'

       DATA_ARRAY     <ADAM_PARNAME>     {structure}
          NAMEPTR        <_CHAR*132>        'cac1d'

       HDSOBJ         <_CHAR*132>        'AST_ROOT:<DATA.DEMO>PSS_DEMO'
       SST_SOURCE     <_CHAR*132>        'TRACE.FOR'

    End of Trace.
\end{verbatim}
\normalsize

This HDS file stores the values of global parameters shared by ADAM
applications.  LUT, GRAPHICS\_DEVICE {\it etc.}\ are structures, and
therefore instead of listing values, {\footnotesize HDSTRACE} writes the
comment {\tt \{structure\}}.  Each of these structures contains one
object, NAMEPTR, which is indented in the trace.  {\footnotesize HDSTRACE} inserts
a blank line after listing the contents at the end of a structure for
clarity.

On VMS the command would be

\small
\begin{verbatim}
    $ HDSTRACE ADAM_USER:GLOBAL VALIND=18
\end{verbatim}
\normalsize
Structures can also be arrays, as in the following example of an NDF.

\small
\begin{verbatim}
    % hdstrace moimp

    MOIMP  <NDF>
       DATA_ARRAY(512,512)  <_REAL>   0,0,0,0,0,0,0,0,0,0,0,0,0,0,0,0,0,0,0,0,
                                      ... 0,0,0,0,0,0,0,0,0,0,0,0,0,0,0,0,0,0,0,0
       TITLE          <_CHAR*16>      'created with pro'
       UNITS          <_CHAR*18>      'NORMALIZED       B'
       AXIS(2)        <AXIS>          {array of structures}

       Contents of AXIS(1)
          DATA_ARRAY(512)  <_REAL>       -3542.08,-3512.186,-3482.291,
                                         ... 11644.38,11674.27,11704.17,11734.06
          LABEL          <_CHAR*18>      'PIXELS           B'

       MORE           <EXT>           {structure}
          FITS(31)       <_CHAR*80>      'SIMPLE  =                    T / St...'
                                         ... 'HISTORY  ESO-DESCRIPTOR...','','END'

    End of Trace.
\end{verbatim}
\normalsize
AXIS is an array of structures.  Therefore it has no values and
{\footnotesize TRACE} puts the comment {\tt \{array of structures\}}.
By default {\footnotesize HDSTRACE} only lists the contents of the first
element of an array of structures.  To obtain a full trace of arrays of
structures you must set the parameter FULL to be true.

\small
\begin{verbatim}
    % hdstrace MOIMP FULL

    MOIMP  <NDF>
       DATA_ARRAY(512,512)  <_REAL>   0,0,0,0,0,0,0,0,0,0,0,0,0,0,0,0,0,0,0,0,
                                      ... 0,0,0,0,0,0,0,0,0,0,0,0,0,0,0,0,0,0,0,0
       TITLE          <_CHAR*16>      'created with pro'
       UNITS          <_CHAR*18>      'NORMALIZED       B'
       AXIS(2)        <AXIS>          {array of structures}

       Contents of AXIS(1)
          DATA_ARRAY(512)  <_REAL>       -3542.08,-3512.186,-3482.291,
                                         ... 11644.38,11674.27,11704.17,11734.06
          LABEL          <_CHAR*18>      'PIXELS           B'

       Contents of AXIS(2)
          DATA_ARRAY(512)  <_REAL>       11734.1,11704.21,11674.31,11644.42,
                                         ... -3482.252,-3512.146,-3542.041
          LABEL          <_CHAR*18>      'PIXELS           B'

       MORE           <EXT>           {structure}
          FITS(31)       <_CHAR*80>      'SIMPLE  =                    T / St...'
                                         ... 'HISTORY  ESO-DESCRIPTOR...','','END'

    End of Trace.
\end{verbatim}
\normalsize

\subsection{Character Arrays}

In MOIMP there is an array of characters---the NDF FITS extension. If
{\footnotesize HDSTRACE} cannot accommodate a string in the space
available it attempts to give the first few characters followed by an
ellipsis to indicate only a part of the string is reported.  Thus in the
above trace only part of the first FITS header card is shown.  Part of
the anti-penultimate header is reported.  The last two lines are blank
and 'END' and are listed in full.  These truncations are not very
convenient. {\footnotesize HDSTRACE} provides a mechanism for placing
each character-array element on a new line.

\small
\begin{verbatim}
    % hdstrace moimp.more.fits newline eachline

    MOIMP.MORE.FITS  <_CHAR*80>
       FITS(31)
        'SIMPLE  =                    T / Standard FITS format', ...
        'END'

    End of Trace.
\end{verbatim}
\normalsize
To list the whole array specify {\tt nlines=all} or give an NLINES with at
least the number of array elements.  Note the object need not
necessarily be the array.  Below we trace the MORE extension.\footnote{
FITSLIST in {\footnotesize KAPPA} offers a more-convenient way of listing the
FITS headers within NDFs.}

\small
\begin{verbatim}
    % hdstrace moimp.more nlines=a newline eachline

    MOIMP.MORE  <EXT>
       FITS(31)       <_CHAR*80>
        'SIMPLE  =                    T / Standard FITS format',
        'BITPIX  =                   32 / No. of bits per pixel',
        'NAXIS   =                    2 / No. of axes in image',
        'NAXIS1  =                  512 / No. of pixels',
        'NAXIS2  =                  512 / No. of pixels',
        'EXTEND  =                    T / FITS extension may be present',
        'BLOCKED =                    T / FITS file may be blocked',
        ' ',
        'BUNIT   = 'NORMALIZED       B  ' / Units of data values',
        'BSCALE  =   7.450580607332E-06 / Scaling factor: r = f*i + z',
        'BZERO   =   1.600000000000E+04 / Zero offset: r = f*i + z',
        ' ',
        'CRPIX1  =   1.000000000000E+00 / Reference pixel',
        'CRVAL1  =  -3.542080000000E+03 / Coordinate at reference pixel',
        'CDELT1  =   2.989460000000E+01 / Coordinate increment per pixel',
        'CTYPE1  = 'PIXELS           B  ' / Units of coordinate',
        'CRPIX2  =   1.000000000000E+00 / Reference pixel',
        'CRVAL2  =   1.173410000000E+04 / Coordinate at reference pixel',
        'CDELT2  =  -2.989460000000E+01 / Coordinate increment per pixel',
        'CTYPE2  = 'PIXELS           B  ' / Units of coordinate',
        ' ',
        'ORIGIN  = 'ESO-MIDAS'          / Written by MIDAS',
        'OBJECT  = 'created with pro'   / MIDAS desc.: IDENT(1)',
        ' ',
        'HISTORY  ESO-DESCRIPTORS START   ................',
        'HISTORY  'HEADER_VERS'    ,'C*1 '   ,    1,   13,'13A1',' ',' '',
        'HISTORY  06-APR-1990',
        'HISTORY',
        'HISTORY  ESO-DESCRIPTORS END     ................',
        ' ',
        'END'

    End of Trace.
\end{verbatim}
\normalsize
The {\tt nlines=all} facility should be used with care; select an individual
component or a structure that does not contain a large array.  The
output from a data array such as found in a typical NDF will be huge.

\subsection{Current Value}

If you wish to examine the same object repeatedly perhaps with other
parameters changed you can use the ACCEPT keyword or its shorthand.
Thus
\small
\begin{verbatim}
    % hdstrace full \\
\end{verbatim}
\normalsize
would give a full trace of the last object traced.

\newpage
\appendix
\section{Reference Section}
\small
\sstroutine{
   HDSTRACE
}{
   Examines the contents of a data-system object
}{
   \sstdescription{
      Data files in ADAM are stored in an hierarchical format (HDS).
      This cannot be read by just typing the file at the terminal or
      spooling it to a printer---a special application is required.
      Now rather than writing separate code to read a variety of
      structures, this application is sufficiently general to examine
      almost all HDS structures or objects.  The examination may also
      be written to an text file as well as being reported to the user.

      For the specified ADAM data-system object ($X$) there are three
      cases which are handled:

      1) $X$ is a {\em primitive}\ object. The value, or the first and
         last few values of $X$ are listed.

      2) $X$ is a {\em structure}.  The contents of the structure are
         listed. If a component is encountered which is itself a
         structure then its contents are listed down to a level of six
         nested structures.

      3) $X$ is an {\em array of structures}.  All elements will be listed
         if parameter FULL is set to TRUE; only the first element will
         be listed when parameter FULL is set to FALSE (default).

      Listings are in the following order: name; dimensions, if any;
      type; and value or comment.  Comments are enclosed in braces.

      The values are normally listed at the end of each line, but may
      start on a new line.  The maximum number of lines of data values
      may also be set.  For all but the smallest arrays where the values
      of all elements can be displayed in the space provided, the last
      few values in the array as well as the first few are presented.
      The last few values appear on a new line, indented the same as
      the line above with the ellipsis notation to indicate any missing
      values.  Note the number of elements shown depends on the number
      of characters that will fit on the line.  The ellipsis notation
      is also used for long character values where there is only room
      to show the first and last few characters.  Bad values appear as
      asterisks.

      The exact layout may be adjusted and is controlled by four
      additional parameters: a) the indentation of the type string with
      respect to the beginning of the name string; b) indentation of the
      value(s) (if not on a new line) with respect to the beginning of
      the type string; and c) the width of the output, 80 or 132.  If
      the name and dimensions do not fit within the space given by
      parameters a) and b), the alignment will be lost because at least
      two spaces will separate the name from the type, or the type from
      the value(s).  The fourth parameter defines how character arrays
      are arranged.  The default is that character-array elements are
      concatenated to fill the available space delimited by commas.  The
      alternative is to write the value of each element on a new line.
      This improves readability for long strings.
   }
   \sstusage{
      HDSTRACE OBJECT [FULL] [NLINES] [TYPIND] [VALIND] [LOGFILE]
         [EACHLINE] [NEWLINE] [WIDEPAGE]
   }
   \sstparameters{
      \sstsubsection{
         EACHLINE = \_LOGICAL (Read)
      }{
         If true the elements of a character array will each appear on
         a separate line.  Otherwise elements fill the available space
         and may span several lines, paragraph style. {\tt [FALSE]}
      }
      \sstsubsection{
         FULL = \_LOGICAL (Read)
      }{
         If true, all the contents of an array of structures will be
         traced, otherwise only the first element is examined. {\tt [FALSE]}
      }
      \sstsubsection{
         LOGFILE = FILENAME (Read)
      }{
         The name of the text file to contain a log of the examination
         of the data object.  Null ({\tt !}) means do not create a log file.
         {\tt [!]}
      }
      \sstsubsection{
         NEWLINE = \_LOGICAL (Read)
      }{
         True indicates that data values are to start on a new line
         below the name and type, and indented from the name.
         Otherwise the values are appended to the same line. {\tt [FALSE]}
      }
      \sstsubsection{
         NLINES = LITERAL (Read)
      }{
         The maximum number of lines in which data values of each
         primitive array component may be displayed, but excluding the
         continuation line used to show the last few values.  Note that
         there may be several data values per line.  There is no
         formatting of the values.  If you require the whole of each
         array use NLINES = {\tt "ALL"}.  Beware this facility can result in
         a large report, so select just the array or arrays you wish to
         trace. {\tt [1]}
      }
      \sstsubsection{
         OBJECT = UNIV (Read)
      }{
         The name of the data-system object to be traced.  This may be
         a whole structure if the name of the container file is given,
         or it may be an object within the container file, or even a
         sub-section of an array component.
      }
      \sstsubsection{
         TYPIND = \_INTEGER (Read)
      }{
         Column indentation of the component's type with respect to
         the current indentation of the component's name.  If the name
         plus dimensions cannot fit in the space provided alignment
         will be lost, since HDSTRACE insists that there be a gap of at
         least two columns.  Note that HDS names can be up to 15
         characters, and the dimension in the format (dim1,dim2,...) is
         abutted to the name. {\tt [15]}
      }
      \sstsubsection{
         VALIND = \_INTEGER (Read)
      }{
         Column indentation of the component's value(s) with respect to
         the current indentation of the component's type provided
         NEWLINE is false.  If, however, NEWLINE is true, VALIND is
         ignored and the value is indented by one column with respect
         to the component's name.  If the type cannot fit in the space
         provided alignment will be lost, since HDSTRACE insists that
         there be a gap of at least two columns.  HDS types can be up
         to 15 characters. {\tt [15]}
      }
      \sstsubsection{
         WIDEPAGE = \_LOGICAL (Read)
      }{
         If true a 132-character-wide format is used to report the
         examination.  Otherwise the format is 80 characters wide.
         {\tt [FALSE]}
      }
   }
   \sstnotes{
      This application allows far more flexibility in layout than
      earlier applications like LS and the original TRACE, though the
      order of the attributes of an object has been fixed and
      rearranged for standardisation, particularly for documentation
      purposes.
   }
}
\end{document}
