\def\hi{\noindent \hangindent=20 pt}
\def\etal{{\it et~al.}}
\def\ie{{\it i.e.} }
\def\eg{{\it e.g.}}
\def\cf{{\it cf.}}
\def\ni{\noindent}
\def\Item{\item {$\bullet$}}
\def\kms{~km\thinspace s$^{-1}$}
\def\cl{\centerline}
\def\Mag{${}^{\rm m}$\llap{.}}
\def\Deg{${}^\circ$\llap{.}}
\def\Arcsec{${}^{\prime\prime}$\llap{.}}
\def\today{\number\year\space \ifcase\month\or
  January\or February\or March\or April\or May\or June\or
  July\or August\or September\or October\or November\or December\fi
  \space\number\day}
\magnification = 1000
\baselineskip = 0.58 true cm
\pretolerance=3000
\tolerance=5000
\vsize=8.25 true in
\hsize=6 true in
\hoffset=1 true cm
\tenrm
\centerline{Notes for people called upon to maintain DAOPHOT:}
\bigskip
     When run, DAOPHOT looks for a file named DAO:DAOPHOT.MSG containing
messages from the local curator for users of DAOPHOT.  This file allows
the curator to notify users of changes in the code or of other items of 
interest.  THEREFORE, I suggest that .EITHER. the directory or 
subdirectory where DAOPHOT-related material resides be given the 
system-wide logical name DAO: (on VMS; set an an environment variable 
dao equal to the directory name in Unix --- DAOPHOT knows how to 
convert ``dao:'' to ``\$dao/''), .OR. all potential users be advised to 
define that logical name in their own LOGIN.COM (or .cshrc) files.

     To the extent possible, I have tried to write DAOPHOT using the
most basic FORTRAN which could perform the job neatly and
comprehensibly.  To those of you without VMS VAXen or Unix Suns:  most
of the changes you have to make will be related to file handling and
will be in the subroutines in VAXSUBS.FOR (sunsubs.f and irasubs.f). I
believe that if you just try to compile everything as is and then
change the cause of each of your error messages to something your
compiler likes, you will get a working copy of DAOPHOT.  I suggest that
you devote a notebook to the recording of all changes that you have to
make to the code, so that if you get a later release of the source
code, you'll already know nearly all of the changes you'll have to make
to arrive at a working copy of the new version.

     For VMS users, I am sending along software which can read a FITS
tape and produce a disk file in the Shortridge/Caltech data-structure
format--  this comprises the files FITS, FITIN, TAPOS, and MTPCKG.  I
am not responsible for this software.  In principle, if you have a VMS
VAX, you should be able to run FITS.EXE.  For FITS to work, the tape
must be positioned at the beginning of a FITS tape file when you
attempt to read the file.  The first thing the program must read from
the tape is ``SIMPLE = T" or it will refuse to read the file.  Issuing
to FITS the command TAPOS will allow you to run another part of the
program which permits fast-forwarding over files, examining the
contents of files record by record, and the like.  The TAPOS part
accepts the command HELP and will then tell you something about what it
does.  When attempting to write pictures to disk in the FITS part of
the program, you must type in the disk filename with no filename
extension; FITS will attach the filename extension ``.DST''.

For Unix users:  DAOPHOT uses IRAF images and libraries.  If you don't
have IRAF, you'll have to do something else.
\vfill
\eject
\noindent This copy of DAOPHOT is being provided to you with the following
understandings:

\item{**}  I don't particularly care whom you allow to use the
executable version of the code at your institution-- it is up to
you to ensure that such people are given enough instruction in the
proper use of the code to avoid wasting your CPU cycles. 

\item{**}  If anyone should ask for a copy of the code, please refer
that person to me.  Do not pass along a copy of the source code, in 
either machine-readable or hard-copy form.  (If a brief examination
of the printout is desired to answer some question which is not
covered in the manual and which you cannot answer off the top of 
your head, that is at your discretion.  What I wish to avoid is for
someone to have a copy of the code for long enough to have any
significant portion of it typed into the computer, or to have it 
photocopied.)  I don't want people modifying the code, passing it along
to other people, who further modify it, and then publish complaints 
about bugs in ``DAOPHOT.''

\item{**}  For obvious reasons, please keep the source code of DAOPHOT
.EITHER. only in a read-protected directory .OR. only on a tape
somewhere. 

\item{**}  If you feel that some changes must be made in the code, 
please label your modifications clearly with C comment statements 
and/or with     ! comments.  This will simplify subsequent bug-hunts
for you and for me.  You probably z    should keep an archival copy of 
the source code exactly as I send it to you.  If you find a bug, or 
think of a better way to do something, please let me know (preferably 
in electronic or hard-copy form) so that I incorporate the improvements
in my copy.

\noindent I am making these requests to avoid having various modified
and remodified versions of the code floating around the astronomical
community, and being used by people who don't really know what it's
doing.  We all know that this sort of thing can and does happen every
day, and it does all of us more harm than good.  Thank you for your
help and understanding.

\vfill
\indent \today
\vfill
\indent Peter B. Stetson 

\indent (604)363--0029$\qquad\qquad$stetson@dao.nrc.ca

\indent Dominion Astrophysical Observatory

\indent 5071 West Saanich Road

\indent Victoria, British Columbia V8X 4M6 
\vfill
\eject
\noindent DAOPHOT and ALLSTAR source code (may be compiled under 
VMS, or may be renamed to $<${\it lower~case\/}$>$.f and compiled under
Unix):

DAOPHOT.FOR

FIND.FOR

FOTOMETRY.FOR

PSF.FOR

PEAK.FOR

NSTAR.FOR

ADDSTAR.FOR

SUBSTAR.FOR

FUDGE.FOR

GROUP.FOR

SORT.FOR

MATHSUBS.FOR

IOSUBS.FOR
\bigskip

ALLSTAR.FOR

ALLSTUBS.FOR

\noindent Subroutines and object libraries for VMS only:

VMSSUBS.FOR

DTA.OLB

ICH.OLB

MTPCKG.OLB

\noindent Subroutines and object libraries for Unix only:

irasubs.f

sunsubs.f

(You will also need the IRAF libraries libimfort, libsys, libvops, and
libos)

\noindent FITS tape reader source code (VMS only):

FITS.FOR

FITIN.FOR

TAPOS.FOR

\noindent VMS executables:

DAOPHOT.EXE

ALLSTAR.EXE

FITS.EXE
\vfill
\eject
\noindent Parameter tables (convert filenames to lower case for Unix):

DAOPHOT.OPT

PHOTO.OPT

ALLSTAR.OPT

\noindent Sample message file:

DAOPHOT.MSG

\noindent Files for making the executables:

DAOLINK.COM  (VMS)

ALLSTLINK.COM (VMS)

FITLINK.COM (VMS)

Makefile (Unix)

\noindent Manuals and other documentation:

CURATOR.TEX (This document.)

DAOPHOTII.TEX

DTA.MEM (VMS only)

MTPCKG.DOC (VMS only)

\end
