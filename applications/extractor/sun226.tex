\documentclass[twoside,11pt]{article}

% ? Specify used packages
% \usepackage{graphicx}        %  Use this one for final production.
% \usepackage[draft]{graphicx} %  Use this one for drafting.
% ? End of specify used packages

\pagestyle{myheadings}

% -----------------------------------------------------------------------------
% ? Document identification
% Fixed part
\newcommand{\stardoccategory}  {Starlink User Note}
\newcommand{\stardocinitials}  {SUN}
\newcommand{\stardocsource}    {sun\stardocnumber}

% Variable part - replace [xxx] as appropriate.
\newcommand{\stardocnumber}    {225.1}
\newcommand{\stardocauthors}   {A.J. Chipperfield\\
                                P.W. Draper}
\newcommand{\stardocdate}      {27 November 1998}
\newcommand{\stardoctitle}     {EXTRACTOR\\
                                An Astronomical Source Detection Program}
\newcommand{\stardocversion}   {1.0}
\newcommand{\stardocmanual}    {User Manual}
\newcommand{\stardocabstract}  {\EXTRACTOR\ will detect sources in
an astronomical image and build a catalogue listing them. It is based on the
popular \SExtractor\ program, has very flexible configuration 
facilities and can handle images and catalogues in a variety of formats.}
% ? End of document identification
% -----------------------------------------------------------------------------

% +
%  Name:
%     sun.tex
%
%  Purpose:
%     Template for Starlink User Note (SUN) documents.
%     Refer to SUN/199
%
%  Authors:
%     AJC: A.J.Chipperfield (Starlink, RAL)
%     BLY: M.J.Bly (Starlink, RAL)
%     PWD: Peter W. Draper (Starlink, Durham University)
%
%  History:
%     17-JAN-1996 (AJC):
%        Original with hypertext macros, based on MDL plain originals.
%     16-JUN-1997 (BLY):
%        Adapted for LaTeX2e.
%        Added picture commands.
%     13-AUG-1998 (PWD):
%        Converted for use with LaTeX2HTML version 98.2 and
%        Star2HTML version 1.3.
%     {Add further history here}
%
% -

\newcommand{\stardocname}{\stardocinitials /\stardocnumber}
\markboth{\stardocname}{\stardocname}
\setlength{\textwidth}{160mm}
\setlength{\textheight}{230mm}
\setlength{\topmargin}{-2mm}
\setlength{\oddsidemargin}{0mm}
\setlength{\evensidemargin}{0mm}
\setlength{\parindent}{0mm}
\setlength{\parskip}{\medskipamount}
\setlength{\unitlength}{1mm}

% -----------------------------------------------------------------------------
%  Hypertext definitions.
%  ======================
%  These are used by the LaTeX2HTML translator in conjunction with star2html.

%  Comment.sty: version 2.0, 19 June 1992
%  Selectively in/exclude pieces of text.
%
%  Author
%    Victor Eijkhout                                      <eijkhout@cs.utk.edu>
%    Department of Computer Science
%    University Tennessee at Knoxville
%    104 Ayres Hall
%    Knoxville, TN 37996
%    USA

%  Do not remove the %begin{latexonly} and %end{latexonly} lines (used by 
%  LaTeX2HTML to signify text it shouldn't process).
%begin{latexonly}
\makeatletter
\def\makeinnocent#1{\catcode`#1=12 }
\def\csarg#1#2{\expandafter#1\csname#2\endcsname}

\def\ThrowAwayComment#1{\begingroup
    \def\CurrentComment{#1}%
    \let\do\makeinnocent \dospecials
    \makeinnocent\^^L% and whatever other special cases
    \endlinechar`\^^M \catcode`\^^M=12 \xComment}
{\catcode`\^^M=12 \endlinechar=-1 %
 \gdef\xComment#1^^M{\def\test{#1}
      \csarg\ifx{PlainEnd\CurrentComment Test}\test
          \let\html@next\endgroup
      \else \csarg\ifx{LaLaEnd\CurrentComment Test}\test
            \edef\html@next{\endgroup\noexpand\end{\CurrentComment}}
      \else \let\html@next\xComment
      \fi \fi \html@next}
}
\makeatother

\def\includecomment
 #1{\expandafter\def\csname#1\endcsname{}%
    \expandafter\def\csname end#1\endcsname{}}
\def\excludecomment
 #1{\expandafter\def\csname#1\endcsname{\ThrowAwayComment{#1}}%
    {\escapechar=-1\relax
     \csarg\xdef{PlainEnd#1Test}{\string\\end#1}%
     \csarg\xdef{LaLaEnd#1Test}{\string\\end\string\{#1\string\}}%
    }}

%  Define environments that ignore their contents.
\excludecomment{comment}
\excludecomment{rawhtml}
\excludecomment{htmlonly}

%  Hypertext commands etc. This is a condensed version of the html.sty
%  file supplied with LaTeX2HTML by: Nikos Drakos <nikos@cbl.leeds.ac.uk> &
%  Jelle van Zeijl <jvzeijl@isou17.estec.esa.nl>. The LaTeX2HTML documentation
%  should be consulted about all commands (and the environments defined above)
%  except \xref and \xlabel which are Starlink specific.

\newcommand{\htmladdnormallinkfoot}[2]{#1\footnote{#2}}
\newcommand{\htmladdnormallink}[2]{#1}
\newcommand{\htmladdimg}[1]{}
\newcommand{\hyperref}[4]{#2\ref{#4}#3}
\newcommand{\htmlref}[2]{#1}
\newcommand{\htmlimage}[1]{}
\newcommand{\htmladdtonavigation}[1]{}

\newenvironment{latexonly}{}{}
\newcommand{\latex}[1]{#1}
\newcommand{\html}[1]{}
\newcommand{\latexhtml}[2]{#1}
\newcommand{\HTMLcode}[2][]{}

%  Starlink cross-references and labels.
\newcommand{\xref}[3]{#1}
\newcommand{\xlabel}[1]{}

%  LaTeX2HTML symbol.
\newcommand{\latextohtml}{\LaTeX2\texttt{HTML}}

%  Define command to re-centre underscore for Latex and leave as normal
%  for HTML (severe problems with \_ in tabbing environments and \_\_
%  generally otherwise).
\renewcommand{\_}{\texttt{\symbol{95}}}

% -----------------------------------------------------------------------------
%  Debugging.
%  =========
%  Remove % on the following to debug links in the HTML version using Latex.

% \newcommand{\hotlink}[2]{\fbox{\begin{tabular}[t]{@{}c@{}}#1\\\hline{\footnotesize #2}\end{tabular}}}
% \renewcommand{\htmladdnormallinkfoot}[2]{\hotlink{#1}{#2}}
% \renewcommand{\htmladdnormallink}[2]{\hotlink{#1}{#2}}
% \renewcommand{\hyperref}[4]{\hotlink{#1}{\S\ref{#4}}}
% \renewcommand{\htmlref}[2]{\hotlink{#1}{\S\ref{#2}}}
% \renewcommand{\xref}[3]{\hotlink{#1}{#2 -- #3}}
%end{latexonly}
% -----------------------------------------------------------------------------
% ? Document specific \newcommand or \newenvironment commands.
\newcommand{\EXTRACTOR}{\texttt{EXTRACTOR}}
\newcommand{\CONVERT}{\texttt{CONVERT}}
\newcommand{\SExtractor}{\texttt{SExtractor}}
\newcommand{\IRAFURL}{http://star-www.rl.ac.uk/iraf/web/iraf-homepage.html}
\newcommand{\FITSURL}{http://fits.gsfc.nasa.gov/}
\newcommand{\MUD}{mud165.ps}
\newcommand{\dash}{--}
\begin{htmlonly}
  \newcommand{\dash}{-}
\end{htmlonly}
%+
%  Name:
%     SST.TEX

%  Purpose:
%     Define LaTeX commands for laying out Starlink routine descriptions.

%  Language:
%     LaTeX

%  Type of Module:
%     LaTeX data file.

%  Description:
%     This file defines LaTeX commands which allow routine documentation
%     produced by the SST application PROLAT to be processed by LaTeX and
%     by LaTeX2html. The contents of this file should be included in the
%     source prior to any statements that make of the sst commnds.

%  Notes:
%     The style file html.sty provided with LaTeX2html needs to be used.
%     This must be before this file.

%  Authors:
%     RFWS: R.F. Warren-Smith (STARLINK)
%     PDRAPER: P.W. Draper (Starlink - Durham University)

%  History:
%     10-SEP-1990 (RFWS):
%        Original version.
%     10-SEP-1990 (RFWS):
%        Added the implementation status section.
%     12-SEP-1990 (RFWS):
%        Added support for the usage section and adjusted various spacings.
%     8-DEC-1994 (PDRAPER):
%        Added support for simplified formatting using LaTeX2html.
%     {enter_further_changes_here}

%  Bugs:
%     {note_any_bugs_here}

%-

%  Define length variables.
\newlength{\sstbannerlength}
\newlength{\sstcaptionlength}
\newlength{\sstexampleslength}
\newlength{\sstexampleswidth}

%  Define a \tt font of the required size.
\latex{\newfont{\ssttt}{cmtt10 scaled 1095}}
\html{\newcommand{\ssttt}{\tt}}

%  Define a command to produce a routine header, including its name,
%  a purpose description and the rest of the routine's documentation.
\newcommand{\sstroutine}[3]{
   \goodbreak
   \rule{\textwidth}{0.5mm}
   \vspace{-7ex}
   \newline
   \settowidth{\sstbannerlength}{{\Large {\bf #1}}}
   \setlength{\sstcaptionlength}{\textwidth}
   \setlength{\sstexampleslength}{\textwidth}
   \addtolength{\sstbannerlength}{0.5em}
   \addtolength{\sstcaptionlength}{-2.0\sstbannerlength}
   \addtolength{\sstcaptionlength}{-5.0pt}
   \settowidth{\sstexampleswidth}{{\bf Examples:}}
   \addtolength{\sstexampleslength}{-\sstexampleswidth}
   \parbox[t]{\sstbannerlength}{\flushleft{\Large {\bf #1}}}
   \parbox[t]{\sstcaptionlength}{\center{\Large #2}}
   \parbox[t]{\sstbannerlength}{\flushright{\Large {\bf #1}}}
   \begin{description}
      #3
   \end{description}
}

%  Format the description section.
\newcommand{\sstdescription}[1]{\item[Description:] #1}

%  Format the usage section.
\newcommand{\sstusage}[1]{\item[Usage:] \mbox{}
\\[1.3ex]{\raggedright \ssttt #1}}

%  Format the invocation section.
\newcommand{\sstinvocation}[1]{\item[Invocation:]\hspace{0.4em}{\tt #1}}

%  Format the arguments section.
\newcommand{\sstarguments}[1]{
   \item[Arguments:] \mbox{} \\
   \vspace{-3.5ex}
   \begin{description}
      #1
   \end{description}
}

%  Format the returned value section (for a function).
\newcommand{\sstreturnedvalue}[1]{
   \item[Returned Value:] \mbox{} \\
   \vspace{-3.5ex}
   \begin{description}
      #1
   \end{description}
}

%  Format the parameters section (for an application).
\newcommand{\sstparameters}[1]{
   \item[Parameters:] \mbox{} \\
   \vspace{-3.5ex}
   \begin{description}
      #1
   \end{description}
}

%  Format the examples section.
\newcommand{\sstexamples}[1]{
   \item[Examples:] \mbox{} \\
   \vspace{-3.5ex}
   \begin{description}
      #1
   \end{description}
}

%  Define the format of a subsection in a normal section.
\newcommand{\sstsubsection}[1]{ \item[{#1}] \mbox{} \\}

%  Define the format of a subsection in the examples section.
\newcommand{\sstexamplesubsection}[2]{\sloppy
\item[\parbox{\sstexampleslength}{\ssttt #1}] \mbox{} \vspace{1.0ex}
\\ #2 }

%  Format the notes section.
\newcommand{\sstnotes}[1]{\item[Notes:] \mbox{} \\[1.3ex] #1}

%  Provide a general-purpose format for additional (DIY) sections.
\newcommand{\sstdiytopic}[2]{\item[{\hspace{-0.35em}#1\hspace{-0.35em}:}]
\mbox{} \\[1.3ex] #2}

%  Format the implementation status section.
\newcommand{\sstimplementationstatus}[1]{
   \item[{Implementation Status:}] \mbox{} \\[1.3ex] #1}

%  Format the bugs section.
\newcommand{\sstbugs}[1]{\item[Bugs:] #1}

%  Format a list of items while in paragraph mode.
\newcommand{\sstitemlist}[1]{
  \mbox{} \\
  \vspace{-3.5ex}
  \begin{itemize}
     #1
  \end{itemize}
}

%  Define the format of an item.
\newcommand{\sstitem}{\item}

%% Now define html equivalents of those already set. These are used by
%  latex2html and are defined in the html.sty files.
\begin{htmlonly}

%  sstroutine.
   \newcommand{\sstroutine}[3]{
      \subsection{#1\xlabel{#1}-\label{#1}#2}
      \begin{description}
         #3
      \end{description}
   }

%  sstdescription
   \newcommand{\sstdescription}[1]{\item[Description:]
      \begin{description}
         #1
      \end{description}
      \\
   }

%  sstusage
   \newcommand{\sstusage}[1]{\item[Usage:]
      \begin{description}
         {\ssttt #1}
      \end{description}
      \\
   }

%  sstinvocation
   \newcommand{\sstinvocation}[1]{\item[Invocation:]
      \begin{description}
         {\ssttt #1}
      \end{description}
      \\
   }

%  sstarguments
   \newcommand{\sstarguments}[1]{
      \item[Arguments:] \\
      \begin{description}
         #1
      \end{description}
      \\
   }

%  sstreturnedvalue
   \newcommand{\sstreturnedvalue}[1]{
      \item[Returned Value:] \\
      \begin{description}
         #1
      \end{description}
      \\
   }

%  sstparameters
   \newcommand{\sstparameters}[1]{
      \item[Parameters:] \\
      \begin{description}
         #1
      \end{description}
      \\
   }

%  sstexamples
   \newcommand{\sstexamples}[1]{
      \item[Examples:] \\
      \begin{description}
         #1
      \end{description}
      \\
   }

%  sstsubsection
   \newcommand{\sstsubsection}[1]{\item[{#1}]}

%  sstexamplesubsection
   \newcommand{\sstexamplesubsection}[2]{\item[{\ssttt #1}] #2}

%  sstnotes
   \newcommand{\sstnotes}[1]{\item[Notes:] #1 }

%  sstdiytopic
   \newcommand{\sstdiytopic}[2]{\item[{#1}] #2 }

%  sstimplementationstatus
   \newcommand{\sstimplementationstatus}[1]{
      \item[Implementation Status:] #1
   }

%  sstitemlist
   \newcommand{\sstitemlist}[1]{
      \begin{itemize}
         #1
      \end{itemize}
      \\
   }
%  sstitem
   \newcommand{\sstitem}{\item}

\end{htmlonly}

%  End of "sst.tex" layout definitions.
%.

% ? End of document specific commands
% -----------------------------------------------------------------------------
%  Title Page.
%  ===========
\renewcommand{\thepage}{\roman{page}}
\begin{document}
\thispagestyle{empty}

%  Latex document header.
%  ======================
\begin{latexonly}
   CCLRC / {\textsc Rutherford Appleton Laboratory} \hfill {\textbf \stardocname}\\
   {\large Particle Physics \& Astronomy Research Council}\\
   {\large Starlink Project\\}
   {\large \stardoccategory\ \stardocnumber}
   \begin{flushright}
   \stardocauthors\\
   \stardocdate
   \end{flushright}
   \vspace{-4mm}
   \rule{\textwidth}{0.5mm}
   \vspace{5mm}
   \begin{center}
   {\Huge\textbf  \stardoctitle \\ [2.5ex]}
   {\LARGE\textbf \stardocversion \\ [4ex]}
   {\Huge\textbf  \stardocmanual}
   \end{center}
   \vspace{5mm}

% ? Add picture here if required for the LaTeX version.
%   e.g. \includegraphics[scale=0.3]{filename.ps}
% ? End of picture

% ? Heading for abstract if used.
   \vspace{10mm}
   \begin{center}
      {\Large\textbf Abstract}
   \end{center}
% ? End of heading for abstract.
\end{latexonly}

%  HTML documentation header.
%  ==========================
\begin{htmlonly}
   \xlabel{}
   \begin{rawhtml} <H1 ALIGN=CENTER> \end{rawhtml}
      \stardoctitle\\
      \stardocversion\\
      \stardocmanual
   \begin{rawhtml} </H1> <HR> \end{rawhtml}

% ? Add picture here if required for the hypertext version.
%   e.g. \includegraphics[scale=0.7]{filename.ps}
% ? End of picture

   \begin{rawhtml} <P> <I> \end{rawhtml}
   \stardoccategory\ \stardocnumber \\
   \stardocauthors \\
   \stardocdate
   \begin{rawhtml} </I> </P> <H3> \end{rawhtml}
      \htmladdnormallink{CCLRC}{http://www.cclrc.ac.uk} /
      \htmladdnormallink{Rutherford Appleton Laboratory}
                        {http://www.cclrc.ac.uk/ral} \\
      \htmladdnormallink{Particle Physics \& Astronomy Research Council}
                        {http://www.pparc.ac.uk} \\
   \begin{rawhtml} </H3> <H2> \end{rawhtml}
      \htmladdnormallink{Starlink Project}{http://star-www.rl.ac.uk/}
   \begin{rawhtml} </H2> \end{rawhtml}
   \htmladdnormallink{\htmladdimg{source.gif} Retrieve hardcopy}
      {http://star-www.rl.ac.uk/cgi-bin/hcserver?\stardocsource}\\

%  HTML document table of contents. 
%  ================================
%  Add table of contents header and a navigation button to return to this 
%  point in the document (this should always go before the abstract \section). 
  \label{stardoccontents}
  \begin{rawhtml} 
    <HR>
    <H2>Contents</H2>
  \end{rawhtml}
  \htmladdtonavigation{\htmlref{\htmladdimg{contents_motif.gif}}
        {stardoccontents}}

% ? New section for abstract if used.
  \section{\xlabel{abstract}Abstract}
% ? End of new section for abstract
\end{htmlonly}

% -----------------------------------------------------------------------------
% ? Document Abstract. (if used)
%  ==================
\stardocabstract
% ? End of document abstract
% -----------------------------------------------------------------------------
% ? Latex document Table of Contents (if used).
%  ===========================================
  \newpage
  \begin{latexonly}
    \setlength{\parskip}{0mm}
    \tableofcontents
    \setlength{\parskip}{\medskipamount}
    \markboth{\stardocname}{\stardocname}
  \end{latexonly}
% ? End of Latex document table of contents
% -----------------------------------------------------------------------------
\cleardoublepage
\renewcommand{\thepage}{\arabic{page}}
\setcounter{page}{1}

% ? Main text
\section{\xlabel{introduction}\label{introduction}Introduction}
\EXTRACTOR\ is basically Emmanuel Bertin's \SExtractor\ (Source-Extractor)
program\footnote{BERTIN E. and ARNOUTS S., 1996, A\&AS 117,393}
packaged as a task for the 
\xref{Starlink Software Environment}{sg4}{}\latex{ (see SG/4)}.
It gets parameters via the Starlink Parameter System and handles image files 
using the Starlink 
\xref{NDF}{sun33}{abstract}
library \latex{(see SUN/33)}
rather than the native FITS handling functions. In this way the program
can read and write different image formats using the NDF library's
\xref{`on-the-fly' conversion facility}{ssn20}{abstract}\latex{ (see SSN/20)}.
You can easily use the Starlink
\xref{\CONVERT}{sun55}{abstract}
package \latex{(see SUN/55)} to set up automatic conversion from or to 
\xref{a large number of different formats}{sun55}{the_default_conversion_commands}
including 
\htmladdnormallink{FITS}{\FITSURL}
and 
\htmladdnormallink{IRAF}{\IRAFURL}
OIF.

Native \SExtractor\ is also included in the distribution of \EXTRACTOR\.
You can find details of \SExtractor's operation, and how to configure it, in 
the 
\htmladdnormallink{\SExtractor\ User's Guide}{\MUD} 
(which is issued as a Starlink Miscellaneous User Document, MUD/165) 
\dash\ you configure \EXTRACTOR\ in much the same way, although 
\htmlref{some restrictions}{some_restrictions}\latex{ (see
Section~\ref{some_restrictions})} will apply.

Here we describe how to run \EXTRACTOR\, and give brief
\htmlref{instructions for running \SExtractor}{running_sextractor}
from the Starlink distribution\latex{ in Section~\ref{running_sextractor}}.

\section{\xlabel{running_extractor}\label{running_extractor}Running \EXTRACTOR}
Like any Starlink Software Environment task, \EXTRACTOR\ may be run from the
Unix shell or from a user-interface such as 
\xref{ICL}{sg5}{abstract}\latex{ (see SG/5)},
where it is controlled via the ADAM message system.
A useful summary of the rich variety of methods of specifying parameter values
for Starlink Environment Programs is given in
\xref{SUN/95}{sun95}{se_param}.

The examples below assume you are running from the Unix shell but the commands
and parameters are exactly the same from ICL.

After normal Starlink startup, at the shell prompt, type:

\begin{quote} \begin{verbatim}
% extract
\end{verbatim} \end{quote}

You will be prompted for the values of parameters CONFIG and IMAGE in turn:

\begin{quote} \begin{verbatim}
CONFIG /'$EXTRACTOR_DIR/config/default.sex'/ >
IMAGE /'image'/ >
\end{verbatim} \end{quote}

where CONFIG is the name of the `preferences' file and IMAGE is the name of 
the image file to be processed.

Suggested values are given between \texttt{//} in the prompt.
You can accept the suggested values by just typing \verb!<RETURN>!, or type in 
new values of your own, terminated by \verb!<RETURN>!.
If you accept the above suggested values, \EXTRACTOR\ will process the NDF
file \texttt{image} using the installed default configuration files to 
produce a catalogue named \texttt{test.cat} (this name is specified in the 
preferences file).

CONFIG has a `default' value of 
\texttt{\$EXTRACTOR\_DIR/config/default.sex}.
This file has some 
\htmlref{sensible defaults}{defaults}
\latex{(described in Section~\ref{defaults})} but you may wish to copy it to 
your own directories and modify it to your taste.
Similarly with the other configuration files. 
Environment variable \texttt{EXTRACTOR\_DIR} is defined, by Starlink startup,
pointing to the directory containing the \EXTRACTOR\ executable image and the
\texttt{config} directory.

Actually, the suggested value given in a prompt will usually be the last value
used. If you want to return to using the `default' value, you can type:
\begin{quote} \begin{verbatim}
% extract reset
\end{verbatim} \end{quote}

You can also provide parameter values on the command line.
Either positionally,
\begin{quote} \begin{verbatim}
% extract image $EXTRACTOR_DIR/config/default.sex
\end{verbatim} \end{quote}
or by keyword,
\begin{quote} \begin{verbatim}
% extract config=$EXTRACTOR_DIR/config/default.sex
\end{verbatim} \end{quote}
(In this case, you would be prompted for IMAGE but not for CONFIG.)

If you want to read an image in one of the 
\xref{formats handled by \CONVERT}{sun55}{the_default_conversion_commands}, 
just start up \CONVERT\ before running \EXTRACTOR\ and specify the image with 
an appropriate extension. For example:

\begin{quote} \begin{verbatim}
% convert

   CONVERT commands are now available -- (Version 1.1-1, 1998 June)

   Defaults for automatic NDF conversion are set.

   Type conhelp for help on CONVERT commands.
   Type "showme sun55" to browse the hypertext documentation.

% extract
CONFIG /'$EXTRACTOR_DIR/config/default.sex'/ >
IMAGE /'image'/ > image.fit
\end{verbatim} \end{quote}
will use the FITS file \texttt{image.fit}, automatically converting it to a
temporary NDF file and then reading the temporary file.

Additional parameters KEYWORDS, NAME and VALUE allow
configuration parameters specified in the preferences file to be overridden 
without editing the file. For example:
\begin{quote} \begin{verbatim}
% extract KEYWORDS
NAME > CATALOG_NAME
VALUE > sky.cat
NAME > CATALOG_TYPE
VALUE > ASCII_SKYCAT
NAME > !
CATALOG_NAME sky.cat
CATALOG_TYPE ASCII_SKYCAT
CONFIG /'$EXTRACTOR_DIR/config/default.sex'/ >
IMAGE /'image'/ >
\end{verbatim} \end{quote}
would change the name and type of the catalogue produced (for this
run only). The list of changes is terminated by replying with the NULL 
response (\texttt{!}) to the prompt.
The 
\htmladdnormallink{\SExtractor\ User's Guide}{\MUD}\latex{ (MUD/165)}
gives a full list of possible parameter names and values but it is only 
sensible to change a few in this way.

\section{\label{running_sextractor}\xlabel{running_sextractor}Running \SExtractor}
The native version of \SExtractor\ is also distributed with \EXTRACTOR.
This works exclusively with FITS files but will provide the full range of
facilities. The executable image will be installed in \texttt{\$EXTRACTOR\_DIR}.
and it is run as described in the
\htmladdnormallink{\SExtractor\ User's Guide}{\MUD}\latex{ (MUD/165)}.

Either put \texttt{\$EXTRACTOR\_DIR} on your \texttt{PATH} or define an alias:
\begin{quote} \begin{verbatim}
% alias sex $EXTRACTOR_DIR/sex
\end{verbatim}\end{quote}
then type:
\begin{quote}
\texttt{\% sex \textit{image} [-c \textit{configuration-file}]} \verb!\! \\
\texttt{? [-\textit{Parameter1 Value1}] [-\textit{Parameter2 Value2}] ...}
\end{quote}
(see the 
\htmladdnormallink{\SExtractor\ User's Guide}{\MUD}\latex{ (MUD/165)}
for details).

Note that if the \texttt{-c} option is omitted, file \texttt{default.sex} in the
current working directory will be used \dash\ if you want to use the installed
default file, you will need to specify it explicitly.

\section{\label{defaults}\xlabel{the_default_configuration}The Default Configuration}
Directory \texttt{\$EXTRACTOR\_DIR/config} contains the default configuration
files and a number of `classical' convolution masks.

The main `preferences' file, \texttt{default.sex} is listed in
Appendix~\ref{default_sex}.
You will see that it uses the default convolution mask and `Neural Network
Weights' file, and specifies that an ASCII\_HEAD-type catalogue named
\texttt{test.cat} is to be produced.

File \texttt{default.param} specifies which parameters are to be placed in the 
catalogue. The defaults are:
\begin{quote}\begin{verbatim}
NUMBER
X_IMAGE
Y_IMAGE
FLUX_ISO
FLUXERR_ISO
FLUX_AUTO
FLUXERR_AUTO
FLUX_MAX
ISOAREA_IMAGE
CXX_IMAGE
CYY_IMAGE
CXY_IMAGE
THETA_IMAGE
ELLIPTICITY
FLAGS
\end{verbatim}\end{quote}

For a full description of the meaning of the configuration parameters, see 
the 
\htmladdnormallink{\SExtractor\ User's Guide}{\MUD}\latex{ (MUD/165)}.

\section{\xlabel{some_restrictions}\label{some_restrictions}Some Restrictions}
\SExtractor\ uses various configuration files to provide numerous processing
options.
Many of these options have not been tried via the \EXTRACTOR\ route so it is
quite possible that problems may arise. Please notify the authors if you
come across any problems in using \EXTRACTOR. Further development and testing
is planned.

\begin{itemize}
\item Some areas which have not been implemented are:
\begin{itemize}
 \item Bad pixel support.
 \item Propagation of anclliary data into check images.
 \item The catalogue cannot be piped to STDOUT.
\end{itemize}
\item Some areas which have not been tried are:
\begin{itemize}
 \item Weight images.
 \item Different measurement images (double-image mode).
 \item Catalogues other than ASCII\_SKYCAT and FITS\_LDAC.
\end{itemize}
\end{itemize}

\section{References}
Bertin E. and Arnouts S., 1996, A\&AS 117,393\\
Bertin E. : MUD/165 : 
SExtractor V2.0 User's Guide (DRAFT).\\
Lawden M.D.: \xref{SG/4}{sg4}{} : 
ADAM \dash\ The Starlink Software Environment.\\
Warren-Smith R.F. : \xref{SUN/33}{sun33}{} : 
NDF \dash\ Routines for Accessing the Extensible
N-Dimensional Data Format.\\
Warren-Smith R.F. : \xref{SSN/20}{ssn20}{} : 
Adding Format Conversion Facilities to the NDF Data
Access Library.\\
Currie M.J. \textit{et al.} : \xref{SUN/55}{sun55}{} : 
CONVERT \dash\ A Format-conversion Package.\\
Bailey J.A. : \xref{SG/5}{sg5}{} : 
ICL \dash\ The Interactive Command Language for ADAM.\\
Currie M.J. : \xref{SUN/95}{sun95}{} : 
KAPPA \dash\ Kernel Application Package.

\newpage
\appendix
\section{Specification of \EXTRACTOR}

\small
\sstroutine{
   EXTRACTOR
}{
   Extracts sources from astronomical images.
}{
   \sstdescription{
      \EXTRACTOR\ is a Starlink Software Environment task which will detect
      sources in an astronomical image and build a catalogue listing them. 
      It is based on the popular \SExtractor\ program, has very flexible 
      configuration facilities and can handle images and catalogues in a 
      variety of formats.

      The program first obtains the name of the image and the preferences 
      file via the Starlink parameter system and reads and writes image files
      using the Starlink NDF library, thus readily handling NDF's and opening
      up the possibility of reading numerous other image formats via the NDF
      library's automatic conversion facility.
   }
   \sstusage{
      extract image config [keywords] [name] [value]
   }
   \sstparameters{
      \sstsubsection{
         IMAGE = LITERAL (Read)
      }{
         The name of the image file to be processed. If it is an NDF,
         the .sdf extension must be omitted.  For other format files
         an appropriate extension (\texttt{.fit}, \texttt{.imh} \textit{etc}.)
         is essential. \texttt{[image]}
      }
      \sstsubsection{
         CONFIG = LITERAL (Read)
      }{
         The name of the preferences file. For details of the preference
         file contents, see \SExtractor\ documentation.
      }
      \sstsubsection{
         KEYWORDS = \_LOGICAL (Read)
      }{
         Whether alternative preferences are to be given to override those
         in the specified preferences file.
         \texttt{[FALSE]}
      }
      \sstsubsection{
         NAME = LITERAL (Read)
      }{
         The name of a preferences file keyword to be overridden. If 
         KEYWORDS is TRUE, pairs of NAME/VALUE parameters will be 
         requested until a null value (\texttt{!}) is given.
      }
      \sstsubsection{
         VALUE = LITERAL (Read)
      }{
         The alternative value for the preferences file keyword specified
         by the corresponding NAME parameter value.
      }
   }
}
\normalsize
\newpage
\section{\label{default_sex}\xlabel{the_default_preferences}The Default
Preferences}
This is a listing of \texttt{\$EXTRACTOR\_DIR/config/default.sex}, the 
installed default preferences file.
\small
\begin{quote} \begin{verbatim}
# Default configuration file for SExtractor V1.2b14 - > 2.0
# EB 23/07/98
# Added environment variables for Starlink release.
# PWD 25/11/98
# (*) indicates parameters which can be omitted from this config file.

#-------------------------------- Catalog ------------------------------------

CATALOG_NAME	test.cat	# name of the output catalog
CATALOG_TYPE	ASCII_HEAD	# "NONE","ASCII_HEAD","ASCII","FITS_1.0"
				# or "FITS_LDAC"

PARAMETERS_NAME	$EXTRACTOR_DIR/config/default.param	
                                # name of the file containing catalog contents

#------------------------------- Extraction ----------------------------------

DETECT_TYPE	CCD		# "CCD" or "PHOTO" (*)
FLAG_IMAGE	flag.fits	# filename for an input FLAG-image
DETECT_MINAREA	5		# minimum number of pixels above threshold
DETECT_THRESH	1.5		# <sigmas> or <threshold>,<ZP> in mag.arcsec-2
ANALYSIS_THRESH	1.5		# <sigmas> or <threshold>,<ZP> in mag.arcsec-2

FILTER		Y		# apply filter for detection ("Y" or "N")?
FILTER_NAME	$EXTRACTOR_DIR/config/default.conv	
                                # name of the file containing the filter

DEBLEND_NTHRESH	32		# Number of deblending sub-thresholds
DEBLEND_MINCONT	0.005		# Minimum contrast parameter for deblending

CLEAN		Y		# Clean spurious detections? (Y or N)?
CLEAN_PARAM	1.0		# Cleaning efficiency

MASK_TYPE	CORRECT		# type of detection MASKing: can be one of
				# "NONE", "BLANK" or "CORRECT"

#------------------------------ Photometry -----------------------------------

PHOT_APERTURES	5		# MAG_APER aperture diameter(s) in pixels
PHOT_AUTOPARAMS	2.5, 3.5	# MAG_AUTO parameters: <Kron_fact>,<min_radius>

SATUR_LEVEL	50000.0		# level (in ADUs) at which arises saturation

MAG_ZEROPOINT	0.0		# magnitude zero-point
MAG_GAMMA	4.0		# gamma of emulsion (for photographic scans)
GAIN		0.0		# detector gain in e-/ADU.
PIXEL_SCALE	1.0		# size of pixel in arcsec (0=use FITS WCS info).

#------------------------- Star/Galaxy Separation ----------------------------

SEEING_FWHM	1.2		# stellar FWHM in arcsec
STARNNW_NAME	$EXTRACTOR_DIR/config/default.nnw	
                                # Neural-Network_Weight table filename

#------------------------------ Background -----------------------------------

BACK_SIZE	64		# Background mesh: <size> or <width>,<height>
BACK_FILTERSIZE	3		# Background filter: <size> or <width>,<height>

BACKPHOTO_TYPE	GLOBAL		# can be "GLOBAL" or "LOCAL" (*)
BACKPHOTO_THICK	24		# thickness of the background LOCAL annulus (*)

#------------------------------ Check Image ----------------------------------

CHECKIMAGE_TYPE	NONE		# can be one of "NONE", "BACKGROUND",
				# "MINIBACKGROUND", "-BACKGROUND", "OBJECTS",
				# "-OBJECTS", "SEGMENTATION", "APERTURES",
				# or "FILTERED" (*)
#CHECKIMAGE_NAME	check		# Filename for the check-image (*)

#--------------------- Memory (change with caution!) -------------------------

MEMORY_OBJSTACK	2000		# number of objects in stack
MEMORY_PIXSTACK	100000		# number of pixels in stack
MEMORY_BUFSIZE	1024		# number of lines in buffer

#----------------------------- Miscellaneous ---------------------------------

VERBOSE_TYPE	NORMAL		# can be "QUIET", "NORMAL" or "FULL" (*)

#------------------------------- New Stuff -----------------------------------
\end{verbatim}\end{quote}
\normalsize

% ? End of main text
\end{document}
