\documentclass[twoside,11pt]{article}

% ? Specify used packages
\usepackage{graphicx}        %  Use this one for final production.
% \usepackage[draft]{graphicx} %  Use this one for drafting.
% ? End of specify used packages

\pagestyle{myheadings}

% -----------------------------------------------------------------------------
% ? Document identification
% Fixed part
\newcommand{\stardoccategory}  {Starlink User Note}
\newcommand{\stardocinitials}  {SUN}
\newcommand{\stardocsource}    {sun\stardocnumber}

% Variable part - replace [xxx] as appropriate.
\newcommand{\stardocnumber}    {225.1}
\newcommand{\stardocauthors}   {A.J. Chipperfield\\
                                P.W. Draper}
\newcommand{\stardocdate}      {1 December 1998}
\newcommand{\stardoctitle}     {EXTRACTOR\\
                                An Astronomical Source Detection Program}
\newcommand{\stardocversion}   {1.0}
\newcommand{\stardocmanual}    {User Manual}
\newcommand{\stardocabstract}  {\EXTRACTOR\ will detect sources in
an astronomical image and build a catalogue listing them. It is based on the
popular \SExtractor\ program, has very flexible configuration
facilities and can handle images and catalogues in a variety of formats.}
% ? End of document identification
% -----------------------------------------------------------------------------

% +
%  Name:
%     sun.tex
%
%  Purpose:
%     Template for Starlink User Note (SUN) documents.
%     Refer to SUN/199
%
%  Authors:
%     AJC: A.J.Chipperfield (Starlink, RAL)
%     BLY: M.J.Bly (Starlink, RAL)
%     PWD: Peter W. Draper (Starlink, Durham University)
%
%  History:
%     17-JAN-1996 (AJC):
%        Original with hypertext macros, based on MDL plain originals.
%     16-JUN-1997 (BLY):
%        Adapted for LaTeX2e.
%        Added picture commands.
%     13-AUG-1998 (PWD):
%        Converted for use with LaTeX2HTML version 98.2 and
%        Star2HTML version 1.3.
%     {Add further history here}
%
% -

\newcommand{\stardocname}{\stardocinitials /\stardocnumber}
\markboth{\stardocname}{\stardocname}
\setlength{\textwidth}{160mm}
\setlength{\textheight}{230mm}
\setlength{\topmargin}{-2mm}
\setlength{\oddsidemargin}{0mm}
\setlength{\evensidemargin}{0mm}
\setlength{\parindent}{0mm}
\setlength{\parskip}{\medskipamount}
\setlength{\unitlength}{1mm}

% -----------------------------------------------------------------------------
%  Hypertext definitions.
%  ======================
%  These are used by the LaTeX2HTML translator in conjunction with star2html.

%  Comment.sty: version 2.0, 19 June 1992
%  Selectively in/exclude pieces of text.
%
%  Author
%    Victor Eijkhout                                      <eijkhout@cs.utk.edu>
%    Department of Computer Science
%    University Tennessee at Knoxville
%    104 Ayres Hall
%    Knoxville, TN 37996
%    USA

%  Do not remove the %begin{latexonly} and %end{latexonly} lines (used by
%  LaTeX2HTML to signify text it shouldn't process).
%begin{latexonly}
\makeatletter
\def\makeinnocent#1{\catcode`#1=12 }
\def\csarg#1#2{\expandafter#1\csname#2\endcsname}

\def\ThrowAwayComment#1{\begingroup
    \def\CurrentComment{#1}%
    \let\do\makeinnocent \dospecials
    \makeinnocent\^^L% and whatever other special cases
    \endlinechar`\^^M \catcode`\^^M=12 \xComment}
{\catcode`\^^M=12 \endlinechar=-1 %
 \gdef\xComment#1^^M{\def\test{#1}
      \csarg\ifx{PlainEnd\CurrentComment Test}\test
          \let\html@next\endgroup
      \else \csarg\ifx{LaLaEnd\CurrentComment Test}\test
            \edef\html@next{\endgroup\noexpand\end{\CurrentComment}}
      \else \let\html@next\xComment
      \fi \fi \html@next}
}
\makeatother

\def\includecomment
 #1{\expandafter\def\csname#1\endcsname{}%
    \expandafter\def\csname end#1\endcsname{}}
\def\excludecomment
 #1{\expandafter\def\csname#1\endcsname{\ThrowAwayComment{#1}}%
    {\escapechar=-1\relax
     \csarg\xdef{PlainEnd#1Test}{\string\\end#1}%
     \csarg\xdef{LaLaEnd#1Test}{\string\\end\string\{#1\string\}}%
    }}

%  Define environments that ignore their contents.
\excludecomment{comment}
\excludecomment{rawhtml}
\excludecomment{htmlonly}

%  Hypertext commands etc. This is a condensed version of the html.sty
%  file supplied with LaTeX2HTML by: Nikos Drakos <nikos@cbl.leeds.ac.uk> &
%  Jelle van Zeijl <jvzeijl@isou17.estec.esa.nl>. The LaTeX2HTML documentation
%  should be consulted about all commands (and the environments defined above)
%  except \xref and \xlabel which are Starlink specific.

\newcommand{\htmladdnormallinkfoot}[2]{#1\footnote{#2}}
\newcommand{\htmladdnormallink}[2]{#1}
\newcommand{\htmladdimg}[1]{}
\newcommand{\hyperref}[4]{#2\ref{#4}#3}
\newcommand{\htmlref}[2]{#1}
\newcommand{\htmlimage}[1]{}
\newcommand{\htmladdtonavigation}[1]{}

\newenvironment{latexonly}{}{}
\newcommand{\latex}[1]{#1}
\newcommand{\html}[1]{}
\newcommand{\latexhtml}[2]{#1}
\newcommand{\HTMLcode}[2][]{}

%  Starlink cross-references and labels.
\newcommand{\xref}[3]{#1}
\newcommand{\xlabel}[1]{}

%  LaTeX2HTML symbol.
\newcommand{\latextohtml}{\LaTeX2\texttt{HTML}}

%  Define command to re-centre underscore for Latex and leave as normal
%  for HTML (severe problems with \_ in tabbing environments and \_\_
%  generally otherwise).
\renewcommand{\_}{\texttt{\symbol{95}}}

% -----------------------------------------------------------------------------
%  Debugging.
%  =========
%  Remove % on the following to debug links in the HTML version using Latex.

% \newcommand{\hotlink}[2]{\fbox{\begin{tabular}[t]{@{}c@{}}#1\\\hline{\footnotesize #2}\end{tabular}}}
% \renewcommand{\htmladdnormallinkfoot}[2]{\hotlink{#1}{#2}}
% \renewcommand{\htmladdnormallink}[2]{\hotlink{#1}{#2}}
% \renewcommand{\hyperref}[4]{\hotlink{#1}{\S\ref{#4}}}
% \renewcommand{\htmlref}[2]{\hotlink{#1}{\S\ref{#2}}}
% \renewcommand{\xref}[3]{\hotlink{#1}{#2 -- #3}}
%end{latexonly}
% -----------------------------------------------------------------------------
% ? Document specific \newcommand or \newenvironment commands.
\newcommand{\EXTRACTOR}{\texttt{EXTRACTOR}}
\newcommand{\CONVERT}{\texttt{CONVERT}}
\newcommand{\SExtractor}{\texttt{SExtractor}}
\newcommand{\IRAFURL}{http://star-www.rl.ac.uk/iraf/web/iraf-homepage.html}
\newcommand{\FITSURL}{http://fits.gsfc.nasa.gov/}
\newcommand{\MUD}{mud165.ps}
\newcommand{\dash}{--}
\begin{htmlonly}
  \newcommand{\dash}{-}
\end{htmlonly}
%+
%  Name:
%     SST.TEX

%  Purpose:
%     Define LaTeX commands for laying out Starlink routine descriptions.

%  Language:
%     LaTeX

%  Type of Module:
%     LaTeX data file.

%  Description:
%     This file defines LaTeX commands which allow routine documentation
%     produced by the SST application PROLAT to be processed by LaTeX and
%     by LaTeX2html. The contents of this file should be included in the
%     source prior to any statements that make of the sst commnds.

%  Notes:
%     The style file html.sty provided with LaTeX2html needs to be used.
%     This must be before this file.

%  Authors:
%     RFWS: R.F. Warren-Smith (STARLINK)
%     PDRAPER: P.W. Draper (Starlink - Durham University)

%  History:
%     10-SEP-1990 (RFWS):
%        Original version.
%     10-SEP-1990 (RFWS):
%        Added the implementation status section.
%     12-SEP-1990 (RFWS):
%        Added support for the usage section and adjusted various spacings.
%     8-DEC-1994 (PDRAPER):
%        Added support for simplified formatting using LaTeX2html.
%     {enter_further_changes_here}

%  Bugs:
%     {note_any_bugs_here}

%-

%  Define length variables.
\newlength{\sstbannerlength}
\newlength{\sstcaptionlength}
\newlength{\sstexampleslength}
\newlength{\sstexampleswidth}

%  Define a \tt font of the required size.
\latex{\newfont{\ssttt}{cmtt10 scaled 1095}}
\html{\newcommand{\ssttt}{\tt}}

%  Define a command to produce a routine header, including its name,
%  a purpose description and the rest of the routine's documentation.
\newcommand{\sstroutine}[3]{
   \goodbreak
   \rule{\textwidth}{0.5mm}
   \vspace{-7ex}
   \newline
   \settowidth{\sstbannerlength}{{\Large {\bf #1}}}
   \setlength{\sstcaptionlength}{\textwidth}
   \setlength{\sstexampleslength}{\textwidth}
   \addtolength{\sstbannerlength}{0.5em}
   \addtolength{\sstcaptionlength}{-2.0\sstbannerlength}
   \addtolength{\sstcaptionlength}{-5.0pt}
   \settowidth{\sstexampleswidth}{{\bf Examples:}}
   \addtolength{\sstexampleslength}{-\sstexampleswidth}
   \parbox[t]{\sstbannerlength}{\flushleft{\Large {\bf #1}}}
   \parbox[t]{\sstcaptionlength}{\center{\Large #2}}
   \parbox[t]{\sstbannerlength}{\flushright{\Large {\bf #1}}}
   \begin{description}
      #3
   \end{description}
}

%  Format the description section.
\newcommand{\sstdescription}[1]{\item[Description:] #1}

%  Format the usage section.
\newcommand{\sstusage}[1]{\item[Usage:] \mbox{}
\\[1.3ex]{\raggedright \ssttt #1}}

%  Format the invocation section.
\newcommand{\sstinvocation}[1]{\item[Invocation:]\hspace{0.4em}{\tt #1}}

%  Format the arguments section.
\newcommand{\sstarguments}[1]{
   \item[Arguments:] \mbox{} \\
   \vspace{-3.5ex}
   \begin{description}
      #1
   \end{description}
}

%  Format the returned value section (for a function).
\newcommand{\sstreturnedvalue}[1]{
   \item[Returned Value:] \mbox{} \\
   \vspace{-3.5ex}
   \begin{description}
      #1
   \end{description}
}

%  Format the parameters section (for an application).
\newcommand{\sstparameters}[1]{
   \item[Parameters:] \mbox{} \\
   \vspace{-3.5ex}
   \begin{description}
      #1
   \end{description}
}

%  Format the examples section.
\newcommand{\sstexamples}[1]{
   \item[Examples:] \mbox{} \\
   \vspace{-3.5ex}
   \begin{description}
      #1
   \end{description}
}

%  Define the format of a subsection in a normal section.
\newcommand{\sstsubsection}[1]{ \item[{#1}] \mbox{} \\}

%  Define the format of a subsection in the examples section.
\newcommand{\sstexamplesubsection}[2]{\sloppy
\item[\parbox{\sstexampleslength}{\ssttt #1}] \mbox{} \vspace{1.0ex}
\\ #2 }

%  Format the notes section.
\newcommand{\sstnotes}[1]{\item[Notes:] \mbox{} \\[1.3ex] #1}

%  Provide a general-purpose format for additional (DIY) sections.
\newcommand{\sstdiytopic}[2]{\item[{\hspace{-0.35em}#1\hspace{-0.35em}:}]
\mbox{} \\[1.3ex] #2}

%  Format the implementation status section.
\newcommand{\sstimplementationstatus}[1]{
   \item[{Implementation Status:}] \mbox{} \\[1.3ex] #1}

%  Format the bugs section.
\newcommand{\sstbugs}[1]{\item[Bugs:] #1}

%  Format a list of items while in paragraph mode.
\newcommand{\sstitemlist}[1]{
  \mbox{} \\
  \vspace{-3.5ex}
  \begin{itemize}
     #1
  \end{itemize}
}

%  Define the format of an item.
\newcommand{\sstitem}{\item}

%% Now define html equivalents of those already set. These are used by
%  latex2html and are defined in the html.sty files.
\begin{htmlonly}

%  sstroutine.
   \newcommand{\sstroutine}[3]{
      \subsection{#1\xlabel{#1}-\label{#1}#2}
      \begin{description}
         #3
      \end{description}
   }

%  sstdescription
   \newcommand{\sstdescription}[1]{\item[Description:]
      \begin{description}
         #1
      \end{description}
      \\
   }

%  sstusage
   \newcommand{\sstusage}[1]{\item[Usage:]
      \begin{description}
         {\ssttt #1}
      \end{description}
      \\
   }

%  sstinvocation
   \newcommand{\sstinvocation}[1]{\item[Invocation:]
      \begin{description}
         {\ssttt #1}
      \end{description}
      \\
   }

%  sstarguments
   \newcommand{\sstarguments}[1]{
      \item[Arguments:] \\
      \begin{description}
         #1
      \end{description}
      \\
   }

%  sstreturnedvalue
   \newcommand{\sstreturnedvalue}[1]{
      \item[Returned Value:] \\
      \begin{description}
         #1
      \end{description}
      \\
   }

%  sstparameters
   \newcommand{\sstparameters}[1]{
      \item[Parameters:] \\
      \begin{description}
         #1
      \end{description}
      \\
   }

%  sstexamples
   \newcommand{\sstexamples}[1]{
      \item[Examples:] \\
      \begin{description}
         #1
      \end{description}
      \\
   }

%  sstsubsection
   \newcommand{\sstsubsection}[1]{\item[{#1}]}

%  sstexamplesubsection
   \newcommand{\sstexamplesubsection}[2]{\item[{\ssttt #1}] #2}

%  sstnotes
   \newcommand{\sstnotes}[1]{\item[Notes:] #1 }

%  sstdiytopic
   \newcommand{\sstdiytopic}[2]{\item[{#1}] #2 }

%  sstimplementationstatus
   \newcommand{\sstimplementationstatus}[1]{
      \item[Implementation Status:] #1
   }

%  sstitemlist
   \newcommand{\sstitemlist}[1]{
      \begin{itemize}
         #1
      \end{itemize}
      \\
   }
%  sstitem
   \newcommand{\sstitem}{\item}

\end{htmlonly}

%  End of "sst.tex" layout definitions.
%.

% ? End of document specific commands
% -----------------------------------------------------------------------------
%  Title Page.
%  ===========
\renewcommand{\thepage}{\roman{page}}
\begin{document}
\thispagestyle{empty}

%  Latex document header.
%  ======================
\begin{latexonly}
   CCLRC / \textsc{Rutherford Appleton Laboratory} \hfill \textbf{\stardocname}\\
   {\large Particle Physics \& Astronomy Research Council}\\
   {\large Starlink Project\\}
   {\large \stardoccategory\ \stardocnumber}
   \begin{flushright}
   \stardocauthors\\
   \stardocdate
   \end{flushright}
   \vspace{-4mm}
   \rule{\textwidth}{0.5mm}
   \vspace{4mm}
   \begin{center}
   {\LARGE\textbf{\stardoctitle} \\ [2.5ex]}
   \vspace{4mm}

% ? Add picture here if required for the LaTeX version.
   \includegraphics[scale=0.6]{sun225fig.ps}
% ? End of picture
   \end{center}

% ? Heading for abstract if used.
   \vspace{5mm}
   \begin{center}
      {\Large\textbf{Abstract}}
   \end{center}
% ? End of heading for abstract.
\end{latexonly}

%  HTML documentation header.
%  ==========================
\begin{htmlonly}
   \xlabel{}
   \begin{rawhtml} <H1 ALIGN=CENTER> \end{rawhtml}
      \stardoctitle\\
      \stardocversion\\
      \stardocmanual
   \begin{rawhtml} </H1> <HR> \end{rawhtml}

% ? Add picture here if required for the hypertext version.
   \htmladdimg{sun225fig.gif}
% ? End of picture

   \begin{rawhtml} <P> <I> \end{rawhtml}
   \stardoccategory\ \stardocnumber \\
   \stardocauthors \\
   \stardocdate
   \begin{rawhtml} </I> </P> <H3> \end{rawhtml}
      \htmladdnormallink{CCLRC}{http://www.cclrc.ac.uk} /
      \htmladdnormallink{Rutherford Appleton Laboratory}
                        {http://www.cclrc.ac.uk/ral} \\
      \htmladdnormallink{Particle Physics \& Astronomy Research Council}
                        {http://www.pparc.ac.uk} \\
   \begin{rawhtml} </H3> <H2> \end{rawhtml}
      \htmladdnormallink{Starlink Project}{http://star-www.rl.ac.uk/}
   \begin{rawhtml} </H2> \end{rawhtml}
   \htmladdnormallink{\htmladdimg{source.gif} Retrieve hardcopy}
      {http://star-www.rl.ac.uk/cgi-bin/hcserver?\stardocsource}\\

%  HTML document table of contents.
%  ================================
%  Add table of contents header and a navigation button to return to this
%  point in the document (this should always go before the abstract \section).
  \label{stardoccontents}
  \begin{rawhtml}
    <HR>
    <H2>Contents</H2>
  \end{rawhtml}
  \htmladdtonavigation{\htmlref{\htmladdimg{contents_motif.gif}}
        {stardoccontents}}

% ? New section for abstract if used.
  \section{\xlabel{abstract}Abstract}
% ? End of new section for abstract
\end{htmlonly}

% -----------------------------------------------------------------------------
% ? Document Abstract. (if used)
%  ==================
\stardocabstract
% ? End of document abstract
% -----------------------------------------------------------------------------
% ? Latex document Table of Contents (if used).
%  ===========================================
  \newpage
  \begin{latexonly}
    \setlength{\parskip}{0mm}
    \tableofcontents
    \setlength{\parskip}{\medskipamount}
    \markboth{\stardocname}{\stardocname}
  \end{latexonly}
% ? End of Latex document table of contents
% -----------------------------------------------------------------------------
\cleardoublepage
\renewcommand{\thepage}{\arabic{page}}
\setcounter{page}{1}

% ? Main text
\section{Introduction}
\EXTRACTOR\ is a program for automatically detecting objects on an
astronomical image and building a catalogue of their properties. It is
particularly suited for the reduction of large scale galaxy-survey
data, but also performs well on other astronomical images.

\EXTRACTOR\ is basically Emmanuel Bertin's \SExtractor\
(Source-Extractor) program\footnote{BERTIN E. and ARNOUTS S., 1996,
A\&AS 117,393} re-packaged for use in the \xref{Starlink Software
Environment}{sg4}{}\latex{ (see SG/4)}.  This means that it uses the
Starlink parameter system, accepts images in the Starlink
\xref{NDF}{sun33}{abstract} format \latex{(see SUN/33)} and uses the
\xref{AST}{sun210}{} library \latex{(SUN/210)} for astrometry.

Extending the program to use NDFs allows it to read and write other
image formats using the NDF library \xref{`on-the-fly' conversion
facility}{ssn20}{abstract} \latex{(see SSN/20)}.  Conversion to and
from many ``standard'' astronomical
\xref{formats}{sun55}{the_default_conversion_commands} are available
by using the \xref{\CONVERT}{sun55}{abstract} package \latex{(see
SUN/55)} -- these include \htmladdnormallink{FITS}{\FITSURL} and
\htmladdnormallink{IRAF}{\IRAFURL} OIF.

Native \SExtractor, version 2.0.15, is also included in the
distribution of \EXTRACTOR.  You can find details of \SExtractor's
operation, and how to configure it, in the
\htmladdnormallink{\SExtractor\ User's Guide}{\MUD} (which is issued
as a Starlink
Miscellaneous User Document, MUD/165\footnote{Note this
document is still in draft consequently you should read this in
conjunction with the A\&AS paper.}) \dash\ you configure \EXTRACTOR\
in much the same way, although \htmlref{some restrictions}{some_restrictions}\latex{ (see
Section~\ref{some_restrictions})} will apply.

This document describes how to run \EXTRACTOR, and gives
\htmlref{instructions for running \SExtractor}{running_sextractor}
from the Starlink distribution.

\section{\EXTRACTOR}

Like any Starlink program, \EXTRACTOR\ may be run from the
Unix shell or from a user-interface such as
\xref{ICL}{sg5}{abstract}\latex{ (see SG/5)}.
A useful summary of the rich variety of methods of specifying parameter values
is given in \xref{SUN/95}{sun95}{se_param}.

The examples below assume you are running from the Unix shell but the commands
and parameters are exactly the same from ICL.

After normal Starlink startup, at the shell prompt, type:

\begin{quote} \begin{verbatim}
% extract
\end{verbatim} \end{quote}

You will be prompted for the values of parameters CONFIG and IMAGE in turn:

\begin{quote} \begin{verbatim}
CONFIG - Configuration file /'$EXTRACTOR_DIR/config/default.sex'/ >
IMAGE - Input image /'image'/ >
\end{verbatim} \end{quote}

where CONFIG is the name of the `preferences' file and IMAGE is the name of
the image file to be processed.

Suggested values are given between \texttt{//} in the prompt.
You can accept the suggested values by just typing \verb!<RETURN>!, or type in
new values of your own.
If you accept the above suggested values, \EXTRACTOR\ will process the NDF
file \texttt{image} using the installed default configuration files to
produce a catalogue named \texttt{test.cat} (this name is specified in the
preferences file).

CONFIG has a `default' value of
\texttt{\$EXTRACTOR\_DIR/config/default.sex}.  This file has some
\htmlref{sensible defaults}{defaults} \latex{(described in
Section~\ref{defaults})} but you will probably want to copy it to your
own directories and modify it to your taste (as you would for the
native \SExtractor program).  Similarly with the other configuration
files.  The environment variable \texttt{EXTRACTOR\_DIR} is defined to
point to the directory containing the \EXTRACTOR\ program and the
\texttt{config} directory.

You can also provide parameter values on the command line.
Either positionally,
\begin{quote} \begin{verbatim}
% extract image $EXTRACTOR_DIR/config/default.sex
\end{verbatim} \end{quote}
or by keyword,
\begin{quote} \begin{verbatim}
% extract config=$EXTRACTOR_DIR/config/default.sex
\end{verbatim} \end{quote}
(In this case, you would be prompted for IMAGE but not for CONFIG.)

The additional parameters KEYWORDS, NAME and VALUE allow
configuration parameters specified in the preferences file to be overridden
without editing the file. For example:
\begin{quote} \begin{verbatim}
% extract KEYWORDS
NAME - Parameter name /!/ > CATALOG_NAME
VALUE - Parameter value /!/ > sky.cat
NAME - Parameter name /!/ > CATALOG_TYPE
VALUE - Parameter value /!/ > ASCII_SKYCAT
NAME - Parameter name /!/ >
CONFIG - Configuration file /'$EXTRACTOR_DIR/config/default.sex'/ >
IMAGE - Input image /'image'/ >
\end{verbatim} \end{quote}
would change the name and type of the catalogue produced (for this
run only). The list of changes is terminated by replying with the NULL
response (\texttt{!}) to the prompt.
The
\htmladdnormallink{\SExtractor\ User's Guide}{\MUD}\latex{ (MUD/165)}
gives a full list of possible parameter names and values but it is only
sensible to change a few in this way.

\section{\label{defaults}\xlabel{defaults}The Default Configuration}

The directory \texttt{\$EXTRACTOR\_DIR/config} contains the default
configuration files.

The main `preferences' file, \texttt{default.sex} is listed in
Appendix~\ref{default_sex}.
You will see that it uses the default convolution mask and `Neural Network
Weights' file, and specifies that an ASCII\_HEAD-type catalogue named
\texttt{test.cat} is to be produced.

File \texttt{default.param} specifies which measurements are to be
made of the objects (these are all written to the output
catalogue). The defaults are:
\begin{quote}\begin{verbatim}
NUMBER
X_IMAGE
Y_IMAGE
FLUX_ISO
FLUXERR_ISO
FLUX_AUTO
FLUXERR_AUTO
FLUX_MAX
ISOAREA_IMAGE
CXX_IMAGE
CYY_IMAGE
CXY_IMAGE
THETA_IMAGE
ELLIPTICITY
FLAGS
\end{verbatim}\end{quote}

For a full description of the meaning of the configuration parameters, see
the
\htmladdnormallink{\SExtractor\ User's Guide}{\MUD}\latex{ (MUD/165)}.

\section{Using images in non-NDF formats}
If you want to read an image in one of the
\xref{formats handled by the
\CONVERT}{sun55}{the_default_conversion_commands} package,
just start up \CONVERT\ before running \EXTRACTOR\ and specify the image with
an appropriate extension. For example:

\begin{quote} \begin{verbatim}
% convert

   CONVERT commands are now available -- (Version 1.1-1, 1998 June)

   Defaults for automatic NDF conversion are set.

   Type conhelp for help on CONVERT commands.
   Type "showme sun55" to browse the hypertext documentation.

% extract
CONFIG - Configuration file /'$EXTRACTOR_DIR/config/default.sex'/ >
IMAGE - Input image /'image'/ > image.fit
\end{verbatim} \end{quote}
will use the FITS file \texttt{image.fit}, automatically converting it to a
temporary NDF file and then reading the temporary file.

\section{Running \SExtractor}
The native version of \SExtractor\ is also distributed with \EXTRACTOR.
This works exclusively with FITS files but will provide the full range of
facilities. The executable image will be installed in \texttt{\$EXTRACTOR\_DIR}.
and it is run as described in the
\htmladdnormallink{\SExtractor\ User's Guide}{\MUD}\latex{ (MUD/165)}.

Either put \texttt{\$EXTRACTOR\_DIR} on your \texttt{PATH} or define an alias:
\begin{quote} \begin{verbatim}
% alias sex $EXTRACTOR_DIR/sex
\end{verbatim}\end{quote}
then type:
\begin{quote}
\texttt{\% sex \textit{image} [-c \textit{configuration-file}]} \verb!\! \\
\texttt{? [-\textit{Parameter1 Value1}] [-\textit{Parameter2 Value2}] ...}
\end{quote}
(see the
\htmladdnormallink{\SExtractor\ User's Guide}{\MUD}\latex{ (MUD/165)}
for details).

Note that if the \texttt{-c} option is omitted, file \texttt{default.sex} in the
current working directory will be used \dash\ if you want to use the installed
default file, you will need to specify it explicitly.

\section{Using GAIA and \EXTRACTOR}
The Graphical Astronomy and Image Analysis Tool (GAIA -
\xref{SUN/214}{sun214}{}), has an interactive toolbox facility that
uses the \EXTRACTOR\ (and \SExtractor) programs. This allows you to
interactively adjust the detection preferences and identifies the
objects that have been detected by drawing suitable ellipses over your
image (in fact GAIA produced the image used on the title page of this
document).

GAIA also displays the results catalogue so you can view the
measurements associated with each object and vice versa (i.e. you can
select one of the displayed ellipses and view the associated
measurements, or select a row of measurements and view the associated
object). The catalogue interface also allows you to sort and select
objects on the basis of their measurements.

\section{\label{some_restrictions}\xlabel{some_restrictions}Some Restrictions}

\SExtractor\ uses various configuration files to provide numerous processing
options.
Many of these options have not been tried via the \EXTRACTOR\ route so it is
quite possible that problems may arise. Please notify the authors if you
come across any problems in using \EXTRACTOR. Further development and testing
is planned.

\begin{itemize}
\item Some areas which have not been implemented are:
\begin{itemize}
 \item Bad pixel support.
 \item Propagation of ancillary data into check images.
 \item The catalogue cannot be piped to STDOUT.
\end{itemize}
\item Some areas which have not been tried are:
\begin{itemize}
 \item Weight images.
 \item Flag images.
\end{itemize}
\end{itemize}

\section{References}
Bailey J.A. : \xref{SG/5}{sg5}{} : ICL \dash\ The Interactive Command Language for ADAM.\\
Bertin E. : MUD/165 : SExtractor V2.0 User's Guide (DRAFT).\\
Bertin E. and Arnouts S., 1996, A\&AS 117,393\\
Currie M.J. : \xref{SUN/95}{sun95}{} : KAPPA \dash\ Kernel Application Package.
Currie M.J. \textit{et al.} : \xref{SUN/55}{sun55}{} : CONVERT \dash\ A Format-conversion Package.\\
Davenhall, A.C. : \xref{SUN/190} : CURSA \dash\ Catalogue and Table Manipulation Applications.\\
Draper, P.W. : \xref{SUN/214} : GAIA \dash\ Graphical Astronomy and Image Analysis Tool.
Lawden M.D.: \xref{SG/4}{sg4}{} : ADAM \dash\ The Starlink Software Environment.\\
Warren-Smith R.F. : \xref{SSN/20}{ssn20}{} : Adding Format Conversion Facilities to the NDF Data Access Library.\\
Warren-Smith R.F. : \xref{SUN/33}{sun33}{} : NDF \dash\ Routines for Accessing the Extensible N-Dimensional Data Format.\\

\newpage
\appendix
\section{Specification of \EXTRACTOR}

\small
\sstroutine{
   EXTRACTOR
}{
   Extracts sources from astronomical images.
}{
   \sstdescription{
      \EXTRACTOR\ is a program for detecting and measuring the
      properties of all the sources on an astronomical image. It
      offers a large range of configuration options for controlling
      the way that objects are detected and the measurements that are
      made of them.

      The source measurements are written to a catalogue (which can be
      of several different formats), so that they can be analysed
      (possibly by a catalogue handling package like CURSA
      -- \xref{SUN190}{sun190}{}).

      \EXTRACTOR, is based on the \SExtractor\ program which is
      described in the \SExtractor's User Guide
      (\xref{MUD/165}{mud165}{}). Consult this about all the various
      options that are available and for the rationale behind the
      program.
   }
   \sstusage{
      extract image config [keywords] [name] [value]
   }
   \sstparameters{
      \sstsubsection{
         IMAGE = LITERAL (Read)
      }{
        The name of the image which contains the objects you wanted
        detected and parameterised. If you have initialised the
        CONVERT package (see \xref{SUN/55}{sun55}{}) then you may
        process foreign formats, such as FITS and IRAF.

        Using this parameter you may give two image files. The
        first image will be used for detection and parameterising
        and the second will be used to actually measure the data
        values. Using this method allows you to measure the same
        objects many images, or two use a high signal to noise image
        to determine the measurement regions on a low signal to noise
        image

        [global\_data\_file]
      }
      \sstsubsection{
         CONFIG = LITERAL (Read)
      }{
        The name of the file that contains the many program
        parameters (things like the threshold for object
        detection). This is initially a file named \texttt{default.sex}
        that can be found in the directory
        \texttt{\$EXTRACTOR\_DIR/config}. To
        modify the parameters used by this program, you must take a
        copy of this file and edit it. Guidance about the values
        that parameters can take may be found in this file as well
        as in the associated \SExtractor\ documentation (see MUD/165).

        The measurements made are determined by a list of parameters
        in the file \\
        \texttt{\$EXTRACTOR\_DIR/config/default.param}. Again if
        you want measurements that are not available by default, you
        must take a copy of this file and edit it. Remember to
        also change \texttt{default.sex} to use this file (otherwise you
        will continue to use the system-wide defaults).

        One off modifications of parameters can be made using the
        KEYWORDS, NAME and VALUE parameters.

        [\texttt{\$EXTRACTOR\_DIR/config/default.sex}]
      }
      \sstsubsection{
         KEYWORDS = \_LOGICAL (Read)
      }{
       Whether you want to enter a series of parameter names and
       values interactively. If TRUE then the parameters NAME and
       VALUE are used to cyclically prompt for program parameters
       and the values you want to use. To end the cycle respond
       with a null symbol (!)

       [FALSE]
      }
      \sstsubsection{
         NAME = LITERAL (Read)
      }{
       The name of a preferences parameter that you want to
       set interactively. Respond with \texttt{!} when you have no
       more to enter.
       [!]
      }
      \sstsubsection{
         VALUE = LITERAL (Read)
      }{
        The value of the parameter you have just specified using the
        NAME prompt.
        [!]
      }
   }
}
\normalsize
\newpage
\section{\label{default_sex}
         \xlabel{default_sex}
         The Default Preferences}

This is a listing of \texttt{\$EXTRACTOR\_DIR/config/default.sex}, the
installed default preferences file.
\small
\begin{quote} \begin{verbatim}
# Default configuration file for SExtractor V1.2b14 - > 2.0
# EB 23/07/98
# Added environment variables for Starlink release.
# PWD 25/11/98
# (*) indicates parameters which can be omitted from this config file.

#-------------------------------- Catalog ------------------------------------

CATALOG_NAME    test.cat        # name of the output catalog
CATALOG_TYPE    ASCII_HEAD      # "NONE","ASCII_HEAD","ASCII","FITS_1.0"
                                # or "FITS_LDAC"

PARAMETERS_NAME $EXTRACTOR_DIR/config/default.param
                                # name of the file containing catalog contents

#------------------------------- Extraction ----------------------------------

DETECT_TYPE     CCD             # "CCD" or "PHOTO" (*)
FLAG_IMAGE      flag.fits       # filename for an input FLAG-image
DETECT_MINAREA  5               # minimum number of pixels above threshold
DETECT_THRESH   1.5             # <sigmas> or <threshold>,<ZP> in mag.arcsec-2
ANALYSIS_THRESH 1.5             # <sigmas> or <threshold>,<ZP> in mag.arcsec-2

FILTER          Y               # apply filter for detection ("Y" or "N")?
FILTER_NAME     $EXTRACTOR_DIR/config/default.conv
                                # name of the file containing the filter

DEBLEND_NTHRESH 32              # Number of deblending sub-thresholds
DEBLEND_MINCONT 0.005           # Minimum contrast parameter for deblending

CLEAN           Y               # Clean spurious detections? (Y or N)?
CLEAN_PARAM     1.0             # Cleaning efficiency

MASK_TYPE       CORRECT         # type of detection MASKing: can be one of
                                # "NONE", "BLANK" or "CORRECT"

#------------------------------ Photometry -----------------------------------

PHOT_APERTURES  5               # MAG_APER aperture diameter(s) in pixels
PHOT_AUTOPARAMS 2.5, 3.5        # MAG_AUTO parameters: <Kron_fact>,<min_radius>

SATUR_LEVEL     50000.0         # level (in ADUs) at which arises saturation

MAG_ZEROPOINT   0.0             # magnitude zero-point
MAG_GAMMA       4.0             # gamma of emulsion (for photographic scans)
GAIN            0.0             # detector gain in e-/ADU.
PIXEL_SCALE     1.0             # size of pixel in arcsec (0=use FITS WCS info).

#------------------------- Star/Galaxy Separation ----------------------------

SEEING_FWHM     1.2             # stellar FWHM in arcsec
STARNNW_NAME    $EXTRACTOR_DIR/config/default.nnw
                                # Neural-Network_Weight table filename

#------------------------------ Background -----------------------------------

BACK_SIZE       64              # Background mesh: <size> or <width>,<height>
BACK_FILTERSIZE 3               # Background filter: <size> or <width>,<height>

BACKPHOTO_TYPE  GLOBAL          # can be "GLOBAL" or "LOCAL" (*)
BACKPHOTO_THICK 24              # thickness of the background LOCAL annulus (*)

#------------------------------ Check Image ----------------------------------

CHECKIMAGE_TYPE NONE            # can be one of "NONE", "BACKGROUND",
                                # "MINIBACKGROUND", "-BACKGROUND", "OBJECTS",
                                # "-OBJECTS", "SEGMENTATION", "APERTURES",
                                # or "FILTERED" (*)
#CHECKIMAGE_NAME        check           # Filename for the check-image (*)

#--------------------- Memory (change with caution!) -------------------------

MEMORY_OBJSTACK 2000            # number of objects in stack
MEMORY_PIXSTACK 100000          # number of pixels in stack
MEMORY_BUFSIZE  1024            # number of lines in buffer

#----------------------------- Miscellaneous ---------------------------------

VERBOSE_TYPE    NORMAL          # can be "QUIET", "NORMAL" or "FULL" (*)

#------------------------------- New Stuff -----------------------------------
\end{verbatim}\end{quote}
\normalsize

% ? End of main text
\end{document}
