%+
%  Name:
%     SST.TEX

%  Purpose:
%     Define LaTeX commands for laying out Starlink routine descriptions.

%  Language:
%     LaTeX

%  Type of Module:
%     LaTeX data file.

%  Description:
%     This file defines LaTeX commands which allow routine documentation
%     produced by the SST application PROLAT to be processed by LaTeX and
%     by LaTeX2html. The contents of this file should be included in the
%     source prior to any statements that make of the sst commnds.

%  Notes:
%     The style file html.sty provided with LaTeX2html needs to be used.
%     This must be before this file.

%  Authors:
%     RFWS: R.F. Warren-Smith (STARLINK)
%     PDRAPER: P.W. Draper (Starlink - Durham University)

%  History:
%     10-SEP-1990 (RFWS):
%        Original version.
%     10-SEP-1990 (RFWS):
%        Added the implementation status section.
%     12-SEP-1990 (RFWS):
%        Added support for the usage section and adjusted various spacings.
%     8-DEC-1994 (PDRAPER):
%        Added support for simplified formatting using LaTeX2html.
%     {enter_further_changes_here}

%  Bugs:
%     {note_any_bugs_here}

%-

%  Define length variables.
\newlength{\sstbannerlength}
\newlength{\sstcaptionlength}
\newlength{\sstexampleslength}
\newlength{\sstexampleswidth}

%  Define a \tt font of the required size.
\latex{\newfont{\ssttt}{cmtt10 scaled 1095}}
\html{\newcommand{\ssttt}{\tt}}

%  Define a command to produce a routine header, including its name,
%  a purpose description and the rest of the routine's documentation.
\newcommand{\sstroutine}[3]{
   \goodbreak
   \rule{\textwidth}{0.5mm}
   \vspace{-7ex}
   \newline
   \settowidth{\sstbannerlength}{{\Large {\bf #1}}}
   \setlength{\sstcaptionlength}{\textwidth}
   \setlength{\sstexampleslength}{\textwidth}
   \addtolength{\sstbannerlength}{0.5em}
   \addtolength{\sstcaptionlength}{-2.0\sstbannerlength}
   \addtolength{\sstcaptionlength}{-5.0pt}
   \settowidth{\sstexampleswidth}{{\bf Examples:}}
   \addtolength{\sstexampleslength}{-\sstexampleswidth}
   \parbox[t]{\sstbannerlength}{\flushleft{\Large {\bf #1}}}
   \parbox[t]{\sstcaptionlength}{\center{\Large #2}}
   \parbox[t]{\sstbannerlength}{\flushright{\Large {\bf #1}}}
   \begin{description}
      #3
   \end{description}
}

%  Format the description section.
\newcommand{\sstdescription}[1]{\item[Description:] #1}

%  Format the usage section.
\newcommand{\sstusage}[1]{\item[Usage:] \mbox{}
\\[1.3ex]{\raggedright \ssttt #1}}

%  Format the invocation section.
\newcommand{\sstinvocation}[1]{\item[Invocation:]\hspace{0.4em}{\tt #1}}

%  Format the arguments section.
\newcommand{\sstarguments}[1]{
   \item[Arguments:] \mbox{} \\
   \vspace{-3.5ex}
   \begin{description}
      #1
   \end{description}
}

%  Format the returned value section (for a function).
\newcommand{\sstreturnedvalue}[1]{
   \item[Returned Value:] \mbox{} \\
   \vspace{-3.5ex}
   \begin{description}
      #1
   \end{description}
}

%  Format the parameters section (for an application).
\newcommand{\sstparameters}[1]{
   \item[Parameters:] \mbox{} \\
   \vspace{-3.5ex}
   \begin{description}
      #1
   \end{description}
}

%  Format the examples section.
\newcommand{\sstexamples}[1]{
   \item[Examples:] \mbox{} \\
   \vspace{-3.5ex}
   \begin{description}
      #1
   \end{description}
}

%  Define the format of a subsection in a normal section.
\newcommand{\sstsubsection}[1]{ \item[{#1}] \mbox{} \\}

%  Define the format of a subsection in the examples section.
\newcommand{\sstexamplesubsection}[2]{\sloppy
\item[\parbox{\sstexampleslength}{\ssttt #1}] \mbox{} \vspace{1.0ex}
\\ #2 }

%  Format the notes section.
\newcommand{\sstnotes}[1]{\item[Notes:] \mbox{} \\[1.3ex] #1}

%  Provide a general-purpose format for additional (DIY) sections.
\newcommand{\sstdiytopic}[2]{\item[{\hspace{-0.35em}#1\hspace{-0.35em}:}]
\mbox{} \\[1.3ex] #2}

%  Format the implementation status section.
\newcommand{\sstimplementationstatus}[1]{
   \item[{Implementation Status:}] \mbox{} \\[1.3ex] #1}

%  Format the bugs section.
\newcommand{\sstbugs}[1]{\item[Bugs:] #1}

%  Format a list of items while in paragraph mode.
\newcommand{\sstitemlist}[1]{
  \mbox{} \\
  \vspace{-3.5ex}
  \begin{itemize}
     #1
  \end{itemize}
}

%  Define the format of an item.
\newcommand{\sstitem}{\item}

%% Now define html equivalents of those already set. These are used by
%  latex2html and are defined in the html.sty files.
\begin{htmlonly}

%  sstroutine.
   \newcommand{\sstroutine}[3]{
      \subsection{#1\xlabel{#1}-\label{#1}#2}
      \begin{description}
         #3
      \end{description}
   }

%  sstdescription
   \newcommand{\sstdescription}[1]{\item[Description:]
      \begin{description}
         #1
      \end{description}
      \\
   }

%  sstusage
   \newcommand{\sstusage}[1]{\item[Usage:]
      \begin{description}
         {\ssttt #1}
      \end{description}
      \\
   }

%  sstinvocation
   \newcommand{\sstinvocation}[1]{\item[Invocation:]
      \begin{description}
         {\ssttt #1}
      \end{description}
      \\
   }

%  sstarguments
   \newcommand{\sstarguments}[1]{
      \item[Arguments:] \\
      \begin{description}
         #1
      \end{description}
      \\
   }

%  sstreturnedvalue
   \newcommand{\sstreturnedvalue}[1]{
      \item[Returned Value:] \\
      \begin{description}
         #1
      \end{description}
      \\
   }

%  sstparameters
   \newcommand{\sstparameters}[1]{
      \item[Parameters:] \\
      \begin{description}
         #1
      \end{description}
      \\
   }

%  sstexamples
   \newcommand{\sstexamples}[1]{
      \item[Examples:] \\
      \begin{description}
         #1
      \end{description}
      \\
   }

%  sstsubsection
   \newcommand{\sstsubsection}[1]{\item[{#1}]}

%  sstexamplesubsection
   \newcommand{\sstexamplesubsection}[2]{\item[{\ssttt #1}] #2}

%  sstnotes
   \newcommand{\sstnotes}[1]{\item[Notes:] #1 }

%  sstdiytopic
   \newcommand{\sstdiytopic}[2]{\item[{#1}] #2 }

%  sstimplementationstatus
   \newcommand{\sstimplementationstatus}[1]{
      \item[Implementation Status:] #1
   }

%  sstitemlist
   \newcommand{\sstitemlist}[1]{
      \begin{itemize}
         #1
      \end{itemize}
      \\
   }
%  sstitem
   \newcommand{\sstitem}{\item}

\end{htmlonly}

%  End of "sst.tex" layout definitions.
%.

