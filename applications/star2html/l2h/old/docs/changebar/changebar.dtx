% \iffalse meta-comment
%
% Copyright 1994-1995 Johannes Braams.  All rights reserved.
% 
% For further copyright information see and any other copyright notices
% in this file.
% 
%   This file is distributed in the hope that it will be useful,
%   but WITHOUT ANY WARRANTY; without even the implied warranty of
%   MERCHANTABILITY or FITNESS FOR A PARTICULAR PURPOSE.
% 
%   For error reports concerning UNCHANGED versions of this file no more
%   than one year old, send a bug report, using latex-bugs.tex to me at
%   the address jlbraams@cistron.nl.
% 
%   Please do not request updates from me directly.  Primary
%   distribution is through the CTAN archives.
% 
% 
% IMPORTANT COPYRIGHT NOTICE:
% 
% You are NOT ALLOWED to distribute this file alone.
% 
% You are allowed to distribute this file under the condition that it is
% distributed together with all the files listed in manifest.txt.
% 
% If you receive only some of these files from someone, complain!
% 
% Permission is granted to copy this file to another file with a clearly
% different name and to customize the declarations in that copy to serve
% the needs of your installation.
% 
% However, NO PERMISSION is granted to produce or to distribute a
% modified version of this file under its original name.
%  
% You are NOT ALLOWED to change this file.
% 
% 
% \fi
% \CheckSum{909}
%
%    This file is built for \LaTeXe, so we make sure an error is
%    generated when it is used with another format
%    \begin{macrocode}
%<+package>\NeedsTeXFormat{LaTeX2e}
%    \end{macrocode}
%
%    Now announce the package name and its version:
%    \begin{macrocode}
%<*dtx>
\ProvidesFile{changebar.dtx}
%</dtx>
%<+package>\ProvidesPackage{changebar}
           [1995/06/27 v3.2b Indicate changes with marginal bars]
%    \end{macrocode}
%
% \section{A driver for this document}
%
%    The next bit of code contains the documentation driver file for
%    \TeX{}, i.e., the file that will produce the documentation you
%    are currently reading. It will be extracted from this file by the
%    \textsc{docstrip} program.
%
%    \begin{macrocode}
%<*driver>
\documentclass{ltxdoc}
%    \end{macrocode}
%    We do want an index, using linenumbers
%    \begin{macrocode}
\EnableCrossrefs
\CodelineIndex
%    \end{macrocode}
%    This document uses some extra \texttt{doc} style macros.
%    \begin{macrocode}
\makeatletter
\def\DescribeVar#1{\leavevmode\@bsphack
    \marginpar{\raggedleft\PrintDescribeEnv{#1}}%
    \SpecialVarUsageIndex{#1}\@esphack\ignorespaces}
\def\SpecialVarMainIndex#1{\@bsphack
    \index{#1\actualchar{\tt #1}\encapchar main}%
    \@esphack}
\def\SpecialVarUsageIndex#1{\@bsphack
    \index{#1\actualchar{\tt #1}\encapchar usage}%
    \@esphack}
\let\@SpecialMainIndex\SpecialMainIndex
\def\Var{\let\SpecialMainIndex\SpecialVarMainIndex\macro}
\def\endVar{\endmacro\let\SpecialMainIndex\@SpecialMainIndex}
\makeatother
%    \end{macrocode}
%    Some commonly used abbreviations
%    \begin{macrocode}
\newcommand\Lopt[1]{\textsf {#1}}
\newcommand\Lenv[1]{\textsf {#1}}
\newcommand\file[1]{\texttt {#1}}
\newcommand\Lcount[1]{\textsl {\small#1}}
\newcommand\pstyle[1]{\textsl {#1}}
%    \end{macrocode}
%    We also want the full details.
%    \begin{macrocode}
\begin{document}
\DocInput{changebar.dtx}
\end{document}
%</driver>
%    \end{macrocode}
% \GetFileInfo{changebar.dtx}
%
% \changes{3.1}{1993/07/02}{Removed a number of typos}
% \changes{3.1a}{1993/08/03}{Uncommented \cs{filedate} and
%    \cs{fileversion}}
% \changes{3.2}{1994/04/21}{Incorporated Robin Fairbairns suggestions
%    for the upgrade for \LaTeX2e}
% \changes{3.2a}{1995/03/20}{Fixed bug about cross-page bars}
% \changes{v3.2b}{1995/06/26}{Removed use of \cs{file...} commands}
% \changes{v3.2b}{1995/06/26}{Use \LaTeX's Warning mechanisms}
%
% \title{The Changebar document style option
%         \thanks{This file has version number \fileversion,
%                 last revised \filedate.}}
%
% \author{Michael Fine\\Distributed Systems Architecture}
%
% \author{Johannes Braams\\
%         Kooienswater 62\\
%         2215 AK Zoetermeer\\
%         The Netherlands\\
%         \texttt{JLBraams@cistron.nl}}
%
% \date{Printed \today}
%
% \maketitle
%
% \begin{multicols}{2}
%  \tableofcontents
% \end{multicols}
%
% \begin{abstract}
%    This document style-option implements a way to indicate
%    modifications in a \LaTeX-document by putting bars in the
%    margin. It realizes this by making use of the |\special|
%    commands supported by `dvi drivers'.  Currently four different
%    drivers are supported. More can easily be added.
% \end{abstract}
%
% \section{Introduction}
%
%    IMPORTANT NOTE: Just as with cross references and labels, you
%    usually need to process the document twice (and sometimes three
%    times) to ensure that the changebars come out correctly.
%    However, a warning will be given if another pass is required.
%
%
%    FEATURES:
%    \begin{itemize}
%    \item Changebars may be nested within each other.  Each level of
%    nesting can be given a different thickness bar. 
%
%    \item Changebars may be nested in other environments including
%    floats and footnotes.
%
%    \item Changebars are applied to all the material within the
%    ``barred'' environment, including floating bodies regardless of
%    where the floats float to.  An exeception to this are margin
%    floats.
%
%    \item Changebars may cross page boundaries.
%    \end{itemize}
%
% \section{The user interface}
%
%    This package has options to specify some details of its
%    operation, and also defines several macros.
%
% \subsection{The package options}
%
%    One set of package options specify the driver that will be used
%    to print the document can be indicated. The driver may be one of:
%    \begin{itemize}
%      \item DVItoLN03
%      \item DVItoPS
%      \item DVIps
%      \item em\TeX
%    \end{itemize}
%    The drivers are represented in the normal typewriter method of
%    typing these names, or by the same entirely in lower case.
%
%    The position of the bars may either be on the inner edge of the
%    page (the left column on a recto or single-sided page, the right
%    column of a verso page) by use of the \Lopt{innerbars} package
%    option (the default), or on the outer edge of the page by use of
%    the the \Lopt{outerbars} package option.
%
%    The package also implements tracing for its own debugging.  The
%    package options \Lopt{traceon} and \Lopt{traceoff} control
%    tracing.
%
% \subsection{Macros defined by the package}
%
% \DescribeMacro{\cbstart}
% \DescribeMacro{\cbend}
%    All material between the macros |\cbstart| and |\cbend| is
%    barred.  The nesting of multiple changebars is allowed.  The
%    macro |\cbstart| has an optional parameter that specifies the
%    width of the bar. The syntax is |\cbstart[|\meta{dimension}|]|.
%    If no width is specified, the current value of the parameter
%    |\changebarwidth| is used.  Note that |\cbstart| and |\cbend| can
%    be used anywhere but must be correctly nested with floats and
%    footnotes.  That is, one cannot have one end of the bar inside a
%    floating insertion and the other outside, but that would be a
%    meaningless thing to do anyhow.
%
% \DescribeEnv{changebar}
%    Apart from the macros |\cbstart| and |\cbend| a proper \LaTeX\
%    environment is defined. The advantage of using the environment
%    whenever possible is that \LaTeX\ will do all the work of
%    checking the correct nesting of different environments.
%
% \DescribeMacro{\cbdelete}
%    The macro |\cbdelete| puts a square bar in the margin to indicate
%    that some text was removed from the document. The macro has an
%    optional argument to specify the width of the bar. When no
%    argument is specified the current value of the parameter
%    |\deletebarwidth| will be used.
%
% \DescribeMacro{\nochangebars}
%    The macro |\nochangebars| disables the changebar commands.
%
% \subsection{Changebar parameters}
%
% \DescribeMacro{\changebarwidth}
%    The width of the changebars is controlled with the \LaTeX\ length
%    parameter |\changebarwidth|.
%    Its value can be changed with the |\setlength| command.
%    Changing the value of |\changebarwidth| affects all subsequent
%    changebars subject to the scoping rules of |\setlength|.
%
% \DescribeMacro{\deletebarwidth}
%    The width of the deletebars is controlled with the \LaTeX\ length
%    parameter |\deletebarwidth|.
%    Its value can be changed with the |\setlength| command.
%    Changing the value of |\changebarwidth| affects all subsequent
%    deletebars subject to the scoping rules of |\setlength|.
%
% \DescribeMacro{\changebarsep}
%    The separation between the text and the changebars is determined
%    by the value of the \LaTeX\ length parameter |\changebarsep|.
%
% \DescribeVar{changebargrey}
%    When one of the supported dvi to PostScript translators is used
%    the `blackness' of the bars can be controlled. The \LaTeX\
%    counter \texttt{changebargrey} is used for this purpose. Its
%    value can be changed with a command like:
%    \begin{verbatim}
%    \setcounter{changebargrey}{85}
%    \end{verbatim}
%    The value of the counter is a percentage, where the value 0 yields
%    black bars, the value 100 yields white bars.
%
% \DescribeVar{outerbars}
%    The changebars will be printed in the `inside' margin of your
%    document.  This means they appear on the left side of the
%    page. When \Lopt{twoside} is in effect the bars will be printed
%    on the right side of even pages.  This behaviour can be changed
%    by including the command |\outerbarstrue| in your document.
%
%
% \section{Deficiencies and bugs}
%
% \begin{itemize}
% \item The macros blindly use specials points |\cb@minpoint| through
%    |\cb@maxpoint|. If this conflicts with another set of macros, the
%    results will be unpredictable.  (What is really needed is a
%    |\newspecialpoint|, analogous to |\newcount| etc.~--- it's not
%    provided because the use of the points is rather rare.)
%
% \item There is a limit of 
%    $(\mbox{\texttt{bslash cb@maxpoint}}-
%    \mbox{\texttt{bslash cb@minpoint}}+1)/2$ bars per page
%    (two special points per bar).  Using more than this number yeilds
%    unpredictable results (but that could be called a feature for a
%    page with so many bars).  This limitation could be increased if
%    desired.
%
% \item Internal macro names are all of the form |\cb@xxxx|.  No
%    checking for conflicts with other macros is done.
%
% \item This implementation does not work with two (or more) column
%    printing.
%
% \item The algorithms may fail if a floating insertion is split over
%    multiple pages.  In \LaTeX\ floats are not split but footnotes
%    may be.  The simplest fix to this is to prevent footnotes from
%    being split but this may make \TeX\ very unhappy.
%
% \item The |\cbend| normally gets ``attached'' to the token after it
%    rather than the one before it.  This may lead to a longer bar
%    than intended.  For example, consider the sequence `word1
%    |\cbend| word2'.  If there is a line break between `word1' and
%    `word2' the bar will incorrectly be extended an extra line.  This
%    particular case can be fixed with the incantation
%    `word1|\cbend{}| word2'.
%  \end{itemize}
%
% \section{The basic algorithm}
%
%    The changebars are implemented using the |\specials| of various
%    \texttt{dvi} interpreting programs like \texttt{DVItoLN03} or
%    \texttt{DVIps}.  In essence, the start of changebar defines a
%    |\special| point in the margin at the current vertical position
%    on the page.  The end of a changebar defines another point and
%    then joins the two using the ``connect'' |\special|.
%
%    This works fine as long as the two points being connected lie on
%    the same page.  However, if they don't, the bar must be
%    artificially terminated at the page break and restarted at the
%    top of the next page.  The only way to do this (that I can think
%    of) is to modify the output routine so that it checks if any bar
%    is in progress when it ships out a page and, if so, adds the
%    necessary artificial end and begin.
%
%    The obvious way to indicate to the output routine that a bar is
%    in progress is to set a flag when the bar is begun and to unset
%    this flag when the bar is ended.  This works most of the time
%    but, because of the asynchronous behavior of the output routine,
%    errors occur if the bar begins or ends near a page break.  To
%    illustrate, consider the following scenario.
%
%    \begin{verbatim}
%    blah blah blah          % page n
%    blah blah blah
%    \cbstart                % this does its thing and set the flag
%    more blah
%               <-------------- pagebreak occurs here
%    more blah
%    \cbend                  % does its thing and unsets flag
%    blah blah
%    \end{verbatim}
%
%    Since \TeX\ processes ahead of the page break before invoking the
%    output routine, it is possible that the |\cbend| is
%    processed, and the flag unset, before the output routine is
%    called.  If this happens, special action is required to generate
%    an artificial end and begin to be added to page $n$ and $n+1$
%    respectively, as it is not possible to use a flag to signal
%    the output routine that a bar crosses a page break.
%
%    The method used by these macros is to create a list of the
%    beginning and end points of each bar in the document together
%    with the page number corresponding to each point.  Then, as a
%    page is completed, a modified output routine checks the list to
%    determine if any bars begun on or before the current page are
%    terminated on subsequent pages, and handles those bars
%    appropriately.  To build the list, information about each
%    changebar is written to the \file{.aux} file as bars are processed.
%    This information is re-read when the document is next processed.
%    Thus, to ensure that changebars are correct, the document must
%    be processed twice.  Luckily, this is generally required for
%    \LaTeX\ anyway.
%
%    This approach is sufficiently general to allow nested bars, bars
%    in floating insertions, and bars around floating insertions.
%    Bars inside floats and footnotes are handled in the same way as
%    bars in regular text.  Bars that encompass floats or footnotes
%    are handled by creating an additional bar that floats with the
%    floating material.  Modifications to the appropriate \LaTeX\
%    macros check for this condition and add the extra bar.
%
%\StopEventually{
% \IndexPrologue{\section*{Index}%
%  \markboth{Index}{Index}%
%     Numbers in \emph{italics} indicate the page where the
%     macro is described, the underlined numbers indicate
%     the number of the line of code where the macro is defined,
%     all other numbers indicate where a macro is used.}
%
% \GlossaryPrologue{\section*{History of changes}%
%    \markboth{History of changes}{History of changes}%
%    The numbers behind the changes indicate the page where
%    the macrocode is described.\hfil\null}
%
%  \PrintIndex
%^^AA \PrintChanges
% }
% \section{The implementation}
%
%    We begin by identifying ourselves by writing a message in the
%    transcript of the session.
%
% \subsection{Declarations And Initializations}
%
% \begin{macro}{\cb@maxpoint}
%    The original version of \texttt{changebar.sty} only supported the
%    \texttt{DVItoLN03} specials. The \texttt{LN03} printer has a
%    maximum number of points that can be defined on a page. Also for
%    some PostScript printers the number of points that can be defined
%    can be limited by the amount of memory used. Therefore, the
%    consecutive numbering of points has to be reset when the maximum
%    is reached. This maximum can be adapted to the printers needs.
%    \begin{macrocode}
%<*package>
\def\cb@maxpoint{80}
%    \end{macrocode}
% \end{macro}
%
% \begin{macro}{\cb@minpoint}
%    When resetting the point number we need to know what to reset it
%    to, this is minimum number is stored in |\cb@minpoint|.
%    \textbf{This number has to be ODD} because the algorithm that
%    decides whether a bar has to be continued on the next page
%    depends on this.
%    \begin{macrocode}
\def\cb@minpoint{1}
%    \end{macrocode}
% \end{macro}
%
% \begin{macro}{\cb@nil}
%    Sometimes a void value for a point has to be returned by one
%    of the macros. For this purpose |\cb@nil| is used.
%    \begin{macrocode}
\def\cb@nil{0}
%    \end{macrocode}
% \end{macro}
%
% \begin{macro}{\cb@nextpoint}
% \begin{macro}{\cb@currentpoint}
%    The number of the next special point is stored in the count
%    register |\cb@nextpoint| and initially equal to
%    |\cb@minpoint|. The number of the current point is stored in
%    |\cb@currentpoint|.
%    \begin{macrocode}
\newcount\cb@nextpoint
\cb@nextpoint=\cb@minpoint
\newcount\cb@currentpoint
%    \end{macrocode}
% \end{macro}
% \end{macro}
%
% \begin{macro}{\cb@page}
% \begin{macro}{\cb@pagecount}
%    The macros need to keep track of the number of pages output so
%    far.  To this end the counter |\cb@pagecount| is used.  When a
%    pagenumber is read from the history list, it is stored in the
%    counter |\cb@page|. The counter |cb@pagecount| is initially $0$;
%    it gets incremented during the call to |\@makebox| (see
%    section~\ref{pagebreak}).
%    \begin{macrocode}
\newcount\cb@page
\newcount\cb@pagecount
\cb@pagecount=0
%    \end{macrocode}
% \end{macro}
% \end{macro}
%
% \begin{Var}{outerbars}
%    A switch is provided to control where the changebars will be
%    printed.
%    \begin{macrocode}
\newif\ifouterbars
%    \end{macrocode}
% \end{Var}
%
%  \begin{Var}{@cb@trace}
%    A switch to enable tracing of the actions of this package
%    \begin{macrocode}
\newif\if@cb@trace
%    \end{macrocode}
%  \end{Var}
%
% \begin{macro}{\cb@positions}
%
%    This macro calculates the (horizontal) positions of the
%    changebars.  The calculations take \Lopt{twoside} into
%    account. Also the setting of the switch \Lopt{outerbars} is
%    checked.
%
%    Because the margins can differ for even and odd pages and because
%    changebars are sometimes on different sides of the paper we need
%    two dimensions to store the result.
%
%    Since the changebars are drawn with the \textsc{PostScript}
%    command \texttt{lineto} and not as \TeX{}-like rules the
%    reference points lie on the center of the changebar, therefore
%    the calculation has to add or subtract half of the width of the
%    bar to keep |\changebarsep| whitespace between the bar and the
%    body text.
%    \begin{macrocode}
\newdimen\cb@odd
\newdimen\cb@even
%    \end{macrocode}
%    First the position for odd pages is calculated. In this case the
%    use of the \Lopt{twoside} option does not effect the positioning
%    of the changebars.
% \changes{3.1}{1993/07/02}{Corrected positioning of PostScript bars}
%    \begin{macrocode}
\def\cb@positions{%
  \global\cb@odd =\hoffset
  \global\advance\cb@odd by \oddsidemargin
  \ifouterbars
    \global\advance\cb@odd by \textwidth
    \global\advance\cb@odd by \changebarsep
    \global\advance\cb@odd by 0.5\changebarwidth
  \else
    \global\advance\cb@odd by -\changebarsep
    \global\advance\cb@odd by -0.5\changebarwidth
  \fi
%    \end{macrocode}
%    On even pages the use of the \Lopt{twoside} option results in a
%    different placement of the bars.
%    \begin{macrocode}
  \global\cb@even =\hoffset
  \global\advance\cb@even by \evensidemargin
  \if@twoside
    \ifouterbars
      \global\advance\cb@even by -\changebarsep
      \global\advance\cb@even by -0.5\changebarwidth
    \else
      \global\advance\cb@even by \textwidth
      \global\advance\cb@even by \changebarsep
      \global\advance\cb@even by 0.5\changebarwidth
    \fi
  \else
    \ifouterbars
      \global\advance\cb@even by \textwidth
      \global\advance\cb@even by \changebarsep
      \global\advance\cb@even by 0.5\changebarwidth
    \else
      \global\advance\cb@even by -\changebarsep
      \global\advance\cb@even by -0.5\changebarwidth
    \fi
  \fi}
%    \end{macrocode}
% \end{macro}
%
% \begin{macro}{\cb@removedim}
%    In PostScript code length specifications are without dimensions.
%    Therefore we need a way to remove the letters `pt' from the
%    result of the operation |\the\|\meta{dimen}.
%    This can be done by defining a command that has a delimited
%    argument like:
%    \begin{verbatim}
%    \def\cb@removedim#1pt{#1}
%    \end{verbatim}
%    We encounter one problem though, the category code of the letters
%    `pt' is 12 when produced as the output from |\the\|\meta{dimen}.
%    Thus the characters that delimit the argument of |\cb@removedim|
%    also have to have category code 12. To keep the changes local
%    the macro |\cb@removedim| is defined in a group.
%    \begin{macrocode}
{\catcode`\p=12\catcode`\t=12 \gdef\cb@removedim#1pt{#1}}
%    \end{macrocode}
% \end{macro}
%
% \subsection{Option Processing}
%
%    The user should select the specials that should be used by
%    specifying the driver name as an option to the
%    |\usepackage| call.
%    Possible choices are:
%    \begin{itemize}
%    \item DVItoLN03
%    \item DVItoPS
%    \item DVIps
%    \item em\TeX
%    \end{itemize}
%
%    The ambition is that the driver names should be case-insensitive,
%    but the following code doesn't achieve this: it only permits the
%    forms given above and their lower-case equivalents.
%    \begin{macrocode}
\DeclareOption{DVItoLN03}{\global\chardef\cb@driver@setup=0\relax}
\DeclareOption{dvitoln03}{\global\chardef\cb@driver@setup=0\relax}
\DeclareOption{DVItoPS}{\global\chardef\cb@driver@setup=1\relax}
\DeclareOption{dvitops}{\global\chardef\cb@driver@setup=1\relax}
\DeclareOption{DVIps}{\global\chardef\cb@driver@setup=2\relax}
\DeclareOption{dvips}{\global\chardef\cb@driver@setup=2\relax}
\DeclareOption{emTeX}{\global\chardef\cb@driver@setup=3\relax}
\DeclareOption{emtex}{\global\chardef\cb@driver@setup=3\relax}
%    \end{macrocode}
%
%    The new features of \LaTeXe\ make it possible to implement the
%    \Lopt{outerbars} option.
%
%    \begin{macrocode}
\DeclareOption{outerbars}{\outerbarstrue}
\DeclareOption{innerbars}{\outerbarsfalse}
%    \end{macrocode}
%
%    It is also possible to specify that the change bars should
%    \emph{always} be printed on either the left or the right side of
%    the text. For this we have the options \Lopt{leftbars} and
%    \Lopt{rightbars}. Specifying \emph{either} of these options will
%    overrule a possible \Lopt{twoside} option at the document level.
% \changes{v3.2b}{1995/06/27}{Added the options\Lopt{leftbars} and
%    \Lopt{rightbars}}
%    \begin{macrocode}
\DeclareOption{leftbars}{%
  \def\cb@positions{%
    \global\cb@odd\hoffset
    \global\advance\cb@odd by \oddsidemargin
    \global\advance\cb@odd by -\changebarsep
    \global\advance\cb@odd by -0.5\changebarwidth
    \global\cb@even\hoffset
    \global\advance\cb@even by \evensidemargin
    \global\advance\cb@even by -\changebarsep
    \global\advance\cb@even by -0.5\changebarwidth
    }}
\DeclareOption{rightbars}{%
  \def\cb@positions{%
    \global\cb@odd =\hoffset
    \global\advance\cb@odd by \oddsidemargin
    \global\advance\cb@odd by \textwidth
    \global\advance\cb@odd by \changebarsep
    \global\advance\cb@odd by 0.5\changebarwidth
    \global\cb@even\hoffset
    \global\advance\cb@even by \evensidemargin
    \global\advance\cb@even by \textwidth
    \global\advance\cb@even by \changebarsep
    \global\advance\cb@even by 0.5\changebarwidth
    }}
%    \end{macrocode}
%    \begin{macrocode}
\DeclareOption{traceon}{\@cb@tracetrue}
\DeclareOption{traceoff}{\@cb@tracefalse}
%    \end{macrocode}
%    Signal an error if an unknown option was specified.
%    \begin{macrocode}
\DeclareOption*{\OptionNotUsed\PackageError
      {Unrecognised option `\CurrentOption'}%
      {known options are dvitoln03, dvitops, dvips
        and emtex,\MessageBreak
        outerbars, innerbars, leftbars and rightbars.}}
%    \end{macrocode}
%
%    The default is to have the change bars on the leftt side of the
%    text on odd pages.
%    \begin{macrocode}
\ExecuteOptions{innerbars,traceoff}
%    \end{macrocode}
%    \begin{macrocode}
\ProcessOptions\relax
%    \end{macrocode}
%
% \subsection{User Level Commands And Parameters}
%
% \begin{macro}{\cb@setup@specials}
%    The macro |\cb@setup@specials| defines macros containing the
%    driver specific |\special| macros. It will be called from
%    within the |\begin{document}| command.
%
%  \begin{macro}{\cb@trace@defpoint}
%    When tracing is on write information about the point being defined
%    to the log file.
%    \begin{macrocode}
\def\cb@trace@defpoint#1#2{%
  \wlog{Changebar trace: defining point \the#1 at position \the#2}
  \wlog{Changebar trace: cb@pagecount: \the\cb@pagecount;
    page \thepage}}
%    \end{macrocode}
%  \end{macro}
%
%
%  \begin{macro}{\cb@trace@connect}
%    When tracing is on write information about the points being
%    connected to the log file.
%    \begin{macrocode}
\def\cb@trace@connect#1#2#3{%
  \wlog{Changebar trace: connecting points \the#1 and \the#2;
      barwidth: \the#3}
  \wlog{Changebar trace: cb@pagecount: \the\cb@pagecount;
    page \thepage}}
%    \end{macrocode}
%  \end{macro}
%
% \begin{macro}{\cb@defpoint}
%    The macro |\cb@defpoint| is used to define one of the two
%    points of a bar. It has two arguments, the number of the point
%    and the distance from the left side of the paper; so its syntax
%    is:
%    |\cb@defpoint{|\meta{number}|}{|\meta{length}|}|.
% \begin{macro}{\cb@resetpoints}
%    The macro |\cb@resetpoints| can be used to instruct the printer
%    driver that it should send a corresponding instruction to the
%    printer. This is really only used for the \textsf{LN03} printer.
%
% \begin{macro}{\cb@connect}
%    The macro |\cb@connect| is used to instruct the printer driver
%    to connect two points with a bar. The syntax is
%    |\cb@connect{|\meta{number}|}{|\meta{number}|}{|\meta{length}|}|
%    The two \meta{number}s indicate the two points to be connected;
%    the \meta{length} is the width of the bar.
%
%    \begin{macrocode}
\def\cb@setup@specials{%
%    \end{macrocode}
%    The control sequence |\cb@driver@setup| expands to a number which
%    indicates the driver that will be used.  The original
%    \file{changebar.sty} was written with only the the |\special|
%    syntax of the program \texttt{DVItoLN03} (actually one of its
%    predecessors, \texttt{ln03dvi}). Therefore this syntax is defined
%    first.
%    \begin{macrocode}
\ifcase\cb@driver@setup
  \def\cb@defpoint##1##2{%
    \special{ln03:defpoint \the##1(\the##2,)}%
    \if@cb@trace\cb@trace@defpoint##1##2\fi}
  \def\cb@connect##1##2##3{%
    \special{ln03:connect \the##1\space\space \the##2\space \the##3}%
    \if@cb@trace\cb@trace@connect##1##2##3\fi}
  \def\cb@resetpoints{%
    \special{ln03:resetpoints \cb@minpoint \space\cb@maxpoint}}
%    \end{macrocode}
%    The first extension to the {changebar} option was for the
%    |\special| syntax of the program \texttt{DVItoPS} by James Clark.
%    \begin{macrocode}
\or
  \def\cb@defpoint##1##2{%
    \special{dvitops: inline
                \expandafter\cb@removedim\the##2\space 6.5536 mul\space
                /CBarX\the##1\space exch def currentpoint exch pop
                /CBarY\the##1\space exch def}%
    \if@cb@trace\cb@trace@defpoint##1##2\fi}
  \def\cb@connect##1##2##3{%
    \special{dvitops: inline
                gsave \thechangebargrey\space 100 div setgray
                \expandafter\cb@removedim\the##3\space 6.5536 mul\space
                CBarX\the##1\space\space CBarY\the##1\space\space moveto
                CBarX\the##2\space\space CBarY\the##2\space\space lineto
                stroke grestore}%
    \if@cb@trace\cb@trace@connect##1##2##3\fi}
  \let\cb@resetpoints\relax
%    \end{macrocode}
%    Also the program \texttt{DVIps} by Thomas Rockicki is
%    supported. The PostScript code is nearly the same as for
%    \texttt{DVItoPS}, but the coordinate space has a different
%    dimension. Also this code has been made resolution independent,
%    whereas the code for \texttt{DVItoPS} might still be resolution
%    dependent.
%
%    So far all the positions have been calculated in \texttt{pt}
%    units.  DVIps uses pixels internally, so we have to convert
%    \texttt{pt}s into pixels which of course is done by dividing by
%    72.27 (\texttt{pt}s per inch) and multiplying by
%    \texttt{Resolution} giving the resolution of the
%    \textsc{PostScript} device in use as a \textsc{PostScript}
%    variable.
%    \begin{macrocode}
\or
  \def\cb@defpoint##1##2{%
     \special{ps:
                \expandafter\cb@removedim\the##2\space
                Resolution\space mul\space 72.27\space div\space
                /CBarX\the##1\space exch def currentpoint exch pop
                /CBarY\the##1\space exch def}%
    \if@cb@trace\cb@trace@defpoint##1##2\fi}
    \def\cb@connect##1##2##3{%
      \special{ps:
                gsave \thechangebargrey\space 100 div setgray
                \expandafter\cb@removedim\the##3\space
                Resolution\space mul\space 72.27\space div\space
                setlinewidth
                CBarX\the##1\space\space CBarY\the##1\space\space moveto
                CBarX\the##2\space\space CBarY\the##2\space\space lineto
                stroke grestore}%
    \if@cb@trace\cb@trace@connect##1##2##3\fi}
    \let\cb@resetpoints\relax
%    \end{macrocode}
%    The latest addition is for the drivers written by Eberhard
%    Mattes.  The |\special| syntax used here will be supported with
%    version 1.5 of his driver programs. This version will probably be
%    released late in 1991, therefore we add a warning for now.
%    \begin{macrocode}
\or
  \PackageWarning{Changebar}%
    {changebars only supported for v1.5+ of dvidrv}%
  \def\cb@defpoint##1##2{%
    \special{em:point \the##1,\the##2}%
    \if@cb@trace\cb@trace@defpoint##1##2\fi}
  \def\cb@connect##1##2##3{%
    \special{em:line \the##1,\the##2,\the##3}%
    \if@cb@trace\cb@trace@connect##1##2##3\fi}
  \let\cb@resetpoints\relax
%    \end{macrocode}
%    When code for other drivers should be added it can be inserted
%    here.  When someone makes a mistake and somehow selects an
%    unknown driver a warning is issued and the macros are defined to
%    be noops.
%    \begin{macrocode}
\else
  \PackageWarning{Changebar}{changebars not supported in unknown setup}
  \def\cb@defpoint##1##2{%
    \if@cb@trace\cb@trace@defpoint##1##2\fi
    }
  \def\cb@connect##1##2##3{%
    \if@cb@trace\cb@trace@connect##1##2##3\fi
    }
  \let\cb@resetpoints\relax
\fi
%    \end{macrocode}
%    The last thing to do is to forget about |\cb@setup@specials|.
%    \begin{macrocode}
\global\let\cb@setup@specials\relax}
%    \end{macrocode}
% \end{macro}
% \end{macro}
% \end{macro}
% \end{macro}
%
% \begin{macro}{\cbstart}
%
%    The macro |\cbstart| starts a new changebar. It has an (optional)
%    argument that will be used to determine the width of the bar.
%    The default width is |\changebarwidth|.
%    \begin{macrocode}
\newcommand\cbstart{\@ifnextchar [{\cb@start}%
                                  {\cb@start[\changebarwidth]}}
%    \end{macrocode}
% \end{macro}
%
% \begin{macro}{\cbend}
%    The macro |\cbend| (surprisingly) ends a changebar. The macros
%    |\cbstart| and |\cbend| can be used when the use of a
%    proper \LaTeX\ environment is not possible.
%    \begin{macrocode}
\newcommand\cbend{\cb@end}
%    \end{macrocode}
% \end{macro}
%
% \begin{macro}{\cbdelete}
%    The macro |\cbdelete| inserts a `deletebar' in the margin.
%    It too has an optional argument to determine the width of the bar.
%    The default width (and length) of it are stored in
%    |\deletebarwidth|.
%    \begin{macrocode}
\newcommand\cbdelete{\@ifnextchar [{\cb@delete}
                                   {\cb@delete[\deletebarwidth]}}
%    \end{macrocode}
%
% \begin{macro}{\cb@delete}
%    Deletebars are implemented as a special `change bar'. The bar
%    is started and immediately ended. It is as long as it is wide.
%    \begin{macrocode}
\def\cb@delete[#1]{\vbox to \z@{\vss\cb@start[#1]\vskip #1\cb@end}}
%    \end{macrocode}
% \end{macro}
% \end{macro}
%
% \begin{macro}{\changebar}
% \begin{macro}{\endchangebar}
%    The macros |\changebar| and |\endchangebar| have the same
%    function as |\cbstart| and |\cbend| but they can be used as a
%    \LaTeX\ environment to enforce correct nesting.  They can
%    \emph{not} be used in the \textsf{tabular} and \textsf{tabbing}
%    environments.
%    \begin{macrocode}
\newenvironment{changebar}%
               {\@ifnextchar [{\cb@start}%
                              {\cb@start[\changebarwidth]}}%
               {\cb@end}
%    \end{macrocode}
% \end{macro}
% \end{macro}
%
% \begin{macro}{\nochangebars}
%    To disable changebars altogether without having to remove them
%    from the document the macro |\nochangebars| is provided. It makes
%    noops from a couple of internal macros.
%    \begin{macrocode}
\newcommand\nochangebars{%
  \def\cb@start[##1]{}%
  \def\cb@delete[##1]{}%
  \let\cb@end\relax}
%    \end{macrocode}
% \end{macro}
%
% \begin{macro}{\changebarwidth}
%    The default width of the changebars is stored in the dimen register
%    |\changebarwidth|.
%    \begin{macrocode}
\newlength{\changebarwidth}
\setlength{\changebarwidth}{2pt}
%    \end{macrocode}
% \end{macro}
%
% \begin{macro}{\deletebarwidth}
%    The default width of the deletebars is stored in the dimen register
%    |\deletebarwidth|.
%    \begin{macrocode}
\newlength{\deletebarwidth}
\setlength{\deletebarwidth}{4pt}
%    \end{macrocode}
% \end{macro}
%
% \begin{macro}{\changebarsep}
%    The default separation between all bars and the text is stored in
%    the dimen register |\changebarsep|.
%    \begin{macrocode}
\newlength{\changebarsep}
\setlength{\changebarsep}{30pt}
%    \end{macrocode}
% \end{macro}
%
% \begin{Var}{changebargrey}
%    When the document is printed using one of the PostScript drivers
%    the bars do not need to be black; with PostScript is possible to
%    have grey bars. The pecentage of greyness of the bar is stored
%    in the count register |\changebargrey|. It can have values
%    between $0$ (meaning white) and $100$ (meaning black).
%
%    \begin{macrocode}
\newcounter{changebargrey}
\setcounter{changebargrey}{65}
%    \end{macrocode}
% \end{Var}
%
% \subsection{Macros for beginning and ending bars}
%
%
% \begin{macro}{\cb@start}
%    This macro starts a change bar.  It assigns a new value to the
%    current point and advances the counter for the next point to be
%    assigned.  It pushes this info onto |\cb@currentlist| and then
%    sets the point by calling |\cb@setBeginPoint| with the point
%    number.  Finally, it writes the aux file.
%
%    \begin{macrocode}
\def\cb@start[#1]{%
  \cb@currentpoint=\cb@nextpoint
%    \end{macrocode}
%    |\cb@push| expects the width of the bar to be stored in
%    |\@tempdima|.
%    \begin{macrocode}
  \@tempdima#1\relax
  \cb@push\cb@currentlist
  \ifvmode 
    \cb@setBeginPoint\cb@currentpoint
  \else
   \vbox to \z@{%
%    \end{macrocode}
%    When we are in horizontal mode we jump up a line to set the
%    starting point of the changebar.
%    \begin{macrocode}
     \vskip -\ht\strutbox
     \cb@setBeginPoint\cb@currentpoint
     \vskip \ht\strutbox}\fi
  \cb@writeAux\cb@advancePoint}
%    \end{macrocode}
% \end{macro}
%
% \begin{macro}{\cb@advancePoint}
%    The macro |\cb@advancePoint| advances the count register
%    |\cb@nextpoint|. When the maximum number is reached, the
%    numbering is reset.
%    \begin{macrocode}
\def\cb@advancePoint{%
  \global\advance\cb@nextpoint by 2\relax
  \ifnum\cb@nextpoint>\cb@maxpoint
    \global\cb@nextpoint=\cb@minpoint\relax
  \fi}
%    \end{macrocode}
% \end{macro}
%
% \begin{macro}{\cb@end}
%    This macro ends a changebar. It pops the current point and
%    nesting level off |\cb@currentlist| and sets the end point by
%    calling |\cb@setEndPoint| with the parameter corresponding to the
%    \textbf{beginning} point number. It writes the \file{.aux} file
%    and joins the points.
%
%    \begin{macrocode}
\def\cb@end{%
  \cb@pop\cb@currentlist
  \ifnum\cb@currentpoint=\cb@nil
    \PackageWarning{Changebar}%
      {Badly nested changebars; Expect erroneous results}%
  \else
    \cb@setEndPoint\cb@currentpoint
    \global\advance\cb@currentpoint by1\cb@writeAux
  \fi}
%    \end{macrocode}
%\end{macro}
%
% \begin{macro}{\cb@setBeginPoint}
%    The macro |\cb@setBeginPoint| assigns a position to point number
%    $\#1$. It determines wether the point is on an even or an odd
%    page and uses the right dimension to position the point.  Keep in
%    mind that the value of |\cb@pagecount| is one less than the value
%    of |\c@page| unless the latter has been reset by the user.
%
%    \begin{macrocode}
\def\cb@setBeginPoint#1{%
  \ifodd\cb@pagecount
    \cb@defpoint{#1}{\cb@even}%
  \else
    \cb@defpoint{#1}{\cb@odd}%
  \fi}
%    \end{macrocode}
% \end{macro}
%
% \begin{macro}{\cb@setEndPoint}
%    The macro |\cb@setEndPoint| assigns a position
%    to point $\#1+1$. It then instructs the driver to connect points
%    $\#1$ and $\#1+1$ with a bar.
%    The macro assumes that the width of bar is stored in |\@tempdima|.
%
%    \begin{macrocode}
\def\cb@setEndPoint#1{%
  \@tempcnta=#1\advance\@tempcnta by1\relax
  \ifodd\cb@pagecount
    \cb@defpoint{\@tempcnta}{\cb@even}%
  \else
    \cb@defpoint{\@tempcnta}{\cb@odd}%
  \fi
  \cb@connect{#1}{\@tempcnta}{\@tempdima}}%
%    \end{macrocode}
% \end{macro}
%
% \begin{macro}{\cb@writeAux}
%    The macro |\cb@writeAux| writes information about a changebar
%    point to the auxiliary file. The number of the point, the
%    pagenumber and the width of the bar are written out as arguments
%    to |\cb@barpoint|. This latter macro will be expanded when the
%    auxiliary file is read in.  The macro assumes that the width of
%    bar is stored in |\@tempdima|.
%
%    The code is only executed when auxiliary files are enabled, as
%    there's no sense in trying to write to an unopened file.
%    \begin{macrocode}
\def\cb@writeAux{%
  \if@filesw
    \begingroup
      \edef\point{\the\cb@currentpoint}%
      \edef\level{\the\@tempdima}%
      \let\the=0%
      \edef\cb@temp{\write\@auxout
        {\string\cb@barpoint{\point}{\the\cb@pagecount}{\level}}}%
      \cb@temp
    \endgroup
  \fi}
%    \end{macrocode}
% \end{macro}
%
% \subsection{Macros for Making It Work Across Page Breaks}
% \label{pagebreak}
%
% \begin{macro}{\@makecol}
% \begin{macro}{\@vtryfc}
%
%    Modifies the \LaTeX\ macros to end the changebars spanning the
%    current page (if any) and restart them on the next page.  Does
%    the following: It resets special points for this page.  Adds
%    begin bars to top of box255.  The bars carried over from the
%    previous page, and hence to be restarted on this page, have been
%    saved in |\cb@beginSaves|, and are then executed.  Then it builds
%    the list |\cb@spanlist|, then calls |\cb@processActive| to
%    process the elements on the list, then it executes the original
%    |\@makecol| (saved as |\cb@makecol|).
%
% \changes{3.2}{1994/04/21}{Added setting of \cs{boxmaxdepth}}
%    \begin{macrocode}
\let\cb@makecol\@makecol
\def\@makecol{%
  \setbox\@cclv \vbox{%
    \cb@resetpoints
    \cb@beginSaves
    \unvbox\@cclv
    \boxmaxdepth\maxdepth}%
  \gdef\cb@beginSaves{}
  \global\advance\cb@pagecount by 1\relax
  \cb@buildActive\cb@processActive
  \cb@makecol}
%    \end{macrocode}
%   When \LaTeX makes a page with only floats it doesn't use
%   |\@makecol|; instead it calls |\@vtryfc|, so we have to modify
%   this macro as well.
%    \begin{macrocode}
\let\cb@vtryfc\@vtryfc
\def\@vtryfc{%
  \setbox\@outputbox \vbox{%
    \cb@resetpoints
    \cb@beginSaves
    \unvbox\@cclv
    \boxmaxdepth\maxdepth}%
  \gdef\cb@beginSaves{}
  \global\advance\cb@pagecount by 1\relax
  \cb@buildActive\cb@processActive
  \cb@vtryfc}
%    \end{macrocode}
% \end{macro}
% \end{macro}
%
% \begin{macro}{\cb@processActive}
%
%    Processes each element on span list. Each element represents a
%    bar that crosses the page break. There could be more than one if
%    bars are nested. Works as folows:
%
% \begin{verbatim}
%       pop top element of span list
%       if point null (i.e., list empty) then done
%       else
%           do an end bar on box255
%           save start for new bar at top of next page in \cb@startSaves
%           push active point back onto history list (need to reprocess
%               on next page).
% \end{verbatim}
%
%    \begin{macrocode}
\def\cb@processActive{%
  \cb@pop\cb@spanlist
  \ifnum\cb@currentpoint=\cb@nil
  \else
    \setbox\@cclv\vbox{%
      \unvbox\@cclv
      \boxmaxdepth\maxdepth
      \advance\cb@pagecount by -1\relax
      \cb@setEndPoint\cb@currentpoint}%
    \cb@saveBeginPoint\cb@currentpoint
    \cb@push\cb@history
    \cb@processActive
  \fi}
%    \end{macrocode}
% \end{macro}
%
% \begin{macro}{\cb@saveBeginPoint}
%
%    Saves a |\special| to begin a point in expanded macro
%    |\cb@beginSaves|.  This is then used to start all spanning bars
%    at the top of the next page.  It is almost the same as
%    |\cb@setBeginPoint|.
%
%    \begin{macrocode}
\def\cb@saveBeginPoint#1{%
  \ifodd\cb@pagecount
    \xdef\cb@beginSaves{\cb@defpoint{#1}{\cb@even}\cb@beginSaves}%
  \else
    \xdef\cb@beginSaves{\cb@defpoint{#1}{\cb@odd}\cb@beginSaves}%
  \fi}
%    \end{macrocode}
%
%    \begin{macrocode}
\def\cb@beginSaves{}                % initially empty
%    \end{macrocode}
% \end{macro}
%
% \begin{macro}{\cb@buildActive}
%
% Initializes the spanlist to null.
%
%    \begin{macrocode}
\def\cb@buildActive{\cb@initlist\cb@spanlist\cb@pushNextActive}
%    \end{macrocode}
% \end{macro}
%
% \begin{macro}{\cb@pushNextActive}
%    Pops top of history list.  If point is on future page,
%    push back onto history list.  If point on current
%    or previous page and odd, add point to span list; if even, pop span
%    list since this bar has terminated on current page.
%    \begin{macrocode}
\def\cb@pushNextActive{%
  \cb@pop\cb@history
  \ifnum\cb@currentpoint=\cb@nil
  \else
    \ifnum\cb@page>\cb@pagecount
      \cb@push\cb@history
    \else
      \ifodd\cb@currentpoint
        \cb@push\cb@spanlist
      \else
        \cb@pop\cb@spanlist
      \fi
      \cb@pushNextActive
    \fi
  \fi}
%    \end{macrocode}
% \end{macro}
%
% \subsection{Macros For Managing The Lists of Bar points}
%
%    The macros make use of three lists corresponding to |\special|
%    defpoints.  Each list takes the form \texttt{<element>
%    ... <element>}
%
%    Each element is of the form \texttt{xxxnyyypzzzl} where
%    \texttt{xxx} is the number of the special point, \texttt{yyy} is
%    the page on which this point is set, and \texttt{zzz} is the
%    dimension used when connecting this point.
%
%    The list |\cb@history| is built from the log information and
%    initially lists all the points.  As pages are processed, points
%    are popped off the list and discarded.
%
%    The list |\cb@spanlist| is a temporary list used by the
%    output routine and contains the list of all bars crossing the
%    current page (there may be more than one with nested bars).
%    It is built by popping elements off the history list.
%
%    The list |\cb@currentlist| contains all the current bars.
%    A |\cb@start| pushes an element onto this list.
%    A |\cb@end| pops the top element off the list and uses the
%    info to terminate the bar.
%
%    For performance and memory reasons, the history list, which can
%    be very long, is special cased and a file is used to store this
%    list rather than an internal macro.  The ``external'' interface
%    to this list is identical to what is described above.  However,
%    when the history list is popped, a line from the file is first
%    read and appended to the macro |\cb@history|.
%
% \begin{macro}{\cb@initlist}
%    A macro to (globally) initialize a list.
%    \begin{macrocode}
\def\cb@initlist#1{\xdef#1{}}
%    \end{macrocode}
% \end{macro}
%
% \begin{macro}{\cb@history}
% \begin{macro}{\cb@write}
% \begin{macro}{\cb@read}
%    We need to initialise a list to store the entries read from the
%    external history file.
%    \begin{macrocode}
\cb@initlist\cb@history
%    \end{macrocode}
%    We also need to allocate a read and a write stream for the
%    history file.
%    \begin{macrocode}
\newwrite\cb@write
\newread\cb@read
%    \end{macrocode}
%    And we open the histroy file for writing.
%    \begin{macrocode}
\immediate\openout\cb@write=\jobname.cb\relax
%    \end{macrocode}
% \end{macro}
% \end{macro}
% \end{macro}
%
% \begin{macro}{\cb@spanlist}
%    Allocate a list for the bars that span the current page.
%    \begin{macrocode}
\cb@initlist\cb@spanlist
%    \end{macrocode}
% \end{macro}
%
% \begin{macro}{\cb@currentlist}
%    And we allocate an extra list that is needed to implement nesting
%    without having to rely on \TeX's grouping mechanism.
%    \begin{macrocode}
\cb@initlist\cb@currentlist
%    \end{macrocode}
% \end{macro}
%
% \begin{macro}{\cb@pop}
%
%    This pops the top element off the named list and puts the point
%    value into |\cb@currentpoint|, the page value into |\cb@page| and
%    the bar width into |\@tempdima|. If the list is empty, returns a
%    void value (|\cb@nil|) in |\cb@currentpoint| and sets
%    |\cb@page=0|.
%
%    \begin{macrocode}
\def\cb@pop#1{%
  \ifx #1\cb@history
    \ifeof\cb@read
    \else
      {\endlinechar=-1\read\cb@read to\@temp
       \xdef\cb@history{\cb@history\@temp}%
      }%
    \fi
  \fi
  \ifx#1\@empty
    \cb@currentpoint=\cb@nil\cb@page=0\relax
  \else
    \expandafter\cb@carcdr#1e#1%
  \fi}
%    \end{macrocode}
% \end{macro}
%
% \begin{macro}{\cb@carcdr}
%    Used to `decode' a list entry.
%    \begin{macrocode}
\def\cb@carcdr#1n#2p#3l#4e#5{%
  \cb@currentpoint=#1\cb@page=#2\relax\@tempdima=#3\xdef#5{#4}}
%    \end{macrocode}
% \end{macro}
%
% \begin{macro}{\cb@push}
%
%    Pushes |\cb@currentpoint|, |\cb@page| and |\@tempdima|
%    onto the top of the named list.
%
%    \begin{macrocode}
\def\cb@push#1{%
  \xdef#1{\the\cb@currentpoint n\the\cb@page p\the\@tempdima l#1}}
%    \end{macrocode}
% \end{macro}
%
% \begin{macro}{\cb@barpoint}
%
%    For populating the history file.  Writes one line to \file{.cb}
%    file which is equivalent to one \meta{element} described above.
%    \begin{macrocode}
\def\cb@barpoint#1#2#3{\immediate\write\cb@write{#1n#2p#3l}}
%    \end{macrocode}
% \end{macro}
%
% \subsection{Macros For Checking That The \file{.aux} File Is Stable}
%
% \begin{macro}{\AtBeginDocument}
%
%    While reading the \file{.aux}-file \LaTeX{} has created the
%    history list in a separate file. We need to close that file and
%    open it for reading. Also the `initialistion' of the
%    |\special| commands has to take place.  While we are
%    modifying the macro we also include the computation of the
%    possible positions of the changebars
%
%    For these actions we need to add to the \LaTeX\ begin-document
%    hook.
% \changes{3.2}{1994/04/21}{Use the \cs{AtBeginDocument} instead of
%    redefining \cs{document}}
%    \begin{macrocode}
\AtBeginDocument{%
  \cb@setup@specials
  \cb@positions
  \immediate\closeout\cb@write
  \immediate\openin\cb@read=\jobname.cb}
%    \end{macrocode}
% \end{macro}
%
% \begin{macro}{\AtEndDocument}
%
%    Does a |\clearpage| to flush rest of document.  (Note that I
%    believe there is contention in this area: are there in fact
%    situations in which the end-document hooks need to be called
%    \emph{before} the final |\clearpage|? --- the documentation of
%    LaTeX itself implies that there are.)  Then closes \file{.cb}
%    file and reopens it for checking.  Inits history list (to be read
%    from file).  Lets |\cb@barpoint=\cb@checkhistory| for checking.
% \changes{3.2}{1994/04/21}{Use \cs{AtEndDocument} now instead of
%    redefining \cs{enddocument}}
%
%    \begin{macrocode}
\AtEndDocument{%
  \clearpage
  \cb@initlist\cb@history
  \immediate\closein\cb@read
  \immediate\openin\cb@read=\jobname.cb%
  \let\cb@barpoint=\cb@checkHistory}
%    \end{macrocode}
% \end{macro}
%
% \begin{macro}{\cb@checkHistory}
%
%    Pops the top of the history list (|\jobname.cb|) and checks to
%    see if the point and page numbers are the same as the arguments
%    $\#1$ and $\#2$ respectively.  Prints a warning message if
%    different.
%
%    \begin{macrocode}
\def\cb@checkHistory#1#2#3{%
  \cb@pop\cb@history
  \ifnum #1=\cb@currentpoint\relax
    \ifnum #2=\cb@page\relax
%    \end{macrocode}
%    Both page and point numbers are equal; do nothing,
%    \begin{macrocode}
    \else
%    \end{macrocode}
%    but generate a warning when page numbers don't mach, or
%    \begin{macrocode}
      \cb@error
    \fi
  \else
%    \end{macrocode}
%    when point numbers don't mach.
%    \begin{macrocode}
    \cb@error
  \fi}
%    \end{macrocode}
% \end{macro}
%
% \begin{macro}{\cb@error}
%    When a mismatch between the changebar information in the
%    auxiliary file and the history list is detected a warning is
%    issued; further checking is disabled.
%    \begin{macrocode}
\def\cb@error{%
  \PackageWarning{Changebar}%
    {Changebar info has changed. Rerun to get the bars right.}
  \gdef\cb@checkHistory##1##2##3{}%
  \let\cb@barpoint=\cb@checkHistory}
%    \end{macrocode}
% \end{macro}
%
% \subsection{Macros For Making It Work With Nested Floats/Footnotes}
%
%
% \begin{macro}{\end@float}
%
%    This is a replacement for the \LaTeX-macro of the same name.  All
%    it does is check to see if changebars are active and, if so, it
%    puts changebars around the box containing the float.  Then it
%    calls the original \LaTeX\ |\end@float|.
%
%    \begin{macrocode}
\let\cb@endfloat=\end@float
%    \end{macrocode}
%    \begin{macrocode}
\def\end@float{%
  \cb@pop\cb@currentlist
  \ifnum\cb@currentpoint=\cb@nil
  \else
    \cb@push\cb@currentlist
    \global\@tempdima=\@tempdima
    \egroup
    \global\setbox\@currbox=%
        \vbox\bgroup\cb@start[\@tempdima]\unvbox\@currbox\cb@end
  \fi
  \cb@endfloat}
%    \end{macrocode}
%    This only works if this new version of |\end@float| is really
%    used. With \LaTeX2.09 the documentstyles used to contain:
% \begin{verbatim}
%    \let\endfigure\end@float
% \end{verbatim}
%    In that case this binding has to be repeated after the
%    redefinition of |\end@float|. However, the \LaTeXe\ class files
%    use |\newenvironment| to define the \Lenv{figure} and
%    \Lenv{table} environments. In that case there is no need to
%    rebind |\endfigure|.
% \changes{3.2}{1994/04/21}{Removed
%    \cs{let}\cs{endfigure}\cs{end@float}, no longer necessary}
% \end{macro}
%
% \begin{macro}{\@footnotetext}
%
%    Replacement for the \LaTeX\ macro of the same name.  Simply
%    checks to see if changebars are active and if so wraps the macro
%    argument (i.e., the footnote) in changebars.
%
% \changes{3.1b}{1993/10/11}{Added missing percent sign to prevent
%    spurious white space in the output.}
%    \begin{macrocode}
\let\cb@footnote=\@footnotetext
%    \end{macrocode}
%    \begin{macrocode}
\long\def\@footnotetext#1{%
  \cb@pop\cb@currentlist
  \ifnum\cb@currentpoint=\cb@nil
    \cb@footnote{#1}%
  \else
    \cb@push\cb@currentlist
    \edef\cb@temp{\the\@tempdima}%
    \cb@footnote{\cb@start[\cb@temp]#1\cb@end}%
  \fi}
%    \end{macrocode}
% \end{macro}
%
% \begin{macro}{\@mpfootnotetext}
%
%    Replacement for the \LaTeX\ macro of the same name.  Same thing
%    as |\@footnotetext|.
%
% \changes{3.1b}{1993/10/11}{Added missing percent sign to prevent
%    spurious white space in the output.}
%    \begin{macrocode}
\let\cb@mpfootnote=\@mpfootnotetext
%    \end{macrocode}
%    \begin{macrocode}
\long\def\@mpfootnotetext#1{%
  \cb@pop\cb@currentlist
  \ifnum\cb@currentpoint=\cb@nil
    \cb@mpfootnote{#1}%
  \else
    \cb@push\cb@currentlist
    \edef\cb@temp{\the\@tempdima}%
    \cb@mpfootnote{\cb@start[\cb@temp]#1\cb@end}%
  \fi}
%</package>
%    \end{macrocode}
% \end{macro}
%
% 
% \Finale
%
\endinput
%% \CharacterTable
%%  {Upper-case    \A\B\C\D\E\F\G\H\I\J\K\L\M\N\O\P\Q\R\S\T\U\V\W\X\Y\Z
%%   Lower-case    \a\b\c\d\e\f\g\h\i\j\k\l\m\n\o\p\q\r\s\t\u\v\w\x\y\z
%%   Digits        \0\1\2\3\4\5\6\7\8\9
%%   Exclamation   \!     Double quote  \"     Hash (number) \#
%%   Dollar        \$     Percent       \%     Ampersand     \&
%%   Acute accent  \'     Left paren    \(     Right paren   \)
%%   Asterisk      \*     Plus          \+     Comma         \,
%%   Minus         \-     Point         \.     Solidus       \/
%%   Colon         \:     Semicolon     \;     Less than     \<
%%   Equals        \=     Greater than  \>     Question mark \?
%%   Commercial at \@     Left bracket  \[     Backslash     \\
%%   Right bracket \]     Circumflex    \^     Underscore    \_
%%   Grave accent  \`     Left brace    \{     Vertical bar  \|
%%   Right brace   \}     Tilde         \~}
%%
%%
