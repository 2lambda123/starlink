%\scrollmode
%\documentclass[dvips,a4paper,twoside]{article}
\documentclass[dvips,a4paper]{article}
\usepackage{html,htmllist,makeidx,enumerate}
\usepackage[dvips]{graphicx,color}
%\usepackage{frames}
% \usepackage{times}

%
% use the url.sty package, by  Donald Arseneau <asnd@triumf.ca>
% to typeset email-addresses, URLs and directory paths in LaTeX ...
%
%begin{latexonly}
 \usepackage[dvips,rightbars]{changebar}
 \usepackage{l2hman}
 \usepackage{url}
 \usepackage{longtable}
 \def\email{\begingroup \urlstyle{tt}\Url}
 \def\Email#1{\email{<#1>}}
 \urldef\onlinedoc\url{http://www-dsed.llnl.gov/files/programs/unix/latex2html/manual/}
 \urldef\onlinedocRM\url{http://www-texdev.mpce.mq.edu.au/l2h/docs/manual/}
 \urldef\EXcolors\url{http://www-texdev.mpce.mq.edu.au/l2h/crayola/}
 \urldef\CVSrepos\url{http://cdc-server.cdc.informatik.tu-darmstadt.de/~latex2html/user/}
 \urldef\CVSsite\url{http://cdc-server.cdc.informatik.tu-darmstadt.de/~latex2html/}
 \urldef\CVSlatest\url{http://cdc-server.cdc.informatik.tu-darmstadt.de/~latex2html/l2h-latest.tar.gz}
 \urldef\patches\url{http://www-dsed.llnl.gov/files/programs/unix/latex2html/}
 \urldef\sourceA\url{http://www-dsed.llnl.gov/files/programs/unix/latex2html/sources/}
% \urldef\sourceB\url{ftp://ftp.mpn.com/pub/nikos/latex2html-98.1.tar.gz}
 \def\sourceB{\CTANtug\CTANA}
 \urldef\sourceC\url{http://ftp.rzg.mpg.de/pub/software/latex2html/sources/}
 \urldef\CTANtug\url{http://ctan.tug.org/ctan/}
 \urldef\CTANA\path{tex-archive/support/latex2html}
 \urldef\tugURL\url{http://www.tug.org/}
 \urldef\danteURL\url{http://www.dante.de/}
 \urldef\ListURL\url{http://www.rosat.mpe-garching.mpg.de/mailing-lists/LaTeX2HTML/}

%
%  a little hack, needed since a footnote occurs in a figure-caption
%
\def\adjustfootnote{\setcounter{footnote}{6}\footnotetext{http://csep1.phy.ornl.gov/csep.html}}

%
% These are needed for the Glossary and Index. 
%
\newenvironment{theglossary}{\begin{list}{}{\setlength{\labelwidth}{20pt}%
 \setlength{\leftmargin}{\labelwidth}\setlength\itemindent{-\labelwidth}%
 \setlength\itemsep{0pt}\setlength\parsep{0pt}\rmfamily}}{\end{list}}
\def\dotfill{\leaders\hbox to.6em{\hss .\hss}\hskip 0pt plus  1fill}%
\def\dotfil{\leaders\hbox to.6em{\hss .\hss}\hfil}%
\def\pfill{\unskip~\dotfill\penalty500\strut\nobreak\dotfil~\ignorespaces}%

\newcommand\Glossary[2]{\glossary{#1@#2}}
\newcommand{\gsl}{\textsl}
\newcommand{\indexentry}[2]{\item #1 #2}

%\newcommand{\latextohtml}{\textup{\LaTeX 2\texttt{HTML}}}%

\DeclareRobustCommand\FoilTeX{{\normalfont{\sffamily Foil}\kern-.03em{\rmfamily\TeX}}}
%
% These macros are built-in to LaTeX2HTML:
%
\newcommand{\Xy}{\leavevmode
 \hbox{\kern-.1em X\kern-.3em\lower.4ex\hbox{Y\kern-.15em}}}
\newcommand{\AmSTeX}{\protect\AmS-\protect\TeX{}}
\newcommand{\AmS}{{\protect\AmSfont A\kern-.1667em\lower.5ex\hbox{M}\kern-.125emS}}
\gdef\AmSfont{\usefont{OMS}{cmsy}{m}{n}}

\newcommand{\sameas}[1]{\ (Same as setting: #1)}
\setcounter{footnote}{0}
%end{latexonly}

\begin{imagesonly}
 \usepackage{l2hman}
\end{imagesonly}
%
\begin{htmlonly}
 \def\makeglossary{}
 \pagecolor[named]{White}
\end{htmlonly}

\input manhtml.tex


% use %sort -f -u manual.idx > manual.index for a primitive index
%
%  NOTE:  You must use LaTeX2e in order to process this document
%       If you do not have LaTeX2e, a PostScript version
%       (manual.ps) is included with this distribution.
%
%%%%%%%%%%%%%%%%%%% No changes beyond this point %%%%%%%%%%%%%%%%%%%%%%%%%%%%%

\makeindex
\makeglossary
\sloppy
%
\setlength{\textwidth}{5.5in}
%\addtolength{\oddsidemargin}{-1in}
%\addtolength{\evensidemargin}{-1in}
\setlength{\changebarwidth}{1pt}

%
% read own internals for sections/contents before any
% from the segments.
%
%\internal[sections]{}
%\internal[contents]{}

\internal[figure]{O}
\internal[figure]{S}
\internal[figure]{M}
\internal[figure]{H}
\internal[figure]{E}%{F}

\internal[table]{O}
\internal[table]{S}
\internal[table]{M}
\internal[table]{H}
\internal[table]{E}%{F}

\internal[sections]{O}
\internal[sections]{S}
\internal[sections]{M}
\internal[sections]{H}
\internal[sections]{E}%{F}
\internal[sections]{P}
%\internal[sections]{C}

\internal[contents]{O}
\internal[contents]{S}
\internal[contents]{M}
\internal[contents]{H}
\internal[contents]{E}%{F}
\internal[contents]{P}
%\internal[contents]{C}


\internal[internals]{O}
\internal[internals]{S}
\internal[internals]{M}
\internal[internals]{H}
\internal[internals]{E}%{F}
\internal[internals]{P}
%\internal[internals]{C}

%\internal[index]{O}
%\internal[index]{S}
%\internal[index]{M}
%\internal[index]{H}
%\internal[index]{E}%{F}
%\internal[index]{P}
%%\internal[index]{C}


\begin{document}
\sloppy
%
%  TITLE-PAGE 
%
\Glossary{latex2html}{\LaTeX2HTML}{}
\title{The \LaTeX2HTML{} Translator}
\author{Nikos Drakos\\Computer Based Learning Unit\\University of Leeds.}
\date{\today}
\index{Computer~Based~Learning~Unit!University of Leeds}%
\maketitle 

\htmlrule
%
\begin{center}{
Documentation revised and updated for \textsc{v97.1} and \texttt{HTML}~3.2;\\
and further revisions for \textsc{v98.1} and \texttt{HTML}~4.0 by:}
\end{center}
\medskip
\begin{center}
%begin{latexonly}
\begin{large}
%end{latexonly}
\RossMoore\\
Mathematics Department\\
\Macquarie, Sydney.
%begin{latexonly}
\end{large}
%end{latexonly}
\end{center}
\bigskip
\htmlrule

%
%  for printed version only
%
\begin{latexonly}
\begin{center}
{\large 
This document accompanies \latextohtml{} version 98.1}%
\footnote{{\bfseries This is a preliminary document for \latextohtml{} 98.1 with
revisions in section \htmlref{``Getting \latextohtml''}{sec:sup},
\htmlref{``Known Problems''}{sec:prb} and
\htmlref{``Environments and Special Features''}{sec:fea}.
A fully updated manual is underway and will be released soon.}

\NikosDrakos' original manuscript was updated for version \textsc{v96.1}\,%,
%as indicated with thin change-bars, 
by Herbert W.\ Swan \Email{lanhws@expl.aai.arco.com} 
and converted for \LaTeXe{} by \Goossens~\Email{goossens@cern.ch}.
Extensive revisions of the manuscript were made by \RossMoore~\Email{ross@mpce.mq.edu.au}
for \textsc{v96.1} \texttt{rev-f}, incorporating also suggestions from \Goossens.
Another major revision was required to adequately describe the new features 
made possible with \texttt{HTML} 3.2\,,
and recent developments in image-generation and macro-handling.
This work was done by \RossMoore, as were most of the revisions for \textsc{v98.1}.
Changes for the \textsc{v97.1} revision are indicated with narrow change-bars.
Wider change-bars indicate where the most recent changes and features are described.}%
\Glossary{latex2e}{\LaTeXe}%
\end{center}
\end{latexonly}

%
%  for HTML version only
%
\begin{htmlonly}

This document accompanies \latextohtml{}\Glossary{latex2html}{\LaTeX2HTML}{}, version 98.1.
\Glossary{latex2e}{\LaTeXe}{}

{\bf This is a preliminary document for \latextohtml{} 98.1 with
revisions in section \htmlref{``Getting \latextohtml''}{sec:sup},
\htmlref{``Known Problems''}{sec:prb} and
\htmlref{``Environments and Special Features''}{sec:fea}.
A fully updated manual is underway and will be released soon.}

The manuscript was updated for version 96.1 %, as indicated with change-bar icons, 
by Herbert W\,Swan \Email{lanhws@expl.aai.arco.com} 
and converted for \LaTeXe{} by \Goossens~\Email{goossens@cern.ch}.

Updates and extensive revisions to the manuscript for version \textsc{v96.1} \texttt{rev-f},
were made by \RossMoore~\Email{ross@mpce.mq.edu.au}, 
also incorporating suggestions from \Goossens. 

\bigskip
Another major revision was required to adequately describe the new features 
made possible with \texttt{HTML} 3.2\,,
and recent developments in image-generation and macro-handling.
This work was done by \RossMoore, as were most of the revisions for \textsc{v98.1}.

Appropriate change-bar icons indicate where newly added features are described.

\htmlrule

%
% customise the URLs below, for the nearest copies of  manual.ps(.gz)
%
\index{PostScript version}
A \htmladdnormallink{PostScript version}%{manual.ps}
{ftp://ftp.mpce.mq.edu.au/pub/maths/l2h/l2hmanual.ps.gz}
is available; also with \htmladdnormallink{no included fonts}%{manual.ps.gz}
{ftp://ftp.mpce.mq.edu.au/pub/maths/l2h/manual.nofonts.ps.gz},
instead with references to PostSript fonts.

\htmlrule

Browse the following links for information concerning:

\begin{htmllist}\htmlitemmark{RedBall}
\item[Contributions from others --- \htmlref{early development}{credits}] 
\item[Contributions from others --- \htmlref{recent developments, 1996}{recent96}] 
\item[Contributions from others --- \htmlref{recent developments, 1997}{recent97}] 
\item[Contributions from others --- \htmlref{recent developments, 1998}{recent98}] 
\item[Proposals for \htmlref{future development}{future}] 
\item[\htmlref{Licensing}{licence}] 
\end{htmllist}

\htmlrule

\textbf{Warning:} The contents of this document are likely to change.\\
It is advisable not to use links to any pages other than the first page (this page).

\end{htmlonly}

\htmlrule
\latex{\newpage\vglue1pt\vfil}
%
%
%  ABSTRACT
%
\Glossary{latex}{\LaTeX}{}%
\Glossary{perl}{\textsl{Perl}}{}%
\glossary{HTML}%
\begin{abstract}%
\latextohtml{} is a conversion tool that allows documents
written in \LaTeX{}  to become part of the World-Wide Web.
In addition, it offers an easy migration path towards
authoring complex hyper-media documents using
familiar word-processing concepts, including the power of a \LaTeX-like
macro language capable of producing correctly structured \texttt{HTML} tags.

\latextohtml{} replicates the basic structure of a \LaTeX{}  document 
as a set of interconnected \texttt{HTML} files which can be explored using
automatically generated navigation panels. 
The cross-references, citations, footnotes, the table-of-contents and the lists
of figures and tables, are also translated into hypertext links. Formatting
information which has equivalent ``tags'' in \texttt{HTML} 
(lists, quotes, paragraph-breaks, type-styles, etc.) 
is also converted appropriately. 
The remaining heavily formatted items
such as mathematical equations, pictures etc. are converted to images
which are placed automatically at the correct position in the
final \texttt{HTML} document.

\latextohtml{} extends \LaTeX{}  by supporting arbitrary hypertext links 
and symbolic cross-references between evolving 
remote documents. It also allows the specification
of \emph{conditional text} and the inclusion of raw \texttt{HTML} commands.
These hyper-media extensions to \LaTeX{}  are available as 
new commands and environments from within a \LaTeX{}  document.

This document presents the main features of \latextohtml{} and
describes how to obtain and install it, and how to use it effectively.
\end{abstract}


%
%  CREDITS
%
\latex{\pagenumbering{roman}}
\clearpage\section*{Credits, 1993--1994\label{credits}}%
\index{Computer~Based~Learning~Unit!University of Leeds}\html{\\}%
Several people have contributed suggestions, ideas, solutions, support
and encouragement. Some of these are \RodWilliams, \AnaPaiva,
\JamilSawar\ and \AndrewCole\ at the \CBLU.

\begin{htmllist}
\htmlitemmark{YellowBall}
\index{CERN!World-Wide Web Project}%
\item [\CERN]
The idea of splitting \LaTeX{}  files
into more than one component, connected via hyperlinks,
was first implemented in \Perl{} by Toni Lantunen at CERN.
Thanks to Robert Cailliau \Email{cailliau@cernnext.cern.ch}
of the World-Wide Web Project, also at CERN,
for providing access to the source code and documentation
(although no part of the original design or the actual code has been used).


\item [Robert S. Thau] \Email{rst@edu.mit.ai}
has contributed the new version of \fn{texexpand}.
Also, in order to translate the ``document from hell'' (!!!)
he has extended the translator to handle \Lc{def} commands,
nested math-mode commands, and has fixed several bugs.

\item [Phillip Conrad and L. Peter Deutsch.]
The \fn{pstogif} \Perl{} script uses the \fn{pstoppm.ps} \PS\ program,
originally written by Phillip Conrad (Perfect Byte, Inc.) and
modified by L. Peter Deutsch (Aladdin Enterprises).

\item [Roderick Williams]
The idea of using existing symbolic labels to provide cross-references
between documents was first conceived during discussions with
\RodWilliams{} \Email{rodw@cbl.leeds.ac.uk}\,.


\item [Eric Carroll] \Email{eric@ca.utoronto.utcc.enfm},
who first suggested providing a command like \Lc{hyperref}\,.

\index{accents!foreign}%

\item [Franz Vojik] \Email{vojik@de.tu-muenchen.informatik}
provided the basic mechanism for handling foreign accents.


\item [Todd Little]
The \Cs{auto\_navigation} option was based on an idea by Todd
\Email{little@com.dec.enet.nuts2u}\,.


\item [Axel Belinfante] \Email{Axel.Belinfante@cs.utwente.nl}
provided the \Perl{} code in the \fn{makeidx.perl} file,
as well as numerous suggestions and bug-reports.

\item
[Verena Umar] \Email{verena@edu.vanderbilt.cas.compsci} (\CSEP)
has been a very patient tester of some early versions of \latextohtml{}
and many of the current features are a result of her suggestions.

\index{mailing list!Argonne National Labs}%

\item [Ian Foster and Bob Olson.]
Thanks to Ian Foster \Email{itf@mcs.anl.gov}
and Bob Olson \Email{olson@mcs.anl.gov}
at the Argonne National Labs, for setting up the \maillist.
\end{htmllist}


\clearpage
\section*{Later Developments, 1995--1996\label{recent96}}%
%
Since 1995 the power and usefulness of \latextohtml{} has been enhanced significantly.
The revisions later than \textsc{v95.1} have been largely due
to the combined efforts of many people, other than the original author.
Interested users have supplied patches to fix a fault,
or implement a feature that previously was not supported.
Often a question or complaint to the discussion-group
(see \hyperref{Getting Support ...}{Section~}{}{support})
has spurred someone else to provide the necessary ``patch''.%

\bigskip\noindent
Arising from this work, special credit is due to:
\begin{htmllist}
\htmlitemmark{GreenBall}
\item [\Hennecke] \Email{hennecke@dbag.ulm.DaimlerBenz.COM}
for his many extensive revisions;

\item [\Noworolski] \Email{jmn@eecs.berkeley.edu}
for coordinating \textsc{v95.3};

\item [\Isani] \Email{isani@cfht.hawaii.edu}
for his improvement in GIF quality;

\index{CERN!World-Wide Web Project}\index{CERN!Michel Goossens}%
\item [\Goossens] \Email{goossens@cern.ch}
was the driving force behind the upgrade to \LaTeXe{} compatibility,
and other features developed at CERN;

\item [Herb Swan] \Email{herb.swan@perc.Arco.COM}
for coordinating \textsc{v96.1} of \latextohtml,
including much of the \Perl{} code
for the new features that were introduced,
and for providing a series of bug-fix revisions
prior to  \textsc{v96.1} \texttt{rev-f};

\item [\RossMoore] \Email{ross@mpce.mq.edu.au}
who has revised and extended this manual, helped design and test the
segmentation strategy, and later revisions of \textsc{v96.1}\,.
Ross organised the release of \textsc{v96.1} \texttt{rev-g}
and provided many of the improvements
incorporated into \textsc{v96.1} \texttt{rev-h}.

\item [\Wilck] \Email{martin@tropos.de}
for the initial work on implementation of \env{frames}.
Also Martin did most of the work implementing the extensive citation and
bibliographic features of the \env{natbib} package, written by \PatrickDaly.
He also provided the \fn{makeseg} \Perl{} script to create Makefiles
for segmented documents.

\item [\Lippmann] \Email{lippmann@cdc.informatik.tu-darmstadt.de}
for organising the releases \textsc{v96.1} \texttt{rev-h} to \textsc{v98.1}.
Jens made significant contributions to
the internal workings of \latextohtml,
as well as cleaning up much of its source code.
\end{htmllist}

\htmlrule


\bigskip\noindent
Many others, too many to mention, contributed bug-reports,
fixes and other suggestions.

\latex{\bigskip}\htmlrule

\index{URL!url package@\env{url} package}%
\noindent
Thanks also to Donald Arseneau \Email{asnd@triumf.ca} for allowing his \fn{url.sty}
to be distributed with this manual.
Similarly, thanks to Johannes Braams \Email{JLBraams@cistron.nl} for \fn{changebar.sty}.
%
\index{change-bars}\html{\\}
Both of these are useful utilities which enhance the appearance of the printed manual.
\begin{latexonly}
In particular, changes introduced with  \textsc{v96.1} and its revisions are denoted
by thin change-bars, while thicker bars denote changes introduced with  \textsc{v97.1}
and later releases.%
\end{latexonly}



\clearpage
\section*{Developments: late 1996 to mid 1997\label{recent97}}%
%
During the latter part of 1996 there was much work on improving the
capabilities of \latextohtml.
Some of this was due to the \WiiiC's proposals for \HTMLiii,
becoming a formal recommendation in November 1996,
and their subsequent acceptance in January 1997.
Existing \LaTeX{} markup for effects such as centering, left-
or right-justification of paragraphs,
flow of text around images, table-layout with formal captions, etc.
could now be given a safe translation into \HTMLiii, compliant with a standard
that would guarantee that browsers would be available to view such effects.

At the same time developers were exploring ways to enhance the overall
performance of \latextohtml.
As a result the current \textsc{v97.1} release has significant improvements in
the following areas:
%
\begin{htmllist}\htmlitemmark{OrangeBall}
%
\item[image-generation]
is much faster, requires less memory
and inline images are aligned more accurately;
%
\item[image quality]
is greatly improved by the use of anti-aliasing effects for on-screen clarity,
in particular with mathematics, text and line-drawings;
%
\item[memory-requirements]
are much reduced, particularly with image-generation;
%
\item[mathematics]
can now be handled using a separate parsing procedure;
images of sub-parts of expressions can be created, rather
than using a single image for the whole formula;
%
\item[macro definitions]
having a more complicated structure than previously allowed,
can now be successfully expanded;
%
\item[counters]
and numbering are no longer entirely dependent on the \texttt{.aux}
file generated by \LaTeX;
%
\item[decisions]
about which environments to include or exclude can now be made;
%
\item[HTML effects]
for which there is no direct \LaTeX{} counterpart
can be requested in a variety of new ways;
%
\item[HTML code]
produced by the translator is much neater and more easily readable,
containing more comments and fewer redundant breaks and \HTMLtag{P} tags.
%
\item[error-detection]
of simple \LaTeX{} errors, such as missing or unmatched braces,
is now performed --- a warning message shows a line or two
of the source code where the error has apparently occurred;
%
\end{htmllist}


\medskip\htmlrule[50\% center]
\noindent
For these developments, thanks goes especially to:
%
\begin{htmllist}
%
\item [\Lippmann] \Email{lippmann@cdc.informatik.tu-darmstadt.de}
for creating and maintaining the CVS repository at \CVSrepos\,.
This has made it much easier for the contributions from different developers
to be collected and maintained as a ``development version'' which
is kept up-to-date and available at all times. Together with \Rouchal\
he produced an extensive rewrite of the \fn{texexpand} utility.


\item [\RossMoore] \Email{ross@mpce.mq.edu.au}
for extensive work on almost all aspects of the \latextohtml{} source
and documentation,
combining code for \LaTeX{}, \Perl{}, \texttt{HTML} and other utilities.
Most of the coding for the new features based on \texttt{HTML} 3.2,
many of the new packages, faster image-generation
and the improved support for mathematics
and other environments, is his work.


\item [\Rouchal] \Email{marek@hl.siemens.de}
for extending the former \fn{pstogif} utility,
transforming it into \fn{pstoimg} which now allows for
alternative image formats, such as \fn{PNG}.
Also he produced the neat \fn{configure-pstoimg} script, which eases
\latextohtml{} installation, and a rewrite of \fn{texexpand}.



\index{portability!Unix systems}\index{latex2html-NG}%
\item [\Hennecke] \Email{hennecke@dbag.ulm.DaimlerBenz.COM}
who has always been there, up-to-date with developments in \texttt{HTML} and
related matters concerning Web publishing,
and tackling the issues involved with portability
of \latextohtml{} to Unix systems on various platforms.

Furthermore Marcus has produced \latextohtmlNG, a version of
\latextohtml{} which handles expansion of macros in a more ``\TeX-like''
fashion. This should lead to further improvements in speed and efficiency,
while allowing complicated macro definitions to work as would be expected
from their expansions under \LaTeX.
(This requires \Perl{ 5}\,,
using some programming features not available with \Perl{ 4}\,.)%


\item [\Popineau] \Email{popineau@esemetz.ese-metz.fr} has produced
an adaptation for the Windows NT platform, of \latextohtml{} \textsc{v97.1}\,.

\item [\Wortmann] \Email{uli12@bonk.ethz.ch}
showed how to configure \appl{Ghostscript} to produce
anti-aliasing effects within images.

\item [\AxelRamge] \Email{axel@ramge.de}
for various suggestions and examples of enhancements,
and the code to avoid a problem with \appl{Ghostscript}.

\end{htmllist}

\medskip\vfil\htmlrule\bigskip\noindent
Thanks also to all those who have made bug-reports, supplied fixes
or offered suggestions as to features that might allow \latextohtml{}
to be used more efficiently in particular circumstances.
Most of these have been incorporated into this new version \textsc{v97.1}\,,
though perhaps not in the form originally envisaged.
Such feedback has contributed enormously to helping make \latextohtml{} the
easy to use, versatile program that it has now become.

\bigskip
\begin{center}
Keep the ideas coming!
\end{center}
\bigskip
\vfil




\subsection*[center]{1st \LaTeX2HTML{} Workshop\\Darmstadt,
15 February 1997\label{darmstadt}}
Thanks again to \Lippmann\ and members of the \LiPS\ for organising this meeting;
also to the \FIDarmstadt\ at \Darmstadt\ for use of their facilities.

\noindent
This was an opportunity for many of the current \latextohtml{} developers to
actually meet for the first time; rather than communication by exchange
of electronic mail messages.
%
\begin{itemize}
\item
\NikosDrakos\ talked about the early development of \latextohtml, while\dots
\item \dots
\RossMoore, \Lippmann\ and \Rouchal\ described recent improvements.
\item
\Goossens\ presented a list of difficulties encountered with earlier
versions of \latextohtml{}, and aspects requiring improvement.
Almost all of these now have been addressed in the \textsc{v97.1} release,
so far as is possible within the bounds inherent in the \HTMLiii\ standard.
\item
\KrisRose\ showed how it is possible to create \texttt{GIF89} animations
from pictures generated by \TeX{} or \LaTeX{}, using the \XypicDK\ graphics
package and extensions, developed by himself and \XypicAUS.
\end{itemize}

\noindent
Also present were representatives from the \DANTE\ \Praesidium\ and
members of the \LaTeXiii\ development team.\html{\\}
In all it was a very pleasant and constructive meeting.

\vfil

\subsection*[center]{TUG'97 --- Workshop on \LaTeX2HTML\\
University of San Francisco, 28 July 1997\label{tug97}}

\noindent
On the Sunday afternoon (2.00pm--5.00pm)
immediately prior to the TUG meeting, there will be a workshop
on \latextohtml, conducted by \RossMoore\footnote{%
Mathematics Department, Macquarie University, Sydney, Australia}.

\begin{itemize}
\item[]
Admission: \$50, includes a printed copy of the latest \latextohtml\ manual.
\end{itemize}

\bigskip

\subsection*[center]{\TeX{}Northeast TUG Conference, \TeX/\LaTeX{} Now\\
March 22--24, 1998, New York City}

\noindent
Includes a workshop/presentation by \RossMoore\footnote{%
Mathematics Department, Macquarie University, Sydney, Australia}.


\subsection*[center]{Euro-\TeX{}'98, 10th European \TeX{} Conference\\
St. Malo, France --- 29--31 March, 1998}

\noindent
Several of the \latextohtml{} developers will be present.
All European (and other) \latextohtml{} users are encouraged to attend.


\vfil


\clearpage
\section*{Developments: late 1997 to early 1998\label{recent98}}%
Much of the work contributed to \latextohtml{} during this time was
related to bug fixing and maintaining the 97.1 release, in order to
reach a more stable and reliable version which produces \fn{HTML} code
conforming to the W3C standards/drafts.
To keep in context with this view, support for \texttt{HTML 4} has been
incorporated into the translator.

\smallskip\noindent
There have been improvements to the way math code is handled, as well
as font-changing and numbering commands. These now are expected to
work much closer to the way that \LaTeX{} handles them.

\smallskip\noindent
Furthermore, missing \LaTeX{} style translations for basic \LaTeX{}
and \AmSTeX{} document classes were added to the distribution:
\texttt{book.perl}, \texttt{report.perl}, \texttt{article.perl},
\texttt{letter.perl}, \texttt{amsbook.perl} and \texttt{amsart.perl}.
New styles implementing \LaTeX{} packages include \texttt{seminar.perl},
\texttt{inputenc.perl} and \texttt{chemsym.perl} naming but a few.

The aim is ultimately to support all \LaTeX{}, \AmSTeX{} etc. packages in the
standard \LaTeX{} distribution, or for which there is published documentation.
At the time of writing this aim has not quite been reached.
To support internationalisation, \Perl{} extensions were provided for
\texttt{HTML} output conforming to ISO-Latin 1, 2, 3, 4, 5, 6,
and Unicode encodings.

\smallskip\noindent
All of the above work was done by \RossMoore.
\bigskip

Additional document formats are now supported, these are Indic\TeX{},
\FoilTeX{}, and \texttt{CWEB} documents.
You may use any of these packages to translate such documents together
with \latextohtml{}, refer to the instructions in the various
\texttt{README} files.

\smallskip\noindent
Thanks go to \RossMoore{} for \IndicHTML{}, to \Veytsman{} for
\FoilTeX{}/\texttt{HTML} and to \Lippmann{} for the \texttt{CWEB} to
\texttt{HTML} translator.

\bigskip
Numerous discussions and efforts have been undertaken to get
\latextohtml{} working independent from the underlying operating
system.
Yet all obstacles are not quite taken, but it is forseeable that we are
OS independent very soon.
This release has been reported to run on OS/2, DOS, and MacOS, besides
Unix-like operating systems.
A former version has also been ported to Amiga OS, but that results
still need to be re-integrated into the source.
Ports for Windows'95 and Windows NT exist, but are not yet
integrated with the main distribution.

\smallskip\noindent
Thanks go to \Hennecke, \AxelRamge, \Rouchal{} and \Wortmann{} for
fruitful and refreshening discussions about that \fn{Override.pm}
loading scheme (which finally made its way after enough chickens and
eggs chased one another to death \mbox{$\mathsmiley$}\,),
and to \Taupin{} for his successful efforts to get \latextohtml{}
running on DOS.

\smallskip\noindent
Thanks go also to \Popineau{} for his port to Windows NT
\footnote{\ctanURL{systems/win32/web2c/l2h-win32.tar.gz}},
and \NikosDrakos{} for a Windows 95 port based on \textsc{v96.1}h
\footnote{\url{ftp://ftp.mpn.com/pub/nikos/latex2html96.1-h-win32.tar.gz}}
(which is mentioned here at last, but not least).

\medskip\noindent
We want to take the opportunity to thank \Nelson{} and the people at
Lawrence Livermore National Laboratory who help to keep up the
\latextohtml{} main archive and the mailing list, and to \Bohnet{} at
the Max Planck Institut fuer extraterrestrische Physik, Garching for
maintaining the list's online archive.
Finally thanks and greetings to all people that contributed to this
release and have not been mentioned here...

\medskip\noindent
You all showed spirit and favour. Thank you for your efforts!

\medskip
\begin{flushright}
\dots\ and don't forget Jens and the LiPS team at Darmstadt!
\end{flushright}

\clearpage
\subsection*{Proposals for Future Development:\label{future}}%



\subsection*{\latextohtmlNG}
Developed by \Hennecke\label{latex2htmlNG} this is a version of
\latextohtml{} that addresses various issues,
not currently handled in the best way by version \textsc{v97.1}\,.
These include:
\begin{itemize}
\item validating the \texttt{HTML} output,
so that only correctly nested tags, and their contents,
can be produced by the translator;
%
\item more \TeX-like order of macro-expansion,
so that macros and their expansions will produce exactly
the results expected from the \TeX{} implementation of \LaTeX;
%
\item faster processing,
by streamlining some of the current \Perl{} code, and allowing
shorter strings to be handled at any given time;
%
\item customisation issues,
allowing easier portability to Unix-like environments on
other platforms.
%
\end{itemize}
Many of these features have been the inspiration for new code
written for \latextohtml~\textsc{v98.1}.

\medskip\noindent
The current version of \latextohtmlNG{} can be obtained from
the developer's \hyperref[page]{repository}{repository, see page~}{}{cvsrepos},
in the directory
\url{http://cdc-server.cdc.informatik.tu-darmstadt.de/~latex2html/ng-user}.
Beware that the files there are \emph{not} compatible with those of the
same name that come with the current version of \latextohtml.



\subsection*{Extended Capabilities in Web browsers}
The following areas are the subject of active development
within the Web community.
Limited support is available within \latextohtml{} for some of these features,
using the \texttt{-html\_version 4.0} command-line switch.
\begin{description}
%
\index{style sheets!CSS}%
\index{style sheets!DSSSL}%
\item [style-sheets: ] \htmladdnormallink{proposals}%
{http://www.w3.org/pub/WWW/TR/WD-style-970324.html}
for a flexible mechanism to allow cascading (CSS) and DSSSL,
within \htmladdnormallinkfoot{\HTMLiv}{http://www.w3.org/pub/Markup/}.
%
\index{extended markup!XML}%
\index{XML!extended markup}%
\item [XML: ] \htmladdnormallinkfoot{eXtensible Markup Language}%
{http://www.w3.org/pub/WWW/TR/WD-xml.html}.

\index{mathematics!markup, MathML}%
\index{MathML!mathematics markup}%
\item [MathML: ] \htmladdnormallinkfoot{Mathematical Markup Language}%
{http://www.w3.org/pub/WWW/TR/WD-math-970515}.

\index{chemical!markup, CML}%
\index{CML!chemical markup}%
\item [CML: ] \htmladdnormallinkfoot{Chemical Markup Language}%
{http://www.venus.co.uk/omf/cml}.

\index{fonts!non-standard encodings}%
\item [Fonts: ] further support for non-standard font encodings.

\index{icons}%
\item [Icons: ] Alternative sets of icons for navigation buttons
and other purposes.
\end{description}
For some background on these technologies read
\Goossens' survey article ``Hyper-activity in the Web-world''
in \textsl{CERN Computer Newsletter}
\htmladdnormallinkfoot{No. 227}{http://wwwinfo.cern.ch/cnls/227/art\_xml.html},
and browse \AxelRamge's
\htmladdnormallinkfoot{site}{http://www.ramge.de/ax/latex2html/latex2html.html}
for ideas on how they could be used with \latextohtml.



\endinput
















\clearpage\section*{General License Agreement and Lack of Warranty\label{licence}}%
This software is distributed in the hope that it will be useful
but \emph{without any warranty}. The author(s) do not accept responsibility 
to anyone for the consequences of using it or for whether it serves 
any particular purpose or works at all. No warranty is made about 
the software or its performance. 
 
Use and copying of this software and the preparation of derivative
works based on this software are permitted, so long as the following
conditions are met:
\begin{itemize}
\item 
The copyright notice and this entire notice are included intact
and prominently carried on all copies and supporting documentation.
\item 
No fees or compensation are charged for use, copies, or
access to this software. You may charge a nominal
distribution fee for the physical act of transferring a
copy, but you may not charge for the program itself. 
\item 
If you modify this software, you must cause the modified
file(s) to carry prominent notices (a \texttt{ChangeLog})
describing the changes, who made the changes, and the date
of those changes.
\item  
Any work distributed or published that in whole or in part
contains or is a derivative of this software or any part 
thereof is subject to the terms of this agreement. The 
aggregation of another unrelated program with this software
or its derivative on a volume of storage or distribution
medium does not bring the other program under the scope
of these terms.
\end{itemize} 
This software is made available \emph{as is}, and is distributed without 
warranty of any kind, either expressed or implied.
In no event will the author(s) or their institutions be liable to you
for damages, including lost profits, lost monies, or other special,
incidental or consequential damages arising out of or in connection
with the use or inability to use (including but not limited to loss of
data or data being rendered inaccurate or losses sustained by third
parties or a failure of the program to operate as documented) the 
program, even if you have been advised of the possibility of such
damages, or for any claim by any other party, whether in an action of
contract, negligence, or other tortuous action.

\smallskip
\index{copyright!Leeds}\html{\\}%
The \latextohtml{} translator is written by Nikos Drakos, 
Computer Based Learning Unit,  University of Leeds,  Leeds,  LS2 9JT.
Copyright \copyright 1993--1997. All rights reserved.

\smallskip
\index{copyright!Macquarie}\html{\\}%
The \textsc{v97.1} revision of the \latextohtml{} translator and this manual
was prepared by Ross Moore, Mathematics Department,  
Macquarie University,  Sydney~2109,  Australia.
Copyright \copyright 1996--1997. All rights reserved.










%
%  CONTENTS, 
%  Lists of figures, tables
%
\clearpage
\tableofcontents
\clearpage
\listoffigures
\listoftables
%begin{latexonly}
 \adjustfootnote
%end{latexonly}
\clearpage


%
%  MAIN MANUAL
%

%begin{latexonly}
\cleardoublepage
\pagenumbering{arabic}\setcounter{page}{1}
%end{latexonly}

\relax   %% this is important, else the next segment doesn't get processed

%%% START XTRACTFAQ (END is somewhere within that segment)
\segment{overview}{section}{Overview}
\latex{\vfil\goodbreak}
%%% START XTRACTFAQ
\segment{support}{section}{Installation and Further Support}
%%% END XTRACTFAQ
\latex{\vfil\goodbreak}
\segment{features}{section}{Environments and Special Features}
\latex{\vfil\goodbreak}
\segment{hypextra}{section}{Hypertext Extensions to \LaTeX{}}
\latex{\vfil\goodbreak}
\segment{userman}{section}{Customising the Layout of HTML pages}
\latex{\vfil\goodbreak}
%%% START XTRACTFAQ
\segment{problems}{section}{Known Problems}
%%% END XTRACTFAQ

%\segment{changes}{section}{Changes from Previous Versions
% \protect\label{sec:chg}\protect\index{changes|(}}

%
%  CHANGES
%
%\begin{htmlonly}
%\relax   %% this is important, else the next section doesn't get handled correctly
%\section{Changes from Previous Versions}
%\input{changes.tex}
%\end{htmlonly}

\begin{htmlonly}
\endsegment
\end{htmlonly}

%%% START XTRACTFAQ
%
%  BIBLIOGRAPHY
%
\bibliographystyle{plain}
\begin{thebibliography}{1}\label{biblio}

\index{LaTeX blue book@\LaTeX{} blue book!Leslie Lamport}%
\htmladdimg[ALIGN=RIGHT]{http://www.awl.com/cseng/images/tri.gif}
\bibitem{lamp:latex}
Leslie Lamport,
\newblock \LaTeX,\textit{A Document Preparation System}. User's Guide \& Reference Manual, 2nd edition.
\newblock ISBN 0--201--52983--1, Paperback 256 pages, %
\htmladdnormallink{Addison--Wesley}%
{http://www.awl.com/cseng/titles/0-201-52983-1}, 1994.
\newblock Online information on {\TeX} and {\LaTeX} is available at \tugURL\ and~\danteURL~.

%begin{latexonly}
\index{Companion|see{The \LaTeX\hfil Companion}}%
\index{LaTeX Companion@\LaTeX\ Companion|see{~The \LaTeX\\ Companion}}%
\index{The LaTeX Companion@\emph{The \textup{\LaTeX} Companion}\label{IIIlatcomp}!Goossens--Mittelbach--Samarin}%
%end{latexonly}
\begin{htmlonly}
\index{Companion|see{\htmlref{~The \textup{\LaTeX} Companion}{IIIlatcomp}}}%
\index{LaTeX Companion@\LaTeX\ Companion|see{\htmlref{The \textup{\LaTeX} Companion}{IIIlatcomp}}}%
\index{The LaTeX Companion@The \textup{\LaTeX} Companion\label{IIIlatcomp}!Goossens--Mittelbach--Samarin}%
\end{htmlonly}
\htmladdimg[ALIGN=RIGHT]{http://www.awl.com/cseng/titles/0-201-54199-8/tlc.gif}
\bibitem{goossens:latex}
Michel Goossens, Frank Mittelbach, Alexander Samarin,
\newblock \textit{The \textup{\LaTeX} Companion}
\newblock ISBN 0--201--54199--8, Paperback 530 pages, %
\htmladdnormallink{Addison--Wesley}%
 {http://www.awl.com/cseng/titles/0-201-54199-8/}, 1994.\medskip

%begin{latexonly}
\index{Graphics Companion|see{~The~\textup{\LaTeX}\\ Graphics Companion}}%
\index{LaTeX Graphics Companion@\LaTeX\ Graphics Companion|see{~The\\ \textup{\LaTeX} Graphics Companion}}%
\index{The LaTeX Graphics Companion@\emph{The \textup{\LaTeX} Graphics Companion}\label{IIIlatgraph}!Goossens--Rahtz--Mittelbach}%
%end{latexonly}
\begin{htmlonly}
\index{Graphics Companion|see{\htmlref{The \textup{\LaTeX}\latex{\hfil} Graphics Companion}{IIIlatgraph}}}%
\index{LaTeX Graphics Companion@\LaTeX{} Graphics Companion|see{\htmlref{The \textup{\LaTeX}\latex{\hfil} Graphics Companion}{IIIlatgraph}}}%
\index{The LaTeX Graphics Companion@The \textup{\LaTeX} Graphics Companion\label{IIIlatgraph}!Goossens--Rahtz--Mittelbach}%
\end{htmlonly}
\htmladdimg[ALIGN=RIGHT]{http://www.awl.com/cseng/titles/0-201-85469-4/coversm.gif}
\bibitem{goossens:latexGraphics}
Michel Goossens, Sebastian Rahtz and Frank Mittelbach,
\newblock \textit{The \textup{\LaTeX} Graphics Companion}.
\newblock ISBN 0--201--85469--4, Softcover 608 pages, %
\htmladdnormallink{Addison--Wesley}{http://www.awl.com/cseng/titles/0-201-85469-4/}, 1997.

\bibitem{drakos:bask}
\NikosDrakos,
\newblock Text to Hypertext conversion with \latextohtml.
\newblock \textit{Baskerville}, December 1993, Vol.\,3, No.\,2, pp 12--15.
\newblock May 1994, CERN, Geneva, Switzerland.
\newblock \url{http://cbl.leeds.ac.uk/nikos/doc/www94/www94.html}

\bibitem{drakos:www94}
\NikosDrakos,
\newblock From Text to Hypertext: A Post-Hoc Rationalisation of \latextohtml.
\newblock Published in ``Proceedings of the 1st World Wide Web Conference'',
\newblock May 1994, CERN, Geneva, Switzerland.
\newblock \url{http://cbl.leeds.ac.uk/nikos/doc/www94/www94.html}

\end{thebibliography}
%%% END XTRACTFAQ


%
%  GLOSSARY
%
% Glossary info stored in:  manual.gls ,  which was created using:
%
%       makeindex -o manual.gls -s l2hglo.ist manual.glo
%       
\begin{latexonly}
\InputIfFileExists{manual.gls}{\clearpage\typeout{^^Jcreating Glossary...}}%
{\typeout{^^JNo Glossary, since  manual.gls  could not be found.^^J}}
\end{latexonly}

\begin{htmlonly}
\section{Glossary of variables and file-names\label{Glossary}}
\begin{htmllist}\htmlitemmark{OrangeBall}
\input l2hfiles.dat     %%%%  <<<<<<<<<<  don't forget, in final version !!!
\end{htmllist}
\end{htmlonly}

%
%  INDEX
%
\internal[index]{O}
\internal[index]{S}
\internal[index]{E}
\internal[index]{H}
\internal[index]{M}
\internal[index]{P}
%
% Index info stored in:  manual.ind ,  which was created using:
%
%       makeindex -s l2hidx.ist manual.idx
%       
\printindex

%
%  Alphabetization and navigation within the index
%  ...these special index entries must come *after* the  \printindex
%  else half of the hyperlinks will point to the preceding page.
%
\begin{htmlonly}
\newcommand{\indexAlpha}[5]{\index{#1@\htmlref{_}{#2}%
 \htmlref{\HTML{SUB}{\LARGE #3}}{AZ}\htmlref{_}{#4}\label{#5}| }}
%
\indexAlpha{\$}{Z}{\$}{dot}{doll}%
\indexAlpha{.}{doll}{~.~}{A}{dot}%
\indexAlpha{A}{dot}{A}{B}{A}%
\indexAlpha{B}{A}{B}{C}{B}%
\indexAlpha{C}{B}{C}{D}{C}%
\indexAlpha{D}{C}{D}{E}{D}%
\indexAlpha{E}{D}{E}{F}{E}%
\indexAlpha{F}{E}{F}{G}{F}%
\indexAlpha{G}{F}{G}{H}{G}%
\indexAlpha{H}{G}{H}{I}{H}%
%\indexAlpha{I}{H}{I}{J}{I}%
\indexAlpha{I}{H}{I}{L}{I}%
\indexAlpha{J}{I}{J, K}{L}{K}%
%\indexAlpha{J}{I}{J}{K}{J}%
%\indexAlpha{K}{J}{K}{L}{K}%
\indexAlpha{L}{K}{L}{M}{L}%
\indexAlpha{M}{L}{M}{N}{M}%
\indexAlpha{N}{M}{N}{O}{N}%
\indexAlpha{O}{N}{O}{P}{O}%
%\indexAlpha{P}{O}{P}{Q}{P}%
\indexAlpha{P}{O}{P}{R}{P}%
\indexAlpha{Q}{P}{Q, R}{S}{R}%
%\indexAlpha{Q}{P}{Q}{R}{Q}%
%\indexAlpha{R}{Q}{R}{S}{R}%
\indexAlpha{S}{R}{S}{T}{S}%
\indexAlpha{T}{S}{T}{U}{T}%
\indexAlpha{U}{T}{U}{V}{U}%
\indexAlpha{V}{U}{V}{W}{V}%
\indexAlpha{W}{V}{W}{X}{W}%
%\indexAlpha{X}{W}{X}{Y}{X}%
\indexAlpha{X}{W}{X}{Z}{X}%
\indexAlpha{Y}{X}{Y, Z}{doll}{Z}%
%\indexAlpha{Y}{X}{Y}{Z}{Y}%
%\indexAlpha{Z}{Y}{Z}{doll}{Z}%
%
%
%% This is an alphabetical navigation panel.
\index{@\label{AZ}\textbf{\LARGE
\htmlref{\$}{doll} \htmlref{.}{dot} \htmlref{ A }{A} 
 \htmlref{B}{B} \htmlref{C}{C} \htmlref{D}{D} \htmlref{E}{E} \htmlref{F}{F}
 \htmlref{G}{G} \htmlref{H}{H} \htmlref{I}{I} \htmlref{J}{K} \htmlref{K}{K}
 \htmlref{L}{L} \htmlref{M}{M} \htmlref{N}{N} \htmlref{O}{O} \htmlref{P}{P}
 \htmlref{Q}{R} \htmlref{R}{R} \htmlref{S}{S} \htmlref{T}{T} \htmlref{U}{U}
 \htmlref{V}{V} \htmlref{W}{W} \htmlref{X}{X} \htmlref{Y}{Z} \htmlref{Z}{Z}\\
\htmlrule[all]| }}

\end{htmlonly}

\end{document}
