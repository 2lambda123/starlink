\documentstyle{article} 
\pagestyle{myheadings}

%------------------------------------------------------------------------------
\newcommand{\stardoccategory}  {Starlink User Note}
\newcommand{\stardocinitials}  {SUN}
\newcommand{\stardocnumber}    {32.12}
\newcommand{\stardocauthors}   {P M Allan}
\newcommand{\stardocdate}      {24 April 1989}
\newcommand{\stardoctitle}     {CHART --- Finding chart and stellar data
system\footnote{This paper was converted to \LaTeX\ format by P M Allan.
The text of the original version by J V Carey \& K F Hartley has only been
slightly modified with the conversion to GKS 7.2.}
}
%------------------------------------------------------------------------------

\newcommand{\stardocname}{\stardocinitials /\stardocnumber}
\markright{\stardocname}
\setlength{\textwidth}{160mm}
\setlength{\textheight}{240mm}
\setlength{\topmargin}{-5mm}
\setlength{\oddsidemargin}{0mm}
\setlength{\evensidemargin}{0mm}
\setlength{\parindent}{0mm}
\setlength{\parskip}{\medskipamount}
\setlength{\unitlength}{1mm}

%------------------------------------------------------------------------------
% Add any \newcommand or \newenvironment commands here
%------------------------------------------------------------------------------

\begin{document}
\thispagestyle{empty}
SCIENCE \& ENGINEERING RESEARCH COUNCIL \hfill \stardocname\\
RUTHERFORD APPLETON LABORATORY\\
{\large\bf Starlink Project\\}
{\large\bf \stardoccategory\ \stardocnumber}
\begin{flushright}
\stardocauthors\\
\stardocdate
\end{flushright}
\vspace{-4mm}
\rule{\textwidth}{0.5mm}
\vspace{5mm}
\begin{center}
{\Large\bf \stardoctitle}
\end{center}
\vspace{5mm}

%------------------------------------------------------------------------------
%  Add this part if you want a table of contents
%  \setlength{\parskip}{0mm}
%  \tableofcontents
%  \setlength{\parskip}{\medskipamount}
%  \markright{\stardocname}
%------------------------------------------------------------------------------

\begin{center}
{\Large\bf Version 3.3}
\end{center}

\section{Introduction}

The CHART package extracts positions, magnitudes and other data from certain
stellar and non-stellar astronomical catalogues.
The results may be listed on a terminal or printer, or used to plot finding
charts or perform astrometric reductions.

There are four modes of operation:
\begin{list}{}{\setlength{\leftmargin}{20mm}
\setlength{\labelsep}{5mm}
\setlength{\labelwidth}{5mm}
}
\item[{\bf CSI}]
Stars can be selected from the Catalogue of Stellar Identifications (CSI) 1979
edition, prepared by the Stellar Data Centre, Strasbourg.
This is a compilation of 434,927 stars, extracted from a wide range of original
source catalogues.
Approximate positions are available for each star, together with cross
references into the original source catalogues (see section \ref{catalogues}).
Magnitudes are available for most of the stars (see section \ref{catalogues}).
Used in this mode, the program can act as a stellar data retrieval system to
find, for example, which stars are (or are not) in a particular catalogue, or
what information is known about the stars in a particular field.
This is also the normal mode of use when preparing overlays or finding charts.
\item[{\bf ASTROMETRIC}]
Stars can be selected from a merged astrometric catalogue (prepared at RGO by
W.~Nicholson).
This contains the whole of the AGK3 catalogue (extending down to Dec -3), the
SAO stars south of the celestial equator, and the complete PERTH70 catalogue.
This catalogue contains fewer stars than the CSI catalogue, but includes
accurate positions and proper motions as well as magnitudes and spectral types.
This mode is most useful when preparing to carry out astrometry.
\item[{\bf NONSTELLAR}]
Nonstellar objects can be selected from the Dixon `Master List of Nonstellar
Objects'.
This was formed by a simple merging of many catalogues and lists (a number of
stellar objects also appear).
The total number is about 185,000.
Approximate angular sizes and magnitudes are given for many objects.
Full details of these three groups of catalogues are given in section 
\ref{catalogues}.
\item[{\bf PRIVATE}]
Stars can be selected from any catalogue that has been produced in the form that
CHART expects for `ASTROMETRIC' mode.
A program is included in CHARTDIR (EXAMPLE.FOR) as an example of how this might
be performed.
\end{list}
You can select which of these four modes to work in and specify a series of
search areas.
In CSI, ASTROMETRIC or PRIVATE modes you can also place magnitude or total
number limits on the search and choose which source catalogues to include or
exclude.
A series of field centres may be input at the terminal or read from a previously
prepared file.

After the search is made, you may list the results at a terminal or on the
printer.
Positional information may be precessed to a specified equinox.
In the case of ASTROMETRIC data, proper motions may be applied up to a specified
epoch.

You may also plot the results in the form of an overlay or finding chart.
Extra objects, i.e.\ with positions supplied by you, may be added to the plot at
this stage.
The plot may be made on any GKS device, and a number of different plot options
are available, such as scale and area, an RA and Dec coordinate grid and various
forms of error box.

This version of CHART allows you to choose between having hard copy 
output plotted automatically, or being in complete control yourself.
This should help CHART users over the network.

You may also use the results as input to an astrometry program, i.e.\ CHART
can be used to select ASTROMETRIC stars as positional references.
In this case, you will be asked to supply x,y positions for the reference
stars from a measuring machine, and can then convert unknown x,y positions to
RA, Dec or vice versa.
The Starlink program ASTROM (see SUN/5) performs the actual astrometry, from
within CHART if necessary.
\section{A Beginner's User Guide}
\subsection{Starting up}
Since CHART creates a number of files in the user's filestore, it is important
that there is sufficient free space available.
A minimum of 500 free blocks is advised.
The program is started up by typing the command:
\begin{quote}
{\tt \$ CHART}
\end{quote}
The command PROMPT will give an introductory summary of the principal commands,
followed by the prompt `CHART$>$'.
\subsection{Help}
Once the CHART environment has been set up in this way, typing HELP now gives
information about CHART, not about VMS commands.
\subsection{Prompts}
When running programs in the CHART environment, you will frequently be prompted
for a value for a parameter, and a default value will be suggested if
appropriate, as in `EQUINOX/1950.0/:='.
Just hitting $<$CR$>$ will give the default value.

To make the programs less verbose for experienced users, no other prompts are
given.
However, a user who does not understand the question being asked or cannot
remember the valid responses may respond with a `?' or HELP, when several lines
of information will be displayed on the terminal.
You will then be re-prompted for the value, with the same default as before.

One exception to the rule that $<$CR$>$ means `take the default value' occurs
when you are being prompted for a list of entries, for example a series of
several field centres. In general, hitting $<$CR$>$ in response to the first
prompt will cause an exit from the loop (see for example sections \ref{create}
and \ref{search}). 

\subsection{Search and output parameters}
The available search and output options are represented by a set of parameters
(listed in section~\ref{defpar}), to which a sensible set of default values 
is assigned when the program is started up.
The values of the parameters may be reviewed at any time by the commands DISPLAY
or DISP.
The parameters are of two kinds, search (S) parameters associated with the
catalogue search phase, and output (O) parameters associated with the output
phases.
The command SSET is used in order to alter any or all of the S parameters and
the command OSET to alter any or all of the O parameters.
Each parameter will be presented in turn with its current value, which you may
change or leave as it is.
A single parameter may be changed by using the new command 1SET.
\subsection{The search phase}
When satisfied with the S parameters, you should issue the command SEARCH to run
the search phase of the program.
The program will prompt for the coordinates of the field centres.
Remember to respond $<$CR$>$ to the prompt `RA:=' when the input of field
centres is complete. 
\subsection{Output options}
The results of the search phase may be presented at the terminal by issuing the
command TYPEO.
Alternatively, a printable file (called OUTPUT.LIS) can be created in your
directory by issuing the command LISTO.
This file can of course be listed on the printer using PRINT OUTPUT.

Results can also be plotted in the form of finding charts on any GKS graphics
device.
The plotting output (O) parameters must first be set up to your satisfaction
(using OSET if necessary).
The command PLOTO will then initiate the plotting.

The results of an ASTROMETRIC mode search may also be used as input to the
Starlink astrometry program ASTROM.
A description of this option appears below under ASTRO in section~\ref{output}.
\subsection{Exiting from CHART}
Exit from the program is made by typing STOP or EXIT.
On exit, two files (CHARTINT.DAT and CHARTPAR.DAT) will have been created in
your directory.
They may be deleted at any time, but if retained might well be useful in a
subsequent CHART run.
Their significance is explained below (sections \ref{env} and \ref{search}).
\section{The Standard Default Parameters}
\label{defpar}
The parameters which are available and their standard default values are as
follows:
\begin{verbatim}
        S LABEL             STARCHART
        S SAREA             2.0
        S EQUINOX           1950.0
        S EPOCH             1950.0
        S MODE              CSI
        S INPUT             CATALOGDIR:ASTROM84.CAT
        S SELECTION         ALL
        S NUMBER            2000
        S MAGNITUDE         26.0
        S CATALOGUES        ALL
        S FIELDS            TERMINAL
        O DEVICE            CANON
        O PAREA             2.0
        O SCALE             67.14
        O SYMBOL            CROSS
        O KEY               YES
        O GRID              NO
        O ERRBOX            NONE
        O RADIUS            2.0
        O COORDS            VERT
        O EXTRA             NONE
        O RATIO             1.0
        O FACTOR            1.0
        O DIRECT            NORMAL
        O CROSS             YES

\end{verbatim}
These parameters are explained below (section~\ref{setpar}), and the on-line 
HELP system is available to define the meaning of each parameter as it occurs
within a program. 

\section{An Advanced User Guide}
\subsection{The CHART environment}
\label{env}
The CHART package is a collection of programs, arranged in a similar way to the
ASPIC package.
These programs may be used as many times as required and in any order.
You must initially set up the `CHART environment'.
This is done by typing the command:
\begin{quote}
{\tt \$ CHART $<$name$>$}
\end{quote}
where $<$name$>$ is optional.
If $<$name$>$ is omitted, it is equivalent to typing  CHART CHARTPAR.

All the parameters needed by the CHART programs are stored in a file in your
current directory.
When you type CHART a connection is made between this file and the CHART
programs, as follows.
If the file name.DAT (or CHARTPAR.DAT) exists in the current directory then that
file is used and {\em all the parameters are exactly as they were when last 
set up.}
If the file does not exist then a copy is made of a standard set and given the
name requested.
Consequently, the next time CHART is executed, the file does exist and those
parameters are used.
This means that you may set up a personal set of parameters once, and use them
for ever.
For example, you may always want charts of a certain size and scale for a
particular telescope, always want to use the CSI catalogue and always want
reversed plots with no central cross.

The point of allowing a user-supplied name is that you may have more than one
standard set of parameters and can choose to use any one of them for a
particular run.

$<$name$>$ may also be a logical name, in which case the file which has been
previously assigned to it is used.

When finished, type the command:
\begin{quote}
{\tt STOP}
\end{quote}
which will return control to the normal command-input mode.
EXIT is a valid alternative to STOP, both leaving everything tidy; other methods
of stopping CHART, including CTRL/Y, will leave unwanted files and logical
names lying around.

Within the CHART environment, each of the CHART programs may be run simply by
typing its name.
Users familiar with ASPIC/DSCL should be aware that the usual DSCL facilities
are available (e.g.\ the standard command line default mechanism will work, most
DCL commands are available, and it is possible to write CHART procedures) but it
is not essential to understand this to use CHART effectively.
The programs will prompt you for all the input that is required.

The programs fall naturally into the following categories:
\subsection{Programs which display parameter values}
* DISPLAY

This simply provides a neatly formatted display on the screen, explaining the
current values of the parameters.
If some of the rarer options are not being used, then DISPLAY does not mention
them.

* DISP

A more succinct display may be generated by simply listing the parameter file
using the command DISP.
\subsection{Programs which set parameters}
\label{setpar}
The parameters are in two groups according to their function.
The groups are identified by the first letter (S or O) in the list given in
section~\ref{defpar}.
Each group has its own program for setting new values.
The two programs are named SSET and OSET, and they set the search (S) parameters
and output (O) parameters respectively.
There is a third program, called 1SET (note the 1, not I), which allows a
single parameter to be changed.

* SSET

SSET sets those parameters which are required by the search process.
They include a specification of the search catalogue, and an indication of the
source of the required field centres.
The parameters are:
\begin{list}{}{\setlength{\leftmargin}{20mm}
\setlength{\labelsep}{5mm}
\setlength{\labelwidth}{5mm}
}
\item[{\bf LABEL}] gives a label to this set of parameters.
\item[{\bf SAREA}] the length (in degrees) of the side of the square area of sky
which is to be searched. 
\item[{\bf EQUINOX}]  the required equinox for the output positions.
\item[{\bf EPOCH}]   the required epoch for the output positions.
This is only relevant in ASTROMETRIC mode, because it is the only one which
gives proper motions.
\item[{\bf MODE}]   selects the main catalogue to be searched.
The choices are CSI (for the CSI catalogue), AST (for ASTROMETRIC mode), PRI
(for a private catalogue) and NSO (for the Nonstellar Object catalogue).
\item[{\bf INPUT}]  specifies the catalogue to be used if PRI has been set as
the mode. The file name of the catalogue is specified. 
\item[{\bf SELECT}]   offers further ways of selecting objects from those found
in the specified area. 
A response of ALL ensures that all stars are stored.
Other possible responses are
N for number (when only a fixed number of stars will be stored),
M for magnitude (when only stars brighter than some magnitude limit will be
stored), or
C for catalogue (when only stars from selected source catalogues within the mode
catalogues will be stored).
If the response is not recognised ALL stars are stored.
SELECT is not available in NONSTELLAR mode.
\item[{\bf NUMBER}]  specifies (if SELECT is set to N) the maximum number of
stars (up to 2000) to be stored. The brightest N will be stored. 
\item[{\bf MAGNITUD}]   is the faint magnitude limit for the stars to be stored
(if SELECT is set to N). Stars in the CSI catalogue with no magnitudes given
will be selected regardless of the magnitude limit. 
\item[{\bf CATALOGU}]   restricts the choice of stars to specified source
catalogues (if SELECT is set to C). This requires different responses in AST and
CSI modes. The on-line information summarises the choices, and examples of the
more complicated possibilities are given in section~\ref{catalogues}. 
\item[{\bf FIELDS}]   where the response should be TERMINAL if the fields are to
be specified at the terminal, or the name of a file which contains a list of
positions. No filetype should be specified as it must be .DAT. If a filename is
given, a suitable file may be created by using the program FCREATE. 
\end{list}

* OSET

This allows you to specify those parameters needed for the output part of the
package.
These parameters include one which identifies whether there are any extra
objects.
The parameters are:

\begin{list}{}{\setlength{\leftmargin}{20mm}
\setlength{\labelsep}{5mm}
\setlength{\labelwidth}{5mm}
}
\item[{\bf DEVICE}]  is the name of the graphics device to be used.
It may be any of the devices supported by GKS and may be specified in several
different ways.
In particular, CHART sets up a list of names for all devices; individual users
or site managers may set up other names for their devices.
Appending an `\_A' to the name will allow the aspect ratio and scale factor to
be changed for some hard-copy devices. 
Further information on device specification will be found in HELP DEVICE; the
maximum size of the actual chart which can be plotted on any device can be found
under HELP SIZES.

\item[{\bf RATIO}]
If the selected device has an `\_A' in it, you can define an aspect ratio
(horizontal:vertical) to correct any distortion introduced by some
hard-copiers.

\item[{\bf FACTOR}]
If the selected device has an `\_A' in it, you can define an additional
scale factor (over and above SCALE) which will help to make the hardcopy of the
plot come out to the scale defined by SCALE.

\item[{\bf PAREA}]
specifies the length (in degrees) of the side of the square which is to
be plotted.
In general, it will be the same as SAREA, but need not be.

\item[{\bf SCALE}]   specifies the plot scale in arcsec/mm.
PAREA (when multiplied by 3600 to convert it to arcseconds) divided by SCALE 
ought to be less than the size given for the selected DEVICE under HELP SIZES.

\item[{\bf SYMBOL}]
specifies what symbol is to be used to represent stars on the plot.
CIRCLE will give circles, CROSS will give exploded crosses, and SPOT will give
filled circles.

\item[{\bf KEY}]
specifies whether magnitude and scale keys will be included on the plot,
and whether stars plotted are to be numbered.
Responses are YES if scales and numbering are required, SCALES if only scale
keys are to be included, NUMBERS if only star numbers are wanted, and NO if
neither scale keys nor numbers are required.

\item[{\bf GRID}]
specifies whether a full RA, Dec coordinate grid, or an abbreviated one
is to be plotted.
Responses are YES, MINI or NO.

\item[{\bf ERRBOX}]
specifies whether an error box is to be plotted, and if so of what type.
The allowed responses are NONE for no error box, CIRC for a circular error box
at the field centre, QUAD for a quadrilateral one, and PROMPT to allow the error
box to be defined in SEARCH for each field.
Further information will be found in the description of SEARCH.

\item[{\bf RADIUS}]
if a circular error box was chosen, this defines its radius in minutes
of arc.

\item[{\bf COORDS}]
if a quadrilateral error box was chosen, this defines the file which
contains the vertices.
No filetype should be specified, as it must be .DAT.
A suitable file may be created by using the program QCREATE.

\item[{\bf EXTRA}]
where the response should be NONE if there are no extra (user supplied)
objects for plotting, otherwise the name of the file in which they are stored.
No filetype should be specified, as it must be .DAT.
If a filename is given, a suitable file can be created by using the program
ECREATE.

\item[{\bf DIRECT}]
allows the RA axis to be plotted in the normal direction (NORMAL) or
reversed (REVERSED) for users who compare charts with plates when the emulsion
is `down', or with the view in instruments with inverting optics.

\item[{\bf CROSS}]
allows the central cross to be plotted (YES) or not (NO).
\end{list}

* 1SET

This allows you to select a single parameter by name and then enter a new value
for it.
However, be careful because this program does not validate your input, unlike
SSET and OSET.
The parameters are:
\begin{list}{}{\setlength{\leftmargin}{20mm}
\setlength{\labelsep}{5mm}
\setlength{\labelwidth}{5mm}
}
\item[{\bf PARAMETER}]
the name of the parameter which is to be altered, as listed by
the command DISP.
A response of `?' or HELP will list the names of all the parameters.

\item[{\bf VALUE}]
the new value for the selected parameter, the default being the current value.
A response of `?' or HELP will give the same information as is available within
SSET and OSET.
\end{list}
It may also be used by typing the command:
\begin{quote}
{\tt 1SET parameter value}
\end{quote}
where `parameter' and `value' are appropriate strings.
If the value has spaces in it must be enclosed in quotes.
\subsection{Programs which create files of coordinates}
\label{create}
There are three files which may be required, each containing lists of RAs and
Decs.
There is a separate program for each of them.

* FCREATE

This creates a file of field centres for use in the SEARCH program.
You must supply a file name; if a file of the given name already exists, you are
asked whether entries are to be added to the file or whether a new file is to be
created.

The program loops around prompting for:-
\begin{list}{}{\setlength{\leftmargin}{20mm}
\setlength{\labelsep}{5mm}
\setlength{\labelwidth}{5mm}
}
\item[{\bf RA}] A null response breaks out of the loop.
\item[{\bf DEC}] \
\item[{\bf EQUINOX}] A response of TODAY will pick up today's date.
\item[{\bf TITLE}] This is used to label output for this field centre.
\end{list}

After each set of values has been entered, you are asked if they are `OK?'.
Responses are $<$CR$>$, T, TRUE, Y, YES for yes and most other things for no.
This program does {\em not\/} set the parameter which would ensure that this
file is used; this is done by program SSET described above. 

* ECREATE

This creates a list of supplementary objects.
You must supply a file name and an equinox which must be the same for all the
objects in the file.
If a file of the given name already exists, you are asked whether entries are to
be added to the file or whether a new file is to be created.
You are then prompted for RA and Dec in turn.
A null response to RA terminates input.
This program does {\em not\/} set the parameter which would ensure that this
file is used; this is done by the program OSET as described above. 

* QCREATE

This may be used to set up the coordinates of the four vertices of a
quadrilateral error box, together with an equinox, and store them in a named
file.
This program does {\em not\/} set the parameter which would ensure that this
file is used; this is done by program OSET as described above. 
\subsection{Files of co-ordinates}
The three programs described in the previous section all create files.
You may find it useful to understand the formats used, so you can write programs
to create such files from different sources.
All the files are variable record length, sequential access files, each record
of which corresponds to one piece of information written exactly as entered, in
ASCII characters.
There is no file terminator, except for end-of-file.

FCREATE, ECREATE and QCREATE create files whose records are in order:

\begin{center}
\begin{tabular}{|l|l|l|}
\hline
FCREATE   &  ECREATE  &  QCREATE \\ \hline
RA        &  EQUINOX  &  EQUINOX \\
DEC       &  RA       &  RA \\      
EQUINOX   &  DEC      &  DEC \\
TITLE     &  RA       &  RA \\
RA        &  ....etc. &  ....etc. \\
DEC       &           &  (4 pairs in all) \\
....etc.  &           & \\
& & \\ \hline
\end{tabular}
\end{center}

The formats allowed for RA and DEC are exactly the same as those allowed for
keyboard entry of the same parameters, i.e.\ decimal hours, or hours and decimal
minutes or hours, minutes and decimal seconds, and so on, as described in the
on-line help.
\subsection{Programs which search catalogues}
\label{search}
* SEARCH

This program reads the current S parameters, searches the catalogue selected and
stores all the results in an intermediate file, called CHARTINT.DAT.
If the parameter FIELD is set to a file name then no user input is required.
Otherwise you are prompted in turn for RA, DEC, EQUINOX and TITLE.
A null response to the RA prompt ends the program.

If the parameter ERRBOX has been set to PROMPT, then some or all of the
following are prompted for after TITLE: TYPE, XDIAM, YDIAM and ORIENT.
TYPE can take the values:
\begin{list}{}{\setlength{\leftmargin}{20mm}
\setlength{\labelsep}{5mm}
\setlength{\labelwidth}{5mm}
}
\item[{\bf C}] for a circular error box
\item[{\bf E}] for an elliptical error box
\item[{\bf Q}] for a rectangular error box
\end{list}
Only those parameters needed for the specified error box type are then prompted
for:
\begin{list}{}{\setlength{\leftmargin}{20mm}
\setlength{\labelsep}{5mm}
\setlength{\labelwidth}{5mm}
}
\item[{\bf XDIAM}]
defines the length of the longer diameter/side in seconds of arc.
\item[{\bf YDIAM}]
defines the length of the shorter diameter/side in seconds of arc.
\item[{\bf ORIENT}]
defines the orientation of the longer diameter/side measured from the
north through east in degrees.
\end{list}
The position parameters here, and everywhere else, may be specified in the same
very flexible way that was used in the earlier CHART program.
It is described in the prompt information on RA and DEC.

This program must have been run before trying to print or plot the output, but
once it has been run the output may be plotted as many times as required, 
e.g.\ changing the output device (or any other O parameter).
However, it must be rerun after a change has been made to any S parameter.

On exit from CHART, the CHARTINT.DAT files will be PURGEd, so the most recently
created version can be preserved and used in a subsequent session if required.

* BSEARCH

This is a variation on the normal SEARCH command, which looks through the
selected catalogue for stars in a band around the sky (from 0 Hr to 24 Hr) at
constant declination, rather than in a square of fixed size centered on the
input field centre.
It uses all the parameters which can be defined by SSET, but interprets the
parameter SAREA as the width (in degrees) of the band to be searched.
Although it is not actually used, it is still essential to enter an RA for the
field centre, to retain compatibility with SEARCH.

Remember that BSEARCH can only store information on 2000 stars.
On reaching this limit it will stop looking, report the fact and end properly.
However, only a part of the selected area will have been searched.
This can be a serious limitation when searching a complete band round the sky.
Consequently, you should think carefully about the width of band to be searched;
the actual value will, of course, depend on the declination of the band.

All the catalogues have positions stored in 1950.0 co-ordinates.
Consequently, the band searched is as defined in those co-ordinates.
The positions found are then precessed to the requested equinox and, if in
ASTROMETRIC mode, proper motions are applied up to the requested epoch.
Consequently, the band of stars found will not strictly be a band at constant
declination at the output equinox.
It would be extremely difficult to get round this problem.

No changes have been made to the output programs TYPEO, LISTO and PLOTO.
PLOTO will therefore not plot all the stars found, as this would be meaningless
in the `tangent plane projection' which it uses.
TYPEO works perfectly, but LISTO gives some `output conversion errors', which
are not fatal, but simply mean that the star in question is too many seconds of
arc away from the field centre to be printed in the given format.
\subsection{Selection strategy}
The SEARCH program allows various methods of selecting stars, but not all of
them are ideal for all jobs.
SEARCH only has room for 2000 stars and it searches in bands of DEC from north
to south.
Consequently, if a large or very densely populated region is being searched and
this limit is reached, only the northern part of the search area will contain
stars.

When NUMBER selection is chosen, all the stars in the region are found and
sorted into order of increasing magnitude.
The list is then truncated, leaving only the N brightest stars.
However, if the 2000 star limit is reached before the whole area has been
searched, only the N brightest stars in the area which had been searched when
the limit was reached will actually be stored.
Clearly, it is not possible to select the N brightest stars in a region until
all the stars in the region have actually been found.

On the other hand, when MAGNITUDE selection is chosen, each star may be
accepted or rejected as it is found.
Therefore, it is much safer to use magnitude limits when there are likely to be
more than 2000 stars in the selected region, provided the magnitude cutoff is
chosen so as to restrict the total number of accepted stars to less than 2000.

You should also be aware of a minor inconsistency when using catalogue selection
in astrometric mode.
As stated elsewhere, the AGK3 catalogue reaches -3 degrees.
However, for southern declinations the same stars will appear in the SAO
catalogue.
Consequently, when both AGK3 and SAO are selected, only AGK3 stars with zero and
positive declinations and SAO stars with negative declinations are used.
However, if only AGK3 stars are requested, they will be found to the declination
limit of the catalogue.
Of course, this is only relevant in the narrow band just south of the equator.
\subsection{Programs which produce output}
\label{output}
Currently, listing, plotting and astrometry are available.
OSET can be used to change the output parameters to produce different output
from the current intermediate file, but there is no need to re-SEARCH unless the
search parameters are changed.

* LISTO

This program reads the intermediate file (and uses SCALE from the current
parameter file) and generates printable output.
No user input is required.

To create a printed copy of the output use the command:
\begin{quote}
{\tt PRINT OUTPUT}
\end{quote}
In all modes, the basic run parameters are first printed, including the field
centre equatorial coordinates both for equinox 1950 and for the required output
equinox.
The field centre is also given in ecliptic and equatorial coordinates.
The approximate coordinates of the field centre on the major sky surveys
(Palomar, UK Schmidt \& ESO) are given in millimetres relative to the south
eastern corner of the relevant chart.

The main listing of the catalogue data is largely self explanatory.
In all modes, offsets are given in arcseconds relative to the field centre.
In ASTROMETRIC mode, these offsets are additionally given in millimetres using
the O parameter SCALE.
In ASTROMETRIC mode, proper motions are listed.
The sense of the x coordinates is different for arcseconds and millimetres:
arc measurements are positive in the same sense as RA increases, as described in
the on-line help.
See section~\ref{catalogues} for more details of the listed catalogue data.

* TYPEO

Like LISTO, this program also reads the intermediate file, but the output
generated is returned to your terminal, one screenful at a time.
The output is similar to LISTO, but is shortened in order to fit the screen.

* PLOTO $<$anything$>$

This program reads the intermediate file and the parameter file and either
generates plotted output on the specified device, or, in the case of the
hard copy devices, generates an output file for subsequent plotting.
If the optional parameter is absent, the output file is automatically submitted
to the appropriate queue and that file deleted.
If the parameter is present (its value is immaterial), the files are left
untouched and nothing is submitted to a print queue.
In this case, it is up to you to handle the files and job submission yourself.
This might include transferring the files over the network.
No graphical output files are deleted or purged.
When the specified device is a graphics terminal, a $<$CR$>$ is
required at the end of each plot to clear the screen; this input is not prompted
for. 

If the output device can plot in colour, the program asks you if you actually
want to use colour. This option is intended for use with pen plotters as it 
may be inappropriate to have a coloured plot, for example, if the plot is to be
reproduced.

A tangent plane projection is plotted; this is sufficiently accurate to be used
as an overlay on most telescope plates or prints.
In stellar modes, each star is plotted either as an exploded cross with a gap at
the exact star position, as a filled circle to simulate a direct view of the
star field, or as a circle centred on the star position.
If requested, a number is drawn beneath each symbol to correspond with the
listings from TYPEO or LISTO.
The symbol sizes are related to the stellar magnitude.
Visual magnitudes are plotted, except for AGK3 stars in ASTROMETRIC mode which
are photographic.
A standard range of symbols representing magnitudes 4--11 is optionally shown
beside the chart as a key, except when a magnitude limit is present, in which
case the magnitude scale only goes to the next faintest magnitude to that limit;
a square demonstrating the linear scale is also drawn.

Nonstellar objects are plotted as a fixed size lozenge.

Supplementary user-defined objects are plotted as squares with diagonal tick
marks.
Their reference numbers are plotted with negative signs to distinguish them from
catalogue objects.

There several other options available, including an RA,DEC grid, an error box
and a central cross.
It is also possible to plot the chart directly or reversed.

If the requested area is larger than can be fitted on the graphics device, only
the central portion is plotted.

* ASTRO

This program creates a file which may be used as input to the standard Starlink
astrometry program ASTROM (see SUN/5).
It does not use the O parameters in the parameter file, but does use other
parameters, which are prompted for, and help is available in response to `?' or
HELP.
For the results to be meaningful, accurate positions and proper motions are
needed; these are only available in ASTROMETRIC mode so you need to have used
MODE=AST when carrying out your SEARCH.

The parameters used are:
\begin{list}{}{\setlength{\leftmargin}{20mm}
\setlength{\labelsep}{5mm}
\setlength{\labelwidth}{5mm}
}
\item[{\bf EQRES}]   the equinox to be used for the output positions.
\item[{\bf TELTYPE}] the type of telescope used.
The responses are AAT2, AAT3, ASTROGRAPH and SCHMIDT.
\item[{\bf RACEN/DECEN}] the position (RA, Dec) of the centre of the plate.
\item[{\bf EQCENTRE}] the equinox of that position.
\item[{\bf EPPLATE}] the epoch when the plate was taken.
\item[{\bf NUMBER}] the sequence number (in the list from LISTO) of one of 
the reference stars.
After receiving a valid sequence number, ASTRO prompts for an X and Y for that
star.
Exit by hitting $<$CR$>$.
\item[{\bf XorR}] indicates whether the next series of entries will be X,Y
positions for conversion to RA, Dec (X) or vice versa (R). Exit from ASTRO by
hitting $<$CR$>$. 
If X is selected, ASTRO loops round prompting for an X coordinate, a Y
coordinate and a name.
If R is selected, ASTRO loops round prompting for an RA, a Dec, an equinox and
a name.
Exit from either loop by hitting $<$CR$>$ in response to X or RA, whichever the
case may be.
On exit you are returned to the XorR prompt.
\end{list}
A file ASTROM.DAT is created, suitable for input to ASTROM.
You may edit this file at this point if required.
When you are happy with the file, type the command ASTROM to run the standard
astrometry program (you do not have to exit from CHART).
Comprehensive results are written to the file ASTROM.LIS and a summary appears
on the terminal, in the way described in SUN/5.

* RGOGS

This command, pronounced RGO Guide Stars, is an alternative to the usual LISTO
command which also computes the positions of all the stars as seen in the
astrometric telescopes at RGO and La Palma.
In particular, positions in the 26-inch Merz guider, the 13-inch guider and the
1-metre are listed.
It writes its output to a file RGOGS.LIS, which may be printed in the usual way.
\section{The Catalogues}
\label{catalogues}
\subsection{The CSI catalogue}
The Catalogue of Stellar Identifications (CSI) has been prepared by the Stellar
Data Centre, Strasbourg.
It contains 434,927 stars extracted from several catalogues merged together (see
references 1--4).
It is believed to be complete to visual magnitude 9.5.
Approximate positions and magnitudes homogenised on the UBV system are available
for each star and several star identifications (HD, AGK, Durchmusterung etc.)
are available explicitly.
The catalogue also indicates whether a star is found in a quite wide range of
other catalogues.
Details of these catalogues are given in the table below.

In the listing from LISTO, the two primary star identifications HD and
Durchmusterung are given explicitly.
Some stars, not in the Durchmusterungen, are given the DM name of `**'.
In these cases, the zone and number are derived from the star's 1950 equatorial
coordinates.
B and V magnitudes are given; a bracketed magnitude indicates an approximate
magnitude.

About 6000 stars in the catalogue have no magnitudes given for B and V. The
search and output programs now pick these stars up and display/plot them.

\begin{tabular}{rl}
\multicolumn{2}{c}{\bf Catalogues indexed in the catalogue of stellar 
identifications} \\
& \\
Abbrev & \multicolumn{1}{c}{Catalogue} \\
&\\
  HD  &  Henry Draper Catalogue (including extensions)\\
  AG  &  AGK2/3\\
  HZ  &  Herz Star Catalogue for Equinox 1890
         (Zone -6 $<$ -10Dec) (Berlin,1906)\\
 CPC  &  Cape Photographic Catalogue (Cape Annals vol. 17--21)\\
 YZ   &  Yale Zones Catalogue\\
CCFS  &  Cape Catalogue of Faint Stars (Zone -40 $<$ -52 Dec)\\
BOSS  &  B Boss Albany General Catalogue\\
 SAO  &  SAO Catalogue\\
 ADS  &  Aitken Double Star Catalogue\\
 IDS  &  Lick Index Catalogue of Visual Double Stars\\
GCRV  &  Wilson General Catalogue of Radial Velocities\\
 YBS  &  Yale Catalogue of Bright Stars\\
 N30  &  Catalogue of 5268 stars based on the normal N30 system\\
 FK4  &  FK4 and FK4 supplement\\
 JSK  &  Jascheck: La Plata Catalogue of Stellar Spectra in
         Morgan Keenan System\\
 A+B  &  Abt \& Biggs Bibliography of Stellar Radial Velocities\\
 KDY  &  Kennedy MK Classification Southern Extension\\
UVBY  &  (uvby) Photometry exists in Hauck \& Mermilliod Catalogue\\
 BAY  &  Bayer or Flamsteed number\\
USNP  &  Blanco Photoelectric Catalogue\\
GCTP  &  Yale General Catalogue of Trigonometric Parallaxes\\
GCVS  &  General Catalogue of Variable Stars (3rd. Edition)\\
 UBV  &  Mermilliod Catalogue of UBV Photoelectric Photometry\\
 NGC  &  Identification in NGC open Cluster\\
 IC   &  Identification in IC open Cluster\\
 CLA  &  Identification in anonymous Cluster
         (In lists of less than 100 Clusters)\\
 CLB  &  Identification in anonymous Cluster
         (In lists of 100 or more Clusters)\\
 LS   &  Luminous stars in Milky Way
         (Hamburg-Bergedorf/Warner Swasey)\\
 IRC  &  Two micron Sky Survey (Neugebauer \& Leighton)\\
 CEL  &  Celescope Catalogue of Ultraviolet Magnitudes\\
 GEN  &  Rufener Catalogue of Stars measured in the Geneva System\\
 U+F  &  Uesugi \& Fukuda Catalogue of Rotational Velocities\\
 SB   &  Batten Spectroscopic Binary Catalogue\\
 BE*  &  Wackerling catalogue of early type stars whose spectra
         have shown emission lines\\
  GL  &  Gliese catalogue of nearby stars\\
JP11  &  Morel UBVRIJKLMNH Photoelectric photometric catalogue\\
HGAM  &  Petrie et al A catalogue of H Gamma measures\\
 MSS  &  Houk \& Cowley. Michigen Catalogue of 2 dimensional
         spectral types for the HD stars\\
  CF  &  Spencer \& Jackson Catalogue of 20554 faint stars in the
         astrographic zone -40 to -52 Dec\\
  UM  & \\
  FE  & \\
  RY  & \\
AMAS  & \\
 DM2  &  Durchmusterung Cross Identification Exists
         (CoD/CPD or BD/CoD)\\
 YZO  &  Overlap in Yale Zones\\
MMAG  &  Magnitudes refer to a multiple system\\
& \\
& \\
\end{tabular}

Details of the cluster identifications (NGC, IC, CLA, CLB) are in reference 5
(sections dealing with code numbers 2 to 5).
This information is also available in the file CATALOGDIR:STARNAMES.LIS.

The listing shows in which catalogues the star may be found, as indicated in the
CSI.
The catalogues are denoted by the abbreviations listed in the table.

If catalogue star selection is chosen when using the CSI (SELECT set to C),
you can select stars that are present in or absent from particular catalogues.
This is done by giving a list of one or more catalogue names for the parameter
CATALOGU.
The names should be supplied in the abbreviated form as given in the table.
By prefacing the list by `NOT', the criterion for selection is reversed,
i.e.\ only those stars which are not in the catalogue(s) specified will be
selected.

Examples of the kind of information which could be obtained are as follows:

\begin{tabbing}
\hspace{10mm}\=Which stars have been photo-electrically \hspace{10mm}\=UBV\\
\>measured in the UBV system?\\ \\

\>Which stars have both MK spectral types \>JSK GCRV \\
\>and radial velocities? \\ \\

\>Which stars are not in the SAO or AGK2/3 \>NOT SAO AGK \\
\>catalogues?
\end{tabbing}

\subsection{The ASTROMETRIC catalogue}
The ASTROMETRIC catalogue contains stars with positions of greater accuracy than
the CSI.
Stars are taken from three catalogues:
the whole of AGK3 (covering the northern hemisphere and extending down to Dec
-3), all SAO stars south of the celestial equator, and the whole of the PERTH70
catalogue.
PERTH70 is a southern hemisphere catalogue (with a few northern FK4 stars) that
is sparser but more accurate than the SAO.
The catalogue includes the epoch of observation and proper motions for each star
in addition to its magnitude and spectral type.

In the listing from LISTO, the star names in the specified catalogue(s) are
given.
Magnitudes printed are visual, except for AGK3 where they are photographic.
\subsection{The NONSTELLAR catalogue}
The nonstellar object catalogue is the Dixon Master List of Nonstellar Objects
(ref 6).
This was formed by a simple merging of many catalogues and lists of
`nonstellar objects'; the total number is about 185,000.
Some catalogues used are largely stellar, e.g.\ Luyten's Faint Blue Star
surveys. The precision of the positions in the list is variable, being
transcribed directly from the source catalogue. 
Approximate angular size and magnitude are given for many objects.
No cross identification was attempted, therefore many objects will have more
than one entry in the list.
A full bibliography of the sources used in compiling the list is available in
the file CHARTDIR:NSOREF.TXT.
This file is arranged alphabetically by catalogue name abbreviation, thus
enabling the object names used in the list to be identified.

In the listing from LISTO, the name of the object is printed as given in the
master list.
The approximate position is always given, and where available the approximate
magnitude and angular size.
A brief description of the object (24 characters or less) is also printed.
\subsection{References}
\begin{enumerate}
\item Ochsenbein, F, Bischoff, M (1975), CDS Inform. Bull. 8, 2.
\item Ochsenbein, F, Egret, D, Bischoff M (1977), Proc. IAU Coll. 35,
    `Compilation, Critical Evaluation \& Distribution of Stellar Data'.
\item Ochsenbein, F (1974), Astron. Astrophys. Suppl. 15, 215.
\item Ochsenbein, F, Bischoff, M, Egret, D (1981), Astron. Astrophys.
    Suppl 43, 259.
\item Mermilliod, J-C (1978), CDS Inform. Bull. 14, 32.
\item Dixon, R S, Sonneborn, G (1980),  A Master List of Non-stellar
    Optical Astronomical Objects (Ohio University Press).
\end{enumerate}
\section{Acknowledgments}
The current version of CHART originates from two programs written at the RGO in
1967, A06E by Bill Nicholson and R19D by Roger Wood.
These were amalgamated by Philip Taylor to run as one program, first on the
1903T and later on the VAX for Starlink.
Leslie Bell documented and made various improvements to the early Starlink
version.
At the request of the Starlink `Database Special Interest Group', Ken Hartley
and Keith Tritton drew up a specification for a major change to the program,
resulting in the present modular structure and the use of SGS/GKS graphics.
This specification was mainly implemented by Ken Hartley (the `CHART
environment') and Joe Vickerstaff (graphical output).

Tim Wilkins (Manchester) did most of the work to convert CHART to use GKS~7.2. 
\end{document}
