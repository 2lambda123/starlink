\documentstyle[11pt]{article} 
\pagestyle{myheadings}

%------------------------------------------------------------------------------
\newcommand{\stardoccategory}  {Starlink User Note}
\newcommand{\stardocinitials}  {SUN}
\newcommand{\stardocnumber}    {105.5}
\newcommand{\stardocauthors}   {R.F. Warren-Smith}
\newcommand{\stardocdate}      {10th December 1991}
\newcommand{\stardoctitle}     {STARLSE \\ [1ex]
                                Starlink Extensions to the VAX \\
                                Language Sensitive Editor}
\newcommand{\stardocversion}   {Version 1.8}
\newcommand{\stardocmanual}    {Programmer's Manual}
%------------------------------------------------------------------------------

\newcommand{\stardocname}{\stardocinitials /\stardocnumber}
\markright{\stardocname}
\setlength{\textwidth}{160mm}
\setlength{\textheight}{230mm}
\setlength{\topmargin}{-2mm}
\setlength{\oddsidemargin}{0mm}
\setlength{\evensidemargin}{0mm}
\setlength{\parindent}{0mm}
\setlength{\parskip}{\medskipamount}
\setlength{\unitlength}{1mm}

\renewcommand{\_}{{\tt\char'137}}
\renewcommand{\thepage}{\roman{page}}

\begin{document}

\thispagestyle{empty}
SCIENCE \& ENGINEERING RESEARCH COUNCIL \hfill \stardocname\\
RUTHERFORD APPLETON LABORATORY\\
{\large\bf Starlink Project\\}
{\large\bf \stardoccategory\ \stardocnumber}
\begin{flushright}
\stardocauthors\\
\stardocdate
\end{flushright}
\vspace{-4mm}
\rule{\textwidth}{0.5mm}
\vspace{5mm}
\begin{center}
{\Huge\bf  \stardoctitle \\ [2.5ex]}
{\LARGE\bf \stardocversion \\ [4ex]}
{\Huge\bf  \stardocmanual}
\end{center}
\vspace{20mm}

%------------------------------------------------------------------------------
%  Package Description
\begin{center}
{\Large\bf Description}
\end{center}

\mbox{STARLSE} is a ``Starlink Sensitive'' editor based on the \mbox{VAX}
Language Sensitive Editor \mbox{LSE}. It exploits the extensibility of
\mbox{LSE} to provide additional features which assist in the writing of
portable Fortran~77 software with a standard Starlink style. 

\mbox{STARLSE} is intended mainly for use by those writing \mbox{ADAM}
applications and subroutine libraries for distribution as part of the Starlink
Software Collection, although it may also be suitable for other software
projects. It is designed to integrate with the SST (Simple Software Tools)
package described in SUN/110. 

%------------------------------------------------------------------------------
%  Add this part if you want a table of contents
\newpage
\markright{\stardocname}
\null\vspace {5mm}
\begin {center}
\rule{80mm}{0.5mm} \\ [1ex]
{\Large\bf \stardoctitle \\ [2.5ex]
           \stardocversion} \\ [2ex]
\rule{80mm}{0.5mm}
\end{center}

\setlength{\parskip}{0mm}
\tableofcontents
\setlength{\parskip}{\medskipamount}

%------------------------------------------------------------------------------
%  Introduction page
\newpage
\markright{\stardocname}
\renewcommand{\thepage}{\arabic{page}}
\setcounter{page}{1}

\null\vspace {5mm}
\begin {center}
\rule{80mm}{0.5mm} \\ [1ex]
{\Large\bf \stardoctitle \\ [2.5ex]
           \stardocversion} \\ [2ex]
\rule{80mm}{0.5mm}
\end{center}
\vspace{30mm}

\section{Introduction}

\mbox{STARLSE} is a ``Starlink Sensitive'' editor based on the \mbox{VAX}
Language Sensitive Editor \mbox{LSE}.\footnote{N.B.\ \mbox{LSE} is part of
the \mbox{VAXset} (\mbox{VAX} Software Engineering Tools) package and
\mbox{STARLSE} can only be used if you are logged in to a machine on which
the \mbox{VAXset} software is installed. You should consult your site
manager if you are unsure which machine to use.} 
 
It exploits the extensibility of \mbox{LSE} to provide additional features which
assist in the writing of portable Fortran~77 software with a standard Starlink
style. 
\mbox{STARLSE} is intended mainly for use by those writing \mbox{ADAM}
applications and subroutine libraries for distribution as part of the Starlink
Software Collection, although it may also be suitable for other software
projects. 

The \mbox{VAX} Language Sensitive Editor is fully described in the
\mbox{VAX} \mbox{LSE/SCA} User and Reference Manuals, which should be
consulted if you have not used \mbox{LSE} before, and in case of difficulty.
This note aims only to describe the additional features provided by
\mbox{STARLSE}. 
In fact, many powerful features of \mbox{LSE}, such as its interaction with
other VAXset tools like \mbox{CMS} and \mbox{SCA}, are not mentioned here at
all, although they are, of course, all available within \mbox{STARLSE}. 
The relevant \mbox{VAX} documentation should be consulted if these other
facilities are required.

Note that STARLSE is also designed to integrate with the SST (Simple
Software Tools) package described in SUN/110. 


\section{Getting Started}

\subsection{Commands}
\label{sect:starting}
Before starting to use \mbox{STARLSE}, you must issue the DCL command:

\begin{verbatim}
   $ STARLSE
\end{verbatim}

which can conveniently go in your \mbox{LOGIN.COM} file.

After this command is complete \mbox{STARLSE} can be invoked simply by using the
\mbox{LSE} command in the same way as with a normal text editor. Thus, to edit
the file \mbox{TEMP.FOR}, you would use the command:

\begin{verbatim}
   $ LSE TEMP.FOR
\end{verbatim}

(But see \S\ref{sect:attaching} for a faster method of access if you are using
STARLSE repeatedly.)


\subsection{Exploring STARLSE}

Perhaps the best way to start using \mbox{STARLSE} is by exploring it
before you try to write anything serious. 
This will probably involve moving between the {\em placeholders} in your
file (these are the things which appear surrounded by [...] or \{...\}) and
{\em expanding} and {\em un-expanding} them to see what they represent. 
If you haven't used \mbox{LSE} before, then you should have the \mbox{VAX}
\mbox{LSE/SCA} User Manual on hand at this stage, as it provides a good
introduction to placeholders and other important \mbox{LSE} concepts. 
You should probably start by trying to master the following keys, which
should give you a good initial feel for what \mbox{STARLSE} provides: 

\begin{quote}
\begin{center}
\begin{tabular}{ll}
ctrl-E & (\underline{E}xpand)\\
ctrl-N & (go to \underline{N}ext placeholder)\\
ctrl-P & (go to \underline{P}revious placeholder)\\
ctrl-K & (\underline{K}ill placeholder)
\end{tabular}
\end{center}
\end{quote}

If you make a mistake, or expand a placeholder into something you don't 
want, then you can go back one step (only) by means of the following keys:

\begin{quote}
\begin{center}
\begin{tabular}{ll}
GOLD ctrl-E & (un-\underline{E}xpand placeholder)\\
GOLD ctrl-K & (un-\underline{K}ill placeholder)
\end{tabular}
\end{center}
\end{quote}


\subsection{A Few Tips}

If you normally use a standard screen editor (like EDT) then you will
probably find that using an \mbox{LSE}-based editor for the first time will
slow you down considerably. 
All those new keys to remember!
However, once you get used to it, you will start to realise how much time
you used to spend doing simple things like moving the cursor, formatting
prologues, indenting lines of code and going to fetch essential
documentation -- all things which \mbox{STARLSE} can do far more
efficiently. 

An important thing to appreciate once you start to use \mbox{STARLSE} in
earnest is that you shouldn't try too hard to use everything it provides
(like religiously expanding {\em all} the placeholders, tokens, {\em
etc.}\,). 
In fact, one of the strengths of \mbox{LSE}-based editing is that it allows
you to pick and choose -- to use the features that you personally find
time-saving, while still being able to type directly over anything which you
find too fussy. 
In this respect \mbox{STARLSE} can be regarded as a sort of interactive
``Manual of Style'' which provides guidelines on layout and programming
standards when you need reminding of them (generally in response to pressing
the {\em expand} key \mbox{ctrl-E}, or the {\em help} key
\mbox{GOLD}~\mbox{PF2}) but lets you work unhindered once you know what you
are doing. 

Nevertheless, you should be aware that some traditional editing techniques
do not carry over to LSE all that effectively. 
In particular, use of the space-bar and the RETURN and arrow keys is not 
nearly as effective for navigating around an LSE editing buffer as the keys 
described above for skipping between placeholders.
This means that it is generally wise not to delete unused placeholders too
early, but to leave most of them in position as ``stepping stones'' until
editing is almost complete. 
This may seem untidy at first, but soon becomes second nature.


\subsection{Faster Access to the Editor}
\label{sect:attaching}
One problem with LSE is that it takes rather more time to start up than most
other editors, and this can be a serious disadvantage if you want to use it to
make frequent minor changes during program development.

If you are fortunate enough to have access to a windowing terminal, then one
solution is to dedicate a window to STARLSE and to leave it running while you do
other work in other windows. Unfortunately, however, not everyone has access to
such terminals, so STARLSE provides a special {\em attach key} facility. This
allows STARLSE to be run in a subprocess to which you may {\em attach} your
terminal whenever you need to use it, thus giving very rapid access to the
editor once it has initially been started. This is a convenient technique to use
with character-cell terminals and terminal emulators.

To make use of this facility, you must nominate a keyboard key to act as your
STARLSE {\em attach key} (the F20 key is recommended, if available). You should
supply the name of your chosen key as an argument when issuing the \$STARLSE
command (\S\ref{sect:starting}), for example:

\begin{verbatim}
   $ STARLSE F20
\end{verbatim}

If you press this key, a subprocess will be created to run STARLSE and your
terminal will be attached to it. You will then be in the editor ready to start
work. Pressing the same key again while in editing mode will return you to your
original process. These steps may be repeated as often as you like to repeatedly
enter and leave the editor. When you finally want to stop STARLSE, issue the
EXIT (or QUIT) command as normal.

When using STARLSE in this mode you cannot specify the  file to be edited on the
command line, so you must issue one of the LSE commands:

\begin{verbatim}
   GOTO FILE filename
\end{verbatim}

or

\begin{verbatim}
   GOTO FILE/CREATE filename
\end{verbatim}

in order to access or create the required files. It is normally convenient to
keep a number of files open in different buffers when working in this way (see
the next section).

STARLSE will flush any modified buffers by writing new versions of the
associated files whenever you use the {\em attach key} to leave the editor. In
fact, this is a quick and convenient way of ensuring that your work is safely
stored on disk if you are interrupted during an editing session. STARLSE will
also ensure that the LSE default directory setting remains set to that of the
process from which it was started. Thus, if you use the SET DEFAULT command from
DCL to change your default directory setting, then STARLSE will adopt this new
setting when you next attach to it.

\subsection{Moving Between Buffers}
\label{sect:buffernavigation}

The most convenient way of using STARLSE to edit several files is to keep each
file in a separate {\em editing buffer}. STARLSE then provides a method of
moving between these buffers using the SB command (short for Show
Buffer).\footnote{This is essentially the same as the standard LSE command SHOW
BUFFER except that the cursor is automatically positioned on the name of the
previous buffer, which is also highlighted.}

This command may be issued either from the LSE command line or by pressing the
F14 key when in editing mode. It will display a list of all your current
editing buffers, with the cursor positioned on the name of the one you were
last editing. The normal cursor-movement keys may then be used to move the
cursor within this list. The ``Select'' key will select a chosen buffer for
editing, while the ``Remove'' key may be used to delete any buffer which is no
longer required.

\subsection{Personalising STARLSE}

If you use \mbox{STARLSE} regularly, you may like to make logical name
definitions such as the following in your \mbox{LOGIN.COM} file: 

\begin{verbatim}
   $ DEFINE STARLSE$PERSONAL_NAME "Fred J. Bloggs"
   $ DEFINE STARLSE$PERSONAL_USERID "FJB"
   $ DEFINE STARLSE$PERSONAL_AFFILIATION "STARLINK, Univ. of Balham"
\end{verbatim}

These definitions allow \mbox{STARLSE} to insert the values you specify at
certain standard places, in response to pressing the {\em expand} (ctrl-E)
key. 
The first value is a string to be used as your personal name (if this is not
defined then you will be prompted whenever it is needed). 
The second is a short form version of your name ({\em e.g.}\ your initials)
-- this will default to your \mbox{VMS} user ID, so you need only define it
if your user ID does not identify you clearly. 
The third value is the institution to which you are affiliated and, if
necessary, your place of work -- this will default simply to `STARLINK'. 

Note that it is best to avoid adding extra information (such as your E-mail
address) to any of these values, because such information soon gets out of
date and becomes little more than a historical curiosity, if not positively
misleading (remember, many pieces of software have life-times measured in
decades). 
The usage suggested above is intended to allow anyone using or supporting
your software in future to trace you if necessary via the institution at
which you did the work. 

\section{The `STARLINK\_FORTRAN' Language}

Perhaps the most important component of \mbox{STARLSE} is a version of the
Fortran~77 programming language called \mbox{STARLINK\_FORTRAN}, which
defines the style of programming that the editor supports. 
This language is used as the default for files of type \mbox{.FOR} and
\mbox{.GEN}. 

The \mbox{STARLINK\_FORTRAN} language is based on the Starlink Application
Programming Standard document SGP/16, and contains only a small number of
approved extensions to the Fortran~77 standard. 
It is therefore much simpler and easier to use than the \mbox{VAX}~Fortran
language which comes with ``native'' \mbox{LSE} (in fact, nearly 80\% of
\mbox{VAX}~Fortran consists of extensions to the Fortran~77 standard!). 
In \mbox{STARLINK\_FORTRAN}, the number of available options and ambiguous
abbreviations is greatly reduced, and most of the common language constructs
can be produced simply by typing a short token, like `DO' or `IF', followed
by \mbox{ctrl-E}. 

The most important features of the \mbox{STARLINK\_FORTRAN} language are
described below.


\subsection{Templates and Prologues.}

\mbox{STARLINK\_FORTRAN} provides templates for the common types of
Fortran~77 program module used in Starlink applications
(\mbox{ADAM}~\mbox{A-tasks}, \mbox{Subroutine}, \mbox{Function} \&
\mbox{Block}~\mbox{Data} program units and ADAM~Monoliths). 
Each of these templates includes a standard {\em prologue} ({\em i.e.}\ a
set of header comments) with a uniform style, and these prologues may be
processed automatically into different forms of user documentation using
applications from the SST (Simple Software Tools) package (see SUN/110 for
details). 
Some examples of completed prologues are given in
Appendix~\ref{section:prologues}. 

The language definition provides assistance when entering information into
these prologues by means of placeholders, which expand into suitable default
values (on pressing \mbox{ctrl-E}) wherever possible.
If no default value is available, then a message is displayed describing the 
information which you should enter.
Many of the prologue sections are optional and may be omitted if
circumstances dictate. 
In addition, some sections in each prologue constitute a {\em maintenance
log} ({\em e.g.}\ the ``Authors'' and ``History'' sections) and certain
placeholders in these sections are intended to remain in place to assist in
making future additions. 

If you do not have access to \mbox{LSE}, or wish to use a different editor
but nevertheless would like to format prologue information in the same style 
as \mbox{STARLSE}, then a set of blank template files is provided to 
assist.
These files can be found in the directory \mbox{STARLSE\_DIR} and have names
as follows: 

\begin{quote}
\begin{center}
\begin{tabular}{|l|l|}
\hline
{\bf Filename} & {\bf Purpose} \\
\hline \hline
ATASK.PRO & {\em ADAM A-task template} \\
SUB.PRO & {\em Subroutine template} \\
FUNC.PRO & {\em Function template} \\
BLOCK.PRO & {\em Block Data routine template} \\
MON.PRO & {\em ADAM Monolith template} \\
\hline
\end{tabular}
\end{center}
\end{quote}

For completeness, these files contain templates for {\bf all} the prologue
items which \mbox{STARLSE} supports. 
In practice, however, only a fairly small subset of these will normally be
required, so you may find it convenient to produce your own ``cut down''
copies of these templates by removing items which you don't normally use. 


\subsection{Subroutine Definitions}

{\em Package definitions} are provided in \mbox{STARLSE} for most of the
standard Starlink/\mbox{ADAM} subroutine libraries, and these are accessible
from the \mbox{STARLINK\_FORTRAN} language. 
These definitions allow you to type the name of a routine from any of these
libraries (or an abbreviation) and to obtain the routine's full name and
argument list simply by pressing \mbox{ctrl-E}. 

This is particularly helpful if you can't remember the name of the routine 
you want, because an ambiguous abbreviation ({\em e.g.}\ just the package
prefix) will result in a {\em menu} in which all matching routines are shown
along with a simple description of what they do. 
The required routine can then be selected from the menu and will appear in
your editing buffer, together with its argument list, with the cursor
positioned ready to enter the first argument. 
This facility can save a great deal of time which would otherwise be spent
referring to documentation. 
It can also virtually eliminate argument-passing errors, which are a
significant source of software bugs. 

The subroutine libraries currently supported by STARLSE package definitions
are shown in Table~\ref{table:status}. 

\begin{table}
\begin{center}
\begin{tabular}{|c|c|c|c|}
\hline
{\bf Library} & {\bf Description} & {\bf Help?} & {\bf References}\\
\hline \hline
AGI & {\em Applications Graphics Interface} & yes & SUN/48 \\
ARY & {\em Array Data Structure Access} & yes & SUN/11 \\
CHR & {\em Character Handling Routines} & yes & SUN/40 \\
EMS & {\em Error Message Service}    & yes & SSN/4 \\
ERR & {\em Error System Routines}    & yes & SUN/104 \\
FIO & {\em File I/O System}          & no & APN/9 \\
GKS & {\em Graphical Kernel System}  & yes & SUN/83 \\
GWM & {\em X Graphics Window Manager}& yes & SUN/130 \\
HDS & {\em Hierarchical Data System} & yes & SUN/92, APN/7 \\
IDI & {\em Image Display Interface}  & yes & SUN/65 \\
MAG & {\em Magnetic Tape Handling} & no & APN/1 \\
MSG & {\em Message System Routines}  & yes & SUN/104 \\
NDF & {\em NDF Data Structure Access} & yes & SUN/33 \\
PAR & {\em ADAM Parameter System}    & no & APN/6 \\
PGPLOT & {\em Graphics Subroutine Library} & yes & SUN/15 \\
PRIMDAT & {\em Primitive Data Processing} & no & SUN/39 \\
PSX & {\em POSIX Interface Routines} & yes & SUN/121 \\
SGS & {\em Simple Graphics System}   & no & SUN/85 \\
SLALIB & {\em Mainly Positional Astronomy} & yes & SUN/67 \\
TRN & {\em Coordinate Transformation} & yes & SUN/61 \\
\hline
\end{tabular}
\caption{The subroutine libraries supported by STARLSE.}
\label{table:status}
\end{center}
\end{table}

\subsection{On-line Help Information}

On-line {\em help} information is also available for certain subroutine
libraries.
To use this facility the cursor should be positioned on the item for which help
is required ({\em i.e.}\ the subroutine name) and the {\em help} key
\mbox{GOLD}~\mbox{PF2} pressed. 
If help is available on the indicated item, it will then be displayed.

Table~\ref{table:status} indicates which subroutine libraries currently have
on-line help available. 


\subsection{Alias Definitions}

A number of \mbox{LSE} {\em alias} definitions are provided in the
\mbox{STARLINK\_FORTRAN} language for expressions which need to be entered
fairly often. 
An alias (or an abbreviation) can simply be typed in, followed by
\mbox{ctrl-E}, in order to enter the equivalent text into the file you are
editing. 
The following are given as examples:

\begin{quote}
\begin{center}
\begin{tabular}{l|l}
{\bf Alias} & {\bf Equivalent Text}\\
\hline
ST & STATUS\\
OK & STATUS .EQ. SAI\_\_OK\\
BAD & STATUS .NE. SAI\_\_OK\\
CLOC & CHARACTER $*$ ( DAT\_\_SZLOC )
\end{tabular}
\end{center}
\end{quote}

A full list of the aliases defined for the current language can be obtained
by using the \mbox{LSE} command: 

\begin{verbatim}
   LSE> SHOW ALIAS
\end{verbatim}


\subsection{ADAM Programming Constructs}
\label{sect:adamconstructs}

A small number of tokens are defined to represent programming constructs 
which are commonly used in ADAM applications and subroutines.
As a simple example, the \mbox{`\_OK\_BLOCK'} token will expand into the
following block: 

\begin{verbatim}
      IF ( STATUS .EQ. SAI__OK ) THEN
         {executable_statement}...
      END IF
\end{verbatim}

which only executes if the global \mbox{STATUS} value is not set. 
As further examples, a token \mbox{`\_CHECK\_STATUS'} exists to facilitate
checking of the \mbox{STATUS} value in \mbox{ADAM} programs, and expands as
follows: 

\begin{verbatim}
      IF ( STATUS .NE. SAI__OK ) GO TO {abort_stmt}
\end{verbatim}

while the \mbox{`\_ERROR\_REPORT'} token provides a template for making a
standard error report using the Starlink \mbox{ERR\_} and \mbox{MSG\_}
routines. 
Its expansion is:

\begin{verbatim}
      STATUS = {error_code}
      [define_message_token]...
      CALL ERR_REP( '{routine_name}_{error_name}',
     :              '{message_text}',
     :              STATUS )
\end{verbatim}

These special \mbox{ADAM} tokens are distinguished by their first character,
which is always an underscore `\_'. 
Thus, a complete list of available tokens which represent \mbox{ADAM}
programming constructs may be obtained by using the \mbox{LSE} command: 

\begin{verbatim}
   LSE> SHOW TOKEN _*
\end{verbatim}

As always, these tokens may be abbreviated (two or three characters normally
suffice) and you can ``un-expand'' them again (using
\mbox{GOLD}~\mbox{ctrl-E}) if you are not happy with the result. 


\subsection{Symbolic Constants, Error Codes and Include Files}

Information is also available in the \mbox{STARLINK\_FORTRAN} language about
the symbolic constants, error codes and include files associated with
certain subroutine libraries. 
In most cases this has been provided by defining the symbolic names of these
components to be {\em routines} (but without arguments) in the appropriate
LSE package definition. 
You can therefore obtain a menu of symbolic constants, {\em etc.}\
associated with a subroutine library in the same way as you would obtain a
menu of the routines themselves; {\em i.e.}\ by typing a suitable
abbreviation followed by \mbox{ctrl-E}. 

This facility is currently available for the GKS, HDS and TRN libraries
only. 


\subsection{Tokens and Menus}

Because many of the items in \mbox{STARLSE} programming templates are 
optional, it often happens that a specific item has been omitted from an 
existing piece of software, but is later required when the software is 
upgraded.

A simple example would occur if a \mbox{DATA} statement were added to an 
existing subroutine.
In this case a new ``Local Data'' section in the prologue might be wanted,
but it could be difficult to remember how such a section should be formatted
(or even what it is called) because the optional placeholder which once
represented it is no longer present. 
Another example would occur if you wanted to add a \mbox{STARLSE} prologue
to an imported routine which did not initially have one.
In this case, you would not be able to remember all the layout details so
it would be essential to insert a template for the prologue into the routine
before starting work. 

\mbox{LSE} provides s solution to this problem in the form of {\em tokens}
(of which the \mbox{ADAM} programming constructs described in
\S\ref{sect:adamconstructs} are a simple example). 
A token (or an abbreviation) can be entered into an editing buffer at any 
point and then expanded (by pressing ctrl-E).
Tokens are provided for the majority of items which exist in the
\mbox{STARLINK\_FORTRAN} language (from single statements, through
individual prologue sections, to complete program units) so there will
almost always be a token representing the particular part of a programming
template you want to use. 
Unfortunately, the sheer number of tokens available can be a problem; you
will soon learn the ones you use frequently, but it can be difficult to
remember (or guess) those which are less frequently needed. 

\mbox{STARLSE} therefore provides a set of hierarchical menus to assist in 
finding the token (or placeholder) you want.
The top level in this menu system is represented by the token:

\begin{verbatim}
   MENU
\end{verbatim}

On entering this token into an editing buffer, and pressing ctrl-E, you will
obtain the top-level menu from which sub-menus (and sub-sub-menus, {\em
etc.}\,) can be selected until the required token or placeholder is found.
This will then be expanded in place of the `MENU' token you initially entered.

Depending on exactly what you are expanding, it may be necessary to adjust 
the indentation of the original `MENU' token (or perhaps add a comment 
character in front of it) to obtain the correct result.
You can use the {\em un-expand} key (GOLD~ctrl-E) to get back to the `MENU' 
token in order to do this.

Note that all the items which appear in these menus are themselves tokens,
so you can jump into the menu hierarchy at any point by using the
appropriate initial token. 
You should also note the name of the token which you eventually find -- you
can use it directly in future instead of going through the menu system. 


\subsection{Enumerated Type Codes}

In certain subroutine libraries (particularly the {\em Graphical Kernel
System} \mbox{GKS}), extensive use is made of symbolic constants to
represent {\em enumerated types} ({\em e.g.}\ a routine argument which may
only take one of a pre-defined set of integer values is ``enumerated'' by
assigning a symbolic name to each of the allowed values). 
Unfortunately, these can sometimes be difficult to use because (a) there may
be a large number of them and (b) the names are typically restricted to a
small number of characters, making them hard to remember. 

To ease this problem in the case of GKS, \mbox{STARLSE} contains menu
definitions for all the enumerated types which the GKS subroutine library
uses. 
Each of these menus is associated with a token whose name is of the form
`GKS\$...'.

Thus, by typing `GKS\$', followed by \mbox{ctrl-E}, you will obtain a menu 
showing all the different enumerated types associated with the \mbox{GKS} 
library.
Having picked the required type from this list, a menu of the symbolic names
assigned to that type (and what they represent) will then be displayed, from
which the appropriate symbolic value may be selected. 

These symbolic values are frequently required as input arguments to
subroutines. 
Where this is likely, the placeholder for the appropriate subroutine argument 
will be indicated by the presence of a `GKS\$...' prefix.
Such placeholders may be expanded (as opposed to simply typing over them)
and will result in a menu showing the symbolic values which may be supplied
at that point. 


\section{The `IFL' Language}

STARLSE also defines a language called IFL, which is the default for file
types of \mbox{.IFL}, and which provides a template for writing interface
files for \mbox{ADAM}~\mbox{A-tasks}. 
This language is far simpler than the STARLINK\_FORTRAN language as it
contains no subroutine libraries, {\em etc.}\,, but it does contain a standard
prologue, together with menus and options to simplify the writing of
interface files by making all necessary information available through the
editor. 

As with other prologues, a blank template is provided to assist with
writing interface files in the same style as \mbox{STARLSE} if another 
editor is to be used.
This template can be found in the \mbox{STARLSE\_DIR} directory in the file
\mbox{IFL.PRO}. 

\section{Additional Commands}
\label{sect:additionalcommands}

Some additional commands are provided by STARLSE to enhance the set already
provided by \mbox{LSE}. 
Some of these are bound to keyboard keys, which means that they may be 
invoked simply by pressing that key.
Where this is so, the key is given in parentheses after the command in the
list below. 
These assignments can easily be changed with the \mbox{DEFINE}~\mbox{KEY}
and \mbox{DELETE}~\mbox{KEY} commands if required (see later). 

The following commands are currently available:

\begin{description}

\item[ALIGN\_COMMENT (ctrl-A)] --- Aligns comment lines as an aid to improving
layout uniformity and to facilitate the integration of foreign code into
Starlink software. 
If the current line is a comment line or contains an end-of-line comment,
then it is aligned so that the comment starts in a standard column.
The comment character is also set to a standard value.
If there is no comment on the line, then this command has no effect. 
If a select range is active, then this process is applied to all the
lines in the select range (be careful not to use this on prologues!).

Note that in \mbox{STARLSE} alignment of end-of-line comments also takes
place implicitly whenever the length of a line is changed by an operation on
a placeholder, such as expanding or killing it. 

\item[COMMENT (ctrl-$\wedge$)\footnotemark]\footnotetext{This is normally
obtained by pressing the ctrl and $\wedge$ keys simultaneously ($\wedge$ is
usually located above the number 6). On some keyboards, however, the shift
key may also need to be pressed at the same time.} --- Inserts a comment
line and a comment placeholder in front of the current line and positions
the cursor on the placeholder, ready to enter a comment. 
A blank line is inserted in front of the comment if there is not one
there already. 
This command has no effect if the current line is already a comment. 

\item[FB] --- ``Flushes'' all modifications to editing buffers, causing the
modified text to be written back to the file associated with each buffer (FB
stands for Flush Buffers).

\item[FIX\_CONTINUATION (GOLD-C)] --- Changes the Fortran continuation
character (if any) on the current line to be a colon `:'.
If a select range is active, then this process is applied to all lines in 
the select range.
This command has no effect unless the \mbox{STARLINK\_FORTRAN} language is in 
use.

Note that when using the \mbox{STARLINK\_FORTRAN} language, this conversion 
process also takes place on the current line whenever any ``special'' key ({\em
i.e.}\ other than a normal character entry key) is pressed.
This is the mechanism by which \mbox{STARLSE} is able to provide `:' as the
continuation character (whereas \mbox{LSE} itself uses the digit `1'). 

\item[GENERIC] --- Runs the Starlink \mbox{GENERIC} utility (SUN/7) on the
contents of the current select range and replaces the select range with the
expanded output. 
The usual \mbox{GENERIC} qualifiers may be specified.

\item[PB (GOLD-P)] --- Causes STARLSE to return to the buffer (and associated
language) you were editing most recently before the current buffer (PB stands
for Previous Buffer).

\item[PRINT] --- Causes the contents of the current editing buffer, or the
current select range (if defined), to be printed as if the \mbox{DCL}
command \mbox{PRINT} had been used, but without leaving the editor. 
The usual \mbox{PRINT} qualifiers may be specified.

\item[SB (F14)] --- Displays a list of all the current editing buffers and
highlights the one which was being edited last. The ``Select'' and ``Remove''
keys may then be used to select a new buffer, or to delete any which are no
longer required. (SB stands for Show Buffer.)

\item[SORT] --- Runs the \mbox{DCL} \mbox{SORT} utility on the lines in the 
current select range.
By default, this sorts them into ascending alphabetical order, although any 
of the usual \mbox{SORT} options may be specified on the command line to 
alter this behaviour.

\item[WHERE (ctrl-W)] --- Displays the current column and line number of the
cursor, and shows how many lines there are in the current buffer and what
percentage of the buffer lies on or above the current line. 

\end{description}

Any of these commands may be bound to a key by using the \mbox{LSE} command
\mbox{DEFINE}~\mbox{KEY}.
For instance, the \mbox{ctrl-W} key has been made to execute the 
\mbox{WHERE} command by issuing the following \mbox{LSE} command: 

\begin{verbatim}
   LSE> DEFINE KEY CTRL_W_KEY "WHERE"
\end{verbatim}

Key definitions such as this would normally be placed in an {\em
initialisation file} for convenience, so that they are available whenever
\mbox{STARLSE} is invoked. 
To do this, you might enter the commands into a file called
\mbox{MYSTUFF.LSE}, which is then made available to \mbox{STARLSE} by means
of a logical name assignment such as: 

\begin{verbatim}
   $ DEFINE LSE$INITIALIZATION DISK$MYDISK:[MYDIR]MYSTUFF.LSE
\end{verbatim}


\section{Acknowledgements}

Thanks are due to the following people who have contributed material or 
ideas for inclusion in \mbox{STARLSE}:

\begin{quote}
\begin{tabular}{l}
Peter Allan \\
Malcolm Currie \\
Nick Eaton \\
Paul Harrison \\
William Lupton \\
Dave Terrett
\end{tabular}
\end{quote}

\newpage
\appendix

\section{Example Prologues}
\label{section:prologues}

\subsection{An ADAM A-task Prologue}

The following is an example of a completed STARLSE prologue for an ADAM 
A-task ({\em i.e.}\ the main routine of an ADAM application program).

\small
\begin{verbatim}
      SUBROUTINE ADD( STATUS )
*+
*  Name:
*     ADD

*  Purpose:
*     Add two NDF data structures.

*  Language:
*     Starlink Fortran 77

*  Type of Module:
*     ADAM A-task

*  Invocation:
*     CALL ADD( STATUS )

*  Arguments:
*     STATUS = INTEGER (Given and Returned)
*        The global status.

*  Description:
*     This application adds two NDF data structures pixel-by-pixel to
*     produce a new NDF.

*  Usage:
*     ADD IN1 IN2 OUT

*  ADAM Parameters:
*     IN1 = NDF (Read)
*        First NDF to be added.
*     IN2 = NDF (Read)
*        Second NDF to be added.
*     OUT = NDF (Write)
*        Output NDF to contain the sum of the two input NDFs.
*     TITLE = LITERAL (Read)
*        Value for the title of the output NDF. A null value (!) will
*        cause the title of the NDF supplied for parameter IN1 to be
*        used instead. ['KAPPA - Add']

*  Examples:
*     ADD A B C
*        Adds the NDF data structures A and B pixel-by-pixel, writing
*        the result to the NDF structure C.
*     ADD IN1=NGC1068 IN2=BIAS OUT=NEW TITLE="NGC1068 with bias added"
*        Adds the two NDF data structures NGC1068 and BIAS
*        pixel-by-pixel to produce a result in the new NDF structure
*        NEW. This output structure is assigned the title "NGC1068 with
*        bias added".

*  Timing:
*     The execution time is approximately proportional to the number of
*     NDF pixels to be added. The time will be approximately doubled if
*     variance components are present in the two input NDFs.

*  Implementation Status:
*     This routine correctly processes the AXIS, DATA, QUALITY, LABEL,
*     TITLE and VARIANCE components of an NDF data structure and
*     propagates all extensions. Bad pixels and all non-complex numeric
*     types can be handled. The UNITS component is not yet supported
*     and is therefore not propagated.

*  Authors:
*     RFWS: R.F. Warren-Smith (STARLINK)
*     {enter_new_authors_here}

*  History:
*     4-APR-1990 (RFWS):
*        Original version, derived from the previous non-NDF routine of
*        the same name.
*     2-OCT-1990 (RFWS):
*        Updated the prologue to demonstrate new features of the
*        STARLSE prologue template.
*     {enter_further_changes_here}

*  Bugs:
*     {note_any_bugs_here}

*-
      
*  Type Definitions:
      IMPLICIT NONE              ! No implicit typing

*  Global Constants:
      INCLUDE 'SAE_PAR'          ! Standard SAE constants
      INCLUDE 'NDF_PAR'          ! NDF_ public constants

*  Status:
      INTEGER STATUS             ! Global status

*  Local Variables:
      CHARACTER * ( NDF__SZTYP ) ITYPE ! Data type for processing
      CHARACTER * ( NDF__SZFTP ) DTYPE ! Data type for output components
      INTEGER EL                 ! Number of mapped elements
      INTEGER NDF1               ! Identifier for 1st NDF (input)
      INTEGER NDF2               ! Identifier for 2nd NDF (input)
      INTEGER NDF3               ! Identifier for 3rd NDF (output)
      INTEGER NERR               ! Number of errors
      INTEGER PNTR1( 1 )         ! Pointer to 1st NDF mapped array
      INTEGER PNTR2( 1 )         ! Pointer to 2nd NDF mapped array
      INTEGER PNTR3( 1 )         ! Pointer to 3rd NDF mapped array
      LOGICAL BAD                ! Need to check for bad pixels?
      LOGICAL VAR1               ! Variance component in 1st input NDF?
      LOGICAL VAR2               ! Variance component in 2nd input NDF?

*.
\end{verbatim}
\normalsize

\subsection{A Subroutine Prologue}

The following is an example of a completed STARLSE prologue for a normal
Fortran~77 subroutine. 

\small
\begin{verbatim}
      SUBROUTINE NDF_ANNUL( INDF, STATUS )
*+
*  Name:
*     NDF_ANNUL

*  Purpose:
*     Annul an NDF identifier.

*  Language:
*     Starlink Fortran 77

*  Invocation:
*     CALL NDF_ANNUL( INDF, STATUS )

*  Description:
*     The routine annuls the NDF identifier supplied so that it is no
*     longer recognised as a valid identifier by the NDF_ routines.
*     Any resources associated with it are released and made available
*     for re-use. If any NDF components are mapped for access, then
*     they are automatically unmapped by this routine.

*  Arguments:
*     INDF = INTEGER (Given and Returned)
*        The NDF identifier to be annulled. A value of NDF__NOID is
*        returned (as defined in the include file NDF_PAR).
*     STATUS = INTEGER (Given and Returned)
*        The global status.

*  Notes:
*     -  This routine attempts to execute even if STATUS is set on
*     entry, although no further error report will be made if it
*     subsequently fails under these circumstances. In particular, it
*     will fail if the identifier supplied is not initially valid, but
*     this will only be reported if STATUS is set to SAI__OK on entry.
*     -  An error will result if an attempt is made to annul the last
*     remaining identifier associated with an NDF whose DATA component
*     has not been defined (unless it is a temporary NDF, in which case
*     it will be deleted at this point).

*  Algorithm:
*     -  Save the error context on entry.
*     -  Import the NDF identifier.
*     -  Annul the associated ACB entry.
*     -  Reset the NDF identifier value.
*     -  Restore the error context.

*  Authors:
*     RFWS: R.F. Warren-Smith (STARLINK)
*     {enter_new_authors_here}

*  History:
*     5-OCT-1989 (RFWS):
*        Original, derived from the equivalent ARY_ system routine.
*     6-OCT-1989 (RFWS):
*        Added STATUS check after calling NDF_$IMPID.
*     {enter_further_changes_here}

*  Bugs:
*     {note_any_bugs_here}

*-
      
*  Type Definitions:
      IMPLICIT NONE              ! No implicit typing

*  Global Constants:
      INCLUDE 'SAE_PAR'          ! Standard SAE constants
      INCLUDE 'NDF_PAR'          ! NDF_ public constants

*  Arguments Given and Returned:
      INTEGER INDF

*  Status:
      INTEGER STATUS             ! Global status

*  Local Variables:
      INTEGER TSTAT              ! Temporary status variable
      INTEGER IACB               ! Index to NDF entry in the ACB

*.
\end{verbatim}
\normalsize

\newpage
\section{Changes and Limitations}

\subsection{Changes since V1.7}

The following describes the main changes which have taken place since the
previous version of STARLSE (V1.7):

\begin{enumerate}

\item A new FB (Flush Buffers) command has been added to force all modified
editing buffers to be updated on disk (see \S\ref{sect:additionalcommands}).

\item A new command SB (Show Buffer) has been added to simplify navigation
between different editing buffers (see \S\ref{sect:buffernavigation}). The SB
command is bound to the F14 key by default.

\item New subroutine package definitions have been added for the PSX library
(POSIX Interface Routines -- SUN/121) and the GWM library (X Graphics Window
Manager -- SUN/130).

\item The definitions for the EMS, ERR, MSG \& NDF subroutine libraries have
been updated to match recent new releases of these systems.

\item The behaviour of the PB command has been improved in cases where the
``previous buffer'' does not exist. This command is now bound to the GOLD-P
key.

\item The handling of the `:' continuation character in the STARLINK\_FORTRAN
language has been improved. In particular, a problem introduced in the previous
version of STARLSE which made it impossible to delete backwards over a
continuation character has been fixed. A work-around has also been installed for
an apparent problem with LSE which could cause the continuation character to
arbitrarily revert to `1'.

\item The behaviour of the ``auto-initialisation'' feature introduced in the
previous version of STARLSE has been modified slightly. The screen width will
not now be altered if you move to an editing buffer where you do not have write
access (the screen width continues to adjust automatically to the current
language if you have write access and other buffer-dependent features continue
to auto-initialise as before). This change has been made to reduce the number
of screen refreshes which occur when using navigation buffers (such as when
reviewing a failed compilation). These were found to be annoying on slow
terminals.

\item The documentation has been revised to reflect the above changes.

\end{enumerate}

\subsection{Current Limitations}

The following limitations are currently imposed by restrictions and features 
within \mbox{LSE} and other contributed material:

\begin{enumerate}

\item When expanding a token which results in a Fortran statement or sequence of
statements, the first of which has a statement label, a number of blank spaces
are appended to the end of the token expansion.  This problem appears to be
related to a problem introduced in LSE version 3.1 which causes Fortran
indentation to be wrongly calculated.  Work-arounds have been installed which
overcome the indentation problem, but the blank spaces have not been
successfully eliminated. 

\item Using the RETURN key when positioned on a placeholder causes the
resulting new line to have the wrong indentation.
This problem arose with the introduction of LSE version 2.3 and the reason for
it is unknown. 

\item When displaying help information, the help screen is displayed twice 
in rapid succession.
This problem arose with the introduction of LSE version 2.3 and the reason for
it is unknown. 

\item If the label placeholder on a Fortran statement is replaced with a
label which is more than one character long, then the indentation on the
statement will be wrong and must be corrected manually.
This appears to be an unavoidable feature of the way LSE handles the label 
field of Fortran statements.

\item The names of subroutine arguments in the \mbox{GKS} help library do
not match the names of the placeholders which appear when \mbox{GKS}
routines are expanded (the latter names being taken from the \mbox{RAL}
\mbox{GKS} manual). 
This problem arises because the GKS specification does not define unique 
names for the routine arguments.

\item After selecting a symbolic constant (for instance) from a displayed
menu, the cursor will usually move on to the next placeholder, which is
often not what is required. 
The use of {\em alias} definitions for these constants (instead of defining
them as {\em routines}) would overcome this, but no descriptive text can
currently be associated with an {\em alias}, so the resulting menu would not
be of use. 

\end{enumerate}


\end{document}
