\documentstyle[11pt]{article} \pagestyle{myheadings}O%------------------------------------------------------------------------------3\newcommand{\stardoccategory}  {Starlink User Note}$\newcommand{\stardocinitials}  {SUN}&\newcommand{\stardocnumber}    {105.5}2\newcommand{\stardocauthors}   {R.F. Warren-Smith}3\newcommand{\stardocdate}      {10th December 1991}0\newcommand{\stardoctitle}     {STARLSE \\ [1ex]A                                Starlink Extensions to the VAX \\:                                Language Sensitive Editor},\newcommand{\stardocversion}   {Version 1.8}4\newcommand{\stardocmanual}    {Programmer's Manual}O%------------------------------------------------------------------------------;\newcommand{\stardocname}{\stardocinitials /\stardocnumber}\markright{\stardocname}\setlength{\textwidth}{160mm}\setlength{\textheight}{230mm}\setlength{\topmargin}{-2mm}\setlength{\oddsidemargin}{0mm} \setlength{\evensidemargin}{0mm}\setlength{\parindent}{0mm}$\setlength{\parskip}{\medskipamount}\setlength{\unitlength}{1mm}!\renewcommand{\_}{{\tt\char'137}}%\renewcommand{\thepage}{\roman{page}}\begin{document}\thispagestyle{empty}=SCIENCE \& ENGINEERING RESEARCH COUNCIL \hfill \stardocname\\ RUTHERFORD APPLETON LABORATORY\\{\large\bf Starlink Project\\},{\large\bf \stardoccategory\ \stardocnumber}\begin{flushright}\stardocauthors\\\stardocdate\end{flushright}
\vspace{-4mm}\rule{\textwidth}{0.5mm}\vspace{5mm}\begin{center}${\Huge\bf  \stardoctitle \\ [2.5ex]}${\LARGE\bf \stardocversion \\ [4ex]}{\Huge\bf  \stardocmanual}\end{center}
\vspace{20mm}O%------------------------------------------------------------------------------%  Package Description\begin{center}{\Large\bf Description}\end{center}I\mbox{STARLSE} is a ``Starlink Sensitive'' editor based on the \mbox{VAX}FLanguage Sensitive Editor \mbox{LSE}. It exploits the extensibility ofH\mbox{LSE} to provide additional features which assist in the writing of=portable Fortran~77 software with a standard Starlink style. F\mbox{STARLSE} is intended mainly for use by those writing \mbox{ADAM}Napplications and subroutine libraries for distribution as part of the StarlinkHSoftware Collection, although it may also be suitable for other softwareJprojects. It is designed to integrate with the SST (Simple Software Tools)package described in SUN/110. O%------------------------------------------------------------------------------0%  Add this part if you want a table of contents\newpage\markright{\stardocname}\null\vspace {5mm}\begin {center}\rule{80mm}{0.5mm} \\ [1ex]#{\Large\bf \stardoctitle \\ [2.5ex]$           \stardocversion} \\ [2ex]\rule{80mm}{0.5mm}\end{center}\setlength{\parskip}{0mm}\tableofcontents$\setlength{\parskip}{\medskipamount}O%------------------------------------------------------------------------------%  Introduction page\newpage\markright{\stardocname}&\renewcommand{\thepage}{\arabic{page}}\setcounter{page}{1}\null\vspace {5mm}\begin {center}\rule{80mm}{0.5mm} \\ [1ex]#{\Large\bf \stardoctitle \\ [2.5ex]$           \stardocversion} \\ [2ex]\rule{80mm}{0.5mm}\end{center}
\vspace{30mm}\section{Introduction}I\mbox{STARLSE} is a ``Starlink Sensitive'' editor based on the \mbox{VAX}JLanguage Sensitive Editor \mbox{LSE}.\footnote{N.B.\ \mbox{LSE} is part ofEthe \mbox{VAXset} (\mbox{VAX} Software Engineering Tools) package andJ\mbox{STARLSE} can only be used if you are logged in to a machine on whichEthe \mbox{VAXset} software is installed. You should consult your site1manager if you are unsure which machine to use.}  PIt exploits the extensibility of \mbox{LSE} to provide additional features whichNassist in the writing of portable Fortran~77 software with a standard Starlinkstyle. F\mbox{STARLSE} is intended mainly for use by those writing \mbox{ADAM}Napplications and subroutine libraries for distribution as part of the StarlinkHSoftware Collection, although it may also be suitable for other software
projects. BThe \mbox{VAX} Language Sensitive Editor is fully described in theE\mbox{VAX} \mbox{LSE/SCA} User and Reference Manuals, which should beLconsulted if you have not used \mbox{LSE} before, and in case of difficulty.CThis note aims only to describe the additional features provided by\mbox{STARLSE}. KIn fact, many powerful features of \mbox{LSE}, such as its interaction withLother VAXset tools like \mbox{CMS} and \mbox{SCA}, are not mentioned here atHall, although they are, of course, all available within \mbox{STARLSE}. HThe relevant \mbox{VAX} documentation should be consulted if these otherfacilities are required.DNote that STARLSE is also designed to integrate with the SST (Simple.Software Tools) package described in SUN/110. \section{Getting Started}\subsection{Commands}\label{sect:starting}FBefore starting to use \mbox{STARLSE}, you must issue the DCL command:\begin{verbatim}   $ STARLSE\end{verbatim}8which can conveniently go in your \mbox{LOGIN.COM} file.PAfter this command is complete \mbox{STARLSE} can be invoked simply by using theN\mbox{LSE} command in the same way as with a normal text editor. Thus, to edit4the file \mbox{TEMP.FOR}, you would use the command:\begin{verbatim}   $ LSE TEMP.FOR\end{verbatim}N(But see \S\ref{sect:attaching} for a faster method of access if you are usingSTARLSE repeatedly.)\subsection{Exploring STARLSE}EPerhaps the best way to start using \mbox{STARLSE} is by exploring it*before you try to write anything serious. HThis will probably involve moving between the {\em placeholders} in yourKfile (these are the things which appear surrounded by [...] or \{...\}) andH{\em expanding} and {\em un-expanding} them to see what they represent. JIf you haven't used \mbox{LSE} before, then you should have the \mbox{VAX}G\mbox{LSE/SCA} User Manual on hand at this stage, as it provides a goodFintroduction to placeholders and other important \mbox{LSE} concepts. GYou should probably start by trying to master the following keys, whichFshould give you a good initial feel for what \mbox{STARLSE} provides: 
\begin{quote}\begin{center}\begin{tabular}{ll}ctrl-E & (\underline{E}xpand)\\/ctrl-N & (go to \underline{N}ext placeholder)\\3ctrl-P & (go to \underline{P}revious placeholder)\\'ctrl-K & (\underline{K}ill placeholder)
\end{tabular}\end{center}\end{quote}HIf you make a mistake, or expand a placeholder into something you don't Jwant, then you can go back one step (only) by means of the following keys:
\begin{quote}\begin{center}\begin{tabular}{ll}3GOLD ctrl-E & (un-\underline{E}xpand placeholder)\\/GOLD ctrl-K & (un-\underline{K}ill placeholder)
\end{tabular}\end{center}\end{quote}\subsection{A Few Tips}EIf you normally use a standard screen editor (like EDT) then you willKprobably find that using an \mbox{LSE}-based editor for the first time willslow you down considerably. All those new keys to remember!IHowever, once you get used to it, you will start to realise how much timeHyou used to spend doing simple things like moving the cursor, formatting?prologues, indenting lines of code and going to fetch essential@documentation -- all things which \mbox{STARLSE} can do far more
efficiently. HAn important thing to appreciate once you start to use \mbox{STARLSE} inHearnest is that you shouldn't try too hard to use everything it providesD(like religiously expanding {\em all} the placeholders, tokens, {\em
etc.}\,). KIn fact, one of the strengths of \mbox{LSE}-based editing is that it allowsFyou to pick and choose -- to use the features that you personally findLtime-saving, while still being able to type directly over anything which youfind too fussy. GIn this respect \mbox{STARLSE} can be regarded as a sort of interactiveG``Manual of Style'' which provides guidelines on layout and programmingLstandards when you need reminding of them (generally in response to pressing9the {\em expand} key \mbox{ctrl-E}, or the {\em help} keyK\mbox{GOLD}~\mbox{PF2}) but lets you work unhindered once you know what youare doing. JNevertheless, you should be aware that some traditional editing techniques/do not carry over to LSE all that effectively. IIn particular, use of the space-bar and the RETURN and arrow keys is not Lnearly as effective for navigating around an LSE editing buffer as the keys 2described above for skipping between placeholders.JThis means that it is generally wise not to delete unused placeholders tooIearly, but to leave most of them in position as ``stepping stones'' untilediting is almost complete. >This may seem untidy at first, but soon becomes second nature.(\subsection{Faster Access to the Editor}\label{sect:attaching}LOne problem with LSE is that it takes rather more time to start up than mostNother editors, and this can be a serious disadvantage if you want to use it to7make frequent minor changes during program development.LIf you are fortunate enough to have access to a windowing terminal, then onePsolution is to dedicate a window to STARLSE and to leave it running while you doOother work in other windows. Unfortunately, however, not everyone has access toMsuch terminals, so STARLSE provides a special {\em attach key} facility. ThisKallows STARLSE to be run in a subprocess to which you may {\em attach} yourJterminal whenever you need to use it, thus giving very rapid access to thePeditor once it has initially been started. This is a convenient technique to use5with character-cell terminals and terminal emulators.MTo make use of this facility, you must nominate a keyboard key to act as yourOSTARLSE {\em attach key} (the F20 key is recommended, if available). You shouldLsupply the name of your chosen key as an argument when issuing the \$STARLSE-command (\S\ref{sect:starting}), for example:\begin{verbatim}   $ STARLSE F20\end{verbatim}KIf you press this key, a subprocess will be created to run STARLSE and yourdNterminal will be attached to it. You will then be in the editor ready to startOwork. Pressing the same key again while in editing mode will return you to yourePoriginal process. These steps may be repeated as often as you like to repeatedlyLenter and leave the editor. When you finally want to stop STARLSE, issue the!EXIT (or QUIT) command as normal.hPWhen using STARLSE in this mode you cannot specify the  file to be edited on the8command line, so you must issue one of the LSE commands:\begin{verbatim}   GOTO FILE filenamee\end{verbatim}or\begin{verbatim}   GOTO FILE/CREATE filename\end{verbatim}Min order to access or create the required files. It is normally convenient to-Nkeep a number of files open in different buffers when working in this way (seethe next section).FSTARLSE will flush any modified buffers by writing new versions of theNassociated files whenever you use the {\em attach key} to leave the editor. InMfact, this is a quick and convenient way of ensuring that your work is safelyaMstored on disk if you are interrupted during an editing session. STARLSE willnMalso ensure that the LSE default directory setting remains set to that of thePprocess from which it was started. Thus, if you use the SET DEFAULT command fromNDCL to change your default directory setting, then STARLSE will adopt this new#setting when you next attach to it.s#\subsection{Moving Between Buffers}u\label{sect:buffernavigation}{NThe most convenient way of using STARLSE to edit several files is to keep eachJfile in a separate {\em editing buffer}. STARLSE then provides a method ofAmoving between these buffers using the SB command (short for Show-OBuffer).\footnote{This is essentially the same as the standard LSE command SHOWLBUFFER except that the cursor is automatically positioned on the name of the,previous buffer, which is also highlighted.}NThis command may be issued either from the LSE command line or by pressing theHF14 key when in editing mode. It will display a list of all your currentKediting buffers, with the cursor positioned on the name of the one you weremJlast editing. The normal cursor-movement keys may then be used to move theKcursor within this list. The ``Select'' key will select a chosen buffer for Nediting, while the ``Remove'' key may be used to delete any buffer which is nolonger required."\subsection{Personalising STARLSE}FIf you use \mbox{STARLSE} regularly, you may like to make logical nameAdefinitions such as the following in your \mbox{LOGIN.COM} file: o\begin{verbatim}2   $ DEFINE STARLSE$PERSONAL_NAME "Fred J. Bloggs")   $ DEFINE STARLSE$PERSONAL_USERID "FJB"aD   $ DEFINE STARLSE$PERSONAL_AFFILIATION "STARLINK, Univ. of Balham"\end{verbatim}JThese definitions allow \mbox{STARLSE} to insert the values you specify atJcertain standard places, in response to pressing the {\em expand} (ctrl-E)key. -LThe first value is a string to be used as your personal name (if this is not:defined then you will be prompted whenever it is needed). KThe second is a short form version of your name ({\em e.g.}\ your initials)K-- this will default to your \mbox{VMS} user ID, so you need only define it/if your user ID does not identify you clearly. FThe third value is the institution to which you are affiliated and, ifInecessary, your place of work -- this will default simply to `STARLINK'. oKNote that it is best to avoid adding extra information (such as your E-mailiJaddress) to any of these values, because such information soon gets out ofKdate and becomes little more than a historical curiosity, if not positivelyoImisleading (remember, many pieces of software have life-times measured int
decades). IThe usage suggested above is intended to allow anyone using or supportingiHyour software in future to trace you if necessary via the institution atwhich you did the work. *\section{The `STARLINK\_FORTRAN' Language}JPerhaps the most important component of \mbox{STARLSE} is a version of theFFortran~77 programming language called \mbox{STARLINK\_FORTRAN}, which;defines the style of programming that the editor supports. dFThis language is used as the default for files of type \mbox{.FOR} and
\mbox{.GEN}. hJThe \mbox{STARLINK\_FORTRAN} language is based on the Starlink ApplicationIProgramming Standard document SGP/16, and contains only a small number ofv0approved extensions to the Fortran~77 standard. JIt is therefore much simpler and easier to use than the \mbox{VAX}~FortranHlanguage which comes with ``native'' \mbox{LSE} (in fact, nearly 80\% ofH\mbox{VAX}~Fortran consists of extensions to the Fortran~77 standard!). JIn \mbox{STARLINK\_FORTRAN}, the number of available options and ambiguousLabbreviations is greatly reduced, and most of the common language constructsKcan be produced simply by typing a short token, like `DO' or `IF', followeddby \mbox{ctrl-E}. HThe most important features of the \mbox{STARLINK\_FORTRAN} language aredescribed below.%\subsection{Templates and Prologues.}EC\mbox{STARLINK\_FORTRAN} provides templates for the common types off7Fortran~77 program module used in Starlink applications B(\mbox{ADAM}~\mbox{A-tasks}, \mbox{Subroutine}, \mbox{Function} \&<\mbox{Block}~\mbox{Data} program units and ADAM~Monoliths). IEach of these templates includes a standard {\em prologue} ({\em i.e.}\ aHset of header comments) with a uniform style, and these prologues may beHprocessed automatically into different forms of user documentation usingJapplications from the SST (Simple Software Tools) package (see SUN/110 for
details). 1Some examples of completed prologues are given inb"Appendix~\ref{section:prologues}. JThe language definition provides assistance when entering information intoLthese prologues by means of placeholders, which expand into suitable default5values (on pressing \mbox{ctrl-E}) wherever possible.rMIf no default value is available, then a message is displayed describing the h#information which you should enter.r@Many of the prologue sections are optional and may be omitted ifcircumstances dictate. rIIn addition, some sections in each prologue constitute a {\em maintenanceiGlog} ({\em e.g.}\ the ``Authors'' and ``History'' sections) and certainnKplaceholders in these sections are intended to remain in place to assist in making future additions. rJIf you do not have access to \mbox{LSE}, or wish to use a different editorMbut nevertheless would like to format prologue information in the same style kEas \mbox{STARLSE}, then a set of blank template files is provided to hassist.nLThese files can be found in the directory \mbox{STARLSE\_DIR} and have namesas follows: 
\begin{quote}l\begin{center}\begin{tabular}{|l|l|}\hline!{\bf Filename} & {\bf Purpose} \\i
\hline \hlineo)ATASK.PRO & {\em ADAM A-task template} \\q&SUB.PRO & {\em Subroutine template} \\%FUNC.PRO & {\em Function template} \\d0BLOCK.PRO & {\em Block Data routine template} \\)MON.PRO & {\em ADAM Monolith template} \\i\hline
\end{tabular}u\end{center}\end{quote}hJFor completeness, these files contain templates for {\bf all} the prologue%items which \mbox{STARLSE} supports. pJIn practice, however, only a fairly small subset of these will normally beHrequired, so you may find it convenient to produce your own ``cut down''Jcopies of these templates by removing items which you don't normally use. #\subsection{Subroutine Definitions}bH{\em Package definitions} are provided in \mbox{STARLSE} for most of theLstandard Starlink/\mbox{ADAM} subroutine libraries, and these are accessible,from the \mbox{STARLINK\_FORTRAN} language. KThese definitions allow you to type the name of a routine from any of theseoHlibraries (or an abbreviation) and to obtain the routine's full name and0argument list simply by pressing \mbox{ctrl-E}. KThis is particularly helpful if you can't remember the name of the routine eIyou want, because an ambiguous abbreviation ({\em e.g.}\ just the packageaLprefix) will result in a {\em menu} in which all matching routines are shown1along with a simple description of what they do. {JThe required routine can then be selected from the menu and will appear inEyour editing buffer, together with its argument list, with the cursort.positioned ready to enter the first argument. JThis facility can save a great deal of time which would otherwise be spentreferring to documentation. DIt can also virtually eliminate argument-passing errors, which are a%significant source of software bugs. kKThe subroutine libraries currently supported by STARLSE package definitionse'are shown in Table~\ref{table:status}. e
\begin{table}i\begin{center}\begin{tabular}{|c|c|c|c|}\hlineD{\bf Library} & {\bf Description} & {\bf Help?} & {\bf References}\\
\hline \hline=AGI & {\em Applications Graphics Interface} & yes & SUN/48 \\n9ARY & {\em Array Data Structure Access} & yes & SUN/11 \\t9CHR & {\em Character Handling Routines} & yes & SUN/40 \\o5EMS & {\em Error Message Service}    & yes & SSN/4 \\ 7ERR & {\em Error System Routines}    & yes & SUN/104 \\f4FIO & {\em File I/O System}          & no & APN/9 \\6GKS & {\em Graphical Kernel System}  & yes & SUN/83 \\7GWM & {\em X Graphics Window Manager}& yes & SUN/130 \\ =HDS & {\em Hierarchical Data System} & yes & SUN/92, APN/7 \\u6IDI & {\em Image Display Interface}  & yes & SUN/65 \\2MAG & {\em Magnetic Tape Handling} & no & APN/1 \\7MSG & {\em Message System Routines}  & yes & SUN/104 \\w7NDF & {\em NDF Data Structure Access} & yes & SUN/33 \\ 4PAR & {\em ADAM Parameter System}    & no & APN/6 \\<PGPLOT & {\em Graphics Subroutine Library} & yes & SUN/15 \\:PRIMDAT & {\em Primitive Data Processing} & no & SUN/39 \\7PSX & {\em POSIX Interface Routines} & yes & SUN/121 \\S5SGS & {\em Simple Graphics System}   & no & SUN/85 \\l<SLALIB & {\em Mainly Positional Astronomy} & yes & SUN/67 \\7TRN & {\em Coordinate Transformation} & yes & SUN/61 \\s\hline
\end{tabular}8\caption{The subroutine libraries supported by STARLSE.}\label{table:status}\end{center}\end{table}e%\subsection{On-line Help Information} GOn-line {\em help} information is also available for certain subroutines
libraries.OTo use this facility the cursor should be positioned on the item for which helpsDis required ({\em i.e.}\ the subroutine name) and the {\em help} key \mbox{GOLD}~\mbox{PF2} pressed. FIf help is available on the indicated item, it will then be displayed.LTable~\ref{table:status} indicates which subroutine libraries currently haveon-line help available. \subsection{Alias Definitions}BA number of \mbox{LSE} {\em alias} definitions are provided in theJ\mbox{STARLINK\_FORTRAN} language for expressions which need to be enteredfairly often. AAn alias (or an abbreviation) can simply be typed in, followed byfJ\mbox{ctrl-E}, in order to enter the equivalent text into the file you are	editing. R$The following are given as examples:
\begin{quote}i\begin{center}\begin{tabular}{l|l}%{\bf Alias} & {\bf Equivalent Text}\\h\hline
ST & STATUS\\sOK & STATUS .EQ. SAI\_\_OK\\BAD & STATUS .NE. SAI\_\_OK\\s%CLOC & CHARACTER $*$ ( DAT\_\_SZLOC )r
\end{tabular} \end{center}\end{quote}lKA full list of the aliases defined for the current language can be obtainedh!by using the \mbox{LSE} command: e\begin{verbatim}   LSE> SHOW ALIAS\end{verbatim}(\subsection{ADAM Programming Constructs}\label{sect:adamconstructs}eIA small number of tokens are defined to represent programming constructs l=which are commonly used in ADAM applications and subroutines.gHAs a simple example, the \mbox{`\_OK\_BLOCK'} token will expand into thefollowing block: L\begin{verbatim}%      IF ( STATUS .EQ. SAI__OK ) THENu"         {executable_statement}...      END IF\end{verbatim}Bwhich only executes if the global \mbox{STATUS} value is not set. JAs further examples, a token \mbox{`\_CHECK\_STATUS'} exists to facilitateKchecking of the \mbox{STATUS} value in \mbox{ADAM} programs, and expands asL	follows: l\begin{verbatim}3      IF ( STATUS .NE. SAI__OK ) GO TO {abort_stmt}i\end{verbatim}Iwhile the \mbox{`\_ERROR\_REPORT'} token provides a template for making auFstandard error report using the Starlink \mbox{ERR\_} and \mbox{MSG\_}
routines. Its expansion is:l\begin{verbatim}      STATUS = {error_code}e      [define_message_token]...v2      CALL ERR_REP( '{routine_name}_{error_name}',%     :              '{message_text}',n     :              STATUS )\end{verbatim}LThese special \mbox{ADAM} tokens are distinguished by their first character,$which is always an underscore `\_'. EThus, a complete list of available tokens which represent \mbox{ADAM}$Hprogramming constructs may be obtained by using the \mbox{LSE} command: \begin{verbatim}   LSE> SHOW TOKEN _*r\end{verbatim}LAs always, these tokens may be abbreviated (two or three characters normally4suffice) and you can ``un-expand'' them again (usingA\mbox{GOLD}~\mbox{ctrl-E}) if you are not happy with the result. p>\subsection{Symbolic Constants, Error Codes and Include Files}LInformation is also available in the \mbox{STARLINK\_FORTRAN} language aboutEthe symbolic constants, error codes and include files associated with certain subroutine libraries. LIn most cases this has been provided by defining the symbolic names of theseJcomponents to be {\em routines} (but without arguments) in the appropriateLSE package definition. BYou can therefore obtain a menu of symbolic constants, {\em etc.}\Jassociated with a subroutine library in the same way as you would obtain aAmenu of the routines themselves; {\em i.e.}\ by typing a suitablec(abbreviation followed by \mbox{ctrl-E}. GThis facility is currently available for the GKS, HDS and TRN libraries)only. \subsection{Tokens and Menus}oFBecause many of the items in \mbox{STARLSE} programming templates are Ioptional, it often happens that a specific item has been omitted from an eGexisting piece of software, but is later required when the software is  	upgraded.SIA simple example would occur if a \mbox{DATA} statement were added to an \existing subroutine.JIn this case a new ``Local Data'' section in the prologue might be wanted,Lbut it could be difficult to remember how such a section should be formattedG(or even what it is called) because the optional placeholder which onceg%represented it is no longer present. oJAnother example would occur if you wanted to add a \mbox{STARLSE} prologue8to an imported routine which did not initially have one.IIn this case, you would not be able to remember all the layout details so Lit would be essential to insert a template for the prologue into the routinebefore starting work. J\mbox{LSE} provides s solution to this problem in the form of {\em tokens}=(of which the \mbox{ADAM} programming constructs described inr3\S\ref{sect:adamconstructs} are a simple example).  JA token (or an abbreviation) can be entered into an editing buffer at any -point and then expanded (by pressing ctrl-E).e@Tokens are provided for the majority of items which exist in theB\mbox{STARLINK\_FORTRAN} language (from single statements, throughFindividual prologue sections, to complete program units) so there willJalmost always be a token representing the particular part of a programmingtemplate you want to use. IUnfortunately, the sheer number of tokens available can be a problem; you\Gwill soon learn the ones you use frequently, but it can be difficult tol<remember (or guess) those which are less frequently needed. K\mbox{STARLSE} therefore provides a set of hierarchical menus to assist in k,finding the token (or placeholder) you want.>The top level in this menu system is represented by the token:\begin{verbatim}   MENUf\end{verbatim}LOn entering this token into an editing buffer, and pressing ctrl-E, you willGobtain the top-level menu from which sub-menus (and sub-sub-menus, {\emEJetc.}\,) can be selected until the required token or placeholder is found.NThis will then be expanded in place of the `MENU' token you initially entered.KDepending on exactly what you are expanding, it may be necessary to adjust nGthe indentation of the original `MENU' token (or perhaps add a comment 7character in front of it) to obtain the correct result.sLYou can use the {\em un-expand} key (GOLD~ctrl-E) to get back to the `MENU' token in order to do this.JNote that all the items which appear in these menus are themselves tokens,Aso you can jump into the menu hierarchy at any point by using therappropriate initial token. sKYou should also note the name of the token which you eventually find -- younHcan use it directly in future instead of going through the menu system. "\subsection{Enumerated Type Codes}GIn certain subroutine libraries (particularly the {\em Graphical Kernel}CSystem} \mbox{GKS}), extensive use is made of symbolic constants to.Jrepresent {\em enumerated types} ({\em e.g.}\ a routine argument which mayIonly take one of a pre-defined set of integer values is ``enumerated'' byi:assigning a symbolic name to each of the allowed values). LUnfortunately, these can sometimes be difficult to use because (a) there mayIbe a large number of them and (b) the names are typically restricted to ay:small number of characters, making them hard to remember. ETo ease this problem in the case of GKS, \mbox{STARLSE} contains menu Idefinitions for all the enumerated types which the GKS subroutine libraryeuses. HEach of these menus is associated with a token whose name is of the form`GKS\$...'.mKThus, by typing `GKS\$', followed by \mbox{ctrl-E}, you will obtain a menu cJshowing all the different enumerated types associated with the \mbox{GKS} library.LHaving picked the required type from this list, a menu of the symbolic namesLassigned to that type (and what they represent) will then be displayed, from6which the appropriate symbolic value may be selected. CThese symbolic values are frequently required as input arguments toa
subroutines. eNWhere this is likely, the placeholder for the appropriate subroutine argument 9will be indicated by the presence of a `GKS\$...' prefix.fISuch placeholders may be expanded (as opposed to simply typing over them)nKand will result in a menu showing the symbolic values which may be suppliedtat that point. i\section{The `IFL' Language}ISTARLSE also defines a language called IFL, which is the default for filesItypes of \mbox{.IFL}, and which provides a template for writing interfacet&files for \mbox{ADAM}~\mbox{A-tasks}. FThis language is far simpler than the STARLINK\_FORTRAN language as itNcontains no subroutine libraries, {\em etc.}\,, but it does contain a standardDprologue, together with menus and options to simplify the writing ofIinterface files by making all necessary information available through theeeditor. DAs with other prologues, a blank template is provided to assist withGwriting interface files in the same style as \mbox{STARLSE} if another Heditor is to be used. KThis template can be found in the \mbox{STARLSE\_DIR} directory in the file\\mbox{IFL.PRO}. \section{Additional Commands}I\label{sect:additionalcommands} KSome additional commands are provided by STARLSE to enhance the set already provided by \mbox{LSE}. GSome of these are bound to keyboard keys, which means that they may be N$invoked simply by pressing that key.JWhere this is so, the key is given in parentheses after the command in thelist below. IThese assignments can easily be changed with the \mbox{DEFINE}~\mbox{KEY}t?and \mbox{DELETE}~\mbox{KEY} commands if required (see later). t/The following commands are currently available:o\begin{description}SN\item[ALIGN\_COMMENT (ctrl-A)] --- Aligns comment lines as an aid to improvingHlayout uniformity and to facilitate the integration of foreign code intoStarlink software. 5IIf the current line is a comment line or contains an end-of-line comment,\Cthen it is aligned so that the comment starts in a standard column.r6The comment character is also set to a standard value.EIf there is no comment on the line, then this command has no effect. nDIf a select range is active, then this process is applied to all theElines in the select range (be careful not to use this on prologues!).rHNote that in \mbox{STARLSE} alignment of end-of-line comments also takesLplace implicitly whenever the length of a line is changed by an operation on0a placeholder, such as expanding or killing it. I\item[COMMENT (ctrl-$\wedge$)\footnotemark]\footnotetext{This is normallynKobtained by pressing the ctrl and $\wedge$ keys simultaneously ($\wedge$ iseJusually located above the number 6). On some keyboards, however, the shiftHkey may also need to be pressed at the same time.} --- Inserts a commentIline and a comment placeholder in front of the current line and positionsn9the cursor on the placeholder, ready to enter a comment. lDA blank line is inserted in front of the comment if there is not onethere already. gEThis command has no effect if the current line is already a comment. uK\item[FB] --- ``Flushes'' all modifications to editing buffers, causing theTLmodified text to be written back to the file associated with each buffer (FBstands for Flush Buffers).F\item[FIX\_CONTINUATION (GOLD-C)] --- Changes the Fortran continuation9character (if any) on the current line to be a colon `:'.iJIf a select range is active, then this process is applied to all lines in the select range.cNThis command has no effect unless the \mbox{STARLINK\_FORTRAN} language is in use.LNote that when using the \mbox{STARLINK\_FORTRAN} language, this conversion Oprocess also takes place on the current line whenever any ``special'' key ({\eml;i.e.}\ other than a normal character entry key) is pressed.TKThis is the mechanism by which \mbox{STARLSE} is able to provide `:' as thecGcontinuation character (whereas \mbox{LSE} itself uses the digit `1'). rJ\item[GENERIC] --- Runs the Starlink \mbox{GENERIC} utility (SUN/7) on theKcontents of the current select range and replaces the select range with theexpanded output. 5The usual \mbox{GENERIC} qualifiers may be specified.eM\item[PB (GOLD-P)] --- Causes STARLSE to return to the buffer (and associatedmMlanguage) you were editing most recently before the current buffer (PB stands}for Previous Buffer).sJ\item[PRINT] --- Causes the contents of the current editing buffer, or theEcurrent select range (if defined), to be printed as if the \mbox{DCL} Dcommand \mbox{PRINT} had been used, but without leaving the editor. 3The usual \mbox{PRINT} qualifiers may be specified.sJ\item[SB (F14)] --- Displays a list of all the current editing buffers andMhighlights the one which was being edited last. The ``Select'' and ``Remove''nKkeys may then be used to select a new buffer, or to delete any which are noa-longer required. (SB stands for Show Buffer.)L\item[SORT] --- Runs the \mbox{DCL} \mbox{SORT} utility on the lines in the current select range.-LBy default, this sorts them into ascending alphabetical order, although any Iof the usual \mbox{SORT} options may be specified on the command line to dalter this behaviour. L\item[WHERE (ctrl-W)] --- Displays the current column and line number of theIcursor, and shows how many lines there are in the current buffer and whate<percentage of the buffer lies on or above the current line. \end{description}fKAny of these commands may be bound to a key by using the \mbox{LSE} commanda\mbox{DEFINE}~\mbox{KEY}. AFor instance, the \mbox{ctrl-W} key has been made to execute the \B\mbox{WHERE} command by issuing the following \mbox{LSE} command: \begin{verbatim}%   LSE> DEFINE KEY CTRL_W_KEY "WHERE"y\end{verbatim}@Key definitions such as this would normally be placed in an {\emIinitialisation file} for convenience, so that they are available whenever{\mbox{STARLSE} is invoked.  ;To do this, you might enter the commands into a file called K\mbox{MYSTUFF.LSE}, which is then made available to \mbox{STARLSE} by means&of a logical name assignment such as: \begin{verbatim}=   $ DEFINE LSE$INITIALIZATION DISK$MYDISK:[MYDIR]MYSTUFF.LSEm\end{verbatim}\section{Acknowledgements}HThanks are due to the following people who have contributed material or &ideas for inclusion in \mbox{STARLSE}:
\begin{quote}u\begin{tabular}{l}Peter Allan \\Malcolm Currie \\a
Nick Eaton \\hPaul Harrison \\William Lupton \\eDave Terrett
\end{tabular}s\end{quote}e\newpage	\appendixf\section{Example Prologues}L\label{section:prologues}d$\subsection{An ADAM A-task Prologue}HThe following is an example of a completed STARLSE prologue for an ADAM EA-task ({\em i.e.}\ the main routine of an ADAM application program).e\small\begin{verbatim}      SUBROUTINE ADD( STATUS )*+*  Name:	*     ADD *  Purpose:m"*     Add two NDF data structures.*  Language:*     Starlink Fortran 77f*  Type of Module:*     ADAM A-taske*  Invocation:*     CALL ADD( STATUS )
*  Arguments:t+*     STATUS = INTEGER (Given and Returned)b*        The global status.a*  Description: E*     This application adds two NDF data structures pixel-by-pixel to *     produce a new NDF.	*  Usage:i*     ADD IN1 IN2 OUT *  ADAM Parameters:t*     IN1 = NDF (Read)*        First NDF to be added.h*     IN2 = NDF (Read) *        Second NDF to be added.*     OUT = NDF (Write)s=*        Output NDF to contain the sum of the two input NDFs.o*     TITLE = LITERAL (Read)E*        Value for the title of the output NDF. A null value (!) will D*        cause the title of the NDF supplied for parameter IN1 to be&*        used instead. ['KAPPA - Add']*  Examples:*     ADD A B CuE*        Adds the NDF data structures A and B pixel-by-pixel, writing+*        the result to the NDF structure C.F*     ADD IN1=NGC1068 IN2=BIAS OUT=NEW TITLE="NGC1068 with bias added":*        Adds the two NDF data structures NGC1068 and BIASD*        pixel-by-pixel to produce a result in the new NDF structureG*        NEW. This output structure is assigned the title "NGC1068 withE*        bias added". 
*  Timing:G*     The execution time is approximately proportional to the number ofnG*     NDF pixels to be added. The time will be approximately doubled if<*     variance components are present in the two input NDFs.*  Implementation Status: F*     This routine correctly processes the AXIS, DATA, QUALITY, LABEL,@*     TITLE and VARIANCE components of an NDF data structure andG*     propagates all extensions. Bad pixels and all non-complex numericnD*     types can be handled. The UNITS component is not yet supported&*     and is therefore not propagated.*  Authors:t(*     RFWS: R.F. Warren-Smith (STARLINK)*     {enter_new_authors_here}*  History:d*     4-APR-1990 (RFWS):G*        Original version, derived from the previous non-NDF routine ofm*        the same name.i*     2-OCT-1990 (RFWS):@*        Updated the prologue to demonstrate new features of the#*        STARLSE prologue template.p"*     {enter_further_changes_here}*  Bugs:*     {note_any_bugs_here}*-      *  Type Definitions:5      IMPLICIT NONE              ! No implicit typingu*  Global Constants:9      INCLUDE 'SAE_PAR'          ! Standard SAE constants 8      INCLUDE 'NDF_PAR'          ! NDF_ public constants
*  Status:0      INTEGER STATUS             ! Global status*  Local Variables:mA      CHARACTER * ( NDF__SZTYP ) ITYPE ! Data type for processingtH      CHARACTER * ( NDF__SZFTP ) DTYPE ! Data type for output components<      INTEGER EL                 ! Number of mapped elementsA      INTEGER NDF1               ! Identifier for 1st NDF (input)hA      INTEGER NDF2               ! Identifier for 2nd NDF (input)rB      INTEGER NDF3               ! Identifier for 3rd NDF (output)3      INTEGER NERR               ! Number of errorspB      INTEGER PNTR1( 1 )         ! Pointer to 1st NDF mapped arrayB      INTEGER PNTR2( 1 )         ! Pointer to 2nd NDF mapped arrayB      INTEGER PNTR3( 1 )         ! Pointer to 3rd NDF mapped array@      LOGICAL BAD                ! Need to check for bad pixels?G      LOGICAL VAR1               ! Variance component in 1st input NDF? G      LOGICAL VAR2               ! Variance component in 2nd input NDF?a*.\end{verbatim}\normalsizeh"\subsection{A Subroutine Prologue}HThe following is an example of a completed STARLSE prologue for a normalFortran~77 subroutine. t\small\begin{verbatim}*      SUBROUTINE NDF_ANNUL( INDF, STATUS )*+*  Name:*     NDF_ANNUL{*  Purpose:a*     Annul an NDF identifier.*  Language:*     Starlink Fortran 77 *  Invocation:$*     CALL NDF_ANNUL( INDF, STATUS )*  Description:tE*     The routine annuls the NDF identifier supplied so that it is nonC*     longer recognised as a valid identifier by the NDF_ routines.hF*     Any resources associated with it are released and made availableC*     for re-use. If any NDF components are mapped for access, thenb6*     they are automatically unmapped by this routine.
*  Arguments:u)*     INDF = INTEGER (Given and Returned)tC*        The NDF identifier to be annulled. A value of NDF__NOID iss;*        returned (as defined in the include file NDF_PAR).i+*     STATUS = INTEGER (Given and Returned)r*        The global status.o	*  Notes:oB*     -  This routine attempts to execute even if STATUS is set on@*     entry, although no further error report will be made if itE*     subsequently fails under these circumstances. In particular, itaF*     will fail if the identifier supplied is not initially valid, butF*     this will only be reported if STATUS is set to SAI__OK on entry.E*     -  An error will result if an attempt is made to annul the lastTF*     remaining identifier associated with an NDF whose DATA componentG*     has not been defined (unless it is a temporary NDF, in which casef(*     it will be deleted at this point).
*  Algorithm: )*     -  Save the error context on entry.d#*     -  Import the NDF identifier.d(*     -  Annul the associated ACB entry.(*     -  Reset the NDF identifier value.#*     -  Restore the error context.n*  Authors:.(*     RFWS: R.F. Warren-Smith (STARLINK)*     {enter_new_authors_here}*  History:l*     5-OCT-1989 (RFWS):C*        Original, derived from the equivalent ARY_ system routine. *     6-OCT-1989 (RFWS):5*        Added STATUS check after calling NDF_$IMPID.e"*     {enter_further_changes_here}*  Bugs:*     {note_any_bugs_here}*-      *  Type Definitions:5      IMPLICIT NONE              ! No implicit typingd*  Global Constants:9      INCLUDE 'SAE_PAR'          ! Standard SAE constantse8      INCLUDE 'NDF_PAR'          ! NDF_ public constants *  Arguments Given and Returned:      INTEGER INDF
*  Status:0      INTEGER STATUS             ! Global status*  Local Variables:o<      INTEGER TSTAT              ! Temporary status variable@      INTEGER IACB               ! Index to NDF entry in the ACB*.\end{verbatim}\normalsizec\newpage!\section{Changes and Limitations}e\subsection{Changes since V1.7}`IThe following describes the main changes which have taken place since theo#previous version of STARLSE (V1.7):e\begin{enumerate}BK\item A new FB (Flush Buffers) command has been added to force all modifiedhLediting buffers to be updated on disk (see \S\ref{sect:additionalcommands}).J\item A new command SB (Show Buffer) has been added to simplify navigationMbetween different editing buffers (see \S\ref{sect:buffernavigation}). The SBO+command is bound to the F14 key by default.nL\item New subroutine package definitions have been added for the PSX libraryL(POSIX Interface Routines -- SUN/121) and the GWM library (X Graphics WindowManager -- SUN/130).L\item The definitions for the EMS, ERR, MSG \& NDF subroutine libraries have;been updated to match recent new releases of these systems.oJ\item The behaviour of the PB command has been improved in cases where theK``previous buffer'' does not exist. This command is now bound to the GOLD-Pdkey.M\item The handling of the `:' continuation character in the STARLINK\_FORTRANiOlanguage has been improved. In particular, a problem introduced in the previouseFversion of STARLSE which made it impossible to delete backwards over aPcontinuation character has been fixed. A work-around has also been installed forLan apparent problem with LSE which could cause the continuation character toarbitrarily revert to `1'.L\item The behaviour of the ``auto-initialisation'' feature introduced in theMprevious version of STARLSE has been modified slightly. The screen width will)Onot now be altered if you move to an editing buffer where you do not have writewIaccess (the screen width continues to adjust automatically to the currenteNlanguage if you have write access and other buffer-dependent features continueMto auto-initialise as before). This change has been made to reduce the number Kof screen refreshes which occur when using navigation buffers (such as whentHreviewing a failed compilation). These were found to be annoying on slow
terminals.F\item The documentation has been revised to reflect the above changes.\end{enumerate}- \subsection{Current Limitations}MThe following limitations are currently imposed by restrictions and features r1within \mbox{LSE} and other contributed material:v\begin{enumerate}P\item When expanding a token which results in a Fortran statement or sequence ofNstatements, the first of which has a statement label, a number of blank spacesKare appended to the end of the token expansion.  This problem appears to benGrelated to a problem introduced in LSE version 3.1 which causes Fortran Mindentation to be wrongly calculated.  Work-arounds have been installed whicheDovercome the indentation problem, but the blank spaces have not beensuccessfully eliminated. TF\item Using the RETURN key when positioned on a placeholder causes the1resulting new line to have the wrong indentation.aNThis problem arose with the introduction of LSE version 2.3 and the reason forit is unknown.  K\item When displaying help information, the help screen is displayed twice in rapid succession.NThis problem arose with the introduction of LSE version 2.3 and the reason forit is unknown. lH\item If the label placeholder on a Fortran statement is replaced with aHlabel which is more than one character long, then the indentation on the7statement will be wrong and must be corrected manually.eKThis appears to be an unavoidable feature of the way LSE handles the label sfield of Fortran statements.I\item The names of subroutine arguments in the \mbox{GKS} help library doADnot match the names of the placeholders which appear when \mbox{GKS}Groutines are expanded (the latter names being taken from the \mbox{RAL}N\mbox{GKS} manual). IThis problem arises because the GKS specification does not define unique i names for the routine arguments.I\item After selecting a symbolic constant (for instance) from a displayedmGmenu, the cursor will usually move on to the next placeholder, which is often not what is required. KThe use of {\em alias} definitions for these constants (instead of defining Hthem as {\em routines}) would overcome this, but no descriptive text canLcurrently be associated with an {\em alias}, so the resulting menu would notbe of use.  \end{enumerate}o\end{document}