\documentstyle[11pt]{article} 
\pagestyle{myheadings}

%------------------------------------------------------------------------------
\newcommand{\stardoccategory}  {History File}

\newcommand{\stardocinitials}  {  }
\newcommand{\stardocsource}    {  }
\newcommand{\stardocnumber}    {  }
\newcommand{\stardocauthors}   {V S Dhillon\\G Privett\\K P Duffey}
\newcommand{\stardocdate}      {12th December 2001}

\newcommand{\stardoctitle}     {PERIOD --- History File}
%------------------------------------------------------------------------------

\newcommand{\stardocname}{\stardocinitials /\stardocnumber}
\markright{\stardocname}
\setlength{\textwidth}{160mm}
\setlength{\textheight}{230mm}
\setlength{\topmargin}{-2mm}
\setlength{\oddsidemargin}{0mm}
\setlength{\evensidemargin}{0mm}
\setlength{\parindent}{0mm}
\setlength{\parskip}{\medskipamount}
\setlength{\unitlength}{1mm}
 
% -----------------------------------------------------------------------------
% Hypertext definitions.
% These are used by the LaTeX2HTML translator in conjuction with star2html.
 
% Comment.sty: version 2.0, 19 June 1992
% Selectively in/exclude pieces of text.
%
% Author
%    Victor Eijkhout                                      <eijkhout@cs.utk.edu>
%    Department of Computer Science
%    University Tennessee at Knoxville
%    104 Ayres Hall
%    Knoxville, TN 37996
%    USA
 
%  Do not remove the %\begin{rawtex} and %\end{rawtex} lines (used by 
%  star2html to signify raw TeX that latex2html cannot process).
%\begin{rawtex}
\makeatletter
\def\makeinnocent#1{\catcode`#1=12 }
\def\csarg#1#2{\expandafter#1\csname#2\endcsname}
 
\def\ThrowAwayComment#1{\begingroup
    \def\CurrentComment{#1}%
    \let\do\makeinnocent \dospecials
    \makeinnocent\^^L% and whatever other special cases
    \endlinechar`\^^M \catcode`\^^M=12 \xComment}
{\catcode`\^^M=12 \endlinechar=-1 %
 \gdef\xComment#1^^M{\def\test{#1}
      \csarg\ifx{PlainEnd\CurrentComment Test}\test
          \let\html@next\endgroup
      \else \csarg\ifx{LaLaEnd\CurrentComment Test}\test
            \edef\html@next{\endgroup\noexpand\end{\CurrentComment}}
      \else \let\html@next\xComment
      \fi \fi \html@next}
}
\makeatother
 

\def\includecomment
 #1{\expandafter\def\csname#1\endcsname{}%
    \expandafter\def\csname end#1\endcsname{}}
\def\excludecomment
 #1{\expandafter\def\csname#1\endcsname{\ThrowAwayComment{#1}}%
    {\escapechar=-1\relax
     \csarg\xdef{PlainEnd#1Test}{\string\\end#1}%
     \csarg\xdef{LaLaEnd#1Test}{\string\\end\string\{#1\string\}}%
    }}
 
%  Define environments that ignore their contents.
\excludecomment{comment}
\excludecomment{rawhtml}
\excludecomment{htmlonly}
%\end{rawtex}
 
%  Hypertext commands etc. This is a condensed version of the html.sty
%  file supplied with LaTeX2HTML by: Nikos Drakos <nikos@cbl.leeds.ac.uk> &
%  Jelle van Zeijl <jvzeijl@isou17.estec.esa.nl>. The LaTeX2HTML documentation
%  should be consulted about all commands (and the environments defined above)
%  except \xref and \xlabel which are Starlink specific.
 
\newcommand{\htmladdnormallinkfoot}[2]{#1\footnote{#2}}
\newcommand{\htmladdnormallink}[2]{#1}
\newcommand{\htmladdimg}[1]{}
\newenvironment{latexonly}{}{}
\newcommand{\hyperref}[4]{#2\ref{#4}#3}
\newcommand{\htmlref}[2]{#1}
\newcommand{\htmlimage}[1]{}
\newcommand{\htmladdtonavigation}[1]{}
 
% Starlink cross-references and labels.
\newcommand{\xref}[3]{#1}
\newcommand{\xlabel}[1]{}
 
%  LaTeX2HTML symbol.
\newcommand{\latextohtml}{{\bf LaTeX}{2}{\tt{HTML}}}
 
%  Define command to recentre underscore for Latex and leave as normal
%  for HTML (severe problems with \_ in tabbing environments and \_\_
%  generally otherwise).
\newcommand{\latex}[1]{#1}
\newcommand{\setunderscore}{\renewcommand{\_}{{\tt\symbol{95}}}}
\latex{\setunderscore}
 
%  Redefine the \tableofcontents command. This procrastination is necessary 
%  to stop the automatic creation of a second table of contents page.
\newcommand{\latexonlytoc}[0]{\tableofcontents}
 

% -----------------------------------------------------------------------------
%  Debugging.
%  =========
%  Un-comment the following to debug links in the HTML version using Latex.
 
% \newcommand{\hotlink}[2]{\fbox{\begin{tabular}[t]{@{}c@{}}#1\\\hline{\footnotesize #2}\end{tabular}}}
% \renewcommand{\htmladdnormallinkfoot}[2]{\hotlink{#1}{#2}}
% \renewcommand{\htmladdnormallink}[2]{\hotlink{#1}{#2}}
% \renewcommand{\hyperref}[4]{\hotlink{#1}{\S\ref{#4}}}
% \renewcommand{\htmlref}[2]{\hotlink{#1}{\S\ref{#2}}}
% \renewcommand{\xref}[3]{\hotlink{#1}{#2 -- #3}}
% -----------------------------------------------------------------------------
% Add any document specific \newcommand or \newenvironment commands here
 
% Environment for indenting and using a small font.
\newenvironment{myquote}{\begin{quote}\begin{small}}{\end{small}\end{quote}}
 
% in-line verbatims
\newcommand{\myverb}[1]{{\small \verb+#1+}}
 
% -----------------------------------------------------------------------------
%  Title Page.
%  ===========
\begin{document}
\thispagestyle{empty}
 
%  Latex document header.
\begin{latexonly}
   CCLRC / {\sc Rutherford Appleton Laboratory} \hfill {\bf \stardocname}\\
   {\large Particle Physics \& Astronomy Research Council}\\
   {\large Starlink Project\\}
   {\large \stardoccategory\ \stardocnumber}
   \begin{flushright}
   \stardocauthors\\
   \stardocdate
   \end{flushright}
   \vspace{-4mm}
   \rule{\textwidth}{0.5mm}
   \vspace{5mm}
   \begin{center}
   {\Large\bf \stardoctitle}
   \end{center}
   \vspace{5mm}
 
%  Add heading for abstract if used.
%   \vspace{10mm}
%   \begin{center}
%      {\Large\bf Description}
%   \end{center}
\end{latexonly}
 

%  HTML documentation header.
\begin{htmlonly}
   \xlabel{}
   \begin{rawhtml} <H1> \end{rawhtml}
      \stardoctitle
   \begin{rawhtml} </H1> \end{rawhtml}
 
%  Add picture here if required.
 
   \begin{rawhtml} <P> <I> \end{rawhtml}
   \stardoccategory \stardocnumber \\
   \stardocauthors \\
   \stardocdate
   \begin{rawhtml} </I> </P> <H3> \end{rawhtml}
      \htmladdnormallink{CCLRC}{http://www.cclrc.ac.uk} /
      \htmladdnormallink{Rutherford Appleton Laboratory}
                        {http://www.cclrc.ac.uk/ral} \\
      Particle Physics \& Astronomy Research Council \\
   \begin{rawhtml} </H3> <H2> \end{rawhtml}
      \htmladdnormallink{Starlink Project}{http://star-www.rl.ac.uk/}
   \begin{rawhtml} </H2> \end{rawhtml}
   \htmladdnormallink{\htmladdimg{source.gif} Retrieve hardcopy}
      {http://star-www.rl.ac.uk/cgi-bin/hcserver?\stardocsource}\\
 
% HTML document table of contents (if used). 
% ==========================================
% Add table of contents header and a navigation button to return 
% to this point in the document (this should always go before the
% abstract \section). This places the table of contents on the title
% page. Do not use this if you want the normal behaviour.
%   \label{stardoccontents}
%   \begin{rawhtml} 
%     <HR>
%     <H2>Contents</H2>
%   \end{rawhtml}
%   \renewcommand{\latexonlytoc}[0]{}
%   \htmladdtonavigation{\htmlref{\htmladdimg{contents_motif.gif}}
%                                            {stardoccontents}}
 
%  Start new section for abstract if used.
%  \section{\xlabel{abstract}Abstract}
 
\end{htmlonly}
 
% -----------------------------------------------------------------------------
%  Document Abstract. (if used)
%  ==================
% -----------------------------------------------------------------------------
%  Latex document Table of Contents. (if used)
%  ===========================================
%  Replace the \latexonlytoc command with \tableofcontents if you're
%  not only having a contents list on the title page.
% \begin{latexonly}
%    \setlength{\parskip}{0mm}
%    \latexonlytoc
%    \setlength{\parskip}{\medskipamount}
%    \markright{\stardocname}
% \end{latexonly}
% -----------------------------------------------------------------------------
 

{\noindent This document lists all of the changes made to the PERIOD
package and the release dates of\newline the various versions.}
\vspace{10mm}



\begin{verbatim}

   DATE       Author            Comments
                           
1992 July 10   VSD       VERSION 1.0 released to Robert Smith (Sussex), 
                         Mark O'Dell (Sussex) and Erik Kuulkers (Amsterdam).
                           
                         *** VERSION 1.0 *** 
                           
1992 July 21   VSD       This history file added to the package.
1992 July 21   VSD       PERIOD_SCARGLE modified to output EFFM instead of
                         PROB and JMAX. This enables PEAKS to calculate te
                         false alarm probability of any peak in the
                         periodogram.
1992 July 21   VSD       PERIOD_PERIOD modified to cope with changes in
                         PERIOD_SCARGLE and calculate the false alarm
                         probability of any peak selected in PEAKS.
                         Also included a trap for invalid values of the 
                         loop gain in CLEAN (greater than 0 and less than 2).
1992 July 21   VSD       PERIOD_INPUT modified to prevent crashes if the 
                         input file contains two contiguous x-axis points 
                         with the same or decreasing values.
1992 July 21   VSD       PERIOD_DOC extended by adding a recipe.
                           
\end{verbatim}



\newpage


\begin{verbatim}
                                                      
   DATE       Author            Comments
                           
1992 July 21   VSD       PERIOD_MAIN, _CLEAN, _DETREND, _FIT, _FOLD, _FT,
                         _INPUT, _NOISE, _PERIOD, _PHASE, _POLYFIT and
                         _STATUS modified in order to handle 10000 data or 
                         frequency points and 40 data files containing 20 
                         columns. This ensures a page file quota of no more 
                         than 34000 is required to run PERIOD.
1992 July 21   VSD       PERIOD_LOG, _READFREE, _STATUS modified by setting
                         large loop variables in data or parameter
                         declarations.
1992 July 21   VSD       Included a MAXCOL parameter in PERIOD_MAIN and
                         PERIOD_INPUT in order to cut down array sizes.
                         Also set size of MAXANT to MXROW in order to 
                         ensure that only the maximum number of permitted
                         rows are loaded.
1992 July 21   VSD       PERIOD_COM:PERIOD_MAIL.COM added in order to make
                         e-mailing new versions of PERIOD easier.
1992 July 21   VSD       VERSION 1.1 of PERIOD released to incorporate the
                         above changes. Copy given to Mark O'Dell (Sussex).
	                   
                         *** VERSION 1.1 *** 
	                   
1992 July 29   VSD       PERIOD_SCARGLE modified to remove calculation of
                         false alarm probability from subroutine.
1992 July 29   VSD       PEAKS in PERIOD_PERIOD modified to include a more 
                         rigorous calculation of the SCARGLE false alarm
                         probability using an analytical expression for the
                         number of independent frequencies.
1992 July 30   VSD       NWK parameter, specifying sizes of the workspace
                         arrays in PERIOD_PERIOD and PERIOD_SCARGLE, reduced
                         from MXROW*100 to MXROW*16 in order to cut down on 
                         memory usage.
1992 July 30   VSD       PERIOD_MOMENT modified to remove redundant skew and
                         kurtosis calculations.
1992 Aug 6     VSD       PERIOD_DOC.TEX split into two documents -- a user
                         document (PERIOD_USER.TEX -- LUN 2.1) and a system 
                         document (PERIOD_SYSTEM.TEX -- LSN 1.1). 
1992 Aug 7     VSD       PERIOD_INTRO modified to give my address as RGO
                         La Palma and my e-mail account as CAVAD::VSD.
1992 Aug 17    VSD       PERIOD_USER.TEX largely rewritten in order to make
                         document clearer and more comprehensive.
1992 Aug 17    VSD       Sampling interval in MEM option of PERIOD_PERIOD 
                         now output in PEAKS instead of in MEM. 
                           
\end{verbatim}


\newpage


\begin{verbatim}
   DATE       Author            Comments                           
                           
1992 Aug 17    VSD       Dimension of NWK in PERIOD_PERIOD changed from 16
                         to 27. This is because the maximum possible workspace
                         usage (assuming 10000 data points and 10000 frequency
                         steps) in PERIOD_SCARGLE is 262144.
1992 Aug 17    VSD       VERSION 1.2 of PERIOD released to incorporate the
                         above changes. Installed on SUSTAR as local 
                         STARLINK software. Copy sent to Erik Kuulkers
                         (Amsterdam).
	                   
                         *** VERSION 1.2 *** 
	                   
1992 Sept 24  VSD       Annoying information written to screen when running
                        PLT removed from PERIOD_PLT.
1993 Mar 3    VSD       Now at ING, not Sussex! PERIOD_PDM and PERIOD_FTEST 
                        subroutines added for the new Phase Dispersion 
                        Minimization (PDM) option.
1993 Mar 3    VSD       PERIOD_PERIOD extensively modified to cope with the
                        new PDM option. The two major changes are the 
                        inclusion of the code to call PERIOD_PDM (which 
                        calculates the PDM statistic) and PEAKS, which now
                        does an F-test to determine the significance of a
                        PDM minimum by calling PERIOD_FTEST.
1993 Mar 3    VSD       Error check on zero period added to PERIOD_FAKE,
                        PERIOD_FIT, PERIOD_PHASE and PERIOD_SINE.
1993 Mar 4    VSD       Reduced the size of the INFILEARRAY string in 
                        PERIOD_PERIOD by shortening 'Frequency' to 'Freq'.
1993 Mar 4    VSD       Adapted PERIOD_SYSTEM.TEX (LSN 1.2) and 
                        PERIOD_USER.TEX (LUN 2.2) to incorporate 
                        new PDM option and other changes to 
                        bring the documentation into line with the 
                        STARLINK release.
1993 Mar 4    VSD       New e-mail address, version number and date
                        added to PERIOD_INTRO.
1993 Mar 4    VSD       PERIOD_COMPILE, PERIOD_LIBRARY, PERIOD_MAIL
                        all modified to include the new PERIOD_PDM and
                        PERIOD_FTEST subroutines. 
1993 Mar 4    VSD       VERSION 2.0 of PERIOD released to incorporate
                        the above changes. Installed on ING Vax and
                        copies sent to STARLINK (RLVAD::STAR), Sussex 
                        (SUSTAR::SYSTEM) and Amsterdam 
                        (PSI\%IBERPAC.DWINGELOO::UVAA01::ERIK) 
	                   
                         *** VERSION 2.0 *** 
	                   
\end{verbatim}


\newpage


\begin{verbatim}
   DATE       Author            Comments                           
                                                      
1993 Mar 16    VSD       Bug in FOLD option found by Martin Still (Sussex).
                         Traced to PROMPT_ZEROPT parameter not setting 
                         ZEROPT parameter correctly. Modified PERIOD_FIT and 
                         PERIOD_PHASE subroutines accordingly (by removing
                         the PROMPT_ZEROPT parameter completely).
1993 Mar 17    VSD       Included an on-line help option. A new subroutine, 
                         PERIOD_HELP, reads a set of help files contained
                         in the new PERIOD subdirectory PERIOD_HLP. The help
                         files are named according to the convention:
                         PERIOD_'COMMAND'.HLP. PERIOD_MAIN and
                         PERIOD_PERIOD have been modified accordingly,
                         as have the various command procedures and 
                         documents (LSN 1.3 and LUN 2.3).
1993 Mar 17    VSD       Minor changes made to PERIOD_INPUT to reduce
                         references made to RUBY. 
1993 Mar 17    VSD       TYPE replaced by WRITE in PERIOD_INPUT and
                         PERIOD_STATUS to conform to standard F77.
1993 Mar 17    VSD       Bug in PERIOD_LOG which prevented the opening
                         of an old log file fixed.
1993 Mar 17    VSD       VERSION 3.0 of PERIOD released to incorporate
                         the above changes. Installed on ING Vax and
                         copies sent to STARLINK (RLVAD::STAR), Sussex 
                         (SUSTAR::SYSTEM) and Amsterdam 
                         (PSI\%IBERPAC.DWINGELOO::UVAA01::ERIK)
                         This was the first version to be released on STARLINK.
	                   
                         *** VERSION 3.0 *** 
	                   
1993 Jul 5    VSD        Included line number as diagnostic of corrupted
                         x-axis data in PERIOD_INPUT (in response to request
                         from Margaret Penston, RGO). 
1993 Aug 21   VSD        Bug in PERIOD_PHASE found by Martin Still (Sussex).
                         When folding data using binning on a number of files
                         the folded data is all placed into a single
                         (incorrect) slot. Problem traced to the use of 
                         the COUNTER variable for two different tasks 
                         (slot number and data array elements). Changed 
                         the data array variable to BINCOUNT. Checked all
                         PERIOD subroutines for a similar error and nothing
                         found.
1993 Aug 2    VSD        Bug in PERIOD_CLEAN found (and corrected) by 
                         Christian Knigge (Oxford). When processing a number
                         of files using CLEAN the first periodogram is
                         calculated correctly, but subsequent files are 
                          
\end{verbatim}


\newpage


\begin{verbatim}
   DATE       Author            Comments                           
	                   
                         wrongly processed, as are files processed if CLEAN
                         is run twice without exiting PERIOD. Problem traced
                         to the C array which is now initialised on each
                         call to PERIOD_CLEAN. Checked all PERIOD_PERIOD 
                         subroutines for a similar error and nothing 
                         found.
1993 Aug 22    VSD       PERIOD_CLEAN made more robust by checking the
                         HWHM calculated by PERIOD_HWHM is not zero before
                         calling PERIOD_FILLB -- this used to cause a crash.
1993 Aug 22    VSD       PEAKS in PERIOD_PERIOD now checks that different 
                         numbers are specified for the periodogram and 
                         time-series. Also now quits if a 0 is entered for
                         the slot number (this did not work properly before).
1993 Aug 22    VSD       Bug in CHAOS option of PERIOD_FAKE did not reset
                         the initial value parameter if processing more
                         than one slot. Problem remedied by introducing
                         a new parameter -- INITVAL. 
1993 Aug 22    VSD       PERIOD_DETREND made more robust by removing the
                         PERIOD_POLYFIT subroutine and calling PERIOD_LSQUAR
                         directly from PERIOD_DETREND. Program now outputs
                         the chi-squared value of the polynomial fit. Also
                         rewrote PERIOD_POLY to handle polynomials of order
                         up to 20 (the previous limit was 12) and removed
                         the redundant subroutine PERIOD_DPOLY. Updated the
                         command procedures in PERIOD_COM by removing any
                         reference to PERIOD_POLYFIT, which has now been 
                         deleted from the PERIOD_FOR subdirectory.
1993 Aug 22    VSD       Bug found in DETREND -- the polynomial fit was
                         divided and not subtracted from the time-series.
                         Have modified PERIOD_DETREND accordingly.
1993 Aug 22    VSD       Added my INTERNET e-mail address to PERIOD_INTRO
                         and incremented the date and version number (3.1).
1993 Aug 27    VSD       Removed string length significance limits from
                         PEAKS in PERIOD_PERIOD since misleading (the 
                         theoretical minimum only applied to an 
                         evenly-spaced sinusoid). Followed comment by
                         Christian Knigge (Oxford). 
1993 Aug 27    VSD       Corrected reduced chi-squared values in the CHISQ 
                         option of PERIOD_PERIOD and PERIOD_FIT by dividing 
                         by the number of points minus the number of free 
                         parameters (=3 in PERIOD_SINFIT). Previously, the 
                         chi-squared value was only divided by the number 
                         of points. Bug spotted by Christian Knigge (Oxford). 
                          
\end{verbatim}


\newpage


\begin{verbatim}
   DATE       Author            Comments                           
                           
1993 Aug 27    VSD       Removed chi-squared value of polynomial fit from
                         PERIOD_DETREND and replaced with simple RMS 
                         calculation. VERSION 3.1 never released.
	                   
                         *** VERSION 3.1 *** 
	                   
1993 Nov 13   VSD       Removed CHAOS option from PERIOD_PERIOD since it is
                        of no use in its present form. Modified PERIOD_HELP
                        and various .COM files and deleted the .HLP file to
                        reflect this change.
1993 Nov 13   VSD       Removed MEM option from PERIOD_PERIOD since it is
                        of no use in its present form. Modified PERIOD_HELP
                        and various .COM files and deleted the .HLP file to
                        reflect this change. Removed the now redundant 
                        subroutines PERIOD_EVLMEM, PERIOD_MEMCOF and 
                        PERIOD_MDIAN1 from the PERIOD_FOR directory and 
                        placed them in the PERIOD_DEV directory.
1993 Nov 13   VSD       Minor bug in PERIOD_NOISE discovered. Removed the 
                        option which allows one to add Poissonian noise 
                        to the data since it incorrectly just added the 
                        SQRT of the data value.
1993 Nov 13   VSD       Removed PERIOD_FTEST from the PERIOD_FOR directory
                        and placed in PERIOD_DEV directory. This routine
                        is now redundant since significances are calculated
                        in PERIOD_PERIOD using a Fisher test (see below).
1993 Nov 14   VSD       Major changes to PERIOD_PERIOD in order to include
                        significance estimate for each periodogram. Removed 
                        the estimates from PEAKS (which were all incorrect),
                        added an extra option SIG and modified each of the 
                        6 remaining PERIOD finding options in order to 
                        perform a Fisher randomisation test. Also included
                        new INFO parameter in PERIOD_FT, PERIOD_SCARGLE 
                        and PERIOD_CLEAN.
1993 Nov 14   VSD       Modified PERIOD_INTRO to reflect new version, date
                        and removed request for acknowledgments.
1993 Nov 14   VSD       Modified PERIOD_DETREND and PERIOD_POLY by setting
                        maximum polynomial order from 20 to 10 (since this
                        was causing underflow errors).
1993 Nov 14   VSD       Changed variable type of SEED in PERIOD_NOISE from
                        REAL to INTEGER, since this caused a crash.
1993 Nov 14   VSD       Changed check of minimum number of points in
                        PERIOD_SINFIT from .LT. 3 to .LE. 3 since this 
                        caused a crash.
1993 Nov 18   VSD       Updated all the .HLP files to reflect the above
                        changes and included a new file for the SIG option.
                           
\end{verbatim}


\newpage

                           
\begin{verbatim}
   DATE       Author            Comments                           

                           
1993 Nov 18    VSD       Restructured PERIOD subdirectories. Now only 4
                         subdirectories -- .FOR, .HLP, .DOC and .SYS. The
                         first three directories contain the fortran source
                         code, help files and user documents (as before). The
                         .SYS directory contains the executable image and the
                         command procedures for initialising and making the 
                         PERIOD package.
1993 Nov 18    VSD       Restructured the command procedures. There are now
                         only two: PERIOD_START.COM and PERIOD_MAKE.COM
                         (which calls PERIOD_MAKE.OPT). This is to bring it
                         more into line with STARLINK standard. 
                         PERIOD_START.COM now defines the new directories 
                         and defines a new logical PERIOD_DISK. 
                         PERIOD_MAKE.COM compiles, creates and fills the 
                         object library, links and then deletes the object 
                         files and object library. 
1993 Nov 18    VSD       To bring into line with STARLINK, the PERIOD 
                         initialisation command is now PERIODSTART (and not
                         PERIOD_START). 
1994 Jan 15    VSD       Included a new subroutine PERIOD_CASE which converts
                         strings from lower to upper case. Replaces calls
                         to the VAX/VMS specific STR\$UPCASE.
1994 Jan 24    VSD       Modified PERIOD_STRING and PERIOD_SINFIT so as not
                         to crash when presented with windowed data. All
                         PERIOD_PERIOD options can now either cope with
                         windowed data or else report an error.
1994 Jan 25    VSD       Changes made to PERIOD_DOC files. This file 
                         converted to \LaTeX\ and renamed PERIOD_HISTORY.TEX.
                         PERIOD_SYSTEM.TEX renamed to SSN167.TEX and
                         rewritten to reflect above changes. SUN167.TEX also
                         rewritten to reflect above changes. PERIOD_USER.TEX
                         (the old Sussex LUN) deleted. 
1994 Jan 26    VSD       Changes made to PERIOD_INPUT so that if it finds
                         x-axis data which is not in ascending order, it
                         prompts users for whether they would like to sort
                         it or not. If they do, it sorts the data using
                         PERIOD_SHELLSORT. If not, it aborts as before.
                         Follows suggestions by Ed Zuiderwijk (RGO).
1994 Feb 1     VSD       Following extensive testing, VERSION 4.0 of PERIOD 
                         released to incorporate the above changes. 
                         Installed on ING Vax and copies sent to STARLINK 
                         (RLVAD::STAR) and Amsterdam (ERIK@astro.uva.nl).
	                   
                         *** VERSION 4.0 *** 
	                   
\end{verbatim}


\newpage


\begin{verbatim}
   DATE       Author            Comments                                                      

1995 June 15  GJP       PERIOD Version 4.0 ported to UNIX. 
                        Functionality remains the same as version 4.0 
                        Now requires an environmental variable to  
                        find the help files. SUN167 and SSN25 re-written 
                        dropping references to VMS. 
1995 Sept 26  GJP       Added the OGIP reading option and modified
                        the documents accordingly. SSN167 becomes SSN25.
1995 Oct  12  GJP       Modified SUN167, SSN25 and PERIOD_HISTORY
                        to be compatible with star2html so that WWW
                        versions can be created by latex2html. Also,
                        removed conversion of input files names to upper
                        case. Submitted to STARLINK.
1995 Dec  15  BLY       Removed dependencies on XANADU versions on PLT and
                        FITSIO by a) providing xanlib.a and plt.dhf for
                        supported systems, b) using Starlink version of 
                        FITSIO.  Makefile and documents modified accordingly.
                        Hypertext versions of documents created.
                        Starlink Release.
                   
                        *** VERSION 4.1 *** 
	                   
1997 Mar  10  GJP       PERIOD 4.1 ported to LINUX. 
                        QDP PLT routine replaced by PGPLOT with some
                        small loss of interactive plotting capability.
                        Slightly clearer error messages in some routines,
                        modified help files and a minor VAXism in 
                        period_output was fixed. A number of routines have
                        been changed to ensure that the variables used
                        are preinitialised to zero.
 
1997 May  07  BLY       Modified period_poly.f to use integer powers not
                        double precision powers when raising X to higher
                        powers since X can be negative.  Old method caused
                        core dump on Digital Unix.

                        *** VERSION 4.2 ***

                        *** VERSION 5.0 ***

2001 Dec  12  KPD       PERIOD implemented in Double Precision.
                        Dynamic array allocation/deallocation introduced
                        for the majority of the input, output and work
                        arrays.
   

\end{verbatim}

\end{document}
