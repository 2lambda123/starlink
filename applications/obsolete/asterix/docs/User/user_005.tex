\documentclass[11pt,fleqn]{article}    %10% larger letters, equns to left
\usepackage{graphics}
\pagestyle{myheadings}
%------------------------------------------------------------------------------
\newcommand{\stardoccategory}  {User Note}
\newcommand{\stardocinitials}  {USER}
\newcommand{\stardocnumber}    {005}
\newcommand{\stardocauthors}   {Richard D Saxton, Jeremy Ashley \& Robert Vallance}
\newcommand{\stardocdate}      {Revised February 2000\\David Geddes}
\newcommand{\stardoctitle}     {The ROSAT XRT interface in ASTERIX}
\newcommand{\stardocname}      {\stardocinitials /\stardocnumber}
%------------------------------------------------------------------------------

\setlength{\textwidth}{160mm}           % Text width 16 cm
\setlength{\textheight}{240mm}          % Text height 24 cm
\setlength{\oddsidemargin}{0pt}         % LH margin width, -1 inch
\setlength{\evensidemargin}{0pt}        % LH margin width, -1 inch
\setlength{\topmargin}{-5mm}            %
\setlength{\headsep}{8mm}               % 
\setlength{\parindent}{0mm}

%    Starlink definitions for \LaTeX\ macros used in MAN output
%
%  Description:
%    As much as possible of the output from the MAN automatic manual generator
%    uses calls to user-alterable macros rather than direct calls to built-in
%    \LaTeX\ macros. This file is a version of the MAN default definitions for
%    these macros modified for Starlink preferences.
%
%  Language:
%    \LaTeX
%
%  Support:
%    William Lupton, {AAO}
%    Alan Chipperfield (RAL)
%-
%  History:
%    16-Nov-88 - WFL - Add definitions to permit hyphenation to work on
%                words containing special characters and in teletype fonts.
%    27-Feb-89 - AJC - Redefine \manroutine
%                      Added \manheadstyle
%                      Switch order of argument descriptors
%    07-Mar-89 - AJC - Narrower box for parameter description
%                      Remove Intro section and other unused bits
%
% permit hyphenation when in teletype font (support 9,10,11,12 point only -
% could extend), define lccodes for special characters so that the hyphen-
% ation algorithm is not switched off. Define underscore character to be
% explicit underscore rather than lots of kerns etc.

\typeout{Starlink MAN macros. Released 27th February 1989}

%\hyphenchar\nintt=`-\hyphenchar\tentt=`-\hyphenchar\elvtt=`-\hyphenchar\twltt=`-\lccode`_=`_\lccode`$=`$

%    Macros used in the .TEX_SUMMARY file
%
%  Description:
%    There is a command to introduce a new section (mansection) and a list-like
%    environment (mansectionroutines) that handles the list of routines in the
%    current section. In addition a mansectionitem command can be used instead
%    of the item command to introduce a new routine in the current section.
%-

\newcommand {\mansection}[2]{\subsection{#1 --- #2}}

\newenvironment {mansectionroutines}{\begin{description}\begin{description}}%
{\end{description}\end{description}}

\newcommand {\mansectionitem}[1]{\item [#1:] \mbox{}}

%    Macros used in the .TEX_DESCR file
%
%  Description:
%    There is a command to introduce a new routine (manroutine) and a list-like
%    environment (manroutinedescription) that handles the list of paragraphs
%    describing the current routine. In addition a manroutineitem command can
%    be used instead of the item command to introduce a new paragraph for the
%    current routine.
%
%    Two-column tables (the ones that can occur anywhere and which are
%    triggered by "=>" as the second token on a line) are bracketed by a
%    new environment (mantwocolumntable). Other sorts of table are introduced
%    by relevant  environments (manparametertable, manfunctiontable and
%    manvaluetable). The definitions of these environments call various other
%    user-alterable commands, thus allowing considerable user control over such
%    tables... (to be filled in when the commands have been written)
%-

\newcommand {\manrule}{\rule{\textwidth}{0.5mm}}

%\newcommand {\manroutine}[2]{\subsection{#1 --- #2}}
\newlength{\speccaption}
\newlength{\specname}
\newcommand{\manroutine}[2]{\goodbreak
                          \rule{\textwidth}{0.5mm}  % draw thick line
                          \settowidth{\specname}{{\Large {\bf #1}}}
                        % left and right box width is text width plus gap
                          \addtolength{\specname}{4ex} 
                        % caption width is width of page less the two names
                        % less than empirical fudge factor
                          \setlength{\speccaption}{\textwidth}
                          \addtolength{\speccaption}{-2.0\specname}
                          \addtolength{\speccaption}{-4.45pt}
                        % move text up the page because \flushleft environ-
                        % ment creates a paragraph
                          \vspace{-7mm}
                          \newline
                          \parbox[t]{\specname}{\flushleft{\Large {\bf #1}}}
                          \parbox[t]{\speccaption}{\flushleft{\Large #2}}
                          \parbox[t]{\specname}{\flushright{\Large {\bf #1}}}
                          }

\newenvironment {manroutinedescription}{\begin{description}}{\end{description}}

\newcommand {\manroutineitem}[2]{\item [#1:] #2\mbox{}}


% parameter tables

\newcommand {\manparametercols}{lllp{90mm}}

\newcommand {\manparameterorder}[3]{#2 & #3 & #1 &}

\newcommand {\manparametertop}{}

\newcommand {\manparameterblank}{\gdef\manparameterzhl{}\gdef\manparameterzss{}}

\newcommand {\manparameterbottom}{}

\newenvironment {manparametertable}{\gdef\manparameterzss{}%
\gdef\manparameterzhl{}\hspace*{\fill}\vspace*{-\partopsep}\begin{trivlist}%
\item[]\begin{tabular}{\manparametercols}\manparametertop}{\manparameterbottom%
\end{tabular}\end{trivlist}}

\newcommand {\manparameterentry}[3]{\manparameterzss\gdef\manparameterzss{\\}%
\gdef\manparameterzhl{\hline}\manparameterorder{#1}{#2}{#3}}


% list environments

\newenvironment {manenumerate}{\begin{enumerate}}{\end{enumerate}}

\newcommand {\manenumerateitem}[1]{\item [#1]}

\newenvironment {manitemize}{\begin{itemize}}{\end{itemize}}

\newcommand {\manitemizeitem}{\item}

\newenvironment {mandescription}{\begin{description}\begin{description}}%
{\end{description}\end{description}}

\newcommand {\mandescriptionitem}[1]{\item [#1]}

\newcommand {\mantt}{\tt}

% manheadstyle for Starlink
\newcommand {\manheadstyle}{}

\catcode`\_=12
\def\cha{{$\langle c1 \rangle$}}
\def\chb{{$\langle c2 \rangle$}}
\def\rornum{{$\langle nnnnnn \rangle$}}
\def\RPF{{$R \langle P/F \rangle$}}
\def\RH{{\it RH}}
\def\RP{{\it RP}}

\newcommand{\docpicture} {\includegraphics{user_005_rosat.ps_tex}}
% End of MAN add-in


\begin{document}                                % Start document
\thispagestyle{empty}
SCHOOL OF PHYSICS AND ASTRONOMY \hfill \stardocname\\
BIRMINGHAM UNIVERSITY\\
{\large\bf ASTERIX Data Analysis\\}
{\large\bf \stardoccategory\ \stardocnumber}
\begin{flushright}
\stardocauthors\\
\stardocdate
\end{flushright}
\vspace{-4mm}
\rule{\textwidth}{0.5mm}
\vspace{5mm}
\begin{center}
{\Large\bf \stardoctitle}
\end{center}
\vspace{5mm}
\begin{center}
\docpicture                             % picture of ROSAT
\end{center}
\parskip=4.0truemm plus 0.5truemm       % Paragraph spacing
\markright{\stardocname}

\newpage
\tableofcontents

\newpage

\section{Introduction}
This guide is for people who want to reduce and analyse data from the ROSAT
X-ray satellite using the Starlink multi-mission data analysis package
ASTERIX. It describes how to get from observation datasets to high level
data products such as spectra, point source lists, etc.

\subsection{Software Version}
This version of  the user guide is  associated with ASTERIX version 2.3, a
beta version of which can be downloaded via the net.  

\begin{verbatim}
       http://sr.bham.ac.uk/asterix
\end{verbatim}

No other formal release plans have yet been made.   This version of ASTERIX
is supported on Solaris, Digital Unix, and Linux.

\subsection{ASTERIX Version 2.3}
ASTERIX version 2.3   includes changes which   allow the XRT   interface to
directly   access  FITS files.  The development   of full FITs compatibility
throughout ASTERIX is also well underway.

\subsection{Getting Started}
All commands described in this document are available automatically as part 
of ASTERIX. If you have ASTERIX installed at your site, it can be started
by typing

\begin{verbatim}
       > aststart
\end{verbatim}

Most ASTERIX commands can be run from the shell.  The exception to this is
the command driven version of image processing.  This requires the
user to start ICL (command: {\tt icl}), although most of the same
functions are available within Isys, the image processing GUI, which
is invoked from the shell.  All other ASTERIX commands will also run
from ICL.

\subsection{Getting Help}
This document provides recipes for specific data analysis tasks, as well
as descriptions of the data formats. It does not provide detailed help on
individual ASTERIX ROSAT commands. 

Such help is provided by the on-line help system, invoked by the command

\begin{verbatim}
       > asthelp
\end{verbatim}

{\tt asthelp} attempts to provide complete information on all ASTERIX commands,
as well as more general guidance on how to get started with ASTERIX,
where to find documentation, information specific to the current ASTERIX
release, etc..

With this release of ASTERIX, a hypertext version of {\tt asthelp} is also
available. This can be invoked with the command 

\begin{verbatim}
       > xasthelp
\end{verbatim}

Or referenced on the WWW at 

\begin{verbatim}
        http://www.sr.bham.ac.uk/asterix
\end{verbatim}

\subsection{Related Documents}

The international ROSAT project has also produced a number of useful
documents, including individual calibration memos and the ROSAT Users
Handbook. These are available from the US anonymous ftp site {\it
legacy.gsfc.nasa.gov}.  More specific references are provided in later
section where appropriate.

News items, including the national ROSAT newsletters can be accessed
via the Leicester ROSAT home page and the HEASARC ROSAT guest observer
facility on the World Wide Web.  Information about ASTERIX is
available on Birmingham University's ASTERIX home page.

\begin{verbatim}
       http://ledas-www.star.le.ac.uk/Forum/rosat-news  
       http://heasarc.gsfc.nasa.gov/docs/rosat/rosgof.html
       http://www.sr.bham.ac.uk/asterix
\end{verbatim}

\subsection{Layout of this Guide}
The following sections give information about the ROSAT satellite (section 2),
the data that you get from ROSAT (section 3), how to read a data tape and
how to get a quick idea of what your data contains (section 4), detailed
descriptions of how to analyse ROSAT PSPC and HRI data (section 5), how to
communicate with the ROSAT project and how to stay informed (section 6).
Finally, there is a number of appendices describing the dataset formats,
giving a full list of ASTERIX XRT commands.

\newpage
\section{ROSAT}
The ROSAT X-ray astronomy satellite (R\"ontgen Satellite) was launched on
June 1, 1990, and was switched off on Friday 12th February 1999. ROSAT 
is a German satellite,which also has scientific instrumentation supplied by the 
USA and the UK. There are scientific data centres at the Max Planck Institut 
f\"ur Extraterrestische Astrophysik (MPE) near Munich in Germany, the 
Goddard Space Flight Center (GSFC) near Washington DC in the US, and 
RAL/University of Leicester in the UK. 

\subsection{Project}
Following a performance verification and calibration phase, the ROSAT
satellite spent 6 months continuously scanning the sky to make an all-sky
survey. This ended on Jan 25 1991, since then observations have consisted
of pointings at individual targets. These targets are selected by
peer-review committees from proposals made in response to annual announcements
of opportunity in the three countries.

\subsection{Satellite}
The ROSAT satellite is in a 96 minute period, low earth orbit having an
inclination of 53$^o$. It is a 3-axis stabilised spacecraft with two
co-aligned telescope systems. The main telescope is the X-Ray Telescope
(XRT), having at its focus a carousel which can place one of two Position
Sensitive Proportional Counters (PSPC) or a micro channel plate High
Resolution Imager (HRI) in the beam. There is also a Wide Field Camera
(WFC) for EUV astronomy co-mounted with the XRT. This document deals
only with the analysis of data from the XRT. SUN 62 describes the analysis
of data from the WFC.

\subsection{Attitude Measurement and Control System}
The ROSAT XRT has two CCD star trackers (one of which failed early in the
mission). The WFC has an independent star tracker. The data from this
star tracker are being incorporated into XRT attitude solutions in the Rev
2 data re-processing.

During PSPC observations the ROSAT spacecraft is wobbled with an
amplitude of typically $\pm$ 3 arcminutes at a period of around 400
seconds. This avoids having the sharply focussed X-ray image of a point
source hitting a detector wire throughout the observation. HRI images are
also wobbled, but with a lower amplitude of typically $\pm$ 1.5 arcminutes.
Although the wobble is strictly periodic, the path of an image across the
detector varies from cycle to cycle.

Attitude reconstruction has not met specification. This can manifest
itself in distortion of the on-axis PSF seen with the HRI, and as offsets
of up to 30 arcseconds between the positions of objects observed after an
interval of months. 

\subsection{XRT Mirrors}
The ROSAT XRT consists of four nested co-focal Wolter type 1 gold-coated
Zerodur (glass) grazing incidence mirrors manufactured by Zeiss in
Germany. The system has a field of view larger than the 2$^o$ diameter of
the PSPC (the HRI is smaller), although vignetting is significant at large
off-axis angles, especially at high energies. The mirror system has a focal
length of 240cm, a focal plane scale of 11.64 $\mu$ per arcsecond, and a
geometric area of 1141 cm$^2$. The point spread function and effective area
is described in the sections on the PSPC and HRI detectors.

\subsection{PSPC}
There are two (nominally identical) position sensitive proportional
counters on ROSAT. PSPC C was in use up to Jan 25, 1991, after which PSPC
B only was used.

The PSPC is a multiwire proportional counter which provides both spectral
and spatial resolution. The counter gas is argon/methane/xenon, enclosed
by a 1 $\mu$ polypropylene/carbon/lexan window supported by a 
grid  with 72\% transmission. A window support structure consists of 8 
radial ribs meeting at a 20 arcminute diameter ring. The PSPC has a 2$^o$
diameter. It has effective area between 0.1 and 2.2 keV, peaking at 330
cm$^2$ just below the carbon edge at 0.28 keV.

In general, both spectral and spatial resolution improve with increasing
energy, but although the PSPC has 256 pha channels, it can make only
$\sim$ 5 independent spectral measures. The spectral resolution varies as
$$ FWHM/E = 0.43 (E/0.93)^{-0.5}$$

The PSPC detector used for observations was swapped (from detector `C' to
detector `B') on 25th January 1991, and the detector gain changed on
11th October 1991, so different combinations of response matrix and
effective area file must be used for observations taken between these dates.

The recommended combinations are: 

before 25th January 1991 - pspcc\_eff.sdf and drm\_06.sdf \\
25th January 1991 to 11th October 1991 - pspcb\_eff.sdf and drm\_06.sdf \\
after 11th October 1991 - pspcb\_eff.sdf and drm\_36.sdf \\

For further information on using these files, see the Leicester Database
and Archive Services (LEDAS) User Guide or the ASTERIX documentation. 

\begin{verbatim}
http://ledas-www.star.le.ac.uk/info/user_guide/node24.html
\end{verbatim}


The PSPC PSF encloses 50\% of counts at 1 keV in a radius of 0.22
arcminutes. A model describing the the PSF as a function of energy and
off-axis angle (within the 20 arcminute ring) is a good fit to the data.
However, at large off-axis angles the PSF can no longer be treated as
circularly symmetric, and at the lowest energies electronic 'ghosting'
occurs, significantly broadening the PSF below 0.19 keV. Neither of these
extreme cases is covered by the PSF model.

The PSPC was occasionally used with a boron filter in the X-ray beam.
This filter has little transmission between the boron edge (0.188 keV)
and the carbon edge (0.28 keV), and thus can provide extra spectral
information in the energy range where the intrinsic spectral resolution
is poorest. The boron filter only covered the central part of the PSPC
field of view.

Since the end of 1994, the gas supply to the PSPC has been exhausted.
The HRI is now permanently at the focus.  A typical PSPC image can be seen
in Figure 1.
  
\subsection{HRI}
The ROSAT HRI is an enhanced version of the HRI flown on the Einstein
satellite. It consists of two cascaded microchannel plates with a crossed
grid readout system. The quantum efficiency of the HRI is maximised by
use of a CsI coating on the front plate.
 
The HRI field of view is the intersection of a circle and a square,
having an effective radius of $\sim$ 18 arcminutes. Unlike the PSPC, it
is able to make full use of the very small mirror PSF, and has an on-axis 
50\% power radius of 3 arcseconds. However, point images also show a large
diameter, low surface brightness halo

The effective area of the HRI is less than that of the PSPC at all
energies, particularly so at the lowest energies. Like the PSPC, it has a
strong carbon edge at 0.28 keV.

The HRI has 16 pha channels, and is easily able to distinguish UV photons from
X-rays. Some X-ray spectral resolution may also be possible, however
this would be very limited. Spatial gain variations 
have not been corrected in the HRI datasets.  A typical HRI image can be
seen in Figure 2.

\newpage
\section{The ROSAT Data}

\subsection{Overview and Origins}
Data from the ROSAT XRT is processed by the SASS (Scientific Analysis
Software System) before being converted to a FITS format and then
distributed to the PI. This processing occurs at MPE for data belonging
to German and UK PIs, and at GSFC for US PIs. Both the SASS and the FITS
conversion software have evolved throughout the mission, but more
fundamentally, the FITS conversion up until May 1994 was different at MPE
from that at GSFC. (This corresponds to SASS versions earlier than 7\_1.) 
Data distributed after that date was FITS converted
using yet another form of the FITS standard, `Rationalized Data Format'
(RDF). This FITS conversion is now common between MPE and GSFC.

Starting at    around the  same  time  as   the FITS    software change,  a
re-processing  of all data using  the evolved SASS   was started.  This has
become   known as the    `Rev   2 re-processing'.  The   output   from  this
re-processing are   RDF files.   Data in   the Leicester   public  archives
currently consist of  a mix of RDF  and pre-RDF FITS files.  As of Feb 2000
around 3 percent of  the old datasets   remain. It is, however,  anticipated
that the database will be completely RDF within a matter of months.

ASTERIX  2.3-0  reads RDF FITS  files directly.   Earlier  ASTERIX versions
converted  all FITS format   files  to HDS files    using the program  {\tt
  xrtconv}.  {\tt xrtconv} has been withdraw in version 2.3-0.

\subsection{The SASS}
The Standard Analysis Software System converts the ROSAT telemetry into
scientifically useful files. It performs standard scientific operations on
these files, and generates postscript output summarising the observation.
The data in the files that make up a ROSAT XRT observation dataset have all
been through the SASS, however, the SASS does not make the FITS files that PIs
and archive users receive, this is done by later software.

The SASS was written by MPE and SAO. Identical copies run at MPE and
GSFC, although it has been continuously updated throughout the mission.  
A brief description of the SASS follows, however, the details of what
specific versions of the SASS have done have proved difficult to uncover.
What is clear is that later versions of the SASS produce significantly more
reliable results than earlier versions. Confidence has risen to the point
that with SASS version 7\_1 the entire set of observations to date are being
re-processed (the so-called `Rev 2 re-processing').

The following modules make up the SASS.

SAV -- copies telemetry tape to disk, deleting duplicate records, and
separates PSPC from HRI data.

TEL -- splits data into continuous observing intervals and identifies
these with the timeline, assigns quality flags to detector and aspect data.

OBI -- applies the aspect solution and screens out bad data. Determines
components of background map.

SEQ (PSPC) -- merges observation, makes images, performs source detection
and identification, makes spectra and light curves and applies
statistical tests to these.

SEQ (HRI) -- as above, but no spectra or scientific light curves made.

\subsection{FITS Conversions}

The User's Guide to  ROSAT Data in the   Rationalized Data Format  (RDF) is
available to download from :

\begin{verbatim}
      http://ledas-www.star.le.ac.uk/rosat-goc/doc.html
\end{verbatim} 

As the conversion of archive datasets into the RDF file format is soon to be 
completed the following section is now only useful as background.

\subsubsection{The Background to FITS File Conversions}
The ROSAT datasets have always been distributed as collections of FITS
files, however there have been three different sets of file formats
within the  FITS convention. The MPE and GSFC FITS files represent the first
generation data format. They were developed independently, and were read by
the respective national analysis packages.  Datasets from the two sites
contained essentially the same information, although the GSFC datasets
included files that had no equivalent in the MPE datasets. UK and German
PIs had data in the MPE format, US PIs had data in the GSFC format.

The fact that there were two data formats, together with the
deficiencies of the individual FITS styles, led to the definition of the
a format common to both MPE and GSFC. This is the `Rationalised Data
Format' (RDF), which is described below. The switch between first
generation FITS files and RDF FITS files took place in April 1994. In
addition to current PI data, all prior data is being re-processed by the
SASS and converted to RDF files.   

Detailed descriptions of the files making up the datasets in the three
formats are given for both the PSPC and HRI in appendix 1. This appendix
also shows the correspondence between FITS files and HDS files. 

\newpage
\section{Getting a Quick Look at your Data}
In this section you will find a description of how to get from the your
data tape or archive dataset useful images, spectra and light curves of
point sources.

\subsection{How to get your Data onto Disk}

\subsubsection{Reading a Data Tape}
The UK ROSAT Data Centre sends out data tapes to PIs. Until May 1994,
these tapes were always in VMS backup format. After that date, the
default format became unix tar.

You can read the unix data tapes to disk with the command

\begin{verbatim}
       > tar -xvf /dev/rmt0l
\end{verbatim} 

where /dev/rmt0l is the device filename for the tape drive.  Extracted
files will be placed in the current directory.

\subsubsection{Accessing the On-line Public Archive}
ROSAT data enters the public archive 1 year and 2 weeks after it was first sent
to the PI. Rev 2 data enters the archive 6 months after distribution.

UK investigators can easily get archive datasets from the on-line service
maintained by the Leicester Database and Archive Service (LEDAS). Access
is described in detail in The LEDAS User Guide but will be describe very
briefly here.

\begin{verbatim}
        http://ledas-www.star.le.ac.uk/info/user_guide
\end{verbatim}

The archive service is currently run from a DEC Alpha and is accessible via
ssh (or telnet) and the WWW.

Connect to the LEDAS machine:

\begin{verbatim}
        > ssh ledas.star.le.ac.uk
\end{verbatim}

Use the username {\tt xray}, no password is required. After giving your
name, you will see a prompt. Type

\begin{verbatim}
        ledas>  browse rospublic
\end{verbatim}

You will then be using the Browse Database Management System, which has
been described in a Users Guide (ESA TM-11, distributed to all Starlink
sites as a MUD). The system is not described here, however, if you know
the position or name of your target, or the original PI name, you can
issue one of the following commands: 

\begin{verbatim}
        rospublic_total_dec> sc 10 20 30 +20 30 40
        rospublic_total_dec> sn crab
        rospublic_total_dec> sif pi smith
\end{verbatim}

These perform searches, the results from which can be seen with the
display sample command {\tt ds}. Once the dataset required is identified,
the command {\tt request} will cause the dataset to be read from the CD
jukebox to a staging area from which you can copy the data over the net
to your home site. The system will mail you when the data is ready for
copying, usually within 5 minutes of the {\tt request} command being
issued.

An alternative method using the WWW at Leicester and the {\it ARNIE}
interface can be found at:

\begin{verbatim}
        http://ledas-www.star.le.ac.uk/arnieV4
\end{verbatim}

\subsection{A Quick Look at the Data}
With your data on a local disk  you are now ready to run ASTERIX on it.
Start ASTERIX with the command:

\begin{verbatim}
        > aststart
\end{verbatim}

If you are using a version of ASTERIX earlier than 2.3-0 you will need to 
convert your FITS data into HDS files using the command:

\begin{verbatim}
        > xrtconv
\end{verbatim}

Begin by getting a summary of the observation with the command {\tt showxrt}.

ROSAT data is in the form of event tables, that is a logical record
containing detection attributes such as position, time, etc. Most
analysis cannot make use of the data in this form, but requires uniformly
binned data. The ASTERIX command to bin data is {\tt xrtsort}. With this
you can make images, spectra, light curves, etc. It's probably
best to make an image first. From this you  can select regions for
further study using {\tt Isys}.

{\tt Isys} or the  command driven image  processing can be used to  inspect
the image.  Specific regions can then be resorted either by entering simple
parameters  into {\tt xrtsort}, such  as positions and  radii of circles or
annuli, or a  regional description can be exported  from {\tt Isys}  in ARD
format (Askii Region Description) then  entered into xrtsort.  Subsequently
the  background   can be  extracted  with  {\tt  xrtsub},  calibrations and
corrections applied with {\tt xtrcorr}, and an energy response applied with
{\tt xrtresp}.  ASTERIX spectral fitting can then  be used directly, if you
have it installed, or the  command {\tt ast2xsp}  can be used to format the
data to a form suitable for {\tt xspec}.

More detailed guidance on data analysis techniques appears in subsequent
sections of this guide.

\newpage
\section{Detailed Analysis}

\subsection{PSPC}

The following is a brief overview of the basic reduction sequence, the
details are shown in the later subsections.

{\bf ROSAT} data may be obtained from a datatape or from the Leicester
data archive (LEDAS).  In either case the raw data should be copied
into a single directory.  It is important that this directory does not
contain any raw datafiles from another observation as this can cause
problems with the pre-processing.

\begin{itemize}

\item{Users of ASTERIX earlier than version 2-3-0 will need to apply 
{\tt xrtconv} to the raw dataset to convert it into ASTERIX format.}

\item{{\tt showxrt} will produce a summary of the observation, listing
information such as target name, pointing direction and exposure time.}

\item{The program {\tt xrthk} may be used to "clean" the data by
producing a file of times when the background was low.}

\item{{\tt xrtsort} can now be used to sort the data directly into an
ASTERIX data file.  If a file of time slots has been created with {\tt
xrthk}  this can be used in  {\tt xrtsort} when  sorting the data (see
Example 2).  A background source  file for input into {\tt xrtsub}  can
also be selected from a source free area of sky  with {\tt xrtsort} by
selecting  the BACK parameter.    Example 1 shows {\tt  xrtsort} being
used to create an image} .

\item{The data can be background subtracted with the program {\tt
xrtsub.} {\tt  xrtsub}  uses a  background file containing  a  selected
source free area to create a model background 
for the  whole dataset. The model  background can then be
subtracted from the source to create a background subtracted source.}

\item{{\tt xrtcorr} is used to exposure correct a dataset.  This
includes corrections for instrumental effects, such as vignetting,
dead time as well as normalisation to on-axis counts/sec.}

\item{A detector matrix needs to be produced with {\tt xrtresp} before
spectral fitting can be performed.  To use {\tt xspec} convert the
files using {\tt ast2xsp}.}
\end{itemize}

\subsection{Background Subtraction}
This is a complex problem and needs to be thought about carefully before
proceeding. The PSPC background consists of a cosmic X-ray background,
a particle induced detector background, solar scattered X-rays,
afterpulses, long and short term enhancements and auroral X-rays. 
A full description is given in Snowden et al, 1993.

The particles contribute $\sim2-3\%$ of the background on average 
($\sim 5\%$ for an observation with the boron filter), with their major
effect being above 1 keV. The needs of background subtraction are as
follows:

{\bf 1.} A representative region of background has to be selected 
from the observation using the {\tt Isys} GUI or the command driven image
processing commands.
The background region should satisfy the following criteria:

\begin{itemize}
       \item{Large enough to give good statistics}
       \item{Free from sources}
       \item{Should not include the ribs or central ring}
       \item{As near to source as possible, to avoid ramps in the background}
\end{itemize}

{\bf 2.} The background data need to be corrected for vignetting effects
before subtraction. This is energy dependent and is complicated by the fact
that the particles in the background have a different radial dependency 
to the photons (i.e. they are not vignetted).

{\bf 3.} The background has to be normalised for the area of the source
and background boxes before subtraction.

{\tt xrtsub} calculates: 
\begin{math}
SUB = SRC - BCK \times \frac {A_{s}} {A_{b}} \left[
B_{phot} \frac {V_{s}} {V_{b}} + B_{part} \frac {P_{s}} {P_{b}} \right]
\end{math}

where A is the area, $B_{phot}$ the photon fraction of the background, 
$B_{part}$ the particle fraction, V is the vignetting function and P is the
position dependence of the particles. The subscripts 's' and 'b' refer to
source and background respectively.


The program {\tt xrtsub} caters for points 2 and 3, so the user only has to
worry about point 1. The program ignores source or background pixels with
bad quality and this feature may be used to create background regions with
complicated shapes, which satisfy the requirements of point 1. e.g. an
annulus at a large radius with sources and the ribs removed. See the 
section on subtracting extended sources for a recipe for creating such a
background file.

The particles in  the background are calculated using  a model developed in
Snowden et al. 1992  and updated by Plucinsky  et al. 1993, which gives the
particle flux as a function  of energy, time  and detector position. Over a
single  observation this model  is entirely  dependent  on the housekeeping
parameter   master\_veto\_rate,  which is  contained  in an    array in the
eventrate file. The model is well constrained for values between 0 and 170,
and this constraint should be applied to the data  using {\tt xrthk} before
attempting  a   difficult subtraction.   The model   parameters  vary  on a
timescales of months.   This effect is   taken into account  by providing a
different set of model parameters depending on the date of the observation.

\subsubsection{Point Source Subtraction}
The background subtraction of point sources tends to be relatively easy
because of the low intrinsic background of the PSPC.  Use {\tt Isys}
to determine a suitable background area and then sort with {\tt xrtsort}.
This should be either an annulus around the source, with an inner radius of
at least 3 arcminutes, or if source confusion makes this impossible, a
circular box as near as possible to the source. The background box should
be as large as possible without including any sources or intersecting the
window support structure.  If a spectrum or time series is being
subtracted, the background datafile must have exactly the same range of
spectral bins and time bins as the source file.

\subsubsection{Extended Source Subtraction}
This is a much trickier problem. The {\tt asthelp} section 
"{\tt xrtsub} advice extended\_sources" gives up to date advice on the best 
subtraction method. The subtraction of an image provides a good illustration 
of the current best technique. Let's say a background subtracted image of 
the whole field between corrected pulse height channels 11 and 250 is
required. First, {\tt xrtsort} should be used to make a {\em spectral\_image}
over this channel range with about 6 spectral bins. A few spectral bins
need to be used to adequately sample the vignetting function which is energy
dependent.
(It is unnecessary to use more than about 10 spectral bins because of the
moderate energy resolution of the instrument. Indeed using a large number
of energy bins can cause problems due to irregularities in the vignetting
function). 

The background area needs to be large but constrained within a narrow range of 
off-axis angles (so that a single vignetting correction may be applied to
the data). The best shape to satisfy these criteria is an annulus at a
reasonably large off-axis angle (e.g. $r=0.6^o - 0.7^o$). To produce such
an annulus with sources and window support structure removed, use the
following procedure:

\begin{itemize}
       \item{Mask support structures using {\tt xspokes} and {\tt ardqual}.}
       \item{Select your background region using {\tt iregion annulus}.}
       \item{Export the remaining areas using {\tt iregion invert} and
{\tt iregion export}.}
       \item{Mask the un-selected regions using {\tt ardqual}.}
\end{itemize}

{\tt xrtsub} may then be run on the source and background spectral images.
The output spectral image can be compressed into an image using
{\tt project}. Alternatively it may be {\tt project}ed into a spectrum for spectral work.
{\em NB:} DO NOT compress the background spectral image into a
spectrum before background subtraction as this will result in the 
spatial information being lost and the program will be unable to normalise for 
the box areas.

The above procedure is fairly robust against spatial variations in the
background.  One problem is that beyond the central ring, it becomes harder
to resolve point sources from the background. This means that faint
sources may be included in the background region when they would have been
ignored in the middle. This can cause a slight over-subtraction and be a
problem for very accurate work. No data should be used beyond a radius of
50 arcminutes due to inaccuracies in the vignetting function at these large
radii. 

\subsubsection{Multiple Point Source Subtraction}
The situation where there are a large number of point sources within the 
central ring, such as a 'deep survey' field, has many similarities to the
case of the extended source. Choosing a nearby background region for each 
source is often impossible because of source confusion and anyway
is time consuming. One method of subtracting these sources is to use an
annulus at a large radius as described above. However, this method may
cause over-subtraction and since many of the sources are likely to be
faint, this can be a serious problem.

An alternative method is to use background from within the central ring.
Although much of the area may be contaminated by sources, there is usually
a reasonable amount of blank sky available. The method for producing such
a background file is very similar to the annular background described in the
'Extended source' section. 

\begin{itemize}
   \item{Produce a square spectral\_image with sides just larger than
the circle required$^{\star}$} 
   \item{Select the regions to be excluded using {\tt iregion}.}
   \item{Export the regions into an ARDFILE using {\tt iregion export}.}
   \item{Use {\tt ardqual} to remove the sources.}
\end{itemize}

$^{\star}$ Never {\tt xrtsort} directly to a circular or annular
spectral\_image.  The pixels at the edge of the circle will be only half
filled and there is no way of subsequently correcting for this in the software.

An example of this method is shown in Figure 3.

A restriction of this method is that a single vignetting correction has
to be applied to the whole background file. This means that the range of 
off-axis angles that can be used with this method is limited. A circular 
background region with off-axis angles 0--12 arcminutes or an annular box with 
angles 10--17 are acceptable. 

When removing   sources from the   background region, careful consideration
must be given to the point spread function of the instrument. Very high and
very  low energies are the  most affected  by photon  scattering  and so if
sources  are  inadequately  removed the   spectral  characteristics  of the
background will be strongly biased. A description  of the PSPC PSF is given
in Hasinger et al.  1992. Faint sources ($<  100$ counts) can be adequately
excluded with a circle  of 1--2 arcminutes radius, centered  on the source. 
Brighter sources should be treated individually. {\tt  Isys} can be used to
obtain a radial profile to ascertain the distance from  the source at which
the source counts  fall below the  level of the background.  Photons should
be excluded from a circle of this radius  centred on the source.  Its often
a good idea to smooth the image, before extracting the profile.

\subsection{Instrument Corrections}
\subsubsection{Spatial/Temporal Gain Correction}
The PSPC suffers from both spatial and temporal gain variations.  An
account of this can be found in {\it OGIP Calibration Memo CAL/ROS/95-003}.
ASTERIX doesn't currently take these into account, however, there is an
ftool, {\it pcpicor} which can be applied to the raw FITS data.
This will only work for data in RDF format.
\subsubsection{Exposure Correction} 
The program {\tt xrtcorr} performs exposure correction on binned datasets. The 
effects listed in the following table are corrected for:

\begin {table}
\caption{Correction factor dependencies}
\label{t:corr}
\begin{center}
\begin{tabular}{|c|c|c|c|}
\hline
  Correction    &  Energy   & Time  &  Position \\ \hline
Vignetting      &   YES     &  NO   &    YES    \\
PSF scattering  &   YES     &  NO   &    YES    \\
Filter absorption &   YES     &  NO   &    NO   \\
Dead time       &   NO      &  YES  &    NO     \\
Wires           &   NO      &  NO   &    NO    \\
\hline
\end{tabular}
\end{center}

\end{table}

All these effects are applied directly to the {\em data} when exposure
correcting a time series or an image, however, only the dead time correction
is applied to a spectrum; in this case the energy dependent corrections 
and the effect of the wires are applied to the detector matrix. 
For a time series or an image, the energy dependent 
corrections, vignetting,
PSF scattering and filter absorption, are applied assuming a mean energy.
The default energy for this correction is 0.2 keV but it may be overridden
on the command line, by :

\begin{verbatim}
        > xrtcorr energy=1.0
\end{verbatim}

The vignetting and filter corrections are read from the effective
area and detector matrix files held in the \$XRTCAL directory. The PSF
correction is calculated from an analytical function and is a strong
function of off-axis angle as well as energy. The wire correction is a
single value (1.27) and dead time is calculated from count rate values
stored in the eventrate file and is usually $\sim 2\%$.

The vignetting, PSF and filter corrections are {\em ENERGY} dependent
and {\em NOT} pulse height dependent. This means that they cannot be
directly applied to the data in a {\em spectrum} but rather must be applied
to the detector matrix used in spectral fitting.
The data in a spectrum are simply multiplied by the dead time factor
and divided by the exposure time during exposure correction. The wire
factor is not applied because this is also contained within the detector 
matrix.

Images and spectral\_images are corrected for dead time and vignetting.
The vignetting correction applied to an image assumes a mean energy
and so is not accurate. The correction applied to a spectral\_image,
uses the approximate energy to pulse height bin mapping of 
$E=(Ch.No.) / 100$ keV. This again is not accurate and under no circumstances
should a spectral image be corrected with {\tt xrtcorr} and then used for
{\em spectral} work by {\tt project}ing the file into a spectrum.

\subsection{Spectral Analysis}
PSPC spectra may be produced over the channel range 1-256.  However the first
7 channels should be ignored in the spectral analysis of sources observed
before 11th October 1991. On this date, the operating voltage of the
detector was reduced in an attempt to lower the detector background. This
led to the first 10 channels becoming useless and so analysis of
observations post 11th October 1991 should ignore the first 10 channels. 

\subsubsection{Single Point Source}
This is the simplest case, a circular box should be used in {\tt xrtsort} to
extract the source counts into a spectrum (BIN\_AXIS=7). There is one 
subtlety which must be 
considered. The PSF of the PSPC is very energy dependent (Hasinger et al 1992),
e,g, on-axis the 95\% radius of a point source is 0.65 arcminutes at 
0.5 keV, but 8.8 arcminutes at 1.7 keV. Below $\sim 0.3 keV$ (channel 30),
PSPC images suffer from an effect known as 'electronic ghost imaging'.
This causes events to be recorded at the intersection of anode wires rather
than at their true position and results in the image appearing like a
rectangular grid rather than a point source. This effect varies
from source to source and so is impossible to correct for in the software.
It has been found that a source radius of 3 arcminutes usually includes
most of the counts at all energies and produces good results as a rule.
If low energies are not important and/or the source is very faint, this
can be reduced to $\sim 1$ arcminute to improve the signal to noise.
If very low energy counts are important for a particular source, e.g.
if it is a very soft source, then the radius which includes most of the soft
counts needs to be found. This may be done by displaying the {\em low energy}
image with {\tt idisplay}, centroiding on the source and producing a radial
profile with {\tt iradial}. The extraction radius should be set to the point
where the source profile merges into the background.

Having produced the source spectrum, it should be background subtracted
and exposure corrected as discussed earlier. After this a detector matrix
needs to be created using the program {\tt xrtresp}. The detector matrix is
unique for each source, being a function of off-axis angle and source
box radius. It is written directly into the spectral datafile. This file
may be used directly in the ASTERIX spectral fitting package, {\tt sfit}, or
within {\tt xspec} after format conversion by {\tt ast2xsp}.

For weak or very soft sources, modelling the particle component of the
background correctly is important. In this case the data should be
"cleaned" before sorting by only using times when the Master Veto Rate
parameter is between 0 and 170. 

\subsubsection{Extended Source Single Spectrum}
Extended sources have higher background and so care needs to be taken
when performing the background subtraction. To help reduce the background
present in the data it is possible to "clean" the data using the Master
Veto Rate parameter, contained in the eventrate file. 
The program {\tt xrthk} may be used to generate a text file of time slots,
for when the MVR value is between 0 and 170. This typically 
results in a reduction in exposure time of about 10 \% but can significantly
reduce the background. The particle model of Snowden et al 1991, used
in the background subtraction assumes that the MVR is no greater than 
170. The text file generated by {\tt xrthk} can be used directly in
{\tt xrtsort}
to specify the time ranges wanted (see Example 2). The same time ranges
{\em must} be used when sorting the background file for the observation.

When analysing part or all of an extended source, e.g. the inner 0.1
degrees of a cluster of galaxies, there are no PSF considerations
when choosing the source box. 
However, the source box must be small enough that a single
vignetting correction (at a single off-axis angle) is appropriate for
the file. A reasonable size source box would be a circle of radius 
0.2 degrees within the inner ring, or an annulus of width 0.1 degrees 
outside the central ring. If a larger box is required, see the 
section on "radial\_spectra".

It is not always obvious which off-axis angle to use for the vignetting
correction of a large source box, centred on the optical axis. 
By default, the off-axis angle is set
as the half radius of a circle or the mid-radius of an annulus. For a point
source with a small source box this is not important, however, with a
large source box this may not be appropriate. For instance, if a circular 
source box was used with a uniform distribution of photons, then the
photon weighted mean off-axis angle would be $R / \sqrt{2}$, where R is
the radius of the box. On the other hand if the photon distribution is
strongly centrally peaked, then an off-axis angle less than the half
radius would be appropriate. In general with a smooth decrease in the
source radial profile, the half radius is the best value to use for the
mean off-axis value, however this default can be overridden on the 
command line if necessary with:

\begin{verbatim}
      > xrtcorr offax1=8.2
\end{verbatim}

Where the off-axis value is in arcminutes.

\subsubsection{Radial Spectra}
To save time analysing different sections of an extended object, it is
possible to produce a single file containing spectra at a number of
different radii. For instance a circular cluster, centred on the optical
axis, of extent 0.5 degrees, could be divided into 5 radial rings 
of width 0.1 degrees. To produce such a source file select a circular source
box in {\tt xrtsort} of radius 0.5 degrees, and select axes 7 and 8 at the
BIN\_AXIS prompt. As with extended sources, it is best to subset the
data by restricting the Master Veto Rate range to 170.

The subtraction should be performed as for an extended source, e.g. an
annular background region of radius 0.7--0.8 degrees might be appropriate.
{\tt xrtsub} and {\tt xrtcorr} will then background subtract and exposure
correct all the spectra in one go. 

If {\tt xspec} is used for spectral fitting, each spectrum may be converted
individually with {\tt ast2xsp}. The ASTERIX spectral fitting software will
work directly on a radial set of spectra, the desired spectra may be
selected with {\tt sdata} before using {\tt sfit}.

\subsubsection{Spectral Analysis of the Background}
To find the spectra of the cosmic background in a PSPC image,
the following procedure may be adopted. It is probably best to just
use data from within the central ring for this purpose to ensure
uniform point source rejection to a well defined threshold.

{\bf 1.} Use {\tt xrthk} to restrict the master veto rate to 170. 

{\bf 2.} Use {\tt xrtsort} to produce a square spectral\_image which includes 
    the inner ring. The spatial binsize should be about 15 arcseconds;
    any number of spectral channels may be used, but 32 or 64 is
    possibly best. A 256 by 256 by 32 spectral channel file should
    produce good results.

{\bf 3.} Use the {\tt quality} setting programs to set pixels outside a 
    radius of upto 17 arcminutes BAD and blank out the sources 
    and window support structure from the spectral image (this
    procedure is described in the section on background subtracting
    multiple point sources). 

{\bf 4.} The spectral\_image can now be compressed into a spectrum
    using {\tt project}.

{\bf 5.} Remove the particles (and detector noise) from the spectrum.
    This is accomplished by using {\tt xrtsub} to produce a spectrum 
    of the particle background by :

\begin{verbatim}
        a) xrtsub spectrum spec_image part_file=parts

           This just uses xrtsub to generate the particle file, throw
           away the subtracted file that it produces. This assumes that the
           spectral file is called spectrum.sdf and the spectral image,
           spec_image.sdf

        b) subtract spectrum parts

           Use to subtract the particles from the background spectrum
\end{verbatim}

{\bf 5.} Exposure correct the spectrum with:

\begin{verbatim}
        > xrtcorr offax1=n.n
\end{verbatim}

{\tt xrtcorr} will use an inappropriate default for the off-axis angle of this
spectrum. It must be supplied with the area weighted mean off-axis
angle of the photons, this is $R/\sqrt{2}$, where R is
the radius of the circular region specified with the quality setting
programs. OFFAX1 is in arcminutes.

{\bf 6.} {\tt xrtresp} will attach a detector response structure using
instrument information calculated by {\tt xrtcorr}.  The default matrix should
be used with observations after mid 1991.  

\subsubsection{Spectra of many Sources}
A common requirement in ROSAT analysis is to improve the signal to 
noise ratio by combining spectra from many similar sources in the same
field, e.g. AGN in a deep exposure field. Strictly speaking this is 
incorrect because a separate detector matrix is required for
each source, depending 
on the off-axis angle and the source extraction radius. However, if this
point is born in mind and the analysis is restricted to sources within
the central ring it is possible to use the technique.

Each source required should be extracted using a circular box with a
fixed radius. If all the sources are faint then {\tt xrtmspec} may be used
to extract each spectrum. This procedure uses an extraction radius of
0.01 degrees which is too small for bright sources ($>100$ counts). If
one or more bright sources are present, then {\tt xrtmspec} should be modified
to use a larger radius or each source should be extracted separately with
{\tt xrtsort}. Each source {\em must} be extracted with the same radius.

The spectra may be added together using:

\begin{verbatim}
        > meandat average=no
\end{verbatim}

This program adds the counts from each input spectrum. 
Subsequent programs will use the
off-axis angle from the first file used in {\tt meandat} to calculate the 
energy dependent corrections for the combined spectrum, so it is 
important that the first datafile is at an average 
off-axis angle for all the sources.
The resultant datafile may be subtracted and corrected as described earlier.

\subsection{Time Series Analysis}

While we would  fully expect the Time Series  Analysis component of ASTERIX
to function as described in this guide, the User should  be aware that 
the use this software has now been superseded, and that there are
no further plans for the development of Time Series software within ASTERIX.

ROSAT flies a near circular low earth orbit. The drawback of
such an orbit is that periodically the satellite must stop observing while
it passes through regions of high particle background such as the South 
Atlantic Anomaly. A {\em ROSAT} time series of greater than 30 minutes duration 
may have gaps in it due to this effect; this causes     difficulties for
traditional time series analysis routines. The ASTERIX document, USER\_006, 
provides a summary of the routines within ASTERIX which have
been designed to overcome this problem.

Analysis of PSPC data is further complicated by the window support
structure. The detector is continually wobbled while observing 
to ensure that sources are not continuously hidden behind the 
fine wire mesh of the 
support structure. This means that a source will continually pass under 
wires and its flux will be modulated at the wobble period, (about 
six minutes). Sources outside the central ring are less prone to this effect
because of the wide PSF at high off-axis angles. 

A time series may be background subtracted if required in a similar
manner to the spectral datasets mentioned earlier. It is likely that 
several of the time bins in the file will be only partly exposed and so 
it is important that the data is exposure corrected with {\tt xrtcorr} before
being analysed. 

Due to the nature of ROSAT observations there can be very long
gaps between data in a time series. To save disk space and processing time,
the gaps in the time series may be removed with the program {\tt streamline}.

The program {\tt lombscar} provides a good technique for searching for peaks
in the power spectrum of 'gappy' data.

\subsubsection{Barycentric Correction}
There is currently no way to correct for the motions of the satellite and
the Earth from within ASTERIX.  However there are two FTOOLS available
{\tt abc} and {\tt bct}, wrapped up in a script {\tt rosbary} which may be
of some use.  These programs will not work on old MPE style datasets.

\subsection{Image Analysis}
Images may be sorted over any energy range with {\tt xrtsort}. Occasionally,
observations are incorrectly processed by the SASS pre-processing
software and aspect errors are introduced. This is often the result of
a poor attitude solution from the star-trackers onboard ROSAT
and results in photons appearing in a different part of the image
from where they were recorded. A sign that this may have happened
is if sources appear to be elongated along the wobble direction.
If it is suspected that an observation contains aspect problems or if
very accurate work, e.g. a search for source extension, is being undertaken,
the problem can be reduced by subsetting the data on the aspect error
(ASP\_ERR) housekeeping parameter, contained in the attitude
file. {\tt xrthk} should be used to
produce a file of times when the aspect error parameter was between 0 and 1.
An image should then be produced with {\tt xrtsort}
between these times. At the same time, the Master Veto Rate parameter
may be used to reduce the background in the image.

The image may be background subtracted using the methods outlined in 
the "Background subtraction" section. 

{\tt xrtcorr} may be used to exposure correct an image. This includes a vignetting
correction, but it should be remembered that this correction
when applied to an image is only approximate because 
vignetting is calibrated in terms of energy and not pulse height channel.

The image may be manipulated in a wide variety of ways using {\tt
Isys} and the command driven image processing, within ASTERIX. 
See "{\tt asthelp} image" for details.

\subsubsection{Energy Mapping}
PSPC data contains energy as well as spatial information. It is possible
to indicate this on an image by producing an {\em energy map}. This may
be done in two ways :

{\bf 1.} Produce a spectral image, with 4-8 spectral channels. \\
{\bf 2.} Run {\tt enmap} with a suitable threshold (See the help entry). \\
{\bf 3.} Plot the result with {\tt idisplay}. \\

or by producing a 3-colour image, \\

{\bf 1.} Produce a low energy (chns: 8-40), medium energy (41-100) 
and high energy (101-200) image of the same region. \\
{\bf 2.} Combine these with {\tt mkcolim}. \\
{\bf 3.} Display after using {\tt icolour rgb}. \\

The first method is suitable for low signal to noise data whereas the 
3-colour image method works well on bright objects.

\subsubsection{Radial Profiling}
A radial profile of a source may be produced with the {\tt iradial}
command. If the region of interest is contaminated by another source
or intersects the window support structure, the image may be cleaned 
before starting the analysis. A typical procedure may be :

\begin{verbatim}
        icl> idisplay im1           ! Display the image 
        icl> iregion                ! Choose the areas to set bad 
        icl> iexclude curr          ! Set these areas bad 
        icl> iradial                ! Produce the radial profile
        icl> isave1d im1_radp       ! Save the radial profile 
        icl> exit
\end{verbatim}

The saved radial profile may be analysed in detail with {\tt qdp}
if required after using the {\tt astqdp} command.

The above example uses the command driven image processing commands.
The same sequence, including the export to {\tt qdp}, can be performed
purely within {\tt Isys}.

\subsubsection{Test for Extension}
A source may be tested for extension using the following procedure:

\begin{verbatim}
        icl> idisplay
        icl> icentroid            ! Find the centre of the source 
        icl> imode key            ! Switch to keyboard mode 
        icl> iradial              ! Produce a radial profile 
        icl> ipsf                 ! Compare with the expected PSF 
\end{verbatim}

{\tt ipsf} allows several PSF functions to be compared against the 
radial profile. The PSPC options may be accessed by selecting XRT\_PSPC.
The best options at present are, ONAXIS\_3 if the source is on
axis or VARPROFILE if it is off axis. The routine gives a $\chi^{2}$
value for the quality of the fit.

NB: if a source appears to be extended, the image should be
sorted over times when the aspect error value was low, to minimise 
possible aspect errors, and the above procedure repeated.

\subsection{Count Rates and Source Searching}

\subsubsection{Quick source counts}
Quick source counts and count rates may be extracted from a PSPC
observation with the image processing system. e.g. 

\begin{verbatim}
        icl> xrtsort              ! Sort an image
        icl> iload image xwin     ! Start the image processing system
        icl> idisplay             ! Display the image 
        icl> icircstats           ! Find the counts in a source
\end{verbatim}

\subsubsection{Accurate source counts}

To determine accurate source counts, the source searching routine, 
{\tt pss} should be used. This routine provides a sensitive means of 
finding the sources in an XRT observation. 
Full details are given in the help and in the document,
USER\_004. It has a large number of operational modes but the
following method works well for finding sources and obtaining count rates
from a PSPC image.

{\bf 1.}  Begin by creating a  spectrum of a  blank  region using {\tt
xrtsort}. The source free area  (ie. a portion of background) selected
will  be used by {\tt  xrtsub}  to create  a model  background for the
whole dataset.

The lower  and upper bounds  in channels {\em must} be {\em
exactly} the same as used  to create the  image to be searched by {\tt
pss}, and  the  spectrum should contain   between 4 and 10   bins. For
instance if  the image was sorted over  the corrected PH  range 8-247,
then the spectrum must be produced over  the same 240 channels and may
consist of 4,6,8, or 10 bins.

Now produce a background model image using {\tt xrtsub}.

Invoke {\tt  xrtsub}  inputting the  image you want  to  search at the
SOURCE   prompt,  and your      background  spectrum at  the    BCKGND
prompt. Supply the program with a file name for the background  model which you wish
to create using the BGMODEL parameter.  {\tt  xrtsub} will also output
a  background subtracted copy  of   the SOURCE  image  using the   OUT
parameter.

{\tt xrtsub}  creates a model of  background by scaling the vignetting
function as a function of energy using  a {\em source free} portion of
the  input  data and  adding an estimate   of  the particle background
(which is not vignetted).

{\tt pss} can accept either the raw image or the background subtracted
image.

{\bf 2.}  Run  {\tt  pss}.  At the  PSF  interface, select POLAR  mode
option, and use  the XRT\_PSPC VARPROFILE option as  the PSF. Give the
background    image generated  by  {\tt  xrtsub}   above  as the model
background at  the BGND parameter.  See Example 3.  {\tt pss} produces
two  outputs, a significance map and  source list.  The first of these
is optional  but useful in diagnosing problems.   It is a map over the
image region searched  of the likelihood ratio  of the two models {\em
background  only} and {\em   background  + point source}.   {\tt  pss}
searches  for  points  in this map  above   a user supplied threshold,
identifies  these as sources   and finds confidence  intervals on  the
parameters describing the source, ie.  flux, X and Y.  The flux output
is in raw counts and takes  into account the fact  that not all of the
psf may be included in the search box.

{\bf 3.} The source   list from {\tt   pss} may  be displayed  to  the
screen, file or printer using the {\tt ssdump} command.

{\bf 4.} The source  counts in the list  produced by {\tt pss} may  be
exposure corrected using {\tt xpsscorr}.

{\tt xpsscorr} can be used to convert  the source counts calculated by
{\tt pss} into a count rate. This  is the count  rate which would have
been observed had the source been on axis. It does this by dividing by
the exposure   time and by using  calibration  files to  determine the
vignetting.  By default it also   multiplies  by a factor 1.0/0.79  to
correct for the absorption by the fine wires in the support structure.

Alternatively the  merged  exposure map which   is provided with  each
observation  may be used to calculate  the exposure  time. This latter
approach has the advantage of coping  with sources which are partially
obscured by  the window support structure,  however, it  is only valid
for images produced  over the full time  range and over the full pulse
height channel  range. Any subset of  these quantities will invalidate
the use of the exposure map.

With    either method,  the wire    correction may be   turned  off by
specifying "wires=no" on  the command  line.  If the count rates   are
going to be  subsequently converted  into  a flux using the  effective
area files or used within spectral fitting, the wire correction should
NOT be applied as it is already included in the detector matrix.

{\bf  5.} Source positions can be  marked on the displayed image using
{\tt  imark}, or added  as  permanent annotations in  the presentation
graphics using {\tt ssanot}.

{\bf 6.} Time series or spectra can be extracted for every source in a
{\tt pss} list using the {\tt  xrtmtime} and {\tt xrtmspec} procedures
respectively.

{\bf 7.}   The whole process  can be  performed iteratively. To refine
your  background estimate those regions  of   the image input to  {\tt
xrtsub} containing  sources can be  masked out using {\tt iregion} and
{\tt iexclude} or {\tt Isys}.   This produces a background unbiased by
the  presence of  point  sources which can then   be used in {\tt pss}
again.

\subsection{HRI}
Before starting to  analyse HRI data you should  obtain a  copy of the
calibration report :

\begin{verbatim}
      ftp://legacy.gsfc.nasa.gov/pub/rosat/doc/hri/HRI_report
\end{verbatim}

HRI data are sorted in an identical manner to PSPC data. The HRI has 
16 pulse height channels, which give very limited energy resolution. It is not
yet possible within ASTERIX to make use of this aspect of the data
except to subset the data on PH channel within {\tt xrtsort}.

Source spectra are significantly more peaked than background spectra
and an improvement in signal to noise may result from restricting the
channel range to 3--8, although the exact channel range will vary
from one source to the next.

\subsubsection {HRI\_Aspect\_Errors}

An error in the SASS processing of aspect time has been recognised in Rosat
HRI observations prior to 17 Jan 1997. The error in calculating aspect time
causes the position of  events to be offset  by upto 3.5 pixels This causes
the PSF to be blurred by the order of ~ 3".

An approximate solution,  using an FTOOLS script, has  been developed 
which is accurate provided that the  roll  angle is not changing too rapidly. A  
detailed description of this problem is provided at :

\begin{verbatim}
       http://ledas-www.star.le.ac.uk/rosat-goc/rosstatus178
       http://hea-www.harvard.edu/rosat/rsdc_www/aspfix.html   
\end{verbatim}

\subsection{Background Subtraction}
Basic background subtraction may be performed with {\tt xrtsub}. This will
subtract the background after normalising for area. At present
the software does not perform any correction for the different detector
efficiencies at the source and background box positions.

Alternatively, since most of the background is due to particles, it would
also be possible simply to subtract a flat background. Software to help
determine the level of this may be obtained from:

\begin{verbatim}
 ftp://legacy.gsfc.nasa.gov/ in directory /rosat/publications/diff_bkgrd/temp
\end{verbatim}

\subsection{Exposure Correction} 
The exposure correction program, {\tt xrtcorr} performs dead-time corrections,
correction for the mirror vignetting, and correction for the quantum
efficiency of the detector.  All corrections are carried out according
to formulae given in the Calibration Report.  The one slight exception to
this is the quantum efficiency correction. The QE of the detector is 
provided as a 2-D map in detector coordinates.  To apply this to data
binned in world coordinates it is radially binned, and the averaged
correction applied only as a function of radial offset from the centre of
the FOV.
\subsection{Source Searching}
The source search program uses the point spread function as specified
in the calibration report.

\newpage
\section{A Description of the PSF}
 
The ROSAT PSF describes the  spatial response of  the instrument to a point
source resulting from the scattering of photons due  to mirror figuring and
irregularities, and physical limitations of the detector. This response also
varies with energy.

The PSF  has  to  be considered   to  perform an  optimal   analysis of  an
observation. 

The PSF results from a number of factors;

ON AXIS effects - see OGIP calibration memo. CAL/ROS/92-001.
\begin{itemize}
\item A scattering effect  - caused by irregularities in the surface of the mirror.
\item A penetration effect - due to the varying depth penetration of X-rays in the counter gas.
\item A resolution effect  - because of the  manner in which electrons are generated.
\item Image ghosting    - in PHA channels less than channel 15. 
\item An ellipsoidal blur    - due to the uncorrected attitude motion.
\end{itemize}

OFF AXIS effect - see OGIP calibration memo. CAL/ROS/93-015.
\begin{itemize}
\item A blur effect - the blur of the XMA.
\end{itemize}

\subsection{ASTERIX PSF}
\subsubsection{ASTERIX Programs using the PSF System}
\begin{itemize}
\item {\tt pss} - in the point source searching program {\tt pss} the PSF is used in conjunction with a background model to make comparisons with an input dataset. (See user_004.)
\item {\tt imsim} - uses the PSF to help create simulated images.
\item {\tt spresp} - attaches a spatial response to an input dataset. (See the Respfile option below).
\item {\tt crepsf} - creates a tabular PSF dataset. (See the Tabular  option below.)
\item {\tt spconvol} - is used in extended source searching to convolve the spatial response with a user supplied source model.
\item {\tt ipsf} - fits the PSF to a radial distribution produced by Iradial and displays the result.
\item {\tt xrad} - is a ROSAT PSPC only program which calculates the radius of a circle which encloses a certain fraction of counts. 
\end{itemize}

\subsubsection {PSF User Interface}

Because ASTERIX  is designed  as a multi-mission program it  facilitates the
inclusion  of instrument dependent code  for the PSF by  the use  of a PSF data
routines  library which  it  accesses   using  PSF system  routines.   (See
{\tt Prog/011})

The PSF system controls access to the numerous PSF options which are
presented to the user in the ASTERIX PSF user interface as follows;

\begin{verbatim}
        PSF system options :
 
        ANALYTIC       ASCA           EXOLE          PWFC           RADIAL
        RESPFILE       TABULAR        WFC            XRT_HRI        XRT_PSPC
 
        CSPEC(PSF/model,spec[,nbin])
        POLAR(PSF,rbin[,abin])        RECT(PSF,xbin[,ybin])

        PSF - Choose PSF to use for source model /'XRT_PSPC'/ >   
        MASK - PSPC PSF option (LIST for descriptions) /'VARPROFILE'/ > 
        Loaded XRT PSF cube with 10 energy channels
        AUX - Mean photon energy in KeV > 
\end{verbatim}

The reason for  the  large number of  PSF  options  list above include:
multiple  missions and  instruments,   recalibrations,  analytical options,  
searching   for extended source, and techniques for improving the speed 
of PSF processing.

\subsubsection{General Considerations}
Although the options all provide a PSF to  the ASTERIX programs in standard
form  it is  important for the   user  to familiarise himself with  the operation  of the
individual PSF which he is using, because there are variations in
the behaviour of each  option.  Fortunately there  are a number of  general
considerations which will hasten understanding of a particular PSF ;

\begin{itemize}
\item Does the PSF option derived from calibration data or an analytical form?
\item Has the PSF been  calculated  from a generally applicable  analytical
  function or is it instrument dependent?
\item Which other PSF options are available?
\item Is the PSF onaxis or offaxis?  
\item Does the PSF simply calculate and apply a single PSF array to a dataset?
\item Is  the  PSF a  variable option  which  attempts to  account for  the
  variation of the PSF  with energy and variation of  the PSF with off-axis
  angle? (see the Respfile Option below.)
\item Is the option intended to be used when working with extended sources?
\item Does the option include techniques to improve the speed of processing PSFs?
\end{itemize}

\subsubsection{Improving PSF Processing Speeds}

ASTERIX stores  PSFs as 2D arrays of  integrated probability. These arrays
are  pixel  based.   When  applying  the PSF  the  size of  the  PSF  array
determines the number of adjacent pixels which have to be visited.  Because
calculating a PSF  for individual pixels,  and then applying  that PSF to each
pixel  in a dataset is  computationally expensive, the PSF system includes
strategies to speed up the process;

\begin{itemize}
\item {{\tt Reduce  the PSF size.} The included energy   fraction of the PSF,
    and hence pixel array size, can be reduced (as in {\tt pss}).}
\item {{\tt Precalculate the PSF.} The spatial response can be pre-calculated
    and included in the dataset (see the Respfile Option below.)}
\item {{\tt Calculate the PSF on a grid.}The number of PSF calculations can
    be reduced by calculating PSFs,  not at each pixel  in the image, but over
    some coarser  grid of positions. This  method can be used either before
    applying the PSF (as in {\tt  spresp}) or on  an `as needed' basis (see
    the Model Option below.)}
\end{itemize}

\subsubsection{PSF and Extended Sources} 

The PSF  system can  also  be used to  search  for or model  extended point
sources.  The pss program has a variable EXTEN which measures the fit to an
extended source using the PSF convolved with a Gaussian. Greater control of
the   process of modifying the  PSF  shape can be   achieved using the {\tt
  crepsf} and {\tt spconvol} programs.

\subsection{PSF Options and Masks}
\subsubsection{Other Satellites and ROSAT Instruments}

The ASCA and EXOLE options refer to other satellite missions, while the WFC,
XRT\_HRI  and  XRT\_PSPC  are  different ROSAT   instruments which each
require their own PSF.

\begin{verbatim}
        PWFC*
        - The WFC in pointed mode 
        WFC 
        - The WFC in survey mode 
        XRT_HRI 
        - The ROSAT XRT, HRI detector 
        XRT_PSPC*!
        - The ROSAT XRT, PSPC 1 and 2 detectors 

        * These psf routines vary with position across the field of view
        ! These psf routines support energy modelling using the CSPEC model  
\end{verbatim}

\subsubsection{Analytic Options}
The ANALYTIC option provides single   analytical descriptions which can  be
used as initial approximations to the PSF. These  are useful pre-flight. 
They  can also be used  to help  create  models when  dealing with extended
sources

A  number of  analytical functions  can  be combined  together   and fit to
calibration data to create a PSF model for a particular instrument (see
the HRI Option below.)

\subsubsection{Tabular and Respfile Options}

The TABULAR and RESPFILE  options do  not  create PSFs  but are methods 
which allow access to PSFs which have been previously created.  

The  TABULAR option reads in  a {\it single} PSF from  a datafile which has
been created by  the {\tt crepsf} application. This  option is useful  when
working with extended sources.

The RESPFILE option is an attempt to increase the processing speed in a PSF
program by allowing access  to a SPATIAL  RESPONSE which is attached to the
datafile.  The spatial response  is an ``array  of PSF arrays.''  A spatial
response  is included in  a datafile as  a precomputed grid of point spread
functions tailored to {\it both} the spatial  and energy sub-space occupied
by a dataset.

If a dataset repeatedly uses a PSF the program {\tt spresp}  can be used to
create a PSF SPatial RESPonse and attach it to the dataset  where it can be
recalled as required.  Because a SPATIAL  RESPONSE is precalculated it also
speeds  processing time. This option  is most  useful in applications which
make very intensive use of the PSF. Attaching a spatial response increase the size of a dataset.

\subsubsection{PSF Models}

The PSF   MODELS are  used  to increase processing  speed.   The MODEL
option places a  grid over the dataset  and subsequently  computes a single
PSF for each grid cell which is to be used for all the enclosed points.
For each cell of the grid the PSF is calculated only once, and only if required.

There   are   three    MODEL      forms  -    POLAR(PSF,rbin[,abin])    and
RECT(PSF,xbin[,ybin])        have      spatial         variations while
CSPEC(PSF/model,spec[,nbin])  includes  energy  dependency.     

MODEL selection {\it must} include a reference to a variable PSF option.

eg.  PSF  -    Choose PSF   to use   for    source  model /'XRT_PSPC'/    >
polar(xrt_pspc,0.02)  

In this example 'polar' is the model,  'xrt_pspc' the variable PSF option and
'0.02' the grid cell size.

\subsubsection{XRT\_PSPC - the Recommended Rosat Option and its Masks}
 
The XRT\_PSPC is  the recommended option for the  ROSAT PSPC and four MASKS
are provided: VARPROFILE, VARGAUSSIAN, ONAXIS\_2, ONAXIS\_3

MASK is probably a slightly misleading term which is used by the PSF system
for a sub-menu or sub-option.

ONAXIS\_3 is a recommended MASK if the source  is on  axis.  

The  ONAXIS\_3  mask returns a   single  on-axis PSF.  It  combines three
functions which account    for three of the  factors   which  make up   the
composite on-axis PSF;
\begin{verbatim}
        - primary electron generation is represented by a Gaussian function
        - penetration of the counter gases which is an exponential function
        - mirror scattering which is a Lorentzian function
\end{verbatim}      
        
At greater  than 20 arc   minutes off axis the  off-axis  blur becomes  the
dominant  factor  in the  PSF   and causes  a  noticeable asymmetry  in the
distribution   of source counts.  Because  of the  blur  effect a PSF which
varies with off-angle is a widely preferred option.

VARPROFILE is the recommended MASK if the source is off axis.

The VARPROFILE mask creates  a variable PSF  which contains {\it both }
off-axis angle and energy dependence.

The VARPROFILE mask uses a `data cube'  - a file  in which calibration data
has been pre-processed and stored.   The values of  radius contained in  US
Rod  Calibration  Tables have been  transformed  into the `data  cube' with
ENERGY, OFF AXIS and RADIAL BIN components.  The `data cube' applies to the
whole ROSAT field of  view and can be used  to calculate an  individual PSF
for each data set.   The   comment `Loaded  XRT  PSF  cube with 10   energy
channels' indicates that the data cube has been loaded. 

\begin{verbatim}
        THE CUBE'S DIMENSIONS

        DATA CUBE   BINS    BIN SIZE

        ENERGY      1-10    0.24keV
        OFF AXIS    1-30    2 arcmin
        RADIAL BIN  1-40    Variable @0.018 arcmin       
\end{verbatim}

With  the VARPROFILE mask  the user will also be prompted for a value of AUX.

The AUX value  provides the program with a  value of mean photon energy  in
keV. For typical source spectra 0.5keV is  a recommended value for AUX. For
very hard or soft sources other values may be required - look at a spectrum
of the source to decide.

If  the  VARPROFILE option   is  being  used with   a MODEL   the  smallest
recommended grid size is 0.02 (where axis values are in degrees).

\subsection{HRI Option}

The HRI model  is an analytic fit  to the in-flight response, performed  by
MPE/GSFC/SAO. The functional form of the on-axis HRI PSF is :
\begin{displaymath}
 PSF(R) = A_{1} e^{-0.5({\frac{R}{S1}})^2} + A_{2} e^{-0.5({\frac{R}{S2}})^2} + e^{(\frac{-R}{S3})}
\end{displaymath}

 where the radial distance R from the psf centre is in arc seconds.
 The constants A and S have values :

    A1 = 0.9638 S1 = 2.1858"
    A2 = 0.1798 S2 = 4.0419"
    A3 = 0.001168 S3 = 31.69"

 The on-axis model provides a reasonable description of the PSF out to
 radii of 100 arcsec, but does not model the azimuthal asymmetry
 present within the PSF core [at a radii less than 12 arcsec] of the PSF.

 The on-axis model does not vary  with position in the field  of view and
consequently " ... the
 PSF starts to degrade beyond 5 arc minutes off axis. Beyond 12 arc minutes
 off     axis,     the      images      become    very      asymmetrical."

"A parametric representation of the azimuthally averaged off-axis HRI PSF
has also been derived which is a simple extension of the on axis expression
... The best fit values of s2 are well parameterised by a cubic polynomial
in the off-axis angle, and are given by 

\begin{displaymath}
\rm   S2 = 3.3 + 0.019\theta - 0.016\theta^2 + 0.0044\theta^3
\end{displaymath}

where T is the angle off-axis in arc minutes. Substituting this into the
HRI PSF gives the general off-axis representation."

The ASTERIX XRT_HRI option uses the off-axis HRI PSF parameterisation. 

Documentation on HRI can be found at :  
\begin{verbatim}
http://hea-www.harvard.edu/rosat/rsdc_www/HRI_CAL_REPORT/node12.html. Feb-97
http://heasarc.gsfc.nasa.gov/docs/rosat/newsletters/hripsf10.html. Mar-96.
\end{verbatim}

\newpage
\section{Communicating with the ROSAT Project}

\subsection{Staying Informed}

News items, including the national ROSAT newsletters can be accessed
via the Leicester ROSAT home page and the HEASARC ROSAT guest observer
facility on the World Wide Web.

\begin{verbatim}
        http://www.star.le.ac.uk/Forum/rosat-news  and
        http://heasarc.gsfc.nasa.gov/docs/rosat/rosgof.html
\end{verbatim}

Documentation on ASTERIX can be found at:

\begin{verbatim}
        http://www.sr.bham.ac.uk/asterix-docs/
\end{verbatim}


\subsection{Reporting Bugs}
An interactive form can be found via the WWW which also lists known bugs,
and the action being taken with regards to them, within ASTERIX project.
This can be found from the main ASTERIX web page.

\newpage
\section{A Quick Guide}

\subsection{PSPC}

This section gives brief details of the programs needed to process an
XRT observation.

ROSAT data may be obtained from a data tape or from the Leicester data
archive. In either case all the raw data files should be copied into
a single directory. It is important that this directory does not
contain any raw datafiles from another observation as this can cause
problems with the pre-processing.

\begin{itemize}
\item{Run {\tt xrtconv } on the raw dataset to convert into ASTERIX format.}
\item{Running program{\tt showxrt} will produce a summary of the observation, listing
information such as target name, pointing direction and exposure time.}
\item{The program {\tt xrthk} may be used to "clean" the data by
producing a file of times when the background was low.}
\item{{\tt xrtsort} can now be used to sort the data directly into an
ASTERIX data file.  Example 1 shows {\tt xrtsort} being used to create an
image.  If a file of time slots has been created with {\tt xrthk} this
can be used in {\tt xrtsort} when sorting the data (see Example 2 below).}
\item{The data can be background subtracted with the program {\tt
xrtsub}.  A background file should first be sorted from a blank area
of sky with {\tt xrtsort}.}
\item{{\tt xrtcorr} is used to exposure correct a dataset.  This
includes corrections for instrumental effects, such as vignetting,
dead time as well as normalisation to on-axis counts/sec.}
\item{A detector matrix needs to be produced with {\tt xrtresp} before
spectral fitting can be performed.  To use {\tt xspec} convert the
files using {\tt ast2xsp}.}
\end{itemize}

\subsection{HRI}
Processing of HRI data proceeds on an identical path to PSPC data,
with the obvious omission of the spectral fitting.

\newpage
\section{Housekeeping and calibration data}
\subsection{Housekeeping Data}
By thresholding on the values of different housekeeping parameters  using the 
XRTHK program
the User  is able exclude  unwanted times in  the  original dataset.  XRTHK
outputs a list of on/off MJD times which can then be passed into XRTSORT.

A  list of  the  all housekeeping  parameters  are  included later in  this
document  (Appendix  4).  The  most   commonly used parameters   (and  some
suggested values) are :

\begin{itemize}
\item MV\_ACO - the master veto rate (1-170). 
\item XACC      - the accepted X-ray rate (1-30).
\item ASP\_QUAL - aspect quality (0-2).
\end{itemize}

The  FTOOL program  {\tt  fv} can be used   to  inspect the housekeeping
parameters in the {\tt evrate} and {\tt aspect} extensions of the ancillary
fits file. {\tt fv} may subsequently be  used to plot one
array in a file against another.

For example - the {\tt xspokes} program  uses the satellite  roll angle to 
determine
the position of the window support structure on the  image. This system can
be flawed if  an observation has been  performed in two  sections, with the
satellite at different roll angles.

To check the  roll angle, start FTOOL  program fv and open  your \_anc.fits
file. Select a  plot on the aspect extension  and use the time and roan\_cas
parameters as your axis.

\subsection{Calibration Data}
The  effective area and  detector response  matrix information is currently
read  from SDF calibration files. {\tt plotxy} can be used to plot SDF information.

eg. The approximate relationship between pulse height channel and energy
for the PSPC is $E=(Ch.No.) / 100$ keV, e.g. channel 100 is approximately 
1.0 keV.

To plot the relative vignetting on-axis against that at 15 arcminutes:

\begin{verbatim}
    plotxy $AST_ETC/pspcb_eff.effar_1 $AST_ETC/pspcb_eff.effar_4
\end{verbatim}

Plots  of the   vignetting function  are  given in  Figures 5   \&  6.  The
vignetting  function is defined  at fourteen different off-axis angles from
0--60 arcminutes.

\newpage
\section{Examples}
{\bf EXAMPLE 1: Sorting an image from PSPC data}

\begin{verbatim}
XRTSORT
xunb> xrtsort
XSORT Version 2.0-0
The following header files are present:
 mrk33_hdr.sdf
ROOTNAME - Rootname of files to sort /'mrk33'/ > 
TYPE - Output data type - Event(E) Binned(B) /'B'/ > B
Binned dataset selection
--------------------------
1 - XPIX
2 - YPIX
3 - XDET
4 - YDET
5 - Time
6 - PHA channel
7 - Corrected PHA channel
8 - Radial
AXES - Enter integer codes of axes required in data array /'12'/ > 1 2
SHAPE - Shape of source area [D,C,A,E,R] /'R'/ > R
RA - RA of the centre of the source region /'     158.13000488281'/ > 
DEC - DEC of the centre of the source region /'     54.400001525879'/ > 
SWIDTH - Width of source box /2.133333/ > 
SHEIGHT - Height of source box /2.133333/ > 
    Data array axes
  *******************
     3) XDET:      0 to    8192 (pixels)
     4) YDET:      0 to    8192 (pixels)
     5) Time:     147.000 to    87964.000 (seconds)
     6) PHA channel:      1 to     256 (chn.)
     7) Corr. PHA chan.      1 to     256 (chn.)
RANGES - Property ranges to change (enter numbers) /!/ > 
 There are 15358 raw sky pixels within the X range selected
NXBIN - Number of output bins in the X axis /512/ > 
There will be 512 X bins, each of 30 raw pixels, starting from -7680
 There are 15358 raw sky pixels within the Y range selected
NYBIN - Number of output bins in the Y axis /512/ > 
There will be 512 Y bins, each of 30 raw pixels, starting from -7680
BACK - Is a background file wanted (Y/N) /FALSE/ > FALSE
OUT - Output source filename /'image1'/ > broad_image1
> 
\end{verbatim}

\newpage
{\bf EXAMPLE 2: Sorting a spectrum from PSPC data with master veto rate
between 0 and 170.}

\begin{verbatim}
> xrthk
XRTHK - Version 1.7-0
RAWDIR - Raw data directory /'/data'/ > 
Index/header file found on  /data
 Found 15 data files on  /data
 /data/mrk33.sdf
 /data/mrk33_att.sdf
 /data/mrk33_dif.sdf
 /data/mrk33_diffuse.sdf
 /data/mrk33_evr.sdf
 /data/mrk33_hdr.sdf
 /data/mrk33_imec.sdf
 /data/mrk33_mex.sdf
 /data/mrk33_orb.sdf
 /data/mrk33_overlay.sdf
 /data/mrk33_pl.sdf
 /data/mrk33_qua.sdf
 /data/mrk33_rz.sdf
 /data/broad_image1.sdf
 /data/aspec_10.sdf
ROOTNAME - Rootname of the calibration files /'mrk33'/ > 
Typical parameters to sort on :
   Master veto rate : ee_mv
   Aspect error : asp_err
   Accepted event rate : ee_axe
HKPAR1 - Name of HK parameter (! to finish) /'ee_mv'/ > 
ee_mv : ranges from 0 : 210
PMIN1 - Minimum value for this parameter /0/ > 
PMAX1 - Maximum value for this parameter /210/ > 170
18 BAD windows - excluding 25.3 % of the data
HKPAR2 - Name of HK parameter (! to finish) /!/ > 
Total exposure time now: 3063 seconds - excluded 25.35 % of the data
FNAME - Name of output time window file /'xrt_times.lis'/ > 
xunb> xrtsort
XSORT Version 2.0-0
The following header files are present:
 mrk33_hdr.sdf
ROOTNAME - Rootname of files to sort /'mrk33'/ > 
TYPE - Output data type - Event(E) Binned(B) /'B'/ > 
Binned dataset selection
--------------------------
1 - XPIX
2 - YPIX
3 - XDET
4 - YDET
5 - Time
6 - PHA channel
7 - Corrected PHA channel
8 - Radial
AXES - Enter integer codes of axes required in data array /'7'/ > 7
SHAPE - Shape of source area [D,C,A,E,R] /'C'/ > C
RA - RA of the centre of the source region /'     158.13000488281'/ > 
DEC - DEC of the centre of the source region /'     54.400001525879'/ > 
RAD - Radius of source box (degrees) /1.066667/ > 0.16
    Data array axes
  *******************
     3) XDET:      0 to    8192 (pixels)
     4) YDET:      0 to    8192 (pixels)
     5) Time:     147.000 to    87964.000 (seconds)
     6) PHA channel:      1 to     256 (chn.)
     7) Corr. PHA chan.      1 to     256 (chn.)
RANGES - Property ranges to change (enter numbers) /!/ > 5
TIMRANGE - Time ranges /'147:87964'/ > xrt_times.lis
ENBIN - Bin width of corrected pulse height axis /1/ > 
There will be 256 energy bins, each of 1 raw channels, starting from 1
BACK - Is a background file wanted (Y/N) /FALSE/ > 
OUT - Output source filename /'aspec_10'/ > aspec_10
> 
\end{verbatim}

\newpage
{\bf EXAMPLE 3: Source searching a PSPC image}

\begin{verbatim}
> pss
PSS Version 2.0-1
INP - Dataset to be searched/tested for sources /'broad_srcsub'/ > broad_image1
EXPERT - Expert mode /TRUE/ > 
PSF system options :
  ANALYTIC       ASCA           EXOLE          PWFC           RADIAL
  RESPFILE       TABULAR        WFC            XRT_HRI        XRT_PSPC
  CSPEC(psf/model,spec[,nbin])
  POLAR(psf,rbin[,abin])        RECT(psf,xbin[,ybin])
PSF - Choose PSF to use for source model /'XRT_PSPC'/ > polar(xrt_pspc,0.02)
MASK - PSPC psf option (LIST for descriptions) /'VARPROFILE'/ > 
Loaded XRT psf cube with 10 energy channels
AUX - Mean photon energy in KeV > 0.9
PSFCON - Assume constant PSF across field /NO/ > 
Energy fraction    Off-axis angle
                    0    32    64   arcmin
      50%         0.9   5.0  13.4   pixels
      68%         1.2   6.4  16.4   pixels
      90%         1.6   8.6  25.7   pixels
      95%         1.8   9.3  28.7   pixels
PSFPIX - Radius of PSF box in pixels /'68%'/ > 
X position axis range is from 1.064584 to -1.064445 degrees
Y position axis range is from -1.064445 to 1.064583 degrees
SLICE - Section of dataset to search /'0.3:-0.3,-0.3:0.3'/ > 
SOPT - Statistic option (CASH,GAUSSIAN) /'CASH'/ > 
BGND - Background model /'broad_bgmodel'/ > 
RESCALE - Re-scale background estimate /NO/ > 
EXTEN - Fit for extension measure /NO/ > 
SAMPLE - Oversampling factor for first pass /1/ > 
First pass - grid spacing 1 pixels
MAP - Significance map /!/ > broad_sigmap
Significance varies from 0 to 12.07309
SIGMIN - Significance threshold /5/ > 4.5
 Src    X        Y     Signif
   1   0.19    -0.14    6.998
   2  -0.22    -0.01    4.528
   3   0.13     0.14    4.072
   4  -0.20     0.16    8.338
   5   0.00     0.00    4.507
   6  -0.07     0.22    8.857
   7  -0.14     0.23    4.872
   8   0.04    -0.19    5.617
   9   0.12     0.02    8.161
  10   0.06     0.06    3.657
  11  -0.14    -0.09   12.073
  12   0.04     0.16    5.764
  13   0.03     0.20    8.549
  14  -0.09    -0.11    4.844
  15  -0.09    -0.09    4.147
  16   0.05    -0.12    4.608
  17   0.04    -0.10    4.981
  18   0.22     0.10    5.942
  19   0.22     0.11    5.974
Second pass
ASYMMETRIC - Asymmetric source parameter errors /NO/ > 
FERL - Flux error confidence level /'1 sigma'/ > 
PERL - Positional error confidence levels /'90%'/ > 
 Src      RA         DEC         X        Y     Signif    Flux
   1  10 31 01.5  +54 23 37    -0.22    -0.01    4.581    17.232
   2  10 31 09.0  +54 33 51    -0.20     0.16    8.362    40.788
   3  10 31 35.1  +54 37 30    -0.14     0.23    4.940    23.224
   4  10 31 35.7  +54 18 30    -0.13    -0.09   12.572    59.422
   5  10 31 54.0  +54 17 06    -0.09    -0.11    4.910    15.302
   6  10 32 03.8  +54 37 04    -0.07     0.22    9.039    48.550
   7  10 32 42.8  +54 35 46     0.03     0.20    8.767    39.222
   8  10 32 47.4  +54 17 50     0.04    -0.10    5.029    15.633
   9  10 32 47.6  +54 12 38     0.04    -0.19    5.624    20.476
  10  10 32 48.7  +54 33 57     0.04     0.17    6.040    23.733
  11  10 32 52.0  +54 16 38     0.05    -0.12    4.739    14.189
  12  10 33 21.5  +54 25 12     0.12     0.02    8.486    34.084
  13  10 33 47.6  +54 15 29     0.19    -0.14    7.144    34.381
  14  10 34 04.0  +54 30 52     0.22     0.11    6.077    30.701
  15  10 34 04.1  +54 30 57     0.22     0.12    6.172    31.177
OUT - Source search results file (! for none) /'srclist'/ > broad_srclist
SSUB - Source subtracted dataset /!/ > broad_srcsub
Creating source subtracted image...
\end{verbatim}


\newpage
\section{Appendix 1: Dataset Filenames and Contents}

The following is a list of files and contents, from dataset of different
origins.

\begin{tabbing}
c1 \= = P if PSPC observation without the Boron filter \\
\> = FP if PSPC observation with the Boron filter \\
\> = H if HRI observation \\
\> = nnnnnn \= six digit ROSAT Observation Request (ROR) number.
\end{tabbing}

%File info.
\small
\begin{table}[h]
\centering
\caption{MPE PSPC files and naming conventions}
\begin{tabular}{| l l |}
\hline
\multicolumn{1}{|c}{Filename} & \multicolumn{1}{c|}{Description} \\
\hline
EVENTS.TFITS    &  Events \\
EVENTRATES.TFITS & Event rates \\
ATTITUDE.TFITS  &  Attitude \\
ORBIT.TFITS     &  Orbit \\
MEXMAP.IFITS    &  Exposure map \\
\hline
IMAGE1.IFITS    &  Broad band image \\
IMAGE2.IFITS    &  Hard band image \\
IMAGE3.IFITS    &  Soft band image \\
IMAGEEC.IFITS   &  Coded energy image \\
DIFEVENTS.TFITS &  Differential events \\
QUALITY.TFITS   &  Quality \\
\RP \rornum.PUBLIC\_CONTENTS & Public contents \\
MOIMP.IFITS     & Modified instrument map \\
DRMPSPC.IFITS   & Detector response \\
EFFAREA\_PSPC.TFITS & Effective area \\
\hline
\end{tabular}
\end{table}

\clearpage

\begin{table}[h]
\centering
\caption{GSFC PSPC files and naming conventions}
\begin{tabular}{| l l |}
\hline
\multicolumn{1}{|c}{Filename2} & \multicolumn{1}{c|}{Description} \\
\hline
\RPF \rornum.FITS & Events \\
\RPF \rornum.EVR & Event rates \\
\RPF \rornum.ASP & Aspect quality \\
\hline
\RPF \rornum.CAS & Corrected aspect \\
\RPF \rornum.HKB & Binned housekeeping \\
\RPF \rornum.MDS & Master data set \\
\RPF \rornum.PAR & Parameter set \\
\RPF \rornum.PUBLIC\_CONTENTS & Public contents \\
\RPF \rornum*.REPORT & Processing report \\
\RPF \rornum.RCR & Radiation calibration \\
\RPF \rornum.SA  & Split aspect \\
\RPF \rornum.SAS & Smoothed aspect \\
\RPF \rornum.SO  & Split orbit \\
\RPF \rornum.STA & Housekeeping status \\
\RPF \rornum\_IM1.FITS & Broad band image \\
\RPF \rornum\_IM2.FITS & Hard band image \\
\RPF \rornum\_IM3.FITS & Soft band image \\
\RPF \rornum\_MEX.FITS & Exposure map \\
\RPF \rornum\_SKY.FITS & Sky catalogue table \\
\RPF \rornum\_SMX.ASC  & Sky catalogue table \\
\RPF \rornum.MOI & Modified instrument map \\
\RPF \rornum.DMP & Detector response \\
\RPF \rornum.OAR & Effective area \\
\RPF \rornum*.PLT & Various Postscipt \\
\hline
\end{tabular}
\end{table}

\clearpage

\begin{table}[h]
\centering
\caption{GSFC HRI files and naming conventions}
\label{GSFC-HRI}
\begin{tabular}{| l l |}
\hline
\multicolumn{1}{|c}{Filename} & \multicolumn{1}{c|}{Description} \\
\hline
\RH \rornum.FITS & Events \\
\hline
\RH \rornum\_IMG.FITS & Image \\
\RH \rornum.AH  & Aspect histogram \\
\RH \rornum.ANP & Processing parameters \\
\RH \rornum.AO  & Aspect offsets \\
\RH \rornum.CPB & Charged particle bgnd map \\
\RH \rornum.CPS & Complete photon store \\
\RH \rornum.DBM & Detector background map\\
\RH \rornum.DMS & Cts/pix**2/s histogram \\
\RH \rornum.HTL & Science frame qualitites \\
\RH \rornum.LTF & Live--time correction factors \\
\RH \rornum.MOB & Merged OBI TSH \\
\RH \rornum.OAN & Observation angles \\
\RH \rornum.OBI & Bright Earth intervals \\
\RH \rornum.OBT & Bright Earth TOE \\
\RH \rornum.OTS & TSH by OBI \\
\RH \rornum.PAR & Output parameter set \\
\RH \rornum.PUBLIC\_CONTENTS & Public contents \\
\RH \rornum*.REPORT & Processing report \\
\RH \rornum.QEG & Quantum efficiency map \\
\RH \rornum.S1D & Sampling histogram \\
\RH \rornum.SAA & SAA rates \\
\RH \rornum.SGI & Standard good intervals \\
\RH \rornum.SO  & Split orbit info \\
\RH \rornum.SPS & Science rates \\
\RH \rornum.SSC & Standard screen results \\
\RH \rornum.TSH & Temporal status history (TSH) \\
\RH \rornum.\_SKY.ASC & Sky catalogue table\\
\RH \rornum.\_SKY.FITS & Sky catalogue table \\
\RH \rornum.\_SKY.PLT & Postscript \\
\RH \rornum.\_STR.PLT & Postscript \\
\hline
\end{tabular}
\end{table}

\clearpage

\begin{table}[h]
\centering
\caption{MPE \& GSFC RDF FITS PSPC files}
\begin{tabular}{| l l |}
\hline
\multicolumn{1}{|c}{Filename} & \multicolumn{1}{c|}{Description} \\
\hline
\RPF \rornum \_bas.fits & Events and good times \\
\RPF \rornum \_anc.fits & Aspect and ancilliary data \\
\RPF \rornum \_im1.fits & Broad band image \\
\RPF \rornum \_im2.fits & Hard band image \\
\RPF \rornum \_im3.fits & Soft band image \\
\RPF \rornum \_bk1.fits & Broad band background \\
\RPF \rornum \_bk2.fits & Hard band background \\
\RPF \rornum \_bk3.fits & Soft band background \\
\RPF \rornum \_mex.fits & Merged exposure map \\
\RPF \rornum \_ime.fits & Energy coded image \\
\RPF \rornum \_src.fits & Source extraction information \\
\RPF \rornum \_his.fits & Processing parameter and log files \\
\RPF \rornum \_prt.fits & Source lists and PostScript plots \\
\RPF \rornum \_ltc.fits & Light curves (if extracted) \\ 
\RPF \rornum .public\_contents & Public contents \\
\hline
\end{tabular}
\end{table}

%\clearpage

\begin{table}[h]
\centering
\caption{MPE \& GSFC RDF FITS HRI files}
\begin{tabular}{| l l |}
\hline
\multicolumn{1}{|c}{Filename} & \multicolumn{1}{c|}{Description} \\
\hline
\RH \rornum \_bas.fits & Events and good times \\
\RH \rornum \_anc.fits & Aspect and ancilliary data \\
\RH \rornum \_im1.fits & HRI image \\
\RH \rornum \_bkg.fits & Background image \\
\RH \rornum \_src.fits & Source extraction information \\
\RH \rornum \_his.fits & Processing parameter and log files \\
\RH \rornum \_prt.fits & Source lists and PostScript plots \\
\RH \rornum .public\_contents & Public contents \\
\hline
\end{tabular}
\end{table}

Following the conversion of the first generation FITS files to the RDF data
format most Users will find the information presented in  Table 5 and Table
6 to be of most interest.

\clearpage

\newpage
\subsection{A Typical ASTERIX File}
\begin{verbatim}
HARD_SPEC3  <UNKNOWN>

   TITLE          <_CHAR*80>      'GB2 1217+348'
   UNITS          <_CHAR*80>      'Counts'
   DATA_ARRAY(128,128,256)  <_REAL>   0,0,0,0,0,0,0,0,0,0,0,0,0,0,0,0,0,0,
                                      ... 0,0,0,0,0,0,0,0,0,0,0,0,0,0,0,0,0,0
   QUALITY        <QUALITY>       {structure}
      QUALITY(128,128,256)  <_UBYTE>   128,128,128,128,128,128,128,128,
                                       ... 128,128,128,128,128,128,128,128,128
      BADBITS        <_UBYTE>        *

   AXIS(3)        <AXIS>          {array of structures}

   Contents of AXIS(1)
      LABEL          <_CHAR*80>      'X position'
      UNITS          <_CHAR*80>      'degrees'
      NORMALISED     <_LOGICAL>      FALSE
      DATA_ARRAY     <ARRAY>         {structure}
         VARIANT        <_CHAR*6>       'SPACED'
         BASE           <_REAL>         1.058334
         SCALE          <_REAL>         -0.016666
         DIMENSION      <_INTEGER>      128

   MORE           <EXT>           {structure}
      ASTERIX        <EXT>           {structure}
         HEADER         <EXTENSION>     {structure}
            TARGET         <_CHAR*20>      'GB2 1298+348'
            OBSERVER       <_CHAR*20>      'WG700001P'
            OBSERVATORY    <_CHAR*5>       'ROSAT'
            INSTRUMENT     <_CHAR*3>       'XRT'
            AXIS_RA        <_DOUBLE>       185.029586791992
            AXIS_DEC       <_DOUBLE>       34.5200004577637
            FIELD_RA       <_DOUBLE>       185.02958679199
            FIELD_DEC      <_DOUBLE>       34.520000457764
            EQUINOX        <_INTEGER>      2000
            POSITION_ANGLE  <_DOUBLE>      0
            BASE_MJD       <_INTEGER>      49129
            BASE_DATE      <_CHAR*11>      '1993-May-22'
            BASE_UTC       <_DOUBLE>       26568.0000002263
            BASE_TAI       <_DOUBLE>       7812.30769675926
            OBS_LENGTH     <_REAL>         7970
            EXPOSURE_TIME  <_REAL>         4281

         INSTRUMENT     <EXTENSION>     {structure}
            PIXEL_SIZE     <_REAL>         0.5
            FILTER         <_CHAR*20>      'OFF'
            DETECTOR       <_CHAR*20>      'PSPCB'
            SC_BASE        <_DOUBLE>       93780976
            SC_CONV        <_DOUBLE>       0.99990789351471
            RAWDATA        <_CHAR*5>       'MPE'
            SASS_VERSION   <_CHAR*25>      '                         '
            SORT           <EXTN>          {structure}
               SPACE(1)       <SPACE>         {structure}
                  SHAPE          <_CHAR*1>       'R'
                  PHI            <_REAL>         0
                  XCENT          <_REAL>         0.000069
                  YCENT          <_REAL>         0.000069
                  XINNER         <_REAL>         0
                  XOUTER         <_REAL>         1.066597
                  YINNER         <_REAL>         0
                  YOUTER         <_REAL>         1.066597

               XDET(1)        <XDET>          {structure}
                  START          <_INTEGER>      0
                  STOP           <_INTEGER>      8192

               YDET(1)        <YDET>          {structure}
                  START          <_INTEGER>      0
                  STOP           <_INTEGER>      8192

               TIME(1)        <TIME>          {structure}
                  START          <_REAL>         2877
                  STOP           <_REAL>         10847

               PH_CHANNEL(1)  <PH_CHANNEL>    {structure}
                  START          <_INTEGER>      1
                  STOP           <_INTEGER>      256

               ENERGY(1)      <ENERGY>        {structure}
                  START          <_INTEGER>      1
                  STOP           <_INTEGER>      256

         PSF            <EXTENSION>     {structure}
            LIBRARY_NAME   <_CHAR*6>       'PSFLIB'
            ROUTINE_NAME   <_CHAR*8>       'XRT_PSPC'

         PROCESSING     <EXTENSION>     {structure}
            BGND_SUBTRACTED  <_LOGICAL>    FALSE
            CORRECTED      <EXTENSION>     {structure}
               DEAD_TIME      <_LOGICAL>      FALSE
               VIGNETTING     <_LOGICAL>      FALSE

   HISTORY        <HISTORY>       {structure}
      CREATED        <_CHAR*18>      ' 9-Jul-96 14:35:25'
      UPDATE_MODE    <_CHAR*10>      'NORMAL'
      EXTEND_SIZE    <_INTEGER>      10
      CURRENT_RECORD  <_INTEGER>     2
      RECORDS(10)    <HIST_REC>      {array of structures}

      Contents of RECORDS(1)
         DATE           <_CHAR*30>      '    9-Jul-1996 14:35:25'
         COMMAND        <_CHAR*30>      'XSORT Version 1.7-1'
         HOST           <_CHAR*30>      'xra1'

End of Trace.
\end{verbatim}


\newpage
\section{Appendix 2: ASTERIX XRT Command List}

\subsection{Preprocessing}
{\large \bf{fits2hds}} - Convert US, MPE and RDF FITS files into ASTERIX
format.\\

\subsection{Data Analysis}
{\large \bf {showxrt}} - Display details of an XRT observation. \\
{\large \bf {xpsscorr}} - Exposure correct a PSS source file. \\
{\large \bf {xrad}} - Calculate the PSF of a PSPC source. \\
{\large \bf {xrtcorr}} - Perform instrument corrections. \\
{\large \bf {xrthk}} - Produce a file of times when the housekeeping
parameters are within a certain range. \\
{\large \bf {xrtresp}} - Calculate the detector matrix for a source. \\
{\large \bf {xrtsub}} - Perform background subtraction. \\
{\large \bf {xrtsort}} - Sort raw data into a binned dataset,
e.g. image, spectrum. \\
{\large \bf {xspokes}} - Produce an ARD file to describe the position
of the PSPC support structure on the image. \\

\subsection{ICL Procedures}

{\large \bf {xpsf}} - Draw the PSF circle on an image. \\
{\large \bf {xrtbox}} - Interactively define a source region 
for data sorting. \\
{\large \bf {xrtbckbox}} - Define a source and background region 
for data sorting. \\
{\large \bf {xrtmspec}} - Produce a spectrum for each source found
with {\tt pss}. \\
{\large \bf {xrtmtime}} - Produce a time series for each source found
with {\tt pss}. \\

\subsection{Associated Commands}

{\large \bf {ardqual}} - Set pixels GOOD/BAD using an ARD file. \\
{\large \bf {arithmetic}} - Basic arithmetic on 2 files. \\
{\large \bf {ast2xsp}} - Convert an ASTERIX spectral file into {\tt xspec} format. \\
{\large \bf {enmap}}   - Creates an energy map. \\
{\large \bf {evbin}}   - Bin up an EVENT dataset. \\
{\large \bf {meandat}} - Average several datafiles. \\
{\large \bf {mkcolim}} - Produce a 3-colour image. \\
{\large \bf {project}} - Reduce file dimensions. \\
{\large \bf {Isys}}    - Image processing GUI.\\

\newpage
\section{Appendix 3: Photon Characteristics}
\begin {table}[h]
\caption{PSPC and HRI Photon characteristics}
\label{t:phot}
\begin{center}
\begin{tabular}{|l|c|c|c|c|}
\hline
Property        &\multicolumn{2}{|c}{PSPC} & \multicolumn{2}{c|}{HRI}  \\ \hline
                &  MPE      &  US          &  MPE   &  US  \\ \hline
XDET$^{a}$      &  0--8192  &  1--8192  &  0--4096 & 1--4096 \\
YDET$^{a}$      &  0--8192  &  1--8192 & 0--4096   & 1--4096 \\
XPIX$^{b}$      & $-7680\rightarrow+7680$ &  1--15360 & $-4095\rightarrow+4097$ & 1--8192 \\
YPIX$^{b}$      & $-7680\rightarrow+7680$ &  1--15360 & $-4095\rightarrow+4097$ & 1--8192 \\
Time            &   S/C time &  S/C time  &  S/C time  & S/C time \\
Raw PHA         & 1--256     &  1--256 & 0--15 & 0--15 \\
Corr. PHA$^{c}$ & 1--256     &  0--255 & 1--1  & 1--1 \\
\hline
\end{tabular}
\end{center}
$^{a}$ Pixel position on the detector. \\
$^{b}$ Pixel position on the sky. i.e. directly related to RA/DEC. \\
$^{c}$ The raw pulse height from the PSPC is corrected for
gain effects during the SASS pre-processing. This corrected PHA is 
related to the energy of the incident photon. \\

The sky pixels are defined and numbered differently in the US and MPE datafiles.
US data increases from bottom to top, so for the PSPC, the 
bottom left hand corner 
pixel is 1,1. MPE pixel numbers decrease with increasing Y value, so
the bottom left hand pixel is -7680,7680 and the top left hand corner
pixel is -7680,-7680. Corrected PH channels have the same meaning in
the two datasets, but the US supply channels 0--255 for PSPC data 
whereas the MPE
data ranges from 1--256. The channel 0 PSPC data in the US files 
may not be used within ASTERIX.

\end{table}

\newpage
\section{Appendix 4: Important Housekeeping Parameters}

By setting the values of  different housekeeping parameters using the XRTHK program
the User  is  able exclude unwanted times  in  the original dataset.  XRTHK
produces a list of on/off MJD's times which are passed in to XRTSORT. 

The following list of parameters upon which the user is able to search in XRTHK
has been compiled from the ASPECT and EVRATE  extension of the ANCILLARY RDF
data file
 
\begin{verbatim}

EVRATE FITS EXTENSION

PARAMETER           DESCRIPTION
TIME                / Spacecraft time of event
MV_ACO              / MV anticoincidence rate       
XTRANSM             / Transmitted X-ray rate  
A1_AL               / Events in anode A1 above low level threshold       
A2_AL               / Events in veto anode A2 above low level thresho        
A1AL_MV             / Events in A1 above low level thresh in antico.
XACC                / Accepted X-ray rate         
SA_RATE             / SA-RATE    
SAADA_RT            / SAA detector A rate   
SAADB_RT            / SAA detector B rate   
A1_AH               / Events in anode A1 above high level threshold      
A2_AH               / Events in veto anode A2 above high level thresh 
|K1_EDGE            / Rate of events in edge strips of front cathode     
K2_EDGE             / Rate of events in edge strips of back cathode K   
K1_PASS             / Rate of events in K1 passing event selection  
K2_PASS             / Rate of events in K2 passing event selection   
MV_CNT              / MV anticoincidence counter
XTR_CNT             / Transmitted X-ray events counter
A1AL_CNT            / Counter of events in A1 above low level threshold
A2AL_CNT            / Counter of events in A2 above low level threshold
XACC_CNT            / Accepted X-ray events
PING_CNT            / Ping counter
A1AH_CNT            / Events in A1 above high level
A2AH_CNT            / Events in A2 above high level
K1ED_CNT            / Events in edge strips of K1
K2ED_CNT            / Events in edge strips of K2
K1PA_CNT            / Events in K1 passing event selection logic
K2PA_CNT            / Events in K2 passing event selection logic  

ASPECT FITS EXTENSION

PARAMETER           DESCRIPTION
TIME                / Spacecraft time of corrected aspect (sec)      
MJD_INT             / Integer portion of UT in MJD     
MJD_FRAC            / Fractional portion of UT in MJD    
RA_CAS              / Corrected right ascension (offset from RA_NOM)   
DEC_CAS             / Corrected declination (offset from DEC_NOM) 
ROAN_CAS            / Corrected roll angle (offset from ROLL_NOM) 
ASP_QUAL            / Aspect quality      
STT_QUAL            / Star tracker flag  
RA_SC               / Right ascension of spacecraft pointing       
DEC_SC              / Declination of spacecraft pointing      
ROAN_SC             / Roll angle of spacecraft pointing     
RA_ERR              / 1-sigma error of RA_SC  
DEC_ERR             / 1-sigma error of DEC_SC   
ROAN_ERR            / 1-sigma error of ROAN_SC    
STATUS              / Aspect status     
\end{verbatim}

\newpage
\section{Glossary}
{\large\bf ARD} - Ascii Region Description. A method of describing
a spatial region in a text file. \\
{\large\bf GUI} - Graphical User Interface.\\
{\large\bf HRI} - High Resolution Imager. \\
{\large\bf LEDAS} - Leicester Database and Archive Service. \\
{\large\bf MPE} - Max-Planck institute at Garching bei Munchen. \\
{\large\bf PSF} - Point spread function. The scattering of photons
from a source due to mirror irregularities and detector response. \\
{\large\bf PSPC} - Position Sensitive Proportional Counter. \\
{\large\bf RDAC} - {\em ROSAT} Data Archive Centre. \\
{\large\bf ROD} - {\em ROSAT} Observation Dataset. \\
{\large\bf US} - United States of America. \\
{\large\bf Vignetting} - The decline in detection efficiency with 
increasing off-axis angle. Primarily due to XRT mirror reflectivity. \\

\subsection{Useful WWW References}

Hypertex version of asthelp:\\
  http://www.sr.bham.ac.uk/asterix\\
  
ROSAT news items:\\
  http://ledas-www.star.le.ac.uk/rosat-goc.html\\
  http://heasarc.gsfc.nasa.gov/\\
  http://www.sr/bham.ac.uk/asterix\\
  
LEDAS user guide:\\
  http://ledas-www.star.le.ac.uk/rosat-goc/doc.html\\

Calibration report:\\
  ftp://legacy.gsfc.nasa.gov/rosat/doc/hri/hri\_report\\


\subsection{Caption List}

The following figures can be found in \$AST\_DOCS as postscript files.

figure 1. "PSPC image" user\_005\_impspc.ps\_tex

figure 2. "HRI image" user\_005\_imhri.ps\_tex

figure 3. "Background extraction technique" user\_005\_srcblk.ps\_tex

figure 4. "HRI detector efficiency" user\_005\_hriqeff.ps\_tex

figure 5. "Vignetting curves at 0,15,30,45 arcminutes of-axis" user\_005\_angvig.ps\_tex

figure 6. "Vignetting at diff. energies" user\_005\_envig.ps\_tex

\newpage
\section{References \& Bibliography}

\subsection{User Guides}
"The UK ROSAT Data Archive Centre User Guide", Sembay, S., Watson, M.G.,
Osborne, J.P.

"The EXSAS users guide", EXSAS development group, ROSAT Scientific Data 
Center, MPE, 8046 Garching, Giessenstrasse, Germany.

"ROSAT Data Products Guide", Downes, R., White, R., Reichert, G., 
Dennerl, K., Englehauser, J., Rosso, C., Voges, W.

\subsection{Instrumentation}

David, L.P., Harnden Jr., F.R., Kearns, K.E., Zombeck, M.V., 
"The ROSAT High Resolution Imager (HRI)", {\em U.S. ROSAT Science Data Center
/SAO}

Plucinsky, P.P., Snowden, S.L., Briel, U.G., Hasinger, B. \& Pfefferman, E.,
1993, {\em Astrophys.J., submitted.}

Snowden, S.L., Plucinsky, P.P, Briel, U., Hasinger, G. \& Pfeffermann, E.
1992, {\em Astrophys.J., {\bf 393}, 819}

Snowden, S.L., McCammon, D., Burrows, D.N. \& Mendenhall, J.A, 1993, 
{\em Astrophys.J., submitted.}

Hasinger, G., Turner, J.T., George, I.M., Boese, G.,
1992, OGIP calibration memo. CAL/ROS/92-001, published in 'Legacy II'.

Hasinger, G., Turner, J.T., George, I.M., Boese, G.,
1992, OGIP calibration memo. CAL/ROS/92-001a.

Hasinger, G., Boese, G., Predehl, P.,
1993, OGIP calibration memo. CAL/ROS/93-015.

\subsection{ASTERIX documents}

APP\_001, "PSS manual", Allan, D.J, Ponman, T.J.

PROG\_002, "Data structures: specification and conventions: ", Ponman, T.J. 

USER\_006, "A guide to time series analysis within ASTERIX", Saxton, R.D. 

SUN98, "ASTERIX - X-ray Data Processing System", Saxton, R.D. 


\end{document}                                  % End
