\documentstyle[11pt,fleqn]{article}     % 10% larger letters, equns to left
\pagestyle{myheadings}
%------------------------------------------------------------------------------
\newcommand{\stardoccategory}  {User Note}
\newcommand{\stardocinitials}  {USER}
\newcommand{\stardocnumber}    {006}
\newcommand{\stardocauthors}   {Richard D Saxton}
\newcommand{\stardocdate}      {17-June-1991}
\newcommand{\stardoctitle}     {Time series analysis}
\newcommand{\stardocname}      {\stardocinitials /\stardocnumber}
%------------------------------------------------------------------------------

\setlength{\textwidth}{160mm}           % Text width 16 cm
\setlength{\textheight}{240mm}          % Text height 24 cm
\setlength{\oddsidemargin}{0pt}         % LH margin width, -1 inch
\setlength{\evensidemargin}{0pt}        % LH margin width, -1 inch
\setlength{\topmargin}{-5mm}            %
\setlength{\headsep}{8mm}               % 
\setlength{\parindent}{0mm}

%    Starlink definitions for \LaTeX\ macros used in MAN output
%
%  Description:
%    As much as possible of the output from the MAN automatic manual generator
%    uses calls to user-alterable macros rather than direct calls to built-in
%    \LaTeX\ macros. This file is a version of the MAN default definitions for
%    these macros modified for Starlink preferences.
%
%  Language:
%    \LaTeX
%
%  Support:
%    William Lupton, {AAO}
%    Alan Chipperfield (RAL)
%-
%  History:
%    16-Nov-88 - WFL - Add definitions to permit hyphenation to work on
%		 words containing special characters and in teletype fonts.
%    27-Feb-89 - AJC - Redefine \manroutine
%                      Added \manheadstyle
%                      Switch order of argument descriptors
%    07-Mar-89 - AJC - Narrower box for parameter description
%                      Remove Intro section and other unused bits
%
% permit hyphenation when in teletype font (support 9,10,11,12 point only -
% could extend), define lccodes for special characters so that the hyphen-
% ation algorithm is not switched off. Define underscore character to be
% explicit underscore rather than lots of kerns etc.

\typeout{Starlink MAN macros. Released 27th February 1989}

\hyphenchar\nintt=`-\hyphenchar\tentt=`-\hyphenchar\elvtt=`-\hyphenchar\twltt=`-

\lccode`_=`_\lccode`$=`$

%    Macros used in the .TEX_SUMMARY file
%
%  Description:
%    There is a command to introduce a new section (mansection) and a list-like
%    environment (mansectionroutines) that handles the list of routines in the
%    current section. In addition a mansectionitem command can be used instead
%    of the item command to introduce a new routine in the current section.
%-

\newcommand {\mansection}[2]{\subsection{#1 --- #2}}

\newenvironment {mansectionroutines}{\begin{description}\begin{description}}%
{\end{description}\end{description}}

\newcommand {\mansectionitem}[1]{\item [#1:] \mbox{}}

%    Macros used in the .TEX_DESCR file
%
%  Description:
%    There is a command to introduce a new routine (manroutine) and a list-like
%    environment (manroutinedescription) that handles the list of paragraphs
%    describing the current routine. In addition a manroutineitem command can
%    be used instead of the item command to introduce a new paragraph for the
%    current routine.
%
%    Two-column tables (the ones that can occur anywhere and which are
%    triggered by "=>" as the second token on a line) are bracketed by a
%    new environment (mantwocolumntable). Other sorts of table are introduced
%    by relevant  environments (manparametertable, manfunctiontable and
%    manvaluetable). The definitions of these environments call various other
%    user-alterable commands, thus allowing considerable user control over such
%    tables... (to be filled in when the commands have been written)
%-

\newcommand {\manrule}{\rule{\textwidth}{0.5mm}}

%\newcommand {\manroutine}[2]{\subsection{#1 --- #2}}
\newlength{\speccaption}
\newlength{\specname}
\newcommand{\manroutine}[2]{\goodbreak
                          \rule{\textwidth}{0.5mm}  % draw thick line
                          \settowidth{\specname}{{\Large {\bf #1}}}
                        % left and right box width is text width plus gap
                          \addtolength{\specname}{4ex} 
                        % caption width is width of page less the two names
                        % less than empirical fudge factor
                          \setlength{\speccaption}{\textwidth}
                          \addtolength{\speccaption}{-2.0\specname}
                          \addtolength{\speccaption}{-4.45pt}
                        % move text up the page because \flushleft environ-
                        % ment creates a paragraph
                          \vspace{-7mm}
                          \newline
                          \parbox[t]{\specname}{\flushleft{\Large {\bf #1}}}
                          \parbox[t]{\speccaption}{\flushleft{\Large #2}}
                          \parbox[t]{\specname}{\flushright{\Large {\bf #1}}}
                          }

\newenvironment {manroutinedescription}{\begin{description}}{\end{description}}

\newcommand {\manroutineitem}[2]{\item [#1:] #2\mbox{}}


% parameter tables

\newcommand {\manparametercols}{lllp{90mm}}

\newcommand {\manparameterorder}[3]{#2 & #3 & #1 &}

\newcommand {\manparametertop}{}

\newcommand {\manparameterblank}{\gdef\manparameterzhl{}\gdef\manparameterzss{}}

\newcommand {\manparameterbottom}{}

\newenvironment {manparametertable}{\gdef\manparameterzss{}%
\gdef\manparameterzhl{}\hspace*{\fill}\vspace*{-\partopsep}\begin{trivlist}%
\item[]\begin{tabular}{\manparametercols}\manparametertop}{\manparameterbottom%
\end{tabular}\end{trivlist}}

\newcommand {\manparameterentry}[3]{\manparameterzss\gdef\manparameterzss{\\}%
\gdef\manparameterzhl{\hline}\manparameterorder{#1}{#2}{#3}}


% list environments

\newenvironment {manenumerate}{\begin{enumerate}}{\end{enumerate}}

\newcommand {\manenumerateitem}[1]{\item [#1]}

\newenvironment {manitemize}{\begin{itemize}}{\end{itemize}}

\newcommand {\manitemizeitem}{\item}

\newenvironment {mandescription}{\begin{description}\begin{description}}%
{\end{description}\end{description}}

\newcommand {\mandescriptionitem}[1]{\item [#1]}

\newcommand {\mantt}{\tt}

% manheadstyle for Starlink
\newcommand {\manheadstyle}{}

\catcode`\_=12

% End of MAN add-in

\begin{document}                        	% Start document
\thispagestyle{empty}
DEPARTMENT OF PHYSICS AND ASTRONOMY \hfill \stardocname\\
LEICESTER UNIVERSITY\\
{\large\bf Asterix Data Analysis\\}
{\large\bf \stardoccategory\ \stardocnumber}
\begin{flushright}
\stardocauthors\\
\stardocdate
\end{flushright}
\vspace{-4mm}
\rule{\textwidth}{0.5mm}
\vspace{5mm}
\begin{center}
{\Large\bf \stardoctitle}
\end{center}
\vspace{5mm}

\parskip=4.0truemm plus 0.5truemm       % Paragraph spacing
\markright{\stardocname}

\tableofcontents

\newpage

\section{Introduction}

This document describes how to analyse a time series
within the Asterix data processing system. Asterix has been primarily
designed to cope with X-ray data, however, the time series analysis
routines within it are not wavelength dependent and so are equally 
capable of processing any time series in the correct format.

A full description of each application in the time series package is
given in the Asterix help file. Type :

\begin{verbatim}

          $ ASTHELP app_name

\end{verbatim}

for further information. This guide is intended to give a brief
description of the available routines and give an indication of
their strengths and weaknesses in various situations.

\section{Input data}

All the applications described in this document work on binned
data with the exception of the barycentric correction program,
BARYCORR, which works on the raw event data.

The input datafile should be a standard Starlink NDF (see the Starlink
document SGP/38). This must contain a component DATA\_ARRAY and a time
axis . In addition the components VARIANCE and QUALITY will be 
used if present. 

Most of the applications support irregularly spaced data and so the
time axis may be either a simple array of times or a regularly spaced
array. An exception is the standard FFT program, POWER, which can
only work on regularly spaced data; it will therefore fail if the time
axis is represented by a simple data array.

Gaps and dropouts in the data should be represented by a QUALITY array.
Most applications support QUALITY and will ignore all bad pixels, 
although as before, POWER can not handle data gaps and will fail if
any pixel is bad.

Datafiles containing magic values are not supported by the software,
however, they may be converted to QUALITY using the command :

\begin{verbatim}

          $ SETQUAL MAGIC=YES

\end{verbatim}

Data currently stored in a textfile may be converted to a standard
NDF, using the command :

\begin{verbatim}

          $ IMPORT

\end{verbatim}

\newpage

\section{The applications}
Time series applications are described briefly below. For more information
on an individual routine see the help entry. 

\subsection{Power spectrum analysis}

\manroutine {{\manheadstyle{POWER}}}{ Standard FFT application.}

\begin{manroutinedescription}
\manroutineitem {Description }{}

Computes the power spectrum of a 1-d dataset using a standard FFT.
This technique works on regularly spaced data with NO gaps.

\manroutineitem {Parameters }{}
\begin{manparametertable}
\manparameterentry {{\mantt{READ}} }{{\mantt{INP}} }{{\mantt{_CHAR}}}
   Name of the input NDF.
\manparameterentry {{\mantt{READ}} }{{\mantt{OUT}} }{{\mantt{UNIV}}}
   Name of the output NDF containing the power spectrum
\manparameterentry {{\mantt{READ}} }{{\mantt{TRUNCATE}} }{{\mantt{\_LOGICAL}}}
   Truncate the data to the nearest power of 2, to increase speed ?
\manparameterentry {{\mantt{READ}} }{{\mantt{TAPER}} }{{\mantt{\_LOGICAL}}}
   Taper the data to minimise the effects of the window fn. ?
\manparameterentry {{\mantt{READ}} }{{\mantt{REMOVE_MEAN}} }{{\mantt{\_LOGICAL}}}
   Subtract the mean before processing ?
\end{manparametertable}

\end{manroutinedescription}

\manroutine {{\manheadstyle{DYNAMICAL}}}{Running FFT algorithm.}

\begin{manroutinedescription}
\manroutineitem {Description }{}

Computes the power spectrum of successive segments of a time
series and stacks the results to form a dynamical power spectrum.

\manroutineitem {Parameters }{}
\begin{manparametertable}
\manparameterentry {{\mantt{READ}} }{{\mantt{INP}} }{{\mantt{UNIV}}}
   Name of the input NDF.
\manparameterentry {{\mantt{READ}} }{{\mantt{OUT}} }{{\mantt{UNIV}}}
   Name of the output NDF containing the power spectrum
\manparameterentry {{\mantt{READ}} }{{\mantt{SECTOR}} }{{\mantt{\_INTEGER}}}
   Number of points in each time series segment
\end{manparametertable}

\end{manroutinedescription}


\manroutine {{\manheadstyle{LOMBSCAR}}}{Power spectrum application using
the Lomb-Scargle algorithm.}

\begin{manroutinedescription}
\manroutineitem {Description }{}

Calculates the power spectrum of a 1-d dataset, using the
Press and Rybicki invocation of the LOMB-SCARGLE algorithm. 
This is a particularly useful technique for a time series containing 
data gaps. 

\manroutineitem {Parameters }{}
\begin{manparametertable}
\manparameterentry {{\mantt{READ}} }{{\mantt{INP}} }{{\mantt{UNIV}}}
   Name of the input NDF.
\manparameterentry {{\mantt{READ}} }{{\mantt{OFAC}} }{{\mantt{\_REAL}}}
   Oversampling factor (to increase the low frequency range)
\manparameterentry {{\mantt{READ}} }{{\mantt{HIFAC}} }{{\mantt{\_REAL}}}
   High frequency factor (to increase the high frequency range)
\manparameterentry {{\mantt{READ}} }{{\mantt{WFREQ}} }{{\mantt{\_REAL}}}
   Frequency of window function 
\manparameterentry {{\mantt{READ}} }{{\mantt{OUT}} }{{\mantt{UNIV}}}
   Name of output NDF to contain the power spectrum
\end{manparametertable}

\end{manroutinedescription}

\manroutine {{\manheadstyle{SINFIT}}}{Least-squares fit of Sine-waves.}

\begin{manroutinedescription}
\manroutineitem {Description }{}

Computes a periodogram of a 1-d dataset by a weighted least squares fit
of Sine waves. It is slow compared with the FFT method in POWER. The
advantage of this technique is that genuine variations in the datapoint
errors are accounted for.

\manroutineitem {Parameters }{}
\begin{manparametertable}
\manparameterentry {{\mantt{READ}} }{{\mantt{INP}} }{{\mantt{UNIV}}}
   Name of the input NDF.
\manparameterentry {{\mantt{READ}} }{{\mantt{OUT}} }{{\mantt{UNIV}}}
   Name of output NDF to contain the power spectrum
\manparameterentry {{\mantt{READ}} }{{\mantt{BASE}} }{{\mantt{\_REAL}}}
   Base frequency 
\manparameterentry {{\mantt{READ}} }{{\mantt{INC}} }{{\mantt{\_REAL}}}
   Frequency increment 
\manparameterentry {{\mantt{READ}} }{{\mantt{PHASE}} }{{\mantt{\_LOGICAL}}}
   Store phase information ?
\end{manparametertable}

\end{manroutinedescription}

\manroutine {{\manheadstyle{CROSSPEC}}}{Calculates a cross-spectrum.}

\begin{manroutinedescription}
\manroutineitem {Description }{}

Computes the cross-spectrum of two 1-d arrays. 

\manroutineitem {Parameters }{}
\begin{manparametertable}
\manparameterentry {{\mantt{READ}} }{{\mantt{INP1}} }{{\mantt{UNIV}}}
   Name of the first input array.
\manparameterentry {{\mantt{READ}} }{{\mantt{INP2}} }{{\mantt{UNIV}}}
   Name of the second input array.
\manparameterentry {{\mantt{READ}} }{{\mantt{OUT}} }{{\mantt{UNIV}}}
   Name of the output NDF.
\manparameterentry {{\mantt{READ}} }{{\mantt{SIGMA}} }{{\mantt{\_INTEGER}}}
   Sigma of the smoothing gaussian
\manparameterentry {{\mantt{READ}} }{{\mantt{TAPER}} }{{\mantt{\_LOGICAL}}}
   Taper data ?
\manparameterentry {{\mantt{READ}} }{{\mantt{FRAC}} }{{\mantt{\_REAL}}}
   Fraction of data tapered at each end
\end{manparametertable}

\end{manroutinedescription}

\newpage

\subsection{Folding}

\manroutine {{\manheadstyle{FOLDLOTS}}}{Performs period search by a
standard folding technique.}

\begin{manroutinedescription}
\manroutineitem {Description }{}

Determines the best folding period of a 1-d dataset by performing
a chi-squared fit to the dataset folded at a number of periods.

\manroutineitem {Parameters }{}
\begin{manparametertable}
\manparameterentry {{\mantt{READ}} }{{\mantt{INP}} }{{\mantt{UNIV}}}
   Name of the input NDF.
\manparameterentry {{\mantt{READ}} }{{\mantt{FOLD\_OBJ}} }{{\mantt{UNIV}}}
   Name of the output file containing data folded at the best period
\manparameterentry {{\mantt{READ}} }{{\mantt{CHI\_OBJ}} }{{\mantt{UNIV}}}
   Name of the output file containing the chi-squared v period data
\manparameterentry {{\mantt{READ}} }{{\mantt{PERIOD}} }{{\mantt{\_REAL}}}
   Base period for folding
\manparameterentry {{\mantt{READ}} }{{\mantt{PINC}} }{{\mantt{\_REAL}}}
   Period increment
\manparameterentry {{\mantt{READ}} }{{\mantt{NPER}} }{{\mantt{\_INTEGER}}}
   Number of periods to fold at
\manparameterentry {{\mantt{READ}} }{{\mantt{PHASE_0_EPOCH}} }{{\mantt{\_DOUBLE}}}
   Epoch of phase zero, in atomic time.
\manparameterentry {{\mantt{READ}} }{{\mantt{N_PHASE_BINS}} }{{\mantt{\_INTEGER}}}
   Number of phase bins to use
\manparameterentry {{\mantt{READ}} }{{\mantt{WEIGHT}} }{{\mantt{\_LOGICAL}}}
   Are weighted means required ?
\end{manparametertable}

\end{manroutinedescription}

\manroutine {{\manheadstyle{FOLDBIN}}}{Folds an n-dimensional dataset
at a given period}

\begin{manroutinedescription}
\manroutineitem {Description }{}

Folds an n-d datafile at a chosen period.

\manroutineitem {Parameters }{}
\begin{manparametertable}
\manparameterentry {{\mantt{READ}} }{{\mantt{INP}} }{{\mantt{UNIV}}}
   Name of the input NDF.
\manparameterentry {{\mantt{READ}} }{{\mantt{OUT}} }{{\mantt{UNIV}}}
   Name of the output file containing folded data
\manparameterentry {{\mantt{READ}} }{{\mantt{PERIOD}} }{{\mantt{\_REAL}}}
   Period for folding
\manparameterentry {{\mantt{READ}} }{{\mantt{EPOCH}} }{{\mantt{\_DOUBLE}}}
   Epoch of phase zero, in atomic time.
\manparameterentry {{\mantt{READ}} }{{\mantt{BINS}} }{{\mantt{\_INTEGER}}}
   Number of phase bins to use
\manparameterentry {{\mantt{READ}} }{{\mantt{WEIGHTED}} }{{\mantt{\_LOGICAL}}}
   Are weighted means required ?
\end{manparametertable}

\end{manroutinedescription}

\manroutine {{\manheadstyle{FOLDAOV}}}{Performs a period search on a
dataset using the 'Analysis of variances' folding technique.}

\begin{manroutinedescription}
\manroutineitem {Description }{}

Determines the best folding period of a 1-d dataset by performing
an analysis of variances chi-squared fit to the dataset folded at a 
number of frequencies.

\manroutineitem {Parameters }{}
\begin{manparametertable}
\manparameterentry {{\mantt{READ}} }{{\mantt{INP}} }{{\mantt{UNIV}}}
   Name of the input NDF.
\manparameterentry {{\mantt{READ}} }{{\mantt{PERIOD}} }{{\mantt{\_REAL}}}
   Base period for folding
\manparameterentry {{\mantt{READ}} }{{\mantt{PINC}} }{{\mantt{\_REAL}}}
   Period increment
\manparameterentry {{\mantt{READ}} }{{\mantt{NPER}} }{{\mantt{\_INTEGER}}}
   Number of periods to fold at
\manparameterentry {{\mantt{READ}} }{{\mantt{NPBIN}} }{{\mantt{\_INTEGER}}}
   Number of phase bins to use
\manparameterentry {{\mantt{READ}} }{{\mantt{PER}} }{{\mantt{UNIV}}}
   Name of output chi-squared file
\manparameterentry {{\mantt{READ}} }{{\mantt{FOLD}} }{{\mantt{UNIV}}}
   Name of output folded file at best frequency
\end{manparametertable}

\end{manroutinedescription}

\manroutine {{\manheadstyle{PHASE}}}{Converts a time series into a
phase series.}

\begin{manroutinedescription}
\manroutineitem {Description }{}

Converts a time series into a `phase series', given the ephemeris of
the periodicity

\manroutineitem {Parameters }{}
\begin{manparametertable}
\manparameterentry {{\mantt{READ}} }{{\mantt{INP}} }{{\mantt{UNIV}}}
   Name of the input NDF.
\manparameterentry {{\mantt{READ}} }{{\mantt{OUT}} }{{\mantt{UNIV}}}
   Name of the output NDF.
\manparameterentry {{\mantt{READ}} }{{\mantt{COEFF1}} }{{\mantt{\_DOUBLE}}}
   Julian date of an occurence of phase zero (days)
\manparameterentry {{\mantt{READ}} }{{\mantt{COEFF2}} }{{\mantt{\_DOUBLE}}}
   Periodicity at the time of coeff1 (days)
\manparameterentry {{\mantt{READ}} }{{\mantt{COEFF3}} }{{\mantt{\_DOUBLE}}}
   dP/dt, first derivative of period wrt time.
\end{manparametertable}

\end{manroutinedescription}

\manroutine {{\manheadstyle{EVPHASE}}}{Produces a phase list for an
event dataset.}

\begin{manroutinedescription}
\manroutineitem {Description }{}

Adds a phase list to an event dataset, gven the ephemeris of
the periodicity.

\manroutineitem {Parameters }{}
\begin{manparametertable}
\manparameterentry {{\mantt{READ}} }{{\mantt{INP}} }{{\mantt{UNIV}}}
   Name of the input event dataset
\manparameterentry {{\mantt{READ}} }{{\mantt{OUT}} }{{\mantt{UNIV}}}
   Name of the output event dataset
\manparameterentry {{\mantt{READ}} }{{\mantt{COEFF1}} }{{\mantt{\_DOUBLE}}}
   Julian date of an occurence of phase zero (days)
\manparameterentry {{\mantt{READ}} }{{\mantt{COEFF2}} }{{\mantt{\_DOUBLE}}}
   Periodicity at the time of coeff1 (days)
\manparameterentry {{\mantt{READ}} }{{\mantt{COEFF3}} }{{\mantt{\_DOUBLE}}}
   dP/dt, first derivative of period wrt time.
\end{manparametertable}

\end{manroutinedescription}

\newpage

\subsection{Correlation routines}

The correlation routines are limited to working on regularly spaced
data.

\manroutine {{\manheadstyle{ACF}}}{Auto-correlation function.}

\begin{manroutinedescription}
\manroutineitem {Description }{}

Calculates a 1-d auto-correlation function.

\manroutineitem {Parameters }{}
\begin{manparametertable}
\manparameterentry {{\mantt{READ}} }{{\mantt{INP}} }{{\mantt{UNIV}}}
   Name of the input NDF.
\manparameterentry {{\mantt{READ}} }{{\mantt{OUT}} }{{\mantt{UNIV}}}
   Name of the output NDF.
\manparameterentry {{\mantt{READ}} }{{\mantt{AXIS}} }{{\mantt{\_INTEGER}}}
   Index number of axis to perform autocorrelation along
\manparameterentry {{\mantt{READ}} }{{\mantt{MXLAG}} }{{\mantt{\_INTEGER}}}
   Maximum lag to calculate ACF at.
\manparameterentry {{\mantt{READ}} }{{\mantt{BIAS}} }{{\mantt{\_LOGICAL}}}
   Bias the ACF ?
\manparameterentry {{\mantt{READ}} }{{\mantt{WEIGHT}} }{{\mantt{\_LOGICAL}}}
   Weight the dataset ?
\end{manparametertable}

\end{manroutinedescription}

\manroutine {{\manheadstyle{CROSSCOR}}}{Cross-correlates two 1-d arrays.}

\begin{manroutinedescription}
\manroutineitem {Description }{}

Computes the cross-correlation of two 1-d arrays. 

\manroutineitem {Parameters }{}
\begin{manparametertable}
\manparameterentry {{\mantt{READ}} }{{\mantt{INP1}} }{{\mantt{UNIV}}}
   Name of the first input array.
\manparameterentry {{\mantt{READ}} }{{\mantt{INP2}} }{{\mantt{UNIV}}}
   Name of the second input array.
\manparameterentry {{\mantt{READ}} }{{\mantt{OUT}} }{{\mantt{UNIV}}}
   Name of the output NDF.
\manparameterentry {{\mantt{READ}} }{{\mantt{LAG}} }{{\mantt{\_INTEGER}}}
   Maximum lag to be computed
\manparameterentry {{\mantt{READ}} }{{\mantt{WEIGHT}} }{{\mantt{\_LOGICAL}}}
   Use variances to weight calculation ?
\manparameterentry {{\mantt{READ}} }{{\mantt{NOISE}} }{{\mantt{\_LOGICAL}}}
   Remove noise variance from calculation ?
\end{manparametertable}

\end{manroutinedescription}

\newpage

\subsection{General}

\manroutine {{\manheadstyle{BARYCORR}}}{Barycentric correction.}

\begin{manroutinedescription}
\manroutineitem {Description }{}

Barycentrically corrects an event dataset. This application corrects
for the motion of the earth and of the particular satellite. A file
giving the position of the satellite with respect to the earth as a 
function of time is required if a spacecraft correction is to be applied.

\manroutineitem {Parameters }{}
\begin{manparametertable}
\manparameterentry {{\mantt{READ}} }{{\mantt{INP}} }{{\mantt{UNIV}}}
   Name of the input event dataset
\manparameterentry {{\mantt{READ}} }{{\mantt{SATCORR}} }{{\mantt{\_LOGICAL}}}
   Correct for the satellite motion ?
\manparameterentry {{\mantt{READ}} }{{\mantt{POS_FILE}} }{{\mantt{CHARACTER}}}
   Name of the satellite orbit file 
\manparameterentry {{\mantt{READ}} }{{\mantt{GO_ON}} }{{\mantt{\_LOGICAL}}}
   Ignore invalid orbit file and perform earth-centred correction ?
\manparameterentry {{\mantt{READ}} }{{\mantt{UPDATE_PRD}} }{{\mantt{\_REAL}}}
   Period after which correction is re-calculated
\manparameterentry {{\mantt{READ}} }{{\mantt{UPDATE_INT}} }{{\mantt{\_REAL}}}
   If no orbit file, the period after which correction is updated
\end{manparametertable}

\end{manroutinedescription}

\manroutine {{\manheadstyle{DIFDAT}}}{Calculates the difference between
adjacent data points in an array.}

\begin{manroutinedescription}
\manroutineitem {Description }{}

Calculates the difference between adjacent data points in a data array.
The object is to reduce low-frequency power, making it effectively a
high pass filter.

\manroutineitem {Parameters }{}
\begin{manparametertable}
\manparameterentry {{\mantt{READ}} }{{\mantt{INP}} }{{\mantt{UNIV}}}
   Name of the input NDF
\manparameterentry {{\mantt{READ}} }{{\mantt{OUT}} }{{\mantt{UNIV}}}
   Name of the output dataset
\end{manparametertable}

\end{manroutinedescription}

\manroutine {{\manheadstyle{TIMSIM}}}{Simulates a time series.}

\begin{manroutinedescription}
\manroutineitem {Description }{}

Simulates an Asterix binned time series. This may consist of
any number of sine waves, square waves, ramps or saw tooths with 
added noise and poisson statistics.

\manroutineitem {Parameters }{}
\begin{manparametertable}
\manparameterentry {{\mantt{READ}} }{{\mantt{OUT}} }{{\mantt{UNIV}}}
   Name of the output NDF.
\manparameterentry {{\mantt{READ}} }{{\mantt{AXIS}} }{{\mantt{\_LOGICAL}}}
   Is the time axis to be taken from another file ?
\manparameterentry {{\mantt{READ}} }{{\mantt{INP}} }{{\mantt{UNIV}}}
   Input file name (if AXIS is true)
\manparameterentry {{\mantt{READ}} }{{\mantt{DUR}} }{{\mantt{\_REAL}}}
   Length of dataset (seconds)
\manparameterentry {{\mantt{READ}} }{{\mantt{WID}} }{{\mantt{\_REAL}}}
   Width of each time bin
\manparameterentry {{\mantt{READ}} }{{\mantt{INIT}} }{{\mantt{\_REAL}}}
   Initial time value
\manparameterentry {{\mantt{READ}} }{{\mantt{OPT}} }{{\mantt{\_CHAR}}}
   Options required in time series, e.g. Sin wave ....
\manparameterentry {{\mantt{READ}} }{{\mantt{LEV1}} }{{\mantt{\_REAL}}}
   Background level
\manparameterentry {{\mantt{READ}} }{{\mantt{RAM1}} }{{\mantt{\_REAL}}}
   Initial level for ramp
\manparameterentry {{\mantt{READ}} }{{\mantt{RAM2}} }{{\mantt{\_REAL}}}
   Final level for ramp
\manparameterentry {{\mantt{READ}} }{{\mantt{NUMSIN}} }{{\mantt{\_INTEGER}}}
   Number of sine waves
\manparameterentry {{\mantt{etc...}} }{{\mantt{etc...}} }{{\mantt{etc...}}}
\end{manparametertable}

\end{manroutinedescription}

\manroutine {{\manheadstyle{STREAMLINE}}}{Removes BAD quality data from a
time series.}

\begin{manroutinedescription}
\manroutineitem {Description }{}

Reduces the size of a time series by removing the bad quality data.
This can be useful for very long, relatively blank datasets which 
take ages to process because of all the paging required.

\manroutineitem {Parameters }{}
\begin{manparametertable}
\manparameterentry {{\mantt{READ}} }{{\mantt{INP}} }{{\mantt{UNIV}}}
   Name of the input NDF.
\manparameterentry {{\mantt{READ}} }{{\mantt{OUT}} }{{\mantt{\_LOGICAL}}}
   Name of the output NDF.

\end{manparametertable}

\end{manroutinedescription}

\manroutine {{\manheadstyle{VARTEST}}}{Tests for variability.}

\begin{manroutinedescription}
\manroutineitem {Description }{}

Tests a 1-d dataset for variability. It first finds the source count 
rate from an input source and background time series and then finds 
the probability that the least likely background point originated from 
a constant source. Both the source and background data need to have
a poissonian distribution for this technique to be valid.

\manroutineitem {Parameters }{}
\begin{manparametertable}
\manparameterentry {{\mantt{READ}} }{{\mantt{INP}} }{{\mantt{UNIV}}}
   Name of the input source file
\manparameterentry {{\mantt{READ}} }{{\mantt{BACK}} }{{\mantt{UNIV}}}
   Name of the input bckground file
\manparameterentry {{\mantt{READ}} }{{\mantt{AREA}} }{{\mantt{\_REAL}}}
   Ratio of the source box to background box areas
\manparameterentry {{\mantt{WRITE}} }{{\mantt{PFUNC}} }{{\mantt{\_REAL}}}
   Probability statistic.
\end{manparametertable}

\end{manroutinedescription}

\newpage

\section{Which application to use}

Table 1 illustrates the response of the main applications
to certain types of input data. This table shows the relative 
performance of the power spectral applications, POWER, LOMBSCAR
and CLEAN and the folding routines, FOLDLOTS and FOLDAOV. CLEAN
is an algorithm which works by identifying the main peak in a
power spectrum and then subtracting the effects of the window function
on the resultant spectrum. It was written by Harry Lehto at Southampton
(SOTON::HL) and is NOT currently available in the Asterix package, but
is incuded here for reference. It is hoped that it will be available 
in Asterix before too long, however, in the meantime users may contact 
Harry directly to use the program.

Each scenario in the table was simulated by adding one or more signals 
to a certain amount of background noise, introducing poissonian 
statistics and then removing a percentage of the data samples.
A noise rate of 95 \% (e.g. scenario 3) means that the mean background 
count rate was 95 per time bin and the amplitude of the signal was
5 per time bin. Data gaps were set to be either periodic or random,
e.g. '50 R' means that 50 percent of the datapoints were removed
randomly from the dataset, '50 P' means that half the datapoints
were removed in a periodic fashion, usually 5 points present, 5 missing,
5 points present etc... Points refers to the number of datapoints
present after the datagaps have been introduced. In the scenario
column, '1 freq' means a single sine wave, '10,12 s' means two sine
waves of period 10 and 12 s, square wave, saw tooth and delta
function refer to the type of input signal.

The key to the performance of each application is as follows:

\begin{itemize}

    \item[{\bf M}] - means that the signal was not detected by the program
    \item[{\bf O}] - means that the signal was obviously detected but not quite
             at the correct frequency.
    \item[{\bf F}] - means that the signal WAS detected by the program
    \item[{\bf G}] - means the signal was well detected by the program

    \item[{\bf S}] - means that spurious peaks were present at a high level
             in the periodogram

    \item[{\bf *}] - means that this was demonstrably the best application
             for this scenario 

    \item[{\bf +}] - means that the performance of this application was
             worse than that of the others in this scenario.

\end{itemize}

From table 1, it can be seen that the basic FFT routine, POWER,
works well on periodic signals contained in evenly sampled datasets
with NO gaps. LOMBSCAR and the CLEAN algorithm both perform well
on broken datasets. CLEAN in general produces less spurious peaks
than the other applications but in one or two cases, 
LOMBSCAR is superior at resolving two 
close frequencies. The folding programs behave acceptably but are
only superior to the power spectrum applications for very sharp
signals.

\subsection{Execution times}

Table 2 shows the execution times of the various programs on an
unloaded Vaxstation 3100. It can be seen that POWER is much the 
quickest program for more than a few thousand samples. LOMBSCAR
is reasonably quick for upto 10000 samples as are the folding routines. 
The version of CLEAN tested here is very slow for more than a few hundred
samples and would ordinarily need to be run in batch for large
datasets.

\section{How to display the data}
 Bob's bit


\section{Example session}

Example 1

This example shows a period analysis of a ROSAT XRT source.

1) Prepare the orbit file for the barycentric correction program.

\begin{verbatim}
$ XRTORB                         
XRTORB  Version 1.4-2
ORBFIL - Name of file containing XRT orbit data /@ORB/ > GX349_ORB
\end{verbatim}

2) Perform barycentric corrections on the raw event data

\begin{verbatim}
$ BARYCORR GX349_EVRAW
 Input data file :
  DISK$<SCRATCH>:[ASTERIX88]GX349_EVRAW.SDF;2
There are 48944 events in this file
SATCORR - Correct for satellite orbit /TRUE/ > 
POS_FILE - Name of orbit file /'ATT_FILE.DAT'/ > 
\end{verbatim}

3) Bin up the event data

\begin{verbatim}
$ EVBIN
 EVBIN Version 1.3-1
INP - Enter name of input dataset /@GX349_EVRAW/ > 
The available lists are:
 1  X_CORR         
 2  Y_CORR         
 3  X_DET          
 4  Y_DET          
 5  RAW_TIMETAG    
 6  PULSE_HEIGHT_CH
 7  CORR_PH_CH     

Select the lists to be binned, by entering the index numbers.
E.g. 1 2 3
LISTS - Index numbers > 5

 RAW_TIMETAG axis:

BASE1 - Enter axis base value /326/ > 
The RAW_TIMETAG data range is 326 to 128435
The intrinsic width is 0.
BINSIZE1 - Enter binsize /250.2129/ > 250.
This will give 512 RAW_TIMETAG bins
OUT - Enter name of the output dataset > GX349_TIM
A total of 48722 events were binned out of 48944 input
\end{verbatim}

4) Exposure correct the binned dataset and set quality of bins
   with zero exposure to bad.

\begin{verbatim}
$ XRTCORR
XRTCORR version 1.4-6
INP - Input dataset name /@GX349_TIMC/ > GX349_TIM
OUT - Output dataset name /@GX349_TIMC/ > GX349_TIMC
RTNAME - Rootname for calibration files /'[HIS.2142]X2142'/ > 
Warning: insufficient eventrate data to calculate dead time correction for 464 t
imebins
Correction for these bins has been set to 1.0
Warning: bad eventrate data for 18 timebins
Correction for these bins has been set to 1.0
Mean dead time correction : 1.001657
Mean vignetting correction : 1.001776
Mean scattering correction : 1.000025
Mean wire correction : 1.388889
\end{verbatim}

5) Do a period search on this file from 1300 to 1800 seconds.

\begin{verbatim}
$ FOLDLOTS
FOLDLOTS - V1.5-1
INP - Enter name of input file /@GX349_TIMC/ > 
Using 63 of the 512 input data values
FOLD_OBJ - Enter name of output fold file /@GX349_FOLD/ > GX349_F
CHI_OBJ - Enter name of output chi-sq vs. period file /@GX349_PER/ > GX349_P
TIME UNITS are seconds
Mean time increment : 250
PERIOD - Enter start period for folding /1400/ > 1300.
PINC - Enter increment period for folding /1/ > 
NPER - Enter number of periods to fold at /400/ > 500
EPOCH - Enter epoch of phase zero /6775.0817013889/ > 
BINS - Enter number of phase bins /5/ > 
WEIGHTED - Are weighted means required? /FALSE/ > 
Best fit period = 1748 seconds
Chi-sq value of best fit to mean = 14631.43
\end{verbatim}

6) Plot the periodogram produced

\begin{verbatim}
$ DRAW GX349_P VWS
DRAW Version 1.0-14

Dataset: DISK$<SCRATCH>:[ASTERIX88]GX349_P
\end{verbatim}

7) Perform a Fourier analysis on the data

\begin{verbatim}
$ LOMBSCAR
LOMBSCAR Version 1.5-1
INP - Input datafile /@GX349_TIMC/ > 
Using 63 of the 512 input data values
OFAC - Oversampling factor /4/ > 
HIFAC - High frequency factor /1/ > 
Peak found at 1.9569472E-06 of amplitude 2.629137 and significance 0.9910592
Graphics device/type (? to see list): CANL
OUT - Output datafile /@GX349_LS/ > 

\end{verbatim}

\section{Conclusions}
I hope that the material in this document will soon become dated
as better techniques are found. Please mail LTVAD::RDS if you 
know of a time series application which is superior or faster than 
those listed here in a given situation. Contributions involving
working software of any standard are especially welcome.

\section{Acknowledgements}
Many thanks to those people who contributed software to this
project, especially, David Holmgren, Koji Mukai, Harry Lehto,
Andy Norton and Simon Duck.



\include{tb_res}
\include{tb_tim}

\end{document}                                  % End
