
\documentstyle[11pt]{article}     % 10% larger letters, equns to left 
\pagestyle{myheadings} 
%------------------------------------------------------------------------------ 
\newcommand{\astdoccategory}  {ASTERIX Programming Note} 
\newcommand{\astdocinitials}  {PROG} 
\newcommand{\astdocnumber}    {011} 
\newcommand{\astdocauthors}   {DJ Allan} 
\newcommand{\astdocdate}      {20 February 1992} 
\newcommand{\astdoctitle}     {The ASTERIX PSF System Routines} 
\newcommand{\astdocname}      {\astdocinitials /\astdocnumber} 

\renewcommand{\_}             {{\tt\char'137}} 
 
\newcommand{\Array}{\mbox{\tt Array}}
\newcommand{\BinDS}{\mbox{\tt BinDS}}
\newcommand{\EventDS}{\mbox{\tt EventDS}}
\newcommand{\File}{\mbox{\tt File}}
\newcommand{\Scalar}{\mbox{\tt Scalar}}
\newcommand{\VecArray}{\mbox{\tt VectorisedArray}}
\newcommand{\VecBinDS}{\mbox{\tt VectorisedBinDS}}
\newcommand{\XYimage}{\mbox{\tt XYimage}}
\newcommand{\Ind}{\hspace{3mm}}
%------------------------------------------------------------------------------ 
 
\setlength{\textwidth}{160mm}           % Text width 16 cm 
\setlength{\textheight}{240mm}          % Text height 24 cm 
\setlength{\oddsidemargin}{0pt}         % LH margin width, -1 inch 
\setlength{\evensidemargin}{0pt}        % LH margin width, -1 inch 
\setlength{\topmargin}{-5mm}            % 
\setlength{\headsep}{8mm}               %  
\setlength{\parindent}{0mm} 
 
\begin{document}                                % Start document 
\thispagestyle{empty} 
DEPARTMENT OF SPACE RESEARCH \hfill \astdocname\\ 
BIRMINGHAM UNIVERSITY\\ 
{\large\bf Asterix Data Analysis\\} 
{\large\bf \astdoccategory\ \astdocnumber} 
\begin{flushright} 
\astdocauthors\\ 
\astdocdate 
\end{flushright} 
\vspace{-4mm} 
\rule{\textwidth}{0.5mm} 
\vspace{5mm} 
\begin{center} 
{\huge\bf \astdoctitle} 
\end{center} 
\vspace{5mm} 
 
\parskip=4.0mm
 
\tableofcontents 
 
\newpage 
\markright{\astdocname} 
 
\section{Introduction} 

     This document describes the ASTERIX PSF system in order that users
     can write applications to use the PSF routines or write their own
     PSF data routines.

     A description of the user interface can be found in the ASTERIX on
     line help system ASTHELP.

\section{Overview}

     The PSF system is designed to allow application software to access
     information which is instrument dependent without incorporating any
     instrument dependent code. The architecture of the system is split
     into 2 distinct sections, the 'system' routines and the 'data'
     routines,

\begin{verbatim}
              application   <->  PSF system  <->  PSF libraries
\end{verbatim}

     The former are the application's sole contact with the PSF system.
     A PSF library is a collection of subroutines which is constructed
     in such a way that its components can be recompiled and the whole
     library relinked without any rebuilding at the application level.
     On VAX systems this functionality is achieved using shareable
     images. The PSF system supports mutliple PSF libraries to enable
     additional instruments to be interfaced without distrurbing other
     libraries. 

\section{The Application Level}

\subsection{Initialisation and Linking}
     
     Any application using the PSF system must perform all access
     between calls to the two system routines PSF\_INIT and PSF\_CLOSE.
     The former ensures that the common block is explicitly initialised
     (vital for monoliths) and the latter that all memory/file resources
     are released by the system before the application returns control 
     to the calling environment. To link with the PSF routines, just
     include the standard ASTERIX options file on the link line, eg.

\begin{verbatim}
       ALINK application,...,ASTOPT/OPT
\end{verbatim}

     Note that the INIT and CLOSE routines, unlike most such pairs
     in ASTERIX subpackages, require a STATUS argument. This is for
     the simple reason that they can generate errors when executing.

\subsection{Associating a Dataset with a PSF}

     The PSF system associates a dataset with a PSF using one of the
     two routines PSF\_ASSOCI, PSF\_ASSOCO, depending on whether read
     or write access is being performed. Both routines return a PSF
     reference number to the application, which should be declared as
     type INTEGER. All subsequent access to PSF system routines should
     pass this reference number as the first argument. More than one
     PSF can be associated with the same dataset, and more than one
     dataset can be associated simultaneously. The total number of
     active references is defined by the symbolic constant PSF\_NMAX
     in the ASTERIX include file PSF\_PAR.

     Before discussion of the routines which return useful data, the
     routine PSF\_RELEASE should be mentioned. This simply frees all
     internal storage used by the PSF system for a specified reference.     

\subsection{Accessing PSF data}

     Discussion of the arguments to PSF routines is not complete 
     without a definition of the coordinate system used. 

\begin{verbatim}
         +---------------------------------------------------+
         |                                                   |     
         |                                                   |     
         |                            ^ +-----------------+  |     
         |                            | |                 |  |     
         |                            | |   +             |  |     
         |                            NY|(X0,Y0) +        |  |     
         |                            | |        (X0+QX,  |  |     
         |                         +  | |         Y0+QY)  |  |     
         |                      (0,0) v +------ARRAY------+  |     
         |                              <--------NX------->  |     
         |                                                   |     
         |                                                   |     
         |                                                   |     
         |                                                   |     
         |                                                   |     
         +---------------- dataset X,Y space ----------------+
\end{verbatim}

     The dataset associated with a PSF must have enough axis information
     to define the X,Y space in angular units. The ASTERIX standards 
     document PROG\_002 defines the pointing direction as (0,0) in this
     space.

     The most primitive PSF access routine is PSF\_2D\_DATA which directly
     drives the PSF library routines. Other system routines such as the
     profilers RADIAL\_PFL and ENERGY\_PFL are built on top of PSF\_2D\_DATA.
     The argument list of PSF\_2D\_DATA is,

\begin{verbatim}
       PSF_2D_DATA( SLOT, X0, Y0, QX, QY, DX, DY, INTEG, NX, NY,
                                                 ARRAY, STATUS )

         SLOT		<int>   The PSF reference
         X0,Y0		<real>  The image position at which the PSF is to
                        	be evaluated, in radians
         QX,QY		<real>  Radian offsets from the position (X0,Y0)
                        	to the centre of the array ARRAY
         DX,DY		<real>  The widths of the bins in ARRAY
         INTEG		<log>   Return integrated probability per pixel?
         NX,NY  	<int>   The dimensions of ARRAY
         ARRAY(NX,NY)	<real>  The 2D probability array
\end{verbatim}

     The units of the values in ARRAY when INTEG is true are integrated 
     probability per unit pixel area DX*DY. Thus, if the image area ARRAY
     is sufficiently large, the sum of the values returned should approach
     or even equal unity. The units of if INTEG is false are currently
     undefined.

     The discussion above has mentioned nothing of potential time or
     energy dependence of the PSF. Most application software is optimised
     to handle spatial variation efficiently, with time and energy being
     treated as extensions. The PSF system is organised to reflect this
     method of programming - the data routines only provide control over
     the spatial variables. To change the energy or time separate calls
     to PSF\_DEF must be made. This allows time and energy bands, as well
     as user definable data to be passed to the PSF data routine. A 
     typical structure for processing such data where the PSF also varies
     with image position might look like,

\begin{verbatim}
       CALL PSF_ASSOCI( LOC, SLOT, STATUS )
       DO E = 1, NCHANNEL
         CALL PSF_DEF( SLOT, ..., STATUS )
         DO X = 1, NX
           DO Y = 1, NY
             CALL PSF_2D_DATA( SLOT, ... STATUS )
           END DO
         END DO 
       END DO
\end{verbatim}

     Note that the PSF system routines know nothing about time or 
     energy. The sole means by which the PSF\_DEF call has any effect
     is by altering the internal state of the PSF library routine
     generating the 2D PSF data.

\section{Creating a PSF Library or Library Routine}

\subsection{PSF system architecture}

     ASTERIX maintains a list of PSF libraries it can access in the
     environment variable AST\_PSF\_LIB\_PATH. This text string contains
     the names of the library or libraries or separated by commas. The
     PSF library containing the PSF's supplied with ASTERIX release is
     called PSFLIB. On VMS systems the PSF library name must be a
     logical name consisting of the filename of the executable image
     (see next section), "EXE" is taken as the default file extension.
     For example,

\begin{verbatim}
       $ ASSIGN DISK$WORK1:[ASTERIX]TESTPSFLIB TESTPSFLIB
       $ AST_PSF_LIB_PATH = "TESTPSFLIB"
\end{verbatim}
     
     would set up TESTPSFLIB.EXE in the directory indicated as the only
     active library in the PSF system. Alternatively, the following

\begin{verbatim}
       $ AST_PSF_LIB_PATH = "TESTPSFLIB,PSFLIB"
\end{verbatim}
 
     adds the library to the ASTERIX library search list. Note that  
     the logical assignment MUST NOT contain any process or job logical 
     names, only SYSTEM names are translated by VMS when translated 
     shareable image names. Details of how to construct a PSF library 
     are dealt with in system specific sections at the end of this note.

\subsection{The PSF Library Structure}

     A PSF library is a shareable image constructed from a series of 
     subroutines, some of which have names and arguments defined by 
     the interface to the PSF system routines. The recommended means
     to maintain a PSF library is to put all the object modules in a
     object library. Consider the hypothetical example of a library 
     called MYLIB which contains two PSF data routines, DET1 and DET2.
     To link this object library into a shareable PSF library, run 
     the ASTERIX development procedure PSFLIBLINK,

\begin{verbatim}
       PSFLIBLINK MYLIB
\end{verbatim}

     This procedure expects to find,

\begin{itemize}
\item
       An object library called MYLIB.OLB
\item
       A link options file called MYLIB.OPT
\end{itemize}

     and creates a file called MYLIB.EXE. The link options file is 
     very important, and the source of most difficulties creating a 
     PSF library. It must contain,

\begin{itemize}
\item
       A UNIVERSAL statement for each externally visible
          subroutine
\item
       A PSECT statement for each named common block (Fortran)
          or global variable (C) used in the PSF library.
\end{itemize}

\subsection{PSF\_SHARE\_INIT and PSF\_SHARE\_CLOSE}

     To continue the example above, here is the beginning of a 
     possible MYLIB.OPT options file

\begin{verbatim}
       UNIVERSAL=PSF_SHARE_INIT,PSF_SHARE_CLOSE
       PSECT=DET1_CMN,NOSHR
       PSECT=DET2_CMN,NOSHR
       ...
\end{verbatim}

     The first line introduces the first 2 special routines. These
     routines are recognised by the PSF system by NAME. The second
     and third lines name common blocks used by the PSF routines,
     and specify them NOSHR. Failure to name a common block will
     not prevent a PSF library linking, but will result in a PSF
     system error "failed to load library NAME".

     The mandatory PSF\_SHARE\_INIT routine is called the library 
     when PSF\_INIT is invoked by the application. Its purpose is
     two-fold; it returns the number and name of the PSFs supplied
     by the library, and performs initialisation tasks for ALL the
     routines in the library, such as zeroing common blocks.

     PSF\_SHARE\_CLOSE, if present, is executed by PSF\_CLOSE for each
     library that has been accessed at least once by the PSF system.

\subsection{A Data routine}

     For a PSF of name $<$name$>$, the data routine is called PSF\_$<$name$>$,
     eg. PSF\_DET1 or PSF\_DET2 in our example. The arguments to this
     routine are specified in Section 6. These routines have to be
     externally visible so we require the following additional lines
     in our example options file,

\begin{verbatim}
       UNIVERSAL=PSF_DET1,PSF_DET2
\end{verbatim}

     The data routine will usually require some ancillary information
     to generate PSF data. Such data usually only needs accessed once
     when the PSF is selected by the user. The means to do this is 
     through the PSF\_$<$name$>$\_INIT routine.

\subsection{Data \_INIT routine}

     When a PSF is selected using one of the PSF\_ASSOC routines, the
     routine PSF\_$<$name$>$\_INIT is invoked if present. This routine is
     supplied the slot number and locator to the associated dataset.
     The former can be used to index a common block inside the library
     if required - the locator is available for any access the library
     routine might need, eg. getting a filter id.

     Note that the \_INIT routine should never annul the locator it
     is given, and should either free any resources it uses or store
     them in COMMON so that the PSF\_$<$name$>$\_CLOSE routine can do so.

     Again, if an \_INIT routine is required, it must be specified in
     the options file, eg.

\begin{verbatim}
       UNIVERSAL=PSF_DET1_INIT
\end{verbatim}

     The PSF system defines two ADAM parameters called MASK and AUX
     for the use of \_INIT routines. These can be used to get auxilliary
     information from the user (see PSF\_ANAL\_INIT in AST\_PSF:PSFLIB.TLB
     for a good example). The interface file parameter type for both
     is "LITERAL" which means that scalar values and filesnames can
     be read using these parameters; vectors, however, must be parsed
     after reading PAR\_GET0C. The prompts require defining using the
     PAR\_PROMT subroutine. \_INIT routines should always PAR\_CANCL any
     environment parameter accessed. This permits multiple calls to an
     \_INIT routine for different PSFs.

\subsection{Data \_CLOSE routine}
  
     This routine, if present, is called by PSF\_RELEASE or PSF\_CLOSE
     when a PSF is being released from the PSF system. Its usual
     function is to free any resources allocated by the $<$name$>$ or
     $<$name$>$\_INIT routines.

     Again, if an \_INIT routine is required, it must be specified in
     the options file, eg.

\begin{verbatim}
       UNIVERSAL=PSF_DET1_CLOSE
\end{verbatim}

\subsection{Data \_DEF routine}

     The PSF\_$<$name$>$\_DEF routine is invoked, if present, by a call to 
     the PSF system routine PSF\_DEF. This link is provided in order 
     that any internal settings stored in COMMON by a PSF data routine
     and associated routines can be changed. The seetings can then
     be access by the data routine.

\subsection{Data \_PFL routine}

     The PSF\_$<$name$>$\_PFL routine provides an mechanism to generate
     energy profile values directly. The PSF system routine 
     PSF\_ENERGY\_PFL has an algorithm to perform energy profiling
     by repeated calls to the routine. However, this routine is
     susceptible to PSFs with discontinuities and is very slow.
     If energy radii can be calculated analytically for your PSF
     it is best to write this routine.

\subsection{Examples}

     A good simple example of a collection of PSF routines is
     provided by the XRT\_PSPC routines in AST\_PSF:PSFLIB.TLB. A
     not so simple one is the ANAL PSF in the same library.

\section{The PSF system routines}

     The following subset of the PSF system routines are for use by
     application software. Note that ";" in argument lists shows
     the break between the input arguments to the left, and output
     arguments to the right.

\rule{\textwidth}{0.5mm}
\begin{verbatim}
       PSF_INIT( STATUS ) 
\end{verbatim}

         Initialise the PSF common block and load the available PSF
         libraries.

\rule{\textwidth}{0.5mm}
\begin{verbatim}
       PSF_CLOSE( STATUS )
\end{verbatim}

         Free all PSF system sources

\rule{\textwidth}{0.5mm}
\begin{verbatim}
       PSF_ASSOCI( LOC; SLOT, STATUS )
       
      	 LOC	<char>	  Locator to dataset
         SLOT   <int>     The PSF reference
\end{verbatim}

         Associate a dataset with a PSF using READ access

\rule{\textwidth}{0.5mm}
\begin{verbatim}
       PSF_ASSOCO( LOC; SLOT, STATUS )
       
      	 LOC	<char>	  Locator to dataset
         SLOT   <int>     The PSF reference
\end{verbatim}

         Associate a dataset with a PSF using WRITE access.The name
         of the PSF selected is written to the dataset.

\rule{\textwidth}{0.5mm}
\begin{verbatim}
       PSF_RELEASE( SLOT, STATUS )

         SLOT   <int>     The PSF reference
\end{verbatim}

         Free any resources used by the PSF system for the PSF SLOT

\rule{\textwidth}{0.5mm}
\begin{verbatim}
       PSF_2D_DATA( SLOT, X0, Y0, QX, QY, DX, DY, INTEG, NX, NY,
                                                 ARRAY, STATUS )

         SLOT   <int>     The PSF reference
         X0,Y0  <real>    The image position at which the PSF is to
                          be evaluated, in radians
         QX,QY  <real>    Radian offsets from the position (X0,Y0) to
                          the centre of the array ARRAY
         DX,DY  <real>    The widths of the bins in ARRAY
         INTEG  <log>     Return integrated probability per pixel?
         NX,NY  <int>     The dimensions of ARRAY
         ARRAY  <real>[]  The 2D probability array
\end{verbatim}

         Return a 2D array of probability per unit pixel for the
         PSF specified by SLOT.

\rule{\textwidth}{0.5mm}
\begin{verbatim}
       PSF_DEF( SLOT, LOWT, HIGHT, LOWE, HIGHE, USERIN, USEROUT,
                                                        STATUS )
         SLOT   <int>     The PSF reference
         LOWT   <dble>    The lower time bound in offset from OBS_TAI
         HIGHT  <dble>    The upper time bound in offset from OBS_TAI
         LOWE   <int>     The lower energy channel bound
         HIGHE  <int>     The upper energy channel bound
         USERIN   *       User input to library DEF routine
         USEROUT  *       User output from library DEF routine
\end{verbatim}

         Define the current energy and time bands for subsequent
         PSF evaluation. The time units are offsets from OBS\_TAI,
         ie. the TAI equivalent of the time origin of the dataset,
         and the energy units are channel numbers. Additional 
         USERIN and USEROUT arguments are provided for other
         possible information which could affect the PSF.

\rule{\textwidth}{0.5mm}
\begin{verbatim}
       PSF_ENERGY_PFL( SLOT, NFRAC, FRAC, X0, Y0, RADII, STATUS )

         SLOT   <int>     The PSF reference
         NFRAC  <int>     Number of energy fractions to be evaluated
         FRAC   <real>    Array of fractions for radii to be found
         X0,Y0  <real>    The image position at which the PSF is to
                          be evaluated, in radians
         RADII  <real>[]  The radii in radians
\end{verbatim}

         Returns the radius in radians at which the specified PSF
         SLOT encloses each of the NFRAC energy fractions in FRAC.
         The PSF is evaluated at the image position (X0,Y0).

\rule{\textwidth}{0.5mm}
\begin{verbatim}
       PSF_QMODEL( SLOT, MODEL, STATUS )

         SLOT   <int>     The PSF reference
         MODEL  <logical> PSF is a model
\end{verbatim}

         Returns a logical flag depending on whether the PSF SLOT
         was defined using a PSF model specification.

\rule{\textwidth}{0.5mm}
\begin{verbatim}
       PSF_RADIAL_PFL( SLOT, X0, Y0, DIM, PROF, STATUS )

         SLOT   <int>     The PSF reference
         X0,Y0  <real>    The image position at which the PSF is to
                          be evaluated, in radians
         DIM    <int>     Number of profile values
         PROF   <real>[]  Profile values
\end{verbatim}

         Returns an normalised surface brightness profile of the
         specified PSF SLOT, evaluating the PSF at (X0,Y0).

\rule{\textwidth}{0.5mm}
\begin{verbatim}
       PSF_RESAMPLE( NX, NY, IN, BORDER, OX, OY, OUT, STATUS )
         NX,NY  <int>     Dimensions of IN and OUT arrays
         IN     <REAL>[]  The input 2D PSF
         BORDER <int>     Border in pixels 
         OUT    <REAL>[]  The input 2D PSF
\end{verbatim}

         Used to shift a 2D array of values by an amount less than
         or equal to one pixel in each dimension. The value of
         BORDER which should be at least 1 prevents edge effects
         affecting the values in OUT within this BORDER. The units
         of OX and OY are fraction pixels.

\section{PSF Library Routine Specifications}

\subsection{Library Routines}

    The special PSF library routines are listed below, with a
    description of the purpose of each.

\rule{\textwidth}{0.5mm}
\begin{verbatim}
       PSF_SHARE_INIT( NNAME, NAMES, STATUS )

         NNAME  <INT>     Number of PSF routines in library
         NAMES  <REAL>[]  Names of the PSFs
\end{verbatim}

       Called only on the first occasion a library is accessed. 
       Should be used to initialise any common blocks used by
       the library.

\rule{\textwidth}{0.5mm}
\begin{verbatim}
       PSF_SHARE_CLOSE( STATUS )
\end{verbatim}

       Called once during PSF\_CLOSE() for each PSF library which
       has been accessed.

\subsection{Data Routines}

    The routines below are defined for a PSF of name <name>. Only
    the PSF\_<name> routine is mandatory.

\rule{\textwidth}{0.5mm}
\begin{verbatim}
       PSF_<name>( SLOT, X0, Y0, QX, QY, DX, DY, INTEG, NX, NY,
                                                ARRAY, STATUS )

         SLOT   <int>     The PSF reference
         X0,Y0  <real>    The image position at which the PSF is to
                          be evaluated, in radians
         QX,QY  <real>    Radian offsets from the position (X0,Y0) to
                          the centre of the array ARRAY
         DX,DY  <real>    The widths of the bins in ARRAY
         INTEG  <log>     Return integrated probability per pixel?
         NX,NY  <int>     The dimensions of ARRAY
         ARRAY  <real>[]  The 2D probability array
\end{verbatim}

         Return a 2D array of probability per unit pixel for the
         PSF specified by SLOT.

\rule{\textwidth}{0.5mm}
\begin{verbatim}
       PSF_<name>_INIT( SLOT, LOC, STATUS )

         SLOT   <int>     The PSF reference
         LOC    <char>    Locator to associated dataset
\end{verbatim}

\rule{\textwidth}{0.5mm}
\begin{verbatim}
       PSF_<name>_DEF( SLOT, SLOT, LOWT, HIGHT, LOWE, HIGHE, USERIN,
                                                   USEROUT, STATUS )

         SLOT   <int>     The PSF reference
         LOWT   <dble>    The lower time bound in offset from OBS_TAI
         HIGHT  <dble>    The upper time bound in offset from OBS_TAI
         LOWE   <int>     The lower energy channel bound
         HIGHE  <int>     The upper energy channel bound
         USERIN   *       User input to library DEF routine
         USEROUT  *       User output from library DEF routine
\end{verbatim}

\rule{\textwidth}{0.5mm}
\begin{verbatim}
       PSF_<name>_CLOSE( SLOT, STATUS )

         SLOT   <int>     The PSF reference
\end{verbatim}

\rule{\textwidth}{0.5mm}
\begin{verbatim}
       PSF_<name>_PFL( SLOT, NFRAC, FRAC, RADII, STATUS )

         SLOT   <int>     The PSF reference
         NFRAC  <int>     The number of radii to evaluate
         FRAC   <real>[]  The energy radii required
         RADII  <real>[]  The radii in radians for each energy fraction
\end{verbatim}

\end{document}
