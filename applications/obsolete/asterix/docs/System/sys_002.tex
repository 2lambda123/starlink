\documentstyle[11pt]{article}     % 10% larger letters, equns to left
\pagestyle{myheadings}
%------------------------------------------------------------------------------
\newcommand{\astdoccategory}  {ASTERIX System Note}
\newcommand{\astdocinitials}  {SYS}
\newcommand{\astdocnumber}    {002}
\newcommand{\astdocauthors}   {David J Allan}
\newcommand{\astdocdate}      {7 Jan 1995}
\newcommand{\astdoctitle}     {The GEN\_make Procedure}
\newcommand{\astdocname}      {\astdocinitials /\astdocnumber}
\renewcommand{\_}             {{\tt\char'137}}

%------------------------------------------------------------------------------

\setlength{\textwidth}{160mm}           % Text width 16 cm
\setlength{\textheight}{240mm}          % Text height 24 cm
\setlength{\oddsidemargin}{0pt}         % LH margin width, -1 inch
\setlength{\evensidemargin}{0pt}        % LH margin width, -1 inch
\setlength{\topmargin}{-5mm}            %
\setlength{\headsep}{8mm}               % 
\setlength{\parindent}{0mm}

\begin{document}                                % Start document
\thispagestyle{empty}
ASTROPHYSICS and SPACE RESEARCH GROUP \hfill \astdocname \\
DEPARTMENT OF PHYSICS AND SPACE RESEARCH\\
UNIVERSITY OF BIRMINGHAM\\
{\large\bf Asterix System Note} \hfill \astdocauthors\\
{\large\bf \astdoccategory\ \astdocnumber} \hfill \astdocdate\\
\vspace{-4mm}
\rule{\textwidth}{0.5mm}
\vspace{5mm}
\begin{center}
{\huge\bf \astdoctitle}
\end{center}
\vspace{5mm}

\parskip=4.0truemm plus 0.5truemm       % Paragraph spacing

\markright{\astdocname}

\section{Introduction}

This document describes the \verb+GEN_make+ Makefile processor -- the Tcl
script which generates ASTERIX \verb+Makefiles+. This script exists because
editing the ASTERIX make files is too error prone to be done by hand. It is
better to reduce the varying information to control files and put the
invariant rules on how to build the various components of the system in 
proto-make files. The reduction in volume thus achieved is about a factor
of 40.
 
In addition to the script itself there are only 3 files involved when 
\verb+GEN_make+ is run. They are named according the the {\em class} of
directory for which the \verb+Makefile+ is being constructed. They are,
\begin{description}
\item[class.config] Contains the configurations of the \verb+Makefile+
  which are peculiar to the directory to which it refers. It might, for
  example, contain the names of the programs to be built.

\item[class.make] The proto-type make file. This is the file which 
  contains the skeleton of the \verb+Makefile+. Things which vary from
  one instance of a directory class to another are parameterised using
  GEN\_make tokens, which have the form $colon$\_$identifier$\_$colon$
  (eg. \verb+:FILES:+). Proto-type include files may also contain
  \verb+#include+ directives which cause file inclusion. This enables
  entire sections common to different classes of make files to be shared,
  further reducing the number of potential errors.

\item[class.base\_config]
  This file, in the same format as the {\bf class.config} file, contains
  default definitions for all the tokens used in the proto-make file.
  This serves two purposes. It enables developers to realise the tokens
  which can be used, and it serves to provide default behaviour which
  may or may not need to be overriden.
\end{description}


\section{How It Works}

Consider a directory in which a file called \verb+etc.config+ exists in
the directory \verb+$AST_ROOT/dev/etc+. How
does GEN\_make go about creating a \verb+Makefile+? To get it started
the user types,
\begin{verbatim}
   % GEN_make etc.config
\end{verbatim}
The file extension may be omitted. GEN\_make then follows the following
algorithm,
\begin{enumerate}
\item The class is derived from the name of the configuration file. In this
  case it is \verb+etc+.
  
\item GEN\_make clears its token substitution array. It defines a number
  of standard tokens which are available for use by all proto-make files.
  They are listed below (without their delimiting colons), with their
  values in parentheses where relevant,
  \begin{description}
  \item[CLASS] The name of the class ({\em etc})
  \item[ROOT] The relative file specification to get to \verb+$AST_ROOT+ ({\em ../..})
  \item[HERE] The location of the \verb+Makefile+ relative to \verb+$AST_ROOT+. ({\em dev/etc})
  \item[SUBSYS] The is full name of
    the subsystem. Same as the HERE token, but with slashes converted to
    underscores. ({\em dev\_etc})
  \item[MODULE] The leftmost directory name in the HERE list ({\em dev})
  \item[MODSUB] The rightmost directory name in the HERE list ({\em etc})
  \item[DATE] The date in \verb+14 Jul 1789+ format
  \item[CREATOR] The version of GEN\_make used to create the \verb+Makefile+
  \item[HOST] The machine on which the \verb+Makefile+ was made ({\em xun9})
  \item[PKG] The package name. Hardwired at the moment ({\em ASTERIX})
  \item[INDENT] A joker this one. Essentially a string of spaces whose
    length is $min(1,2 * depth)$. Useful for formatting recursive
    \verb+Makefile+ text.
  \end{description}

\item GEN\_make loads the value of the \verb+AST_MAKE_PATH+ variable. This
  is a colon separated list of the directories which will be searched for
  proto-make files.

\item The script checks that the {\bf $class$.make} and 
  {\bf $class$.base\_config} files
  exist somewhere in \verb+$AST_MAKE_PATH+.
  
\item The token translations in the files {\bf global.base\_config}, 
  {\bf $class$.base\_config} and
  {\bf $class$.config} files are loaded in that order. See the next
  section for a discussion of the format.
  
\item The proto-make file is processed to remove all \verb+#include+
  directives -- the file name argument following the \verb+#include+
  specifies a file which {\em must} lie somewhere in the \verb+$AST_MAKE_PATH+
  directory list. This file flattening process is performed for speed's
  sake for a C tool which lives in the kernel called \verb+at_strm+.

\item GEN\_make now reads this flattened file looking for occurrances of
  its tokens. When it finds one it substitutes the token value defined
  by the configuration files and/or its built in tokens. This process
  continues until each line is free of tokens. The only further
  complication is the \verb+#foreach+ directive. This is a control
  structure within the proto-make file for generating multiple copies
  of the same bit of make text. An example is,
  
  \begin{verbatim}
  #foreach &X& :TASKS:
  $(BIN)/&x&: &x&.f
      alink $? `ast_link_adam`
      mv &x& $@
	
  #end
  \end{verbatim}
  This directive  expands {\em zero or more} times, depending on the 
  number of space separated values in the translation of the token
  specified as the second argument. The first argument is the loop
  variable name -- it must be enclosed in ampersands. This apparently
  silly choice of ampersands and colons was done deliberately to make
  GEN\_make stuff stand out from normal \verb+Makefile+ text, and
  ensure \verb+make+ will break if you leave any such tokens untranslated. 
  Upper
  case instances of the loop variable expand to upper case copies of
  the elements of the second and subsequent arguments, and lower to 
  lower case copies.
  
  The utility of this directive can only be appreciated after a thorough
  understanding of the differences of OSF, Solaris and Linux `make'
  syntax. The above text of a good example of why it useful. The final
  target is in another directory and we only want to rebuild it if it
  is out of date with respect to its source code.

  The (entirely) blank line before the \verb+#end+ marker is required
  to generate correct \verb+Makefile+ source if there is more than one
  repetition of the directive.
  
  The \verb+#foreach+ directives cannot be nested.
  
\item The output file is renamed to be \verb+Makefile+. Any pre-existing
  file of that name is deleted.
\end{enumerate}

\section{Configuration File Syntax}

To recap the behaviour of GEN\_make is quite simple. It translates named
tokens in an input file to token values in an output file. The job of
the configuration files is to manipulate the GEN\_make token translation
table. GEN\_make is written in {\bf Tcl} and the token translation table
is just an associative array called \verb+Subs+.

GEN\_make loads configuration information by \verb+source+ing them, ie. a
configuration file is itself a Tcl script! This cunning trick is a 
generally useful technique and I commend it to you. There are currently
two configuration file commands, called \verb+files+ and \verb+const+.
These are there for historical reasons, in the current implemention
there effect is similar except for the fact that \verb+const+ can have only
one argument, whereas \verb+files+ can have many. The syntax is,
\begin{verbatim}
   files TOKEN values
   
   const TOKEN [value]
\end{verbatim}
In the \verb+const+ case the value defaults to the null string. In 
addition to simply listing file names there are some `specials'. 
These have the following forms,
\begin{verbatim}
   files NEW_TOKEN %remext OLD_TOKEN

   files NEW_TOKEN %reverse OLD_TOKEN
   
   files NEW_TOKEN %upper OLD_TOKEN
   
   files NEW_TOKE %lower OLD_TOKEN
   
   files NEW_TOKEN %prefix <prefix> OLD_TOKEN
   
   files NEW_TOKEN %suffix OLD_TOKEN <suffix>
   
   files NEW_TOKEN %presuf <prefix> OLD_TOKEN <suffix>
\end{verbatim}
The use of these is hopefully clear! They add or subtract bits from
file name to create new tokens, or reorder the elements of a token. 
If the bit you are adding contains
characters special to Tcl (eg dollars) then you should enclose the
bit in braces, eg.

\begin{verbatim}
   # The task names
   
   files TASKS  hdir hcreate hretype

   # The Fortran files
   
   files FORTRAN %suffix .f TASKS
   
   # The executables
   
   files EXECS %prefix {$(BIN)/} TASKS
\end{verbatim}

It is important to realise that all token translation is delayed until
the proto-make file is processed, ie. it is quite valid to use one of
the specials above before \verb+TOKEN2+ is defined.

The ability to do this is extremely useful. We only have to name the tasks
{\em once} -- all the rest is rule based. In Starlink make files task names
often appear in 4 or 5 places. 

\section{Adding a Directory Class}

To add a directory class you must create a \verb+.base_config+ and a
\verb+.make+ file in the \verb+dev/etc+ directory. Then edit the 
\verb+etc.config+ file in that directory, build it, and then you are
ready. 

\end{document}
