\documentstyle[11pt]{article}     % 10% larger letters, equns to left
\pagestyle{myheadings}
%------------------------------------------------------------------------------
\newcommand{\astdoccategory}  {ASTERIX System Note}
\newcommand{\astdocinitials}  {SYS}
\newcommand{\astdocnumber}    {001}
\newcommand{\astdocauthors}   {David J Allan}
\newcommand{\astdocdate}      {27 May 1996}
\newcommand{\astdoctitle}     {ASTERIX System Architecture}
\newcommand{\astdocname}      {\astdocinitials /\astdocnumber}
\renewcommand{\_}             {{\tt\char'137}}

%------------------------------------------------------------------------------

\setlength{\textwidth}{160mm}           % Text width 16 cm
\setlength{\textheight}{240mm}          % Text height 24 cm
\setlength{\oddsidemargin}{0pt}         % LH margin width, -1 inch
\setlength{\evensidemargin}{0pt}        % LH margin width, -1 inch
\setlength{\topmargin}{-5mm}            %
\setlength{\headsep}{8mm}               % 
\setlength{\parindent}{0mm}

\begin{document}                                % Start document
\thispagestyle{empty}
ASTROPHYSICS and SPACE RESEARCH GROUP \hfill \astdocname \\
DEPARTMENT OF PHYSICS AND SPACE RESEARCH\\
UNIVERSITY OF BIRMINGHAM\\
{\large\bf Asterix System Note} \hfill \astdocauthors\\
{\large\bf \astdoccategory\ \astdocnumber} \hfill \astdocdate\\
\vspace{-4mm}
\rule{\textwidth}{0.5mm}
\vspace{5mm}
\begin{center}
{\huge\bf \astdoctitle}
\end{center}
\vspace{5mm}

\parskip=4.0truemm plus 0.5truemm       % Paragraph spacing

\markright{\astdocname}
\tableofcontents
\newpage

\section{Introduction}

This document explains the system architecture of the ASTERIX package
and in particular the way in which make files are used. It also 
highlights some of the reasons why things are done the way they are.
Sometimes things me seem more complicated than they need be, but before
changing anything fundamental be sure to test that it will work on all
the machines supported (be especially wary of vendor specific make file
features), and that it will work in both the development and publicly 
released systems.

\subsection{The Basics}

All Starlink software uses the \verb+SYSTEM+ environment variable
to control machine dependent aspects of software construction. It
is normal for accounts doing software development to set this up
in their \verb+.cshrc+ file. The values {\em currently} in use
are,
\begin{description}
\item[sun4\_Solaris] SUN SPARCs running Solaris
\item[alpha\_OSF1] DEC Alphas running OSF1
\item[ix86\_Linux] Generic PC architecture running Linux
\end{description}
In addition we have supported {\bf sun4} (SUNs running SunOS) and
{\bf mips} (DECstations running Ultrix) in the past, although we
don't any longer.

All Starlink software uses the UNIX system utility \verb+make+ to build
its software. The associated make files are written in a generic fashion
with the aim of containing no system dependent statements. Operations
which are potentially non-portable (eg. the command to replace an object
module in an archive) are tokenised. Each Starlink package then has a
Bourne shell script called \verb+mk+ which sets up the values of these
tokens depending on the value of \verb+$SYSTEM+. All make files are 
therefore invoked using this \verb+mk+ script rather than running
\verb+make+ directly.

Make files define a number of targets. The ones used by Starlink
are usually high level operations such as `build the system from the
source code', or `install this package into the Starlink tree'. These
high level targets are called {\em primary targets}. The {\em secondary}
targets are the actual files which are built or modified by the system.
Generally it is good practise not to access the secondary targets directly.
A \verb+make+ target is activated if the file which the target specifies 
is to be built is out of date with respect to its dependencies. A target
with no dependencies is always activated.

For primary targets it is not unusual to have dummy or zero length
files which due to their date stamp in the file system can be used
as date carrying flags. Starlink software usually maintains 3 such files
in each directory called \verb+.BUILT+, \verb+.INSTALLED+ and \verb+.TESTED+.
These flag files tell the system in which of 4 potential states the package
exists. In ASTERIX, because we build our volatile files from one copy of
the source code, these files are suffixed with the value of \verb+$SYSTEM+.
Starlink aren't too keen on this, but who cares, and they get flack for
having multiple copies of software.

\subsection{Revision Control}
The ASTERIX development system uses {\bf sccs}, the UNIX {\em s}ource 
{\em c}ode {\em c}ontrol {\em s}ystem. This is done for two very different
reasons,
\begin{itemize}
\item Almost unbelievably the text archive format is not portable between
  SunOS and Solaris, and not up-datable between Solaris and OSF. A container
  is required for text modules, and updates to \verb+tar+ archives are not
  universally supported.

\item We need some way of preserving old versions of code which accessible
  with resorting to system backups.
\end{itemize}
The result is that each library or application group subdirectory in ASTERIX
contains an SCCS archive. Some useful SCCS commands are,
\begin{center}
\begin{tabular}{|l|l|} \hline
Command & Purpose \\ \hline
\verb+mkdir SCCS+            & Start a new SCCS archive \\
\verb+sccs create <module>+  & Create a new module \\
\verb+sccs get <module>+     & Extract a read only copy of a module \\
\verb+sccs edit <module>+    & Extract a writable copy of a module \\
\verb+sccs prs <module>+     & Display revision history of a module \\
\verb+sccs get -r<rev> <module>+ & Extract a particular revision \\
\verb+sccs -r<rev> SCCS+     & Increment version counter in archive \\ \hline
\end{tabular}
\end{center}

Typical mistakes using SCCS include,
\begin{itemize}
\item Making a read only copy of a file, editing it (because \verb+jed+ allows
  this), and \verb+replace+ing it. Effect is to change object module correctly
  but {\em does not} update the source archive. Result is that next time
  you rebuild the library the change goes away!
  
\item Deleting the writable copy of a module. Simply re-extract the
  module using the \verb+sccs get -r+ mode (a simple \verb+sccs get+
  fails because SCCS believes the module is write locked), change the
  protection using \verb+chmod +w+ and then replace it.
\end{itemize}

If you make a change which subsequently turns out to have been ill-advised
then you can recover old versions by discarding subsequent revisions. See
the manual for details. You can also recover the whole archive to a
particular date.

It is important to realise that only the development system uses SCCS --
Starlink just get \verb+tar+ files containing the source modules and
build it from those. This duality is controlled by the \verb+AST_SYS_DEV+
environment variable which causes the \verb+mk+ script to make alternate
token substitutions. Starlink build ASTERIX without setting this variable,
we define a value for it (which is itself irrelevant)  in the \verb+sysdev+
procedure.

\subsection{The Primary Targets}

Lets go back to the state files \verb+BUILT_$SYSTEM+ etc. A Starlink
package (and for us, any ASTERIX directory) can exist in 4 states, unbuilt
(or source), built, installed and tested. The system is in the first state
if you unpack the source code distribution. The primary targets can now
be understood in terms of transitions between these states. 
\begin{description}
\item[build] Changes the state from unbuilt to built by constructing 
  all the volatile, usually system dependent, files (eg executables)
  Success is marked by the presence of the \verb+.BUILT_$SYSTEM+ file.
  
\item[unbuild] Goes from built back to unbuilt by deleting the volatile
  files and the built flag file.
  
\item[check] Check that all the files that will be needed for a `build'
  are in fact present. Generally a test that the `tar' file was not
  corrupted
  
\item[clean] Deletes any temporary files created by the `build' target. 
  Generally a null task in ASTERIX

\item[install] Change the state from built to installed. Involves moving
  and sometimes editing the volatile files into different directories.
  Create the \verb+.INSTALLED_$SYSTEM+ flag file. Generally the built
  volatiles are replaced with soft links to their installed counterparts
  to save disk space.
  
\item[deinstall] Recover the built system from the installed state. It
  is quite important that this pair of targets are well written such 
  that if the install partially fails then the built system can be
  correctly recovered.

\item[export] Builds a \verb+tar+ file containing all the built files for
  the architecture specified by the \verb+SYSTEM+ environment variable.
  The tar file is created compressed in the top-level directory with 
  name \verb+asterix_$SYSTEM.tar.Z+.
  
\item[export\_source] Builds a \verb+tar+ file containing all the source
  and make files required to generate the binary built system and the
  developer system. The file is created compressed in the top-level 
  directory with name \verb+asterix.tar.Z+.
  
\item[test] Run a test procedure and create the test flag file. Not used
  by ASTERIX.

\end{description}
In addition ASTERIX make files support a few more primary targets,
\begin{description}
\item[help] List the primary targets available to this directory
\item[developer] Create the development system from the normal source
  distribution. This involves setting up the SCCS directories.
\item[makes] Recreate the \verb+Makefile+ for each of the dependent 
  sub-directories of the current directory. Invoked at the top level
  of the ASTERIX distribution this will rebuild all the make files except the
  top level one (which is a special case anyhow).
\item[showtree] Display the directory structure from the current directory
  downwards. Invoked from the top-level shows the ASTERIX directory tree
\item[rebuild] Just performs a {\bf unbuild}, then a {\bf build} and
  finally a {\bf clean}.
\item[get\_s\_files] Lists source files in the directory and all
  sub-directories. This target is used by the \verb+build_export+ target
  to construct the releasable files.
\item[get\_b\_files] Lists built files in the directory and all sub-directories.
  Used for building the binary release.
\end{description}




\section{The ASTERIX File Structure}

The ASTERIX top level development directory contains a number of {\em modules}
each of which has its own directory, plus one directory for each architecture
on which the system is built. The latter directories have the same name as
their corresponding \verb+$SYSTEM+ value. Source files for modules are broken
up into different directories depending on their type. The system has been
designed so that a directory contains only one class of file (eg.
applications, script or data) which enables the make-files required to build
those files be parameterised using the GEN\_make facility. 

\subsection{The {\tt build} system}

Each module sub-directory constructs built files from the source files (even
if they are only copies), and moves them into the appropriate sub-directory
of the system dependent directory for that architecture. There are only 5 
such sub-directories.
\begin{description}
\item[dates] Text date stamp files. These are using to store details
  of build times and macro definitions extant at the time of build. One
  for every directory in ASTERIX.
\item[bin] Executables and scripts. This is the directory which is
  added to the user's \verb+PATH+ string.
\item[etc] Everything else needed by the run-time system, apart from
  help text and documentation. Contains data files, non-executable
  scripts (eg. Tcl routines) and anything else you care to mention.
\item[etc/sys] This contains a small subset of files needed by the ASTERIX
  system. They are those which control the modular aspect of the system,
  and include the module start up scripts and include file tables. Their
  presence in this directory greatly speeds up the ASTERIX startup because
  many fewer directory entries need to be scanned compared with keeping
  them in {\tt etc}.
\item[etc/docs] Contains all ASTERIX internal documentation, apart from 
  SUN 98, the guide to the whole package.
\item[lib] Object archives and object modules
\item[help] Online help info
\end{description}
The entire ASTERIX system runs from within these directories, and relies
on nothing outside. This results in duplication of the contents of the 
\verb+etc+ directory which is unnecessary as the files are portable, but
greatly simplifies the overall architecture of the system, and satisfies
Starlink requirements for architecture separation.

\subsection{The {\tt install} system}

The ASTERIX development system runs in the built state, but for 
external consumption Starlink insist we distribute our files into
the Starlink directory structure. This is done by moving the built
files from the build system directory \verb+xxx+ to \verb+/star/xxx/asterix+.
Most files can simply be copied, but some scripts have to be edited.
When the {\bf install} target moves the files is replaces the copies
in the build directories with soft links to the installed files. The
{\bf deinstall} target destroys these links and moves the files back
to the build directories.

\subsection{Modules}

The module is the smallest optional unit in ASTERIX, ie. it is the unit
in which sub-packages of ASTERIX are distributed. All modules reside at
the top level of the file structure, although not all top level directories
need be modules.

\subsection{The Directory Classes}

The various classes of ASTERIX directories are discussed below. There
is nothing to stop new classes being defined, but you should ensure
that they behave according to the general principles.

\subsubsection{The General Principles}

All ASTERIX directories can contain sub-directories. The list of such 
directories is defined by the token {\bf SUBDIRS}. The order specified
in this token is the order in which the sub-directories are built and
installed -- these operations take place {\em before} the build and
install phases of the directory which is the container. The {\bf SUBDIRS}
token defines another token called {\bf REV\_SUBDIRS}, which is the
order in which sub-directories are unbuilt and deinstalled.

A useful target supported by {\em all} ASTERIX directory classes is the
\verb+showtree+ target. This recursively displays the names of all the
directories in the {\bf SUBDIRS} list, and all their sub-directories,
and so on until a directory is reached which has no sub-directories. The
output looks something like,

\begin{verbatim}
   % cd $AST_ROOT
   % mk showtree

     dev/
       etc
       scripts
     kernel/
       hed/
         scripts
       lib/
         etc
         inc
         scripts
       util/
         scripts
     ....
     rosat/
       scripts
\end{verbatim}

\subsubsection{The {\tt etc} Class}

The {\tt etc} class is the simplest directory class. Source files are
simply copied to the {\tt etc} sub-directory of \verb+$AST_ROOT/$SYSTEM+
in the build phase, and \verb+$INSTALL/etc/asterix+ in the install phase.

Often the files which are specified by the {\tt etc} directory are
calibration files, which are perhaps computed by some application or
supplied by an external agency. Each such file should have a subdirectory
containing either the code required to make the file or relevant
documentation. This directory
is not part of the system as far as building and installing is
concerned, and is ignored by the \verb+mk+ system. An example of this
is the ASCA SIS psf calibration, which lives in \verb+kernel/etc+. 
A subdirectory called \verb+SISpsfGenerator+ exists which contains
the code which makes this file.

The only mandatory token required by the \verb+etc+ class make file
is one called {\bf ETC\_FILES}. You can list all the files under this token
or you can split them, which is neater. For example,
\begin{verbatim}
   files BITMAPS marilyn.xbm pamella.xbm claudia.xbm
  
   files COLTABS mary.sdf mungo.sdf midge.sdf
  
   #
   # The mandatory token
   #
   files ETC_FILES :BITMAPS: :COLTABS:
\end{verbatim}
  
\subsubsection{The {\tt docs} Class}

A simple variant of the \verb+etc+ class which simply copies its files
into the directory \verb+$AST_ROOT/$SYSTEM/etc/docs+. The only token used
is {\bf DOC\_FILES} which is a list of the documents to be copied. Current
practice in ASTERIX is to have all user documents available in an installation
regardless of the modules actually installed. System maintainance and
programmer documentation may however be module specific.

\subsubsection{The {\tt scripts} Class}

Contains {\em public} and {\em private} scripts. Public scripts are
executable files which are built to the directory
\verb+$AST_ROOT/$SYSTEM/bin+,
and installed to \verb+$INSTALL/bin/asterix+. Private scripts are
auxiliary files (``subroutines''), which are not intended to be
directly referenced by the user. These are built to 
\verb+$AST_ROOT/$SYSTEM/etc+ and installed to \verb+$INSTALL/etc/asterix+.

There are 4 types of supported scripts -- C shell scripts, Bourne shell
scripts, ICL procedures and Tcl procedures. The first three require no
special treatment by the \verb+mk+ system, but the Tcl files require
the location of the \verb+tclsh+ interpreter and the ADAM Tk 
library edited in. These locations are specified by token substitutions
in the \verb+mk+ script.

Only those tokens denoting source for the class of scripts in a particular
directory are mandatory, eg.
\begin{verbatim}
   files PRIVATE_TCL_SRC sub1.tcl doit.tcl
   
   files PUBLIC_ICL_SRC  mickey.icl mouse.icl language.icl
\end{verbatim}
We have no Bourne shell scripst so we don't have to specify the
{\bf PUBLIC\_SH\_SRC} or the {\bf PRIVATE~\_SH\_SRC} tokens.

\subsubsection{The {\tt tools} Class}

The {\tt tools} class supports C and Fortran stand-alone programs. These
tools are used to support ASTERIX operations. Examples include the file
streamer \verb+at_strm+ (used by GEN\_make) and the help reader. The
requirement on the linking of these tools is that they must {\em not}
depend on any other part of ASTERIX. They may require linking against
Starlink libraries. Executables are built to the \verb+$AST_ROOT/$SYSTEM/bin+
directory and installed in \verb+$INSTALL/bin/asterix+.

\subsubsection{The {\tt inc} Class}

The purpose of this class is to support C and Fortran include files.
The C ones are no problem, it's the Fortran ones that cause the hassle.
All ASTERIX Fortran uses the old VMS style logical names for include
files. These are translated by the \verb+v2u_include+ script into
absolute UNIX file names. This process is performed by the Starlink
\verb+forconv+ program which reads in a file called \verb+include.sub+
which is a two column text file containing the mappings from VMS style
include file names to UNIX ones. The file of these mappings is constructed
on the fly by \verb+v2u_include+, and destroyed after translation (although 
their is a persistent mode to support bulk compilation). The mapping
file is made by concatenating all the mapping files generated for each
of the ASTERIX \verb+inc+ directories. These mapping files are stored
in \verb+$AST_ROOT/$SYSTEM/etc/sys+ in the built system.

It is worth mentioning that C modules in ASTERIX have to have their
include file paths specified somehow. This is done in two ways. Include
paths which are system dependent are included in the {\bf mk} script 
using the \verb+CFLAGS+ environment variable. The \verb+replace+ script
in \verb+dev/scripts+ then adds the current directory, the ASTERIX library
include directory and the Starlink include directory to the include 
path list, ie.the equivalent of 
\begin{verbatim}
   -I. -I$AST_ROOT/kernel/lib/inc -I/star/include
\end{verbatim}
on the command line. The strategy of C use in ASTERIX has been to restrict
it to the kernel routines -- there are only two exceptions, 
\verb+fit__misc_c.c+ in the spectral fitting and \verb+htrace_int.c+
in the HSD editor packages. If it became necessary to start adding more
C routines then there might then be a case for a local \verb+GEN_make+ token
containing local C include directives to give the same flexibility as 
is already implemented in the Fortran case.

In summary there are only two tokens in this directory class, namely
{\bf F\_INCS} and {\bf C\_INCS}. The only built file is the Fortran
mapping file which ends up in \verb+$AST_ETC/sys/<subsystem>.includes+
where $subsystem$ is the full subsystem name of the include file diretory.

\subsubsection{The {\tt lib} Class}

The {\tt lib} class handles the construction of one or more object libraries

\subsubsection{The {\tt appgrp} Class}

Handles the code associated with an ADAM application group.

\subsubsection{The {\tt module} Class}

The module directory class supports both the \verb+export+ and
\verb+export_source+ targets, just like the top level. This enables
individual modules to be released as separate entities. 

The mandatory tokens supported by the module class are {\bf VERSION}
and {\bf DESCRIPTION}. The values of these tokens are used to 
provide information for the selection of modules at user ASTERIX startup.
A module must supply a startup script for each supported shell (currently 
only \verb+csh+). This script is \verb+sourced+ for each module that is
to be initialised.

The module class obeys the {\bf EXPORT} and {\bf EXPORT\_SOURCE} targets
which construct binary+source and source only module releases respectively. The
module release is written to a compressed \verb+tar+ file, the location of which
is specified using the \verb+EXPORT+ environment variable, the default
being the current directory, ie. the module top level directory. This 
facility can be used to release modules of ASTERIX separately of other
modules.

\subsubsection{The {\tt master} Class}

The master class is the template for the top level package. The only
build function performed by such a directory is the creation of the
system Tcl index file in \verb+$AST_ETC/tclIndex+. This is done after
building all the sub-directories which are all modules at this level.

The master class obeys the {\bf EXPORT} and {\bf EXPORT\_SOURCE} targets
which construct binary+source and source only releases respectively. The
release is written to a compressed \verb+tar+ file, the location of which
is specified using the \verb+EXPORT+ environment variable, the default
being the current directory.

\subsection{The {\tt cmk} script}

Now that the directory class structure has been introduced we can mention
the \verb+cmk+ script without hopelessly confusing the reader. \verb+cmk+
takes two or more arguments. The first argument is the name of a directory 
class (eg. \verb+script+, \verb+appgrp+ etc). Subsequent arguments are
concatenated to form a command. \verb+cmk+ then effectively runs \verb+mk+
with that command {\em only} on directories whose class is the same as
the first argument from the current directory downwards. For example,
\begin{verbatim}
   % cd $AST_ROOT
   
   % cmk scripts unbuild
\end{verbatim}
Unbuilds every scripts directory in ASTERIX. This is an extremely useful
thing to be able to do. If you have to make a change to the way the
script directory class is handled, you can unbuild all the script directories,
change the \verb+.base_config+ and/or \verb+.make+ files, reconstruct
the \verb+Makefiles+ using \verb+mk makes+ then rebuild.
Thus although the whole system is rather more fragmented then it used to
be, giving a finer degree of control of parts of the system, there is
also more opportunity for labour saving tricks.

\subsection{Bootstrapping}

It is inevitable that at some point you will break the \verb+GEN_make+
script in such a way that the make files it generates contain a flaw
which makes it impossible to rebuild the make files using the {\bf makes}
target. A procedure has
been written which gets around this problem. Assuming that you have
identified the flaw in the \verb+GEN_make+ source in \verb+dev/scripts+,
you can execute the script \verb+GEN_bootstrap+ {\em in the top level
ASTERIX directory}. This rebuilds the ASTERIX make files using nothing 
except the configuration files, \verb+dev/scripts/GEN_make+ and
\verb+kernel/tools/at_strm.c+.

\section{Building a Release}

To release ASTERIX we generate an exportable copy of the source code
using the \verb+export_source+ top level target. We supply this file
plus the \verb+mk+ script to Starlink who build it on each system. 
They check that the \verb+install+
and \verb+deinstall+ targets work ok, then they build the binary export 
version using \verb+export+. It is this machine specific file which is
released, along with a news item and update to SUN/98.

To test a release go through the same process, building ASTERIX on a 
scratch area. It is essential that you do the following before you start,
\begin{itemize}
\item Make sure that the current directory specifier "." is not in your
  \verb+PATH+ variable before building the package. For security reasons
  most people work like this, and its is important that the ASTERIX make
  files do not make the assumption that this directory is in the path.

\item On Solaris ensure that {\bf LD\_LIBRARY\_PATH} is set correctly.
  It should be 
  \begin{verbatim}
  /star/lib:/usr/openwin/lib:/usr/ucblib:/opt/X11R5/lib:/star/share
  \end{verbatim}
  It absolutely must not contain anything to with XANADU or FTOOLS when
  you build ASTERIX, or dire consequences will ensue.

\item Make sure that no ASTERIX environment variables are active. 
  You can force this using,
  \begin{verbatim}
  unsetenv AST*
  \end{verbatim}
  If you don't do this then the build system might look for the source
  code in the SCCS libraries, which of course won't exist.  

\item Set the value of the {\bf TMPDIR} environment variable to a scratch
  disk area, preferably not auto-mounted on the machine your doing the
  build on.
\end{itemize}
Once you've done this set up, invoke the top level build target, logging
both standard output and errors. Here's the whole thing to this point,
starting from building the release.
\begin{verbatim}

   # Create test directory and tell mk that this is where the exported
   # source will go

   % mkdir /scratch/asterix/test_build
   % setenv EXPORT /scratch/asterix/test_build

   # Build the release and copy the 'mk' script

   % cd /star/asterix
   % mk export_source
   % cp mk /scratch/asterix/test_build/

   # Unpack the source archive

   % cd /scratch/asterix/test_build/
   % zcat asterix.tar | tar xvf -

   # Build the system

   % ./mk build >&build.log &
\end{verbatim}
Now inspect the log file. Look for the word 'error' using an editor. The
only occurance of this word should be in the names of routines, eg.
\verb+serror+.
Once this has worked try the install and deinstall phases.
\begin{verbatim}
   # Set installation directory

   % setenv INSTALL /scratch/asterix/test_install

   % ./mk install >&install.log &
\end{verbatim}
Again look for errors in the log file. Once installed you should be able
to redefine the \verb+aststart+ command to run the installed version. You
must try this before you release the system, as this is the mode in which
everyone else in the world uses the system. Once you've tested it, check
the deinstall target in the same way -- this restores the system to the
built state.


\section{Open Issues}

This is a list of those things which haven't been finished in ASTERIX V2.
\begin{itemize}
\item The documentation directory. I have implemented this as a simple
variant of the \verb+etc+ directory class, with the documents being copied
into \verb+$AST_ETC/docs+. Starlink doesn't really have a system for recognising
internal package documents, so I propose we simply don't tell them about
this.

\item The help system.
\end{itemize}

\end{document}
