\documentstyle[11pt]{article}
\pagestyle{myheadings}

% -----------------------------------------------------------------------------
% ? Document identification
\newcommand{\stardoccategory}  {Starlink User Note}
\newcommand{\stardocinitials}  {SUN}
\newcommand{\stardocsource}    {sun18.6}
\newcommand{\stardocnumber}    {18.6}
\newcommand{\stardocauthors}   {Martin Ricketts}
\newcommand{\stardocdate}      {12 March 1996}
\newcommand{\stardoctitle}     {RPS \\[1ex] ROSAT Proposal Submission}
\newcommand{\stardocversion}   {Version 5.4}
\newcommand{\stardocmanual}    {User's Guide}
% ? End of document identification
% -----------------------------------------------------------------------------

\newcommand{\stardocname}{\stardocinitials /\stardocnumber}
\markright{\stardocname}
\setlength{\textwidth}{160mm}
\setlength{\textheight}{230mm}
\setlength{\topmargin}{-2mm}
\setlength{\oddsidemargin}{0mm}
\setlength{\evensidemargin}{0mm}
\setlength{\parindent}{0mm}
\setlength{\parskip}{\medskipamount}
\setlength{\unitlength}{1mm}

% -----------------------------------------------------------------------------
%  Hypertext definitions.
%  ======================
%  These are used by the LaTeX2HTML translator in conjunction with star2html.

%  Comment.sty: version 2.0, 19 June 1992
%  Selectively in/exclude pieces of text.
%
%  Author
%    Victor Eijkhout                                      <eijkhout@cs.utk.edu>
%    Department of Computer Science
%    University Tennessee at Knoxville
%    104 Ayres Hall
%    Knoxville, TN 37996
%    USA

%  Do not remove the %\begin{rawtex} and %\end{rawtex} lines (used by 
%  star2html to signify raw TeX that latex2html cannot process).
%\begin{rawtex}
\makeatletter
\def\makeinnocent#1{\catcode`#1=12 }
\def\csarg#1#2{\expandafter#1\csname#2\endcsname}

\def\ThrowAwayComment#1{\begingroup
    \def\CurrentComment{#1}%
    \let\do\makeinnocent \dospecials
    \makeinnocent\^^L% and whatever other special cases
    \endlinechar`\^^M \catcode`\^^M=12 \xComment}
{\catcode`\^^M=12 \endlinechar=-1 %
 \gdef\xComment#1^^M{\def\test{#1}
      \csarg\ifx{PlainEnd\CurrentComment Test}\test
          \let\html@next\endgroup
      \else \csarg\ifx{LaLaEnd\CurrentComment Test}\test
            \edef\html@next{\endgroup\noexpand\end{\CurrentComment}}
      \else \let\html@next\xComment
      \fi \fi \html@next}
}
\makeatother

\def\includecomment
 #1{\expandafter\def\csname#1\endcsname{}%
    \expandafter\def\csname end#1\endcsname{}}
\def\excludecomment
 #1{\expandafter\def\csname#1\endcsname{\ThrowAwayComment{#1}}%
    {\escapechar=-1\relax
     \csarg\xdef{PlainEnd#1Test}{\string\\end#1}%
     \csarg\xdef{LaLaEnd#1Test}{\string\\end\string\{#1\string\}}%
    }}

%  Define environments that ignore their contents.
\excludecomment{comment}
\excludecomment{rawhtml}
\excludecomment{htmlonly}
%\end{rawtex}

%  Hypertext commands etc. This is a condensed version of the html.sty
%  file supplied with LaTeX2HTML by: Nikos Drakos <nikos@cbl.leeds.ac.uk> &
%  Jelle van Zeijl <jvzeijl@isou17.estec.esa.nl>. The LaTeX2HTML documentation
%  should be consulted about all commands (and the environments defined above)
%  except \xref and \xlabel which are Starlink specific.

\newcommand{\htmladdnormallinkfoot}[2]{#1\footnote{#2}}
\newcommand{\htmladdnormallink}[2]{#1}
\newcommand{\htmladdimg}[1]{}
\newenvironment{latexonly}{}{}
\newcommand{\hyperref}[4]{#2\ref{#4}#3}
\newcommand{\htmlref}[2]{#1}
\newcommand{\htmlimage}[1]{}
\newcommand{\htmladdtonavigation}[1]{}

%  Starlink cross-references and labels.
\newcommand{\xref}[3]{#1}
\newcommand{\xlabel}[1]{}

%  LaTeX2HTML symbol.
\newcommand{\latextohtml}{{\bf LaTeX}{2}{\tt{HTML}}}

%  Define command to re-centre underscore for Latex and leave as normal
%  for HTML (severe problems with \_ in tabbing environments and \_\_
%  generally otherwise).
\newcommand{\latex}[1]{#1}
\newcommand{\setunderscore}{\renewcommand{\_}{{\tt\symbol{95}}}}
\latex{\setunderscore}

%  Redefine the \tableofcontents command. This procrastination is necessary 
%  to stop the automatic creation of a second table of contents page
%  by latex2html.
\newcommand{\latexonlytoc}[0]{\tableofcontents}

% -----------------------------------------------------------------------------
%  Debugging.
%  =========
%  Remove % on the following to debug links in the HTML version using Latex.

% \newcommand{\hotlink}[2]{\fbox{\begin{tabular}[t]{@{}c@{}}#1\\\hline{\footnotesize #2}\end{tabular}}}
% \renewcommand{\htmladdnormallinkfoot}[2]{\hotlink{#1}{#2}}
% \renewcommand{\htmladdnormallink}[2]{\hotlink{#1}{#2}}
% \renewcommand{\hyperref}[4]{\hotlink{#1}{\S\ref{#4}}}
% \renewcommand{\htmlref}[2]{\hotlink{#1}{\S\ref{#2}}}
% \renewcommand{\xref}[3]{\hotlink{#1}{#2 -- #3}}
% -----------------------------------------------------------------------------
% ? Document specific \newcommand or \newenvironment commands.
% ? End of document specific commands
% -----------------------------------------------------------------------------
%  Title Page.
%  ===========
\renewcommand{\thepage}{\roman{page}}
\begin{document}
\thispagestyle{empty}

%  Latex document header.
%  ======================
\begin{latexonly}
   CCLRC / {\sc Rutherford Appleton Laboratory} \hfill {\bf \stardocname}\\
   {\large Particle Physics \& Astronomy Research Council}\\
   {\large Starlink Project\\}
   {\large \stardoccategory\ \stardocnumber}
   \begin{flushright}
   \stardocauthors\\
   \stardocdate
   \end{flushright}
   \vspace{-4mm}
   \rule{\textwidth}{0.5mm}
   \vspace{5mm}
   \begin{center}
   {\Huge\bf  \stardoctitle \\ [2.5ex]}
   {\LARGE\bf \stardocversion \\ [4ex]}
   {\Huge\bf  \stardocmanual}
   \end{center}
   \vspace{5mm}

% ? Heading for abstract if used.
   \vspace{10mm}
   \begin{center}
      {\Large\bf Abstract}
   \end{center}
% ? End of heading for abstract.
\end{latexonly}

%  HTML documentation header.
%  ==========================
\begin{htmlonly}
   \xlabel{}
   \begin{rawhtml} <H1> \end{rawhtml}
      \stardoctitle\\
      \stardocversion\\
      \stardocmanual
   \begin{rawhtml} </H1> \end{rawhtml}

% ? Add picture here if required.
% ? End of picture

   \begin{rawhtml} <P> <I> \end{rawhtml}
   \stardoccategory \stardocnumber \\
   \stardocauthors \\
   \stardocdate
   \begin{rawhtml} </I> </P> <H3> \end{rawhtml}
      \htmladdnormallink{CCLRC}{http://www.cclrc.ac.uk} /
      \htmladdnormallink{Rutherford Appleton Laboratory}
                        {http://www.cclrc.ac.uk/ral} \\
      \htmladdnormallink{Particle Physics \& Astronomy Research Council}
                        {http://www.pparc.ac.uk} \\
   \begin{rawhtml} </H3> <H2> \end{rawhtml}
      \htmladdnormallink{Starlink Project}{http://star-www.rl.ac.uk/}
   \begin{rawhtml} </H2> \end{rawhtml}
   \htmladdnormallink{\htmladdimg{source.gif} Retrieve hardcopy}
      {http://star-www.rl.ac.uk/cgi-bin/hcserver?\stardocsource}\\

%  HTML document table of contents. 
%  ================================
%  Add table of contents header and a navigation button to return to this 
%  point in the document (this should always go before the abstract \section). 
  \label{stardoccontents}
  \begin{rawhtml} 
    <HR>
    <H2>Contents</H2>
  \end{rawhtml}
  \renewcommand{\latexonlytoc}[0]{}
  \htmladdtonavigation{\htmlref{\htmladdimg{contents_motif.gif}}
        {stardoccontents}}

% ? New section for abstract if used.
  \section{\xlabel{abstract}Abstract}
% ? End of new section for abstract
\end{htmlonly}

% -----------------------------------------------------------------------------
% ? Document Abstract. (if used)
%   ==================

This Note describes RPS released to support proposal submission for
AO-7 of the Rosat Pointed Observation Programme, which has a deadline
for submission of May 15th 1996.  AO-7 covers observations with the HRI
and WFC. RPS provides data-checking, proposal printing, and electronic
proposal submission.  Details of the AO-7 Proposer's Guide are
available from the Leicester Database and Archive Service WWW page:
\htmladdnormallink{{\tt http://ledas-www.star.le.ac.uk/}}{http://ledas-www.star.le.ac.uk/}

% ? End of document abstract
% -----------------------------------------------------------------------------
% ? Latex document Table of Contents (if used).
%  ===========================================
 \newpage
 \begin{latexonly}
   \setlength{\parskip}{0mm}
   \latexonlytoc
   \setlength{\parskip}{\medskipamount}
   \markright{\stardocname}
 \end{latexonly}
% ? End of Latex document table of contents
% -----------------------------------------------------------------------------
\newpage
\renewcommand{\thepage}{\arabic{page}}
\setcounter{page}{1}
 
\section{\label{se:introduction}\xlabel{introduction}Introduction}

\subsection{Overview}

The RPS package provides all the tools necessary for the generation and
electronic submission of observing proposals for the UK ROSAT Pointed 
Observation Programme. Briefly RPS allows the user to do the following:

\begin{enumerate}

\item create a new ROSAT Proposal Form (RPF) by `filling it in'
interactively

\item edit an existing RPF, for example to make minor changes to
observation details

\item summarise the details of an RPF

\item produce a laser printer version of the RPF for submission to the
UK Programme

\item produce a version of the RPF that can be transmitted over the
network and submit this electronically to the ROSAT UK Data Centre
(UKDC) at RAL

\item check when a target will be observable by Rosat and get an
estimate of the survey exposure

\end{enumerate}

Before starting to use RPS, the user ought to be familar with the
requirements of the UK ROSAT Pointed Observation Programme as laid out
in the Announcement of Opportunity (AO) cover letter and the Technical
Description. Copies of these documents are available from the UKDC at
RAL. 

The RPS package is designed to enable guest observers to fulfil the
most important requirements of the UK Programme, namely to submit a
computer-readable version of their RPF to the UKDC and also to submit 2
paper copies of the complete proposal --- consisting of the RPF plus
the scientific case for the proposal.

The use of the RPS package offers a number of advantages to users, for
example it will check the data you enter into each form, ensuring that
the form you submit will not be rejected because of incorrectly entered
or incomplete data.  There is now only the {\bf Unix} version of the
package which works in line-entry mode.  If you have any queries
regarding RPS then contact the UKDC, Internet mail address  {\tt
rosatmail@sdc1.bnsc.rl.ac.uk}, or Janet addresses {\tt
savax::rosatmail} or {\tt rosatmail@uk.ac.rl.savax}.

\subsection{Proposal Forms}
\label{sse:forms}

The ROSAT Proposal Form (RPF) for one Proposal consists of four
sections, namely Cover, General, Target and Constraints `forms',
as follows:

\begin{enumerate}

\item The {\bf Cover Page} which contains details of the Principal
Investigator (PI), the proposal title, subject category and the number
of targets proposed. The printed cover page also includes the abstract.

\item The {\bf General Form} which contains the
details of any Co-Investigators (Co-Is), the official endorsement of
the PI's institution and the PI declaration.

\item One or more {\bf Target Summary} sheets which contain the
technical details of the proposed observation(s) of each target.

\item If any targets have time constraints, these are summarised on a
{\bf Constraints Summary} sheet. The types of time-constraint that can
be applied to an observation are explained in Chapter 9 of the
Technical Description.

\item If there are any remarks, these are printed on one or more separate 
{\bf Remarks Summary} sheets.

\end{enumerate}

In addition, the abstract of the scientific case has to be entered; an
editor is provided for this.

\subsection{File names}

RPS uses various files to store and output the RPF, as follows, where
{\tt propname} is the name you enter when starting up the program:

\begin{list}%
{}{\setlength{\leftmargin}{53mm} \setlength{\labelwidth}{40mm} \setlength{\labelsep}{6mm}
\setlength{\listparindent}{0mm} }

\item[{\tt propname} \hfill]  holds the cover and general form
data, where {\tt propname} is the name entered by the user when
first creating the RPF. 

{\bf NB.} \emph{These are binary files in {\tt FACTS}
format, as used within the SCAR package, and are {\bf not} transferable
between different platforms.}

\item[{\tt propname\_target} \hfill] holds the target data. 

\item[{\tt propname.abstract} \hfill] may be created within RPS.
This must be a simple ASCII text file ({\em i.e.}, without \LaTeX\ 
commands).  Without an abstract file the RPF can be printed, but not
submitted by electronic mail.

\item[{\tt propname.tex} \hfill] \LaTeX\ input file. This file can be
processed outside RPS in the usual way if desired.

\item[{\tt propname.post} \hfill] computer-readable version of the
RPF for submission to the UKDC.  If possible users should submit their
RPFs to the UKDC using the option within RPS.

\item[{\tt propname.lis} \hfill] will be created if you want the
{\em Summarise} output in a file rather than to the terminal.  Can be
printed.

\end{list}

\section{\label{se:getting_started}\xlabel{getting_started}Getting Started}
 
There is just one environment variable, {\tt RPS\_AUX}, which prefixes
the file names.  At Starlink sites, if RPS is installed, this is set by
{\tt /star/etc/login}.

The program is initiated by typing:

\begin{verbatim}
      % rps
\end{verbatim}

where {\tt \%} is the Unix shell prompt.

You are asked initially to enter a filename (without suffix) to be used
by RPS for all files created. This is {\tt propname} referred to
above.  Although there is a default (\verb+rps_form+), for all
purposes other than {\em Check Target} you should enter a suitable
name.

Following this you will be presented with the main menu options, which
include accessing the Help library.

When form-filling, {\bf Help} is accessed by entering {\tt `?'} in any
field.  This gives a description of the field and the options allowed
where appropriate.  The main menu {\bf Help} item gives access to the whole
of the {\tt Help} file.

\section{\label{se:main_options}\xlabel{main_options}Main Options}

\subsection{Create New File}

This lets you create a new Proposal Form, using the Cover Page data
from a previous file if you supply a filename.  The fields from the
Cover page are presented, with defaults where appropriate.  See section
\ref{se:form_filling} on \htmlref{Form Filling}{se:form_filling}
for a further description of the facilities available.

After completing the cover page the validation is carried out.  If you
have entered the necessary fields such as your name, Institute, the
Proposal name and category, {\em etc.},  the page will be validated.
If the validation fails, you will be given the first field where an
error was detected and you can go back to it by taking the default.  If
you wish to exit from the form-filling, enter {\tt `E'} instead.

When you have finished with the Cover page, the General form will be
displayed.  Some of these details are only required on the paper
copies, but you can get them printed by entering them here.

Following this the first Target form is entered. Fill in the details of
your first target.  On completion the validation will be done.  If this
indicates the form is unsatisfactory, then correct the appropriate
field and try again. As before, you can exit from an incomplete or
incorrect form by entering {\tt `E'} in reply to the prompt.

If the Constraints flag is set on the Target record then after you have
completed it the Constraints page is entered for you to fill the
relevant fields.  If the validation fails twice on the constraints
form, the program returns to the Target form, in case you wish to
change anything there. There is also a warning if the moon constraint
affects a coordinated observation; you have the options of ignoring it,
or returning to the form, {\em e.g.}, to change the time range.  On
completion of a target / constraints form you are asked if you want
another target. When you reply in the negative you will get the main
menu again.

\subsection{Edit Old File}

Once a form has been created, the form-filling procedures, outlined
above, can be used to change the contents. Also, on each Target form
there is a Target number; this is just incremented when a Form is
created; since this number may decide the precedence when it comes to
scheduling, you may wish to alter it.  The `Review' option allows
deletion of records and changing of Target numbers.

The options are:

\begin{description}
\item[\mbox{}]\mbox{}
\begin{description}

\item [Edit Cover Page] --- Although the Cover page contains the number
of Targets, this is not displayed with the other fields as it is
automatically adjusted whenever records are added or deleted.

\item [Edit General Page] 

\item [Edit Target Data] --- The Constraints form is accessed after the
main Target form if selected.

\item [Add Target] --- to enter data for another target.

\item [Review Targets] --- This displays the Target numbers and names
for up to 20 Targets and then allows two main options. To change the
Target number of record {\tt n} so it becomes Target {\tt m} enter
after the prompt:

\begin{verbatim}
     > Cn,m
\end{verbatim}

and to delete {\em record} {\tt n} ({\bf not} {\em target number} {\tt n}):

\begin{verbatim}
     > Dn
\end{verbatim}

Deleting a Target record does not automatically change any of the other
target numbers, but the records are concatenated.

\end{description}
\end{description}

When exiting from RPS after editing, you will be given the option of
deleting the old (default) version or the new one.

\subsection{Summarise}

This writes a few lines for each target in the form (as many as you
have filters selected).  The output can be to the screen (default) or a
listing file.  The details of the format are as follows:

\begin{center}
\begin{tabular}{rl}
\setlength{\leftmargin}{40mm}
\hspace{20mm}Heading	& Contents\\ \hline
Rec No		& Record number\\
Target Name	& Target name (16 chars)\\
Targ No		& Target number\\
Qual		& Quality indicator - `OK', or `F' if record failed check\\
R.A.		& RA - HH MM SS.S\\
Dec		& dec - DD MM SS.S\\
Start		& start of visible period, due to Sun constraint\\
End		& end visible period\\
		& (there may be two periods listed)\\
No. obs		& Number of observations\\
TC		& type of time constraint, or `none'\\
Time Ksec	& Time requested, ksec (to nearest integer)\\
2 $\times$ srv exp XRT	& Twice estimated XRT survey exposure, Ksec (cf AO fig. 6.1)\\
2 $\times$ srv exp WFC	& Twice estimated WFC survey exposure, Ksec \\
		& Data for each filter:\\
Instr		& Instrument. Z after WFC indicates Zoom on\\
Flt		& Filter\\
\%		& Time percentage requested\\
sec		& Time, total or for particular WFC filter \\
Det cps		& estimated cps for detection:\\
		& WFC:  see AO section 12.2, uses Bcell = 0.01\\
		& HRI:  see AO section 11.2, uses Rb = 0.000013\\
\end{tabular}
\end{center}

\subsection{Check Target}

This can be used to check the dates when a target satisfies the sun
constraint, and also to find what the expected survey exposure,
{\em etc.}, is likely to be; these outputs are as for the {\em Summarise}
option.  Enter the RA and dec separately on request.  These can be in
any format that is valid for the form-filling procedure --- {\tt `?'} will
give you the help information on the field. After one display typing
{\tt `Y'} enables a further target to be tested. the default is to exit to
the main menu.

\subsection{Print File}

All the \LaTeX\ options create a file which you process after leaving RPS.

The options are:

\begin{description}

\item [All forms] --- \LaTeX\ file of whole proposal produced.

\item [Selected Pages] --- \LaTeX\ particular pages,

\item [Blank Forms] --- \LaTeX\ gives you a blank copy of each page
as given in the AO document.

\end{description}

\subsection{Proposal Submission}

This will try to create the {\tt .post} file from the proposal form. It might
fail because:

\begin{enumerate}

\item Errors are detected in the one or more of the pages. Use {\bf
Review Targets} or {\bf Summarise} for more information or enter the
Target or Cover Edit and type {\tt \%} to exit and initiate the
verification.

\item The Target numbers do not form a sequence from 1 to the number of
target records.

\item There is no abstract file yet, or it is longer than 800 characters.

\end{enumerate}

You can choose to let the program mail it to the UKDC, the address being
given on the file {\tt address.dis} in the {\tt RPS\_AUX} directory.
Alternatively, you can exit from RPS and
send it yourself.  A checksum is used to ensure that the paper copies
match the file sent over the network.

{\bf Two paper copies must also be sent to RAL, which have attached the
scientific case (see the AO document). } If you do not have the means
to produce the \LaTeX\ hard-copy version of the RPF forms at your
Institute, please produce your RPF by filling in your entries on a
photocopy of the RPF form as they appear in the AO Technical
Description. You should nevertheless submit the RPF electronically to
the UKDC.

The {\bf Data Protection Act 1984} applies to the database of
proposals; by submitting a proposal it is assumed that you accept its
provisions.

\section{\label{se:form_filling}\xlabel{form_filling}Form Filling Procedure}

For each section, you will be presented with each field name in turn.
The line mode options, apart from moving to the next field
with \verb+<return>+ are as follows:

\begin{center}
\begin{tabular}{|ll|ll|} \hline
Input	&Action&Input	&Action\\ \hline
\verb+%+&to exit out of the form at any time.		&\verb+&+&to go to the previous field.\\
\verb+^+&to erase information in character fields	&\verb+?+&for field help.\\
        &entering spaces will NOT erase!		&\verb+??+&for key help.\\
\verb+#+&to skip groups of similar fields.		&\verb+???+&for screen help.\\ \hline
\end{tabular}
\end{center}

For array entries a null entry causes a skip to the next array. Also
for each of the four types of Constraint, the data fields are skipped
if that constraint is not applicable.

\sloppy

\section{\label{se:form_contents}\xlabel{form_contents}Form contents}

The table below shows the fields for each section:

\footnotesize
\begin{tabular}{llrlr}
{\bf Form}	&{\bf Field name} &{\bf Format}&{\bf Field name}	&{\bf Format}\\[2mm]
{\bf Cover}	&LAST.NAME            &C*27	&FIRST.NAME           &C*17	\\
	&MIDDLE.NAME          &C*12	&PI.TITLE             &C*12  \\
	&DEPARTMENT          &C*60   	&INSTITUTE           &C*60   \\
	&ADDRESS             &C*60   	&CITY.TOWN           &C*32   \\
	&COUNTY              &C*32   	&POSTAL.CODE         &C*12   \\
	&COUNTRY             &C*24   	&TELEPHONE.NUMBER    &C*24   \\
	&TELEX.NUMBER        &C*20   	&FAX.NUMBER          &C*24   \\
	&NETWORK.NAME        &C*10   	&NETWORK.ADDRESS     &C*25   \\
	&PROPOSAL.TITLE(2)   &C*60   	&SUBJECT.CATEGORY    &I*1    \\
	&NUMBER.OF.TARGETS   &I*1    	&			& \\ [1mm]
{\bf General}	&COI.NAME(6)         &C*32  	&COI.INSTITUTE(6)    &C*32   \\
	&COI.CNTRY(6)        &C*24   	&ADMIN.NAME         &C*32    \\
	&ADMIN.POST         &C*60    	&AGENCY             &C*6     \\ [1mm]
{\bf Target}	&TARGET.NUMBER      &I*2 &TARGET.NAME        &C*20     \\
	&ALTERNATIVE.NAME   &C*20     	&TARGET.RA          &C*11     \\
	&TARGET.DEC         &C*11     	&TOTAL.OBS.TIME     &R*4      \\
	&NUMBER.OBS         &I*1      	&TIME.CRITICAL      &L*1      \\
	&HRI.CODE           &I*1      	&WFC.CODE           &I*1      \\
	&WFC.ZOOM.ON        &L*1      	&WFC.FILT.CODE(8)   &C*3      \\	
	&WFC.FILT.PCNT(8)   &I*1      	&WFC.FILT.MINT(8)   &R*4      \\
	&REMARKS(4)        &C*50     	&			&	\\[1mm]
{\bf Constraints}&COORD.OBSERVATION  &L*1 	&START.YEAR        &I*2      \\
	&START.MONTH        &I*1      	&START.DAY          &I*2      \\
	&START.HOUR         &I*1      	&START.MINUTE       &R*4      \\
	&END.YEAR          &I*2      	&END.MONTH          &I*1      \\
	&END.DAY            &I*2      	&END.HOUR           &I*1      \\
	&END.MINUTE         &R*4      	&MONITOR           &L*1      \\
	&TIME.INTERVAL      &R*4      	&PHASE.DEPENDENT    &L*1      \\
	&EPOCH              &R*4      	&PERIOD             &R*4      \\
	&CONTIGUOUS.OBS     &L*1      	&NUMBER.INTERVALS   &I*2      \\
\end{tabular}
\normalsize

\section{\label{se:acknowledgements}\xlabel{acknowledgements}Acknowledgements}

Assistance has been provided by M Watson, of Leicester University
and at RAL: Brian Stewart, Mark Jefferies, Martin Bush, and Dave Ewart.
For the Unix version, a lot of the work was done by Margo Duesterhaus and
Phillip Brisco at NASA Goddard.

\end{document}
