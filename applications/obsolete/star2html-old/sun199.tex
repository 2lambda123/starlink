\documentclass[twoside,11pt]{article}

% ? Specify used packages
% \usepackage{graphicx}        %  Use this one for final production.
% \usepackage[draft]{graphicx} %  Use this one for drafting.
% ? End of specify used packages

\pagestyle{myheadings}

% -----------------------------------------------------------------------------
% ? Document identification
% Fixed part
\newcommand{\stardoccategory}  {Starlink User Note}
\newcommand{\stardocinitials}  {SUN}
\newcommand{\stardocsource}    {sun\stardocnumber}
\newcommand{\stardoccopyright}
{Copyright \copyright\ 2000-2004 Council for the Central Laboratory of the Research Councils
<br>
Copyright \copyright\ 2008 Science and Technology Facilities Council}

% Variable part - replace [xxx] as appropriate.
\newcommand{\stardocnumber}    {199.10}
\newcommand{\stardocauthors}   {P W Draper\\
                                A J Chipperfield\\
                                M D Lawden}
\newcommand{\stardocdate}      {15 January 2008}
\newcommand{\stardoctitle}     {Star2HTML\\[1ex]
                                Converting Starlink Documents to Hypertext}
\newcommand{\stardocversion}   {Version 1.5}
\newcommand{\stardocmanual}    {User's Manual}
\newcommand{\stardocabstract}{
Star2HTML lets you write (or convert) a
\xref{Starlink document}{sug}{}
 so that you can create two versions of it from a single source file.
A paper version is produced by
\xref{\LaTeX}{sun9}{},
and a hypertext version (suitable for browsing on the web) is produced by
\htmladdnormallink{\latextohtml}{\LtoHURL}.
You can tailor each version to its own medium by marking selected text as
\emph{\LaTeX-only}\/ or \emph{HTML-only}.

Star2HTML also includes a set of document templates for producing
Starlink documents in a standard style (such as \emph{Starlink User Note}).
They also define new \LaTeX\ commands for adding extra links to the
hypertext version of your document (without affecting the paper version).

This document explains these new facilities, and gives advice on good
practice and on how to deal with some specific formatting problems when
converting a document to hypertext.
You are assumed to be familiar with \LaTeX.
}
% ? End of document identification
% -----------------------------------------------------------------------------

% +
%  Name:
%     sun.tex
%
%  Purpose:
%     Template for Starlink User Note (SUN) documents.
%     Refer to SUN/199
%
%  Authors:
%     AJC: A.J.Chipperfield (Starlink, RAL)
%     BLY: M.J.Bly (Starlink, RAL)
%     PWD: Peter W. Draper (Starlink, Durham University)
%
%  History:
%     17-JAN-1996 (AJC):
%        Original with hypertext macros, based on MDL plain originals.
%     16-JUN-1997 (BLY):
%        Adapted for LaTeX2e.
%        Added picture commands.
%     13-AUG-1998 (PWD):
%        Converted for use with LaTeX2HTML version 98.1p7 and
%        Star2HTML version 1.6.
%     {Add further history here}
%
% -

\newcommand{\stardocname}{\stardocinitials /\stardocnumber}
\markboth{\stardocname}{\stardocname}
\setlength{\textwidth}{160mm}
\setlength{\textheight}{230mm}
\setlength{\topmargin}{-2mm}
\setlength{\oddsidemargin}{0mm}
\setlength{\evensidemargin}{0mm}
\setlength{\parindent}{0mm}
\setlength{\parskip}{\medskipamount}
\setlength{\unitlength}{1mm}

% -----------------------------------------------------------------------------
%  Hypertext definitions.
%  ======================
%  These are used by the LaTeX2HTML translator in conjunction with Star2HTML.

%  Comment.sty: version 2.0, 19 June 1992
%  Selectively in/exclude pieces of text.
%
%  Author
%    Victor Eijkhout                                      <eijkhout@cs.utk.edu>
%    Department of Computer Science
%    University Tennessee at Knoxville
%    104 Ayres Hall
%    Knoxville, TN 37996
%    USA

%  Do not remove the %begin{latexonly} and %end{latexonly} lines (used by
%  LaTeX2HTML to signify text it shouldn't process, in this case raw
%  TeX and some command redefinitions that we only want when using LaTeX).
%begin{latexonly}
\makeatletter
\def\makeinnocent#1{\catcode`#1=12 }
\def\csarg#1#2{\expandafter#1\csname#2\endcsname}

\def\ThrowAwayComment#1{\begingroup
    \def\CurrentComment{#1}%
    \let\do\makeinnocent \dospecials
    \makeinnocent\^^L% and whatever other special cases
    \endlinechar`\^^M \catcode`\^^M=12 \xComment}
{\catcode`\^^M=12 \endlinechar=-1 %
 \gdef\xComment#1^^M{\def\test{#1}
      \csarg\ifx{PlainEnd\CurrentComment Test}\test
          \let\html@next\endgroup
      \else \csarg\ifx{LaLaEnd\CurrentComment Test}\test
            \edef\html@next{\endgroup\noexpand\end{\CurrentComment}}
      \else \let\html@next\xComment
      \fi \fi \html@next}
}
\makeatother

\def\includecomment
 #1{\expandafter\def\csname#1\endcsname{}%
    \expandafter\def\csname end#1\endcsname{}}
\def\excludecomment
 #1{\expandafter\def\csname#1\endcsname{\ThrowAwayComment{#1}}%
    {\escapechar=-1\relax
     \csarg\xdef{PlainEnd#1Test}{\string\\end#1}%
     \csarg\xdef{LaLaEnd#1Test}{\string\\end\string\{#1\string\}}%
    }}

%  Define environments that ignore their contents.
\excludecomment{comment}
\excludecomment{rawhtml}
\excludecomment{htmlonly}

%  Hypertext commands etc. This is a condensed version of the html.sty
%  file supplied with LaTeX2HTML by: Nikos Drakos <nikos@cbl.leeds.ac.uk> &
%  Jelle van Zeijl <jvzeijl@isou17.estec.esa.nl>. The LaTeX2HTML documentation
%  should be consulted about all commands (and the environments defined above)
%  except \xref and \xlabel which are Starlink specific.

\newcommand{\htmladdnormallinkfoot}[2]{#1\footnote{#2}}
\newcommand{\htmladdnormallink}[2]{#1}
\newcommand{\htmladdimg}[1]{}
\newcommand{\hyperref}[4]{#2\ref{#4}#3}
\newcommand{\htmlref}[2]{#1}
\newcommand{\htmlimage}[1]{}
\newcommand{\htmladdtonavigation}[1]{}

\newenvironment{latexonly}{}{}
\newcommand{\latex}[1]{#1}
\newcommand{\html}[1]{}
\newcommand{\latexhtml}[2]{#1}
\newcommand{\HTMLcode}[2][]{}

%  Starlink cross-references and labels.
\newcommand{\xref}[3]{#1}
\newcommand{\xlabel}[1]{}

%  LaTeX2HTML symbol.
\newcommand{\latextohtml}{\LaTeX2\texttt{HTML}}

%  Define command to re-centre underscore for Latex and leave as normal
%  for HTML (severe problems with \_ in tabbing environments and \_\_
%  generally otherwise).
\renewcommand{\_}{\texttt{\symbol{95}}}

% -----------------------------------------------------------------------------
%  Debugging.
%  =========
%  Remove % on the following to debug links in the HTML version using Latex.

% \newcommand{\hotlink}[2]{\fbox{\begin{tabular}[t]{@{}c@{}}#1\\\hline{\footnotesize #2}\end{tabular}}}
% \renewcommand{\htmladdnormallinkfoot}[2]{\hotlink{#1}{#2}}
% \renewcommand{\htmladdnormallink}[2]{\hotlink{#1}{#2}}
% \renewcommand{\hyperref}[4]{\hotlink{#1}{\S\ref{#4}}}
% \renewcommand{\htmlref}[2]{\hotlink{#1}{\S\ref{#2}}}
% \renewcommand{\xref}[3]{\hotlink{#1}{#2 -- #3}}

%  End of latexonly comment section -- do not remove.
%end{latexonly}
% -----------------------------------------------------------------------------
% ? Document specific \newcommand or \newenvironment commands.
\newcommand{\dash}{--}
\begin{htmlonly}
  \newcommand{\dash}{-}
\end{htmlonly}
\newcommand{\StarlinkURL}{http://www.starlink.ac.uk/}
\newcommand{\StoreURL}{http://www.starlink.ac.uk/cgi-store/storetop}
\newcommand{\StarURL}{http://www.starlink.ac.uk/}
\newcommand{\PerlURL}{http://pubweb.nexor.co.uk/public/perl/man/html/perl.html}
\newcommand{\WWWURL}{http://www.w3.org/hypertext/WWW/}
\newcommand{\HTMLURL}{http://www.w3.org/hypertext/WWW/MarkUp/MarkUp.html}
\newcommand{\LtoHURL}{http://www.tex.ac.uk/tex-archive/support/latex2html/}
\newcommand{\LtoHManURL}{http://star-www.dur.ac.uk/~{}pdraper/latex2html/manual/}
\newcommand{\supportURL}{http://star-www.dur.ac.uk/\~{}pdraper/star2html/star2html.html}
% ? End of document specific commands
% -----------------------------------------------------------------------------
%  Title Page.
%  ===========
\renewcommand{\thepage}{\roman{page}}
\begin{document}
\thispagestyle{empty}

%  Latex document header.
%  ======================
\begin{latexonly}
   CCLRC / \textsc{Rutherford Appleton Laboratory} \hfill \textbf{\stardocname}\\
   {\large Particle Physics \& Astronomy Research Council}\\
   {\large Starlink Project\\}
   {\large \stardoccategory\ \stardocnumber}
   \begin{flushright}
   \stardocauthors\\
   \stardocdate
   \end{flushright}
   \vspace{-4mm}
   \rule{\textwidth}{0.5mm}
   \vspace{5mm}
   \begin{center}
   {\Huge\textbf{\stardoctitle \\ [2.5ex]}}
   {\LARGE\textbf{\stardocversion \\ [4ex]}}
   {\Huge\textbf{\stardocmanual}}
   \end{center}
   \vspace{5mm}

% ? Add picture here if required for the LaTeX version.
%   e.g. \includegraphics[scale=0.3]{filename.ps}
% ? End of picture

% ? Heading for abstract if used.
   \vspace{10mm}
   \begin{center}
      {\Large\textbf{Abstract}}
   \end{center}
% ? End of heading for abstract.
\end{latexonly}

%  HTML documentation header.
%  ==========================
\begin{htmlonly}
   \xlabel{}
   \begin{rawhtml} <H1> \end{rawhtml}
      \stardoctitle\\
      \stardocversion\\
      \stardocmanual
   \begin{rawhtml} </H1> <HR> \end{rawhtml}

% ? Add picture here if required for the hypertext version.
%   e.g. \includegraphics[scale=0.7]{filename.ps}
% ? End of picture

   \begin{rawhtml} <P> <I> \end{rawhtml}
   \stardoccategory\ \stardocnumber \\
   \stardocauthors \\
   \stardocdate
   \begin{rawhtml} </I> </P> <H3> \end{rawhtml}
      \htmladdnormallink{CCLRC / Rutherford Appleton Laboratory}
                        {http://www.cclrc.ac.uk} \\
      \htmladdnormallink{Particle Physics \& Astronomy Research Council}
                        {http://www.pparc.ac.uk} \\
   \begin{rawhtml} </H3> <H2> \end{rawhtml}
      \htmladdnormallink{Starlink Project}{http://www.starlink.ac.uk/}
   \begin{rawhtml} </H2> \end{rawhtml}
   \htmladdnormallink{\htmladdimg{source.gif} Retrieve hardcopy}
      {http://www.starlink.ac.uk/cgi-bin/hcserver?\stardocsource}\\

%  HTML document table of contents.
%  ================================
%  Add table of contents header and a navigation button to return to this
%  point in the document (this should always go before the abstract \section).
  \label{stardoccontents}
  \begin{rawhtml}
    <HR>
    <H2>Contents</H2>
  \end{rawhtml}
  \htmladdtonavigation{\htmlref{\htmladdimg{contents_motif.gif}}
        {stardoccontents}}

% ? New section for abstract if used.
  \section{\xlabel{abstract}Abstract}
% ? End of new section for abstract
\end{htmlonly}

% -----------------------------------------------------------------------------
% ? Document Abstract. (if used)
%  ==================
\stardocabstract
% ? End of document abstract

% -----------------------------------------------------------------------------
% ? LateX Copyright Statement
%  =========================
\begin{latexonly}
\newpage
\vspace*{\fill}
\stardoccopyright
\end{latexonly}
% ? End of Latex copyright statement

% -----------------------------------------------------------------------------
% ? Latex document Table of Contents (if used).
%  ===========================================
  \newpage
  \begin{latexonly}
    \setlength{\parskip}{0mm}
    \tableofcontents
    \setlength{\parskip}{\medskipamount}
    \markboth{\stardocname}{\stardocname}
  \end{latexonly}
% ? End of Latex document table of contents
% -----------------------------------------------------------------------------
\cleardoublepage
\renewcommand{\thepage}{\arabic{page}}
\setcounter{page}{1}

\section{\xlabel{introduction}Introduction}

\xref{Starlink documents}{sug}{} (SUN, SGP, \textit{etc}.) are produced by
\xref{\LaTeX}{sun9}{}.
Star2HTML converts Starlink documents to
\htmladdnormallink{HTML}{\HTMLURL}
for use on the World Wide Web.
It has the potential to make the entire set of Starlink documents
appear as a single hypertext document by linking all the separate documents
together.
It allows a single document source file (\texttt{.tex}) to produce a paper
version (when processed by \LaTeX) and a hypertext version (when processed by
\htmladdnormallink{\latextohtml}{\LtoHURL}).

\subsection{\xlabel{getting_started}\label{getting_started}Getting started}

Copy one of the Starlink template documents into a working directory:

\begin{quote}
\begin{verbatim}
# cp /star/share/sun.tex .
\end{verbatim}
\end{quote}

This file is a skeleton \LaTeX\ source file and will give your document a
standard style and extra hypertext facilities.

Get a paper copy to look at (or use \texttt{xdvi} if you prefer):

\begin{quote}
\begin{verbatim}
# latex sun
# dvips sun
# lpr -P1 sun.ps
\end{verbatim}
\end{quote}

You've produced a Starlink User Note!
It's got a title page with an abstract, a contents page, and third page
(page 1) containing the body of the document.

You've got a paper version of the document.
Now use Star2HTML to produce a hypertext version:

\begin{quote}
\begin{verbatim}
# star2html sun
\end{verbatim}
\end{quote}

A lot of alarming messages, dots, and general gobbledygook appear on your
screen.
Don't worry about it -- this is Unix not Windows.
You now have a hypertext version of the document.
It's stored in a new directory called \texttt{sun.htx} (which will
have a lot of files in it, none of which need concern you).
Notice that you've produced a paper version and a hypertext version from the
same source file (\texttt{sun.tex}).

Now use a web browser to look at the hypertext version:

\begin{quote}
\begin{verbatim}
# showme ./sun
\end{verbatim}
\end{quote}

Compare the two versions.

The same information is there, it's just been rearranged on the page.
Some extra information has been added to the hypertext version.
There's a navigation band at the top (you may have to scroll up to see it), a
button to get hardcopy, a couple of entries in the contents list, and some
extra identification at the bottom.
Also, some elements in the header (like \texttt{CCLRC}) appear as hypertext
links to external pages that describe what they are.

Star2HTML and the templates do all this automatically.
All you have to worry about is filling in the template (and deleting the
bits you don't want), replacing those place-holders (like \texttt{[author]} and
\texttt{[title]} with meaningful text, and writing the document.
The following sections will help you do this.

\subsection{\xlabel{converting_an_existing_document}Converting an existing document}

A basic conversion of an existing \LaTeX\ document can be done very quickly.
It consists of incorporating the appropriate template in the source file, and
then changing (or deleting) the place-holders. However, to get the best from
the system, some additional tasks are required.

To convert an existing \LaTeX\ document, edit it to:

\begin{enumerate}
\item Incorporate one of the Star2HTML document templates and adapt it to
 your document.
\item Add extra hypertext links to your document and allow other Starlink
hypertext documents to link to your document.
\item Adapt the two versions (paper and hypertext) of your document to their
 respective media.
\item Correct any formatting problems in the hypertext version that result from
 the conversion process.
\end{enumerate}

These four tasks are described in later sections of this document.

Start with task 1 (see
\htmlref{``Using a template''}{using_a_template}\latex{,
Section~\ref{using_a_template}}), which will produce a basic conversion of your
document.
You can sort out any linking, formatting, and media problems later.

Run the converted document through \LaTeX\ to make sure you haven't
harmed the paper version.
Then produce a hypertext version, using \texttt{star2html}, and check the
result, using \texttt{showme}.
Use the commands shown in
\htmlref{``Getting started''}{getting_started}\latex{ (Section~\ref{getting_started})}.

Sometimes the hypertext version doesn't look right on the screen.
You've got formatting problems.
Check the
\htmlref{common problems}{common_problems}
\latex{in Section~\ref{common_problems}} for possible solutions.

\subsection{\xlabel{some_background}Some background}

Star2HTML comprises two Perl scripts and a set of templates.

The
\htmlref{\texttt{star2html}}{star2html}
command invokes one of the Perl scripts which calls the
\htmladdnormallink{\latextohtml}{\LtoHURL}
utility to do the actual conversion to hypertext format.
\latextohtml\ is described in its own
\htmladdnormallinkfoot{manual}{\LtoHManURL}\latex{ (MUD/152)}.
However, you shouldn't need to refer to it unless you have specialised needs.
Always use the \texttt{star2html} command rather than the
\texttt{latex2html} command to convert Starlink documents to hypertext.

The templates have two functions:
\begin{itemize}
\item Provide \LaTeX\ definitions for standard styles of Starlink document.
\item Add extra \LaTeX\ commands to the basic repertoire provided by
\latextohtml.
These are used to link Starlink documents together.
\end{itemize}

The templates are based on \LaTeXe\, and all documents should now be
produced using this.

The second Perl script,
\htmlref{\texttt{xlabel}}{xlabel},
is a utility to help in labelling sections of the hypertext document so that
other documents can link to them.
Its use is described in
\htmlref{``Labelling sections''}{labelling_sections}
\latex{ (Section~\ref{labelling_sections})}.

\section{\xlabel{using_a_template}\label{using_a_template}Using a template}

The templates are skeleton \LaTeX\ documents with preambles giving
the document a standard style and extra hypertext facilities.
They are stored in directory \texttt{/star/share} and have filenames
which are based on the name of the Starlink document class for which they
have been designed (\texttt{sun.tex}, \texttt{ssn.tex}, \textit{etc.}).

Changing the place-holders in the template will be trivial for \LaTeX\
experts, but it can be intimidating for novices, so the process is
described in detail below.

\subsection{\xlabel{changing_placeholders}Changing place-holders}
The sections of the template you need to change are contained between comments
like:

\begin{quote}
\verb+% ? Section description+\\
\ldots\\
\verb+% ? End of section description+
\end{quote}

Some just indicate where you should put any special commands required by
your document, others contain sections of code which should be commented
in or out (with \%) depending upon your requirements and yet others contain
place-holders indicated by being enclosed in brackets,
\texttt{[number]} for example.
Change the place-holders as appropriate for your document.
Some place-holders may not be relevant to your document, in which case
delete the lines that contain them.

The section headings to look for are:


\vspace{6mm}

\verb+% ? Specify used packages+

\begin{quote}
Any \LaTeX\ extension packages which are used in your document should
be listed here in a {\tt \\usepackage} command like:
\begin{quote}
\verb+\usepackage{graphicx}+
\end{quote}
The section contains commented out commands for the commonly-used
\verb+graphicx+ package \dash\ don't forget to use the \texttt{draft} option of
the package to save time when writing your document.
\end{quote}

\vspace{6mm}

\verb+% ? Document identification+

\begin{quote}
Replace or delete the following place-holders:
\begin{itemize}
\item \texttt{[number].[version]} \dash\ replace with the sequence number and
 version number of the document code (like \texttt{199.3}).
\item \texttt{[author]} \dash\ the name of the author (or authors).
\item \texttt{[date]} \dash\ the date of issue (like \texttt{3 June 1997}).
\item \texttt{[title]} \dash\ the main title of the document.
\item \texttt{[software-version]} \dash\ the version number of the software
being described (like \texttt{Version 1.1-2}).
 If you are not writing about a software package, or there is no version
 number, delete this line.
\item \texttt{[manual-type]} \dash\ the type of manual you are writing
 (like \texttt{User's Manual}).
 If this line isn't appropriate, delete it.
\item \texttt{[Text of abstract]} \dash\ every document, except the most
 trivial, should have an abstract.
 Put the text of the abstract here.
 It will appear under the heading ``Abstract'' on the front page of the paper
 version.
 In the hypertext version, it will appear in the first section of the document
 and will be given the title \emph{Abstract}.
 If you don't want to supply an abstract, delete this line, and also delete
 the other sections which are used to create abstracts:
\begin{quote}
\verb#% ? Heading for abstract if used#\\
\ldots\\
\verb#% ? End of heading for abstract#
\end{quote}
 and
\begin{quote}
\verb#% ? New section for abstract if used#\\
\ldots\\
\verb#% ? End of new section for abstract#
\end{quote}
 and
\begin{quote}
\verb#% ? Document Abstract. (if used)#\\
\ldots\\
\verb#% ? End of document abstract#
\end{quote}
One point to note is that if you want to include a paragraph break in your
abstract, you must use the \verb+\par+ command rather than just leaving a
blank line in the text.
\end{itemize}
\end{quote}

\verb+% ? Document specific \newcommand or \newenvironment commands+

\begin{quote}
Use this section to add any extra definitions that you want to use in the body
of your text.
Most documents don't have anything added to this section, but some
({\em{e.g.}}\ SUN/95) define a large number of new commands here.
Just leave this section alone if you don't want to add anything to it.
\end{quote}
\begin{verbatim}
% ? Add picture here if required for the LaTeX version
% ? Add picture here if required for the hypertext version
\end{verbatim}
\begin{quote}
Use these sections to add any pictures that you want to display on the front
page of your document.
Just leave the sections alone if you don't want to add a picture in this
position.

\xref{SUN/9}{sun9}{including_ps_in_latex}
gives details of how to include PostScript files in \LaTeX\ documents
and there are examples of simple \verb+\includegraphics+ commands in both
sections but you will probably need something more elaborate.
PostScript files can be included in both versions
or you can use \verb+\htmladdimg{<URL>}+ to include a \texttt{.gif} file for
the hypertext document.
\end{quote}

\verb+% ? Latex Copyright Statement+

\begin{quote}
This section is required for Starlink documents but may be deleted by others.
The statement specified in the \verb+\stardoccopyright+ command will appear at
the bottom of the second page of the paper version; the HTML version will, by
default, have the copyright statement in the footer of each page.

Copyright is assigned to STFC by default.
\end{quote}

\verb+% ? Latex document Table of Contents (if used)+

\begin{quote}
Delete this entire section if you don't want a table of contents in the paper
version (you will always get one in the hypertext version).
We recommend that you always include this section unless your document is short
and unstructured.
\end{quote}

\subsection{\xlabel{using_sst_macros}Using SST macros}

If your document uses the standard macros defined for the Simple
Software Tools (SST) package on VMS, then you should add the command:
\begin{quote}
\verb+\+\verb+input{/star/share/sst}+
\end{quote}
to the ``Document specific \ldots commands'' section of the template.
The new definitions are a direct replacement for the old macros and should not
cause any problems for existing documents unless you have modified the old
macros.
If you have done so, you must work out how to incorporate your changes into
the new definitions.

\section{\xlabel{adding_links}Adding links}

Star2HTML automatically adds some links to a document it converts; for example,
the section headings in the contents list link to the section text.
However, you can add extra links if you wish.
These can be to other parts of your document (Internal links), or to
other pages on the web (External links).
A type of External link that is very common is one to another Starlink
document; a special command is provided to create this.

\subsection{\xlabel{internal_links}Internal links}

If your document uses the cross-referencing facilities of \LaTeX\
({\em{i.e.}}\ the \verb+\label{key}+, \verb+\ref{key}+ commands), then
Star2HTML will automatically create a link from the \verb+\ref+ to the
\verb+\label+.
The link will normally be indicated by a small button, when viewed with a
browser (the actual value will be used if the \texttt{-aux} switch of the
\texttt{star2html} command is used, but these values aren't normally useful
in hypertext).

If you want to use some descriptive text for the link instead of the button
(or if you want to add a new internal link), use the command:

\begin{quote}
\begin{verbatim}
\htmlref{text}{label}
\end{verbatim}
\end{quote}

\texttt{text} is the text that will be highlighted as the link (it will also
appear in the paper version of your document as ordinary text).
\texttt{label} is the name of a label defined somewhere in your document by
the \verb#\label# command.

For example, the source code for the heading of the current top-level section
might be:

\begin{quote}
\verb+\section{\label{adding_links}Adding links}+
\end{quote}

You could then create a link to this section by a reference like:

\begin{quote}
\verb+... refer to the \htmlref{Adding links}{adding_links} section.+
\end{quote}

This will appear in both versions as:

\begin{quote}
\ldots refer to the Adding links section.
\end{quote}

but in the hypertext version, ``Adding links'' is indicated as a link.
When activated, it will take the reader to that section.

As another example, the standard SST macros define a label for every routine
description.
If a reference is made to a routine, say PISAPLOT, in the main text, a link
to the full description of the routine can be made by:

\begin{quote}
\begin{verbatim}
\htmlref{PISAPLOT}{PISAPLOT}
\end{verbatim}
\end{quote}

The word ``PISAPLOT'' will appear in both versions, but in the hypertext version
it will be a link to the description of the routine.

\subsection{\xlabel{external_links}External links}

To add a link to an external web page, use the command:

\begin{quote}
\begin{verbatim}
\htmladdnormallink{text}{URL}
\end{verbatim}
\end{quote}

\texttt{text} will appear in both versions, but in the hypertext version it
will be a link to the web page with the specified
\htmladdnormallink{\texttt{URL}}
{http://www.w3.org/hypertext/WWW/Addressing/Addressing.html}.

To add the URL as a footnote in the paper version, use the following command
instead:

\begin{quote}
\begin{verbatim}
\htmladdnormallinkfoot{text}{URL}
\end{verbatim}
\end{quote}

\subsection{\xlabel{links_between_starlink_documents}Links between Starlink documents}

The \LaTeX\ command \verb#\label# allows cross-references within normal
\LaTeX\ documents.
Similarly, the Star2HTML command \verb#\xlabel# allows hypertext links to be
set up from other Starlink documents to your document.
Its format is:

\begin{quote}
\begin{verbatim}
\xlabel{label}
\end{verbatim}
\end{quote}

Other documents can create a link to it by using \verb+\xref+.
The format of \verb+\xref+ is:

\begin{quote}
\begin{verbatim}
\xref{text}{doc}{label}
\end{verbatim}
\end{quote}

\texttt{text} will appear in both versions, but in the hypertext version it
will be a link to the place labelled \texttt{label} (with \verb+\xlabel+) in
the Starlink document \texttt{doc}.

Obviously, if other Starlink documents contain \verb+\xlabel+ commands, you
can use \verb+\xref+ in your document to add links to them.

It will often be necessary to include both a \verb#\label# and an
\verb#\xlabel# command at the same point in a document in order to anchor
internal and external links to the same place.
In this case, the name of the label can be the same in both commands since
\verb#\xlabel# modifies it to produce a different anchor name.

You can create a link to any Starlink document by a command like:

\begin{quote}
\begin{verbatim}
\xref{SUN/x}{sunx}{}
\end{verbatim}
\end{quote}

This link will take you to the beginning of the document as the templates
contain an \verb#\xlabel# with a blank label on the title page of the document.
Not all Starlink documents are available in hypertext form but the link will
even work in that case \dash\  browsers will usually just display the paper
version.

There is one further step required to create a link to another local
Starlink document \dash\ you must run the hypertext linker \texttt{hlink},
after running \texttt{star2html} on your source file, so that the links
between the two documents can be resolved. If you do not do this then all the
cross references will be passed to the Starlink central document server for
resolution.
Thus, if your document contains links to other Starlink documents, you
can convert, link it locally, and view it by:

\begin{quote}
\begin{verbatim}
# star2html sunx
# hlink
# showme ./sunx
\end{verbatim}
\end{quote}

The Starlink hypertext linker, \texttt{hlink}, is more fully described in
\xref{SUN/188}{sun188}{}.

\subsection{\xlabel{labelling_sections}\label{labelling_sections}Labelling sections}

If you want to label a complete section, for other documents to refer to,
you should include an \verb+\xlabel+ as part of the section title.
This is because the link will be to the \emph{exact point}\/ in the document
where the \verb#\xlabel# occurs, rather than to the start of the
section (as with normal \verb#\ref# and \verb#\label#).

So that others can easily link to sections of your document,
we recommend that you \verb+\xlabel+ all sections and subsections
with the section title in lower case, non-alphanumeric characters removed and
words separated by underscore.

For example, the subsection you are now reading has been labelled as follows:

\begin{quote}
\verb+\subsection{\xlabel{labelling_sections}Labelling sections}+
\end{quote}

and other documents can create a link to it by using \verb+\xref+ as in:

\begin{quote}
\begin{verbatim}
See SUN/199 for more on
\xref{labelling sections}{sun199}{labelling_sections}.
\end{verbatim}
\end{quote}

You can use the \texttt{xlabel} command from the Unix shell to ensure that your
document has the recommended \verb+\xlabel+'s for each section and subsection.
The command:
\begin{quote}
\texttt{\# xlabel} \textit{document}
\end{quote}
Will produce a new version of the document, in file
\texttt{new}\textit{document},
containing any additional required \verb+\xlabel+ commands.
It will also warn you if there are any duplicated section labels or if an
existing  \verb+\xlabel+ differs from the recommendation.
See the
\htmlref{description of \texttt{xlabel}}{xlabel}
\latex{(in Appendix \ref{xlabel})} for more information on this.

When you are happy with the result, copy \texttt{new}\textit{document} back to
\textit{document}.

\subsection{\xlabel{links_to_routine_descriptions}Links to routine descriptions}

The SST routine description macros include an \verb#\xlabel# command for
each routine so, if you use them, your routines will
already be labelled to allow links from other documents. By the same token,
you can include links to routine descriptions in other documents which have
used the SST macros.

For example:
\begin{quote}
\begin{verbatim}
\xref{CHR\_LEN}{sun40}{CHR_LEN} finds the used length of a string.
\end{verbatim}
\end{quote}
\begin{htmlonly}
would produce:
\begin{quote}
\xref{CHR\_LEN}{sun40}{CHR_LEN} finds the used length of a string.
\end{quote}
\end{htmlonly}


\section{\xlabel{adapting_to_the_media}\label{conditional_text}Adapting to the media}

You may wish the hypertext version of your document to differ from the paper
version.
To restrict text to the \emph{paper}\/ version, use the environment:

\begin{quote}
\verb!\begin!\verb!{latexonly}!\\
This text will only appear in the \LaTeX\ version. \\
\verb!\end!\verb!{latexonly}!
\end{quote}

To restrict text to the \emph{hypertext}\/ version, use the environment:

\begin{quote}
\verb!\begin!\verb!{htmlonly}!\\
This text will only appear in the hypertext version. \\
\verb!\end!\verb!{htmlonly}!
\end{quote}

Note:

\begin{itemize}
\item The \verb!\begin!\verb!{htmlonly}!, \verb!\end!\verb!{htmlonly}!,
  \verb!\begin!\verb!{latexonly}! and \verb!\end!\verb!{latexonly}!
  statements should be on separate lines from other text or \LaTeX\ commands.
\item Both text and commands may be included within the environments,
  but problems can arise if they contain any of the \LaTeX\ sectioning commands
  (see
  \htmlref{``Selecting sections''}{selecting_sections}\latex{, Section~\ref{selecting_sections}}).
\end{itemize}

Single lines of HTML-only and \LaTeX\--only text can be controlled
using the commands:
\begin{quote}
\verb!\html!\{text for the hypertext version\}
\end{quote}
and
\begin{quote}
\verb!\latex!\{text for the paper version\}
\end{quote}
Or you could use the conditional statement:
\begin{quote}
text for the \verb!\latexhtml!\{paper\}\{hypertext\} version.
\end{quote}

\subsection{\xlabel{selecting_sections}\label{selecting_sections}Selecting sections}

If you want to have a section that only appears in the paper version
of your document, then your document preamble must define new commands
to replace the \LaTeX\ sectioning commands, for example:

\begin{latexonly}
\begin{quote}
\begin{verbatim}
% Latex only sections, subsections etc.
% Use these in a latexonly environment.
\newcommand{\latexonlysection}[1]{\section{#1}}
\newcommand{\latexonlysubsection}[1]{\subsection{#1}}
\newcommand{\latexonlysubsubsection}[1]{\subsubsection{#1}}
\begin{htmlonly}
   \newcommand{\latexonlysection}[1]{#1}
   \newcommand{\latexonlysubsection}[1]{#1}
   \newcommand{\latexonlysubsubsection}[1]{#1}
\end{htmlonly}
\end{verbatim}
\end{quote}
\end{latexonly}
%  Repeat for HTML version (begin/end htmlonly cause problems).
\begin{rawhtml}
<P>
<BLOCKQUOTE>
<PRE>
% Latex only sections, subsections etc.
% Use these in a latexonly environment.
\newcommand{\latexonlysection}[1]{\section{#1}}
\newcommand{\latexonlysubsection}[1]{\subsection{#1}}
\newcommand{\latexonlysubsubsection}[1]{\subsubsection{#1}}
<TT>\begin</TT><TT>{htmlonly}</TT>
   \newcommand{\latexonlysection}[1]{#1}
   \newcommand{\latexonlysubsection}[1]{#1}
   \newcommand{\latexonlysubsubsection}[1]{#1}
<TT>\end</TT><TT>{htmlonly}</TT>
</PRE>
</BLOCKQUOTE>
\end{rawhtml}

You must then use these new commands instead of the normal sectioning
commands, for example:

\begin{quote}
\verb!\begin!\verb!{latexonly}!\\
\verb!\latexonlysection{LaTeX readers beware...}!\\
LaTeX readers are missing out by not reading the HTML version of
this document.\\
\verb!\end!\verb!{latexonly}!
\end{quote}

If a sectioning command appears in an \texttt{htmlonly} environment, you get
the warning message ``\verb#Cannot find \end#\verb#{htmlonly}#'', but
the result should be satisfactory.

\section{\xlabel{common_conversion_problems}\label{common_problems}Common conversion problems}

We have already mentioned that Star2HTML uses \latextohtml\ to do the
conversion to hypertext.
\latextohtml\ usually makes a good attempt at this conversion.
However, it can produce disappointing results, particularly for complex
sections, such as the cover page and subroutine/application descriptions in
Starlink documents.

Some \LaTeX\ structures cannot be translated into HTML.
Normally, \latextohtml\ handles these by either ignoring them, or by passing
them back to \LaTeX\ for processing and then converting the result to
in-line images (complex maths symbols, for example).
The result occasionally requires tweaking by hand.
Sometimes the result is so poor that you need to re-write parts of your
document in order to generate a more acceptable result.

\subsection{\xlabel{things_to_avoid}Things to avoid}

\subsubsection{Raw \TeX}

\latextohtml\ is designed to deal with \LaTeX\ commands of the form
\verb#\command{arg1}{arg2}...# and cannot, in general, deal satisfactorily
with raw \TeX\ commands.
If this problem occurs, you'll notice ``='' signs in inappropriate places, or
missing parts in your hypertext version.
The only thing to do is to use \LaTeX\ equivalents of the \TeX\ commands.

For example use:
\begin{quote}
\begin{verbatim}
\newfont{\ssttt}{cmtt10 scaled 1095}
\end{verbatim}
\end{quote}
instead of:
\begin{quote}
\begin{verbatim}
\font\ssttt=cmtt10 scaled 1095
\end{verbatim}
\end{quote}
or
\begin{quote}
\begin{verbatim}
\newcommand{\_}{\texttt\symbol{95}}
\end{verbatim}
\end{quote}
instead of:
\begin{quote}
\begin{verbatim}
\newcommand{\_}{\tt\char'137}
\end{verbatim}
\end{quote}
and so on.

\subsubsection{Parboxes}

These are always translated into GIFs and often have an incorrect
size.
Try to avoid them if possible.
If you can't, you might have to fix things by hand.

\subsection{\xlabel{things_that_rarely_look_right}Things that rarely look right}

\subsubsection{Tabbing environments}

Don't use the \verb#tabbing# environment \dash\ the converter makes a
mess of it and the only solution is to edit the HTML files.
Use the \verb#tabular# environment instead.

\subsection{\xlabel{irritants}Irritants}

\subsubsection{\xlabel{suppressing_blank_lines}Unwanted blank lines}

Consider the source text:
\begin{quote}
{\tt
Some text\\
\verb+\latex{for latex readers.}+\\
\verb+\html{HTML readers.}+
}
\end{quote}
You'd expect the generated text to look like:
\begin{quote}
\begin{verbatim}
Some text for latex readers.
\end{verbatim}
\end{quote}
in the paper version, and
\begin{quote}
\begin{verbatim}
Some text for HTML readers.
\end{verbatim}
\end{quote}
in the hypertext version.
However, the hypertext version comes out like:
\begin{quote}
\begin{verbatim}
Some text

for HTML readers.
\end{verbatim}
\end{quote}
because the \verb+\latex+ line is replaced by a blank line and
consequently ends up as a paragraph break.
To get around this you must include some text on the \verb+\latex+ line,
as in:
\begin{quote}
{\tt
Some text
\verb+\latex{for latex readers.} \html{for HTML readers.}+
}
\end{quote}
Or you could get the same effect using the \verb+\latexhtml+ command:
\begin{quote}
{\tt
Some text
\verb+\latexhtml{for latex readers.}{for HTML readers.}+
}
\end{quote}


\subsubsection{Tildas in URLs}

To get a tilda into a URL, use:
\begin{quote}
\begin{verbatim}
\htmladdnormallink{My home page at RAL}{http://star.rl.ac.uk/\~{}me}
\end{verbatim}
\end{quote}
instead of:
\begin{quote}
\begin{verbatim}
\htmladdnormallink{My home page at RAL}{http://star.rl.ac.uk/~me}
\end{verbatim}
\end{quote}
which just throws away the tilda.

\subsection{\xlabel{using_the_htmladdimg_command_to_replace_figures}Using the {\texttt{htmladdimg}} command to replace figures}

The \verb+\htmladdimg+ command is used to display an image stored in an
external file.
Its form is:
\begin{quote}
\begin{verbatim}
\htmladdimg{<URL>}
\end{verbatim}
\end{quote}
where the \texttt{URL} is the hypertext address of the image (probably just
the name of an \texttt{image.gif} file in the default directory).

It is tempting to use this command to replace existing figures with better
colour graphics.
However, you cannot use this command inside a \texttt{figure} environment.
Instead, you should use something like:
%begin{latexonly}
\begin{quote}
\begin{verbatim}
\begin{latexonly}
  \begin{figure}
    \label{This figure}
\end{verbatim}
\hspace{3cm}\textit{body of figure}
\begin{verbatim}
    \caption{Latex eps version}
  \end{figure}
\end{latexonly}
\begin{htmlonly}
  \label{This figure}
  \htmladdimg{figure_replacement.gif}
  Figure: HTML replacement
\end{htmlonly}
\end{verbatim}
\end{quote}
%end{latexonly}
%  Repeat for HTML version (begin/end htmlonly cause problems).
\begin{rawhtml}
<BLOCKQUOTE>
<PRE>
<TT>\begin</TT>{latexonly}
  \begin{figure}
    \label{This figure}
        <I>body of figure</I>
   \caption{Latex eps version}
  \end{figure}
<TT>\end</TT>{latexonly}
<TT>\begin</TT>{htmlonly}
    \label{This figure}
    <TT>\htmladdimg</TT>{figure_replacement.gif}
    Figure: HTML replacement
<TT>\end</TT>{htmlonly}
</PRE>
</BLOCKQUOTE>
\end{rawhtml}
However, you will not get a figure number in the caption.

\subsection{\xlabel{hints_from_the_starlink_software_librarian}Hints from the Starlink Software Librarian}

The following list of recommendations was compiled by the Starlink Software
Librarian, Martin Bly, who has issued many Starlink documents submitted for
release and has noted the things that have caused him trouble:

\begin{itemize}

\item Don't reference specific version numbers of documents, \textit{e.g.}\ use
``SSN/26'' rather than ``SSN/26.1''.

\item Use a \verb+/+ to separate document-type from
document-number, \textit{e.g.}\ use ``SSN/26'' rather than ``SSN26'' or
``SSN 26''.

\item In the paper version of your document, the first page (page 1)
of the main body of your text should appear on the right-hand-side
when it is printed double-sided (or sent for photocopying), otherwise
your document looks amateurish.  This main body is preceded by the
title page, abstract, and contents list (which have page numbers
written in Roman numerals).

You can ensure the correct layout by including a \verb+\cleardoublepage+
immediately before your ``page 1''.
The latest templates include this automatically but you should check existing
documents for this problem.

\item Use \verb+\begin{quote}+ \ldots \verb+\end{quote}+ to surround
\verb+\+\verb+begin{verbatim}+ \ldots \verb+\+\verb+end{verbatim}+ texts.
If the area of text is large or long, use \verb+\small+ inside the quote
environment.

\end{itemize}

\section{\xlabel{finally}Finally}

If you've spent many hours fine-tuning the layout of your \LaTeX\
document, you will probably be disappointed with the
\texttt{Star2HTML} conversion \dash\ most of your extra bells and
whistles will be thrown away without trace.  Remember, it's not yet
possible to obtain the level of formatting control in HTML that
\LaTeX\ offers (although this is situation is changing as HTML
develops).

More problems which may occur with \latextohtml\ are described in
\htmladdnormallinkfoot{\latextohtml\ manual}{\LtoHManURL}\latex{ (see
MUD/152)}.  If you run across undocumented problems, please pass them
on (hopefully with fixes) to the
\htmladdnormallink{Starlink Librarian}{mailto:starlink@jiscmail.ac.uk}.
A \htmladdnormallinkfoot{support page}{\supportURL} for this
package is maintained and will list any new problems.

\section{\xlabel{references}References}

The following documents give further information of value to authors of
Starlink documents:

\begin{latexonly}
\begin {tabular}{lll}
Charles, A.C. & \xref{SC/9}{sc9}{}
 & \LaTeX\ cookbook.\\
Lawden, M.D. & \xref{SGP/28}{sgp28}{}
 & How to write good documents for Starlink.\\
Lawden, M.D. & \xref{SGP/50}{sgp50}{}
 & Starlink document styles.\\
Charles, Anne & \xref{SUN/9}{sun9}{}
 & \LaTeX\ -- A document preparation system.\\
Warren-Smith, R.F. & \xref{SUN/188}{sun188}{}
 & HTX -- Hypertext cross-reference utilities.\\
Bly, M.J. & \xref{SUN/201}{sun201}{}
 & \latextohtml\ -- \LaTeX\ to HTML conversion.\\
Drakos, Nikos & MUD/152
 & A \LaTeX\ to HTML translator.\\
Lamport, Leslie & & \LaTeX\ User's Guide \& Reference Manual,\\
 & & 2nd ed, Addison-Wesley, 1994.\\
\end {tabular}
\end{latexonly}

\begin{htmlonly}
Charles, A.C. : \xref{SC/9}{sc9}{} :
 \LaTeX\ cookbook.\\
Lawden, M.D. : \xref{SGP/28}{sgp28}{} :
 How to write good documents for Starlink.\\
Lawden, M.D. : \xref{SGP/50}{sgp50}{} :
 Starlink document styles.\\
Charles, Anne : \xref{SUN/9}{sun9}{} :
 \LaTeX\ - A Document Preparation System.\\
Warren-Smith, R.F. : \xref{SUN/188}{sun188}{} :
 HTX - Hypertext Cross-Reference Utilities.\\
Bly, M.J. : \xref{SUN/201}{sun201}{} :
\latextohtml\ - \LaTeX\ to HTML Conversion.\\
Drakos, Nikos : MUD/152 :
 A \LaTeX\ to HTML Translator.\\
Lamport, Leslie : A Document Preparation System.
 \LaTeX\ User's Guide \& Reference Manual, 2nd ed, Addison-Wesley, 1994.
\end{htmlonly}

\newpage
\appendix
\section{\xlabel{changes_in_this_release}\label{changes}Changes in release 1.4}
The document templates have been updated to correct an error that centered
the HTML document title, type and version and also to remove the Rutherford
Appleton Laboratory URL, which is now invalid. The string ``CCLRC /
Rutherford Appleton Laboratory'' is now a single hypertext link to CCLRC.

The templates also now include a copyright statement for the paper document
and the star2html script will add a copyright statement to the footer of each
page unless it is specifically prevented. (The action of the -c switch and
HTX\_COPYRIGHT environment variable has changed.) By default the copyright
statement will assign copyright to STFC.

This document has been updated to describe the new scheme.

\section{\xlabel{changes13}\label{changes13}Changes in release 1.3}
In this release Star2HTML has been upgraded to include \latextohtml\
versions 98.2 and 97.1. Since the time of the last Star2HTML release
\latextohtml\ itself has been under constant development (by a keen
group of users organised using a discussion group) and there have been
many changes which produce a generally better translation. The key
features of these changes are:
\begin{itemize}
\item Much improved image generation. Images are now anti-aliased,
      resulting in much clearer text, and are generated much quicker.
\item Images of the same text are now shared. This reduces the number
      of images required.
\item \LaTeX\ tables are now translated into HTML ones.
\item HTML 3.2 is now generated (this allows improved formatting like:
      centering, left-right justification, table captions etc.).
\item \LaTeXe\ is now supported.
\item Many more \LaTeXe\ classes and packages are now supported (this
      means less images and more intelligent translations). *
\item Large speed and memory usage improvements. These mean
      less waiting around for a translation to complete.
\item Maths formulae are more carefully translated (\textit{e.g.}
      plain text is used whenever possible, HTML superscript and
      subscript are also used).
\item \verb#\newcommands# can now be more complex (\textit{i.e.} can
      include other newcommands). This means existing documents can be
      translated with less effort.
\item The HTML produced is generally much cleaner (redundant \verb#<P>#'s
      have mostly gone).
\item HTML 4 support is available. *
\item ISO-Latin and unicode characters are now translated. *
\end{itemize}

Changes in Star2HTML, other than to support \latextohtml, are the
addition of the ``\verb#-s#'', ``\verb#-c#'' and ``\verb#-old#'' flags
and a new command \verb#latex2html#.

The ``\verb#-s#'' flag (which is on by default) greatly speeds up
document conversion by splitting it up into smaller parts.
The ``\verb#-c#'' flag (which has no effect by default) adds a
copyright statement to the document footer (this can be set
permanently by the \verb#HTX_COPYRIGHT# environment variable).
The ``\verb#-old#'' flag forces the use of \latextohtml\ version
97.1. The \verb#latex2html# command invokes one of the builtin
versions of \latextohtml.

\subsection{Incompatibilities with previous version of Star2HTML}

Unfortunately changes in \latextohtml\ version 98.2 mean that several
major incompatibilities with previous versions of Star2HTML documents
have been introduced.

To ease the transition process -- so that you can still process your
existing documents -- Star2HTML is accompanied by two versions of
\latextohtml\ (which are now also builtin into the package) -- 97.1
and 98.2. 97.1 offers many of the the facilities of 98.2 (in fact all
except those marked with an asterisk in the previous
\htmlref{list}{changes13}) and will also process
existing documents. To make use of this facility just invoke the
\verb#star2html# command as usual.

However, if you want to take full advantage of the new facilities
you'll need to modify your documents so that they use the new
templates provided in this release. You may also need to slightly
modify any media sensitive commands that you have added.

The simplest way to upgrade your document is to change it to use the
appropriate new template (\textit{e.g.} \verb#sun.tex#) and to use the
new version of \verb#sst.tex#, if appropriate. Another change you must
make is to modify any \verb#\latexonly{}# commands to use the
equivalent \verb#\latex{}# command. Note that \verb#\latexonly# as an
environment (\verb#\begin{latexonly} \end{latexonly}#) can continue to be
used, this is just a change to the command form.

Note that if you are using \verb#sst.tex# then it \emph{must} now be
included in the document prologue, not in the text. Now if you have
not made any other changes to differentiate your on-line and paper
documents, then you should be able to convert it..

If you have added customized \verb#\newcommands# to your document then
you will also need to change any commands that look like:
\begin{quote}
\verb!\newcommand{\mycommand}{this version is only for LaTeX}!\\
\verb!\begin!\verb!{htmlonly}!\\
\enspace\enspace\hspace*{2em}\verb!\renewcommand{\mycommand}{this version is only for LaTeX2HTML}!\\
\verb!\end!\verb!{htmlonly}!
\end{quote}
to:
\begin{quote}
\verb!\newcommand{\mycommand}{this version is only for LaTeX}!\\
\verb!\begin!\verb!{htmlonly}!\\
\enspace\enspace\hspace*{2em}\verb!\newcommand{\mycommand}{this version is only for LaTeX2HTML}!\\
\verb!\end!\verb!{htmlonly}!
\end{quote}
This is especially important if you have any section control commands
in the redefinitions (\verb#\newcommand# and \verb#\renewcommand# are
now dealt with at different stages in the conversion process). In
general if your \verb#\renewcommand# is just a replacement command then
you should now use \verb#\newcommand#.

If your document is very customized and you do not want to change the
whole prologue, then try the following sequence of changes:
\begin{itemize}
\item Move the \verb#%#\verb#\end{rawtex}# line to the end of the
      ``debugging'' section (all commands visible in your document now
      override any that \latextohtml\ may know about in any fashion,
      i.e. builtins, from style files etc., so must be hidden). Now
      rename \verb#%#\verb#\end{rawtex}# to \verb#%#\verb#end{latexonly}#
      and \verb#%#\verb#\begin{rawtex}# to \verb#%#\verb#begin{latexonly}#.
\item Move your \verb#sst.tex# definitions into the prologue.
\item Change any \verb#\renewcommands# in \verb#htmlonly# environments
      to \verb#\newcommands# (this includes the SST definitions).
\item Change any \verb#\latexonly{}# commands to \verb#\latex{}#
(leave your \verb#latexonly# environments as they are).
\item Modify the font declaration in the SST definitions to:
\begin{quote}
\begin{verbatim}
  \latex{\newfont{\ssttt}{cmtt10 scaled 1095}}
  \html{\newcommand{\ssttt}{\tt}}
\end{verbatim}
\end{quote}
Removing any other such definitions and adding a command:
\begin{quote}
\begin{verbatim}
  \newcommand{\html}[1]{}
\end{verbatim}
\end{quote}
To the standard part of the prologue.
\item An additional space issue means you also need to change:
\begin{quote}
\begin{verbatim}
  \stardoccategory \stardocnumber \\
\end{verbatim}
\end{quote}
to:
\begin{quote}
\begin{verbatim}
  \stardoccategory\ \stardocnumber \\
\end{verbatim}
\end{quote}
\item And finally add the following definition to the standard part of
      the prologue.
\begin{quote}
\begin{verbatim}
 \newcommand{\latexhtml}[2]{#1}
\end{verbatim}
\end{quote}
If you don't do this then your document will be converted using
\latextohtml\ version 97.1.
\end{itemize}
If all this doesn't work then you'll have to do the conversion properly.

\newpage
\section{\xlabel{the_star2html_command}The \texttt{star2html} command\label{star2html}}

The \texttt{star2html} command sets up a Starlink-specific environment before
calling \latextohtml\ to produce a hypertext version of a Starlink document.
A typical invocation, is simply:

\begin{quote}
\texttt{\# star2html sun199}
\end{quote}

However, there are some optional switches which you might find useful.
The full format of the command is:

\begin{quote}
\begin{verbatim}
# star2html [-a author(s)] [-aux] [-c copyright_statement]
            [-d document_date] [-m email_address] [-s 0|1]
            [-old] [-obsolete] [-t title] [arguments for LaTeX2HTML] document
\end{verbatim}
\end{quote}

The switches (all of which are optional) have the following meanings:

\begin{description}
\item[\texttt{-a authors}] \dash\
   The names of the document authors.
   If not supplied, the value of the command \verb#\stardocauthors# will be
   used.
   If this doesn't exist, the name of the person running the application will
   be used.
   The value appears in the footer of every page of the hypertext version.
\item[\texttt{-aux}] \dash\
   This controls whether or not the document \texttt{.aux} file is used.
   (This file is produced by running LaTeX on the source file, and contains a
   description of the section numbering).
   If this switch is omitted, the \texttt{.aux} file is not used.
   This default is sensible because the structure of the paper version
   may differ from the structure of the hypertext version.
   For example, optional diagrams and/or sections may be specified by
   \verb+\htmlonly+ and \verb+\latexonly+ environments.
\item[\texttt{-c copyright\_statement}] \dash\
   A copyright statement to override the default which will appear
   in the footer of each page. The default statement is found from
   the \verb#\stardoccopyright# command if there is one in the document, or
   failing that from the environment variable \verb#HTX_COPYRIGHT# if defined,
   or failing that the default is: ``Copyright \copyright\ YYYY Council for
   the Central Laboratory of the Research Councils'' (where YYYY is the
   current year).
   To prevent any copyright statement from appearing, specify \verb#"-c none"#
   or, in the absence of a \verb#\stardoccopyright# command or explicit
   \verb#-c "copyright_statement"#, set  \verb#HTX_COPYRIGHT# to \verb#"none"#.
\item[\texttt{-d document\_date}] \dash\
   The date when the document was created.
   If not supplied, the value of the \verb#\stardocdate# command will be used.
   If this doesn't exist, today's date will be used.
   The value appears in the footer of every page of the hypertext version.
\item[\texttt{-m e-mail\_address}] \dash\
   A contact e-mail address for the document.
   If not supplied, the value of the environment variable
   \verb#HTX_EMAIL# will be used.
   If this isn't set, then the e-mail address
   ``\texttt{starlink@jiscmail.ac.uk}'' will be used.
   The value appears in the footer of every page of the hypertext version.
   If you are only responsible for producing your own documents, you may find
   it convenient to set this environment variable in your \texttt{.login}
   script.
\item[\texttt{-obsolete}] \dash\
   This flag indicates you want to mark the document as obsolete. This is
   done by displaying an image with the word OBSOLETE written into it as
   the background.
\item[\texttt{-old}] \dash\
   This flag indicates that you want to use the ``old''
   compatibility version of \latextohtml. Normally the \verb#star2html#
   command decides which version to use by looking for unique
   signatures in your document prologue. At this time the
   presence of the \verb#\latexhtml# command signifies a document that
   complies with the current standard.
\item[\texttt{-s 1|0}] \dash\
   This switch controls whether the input file will be split
   into parts, or not, before being processed by \latextohtml.
   The split occurs at \verb#\section#, \verb#\subsection#,
   \verb#\sstroutine# and \verb#\routine# statements.
   Splitting up the document may considerable speed up processing and
   consequently this is the default.
\item[\texttt{-t title}] \dash\
   The title of the document.
   If not supplied, the value of the \verb#\stardoctitle# command will be used.
   If this fails, a name generated from the document type will be used.
   The value is used as a title for the first page, and appears in the footer of
   every page of the hypertext version.
\item[\texttt{LATEX2HTML arguments}] \dash\
   Any unknown arguments before the file name will be passed to \latextohtml\
   (do not use the \verb#-dir# and \verb#-no_subdir# arguments).
\item[\texttt{document}] \dash\
   The name of the document to be converted into HTML (the file type
   \texttt{.tex} is optional).

\end{description}

Some error messages always appear, but they can usually be ignored.

\newpage
\section{\xlabel{the_xlabel_command}\label{xlabel}The \texttt{xlabel} command}
The \texttt{xlabel} command is used to ensure that all sections and subsections
in the document include an \verb+\xlabel+ command labelling them with the
section heading in lower case with all non-alphanumeric characters removed and
words separated by underscore.

The format of the command is:
\begin{quote}
\verb+# xlabel+ \textit{document}
\end{quote}

This will process the file \textit{document} and create a new version,
\texttt{new}\textit{document}, containing any additional \verb+\xlabel+
commands required. The file type \texttt{.tex} is optional and will be added
if it is omitted.

A label is created from the heading of each \verb+\section+ and
\verb+\subsection+ in the specified document by removing common font-changing
commands (such as \verb+\texttt+, \verb+\texttt+, \verb+\em+ and \verb+\emph+),
replacing all spaces by underscore, removing all non-alphanumeric characters
(underscore counts as alphanumeric here) and folding to lower case.
If the sectioning command does not already contain an \verb+\xlabel+ command
specifying the generated label, one is inserted.

The labels generated will be displayed and should be inspected for
peculiarities, probably caused by unusual \LaTeX\ commands in the heading.

Warnings will be displayed if:
\begin{itemize}
\item The document contains an \verb+\xlabel+ command specifying the same
label for a different section. Note that duplicates for \verb+\xlabel+'s not
in section or subsection headings will not be detected.
\item The section already contains an \verb+\xlabel+ command specifying
a different label.
\item The section heading resulted in an empty label.
\end{itemize}

When you are happy with the result, copy \texttt{new}\textit{document} back to
\textit{document}.

\section{\xlabel{the_latex2html_command}\label{latex2html_command}
         The \texttt{latex2html} command}
The \verb#latex2html# command runs the current version of
\latextohtml\ using the command-line options supplied (see the
\latextohtml\ \htmladdnormallinkfoot{manual}{\LtoHManURL}
\latex{(MUD/152)}  for a complete list of these). If you want to run
\latextohtml\ version 97.1, then add the extra parameter: \verb#-vers old#.



\newpage

\section{\xlabel{the_structure_of_a_converted_hypertext_document}The structure of a converted hypertext document}

Star2HTML bases the structure of the hypertext version of a \LaTeX\ document
on the original structure defined by commands like \verb+\section+.
However, it changes the format of the title page, adds buttons and links
to aid navigation within the document, and adds identification information
to each page.
It does this automatically.

It splits a document up into pages.
Each section, subsection, and so on, is stored in its own page.
Each page is divided into bands, separated by a horizontal line.
The top-level page is the \emph{Title Page}.
This is the first page you see when you start reading the document.

The Title Page is divided into the following bands:

\begin{enumerate}
\item A \emph{Navigation Band}.
 This contains a row of navigation buttons (some of which are blanked out),
 followed by a line containing a link to the document Abstract (by default).
\item A \emph{Title Band}.
 This contains information from the title page of the paper document (title,
 document number, author, date, \textit{etc.}), but presented in a different
 order and style.
 References to CCLRC/RAL, PPARC, and Starlink are converted into links
 to their associate web pages.
 A button is added to help people copy the document.
\item A \emph{Contents Band}.
 This looks like two bands because a horizontal line appears between the heading
 ``Contents'' and the list of section titles (which are links to the
 section pages).
\item A \emph{Navigation Band}.
 This is a duplicate of the band at the top of the page.
 It only appears in long pages.
\item An \emph{Identification Band}.
 Shows the document title, code, author, date, and e-mail address.
\end{enumerate}

The other pages are made up of similar bands, some of which don't always
appear.

\begin{enumerate}
\item A \emph{Navigation Band}.
The lines below the navigation buttons show the titles of the Next, Up, and
Previous pages.
``Up'' means the next level up in the structural hierarchy, thus if the page is
a subsection, it will point to the containing section.
\item A \emph{Text Band}.
This contains the author's text for the section.
\item A \emph{Structure Band}.
This shows the titles of the subsections in this section, with links to the
pages containing the text.
If the section has no subsections, this band won't appear.
\item A \emph{Navigation Band}.
This is a duplicate of the band at the top of the page.
It only appears in long pages.
\item An \emph{Identification Band}.
 Same as for the Title Page.
\end{enumerate}

If your document has an index it will be converted to an index page with
links from each entry to the indexed entry.
The last page in the hypertext document will be called ``About this
document \ldots''.
This gives details about how and when the hypertext document was produced.
It has no equivalent in the paper version.

\end{document}
