\begin{htmlonly}
\documentclass[dvips,a4paper]{article}
\usepackage{html,htmllist,makeidx}

%this font is used in  features.tex
\font\wncyr = wncyr10


\begin{htmlonly}
 \usepackage[dvips]{graphicx}
 \usepackage[dvips]{color}
 \usepackage[dvips,leftbars]{changebar}
 \newcommand{\FoilTeX}{\env{FoilTeX}}
%
 \def\url#1{\htmladdnormallink{#1}{#1}}
 \def\Email#1{\htmladdnormallink{<#1>}{mailto:#1}}
 \def\path#1{\texttt{#1}}
% \def\urldef#1#2#3{\def#1{#2{#3}}}
 \def\onlinedoc{\url{http://www-dsed.llnl.gov/files/programs/unix/latex2html/manual/}}
 \def\patches{\url{http://www-dsed.llnl.gov/files/programs/unix/latex2html/}}
 \def\sourceA{%
  \url{http://www-dsed.llnl.gov/files/programs/unix/latex2html/sources/}}
% \def\sourceB{\url{ftp://ftp.mpn.com/pub/nikos/latex2html-98.1.tar.gz}}
 \def\sourceB{\CTANtug{\CTANA}}
 \def\sourceC{\url{http://ftp.rzg.mpg.de/pub/software/latex2html/sources/}}
 \def\CVSsite{\url{http://cdc-server.cdc.informatik.tu-darmstadt.de/\~{}latex2html/}}
 \def\CVSrepos{\url{http://cdc-server.cdc.informatik.tu-darmstadt.de/\~{}latex2html/user/}}
 \def\CVSlatest{\url{http://cdc-server.cdc.informatik.tu-darmstadt.de/\~{}latex2html/l2h-latest.tar.gz}}
 \newcommand{\CTANtug}[1]{\path{http://ctan.tug.org/ctan/#1}}
 \def\CTANA{tex-archive/support/latex2html}
 \def\tugURL{\url{http://www.tug.org}}
 \def\danteURL{\url{http://www.dante.de}}
 \def\ListURL{\url{http://www.rosat.mpe-garching.mpg.de/mailing-lists/LaTeX2HTML/}}
%
 \def\glossary#1{\index{#1@\texttt{#1} \label{III#1}\htmlref{(G)}{GGG#1}}}
 \def\Glossary#1#2{\index{#1@{#2} \label{III#1}\htmlref{(G)}{GGG#1}}}
%
% hack to suppress changebar entries read from .aux file  !!! bug in changebar.perl !!! 
 \makeatletter
 \def\cb@barpoint#1#2#3{}
 \makeatother
%
\newcommand{\sameas}[1]{\textcolor{red}{Same as setting: #1}}
\newcommand{\onlinedocRM}{\url{http://www-math.mpce.mq.edu.au/\~{}ross/latex/manual/manual.html}}
\newcommand{\EXcolors}{\url{http://www-math.mpce.mq.edu.au/\~{}ross/latex/crayola/crayola.html}}
%
\newcommand{\Lc}[1]{\texttt{\char92#1}} % LaTeX command
\newcommand{\Tc}[1]{\texttt{\char92#1}} % TeX command
\newcommand{\Cs}[1]{ \texttt{-#1} }   % command-line switch
\newcommand{\Ve}[1]{\index{#1@\texttt{#1}}\texttt{#1}} % version extension
\newcommand{\gsl}[1]{#1}
\newcommand{\indexentry}[2]{\item #1 #2}

%
\internal{}%
\internal{O}%
\internal{S}%
\internal{E}%
\internal{M}%
\internal{H}%
\internal{P}%

\newcommand{\texdev}[1]{\htmladdnormallink
 {\texdevURL/#1}{\texdevURL/#1}}
\newcommand{\ctan}{http://ctan.tug.org/ctan}
\newcommand{\ctanTUG}{\htmladdnormallinkfoot
 {TUG's searchable CTAN site}{\ctan}}
\newcommand{\ctanURL}[1]{\htmladdnormallink
 {CTAN:\texttt{.../#1}}{\ctan/tex-archive/#1}}
\newcommand{\ctanURLbr}[1]{\htmladdnormallink
 {CTAN:\texttt{.../#1}}{\ctan/tex-archive/#1}}
\newcommand{\ctanTUGurl}[2]{\ctanURL{#2}}
\newcommand{\indichtml}{l2h/indic/IndicHTML/}

\end{htmlonly}

\newcommand{\texdevURL}{http://www-texdev.mpce.mq.edu.au/}
\newcommand{\Unicode}{\htmladdnormallinkfoot
 {Unicode}{http://www.unicode.org/}}
\newcommand{\IndicHTML}{\htmladdnormallinkfoot
 {Indic\TeX/HTML}{\texdevURL{\indichtml}}}

%begin{latexonly}
\usepackage{array}

\newcommand{\Cs}[1]{{\upshape`\,\texttt{-#1}\,'}}
\newcommand{\Ve}[1]{\index{#1@\texttt{#1}}{\upshape`\,\texttt{#1}\,'}}
\newcommand{\Lc}[1]{{\upshape\ttfamily\char92#1}}
\newcommand{\Tc}[1]{{\upshape\ttfamily\char92#1}}

\newcommand{\ctanTUG}[1]{TUG's searchable CTAN site\footnote{\ctanTUGurl}}
\newcommand{\ctanURL}{CTAN: \penalty-200\ctanurl}
\newcommand{\ctanURLbr}{CTAN: \newline\ctanurl}
\urldef\texdev\url{http://www-texdev.mpce.mq.edu.au/}
\urldef\ctanurl\url{.../tex-archive/}
\urldef\ctanTUGurl\url{http://ctan.tug.org/ctan/tex-archive/}
\urldef\indichtml\url{l2h/indic/IndicHTML/}

\def\mathsmiley{\smiley}
%end{latexonly}

% thanks to \KrisRose for this macro:
\def\smiley{\hbox{\rlap{$\bigcirc$}\kern1.3pt$\scriptstyle\ddot\smile$}}

\begin{imagesonly}
\def\mathsmiley{\smiley}
\end{imagesonly}


\newcommand{\Lcs}[1]{{\upshape\ttfamily\char92#1}}

\renewcommand{\thefootnote}{\arabic{footnote}}

%\newcommand{\godown}[1]{{\htmlref
% {\htmladdimg[left BORDER=0]{../psfiles/dn.gif}}{#1}}}
\newcommand{\godown}[1]{}

%\newcommand{\goback}[1]{{\htmlref
% {\htmladdimg[left BORDER=0]{../psfiles/up.gif}}{#1}}}
\newcommand{\goback}[1]{}


\newcommand{\WiiiC}{\htmladdnormallinkfoot{World Wide Web Consortium}{http://www.w3c.org/}}
\newcommand{\latextohtmlNG}{\latextohtml-NG}
\newcommand{\maillist}{\htmladdnormallinkfoot{\latextohtml{} mailing list}%
{http://cbl.leeds.ac.uk/nikos/tex2html/doc/mail/mail.html}}
%
\newcommand{\HTMLiii}{\textup{\texttt{HTML} 3.2}}%
\newcommand{\HTMLIII}{\htmladdnormallink{\HTMLiii}%
{http://www.w3.org/TR/REC-html32.html}}%

\newcommand{\HTMLiv}{\textup{\texttt{HTML} 4.0}}%
\newcommand{\HTMLIV}{\htmladdnormallink{\HTMLiv}%
{http://www.w3.org/TR/REC-html40/}}%

%\newcommand{\latextohtml}{\textup{\LaTeX 2{\ttfamily HTML}}}%
\newcommand{\Perl}{\htmlref{\textsl{Perl}}{GGGPerl}}%  
\newcommand{\PS}{\htmlref{\textup{Post\-Script}}{GGGpostscript}\Glossary{PostScript}{PostScript}{}}%
\newcommand{\MF}{\htmlref{\textsl{Metafont}}{GGGMetafont}\Glossary{Metafont}{\textsl{Metafont}}{}}%
\newcommand{\fn}[1]{\htmlref{\texttt{#1}}{GGG#1}\glossary{#1}}%  file names, with link to glossary
\newcommand{\gn}[1]{\texttt{#1}\label{GGG#1}\htmlref{\^{}}{III#1}}%  file names, labelled within glossary
%\newcommand{\appl}[1]{\htmlref{\textsl{#1}}{GGG#1}\Glossary{#1}{\textsl{#1}}{}}%  application software names
\newcommand{\appl}[1]{\htmlref{\textsl{#1}}{GGG#1}\Glossary{#1}{\gsl{#1}}{}}%  application software names
%
\newcommand{\env}[1]{{\upshape\sffamily #1}}%  LaTeX environment and package names
\newcommand{\HTMLtag}[1]{\path{<#1>}}%  HTML tag
\newcommand{\Meta}[1]{\texttt{\upshape<\textit{#1}>}}%  Meta string

%
% developer names and addresses:
%
\newcommand{\NikosDrakos}{\index{Nikos Drakos}%\Email{nikos@cbl.leeds.ac.uk}
\htmladdnormallink{Nikos Drakos}{http://www.cbl.leeds.ac.uk/nikos/personal.html}}
%
\newcommand{\RossMoore}{\index{Ross Moore}%\Email{ross@mpce.mq.edu.au}
\htmladdnormallink{Ross Moore}{http://www.mpce.mq.edu.au/\~{}ross/}}
\newcommand{\Macquarie}{\htmladdnormallink
{Macquarie University}{http://www.mq.edu.au/}}
%
\newcommand{\Hennecke}{\index{Marcus Hennecke}%\Email{hennecke@dbag.ulm.DaimlerBenz.COM}
\htmladdnormallink{Marcus Hennecke}{http://www.crc.ricoh.com/\~{}marcush/}}
%
\newcommand{\Noworolski}{\index{Mark Noworolski}%\Email{jmn@eecs.berkeley.edu}
\htmladdnormallink{Mark Noworolski}{http://www-power.eecs.berkeley.edu/\~{}jmn/}}
%
\newcommand{\Isani}{\index{Sidik Isani}%\Email{isani@cfht.hawaii.edu}
\htmladdnormallink{Sidik Isani}{http://www.cfht.hawaii.edu/\~{}isani/si.html}}
%
\newcommand{\Goossens}{\index{Michel Goossens}%\Email{goossens@cern.ch}
\htmladdnormallink{Michel Goossens}{http://wwwcn1.cern.ch/\~{}goossens/}}
%
\newcommand{\Wilck}{\index{Martin Wilck}%\Email{martin@tropos.de}
\htmladdnormallink{Martin Wilck}{http://www.tropos.de/personal/wilck.html}}
%
\newcommand{\PatrickDaly}{\index{Patrick Daly}%\Email{daly@linmpi.dnet.gwdg.de}
\htmladdnormallink{Patrick Daly}{mailto:daly@linmpi.dnet.gwdg.de}}
%
\newcommand{\HerbSwan}{\index{Herb Swan}%\Email{herb.swan@perc.Arco.COM}
%\htmladdnormallink{Herb Swan}{mailto:herb.swan@perc.Arco.COM}}
\htmladdnormallink{Herb Swan}{mailto:lanhws@expl.aai.arco.com}}
%
\newcommand{\Lippmann}{\index{Jens Lippmann}%\Email{lippmann@cdc.informatik.tu-darmstadt.de}
\htmladdnormallink{Jens Lippmann}{http://www-jb.cs.uni-sb.de/\~{}www/people/lippmann}}
%
\newcommand{\Rouchal}{\index{Marek Rouchal}%\Email{marek@hl.siemens.de}
\htmladdnormallink{Marek Rouchal}{mailto:marek@hl.siemens.de}}
%
\newcommand{\Bohnet}{\index{Achim Bohnet}%\Email{ach@rosat.mpe-garching.mpg.de}
\htmladdnormallink{Achim Bohnet}{mailto:ach@rosat.mpe-garching.mpg.de}}
%
\newcommand{\Nelson}{\index{Scott Nelson}%\Email{nelson18@llnl.gov}
\htmladdnormallink{Scott Nelson}{mailto:nelson18@llnl.gov}}
%
\newcommand{\AxelRamge}{\index{Axel Ramge}%\Email{axel@ramge.de}
\htmladdnormallink{Axel Ramge}{mailto:axel@ramge.de}}
%
\newcommand{\Popineau}{\index{Fabrice Popineau}%\Email{popineau@esemetz.ese-metz.fr}
\htmladdnormallink{Fabrice Popineau}{mailto:popineau@esemetz.ese-metz.fr}}
%
\newcommand{\Wortmann}{\index{Uli Wortmann}%\Email{uli12@bonk.ethz.ch}
\htmladdnormallink{Uli Wortmann}{mailto:uli12@bonk.ethz.ch}}
%
\newcommand{\Veytsman}{\index{Boris Veytsman}%\Email{boris@plmsc.psu.edu}
\htmladdnormallink{Boris Veytsman}{mailto:boris@plmsc.psu.edu}}
%
\newcommand{\Taupin}{\index{Daniel Taupin}%\Email{taupin@lps.u-psud.fr}
\htmladdnormallink{Daniel Taupin}{mailto:taupin@lps.u-psud.fr}}
%
%\endinput

%
% (La)TeX related URLs
%
\newcommand{\CSEP}{\index{Computer Science Education Project!CSEP}%
\htmladdnormallinkfoot{Computer Science Education Project}%
{http://csep1.phy.ornl.gov/csep.html}} 
%
\newcommand{\CBLU}{\index{Computer Based Learning Unit!CBLU}%
\htmladdnormallinkfoot{Computer Based Learning Unit}%
{http://cbl.leeds.ac.uk/\~{}www/home.html}}
%
\newcommand{\CERN}{\index{CERN}%
\htmladdnormallink{CERN}{http://wwwcn.cern.ch/Welcome.html}} 
%
\newcommand{\KrisRose}{\index{Kristoffer Rose}%\Email{krisrose@brics.dk}
\htmladdnormallink{Kristoffer Rose}{http://www.brics.dk/\~{}krisrose/}}
%
\newcommand{\XypicDK}{\index{Xy-pic@\protect\Xy-pic graphics package}%
\index{Xy-pic@\protect\Xy-pic graphics package!home page}%
\htmladdnormallink{Xy-pic}{http://www.brics.dk/\~{}krisrose/Xy-pic.html}}
\newcommand{\XypicAUS}{\index{Xy-pic@\protect\Xy-pic graphics package!home page, down under}%
\htmladdnormallink{Ross Moore}{http://www.mpce.mq.edu.au/\~{}ross/Xy-pic.html}}
%
\newcommand{\LiPS}{\index{LiPS Design Team}%
\htmladdnormallink{LiPS Design Team}{http://www-jb.cs.uni-sb.de/LiPS/node2.html}}
\newcommand{\FIDarmstadt}{\index{Fachbereich Informatik, Darmstadt}%
\htmladdnormallink{Fachbereich Informatik}{http://www.informatik.tu-darmstadt.de/}}
\newcommand{\Darmstadt}{\index{Darmstadt!Fachbereich Informatik}%
\htmladdnormallink{Darmstadt}{http://www.tu-darmstadt.de/Welcome.de.html}}
%
\newcommand{\DANTE}{\index{DANTE}%
\htmladdnormallink{DANTE e.V.}{http://www.dante.de/}}
\newcommand{\Praesidium}{\index{DANTE!Praesidium}%
\htmladdnormallink{Praesidium}{http://www.dante.de/dante/Organe.html}}
\newcommand{\LaTeXiii}{\index{LaTeX@\LaTeX!LaTeX3@\LaTeX{3}}%
\htmladdnormallink{\LaTeX{3}}{http://www.tex.ac.uk/CTAN/latex/latex3}}
%
\newcommand{\Engberg}{\index{Uffe Engberg}%
\htmladdnormallinkfoot{Uffe Engberg}{http://www.brics.dk/\~{}engberg}} 

\newcommand{\texdevINDIC}{\texdev\indic}
%\endinput


%% earlier contributions from...
\newcommand{\AndrewCole}{\index{Computer Based Learning Unit!Andrew Cole}%
\htmladdnormallink{Andrew Cole}{http://www.cbl.leeds.ac.uk/ajcole/personal.html}}
%
\newcommand{\AnaPaiva}{\index{Computer Based Learning Unit!Ana Maria Paiva}%
\htmladdnormallink{Ana Maria Paiva}{http://www.cbl.leeds.ac.uk/amp/personal.html}}
%
\newcommand{\RodWilliams}{\index{Computer Based Learning Unit!Roderick Williams}%
\htmladdnormallink{Roderick Williams}{http://www.cbl.leeds.ac.uk/rodw/personal.html}}
%
\newcommand{\JamilSawar}{\index{Computer Based Learning Unit!Jamil Sawar}%
\htmladdnormallink{Jamil Sawar}{http://www.cbl.leeds.ac.uk/sawar/personal.html}}

\newcommand{\RobertThau}{\index{Robert S. Thau}%
\htmladdnormallink{Robert S. Thau}{mailto: rst@edu.mit.ai}}

\endinput




%
\newcommand{\RobertThau}{\index{Robert S. Thau}%
\htmladdnormallink{Robert S. Thau}{mailto: rst@edu.mit.ai}}


%
% Ian Foster \Email{itf@mcs.anl.gov}
% Bob Olson \Email{olson@mcs.anl.gov}
% Verena Umar \Email{verena@edu.vanderbilt.cas.compsci}
% Axel Belinfante \Email{Axel.Belinfante@cs.utwente.nl}
% Todd Little \Email{little@com.dec.enet.nuts2u}
% Franz Vojik \Email{vojik@de.tu-muenchen.informatik}
% Eric Carroll \Email{eric@ca.utoronto.utcc.enfm}
% Roderick Williams \Email{rodw@cbl.leeds.ac.uk}
% Robert Cailliau \Email{cailliau@cernnext.cern.ch}
% Toni Lantunen at CERN
% Ana Maria Paiva, Jamil Sawar, Andrew Cole at CBLU Leeds
% Phillip Conrad (Perfect Byte, Inc.)
% L. Peter Deutsch (

%
\begin{htmlonly}
\documentclass[dvips,a4paper]{article}
\usepackage{html,htmllist,makeidx}

%this font is used in  features.tex
\font\wncyr = wncyr10


\begin{htmlonly}
 \usepackage[dvips]{graphicx}
 \usepackage[dvips]{color}
 \usepackage[dvips,leftbars]{changebar}
 \newcommand{\FoilTeX}{\env{FoilTeX}}
%
 \def\url#1{\htmladdnormallink{#1}{#1}}
 \def\Email#1{\htmladdnormallink{<#1>}{mailto:#1}}
 \def\path#1{\texttt{#1}}
% \def\urldef#1#2#3{\def#1{#2{#3}}}
 \def\onlinedoc{\url{http://www-dsed.llnl.gov/files/programs/unix/latex2html/manual/}}
 \def\patches{\url{http://www-dsed.llnl.gov/files/programs/unix/latex2html/}}
 \def\sourceA{%
  \url{http://www-dsed.llnl.gov/files/programs/unix/latex2html/sources/}}
% \def\sourceB{\url{ftp://ftp.mpn.com/pub/nikos/latex2html-98.1.tar.gz}}
 \def\sourceB{\CTANtug{\CTANA}}
 \def\sourceC{\url{http://ftp.rzg.mpg.de/pub/software/latex2html/sources/}}
 \def\CVSsite{\url{http://cdc-server.cdc.informatik.tu-darmstadt.de/\~{}latex2html/}}
 \def\CVSrepos{\url{http://cdc-server.cdc.informatik.tu-darmstadt.de/\~{}latex2html/user/}}
 \def\CVSlatest{\url{http://cdc-server.cdc.informatik.tu-darmstadt.de/\~{}latex2html/l2h-latest.tar.gz}}
 \newcommand{\CTANtug}[1]{\path{http://ctan.tug.org/ctan/#1}}
 \def\CTANA{tex-archive/support/latex2html}
 \def\tugURL{\url{http://www.tug.org}}
 \def\danteURL{\url{http://www.dante.de}}
 \def\ListURL{\url{http://www.rosat.mpe-garching.mpg.de/mailing-lists/LaTeX2HTML/}}
%
 \def\glossary#1{\index{#1@\texttt{#1} \label{III#1}\htmlref{(G)}{GGG#1}}}
 \def\Glossary#1#2{\index{#1@{#2} \label{III#1}\htmlref{(G)}{GGG#1}}}
%
% hack to suppress changebar entries read from .aux file  !!! bug in changebar.perl !!! 
 \makeatletter
 \def\cb@barpoint#1#2#3{}
 \makeatother
%
\newcommand{\sameas}[1]{\textcolor{red}{Same as setting: #1}}
\newcommand{\onlinedocRM}{\url{http://www-math.mpce.mq.edu.au/\~{}ross/latex/manual/manual.html}}
\newcommand{\EXcolors}{\url{http://www-math.mpce.mq.edu.au/\~{}ross/latex/crayola/crayola.html}}
%
\newcommand{\Lc}[1]{\texttt{\char92#1}} % LaTeX command
\newcommand{\Tc}[1]{\texttt{\char92#1}} % TeX command
\newcommand{\Cs}[1]{ \texttt{-#1} }   % command-line switch
\newcommand{\Ve}[1]{\index{#1@\texttt{#1}}\texttt{#1}} % version extension
\newcommand{\gsl}[1]{#1}
\newcommand{\indexentry}[2]{\item #1 #2}

%
\internal{}%
\internal{O}%
\internal{S}%
\internal{E}%
\internal{M}%
\internal{H}%
\internal{P}%

\newcommand{\texdev}[1]{\htmladdnormallink
 {\texdevURL/#1}{\texdevURL/#1}}
\newcommand{\ctan}{http://ctan.tug.org/ctan}
\newcommand{\ctanTUG}{\htmladdnormallinkfoot
 {TUG's searchable CTAN site}{\ctan}}
\newcommand{\ctanURL}[1]{\htmladdnormallink
 {CTAN:\texttt{.../#1}}{\ctan/tex-archive/#1}}
\newcommand{\ctanURLbr}[1]{\htmladdnormallink
 {CTAN:\texttt{.../#1}}{\ctan/tex-archive/#1}}
\newcommand{\ctanTUGurl}[2]{\ctanURL{#2}}
\newcommand{\indichtml}{l2h/indic/IndicHTML/}

\end{htmlonly}

\newcommand{\texdevURL}{http://www-texdev.mpce.mq.edu.au/}
\newcommand{\Unicode}{\htmladdnormallinkfoot
 {Unicode}{http://www.unicode.org/}}
\newcommand{\IndicHTML}{\htmladdnormallinkfoot
 {Indic\TeX/HTML}{\texdevURL{\indichtml}}}

%begin{latexonly}
\usepackage{array}

\newcommand{\Cs}[1]{{\upshape`\,\texttt{-#1}\,'}}
\newcommand{\Ve}[1]{\index{#1@\texttt{#1}}{\upshape`\,\texttt{#1}\,'}}
\newcommand{\Lc}[1]{{\upshape\ttfamily\char92#1}}
\newcommand{\Tc}[1]{{\upshape\ttfamily\char92#1}}

\newcommand{\ctanTUG}[1]{TUG's searchable CTAN site\footnote{\ctanTUGurl}}
\newcommand{\ctanURL}{CTAN: \penalty-200\ctanurl}
\newcommand{\ctanURLbr}{CTAN: \newline\ctanurl}
\urldef\texdev\url{http://www-texdev.mpce.mq.edu.au/}
\urldef\ctanurl\url{.../tex-archive/}
\urldef\ctanTUGurl\url{http://ctan.tug.org/ctan/tex-archive/}
\urldef\indichtml\url{l2h/indic/IndicHTML/}

\def\mathsmiley{\smiley}
%end{latexonly}

% thanks to \KrisRose for this macro:
\def\smiley{\hbox{\rlap{$\bigcirc$}\kern1.3pt$\scriptstyle\ddot\smile$}}

\begin{imagesonly}
\def\mathsmiley{\smiley}
\end{imagesonly}


\newcommand{\Lcs}[1]{{\upshape\ttfamily\char92#1}}

\renewcommand{\thefootnote}{\arabic{footnote}}

%\newcommand{\godown}[1]{{\htmlref
% {\htmladdimg[left BORDER=0]{../psfiles/dn.gif}}{#1}}}
\newcommand{\godown}[1]{}

%\newcommand{\goback}[1]{{\htmlref
% {\htmladdimg[left BORDER=0]{../psfiles/up.gif}}{#1}}}
\newcommand{\goback}[1]{}


\newcommand{\WiiiC}{\htmladdnormallinkfoot{World Wide Web Consortium}{http://www.w3c.org/}}
\newcommand{\latextohtmlNG}{\latextohtml-NG}
\newcommand{\maillist}{\htmladdnormallinkfoot{\latextohtml{} mailing list}%
{http://cbl.leeds.ac.uk/nikos/tex2html/doc/mail/mail.html}}
%
\newcommand{\HTMLiii}{\textup{\texttt{HTML} 3.2}}%
\newcommand{\HTMLIII}{\htmladdnormallink{\HTMLiii}%
{http://www.w3.org/TR/REC-html32.html}}%

\newcommand{\HTMLiv}{\textup{\texttt{HTML} 4.0}}%
\newcommand{\HTMLIV}{\htmladdnormallink{\HTMLiv}%
{http://www.w3.org/TR/REC-html40/}}%

%\newcommand{\latextohtml}{\textup{\LaTeX 2{\ttfamily HTML}}}%
\newcommand{\Perl}{\htmlref{\textsl{Perl}}{GGGPerl}}%  
\newcommand{\PS}{\htmlref{\textup{Post\-Script}}{GGGpostscript}\Glossary{PostScript}{PostScript}{}}%
\newcommand{\MF}{\htmlref{\textsl{Metafont}}{GGGMetafont}\Glossary{Metafont}{\textsl{Metafont}}{}}%
\newcommand{\fn}[1]{\htmlref{\texttt{#1}}{GGG#1}\glossary{#1}}%  file names, with link to glossary
\newcommand{\gn}[1]{\texttt{#1}\label{GGG#1}\htmlref{\^{}}{III#1}}%  file names, labelled within glossary
%\newcommand{\appl}[1]{\htmlref{\textsl{#1}}{GGG#1}\Glossary{#1}{\textsl{#1}}{}}%  application software names
\newcommand{\appl}[1]{\htmlref{\textsl{#1}}{GGG#1}\Glossary{#1}{\gsl{#1}}{}}%  application software names
%
\newcommand{\env}[1]{{\upshape\sffamily #1}}%  LaTeX environment and package names
\newcommand{\HTMLtag}[1]{\path{<#1>}}%  HTML tag
\newcommand{\Meta}[1]{\texttt{\upshape<\textit{#1}>}}%  Meta string

%
% developer names and addresses:
%
\newcommand{\NikosDrakos}{\index{Nikos Drakos}%\Email{nikos@cbl.leeds.ac.uk}
\htmladdnormallink{Nikos Drakos}{http://www.cbl.leeds.ac.uk/nikos/personal.html}}
%
\newcommand{\RossMoore}{\index{Ross Moore}%\Email{ross@mpce.mq.edu.au}
\htmladdnormallink{Ross Moore}{http://www.mpce.mq.edu.au/\~{}ross/}}
\newcommand{\Macquarie}{\htmladdnormallink
{Macquarie University}{http://www.mq.edu.au/}}
%
\newcommand{\Hennecke}{\index{Marcus Hennecke}%\Email{hennecke@dbag.ulm.DaimlerBenz.COM}
\htmladdnormallink{Marcus Hennecke}{http://www.crc.ricoh.com/\~{}marcush/}}
%
\newcommand{\Noworolski}{\index{Mark Noworolski}%\Email{jmn@eecs.berkeley.edu}
\htmladdnormallink{Mark Noworolski}{http://www-power.eecs.berkeley.edu/\~{}jmn/}}
%
\newcommand{\Isani}{\index{Sidik Isani}%\Email{isani@cfht.hawaii.edu}
\htmladdnormallink{Sidik Isani}{http://www.cfht.hawaii.edu/\~{}isani/si.html}}
%
\newcommand{\Goossens}{\index{Michel Goossens}%\Email{goossens@cern.ch}
\htmladdnormallink{Michel Goossens}{http://wwwcn1.cern.ch/\~{}goossens/}}
%
\newcommand{\Wilck}{\index{Martin Wilck}%\Email{martin@tropos.de}
\htmladdnormallink{Martin Wilck}{http://www.tropos.de/personal/wilck.html}}
%
\newcommand{\PatrickDaly}{\index{Patrick Daly}%\Email{daly@linmpi.dnet.gwdg.de}
\htmladdnormallink{Patrick Daly}{mailto:daly@linmpi.dnet.gwdg.de}}
%
\newcommand{\HerbSwan}{\index{Herb Swan}%\Email{herb.swan@perc.Arco.COM}
%\htmladdnormallink{Herb Swan}{mailto:herb.swan@perc.Arco.COM}}
\htmladdnormallink{Herb Swan}{mailto:lanhws@expl.aai.arco.com}}
%
\newcommand{\Lippmann}{\index{Jens Lippmann}%\Email{lippmann@cdc.informatik.tu-darmstadt.de}
\htmladdnormallink{Jens Lippmann}{http://www-jb.cs.uni-sb.de/\~{}www/people/lippmann}}
%
\newcommand{\Rouchal}{\index{Marek Rouchal}%\Email{marek@hl.siemens.de}
\htmladdnormallink{Marek Rouchal}{mailto:marek@hl.siemens.de}}
%
\newcommand{\Bohnet}{\index{Achim Bohnet}%\Email{ach@rosat.mpe-garching.mpg.de}
\htmladdnormallink{Achim Bohnet}{mailto:ach@rosat.mpe-garching.mpg.de}}
%
\newcommand{\Nelson}{\index{Scott Nelson}%\Email{nelson18@llnl.gov}
\htmladdnormallink{Scott Nelson}{mailto:nelson18@llnl.gov}}
%
\newcommand{\AxelRamge}{\index{Axel Ramge}%\Email{axel@ramge.de}
\htmladdnormallink{Axel Ramge}{mailto:axel@ramge.de}}
%
\newcommand{\Popineau}{\index{Fabrice Popineau}%\Email{popineau@esemetz.ese-metz.fr}
\htmladdnormallink{Fabrice Popineau}{mailto:popineau@esemetz.ese-metz.fr}}
%
\newcommand{\Wortmann}{\index{Uli Wortmann}%\Email{uli12@bonk.ethz.ch}
\htmladdnormallink{Uli Wortmann}{mailto:uli12@bonk.ethz.ch}}
%
\newcommand{\Veytsman}{\index{Boris Veytsman}%\Email{boris@plmsc.psu.edu}
\htmladdnormallink{Boris Veytsman}{mailto:boris@plmsc.psu.edu}}
%
\newcommand{\Taupin}{\index{Daniel Taupin}%\Email{taupin@lps.u-psud.fr}
\htmladdnormallink{Daniel Taupin}{mailto:taupin@lps.u-psud.fr}}
%
%\endinput

%
% (La)TeX related URLs
%
\newcommand{\CSEP}{\index{Computer Science Education Project!CSEP}%
\htmladdnormallinkfoot{Computer Science Education Project}%
{http://csep1.phy.ornl.gov/csep.html}} 
%
\newcommand{\CBLU}{\index{Computer Based Learning Unit!CBLU}%
\htmladdnormallinkfoot{Computer Based Learning Unit}%
{http://cbl.leeds.ac.uk/\~{}www/home.html}}
%
\newcommand{\CERN}{\index{CERN}%
\htmladdnormallink{CERN}{http://wwwcn.cern.ch/Welcome.html}} 
%
\newcommand{\KrisRose}{\index{Kristoffer Rose}%\Email{krisrose@brics.dk}
\htmladdnormallink{Kristoffer Rose}{http://www.brics.dk/\~{}krisrose/}}
%
\newcommand{\XypicDK}{\index{Xy-pic@\protect\Xy-pic graphics package}%
\index{Xy-pic@\protect\Xy-pic graphics package!home page}%
\htmladdnormallink{Xy-pic}{http://www.brics.dk/\~{}krisrose/Xy-pic.html}}
\newcommand{\XypicAUS}{\index{Xy-pic@\protect\Xy-pic graphics package!home page, down under}%
\htmladdnormallink{Ross Moore}{http://www.mpce.mq.edu.au/\~{}ross/Xy-pic.html}}
%
\newcommand{\LiPS}{\index{LiPS Design Team}%
\htmladdnormallink{LiPS Design Team}{http://www-jb.cs.uni-sb.de/LiPS/node2.html}}
\newcommand{\FIDarmstadt}{\index{Fachbereich Informatik, Darmstadt}%
\htmladdnormallink{Fachbereich Informatik}{http://www.informatik.tu-darmstadt.de/}}
\newcommand{\Darmstadt}{\index{Darmstadt!Fachbereich Informatik}%
\htmladdnormallink{Darmstadt}{http://www.tu-darmstadt.de/Welcome.de.html}}
%
\newcommand{\DANTE}{\index{DANTE}%
\htmladdnormallink{DANTE e.V.}{http://www.dante.de/}}
\newcommand{\Praesidium}{\index{DANTE!Praesidium}%
\htmladdnormallink{Praesidium}{http://www.dante.de/dante/Organe.html}}
\newcommand{\LaTeXiii}{\index{LaTeX@\LaTeX!LaTeX3@\LaTeX{3}}%
\htmladdnormallink{\LaTeX{3}}{http://www.tex.ac.uk/CTAN/latex/latex3}}
%
\newcommand{\Engberg}{\index{Uffe Engberg}%
\htmladdnormallinkfoot{Uffe Engberg}{http://www.brics.dk/\~{}engberg}} 

\newcommand{\texdevINDIC}{\texdev\indic}
%\endinput


%% earlier contributions from...
\newcommand{\AndrewCole}{\index{Computer Based Learning Unit!Andrew Cole}%
\htmladdnormallink{Andrew Cole}{http://www.cbl.leeds.ac.uk/ajcole/personal.html}}
%
\newcommand{\AnaPaiva}{\index{Computer Based Learning Unit!Ana Maria Paiva}%
\htmladdnormallink{Ana Maria Paiva}{http://www.cbl.leeds.ac.uk/amp/personal.html}}
%
\newcommand{\RodWilliams}{\index{Computer Based Learning Unit!Roderick Williams}%
\htmladdnormallink{Roderick Williams}{http://www.cbl.leeds.ac.uk/rodw/personal.html}}
%
\newcommand{\JamilSawar}{\index{Computer Based Learning Unit!Jamil Sawar}%
\htmladdnormallink{Jamil Sawar}{http://www.cbl.leeds.ac.uk/sawar/personal.html}}

\newcommand{\RobertThau}{\index{Robert S. Thau}%
\htmladdnormallink{Robert S. Thau}{mailto: rst@edu.mit.ai}}

\endinput




%
\newcommand{\RobertThau}{\index{Robert S. Thau}%
\htmladdnormallink{Robert S. Thau}{mailto: rst@edu.mit.ai}}


%
% Ian Foster \Email{itf@mcs.anl.gov}
% Bob Olson \Email{olson@mcs.anl.gov}
% Verena Umar \Email{verena@edu.vanderbilt.cas.compsci}
% Axel Belinfante \Email{Axel.Belinfante@cs.utwente.nl}
% Todd Little \Email{little@com.dec.enet.nuts2u}
% Franz Vojik \Email{vojik@de.tu-muenchen.informatik}
% Eric Carroll \Email{eric@ca.utoronto.utcc.enfm}
% Roderick Williams \Email{rodw@cbl.leeds.ac.uk}
% Robert Cailliau \Email{cailliau@cernnext.cern.ch}
% Toni Lantunen at CERN
% Ana Maria Paiva, Jamil Sawar, Andrew Cole at CBLU Leeds
% Phillip Conrad (Perfect Byte, Inc.)
% L. Peter Deutsch (

%
\begin{htmlonly}
\documentclass[dvips,a4paper]{article}
\usepackage{html,htmllist,makeidx}

%this font is used in  features.tex
\font\wncyr = wncyr10


\begin{htmlonly}
 \usepackage[dvips]{graphicx}
 \usepackage[dvips]{color}
 \usepackage[dvips,leftbars]{changebar}
 \newcommand{\FoilTeX}{\env{FoilTeX}}
%
 \def\url#1{\htmladdnormallink{#1}{#1}}
 \def\Email#1{\htmladdnormallink{<#1>}{mailto:#1}}
 \def\path#1{\texttt{#1}}
% \def\urldef#1#2#3{\def#1{#2{#3}}}
 \def\onlinedoc{\url{http://www-dsed.llnl.gov/files/programs/unix/latex2html/manual/}}
 \def\patches{\url{http://www-dsed.llnl.gov/files/programs/unix/latex2html/}}
 \def\sourceA{%
  \url{http://www-dsed.llnl.gov/files/programs/unix/latex2html/sources/}}
% \def\sourceB{\url{ftp://ftp.mpn.com/pub/nikos/latex2html-98.1.tar.gz}}
 \def\sourceB{\CTANtug{\CTANA}}
 \def\sourceC{\url{http://ftp.rzg.mpg.de/pub/software/latex2html/sources/}}
 \def\CVSsite{\url{http://cdc-server.cdc.informatik.tu-darmstadt.de/\~{}latex2html/}}
 \def\CVSrepos{\url{http://cdc-server.cdc.informatik.tu-darmstadt.de/\~{}latex2html/user/}}
 \def\CVSlatest{\url{http://cdc-server.cdc.informatik.tu-darmstadt.de/\~{}latex2html/l2h-latest.tar.gz}}
 \newcommand{\CTANtug}[1]{\path{http://ctan.tug.org/ctan/#1}}
 \def\CTANA{tex-archive/support/latex2html}
 \def\tugURL{\url{http://www.tug.org}}
 \def\danteURL{\url{http://www.dante.de}}
 \def\ListURL{\url{http://www.rosat.mpe-garching.mpg.de/mailing-lists/LaTeX2HTML/}}
%
 \def\glossary#1{\index{#1@\texttt{#1} \label{III#1}\htmlref{(G)}{GGG#1}}}
 \def\Glossary#1#2{\index{#1@{#2} \label{III#1}\htmlref{(G)}{GGG#1}}}
%
% hack to suppress changebar entries read from .aux file  !!! bug in changebar.perl !!! 
 \makeatletter
 \def\cb@barpoint#1#2#3{}
 \makeatother
%
\newcommand{\sameas}[1]{\textcolor{red}{Same as setting: #1}}
\newcommand{\onlinedocRM}{\url{http://www-math.mpce.mq.edu.au/\~{}ross/latex/manual/manual.html}}
\newcommand{\EXcolors}{\url{http://www-math.mpce.mq.edu.au/\~{}ross/latex/crayola/crayola.html}}
%
\newcommand{\Lc}[1]{\texttt{\char92#1}} % LaTeX command
\newcommand{\Tc}[1]{\texttt{\char92#1}} % TeX command
\newcommand{\Cs}[1]{ \texttt{-#1} }   % command-line switch
\newcommand{\Ve}[1]{\index{#1@\texttt{#1}}\texttt{#1}} % version extension
\newcommand{\gsl}[1]{#1}
\newcommand{\indexentry}[2]{\item #1 #2}

%
\internal{}%
\internal{O}%
\internal{S}%
\internal{E}%
\internal{M}%
\internal{H}%
\internal{P}%

\newcommand{\texdev}[1]{\htmladdnormallink
 {\texdevURL/#1}{\texdevURL/#1}}
\newcommand{\ctan}{http://ctan.tug.org/ctan}
\newcommand{\ctanTUG}{\htmladdnormallinkfoot
 {TUG's searchable CTAN site}{\ctan}}
\newcommand{\ctanURL}[1]{\htmladdnormallink
 {CTAN:\texttt{.../#1}}{\ctan/tex-archive/#1}}
\newcommand{\ctanURLbr}[1]{\htmladdnormallink
 {CTAN:\texttt{.../#1}}{\ctan/tex-archive/#1}}
\newcommand{\ctanTUGurl}[2]{\ctanURL{#2}}
\newcommand{\indichtml}{l2h/indic/IndicHTML/}

\end{htmlonly}

\newcommand{\texdevURL}{http://www-texdev.mpce.mq.edu.au/}
\newcommand{\Unicode}{\htmladdnormallinkfoot
 {Unicode}{http://www.unicode.org/}}
\newcommand{\IndicHTML}{\htmladdnormallinkfoot
 {Indic\TeX/HTML}{\texdevURL{\indichtml}}}

%begin{latexonly}
\usepackage{array}

\newcommand{\Cs}[1]{{\upshape`\,\texttt{-#1}\,'}}
\newcommand{\Ve}[1]{\index{#1@\texttt{#1}}{\upshape`\,\texttt{#1}\,'}}
\newcommand{\Lc}[1]{{\upshape\ttfamily\char92#1}}
\newcommand{\Tc}[1]{{\upshape\ttfamily\char92#1}}

\newcommand{\ctanTUG}[1]{TUG's searchable CTAN site\footnote{\ctanTUGurl}}
\newcommand{\ctanURL}{CTAN: \penalty-200\ctanurl}
\newcommand{\ctanURLbr}{CTAN: \newline\ctanurl}
\urldef\texdev\url{http://www-texdev.mpce.mq.edu.au/}
\urldef\ctanurl\url{.../tex-archive/}
\urldef\ctanTUGurl\url{http://ctan.tug.org/ctan/tex-archive/}
\urldef\indichtml\url{l2h/indic/IndicHTML/}

\def\mathsmiley{\smiley}
%end{latexonly}

% thanks to \KrisRose for this macro:
\def\smiley{\hbox{\rlap{$\bigcirc$}\kern1.3pt$\scriptstyle\ddot\smile$}}

\begin{imagesonly}
\def\mathsmiley{\smiley}
\end{imagesonly}


\newcommand{\Lcs}[1]{{\upshape\ttfamily\char92#1}}

\renewcommand{\thefootnote}{\arabic{footnote}}

%\newcommand{\godown}[1]{{\htmlref
% {\htmladdimg[left BORDER=0]{../psfiles/dn.gif}}{#1}}}
\newcommand{\godown}[1]{}

%\newcommand{\goback}[1]{{\htmlref
% {\htmladdimg[left BORDER=0]{../psfiles/up.gif}}{#1}}}
\newcommand{\goback}[1]{}


\newcommand{\WiiiC}{\htmladdnormallinkfoot{World Wide Web Consortium}{http://www.w3c.org/}}
\newcommand{\latextohtmlNG}{\latextohtml-NG}
\newcommand{\maillist}{\htmladdnormallinkfoot{\latextohtml{} mailing list}%
{http://cbl.leeds.ac.uk/nikos/tex2html/doc/mail/mail.html}}
%
\newcommand{\HTMLiii}{\textup{\texttt{HTML} 3.2}}%
\newcommand{\HTMLIII}{\htmladdnormallink{\HTMLiii}%
{http://www.w3.org/TR/REC-html32.html}}%

\newcommand{\HTMLiv}{\textup{\texttt{HTML} 4.0}}%
\newcommand{\HTMLIV}{\htmladdnormallink{\HTMLiv}%
{http://www.w3.org/TR/REC-html40/}}%

%\newcommand{\latextohtml}{\textup{\LaTeX 2{\ttfamily HTML}}}%
\newcommand{\Perl}{\htmlref{\textsl{Perl}}{GGGPerl}}%  
\newcommand{\PS}{\htmlref{\textup{Post\-Script}}{GGGpostscript}\Glossary{PostScript}{PostScript}{}}%
\newcommand{\MF}{\htmlref{\textsl{Metafont}}{GGGMetafont}\Glossary{Metafont}{\textsl{Metafont}}{}}%
\newcommand{\fn}[1]{\htmlref{\texttt{#1}}{GGG#1}\glossary{#1}}%  file names, with link to glossary
\newcommand{\gn}[1]{\texttt{#1}\label{GGG#1}\htmlref{\^{}}{III#1}}%  file names, labelled within glossary
%\newcommand{\appl}[1]{\htmlref{\textsl{#1}}{GGG#1}\Glossary{#1}{\textsl{#1}}{}}%  application software names
\newcommand{\appl}[1]{\htmlref{\textsl{#1}}{GGG#1}\Glossary{#1}{\gsl{#1}}{}}%  application software names
%
\newcommand{\env}[1]{{\upshape\sffamily #1}}%  LaTeX environment and package names
\newcommand{\HTMLtag}[1]{\path{<#1>}}%  HTML tag
\newcommand{\Meta}[1]{\texttt{\upshape<\textit{#1}>}}%  Meta string

%
% developer names and addresses:
%
\newcommand{\NikosDrakos}{\index{Nikos Drakos}%\Email{nikos@cbl.leeds.ac.uk}
\htmladdnormallink{Nikos Drakos}{http://www.cbl.leeds.ac.uk/nikos/personal.html}}
%
\newcommand{\RossMoore}{\index{Ross Moore}%\Email{ross@mpce.mq.edu.au}
\htmladdnormallink{Ross Moore}{http://www.mpce.mq.edu.au/\~{}ross/}}
\newcommand{\Macquarie}{\htmladdnormallink
{Macquarie University}{http://www.mq.edu.au/}}
%
\newcommand{\Hennecke}{\index{Marcus Hennecke}%\Email{hennecke@dbag.ulm.DaimlerBenz.COM}
\htmladdnormallink{Marcus Hennecke}{http://www.crc.ricoh.com/\~{}marcush/}}
%
\newcommand{\Noworolski}{\index{Mark Noworolski}%\Email{jmn@eecs.berkeley.edu}
\htmladdnormallink{Mark Noworolski}{http://www-power.eecs.berkeley.edu/\~{}jmn/}}
%
\newcommand{\Isani}{\index{Sidik Isani}%\Email{isani@cfht.hawaii.edu}
\htmladdnormallink{Sidik Isani}{http://www.cfht.hawaii.edu/\~{}isani/si.html}}
%
\newcommand{\Goossens}{\index{Michel Goossens}%\Email{goossens@cern.ch}
\htmladdnormallink{Michel Goossens}{http://wwwcn1.cern.ch/\~{}goossens/}}
%
\newcommand{\Wilck}{\index{Martin Wilck}%\Email{martin@tropos.de}
\htmladdnormallink{Martin Wilck}{http://www.tropos.de/personal/wilck.html}}
%
\newcommand{\PatrickDaly}{\index{Patrick Daly}%\Email{daly@linmpi.dnet.gwdg.de}
\htmladdnormallink{Patrick Daly}{mailto:daly@linmpi.dnet.gwdg.de}}
%
\newcommand{\HerbSwan}{\index{Herb Swan}%\Email{herb.swan@perc.Arco.COM}
%\htmladdnormallink{Herb Swan}{mailto:herb.swan@perc.Arco.COM}}
\htmladdnormallink{Herb Swan}{mailto:lanhws@expl.aai.arco.com}}
%
\newcommand{\Lippmann}{\index{Jens Lippmann}%\Email{lippmann@cdc.informatik.tu-darmstadt.de}
\htmladdnormallink{Jens Lippmann}{http://www-jb.cs.uni-sb.de/\~{}www/people/lippmann}}
%
\newcommand{\Rouchal}{\index{Marek Rouchal}%\Email{marek@hl.siemens.de}
\htmladdnormallink{Marek Rouchal}{mailto:marek@hl.siemens.de}}
%
\newcommand{\Bohnet}{\index{Achim Bohnet}%\Email{ach@rosat.mpe-garching.mpg.de}
\htmladdnormallink{Achim Bohnet}{mailto:ach@rosat.mpe-garching.mpg.de}}
%
\newcommand{\Nelson}{\index{Scott Nelson}%\Email{nelson18@llnl.gov}
\htmladdnormallink{Scott Nelson}{mailto:nelson18@llnl.gov}}
%
\newcommand{\AxelRamge}{\index{Axel Ramge}%\Email{axel@ramge.de}
\htmladdnormallink{Axel Ramge}{mailto:axel@ramge.de}}
%
\newcommand{\Popineau}{\index{Fabrice Popineau}%\Email{popineau@esemetz.ese-metz.fr}
\htmladdnormallink{Fabrice Popineau}{mailto:popineau@esemetz.ese-metz.fr}}
%
\newcommand{\Wortmann}{\index{Uli Wortmann}%\Email{uli12@bonk.ethz.ch}
\htmladdnormallink{Uli Wortmann}{mailto:uli12@bonk.ethz.ch}}
%
\newcommand{\Veytsman}{\index{Boris Veytsman}%\Email{boris@plmsc.psu.edu}
\htmladdnormallink{Boris Veytsman}{mailto:boris@plmsc.psu.edu}}
%
\newcommand{\Taupin}{\index{Daniel Taupin}%\Email{taupin@lps.u-psud.fr}
\htmladdnormallink{Daniel Taupin}{mailto:taupin@lps.u-psud.fr}}
%
%\endinput

%
% (La)TeX related URLs
%
\newcommand{\CSEP}{\index{Computer Science Education Project!CSEP}%
\htmladdnormallinkfoot{Computer Science Education Project}%
{http://csep1.phy.ornl.gov/csep.html}} 
%
\newcommand{\CBLU}{\index{Computer Based Learning Unit!CBLU}%
\htmladdnormallinkfoot{Computer Based Learning Unit}%
{http://cbl.leeds.ac.uk/\~{}www/home.html}}
%
\newcommand{\CERN}{\index{CERN}%
\htmladdnormallink{CERN}{http://wwwcn.cern.ch/Welcome.html}} 
%
\newcommand{\KrisRose}{\index{Kristoffer Rose}%\Email{krisrose@brics.dk}
\htmladdnormallink{Kristoffer Rose}{http://www.brics.dk/\~{}krisrose/}}
%
\newcommand{\XypicDK}{\index{Xy-pic@\protect\Xy-pic graphics package}%
\index{Xy-pic@\protect\Xy-pic graphics package!home page}%
\htmladdnormallink{Xy-pic}{http://www.brics.dk/\~{}krisrose/Xy-pic.html}}
\newcommand{\XypicAUS}{\index{Xy-pic@\protect\Xy-pic graphics package!home page, down under}%
\htmladdnormallink{Ross Moore}{http://www.mpce.mq.edu.au/\~{}ross/Xy-pic.html}}
%
\newcommand{\LiPS}{\index{LiPS Design Team}%
\htmladdnormallink{LiPS Design Team}{http://www-jb.cs.uni-sb.de/LiPS/node2.html}}
\newcommand{\FIDarmstadt}{\index{Fachbereich Informatik, Darmstadt}%
\htmladdnormallink{Fachbereich Informatik}{http://www.informatik.tu-darmstadt.de/}}
\newcommand{\Darmstadt}{\index{Darmstadt!Fachbereich Informatik}%
\htmladdnormallink{Darmstadt}{http://www.tu-darmstadt.de/Welcome.de.html}}
%
\newcommand{\DANTE}{\index{DANTE}%
\htmladdnormallink{DANTE e.V.}{http://www.dante.de/}}
\newcommand{\Praesidium}{\index{DANTE!Praesidium}%
\htmladdnormallink{Praesidium}{http://www.dante.de/dante/Organe.html}}
\newcommand{\LaTeXiii}{\index{LaTeX@\LaTeX!LaTeX3@\LaTeX{3}}%
\htmladdnormallink{\LaTeX{3}}{http://www.tex.ac.uk/CTAN/latex/latex3}}
%
\newcommand{\Engberg}{\index{Uffe Engberg}%
\htmladdnormallinkfoot{Uffe Engberg}{http://www.brics.dk/\~{}engberg}} 

\newcommand{\texdevINDIC}{\texdev\indic}
%\endinput


%% earlier contributions from...
\newcommand{\AndrewCole}{\index{Computer Based Learning Unit!Andrew Cole}%
\htmladdnormallink{Andrew Cole}{http://www.cbl.leeds.ac.uk/ajcole/personal.html}}
%
\newcommand{\AnaPaiva}{\index{Computer Based Learning Unit!Ana Maria Paiva}%
\htmladdnormallink{Ana Maria Paiva}{http://www.cbl.leeds.ac.uk/amp/personal.html}}
%
\newcommand{\RodWilliams}{\index{Computer Based Learning Unit!Roderick Williams}%
\htmladdnormallink{Roderick Williams}{http://www.cbl.leeds.ac.uk/rodw/personal.html}}
%
\newcommand{\JamilSawar}{\index{Computer Based Learning Unit!Jamil Sawar}%
\htmladdnormallink{Jamil Sawar}{http://www.cbl.leeds.ac.uk/sawar/personal.html}}

\newcommand{\RobertThau}{\index{Robert S. Thau}%
\htmladdnormallink{Robert S. Thau}{mailto: rst@edu.mit.ai}}

\endinput




%
\newcommand{\RobertThau}{\index{Robert S. Thau}%
\htmladdnormallink{Robert S. Thau}{mailto: rst@edu.mit.ai}}


%
% Ian Foster \Email{itf@mcs.anl.gov}
% Bob Olson \Email{olson@mcs.anl.gov}
% Verena Umar \Email{verena@edu.vanderbilt.cas.compsci}
% Axel Belinfante \Email{Axel.Belinfante@cs.utwente.nl}
% Todd Little \Email{little@com.dec.enet.nuts2u}
% Franz Vojik \Email{vojik@de.tu-muenchen.informatik}
% Eric Carroll \Email{eric@ca.utoronto.utcc.enfm}
% Roderick Williams \Email{rodw@cbl.leeds.ac.uk}
% Robert Cailliau \Email{cailliau@cernnext.cern.ch}
% Toni Lantunen at CERN
% Ana Maria Paiva, Jamil Sawar, Andrew Cole at CBLU Leeds
% Phillip Conrad (Perfect Byte, Inc.)
% L. Peter Deutsch (

%
\begin{htmlonly}
\documentclass[dvips,a4paper]{article}
\usepackage{html,htmllist,makeidx}
\input{manhtml.tex}
%
\input{support.ptr}             % Input counters and section
\end{htmlonly}
%\internal{}%
%\internal{M}%
%\internal{H}%
%\internal{F}%
\startdocument
%
%\section{Installation and Further Support\label{sec:sup}\index{install}}%
\subsection[center]{Getting \protect\latextohtml}
%\section[center]{Getting \protect\latextohtml}
\tableofchildlinks*
\htmlrule
\index{source code}%
\index{source code!from CTAN}%
\index{CTAN}%
\cbversion{96.1}\begin{changebar}%
One way \latextohtml may be obtained is through one of the three
\htmladdnormallink{Comprehensive \TeX{} Archive Network (CTAN)}%
{http://jasper.ora.com/ctan.html} sites nearest you.  They are
located in the United States
\htmladdnormallink{\path{<ftp.shsu.edu>}}{ftp://ftp.shsu.edu/\CTANA}\,,
the United Kingdom
\htmladdnormallink{\path{<ftp.tex.ac.uk>}}{ftp://ftp.tex.ac.uk/\CTANA}
and Germany
\htmladdnormallink{\path{<ftp.dante.de>}}{ftp://ftp.dante.de/\CTANA}\,.
In the directory \CTAN\ should be the latest version, uncompressed.

\smallskip\noindent
\cbversion{97.1}\begin{changebar}
The site at \CVSsite\ is a convenient alternative for European users.
This is connected to the \htmlref{developer's repository}{cvsrepos},
so should always have the most recent release.
\end{changebar}

\index{source code!home site}%
\smallskip\noindent
Alternatively, a compressed \fn{tar} file of the source and related files
may be obtained via \appl{anonymous ftp} to \sourceA\,.

\smallskip\noindent
Two other \appl{ftp}-sites are \sourceB\ and \sourceC\,.

\index{source code!using Archie@using \textsl{Archie}}%
\index{source code!using FTP search@using \textsl{FTP search}}%
\smallskip\noindent
Other \appl{ftp}-sites nearer to you can be found using \appl{Archie} at
\url{http://hoohoo.ncsa.uiuc.edu/archie.html} or
\url{http://www.pvv.unit.no/archie/} (faster)
or more recent Web-searching tools such as \appl{FTP search}
in \htmladdnormallink{Norway}{http://ftpsearch.ntnu.no/ftpsearch}.
\begin{quote}
\textbf{Warning: }%
Some \appl{ftp}-sites may not carry the latest version.
\end{quote}

\index{source code!patches}%
\smallskip\noindent
Updates and patches are posted on the \latextohtml{} server at \patches\,.
\end{changebar}

\bigskip
\index{source code!for Windows NT platform}%
\noindent
\cbversion{97.1}\begin{changebar}%
For users of Windows NT, there is a port of \latextohtml{} obtainable
from \url{ftp://ftp.ese-metz.fr/pub/TeX/win32}\,. Obtain the release from
this site and follow the instructions in the accompanying file
\texttt{README.win32}. \html{\\}Thanks to \Popineau\ for this work.\html{\\}
In future it is planned to merge this code with the main distribution.
\end{changebar}

\bigskip
\index{source code!developer's repository}\label{cvsrepos}%
\noindent
\cbversion{97.1}\begin{changebar}%
Finally there is the \latextohtml{} developers' CVS repository, at \CVSrepos\,.\\
The files to be found here are the most up-to-date with current developments,
but they cannot be guaranteed to be fully reliable. New features may be
still under development and not yet sufficiently tested for release.

\begin{quote}
\textbf{Warning: }Use the files from this site at your own risk.
\end{quote}%
\end{changebar}

\htmlrule\index{source code!compressed}%
\medskip\noindent
Having obtained a compressed \fn{tar} version, save it into a file
\fn{latex2html-97.1.tar.gz} say,
then extract its contents with
%begin{latexonly}
\begin{small}
%end{latexonly}
\begin{verbatim}
% gzip -d latex2html-97.1.tar.gz
% tar xvf latex2html-97.1.tar
\end{verbatim}
%begin{latexonly}
\end{small}
%end{latexonly}
\index{source code!listing}%

\noindent
You should then have the following:
\begin{itemize}
\item \fn{README} file;
\begin{htmlonly}
\item \strikeout{\fn{Changes} file} (no longer supplied);
\end{htmlonly}
\item \fn{latex2html} \Perl{} script;
\item \fn{texexpand} \Perl{} script\footnote{Initially written
by \RobertThau\ and substantially modified by \Lippman.};
\item \fn{latex2html.config} configuration file;
\item \fn{install-test} \Perl{} script, for installation and testing;
\item \fn{dot.latex2html-init} sample initialisation file;
%
\cbversion{97.1}\begin{changebar}%
\item \fn{texinputs/ } subdirectory, containing various
\LaTeX{} style-files;
\item \fn{versions/ } subdirectory, containing code for specific
\texttt{HTML} versions;
\item \fn{makemap} \Perl{} script;
\item \fn{example/ } subdirectory, containing the segmentation example,
described in detail in
\hyperref{a later section}{Section~}{}{Segmentation};
\item \fn{.dvipsrc} file;%
\begin{htmlonly}
\item \strikeout{\fn{pstogif} \Perl{} script} (no longer supplied);
\end{htmlonly}
\end{changebar}
\cbversion{97.1}\begin{changebar}%
\item \fn{pstoimg} \Perl{} script (replaces \fn{pstogif});
\item \fn{configure-pstoimg} \Perl{} script;
\item \fn{local.pm} \Perl{} input file;
\item \fn{icons.gif/ } subdirectory, containing icons in GIF format;
\item \fn{icons.png/ } subdirectory, containing icons in PNG format;
\end{changebar}
\item \fn{docs/ } subdirectory, containing the files needed to create
a version of this manual;
\item \fn{styles/ } subdirectory, containing \Perl{} code for handling
some style-files.
\end{itemize}

\htmlrule
\tableofchildlinks

\subsubsection{Requirements}%
%\subsection{Requirements}%
\index{requirements|(}\html{\\}%
The translator makes use of several utilities all of which
are freely available on most platforms.
You may use \appl{Archie}\htmladdnormallinkfoot{\html{ }}%
{http://www.pvv.unit.no/archie/},
or other Web-searching tools such as \appl{FTP search}%
\htmladdnormallinkfoot{\html{ }}{http://ftpsearch.ntnu.no/ftpsearch},
to find the source code of any utilities you might need.

\medskip\noindent
For the best use of \latextohtml{} you want to get the latest
versions of all the utilities that it uses. (It will still work
with earlier versions, but some  special effects may not be possible.
The specific requirements are discussed below.)
%
\begin{itemize}
\item \Perl{} version 5.002, or later;
\item \appl{DBM} or \appl{NDBM}, the Unix DataBase Management system;
\item \LaTeX, meaning \LaTeXe{} at patchlevel 4, or later;
\item \fn{dvips} or \fn{dvipsk}, at version 5.58 or later;
\item \appl{Ghostscript} at version 4.02 or later;
\item the \fn{netpbm} library of graphics utilities.
\end{itemize}
\Perl{} should be compiled to use the \fn{csh} or \fn{tcsh} shell,
though \latextohtml{} can also be made to work with the \fn{bash} shell.

\medskip\htmlrule
\medskip\noindent
More specific requirements for using \latextohtml{}
depend on the kind of translation you would like to perform, as follows:
%
\begin{enumerate}
\item
\index{requirements!minimal}%
\textbf{\LaTeX{}  commands but without equations, figures, tables, etc.} \hfill
\begin{itemize}
\item
\htmladdnormallink{Perl}{ftp://ftp.uu.net/languages/perl/}
\begin{small}
(version 4.0 - RCSfile: perl.c,v - Revision: 4.0.1.8 - Date:
1993/02/05 19:39:30 - Patch level: 36).
\end{small}\html{\smallskip}

\textbf{Warning~1: }%
You really \emph{do} need \Perl{} at patch level 36 or later.\\
\strikeout{Versions of \latextohtml{} earlier than 0.7a4 work \emph{only} with
\Perl{ 4} at patch level 36.\\}
Versions of \latextohtml{} up to \textsc{v97.1} work
both with \Perl{ 4} at patch level 36 and \Perl{ 5}\,,
though some of the packages may only work with \Perl{ 5}\,.\html{\\}
\emph{No}~version
of \latextohtml{} will work  with \Perl{ 4} at earlier patch levels.\html{\\}
Future versions, from \textsc{v97.2} onwards,
may require \Perl{ 5} in order to operate at all.


\textbf{Warning~2: }\index{requirements!Unix shell}%
Various aspects of \Perl{}, which are used by \latextohtml{}, assume
certain system commands to be provided by the operating system shell.
If \fn{csh} or \fn{tcsh} is used to invoke \latextohtml{}
then everything should work properly.
\Perl{ 5.003} eliminates this requirement on the shell.

\textbf{Warning~3: }\index{requirements!special packages}%
Some of the packages which implement advanced features,
such as the \env{natbib} and \env{frames} packages,
require \Perl{ 5}\,.

\index{requirements!DataBase Management system}%
\item
\appl{DBM} or \appl{NDBM}, the Unix DataBase Management system.
\end{itemize}

\index{requirements!for full graphics}%
\index{tables!as images}\index{images!tables}%
\index{images!figures}\index{images!equations}%
\item
\textbf{\LaTeX{}  commands with equations, figures, tables, etc.} \\
As above plus \dots
%
\begin{itemize}
\item \fn{latex};
%
\index{dvips@\texttt{dvips} version}
\item
\htmladdnormallink{\texttt{dvips}}
{ftp://ftp.tex.ac.uk/pub/archive/dviware/dvips}
(version~5.516 or later) or \fn{dvipsk};
%
\index{Ghostscript@\textsl{Ghostscript} version}
\item
\fn{gs} \appl{Ghostscript} (version 2.6.1 works;
versions from 3.33 onwards are preferable, up to at least 5.01);
%
\index{image conversion!Postscript@\PS\ to GIF}%
\index{GIF!image conversion}%
\item
The \htmladdnormallink{\texttt{pbmplus}}{ftp://ftp.x.org/R5contrib/}
or \htmladdnormallink{\texttt{netpbm}}{ftp://ftp.x.org/R5contrib/} library.
Some of the filters in those libraries are used during the \PS\ to
GIF conversion.
\end{itemize}


\index{requirements!for segmentation feature}%
\index{segmentation!needs latex2e@needs \LaTeXe}%
\item
\textbf{\htmlref{Segmentation}{Segmentation} of large documents}\\
If you wish to use this feature, you will have to upgrade your
\LaTeX{} to \LaTeXe\,.
Some other hyperlinking features also require \LaTeXe\,.

\index{requirements!for transparent images}%
\index{images!transparent}%
\item
\textbf{\htmladdnormallinkfoot{Transparent inlined images}%
{http://melmac.corp.harris.com/transparent\_images.html}}\\
If you dislike the white background color of the
generated inlined images then you should get either
the \fn{netpbm} library (instead of the older \fn{pbmplus})
or install the \htmladdnormallinkfoot{\texttt{giftrans}}%
{ftp://ftp.rz.uni-karlsruhe.de/pub/net/www/tools/giftrans.c}
filter by Andreas Ley \Email{ley@rz.uni-karlsruhe.de}.
Version 1.10.2 is known to work without problems,
but later versions should also be OK.
%
\end{enumerate}

\index{requirements!without images}%

\noindent
If \appl{Ghostscript} or the \fn{pbmplus} (or \fn{netpbm}) library are not
available, it is still possible to
use the translator with the \Cs{no\_images} option.

\index{requirements!for special features}\index{special!features}%
\index{html.sty@\texttt{html.sty} style-file!needed for special features}%
\html{\\}%

If you intend to use any of the \htmlref{special features}{sec:hyp}
of the translator \latex{(see page~\pageref{sec:hyp})}
then you have to include the \fn{html.sty} file
in any \LaTeX{}  documents that use them.

\index{browser!supports images}%
\index{Mosaic@\textsl{Mosaic}|see{\htmlref{browser}{IIIbrowser}}}%
\index{NCSA Mosaic@\textsl{NCSA Mosaic}|see{\htmlref{browser}{IIIbrowser}}}%
\index{Netscape@\textsl{Netscape Navigator}|see{\htmlref{browser}{IIIbrowser}}}%
\index{browser!\textsl{NCSA Mosaic}}%
\index{browser!\textsl{Netscape Navigator}}%
\html{\\}%

Since by default the translator makes use of inlined images in the final
\texttt{HTML} output, it would be better to have a viewer
which supports the \HTMLtag{IMG} tag, such as \htmladdnormallink{\textsl{NCSA Mosaic}}%
{http://www.ncsa.uiuc.edu/SDG/Software/Mosaic/Docs/help-about.html}
or \htmladdnormallink{\textsl{Netscape Navigator}}{http://home.netscape.com}.
\cbversion{97.1}\begin{changebar}
Any browser which claims to be compatible with \HTMLiii{} should meet
this requirement.
\end{changebar}

\index{browser!character-based}%
\index{browser!character-based}\index{browser!lynx@\textsl{lynx}}%
\html{\\}%

If only a character-based browser, such as \appl{lynx}, is available,
or if you want the generated documents to be more portable,
then the translator can be used with the \Cs{ascii\_mode}
\hyperref{option}{option (see Section~}{)}{asciimode}.

\index{requirements|)}


\subsubsection{Installing \protect\latextohtml}%
%\subsection{Installing \protect\latextohtml}%
\index{installation}\html{\\}%
To install \latextohtml{} you \textbf{MUST} do the following:
%
\begin{enumerate}
\item
\textbf{Specify where \Perl{} is on your system}. \\
In each of the files \fn{latex2html}, \fn{texexpand}, \fn{pstoimg},
\fn{install-test} and \fn{makemap},
modify the first line saying where \Perl{} is on your system.

\index{installation!without Perl@without \Perl{} shell scripts}

\noindent
Some system administrators do not allow \Perl{} programs to run as shell scripts.
This means that you may not be able to run any of the above programs.
\emph{In this case change the first line in each of these programs from }
\html{\smallskip}
%begin{latexonly}
\begin{small}
%end{latexonly}
\verb|#!/usr/local/bin/perl |
%begin{latexonly}
\end{small}
%end{latexonly}
\html{\smallskip}\emph{to}:
%begin{latexonly}
\begin{small}
%end{latexonly}
\begin{verbatim}
# *-*-perl-*-*
    eval 'exec perl -S  $0 "$@"'
    if $running_under_some_shell;
\end{verbatim}
%begin{latexonly}
\end{small}
%end{latexonly}

\index{installation!change configuration}%
\item
In the file \fn{latex2html.config} give the correct path-names for
some directories (the \fn{LATEX2HTMLDIR} directory \strikeout{and the
\fn{pbmplus} or \fn{netpbm} library})
and some executables (\fn{latex}, \fn{dvips}\strikeout{, \fn{gs}}).

Choose or set up the icon server as explained in the comments.

\index{installation!change defaults}

While you're at it you may want to change some default
options in the same file.

\index{installation!check path-names}%
\index{installation!graphics utilities}%
\item
\textbf{Run \fn{install-test}\,.} \\
This \Perl{} script will make some changes in the \fn{latex2html} file
and then check whether the path-names to any external utilities
required by \fn{latex2html} are correct.
It will not actually install the external utilities.
\cbversion{97.1}\begin{changebar}
\fn{install-test} asks you whether to configure for \fn{GIF} or
\fn{PNG} image generation.
Finally it creates the file \fn{local.pm} which houses pathnames for the
external utilities determined earlier.
\end{changebar}

Don't forget to make \fn{install-test} executable
(using the \fn{chmod} command) if necessary, before using it.
You may also need to make the files \fn{pstogif},
\fn{texexpand}, \fn{configure-pstoimg} and \fn{latex2html} executable
if \fn{install-test} fails to do it for you.

\index{installation!system installation}%
\item If you didn't have done it before, copy the files to the named
\fn{LATEX2HTMLDIR} directory.
The executable script \fn{latex2html} may reside outside this directory.


%%\cbversion{97.1}\begin{changebar}
%%\textbf{Run \fn{configure-pstoimg} }\Meta{format} \\
%%The \Meta{format} should be either \Cs{gif} or \Cs{png}
%%according to the type of images that you wish to be generated by \latextohtml.
%%
%%This \Perl{} script creates a file called \fn{local.pm} which contains
%%complete paths to the various  utilities which \fn{pstoimg}
%%will use as it generates graphics. This file can be edited later if new
%%the script fails to find a particular utility, or a new version
%%becomes available. In particular a variable \fn{\$TMP} can also be set
%%to indicate a temporary directory to use while generating images.\html{\\}
%%Alternatively variables can be assigned values on the
%%command-line; run
%%\html{\smallskip}
%%%begin{latexonly}
%%\begin{small}
%%%end{latexonly}
%%\verb| configure-pstoimg -help |
%%%begin{latexonly}
%%\end{small}
%%%end{latexonly}
%%\html{\smallskip}%
%%to see the full range of options.
%%\end{changebar}
%%\html{\smallskip}%

\item
\index{installation!LaTeX packages}%
\textbf{\LaTeX\ packages:} Copy the contents of the \fn{texinputs/ }
directory to a place where they will be found by \LaTeX.
\end{enumerate}

\cbversion{97.1}\begin{changebar}
Note that you must run \fn{install-test} now (formerly you needn't).
If you want to reconfigure \latextohtml{} for \fn{GIF}/\fn{PNG} image
generation or because some of the external tools changed the location, run
\fn{install-test} again.
\end{changebar}

\medskip\htmlrule[50\% center]
\noindent
This is usually enough for the main installation, but you may also
want to do some of the following, to ensure that advanced features
of \latextohtml{} work correctly on your system:
\begin{itemize}
\index{html.sty@\texttt{html.sty} style-file!location}%
\index{installation!environment variable}%
\item
\textbf{To use the new \LaTeX{}  commands
which are defined in \fn{html.sty}:}\\
Make sure that \LaTeX{}  knows where the \fn{html.sty} file is,
either by putting it in the same place as the other style-files on your system,
or by changing your \fn{TEXINPUTS} shell environment variable,
or by copying \fn{html.sty} into the same directory as your \LaTeX{}  source file.
\index{html.sty@\texttt{html.sty} style-file!symbolic link}%
\cbversion{96.1}\begin{changebar}
\strikeout{If you are still using \LaTeX{} V2.09,
you must change the symbolic link \fn{html.sty}
from \fn{html2e.sty} to \fn{html209.sty}.}\\
\emph{Also make sure that} \fn{TEXINPUTS} \emph{includes} `\texttt{..}'
else \latextohtml{} will not work properly.
(On some systems, the command \fn{latex} is really a shell script
which sets some environment variables and calls the real \fn{latex}.
If this is so for you,
make sure that this shell script sets \fn{TEXINPUTS} properly.)
This environment variable is not to be confused with
the \latextohtml{} installation variable \fn{\$TEXINPUTS} described next.
\par

\index{installation!input-path variable}%
\item
There is an installation variable in \fn{latex2html.config}
called \fn{\$TEXINPUTS},
which tells \latextohtml{} where to look for \LaTeX{} style-files to process.
It can also affect the input-path of \LaTeX{} when called by \latextohtml,
unless the command \fn{latex} is really a script
which overwrites the \fn{\$TEXINPUTS} variable
prior to calling the real \fn{latex}.
This variable is overridden by the environment variable of the same name
if it is set.

\index{installation!fonts-path variable}%
\item
The installation variable \fn{\$PK\_GENERATION} specifies which
fonts are used in the generation of mathematical equations.  A value
of ``\texttt{0}'' causes the same fonts to be used as those for the default
printer.  Because they were designed for a printer of much greater
resolution than the screen, equations will generally appear to be
of a lower quality than is otherwise possible.  To cause \latextohtml{} to
dynamically generate fonts that are designed specifically for the
screen, you should specify a value of ``\texttt{1}'' for this variable.
If you do, then check to see whether your version of \texttt{dvips}
supports the command-line option \Cs{mode}\,.  If it does,
then also set the installation variable \fn{\$DVIPS\_MODE} to
a low resolution entry from \fn{modes.mf}, such as \fn{toshiba}.
\cbversion{96.1f}\begin{changebar}%
It may also be necessary to edit the \fn{MakeTeXPK} script,
to recognise this mode at the appropriate resolution.%
\end{changebar}
\cbversion{97.1}\begin{changebar}%
If you have \PS\ fonts available for use with \LaTeX{} and \fn{dvips}
then you can probably ignore the above complications and simply
set \fn{\$PK\_GENERATION} to ``\texttt{0}''
and \fn{\$DVIPS\_MODE} to \texttt{\char34\char34} (the empty string).
You must also make sure that \fn{gs} has the locations of the
fonts recorded in its \fn{gs\_fonts.ps} file. This should already be the case
where \appl{GS-Preview} is installed as the viewer for \texttt{.dvi}-files,
using the \PS\ fonts.
\end{changebar}

If \fn{dvips} does \emph{not} support the \Cs{mode} switch,
then leave \fn{\$DVIPS\_MODE} undefined, and verify that the
\fn{.dvipsrc} file points to the correct screen device and its
resolution.

\index{installation!filename-prefix}\label{autoprefix}%
\item
The installation variable \fn{\$AUTO\_PREFIX}
allows the filename-prefix to be automatically set
to the base filename-prefix of the document being translated.
This can be especially useful for multiple-segment documents.

\index{installation!Linux@\textsl{Linux} systems}%
\item
On certain \appl{Linux} systems it is necessary to uncomment
the line ``\texttt{use GDBM\_File}'' in the \fn{latex2html}
and \fn{install-test} scripts, to define the interface to the
data\-base-management routines.

\index{installation!makemap script@\texttt{makemap} script}%
\index{CERN!image-map server}\index{NCSA!image-map server}%
\item
The \fn{makemap} script also has a configuration variable \fn{\$SERVER},
which must be set to either \texttt{CERN} or \texttt{NCSA},
depending on the type of Web-server you are using.
\end{changebar}\par

\index{installation!initialization files}%
\index{initialization file!per user}%
\label{initfile}%
\item \textbf{To set up different initialization files:}\\
For a ``per user'' initialization file,
copy the file \fn{dot.latex2html-init} in the home directory
of any user that wants it, modify it according to her preferences and
rename it as \ \fn{.latex2html-init}. At runtime, both the
\fn{latex2html.config} file and \fn{\$HOME/.latex2html-init} file will be
loaded, but the latter will take precedence.

\index{initialization file!per directory}%

You can also set up a ``per directory'' initialization file by
copying a version of \ \fn{.latex2html-init} in each directory you
would like it to be effective. An initialization file
\path{/X/Y/Z/.latex2html-init} will take precedence over all other
initialization files if \path{/X/Y/Z} is the ``current directory'' when
\latextohtml{} is invoked.

\index{initialization file!incompatible with early versions}%
\begin{quotation}\noindent
\textbf{Warning: }%
This initialization file is incompatible with
any version of \latextohtml\ prior to \textsc{v96.1}\,.
Users must either update this file in their home directory,
or delete it altogether.
\end{quotation}

\index{installation!icons subdirectory@\texttt{icons/ } subdirectory}%
\item \label{icondir}%
\textbf{To make your own local copies of the \latextohtml{} icons:} \\
Please copy the \fn{icons/ } subdirectory to a
place under your WWW tree
where they can be served by your server.
Then modify the value of the \fn{\$ICONSERVER} variable in
\fn{latex2html.config} accordingly.
\index{installation!local icons}%
\cbversion{97.1}\begin{changebar}%
Alternatively, a local copy of the icons can be included within
the subdirectory containing your completed \texttt{HTML} documents.
This is most easily done using the \Cs{local\_icons} command-line switch,
or by setting \fn{\$LOCAL\_ICONS} to ``\texttt{1}'' in \fn{latex2html.config}
or within an initialization file, as described \htmlref{above}{initfile}.
\end{changebar}

\index{Livermore, California }%
\begin{quotation}\noindent
\textbf{Warnings: }%
If you cannot do that, bear in mind that these icons will have
to travel from Livermore, California!!!
Also note that several more icons were added in \textsc{v96.1}
that were not present in earlier versions of \latextohtml.
\end{quotation}

\index{installation!create manual}%
\index{documentation!test of installation}%
\item
\textbf{To make your own local copy of the \latextohtml{}
documentation:} \\
This will also be a good test of your installation.
% To do it run \latextohtml{} on the file \fn{docs/manual.tex}.
% You will get better results if you run \LaTeX{} first on the
% same file in order to create some auxiliary files.
%
\index{documentation!dvi version@\texttt{.dvi} version}%
\cbversion{96.1f}\begin{changebar}%
\noindent
Firstly, to obtain the \texttt{.dvi} version for printing,
from within the \fn{docs/ } directory it is sufficient to type:

%begin{latexonly}
\begin{small}
%end{latexonly}
\texttt{ make manual.dvi}
%begin{latexonly}
\end{small}
%end{latexonly}

\noindent
This initiates the following sequence of commands:
%begin{latexonly}
\begin{small}
%end{latexonly}
\begin{verbatim}
latex manual.tex
makeindex -s l2hidx.ist manual.idx
makeindex -s l2hglo.ist -o manual.gls manual.glo
latex manual.tex
latex manual.tex
\end{verbatim}
%begin{latexonly}
\end{small}
%end{latexonly}
\index{documentation!using makeindex@using \texttt{makeindex}}%
\index{documentation!index and glossary}%
...in which the two configuration files \fn{l2hidx.ist} and \fn{l2hglo.ist}
for the \fn{makeindex} program, are used to create the index and glossary respectively.
The 2nd run of \fn{latex} is needed to assimilate references, etc.
and include the index and glossary.\html{\\}%
\index{documentation!without makeindex@without \texttt{makeindex}}%
\html{\\}
(In case \fn{makeindex} is not available, a copy of its outputs \fn{manual.ind}
and \fn{manual.gls} are included in the \fn{docs/ } subdirectory,
along with \fn{manual.aux}\,.)\html{\\}
The 3rd run of \fn{latex} is needed to adjust page-numbering for the Index
and Glossary within the Table-of-Contents.

\noindent
Next, the \texttt{HTML} version is obtained by typing:

%begin{latexonly}
\begin{small}
%end{latexonly}
\texttt{make manual.html}
%begin{latexonly}
\end{small}
%end{latexonly}

\noindent
This initiates a series of calls to \latextohtml{} on the separate
segments of the manual;
the full manual is thus created as a ``segmented document''
(see \hyperref{a later section}{Section~}{}{Segmentation}).
The whole process may take quite some time,
as each segment needs to be processed at least twice,
to collect the cross-references from other segments.

\medskip
\index{documentation!requirements}\html{\\}%
\noindent
The files necessary for correct typesetting of the manual to be
found within the \fn{docs/ } subdirectory.
They are as follows:
\begin{itemize}
%
\index{html.sty@\texttt{html.sty} style-file}%
\index{documentation!style-files}%
\item
style-files:
 \fn{l2hman.sty}, \fn{html.sty}, \fn{htmllist.sty}, \fn{justify.sty},\\
  \fn{changebar.sty} and \fn{url.sty}
%
\index{documentation!input files}%
\item
inputs:
 \strikeout{\fn{changes.tex},} \fn{credits.tex}, \fn{features.tex}, \fn{hypextra.tex},\\
 \fn{licence.tex}, \fn{manhtml.tex}, \fn{manual.tex}, \fn{overview.tex},\\
 \fn{problems.tex}, \fn{support.tex} and \fn{userman.tex}
%
\index{documentation!graphics}%
\item
sub-directory:
 \fn{psfiles/ } containing \PS\ graphics
 used in the printed version of this manual
%
\index{documentation!images}%
\item
images of small curved arrows: \fn{up.gif}, \fn{dn.gif}
%
\index{documentation!filename data}%
\item
filename data:
 \fn{l2hfiles.dat}
%
\index{documentation!auxiliaries}%
\item
auxiliaries:
 \fn{manual.aux}, \fn{manual.ind}, \fn{manual.gls}
%
\end{itemize}

The last three can be derived from the others, but are included for convenience.
\par

\index{documentation!Changes section}%
\item

\textbf{To get a printed version of the `Changes' section: }\\
Due to the burgeoning size of the \fn{Changes} file with successive
revisions of \latextohtml, the `Changes' section is no longer
normally included as part of the printed version of the manual.
If you want this, then find the line \html{\smallskip}
%begin{latexonly}
\begin{small}
%end{latexonly}
\Lc{input}\verb|{changes.tex}|
%begin{latexonly}
\end{small}
%end{latexonly}
\html{\smallskip}
within \fn{docs/manual.tex}\,,
and comment-out the surrounding \env{htmlonly} environment.
Remake the manual as in the preceding item;
the repeated runs of \fn{latex} are required to adjust the index,
glossary and table-of-contents for the extra information.
Up to 20 extra pages may be added.%
\cbversion{97.1}\begin{changebar}%
The changes made for the \textsc{v97.1} release are far too extensive
to be included here. Instead they can be obtained from the
\htmlref{developer's repository}{cvsrepos}%
using the \appl{CVS} version-control software.
As yet there is no typeset version.
\end{changebar}\end{changebar}


\index{discussion group}\index{bugs!bug reports}%
\item
\textbf{To join the community of \latextohtml{} users:} \\
More information on a mailing list, discussion archives, bug reporting
forms and more is available at
\url{http://cbl.leeds.ac.uk/nikos/tex2html/doc/latex2html/latex2html.html}
\end{itemize}


\subsection[center]{Getting Support and More Information\label{support}}%
%\section[center]{Getting Support and More Information\label{support}}%
\index{support}%
\index{support!mailing list}\index{mailing list!Argonne National Labs}

A \htmladdnormallink{\latextohtml{} mailing list}%
{mailto:latex2html-request@mcs.anl.gov}
has been set up at the Argonne National Labs.
The \htmladdnormallinkfoot{\latextohtml{} mailing list archive}%
{http://cbl.leeds.ac.uk/nikos/tex2html/doc/mail/mail.html} is available.
\html{\\}
(Thanks to Ian Foster \Email{itf@mcs.anl.gov}
and Bob Olson \Email{olson@mcs.anl.gov}.)

\smallskip\noindent
To join send a message to: \Email{latex2html-request@mcs.anl.gov }
\index{mailing list!subscribe}\\
with the contents:~~\texttt{ subscribe }\latex{.}


\smallskip\noindent
To be removed from the list send a message to:
\Email{latex2html-request@mcs.anl.gov}
\index{mailing list!unsubscribe}\\
with the contents:~~\texttt{ unsubscribe }\latex{.}




             % Input counters and section
\end{htmlonly}
%\internal{}%
%\internal{M}%
%\internal{H}%
%\internal{F}%
\startdocument
%
%\section{Installation and Further Support\label{sec:sup}\index{install}}%
\subsection[center]{Getting \protect\latextohtml}
%\section[center]{Getting \protect\latextohtml}
\tableofchildlinks*
\htmlrule
\index{source code}%
\index{source code!from CTAN}%
\index{CTAN}%
\cbversion{96.1}\begin{changebar}%
One way \latextohtml may be obtained is through one of the three
\htmladdnormallink{Comprehensive \TeX{} Archive Network (CTAN)}%
{http://jasper.ora.com/ctan.html} sites nearest you.  They are
located in the United States
\htmladdnormallink{\path{<ftp.shsu.edu>}}{ftp://ftp.shsu.edu/\CTANA}\,,
the United Kingdom
\htmladdnormallink{\path{<ftp.tex.ac.uk>}}{ftp://ftp.tex.ac.uk/\CTANA}
and Germany
\htmladdnormallink{\path{<ftp.dante.de>}}{ftp://ftp.dante.de/\CTANA}\,.
In the directory \CTAN\ should be the latest version, uncompressed.

\smallskip\noindent
\cbversion{97.1}\begin{changebar}
The site at \CVSsite\ is a convenient alternative for European users.
This is connected to the \htmlref{developer's repository}{cvsrepos},
so should always have the most recent release.
\end{changebar}

\index{source code!home site}%
\smallskip\noindent
Alternatively, a compressed \fn{tar} file of the source and related files
may be obtained via \appl{anonymous ftp} to \sourceA\,.

\smallskip\noindent
Two other \appl{ftp}-sites are \sourceB\ and \sourceC\,.

\index{source code!using Archie@using \textsl{Archie}}%
\index{source code!using FTP search@using \textsl{FTP search}}%
\smallskip\noindent
Other \appl{ftp}-sites nearer to you can be found using \appl{Archie} at
\url{http://hoohoo.ncsa.uiuc.edu/archie.html} or
\url{http://www.pvv.unit.no/archie/} (faster)
or more recent Web-searching tools such as \appl{FTP search}
in \htmladdnormallink{Norway}{http://ftpsearch.ntnu.no/ftpsearch}.
\begin{quote}
\textbf{Warning: }%
Some \appl{ftp}-sites may not carry the latest version.
\end{quote}

\index{source code!patches}%
\smallskip\noindent
Updates and patches are posted on the \latextohtml{} server at \patches\,.
\end{changebar}

\bigskip
\index{source code!for Windows NT platform}%
\noindent
\cbversion{97.1}\begin{changebar}%
For users of Windows NT, there is a port of \latextohtml{} obtainable
from \url{ftp://ftp.ese-metz.fr/pub/TeX/win32}\,. Obtain the release from
this site and follow the instructions in the accompanying file
\texttt{README.win32}. \html{\\}Thanks to \Popineau\ for this work.\html{\\}
In future it is planned to merge this code with the main distribution.
\end{changebar}

\bigskip
\index{source code!developer's repository}\label{cvsrepos}%
\noindent
\cbversion{97.1}\begin{changebar}%
Finally there is the \latextohtml{} developers' CVS repository, at \CVSrepos\,.\\
The files to be found here are the most up-to-date with current developments,
but they cannot be guaranteed to be fully reliable. New features may be
still under development and not yet sufficiently tested for release.

\begin{quote}
\textbf{Warning: }Use the files from this site at your own risk.
\end{quote}%
\end{changebar}

\htmlrule\index{source code!compressed}%
\medskip\noindent
Having obtained a compressed \fn{tar} version, save it into a file
\fn{latex2html-97.1.tar.gz} say,
then extract its contents with
%begin{latexonly}
\begin{small}
%end{latexonly}
\begin{verbatim}
% gzip -d latex2html-97.1.tar.gz
% tar xvf latex2html-97.1.tar
\end{verbatim}
%begin{latexonly}
\end{small}
%end{latexonly}
\index{source code!listing}%

\noindent
You should then have the following:
\begin{itemize}
\item \fn{README} file;
\begin{htmlonly}
\item \strikeout{\fn{Changes} file} (no longer supplied);
\end{htmlonly}
\item \fn{latex2html} \Perl{} script;
\item \fn{texexpand} \Perl{} script\footnote{Initially written
by \RobertThau\ and substantially modified by \Lippman.};
\item \fn{latex2html.config} configuration file;
\item \fn{install-test} \Perl{} script, for installation and testing;
\item \fn{dot.latex2html-init} sample initialisation file;
%
\cbversion{97.1}\begin{changebar}%
\item \fn{texinputs/ } subdirectory, containing various
\LaTeX{} style-files;
\item \fn{versions/ } subdirectory, containing code for specific
\texttt{HTML} versions;
\item \fn{makemap} \Perl{} script;
\item \fn{example/ } subdirectory, containing the segmentation example,
described in detail in
\hyperref{a later section}{Section~}{}{Segmentation};
\item \fn{.dvipsrc} file;%
\begin{htmlonly}
\item \strikeout{\fn{pstogif} \Perl{} script} (no longer supplied);
\end{htmlonly}
\end{changebar}
\cbversion{97.1}\begin{changebar}%
\item \fn{pstoimg} \Perl{} script (replaces \fn{pstogif});
\item \fn{configure-pstoimg} \Perl{} script;
\item \fn{local.pm} \Perl{} input file;
\item \fn{icons.gif/ } subdirectory, containing icons in GIF format;
\item \fn{icons.png/ } subdirectory, containing icons in PNG format;
\end{changebar}
\item \fn{docs/ } subdirectory, containing the files needed to create
a version of this manual;
\item \fn{styles/ } subdirectory, containing \Perl{} code for handling
some style-files.
\end{itemize}

\htmlrule
\tableofchildlinks

\subsubsection{Requirements}%
%\subsection{Requirements}%
\index{requirements|(}\html{\\}%
The translator makes use of several utilities all of which
are freely available on most platforms.
You may use \appl{Archie}\htmladdnormallinkfoot{\html{ }}%
{http://www.pvv.unit.no/archie/},
or other Web-searching tools such as \appl{FTP search}%
\htmladdnormallinkfoot{\html{ }}{http://ftpsearch.ntnu.no/ftpsearch},
to find the source code of any utilities you might need.

\medskip\noindent
For the best use of \latextohtml{} you want to get the latest
versions of all the utilities that it uses. (It will still work
with earlier versions, but some  special effects may not be possible.
The specific requirements are discussed below.)
%
\begin{itemize}
\item \Perl{} version 5.002, or later;
\item \appl{DBM} or \appl{NDBM}, the Unix DataBase Management system;
\item \LaTeX, meaning \LaTeXe{} at patchlevel 4, or later;
\item \fn{dvips} or \fn{dvipsk}, at version 5.58 or later;
\item \appl{Ghostscript} at version 4.02 or later;
\item the \fn{netpbm} library of graphics utilities.
\end{itemize}
\Perl{} should be compiled to use the \fn{csh} or \fn{tcsh} shell,
though \latextohtml{} can also be made to work with the \fn{bash} shell.

\medskip\htmlrule
\medskip\noindent
More specific requirements for using \latextohtml{}
depend on the kind of translation you would like to perform, as follows:
%
\begin{enumerate}
\item
\index{requirements!minimal}%
\textbf{\LaTeX{}  commands but without equations, figures, tables, etc.} \hfill
\begin{itemize}
\item
\htmladdnormallink{Perl}{ftp://ftp.uu.net/languages/perl/}
\begin{small}
(version 4.0 - RCSfile: perl.c,v - Revision: 4.0.1.8 - Date:
1993/02/05 19:39:30 - Patch level: 36).
\end{small}\html{\smallskip}

\textbf{Warning~1: }%
You really \emph{do} need \Perl{} at patch level 36 or later.\\
\strikeout{Versions of \latextohtml{} earlier than 0.7a4 work \emph{only} with
\Perl{ 4} at patch level 36.\\}
Versions of \latextohtml{} up to \textsc{v97.1} work
both with \Perl{ 4} at patch level 36 and \Perl{ 5}\,,
though some of the packages may only work with \Perl{ 5}\,.\html{\\}
\emph{No}~version
of \latextohtml{} will work  with \Perl{ 4} at earlier patch levels.\html{\\}
Future versions, from \textsc{v97.2} onwards,
may require \Perl{ 5} in order to operate at all.


\textbf{Warning~2: }\index{requirements!Unix shell}%
Various aspects of \Perl{}, which are used by \latextohtml{}, assume
certain system commands to be provided by the operating system shell.
If \fn{csh} or \fn{tcsh} is used to invoke \latextohtml{}
then everything should work properly.
\Perl{ 5.003} eliminates this requirement on the shell.

\textbf{Warning~3: }\index{requirements!special packages}%
Some of the packages which implement advanced features,
such as the \env{natbib} and \env{frames} packages,
require \Perl{ 5}\,.

\index{requirements!DataBase Management system}%
\item
\appl{DBM} or \appl{NDBM}, the Unix DataBase Management system.
\end{itemize}

\index{requirements!for full graphics}%
\index{tables!as images}\index{images!tables}%
\index{images!figures}\index{images!equations}%
\item
\textbf{\LaTeX{}  commands with equations, figures, tables, etc.} \\
As above plus \dots
%
\begin{itemize}
\item \fn{latex};
%
\index{dvips@\texttt{dvips} version}
\item
\htmladdnormallink{\texttt{dvips}}
{ftp://ftp.tex.ac.uk/pub/archive/dviware/dvips}
(version~5.516 or later) or \fn{dvipsk};
%
\index{Ghostscript@\textsl{Ghostscript} version}
\item
\fn{gs} \appl{Ghostscript} (version 2.6.1 works;
versions from 3.33 onwards are preferable, up to at least 5.01);
%
\index{image conversion!Postscript@\PS\ to GIF}%
\index{GIF!image conversion}%
\item
The \htmladdnormallink{\texttt{pbmplus}}{ftp://ftp.x.org/R5contrib/}
or \htmladdnormallink{\texttt{netpbm}}{ftp://ftp.x.org/R5contrib/} library.
Some of the filters in those libraries are used during the \PS\ to
GIF conversion.
\end{itemize}


\index{requirements!for segmentation feature}%
\index{segmentation!needs latex2e@needs \LaTeXe}%
\item
\textbf{\htmlref{Segmentation}{Segmentation} of large documents}\\
If you wish to use this feature, you will have to upgrade your
\LaTeX{} to \LaTeXe\,.
Some other hyperlinking features also require \LaTeXe\,.

\index{requirements!for transparent images}%
\index{images!transparent}%
\item
\textbf{\htmladdnormallinkfoot{Transparent inlined images}%
{http://melmac.corp.harris.com/transparent\_images.html}}\\
If you dislike the white background color of the
generated inlined images then you should get either
the \fn{netpbm} library (instead of the older \fn{pbmplus})
or install the \htmladdnormallinkfoot{\texttt{giftrans}}%
{ftp://ftp.rz.uni-karlsruhe.de/pub/net/www/tools/giftrans.c}
filter by Andreas Ley \Email{ley@rz.uni-karlsruhe.de}.
Version 1.10.2 is known to work without problems,
but later versions should also be OK.
%
\end{enumerate}

\index{requirements!without images}%

\noindent
If \appl{Ghostscript} or the \fn{pbmplus} (or \fn{netpbm}) library are not
available, it is still possible to
use the translator with the \Cs{no\_images} option.

\index{requirements!for special features}\index{special!features}%
\index{html.sty@\texttt{html.sty} style-file!needed for special features}%
\html{\\}%

If you intend to use any of the \htmlref{special features}{sec:hyp}
of the translator \latex{(see page~\pageref{sec:hyp})}
then you have to include the \fn{html.sty} file
in any \LaTeX{}  documents that use them.

\index{browser!supports images}%
\index{Mosaic@\textsl{Mosaic}|see{\htmlref{browser}{IIIbrowser}}}%
\index{NCSA Mosaic@\textsl{NCSA Mosaic}|see{\htmlref{browser}{IIIbrowser}}}%
\index{Netscape@\textsl{Netscape Navigator}|see{\htmlref{browser}{IIIbrowser}}}%
\index{browser!\textsl{NCSA Mosaic}}%
\index{browser!\textsl{Netscape Navigator}}%
\html{\\}%

Since by default the translator makes use of inlined images in the final
\texttt{HTML} output, it would be better to have a viewer
which supports the \HTMLtag{IMG} tag, such as \htmladdnormallink{\textsl{NCSA Mosaic}}%
{http://www.ncsa.uiuc.edu/SDG/Software/Mosaic/Docs/help-about.html}
or \htmladdnormallink{\textsl{Netscape Navigator}}{http://home.netscape.com}.
\cbversion{97.1}\begin{changebar}
Any browser which claims to be compatible with \HTMLiii{} should meet
this requirement.
\end{changebar}

\index{browser!character-based}%
\index{browser!character-based}\index{browser!lynx@\textsl{lynx}}%
\html{\\}%

If only a character-based browser, such as \appl{lynx}, is available,
or if you want the generated documents to be more portable,
then the translator can be used with the \Cs{ascii\_mode}
\hyperref{option}{option (see Section~}{)}{asciimode}.

\index{requirements|)}


\subsubsection{Installing \protect\latextohtml}%
%\subsection{Installing \protect\latextohtml}%
\index{installation}\html{\\}%
To install \latextohtml{} you \textbf{MUST} do the following:
%
\begin{enumerate}
\item
\textbf{Specify where \Perl{} is on your system}. \\
In each of the files \fn{latex2html}, \fn{texexpand}, \fn{pstoimg},
\fn{install-test} and \fn{makemap},
modify the first line saying where \Perl{} is on your system.

\index{installation!without Perl@without \Perl{} shell scripts}

\noindent
Some system administrators do not allow \Perl{} programs to run as shell scripts.
This means that you may not be able to run any of the above programs.
\emph{In this case change the first line in each of these programs from }
\html{\smallskip}
%begin{latexonly}
\begin{small}
%end{latexonly}
\verb|#!/usr/local/bin/perl |
%begin{latexonly}
\end{small}
%end{latexonly}
\html{\smallskip}\emph{to}:
%begin{latexonly}
\begin{small}
%end{latexonly}
\begin{verbatim}
# *-*-perl-*-*
    eval 'exec perl -S  $0 "$@"'
    if $running_under_some_shell;
\end{verbatim}
%begin{latexonly}
\end{small}
%end{latexonly}

\index{installation!change configuration}%
\item
In the file \fn{latex2html.config} give the correct path-names for
some directories (the \fn{LATEX2HTMLDIR} directory \strikeout{and the
\fn{pbmplus} or \fn{netpbm} library})
and some executables (\fn{latex}, \fn{dvips}\strikeout{, \fn{gs}}).

Choose or set up the icon server as explained in the comments.

\index{installation!change defaults}

While you're at it you may want to change some default
options in the same file.

\index{installation!check path-names}%
\index{installation!graphics utilities}%
\item
\textbf{Run \fn{install-test}\,.} \\
This \Perl{} script will make some changes in the \fn{latex2html} file
and then check whether the path-names to any external utilities
required by \fn{latex2html} are correct.
It will not actually install the external utilities.
\cbversion{97.1}\begin{changebar}
\fn{install-test} asks you whether to configure for \fn{GIF} or
\fn{PNG} image generation.
Finally it creates the file \fn{local.pm} which houses pathnames for the
external utilities determined earlier.
\end{changebar}

Don't forget to make \fn{install-test} executable
(using the \fn{chmod} command) if necessary, before using it.
You may also need to make the files \fn{pstogif},
\fn{texexpand}, \fn{configure-pstoimg} and \fn{latex2html} executable
if \fn{install-test} fails to do it for you.

\index{installation!system installation}%
\item If you didn't have done it before, copy the files to the named
\fn{LATEX2HTMLDIR} directory.
The executable script \fn{latex2html} may reside outside this directory.


%%\cbversion{97.1}\begin{changebar}
%%\textbf{Run \fn{configure-pstoimg} }\Meta{format} \\
%%The \Meta{format} should be either \Cs{gif} or \Cs{png}
%%according to the type of images that you wish to be generated by \latextohtml.
%%
%%This \Perl{} script creates a file called \fn{local.pm} which contains
%%complete paths to the various  utilities which \fn{pstoimg}
%%will use as it generates graphics. This file can be edited later if new
%%the script fails to find a particular utility, or a new version
%%becomes available. In particular a variable \fn{\$TMP} can also be set
%%to indicate a temporary directory to use while generating images.\html{\\}
%%Alternatively variables can be assigned values on the
%%command-line; run
%%\html{\smallskip}
%%%begin{latexonly}
%%\begin{small}
%%%end{latexonly}
%%\verb| configure-pstoimg -help |
%%%begin{latexonly}
%%\end{small}
%%%end{latexonly}
%%\html{\smallskip}%
%%to see the full range of options.
%%\end{changebar}
%%\html{\smallskip}%

\item
\index{installation!LaTeX packages}%
\textbf{\LaTeX\ packages:} Copy the contents of the \fn{texinputs/ }
directory to a place where they will be found by \LaTeX.
\end{enumerate}

\cbversion{97.1}\begin{changebar}
Note that you must run \fn{install-test} now (formerly you needn't).
If you want to reconfigure \latextohtml{} for \fn{GIF}/\fn{PNG} image
generation or because some of the external tools changed the location, run
\fn{install-test} again.
\end{changebar}

\medskip\htmlrule[50\% center]
\noindent
This is usually enough for the main installation, but you may also
want to do some of the following, to ensure that advanced features
of \latextohtml{} work correctly on your system:
\begin{itemize}
\index{html.sty@\texttt{html.sty} style-file!location}%
\index{installation!environment variable}%
\item
\textbf{To use the new \LaTeX{}  commands
which are defined in \fn{html.sty}:}\\
Make sure that \LaTeX{}  knows where the \fn{html.sty} file is,
either by putting it in the same place as the other style-files on your system,
or by changing your \fn{TEXINPUTS} shell environment variable,
or by copying \fn{html.sty} into the same directory as your \LaTeX{}  source file.
\index{html.sty@\texttt{html.sty} style-file!symbolic link}%
\cbversion{96.1}\begin{changebar}
\strikeout{If you are still using \LaTeX{} V2.09,
you must change the symbolic link \fn{html.sty}
from \fn{html2e.sty} to \fn{html209.sty}.}\\
\emph{Also make sure that} \fn{TEXINPUTS} \emph{includes} `\texttt{..}'
else \latextohtml{} will not work properly.
(On some systems, the command \fn{latex} is really a shell script
which sets some environment variables and calls the real \fn{latex}.
If this is so for you,
make sure that this shell script sets \fn{TEXINPUTS} properly.)
This environment variable is not to be confused with
the \latextohtml{} installation variable \fn{\$TEXINPUTS} described next.
\par

\index{installation!input-path variable}%
\item
There is an installation variable in \fn{latex2html.config}
called \fn{\$TEXINPUTS},
which tells \latextohtml{} where to look for \LaTeX{} style-files to process.
It can also affect the input-path of \LaTeX{} when called by \latextohtml,
unless the command \fn{latex} is really a script
which overwrites the \fn{\$TEXINPUTS} variable
prior to calling the real \fn{latex}.
This variable is overridden by the environment variable of the same name
if it is set.

\index{installation!fonts-path variable}%
\item
The installation variable \fn{\$PK\_GENERATION} specifies which
fonts are used in the generation of mathematical equations.  A value
of ``\texttt{0}'' causes the same fonts to be used as those for the default
printer.  Because they were designed for a printer of much greater
resolution than the screen, equations will generally appear to be
of a lower quality than is otherwise possible.  To cause \latextohtml{} to
dynamically generate fonts that are designed specifically for the
screen, you should specify a value of ``\texttt{1}'' for this variable.
If you do, then check to see whether your version of \texttt{dvips}
supports the command-line option \Cs{mode}\,.  If it does,
then also set the installation variable \fn{\$DVIPS\_MODE} to
a low resolution entry from \fn{modes.mf}, such as \fn{toshiba}.
\cbversion{96.1f}\begin{changebar}%
It may also be necessary to edit the \fn{MakeTeXPK} script,
to recognise this mode at the appropriate resolution.%
\end{changebar}
\cbversion{97.1}\begin{changebar}%
If you have \PS\ fonts available for use with \LaTeX{} and \fn{dvips}
then you can probably ignore the above complications and simply
set \fn{\$PK\_GENERATION} to ``\texttt{0}''
and \fn{\$DVIPS\_MODE} to \texttt{\char34\char34} (the empty string).
You must also make sure that \fn{gs} has the locations of the
fonts recorded in its \fn{gs\_fonts.ps} file. This should already be the case
where \appl{GS-Preview} is installed as the viewer for \texttt{.dvi}-files,
using the \PS\ fonts.
\end{changebar}

If \fn{dvips} does \emph{not} support the \Cs{mode} switch,
then leave \fn{\$DVIPS\_MODE} undefined, and verify that the
\fn{.dvipsrc} file points to the correct screen device and its
resolution.

\index{installation!filename-prefix}\label{autoprefix}%
\item
The installation variable \fn{\$AUTO\_PREFIX}
allows the filename-prefix to be automatically set
to the base filename-prefix of the document being translated.
This can be especially useful for multiple-segment documents.

\index{installation!Linux@\textsl{Linux} systems}%
\item
On certain \appl{Linux} systems it is necessary to uncomment
the line ``\texttt{use GDBM\_File}'' in the \fn{latex2html}
and \fn{install-test} scripts, to define the interface to the
data\-base-management routines.

\index{installation!makemap script@\texttt{makemap} script}%
\index{CERN!image-map server}\index{NCSA!image-map server}%
\item
The \fn{makemap} script also has a configuration variable \fn{\$SERVER},
which must be set to either \texttt{CERN} or \texttt{NCSA},
depending on the type of Web-server you are using.
\end{changebar}\par

\index{installation!initialization files}%
\index{initialization file!per user}%
\label{initfile}%
\item \textbf{To set up different initialization files:}\\
For a ``per user'' initialization file,
copy the file \fn{dot.latex2html-init} in the home directory
of any user that wants it, modify it according to her preferences and
rename it as \ \fn{.latex2html-init}. At runtime, both the
\fn{latex2html.config} file and \fn{\$HOME/.latex2html-init} file will be
loaded, but the latter will take precedence.

\index{initialization file!per directory}%

You can also set up a ``per directory'' initialization file by
copying a version of \ \fn{.latex2html-init} in each directory you
would like it to be effective. An initialization file
\path{/X/Y/Z/.latex2html-init} will take precedence over all other
initialization files if \path{/X/Y/Z} is the ``current directory'' when
\latextohtml{} is invoked.

\index{initialization file!incompatible with early versions}%
\begin{quotation}\noindent
\textbf{Warning: }%
This initialization file is incompatible with
any version of \latextohtml\ prior to \textsc{v96.1}\,.
Users must either update this file in their home directory,
or delete it altogether.
\end{quotation}

\index{installation!icons subdirectory@\texttt{icons/ } subdirectory}%
\item \label{icondir}%
\textbf{To make your own local copies of the \latextohtml{} icons:} \\
Please copy the \fn{icons/ } subdirectory to a
place under your WWW tree
where they can be served by your server.
Then modify the value of the \fn{\$ICONSERVER} variable in
\fn{latex2html.config} accordingly.
\index{installation!local icons}%
\cbversion{97.1}\begin{changebar}%
Alternatively, a local copy of the icons can be included within
the subdirectory containing your completed \texttt{HTML} documents.
This is most easily done using the \Cs{local\_icons} command-line switch,
or by setting \fn{\$LOCAL\_ICONS} to ``\texttt{1}'' in \fn{latex2html.config}
or within an initialization file, as described \htmlref{above}{initfile}.
\end{changebar}

\index{Livermore, California }%
\begin{quotation}\noindent
\textbf{Warnings: }%
If you cannot do that, bear in mind that these icons will have
to travel from Livermore, California!!!
Also note that several more icons were added in \textsc{v96.1}
that were not present in earlier versions of \latextohtml.
\end{quotation}

\index{installation!create manual}%
\index{documentation!test of installation}%
\item
\textbf{To make your own local copy of the \latextohtml{}
documentation:} \\
This will also be a good test of your installation.
% To do it run \latextohtml{} on the file \fn{docs/manual.tex}.
% You will get better results if you run \LaTeX{} first on the
% same file in order to create some auxiliary files.
%
\index{documentation!dvi version@\texttt{.dvi} version}%
\cbversion{96.1f}\begin{changebar}%
\noindent
Firstly, to obtain the \texttt{.dvi} version for printing,
from within the \fn{docs/ } directory it is sufficient to type:

%begin{latexonly}
\begin{small}
%end{latexonly}
\texttt{ make manual.dvi}
%begin{latexonly}
\end{small}
%end{latexonly}

\noindent
This initiates the following sequence of commands:
%begin{latexonly}
\begin{small}
%end{latexonly}
\begin{verbatim}
latex manual.tex
makeindex -s l2hidx.ist manual.idx
makeindex -s l2hglo.ist -o manual.gls manual.glo
latex manual.tex
latex manual.tex
\end{verbatim}
%begin{latexonly}
\end{small}
%end{latexonly}
\index{documentation!using makeindex@using \texttt{makeindex}}%
\index{documentation!index and glossary}%
...in which the two configuration files \fn{l2hidx.ist} and \fn{l2hglo.ist}
for the \fn{makeindex} program, are used to create the index and glossary respectively.
The 2nd run of \fn{latex} is needed to assimilate references, etc.
and include the index and glossary.\html{\\}%
\index{documentation!without makeindex@without \texttt{makeindex}}%
\html{\\}
(In case \fn{makeindex} is not available, a copy of its outputs \fn{manual.ind}
and \fn{manual.gls} are included in the \fn{docs/ } subdirectory,
along with \fn{manual.aux}\,.)\html{\\}
The 3rd run of \fn{latex} is needed to adjust page-numbering for the Index
and Glossary within the Table-of-Contents.

\noindent
Next, the \texttt{HTML} version is obtained by typing:

%begin{latexonly}
\begin{small}
%end{latexonly}
\texttt{make manual.html}
%begin{latexonly}
\end{small}
%end{latexonly}

\noindent
This initiates a series of calls to \latextohtml{} on the separate
segments of the manual;
the full manual is thus created as a ``segmented document''
(see \hyperref{a later section}{Section~}{}{Segmentation}).
The whole process may take quite some time,
as each segment needs to be processed at least twice,
to collect the cross-references from other segments.

\medskip
\index{documentation!requirements}\html{\\}%
\noindent
The files necessary for correct typesetting of the manual to be
found within the \fn{docs/ } subdirectory.
They are as follows:
\begin{itemize}
%
\index{html.sty@\texttt{html.sty} style-file}%
\index{documentation!style-files}%
\item
style-files:
 \fn{l2hman.sty}, \fn{html.sty}, \fn{htmllist.sty}, \fn{justify.sty},\\
  \fn{changebar.sty} and \fn{url.sty}
%
\index{documentation!input files}%
\item
inputs:
 \strikeout{\fn{changes.tex},} \fn{credits.tex}, \fn{features.tex}, \fn{hypextra.tex},\\
 \fn{licence.tex}, \fn{manhtml.tex}, \fn{manual.tex}, \fn{overview.tex},\\
 \fn{problems.tex}, \fn{support.tex} and \fn{userman.tex}
%
\index{documentation!graphics}%
\item
sub-directory:
 \fn{psfiles/ } containing \PS\ graphics
 used in the printed version of this manual
%
\index{documentation!images}%
\item
images of small curved arrows: \fn{up.gif}, \fn{dn.gif}
%
\index{documentation!filename data}%
\item
filename data:
 \fn{l2hfiles.dat}
%
\index{documentation!auxiliaries}%
\item
auxiliaries:
 \fn{manual.aux}, \fn{manual.ind}, \fn{manual.gls}
%
\end{itemize}

The last three can be derived from the others, but are included for convenience.
\par

\index{documentation!Changes section}%
\item

\textbf{To get a printed version of the `Changes' section: }\\
Due to the burgeoning size of the \fn{Changes} file with successive
revisions of \latextohtml, the `Changes' section is no longer
normally included as part of the printed version of the manual.
If you want this, then find the line \html{\smallskip}
%begin{latexonly}
\begin{small}
%end{latexonly}
\Lc{input}\verb|{changes.tex}|
%begin{latexonly}
\end{small}
%end{latexonly}
\html{\smallskip}
within \fn{docs/manual.tex}\,,
and comment-out the surrounding \env{htmlonly} environment.
Remake the manual as in the preceding item;
the repeated runs of \fn{latex} are required to adjust the index,
glossary and table-of-contents for the extra information.
Up to 20 extra pages may be added.%
\cbversion{97.1}\begin{changebar}%
The changes made for the \textsc{v97.1} release are far too extensive
to be included here. Instead they can be obtained from the
\htmlref{developer's repository}{cvsrepos}%
using the \appl{CVS} version-control software.
As yet there is no typeset version.
\end{changebar}\end{changebar}


\index{discussion group}\index{bugs!bug reports}%
\item
\textbf{To join the community of \latextohtml{} users:} \\
More information on a mailing list, discussion archives, bug reporting
forms and more is available at
\url{http://cbl.leeds.ac.uk/nikos/tex2html/doc/latex2html/latex2html.html}
\end{itemize}


\subsection[center]{Getting Support and More Information\label{support}}%
%\section[center]{Getting Support and More Information\label{support}}%
\index{support}%
\index{support!mailing list}\index{mailing list!Argonne National Labs}

A \htmladdnormallink{\latextohtml{} mailing list}%
{mailto:latex2html-request@mcs.anl.gov}
has been set up at the Argonne National Labs.
The \htmladdnormallinkfoot{\latextohtml{} mailing list archive}%
{http://cbl.leeds.ac.uk/nikos/tex2html/doc/mail/mail.html} is available.
\html{\\}
(Thanks to Ian Foster \Email{itf@mcs.anl.gov}
and Bob Olson \Email{olson@mcs.anl.gov}.)

\smallskip\noindent
To join send a message to: \Email{latex2html-request@mcs.anl.gov }
\index{mailing list!subscribe}\\
with the contents:~~\texttt{ subscribe }\latex{.}


\smallskip\noindent
To be removed from the list send a message to:
\Email{latex2html-request@mcs.anl.gov}
\index{mailing list!unsubscribe}\\
with the contents:~~\texttt{ unsubscribe }\latex{.}




             % Input counters and section
\end{htmlonly}
%\internal{}%
%\internal{M}%
%\internal{H}%
%\internal{F}%
\startdocument
%
%\section{Installation and Further Support\label{sec:sup}\index{install}}%
\subsection[center]{Getting \protect\latextohtml}
%\section[center]{Getting \protect\latextohtml}
\tableofchildlinks*
\htmlrule
\index{source code}%
\index{source code!from CTAN}%
\index{CTAN}%
\cbversion{96.1}\begin{changebar}%
One way \latextohtml may be obtained is through one of the three
\htmladdnormallink{Comprehensive \TeX{} Archive Network (CTAN)}%
{http://jasper.ora.com/ctan.html} sites nearest you.  They are
located in the United States
\htmladdnormallink{\path{<ftp.shsu.edu>}}{ftp://ftp.shsu.edu/\CTANA}\,,
the United Kingdom
\htmladdnormallink{\path{<ftp.tex.ac.uk>}}{ftp://ftp.tex.ac.uk/\CTANA}
and Germany
\htmladdnormallink{\path{<ftp.dante.de>}}{ftp://ftp.dante.de/\CTANA}\,.
In the directory \CTAN\ should be the latest version, uncompressed.

\smallskip\noindent
\cbversion{97.1}\begin{changebar}
The site at \CVSsite\ is a convenient alternative for European users.
This is connected to the \htmlref{developer's repository}{cvsrepos},
so should always have the most recent release.
\end{changebar}

\index{source code!home site}%
\smallskip\noindent
Alternatively, a compressed \fn{tar} file of the source and related files
may be obtained via \appl{anonymous ftp} to \sourceA\,.

\smallskip\noindent
Two other \appl{ftp}-sites are \sourceB\ and \sourceC\,.

\index{source code!using Archie@using \textsl{Archie}}%
\index{source code!using FTP search@using \textsl{FTP search}}%
\smallskip\noindent
Other \appl{ftp}-sites nearer to you can be found using \appl{Archie} at
\url{http://hoohoo.ncsa.uiuc.edu/archie.html} or
\url{http://www.pvv.unit.no/archie/} (faster)
or more recent Web-searching tools such as \appl{FTP search}
in \htmladdnormallink{Norway}{http://ftpsearch.ntnu.no/ftpsearch}.
\begin{quote}
\textbf{Warning: }%
Some \appl{ftp}-sites may not carry the latest version.
\end{quote}

\index{source code!patches}%
\smallskip\noindent
Updates and patches are posted on the \latextohtml{} server at \patches\,.
\end{changebar}

\bigskip
\index{source code!for Windows NT platform}%
\noindent
\cbversion{97.1}\begin{changebar}%
For users of Windows NT, there is a port of \latextohtml{} obtainable
from \url{ftp://ftp.ese-metz.fr/pub/TeX/win32}\,. Obtain the release from
this site and follow the instructions in the accompanying file
\texttt{README.win32}. \html{\\}Thanks to \Popineau\ for this work.\html{\\}
In future it is planned to merge this code with the main distribution.
\end{changebar}

\bigskip
\index{source code!developer's repository}\label{cvsrepos}%
\noindent
\cbversion{97.1}\begin{changebar}%
Finally there is the \latextohtml{} developers' CVS repository, at \CVSrepos\,.\\
The files to be found here are the most up-to-date with current developments,
but they cannot be guaranteed to be fully reliable. New features may be
still under development and not yet sufficiently tested for release.

\begin{quote}
\textbf{Warning: }Use the files from this site at your own risk.
\end{quote}%
\end{changebar}

\htmlrule\index{source code!compressed}%
\medskip\noindent
Having obtained a compressed \fn{tar} version, save it into a file
\fn{latex2html-97.1.tar.gz} say,
then extract its contents with
%begin{latexonly}
\begin{small}
%end{latexonly}
\begin{verbatim}
% gzip -d latex2html-97.1.tar.gz
% tar xvf latex2html-97.1.tar
\end{verbatim}
%begin{latexonly}
\end{small}
%end{latexonly}
\index{source code!listing}%

\noindent
You should then have the following:
\begin{itemize}
\item \fn{README} file;
\begin{htmlonly}
\item \strikeout{\fn{Changes} file} (no longer supplied);
\end{htmlonly}
\item \fn{latex2html} \Perl{} script;
\item \fn{texexpand} \Perl{} script\footnote{Initially written
by \RobertThau\ and substantially modified by \Lippman.};
\item \fn{latex2html.config} configuration file;
\item \fn{install-test} \Perl{} script, for installation and testing;
\item \fn{dot.latex2html-init} sample initialisation file;
%
\cbversion{97.1}\begin{changebar}%
\item \fn{texinputs/ } subdirectory, containing various
\LaTeX{} style-files;
\item \fn{versions/ } subdirectory, containing code for specific
\texttt{HTML} versions;
\item \fn{makemap} \Perl{} script;
\item \fn{example/ } subdirectory, containing the segmentation example,
described in detail in
\hyperref{a later section}{Section~}{}{Segmentation};
\item \fn{.dvipsrc} file;%
\begin{htmlonly}
\item \strikeout{\fn{pstogif} \Perl{} script} (no longer supplied);
\end{htmlonly}
\end{changebar}
\cbversion{97.1}\begin{changebar}%
\item \fn{pstoimg} \Perl{} script (replaces \fn{pstogif});
\item \fn{configure-pstoimg} \Perl{} script;
\item \fn{local.pm} \Perl{} input file;
\item \fn{icons.gif/ } subdirectory, containing icons in GIF format;
\item \fn{icons.png/ } subdirectory, containing icons in PNG format;
\end{changebar}
\item \fn{docs/ } subdirectory, containing the files needed to create
a version of this manual;
\item \fn{styles/ } subdirectory, containing \Perl{} code for handling
some style-files.
\end{itemize}

\htmlrule
\tableofchildlinks

\subsubsection{Requirements}%
%\subsection{Requirements}%
\index{requirements|(}\html{\\}%
The translator makes use of several utilities all of which
are freely available on most platforms.
You may use \appl{Archie}\htmladdnormallinkfoot{\html{ }}%
{http://www.pvv.unit.no/archie/},
or other Web-searching tools such as \appl{FTP search}%
\htmladdnormallinkfoot{\html{ }}{http://ftpsearch.ntnu.no/ftpsearch},
to find the source code of any utilities you might need.

\medskip\noindent
For the best use of \latextohtml{} you want to get the latest
versions of all the utilities that it uses. (It will still work
with earlier versions, but some  special effects may not be possible.
The specific requirements are discussed below.)
%
\begin{itemize}
\item \Perl{} version 5.002, or later;
\item \appl{DBM} or \appl{NDBM}, the Unix DataBase Management system;
\item \LaTeX, meaning \LaTeXe{} at patchlevel 4, or later;
\item \fn{dvips} or \fn{dvipsk}, at version 5.58 or later;
\item \appl{Ghostscript} at version 4.02 or later;
\item the \fn{netpbm} library of graphics utilities.
\end{itemize}
\Perl{} should be compiled to use the \fn{csh} or \fn{tcsh} shell,
though \latextohtml{} can also be made to work with the \fn{bash} shell.

\medskip\htmlrule
\medskip\noindent
More specific requirements for using \latextohtml{}
depend on the kind of translation you would like to perform, as follows:
%
\begin{enumerate}
\item
\index{requirements!minimal}%
\textbf{\LaTeX{}  commands but without equations, figures, tables, etc.} \hfill
\begin{itemize}
\item
\htmladdnormallink{Perl}{ftp://ftp.uu.net/languages/perl/}
\begin{small}
(version 4.0 - RCSfile: perl.c,v - Revision: 4.0.1.8 - Date:
1993/02/05 19:39:30 - Patch level: 36).
\end{small}\html{\smallskip}

\textbf{Warning~1: }%
You really \emph{do} need \Perl{} at patch level 36 or later.\\
\strikeout{Versions of \latextohtml{} earlier than 0.7a4 work \emph{only} with
\Perl{ 4} at patch level 36.\\}
Versions of \latextohtml{} up to \textsc{v97.1} work
both with \Perl{ 4} at patch level 36 and \Perl{ 5}\,,
though some of the packages may only work with \Perl{ 5}\,.\html{\\}
\emph{No}~version
of \latextohtml{} will work  with \Perl{ 4} at earlier patch levels.\html{\\}
Future versions, from \textsc{v97.2} onwards,
may require \Perl{ 5} in order to operate at all.


\textbf{Warning~2: }\index{requirements!Unix shell}%
Various aspects of \Perl{}, which are used by \latextohtml{}, assume
certain system commands to be provided by the operating system shell.
If \fn{csh} or \fn{tcsh} is used to invoke \latextohtml{}
then everything should work properly.
\Perl{ 5.003} eliminates this requirement on the shell.

\textbf{Warning~3: }\index{requirements!special packages}%
Some of the packages which implement advanced features,
such as the \env{natbib} and \env{frames} packages,
require \Perl{ 5}\,.

\index{requirements!DataBase Management system}%
\item
\appl{DBM} or \appl{NDBM}, the Unix DataBase Management system.
\end{itemize}

\index{requirements!for full graphics}%
\index{tables!as images}\index{images!tables}%
\index{images!figures}\index{images!equations}%
\item
\textbf{\LaTeX{}  commands with equations, figures, tables, etc.} \\
As above plus \dots
%
\begin{itemize}
\item \fn{latex};
%
\index{dvips@\texttt{dvips} version}
\item
\htmladdnormallink{\texttt{dvips}}
{ftp://ftp.tex.ac.uk/pub/archive/dviware/dvips}
(version~5.516 or later) or \fn{dvipsk};
%
\index{Ghostscript@\textsl{Ghostscript} version}
\item
\fn{gs} \appl{Ghostscript} (version 2.6.1 works;
versions from 3.33 onwards are preferable, up to at least 5.01);
%
\index{image conversion!Postscript@\PS\ to GIF}%
\index{GIF!image conversion}%
\item
The \htmladdnormallink{\texttt{pbmplus}}{ftp://ftp.x.org/R5contrib/}
or \htmladdnormallink{\texttt{netpbm}}{ftp://ftp.x.org/R5contrib/} library.
Some of the filters in those libraries are used during the \PS\ to
GIF conversion.
\end{itemize}


\index{requirements!for segmentation feature}%
\index{segmentation!needs latex2e@needs \LaTeXe}%
\item
\textbf{\htmlref{Segmentation}{Segmentation} of large documents}\\
If you wish to use this feature, you will have to upgrade your
\LaTeX{} to \LaTeXe\,.
Some other hyperlinking features also require \LaTeXe\,.

\index{requirements!for transparent images}%
\index{images!transparent}%
\item
\textbf{\htmladdnormallinkfoot{Transparent inlined images}%
{http://melmac.corp.harris.com/transparent\_images.html}}\\
If you dislike the white background color of the
generated inlined images then you should get either
the \fn{netpbm} library (instead of the older \fn{pbmplus})
or install the \htmladdnormallinkfoot{\texttt{giftrans}}%
{ftp://ftp.rz.uni-karlsruhe.de/pub/net/www/tools/giftrans.c}
filter by Andreas Ley \Email{ley@rz.uni-karlsruhe.de}.
Version 1.10.2 is known to work without problems,
but later versions should also be OK.
%
\end{enumerate}

\index{requirements!without images}%

\noindent
If \appl{Ghostscript} or the \fn{pbmplus} (or \fn{netpbm}) library are not
available, it is still possible to
use the translator with the \Cs{no\_images} option.

\index{requirements!for special features}\index{special!features}%
\index{html.sty@\texttt{html.sty} style-file!needed for special features}%
\html{\\}%

If you intend to use any of the \htmlref{special features}{sec:hyp}
of the translator \latex{(see page~\pageref{sec:hyp})}
then you have to include the \fn{html.sty} file
in any \LaTeX{}  documents that use them.

\index{browser!supports images}%
\index{Mosaic@\textsl{Mosaic}|see{\htmlref{browser}{IIIbrowser}}}%
\index{NCSA Mosaic@\textsl{NCSA Mosaic}|see{\htmlref{browser}{IIIbrowser}}}%
\index{Netscape@\textsl{Netscape Navigator}|see{\htmlref{browser}{IIIbrowser}}}%
\index{browser!\textsl{NCSA Mosaic}}%
\index{browser!\textsl{Netscape Navigator}}%
\html{\\}%

Since by default the translator makes use of inlined images in the final
\texttt{HTML} output, it would be better to have a viewer
which supports the \HTMLtag{IMG} tag, such as \htmladdnormallink{\textsl{NCSA Mosaic}}%
{http://www.ncsa.uiuc.edu/SDG/Software/Mosaic/Docs/help-about.html}
or \htmladdnormallink{\textsl{Netscape Navigator}}{http://home.netscape.com}.
\cbversion{97.1}\begin{changebar}
Any browser which claims to be compatible with \HTMLiii{} should meet
this requirement.
\end{changebar}

\index{browser!character-based}%
\index{browser!character-based}\index{browser!lynx@\textsl{lynx}}%
\html{\\}%

If only a character-based browser, such as \appl{lynx}, is available,
or if you want the generated documents to be more portable,
then the translator can be used with the \Cs{ascii\_mode}
\hyperref{option}{option (see Section~}{)}{asciimode}.

\index{requirements|)}


\subsubsection{Installing \protect\latextohtml}%
%\subsection{Installing \protect\latextohtml}%
\index{installation}\html{\\}%
To install \latextohtml{} you \textbf{MUST} do the following:
%
\begin{enumerate}
\item
\textbf{Specify where \Perl{} is on your system}. \\
In each of the files \fn{latex2html}, \fn{texexpand}, \fn{pstoimg},
\fn{install-test} and \fn{makemap},
modify the first line saying where \Perl{} is on your system.

\index{installation!without Perl@without \Perl{} shell scripts}

\noindent
Some system administrators do not allow \Perl{} programs to run as shell scripts.
This means that you may not be able to run any of the above programs.
\emph{In this case change the first line in each of these programs from }
\html{\smallskip}
%begin{latexonly}
\begin{small}
%end{latexonly}
\verb|#!/usr/local/bin/perl |
%begin{latexonly}
\end{small}
%end{latexonly}
\html{\smallskip}\emph{to}:
%begin{latexonly}
\begin{small}
%end{latexonly}
\begin{verbatim}
# *-*-perl-*-*
    eval 'exec perl -S  $0 "$@"'
    if $running_under_some_shell;
\end{verbatim}
%begin{latexonly}
\end{small}
%end{latexonly}

\index{installation!change configuration}%
\item
In the file \fn{latex2html.config} give the correct path-names for
some directories (the \fn{LATEX2HTMLDIR} directory \strikeout{and the
\fn{pbmplus} or \fn{netpbm} library})
and some executables (\fn{latex}, \fn{dvips}\strikeout{, \fn{gs}}).

Choose or set up the icon server as explained in the comments.

\index{installation!change defaults}

While you're at it you may want to change some default
options in the same file.

\index{installation!check path-names}%
\index{installation!graphics utilities}%
\item
\textbf{Run \fn{install-test}\,.} \\
This \Perl{} script will make some changes in the \fn{latex2html} file
and then check whether the path-names to any external utilities
required by \fn{latex2html} are correct.
It will not actually install the external utilities.
\cbversion{97.1}\begin{changebar}
\fn{install-test} asks you whether to configure for \fn{GIF} or
\fn{PNG} image generation.
Finally it creates the file \fn{local.pm} which houses pathnames for the
external utilities determined earlier.
\end{changebar}

Don't forget to make \fn{install-test} executable
(using the \fn{chmod} command) if necessary, before using it.
You may also need to make the files \fn{pstogif},
\fn{texexpand}, \fn{configure-pstoimg} and \fn{latex2html} executable
if \fn{install-test} fails to do it for you.

\index{installation!system installation}%
\item If you didn't have done it before, copy the files to the named
\fn{LATEX2HTMLDIR} directory.
The executable script \fn{latex2html} may reside outside this directory.


%%\cbversion{97.1}\begin{changebar}
%%\textbf{Run \fn{configure-pstoimg} }\Meta{format} \\
%%The \Meta{format} should be either \Cs{gif} or \Cs{png}
%%according to the type of images that you wish to be generated by \latextohtml.
%%
%%This \Perl{} script creates a file called \fn{local.pm} which contains
%%complete paths to the various  utilities which \fn{pstoimg}
%%will use as it generates graphics. This file can be edited later if new
%%the script fails to find a particular utility, or a new version
%%becomes available. In particular a variable \fn{\$TMP} can also be set
%%to indicate a temporary directory to use while generating images.\html{\\}
%%Alternatively variables can be assigned values on the
%%command-line; run
%%\html{\smallskip}
%%%begin{latexonly}
%%\begin{small}
%%%end{latexonly}
%%\verb| configure-pstoimg -help |
%%%begin{latexonly}
%%\end{small}
%%%end{latexonly}
%%\html{\smallskip}%
%%to see the full range of options.
%%\end{changebar}
%%\html{\smallskip}%

\item
\index{installation!LaTeX packages}%
\textbf{\LaTeX\ packages:} Copy the contents of the \fn{texinputs/ }
directory to a place where they will be found by \LaTeX.
\end{enumerate}

\cbversion{97.1}\begin{changebar}
Note that you must run \fn{install-test} now (formerly you needn't).
If you want to reconfigure \latextohtml{} for \fn{GIF}/\fn{PNG} image
generation or because some of the external tools changed the location, run
\fn{install-test} again.
\end{changebar}

\medskip\htmlrule[50\% center]
\noindent
This is usually enough for the main installation, but you may also
want to do some of the following, to ensure that advanced features
of \latextohtml{} work correctly on your system:
\begin{itemize}
\index{html.sty@\texttt{html.sty} style-file!location}%
\index{installation!environment variable}%
\item
\textbf{To use the new \LaTeX{}  commands
which are defined in \fn{html.sty}:}\\
Make sure that \LaTeX{}  knows where the \fn{html.sty} file is,
either by putting it in the same place as the other style-files on your system,
or by changing your \fn{TEXINPUTS} shell environment variable,
or by copying \fn{html.sty} into the same directory as your \LaTeX{}  source file.
\index{html.sty@\texttt{html.sty} style-file!symbolic link}%
\cbversion{96.1}\begin{changebar}
\strikeout{If you are still using \LaTeX{} V2.09,
you must change the symbolic link \fn{html.sty}
from \fn{html2e.sty} to \fn{html209.sty}.}\\
\emph{Also make sure that} \fn{TEXINPUTS} \emph{includes} `\texttt{..}'
else \latextohtml{} will not work properly.
(On some systems, the command \fn{latex} is really a shell script
which sets some environment variables and calls the real \fn{latex}.
If this is so for you,
make sure that this shell script sets \fn{TEXINPUTS} properly.)
This environment variable is not to be confused with
the \latextohtml{} installation variable \fn{\$TEXINPUTS} described next.
\par

\index{installation!input-path variable}%
\item
There is an installation variable in \fn{latex2html.config}
called \fn{\$TEXINPUTS},
which tells \latextohtml{} where to look for \LaTeX{} style-files to process.
It can also affect the input-path of \LaTeX{} when called by \latextohtml,
unless the command \fn{latex} is really a script
which overwrites the \fn{\$TEXINPUTS} variable
prior to calling the real \fn{latex}.
This variable is overridden by the environment variable of the same name
if it is set.

\index{installation!fonts-path variable}%
\item
The installation variable \fn{\$PK\_GENERATION} specifies which
fonts are used in the generation of mathematical equations.  A value
of ``\texttt{0}'' causes the same fonts to be used as those for the default
printer.  Because they were designed for a printer of much greater
resolution than the screen, equations will generally appear to be
of a lower quality than is otherwise possible.  To cause \latextohtml{} to
dynamically generate fonts that are designed specifically for the
screen, you should specify a value of ``\texttt{1}'' for this variable.
If you do, then check to see whether your version of \texttt{dvips}
supports the command-line option \Cs{mode}\,.  If it does,
then also set the installation variable \fn{\$DVIPS\_MODE} to
a low resolution entry from \fn{modes.mf}, such as \fn{toshiba}.
\cbversion{96.1f}\begin{changebar}%
It may also be necessary to edit the \fn{MakeTeXPK} script,
to recognise this mode at the appropriate resolution.%
\end{changebar}
\cbversion{97.1}\begin{changebar}%
If you have \PS\ fonts available for use with \LaTeX{} and \fn{dvips}
then you can probably ignore the above complications and simply
set \fn{\$PK\_GENERATION} to ``\texttt{0}''
and \fn{\$DVIPS\_MODE} to \texttt{\char34\char34} (the empty string).
You must also make sure that \fn{gs} has the locations of the
fonts recorded in its \fn{gs\_fonts.ps} file. This should already be the case
where \appl{GS-Preview} is installed as the viewer for \texttt{.dvi}-files,
using the \PS\ fonts.
\end{changebar}

If \fn{dvips} does \emph{not} support the \Cs{mode} switch,
then leave \fn{\$DVIPS\_MODE} undefined, and verify that the
\fn{.dvipsrc} file points to the correct screen device and its
resolution.

\index{installation!filename-prefix}\label{autoprefix}%
\item
The installation variable \fn{\$AUTO\_PREFIX}
allows the filename-prefix to be automatically set
to the base filename-prefix of the document being translated.
This can be especially useful for multiple-segment documents.

\index{installation!Linux@\textsl{Linux} systems}%
\item
On certain \appl{Linux} systems it is necessary to uncomment
the line ``\texttt{use GDBM\_File}'' in the \fn{latex2html}
and \fn{install-test} scripts, to define the interface to the
data\-base-management routines.

\index{installation!makemap script@\texttt{makemap} script}%
\index{CERN!image-map server}\index{NCSA!image-map server}%
\item
The \fn{makemap} script also has a configuration variable \fn{\$SERVER},
which must be set to either \texttt{CERN} or \texttt{NCSA},
depending on the type of Web-server you are using.
\end{changebar}\par

\index{installation!initialization files}%
\index{initialization file!per user}%
\label{initfile}%
\item \textbf{To set up different initialization files:}\\
For a ``per user'' initialization file,
copy the file \fn{dot.latex2html-init} in the home directory
of any user that wants it, modify it according to her preferences and
rename it as \ \fn{.latex2html-init}. At runtime, both the
\fn{latex2html.config} file and \fn{\$HOME/.latex2html-init} file will be
loaded, but the latter will take precedence.

\index{initialization file!per directory}%

You can also set up a ``per directory'' initialization file by
copying a version of \ \fn{.latex2html-init} in each directory you
would like it to be effective. An initialization file
\path{/X/Y/Z/.latex2html-init} will take precedence over all other
initialization files if \path{/X/Y/Z} is the ``current directory'' when
\latextohtml{} is invoked.

\index{initialization file!incompatible with early versions}%
\begin{quotation}\noindent
\textbf{Warning: }%
This initialization file is incompatible with
any version of \latextohtml\ prior to \textsc{v96.1}\,.
Users must either update this file in their home directory,
or delete it altogether.
\end{quotation}

\index{installation!icons subdirectory@\texttt{icons/ } subdirectory}%
\item \label{icondir}%
\textbf{To make your own local copies of the \latextohtml{} icons:} \\
Please copy the \fn{icons/ } subdirectory to a
place under your WWW tree
where they can be served by your server.
Then modify the value of the \fn{\$ICONSERVER} variable in
\fn{latex2html.config} accordingly.
\index{installation!local icons}%
\cbversion{97.1}\begin{changebar}%
Alternatively, a local copy of the icons can be included within
the subdirectory containing your completed \texttt{HTML} documents.
This is most easily done using the \Cs{local\_icons} command-line switch,
or by setting \fn{\$LOCAL\_ICONS} to ``\texttt{1}'' in \fn{latex2html.config}
or within an initialization file, as described \htmlref{above}{initfile}.
\end{changebar}

\index{Livermore, California }%
\begin{quotation}\noindent
\textbf{Warnings: }%
If you cannot do that, bear in mind that these icons will have
to travel from Livermore, California!!!
Also note that several more icons were added in \textsc{v96.1}
that were not present in earlier versions of \latextohtml.
\end{quotation}

\index{installation!create manual}%
\index{documentation!test of installation}%
\item
\textbf{To make your own local copy of the \latextohtml{}
documentation:} \\
This will also be a good test of your installation.
% To do it run \latextohtml{} on the file \fn{docs/manual.tex}.
% You will get better results if you run \LaTeX{} first on the
% same file in order to create some auxiliary files.
%
\index{documentation!dvi version@\texttt{.dvi} version}%
\cbversion{96.1f}\begin{changebar}%
\noindent
Firstly, to obtain the \texttt{.dvi} version for printing,
from within the \fn{docs/ } directory it is sufficient to type:

%begin{latexonly}
\begin{small}
%end{latexonly}
\texttt{ make manual.dvi}
%begin{latexonly}
\end{small}
%end{latexonly}

\noindent
This initiates the following sequence of commands:
%begin{latexonly}
\begin{small}
%end{latexonly}
\begin{verbatim}
latex manual.tex
makeindex -s l2hidx.ist manual.idx
makeindex -s l2hglo.ist -o manual.gls manual.glo
latex manual.tex
latex manual.tex
\end{verbatim}
%begin{latexonly}
\end{small}
%end{latexonly}
\index{documentation!using makeindex@using \texttt{makeindex}}%
\index{documentation!index and glossary}%
...in which the two configuration files \fn{l2hidx.ist} and \fn{l2hglo.ist}
for the \fn{makeindex} program, are used to create the index and glossary respectively.
The 2nd run of \fn{latex} is needed to assimilate references, etc.
and include the index and glossary.\html{\\}%
\index{documentation!without makeindex@without \texttt{makeindex}}%
\html{\\}
(In case \fn{makeindex} is not available, a copy of its outputs \fn{manual.ind}
and \fn{manual.gls} are included in the \fn{docs/ } subdirectory,
along with \fn{manual.aux}\,.)\html{\\}
The 3rd run of \fn{latex} is needed to adjust page-numbering for the Index
and Glossary within the Table-of-Contents.

\noindent
Next, the \texttt{HTML} version is obtained by typing:

%begin{latexonly}
\begin{small}
%end{latexonly}
\texttt{make manual.html}
%begin{latexonly}
\end{small}
%end{latexonly}

\noindent
This initiates a series of calls to \latextohtml{} on the separate
segments of the manual;
the full manual is thus created as a ``segmented document''
(see \hyperref{a later section}{Section~}{}{Segmentation}).
The whole process may take quite some time,
as each segment needs to be processed at least twice,
to collect the cross-references from other segments.

\medskip
\index{documentation!requirements}\html{\\}%
\noindent
The files necessary for correct typesetting of the manual to be
found within the \fn{docs/ } subdirectory.
They are as follows:
\begin{itemize}
%
\index{html.sty@\texttt{html.sty} style-file}%
\index{documentation!style-files}%
\item
style-files:
 \fn{l2hman.sty}, \fn{html.sty}, \fn{htmllist.sty}, \fn{justify.sty},\\
  \fn{changebar.sty} and \fn{url.sty}
%
\index{documentation!input files}%
\item
inputs:
 \strikeout{\fn{changes.tex},} \fn{credits.tex}, \fn{features.tex}, \fn{hypextra.tex},\\
 \fn{licence.tex}, \fn{manhtml.tex}, \fn{manual.tex}, \fn{overview.tex},\\
 \fn{problems.tex}, \fn{support.tex} and \fn{userman.tex}
%
\index{documentation!graphics}%
\item
sub-directory:
 \fn{psfiles/ } containing \PS\ graphics
 used in the printed version of this manual
%
\index{documentation!images}%
\item
images of small curved arrows: \fn{up.gif}, \fn{dn.gif}
%
\index{documentation!filename data}%
\item
filename data:
 \fn{l2hfiles.dat}
%
\index{documentation!auxiliaries}%
\item
auxiliaries:
 \fn{manual.aux}, \fn{manual.ind}, \fn{manual.gls}
%
\end{itemize}

The last three can be derived from the others, but are included for convenience.
\par

\index{documentation!Changes section}%
\item

\textbf{To get a printed version of the `Changes' section: }\\
Due to the burgeoning size of the \fn{Changes} file with successive
revisions of \latextohtml, the `Changes' section is no longer
normally included as part of the printed version of the manual.
If you want this, then find the line \html{\smallskip}
%begin{latexonly}
\begin{small}
%end{latexonly}
\Lc{input}\verb|{changes.tex}|
%begin{latexonly}
\end{small}
%end{latexonly}
\html{\smallskip}
within \fn{docs/manual.tex}\,,
and comment-out the surrounding \env{htmlonly} environment.
Remake the manual as in the preceding item;
the repeated runs of \fn{latex} are required to adjust the index,
glossary and table-of-contents for the extra information.
Up to 20 extra pages may be added.%
\cbversion{97.1}\begin{changebar}%
The changes made for the \textsc{v97.1} release are far too extensive
to be included here. Instead they can be obtained from the
\htmlref{developer's repository}{cvsrepos}%
using the \appl{CVS} version-control software.
As yet there is no typeset version.
\end{changebar}\end{changebar}


\index{discussion group}\index{bugs!bug reports}%
\item
\textbf{To join the community of \latextohtml{} users:} \\
More information on a mailing list, discussion archives, bug reporting
forms and more is available at
\url{http://cbl.leeds.ac.uk/nikos/tex2html/doc/latex2html/latex2html.html}
\end{itemize}


\subsection[center]{Getting Support and More Information\label{support}}%
%\section[center]{Getting Support and More Information\label{support}}%
\index{support}%
\index{support!mailing list}\index{mailing list!Argonne National Labs}

A \htmladdnormallink{\latextohtml{} mailing list}%
{mailto:latex2html-request@mcs.anl.gov}
has been set up at the Argonne National Labs.
The \htmladdnormallinkfoot{\latextohtml{} mailing list archive}%
{http://cbl.leeds.ac.uk/nikos/tex2html/doc/mail/mail.html} is available.
\html{\\}
(Thanks to Ian Foster \Email{itf@mcs.anl.gov}
and Bob Olson \Email{olson@mcs.anl.gov}.)

\smallskip\noindent
To join send a message to: \Email{latex2html-request@mcs.anl.gov }
\index{mailing list!subscribe}\\
with the contents:~~\texttt{ subscribe }\latex{.}


\smallskip\noindent
To be removed from the list send a message to:
\Email{latex2html-request@mcs.anl.gov}
\index{mailing list!unsubscribe}\\
with the contents:~~\texttt{ unsubscribe }\latex{.}




             % Input counters and section
\end{htmlonly}
%\internal{}%
%\internal{M}%
%\internal{H}%
%\internal{F}%
\startdocument
%
%\section{Installation and Further Support\label{sec:sup}\index{install}}%
\subsection[center]{Getting \protect\latextohtml}
%\section[center]{Getting \protect\latextohtml}
\label{sec:sup}\index{install}
\tableofchildlinks*
\htmlrule
\index{source code}%
\index{source code!from CTAN}%
\index{CTAN}%
\cbversion{98.1}\begin{changebar}%
One way \latextohtml may be obtained is through one of the three
\htmladdnormallink{Comprehensive \TeX{} Archive Network (CTAN)}%
{\CTANtug{}} sites.
They are located at
\begin{htmllist}\htmlitemmark{GreenBall}
\item [US ]
United States:
\htmladdnormallinkfoot{\CTANtug{}}{\CTANtug{\CTANA}}\,,
\item [UK ]
United Kingdom:
\htmladdnormallinkfoot{\path{http://www.tex.ac.uk/}}%
{http://www.tex.ac.uk/\CTANA}
\item [DE ]
Germany:
\htmladdnormallinkfoot{\path{ftp://ftp.dante.de/}}%
{ftp://ftp.dante.de/\CTANA}\,.
\end{htmllist}
In the directory \htmladdnormallink{\CTANA/}{\CTANtug{\CTANA}}
should be the latest version, as a compressed archive.

\smallskip\noindent
There are also many mirrors.
To find the nearest to you, get a listing via the command:
\begin{quote}
\texttt{finger ctan@www.tug.org}
\end{quote}

\cbversion{97.1}\begin{changebar}\noindent
The site at \CVSsite\ is a convenient alternative for European users.
This is connected to the \htmlref{developer's repository}{cvsrepos},
so should always have the most recent release.
\end{changebar}

\index{source code!home site}%
\smallskip\noindent
Alternatively, a compressed \fn{tar} file of the source and related files
may be obtained via \appl{anonymous ftp} to \sourceA\,.

\smallskip\noindent
Two other \appl{ftp}-sites are:
\begin{itemize}
\item \sourceB{}
\item \sourceC{}
\end{itemize}

\index{source code!using Archie@using \textsl{Archie}}%
\index{source code!using FTP search@using \textsl{FTP search}}%
\smallskip\noindent
Other \appl{ftp}-sites nearer to you can be found using \appl{Archie} at
\url{http://hoohoo.ncsa.uiuc.edu/archie.html} or
\url{http://www.pvv.unit.no/archie/} (faster)
or more recent Web-searching tools such as \appl{FTP search}
in \htmladdnormallink{Norway}{http://ftpsearch.ntnu.no/ftpsearch}.
\begin{quote}
\textbf{Warning: }%
Some \appl{ftp}-sites may not carry the latest version.
\end{quote}

\index{source code!patches}%
\smallskip\noindent
Updates and patches are posted on the \latextohtml{} server at \patches\,.
\end{changebar}

\bigskip
\index{source code!for Windows NT platform}%
\noindent
\cbversion{97.1}\begin{changebar}%
For users of Windows NT, there is a port of \latextohtml{} obtainable
from \url{ftp://ftp.ese-metz.fr/pub/TeX/win32}\,. Obtain the release from
this site and follow the instructions in the accompanying file
\texttt{README.win32}. \html{\\}Thanks to \Popineau\ for this work.\html{\\}
In future it is planned to merge this code with the main distribution.
\end{changebar}

\bigskip
\index{source code!developer's repository}\label{cvsrepos}%
\noindent
\cbversion{97.1}\begin{changebar}%
Finally there is the \latextohtml{} developers' CVS repository, at \CVSrepos\,.\\
The files to be found here are the most up-to-date with current developments,
but they cannot be guaranteed to be fully reliable. New features may be
still under development and not yet sufficiently tested for release.
A daily updated compressed archive of the developers' work may be
downloaded from \CVSlatest\,.

\begin{quote}
\textbf{Warning: }Use the files from this site at your own risk.
\end{quote}%
\end{changebar}

\htmlrule\index{source code!compressed}%
\medskip\noindent
Having obtained a compressed \fn{tar} version, save it into a file
\fn{latex2html-98.1.tar.gz} say,
then extract its contents with
%begin{latexonly}
\begin{small}
%end{latexonly}
\begin{verbatim}
% gzip -d latex2html-98.1.tar.gz
% tar xvf latex2html-98.1.tar
\end{verbatim}
%begin{latexonly}
\end{small}
%end{latexonly}
\index{source code!listing}%

\noindent
You should then have the following:
\begin{itemize}
\item \fn{README} file;
\cbversion{98.1}\begin{changebar}%
\item \fn{Changes} index with latest changes;
\item \strikeout{\fn{Changes.detailed}} (no longer supplied);
\end{changebar}
\item \fn{latex2html} \Perl{} script;
\item \fn{texexpand} \Perl{} script\footnote{Initially written
by \RobertThau, completely rewritten by \Rouchal\ and \Lippmann.};
\item \fn{latex2html.config} configuration file;
\item \fn{install-test} \Perl{} script, for installation and testing;
\item \fn{dot.latex2html-init} sample initialisation file;
%
\cbversion{97.1}\begin{changebar}%
\item \fn{texinputs/} subdirectory, containing various
\LaTeX{} style-files;
\item \fn{versions/} subdirectory, containing code for specific
\texttt{HTML} versions;
\item \fn{makemap} \Perl{} script;
\item \fn{example/} subdirectory, containing the segmentation example,
described in detail in
\hyperref{a later section}{Section~}{}{Segmentation};
\item \fn{.dvipsrc} file;%
\begin{htmlonly}
\item \strikeout{\fn{pstogif} \Perl{} script} (no longer supplied);
\end{htmlonly}
\end{changebar}
\cbversion{97.1}\begin{changebar}%
\item \fn{pstoimg} \Perl{} script for image conversion (replaces \fn{pstogif});
\item \fn{configure-pstoimg} \Perl{} script for installation;
\item \fn{local.pm} \Perl{} input file;
\item \fn{icons.gif/} subdirectory, containing icons in GIF format;
\item \fn{icons.png/} subdirectory, containing icons in PNG format;
\item \fn{makeseg} \Perl{} script and examples to handle segmented
documents via a generated Makefile, see \verb/makeseg.tex/;
\end{changebar}
\cbversion{98.1}\begin{changebar}
\item \fn{docs/foilhtml/} contains \LaTeX{} package and \Perl{} implementation by \Veytsman,
to support \FoilTeX{} to \fn{HTML} translation;
\item \fn{IndicTeX-HTML/} package that contains \Perl{} and
 \LaTeX{} code for translating Indic\TeX{} documents (see README file);
\end{changebar}
\item \fn{docs/} subdirectory, containing the files needed to create
a version of this manual;
\item \fn{styles/} subdirectory, containing \Perl{} code for handling
some style-files;
\item \fn{tests/} contains some test documents for \latextohtml.
\end{itemize}

\htmlrule
\tableofchildlinks

\subsubsection{Requirements}%
%\subsection{Requirements}%
\index{requirements|(}\html{\\}%
The translator makes use of several utilities all of which
are freely available on most platforms.
You may use \appl{Archie}\htmladdnormallinkfoot{\html{ }}%
{http://www.pvv.unit.no/archie/},
or other Web-searching tools such as \appl{FTP search}%
\htmladdnormallinkfoot{\html{ }}{http://ftpsearch.ntnu.no/ftpsearch},
to find the source code of any utilities you might need.

\medskip\noindent
For the best use of \latextohtml{} you want to get the latest
versions of all the utilities that it uses. (It will still work
with earlier versions, but some  special effects may not be possible.
The specific requirements are discussed below.)
%
\begin{itemize}
\cbversion{98.1}\begin{changebar}
\item \Perl{} version 5.002, or later (check with \verb/perl -v/);\\
\Perl{} should be compiled to use the \fn{csh} or \fn{tcsh} shell,
though \latextohtml{} can also work with the \fn{bash} shell
if \Perl{} is recompiled to use it as ``full csh''.
Don't worry about this; any missing programs should be reported
by \fn{install-test} upon installation.
\end{changebar}

\item \LaTeX, meaning \LaTeXe{} dated \texttt{<1995/06/01>}, or later;
\item \fn{dvips} or \fn{dvipsk}, at version 5.58 or later;
\item \appl{Ghostscript} at version 4.02 or later;
\item \fn{netpbm} library of graphics utilities, version \textsc{1-mar-94}\\
  (check with \fn{pnmcrop} \Cs{version}).
\end{itemize}

\medskip\htmlrule
\medskip\noindent
More specific requirements for using \latextohtml{}
depend on the kind of translation you would like to perform, as follows:
%
\begin{enumerate}
\item
\index{requirements!minimal}%
\textbf{\LaTeX{}  commands but without equations, figures, tables, etc.} \hfill
\begin{itemize}
\item
\htmladdnormallink{Perl}{http://perl.com/}
\begin{small}
\cbversion{98.1}\begin{changebar}
{\bf Note:} \latextohtml{} {\bf requires \Perl{ 5} to operate}.
\end{changebar}
\end{small}\html{\smallskip}

\textbf{Warning~1: }%
You really \emph{do} need \Perl{ 5}.\\
Versions of \latextohtml{} up to \textsc{v96.1}h work
both with \Perl{ 4} at patch level 36 and \Perl{ 5}\,,
though some of the packages may only work with \Perl{ 5}\,.\html{\\}


\textbf{Warning~2: }\index{requirements!Unix shell}%
Various aspects of \Perl{}, which are used by \latextohtml{}, assume
certain system commands to be provided by the operating system shell.
If \fn{csh} or \fn{tcsh} is used to invoke \latextohtml{}
then everything should work properly.\\
\Perl{ 5} eliminates this requirement on the shell.

%%\textbf{Warning~3: }\index{requirements!special packages}%
%%Some of the packages which implement advanced features,
%%such as the \env{natbib} and \env{frames} packages,
%%require \Perl{ 5}\,.

\index{requirements!DataBase Management system}%
\item
\appl{DBM} or \appl{NDBM}, the Unix DataBase Management system,
or \appl{GDBM}, the GNU database manager.

\textbf{Note: }Some systems lack any DBM support.
\Perl{ 5} comes with its own database system SDBM, but it is sometimes
not part of some Perl distributions.

\cbversion{98.1}\begin{changebar}
The installation script \fn{install-test} will check that for you.
If no database system is found, you will have to install Perl
properly.
\end{changebar}
\end{itemize}

\index{requirements!for full graphics}%
\index{tables!as images}\index{images!tables}%
\index{images!figures}\index{images!equations}%
\item
\textbf{\LaTeX{}  commands with equations, figures, tables, etc.} \\
As above plus \dots
%
\begin{itemize}
\item \fn{latex} (version 2e recommended but 2.09 will work --- with
reduced ability to support styles and packages);
%
\index{dvips@\texttt{dvips} version}
\item
\htmladdnormallink{\texttt{dvips}}
{ftp://ftp.tex.ac.uk/pub/archive/dviware/dvips}
(version~5.516 or later) or \fn{dvipsk}\\
\cbversion{98.1}\begin{changebar}
  Version 5.62 or higher enhances the performance of image creation
  with a {\bf significant} speed-up. See latex2html.config for this
  after you are done with the installation.
  Do not use the 'dvips -E' feature unless you have 5.62, else you
  will get broken images.
\end{changebar}
%
\index{Ghostscript@\textsl{Ghostscript} version}
\item
\cbversion{98.1}\begin{changebar}
\fn{gs} \appl{Ghostscript} (version 4.03 or later);
  with the ppmraw device driver, or even better pnmraw.
  Upgrade to 5.10 or later if you want to go sure about seldom problems
  with 4.03 to avoid (yet unclarified).
\end{changebar}
%
\index{image conversion!Postscript@\PS\ to GIF}%
\index{GIF!image conversion}%
\item
\cbversion{98.1}\begin{changebar}
The \htmladdnormallink{\texttt{netpbm}}
{ftp://ftp.cs.ubc.ca/pub/archive/netpbm/netpbm-1mar1994.tar.gz}
library of graphics utilities; \fn{netpbm} dated 1 March 1994
is required, else part of the image creation process will fail.\\
Check with: \fn{pnmcrop} \Cs{version}.
\end{changebar}

Several of the filters in those libraries are used during the \PS\ to
GIF conversion.
\index{image conversion!Postscript@\PS\ to PNG}%
\index{PNG!image conversion}%
\item
\cbversion{98.1}\begin{changebar}
If you want PNG images, you need \fn{pnmtopng} (current version is 2.31).
It is not part of \fn{netpbm} and requires \fn{libpng-0.89c.tar.gz} and
\fn{libz} (1.0.4) (or later versions).
\fn{pnmtopng} supports transparency and interlace mode.\\
{\bf Hooray!!!}~~
\appl{Netscape Navigator} v4.04 has been reported to grok PNG images!\\
This means that the PNG option is no longer ahead of its time!
\end{changebar}
\end{itemize}


\index{requirements!for segmentation feature}%
\index{segmentation!needs latex2e@needs \LaTeXe}%
\item
\textbf{\htmlref{Segmentation}{Segmentation} of large documents}\\
If you wish to use this feature, you will have to upgrade your
\LaTeX{} to \LaTeXe\,.
Some other hyperlinking features also require \LaTeXe\,.

\index{requirements!for transparent images}%
\index{images!transparent}%
\item
\textbf{\htmladdnormallinkfoot{Transparent inlined images}%
{http://melmac.corp.harris.com/transparent\_images.html}}\\
If you dislike the white background color of the
generated inlined images then you should get either
the \fn{netpbm} library (instead of the older \fn{pbmplus})
or install the \htmladdnormallinkfoot{\texttt{giftrans}}%
{ftp://ftp.uni-stuttgart.de/pub/comm/infosystems/www/tools/imaging/giftrans/giftrans-1.11.1.tar.gz}
filter by Andreas Ley \Email{ley@rz.uni-karlsruhe.de}.
\latextohtml{} now supports the shareware program \fn{giftool}
(by Home Pages, Inc., version 1.0), too.
It can also create interlaced GIFs.
%
\end{enumerate}

\index{requirements!without images}%

\noindent
If \appl{Ghostscript} or the \fn{netpbm} library are
not available, it is still possible to use the translator with the
\Cs{no\_images} option.

\index{requirements!for special features}\index{special!features}%
\index{html.sty@\texttt{html.sty} style-file!needed for special features}%
\html{\\}%

If you intend to use any of the \htmlref{special features}{sec:hyp}
of the translator \latex{(see page~\pageref{sec:hyp})}
then you have to include the \fn{html.sty} file
in any \LaTeX{}  documents that use them.

\index{browser!supports images}%
\index{Mosaic@\textsl{Mosaic}|see{\htmlref{browser}{IIIbrowser}}}%
\index{NCSA Mosaic@\textsl{NCSA Mosaic}|see{\htmlref{browser}{IIIbrowser}}}%
\index{Netscape@\textsl{Netscape Navigator}|see{\htmlref{browser}{IIIbrowser}}}%
\index{browser!\textsl{NCSA Mosaic}}%
\index{browser!\textsl{Netscape Navigator}}%
\html{\\}%

Since by default the translator makes use of inlined images in the final
\texttt{HTML} output, it would be better to have a viewer
which supports the \HTMLtag{IMG} tag, such as \htmladdnormallink{\textsl{NCSA Mosaic}}%
{http://www.ncsa.uiuc.edu/SDG/Software/Mosaic/Docs/help-about.html}
or \htmladdnormallink{\textsl{Netscape Navigator}}{http://home.netscape.com}.
\cbversion{97.1}\begin{changebar}
Any browser which claims to be compatible with \HTMLiii{} should meet
this requirement.
\end{changebar}

\index{browser!character-based}%
\index{browser!character-based}\index{browser!lynx@\textsl{lynx}}%
\html{\\}%

If only a character-based browser, such as \appl{lynx}, is available,
or if you want the generated documents to be more portable,
then the translator can be used with the \Cs{ascii\_mode}
\hyperref{option}{option (see Section~}{)}{cs_asciimode}.

\index{requirements|)}


\subsubsection{Installing \protect\latextohtml}%
%\subsection{Installing \protect\latextohtml}%
\index{installation}\html{\\}%
\cbversion{98.1}\begin{changebar}
To install \latextohtml{} you \textbf{MUST} do the following:
%
\begin{enumerate}
\item
\textbf{Specify where \Perl{} is on your system}. \\
In each of the files \fn{latex2html}, \fn{texexpand}, \fn{pstoimg},
\fn{install-test} and \fn{makemap},
modify the first line saying where \Perl{} is on your system.

\index{installation!without Perl@without \Perl{} shell scripts}

\noindent
Some system administrators do not allow \Perl{} programs to run as shell scripts.
This means that you may not be able to run any of the above programs.
\emph{In this case change the first line in each of these programs from }
\html{\smallskip}
%begin{latexonly}
\begin{small}
%end{latexonly}
\verb|#!/usr/local/bin/perl |
%begin{latexonly}
\end{small}
%end{latexonly}
\html{\smallskip}\emph{to}:
%begin{latexonly}
\begin{small}
%end{latexonly}
\begin{verbatim}
# *-*-perl-*-*
    eval 'exec perl -S  $0 "$@"'
    if $running_under_some_shell;
\end{verbatim}
%begin{latexonly}
\end{small}
%end{latexonly}

\index{installation!system installation}%
\index{installation!LaTeX packages}%
\item
Copy the files to the destination directory.\\
Copy the contents of the \fn{texinputs/} directory to a place where they
will be found by \LaTeX, or set up your \fn{TEXINPUTS} variable to point
to that directory.

\index{installation!check path-names}%
\index{installation!graphics utilities}%
\item
\textbf{Run \fn{install-test}\,.} \\
This \Perl{} script will make some changes in the \fn{latex2html} file
and then check whether the path-names to any external utilities
required by \fn{latex2html} are correct.
It will not actually install the external utilities.
\fn{install-test} asks you whether to configure for GIF or
\fn{PNG} image generation.
Finally it creates the file \fn{local.pm} which houses pathnames for the
external utilities determined earlier.

\smallskip\noindent
You might need to make \fn{install-test} executable before using it.\\
Use \verb/chmod +x install-test/ to do this.

\smallskip\noindent
You may also need to make the files \fn{pstogif},
\fn{texexpand}, \fn{configure-pstoimg} and \fn{latex2html} executable
if \fn{install-test} fails to do it for you.

\index{installation!system installation}%
\item
If you like so, copy or move the \fn{latex2html} executable script to
some location outside the \fn{\$LATEX2HTMLDIR} directory.

\index{installation!change configuration}%
\index{installation!change defaults}
\item
You might want to edit \fn{latex2html.config} to reflect your needs.
Read the instructions about \fn{\$ICONSERVER} carefully to make sure your
\fn{HTML} documents will be displayed right via the Web server.\\
While you're at it you may want to change some of the default
options in this file.\\
If you do a system installation for many users, only cope with general
aspects; let the user override these with \verb|$HOME/.latex2html-init|.
%
\end{enumerate}


Note that you \emph{must} run \fn{install-test} now;
formerly you could manage without.
If you want to reconfigure \latextohtml{} for GIF/PNG image
generation, or because some of the external tools changed the location,
simply rerun \fn{configure-pstoimg}.
\end{changebar}


\medskip\htmlrule[50\% center]
\noindent
This is usually enough for the main installation, but you may also
want to do some of the following, to ensure that advanced features
of \latextohtml{} work correctly on your system:
\begin{itemize}
\index{html.sty@\texttt{html.sty} style-file!location}%
\index{installation!environment variable}%
\item
\textbf{To use the new \LaTeX{}  commands
which are defined in \fn{html.sty}:}\\
Make sure that \LaTeX{}  knows where the \fn{html.sty} file is,
either by putting it in the same place as the other style-files on your system,
or by changing your \fn{TEXINPUTS} shell environment variable,
or by copying \fn{html.sty} into the same directory as your \LaTeX{}  source file.

The environment variable \fn{TEXINPUTS} is not to be confused with
the \latextohtml{} installation variable \fn{\$TEXINPUTS} described next.
\par

\index{installation!input-path variable}%
\item
There is an installation variable in \fn{latex2html.config}
called \fn{\$TEXINPUTS},
which tells \latextohtml{} where to look for \LaTeX{} style-files to process.
It can also affect the input-path of \LaTeX{} when called by \latextohtml,
unless the command \fn{latex} is really a script
which overwrites the \fn{\$TEXINPUTS} variable
prior to calling the real \fn{latex}.
This variable is overridden by the environment variable of the same name
if it is set.

\index{installation!fonts-path variable}%
\item
The installation variable \fn{\$PK\_GENERATION} specifies which
fonts are used in the generation of mathematical equations.  A value
of ``\texttt{0}'' causes the same fonts to be used as those for the default
printer.  Because they were designed for a printer of much greater
resolution than the screen, equations will generally appear to be
of a lower quality than is otherwise possible.  To cause \latextohtml{} to
dynamically generate fonts that are designed specifically for the
screen, you should specify a value of ``\texttt{1}'' for this variable.
If you do, then check to see whether your version of \texttt{dvips}
supports the command-line option \Cs{mode}\,.  If it does,
then also set the installation variable \fn{\$DVIPS\_MODE} to
a low resolution entry from \fn{modes.mf}, such as \fn{toshiba}.

It may also be necessary to edit the \fn{MakeTeXPK} script,
to recognise this mode at the appropriate resolution.

\cbversion{97.1}\begin{changebar}%
If you have \PS\ fonts available for use with \LaTeX{} and \fn{dvips}
then you can probably ignore the above complications and simply
set \fn{\$PK\_GENERATION} to ``\texttt{0}''
and \fn{\$DVIPS\_MODE} to \texttt{\char34\char34} (the empty string).
You must also make sure that \fn{gs} has the locations of the
fonts recorded in its \fn{gs\_fonts.ps} file. This should already be the case
where \appl{GS-Preview} is installed as the viewer for \texttt{.dvi}-files,
using the \PS\ fonts.
\end{changebar}

If \fn{dvips} does \emph{not} support the \Cs{mode} switch,
then leave \fn{\$DVIPS\_MODE} undefined, and verify that the
\fn{.dvipsrc} file points to the correct screen device and its
resolution.

\index{installation!filename-prefix}\label{autoprefix}%
\item
The installation variable \fn{\$AUTO\_PREFIX}
allows the filename-prefix to be automatically set
to the base filename-prefix of the document being translated.
This can be especially useful for multiple-segment documents.

%%JCL obsolete with Perl5
%%\index{installation!Linux@\textsl{Linux} systems}%
%%\item
%%On certain \appl{Linux} systems it is necessary to uncomment
%%the line ``\texttt{use GDBM\_File}'' in the \fn{latex2html}
%%and \fn{install-test} scripts, to define the interface to the
%%data\-base-management routines.

\index{installation!makemap script@\texttt{makemap} script}%
\index{CERN!image-map server}\index{NCSA!image-map server}%
\item
The \fn{makemap} script also has a configuration variable \fn{\$SERVER},
which must be set to either \texttt{CERN} or \texttt{NCSA},
depending on the type of Web-server you are using.


\index{installation!initialization files}%
\index{initialization file!per user}%
\label{initfile}%
\item \textbf{To set up different initialization files:}\\
For a ``per user'' initialization file,
copy the file \fn{dot.latex2html-init} in the home directory
of any user that wants it, modify it according to her preferences and
rename it as \ \fn{.latex2html-init}. At runtime, both the
\fn{latex2html.config} file and \fn{\$HOME/.latex2html-init} file will be
loaded, but the latter will take precedence.

\index{initialization file!per directory}%

You can also set up a ``per directory'' initialization file by
copying a version of \ \fn{.latex2html-init} in each directory you
would like it to be effective. An initialization file
\path{/X/Y/Z/.latex2html-init} will take precedence over all other
initialization files if \path{/X/Y/Z} is the ``current directory'' when
\latextohtml{} is invoked.

\index{initialization file!incompatible with early versions}%
\begin{quotation}\noindent
\textbf{Warning: }%
This initialization file is incompatible with
any version of \latextohtml\ prior to \textsc{v96.1}\,.
Users must either update this file in their home directory,
or delete it altogether.
\end{quotation}

\index{installation!icons subdirectory@\texttt{icons/} subdirectory}%
\item \label{icondir}%
\textbf{To make your own local copies of the \latextohtml{} icons:} \\
Please copy the \fn{icons/} subdirectory to a
place under your WWW tree
where they can be served by your server.
Then modify the value of the \fn{\$ICONSERVER} variable in
\fn{latex2html.config} accordingly.
\index{installation!local icons}%
\cbversion{97.1}\begin{changebar}%
Alternatively, a local copy of the icons can be included within
the subdirectory containing your completed \texttt{HTML} documents.
This is most easily done using the \Cs{local\_icons} command-line switch,
or by setting \fn{\$LOCAL\_ICONS} to ``\texttt{1}'' in \fn{latex2html.config}
or within an initialization file, as described \htmlref{above}{initfile}.
\end{changebar}

\index{Livermore, California }%
\begin{quotation}\noindent
\textbf{Warnings: }%
If you cannot do that, bear in mind that these icons will have
to travel from Livermore, California!!!
Also note that several more icons were added in \textsc{v96.1}
that were not present in earlier versions of \latextohtml.
\end{quotation}

\index{installation!create manual}%
\index{documentation!test of installation}%
\item
\textbf{To make your own local copy of the \latextohtml{}
documentation:} \\
This will also be a good test of your installation.
% To do it run \latextohtml{} on the file \fn{docs/manual.tex}.
% You will get better results if you run \LaTeX{} first on the
% same file in order to create some auxiliary files.
%
\index{documentation!dvi version@\texttt{.dvi} version}%
\noindent
Firstly, to obtain the \texttt{.dvi} version for printing,
from within the \fn{docs/} directory it is sufficient to type:

%begin{latexonly}
\begin{small}
%end{latexonly}
\texttt{ make manual.dvi}
%begin{latexonly}
\end{small}
%end{latexonly}

\noindent
This initiates the following sequence of commands:
%begin{latexonly}
\begin{small}
%end{latexonly}
\begin{verbatim}
latex manual.tex
makeindex -s l2hidx.ist manual.idx
makeindex -s l2hglo.ist -o manual.gls manual.glo
latex manual.tex
latex manual.tex
\end{verbatim}
%begin{latexonly}
\end{small}
%end{latexonly}
\index{documentation!using makeindex@using \texttt{makeindex}}%
\index{documentation!index and glossary}%
...in which the two configuration files \fn{l2hidx.ist} and \fn{l2hglo.ist}
for the \fn{makeindex} program, are used to create the index and glossary respectively.
The 2nd run of \fn{latex} is needed to assimilate references, etc.
and include the index and glossary.\html{\\}%
\index{documentation!without makeindex@without \texttt{makeindex}}%
\html{\\}
(In case \fn{makeindex} is not available, a copy of its outputs \fn{manual.ind}
and \fn{manual.gls} are included in the \fn{docs/} subdirectory,
along with \fn{manual.aux}\,.)\html{\\}
The 3rd run of \fn{latex} is needed to adjust page-numbering for the Index
and Glossary within the Table-of-Contents.

\noindent
Next, the \texttt{HTML} version is obtained by typing:

%begin{latexonly}
\begin{small}
%end{latexonly}
\texttt{make manual.html}
%begin{latexonly}
\end{small}
%end{latexonly}

\noindent
This initiates a series of calls to \latextohtml{} on the separate
segments of the manual;
the full manual is thus created as a ``segmented document''
(see \hyperref{a later section}{Section~}{}{Segmentation}).
The whole process may take quite some time,
as each segment needs to be processed at least twice,
to collect the cross-references from other segments.

\medskip
\index{documentation!requirements}\html{\\}%
\noindent
The files necessary for correct typesetting of the manual to be
found within the \fn{docs/} subdirectory.
They are as follows:
\begin{itemize}
%
\index{html.sty@\texttt{html.sty} style-file}%
\index{documentation!style-files}%
\item
style-files:
 \fn{l2hman.sty}, \fn{html.sty}, \fn{htmllist.sty}, \fn{justify.sty},\\
  \fn{changebar.sty} and \fn{url.sty}
%
\index{documentation!input files}%
\item
inputs:
 \strikeout{\fn{changes.tex},} \fn{credits.tex}, \fn{features.tex}, \fn{hypextra.tex},\\
 \fn{licence.tex}, \fn{manhtml.tex}, \fn{manual.tex}, \fn{overview.tex},\\
 \fn{problems.tex}, \fn{support.tex} and \fn{userman.tex}
%
\index{documentation!graphics}%
\item
sub-directory:
 \fn{psfiles/} containing \PS\ graphics
 used in the printed version of this manual
%
\index{documentation!images}%
\item
images of small curved arrows: \fn{up.gif}, \fn{dn.gif}
%
\index{documentation!filename data}%
\item
filename data:
 \fn{l2hfiles.dat}
%
\index{documentation!auxiliaries}%
\item
auxiliaries:
 \fn{manual.aux}, \fn{manual.ind}, \fn{manual.gls}
%
\end{itemize}

The last three can be derived from the others, but are included for convenience.
\par

\index{documentation!Changes section}%
\item

\cbversion{98.1}\begin{changebar}
\textbf{To get a printed version of the `Changes' section: }\\
Due to the burgeoning size of the \fn{Changes} file with successive
revisions of \latextohtml, the `Changes' section is no longer
supported for the manual.
Please refer to text file \fn{Changes} instead which is part of the
distribution.
\end{changebar}


\index{discussion group}\index{bugs!bug reports}%
\item
\textbf{To join the community of \latextohtml{} users:} \\
More information on a mailing list, discussion archives, bug reporting
forms and more is available at
\url{http://cbl.leeds.ac.uk/nikos/tex2html/doc/latex2html/latex2html.html}
\end{itemize}


\subsection[center]{Getting Support and More Information\label{support}}%
%\section[center]{Getting Support and More Information\label{support}}%
\index{support}%
\index{support!mailing list}\index{mailing list!Argonne National Labs}

A \htmladdnormallink{\latextohtml{} mailing list}%
{mailto:latex2html-request@mcs.anl.gov}
has been set up at the Argonne National Labs.
The \htmladdnormallinkfoot{\latextohtml{} mailing list archive}%
{http://cbl.leeds.ac.uk/nikos/tex2html/doc/mail/mail.html} is available.
\html{\\}
(Thanks to Ian Foster \Email{itf@mcs.anl.gov}
and Bob Olson \Email{olson@mcs.anl.gov}.)

\smallskip\noindent
To join send a message to: \Email{latex2html-request@mcs.anl.gov }
\index{mailing list!subscribe}\\
with the contents:~~\texttt{ subscribe }\latex{.}


\smallskip\noindent
To be removed from the list send a message to:
\Email{latex2html-request@mcs.anl.gov}
\index{mailing list!unsubscribe}\\
with the contents:~~\texttt{ unsubscribe }\latex{.}


\bigskip\noindent
The mailing list also has a searchable online archive at \ListURL.
It recommendable to start with that URL first, to get in touch with
the topics actually discussed and to search for articles related with
your interests.

\bigskip
Enjoy!

