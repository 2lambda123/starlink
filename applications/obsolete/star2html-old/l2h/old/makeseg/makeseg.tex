
\documentclass[a4paper,12pt]{article}

\usepackage{natbib}
\usepackage{html}
\usepackage{latexsym}
\usepackage{makeidx}

\makeindex
\citeindextrue

\newcommand{\htm}{{\tt HTML}}
\newcommand{\lhh}{{\rm \LaTeX2}\htm}
\newcommand\bef{\begin{figure}}

\begin
{document}
\sloppy

\title{Makesegments - a segmentation tool for \lhh}
\author{Martin Wilck}

\date{Institut f{\"u}r Troposph{\"a}renforschung \\
  04303 Leipzig, Germany}

\maketitle

\tableofcontents

\section{Introduction}
\label{intro}

\lhh\ \index{latex2html@\lhh} is a powerful tool for translating \LaTeX\ documents into a
hierarchy of linked \htm\ documents suitable for presentation on the
World Wide Web \cite[]{drakos:l2h}. Users of \lhh\ will encounter
problems if their \LaTeX\ documents are very large
\index{document!large}
--- the translation
process will eventually fail due to lack of memory. Therefore the
document segmentation mechanism\index{segmentation}, described in detail by
\citeauthor{drakos:l2h}, was developed. It serves to split the
document into peaces called ``segments'', which may be translated by
\lhh\ seperately. Makesegments is a utility to produce a suitably
segmented document from a non-segmented \LaTeX\ file.

\subsection{Features}
\label{features}

Makesegments
\begin{itemize}
\item takes care to preserve the \LaTeX\ structure of your document,
\item reads \verb:\input: files and inserts their
  contents,\index{input@\verb+\input+}\index{input files}
\item handles already segmented documents, eventually improving their
  segment structure or changing the level of it,
\item Handles \verb:\verb: statements and \verb+verbatim+ environments
  correctly,
\item automatically inserts the preamble into all segments, stripping
  off useless \verb+\internal+ statements and \verb+htmlonly+ environments,
\item cares for the passing of \lhh\ internal information by inserting
  appropriate \verb:\internal: statements,
\item places special sections like \verb:\tableofcontents: or
  \verb:\bibliography: in the master file (unless the
  \hyperref{\texttt{ -e} option}{\texttt{-e} option, see }{}{options}
  is given),
  \index{special sections}
\item generates a Makefile for the segmented document, which may be
  \index{Makefile}
  edited for customization of \htm\ titles etc. (see \ref{makefile}),
\item allows for unconventional \LaTeX\ coding, e.~g. sections that
  are enclosed (or even irregularly devided) by \LaTeX\ blocks (see
  \ref{caveats} and \ref{blocks}), \index{latex blocks@\LaTeX blocks}
\item copies (or links) input files to the target directory
  automatically, \index{input files!copying}
\item optionally checks if requested \LaTeX\ style files can be found.
\end{itemize}

\section{Usage}
\label{usage}

If you have written a long \LaTeX\ document, which you would like to
translate into \htm, you may use Makesegments to produce a segmented
version of this document that can be comfortably processed by \lhh.
It moves the text of each section to a seperate file, leaving only
the preamble, the \verb:\segment:
commands\index{segment@\verb+\segment+}, and necessary global
information in the main file.

\subsection{Arguments and command line options}

Makesegments is called in the following way:\index{usage}

\begin
{quote}

\begin{verbatim}
$ makesegments [options] inputfile
\end{verbatim}
\end{quote}

Note that you shold change to the directory containig the input file
before calling Makesegments.
The \verb:.tex: extension may be ommited from filenames.

\subsubsection{Options}
\label{options}\index{command line options}\index{invocation}\index{options}

\begin{description}
\item[\texttt{ -level sectioning-level} (short \texttt{ -l})]
 \index{options!level@\texttt{ -level/-l}}
  specifies the level of sectioning commands down to that segmentation is
  carried out. Possible values are
  \verb+document+, \verb+part+, \verb+chapter+, \verb+section+,
  \verb+subsection+, and \verb+subsubsection+. If you specify
  \verb+-l document+, Makesegments will --- against its usual purpose
  --- merge all input files and predefined segments into one big file.
\item[\texttt{ -dir directory} (short \texttt{ -d})]  specifies the
\index{options!dir@\texttt{ -dir/-d}}
  target directory, where all segments and all other necessary files
  are going to be written. Default is \verb+./segmented.+
\item[\texttt{ -output output-file} (short \texttt{ -o})]
  \index{options!dir@\texttt{ -dir/-d}}
  specifies the name of the master file for output (usually without
  \verb+.tex+ extension).
\item[\texttt{ -config configuration-file} (short \texttt{ -c})]
  \index{options!config@\texttt{-config/-c}}
  is the name of a \hyperref{configuration file}{configuration file
    (see }{)}{config} for
  Makesegments. This file may \index{configuration file}
  define several configuration variables.
\item[\texttt{ -zero string} (short \texttt{ -z})]
  \index{options!zero@\texttt{ -zero/-z}} is the name of the ``segment
  zero''. Makesegments will increment this string in order to obtain
  segment names. That is, if you specify \texttt{ -z seg00}, the
  segments will be named \texttt{ seg01.tex}, \texttt{ seg02.tex}, and so
  forth. If you specify \texttt{ -z master}, the segment names will be
  \texttt{ mastes}, \texttt{ -mastet}, \ldots, \texttt{ mastez},
  \texttt{ mastfa}, \ldots. See the \verb+perlop+ Manpage for details
  on string incrementation.
\item[\texttt{ -no\_makefile} (short \texttt{ -n})] suppresses
  Makefile generation.\index{options!nomakefile@\texttt{ -no\_makefile/-n}}\index{Makefile}
\item[\texttt{ -check\_latex\_styles} (short \texttt{ -s})]
  \index{options!check latex@\texttt{ -check\_latex\_styles/-s}}
  causes Makesegments to look for \LaTeX\ style files. This requires
  the correct setting of the environment variables for \LaTeX\ input
  file paths (mainly \texttt{ TEXINPUTS}) or the related
  \hyperref{configuration variables}{configuration variables
  (see }{)}{texinputs}. Setting this variable has two side effects:
  1.~\verb+\usepackage+ commands requesting a \LaTeX\ package that
  can't be found are deleted; 2.~``local'' style files that are in the
  current directory are copied or linked to the target directory.
\item[\texttt{ -ignore\_inputs} (short \texttt{ -i}):]   \index{input@\verb+\input+}
  \index{options!ignore@\texttt{ -ignore\_inputs/-i}} With
  this option, Makesegments will keep \verb+\input+ and
  \verb+\include+ statements in their place rather than repacing them
  with the contents of the input files (of course, sectioning commands
  inside the input files won't be found!)
\item[\texttt{ -use\_links} (short \texttt{ -u}):]
  \index{options!use@\texttt{ -use\_links/-u}} Usually, Makesegments
  copies input files to the target directory unless it is the same as
  the current. This holds for files included with \verb+\input+ and
  \verb+\include+ if the \verb+-i+ option is set, for ``local'' style
  files if the \verb+-s+ option is set, for bibliography files
  (\verb+.bib+, \verb+.bst+) and
  \hyperref{image files}{image files(see }{)}{epsfig}
  as long as they are given
  by a path {\em relative to the current directory}\/. The reason is
  that these input files will not be found by \LaTeX\ or \lhh\ in the
  target directory. With the \verb+-u+ option, Makesegments creates
  symbolic links in the target directory instead.
\item[\texttt{ -dont\_copy} (short \texttt{ -y}):]
  \index{options!dontcopy@\texttt{ -dont\_copy/-y}} If this flag is set,
  Makesegments will neither copy nor link the input files to the
  target directory.
\item[\texttt{ -specialsegments} (short \texttt{ -e}):]
  \index{options!specialsegments@\texttt{ -specialsegments/-e}}
  \index{special sections}
  Normally Makesegments places all ``special sections'' of \LaTeX in
  the master segment. With the \verb+-e+ option, it will create extra
  files for them. If \verb+-e+ is set, the configuration variable
  \hyperref{\texttt{\%SPECIALNAMES}}{\texttt{\%SPECIALNAMES} (see
    }{)}{config}\
  determines which of the special sections will
  be treated this way, and which file names they'll get.

  Doing this is
  useful especially for the table of contents
  (\verb+\tableofcontents+) and the index (\verb+\printindex+), since
  in the \htm\ version
  these may be linked to the other segments by the ``Contents'' and
  ``Index'' buttons in the navigation panel. If \verb+-e+ is not
  specified, Makesegments can't figure out the URL's of the contents
  and index pages, and therefore can't include this information in the
  Makefile; with \verb+-e+, the URL's are determined before Makefile
  generation, and Makesegments is able to set the \verb+-contents+ and
  \verb+-index+ switches for the segments correctly.
\index{latex2html@\lhh!prev@\verb+-contents+ option}
\index{latex2html@\lhh!prev@\verb+-index+ option}
\item[\texttt{ -help} (short \texttt{ -h} or \texttt{ -?}):]
  \index{options!help@\texttt{ -help/-h/-?}} Display a help message
  with overview over the options.

\end{description}

\subsection{Caveats}
\label{caveats}

What you should avoid:

\begin{itemize}
\item misaligned \LaTeX\ blocks, such as \verb:{\small \begin{figure}...}:.
    (\LaTeX
    will also complain about this!)\index{latex blocks@\LaTeX blocks}
\item use of plain \TeX\ commands such as \verb+\def+.
\index{tex commands@\TeX\ commands}
\end{itemize}

The more ``structured'' your \LaTeX\ document, the easier and better
the outcome of Makesegments will be.

The \verb+\segment+\index{segment@\verb+\segment+} command, that is
defined in the \lhh\ file \verb+html.sty+, does not account for the
optional argument to sectioning commands (short version of the section
title). \index{sectioning commands!optional argument} These arguments are
therefore discarded by Makesegments.

\subsection{After using Makesegments}
\label{after}

Ideally, after the segmentation procedure has successfully completed,
you just need to change to your target directory and type
\verb:make:.\index{make@\verb+make+} This will produce the complete
set of \htm\ pages. Type \verb:make all: if you want DVI and
PostScript output, too.\index{DVI output}
\index{PostScript output} Then change to the
subdirectory called \verb:inputfile: (without \verb:.tex: extension),
and start your \htm\ browser on \verb:inputfile.html:\index{html
  output@\htm\ output}. In reality,
you will probably need to edit the \index{Makefile}
\hyperref{Makefile}{Makefile (see }{)}{makefile} and perhaps even some of
the segments to obtain the desired \htm\ output.

\section{Internals}
\label{internals}

Makesegments, as well as \lhh\ itself, is written in Perl. That is, it
is easy to customize for people with a basic knowledge of Perl.
\index{Perl} This section gives a brief overview of how Makesegments works.

\subsection{Block structure}
\label{blocks}

As mentioned in \ref{features}, Makesegment takes special care to
avoid confusion by intermediate \LaTeX\ blocks.
Consider the following file:
\index{latex blocks@\LaTeX blocks}

\begin{verbatim}
\documentclass{article}
<preamble>
{
  \begin{document}
    <leading text>
  \section{1}
    <text of 1>
    {\small
      <more text of 1>
      \section{2}
        <text of 2>
        \texttt{
          <more text of 2>
          \listoffigures
        }
    }
  \end{document}
}
\end{verbatim}

How should this be divided reasonably into segments? Makesegments will
produce the following master file:

\begin{verbatim}
\documentclass{article}
<preamble>
{
  \begin{document}
    <leading text>
  \segment{a1}{1}        %  -> that's where <text of 1> goes
    {\small
      <more text of 1>
      \segment{a2}{2}    %  -> that's where <text of 2> goes
        \texttt{
          <more text of 2>
          \listoffigures
        }
    }
  \end{document}
}
\end{verbatim}

Note that all blocks that contain a command that is relevant to
segmentation (\verb+\section+ and \verb+\listoffigures+) have been
moved to the master segment. This ensures that the \LaTeX\ block
structure is unharmed and that \LaTeX\ will produce the same DVI output
from the segmented file as from the original document. \index{DVI output}

But it is also clear that \lhh\ will not see the \verb+\small+
command when it processes segment 1.
Also, the text of both sections is divided between the segments and
the master file, leading to a strange structure in the \htm\ document.

Thus, even though Makesegments
can still produce a fairly reasonable segmentation from such a file,
it is not recommended to write \LaTeX\ code this way (if it is
supposed to be segmented, at least).

\subsection{Makefile}
\label{makefile}

The Makefile defines several variables that may be customized.
For each segment there is a variable that specifies the title of the
corresponding URL (\verb+A3TITLE+ for segment \verb:a3.tex:) and one
that determines the command options for \lhh\ when it is invoked on
this segment (\verb:L2HA3:).  Segment titles are derived from the
section titles they were made of, with all \LaTeX\ commands and their
arguments stripped, so it's likely that the URL titles are not what
you desire.

Makesegments tries to guess the correct
neighbour segments and link them with the \verb+-prev_url+, \verb+-up_url+ and
\verb+-down_url+ options to \lhh.\index{latex2html@\lhh!down@\verb+-down_url+ option}
\index{latex2html@\lhh!up@\verb+-up_url+ option}
\index{latex2html@\lhh!prev@\verb+-prev_url+ option}

This may easily be modified by changing the corresponding variables.
Note that Makesegments can only set the \verb+-index+ and \verb+-contents+ options
\index{latex2html@\lhh!prev@\verb+-contents+ option}
\index{latex2html@\lhh!prev@\verb+-index+ option}
in the Makefile correctly if the file is processed
with the \verb+-e+ option. Otherwise, you'll have to find out the
URL's of these sections ``by hand'' using a \htm\ browser.

\subsection{The configuration file}
\label{config}\index{configuration file}

Makesegments looks for configuration files in three places:
\begin{itemize}
\item in the file specified with the \verb:-c: option,
\item in the file \verb:.makesegments.cnf: in the current directory,
\item in \verb:$HOME/.makesegments.cnf:.
\end{itemize}

If no configuration file is found, Makesegments sets its own defaults.

The files are executed in the order specified above, so if you build a
configuration file in your home directory, take care that it doesn't
override settings that have been made before. The same holds for
system operators that want to change Makesegents' behaviour by editing
the makesegments script itself.

The easiest way to create a config file is copying the configuration
part of makesegments (from ``Site configuration'' to ``End of
configuration options'') to a seperate file and edit.

\subsubsection{Path variables for \LaTeX\ inputs}
\label{texinputs}\index{path!latex@\LaTeX inputs}

Setting these is very impotant if you want to use the \texttt{
  -check\_latex\_styles} command line option.
  \index{options!check latex@\texttt{ -check\_latex\_styles/-s}}
Makesegments tries to read the environment variables \verb+TEXINPUTS+
(for style files as well as text inluded with \verb+\input+ and
\verb+\include+ statements), \verb+BIBINPUTS+ (for bibliography
databases) and \verb+BSTINPUTS+ (for bibliography styles). But on many
Unix systems these environment variables aren't set, since they have
been compiled into the \verb+kpathsea+ library, that searches files
for \TeX. If you work on such a system, set the corresponding Perl
variables \verb+$TEXINPUTS+ etc.

\subsubsection{Path variables for executables}
\label{executables}\index{path!executables}

These variables are only needed for Makefile
generation\index{Makefile}. You should set them if your executables
for \LaTeX, Bib\TeX, Makeindex, dvips, \lhh, make or touch are
different then usual. The variable names are \verb+$LATEX+,
\verb+$BIBTEX+ etc. Additionally, Makesegments defines two variables
\verb+$TEXENV+ and \verb+$DVIPSENV+ for environment information that should
be passed explicitly to these programs, e.~g.:
\begin{verbatim}
$TEXENV="TEXINPUTS=$TEXINPUTS:/home/myname/mylatex/inputs/ ".
        "PKFONTS=/home/myname/mylatex//pk//";
$LATEX="$TEXENV /home/myname/bin/latex";
\end{verbatim}
This will produce a command line in the Makefile that sets the desired
environment variables (note that the above example just {\em adds} a
directory to the Perl variable \verb+$TEXINPUTS+) and calls the special
executable.

\subsubsection{Variables corresponding to command line options}
\index{options}

Those settings in the configuration files that correspond to command
line options are always {\em overriden} by the latter. But be aware
that some command line options can only switch a feature {\em on}. For
example, if you specify \verb+$DONTCOPY=1;+ Makesegments will not copy
any input files and there's no command line option to override this.

\index{options!specialsegments@\texttt{ -specialsegments/-e}}
The configuration variable \verb+$SPECIAL+ corresponds to the
\verb+-specialsegments+ option. However, if this is set, you have
further customization possibilities. The hash table
\verb+%SPECIALNAMES+ contains the default file names (without
extension) for the \LaTeX\ commands that you'd like to have a special
segment for. The Makesegments script defines
\begin{verbatim}
%SPECIALNAMES=(
    "tableofcontents" => "toc",
    "printindex" => "ind",
#    "listoffigures" => "lof",
#    "listoftables" => "lot",
#    "bibliography" => "bbl",
#    "thebibliography" => "tbl",
               );
\end{verbatim}
Note that the last four entries are commented out: no entries are set
here (this is different from setting one of those entries to an empty
string!). With this (default) setting, Makesegments will only create
special segments for the index and for the table of contents.
The hash table \verb+%SPECIALTITLES+ takes the same keys as
\verb+%SPECIALNAMES+; the values are the URL titles that you'd like
for these segments. This is mainly intended for language
customization.

\subsubsection{Adding support for \LaTeX\ commands}
\label{commands}\index{adding new \LaTeX\ commands}

The ``Experts only'' section of the configuration section in
Makesegments is intended to help users to add support for special
\LaTeX\ commands. This may be helpful e.~g. for
\begin{itemize}
\item commands that take file arguments that you'd like to have
  copied;
\item commands that take weird arguments that should not be parsed
  (Makesegment parses command arguments with a simpler algorithm than
  the normal text: commands and environments are not recognized inside
  these arguments. This is useful for the \verb+\newcommand+ command,
  for example, because Makesegments would usually complain about a
  construct like
\begin{verbatim}
       \newcommand\beq{\begin{equation}}
\end{verbatim}
\index{newcommand@\verb+\newcommand+}
since it would think that the \verb+equation+ environment wasn't regularly
closed);
\item commands that should initiate any special operation by makesegments.
\end{itemize}

If you want Makesegments to recognize a command, you have to add it to
the table of known commands. This may be done by adding entries to the
hash table \verb+%USER_CMDS+. The syntax is as follows:
\begin{verbatim}
    %USER_CMDS=(
                "command1" => [argument-list],
                "command2" => [argument-list],
                ...
                );
\end{verbatim}
\index{usercmds@\verb+%USER_CMDS+}
The command names should be given without the leading backslash.
\verb+argument-list+ is a comma-separated list of the two items
\verb+$m+ for mandatory and \verb+$o+ for optional arguments.
Example:
\begin{verbatim}
    %USER_CMDS=(
                "parbox" => [$o, $m, $m],
                "newpage" => [],
                );
\end{verbatim}
See \htmlref{the next section}{example} for another example. If a command
takes no arguments, specify an empty array. See also the
Makesegments script and look at the definition of \verb+%misc_cmds+.

Usually one entry in the hash suffices for both the normal and the
starred version of the command (if there is one). You only need to add
an extra entry for the starred version if it takes other arguments (or
in different order) than the unstarred one.

Adding a command to \verb+%USER_CMDS+ suffices if you just want
Makesegments to ignore this command's arguments (such as in the
\verb+\newcommand+ example above).

If you want Makesegments to do anything special when the command is
encountered, you must define a subroutine called \verb+do_mycommand+
(but you {\em always} need  the  \verb+%USER_CMDS+ entry!).

All routines of the type \verb+do_mycommand+ have access to the
following local variables:
\begin{description}
\item[\texttt{\$command}] contains the command name, without backslash
  and (eventual) star.
\item[\texttt{\$star}] contains "\verb+*+" if the command was starred.
\item[\texttt{@arg}] is the array of arguments in the order given, with
  brackets.
\item[\texttt{@mand}] is the array of mandatory arguments in given order,
  brackets stripped.
\item[\texttt{@opt}] is the array of optional arguments.
\end{description}

You may delete the
command from the text output of Makesegments by setting
\verb+$command=""; @arg=();+

\subsubsection{An example}\label{example}
\label{epsfig}
\index{adding new \LaTeX\ commands}
\index{epsfig@\verb+\epsfig+}

As an example, the following code shows how to add support for the
\verb+\epsfig+ command. The only graphics command that is supported by
default by Makesgements is \verb+\includegraphics+. But a user might
wish that Makesegments copy the graphics file specified by an
\verb+\epsfig+ command to the target directory.

\begin{verbatim}
# First, add epsfig to %USER_CMDS

    %USER_CMDS=(
# epsfig takes only one mandatory argument
                "epsfig" => [$m],
                ...
                );

# Now the subroutine definition
    sub do_epsfig {

# Check if the epsfig package was loaded
# If not, do nothing

        if (defined $packages{"epsfig"}) {

# Do a pattern match with the (first and only) mandatory
# argument to retrieve the filename

            $mand[0] =~ /file=\s*([^\s,]+)\s*,/;
            my $file=$1;

# Use the &find_texinput routine to look where the file
# actually is. The routine needs two array references
# as input: One for the directories to search,
# one for possible extensions (if ommitted in the text)
# It returns the full path name or nothing if the
# file wasn't found.

            $file=&find_texinput
                ($file,\@graphicsinputs,\@graphicsextensions);
            if ($file) {

# If the returned filename is a full pathname
# (i.e. starts with / or ~), it shouldn't be copied
# (it will be found from the target directory also)

                push @files_to_copy,$file
                             unless file=~ m:^(/|~):;

            } else {

# Complain if the file wasn't found

                print STDERR "epsfig input file not found!\n";
            };
        };
    };
\end{verbatim}

\subsection{Support}\label{support}

If you have trouble with makesegments, mail me
    (\htmladdnormallink{martin@tropos.de}{mailto:martin@tropos.de}).

\section*{Acknowledgements}

The segmentation mechanism of \lhh\ has been invented an programmed by
Herb Swan and Ross Moore. Thanks to them, to Nikos Drakos and all the
others who made \lhh. \index{Moore, Ross}\index{Drakos,
  Nikos}\index{Swan, Herb}

\printindex

\bibliographystyle{plainnat}
\bibliography{rep,harvard}

\end{document}
