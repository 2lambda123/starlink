\documentclass[twoside,11pt]{article}
\pagestyle{myheadings}

% -----------------------------------------------------------------------------
% ? Document identification
% Fixed part
\newcommand{\stardoccategory}  {Starlink User Note}
\newcommand{\stardocinitials}  {SUN}
\newcommand{\stardocsource}    {sun\stardocnumber}

% Variable part
\newcommand{\stardocnumber}    {217.2}
\newcommand{\stardocauthors}   {A J Chipperfield}
\newcommand{\stardocdate}      {11 December 1997}
\newcommand{\stardoctitle}     {Running Starlink Applications from IRAF CL}
\newcommand{\stardocversion}   {}
\newcommand{\stardocmanual}    {}
\newcommand{\stardocabstract}  {
To extend the choice of data-reduction software for
\htmladdnormallink{IRAF}{\IRAFURL}
users, a system has been developed which enables applications written for the
\htmladdnormallink{Starlink}{\STARURL}
software environment to be run from the IRAF CL. 
Data conversion between the IRAF image format and the Starlink Extensible 
N-Dimensional Data Format (NDF) can be done without the need for additional 
conversion steps so IRAF and Starlink applications can be intermixed easily.
\par
This document assumes that the reader is familiar with running IRAF 
applications, and describes in general terms how to install and run Starlink 
applications for use with IRAF.
Information about which application packages are available, and more specific
information relating to particular packages may be found elsewhere.
}
% ? End of document identification

% -----------------------------------------------------------------------------

\newcommand{\stardocname}{\stardocinitials /\stardocnumber}
\markboth{\stardocname}{\stardocname}
\setlength{\textwidth}{160mm}
\setlength{\textheight}{230mm}
\setlength{\topmargin}{-2mm}
\setlength{\oddsidemargin}{0mm}
\setlength{\evensidemargin}{0mm}
\setlength{\parindent}{0mm}
\setlength{\parskip}{\medskipamount}
\setlength{\unitlength}{1mm}

% -----------------------------------------------------------------------------
%  Hypertext definitions.
%  ======================
%  These are used by the LaTeX2HTML translator in conjunction with star2html.

%  Comment.sty: version 2.0, 19 June 1992
%  Selectively in/exclude pieces of text.
%
%  Author
%    Victor Eijkhout                                      <eijkhout@cs.utk.edu>
%    Department of Computer Science
%    University Tennessee at Knoxville
%    104 Ayres Hall
%    Knoxville, TN 37996
%    USA

%  Do not remove the %\begin{rawtex} and %\end{rawtex} lines (used by 
%  star2html to signify raw TeX that latex2html cannot process).
%\begin{rawtex}
\makeatletter
\def\makeinnocent#1{\catcode`#1=12 }
\def\csarg#1#2{\expandafter#1\csname#2\endcsname}

\def\ThrowAwayComment#1{\begingroup
    \def\CurrentComment{#1}%
    \let\do\makeinnocent \dospecials
    \makeinnocent\^^L% and whatever other special cases
    \endlinechar`\^^M \catcode`\^^M=12 \xComment}
{\catcode`\^^M=12 \endlinechar=-1 %
 \gdef\xComment#1^^M{\def\test{#1}
      \csarg\ifx{PlainEnd\CurrentComment Test}\test
          \let\html@next\endgroup
      \else \csarg\ifx{LaLaEnd\CurrentComment Test}\test
            \edef\html@next{\endgroup\noexpand\end{\CurrentComment}}
      \else \let\html@next\xComment
      \fi \fi \html@next}
}
\makeatother

\def\includecomment
 #1{\expandafter\def\csname#1\endcsname{}%
    \expandafter\def\csname end#1\endcsname{}}
\def\excludecomment
 #1{\expandafter\def\csname#1\endcsname{\ThrowAwayComment{#1}}%
    {\escapechar=-1\relax
     \csarg\xdef{PlainEnd#1Test}{\string\\end#1}%
     \csarg\xdef{LaLaEnd#1Test}{\string\\end\string\{#1\string\}}%
    }}

%  Define environments that ignore their contents.
\excludecomment{comment}
\excludecomment{rawhtml}
\excludecomment{htmlonly}
%\end{rawtex}

%  Hypertext commands etc. This is a condensed version of the html.sty
%  file supplied with LaTeX2HTML by: Nikos Drakos <nikos@cbl.leeds.ac.uk> &
%  Jelle van Zeijl <jvzeijl@isou17.estec.esa.nl>. The LaTeX2HTML documentation
%  should be consulted about all commands (and the environments defined above)
%  except \xref and \xlabel which are Starlink specific.

\newcommand{\htmladdnormallinkfoot}[2]{#1\footnote{#2}}
\newcommand{\htmladdnormallink}[2]{#1}
\newcommand{\htmladdimg}[1]{}
\newenvironment{latexonly}{}{}
\newcommand{\hyperref}[4]{#2\ref{#4}#3}
\newcommand{\htmlref}[2]{#1}
\newcommand{\htmlimage}[1]{}
\newcommand{\htmladdtonavigation}[1]{}

%  Starlink cross-references and labels.
\newcommand{\xref}[3]{#1}
\newcommand{\xlabel}[1]{}

%  LaTeX2HTML symbol.
\newcommand{\latextohtml}{{\bfseries LaTeX}{2}{\texttt{HTML}}}

%  Define command to re-centre underscore for Latex and leave as normal
%  for HTML (severe problems with \_ in tabbing environments and \_\_
%  generally otherwise).
\newcommand{\latex}[1]{#1}
\newcommand{\setunderscore}{\renewcommand{\_}{{\texttt{\symbol{95}}}}}
\latex{\setunderscore}

%  Redefine the \tableofcontents command. This procrastination is necessary 
%  to stop the automatic creation of a second table of contents page
%  by latex2html.
\newcommand{\latexonlytoc}[0]{\tableofcontents}

% -----------------------------------------------------------------------------
%  Debugging.
%  =========
%  Remove % on the following to debug links in the HTML version using Latex.

% \newcommand{\hotlink}[2]{\fbox{\begin{tabular}[t]{@{}c@{}}#1\\\hline{\footnotesize #2}\end{tabular}}}
% \renewcommand{\htmladdnormallinkfoot}[2]{\hotlink{#1}{#2}}
% \renewcommand{\htmladdnormallink}[2]{\hotlink{#1}{#2}}
% \renewcommand{\hyperref}[4]{\hotlink{#1}{\S\ref{#4}}}
% \renewcommand{\htmlref}[2]{\hotlink{#1}{\S\ref{#2}}}
% \renewcommand{\xref}[3]{\hotlink{#1}{#2 -- #3}}
% -----------------------------------------------------------------------------
% ? Document specific \newcommand or \newenvironment commands.
\newcommand{\dash}{--}
\begin{htmlonly}
\renewcommand{\dash}{-}
\end{htmlonly}
\newcommand{\STARURL}{http://www.starlink.ac.uk}
\newcommand{\IRAFURL}{http://www.starlink.ac.uk/iraf/web/iraf-homepage.html}
\newcommand{\TCLURL}{http://sunscript.sun.com/tcltext.html}
\newcommand{\STORE}{http://www.starlink.ac.uk/cgi-bin/storetop}
% ? End of document specific commands
% -----------------------------------------------------------------------------
%  Title Page.
%  ===========
\renewcommand{\thepage}{\roman{page}}
\begin{document}
\thispagestyle{empty}

%  Latex document header.
%  ======================
\begin{latexonly}
   CCLRC / {\sc Rutherford Appleton Laboratory} \hfill {\bfseries \stardocname}\\
   {\large Particle Physics \& Astronomy Research Council}\\
   {\large Starlink Project\\}
   {\large \stardoccategory\ \stardocnumber}
   \begin{flushright}
   \stardocauthors\\
   \stardocdate
   \end{flushright}
   \vspace{-4mm}
   \rule{\textwidth}{0.5mm}
   \vspace{5mm}
   \begin{center}
   {\Huge\bfseries  \stardoctitle \\ [2.5ex]}
%   {\LARGE\bfseries \stardocversion \\ [4ex]}
   {\Huge\bfseries  \stardocmanual}
   \end{center}
   \vspace{5mm}

% ? Heading for abstract if used.
   \vspace{10mm}
   \begin{center}
      {\Large\bfseries Abstract}
   \end{center}
% ? End of heading for abstract.
\end{latexonly}

%  HTML documentation header.
%  ==========================
\begin{htmlonly}
   \xlabel{}
   \begin{rawhtml} <H1> \end{rawhtml}
      \stardoctitle\\
%      \stardocversion\\
      \stardocmanual
   \begin{rawhtml} </H1> \end{rawhtml}

% ? Add picture here if required.
% ? End of picture

   \begin{rawhtml} <P> <I> \end{rawhtml}
   \stardoccategory \stardocnumber \\
   \stardocauthors \\
   \stardocdate
   \begin{rawhtml} </I> </P> <H3> \end{rawhtml}
      \htmladdnormallink{CCLRC}{http://www.cclrc.ac.uk} /
      \htmladdnormallink{Rutherford Appleton Laboratory}
                        {http://www.cclrc.ac.uk/ral} \\
      \htmladdnormallink{Particle Physics \& Astronomy Research Council}
                        {http://www.pparc.ac.uk} \\
   \begin{rawhtml} </H3> <H2> \end{rawhtml}
      \htmladdnormallink{Starlink Project}{http://www.starlink.ac.uk/}
   \begin{rawhtml} </H2> \end{rawhtml}
   \htmladdnormallink{\htmladdimg{source.gif} Retrieve hardcopy}
      {http://www.starlink.ac.uk/cgi-bin/hcserver?\stardocsource}\\

%  HTML document table of contents. 
%  ================================
%  Add table of contents header and a navigation button to return to this 
%  point in the document (this should always go before the abstract \section). 
  \label{stardoccontents}
  \begin{rawhtml} 
    <HR>
    <H2>Contents</H2>
  \end{rawhtml}
  \renewcommand{\latexonlytoc}[0]{}
  \htmladdtonavigation{\htmlref{\htmladdimg{contents_motif.gif}}
        {stardoccontents}}

% ? New section for abstract if used.
  \section{\xlabel{abstract}Abstract}
% ? End of new section for abstract
\end{htmlonly}

% -----------------------------------------------------------------------------
% ? Document Abstract. (if used)
%  ==================
\stardocabstract
% ? End of document abstract
% -----------------------------------------------------------------------------
% ? Latex document Table of Contents (if used).
%  ===========================================
 \newpage
 \begin{latexonly}
   \setlength{\parskip}{0mm}
   \latexonlytoc
   \setlength{\parskip}{\medskipamount}
   \markboth{\stardocname}{\stardocname}
 \end{latexonly}
% ? End of Latex document table of contents
% -----------------------------------------------------------------------------
\cleardoublepage
\renewcommand{\thepage}{\arabic{page}}
\setcounter{page}{1}

\section{\xlabel{introduction}Introduction}
The UK 
\htmladdnormallink{Starlink Project}{\STARURL} 
provides a collection of software to reduce and analyse astronomical data.
\latexonly{Full details of the project and software available can be found at
our web site, \texttt{http://www.starlink.ac.uk}.}
The software is described in a series of Starlink User Notes (SUN's), Starlink
System Notes (SSN's), Starlink Guides (SG's) and Starlink Cookbooks (SC's),
available in hardcopy or hypertext form.

For some time 
Starlink application programs have been able to
optionally read and write IRAF image ({\texttt{.imh}}) or FITS files instead 
of the standard Starlink
\xref{Extensible N-Dimensional Data Format (NDF)}{sun33}{} files.
Now a system has been developed to enable Starlink applications to be
run from the 
\htmladdnormallink{IRAF}{\IRAFURL}
CL, allowing applications written for the two different environments to be
intermixed easily.

The intention is to make Starlink applications appear as much like normal
IRAF applications as possible but there are some differences which are discussed
in this document.
The system is still under development and we would be interested to hear of
any problems or comments you may have. Please e-mail to The Starlink Software
Librarian (\texttt{ussc@star.rl.ac.uk}).

\section{\xlabel{setting_up}Setting Up}
At Starlink sites, the Site Manager will have installed (or can be requested 
to install) the necessary software and all required environment variables are 
set when the \texttt{/star/etc/login} file is `sourced' \dash\ the normal way 
for running any Starlink software. 
For other sites, consult the
\htmlref{installation instructions}{installing}\latexonly{ in Appendix
\ref{installing}} for advice on obtaining and installing the software.

\section{\xlabel{basic_usage}Basic Usage}
Assuming that the software has been set up in the standard way and CL started,
the Starlink package is entered by typing its name (in lower case). The welcome
message and menu of commands is then displayed
For example:
\begin{quote} \begin{verbatim}
cl> figaro

--------------------------------------------------
          Welcome to Figaro for IRAF
            Version 5.2-1 - 24Sep97
      For support contact ussc@star.rl.ac.uk
--------------------------------------------------

      abconv      cset        fscrunch    image       lyset       slice
      abline      cspike      fwconv      imult       mask        soft
      adjoin      delobj      gauss       interp      maskext     spflux

                           etc.. 
fi>
\end{verbatim} \end{quote}
and applications may be run.

Note that the Figaro package is used here to give examples of running Starlink
application programs from IRAF CL. This is not intended as an introduction to
the wealth of applications available in Figaro \dash\ for that, consult
\xref{SUN/86}{sun86}{}
or, if you have Figaro installed, run the Figaro demonstration:
\begin{quote} \begin{verbatim}
cl> task $figdemo = figaro$figdemo.cl
cl> figdemo
\end{verbatim} \end{quote}

Starlink applications tend to have more parameters than IRAF applications but
many of them will never require altering by the user.
In most cases parameters will be handled as expected but, because of the 
differences between the Starlink and IRAF parameter systems, there are a few 
cases where the handling of parameters may surprise IRAF users. 
These are discussed in the following sections.

\subsection{\xlabel{data_format_conversion}Data Format Conversion}
As a simple example, Figaro contains an application \texttt{istat} which 
provides statistics on an image file.
\begin{quote} \begin{verbatim}
fi> istat comwest

Y-range 1 to 256
X-range 1 to 256
Total (over 65536.0 pixels) = 1.1196E+7
Max   = 245.94   in pixel (248,45)
Min   = 3.8906   in pixel (59,83)
Mean  = 170.84
Sigma = 63.4732

fi>
\end{verbatim} \end{quote}

Unless there is a genuine NDF ({\texttt{comwest.sdf}}) present, the application 
will attempt to open an IRAF image file, \texttt{comwest.imh} (or failing that
a FITS file, \texttt{comwest.fit} or \texttt{comwest.fits}), use an
\htmlref{automatic conversion process}{more_on_automatic_data_conversion} 
to convert it to a temporary NDF file and process that to provide the 
statistics.

Where applications create image files, IRAF image files will be created by 
default.
\latexonly{The automatic data conversion is discussed more in Section 
\ref{more_on_automatic_data_conversion}.}

Note that filenames must be specified explicitly for Starlink applications,
IRAF filename templates cannot be used.
Furthermore, only IRAF short-integer and real datatype images can be handled.

\subsection{\xlabel{data_subsets}Data Subsets}
You can specify a subset of an image as input. For example:
\begin{quote} \begin{verbatim}
fi> istat comwest[50:200,100:250]

Y-range 1 to 151
X-range 1 to 151
Total (over 22801 pixels) = 3.6895E+6
Max   = 236.58   in pixel (8,150)
Min   = 12.813   in pixel (28,1)
Mean  = 161.81
Sigma = 62.4827

fi>
\end{verbatim} \end{quote}
\emph{Note that the subset indices must be 
\xref{NDF-type indices}{sun33}{using_subscripts_to_access_ndf_sections}
\latexonly{, as described in SUN/33,} but enclosed in \texttt{[]} rather than 
\texttt{()}.}

In the Starlink notation, omitting any of the bounds from the subscript 
expression will cause the appropriate (lower or upper) bound of the NDF to be 
used instead. 
If the separating `:' is also omitted, then the lower and upper bounds of the 
section will both be set to the same value, so that a single pixel will be 
selected for that dimension. Omitting the bounds entirely for a dimension (but
still retaining the comma) will cause the entire extent of that dimension to 
be used.
Thus, 
\begin{quote} \begin{verbatim}
image[50:,64]
\end{verbatim} \end{quote}
could be used to specify pixel 50 onwards of line 64 of a 2-dimensional image,
and
\begin{quote} \begin{verbatim}
image[,64]
\end{verbatim} \end{quote}
would specify the whole of line 64.
\emph{Note that \texttt{*} is not used to mean the whole extent}.

In addition to the lower and upper bounds form in the example, the form
\verb!c~e! will specify the centre and extent of a subset. For example:
\begin{quote} \begin{verbatim}
name[100~11,200~5] 
\end{verbatim} \end{quote}
would refer to an 11 $\times$ 5 pixel region of an image centred on pixel 
(100, 200). 
The two forms can be combined in one subset specification.

\subsection{\xlabel{dynamic_parameters}Dynamic Parameters}
The \texttt{istat} parameters \texttt{ystart}, \texttt{yend}, \texttt{xstart} 
and \texttt{xend} provide an alternative way of specifying a subset of an image
so you could analyse the subset \texttt{comwest[50:200,100:250]} equally well 
by:
\begin{quote} \begin{verbatim}
fi> istat comwest xstart=50 xend=200 ystart=100 yend=250

Y-range 100 to 250
X-range 50 to 200
Total (over 22801 pixels) = 3.6895E+6
Max   = 236.58   in pixel (57,249)
Min   = 12.813   in pixel (77,100)
Mean  = 161.81
Sigma = 62.4827

fi>
\end{verbatim} \end{quote}
(but note that the pixels indices here are relative to the whole image whereas 
in the previous example they were relative to the subset).

However, these parameters will usually require to be set to the lower and
upper bounds for the particular image \dash\ values determined by the application
at runtime \dash\ so they have to be treated as so-called
\htmlref{`dynamic parameters'}{more_on_dynamic_parameters}. 
If a value is not specified on the command line, dynamic parameters
will take a value suggested by the program at runtime (in this
case the lower and upper bounds for the image) without reference to the value 
of the corresponding IRAF parameter. Editing the values with \texttt{eparam} 
will have no effect but the IRAF parameter will be set to the value actually 
used.

A number of Figaro applications have these start and end of image parameters 
and they are always treated in the same way.
\latexonly{Dynamic parameters are discussed more in Section 
\ref{more_on_dynamic_parameters}.} 

\subsection{\xlabel{output_parameters}Output Parameters}
Starlink applications sometimes make use of output parameters. 
The values of the corresponding IRAF parameters will be updated after the 
application has run. For example, \texttt{istat} writes the values calculated 
to output parameters as well as displaying them.
\begin{quote} \begin{verbatim}
fi> istat comwest

Y-range 1 to 256
X-range 1 to 256
Total (over 65536.0 pixels) = 1.1196E+7
Max   = 245.94   in pixel (248,45)
Min   = 3.8906   in pixel (59,83)
Mean  = 170.84
Sigma = 63.4732

fi> =istat.stat_max
245.9375
fi> =istat.stat_min
3.890625
fi>
\end{verbatim} \end{quote}
The documentation for applications will list their output parameters.

\subsection{\xlabel{graphical_output}Graphical Output}
For graphical and image display, Starlink packages do not use the normal IRAF
\texttt{stdimage} with the \texttt{imtool} or \texttt{saoimage} systems. 
Instead, the name of the desired device is given as a parameter to the 
application using a name from the Starlink
\xref{Graphics Name Service}{sun57}{}\latexonly{ (see SUN/57)}.

Figaro provides the \texttt{soft} task to open a device and specify it as the
default to be used by subsequent Figaro applications. So for example:
\begin{quote} \begin{verbatim}
fi> soft xw
\end{verbatim} \end{quote}
specifies an X-windows device. This will cause a window to be opened under the
control of the Starlink
\xref{TCL/GWM Xdisplay Widget}{sun219}{}
so that it can be used by other applications.
\latexonly{ (TCL/GWM is described in SUN/219.)}

You can override the device specified by the \texttt{soft} task for subsequent
applications by specifying a device parameter on the command line but not
by editing in \texttt{eparam} (this is another example of a 
\htmlref{`dynamic'}{dynamic_parameters}
parameter).

You can get a complete list of the devices available by using 
\texttt{soft} as follows:
\begin{quote} \begin{verbatim}
fi> soft options+

Possible devices are:

mfile_input Metafile input
..
..
xwindows   X-windows
x2windows  X-windows
..
xoverlay   X-overlay
x2overlay  X-overlay
..

(SOftdev) Device to use for soft plots (xw):
fi>
\end{verbatim} \end{quote}
Note that the names can be abbreviated to an unambiguous value so that 
\texttt{xw} is short for \texttt{xwindows}.

\subsection{\xlabel{boolean_parameters}Boolean Parameters}
The \texttt{options} parameter of \texttt{soft} is an example of using a 
boolean parameter.
When running from IRAF CL, the values \texttt{yes} and \texttt{no} can be 
given and, on the command line, \texttt{parname+} and \texttt{parname-} may be
used as equivalents for \texttt{parname=yes} and \texttt{parname=no} in the 
normal way.

However, in the Starlink environment, boolean parameters are known as logical
parameters and may take the values \texttt{true} or \texttt{false} as 
equivalents to \texttt{yes} or \texttt{no}. 
The single letters \texttt{t}, \texttt{f}, \texttt{y} or \texttt{n} are also 
accepted and case is not significant. 
For logical parameters on the Starlink command line, 
\texttt{parname} is equivalent to \texttt{parname=yes} and 
\texttt{noparname} is equivalent to \texttt{parname=no}.
This may be reflected in help text and messages or prompts but the IRAF rules
must be applied when running from IRAF CL.

\subsection{\xlabel{array_parameters}Array Parameters}
The IRAF parameter corresponding to a Starlink vector or array parameter is 
defined as type struct and its value must be given in the form supported by
Starlink, \textit{e.g.}\ a 2 $\times$ 2 array might be given as 
\texttt{[[1,2][3,4]]}.
The outermost \texttt{[]} may be omitted and elements can be
separated by space or comma but, if separated by space on the command line,
the whole lot must be enclosed in \texttt{""}.

The Figaro contouring program \texttt{icont}, provides 
an example of using a vector parameter (and the default graphics device set up
by \texttt{soft}).

If you want to specify the contour levels for \texttt{icont}, set
parameter \texttt{byvalue} to \texttt{yes}, specify the number of contours 
and give the required contour levels as a 1-D array. You will be prompted for
other parameters which \texttt{icont} requires.
\begin{quote} \begin{verbatim}
fi> icont comwest byvalue=yes contours=4 levels="50 100 150 200"
(ADjust) Adjust scales so as to fill display? (yes):
(HArdcopy) Generate a hardcopy plot? (no):
(LABel) Label for plot: comwest
fi>
\end{verbatim} \end{quote}
A contour plot of the \texttt{comwest} image will be displayed on the default 
graphics device.

Don't worry about the capitalization of the first few characters of the 
parameter names in the prompts \dash\ this is historical from an early system
of parameter name abbreviation. You can use the IRAF system of specifying the
minimum unambiguous string. For example:
\begin{quote} \begin{verbatim}
icont comwest b=yes c=4 le="50 100 150 200" a=yes hardc=no la=comwest
\end{verbatim} \end{quote}
would achieve the same result as the previous example.

\subsection{\xlabel{the_indef_and_null_values}\label{indef_and_null}The INDEF and NULL Values}
The Starlink parameter system allows parameters to be defined as NULL 
(specified by a parameter value of \texttt{!}). 
References to NULL and/or \texttt{!}\ may occur, particularly in error messages.

IRAF will not in general accept \texttt{!}\ as a parameter value but if INDEF 
is specified, it will get translated to \texttt{!}\ before being passed to the 
Starlink application, thus setting the Starlink parameter to NULL. 
The precise effect of this depends upon the application but it is sometimes
used to terminate the program and will often be treated as an error.

For example, if having got a list of available graphics devices from
\texttt{soft}, you do not want to open a device, you could reply with INDEF to
the prompt and will get message as follows:
\begin{quote} \begin{verbatim}
fi> soft options+
   ..
   ..

(SOftdev) Device to use for soft plots (xw): INDEF
 SUBPAR: Null (!) response to prompt for parameter SOFTDEV
fi>
\end{verbatim} \end{quote}

\subsection{\xlabel{error_messages}\label{error_messages}Error Messages}
Error messages from Starlink applications are often of the form:
\begin{quote}
\textit{subsystem}\texttt{:} \textit{message}
\end{quote}
where \textit{subsystem}
indicates where the error was detected. For example, \texttt{NDF\_OPEN:}
indicates that the subroutine NDF\_OPEN detected the error; \texttt{SUBPAR:}
indicates that the Starlink parameter system detected the error. There will
often be a string of messages giving more precise information about what 
happened.

\subsection{\xlabel{help}Help}
Help is available for the Starlink applications using the \texttt{help} command
in the normal way but in some cases the text may be oriented toward Starlink 
usage with use of NULL \textit{etc.}\ as
\htmlref{already described}{indef_and_null}. 
Similarly, displayed messages, particularly error messages, may appear slightly 
strange to users familiar with IRAF applications but we hope that their meaning 
will be obvious. 

Please contact The Starlink Software Librarian (\texttt{ussc@star.rl.ac.uk}) if 
you have any problems.

\section{\xlabel{for_starlink_users}For Starlink Users}
If you are familiar with running applications in the Starlink environment, you
should note that a number of commonly-used facilities are not available when 
running from IRAF CL.
\begin{description}
\item[Special Keywords] The special keywords, PROMPT, RESET and ACCEPT 
(\verb!\!) may not be given on the command line, nor ACCEPT (\verb!\!) in 
response to a prompt.
IRAF has its own ways of achieving similar effects.
\item[MIN/MAX] `MIN' and `MAX' may not be specified as parameter values.
Range information from the Starlink interface file will be inserted in the
min and max positions for the corresponding IRAF parameter but there can be
no other interaction between the two systems.
\item[Parameter Help] The \texttt{?}\ and \texttt{??}\ responses may not be 
given to obtain help at parameter prompts.
\item[Missing Applications] Some applications may not have been made available
for IRAF. In particular, the help applications such as \texttt{FIGHELP} and 
\texttt{KAPHELP} are likely to be missing.
\item[No VPATH] The Starlink \texttt{VPATH} mechanism is overridden so that 
values are likely to be requested in IRAF which were not requested in the 
Starlink environment. Furthermore, the default values may be different.
\item[ABORT ({\texttt{!!}})] The abort response ({\texttt{!!}}) may not be 
given in response to a parameter prompt. 
If applications rely on this then specific solutions will be required when 
running from IRAF \dash\ consult the package documentation.
\end{description}

\section{\xlabel{more_on_automatic_data_conversion}\label{more_on_automatic_data_conversion}More on Automatic Data Conversion}
The standard data format for Starlink applications is the 
\xref{Extensible N-Dimensional Data Format (NDF)}{sun33}{}
and the library of routines for handling NDFs has built-in facilities for
converting between NDFs and other data formats (notably IRAF \texttt{.imh} 
format and FITS) `on-the-fly'. 
Applications using the NDF library can therefore read and write data files in 
the foreign formats without a separate conversion step.
The NDF conversion facilities are controlled by setting environment variables
and are described fully in \xref{SSN/20}{ssn20}{}.

\subsection{\xlabel{default_file_types}Default File Types}
The default situation for Starlink applications running from IRAF CL is that
given an input NDF filename, \textit{filename}, they will attempt to open a 
genuine NDF file, \textit{filename}{\texttt{.sdf}}; failing that they will 
attempt to open and convert an IRAF image file, 
\textit{filename}{\texttt{.imh}} and, failing that, a FITS file, 
\textit{filename}{\texttt{.fit}} or \textit{filename}{\texttt{.fits}}. 

For output NDFs, the default is to produce IRAF image files automatically but
you can change this with the command:
\begin{quote} \begin{verbatim}
fi> use_ndf
\end{verbatim} \end{quote}
or
\begin{quote} \begin{verbatim}
fi> use_fits
\end{verbatim} \end{quote}
(\texttt{use\_fits} will produce a file with extension \texttt{.fit}.)

You can change back to producing IRAF \texttt{.imh} files with:
\begin{quote} \begin{verbatim}
fi> use_imh
\end{verbatim} \end{quote}

The system will attempt to delete any existing file with the same name as one
to be output.

The default values of the environment variables associated with the automatic
conversion system are set when the first Starlink package is loaded and will
not be reset by loading other packages.

\subsection{\xlabel{using_explicit_file_extensions}Using Explicit File Extensions}
You can also specify filetypes (other than NDF) by giving a recognised 
extension on the filename. For example, if \texttt{wrndf} is a program which
creates an NDF named by the `out' parameter, under the normal default conditions
\begin{quote} \begin{verbatim}
fi> wrndf out=file
\end{verbatim} \end{quote}
will produce an IRAF image file, \texttt{file.imh}, while
\begin{quote} \begin{verbatim}
fi> wrndf out=file.fits
\end{verbatim} \end{quote}
will produce a FITS file, \texttt{file.fits}.

\subsection{\xlabel{some_pitfalls}Some Pitfalls}
You should be aware that a few applications do not use the NDF library so 
this conversion is not always possible. 
Furthermore, some applications require or write components of NDFs which do 
not exist in other formats and so get lost or are not available in the 
conversion process. 
In particular quality, variance and axis information may not be propagated
correctly.

There is some discussion of these matters, with some possible solutions, under
\xref{Data Extensibility}{ssn20}{general_principles_of_extensibility}
in SSN/20.

Where components are lost in converting from NDF, you may be able
to string together a sequence of Starlink applications starting and 
ending with IRAF-format files but using NDFs for intermediate processing.

By default, conversion is done by the appropriate Starlink applications from 
the
\xref{CONVERT}{sun55}{}
package\latexonly{ described in SUN/55} \dash\ the default values of their 
parameters are used (consult their documentation for more details). 
If this does not suit your requirements, you can change the behaviour by 
resetting the appropriate environment variables as detailed in
\xref{SSN/20}{ssn20}{}.

Also note that currently the conversion process with
\xref{\texttt{iraf2ndf}}{sun55}{IRAF2NDF}
and
\xref{\texttt{ndf2iraf}}{sun55}{NDF2IRAF},
is subject to the limitations of the IRAF IMFORT library, \textit{i.e.}\ only 
IRAF short-integer and real datatype images can be handled.

\section{\xlabel{more_on_dynamic_parameters}\label{more_on_dynamic_parameters}More on Dynamic Parameters}
The Starlink software environment allows parameter default values to be
calculated at runtime by the application or to be set to the value of
`global' parameters set by other applications. 
For correct operation of the applications
it has been necessary for some parameters to take these default values without
reference to the corresponding IRAF parameter value. 
In most cases you will not want to specify a value for these parameters
anyway but the default can always be overridden by including the parameter on 
the command line (but not by editing the parameter file with \texttt{eparam}). 
In all cases the IRAF parameter value will be updated to the value actually
used. 

It is therefore possible that you might set a value in \texttt{eparam} but the
application uses a different value. To minimise this risk, the prompt string 
as displayed by \texttt{eparam} and \texttt{lparam} will be prefixed with 
`{\texttt{*!\ }}' to indicate that the parameter is dynamic.

\section{\xlabel{final_note}Final Note}
Users of Starlink applications from IRAF need not be too concerned with the
details of how the system works. If you are interested, it was described by 
Dave Terrett in
\htmladdnormallink{Starlink Bulletin 16}
{http://www.starlink.ac.uk/bulletin/95dec/95dec.html}
and the way in which packages are set up is described in
\xref{SSN/35}{ssn35}{}.
It is probably sufficient to know that communications between CL and the task
are intercepted by an adaptor process which does such things as changing data
values between IRAF and Starlink (ADAM) format, and handling parameters in
an appropriate way before passing the messages on.  


\appendix
\newpage
\section{\xlabel{installing_the_starlink_software}Installing the Starlink Software\label{installing}}
\subsection{\xlabel{general}General}
No special privileges are required to install Starlink software. Installation
is a two-stage process. First the software item is built in a source directory
and then the built files are moved (`installed' in Starlink parlance) to 
directories beneath a top-level Starlink installation directory using 
\texttt{make}. 
At Starlink sites, this top-level directory is \texttt{/star} but it can be 
anywhere, defined for \texttt{make} by the environment variable 
\texttt{INSTALL}.

\emph{ N.B. `\texttt{/star}' is used in the examples below to indicate this 
top-level \texttt{INSTALL} directory and should be substituted appropriately.}

Makefiles are provided with targets \texttt{build}, \texttt{install}, 
\texttt{unbuild}, \texttt{deinstall} \textit{etc.}\ but software will usually 
be obtained in a ready-built form, probably via the
\htmladdnormallink{Starlink Software Store}{\STORE}\latexonly{\footnote{\STORE}}
which handles the whole installation process with just a few prompts to the
user. The Software Store also handles the problem of knowing which items
of Starlink infrastructure are required to support a particular application
and bundles them together for delivery. These installation instructions assume
that you are obtaining ready-built systems.

\subsection{\xlabel{starlink_application_packages}Starlink Application Packages}
This document is written assuming that the Starlink applications to be run are
part of a larger package, Figaro or KAPPA for example.

The IRAF-special files associated with Starlink packages (the 
\textit{package}{\texttt{.cl}} and \texttt{.par} files \textit{etc.}) will be 
installed in directory \texttt{/star/iraf/{\textit{package}}}, where 
\textit{package} is the package (Software Item) name in lower case.

In the long run these files will be part of the normal Starlink installation
of the package, but until full integration is done they may be released as
a separate item.

\subsection{\xlabel{other_required_software_items}Other Required Software Items}
Most Starlink application packages will require certain other Starlink software
items to be installed. For ease of installation and operation, all Starlink 
software should be installed beneath the same top-level \texttt{INSTALL} 
directory.

The required items are:
\begin{description}
\item[\xref{IRAFSTAR}{ssn35}{}] The IRAF/Starlink inter-operability 
infrastructure\latexonly{ (see SSN/35)}. IRAFSTAR
will be required by any Starlink application being run from IRAF CL but does
not include any applications itself. 
\item[\xref{TCLSYS}{sun200}{}] This is the Starlink distribution of 
\htmladdnormallink{Tcl}{\TCLURL}
and associated items\latexonly{ (see SUN/200)}. 
Tcl is used by IRAFSTAR which expects it to be the Starlink distribution
installed in \texttt{/star}. 
If you already have Tcl installed elsewhere, you may not need this Starlink 
item, but you will need to check the versions and will have to set the 
TCL\_LIBRARY environment variable appropriately.
\item[\xref{STARTCL}{sun186}{}] Starlink Tcl extensions \dash\ includes the
TCL/GWM Xdisplay Widget\latexonly{ (see SUN/186)}.
\item[RUNTIME] Miscellaneous items of software required by most applications
at runtime.
\item[\xref{CONVERT}{sun55}{}] The data-conversion software required if the 
automatic data-conversion facilities are to be used\latexonly{ (see SUN/55)}.
\item[\xref{HTX}{sun188}{}] Document cross-referencing software required during 
installation of Starlink software\latexonly{ (see SUN/188)}.
\end{description}

\subsection{\xlabel{making_starlink_known_to_iraf}Making Starlink Known to IRAF}
The preferred way of defining new external packages for IRAF is to edit
the \texttt{hlib\$extern.pkg} file. To define Starlink packages, add entries as 
follows:
\begin{quote} \begin{verbatim}
reset   starlink            = /star/iraf/
\end{verbatim} \end{quote}
and add `{\texttt{,starlink\$irafstar/helpdb.mip}}' to the \texttt{helpdb} 
definition.

Then add definitions for the Starlink applications packages in use. 
For example:
\begin{quote} \begin{verbatim}
reset   figaro            = starlink$figaro/
task    figaro.pkg        = figaro$figaro.cl
\end{verbatim} \end{quote}
and add `{\texttt{,figaro\$helpdb.mip}}' to the \texttt{helpdb} definition.

If you do not have access to the \texttt{hlib\$extern.pkg} file, the above 
commands can be added to your \texttt{login.cl} file. In that case, the
\texttt{helpdb} definition can be extended by a command like:
\begin{quote} \begin{verbatim}
reset helpdb = (envget("helpdb") // ",figaro$helpdb.mip")
\end{verbatim} \end{quote}

\subsection{\xlabel{environment_variables}Environment Variables}
Certain environment variables must be set for the inter-operability system to 
run correctly. At Starlink sites all this will be done by 
\begin{quote} \begin{verbatim}
% source /star/etc/login
\end{verbatim} \end{quote}
normally done from your \texttt{.login} file.

If you have to set them yourself, remember that \texttt{/star} here should be 
replaced by the top-level Starlink installation directory at your site.

\begin{description}
\item[PATH] The Starlink binaries directories must be added to your 
\texttt{PATH}.
\begin{quote} \begin{verbatim}
% setenv PATH ${PATH}:/star/bin/startcl:/star/bin
\end{verbatim} \end{quote}

\item[TCL\_LIBRARY] must be set to the directory containing the 
architecture-independent Tcl files for your installation.
\begin{quote} \begin{verbatim}
% setenv TCL_LIBRARY /star/lib/tcl7.6
\end{verbatim} \end{quote}
This step is unnecessary if your \texttt{/star} directory really is 
\texttt{/star} 
and you are using the Starlink distribution of Tcl.

\item[CONVERT\_DIR] must be set to the binaries directory of
the CONVERT package.
\begin{quote} \begin{verbatim}
% setenv CONVERT_DIR /star/bin/convert
\end{verbatim} \end{quote}

\item[ADAM\_USER] which defines where the Starlink environment 
parameter files will be created, is set to \texttt{home\$uparm} within the 
package startup scripts.

\end{description}

\end{document}
