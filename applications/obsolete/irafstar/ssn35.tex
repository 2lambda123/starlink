\documentclass[twoside,11pt]{article}
\pagestyle{myheadings}

% -----------------------------------------------------------------------------
% ? Document identification
% Fixed part
\newcommand{\stardoccategory}  {Starlink System Note}
\newcommand{\stardocinitials}  {SSN}
\newcommand{\stardocsource}    {ssn\stardocnumber}

% Variable part
\newcommand{\stardocnumber}    {35.3}
\newcommand{\stardocauthors}   {A J Chipperfield\\D L Terrett}
\newcommand{\stardocdate}      {14 October 1997}
\newcommand{\stardoctitle}     {IRAFSTAR\\[1ex]
                             The IRAF/Starlink Inter-operability Infrastructure}
\newcommand{\stardocversion}   {1.2}
\newcommand{\stardocmanual}    {}
\newcommand{\stardocabstract}  {
IRAFSTAR contains the infrastructure which enables applications written for
the
\xref{Starlink Software Environment (ADAM)}{sg4}{} to be run in the
\htmladdnormallink{IRAF}{\IRAFURL} software environment.
IRAFSTAR does not include any applications itself.
\par
Communications between the application and IRAF are handled by an `adaptor'
process and conversion between Starlink NDF and IRAF image or FITS files may
be done transparently, making it easy to intermix IRAF and Starlink
applications.
\par
This document is intended for software developers and describes the system
and the additional files required by Starlink applications to be used with it.
Some knowledge of both ADAM and IRAF is assumed.
\par
Users who just want to run to run Starlink applications
in IRAF should read
\xref{SUN/217}{sun217}{}
rather than this document.}
% ? End of document identification
% -----------------------------------------------------------------------------


\newcommand{\stardocname}{\stardocinitials /\stardocnumber}
\markboth{\stardocname}{\stardocname}
\setlength{\textwidth}{160mm}
\setlength{\textheight}{230mm}
\setlength{\topmargin}{-2mm}
\setlength{\oddsidemargin}{0mm}
\setlength{\evensidemargin}{0mm}
\setlength{\parindent}{0mm}
\setlength{\parskip}{\medskipamount}
\setlength{\unitlength}{1mm}

% -----------------------------------------------------------------------------
%  Hypertext definitions.
%  ======================
%  These are used by the LaTeX2HTML translator in conjunction with star2html.

%  Comment.sty: version 2.0, 19 June 1992
%  Selectively in/exclude pieces of text.
%
%  Author
%    Victor Eijkhout                                      <eijkhout@cs.utk.edu>
%    Department of Computer Science
%    University Tennessee at Knoxville
%    104 Ayres Hall
%    Knoxville, TN 37996
%    USA

%  Do not remove the %\begin{rawtex} and %\end{rawtex} lines (used by
%  star2html to signify raw TeX that latex2html cannot process).
%\begin{rawtex}
\makeatletter
\def\makeinnocent#1{\catcode`#1=12 }
\def\csarg#1#2{\expandafter#1\csname#2\endcsname}

\def\ThrowAwayComment#1{\begingroup
    \def\CurrentComment{#1}%
    \let\do\makeinnocent \dospecials
    \makeinnocent\^^L% and whatever other special cases
    \endlinechar`\^^M \catcode`\^^M=12 \xComment}
{\catcode`\^^M=12 \endlinechar=-1 %
 \gdef\xComment#1^^M{\def\test{#1}
      \csarg\ifx{PlainEnd\CurrentComment Test}\test
          \let\html@next\endgroup
      \else \csarg\ifx{LaLaEnd\CurrentComment Test}\test
            \edef\html@next{\endgroup\noexpand\end{\CurrentComment}}
      \else \let\html@next\xComment
      \fi \fi \html@next}
}
\makeatother

\def\includecomment
 #1{\expandafter\def\csname#1\endcsname{}%
    \expandafter\def\csname end#1\endcsname{}}
\def\excludecomment
 #1{\expandafter\def\csname#1\endcsname{\ThrowAwayComment{#1}}%
    {\escapechar=-1\relax
     \csarg\xdef{PlainEnd#1Test}{\string\\end#1}%
     \csarg\xdef{LaLaEnd#1Test}{\string\\end\string\{#1\string\}}%
    }}

%  Define environments that ignore their contents.
\excludecomment{comment}
\excludecomment{rawhtml}
\excludecomment{htmlonly}
%\end{rawtex}

%  Hypertext commands etc. This is a condensed version of the html.sty
%  file supplied with LaTeX2HTML by: Nikos Drakos <nikos@cbl.leeds.ac.uk> &
%  Jelle van Zeijl <jvzeijl@isou17.estec.esa.nl>. The LaTeX2HTML documentation
%  should be consulted about all commands (and the environments defined above)
%  except \xref and \xlabel which are Starlink specific.

\newcommand{\htmladdnormallinkfoot}[2]{#1\footnote{#2}}
\newcommand{\htmladdnormallink}[2]{#1}
\newcommand{\htmladdimg}[1]{}
\newenvironment{latexonly}{}{}
\newcommand{\hyperref}[4]{#2\ref{#4}#3}
\newcommand{\htmlref}[2]{#1}
\newcommand{\htmlimage}[1]{}
\newcommand{\htmladdtonavigation}[1]{}

%  Starlink cross-references and labels.
\newcommand{\xref}[3]{#1}
\newcommand{\xlabel}[1]{}

%  LaTeX2HTML symbol.
\newcommand{\latextohtml}{{\bfseries LaTeX}{2}{\texttt{HTML}}}

%  Define command to re-centre underscore for Latex and leave as normal
%  for HTML (severe problems with \_ in tabbing environments and \_\_
%  generally otherwise).
\newcommand{\latex}[1]{#1}
\newcommand{\setunderscore}{\renewcommand{\_}{\texttt{\symbol{95}}}}
\latex{\setunderscore}

%  Redefine the \tableofcontents command. This procrastination is necessary
%  to stop the automatic creation of a second table of contents page
%  by latex2html.
\newcommand{\latexonlytoc}[0]{\tableofcontents}

% -----------------------------------------------------------------------------
%  Debugging.
%  =========
%  Remove % on the following to debug links in the HTML version using Latex.

% \newcommand{\hotlink}[2]{\fbox{\begin{tabular}[t]{@{}c@{}}#1\\\hline{\footnotesize #2}\end{tabular}}}
% \renewcommand{\htmladdnormallinkfoot}[2]{\hotlink{#1}{#2}}
% \renewcommand{\htmladdnormallink}[2]{\hotlink{#1}{#2}}
% \renewcommand{\hyperref}[4]{\hotlink{#1}{\S\ref{#4}}}
% \renewcommand{\htmlref}[2]{\hotlink{#1}{\S\ref{#2}}}
% \renewcommand{\xref}[3]{\hotlink{#1}{#2 -- #3}}
% -----------------------------------------------------------------------------
% ? Document specific \newcommand or \newenvironment commands.
\newcommand{\itfile}[2]{\texttt{\textit{#1}#2}}
\newcommand{\cl}{\texttt{cl}}
\newcommand{\dash}{--}
\begin{htmlonly}
\renewcommand{\dash}{-}
\end{htmlonly}
\newcommand{\IRAFURL}{http://www.starlink.ac.uk/iraf/web/iraf-homepage.html}
% ? End of document specific commands
% -----------------------------------------------------------------------------
%  Title Page.
%  ===========
\renewcommand{\thepage}{\roman{page}}
\begin{document}
\thispagestyle{empty}

%  Latex document header.
%  ======================
\begin{latexonly}
   CCLRC / {\sc Rutherford Appleton Laboratory} \hfill {\bfseries \stardocname}\\
   {\large Particle Physics \& Astronomy Research Council}\\
   {\large Starlink Project\\}
   {\large \stardoccategory\ \stardocnumber}
   \begin{flushright}
   \stardocauthors\\
   \stardocdate
   \end{flushright}
   \vspace{-4mm}
   \rule{\textwidth}{0.5mm}
   \vspace{5mm}
   \begin{center}
   {\Huge\bfseries  \stardoctitle \\ [2.5ex]}
   {\LARGE\bfseries \stardocversion \\ [4ex]}
   {\Huge\bfseries  \stardocmanual}
   \end{center}
   \vspace{5mm}

% ? Heading for abstract if used.
   \vspace{10mm}
   \begin{center}
      {\Large\bfseries Abstract}
   \end{center}
% ? End of heading for abstract.
\end{latexonly}

%  HTML documentation header.
%  ==========================
\begin{htmlonly}
   \xlabel{}
   \begin{rawhtml} <H1> \end{rawhtml}
      \stardoctitle\\
      \stardocversion\\
      \stardocmanual
   \begin{rawhtml} </H1> \end{rawhtml}

% ? Add picture here if required.
% ? End of picture

   \begin{rawhtml} <P> <I> \end{rawhtml}
   \stardoccategory \stardocnumber \\
   \stardocauthors \\
   \stardocdate
   \begin{rawhtml} </I> </P> <H3> \end{rawhtml}
      \htmladdnormallink{CCLRC}{http://www.cclrc.ac.uk} /
      \htmladdnormallink{Rutherford Appleton Laboratory}
                        {http://www.cclrc.ac.uk/ral} \\
      \htmladdnormallink{Particle Physics \& Astronomy Research Council}
                        {http://www.pparc.ac.uk} \\
   \begin{rawhtml} </H3> <H2> \end{rawhtml}
      \htmladdnormallink{Starlink Project}{http://www.starlink.ac.uk/}
   \begin{rawhtml} </H2> \end{rawhtml}
   \htmladdnormallink{\htmladdimg{source.gif} Retrieve hardcopy}
      {http://www.starlink.ac.uk/cgi-bin/hcserver?\stardocsource}\\

%  HTML document table of contents.
%  ================================
%  Add table of contents header and a navigation button to return to this
%  point in the document (this should always go before the abstract \section).
  \label{stardoccontents}
  \begin{rawhtml}
    <HR>
    <H2>Contents</H2>
  \end{rawhtml}
  \renewcommand{\latexonlytoc}[0]{}
  \htmladdtonavigation{\htmlref{\htmladdimg{contents_motif.gif}}
        {stardoccontents}}

% ? New section for abstract if used.
  \section{\xlabel{abstract}Abstract}
% ? End of new section for abstract
\end{htmlonly}

% -----------------------------------------------------------------------------
% ? Document Abstract. (if used)
%  ==================
\stardocabstract
% ? End of document abstract
% -----------------------------------------------------------------------------
% ? Latex document Table of Contents (if used).
%  ===========================================
 \newpage
 \begin{latexonly}
   \setlength{\parskip}{0mm}
   \latexonlytoc
   \setlength{\parskip}{\medskipamount}
   \markboth{\stardocname}{\stardocname}
 \end{latexonly}
% ? End of Latex document table of contents
% -----------------------------------------------------------------------------
\cleardoublepage
\renewcommand{\thepage}{\arabic{page}}
\setcounter{page}{1}

\section{\xlabel{introduction}Introduction}
IRAFSTAR contains the infrastructure which allows Starlink applications to
be run from from the
\htmladdnormallink{IRAF}{\IRAFURL} user-interface, \cl.
Where these applications use the Starlink
\xref{NDF}{sun33}{}
library, conversion between Starlink NDF and IRAF \texttt{.imh} or FITS format
data files can be done automatically, making the intermixing of IRAF and
Starlink applications easy.

The main component of IRAFSTAR is an `adaptor' process which sits between
the
\htmladdnormallink{IRAF}{\IRAFURL} \cl\ and the Starlink task intercepting
communications in both directions and handling them appropriately.
In some cases this will be a simple format conversion and in others a more
complex task.

Also in IRAFSTAR are IRAF scripts defining some environment variables required
to run Starlink applications and tasks which can be used to set the format
(NDF or IRAF or FITS) of images output by applications which use the Starlink
NDF library.

IRAFSTAR does not include any applications itself.
Section \latexonly{\ref{installing}}
\htmlref{`Installing Starlink Applications for IRAF'}{installing}
describes for programmers how to adapt Starlink applications so that they can
be run with the system.
\xref{SUN/217}{sun217}{} describes how to use Starlink applications
in IRAF and includes a section on obtaining and installing the software at
non-Starlink sites.

\section{\xlabel{the_adaptor_process}The Adaptor Process}
Starlink applications are defined for the \cl\ so that when they are invoked,
\cl\ actually starts the adaptor process which in turn runs a script which
loads and controls the required Starlink executable (which will usually be the
normal Starlink version using the normal Starlink interface module).
The script is written in
\xref{Tcl with Starlink extensions}{sun186}{}\latexonly{ (see SUN/186)}
and uses the ADAM message system interface to control the application.
Having `loaded' the Starlink application, the script `sets' the value of
any parameters which were specified on the command line as received from IRAF.
There will be specifications for any parameters entered by the user and any
IRAF `hidden' parameters (except strings with no default value).

The ADAM task is then `obeyed' with just the special keyword `PROMPT' on the
command line.
This forces the ADAM interface file VPATH to be ignored and ADAM parameter
request messages to be issued for any parameters not already specified.
The parameter request message is intercepted by the adaptor which requests
a value from IRAF or, if the parameter is classed as
\htmlref{`dynamic'}{dynamic_parameters},
uses the ADAM
suggested value. \latexonly{More information on `dynamic' parameters is given
in Section \ref{dynamic_parameters}.}
If a value is requested from IRAF, a prompt may or may not be issued depending
upon the effective mode of the parameter and whether or not it has a default
value.

Any informational messages from the ADAM task are passed on to the IRAF
\cl\ for display to the user.

\section{\xlabel{required_environment_variables}Required Environment Variables}
In order to run any Starlink (ADAM) application from the IRAF \cl, certain
environment variables must be set. These will be set to standard values by
obeying the IRAFSTAR script \texttt{zzsetenv.def}, normally in the
\htmlref{package definition file}
{pkg_def_files}\latexonly{ (see Section \ref{pkg_def_files})}.
The \texttt{zzsetenv.def} script sets the following variables:
\begin{description}
\item[ADAM\_USER] This is set to the user's \texttt{uparm} directory.
\item[NDF Conversion Variables] The environment variables associated with the
\xref{NDF Format Conversion Facilities}{ssn20}{}
are set so that given a filename without a recognised extension, applications
using the NDF library will:
\begin{itemize}
\item Input an IRAF image file \itfile{filename}{.imh},
or failing that a FITS file \itfile{filename}{.fit},
(or \itfile{filename}{.fits}),
and automatically convert it if an NDF,
\itfile{filename}{.sdf} is not found.
\item Output IRAF image file \itfile{filename}{.imh} instead of NDF
\itfile{filename}{.sdf}.
\end{itemize}
\end{description}
In addition, the IRAF tasks, \texttt{use\_ndf}, \texttt{use\_imh} and
\texttt{use\_fits} are defined.
These can be used to switch the conversion behaviour so that Starlink NDF or
IRAF image or FITS files respectively are output by default.

IRAF Image and FITS files can be input or output regardless of the default by
including a recognised extension on the filename.

\section{\xlabel{installing_starlink_applications_for_iraf}\label{installing}Installing Starlink Applications for IRAF}
This section describes how to set up a Starlink (ADAM) package for use from
IRAF.
It assumes that the Starlink IRAF inter-operability infrastructure (IRAFSTAR)
and the Interface Definition File system (IFD) have already been installed.
\begin{quote}
\emph{The Starlink application must have been linked against the Parameter
and Communications Subsystem (PCS) version 3.2-1\footnote{3.2-3 on Linux}
(released as part of USSC update 181, 17/12/96) or later.}
\end{quote}
The files associated with running Starlink packages from IRAF are normally
installed beneath directory \texttt{/star/iraf} and an IRAF environment
variable \texttt{starlink} is set to point to this top-level directory.
The files associated with a particular package will be installed in
\texttt{/star/iraf/\textit{package}}.
IRAF environment variables \texttt{\textit{package}}
will be set to point to the individual package directories.

References to \texttt{/star} here may be taken to mean the top-level
installation directory for the software item in question. For development
systems, or at non-Starlink sites, it may be something other than
\texttt{/star}.
References to \texttt{\textit{package}} mean the package name in lower case,
\textit{e.g.}\ \texttt{figaro}, \texttt{kappa}, \textit{etc.}

\subsection{\xlabel{overview}Overview}
The makefile for a Starlink package to be used with IRAF will typically
install the following files in \texttt{/star/iraf/\textit{package}}:
\begin{itemize}
\item The Starlink executable image files \dash\ usually these will just be
soft links to the executables in the Starlink software.
\item The Starlink interface ({\texttt{.ifc}}) files.
These will also usually be soft links to the Starlink software versions but
may be real files if it turns out to be necessary to have different interface
files for running under IRAF.
\item The IRAF tasks. These have the same names as the Starlink executables but
with \texttt{.e} appended to the name and are soft links to the Starlink/IRAF
adaptor, {\texttt{aitclsh}}.
\item An IRAF package definition file (\itfile{package}{.cl}).
\item An IRAF package parameter file (\itfile{package}{.par}).
\item An IRAF parameter (\texttt{.par}) file for each command in the package.
\item An Output Parameters File for each executable to list the task
parameters which are:
\begin{enumerate}
\item `output' and should therefore have their values set in the
IRAF parameter file when a command completes.
\item `dynamic' and should therefore take the value suggested by the ADAM task.
\end{enumerate}
\item IRAF help files (\itfile{\_package}{.hd},
\itfile{package}{.hd},
\itfile{package}{.men}, \texttt{helpdb.mip} and \texttt{root.hd}) and the
subdirectory (\texttt{doc}) containing the individual help topic
(\texttt{.hlp}) files.
\end{itemize}

\subsection{\xlabel{the_interface_definition_file}The Interface Definition File}
The required files will initially be produced in two stages. First, an
{\bfseries Interface Definition File} (\texttt{.ifd}) is produced for the
package.
Interface Definition Files and the scripts to produce and handle them are fully
described in
\xref{SSN/68}{ssn68}{}.

If there is an existing Starlink Interface File source (\texttt{.ifl}) for
the package, the Tcl script
\xref{\texttt{ifl2ifd}}{ssn68}{creating_ifds_from_interface_files}
may be used to read it and create a basic
Interface Definition File.
After modification, this file may be used as the source for the package
definition files required by both IRAF and ADAM.
The script assumes a single monolith for the package \dash\ if there is more
than one, the resulting \texttt{.ifd}'s have to be combined by hand.

Secondly, the Tcl script
\xref{\texttt{ifd2iraf}}{ssn68}{producing_iraf_files_from_an_ifd}
will read the Interface Definition File
and produce a package definition file and an IRAF parameter file for
the package and for each command defined. It also creates an Output Parameters
File for each executable, listing the `output' and
`dynamic' parameters for each action in the executable.

The files resulting from this automatic process will probably not describe
the package correctly in all cases for IRAF.
Likely problems are discussed in the following sections and should be cured by
altering and reprocessing the Interface Definition File.

Sometimes a special version of the ADAM interface file will be required for
use with IRAF. In this case conditional statements are added to the
Interface Definition File, and it is processed by the Tcl script
\xref{\texttt{ifd2irafifl}}{ssn68}{ifd2irafifl} to produce the required
interface file.

\subsection{\xlabel{package_definition_files}Package Definition Files\label{pkg_def_files}}
The Package Definition File is a \cl\ script which will look something like:
\begin{quote} \begin{verbatim}
# IRAF package initialisation script for the ADAM kappa package

package kappa

cl < "starlink$irafstar/zzsetenv.def"

task add = "kappadir$kappa_pm.e"
task aperadd = "kappadir$kappa_pm.e"
task tweak = "kappadir$kappa_pm.e"
task zaplin = "kappadir$kappa_pm.e"

print ""
print "--------------------------------------------------"
print "    Welcome to The Starlink Software Collection"
print "                KAPPA for IRAF"
print "            Version " (version)
print "      For support contact ussc@star.rl.ac.uk"
print "--------------------------------------------------"
print ""

clbye()
\end{verbatim} \end{quote}
If the list of tasks generated from the initial \texttt{.ifd} contains
task statements for commands that don't make any sense when running as part
of IRAF (\textit{e.g.}\ the package `help' command), the \texttt{.ifd} can be
altered to use
\xref{conditional sections}{ssn68}{conditional_sections}\latexonly{ (see
SSN/68)}
so that they are omitted when processing for IRAF.

\subsection{\xlabel{package_parameter_files}Package Parameter Files}
These will normally contain just the definition of the parameter
\texttt{version} as follows:
\begin{quote} \begin{verbatim}
# Package parameters for the PKG package
version,s,h,"V1.3 - 25/4/97"
\end{verbatim} \end{quote}
The
\xref{\texttt{ifd2iraf}}{ssn68}{producing_iraf_files_from_an_ifd}
script will set the parameter value to the argument of
the
\xref{\texttt{version}}{ssn68}{version}
keyword in the Interface Definition File (or to `PKG\_VERS' in default) plus
the current date. This text may be edited to your taste, possibly by the
\texttt{install} target of the package makefile replacing `PKG\_VERS' with
\$(PKG\_VERS).

\subsection{\xlabel{task_parameter_files}\label{parfiles}Task Parameter Files}
The task parameter (\texttt{.par}) files created by
\xref{\texttt{ifd2iraf}}{ssn68}{producing_iraf_files_from_an_ifd}
need to be
inspected before they are ready for use because parameters that might
potentially be set to an array rather than a single value must be given
an IRAF type of `struct' and there is no way of determining whether this
is the case or not from the Starlink interface file unless the parameter has
a default value that is an array.

For example, the parameter file for the kappa command \texttt{ffclean} as
generated by
\xref{\texttt{ifl2ifd}}{ssn68}{creating_ifds_from_interface_files}
followed by
\xref{\texttt{ifd2iraf}}{ssn68}{producing_iraf_files_from_an_ifd}
is:
\begin{quote} \begin{verbatim}
box,struct,a,,,,Smoothing box size
3,3
clip,struct,a,,,,Clipping levels
3.0,3.0,3.0
in,f,a,,,,Give image to be cleaned
out,f,a,,,,NDF to hold the cleaned image
ilevel,i,h,2,,,Interaction level
sigma,r,h,0.0,,,Object to contain the RMS noise per pixel
title,s,h,KAPPA - Cleaner,,,Title for output NDF
thresh,r,h,INDEF,,,Data-value limits
wlim,r,h,INDEF,,,Weight limit for good output pixels
\end{verbatim} \end{quote}
but the parameter `thresh' is an array so the definition of `thresh'
must be changed to:
\begin{quote} \begin{verbatim}
thresh,struct,h,,,,Data-value limits
INDEF
\end{verbatim} \end{quote}
\textit{i.e.}\ the data type (the second field) is changed to `struct' and the
default value (the fourth field) moved onto a line following the definition.
If there is no default, then the line following should be blank.
This situation should be corrected permanently by inserting a
\xref{\texttt{size}}{ssn68}{size} directive into the parameter description in
the \texttt{.ifd} file.

Further modifications to the parameter files will probably be seen to be
necessary when the package is tested.

\subsection{\xlabel{the_output_parameters_file}The Output Parameters File}
\subsubsection{\xlabel{output_parameters}Output Parameters}
The Starlink/IRAF adaptor process needs to know which ADAM parameters have
their values set by the application so that their values can be copied
into the corresponding IRAF parameters when the application has finished
executing. This information cannot be reliably deduced from most interface
files as the default access mode (`UPDATE') is often used for parameters which
are actually read-only.
Furthermore, parameters whose values are the names of files or devices to be
written to need to have access `WRITE' or `UPDATE' specified whereas the
parameter value (the filename or device name) is usually read-only and
therefore does not require the IRAF parameter to be updated when the
application has finished.
The
\xref{\texttt{ifd2iraf}}{ssn68}{producing_iraf_files_from_an_ifd}
script assumes that a parameter is an output parameter
if the \texttt{.ifd} contains an
\xref{\texttt{outputpar}}{ssn68}{outputpar}
keyword for it or if it is a primitive type with an
\xref{\texttt{access}}{ssn68}{access}
keyword specifying WRITE or UPDATE.

\subsubsection{\xlabel{dynamic_parameters}\label{dynamic_parameters}Dynamic
Parameters}
The ADAM environment permits parameter values to be taken from GLOBAL
parameters (usually set up by another task) or from values calculated by the
current task itself. Where this is the case and there is no obvious default
value and the user cannot reasonably be expected to provide the value, the
parameter is classed as `dynamic' by the IRAF/Starlink inter-operability
system. If the ADAM task prompts for a dynamic parameter value, the prompt
is intercepted by the adaptor process which replies with the ADAM suggested
value. The success of this method depends upon:
\begin{enumerate}
\item The parameter not being provided on the IRAF command line. That is,
the mode of the IRAF parameter cannot be `hidden'. If a value is provided on
the command line by the user, that value will be used in preference to the
suggested value.
\item The ADAM task must prompt with the correct suggested value. This implies
that the PPATH starts with GLOBAL or DYNAMIC.
\item The adaptor process must know that the parameter is `dynamic'.
\end{enumerate}
The
\xref{\texttt{ifd2iraf}}{ssn68}{producing_iraf_files_from_an_ifd}
script assumes that a parameter is dynamic if there is a
\xref{`\texttt{dynamic yes}'}{ssn68}{dynamic}
keyword for it or if it has a
\xref{\texttt{vpath}}{ssn68}{vpath}
starting with GLOBAL.

\emph{Parameters with \texttt{vpath DYNAMIC} are not assumed to be dynamic by
default because they are often using a fixed value which is hardwired in the
code. Such parameters are usually better handled by specifying that fixed
value as the static default value with a
\xref{\texttt{default}}{ssn68}{default}
keyword.
If it really is a dynamic parameter, the \texttt{.ifd} should contain a
\xref{`\texttt{dynamic yes}'}{ssn68}{dynamic}
keyword for it.}

\subsubsection{\xlabel{creating_the_output_parameters_file}Creating the Output Parameters File}
There should be a separate Output Parameters File for each executable image in
the package.
This file is read by the adaptor process and should contain Tcl statements that
set the elements of arrays OutputParList and DynParList, indexed by command
name, to a list of the output and dynamic parameters respectively.

For example:
\begin{quote} \begin{verbatim}
set OutputParList(aperadd) {numpix total mean sigma noise}
set DynParList(aperadd) {}
set OutputParList(centroid) {centre xcen ycen}
set DynParList(centroid) {cosys device mode}
\end{verbatim} \end{quote}

If a command doesn't have any output or dynamic parameters the relevant entry
can be omitted entirely \dash\ you don't have to set the appropriate array
element to an empty list.

An attempt is made by the
\xref{\texttt{ifd2iraf}}{ssn68}{producing_iraf_files_from_an_ifd}
script to list the output and dynamic parameters in file \itfile{adamexe}{.tcl}
(where \texttt{\textit{adamexe}} is the name of the executable image) but
this file is likely to need editing until the \texttt{.ifd} has the correct
entries.

\subsection{\xlabel{help_files}Help Files}
The script \texttt{ifd\_irafhlpgen} converts a Starlink help (\texttt{.hlp})
file to the files necessary to generate an IRAF help database. Each `level 1'
topic (and its sub-topics) is converted into an IRAF help file (in the
subdirectory \texttt{doc}) and a package menu (\texttt{.men}) file created
with one line for each topic; the line following the topic is used
as the description. A typical menu file looks like:

\begin{quote} \begin{verbatim}
       add - Adds two NDF data structures.
   aperadd - Derives statistics of pixels within a specified circled.
     tweak - Interactively adjusts a colour table.
using_help -
    zaplin - Replaces regions by bad values or by linear interpolation.
\end{verbatim} \end{quote}
This file will almost certainly need to be edited to remove irrelevant
topics (such as `using\_help' in the above example) and to tidy up the
descriptions.

The package \texttt{.hd} file (\textit{e.g.}\ \texttt{kappa.hd}) looks like:
\begin{quote} \begin{verbatim}
#  Help directory for the kappa package

$doc = "kappa$doc/"

add        hlp = doc$add.hlp, src = add.f
aperadd        hlp = doc$aperadd.hlp, src = aperadd.f
using_help        hlp = doc$using_help.hlp, src = using_help.f
zaplin        hlp = doc$zaplin.hlp, src = zaplin.f
\end{verbatim} \end{quote}
will also need to have irrelevant topics removed and possibly directory
specifications altered.

Having created the above files, the help database for the package must be
created. Start the \cl\ (probably in your normal IRAF home directory) and then:

\begin{quote}
\texttt{cl> set \textit{package} = \textit{wherever}/\\
cl> so\\
so> mkhelpdb \textit{package}\$root.hd helpdb.mip}
\end{quote}

\subsection{\xlabel{completing_the_installation}Completing the Installation}

It is assumed that the Starlink software is installed in the standard Starlink
way beneath directory \texttt{/star}. If this is not the case, suitable
modifications must be made to the following instructions.

\begin{enumerate}
\item The \texttt{/star/iraf/\textit{package}} directory should contain the
package definition (\texttt{.cl}) file, the package and task parameter
(\texttt{.par}) files, the Output Parameter File(s) and the IRAF Help Files.
The following links must then be set up in the same directory:
\begin{itemize}
\item The Starlink executable and interface-file soft links:
\begin{quote}
\texttt{ln -s /star/bin/\textit{package}/\textit{monolith} \textit{monolith}\\
ln -s /star/bin/\textit{package}/\textit{monolith}.ifc \textit{monolith}.ifc}
\end{quote}
There will be a set for each monolith in the package. If it proves necessary
to have special versions of either the monolith or interface file for use with
IRAF, the actual files should be installed here.
\item The IRAF executable soft link:
\begin{quote}
\texttt{ln -s /star/iraf/irafstar/aitclsh \emph{monolith}.e}
\end{quote}
\end{itemize}

\item Make your package known to IRAF.
The preferred way of adding new external packages for IRAF is for the site
manager to add entries in the \texttt{hlib\$extern.pkg} file, starting with the
definition of the top-level \texttt{starlink} directory:
\begin{quote} \begin{verbatim}
reset   starlink          = /star/iraf/
\end{verbatim} \end{quote}
and add `\texttt{,starlink\$irafstar/helpdb.mip}' to the \texttt{helpdb}
definition.

Then, for example:
\begin{quote} \begin{verbatim}
reset   figaro            = starlink$figaro/
task    figaro.pkg        = figaro$figaro.cl
\end{verbatim} \end{quote}
and add `\texttt{,figaro\$helpdb.mip}' to the \texttt{helpdb} definition.
Similar definitions should be added for all the Starlink applications packages
in use.

If you do not have access to the \texttt{hlib\$extern.pkg} file, or whilst
testing, the above commands can be added to your \texttt{login.cl} file, just
before the \texttt{user} package definition.
In that case, the \texttt{helpdb} definition can be extended by a command like:
\begin{quote} \begin{verbatim}
reset helpdb = (envget("helpdb") // ",figaro$helpdb.mip")
\end{verbatim} \end{quote}
\end{enumerate}
The package can now be loaded and tested.

\subsection{\xlabel{testing}Testing}

During the testing phase every application in the package needs to be run
and the following questions addressed:

\begin{enumerate}

\item \label{pardefs}Are the parameter defaults appropriate
(view them with \texttt{lparam}
after resetting the parameters to their initial state with \texttt{unlearn})?

\item \label{dynamic} Are
\htmlref{`dynamic' parameters}{dynamic_parameters}
used where required and only where required?

\item \label{repeated} Are parameters which will be reprompted for set to
`query' mode to force an IRAF prompt?

\item \label{prompt}Is the prompt text sensible in the context of IRAF?

\item \label{images}Will the application read IRAF image files
(where appropriate)? If not,
either the application does not use the NDF library or does some some
peculiar manipulation of HDS files that defeats the NDF data conversion
mechanism.

\item \label{error}Do error messages look sensible?

\item \label{promptdefs}Are the default values in prompts what
an IRAF user would expect?
For example, if you run an IRAF application a second time the defaults are
usually simply the values used last time; the `global association' mechanism
used by some Starlink applications to set the default for an input image to be
the output image created by the preceding application gives `surprising'
behaviour for an IRAF user.

\end{enumerate}

Most of these problems can be addressed by modifying the interface definition
file to produce the required \texttt{.par} files but in some cases, particularly
with `dynamic' parameters, a special ADAM interface file may be required for
use with IRAF \dash\ to produce the required suggested value, for example.

Problems 5 and 6 may require modification of the application
source code, but this should be done in a way that is compatible with running
the application in the normal way so that only one version of the source
code and executable is needed.  Remember that the ADAM message system does
permit message text to be defined in the interface file.
\end{document}
