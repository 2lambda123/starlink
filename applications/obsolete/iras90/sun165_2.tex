\documentstyle[11pt]{article}
\pagestyle{myheadings}

%------------------------------------------------------------------------------
\newcommand{\stardoccategory}  {Starlink User Note}
\newcommand{\stardocinitials}  {SUN}
\newcommand{\stardocnumber}    {165.2}
\newcommand{\stardocauthors}   {David S. Berry}
\newcommand{\stardocdate}      {7 February 1994}
\newcommand{\stardoctitle}     {IRAS90 --- Programmers Guide}
%------------------------------------------------------------------------------

\newcommand{\stardocname}{\stardocinitials /\stardocnumber}
\renewcommand{\_}{{\tt\char'137}}     % re-centres the underscore
\markright{\stardocname}
\setlength{\textwidth}{160mm}
\setlength{\textheight}{230mm}
\setlength{\topmargin}{-2mm}
\setlength{\oddsidemargin}{0mm}
\setlength{\evensidemargin}{0mm}
\setlength{\parindent}{0mm}
\setlength{\parskip}{\medskipamount}
\setlength{\unitlength}{1mm}
\setlength{\unitlength}{1mm}

%------------------------------------------------------------------------------
% Add any \newcommand or \newenvironment commands here

% degrees symbol
\newcommand{\dgs}{\hbox{$^\circ$}} 
% centre an asterisk
\newcommand{\lsk}{\raisebox{-0.4ex}{\rm *}}
% A kind of list item, like description, but with an easily adjustable
% item separation.
\newcommand{\menuitem}[2]
  {{\bf #1}: \addtolength{\baselineskip}{-0.4ex}
  \parbox[t]{128mm}{#2} \addtolength{\baselineskip}{0.4ex} \\ \hspace{-5pt}}
% an environment for references
\newenvironment{refs}{\goodbreak
                      \vspace{3ex}
                      \begin{list}{}{\setlength{\topsep}{0mm}
                                     \setlength{\partopsep}{0mm}
                                     \setlength{\itemsep}{0mm}
                                     \setlength{\parsep}{0mm}
                                     \setlength{\leftmargin}{1.5em}
                                     \setlength{\itemindent}{-\leftmargin}
                                     \setlength{\labelsep}{0mm}
                                     \setlength{\labelwidth}{0mm}}
                    }{\end{list}}

\begin{document}
\thispagestyle{empty}
SCIENCE \& ENGINEERING RESEARCH COUNCIL \hfill \stardocname\\
RUTHERFORD APPLETON LABORATORY\\
{\large\bf Starlink Project\\}
{\large\bf \stardoccategory\ \stardocnumber}
\begin{flushright}
\stardocauthors\\
\stardocdate
\end{flushright}
\vspace{-4mm}
\rule{\textwidth}{0.5mm}
\vspace{5mm}
\begin{center}
{\Large\bf \stardoctitle}
\end{center}
\vspace{5mm}

\setlength{\parskip}{0mm} \tableofcontents
\setlength{\parskip}{\medskipamount} \markright{\stardocname}

\newpage

\section{Introduction}
{\small IRAS90} is a package of applications which provides facilities for the
reduction and analysis of {\small IRAS} Survey and Pointed Observation (PO) data
(also known as ``Additional Observation'', or ``AO'' data). It does not provide
facilities for handling Low Resolution Spectrometer (LRS) data, or catalogue
data. For more information on existing {\small IRAS90} applications see SUN/163.
This document gives some guide-lines for users who want to produce their own
{\small IRAS90} applications. 

The {\small IRAS90} system is written in {\small FORTRAN} and is integrated with
the Starlink {\small ADAM} environment. Anyone intending to create their own 
{\small IRAS90} applications should therefore be familiar with {\small FORTRAN} 
and with the {\small ADAM} programming interface (see SUN/101). It is 
recommended that programmers conform to the Starlink programming standards 
described in SGP/16 when producing {\small IRAS90} code as this will help to 
produce portable, maintainable, reliable code which is consistent with the 
existing {\small IRAS90} code. Adherence to the Starlink programming standards 
is particularly important if you intend to submit your applications for 
inclusion in the {\small IRAS90} package. Use of the ``Language Sensitive 
Editor'', {\small STARLSE}, is an enormous help in creating such code (see 
SUN/105).

\section{First Steps}
It is strongly recommended that potential {\small IRAS90} programmers look at some of
the existing {\small IRAS90} code before starting to create their own software.
Particularly, programmers should try to identify an existing application which
has a similar interface to the application they wish to write. One or two
aspects of the interface worth checking are: 

\begin{itemize}
\item The type of input and output data. Input and output {\small NDF}s usually 
hold either images or {\small CRDD}.
\item The number of input and output NDFs. Do you want to write an application 
which operates on a single input {\small NDF} to create a single output {\small 
NDF}, or one which combines several input {\small NDF}s into a single output
{\small NDF}? There are obviously other possibilities.
\item Access to graphics devices.
\end{itemize}

The idea is that you find an application which looks as similar as possible to 
the one you want to write, and use the existing code as the basis for your new
application.

\section{Locating Existing Source Code}
Existing source code is stored within the released {\small IRAS90} system. 
The majority of routines can be classified as follows:

\begin{description}
\item [A-task routines :] These are the top-level routines for each application 
and include user documentation in the form of a prologue.
\item [Other application-specific routines :] These are routines which are 
specific to a particular application and which are called from the applications
A-task routine (or from another application-specific routine). In general they 
perform a specific operation which is of relevance only to a single application.
\item [Utility routines :] These are routines which perform commonly used
operations. They are not associated with any one application and can be
called by any. 
\end{description}

In general, only the first two categories will be of interest. If you have
written a routine in the third category, which you think may be of use to other
applications, you should contact the {\small IRAS90} support team with a view to
getting your routine included in the utility routine libraries. 

The location of these routines depends on the operating system being used.

\subsection{On VMS Systems}
On {\small VMS} systems A-task routines are stored in the text library {\small 
IRAS90\_DIR:IRAS90.TLB}, and other application specific routines are stored in 
he text library {\small IRAS90\_DIR:IRAS90\_SUB.TLB}. For instance, to list the 
existing application specific routines, use the command:

\small
\begin{verbatim}
   $ LIB/LIST IRAS90_DIR:IRAS90_SUB.TLB
\end{verbatim}
\normalsize

To get the A-task source code for the application {\small DESTCRDD}, use the 
command:

\small
\begin{verbatim}
   $ LIB/EXT=DESTCRDD/OUT=DESTCRDD.FOR IRAS90_DIR:IRAS90.TLB
\end{verbatim}
\normalsize

Interface files are stored in the text library {\small
IRAS90\_DIR:IRAS90\_IFL.TLB}, so to extract the interface file for application
{\small DESTCRDD}, use the command: 

\small
\begin{verbatim}
   $ LIB/EXT=DESTCRDD/OUT=DESTCRDD.IFL IRAS90_DIR:IRAS90_IFL.TLB
\end{verbatim}
\normalsize

\subsection{On UNIX Systems}
All A-task routines, applications specific routines and interface files are
stored in the tar file \verb+iras90_source.tar+. This archive file is stored
within the {\small IRAS90} source directory (probably \verb+/star/iras90+
although this could in principle change). 

The contents of the archive file can be listed using the command: 

\small
\begin{verbatim}
   % tar -tf /star/iras90/iras90_source.tar
\end{verbatim}
\normalsize

To get the A-task
source code for the application {\small DESTCRDD}, use the command: 

\small
\begin{verbatim}
   % tar -xf /star/iras90/iras90_source.tar destcrdd.f 
\end{verbatim}
\normalsize

To extract the interface file for application {\small DESTCRDD}, use the
command: 

\small
\begin{verbatim}
   % tar -xf /star/iras90/iras90_source.tar destcrdd.ifl 
\end{verbatim}
\normalsize

\section{Use of Utility Routines}
\label{SEC:UTIL}
Many general purpose utility routines have been written as part of the {\small 
IRAS90} package. Use of these routines can save much effort, and is indeed 
essential for many operations. The utility routines are divided into several 
``libraries'', each library containing routines associated with a specific 
function. The libraries all have three letter names beginning with ``IR'':

\begin{description}
\item [IRA :] contains routines associated with accessing astrometry information
within two dimensional images, performing spherical trigonometry, handling of
formatted sky coordinate values (eg getting sky coordinates from the user), and
generation of astrometric graphics (eg sky coordinate grids, meridians,
parallels, etc). {\small IRA} documentation is in file {\small ID2.TEX}.
\item [IRC :] contains routines used for accessing descriptive information 
associated with {\small CRDD} files. {\small IRC} documentation is in file 
{\small ID1.TEX}.
\item [IRI :] contains routines for controlling the descriptive information
(except astrometry which is handled by {\small IRA}) associated with {\small
IRAS90} images. {\small IRI} documentation is in file {\small ID12.TEX}.
\item [IRM :] contains a miscellaneous selection of routines, covering a wide 
range of general tasks. {\small IRM} documentation is in file {\small ID8.TEX}.
\item [IRQ :] contains routines for accessing and using quality information
within {\small IRAS90} images and {\small CRDD} files. {\small IRA}
documentation is in file {\small ID6.TEX}. 
\end{description}

In addition, there is a sub-system called I90 which consists of a set of include
files, and block data files, containing definitions of a large number of 
symbolic constants and array values. These contain an assortment of data 
describing the {\small IRAS} satellite and mission, and the {\small IRAS90} 
package. It is documented in file {\small ID20.TEX}.

Each sub-system is described in a separate document distributed with the {\small
IRAS90} package, as listed above. These files give a general over-view of the
purpose of the sub-system and also included specific details about the routines
contained in the sub-system. They are located in different places, depending on
the operating system: 

\begin{description}
\item [UNIX :] The files are in the directory \verb+/star/iras90/doc+.
\item [VMS :] Each sub-system has a subdirectory within the main {\small IRAS90}
source directory (pointed to by logical name {\small IRAS90\_DIR}). Thus, for
example, the documentation for the {\small IRC} sub-system is in [.IRC]ID1.TEX
(assuming that the current default directory is {\small IRAS90\_DIR}). 
\end{description}

Anyone intending to write an {\small IRAS90} application will first need to
obtain copies of these documents by processing the distributed files using
{\small LATEX}, etc. 

\subsection{Utility Routine Templates within STARLSE}
The {\small STARLSE} package (see SUN/105) provides facilities for initialising
the {\small DEC} Language Sensitive Editor ({\small LSE}) to simplify the
generation of Fortran code conforming to the Starlink programming standard. One
of the facilities provided by {\small LSE} is the automatic production of
argument lists and help for subroutine calls. Templates for all the {\small
IRAS90} utility subroutines can be made available within {\small LSE} (on
{\small VMS} systems only at the moment) by performing the following steps
within {\small LSE}: 

\begin{enumerate}
\item Issue the {\small LSE} command {\small GOTO FILE/READ IRAS90\_DIR:IRAS90.LSE}
\item Issue the {\small LSE} command {\small DO}
\item Issue the {\small LSE} command {\small DELETE BUFFER IRAS90.LSE}
\item Move to a buffer holding a {\small .FOR} or a {\small .GEN} file in the 
usual way.
\item {\small IRAS90} subroutine templates are then available by typing in the
name of an {\small IRAS90} utility subroutine (or an abbreviation) and expanding
it using {\small CTRL-E}. 
\item Help on the subroutine and its arguments can be obtained by placing the
cursor at a point in the buffer at which the subroutine name has been entered 
and pressing {\small GOLD-PF2}.
\end{enumerate}

\subsection{Status Values Returned by Utility Routines}
The utility routines (and all other {\small IRAS90} routines) use the 
``inherited status'' error system described in SUN/104. Utility routines flag 
errors by reporting an error message and setting the returned status to some 
specific value indicating the type of error which has occurred. These status 
values are unique amongst {\em all} status values used by any {\small ADAM} 
software, and are calculated using the expression:

\small
\begin{verbatim}
     134250498 + 65536*<facility number> + 8*<err_num>
\end{verbatim}
\normalsize

where \verb+<facility number>+ is a number reserved by Starlink for the {\small
IRAS90} package (currently 1507), and \verb+<err_num>+ is an error number
internal to IRAS90, in the range 1 - 4095. Each utility package has been
assigned a range of error numbers as follows: 

\begin{description}
\item [1 -   30] : IRA
\item [31 -   60] : IRQ
\item [61 -   90] : NDG
\item [91 -  120] : IRC
\item [121 -  200] : IRM
\item [201 -  230] : Not currently used.
\item [231 -  241] : IRQ (extra to 31 - 60)
\item [242 -  300] : IRA (extra to 1 - 30)
\item [301 -  330] : IRI
\item [331 - 4095] : Not currently used.
\end{description}

\section{IRAS90 Programming Conventions}
\label{SEC:CONV}
This section contains descriptions of {\em some} of the conventions which have
been adhered to in the production of the released {\small IRAS90} applications.
If you intend to submit your applications for inclusion in {\small IRAS90} you
should ensure that they also conforms to these conventions: 

\begin{enumerate}
\item If an application needs to read or write a list of character strings, the
facilities of the {\small GRP} package (SUN/150) should be used if possible.
{\small GRP} provides the user with a very versatile system for specifying the
strings, and avoids the use of explicit file handling routines. As an example,
see the way that {\small REMQUAL} accesses the {\small QNAMES} parameter, or the
way that {\small SKYPOS} handles the files associated with parameters {\small
FILE} and {\small LOGFILE}. 

\item If multiple {\small NDF}s are processed or created, then the list of
{\small NDF}s should be obtained using the facilities of the {\small IRM} and
{\small NDG} packages (see section \ref{SEC:UTIL} and appendix \ref{APP:NDG}).
As an example, see the way that {\small MAPCRDD} accesses its list of input
{\small CRDD} files. 

\item If the application processes a single input {\small NDF} to create a
single output {\small NDF}, then it should be written to allow the user to
specify a group of input {\small NDF}s, each being processed to create a 
corresponding output {\small NDF}. See {\small DESTCRDD, BACKCRDD} etc.

\item Applications should in general always check the {\small UNITS} components
of the input {\small NDF}s to ensure that the data is in a system of units which
the application knows how to process. The {\small IRAS90} package has a set of 
``standard'' units which is described in the documentation for the {\small IRI} 
sub-system (see section \ref{SEC:UTIL}). The {\small IRM} sub-system contains routines 
for converting data values between system of units.

\item If appropriate, applications should allow the user to specify which pixels
or samples within the input {\small NDF}s are to be processed, by prompting for
a ``quality expression'' (these are described in the {\small IRQ} sub-system
documentation). See {\small DESTCRDD} for an example of the use of quality
expressions. 

\item \label{IT:NDFLIST} If the application can produce more than one output
{\small NDF} it should create a text file holding a list of the created output
{\small NDF}s and associate this file with global parameter {\small
GLOBAL.DATA\_ARRAY}. This can be done using utility routine {\small IRM\_LISTN},
specifying {\small ADAM} parameter ``{\small NDFLIST}''. See {\small DESTCRDD}
for an example (also look at the specification of parameters {\small OUT} and
{\small NDFLIST} in the interface file for {\small DESTCRDD}). 

\item Checks for bad pixel values should always be made unless a call to 
{\small NDF\_BAD} has been made to confirm that the {\small NDF} contains 
no bad values.

\item If convenient, the bad pixel flag should be set in all output {\small 
NDF}s, using routine {\small NDF\_SBAD}.

\item The {\small PSX} library should be used for getting dynamic workspace
rather than the {\small ARY} library.

\item Application specific subroutines should have names in the form {\em 
xxxxyz}, where {\em xxxx} are the first four characters from the application 
name, {\em y} is a single letter (a-z), and {\em z} is a single digit (0-9).
If there already exists an application with the same first four letters in its 
name, then the preference for {\em xxxx} is the first, fifth, sixth and seventh
letters of the application name. If this cannot be done, then any unique string 
can be used, but it should indicate which application the routine is associated 
with. For instance, the routines associated with {\small MAPCRDD} are named
{\small MAPCA0}, {\small MAPCA1}, {\small MAPCB0}, etc. 

\item If the application processes multiple input {\small NDF}s then the user
should be told which {\small NDF} is currently being processed using a message 
of the form "  Processing \verb+^+NDF..." where \verb+^+NDF is replaced by the name of the 
NDF (see {\small DESTCRDD} for instance).

\item If the application processes multiple input {\small NDF}s, then any errors
which occur while processing an input {\small NDF} should be flushed (using
routine ERR\_FLUSH) and the application should continue to process the other
input {\small NDF}s (see {\small DESTCRDD} for instance). 

\item The following names should be used for parameters associated with 
{\small NDF}s:
\begin{description}
\item [IN] - if {\small READ} access is required to the {\small NDF} (i.e. if 
the {\small NDF} is read but not modified).
\item [OUT] - if {\small WRITE} access is required to the {\small NDF} (i.e. if 
a new {\small NDF} is created).
\item [NDF] - if {\small UPDATE} access is required to the {\small NDF} (i.e. if
the {\small NDF} is read and then modified). 
\end{description}

\item All groups identified by {\small GRP} identifiers should be deleted using
routine {\small GRP\_DELET} before the application finishes (even if the 
application aborts due to an error).

\item If appropriate, applications should propagate history from the input to 
the output {\small NDF}s.

\item \label{IT:HIS} Applications should add history information to the output {\small NDF}s
using routine {\small IRM\_HIST}, specifying {\small ADAM} parameter ``{\small
HISTORY}''. The value supplied for argument {\small COMMAND} should commence
with the string ```IRAS90:''. The history should contain all input and output
{\small NDF} names, together with key parameter values, etc. See {\small
DESTCRDD} for an example. 

\item \label{IT:MSG_FILTER} Each application should set up a conditional message
filtering level by calling routine {\small MSG\_IFGET}, specifying {\small ADAM}
parameter {\small MSG\_FILTER}. 

\item All messages displayed with routines {\small MSG\_OUTIF} and {\small
MSG\_OUT} should correspond to parameters with names of the form {\em
xxxxx\_MSGy} where {\em xxxxxx} is the name of the routine, and {\em y} is a
number which increases monotonically for each message produced by the routine
(starting at 1 for the first message). 

\item All messages should start with at least two spaces.

\item All explicit assignments to the inherited status value ``{\small STATUS}''
should be guarded against the possibility of a pre-existing error value being 
over-written, as in the following code fragment:

\small
\begin{verbatim}
   *  If the pixel size is zero, report an error.
         IF( PIXSIZ .EQ. 0.0 .AND. STATUS .EQ. SAI__OK ) THEN
            STATUS = SAI__ERROR
            CALL ERR_REP( 'CTEMA0_ERR1', 
        :                 'CTEMA0: Zero pixel size encountered',
        :                 STATUS )
            GO TO 999
         END IF
\end{verbatim}
\normalsize

\item All ``{\small DO WHILE}'' loops should terminate if {\small STATUS} 
indicates an error has occurred:

\small
\begin{verbatim}
   *  Loop until all values have been done, or an error occurs.
         MORE = .TRUE.
         DO WHILE ( MORE .AND. STATUS .EQ. SAI__OK ) 
            ...
            ...
         END DO
\end{verbatim}
\normalsize

\item Any subroutine call which includes variable array sizes within its
list of arguments should be preceded with a status check (to avoid adjustable 
array dimensions errors occurring if an array size of zero is passed to the 
subroutine because of the error condition):

\small
\begin{verbatim}
   *  Map the data array.
         CALL NDF_MAP( INDF, 'DATA', '_REAL', 'READ', IPIN, EL, STATUS )

   *  Abort if an error has occurred.
         IF( STATUS .NE. SAI__OK ) GO TO 999

   *  Find the total data sum in the data array.
         CALL CTEMA1( EL, %VAL( IPIN ), SUM, STATUS )
\end{verbatim}
\normalsize

\item If a subroutine call uses the {\small \%VAL} construct to pass a mapped
character array to the subroutine, the final argument in the argument list
should be the length of each character string in the array. This length should
be passed using the {\small \%VAL} construct. Particular care is needed if the
argument list contains other character values as well (see SUN/92, section
``Mapping Character Data''). 

\item If a constant value is used in an argument list for an argument which 
corresponds to a double precision value, then the constant must have the correct 
double precision form (i.e. must terminate with a ``D'' exponent 
specifier, eg 1.456D2).

\item Care should be taken that calls to {\small ERR\_ANNUL} cannot annul the 
wrong error. One way to do this is to include a status check before 
the subroutine call which generates the status to be annulled:

\small
\begin{verbatim}
   *  Get a value for parameter BOXSIZE.
         CALL PAR_GET0R( 'BOXSIZE', BOXSIZ, STATUS )

   *  Abort if an error has occurred.
         IF( STATUS .NE. SAI__OK ) GO TO 999

   *  Get a value for parameter SIGMA.
         CALL PAR_GET0R( 'SIGMA', SIGMA, STATUS )

   *  If a null value was supplied, annul the error and indicate that
   *  no filtering is required.
         IF( STATUS .EQ. PAR__NULL ) THEN
            CALL ERR_ANNUL( STATUS )
            FILT = .FALSE.
         ELSE
            FILT = .TRUE.
         END IF
\end{verbatim}
\normalsize

\item All error reports should correspond to parameters with names of the form
{\em xxxxx\_ERRy} where {\em xxxxxx} is the name of the routine, and {\em y} is
a number which increases monotonically for each call to {\small ERR\_REP} made
by the routine (starting at 1 for the first error). 

\item The text of error reports should start with the name of the routine in 
which the call to {\small ERR\_REP} is made, followed by a colon and a space. 
They should include as much information as possible about the conditions under 
which the error occurred:

\small
\begin{verbatim}
            CALL NDF_MSG( 'NDF', INDF )
            CALL MSG_SETI( 'C', COLUMN )
            CALL MSG_SETI( 'R', ROW )
            CALL MSG_SETR( 'D', DATA( COLUMN, ROW ) )
            CALL MSG_SETR( 'L', MAXDAT )

            CALL ERR_REP( 'CTEMA2_ERR1', 
        : 'CTEMA2: Illegal value (^D) found at column ^C, row ^R of '//
        : '^NDF. Maximum legal value is ^L.', STATUS )
\end{verbatim}
\normalsize

\item Checks for the error conditions {\small PAR\_\_NULL} and {\small
PAR\_\_ABORT} should be made and, if found, these errors should be annulled
before returning from an application. 

\item The very last thing that an application does before returning should be to
check the status and issue a contextual report if an error has occurred. 

\item Output {\small NDF}s should be deleted if an error occurs while output 
data is being calculated.

\small
\begin{verbatim}
   *  Create the output NDF.
         CALL NDF_PROP( INDF1, ' ', 'OUT', INDF2, STATUS )

   *  Map the output data array.
         CALL NDF_MAP( INDF2, 'DATA', '_REAL', 'READ', IPOUT, EL, STATUS )

   *  Abort if an error has occurred.
         IF( STATUS .NE. SAI__OK ) GO TO 999

   *  Fill the output data array with values.
         CALL CTEMC3( EL, %VAL( IPOUT ), STATUS )

   *  If an error has occurred, delete the output NDF.
    999  CONTINUE
         IF( STATUS .NE. SAI__OK ) CALL NDF_DELET( INDF2, STATUS )
\end{verbatim}
\normalsize

\item The A-task prologue should conform to the style generated by the Language 
Sensitive Editor {\small STARLSE} (see SUN/105).

\item The first two words of the A-task prologue ``Description'' section should be ``This 
routine''.

\item The ``Description'' section should be as short as possible and give just a
general over-view of what the application does. Different ways of using the
application should not be included in the ``Description'' but should be put into
separate ``DIY topics'' sections, following the ``Examples'' section (eg see
{\small TRACECRDD} prologue). 

\item Any general information included in the A-task prologue which will need to
be duplicated in other applications should be removed from the prologue and put
into the on-line help library. The prologue should contain a reference to the
relevant topic within the help library. (This will only be of relevance to
people updating the released {\small IRAS90} system). 

\item References to the on-line help library should be made if any potentially
unfamiliar general terms or concepts are mentioned in the prologue (rather than
duplicating descriptions within the applications prologue). For example,
``Detector numbers should be given in the form of a group expression (see help
on ``Group\_Expressions'')''. 

\item The A-task prologue should contain a description of how {\em each} {\small NDF}
component is processed, particularly a description of what happens to the
{\small QUALITY} and {\small VARIANCE} components should be included.

\item The ``Usage'' and ``Examples'' sections of the standard {\small STARLSE}
prologue should be filled in. 

\item Parameter should be described in alphabetical order.

\item {\em All} parameters used by the application should be described in the 
prologue (except for {\small NDFLIST}, see item \ref{IT:NDFLIST}).

\item If a parameter is normally defaulted, the description of the parameter
should include the default value by placing it in square brackets at the end of
the parameter description. 

\item The parameter descriptions should make it clear if a scalar value, array 
or group expression is required for the parameter.

\item Information should not be duplicated in different sections of the 
prologue, but should be included once only, with relevant references in other 
sections.

\item Parameters associated with groups of {\small NDF}s should refer to a 
``group'' of {\small NDF}s (not a ``set'', or ``list'', etc).

\item If a parameter is associated with a group expression, and if modification
elements are allowed within the group expression, then a description of what the
``{\small NAME\_TOKEN}'' character (usually an asterisk) refers to should be
included in the parameter description. 

\item There should be no occurrences of the word ``{\small ADAM}'' within the 
prologue.

\item Documentation generated from the A-task prologue using {\small PROLAT}
(see SUN/110) should successfully pass through a spelling checker such as {\small 
SPELL}.

\item All parameters used within the application should have entries in the 
interface file.

\item Parameters should be listed in alphabetical order in the interface file.

\item The {\small HELPLIB} field in the interface file should have the value
``{\small IRAS90\_DIR:IRAS90\_HELP}'' (on {\small VMS} systems), or 
``{\small \$IRAS90\_DIR/iras90\_help.shl}'' (on {\small UNIX} systems).

\item Each parameter should have a {\small HELPKEY} field with the value 
``\lsk''.

\item The following associations with ``global'' parameters should be used:
\begin{description}
\item [GLOBAL.DATA\_ARRAY ] should be read by parameters associated with input
{\small NDF}s, and written by ``{\small NDFLIST}'' parameters (see item
\ref{IT:NDFLIST}), and {\em not} by parameters associated with output {\small
NDF}s unless there is no {\small NDFLIST} parameter. 
\item [GLOBAL.MSG\_FILTER ] should be read and written by ``{\small 
MSG\_FILTER}'' parameters (see item \ref{IT:MSG_FILTER}).
\item [GLOBAL.QUALITY\_EXP ] should be read and written by parameters associated
with quality expression (but only used to provide the suggested default, i.e.
the {\small PPATH}, not the {\small VPATH}). 
\item [GLOBAL.IRAS90\_HISTORY] should be read and written by all ``{\small 
HISTORY} parameters'' (see item \ref{IT:HIS}).
\item [GLOBAL.SKY\_COORDS] should be read and written by all parameters 
associated with a sky coordinate system.
\end{description}

\item As many parameters as possible should be normally defaulted.

\item Positional parameters should be given positions in the same order in which
they are prompted for if not supplied on the command line. 

\item {\small IRAS} wave-bands should be obtained using the routine {\small
IRM\_GTBND} specifying the parameter ``{\small BAND}''. 

\end{enumerate}

\section{Compiling, Linking and Running}
\subsection{On VMS Systems}
Before starting to develop {\small IRAS90} software the following commands 
should be issued:

\small
\begin{verbatim}
   $ ADAMSTART
   $ ADAMDEV
   $ IRAS90
   $ IRAS90_DEV
\end{verbatim}
\normalsize

The A-task routine and all the application specific routines should then be
compiled as usual using the {\small FORTRAN} command. An object library should
then be created and the object modules for all the application specific routines
({\em not} the A-task object module) should be inserted into the library. If the
application name is {\small FRED}, the commands would be: 

\small 
\begin{verbatim}
   $ LIB/CRE FRED.OLB
   $ LIB FRED FREDA0
   $ LIB FRED FREDA1
     ...  
     (etc)
     ...
\end{verbatim}
\normalsize

This assumes that the application conforms to the conventions described in
section \ref{SEC:CONV} in that the names of the application specific routines
end in {\small A0, A1, ... B0, B1, ... } etc. The application FRED can then be
linked using the following command: 

\small
\begin{verbatim}
   $ ALINK FRED,FRED/LIB,IRAS90_LINK_ADAM/OPT
\end{verbatim}
\normalsize

Before running the application an interface file should be created 
(a text file with the same name as the application and a file type 
of {\small .IFL}). The contents of the interface file should conform to the 
conventions described in section {\ref{SEC:CONV}. Again, it is a good idea to 
use an existing {\small IRAS90} interface file as a basis for a new one. The 
simplest way to run the application is just to use the {\small RUN} command:

\small
\begin{verbatim}
   $ RUN FRED
\end{verbatim}
\normalsize

However, the {\small RUN} command does not allow anything to appear after the 
application name on the command line, and so command line specification of 
parameter values and keywords is not possible. To get round this a ``foreign 
command'' must be set up to run the application. To do this first ensure that a 
logical name exists pointing to the directory containing files {\small FRED.EXE} 
and {\small FRED.IFL}. If this logical name is {\small HERE}, then the command:

\small
\begin{verbatim}
   $ FRED:==$HERE:FRED
\end{verbatim}
\normalsize

will define the foreign command {\small FRED}, so that the application can 
be run (from any directory) with command line parameter assignments:

\small
\begin{verbatim}
   $ FRED IN=M31 OUT=*_PROC
\end{verbatim}
\normalsize

To run the application from {\small ICL}, first start up {\small ICL} and then 
issue the command:

\small
\begin{verbatim}
   ICL> DEFINE FRED HERE:FRED
\end{verbatim}
\normalsize

Again, {\small HERE} is a logical name pointing to the directory containing
{\small FRED.EXE} and {\small FRED.IFL}. Note, this logical name must be a {\em 
job} logical name, set up prior to starting {\small ICL} with the command:

\small
\begin{verbatim}
   $ DEFINE/JOB HERE <directory specification>
\end{verbatim}
\normalsize

\subsection{On UNIX Systems}
Before starting to develop {\small IRAS90} software the following commands 
should be issued:

\small
\begin{verbatim}
   % iras90
   % iras90_dev
\end{verbatim}
\normalsize

Note, to develop {\small IRAS90} applications on {\small UNIX} systems requires
the {\small IRAS90} ``source directory'' (usually \verb+/star/iras90+) to be
present. It is possible for system managers to delete the source directory in
order to save disk space, without stopping the released {\small IRAS90}
applications from working. If the necessary {\small IRAS90} files do not exist
on your system the \verb+iras90_dev+ script will issue a warning. 

All the application specific routines should now be compiled to create
corresponding object modules using the \verb+f77+ command. If the application
name is \verb+fred+, the commands could be: 

\small 
\begin{verbatim}
   % f77 -O -c freda0.f
   % f77 -O -c freda1.f
     ...  
     (etc)
     ...
\end{verbatim}
\normalsize

This assumes that the application conforms to the conventions described in
section \ref{SEC:CONV} in that the names of the application specific routines
end in \verb+a0+, \verb+a1+, ... \verb+b0+, \verb+b1+, ... etc. An archive
library should then be created and the object modules for all the application
specific routines ({\em not} the A-task object module) should be inserted into
the archive. The archive should finally be ``randomised'': 

\small 
\begin{verbatim}
   % ar c fred.a
   % ar r fred.a freda0.o
   % ar r fred.a freda1.o
     ...  
     (etc)
     ...
   % ranlib fred.a
\end{verbatim}
\normalsize

The application \verb+fred+ can then be linked using the following command: 

\small
\begin{verbatim}
   % alink fred.f fred.a -L/star/lib -L/star/iras90/lib \
     `iras90_link`
\end{verbatim}
\normalsize

Note the use of left quote marks (\verb+`+) rather than the more usual right
quote marks (\verb+'+). After completing all compilations and linkings, the
command: 

\small
\begin{verbatim}
   % iras90_dev remove
\end{verbatim}
\normalsize

should be executed in order to remove unnecessary files (actually soft links to
other files) created by the initial \verb+iras90_dev+ command. The application
can be run by just typing its name together with any command line parameter
assignments or keywords: 

\small
\begin{verbatim}
   % fred in=m31 out=\*_proc
\end{verbatim}
\normalsize

If the application is to be run from a directory other than the one containing
the executable and interface files, then an alias can be set up for it using the
command: 

\small
\begin{verbatim}
   % alias fred <directory path>/fred
\end{verbatim}
\normalsize
\appendix
\section{The NDG Library}
\label{APP:NDG}
The {\small NDG} library (which is used for accessing multiple {\small NDF}s
using a single parameter) started life as an {\small IRAS90} utility library
called {\small IRG}. It will eventually be released as a Starlink subroutine
library, independent of {\small IRAS90}. Until then, it continues to be
distributed as part of the {\small IRAS90} package in the same way as the other
utility libraries. The draft documentation is in the file {\small SUN2.TEX}
which is stored in the same places as the other utility library documentation.
This documentation is rather sketchy at the moment, but hopefully it will be
sufficient, when combined with studies of existing {\small IRAS90} application,
to enable programmers to use {\small NDG}. 

\end{document}
