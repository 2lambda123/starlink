\documentstyle{article}
\pagestyle{myheadings}

%------------------------------------------------------------------------------
\newcommand{\irasdoccategory}  {IRAS90 Document}
\newcommand{\irasdocinitials}  {ID}
\newcommand{\irasdocnumber}    {12.4}
\newcommand{\irasdocauthors}   {D.S. Berry}
\newcommand{\irasdocdate}      {24th May 1993}
\newcommand{\irasdoctitle}     {IRI: IRAS90 Image Handling Routines}
%------------------------------------------------------------------------------

\newcommand{\numcir}[1]{\mbox{\hspace{3ex}$\bigcirc$\hspace{-1.7ex}{\small #1}}}
\newcommand{\irasdocname}{\irasdocinitials /\irasdocnumber}
\renewcommand{\_}{{\tt\char'137}}     % re-centres the underscore
\markright{\irasdocname}
\setlength{\textwidth}{160mm}
\setlength{\textheight}{240mm}
\setlength{\topmargin}{-5mm}
\setlength{\oddsidemargin}{0mm}
\setlength{\evensidemargin}{0mm}
\setlength{\parindent}{0mm}
\setlength{\parskip}{\medskipamount}
\setlength{\unitlength}{1mm}

%------------------------------------------------------------------------------
% Add any \newcommand or \newenvironment commands here
%------------------------------------------------------------------------------

\begin{document}
\thispagestyle{empty}
SCIENCE \& ENGINEERING RESEARCH COUNCIL \hfill \irasdocname\\
RUTHERFORD APPLETON LABORATORY\\
{\large\bf IRAS90\\}
{\large\bf \irasdoccategory\ \irasdocnumber}
\begin{flushright}
\irasdocauthors\\
\irasdocdate
\end{flushright}
\vspace{-4mm}
\rule{\textwidth}{0.5mm}
\vspace{5mm}
\begin{center}
{\Large\bf \irasdoctitle}
\end{center}
\vspace{5mm}
\setlength{\parskip}{0mm}
\tableofcontents
\setlength{\parskip}{\medskipamount}
\markright{\irasdocname}

\section{Introduction}
IRAS90 applications such as PREPARE and MAPCRDD produce images in which
IRAS-specific global properties of the image (such as waveband, etc) are stored
in the IRAS extension of the NDF holding the image. This document describes the
components of the IRAS extension used to hold such information, and also
describes the package of subroutines which should be used for accessing IRAS
images. Following Starlink programming standards, these subroutines are
identified by names of up to five characters, together with the prefix
``IRI\_''.

It is assumed that the reader is familiar with the following:
\begin{itemize}
\item NDF structures and access routines as described in SUN/33.
\item The HDS system as described in SUN/92.
\item The IRAS90 astrometry package as described in ID/2.
\item The NDF group access package as described in SUN/2.
\end{itemize}

\section{Structure of an IRAS Image}
In order to be a ``legal IRAS90 image'', an NDF must satisfy the following
criteria:

\begin{enumerate}

\item The DATA array must be two dimensional. A two dimensional NDF section is
also acceptable.

\item The NDF must contain an extension called ``IRAS''.

\item The IRAS extension must contain a component structure named ASTROMETRY
containing astrometry information as produced by the IRA package.

\item The IRAS extension must contain a component structure named IMAGE\_INFO,
with HDS {\em type} IMAGE\_INFO.

\item The IMAGE\_INFO structure must not contain any components other than those
listed in appendix \ref{APP:ITEMS}.

\item The IMAGE\_INFO structure {\bf {\em must} } contain the components
INSTRUMENT, BAND and TYPE (included in the list in appendix \ref{APP:ITEMS}).

\item The IMAGE\_INFO structure must not contain any components which are not
relevant to the image type specified by component TYPE. Appendix \ref{APP:TYPE}
identifies such components.
\end{enumerate}

These are the only restrictions imposed upon IRAS images. In particular,
note the following:

\begin{itemize}

\item NDFs holding IRAS images may legally contain data in any units (as
indicated by the NDF component UNITS). However, it should be realised that if an
application produces an image with unusual units, then later application may not
be able to process it. In order to encourage consistency in the choice of units,
a set of ``standard'' units is listed in appendix \ref{APP:UNITS}. Applications
should where-ever possible produce images in one of these standard systems of
units. If future applications require some system not included in this list,
then consideration should be given to including the new system in the list, and
modifying the IRI package (and other applications) to recognise it. The routine
IRI\_GETUN can be used to get a specification for one of the standard system of
units from the environment.

\item No restrictions are placed on the contents of the LABEL component of NDFs
holding IRAS images. However, as a general guide line, LABEL should be used to
hold a description of the physical quantity stored in the data array. For
instance, typical values may be {\em Surface Brightness}, {\em Flux}, etc.

\item Applications are free to add any components from the list in appendix
\ref{APP:ITEMS} to the IMAGE\_INFO structure, so long as they are mutually
compatible. In general, an application should add as many of these components as
it can, given the information available to the application regarding the image.

\item Images which were originally imported using a FITS medium may have an NDF
extension called FITS, holding the FITS header cards. Useful information about
the image may be obtained from this extension. The information stored in
IMAGE\_INFO may contain values derived from these FITS header cards.

\item If applications want to store information about the image for which there
is no corresponding item in the list given in appendix \ref{APP:ITEMS}, then the
application should store the information in one or more components within the
IRAS extension. Such information must {\em not} be stored within the IMAGE\_INFO
structure. However, if appropriate, consideration should be given to formally
adding a new component to the list of legal IMAGE\_INFO components in which to
store such information, and modifying IRI (and other applications) to recognise
it.

\end{itemize}


\section{Facilities Provided by the IRI Package}
The IRI package provides facilities for the following:
\begin{itemize}
\item Checking that an existing NDF is a valid IRAS image.
\item Adding mandatory items to an existing NDF to establish it as a valid
IRAS image.
\item Obtaining the values of mandatory IMAGE\_INFO components.
\item Obtaining an HDS locator to the IMAGE\_INFO structure.
\item Reading and writing values for the NDF UNITS component.
\item Obtaining and checking specifications of standard units.
\item Displaying the contents of an IMAGE\_INFO structure in a user-friendly
format.
\end{itemize}

\begin{quote}
\begin{center}
{\em
Note, IRI {\em does not} provide facilities for reading or writing any of the
optional items listed in appendix \ref{APP:ITEMS}. Applications should use calls
to HDS routines to do this, using the HDS locator provided by the appropriate
IRI routine.
}
\end{center}
\end{quote}


\section{Using IRI Subroutines}

\subsection {Constants and Error Values}
\label {SEC:CON}
The IRI package has associated with it various symbolic constants defining such
things as standard UNITS, etc. These values consist of a name of up to 5
characters prefixed by ``IRI\_\_''  (note the {\em double} underscore), and can
be made available to an application by including the following line at the
start of the routines which uses them:

\begin{verbatim}
      INCLUDE 'IRI_PAR'
\end{verbatim}

Another set of symbolic constants is made available by the statement

\begin{verbatim}
      INCLUDE 'IRI_ERR'
\end{verbatim}

These values have the same format of those contained in IRI\_PAR, put define
various error conditions which can be generated within the IRI package.
Applications can compare the $STATUS$ argument with these values to check for
specific error conditions. These values are described in appendix \ref
{APP:ERRORS}.

\subsection{Gaining Access to Existing IRAS Image Information}
Applications which process existing IRAS images need to consider the following
steps:
\begin{itemize}

\item Obtain an NDF identifier for the image, for instance using the NDF\_ or
NDG\_ packages.

\item Call routine IRI\_OLD to check that the NDF is a valid IRAS image, and
obtain a locator to the IMAGE\_INFO structure. This call also returns the
mandatory IMAGE\_INFO components, and the value of the NDF
component UNITS.

\item The application must then check that the image is of a form which the
application knows how to process. This will usually involve comparing the value
of the mandatory IMAGE\_INFO component TYPE against a list of values which can
be processed.

\item Applications should always give consideration to the value of the NDF
UNITS component, to decide if the data is in a system of units which the
application can meaningfully process. IRI\_OLD does not check that the image is
in one of the``standard'' system of units (as listed in appendix
\ref{APP:UNITS}), so applications should be aware of the possibility of images
having arbitrary units. When comparing the value of the UNITS component with the
standard UNITS values, applications should always use an IRI symbolic constant
(listed in appendix \ref{APP:UNITS}) rather than literal strings. This removes
the possibility of unintentional mis-spelling of standard UNITS values which
might otherwise cause problems.

\item When the information stored in IMAGE\_INFO is no longer needed, the
locator should be annulled by calling DAT\_ANNUL.

\end{itemize}

\subsection{Creating New IRAS Image Information}
Applications which create IRAS images need to consider the following
steps:
\begin{itemize}

\item Obtain an NDF identifier for the image, for instance using the NDF\_ or
NDG\_ packages.

\item Write values to the standard NDF components DATA, VARIANCE, etc.

\item Decide what units to use. The routine IRI\_GETUN can be used to obtain a
standard system of units from the environment. Routine IRI\_CHECK can be used to
check that a given system of units is standard, and to set the case of each
character within the units specification to the standard form.

\item Call routine IRI\_NEW to create the IRAS extension, and the IMAGE\_INFO
structure within the IRAS extension, and return an HDS locator to the
IMAGE\_INFO structure. IRI\_NEW also stores values for the mandatory IMAGE\_INFO
components, and the NDF component UNITS. Note, an error is reported if a
non-standard UNITS value is given. This error should be annulled if the
application {\em must} use non-standard units.

\item The application can then add any set of mutually compatible
items to the IMAGE\_INFO structure, selected from the list given in appendix
\ref{APP:ITEMS}. HDS calls should be used to do this (eg DAT\_NEWxx and
CMP\_PUTxx). In general, an application should add all items for which it has
the relevant information.

\item When the IMAGE\_INFO structure is complete, the
locator should be annulled by calling DAT\_ANNUL.

\item An astrometry structure should be added to the IRAS extension (for
instance by calling IRA\_CREAT) before the NDF identifier is annulled. If this
is not done, later IRAS90 applications will not be able to process the image.

\end{itemize}

\section {Compiling and Linking with IRI}
\label{SEC:LINK}
This section describes how to compile and link applications which use IRI
subroutines, on both VMS and UNIX systems. It is assumed that the IRAS90 package
is installed as part of the Starlink Software Collection.

\subsection{VMS}
Each terminal session which is to include the compilation or linking of
applications which use the IRI package should start by issuing the commands:

\begin{verbatim}
$ ADAMSTART
$ ADAMDEV
$ IRAS90
$ IRAS90_DEV
\end{verbatim}

These commands set up logical names related to all the IRAS90
subsystems, including IRI.

To link a VMS ADAM application with the IRI package, the linker options file
IRAS90\_LINK\_ADAM should be used. For example, to compile and link an ADAM
application called PROG with the IRI library, the following commands should be
used:

\begin{verbatim}
$ ADAMSTART
$ ADAMDEV
$ IRAS90
$ IRAS90_DEV
$ FORT PROG
$ ALINK PROG, IRAS90_LINK_ADAM/OPT
\end{verbatim}

Stand-alone (i.e. non-ADAM) applications can be linked with the ``standalone''
version of IRI. This version excludes the routines described in appendix
\ref{APP:ADAM}. To do this the link options file IRAS90\_LINK should be used
instead of IRAS90\_LINK\_ADAM. Thus to compile and link a stand-alone
application with the IRI package, the following commands should be given:

\begin{verbatim}
$ IRAS90
$ IRAS90_DEV
$ FORT PROG
$ LINK PROG, IRAS90_LINK/OPT
\end{verbatim}

\subsection{UNIX}


Each terminal session which is to include the compilation or linking of
applications which use any IRAS90 sub-package should do the following:

\begin{enumerate}
\item Execute the {\bf iras90} command. This sets up an alias for the
{\bf iras90\_dev} command.

\item Execute the {\bf iras90\_dev} command.  This creates soft links
within the current directory to
all the iras90 include files, defines an environment variable IRAS90\_SRC
pointing to the IRAS90 source directory, adds the {\bf bin}
sub-directory of IRAS90\_SRC on to the end of the current value for the
environment variable PATH, and adds the {\bf lib} sub-directory of
IRAS90\_SRC on to the end of the current value for the environment
variable LD\_LIBRARY\_PATH. The command {\bf iras90\_dev star} will do
the same but will also create soft links to all starlink include files
(i.e. all files in the directory /star/include).

\item If a UNIX application accesses any include files (such as
IRI\_PAR) then they should be specified (within the source file) in upper case
without any directory
path. The I90\_PAR file (for instance) can be included using the statement

\verb+      INCLUDE 'I90_PAR'+\\

\item Soft links can be deleted (using the {\bf rm} command) when no longer
needed.
The command {\bf iras90\_dev remove} will remove all IRAS90 soft links from
the current directory. The command {\bf iras90\_dev remove star} will
remove all IRAS90 and Starlink soft links from the current directory.
Note, soft links are only accessable in the directory from which the {bf ln}
or {\bf iras90\_dev} command was issued.
\item For ADAM applications the following {\bf alink} command should be used:

\verb+% alink prog.f -L$IRAS90_SRC/lib `iri_link_adam`+\\

where {\bf prog.f} is the fortran source file for the ATASK.
Note the use of opening apostrophies (`) instead of the more usual closing
apostrophy (') in the above {\bf alink} command.

\item For a stand-alone program the following {\bf f77} command should be used:

\verb+% f77 prog.f -o prog -L$IRAS90_SRC/lib `iri_link`+\\
\end{enumerate}

\appendix
\section{Routine Descriptions}
% Command for displaying routines in routine lists:
% =================================================

\newcommand{\noteroutine}[2]{{\small \bf #1} \\
                              \hspace*{3em} {\em #2} \\[1.5ex]}

\subsection {Routine List}
\noteroutine{IRI\_CHECK( UNITS, OK, STATUS )}
   {Check a system of units is standard, and correct the case of each character.}
\noteroutine{IRI\_GETUN( PARAM, DEFLT, UNITS, STATUS )}
   {Get a standard system of units from the environment.}
\noteroutine{IRI\_NEW( INDF, INSTRM, BAND, TYPE, UNITS, LOC, STATUS)}
   {Create new image information within an NDF.}
\noteroutine{IRI\_OLD( INDF, INSTRM, BAND, TYPE, UNITS, LOC, STATUS)}
   {Access existing image information within an NDF.}
\noteroutine{IRI\_TRACE( LOC, ROUTNE, STATUS)}
   {Display a specified IMAGE\_INFO structure.}

\section{Full Routine Specifications}
\label {SEC:FULLSPEC}
\newlength{\sstbannerlength}
\newlength{\sstcaptionlength}
\newcommand{\sstroutine}[3]{
   \goodbreak
   \rule{\textwidth}{0.5mm}
   \vspace{-7ex}
   \newline
   \settowidth{\sstbannerlength}{{\Large {\bf #1}}}
   \setlength{\sstcaptionlength}{\textwidth}
   \addtolength{\sstbannerlength}{0.5em}
   \addtolength{\sstcaptionlength}{-2.0\sstbannerlength}
   \addtolength{\sstcaptionlength}{-4.45pt}
   \parbox[t]{\sstbannerlength}{\flushleft{\Large {\bf #1}}}
   \parbox[t]{\sstcaptionlength}{\center{\Large #2}}
   \parbox[t]{\sstbannerlength}{\flushright{\Large {\bf #1}}}
   \begin{description}
      #3
   \end{description}
}

\newcommand{\sstdescription}[1]{\item[Description:] #1}

\newcommand{\sstinvocation}[1]{\item[Invocation:]\hspace{0.4em}{\tt #1}}

\newcommand{\sstarguments}[1]{
   \item[Arguments:] \mbox{} \\
   \vspace{-3.5ex}
   \begin{description}
      #1
   \end{description}
}

\newcommand{\sstreturnedvalue}[1]{
   \item[Returned Value:] \mbox{} \\
   \vspace{-3.5ex}
   \begin{description}
      #1
   \end{description}
}

\newcommand{\sstparameters}[1]{
   \item[Parameters:] \mbox{} \\
   \vspace{-3.5ex}
   \begin{description}
      #1
   \end{description}
}

\newcommand{\sstexamples}[1]{
   \item[Examples:] \mbox{} \\
   \vspace{-3.5ex}
   \begin{description}
      #1
   \end{description}
}

\newcommand{\sstsubsection}[1]{\item[{#1}] \mbox{} \\}

\newcommand{\sstnotes}[1]{\item[Notes:] \mbox{} \\[1.3ex] #1}

\newcommand{\sstdiytopic}[2]{\item[{\hspace{-0.35em}#1\hspace{-0.35em}:}] \mbox{} \\[1.3ex] #2}

\newcommand{\sstimplementationstatus}[1]{
   \item[{Implementation Status:}] \mbox{} \\[1.3ex] #1}

\newcommand{\sstbugs}[1]{\item[Bugs:] #1}

\newcommand{\sstitemlist}[1]{
  \mbox{} \\
  \vspace{-3.5ex}
  \begin{itemize}
     #1
  \end{itemize}
}

\newcommand{\sstitem}{\item}
\sstroutine{
   IRI\_CHECK
}{
   Check that a units system is one of the standard systems
}{
   \sstdescription{
      This routine checks that the supplied units system belongs to the
      set of standard units system. The comparisons are case
      insensitive, but the supplied units system is replaced by a
      string in which the case of each letter corresponds to that of
      the appropriate standard system (e.g. if {\tt "}JY/SR{\tt "} is supplied,
      {\tt "}Jy/sr{\tt "} is returned). If the units system is recognised, OK is
      returned .true., otherwise it is returned .false.
   }
   \sstinvocation{
      CALL IRI\_CHECK( UNITS, OK, STATUS )
   }
   \sstarguments{
      \sstsubsection{
         UNITS = CHARACTER $*$ ( $*$ ) (Given and Returned)
      }{
         The units.
      }
      \sstsubsection{
         OK = LOGICAL (Returned)
      }{
         Flag indicating if the units are recognised.
      }
      \sstsubsection{
         STATUS = INTEGER (Given and Returned)
      }{
         The global status.
      }
   }
}
\sstroutine{
   IRI\_GETUN
}{
   Get a standard units system from the environment
}{
   \sstdescription{
      This routine uses the specified parameter to obtain a string
      identifying one of the standard system of image units. The
      returned string is equal to the value of one of the IRI symbolic
      constants (including case).
   }
   \sstinvocation{
      CALL IRI\_GETUN( PARAM, DEFLT, UNITS, STATUS )
   }
   \sstarguments{
      \sstsubsection{
         PARAM = CHARACTER $*$ ( $*$ ) (Given)
      }{
         The parameter to use.
      }
      \sstsubsection{
         DEFLT = CHARACTER $*$ ( $*$ ) (Given)
      }{
         The default value to suggest to the user.
      }
      \sstsubsection{
         UNITS = CHARACTER $*$ ( $*$ ) (Returned)
      }{
         The obtained units.
      }
      \sstsubsection{
         STATUS = INTEGER (Given and Returned)
      }{
         The global status.
      }
   }
}
\sstroutine{
   IRI\_NEW
}{
   Create an IMAGE\_INFO structure within an existing NDF
}{
   \sstdescription{
      This routine creates structures describing an IRAS image within a
      specified NDF. If the NDF does not contain an IRAS extension, one
      is created. Any IMAGE\_INFO structure which currently exists is
      deleted and a new one created with an HDS locator to it being
      returned to the calling application. Values are stored for the
      mandatory IMAGE\_INFO components BAND, INSTRUMENT and TYPE, and the
      NDF UNITS component. Note, no ASTROMETRY structure is created by
      this routine. The IRA package must be used to create such a
      structure before the image can be used by other IRAS90
      applications.
   }
   \sstinvocation{
      CALL IRI\_NEW( INDF, INSTRM, BAND, TYPE, UNITS, LOC, STATUS )
   }
   \sstarguments{
      \sstsubsection{
         INDF = INTEGER (Given)
      }{
         An identifier for the NDF holding the image.
      }
      \sstsubsection{
         INSTRM = CHARACTER $*$ ( $*$ ) (Given)
      }{
         The instrument from which the data originated; CPC or
         SURVEY.
      }
      \sstsubsection{
         BAND = INTEGER (Given)
      }{
         The IRAS waveband no of the data in the image. This must be
         in the range 1 - 4 for data from the survey array and 1 - 2
         for data from the CPC.
      }
      \sstsubsection{
         TYPE = CHARACTER $*$ ( $*$ ) (Given)
      }{
         The image type. Legal values are listed in ID/12. An error is
         reported if an unknown value is supplied.
      }
      \sstsubsection{
         UNITS = CHARACTER $*$ ( $*$ ) (Given)
      }{
         A string describing the system of units in which the DATA array
         values are stored. This string is stored as the NDF UNITS
         component. The supplied value must be on of the standard values
         listed in ID12, otherwise an error is reported. Note, in this
         case the error is reported after the units have been stored,
         so that annulling the error will enable an application to
         proceed without needing to store a new value for UNITS.
      }
      \sstsubsection{
         LOC = CHARACTER $*$ ( $*$ ) (Returned)
      }{
         An HDS locator to the created IMAGE\_INFO structure residing
         within the IRAS extension of the NDF. This locator should be
         annulled using DAT\_ANNUL when it is no longer needed.
      }
      \sstsubsection{
         STATUS = INTEGER (Given and Returned)
      }{
         The global status.
      }
   }
}
\sstroutine{
   IRI\_OLD
}{
   Get a locator to the IMAGE\_INFO structure of an existing
   NDF
}{
   \sstdescription{
      This routine checks that the supplied NDF holds a valid IRAS
      image. If the NDF is not a legal IRAS image then an error is
      reported. The values of the mandatory IMAGE\_INFO components
      BAND, INSTRUMENT and TYPE are returned, together with a locator
      to the IMAGE\_INFO structure. The value of the NDF component
      UNITS is also returned.
   }
   \sstinvocation{
      CALL IRI\_OLD( INDF, INSTRM, BAND, TYPE, UNITS, LOC, STATUS )
   }
   \sstarguments{
      \sstsubsection{
         INDF = INTEGER (Given)
      }{
         An identifier for the NDF holding the image.
      }
      \sstsubsection{
         INSTRM = CHARACTER $*$ ( $*$ ) (Returned)
      }{
         The instrument from which the data originated; CPC or
         SURVEY. The supplied variable should have a length equal
         to symbolic constant IRI\_\_SZINS. If the variable is too short
         the returned string will be truncated but no error will be
         reported.
      }
      \sstsubsection{
         BAND = INTEGER (Returned)
      }{
         The IRAS waveband no of the data in the image. This will be
         in the range 1 - 4 for data from the survey array and 1 - 2
         for data from the CPC.
      }
      \sstsubsection{
         TYPE = CHARACTER $*$ ( $*$ ) (Returned)
      }{
         The type of IRAS image; see ID/12 for a list of values. The
         supplied variable should have a length equal to symbolic
         constant IRI\_\_SZTYP. If the variable is too short the returned
         string will be truncated but no error will be reported.
      }
      \sstsubsection{
         UNITS = CHARACTER $*$ ( $*$ ) (Returned)
      }{
         The value of the NDF UNITS component. The supplied variable
         should have a length equal to symbolic constant IRI\_\_SZUNI. If
         the variable is too short the returned string will be
         truncated but no error will be reported.
      }
      \sstsubsection{
         LOC = CHARACTER $*$ ( $*$ ) (Returned)
      }{
         An HDS locator to the IMAGE\_INFO structure residing within the
         IRAS extension of the NDF. This locator should be annulled
         using DAT\_ANNUL when it is no longer needed.
      }
      \sstsubsection{
         STATUS = INTEGER (Given and Returned)
      }{
         The global status.
      }
   }
}
\sstroutine{
   IRI\_TRACE
}{
   Display information about an IRAS image
}{
   \sstdescription{
      This routine displays the information stored in the IMAGE\_INFO
      structure using the supplied routine to display each line of text.
   }
   \sstinvocation{
      CALL IRI\_TRACE( LOC, ROUTNE, STATUS )
   }
   \sstarguments{
      \sstsubsection{
         LOC = CHARACTER $*$ ( $*$ ) (Given)
      }{
         An HDS locator to the IMAGE\_INFO structure. This should have
         been obtained using routine IRI\_OLD or IRI\_NEW.
      }
      \sstsubsection{
         ROUTNE = EXTERNAL (Given)
      }{
         A routine to which is passed each line of text for display.
         It should have the same argument list as MSG\_\_OUTIF (see
         SUN/104), and should be declared EXTERNAL in the calling
         routine. All calls to this routine are made with priority
         MSG\_\_NORM.
      }
      \sstsubsection{
         STATUS = INTEGER (Given and Returned)
      }{
         The global status.
      }
   }
}

\section{Legal Components of the IMAGE\_INFO Structure}
\label{APP:ITEMS}
This appendix contains a list of the legal items of image information which may
be stored as components of the IMAGE\_INFO structure. The HDS {\em name} of each
item is given, followed by the HDS {\em type}, and a description of the values
which the component takes.

\begin{description}

\item [COLCOR $\langle$\_CHAR$\rangle$] - A comment describing any colour
correction that has been applied to the image. If this component is present then
theimage has been colour corrected, otherwise it has not. The content of the
string is arbitrary.

\item [BAND $\langle$\_INTEGER$\rangle$] - The IRAS waveband index. This will be in the range 1
to 4 if item INSTRUMENT has the value ``SURVEY'' (representing the 12, 25, 60
and 100 $\mu m$ bands), and in the range 1 to 2 if item INSTRUMENT has the value
``CPC'' (representing the 50 and 100 $\mu m$ bands).

\item [CPCRAW $\langle$\_LOGICAL$\rangle$] - True if the image is a raw CPC image. False if it
is a cleaned CPC image. Only applicable to CPC images.

\item [FIELDLAT $\langle$\_DOUBLE$\rangle$] - The sky latitude (eg Declination, Galactic
latitude, etc) of the reference position for the image. The value should be in
radians. If FIELDLAT is present, then FIELDLON and FIELDSCS must also be
present.

\item [FIELDLON $\langle$\_DOUBLE$\rangle$] - The sky longitude (eg Right Ascension, Galactic
longitude, etc) of a reference position for the image. The exact location of the
reference point is left up to each application to decide but may, for instance,
be the position of some object of interest within the image. The value should be
in radians. If FIELDLON is present, then FIELDLAT and FIELDSCS must also be
present.

\item [FIELDSCS $\langle$\_CHAR*(IRA\_\_SZSCS)$\rangle$] - The sky coordinate system in which
FIELDLON and FIELDLAT are given (eg EQUATORIAL, GALACTIC, etc). This string
should represent a valid IRA Sky Coordinate System. If FIELDSCS is present, then
FIELDLAT and FIELDLON must also be present.

\item [GALCEN $\langle$\_GALCEN$\rangle$] - If true then the image is a low resolution all-sky
map centred on the galactic centre. If false, then the image is a low resolution
all-sky map centred on the galactic anti-centre. This item is only applicable to
the ``Low Resolution All-sky Maps'' distributed by IPAC.

\item [GALMAP $\langle$\_INTEGER$\rangle$] - The map number of a galactic plane map. This item
is only applicable to the ``Galactic Plane Maps'' distributed by IPAC.

\item [HCON $\langle$\_INTEGER$\rangle$] - The HCON number of the data. This is only applicable
to survey data which contains data from a single HCON.

\item [INSTRUMENT $\langle$\_CHAR*(IRI\_\_SZINS)$\rangle$] - The IRAS instrument
which gave rise to the data. Can take the value ``CPC'' or ``SURVEY'' (the LRS
instrument is not suppored by IRAS90).

\item [ISSAFLD $\langle$\_INTEGER$\rangle$] - The IRAS Sky Survey Atlas
field number. Only applicable to ISSA images.

\item [MAPCRDD $\langle$\_LOGICAL$\rangle$] - If true, then the image was
created by IRAS90 application MAPCRDD.

\item [MAXSOP $\langle$\_INTEGER$\rangle$] - The maximum SOP number which contributed to the
data in the image. If item MINSOP does not also exist, then all the data is
assumed to come from the SOP identified by MAXSOP. If item MINSOP does exist,
then MINSOP must be less than or equal to MAXSOP.

\item [MINSOP $\langle$\_INTEGER$\rangle$] - The minimum SOP number which contributed to the
data in the image. If item MAXSOP does not also exist, then all the data is
assumed to come from the SOP identified by MINSOP. If item MAXSOP does exist,
then MINSOP must be less than or equal to MAXSOP.

\item [OBSNO $\langle$\_INTEGER$\rangle$] - The number of the observation from which the data in
the image was derived. This can only be used if all the data came from the same
observation, and is thus not allowed if items MAXSOP and MINSOP indicate that a
range of SOPs are represented in the image.

\item [POFLUX $\langle$\_LOGICAL$\rangle$] - If true, then the image is a PO
flux map. Otherwise it is an intensity PO map. Only applicable to PO
images.

\item [PONMAP $\langle$\_LOGICAL$\rangle$] - If true, then the image is a PO
noise map. Only applicable to PO images.

\item [PONOISE $\langle$\_REAL$\rangle$] - The median noise estimate within a
pointed observation image. Only applicable to PO images. This should be in the
units given by component POUNITS. If present, POUNITS must also be present.

\item [POUNITS $\langle$\_CHAR*(IRI\_\_SZUNI)$\rangle$] - The units in which
PONOISE is stored. If present, PONOISE must also be present.

\item [SKYFLUX $\langle$\_INTEGER$\rangle$] - The SKYFLUX plate number from which the
image was derived. This item is only applicable to the ``SKYFLUX'' images
distributed by IPAC.

\item [SKYWEIGHT $\langle$\_LOGICAL$\rangle$] - If true, then the image is a
SKYFLUX weight image. If false, it is a SKYFLUX intensity image. This item is
only applicable to the ``SKYFLUX'' images distributed by IPAC.

\item [TYPE $\langle$\_CHAR*(IRI\_\_SZTYP)$\rangle$] - The type of IRAS image.
Legal values are listed in appendix /ref{APP:TYPE}.

\item [YORTYPE $\langle$\_CHAR*(IRI\_\_SZYOR)$\rangle$] - Specifies the type of
data stored in a YORIC/HIRES image. It can take values given by any of the
following IRI parameters:
\begin{description}
\item [IRI\_\_YOIMG] - Surface brightness image.
\item [IRI\_\_YOPHN] - Photometric noise image.
\item [IRI\_\_YOCVG] - Coverage image.
\item [IRI\_\_YOCFV] - Correction factor variance image.
\item [IRI\_\_YORES] - Resolution image.
\end{description}
This item is only applicable to the ``YORIC'' or ``HIRES''  images distributed
by IPAC.

\end{description}

\section{Standard UNITS for IRAS Images}
\label{APP:UNITS}
This appendix gives a list of standard units recognised by routine IRI\_NEW.
The different system of units should be referred to by means of symbolic
constants listed below (see section \ref{SEC:CON}).

\begin{description}
\item [IRI\_\_JPS] - Janksy's per Steradian.
\item [IRI\_\_MJPS] - Mega-Janksy's per Steradian.
\item [IRI\_\_JPP] - Jansky' per pixel (note, care must be taken when dealing
with these units to ensure that the change in pixel size across large images is
taken account of).
\item [IRI\_\_FPS] - Flux (in Pico-Watts per square metre) per Steradian (this
system of units is independent of colour correction).
\item [IRI\_\_FPP] - Flux (in Pico-Watts per square metre) per pixel.
\end{description}

In addition the symbolic constant IRI\_\_UNITS contains a concatenated list of
all the standard units, seperated by commas.

\section{Legal Values for the TYPE Component}
\label{APP:TYPE}
The TYPE component of the IMAGE\_INFO structure holds a string identifying the
type of IRAS image. The legal values for these strings are given by the
following symbolic constants:
\begin{description}

\item [IRI\_\_CPC] - An image created from data obtained from the Chopped
Photometric Channel instrument. The IMAGE\_INFO component CPCRAW can be used to
give further information about such images.

\item [IRI\_\_SKYFL] - A SKYFLUX map. The IMAGE\_INFO components SKYFLUX and
SKYWEIGHT can be used to give further information about such images.

\item [IRI\_\_GALPL] - A galactic plane map. The IMAGE\_INFO component GALMAP
can be used to give further information about such images.

\item [IRI\_\_ALLSK] - An all sky  map. The IMAGE\_INFO component GALCEN can be
used to give further information about such images.

\item [IRI\_\_DSCO] - A Pointed Observation map produced using the IPAC DSCO
processor. The IMAGE\_INFO components POFLUX, PONOISE, PONMAP and POUNITS can be
used to give further information about such images.

\item [IRI\_\_ISSA] - An IRAS Sky Survey Atlas map. The IMAGE\_INFO component
ISSAFLD can be used to give further information about such images.

\item [IRI\_\_YORIC] - A map produced using the IPAC YORIC or HIRES processor.

\item [IRI\_\_MAPCR] - A map produced using the IRAS90 MAPCRDD application.

\item [IRI\_\_COLC] - A map produced using the IRAS90 COLCORR application.

\item [IRI\_\_NONAM] - Specifies that the map was created by some other process.

In addition the symbolic constant IRI\_\_TYPES contains a concatenated list of
all the legal types, seperated by commas.

\end{description}

\section{Templates for IRI Routines Within the VAX LSE Editor}
The STARLSE package (see SUN/105) provides facilities for initialising the VAX
Language Sensitive Editor (LSE) to simplify the generation of Fortran
code conforming to the Starlink programming standard (see SGP/16). One of the
facilities provided by LSE is the automatic production of argument lists for
subroutine calls. Templates for all the subroutines in the IRI package can be
made available within LSE by performing the following steps (within LSE):
\begin{enumerate}
\item Issue the LSE command GOTO FILE/READ IRI\_DIR:IRI.LSE
\item Issue the LSE command DO
\item Issue the LSE command DELETE BUFFER
\item Move to a buffer holding a .FOR of a .GEN file in the usual way.
\item IRI subroutine templates are then available by typing in the name of an
IRI subroutine (or an abbreviation) and expanding it (CTRL-E).
\item Help on the subroutine and its arguments can be obtained by placing the
cursor at a point in the buffer at which the subroutine name has been entered
and pressing GOLD-PF2.
\end{enumerate}

\section{Packages Called by IRI}
IRI\_ makes calls to the following packages:
\begin {description}
\item [IRA\_] - IRAS90 astrometry system.
\item [I90\_] - IRAS90 data values.
\item [CMP\_] - HDS; see SUN/92.
\item [DAT\_] - HDS; see SUN/92.
\item [ERR\_] - The Starlink error reporting package; see SUN/104.
\item [MSG\_] - The Starlink message reporting package; see SUN/104.
\item [NDF\_] - The NDF access package; see SUN/33.
\end{description}

Access to these packages, together with packages called from within these
packages, is necessary to use IRI.

\section{IRI Error Codes}
\label {APP:ERRORS}
IRI routines can return any $STATUS$ value generated by the subroutine packages
which it calls. In addition it can return the following IRI-specific values:

\begin{description}

\item {\bf IRI\_\_BADAS}\\
The astrometry structure store within the IRAS extension has an incorrect
HDS type (should be IRAS\_ASTROMETRY).

\item {\bf IRI\_\_BADBN}\\
A wave-band index supplied as an argument or retrieved from an NDF is out of
the legal range, given the value of INSTRUMENT.

\item {\bf IRI\_\_BADCM}\\
An illegal component has been found within the IMAGE\_INFO structure.

\item {\bf IRI\_\_BADII}\\
The IMAGE\_INFO structure has an illegal HDS type.

\item {\bf IRI\_\_BADIN}\\
An illegal value for INSTRUMENT has been found.

\item {\bf IRI\_\_BADOB}\\
The IMAGE\_INFO component OBSNO was found within an NDF which also contains
MAXSOP and MINSOP values indicating that data comes from more than one SOP.

\item {\bf IRI\_\_BADSO}\\
The value of IMAGE\_INFO component MAXSOP is less than MINSOP.

\item {\bf IRI\_\_BADTY}\\
The value of IMAGE\_INFO component TYPE is un-recognised.

\item {\bf IRI\_\_BADUN}\\
A non-standard system of units was supplied.

\item {\bf IRI\_\_INCOM}\\
Incompatible components were found within in IMAGE\_INFO.

\item {\bf IRI\_\_MISF}\\
Some, but not all, of the IMAGE\_INFO components FIELDLAT, FIELDLON and FIELDSCS
were found within an NDF.

\item {\bf IRI\_\_NOT2D}\\
The NDF is not two dimensional.

\end{description}

\section{Changes Introduced in the Current Version of this Document}
\label {SEC:CHANGES}

Changes introduced in version 4 of ID12:
\begin{itemize}
\item The new optional IMAGE\_INFO component COLCOR has been added.
\item The new image type IRI\_\_COLC has been added.
\end{itemize}

Changes introduced in version 3 of ID12:
\begin{itemize}
\item The new mandatory IMAGE\_INFO component TYPE has been added.
\item IMAGE\_INFO components CPCOBJ, POGRID, POMACRO and POOBSID have been
removed.
\item IMAGE\_INFO components PONMAP and POUNITS added.
\item If a non-standard system of units is supplied to IRI\_NEW, the routine
completes before reporting the error.
\item New routine IRI\_CHECK has been added.
\end{itemize}

Changes introduced in version 2 of ID12:
\begin{itemize}
\item The IRI package has been radically changed since the last version. The new
document should be read in its entirety and all applications which use IRI will
need to be modified.
\end{itemize}

\end {document}
