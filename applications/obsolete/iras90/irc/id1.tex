\documentstyle{article} 
\pagestyle{myheadings}

%------------------------------------------------------------------------------
\newcommand{\irasdoccategory}  {IRAS90 Document}
\newcommand{\irasdocinitials}  {ID}
\newcommand{\irasdocnumber}    {1.14}
\newcommand{\irasdocauthors}   {D.S. Berry}
\newcommand{\irasdocdate}      {25th August 1993}
\newcommand{\irasdoctitle}     {IRC: A CRDD handling subroutine package}
%------------------------------------------------------------------------------

\newcommand{\numcir}[1]{\mbox{\hspace{3ex}$\bigcirc$\hspace{-1.7ex}{\small #1}}}
\newcommand{\irasdocname}{\irasdocinitials /\irasdocnumber}
\renewcommand{\_}{{\tt\char'137}}     % re-centres the underscore
\markright{\irasdocname}
\setlength{\textwidth}{160mm}
\setlength{\textheight}{240mm}
\setlength{\topmargin}{-5mm}
\setlength{\oddsidemargin}{0mm}
\setlength{\evensidemargin}{0mm}
\setlength{\parindent}{0mm}
\setlength{\parskip}{\medskipamount}
\setlength{\unitlength}{1mm}

%------------------------------------------------------------------------------
% Add any \newcommand or \newenvironment commands here
%------------------------------------------------------------------------------

\begin{document}
\thispagestyle{empty}
SCIENCE \& ENGINEERING RESEARCH COUNCIL \hfill \irasdocname\\
RUTHERFORD APPLETON LABORATORY\\
{\large\bf IRAS90\\}
{\large\bf \irasdoccategory\ \irasdocnumber}
\begin{flushright}
\irasdocauthors\\
\irasdocdate
\end{flushright}
\vspace{-4mm}
\rule{\textwidth}{0.5mm}
\vspace{5mm}
\begin{center}
{\Large\bf \irasdoctitle}
\end{center}
\vspace{5mm}
\setlength{\parskip}{0mm}
\tableofcontents
\setlength{\parskip}{\medskipamount}
\markright{\irasdocname}
\section {Introduction}
The IRC\_ package comprises a set of utility subroutines which allow
applications running under either VMS or UNIX to access IRAS CRDD (Calibrated
Reconstructed Detector Data) in a way which makes as few assumptions as possible
about how the information is stored in the NDF. The main body of this document
is intended as a guide for {\em users } of the IRC\_ package, but a set of
appendices will be included containing information needed by {\em maintainers }
of the package. These appendices include specific details of the format in which
CRDD of different types is stored, and notes on the implementation of the
package. Users of IRC\_ should not need to know about these specific details, as
all access to CRDD information (except the DATA array) should be done
using IRC\_. Following Starlink programming standards, all the subroutines are
identified by names of up to five characters, together with the prefix
``IRC\_''. 

The routines within IRC\_ all apply to {\em reading } CRDD files. New CRDD files
can be created using the NDF\_ system, based on previously existing input CRDD
files. Such input CRDD files will exist for most applications {\em with the
exception of the application which extracts CRDD from the tape archives}. 

Access to the sample values which form the bulk data in a CRDD file can be
gained through the NDF\_ routines described in 
SUN/33. The IRG\_ package (see ID/5 ) provides facilities for acquiring NDF 
identifiers from a ``free format'' list of CRDD files, potentially including wild 
cards, explicit multiple names, indirection through text file, etc.

Throughout this document the term ``CRDD file'' is used to mean any file which 
holds data in the form of time ordered data streams from separate detectors. 
Technically, data is no longer true CRDD once any operation has been performed 
on, since it is no longer in the original form distributed by IPAC.

\section {CRDD Type Independency}
\label {SEC:INDEP}
Currently, most CRDD used in the UK is SURVEY data, and is distributed in a form
in which the pointing information is stored as a set of samples of boresight
angles. This may not always be true. The CRDD for Pointed Observations
(POs) may eventually be available in a form known as ``footprints'', in
which the pointing information is stored explicitly for each data sample. Other
forms of CRDD may become available in the future. It would obviously be a large
overhead if a new set of CRDD handling applications need to be written each 
time this happens. 

For this reason, a major design goal for the IRC\_ package was ``CRDD type
independency''. This means that applications need not know what type of CRDD is
being handled. Obviously, internally IRC\_ needs to know all the details of each
specific CRDD type, but users of the package should make very sure that the
applications they write {\em do not depend } on such information. This will be
the case if {\em all} access to CRDD information is done using IRC\_. Applications
should never use HDS to directly access the CRDD information held in the IRAS
NDF extension (if IRC\_ doesn't provide the function you require, then IRC\_
should be modified). If this rule is strictly adhered to, then future new types
of CRDD can be accommodated simply by modifying the IRC\_ routines. For this
reason, the main part of this document describes CRDD in terms of concepts
common to {\em all} CRDD types. All specific details of individual CRDD types
are contained in the appendices. 

\section {CRDD Support Information}
\label {SEC:SUPP}
Some types of CRDD file may carry round with them extra information over and
above that needed for the operation of IRC. This information is called ``support
information'' within this document. Examples of support information may be the
geographic longitude and latitude of the boresight at various times through the
scan, or uncertainties in satellite coordinates. The exact items of support
information available in a CRDD file will depend on the CRDD type, and they are
listed in the relevant appendices at the end of this document. The only support
provided by IRC for handling this information is the routine IRC\_SUPP which
returns an HDS locator to a requested item of support information (if it
exists). It is the responsibility of the calling application to ensure that this
information is correctly interpreted. Access to this information should be kept
to a minimum, as its use makes an application CRDD-type dependant. Applications 
should always allow for the possibility that a required item of support 
information may not be present in the supplied CRDD file.

\section {Terms Used to Describe CRDD Files}
\label {SEC:TERMS}
This section aims to define terms needed to describe a CRDD file. Again, care
has been taken to ensure that the concepts described are general enough
as to be applicable to forseeable future CRDD types.

For the purposes of the IRC\_ routines, a CRDD file is an NDF in which the DATA
array holds the calibrated reconstructed detector data values from a single
``scan''. The meaning of the word ``scan'' depends on the individual CRDD type
and definitions are contained in the corresponding appendices. Two examples are
a survey scan of arbitrary length, or a single leg of a pointed observation. The
data values in a CRDD file must all relate to the same IRAS wave band, must come
from the same Satellite Observation Plan (SOP), and must come from the same
observation (OBS) within the SOP. The data array is two dimensional (a size of
one for the second dimension is allowed). Each row of the array corresponds to
the data stream from a single detector, and can have an arbitrary time offset
relative to the other rows (in other words, the first sample from each detector
were not necessarily taken at the same time). Thus the first dimension steps
though sample numbers, and the second steps though detectors. Note, there is no
need for the DATA array to have a row for {\em each} detector in the band, the
number of rows can be less than the number of detectors in the band if some
detectors are to be excluded. 

Each element in the data array is called a ``sample'', not a pixel. The word
``pixel'' means a ``picture element'', so ``sample'' is used to emphasise that
the data is {\em not} a picture, but a set of time order data streams. Each 
sample can be characterised by the following general properties: 

\begin{description}

\item [Sample value] - This would usually be a value in units of Pico-Watts
($10^{-12}$ Watts) per square metre, but the UNITS component of the NDF should
be checked to confirm this. The sample values could conceivably be divided by
the effective detector solid angles to get a CRDD file in which the values were
surface brightness estimates rather than flux estimates. In fact applications
could in principle perform {\em any} operation on the sample values.
Applications should check UNITS if the nature of the sample values effects the
algorithm. Standard values for UNITS are listed in appendix \ref {APP:UNITS}. No
calibration corrections are applied by applications which use these values. 

\item [Sample number] - This is a value varying between the bounds of the
first dimension of the data array. These bounds need not necessarily start at 1
(see the NDF\_ system documentation, SUN/33). For instance, applications may
create an NDF {\em section} (see SUN/33) from a CRDD file, containing a subset
of the entire DATA array. In this case the first sample in the section retains
the sample number it had in the original {\em base} NDF, which in general will
not be 1. However, applications should always create {\em new} CRDD files 
with a lower bound of 1 (except if they are propagated from input CRDD file, in
which case they should inherit the bounds of the input CRDD file). 

In order to index the data array, an {\em integer} sample number must obviously
be used. However, several IRC routines use {\em floating point} sample numbers
which can have fractional values. This allows positions to be specified to a
greater accuracy. Applications can choose whether to just use the nearest
integer to index the data array, or whether to use some form of interpolation
between the sample values on either side of the fractional sample number. 

Sample numbers may extend beyond the bounds of the first dimension of the NDF
DATA array. Obviously in such situations, no actual {\em data values} are
available for the sample, but IRC will provide {\em positional} information for
such samples. This facility is often useful for positioning a scan relative to
some given position on the sky. The position of a sample which lies outside the
DATA array is obtained by extrapolation from the samples within the DATA array.
If an accurate extrapolated position cannot be obtained, IRC will return an
error status, and generate an error report. The calling application must then
decide how to handle this situation. 

\item [Time] - For each sample, this is the difference in UTC between the moment
each sample was taken and the moment sample number 1 was taken, in seconds.
The time of a given sample number can vary from detector to detector.

\item [Detector index] - An integer varying between the bounds of the second
dimension of the data array, and giving the row at which the detectors data
is stored. Note, the detector {\em index } does not explicitly 
identify a detector in terms of the detector {\em numbers } used in the 
Explanatory Supplement. The detector {\em index } locates the detector
within the NDF data array (i.e the row number holding the detectors data 
stream), the detector {\em number} locates it within the IRAS focal plane.
No assumptions are made about the order in which detectors are stored in the
NDF.

\item [Detector number] - Each of the 62 IRAS detectors was identified by a 
``Detector number'', an integer varying from 1 to 62. These are shown in the 
Explanatory Supplement figure II.C.6. The detector which gave rise to a
particular sample can be identified by these numbers. A vector of
detector numbers will be stored in the IRAS extension, giving an ``absolute''
identity to each detector index value. Each detector can have a maximum of 
one row of data in a CRDD file.

\item [Position] - This is the position on the sky corresponding to the detector
centre at the moment the sample was taken. All sky positions are given and
returned as double precision Right Ascension and Declination values (FK4,
B1950), in radians. If positions in other sky coordinate systems are required,
the IRAS astrometry package (IRA\_, see ID/2) can be used to do the conversion. 

\item [Scan angle] - This is the orientation of the path being traced across the
sky by a detector, and can vary from sample to sample. Near the poles, it can
also vary significantly from detector to detector. The angle is measured from
equatorial north to a line passing through the detector centre in the direction
of the positive focal plane Y axis. The angle is in radians and is measured
positive in the same sense as rotation from north to east (this is the standard
astronomical Position Angle (PA) of the positive Y axis). Scan angles are always
double precision values. See appendix \ref {APP:FOCALP} for a description of the
focal plane (Z,Y) coordinate system. 

\item [Scan speed] - This is the speed at which a source is moving across the
focal plane, in radians per second. This is a signed value to indicate if the
motion is ``with-survey'' (i.e in the same sense as the all sky survey), or
``anti-survey'' (i.e. in the opposite direction). With-survey motion is positive
and corresponds to sources moving across the focal plane in the positive Y
direction. Anti-survey motion is negative and corresponds to sources moving
across the focal plane in the negative Y direction (see appendix \ref 
{APP:FOCALP}).
\end{description}

\section {Using the IRC\_ Routines}
See appendix \ref {APP:EXAMS} for some code fragments showing typical examples 
of the use of IRC.

\subsection {Assumptions to be Avoided when Using IRC}
Applications which process CRDD files should be designed so that they relate to
the CRDD in terms of the general properties listed above. Concepts which may not
be defined for all types of CRDD should be avoided where ever possible (eg
``plates'', ``frames'' etc). Care should be taken to avoid making unjustified
assumptions about the relationship between data samples. Some particularly
assumption which used to be valid for BDF CRDD files but which should now be
avoided are:

\begin {enumerate}

\item {\em ``CRDD files always contain values in the same units''}. CRDD files
may contain data in any of the units listed in appendix \ref {APP:UNITS}, and
applications should check the UNITS component before deciding how to process the
data. If the application cannot process data with the supplied units, then a
suitable error should be reported. Applications which {\em write} CRDD files
should always produce values in one of the supported units systems. 

\item {\em ``Taking the same sample number from each row results in a set of
simultaneous samples''}. This may not be the case for all types of CRDD (eg PO
footprint data). However, substantial CPU savings can sometimes be made by an
application if each column of samples {\em are} in fact obtained simultaneously.
For this reason routine IRC\_SIMUL is provided which will return a logical value
indicating if the detector data samples are simultaneous or not. 

\item {\em ``Adjacent samples from a single detector are all equally spaced on 
the sky''}. Again, this may not be the case for all types of CRDD (eg PO
footprint data). It {\em can} be assumed that adjacent samples are equally
spaced in {\em time}, but not in {\em position}. 

\item {\em ``The first sample in the array is sample number 1''}. This need not 
be true now that NDFs are being used to store the CRDD (see the description of 
`` Sample Numbers'' in section \ref {SEC:TERMS} ).

\item {\em ``The detectors are stored in cross-scan order''}. This need not be
true, detectors can be stored in any order. The IRAS detectors numbers
identifying the detectors from which the data in each row of the NDF was created
are explicitly stored in CRDD file. The relationship between row numbers in the
DATA array (or ``detector index'', see section \ref {SEC:TERMS}) and IRAS
detector numbers can be obtained using IRC\_DETNO and IRC\_DETIN. 

\item {\em ``The scan angle is constant''}. This also is not true. Scan angle
(see section \ref{SEC:TERMS}) can vary along the length of a scan. {\em It can
also vary from detector to detector at a given point in the scan.} This latter
effect is only significant near the poles, but applications should not be
written which rely on the scan angle being the same for all detectors. 

\end {enumerate}

\subsection {Constants and Error Values}
The IRC package has associated with it various symbolic constants defining such
things as the required length of various character variables. These values
consist of a name of up to 5 characters prefixed by ``IRC\_\_''  (note the {\em
double} underscore), and can be made available to an application by including
the following lines at the start of the routines which uses them: 


\begin{verbatim}
      INCLUDE 'I90_DAT'
      INCLUDE 'IRC_PAR'
\end{verbatim}

The values thus defined are described in the following sections, and also in the
subroutine specifications. Another set of symbolic constants is made available
by the statement 

\begin{verbatim}
      INCLUDE 'IRC_ERR'
\end{verbatim}

These values have the same format of those contained in IRC\_PAR, put define
various error conditions which can be generated within the IRC package.
Applications can compare the $STATUS$ argument with these values to check for
specific error conditions. These values are described in appendix \ref
{APP:ERRORS}. 

\subsection {Initialising the IRC\_ System}
A call to IRC\_INIT must be made before calling any other IRC\_ routines. This 
routine initializes various common blocks.

\subsection {Obtaining  NDF Identifiers for Input CRDD Files}
The next step that most applications will need to do is to obtain a set of NDF 
identifiers for all the input CRDD files. These identifiers are given as input
to many NDF\_ routines and to IRC\_IMPRT, to identify the CRDD file to be 
operated on.
There is a finite limit to the number of NDF identifiers which can be 
defined simultaneously. Therefore it is a good idea to process CRDD files one 
at a time, rather than all together, so that resources can be released and 
re-used. Obviously this may not always be possible, in which case the possibility 
of hitting the limit should be born in mind. The actual value of the limit is 
hard to define as it depends on many things, but it will probably be of the 
order of one or two hundred, if the users open file quota does not impose a 
lower limit. If this presents a problem, the facilities provided by the IRG\_ 
(ID/5) package may be of help.

NDF identifiers for existing CRDD files can be obtained by calling routine
IRM\_CRDDF (see ID/8), followed by IRG\_NDFEX (see ID/5). IRM\_CRDDF uses the
facilities of the IRG package to obtain the name of a single CRDD file or the
names of a group of CRDD files. IRG\_NDFEX returns an NDF identifier for a
specified name within the group of names returned by IRM\_CRDDF. A CRDD file
must next be ``imported'' into the IRC\_ system by calling the IRC\_IMPRT
routine. This routine checks that the NDF is in fact a CRDD file, and reads data
from the CRDD\_INFO extension (see below) into various internal common blocks.
Note, it is possible to import an NDF {\em section} into IRC as well as a base
NDF. Indeed, there is nothing to stop the user specifying a section when giving
the NDF name in reponse to the parameter prompt from IRM\_CRDDF. However, if 
an NDF section is specified which is a {\em super}-set in the second dimension 
(i.e. if the bounds of the second dimension of the section extend beyond the 
bounds of the second dimension of the base NDF), then there will be no 
detector numbers corresponding to the extra rows used to pad the section out to 
the required size. In this case IRC\_IMPRT will report an error. No such error 
is reported if the section is a {\em sub}-set of the base NDF in the second 
dimension.

\subsection{IRC identifiers}
Routine IRC\_IMPRT returns an ``IRC identifier'' for the CRDD information 
associated with the given NDF. This identifier is passed to other IRC\_ routines 
to identify the CRDD file. IRC\_ANNUL should be called to annul the IRC 
identifier when access to the CRDD information is no longer needed.

\subsection {Accessing Sample Values}
The CRDD sample values are stored in the standard DATA component of the NDF, and 
can be accessed using the usual NDF\_ routines such as NDF\_MAP. Before 
processing the CRDD, applications should first get the value of the UNITS 
component using NDF\_CGET, and check that the data represents a physical 
quantity which the application knows how to handle. The legality of both the
UNITS and LABEL component values are checked by IRC\_IMPRT. If the stored values
are not recognised then an error report is made (see appendices 
\ref {APP:UNITS} and \ref {APP:LABEL} for lists of legal values).

\subsection {Obtaining Pointing Information}
Pointing information can be requested in several ways. IRC\_DPOS returns the sky
positions, scan angles and scan speeds for a specified small set of detector
samples. IRC\_BPOS is like IRC\_DPOS except that the returned position and scan
angle refer to the boresight instead of any given detector. (Note, IRC does not
impose the restriction that each column of the DATA array must contain
simultaneously obtained detector samples. For this reason IRC\_BPOS requires a
detector index as a given argument to establish the relationship between sample
number and boresight position.) IRC\_POSMP returns three arrays of the same size
and shape as the main data array, holding the two sky coordinates and scan 
angle at every detector sample. 

Positions of samples which lie outside the bounds of the data array can be
estimated using IRC\_BPOS and IRC\_DPOS (but {\em not} IRC\_POSMP). These
positions are based on extrapolation of the samples within the DATA array (for
instance SURVEY scans with boresight pointing data are extrapolated by assuming
constant cone angle and constant rate of change of clock angle. See the
Explanatory Supplement, chapter III). If an accurate extrapolated position is
not available, the subroutines inherited STATUS value is set to the symbolic
value IRC\_\_BADEX (see appendix \ref {APP:ERRORS} ) and an error report is
generated. The calling application can then decide how to handle this error. 

Routine IRC\_IMCO combines the above routines with various IRA (see ID/2)
routines, to provide the coefficients of a linear transformation which maps
focal plane coordinates (relative to the detector centre) to the image
coordinates system defined by a given ``IRA identifier''. This identifier
defines the mapping between sky coordinates and image coordinates, and is
usually obtained by calling one of the IRA routines IRA\_CREAT or IRA\_IMPRT.
The returned coefficients can either be used directly to transform focal plane
coordinates to image coordinates, or can be used to derive values such as the
position angle of the focal plane Y axis, the pixel dimensions, etc. 

If all that is required is a rough estimate of where abouts on the sky the data
is, the routine IRC\_INFO can be called which returns (amongst other things) the
position of the CRDD file ``reference point'', which is usually the
field centre requested when the data was extracted from the tape archives. 

\subsection {Transforming Between Focal Plane and Sky Coordinates}
At the moment a given detector sample is taken, there exists a transformation
between focal plane coordinates (see appendix \ref {APP:FOCALP}) and sky
coordinates. This transformation obviously changes as the boresight moves across
the sky. Routines IRC\_FPCO1 and IRC\_FPCO2 will perform the transformation from
sky coordinates to focal plane coordinates using full spherical geometry,
accurate for {\em any} point on the sky. The inverse transformation can be
performed by IRC\_SKCO1 and IRC\_SKCO2. These routines can be used to find the
in-scan or cross-scan distance (Y or Z focal plane values) from the boresight to
any point on the sky. 

\subsection {Moving Between Detector Samples}
Routine IRC\_DIST will find the in-scan distance joining two specified detector
samples. IRC\_OFFST will find the sample which is a given in-scan distance away
from a specified sample. IRC\_DCLAP will find the sample number which
corresponds to the closest approach of a given detector track to a specified sky
position. IRC\_BCLAP will do the same for the boresight track.(Note, IRC\_BCLAP
requires a detector index as a given argument to establish the relationship
between sample number and boresight position, which can vary from detector to
detector.) 

\subsection {Accessing Detector Numbers and Detector Indices}
The detector number corresponding to a given row in the DATA array can be found
using IRC\_DETNO. This is an integer {\em function}, not a subroutine. The
inverse of this operation is performed by function IRC\_DETIN which returns a
detector index (i.e. row number) which corresponds to a given detector number.
If the requested detector number is not included in the CRDD file then the
Starlink ``BAD'' value (VAL\_\_BADI) is returned. In addition, the routine
IRC\_DINDS will convert a list of detector numbers to the corresponding detector
indices, ignoring any detector numbers for which there is no data in the CRDD
file. 

\subsection {Accessing Other Information}
Other information is available by calling IRC\_INFO. This includes:

\begin{itemize}

\item The IRAS band number.

\item The position of the reference position.

\item The SOP number.

\item The OBS number.

\item A nominal scan speed.

\end{itemize}

These, and other values, are defined in section \ref {SEC:CRDD_INFO}. A locator
to an item of ``support information'' (see section \ref{SEC:SUPP}) may be
obtained using routine IRC\_SUPP. 

\subsection {Other Functions}
More routines are being added to IRC\_ to perform other functions. See
the routine specifications contained in appendix \ref {APP:ROUTS} for
a complete list.

\section {Data Structures}
A CRDD file makes use of some of the standard NDF components, but 
needs extra information held in the IRAS extension.

\subsection {NDF Components}
The following meanings are attached to the standard NDF components.

\begin{description}

\item [DATA] - Holds the CRDD sample values as described above.

\item [UNITS] - A character string describing the sample values held in the data
array. See appendix \ref {APP:UNITS} for a list of legal values and their
meanings. Applications need to check the UNITS value before deciding how to use
the sample values. A check is made by IRC\_IMPRT that the value held in UNITS 
corresponds to one of the legal values.

\item [TITLE] - A string which usually names the object observed (eg "M51", 
"NGC 5128" etc).

\item [LABEL] - A string describing the type of data; ``Pointed Observation
CRDD'' or ``Survey CRDD''. Legal values are listed in appendix \ref{APP:LABEL}.
A check is made by IRC\_IMPRT that the value held in LABEL corresponds to one of
the legal values. 

\item [VARIANCE] - This is optional, but could be used to store the variance of
each CRDD sample. General purpose Starlink software assume that these variance 
values represent independent, Gaussian noise samples. The limitations imposed 
by these assumptions are recognised, but this component can still be useful, 
especially within the IRAS package, where the limitations will be understood.

\item [QUALITY] - This is optional, but can be used to distinguish between 
different samples. For instance, quality could be used to identify samples
which are to be used when calculating destriping parameters. The IRQ\_ package 
(see ID/6) provides facilities for handling quality values in IRAS90.

\item [HISTORY] - This component is used to record any significant processes 
performed on the CRDD, and any important numerical values used (eg solid angles, 
bandwidths, etc). What goes into HISTORY is left to the discretion of each 
application.

\item [AXIS] - This component is left undefined within a CRDD file.

\end{description}

\subsection {Components of the CRDD\_INFO Structure}
\label{SEC:CRDD_INFO}
NDFs which contain CRDD have an extension with a {\em name} IRAS and
an HDS {\em type} of IRAS. This is an HDS structure used to store IRAS specific
information. It can potentially contain several components or sub-structures.
The IRC\_ package is associated with a particular sub-structure {\em named} 
``CRDD\_INFO''. CRDD\_INFO is an HDS object and has a {\em type} of
``CRDD\_INFO''. This object stores all the information required by the IRC\_
package. It contains the following components: 

\begin {description}

\item [DET\_NUMBERS] - A 1 dimensional \_INTEGER array. The size of this array
must be equal to the size of the second dimension of the {\em base} NDF. Each
element will hold the detector number of the corresponding row in the DATA
array. Thus if the bounds of the second dimension of the base NDF were [-5,3]
then DET\_NUMBERS(1) would hold the detector number corresponding to row -5,
DET\_NUMBERS(2) would hold the detector number corresponding to row -4, etc. 

\item [DET\_ORIGIN] - An \_INTEGER scalar giving the index within the 
DET\_NUMBERS array corresponding to the origin of the second dimension of the 
NDF. Thus if the bounds of the second dimension of the base NDF were [-5,3]
so that DET\_NUMBERS(1) holds the detector number corresponding to row -5, then
DET\_NUMBERS(6) would hold the detector number corresponding to row zero, and
so DET\_ORIGIN should have the value 6.

\item [BAND] - An \_INTEGER scalar in the range 1 to 4, giving the IRAS band 
number. Storing the band {\em number} (rather than wavelength) allows the
value of BAND to be used directly as an index into arrays holding band-specific 
information. On the other hand, applications should always use wavelengths
(12, 25, 60, 100 $\mu$m) rather than band numbers when communicating with the user.

\item [REF\_RA] - A \_DOUBLE scalar giving the Right Ascension (equinox B1950.0)
of a reference position in the vicinity of the sky region covered by the CRDD
file, given in units of radians. The reference position will usually be the
source position requested by the user when the CRDD file was created. 

\item [REF\_DEC] - A \_DOUBLE scalar giving the Declination (equinox B1950.0) 
of the reference position in radians.

\item [SOP] - An \_INTEGER scalar giving the Satellite Observation Plan number 
from which the data is derived.

\item [OBS] - An \_INTEGER scalar giving the Observation number within the
SOP from which the data is derived.

\item [NOM\_SPEED] - A real value giving the scan speed on the sky at the 
centre of the scan (see section \ref {SEC:TERMS}). This value will usually
be the equivalent of 3.85 arc-minutes per second for survey data, but may be
different for PO data. Also, the exact scan speed may vary with sample number
in a PO scan. 

\item [DETAILS] - All the above components are meaningful for any type of CRDD.
However, there will be information which is only meaningful in the context of
one particular type of CRDD (pointing data for instance). A set of HDS
structures will be defined to hold such information. They have differing HDS
{\em types} (restricted to 15 characters e.g. ``SURVEY\_BSIGHT'',
``PO\_FOOTPRINT'', etc). The component named DETAILS within the CRDD\_INFO
structure contains one (and only one) of these structures. The names and
contents of each of these structure types are described in the relevant
appendix. Any available support information is stored in a component of the 
DETAILS structure named SUPPORT\_INFO.

\end {description}
\newpage

The structure of a typical CRDD file is as follows:

\setlength{\textwidth}{190mm}
\setlength{\unitlength}{0.8mm}
\begin{picture}(190,170)
\put(5,150){\framebox(30,15){NDF}}
\put(20,150){\line(0,-1){15}}
\put(20,140){\line(1,0){150}}
\put(5,120){\framebox(30,15){MORE}}
\put(70,140){\line(0,-1){5}}
\put(55,120){\framebox(30,15){DATA}}
\put(120,140){\line(0,-1){5}}
\put(105,120){\framebox(30,15){VARIANCE}}
\put(170,140){\line(0,-1){5}}
\put(155,120){\dashbox(30,15){(etc)}}
\put(20,120){\line(0,-1){15}}
\put(20,110){\line(1,0){100}}
\put(5,90){\framebox(30,15){IRAS}}
\put(70,110){\line(0,-1){5}}
\put(55,90){\framebox(30,15){FITS}}
\put(120,110){\line(0,-1){5}}
\put(105,90){\dashbox(30,15){(etc)}}
\put(20,90){\line(0,-1){15}}
\put(20,80){\line(1,0){150}}
\put(5,60){\framebox(30,15){CRDD\_INFO}}
\put(70,80){\line(0,-1){5}}
\put(55,60){\framebox(30,15){DESTRIPE}}
\put(120,80){\line(0,-1){5}}
\put(105,60){\framebox(30,15){MEMCRDD}}
\put(170,80){\line(0,-1){5}}
\put(155,60){\dashbox(30,15){(etc)}}
\put(20,60){\line(0,-1){15}}
\put(20,50){\line(1,0){150}}
\put(5,30){\framebox(30,15){DETAILS}}
\put(70,50){\line(0,-1){5}}
\put(55,30){\framebox(30,15){BAND}}
\put(120,50){\line(0,-1){5}}
\put(105,30){\framebox(30,15){REF\_RA}}
\put(170,50){\line(0,-1){5}}
\put(155,30){\dashbox(30,15){(etc)}}
\put(20,30){\line(0,-1){15}}
\put(20,20){\line(1,0){50}}
\put(5,0){\framebox(30,15){ }}
\put(5,0){\shortstack{Type-specific\\details\\ \rule{0mm}{2mm}}}
\put(70,20){\line(0,-1){5}}
\put(55,0){\dashbox(30,15){ }}
\put(55,0){\shortstack{Support\\information\\ \rule{0mm}{2mm}}}
\end{picture}
\setlength{\textwidth}{160mm}

The IRC\_ routines access the components of the fifth and sixth rows, and 
NDF\_ routines access components of the second and third rows.

The {\bf DESTRIPE} and {\bf MEMCRDD} components are shown as examples of 
components of the IRAS structure which may be created by CRDD handling 
applications to store supplementary information about the CRDD file. Such 
structures will be defined in the documentation of the relevant application and 
{\em do not } form part of the standard structure of a CRDD file, as defined by this 
document. Routines to access such structures will form part of the individual 
application.

\section {Compiling and Linking with IRC}
\label{SEC:LINK}
This section describes how to compile and link applications which use IRC
subroutines, on both VMS and UNIX systems. It is assumed that the IRAS90 package
is installed as part of the Starlink Software Collection. 

\subsection{VMS}
Each terminal session which is to include the compilation or linking of 
applications which use the IRC package should start by issuing the commands:

\begin{verbatim}
$ ADAMSTART
$ ADAMDEV
$ IRAS90
$ IRAS90_DEV
\end{verbatim}

These commands set up logical names related to all the IRAS90
subsystems, including IRC. 

To link a VMS ADAM application with the IRC package, the linker options file
IRAS90\_LINK\_ADAM should be used. For example, to compile and link an ADAM
application called PROG with the IRC library, the following commands should be
used: 

\begin{verbatim}
$ ADAMSTART
$ ADAMDEV
$ IRAS90
$ IRAS90_DEV
$ FORT PROG
$ ALINK PROG, IRAS90_LINK_ADAM/OPT 
\end{verbatim}

Stand-alone (i.e. non-ADAM) applications can be linked with the ``standalone''
version of IRC. To do this the link options file IRAS90\_LINK should be used
instead of IRAS90\_LINK\_ADAM. Thus to compile and link a stand-alone
application with the IRC package, the following commands should be given: 

\begin{verbatim}
$ IRAS90
$ IRAS90_DEV
$ FORT PROG
$ LINK PROG, IRAS90_LINK/OPT 
\end{verbatim}

\subsection{UNIX}

Each terminal session which is to include the compilation or linking of 
applications which use any IRAS90 sub-package should do the following:

\begin{enumerate}
\item Execute the {\bf iras90} command. This sets up an alias for the 
{\bf iras90\_dev} command.

\item Execute the {\bf iras90\_dev} command.  This creates soft links
within the current directory to
all the iras90 include files, defines an environment variable IRAS90\_SRC
pointing to the IRAS90 source directory, adds the {\bf bin}
sub-directory of IRAS90\_SRC on to the end of the current value for the
environment variable PATH, and adds the {\bf lib} sub-directory of 
IRAS90\_SRC on to the end of the current value for the environment 
variable LD\_LIBRARY\_PATH. The command {\bf iras90\_dev star} will do
the same but will also create soft links to all starlink include files
(i.e. all files in the directory /star/include).

\item If a UNIX application accesses any include files (such as
IRC\_PAR) then they should be specified (within the source file) in upper case 
without any directory
path. The I90\_PAR file (for instance) can be included using the statement

\verb+      INCLUDE 'I90_PAR'+\\

\item Soft links can be deleted (using the {\bf rm} command) when no longer 
needed.
The command {\bf iras90\_dev remove} will remove all IRAS90 soft links from
the current directory. The command {\bf iras90\_dev remove star} will
remove all IRAS90 and Starlink soft links from the current directory.
Note, soft links are only accessable in the directory from which the {bf ln} 
or {\bf iras90\_dev} command was issued. 
\item For ADAM applications the following {\bf alink} command should be used:

\verb+% alink prog.f -L$IRAS90_SRC/lib `irc_link_adam`+\\

where {\bf prog.f} is the fortran source file for the ATASK.
Note the use of opening apostrophies (`) instead of the more usual closing
apostrophy (') in the above {\bf alink} command.

\item For a stand-alone program the following {\bf f77} command should be used:

\verb+% f77 prog.f -o prog -L$IRAS90_SRC/lib `irc_link`+\\
\end{enumerate}


\appendix
\section {Routine Descriptions}
\label {APP:ROUTS}

\subsection {Routine List}
In this list, the term {\em sample} is used to refer to a combination of a {\em
sample number} and a {detector index} (see section \ref{SEC:TERMS}).

% Command for displaying routines in routine lists:
% =================================================
\newcommand{\noteroutine}[2]{{\small \bf #1} \\
                              \hspace*{3em} {\em #2} \\[1.5ex]}
\noteroutine{IRC\_ANNUL( IDC, STATUS )}
   {Release a CRDD file from the IRC\_ system.}
\noteroutine{IRC\_BCLAP( IDC, DETIND, RA, DEC, CLSAMP, CLZFP, STATUS )}
   {Find the sample number corresponding to the closest approach of the 
    boresight to a given sky position.}
\noteroutine{IRC\_BPOS( IDC, NVAL, SAMPLE, DETIND, RA, DEC, ANGLE, SPEED, 
                        STATUS )}
   {Returns the position, speed and scan angle of the boresight at a set of 
    samples.}
\noteroutine{IRC\_CLOSE( STATUS )}
   {Close down the IRC\_ system.}
\noteroutine{IRC\_DCLAP( IDC, DETIND, RA, DEC, CLSAMP, CLZFP, STATUS )}
   {Find the sample number corresponding to the closest approach of a given
    detector to a given sky position.}
\noteroutine{IRC\_DETIN( IDC, DETNO, STATUS )}
   {Return a detector index for a given detector number.}
\noteroutine{IRC\_DETNO( IDC, DETIND, STATUS )}
   {Return a detector number for a given detector index.}
\noteroutine{IRC\_DINDS( IDC, NDETNO, DETNO, DETIN, NDETIN, STATUS )}
   {Converts a list of detector numbers to the corresponding detector indices.}
\noteroutine{IRC\_DIST( IDC, SAMP1, DETIN1, SAMP2, DETIN2, DIST, STATUS )}
   {Return the in-scan distance between two samples within a scan.}
\noteroutine{IRC\_DPOS( IDC, NVAL, SAMPLE, DETIND, RA, DEC, ANGLE, SPEED, 
                        STATUS )}
   {Returns the position, speed and scan angle of a given detector at a set 
    of sample numbers.}
\noteroutine{IRC\_FPCO1( IDC, SAMPLE, DETIND, NVAL, RA, DEC, ZFP, YFP, STATUS )}
   {Return the focal plane coordinates of many sky positions at the time 
    determined by a given sample.}
\noteroutine{IRC\_FPCO2( IDC, NVAL, SAMPLE, DETIND, RA, DEC, W1, W2, W3, 
                         W4, ZFP, YFP, STATUS )}
    {Return the focal plane coordinates of a single sky position at a set of
     times determined by the given set of samples.}
\noteroutine{IRC\_ILAB( LIST, STATUS )}
   {Return a list of legal values for NDF component LABEL.}
\noteroutine{IRC\_IMCO( IDC, SAMP, DETIN, IDA, INIT, C, STATUS ))}
   {Locate a detector sample within an image frame.}
\noteroutine{IRC\_IMPRT( INDF, IDC, STATUS )}
   {Imports a CRDD file into the IRC\_ system, given its NDF identifier.}
\noteroutine{IRC\_INFO( IDC, BAND, REFRA, REFDEC, NOMSPD, SOP, OBS, STATUS )}
   {Obtain values for the global properties of a CRDD file. }
\noteroutine{IRC\_INIT( STATUS )}
   {Initialises the IRC\_ system.}
\noteroutine{IRC\_IUNIT( LIST, STATUS )}
   {Return a list of legal values for NDF component UNITS.}
\noteroutine{IRC\_LIMIT( IDC, NDETS, DETIND, TRUNC, IDA, LBND, UBND, STATUS )}
   {Determine the image area covered by a CRDD file.}
\noteroutine{IRC\_OFFST( IDC, SAMP1, DETIN1, DETIN2, DIST, SAMP2, STATUS )}
   {Find a sample which is a given in-scan distance away from a given sample.}
\noteroutine{IRC\_POSMP( IDC, NEWNDF, PNTR, STATUS )}
   {Produce a mapped array holding the position and orientation of each sample.}
\noteroutine{IRC\_SATCO( IDC, NVAL, SAMPLE, DETIND, PSI, THETA, SOLONG, UTCS,
                        STATUS )}
   {Returns the satellite coordinates, solar longitude and UTCS values at a 
    set of samples.}
\noteroutine{IRC\_SIMUL( IDC, SIMUL, STATUS )}
   {See if the CRDD file contains simultaneous detector samples or not.}
\noteroutine{IRC\_SKCO1( IDC, SAMPLE, DETIND, NVAL, ZFP, YFP, RA, DEC, STATUS )}
   {Find sky coordinates for a set of focal plane positions at a time specified 
    by the given sample.}
\noteroutine{IRC\_SKCO2( IDC, NVAL, SAMPLE, DETIND, ZFP, YFP, RA, DEC, STATUS )}
   {Find sky coordinates for a single focal plane position at a set of times
    specified by a given set of samples.}
\noteroutine{IRC\_SUPP( IDC, NAME, THERE, LOC, STATUS )}
   {See if an item of support information exists, and if so, return an HDS 
    locator to it.}
\noteroutine{IRC\_TRACE( IDC, ROUTNE, STATUS )}
   {Display the contents of the CRDD\_INFO structure.}
\noteroutine{IRC\_TRUNC( IDC, NDETS, DETIND, SAMPLO, SAMPHI, STATUS )}
   {Find section of scan which is covered by data from all detectors.}

\subsection {Full Routine Specifications}
\newlength{\sstbannerlength}
\newlength{\sstcaptionlength}

\font\ssttt=cmtt10 scaled 1095

\newcommand{\sstroutine}[3]{
   \goodbreak
   \rule{\textwidth}{0.5mm}
   \vspace{-7ex}
   \newline
   \settowidth{\sstbannerlength}{{\Large {\bf #1}}}
   \setlength{\sstcaptionlength}{\textwidth}
   \addtolength{\sstbannerlength}{0.5em} 
   \addtolength{\sstcaptionlength}{-2.0\sstbannerlength}
   \addtolength{\sstcaptionlength}{-4.45pt}
   \parbox[t]{\sstbannerlength}{\flushleft{\Large {\bf #1}}}
   \parbox[t]{\sstcaptionlength}{\center{\Large #2}}
   \parbox[t]{\sstbannerlength}{\flushright{\Large {\bf #1}}}
   \begin{description}
      #3
   \end{description}
}

\newcommand{\sstdescription}[1]{\item[Description:] #1}

\newcommand{\sstusage}[1]{\item[Usage:] \mbox{} \\[1.3ex] {\ssttt #1}}

\newcommand{\sstinvocation}[1]{\item[Invocation:]\hspace{0.4em}{\tt #1}}

\newcommand{\sstarguments}[1]{
   \item[Arguments:] \mbox{} \\
   \vspace{-3.5ex}
   \begin{description}
      #1
   \end{description}
}

\newcommand{\sstreturnedvalue}[1]{
   \item[Returned Value:] \mbox{} \\
   \vspace{-3.5ex}
   \begin{description}
      #1
   \end{description}
}

\newcommand{\sstparameters}[1]{
   \item[Parameters:] \mbox{} \\
   \vspace{-3.5ex}
   \begin{description}
      #1
   \end{description}
}

\newcommand{\sstexamples}[1]{
   \item[Examples:] \mbox{} \\
   \vspace{-3.5ex}
   \begin{description}
      #1
   \end{description}
}

\newcommand{\sstsubsection}[1]{\item[{#1}] \mbox{} \\}

\newcommand{\sstexamplesubsection}[1]{\item[{\ssttt #1}] \mbox{} \\}

\newcommand{\sstnotes}[1]{\item[Notes:] \mbox{} \\[1.3ex] #1}

\newcommand{\sstdiytopic}[2]{\item[{\hspace{-0.35em}#1\hspace{-0.35em}:}] \mbox{} \\[1.3ex] #2}

\newcommand{\sstimplementationstatus}[1]{
   \item[{Implementation Status:}] \mbox{} \\[1.3ex] #1}

\newcommand{\sstbugs}[1]{\item[Bugs:] #1}

\newcommand{\sstitemlist}[1]{
  \mbox{} \\
  \vspace{-3.5ex}
  \begin{itemize}
     #1
  \end{itemize}
}

\newcommand{\sstitem}{\item}

\setlength{\textwidth}{160mm}
\setlength{\textheight}{240mm}
\setlength{\topmargin}{-5mm}
\setlength{\oddsidemargin}{0mm}
\setlength{\evensidemargin}{0mm}
\setlength{\parindent}{0mm}
\setlength{\parskip}{\medskipamount}
\setlength{\unitlength}{1mm}

\renewcommand{\_}{{\tt\char'137}}


\sstroutine{
   IRC\_ANNUL
}{
   Release a CRDD file from the IRC\_ system
}{
   \sstdescription{
      The IRC identifier is annuled and all resources reserved by the
      IRC\_ system for this CRDD file are released. NB, this routine does
      not effect the associated NDF identifier.
   }
   \sstinvocation{
      CALL IRC\_ANNUL( IDC, STATUS )
   }
   \sstarguments{
      \sstsubsection{
         IDC = INTEGER (Given)
      }{
         The IRC identifier of the CRDD file.
      }
      \sstsubsection{
         STATUS = INTEGER (Given and Returned)
      }{
         The global status.
      }
   }
   \sstnotes{
      \sstitemlist{

         \sstitem
         This routine attempts to execute even if STATUS is set on
         entry, although no further error report will be made if it
         subsequently fails under these circumstances. In particular, it
         will fail if the identifier supplied is not initially valid, but
         this will only be reported if STATUS is set to SAI\_\_OK on entry.
      }
   }
}
\sstroutine{
   IRC\_BCLAP
}{
   Find the sample number at the closest approach of the boresight
   to a given sky position
}{
   \sstdescription{
      The path of a detector centre as it passes over the sky is called
      the {\tt "}detector track{\tt "}. The corresponding boresight positions form
      the {\tt "}boresight track{\tt "}. The boresight track is the same for all
      detectors, EXCEPT for a possible shift in sample number. Such a
      shift will exist if the detector samples in a given column of the
      NDF DATA array were not obtained simultaneously. This will depend
      on what type of CRDD file is being processed. CRDD files of type
      SURVEY\_BSIGHT (see ID1 Appendix F) do in fact have simultaneous
      samples, but other types may not (See routine IRC\_SIMUL). For
      this reason a detector index is needed to completely specify the
      boresight track.

      This routine finds the fractional sample number from a specified
      detector which corresponds to the position of closest approach of
      the boresight track to a given sky position. The focal plane z
      coordinate (see ID1 Appendix E) of the given position, at the
      moment at which the closest approach is reached, is also returned.
      If the position of closest approach lies outside the bounds of the
      first dimension of the NDF, then the sample number returned
      represents an extrapolated position. If it is not possible to
      produce a reliable extrapolated sample number, then STATUS is
      returned with the value IRC\_\_BADEX, and an error report is made.
      It is then the responsibility of the calling routine to handle
      the situation.
   }
   \sstinvocation{
      CALL IRC\_BCLAP( IDC, DETIND, RA, DEC, CLSAMP, CLZFP, STATUS )
   }
   \sstarguments{
      \sstsubsection{
         IDC = INTEGER (Given)
      }{
         An IRC identifier for the CRDD file.
      }
      \sstsubsection{
         DETIND = INTEGER (Given)
      }{
         The detector index to which the returned sample number (CLSAMP)
         should refer.
      }
      \sstsubsection{
         RA = DOUBLE PRECISION (Given)
      }{
         The Right Ascension (B1950, FK4) of the given sky position.
         If the Starlink {\tt "}BAD{\tt "} value (VAL\_\_BADD) is given then CLSAMP
         and CLZFP are also returned with the bad value.
      }
      \sstsubsection{
         DEC = DOUBLE PRECISION (Given)
      }{
         The Declination (B1950, FK4) of the given sky position.
         If the Starlink {\tt "}BAD{\tt "} value (VAL\_\_BADD) is given then CLSAMP
         and CLZFP are also returned with the bad value.
      }
      \sstsubsection{
         CLSAMP = REAL (Returned)
      }{
         The fractional sample number at which the point of closest
         approach of the boresight track to the given sky position is
         reached. Accurate to about a tenth of a sample.
      }
      \sstsubsection{
         CLZFP = REAL (Returned)
      }{
         The focal plane Z coordinate of the given position at the point
         of closest approach. In radians.
      }
      \sstsubsection{
         STATUS = INTEGER (Given and Returned)
      }{
         The global status.
      }
   }
}
\sstroutine{
   IRC\_BPOS
}{
   Returns boresight positions at a set of samples
}{
   \sstdescription{
      The calling routine specifies a list of samples by giving
      the sample number and detector index of each sample. For
      each such sample, various items of information about the boresight
      position are returned, as listed in the argument list below. If a
      sample number lies outside the bounds of the first dimension of
      the NDF, then an extrapolated position is returned if possible.
      If this is not possible, the STATUS value is set to IRC\_\_BADEX
      and an error report is generated.
   }
   \sstinvocation{
      CALL IRC\_BPOS( IDC, NVAL, SAMPLE, DETIND, RA, DEC, ANGLE,
                     SPEED, STATUS )
   }
   \sstarguments{
      \sstsubsection{
         IDC = INTEGER (Given)
      }{
         The IRC identifier for the CRDD file.
      }
      \sstsubsection{
         NVAL = INTEGER (Given)
      }{
         The number of samples in the input and output lists.
      }
      \sstsubsection{
         SAMPLE( NVAL ) = REAL (Given)
      }{
         A list of fractional sample numbers. If any sample number is
         equal to the Starlink {\tt "}BAD{\tt "} value (VAL\_\_BADR) then the
         corresponding elements of the returned arrays are set to the
         bad value.
      }
      \sstsubsection{
         DETIND( NVAL ) = INTEGER (Given)
      }{
         A list of detector indices.
      }
      \sstsubsection{
         RA( NVAL ) = DOUBLE PRECISION (Returned)
      }{
         An array holding the Right Ascension (B1950 FK4) of the
         boresight at the moment each sample specified in the input
         lists was taken (radians).
      }
      \sstsubsection{
         DEC( NVAL ) = DOUBLE PRECISION (Returned)
      }{
         An array holding the Declination (B1950 FK4) of the
         boresight at the moment each sample specified in the input
         lists was taken (radians).
      }
      \sstsubsection{
         ANGLE( NVAL ) = DOUBLE PRECISION (Returned)
      }{
         The scan angle (see ID1 section 3) at the boresight. This is
         measured from equatorial north to the positive focal plane Y
         axis (see ID1 Appendix E).  The angle is in radians and is
         measured positive in the same sense as rotation from north to
         east.
      }
      \sstsubsection{
         SPEED( NVAL ) = REAL (Returned)
      }{
         The scan speed in radians per second. Positive values imply
         that sources move across the focal plane in the positive Y
         direction (i.e in the {\tt "}with-survey{\tt "} direction). Negative
         values imply that sources move in the negative Y direction
         (i.e. {\tt "}anti-survey{\tt "}).
      }
      \sstsubsection{
         STATUS = INTEGER (Given and Returned)
      }{
         The global status.
      }
   }
}
\sstroutine{
   IRC\_CLOSE
}{
   Close down the IRC CRDD handling package
}{
   \sstdescription{
      This routine should be called when all other IRC routines have
      been finished with. It annuls any remaining valid IRC identifiers.
   }
   \sstinvocation{
      CALL IRC\_CLOSE( STATUS )
   }
   \sstarguments{
      \sstsubsection{
         STATUS = INTEGER (Given and Returned)
      }{
         The global status.
      }
   }
   \sstnotes{
      \sstitemlist{

         \sstitem
         This routine attempts to execute even if STATUS is set on
         entry, although no further error report will be made if it
      }
   }
}
\sstroutine{
   IRC\_DCLAP
}{
   Find the position of closest approach of a detector track to a
   given sky position
}{
   \sstdescription{
      The path of a detector centre as it passes over the sky is called
      the {\tt "}detector track{\tt "}. This routine finds the fractional sample
      number (from the row specified by the given detector index) at
      which the closest approach of the detector to a given sky
      position is reached.  The focal plane z coordinate of the given
      position, at the moment corresponding to the closest approach is
      also returned.  If the position of closest approach lies outside
      the bounds of the first dimension of the NDF, then the sample
      number returned represents an extrapolated position. If it is not
      possible to produce a reliable extrapolated sample number, then
      STATUS is returned with the value IRC\_\_BADEX, and an error report
      is made.  It is then the responsibility of the calling routine to
      handle the situation.
   }
   \sstinvocation{
      CALL IRC\_DCLAP( IDC, DETIND, RA, DEC, CLSAMP, CLZFP, STATUS )
   }
   \sstarguments{
      \sstsubsection{
         IDC = INTEGER (Given)
      }{
         An IRC identifier for the CRDD file.
      }
      \sstsubsection{
         DETIND = INTEGER (Given)
      }{
         The detector index. The returned sample number (CLSAMP) refers
         to this detector.
      }
      \sstsubsection{
         RA = DOUBLE PRECISION (Given)
      }{
         The Right Ascension (B1950, FK4) of the given sky position.
         If this has the Starlink {\tt "}BAD{\tt "} value (VAL\_\_BADD) then CLSAMP
         and CLZFP are returned with the bad value.
      }
      \sstsubsection{
         DEC = DOUBLE PRECISION (Given)
      }{
         The Declination (B1950, FK4) of the given sky position.
         If this has the Starlink {\tt "}BAD{\tt "} value (VAL\_\_BADD) then CLSAMP
         and CLZFP are returned with the bad value.
      }
      \sstsubsection{
         CLSAMP = REAL (Returned)
      }{
         The fractional sample number at which the point of closest
         approach of the detector track to the given sky position is
         reached. Accurate to about a tenth of a sample.
      }
      \sstsubsection{
         CLZFP = REAL (Returned)
      }{
         The focal plane Z coordinate of the given position at the point
         of closest approach. In radians.
      }
      \sstsubsection{
         STATUS = INTEGER (Given and Returned)
      }{
         The global status.
      }
   }
}
\sstroutine{
   IRC\_DETIN
}{
   Get the detector index for a given detector number
}{
   \sstdescription{
      The detector index (if any) which corresponds to the given
      detector number (in the range 1 to 62) is returned.  The detector
      index is just the row number within the NDF DATA array, and
      varies between the bounds of the second dimension of the NDF. If
      the requested detector number is not included in the CRDD file,
      the Starlink {\tt "}BAD{\tt "} value (VAL\_\_BADI) is returned, but no error is
      reported. If an error state exists on entry, or if any error
      occurs within the function, VAL\_\_BADI is returned.
   }
   \sstinvocation{
      RESULT = IRC\_DETIN( IDC, DETNO, STATUS )
   }
   \sstarguments{
      \sstsubsection{
         IDC = INTEGER (Given)
      }{
         The IRC identifier for the CRDD file.
      }
      \sstsubsection{
         DETNO = INTEGER (Given)
      }{
         The detector number for which the corresponding detector index
         is required.
      }
      \sstsubsection{
         STATUS = INTEGER (Given and Returned)
      }{
         The global status.
      }
   }
   \sstreturnedvalue{
      \sstsubsection{
         IRC\_DETIN = INTEGER
      }{
         The detector index corresponding to the given detector number.
         This is set to VAL\_\_BADI if STATUS indicates an error on entry,
         or if an error is generated within IRC\_DETIN.
      }
   }
}
\sstroutine{
   IRC\_DETNO
}{
   Get the detector number for a given detector index
}{
   \sstdescription{
      The detector number (in the range 1 to 62) of the detector which
      generated the data held at the detector index given by argument
      DETIND is found. The detector index is just the row number within
      the NDF DATA array, and varies between the bounds of the second
      dimension of the NDF. An error is reported if a detector index is
      given outside this range. If any error occurs, the function
      returns the Starlink {\tt "}BAD{\tt "} values (VAL\_\_BADI).
   }
   \sstinvocation{
      RESULT = IRC\_DETNO( IDC, DETIND, STATUS )
   }
   \sstarguments{
      \sstsubsection{
         IDC = INTEGER (Given)
      }{
         The IRC identifier for the CRDD file.
      }
      \sstsubsection{
         DETIND = INTEGER (Given)
      }{
         The detector index for which the corresponding detector number
         is required.
      }
      \sstsubsection{
         STATUS = INTEGER (Given and Returned)
      }{
         The global status.
      }
   }
   \sstreturnedvalue{
      \sstsubsection{
         IRC\_DETNO = INTEGER
      }{
         The detector number corresponding to the given detector index.
         This is set to VAL\_\_BADI if STATUS indicates an error on entry,
         or if an error is generated within IRC\_DETNO.
      }
   }
}
\sstroutine{
   IRC\_DINDS
}{
   Convert a list of detector numbers to detector indices
}{
   \sstdescription{
      The detector index corresponding to each supplied detector number
      is found and returned in DETIN. If the CRDD file contains no data
      for any of the supplied detector numbers, or if any of the
      detector number are equal to the Starlink {\tt "}BAD{\tt "} value
      (VAL\_\_BADI), then no corresponding value is stored in DETIN. The
      number of values actually stored in DETIN is returned in NDETIN.
      These values occupy elements 1 to NDETIN of the DETIN array.
   }
   \sstinvocation{
      CALL IRC\_DINDS( IDC, NDETNO, DETNO, DETIN, NDETIN, STATUS )
   }
   \sstarguments{
      \sstsubsection{
         IDC = INTEGER  (Given)
      }{
         An IRC identifier for the CRDD file.
      }
      \sstsubsection{
         NDETNO = INTEGER (Given)
      }{
         The size of the DETNO and DETIN arrays.
      }
      \sstsubsection{
         DETNO( NDETNO ) = INTEGER (Given)
      }{
         The array of detector numbers.
      }
      \sstsubsection{
         DETIN( NDETNO ) = INTEGER (Returned)
      }{
         The list of detector indices corresponding to the detector
         numbers in DETNO.
      }
      \sstsubsection{
         NDETIN = INTEGER (Returned)
      }{
         The number of values returned in DETIN. This will be less than
         DETNO if any {\tt "}BAD{\tt "} detector numbers are supplied, or if the
         CRDD file contains no data for any of the supplied detectors.
      }
      \sstsubsection{
         STATUS = INTEGER (Given and Returned)
      }{
         The global status.
      }
   }
}
\sstroutine{
   IRC\_DIST
}{
   Find the in-scan distance between two samples
}{
   \sstdescription{
      The in-scan distance from the sample specified by SAMP1 and DETIN1
      to the sample specified by SAMP2 and DETIN2 is found. Positive
      values imply that the displacement from the first sample to the
      second sample is in the direction of the positive focal plane
      Y axis (see ID1 Appendix E).
   }
   \sstinvocation{
      CALL IRC\_DIST( IDC, SAMP1, DETIN1, SAMP2, DETIN2, DIST, STATUS )
   }
   \sstarguments{
      \sstsubsection{
         IDC = INTEGER (Given)
      }{
         An IRC identifier for the CRDD file.
      }
      \sstsubsection{
         SAMP1 = REAL (Given)
      }{
         The first fractional sample number. If this has the Starlink
         {\tt "}BAD{\tt "} value (VAL\_\_BADR) then DIST is returned set to the bad
         value.
      }
      \sstsubsection{
         DETIN1 = INTEGER (Given)
      }{
         The detector index to which SAMP1 refers.
      }
      \sstsubsection{
         SAMP2 = REAL (Given)
      }{
         The second fractional sample number. If this has the Starlink
         {\tt "}BAD{\tt "} value (VAL\_\_BADR) then DIST is returned set to the bad
         value.
      }
      \sstsubsection{
         DETIN2 = INTEGER (Given)
      }{
         The detector index to which SAMP2 refers.
      }
      \sstsubsection{
         DIST = REAL (Returned)
      }{
         The in-scan distance between the two samples, in radians.
         Positive if the displacement from SAMP1 to SAMP2 is in the
         same direction as the positive focal plane Y axis.
      }
      \sstsubsection{
         STATUS = INTEGER (Given and Returned)
      }{
         The global status.
      }
   }
}
\sstroutine{
   IRC\_DPOS
}{
   Returns positional information about a set of samples
}{
   \sstdescription{
      The calling routine specifies a list of samples by giving
      the sample number and detector index of each sample. For
      each such sample, various items of information are returned, as
      listed in the argument list below. Extrapolated values are
      returned for samples which lie outside the bounds of the first
      dimension of the NDF.
   }
   \sstinvocation{
      CALL IRC\_DPOS( IDC, NVAL, SAMPLE, DETIND, RA, DEC, ANGLE,
                     SPEED, STATUS )
   }
   \sstarguments{
      \sstsubsection{
         IDC = INTEGER (Given)
      }{
         The IRC identifier for the CRDD file.
      }
      \sstsubsection{
         NVAL = INTEGER (Given)
      }{
         The number of samples in the input and output lists.
      }
      \sstsubsection{
         SAMPLE( NVAL ) = REAL (Given)
      }{
         A list of fractional sample numbers. If any sample number has
         the Starlink {\tt "}BAD{\tt "} value (VAL\_\_BADR) then the corresponding
         elements of the returned arrays are set to the bad value.
      }
      \sstsubsection{
         DETIND( NVAL ) = INTEGER (Given)
      }{
         A list of detector indices.
      }
      \sstsubsection{
         RA( NVAL ) = DOUBLE PRECISION (Returned)
      }{
         An array holding the Right Ascension (B1950 FK4) of each
         detector centre specified by the input lists (radians).
      }
      \sstsubsection{
         DEC( NVAL ) = DOUBLE PRECISION (Returned)
      }{
         An array holding the Declination (B1950 FK4) of each
         detector centre specified by the input lists (radians).
      }
      \sstsubsection{
         ANGLE( NVAL ) = DOUBLE PRECISION (Returned)
      }{
         The scan angle at each detector centre. This is measured from
         equatorial north to a line parallel to the focal plane Y axis.
         The angle is in radians and is measured positive in the same
         sense as rotation from north to east.
      }
      \sstsubsection{
         SPEED( NVAL ) = REAL (Returned)
      }{
         The scan speed in radians per second. Positive values imply
         that sources move in the positive focal plane Y direction (i.e
         in the {\tt "}with-survey{\tt "} direction). Negative values imply that
         sources move in the negative Y direction (i.e. {\tt "}anti-survey{\tt "}).
      }
      \sstsubsection{
         STATUS = INTEGER (Given and Returned)
      }{
         The global status.
      }
   }
}
\sstroutine{
   IRC\_FPCO1
}{
   Find focal plane coordinates of many sky positions at a time
   determined by a given sample
}{
   \sstdescription{
      This routine returns the focal plane Z and Y coordinate
      values (see ID1 Appendix E) of a list of sky positions, at the
      moment the specified sample was taken.
   }
   \sstinvocation{
      CALL IRC\_FPCO1( IDC, SAMPLE, DETIND, NVAL, RA, DEC, ZFP, YFP,
                      STATUS )
   }
   \sstarguments{
      \sstsubsection{
         IDC = INTEGER (Given)
      }{
         An IRC identifier for the CRDD file.
      }
      \sstsubsection{
         SAMPLE = REAL (Given)
      }{
         The fractional sample number. If this has the Starlink {\tt "}BAD{\tt "}
         value, then all returned focal plane coordinates are set bad.
      }
      \sstsubsection{
         DETIND = INTEGER (Given)
      }{
         The detector index to which the sample number refers.
      }
      \sstsubsection{
         NVAL = INTEGER (Given)
      }{
         The number of sky positions to be converted.
      }
      \sstsubsection{
         RA( NVAL ) = DOUBLE PRECISION (Given)
      }{
         The Right Ascension values (B1950 FK4) of the sky positions,
         in radians. If a Starlink {\tt "}BAD{\tt "} value (VAL\_\_BADD) is found,
         the corresponding element of array FPCO is set to the bad
         value.
      }
      \sstsubsection{
         DEC( NVAL ) = DOUBLE PRECISION (Given)
      }{
         The Declination values (B1950 FK4) of the sky positions, in
         radians. If a Starlink {\tt "}BAD{\tt "} value (VAL\_\_BADD) is found, the
         corresponding element of array FPCO is set to the bad value.
      }
      \sstsubsection{
         ZFP( NVAL ) = REAL (Returned)
      }{
         The focal plane Z coordinate values, in radians.
      }
      \sstsubsection{
         YFP( NVAL ) = REAL (Returned)
      }{
         The focal plane Y coordinate values, in radians.
      }
      \sstsubsection{
         STATUS = INTEGER (Given and Returned)
      }{
         The global status.
      }
   }
}
\sstroutine{
   IRC\_FPCO2
}{
   Find focal plane coordinates of a single sky position at a set of
   different times
}{
   \sstdescription{
      This routine returns the focal plane Z and Y coordinate
      values of a given sky position, at a set of different times,
      specified by a set of sample numbers and detector indices.
   }
   \sstinvocation{
      CALL IRC\_FPCO2( IDC, NVAL, SAMPLE, DETIND, RA, DEC, DPW1,
                      DPW2, DPW3, RW4, ZFP, YFP, STATUS )
   }
   \sstarguments{
      \sstsubsection{
         IDC = INTEGER (Given)
      }{
         An IRC identifier for the CRDD file.
      }
      \sstsubsection{
         NVAL = INTEGER (Given)
      }{
         The number of samples at which to convert the given sky
         position.
      }
      \sstsubsection{
         SAMPLE( NVAL ) = REAL (Given)
      }{
         The fractional sample numbers at which to do the conversion.
         If any sample number has the Starlink {\tt "}BAD{\tt "} value (VAL\_\_BADR),
         then the corresponding element of the array FPCO is returned
         bad.
      }
      \sstsubsection{
         DETIND( NVAL ) = INTEGER (Given)
      }{
         The detector indices to which the sample numbers refer. These
         must all be within the bounds of the second dimension of the
         NDF.
      }
      \sstsubsection{
         RA = DOUBLE PRECISION (Given)
      }{
         The Right Ascension (B1950 FK4) of the sky position, in
         radians. If this is equal to the Starlink {\tt "}BAD{\tt "} value
         (VAL\_\_BADD), then the FPCO array is returned full of bad
         values.
      }
      \sstsubsection{
         DEC = DOUBLE PRECISION (Given)
      }{
         The Declination (B1950 FK4) of the sky position. If this is
         equal to the Starlink {\tt "}BAD{\tt "} value (VAL\_\_BADD), then the FPCO
         array is returned full of bad values.
      }
      \sstsubsection{
         DPW1( NVAL ) = DOUBLE PRECISION (Given and Returned)
      }{
         Double precision workspace.
      }
      \sstsubsection{
         DPW2( NVAL ) = DOUBLE PRECISION (Given and Returned)
      }{
         Double precision workspace.
      }
      \sstsubsection{
         DPW3( NVAL ) = DOUBLE PRECISION (Given and Returned)
      }{
         Double precision workspace.
      }
      \sstsubsection{
         RW4( NVAL ) = REAL (Given and Returned)
      }{
         Real workspace.
      }
      \sstsubsection{
         ZFP( NVAL ) = REAL (Returned)
      }{
         The focal plane Z coordinate values, in radians.
      }
      \sstsubsection{
         YFP( NVAL ) = REAL (Returned)
      }{
         The focal plane Y coordinate values, in radians.
      }
      \sstsubsection{
         STATUS = INTEGER (Given and Returned)
      }{
         The global status.
      }
   }
}
\sstroutine{
   IRC\_ILAB
}{
   Return a list of legal values for NDF component LABEL
}{
   \sstdescription{
      A string is returned containing a list of names identifying the
      legal values of the NDF component LABEL. The values are separated
      by commas. The currently recognised values are:

      1) {\tt "}Survey CRDD{\tt "}
   }
   \sstinvocation{
      CALL IRC\_ILAB( LIST, STATUS )
   }
   \sstarguments{
      \sstsubsection{
         LIST = CHARACTER $*$ ( $*$ ) (Returned)
      }{
         The list of recognized values for NDF component LABEL. The
         character variable supplied for this argument should have a
         declared size equal to the value of parameter IRC\_\_SZLLS. If
         the supplied string is not long enough to hold all the names, a
         warning message is given, but no error status is returned.
      }
      \sstsubsection{
         STATUS = INTEGER (Given and Returned)
      }{
         The global status.
      }
   }
}
\sstroutine{
   IRC\_IMCO
}{
   Locate a given sample within an image frame
}{
   \sstdescription{
      This routine finds the position and orientation of a given sample
      within an image frame defined by an IRA identifier (see ID2). The
      information is returned in the form of six values giving the
      coefficients of a linear transformation which maps an offset from
      the sample centre given in focal plane coordinates, to the
      corresponding pixel coordinates within the image frame defined by
      the supplied IRA identifier. Specifically, the returned vector C
      holds values C(1) to C(6) where:

         X = C(1) $+$ C(2)$*$DZfp $+$ C(3)$*$DYfp

         Y = C(4) $+$ C(5)$*$DZfp $+$ C(6)$*$Dyfp

      In these expressions, DZfp and DYfp are displacements (in
      radians, NOT arc-minutes!) parallel to the focal plane Z and Y
      axes, such that (DZfp,DYfp)=(0,0) corresponds to the detector
      centre. (X,Y) are the image coordinates which correspond to the
      position with focal plane offset (DZfp,DYfp). Thus C(1) and C(4)
      are the image coordinates of the detector centre, and the
      anticlockwise angle from the image Y axis to the focal plane Y
      axis is ATAN2( C(5), C(6) ), or equivalently ATAN2( -C(3),C(2) ).
      The tendency for pixel size to vary across a large image is taken
      into account in the returned coefficients.

      The use of a linear transformation is only valid for small
      displacements away from the detector centre. This routine is
      intended for situations in which the displacement is less than
      an arc-degree.

      Various forms of initialisation need to be performed each time a
      new combination of CRDD file (as specified by IDC) and astrometry
      information (as specified by IDA) is used. Setting the argument
      INIT to .TRUE. causes this initialisation to be done, before
      going on to calculate the returned information. On subsequent
      calls to this routine INIT should set to .FALSE., until either a
      new CRDD file or new astrometry information is used. Note, the
      initialisation procedure is also performed (even if INIT is
      false) if the supplied detector index is different to the value
      supplied for the previous call to this routine. For this reason,
      applications should use this routine to process all samples from
      one detector before moving on to do another detector. The
      alternative approach (processing the same sample from all
      detectors before moving on to do another sample) would cause much
      time to be wasted doing unnecessary initialisations.

      ID2 should be consulted for information about setting up
      astrometry information.
   }
   \sstinvocation{
      CALL IRC\_IMCO( IDC, SAMP, DETIN, IDA, INIT, C, STATUS )
   }
   \sstarguments{
      \sstsubsection{
         IDC = INTEGER (Given)
      }{
         An IRC identifier for the CRDD file.
      }
      \sstsubsection{
         SAMP = REAL (Given)
      }{
         The fractional sample number. If this has the Starlink {\tt "}BAD{\tt "}
         value (VAL\_\_BADR) then the returned coefficients are all set to
         the bad value.
      }
      \sstsubsection{
         DETIN = INTEGER (Given)
      }{
         The detector index to which SAMP refers.
      }
      \sstsubsection{
         IDA = INTEGER (Given)
      }{
         An IRA identifier for the astrometry information which defines
         the mapping from sky coordinates to image coordinates (see
         ID2).
      }
      \sstsubsection{
         INIT = LOGICAL (Given)
      }{
         INIT should be set true on the first call to this routine, and
         false on subsequent calls. If either a new CRDD file, or new
         astrometry information is supplied, then INIT should be set
         true again for one call to this routine.
      }
      \sstsubsection{
         C( 6 ) = REAL (Returned)
      }{
         The six coefficients defining the linear transformation from
         focal plane offsets (in radians) to image coordinates (in
         pixels).
      }
      \sstsubsection{
         STATUS = INTEGER (Given and Returned)
      }{
         The global status.
      }
   }
}
\sstroutine{
   IRC\_IMPRT
}{
   Import an existing NDF containing CRDD into the IRC\_ system
}{
   \sstdescription{
      This routine checks that the given NDF is two dimensional and
      contains an IRAS extension, which in turn contains a CRDD\_INFO
      component. If it doesn{\tt '}t, an error is reported. If it does, a
      check is made that the NDF contains CRDD of a recognised type.  A
      record is then made of the CRDD file in the IRC\_ common blocks
      and an IRC identifier returned for this record. This identifier
      can be passed to other IRC routines to access the CRDD
      descriptive information. The bulk data values (contained in the
      NDF DATA component) should be accessed using the normal NDF\_
      routines (see SUN/33).
   }
   \sstinvocation{
      CALL IRC\_IMPRT( INDF, IDC, STATUS )
   }
   \sstarguments{
      \sstsubsection{
         INDF = INTEGER (Given)
      }{
         The NDF identifier for the CRDD file. This can be an NDF
         section so long as the bounds of the second dimension do not
         extend beyond the bounds of the second dimension of the base
         NDF.
      }
      \sstsubsection{
         IDC  = INTEGER (Returned)
      }{
         The IRC identifier for the CRDD information.
      }
      \sstsubsection{
         STATUS = INTEGER (Given and Returned)
      }{
         The global status.
      }
   }
}
\sstroutine{
   IRC\_INFO
}{
   Obtain values for the global properties of a CRDD file
}{
   \sstdescription{
      The values of various global parameters are returned for the CRDD
      file identified by the given IRC identifier. Other global
      properties (such as the size and shape of the data array) can be
      obtained using the NDF\_ routines (see SUN/33).
   }
   \sstinvocation{
      CALL IRC\_INFO( IDC, BAND, REFRA, REFDEC, NOMSPD, SOP, OBS,
                     STATUS )
   }
   \sstarguments{
      \sstsubsection{
         IRC = INTEGER (Given)
      }{
         The IRC identifier for the CRDD file.
      }
      \sstsubsection{
         BAND = INTEGER (Returned)
      }{
         The IRAS band number of the data, in the range 1 to 4.
      }
      \sstsubsection{
         REFRA = DOUBLE PRECISION (Returned)
      }{
         The Right Ascension of the reference position, given in units
         of radians (FK4 B1950.0).
      }
      \sstsubsection{
         REFDEC = DOUBLE PRECISION (Returned)
      }{
         The Declination of the reference position, given in units of
         radians (FK4 B1950.0).
      }
      \sstsubsection{
         NOMSPD = REAL (Returned)
      }{
         The nominal scan speed in arcminutes per second (the exact
         scan speed may vary slightly from sample to sample). Positive
         values imply that sources move in the positive focal plane Y
         direction (i.e in the {\tt "}with-survey{\tt "} direction). Negative values
         imply that sources move in the negative Y direction (i.e.
         {\tt "}anti-survey{\tt "}). See ID1 Appendix E for a description of the
         focal plane coordinate system).
      }
      \sstsubsection{
         SOP = INTEGER (Returned)
      }{
         The SOP number from which the crdd was derived.
      }
      \sstsubsection{
         OBS  = INTEGER (Returned)
      }{
         The Observation number within the SOP from which the data was
         derived.
      }
      \sstsubsection{
         STATUS = INTEGER (Given and Returned)
      }{
         The global status.
      }
   }
}
\sstroutine{
   IRC\_INIT
}{
   Initialise the IRC system
}{
   \sstdescription{
      This routine should be called before calling any other IRC
      routines.  If IRC has not previously been initialised, or has
      been closed down (by calling IRC\_CLOSE), then all IRC identifiers
      are released but no check is made to see if they are valid (since
      such checks could give spurious results).  If IRC has previously
      been initialised but has not yet been closed down then all data
      structures associated with currently valid IRC identifiers are
      annuled and the identifiers are released.
   }
   \sstinvocation{
      CALL IRC\_INIT( STATUS )
   }
   \sstarguments{
      \sstsubsection{
         STATUS = INTEGER (Given and Returned)
      }{
         The global status.
      }
   }
}
\sstroutine{
   IRC\_IUNIT
}{
   Return a list of legal values for NDF component UNITS
}{
   \sstdescription{
      A string is returned containing a list of names identifying the
      legal values of the NDF component UNITS. The values are separated
      by commas. The currently recognised values are given by the
      following symbolic constants:

      IRC\_\_F      - Flux values in units of Pico-Watts (i.e. 1.0E-12 of
                      a Watt) per square metre.
 
      IRC\_\_J      - Flux density values in Janskys.

      IRC\_\_JPS    - Surface brightness in Janskys per steradian.

      IRC\_\_MJPS   - Surface brightness in Mega-Janskys per steradian.

      IRC\_\_FPS    - Surface brightness in Pico-Watts per square metre,
                      per steradian.
   }
   \sstinvocation{
      CALL IRC\_IUNIT( LIST, STATUS )
   }
   \sstarguments{
      \sstsubsection{
         LIST = CHARACTER $*$ ( $*$ ) (Returned)
      }{
         The list of recognized values for NDF component UNITS. The
         character variable supplied for this argument should have a
         declared size equal to the value of parameter IRC\_\_SZULS. If
         the supplied string is not long enough to hold all the names, a
         warning message is given, but no error status is returned. Each
         individual value within the string has a maximum length given
         by the symbolic constant IRC\_\_SZUNI.
      }
      \sstsubsection{
         STATUS = INTEGER (Given and Returned)
      }{
         The global status.
      }
   }
}
\sstroutine{
   IRC\_LIMIT
}{
   Determine the image area covered by a CRDD file
}{
   \sstdescription{
      The positions of selected boundary samples within the supplied
      CRDD file are converted to image coordinates using the supplied
      IRA identifier (see ID2). If any of the samples fall outside the
      image area defined by the supplied bounds, then the bounds are
      updated so that the data will just fit in. The calling routine
      supplies a list of the detectors which are to be included in the
      image area.  The detector data streams in a CRDD file usually
      start and finish at different in-scan positions, giving a
      `ragged{\tt '} end to a scan.  If the argument TRUNC is supplied true,
      then a short section from each end of each data stream is
      excluded from the process, so that all detector data streams
      start and finish at the same in-scan position, resulting in a
      `straight{\tt '} end to the scan.
   }
   \sstinvocation{
      CALL IRC\_LIMIT( IDC, NDETS, DETIND, TRUNC, IDA, LBND, UBND,
                      STATUS )
   }
   \sstarguments{
      \sstsubsection{
         IDC = INTEGER (Given)
      }{
         The IRC identifier for the CRDD file.
      }
      \sstsubsection{
         NDETS = INTEGER (Given)
      }{
         The number of detector data streams to be included in the image
         area.
      }
      \sstsubsection{
         DETIND( NDETS ) = INTEGER (Given)
      }{
         A list of detector indices defining the detector data streams
         to be included in the image area.
      }
      \sstsubsection{
         TRUNC = LOGICAL (Given)
      }{
         True if the detector data streams are to be truncated so that
         they all start and finish at the same in-scan position.  This
         produces a `straight edge{\tt '} across each end of the scan. If
         TRUNC is false, then all data is included in the image area
         from each detector data stream. This will in general result in
         `ragged edges{\tt '} across each end of the scan.
      }
      \sstsubsection{
         IDA = INTEGER (Given)
      }{
         An IRA identifier for the astrometry information which is to be
         used to define the projection from sky coordinates to image
         coordinates. See ID2.
      }
      \sstsubsection{
         LBND( 2 ) = DOUBLE PRECISION (Given and Returned)
      }{
         On entry, LBND holds the lower bounds on the two image
         axes of the current image area (as pixel coordinates, not
         indices). On exit, these values are updated if the given CRDD
         file could not be fitted into the image area defined by the
         values of LBND on entry.
      }
      \sstsubsection{
         UBND( 2 ) = DOUBLE PRECISION (Given and Returned)
      }{
         On entry, UBND holds the upper bounds on the two image
         axes of the current image area (as pixel coordinates, not
         indices). On exit, these values are updated if the given CRDD
         file could not be fitted into the image area defined by the
         values of UBND on entry.
      }
      \sstsubsection{
         STATUS = INTEGER (Given and Returned)
      }{
         The global status.
      }
   }
}
\sstroutine{
   IRC\_OFFST
}{
   Find a sample which is a given in-scan distance away from a given
   sample
}{
   \sstdescription{
      The calling routine specifies a detector index and sample number
      (DETIN1 and SAMP1) to define a starting point, and a distance
      (DIST) by which to move away from the starting point in the focal
      plane Y direction (see ID1 appendix E). Firstly, the sample
      number (from the same detector) is found which is the given
      distance away from the starting point. If DETIN2 is the same as
      DETIN1, this sample number is returned in SAMP2. If not, the
      sample number from detector specified by DETIN2 is returned at
      which that detector reaches the same in-scan position.
   }
   \sstinvocation{
      CALL IRC\_OFFST( IDC, SAMP1, DETIN1, DETIN2, DIST, SAMP2, STATUS )
   }
   \sstarguments{
      \sstsubsection{
         IDC = INTEGER (Given)
      }{
         An IRC identifier for the CRDD file.
      }
      \sstsubsection{
         SAMP1 = REAL (Given)
      }{
         A fractional sample number which gives the starting point of
         the offset operation. If SAMP1 has the Starlink {\tt "}BAD{\tt "} value
         (VAL\_\_BADR) then SAMP2 is returned bad.
      }
      \sstsubsection{
         DETIN1 = INTEGER (Given)
      }{
         The detector index to which SAMP1 refers.
      }
      \sstsubsection{
         DETIN2 = INTEGER (Given)
      }{
         The detector index to which SAMP2 should refer.
      }
      \sstsubsection{
         DIST = REAL (Given)
      }{
         The arc-distance to offset away from SAMP1, in radians. A
         positive offset moves in the focal plane Y direction. If
         DIST has the Starlink {\tt "}BAD{\tt "} value (VAL\_\_BADR) then SAMP2 is
         returned bad.
      }
      \sstsubsection{
         SAMP2 = REAL (Returned)
      }{
         The fractional sample number from detector DETIN2 which is the
         required in-scan distance away from SAMP1 from detector
         DETIN1.
      }
      \sstsubsection{
         STATUS = INTEGER (Given and Returned)
      }{
         The global status.
      }
   }
}
\sstroutine{
   IRC\_POSMP
}{
   Get arrays containing sky coordinates and scan angles for
   every sample
}{
   \sstdescription{
      This routine produces a temporary double precision NDF in which
      the data array is three dimensional, having three {\tt "}planes{\tt "} each
      matching the 2D data array of the input NDF (a CRDD file). The
      lower plane (index 1) holds the Right Ascension of the centre of
      each detector sample (B1950 FK4), the middle plane (index 2) holds
      the Declination, and the upper plane (index 3) holds the scan
      angle (see ID1 section 3) at the detector centre. This routine
      also maps the data array of the created temporary NDF, and returns
      a pointer to it. The created NDF should be annulled (using
      NDF\_ANNUL) when it is no longer required.

      NOTE, unlike IRC\_BPOS and IRC\_DPOS extrapolated sample positions
      cannot be found using IRC\_POSMP. The returned arrays only contain
      information for samples lying within the bounds of the DATA array.
   }
   \sstinvocation{
      CALL IRC\_POSMP( IDC, NEWNDF, PNTR, STATUS )
   }
   \sstarguments{
      \sstsubsection{
         IDC = INTEGER (Given)
      }{
         The IRC identifier for the CRDD file.
      }
      \sstsubsection{
         NEWNDF = INTEGER (Returned)
      }{
         The NDF identifier for the created temporary NDF.
      }
      \sstsubsection{
         PNTR = INTEGER (Returned)
      }{
         A pointer to the mapped double precision array holding the
         position and orientation of each CRDD sample. The array bounds
         for the first two dimensions are the same as those of the input
         NDF. The third dimension has bounds (1:3). Note, many IRC
         routines allow for {\tt "}extrapolated{\tt "} fractional sample number
         (i.e. sample numbers which are outside the bound of the first
         NDF dimension and which may not be whole numbers). The array
         returned by this routine is ONLY defined for integer sample
         numbers lying within the bounds of the second dimension of the
         NDF.
      }
      \sstsubsection{
         STATUS = INTEGER (Given and returned)
      }{
         The global status.
      }
   }
}
\sstroutine{
   IRC\_SATCO
}{
   Returns satellite coordinates at a set of samples
}{
   \sstdescription{
      The calling routine specifies a list of samples by giving
      the sample number and detector index of each sample. For
      each such sample, various items of information about the boresight
      position are returned, as listed in the argument list below. If a
      sample number lies outside the bounds of the first dimension of
      the NDF, then an extrapolated position is returned if possible.
      If this is not possible, the STATUS value is set to IRC\_\_BADEX
      and an error report is generated.
   }
   \sstinvocation{
      CALL IRC\_SATCO( IDC, NVAL, SAMPLE, DETIND, PSI, THETA, SOLONG,
                      UTCS, STATUS )
   }
   \sstarguments{
      \sstsubsection{
         IDC = INTEGER (Given)
      }{
         The IRC identifier for the CRDD file.
      }
      \sstsubsection{
         NVAL = INTEGER (Given)
      }{
         The number of samples in the input and output lists.
      }
      \sstsubsection{
         SAMPLE( NVAL ) = REAL (Given)
      }{
         A list of fractional sample numbers. If any sample number is
         equal to the Starlink {\tt "}BAD{\tt "} value (VAL\_\_BADR) then the
         corresponding elements of the returned arrays are set to the
         bad value.
      }
      \sstsubsection{
         DETIND( NVAL ) = INTEGER (Given)
      }{
         A list of detector indices.
      }
      \sstsubsection{
         PSI( NVAL ) = REAL (Returned)
      }{
         An array holding the clock angle of the boresight at the
         moment each sample specified in the input lists was taken
         (radians).
      }
      \sstsubsection{
         THETA( NVAL ) = REAL (Returned)
      }{
         An array holding the cone angle of the boresight at the moment
         each sample specified in the input lists was taken (radians).
      }
      \sstsubsection{
         SOLONG( NVAL ) = REAL (Returned)
      }{
         An array holding the solar longitude at the moment each sample
         specified in the input lists was taken (radians). These values
         are refered to the mean equator and equinox of the
         corresponding time returned in UTCS.
      }
      \sstsubsection{
         UTCS( NVAL ) = DOUBLE PRECISION (Returned)
      }{
         An array holding the UTCS at the moment each sample specified
         in the input lists was taken (seconds).
      }
      \sstsubsection{
         STATUS = INTEGER (Given and Returned)
      }{
         The global status.
      }
   }
   \sstnotes{
      \sstitemlist{

         \sstitem
         The relationships between clock angle, cone angle, solar
         longitude and UTCS are desribed in the ID/1 appendix, {\tt "}Satellite
         Coordinates{\tt "}.

         \sstitem
         This routine uses PSI, not PHI, as the clock angle.
         ( PSI=2.PI-PHI ).
      }
   }
}
\sstroutine{
   IRC\_SIMUL
}{
   See if a CRDD file holds simultaneous detector samples
}{
   \sstdescription{
      A column of data from the NDF DATA array consists of a set of
      samples, one from each detector, which all have the same sample
      number. This routine returns a true value for argument SIMUL if
      the detector samples in each column were obtained simultaneously.
      If this is true, it can be assumed that the boresight position at
      a given sample number is the same for every detector (which can
      reduce the amount of computation required by an application).
   }
   \sstinvocation{
      CALL IRC\_SIMUL( IDC, SIMUL, STATUS )
   }
   \sstarguments{
      \sstsubsection{
         IDC = INTEGER (Given)
      }{
         The IRC identifier for the CRDD file.
      }
      \sstsubsection{
         SIMUL = LOGICAL (Returned)
      }{
         True if all samples in each column of the DATA array were
         obtained simultaneously.
      }
      \sstsubsection{
         STATUS = INTEGER (Given and Returned)
      }{
         The global status.
      }
   }
}
\sstroutine{
   IRC\_SKCO1
}{
   Find sky coordinates of a set of focal plane positions at
   a time specified by a given sample
}{
   \sstdescription{
      This routine returns the Right Ascension and Declination
      of each point in a list of focal plane positions, at the
      moment specified by the given sample number and detector index.
      See ID1 Appendix E for a description of the focal plane (Z,Y)
      coordinate system.
   }
   \sstinvocation{
      CALL IRC\_SKCO1( IDC, SAMPLE, DETIND, NVAL, ZFP, YFP, RA, DEC,
                      STATUS )
   }
   \sstarguments{
      \sstsubsection{
         IDC = INTEGER (Given)
      }{
         An IRC identifier for the CRDD file.
      }
      \sstsubsection{
         SAMPLE = REAL (Given)
      }{
         The fractional sample number at which to do the conversion.
         If this has the Starlink {\tt "}BAD{\tt "} value (VAL\_\_BADR) then all
         elements of the RA and DEC arrays are returned bad.
      }
      \sstsubsection{
         DETIND = INTEGER (Given)
      }{
         The detector index to which the given sample number refers.
         This must be within the bounds of the second dimension of the
         NDF.
      }
      \sstsubsection{
         NVAL = INTEGER (Given)
      }{
         The number of focal plane positions to be converted.
      }
      \sstsubsection{
         ZFP( NVAL ) = REAL (Given)
      }{
         The focal plane Z coordinate values, in radians. If any value
         is equal to the Starlink {\tt "}BAD{\tt "} value (VAL\_\_BADR) then the
         corresponding value in both the RA and DEC arrays are set to
         the bad value VAL\_BADD.
      }
      \sstsubsection{
         YFP( NVAL ) = REAL (Given)
      }{
         The focal plane Y coordinate values, in radians. If any value
         is equal to the Starlink {\tt "}BAD{\tt "} value (VAL\_\_BADR) then the
         corresponding value in both the RA and DEC arrays are set to
         the bad value VAL\_\_BADD.
      }
      \sstsubsection{
         RA( NVAL ) = DOUBLE PRECISION (Returned)
      }{
         The Right Ascension (B1950 FK4) of each focal plane positions,
         in radians.
      }
      \sstsubsection{
         DEC( NVAL ) = DOUBLE PRECISION (Returned)
      }{
         The Declination (B1950 FK4) of each focal plane positions,
         in radians.
      }
      \sstsubsection{
         STATUS = INTEGER (Given and Returned)
      }{
         The global status.
      }
   }
}
\sstroutine{
   IRC\_SKCO2
}{
   Find sky coordinates of a single focal plane position, at a set
   of different times
}{
   \sstdescription{
      This routine returns the Right Ascension and Declination values
      of the given focal plane position, at times specified by the
      given sample numbers and detector indices. See ID1 Appendix E for
      a description of the focal plane (Z,Y) coordinate system.
   }
   \sstinvocation{
      CALL IRC\_SKCO2( IDC, NVAL, SAMPLE, DETIND, ZFP, YFP, DPW1,
                      DPW2, DPW3, RW4, RA, DEC, STATUS )
   }
   \sstarguments{
      \sstsubsection{
         IDC = INTEGER (Given)
      }{
         An IRC identifier for the CRDD file.
      }
      \sstsubsection{
         NVAL = INTEGER (Given)
      }{
         The number of times at which the given focal plane position is
         to be converted.
      }
      \sstsubsection{
         SAMPLE( NVAL ) = REAL (Given)
      }{
         The fractional sample numbers at which to do the conversion.
         If any sample is equal to the Starlink {\tt "}BAD{\tt "} value (VAL\_\_BADR)
         then the corresponding elements of the RA and DEC arrays are
         returned bad.
      }
      \sstsubsection{
         DETIND( NVAL ) = INTEGER (Given)
      }{
         The detector indices to which the given sample numbers refer.
         These must be within the bounds of the second dimension of the
         NDF.
      }
      \sstsubsection{
         ZFP = REAL (Given)
      }{
         The focal plane Z coordinate value, in radians. If this is
         equal to the Starlink {\tt "}BAD{\tt "} data value (VAL\_\_BADR), then both
         output arrays (RA and DEC) are filled with the bad value
         VAL\_\_BADD.
      }
      \sstsubsection{
         YFP = REAL (Given)
      }{
         The focal plane Y coordinate value, in radians. If this is
         equal to the Starlink {\tt "}BAD{\tt "} data value (VAL\_\_BADR), then both
         output arrays (RA and DEC) are filled with the bad value
         VAL\_\_BADD.
      }
      \sstsubsection{
         DPW1( NVAL ) = DOUBLE PRECISION (Returned)
      }{
         Double precision workspace.
      }
      \sstsubsection{
         DPW2( NVAL ) = DOUBLE PRECISION (Returned)
      }{
         Double precision workspace.
      }
      \sstsubsection{
         DPW3( NVAL ) = DOUBLE PRECISION (Returned)
      }{
         Double precision workspace.
      }
      \sstsubsection{
         RW4( NVAL ) = REAL (Returned)
      }{
         Real workspace.
      }
      \sstsubsection{
         RA( NVAL ) = DOUBLE PRECISION (Returned)
      }{
         The Right Ascension (B1950 FK4) of the focal plane position,
         (in radians) at each boresight position.
      }
      \sstsubsection{
         DEC( NVAL ) = DOUBLE PRECISION (Returned)
      }{
         The Declination (B1950 FK4) of the focal plane position,
         (in radians) at each boresight position.
      }
      \sstsubsection{
         STATUS = INTEGER (Given and Returned)
      }{
         The global status.
      }
   }
}
\sstroutine{
   IRC\_SUPP
}{
   Locate an item of support information
}{
   \sstdescription{
      CRDD files may contain {\tt "}support information{\tt "}. This is information
      which is supplied with the data but which is not actually needed
      for the operation of any other IRC routine. The support
      information available will in general depend on the type of CRDD
      contained in the CRDD file, but may contain things like the
      geographic longitude and latitude at various times throughout the
      scan. The exact support information available is listed in the
      appendices of ID1 describing each individual CRDD type. This
      routine returns an HDS locator to a particular item of support
      information. No error is reported if the item does not exist, but
      the argument THERE is returned false. The correct interpretation
      of the support information is the responsibility of the calling
      application.
   }
   \sstinvocation{
      CALL IRC\_SUPP( IDC, NAME, THERE, LOC, STATUS )
   }
   \sstarguments{
      \sstsubsection{
         IDC = INTEGER (Given)
      }{
         An IRC identifier for the CRDD file.
      }
      \sstsubsection{
         NAME = CHARACTER $*$ ( $*$ ) (Given)
      }{
         The HDS name of the required item of support information.
      }
      \sstsubsection{
         THERE = LOGICAL (Returned)
      }{
         Returned true if the item of support information was found, and
         false otherwise.
      }
      \sstsubsection{
         LOC = CHARACTER $*$ ( $*$ ) (Returned)
      }{
         An HDS locator to the required item of support information. The
         variable supplied for this argument should have a declared
         length given by the symbolic constant DAT\_\_SZLOC.
      }
      \sstsubsection{
         STATUS = INTEGER (Given and Returned)
      }{
         The global status.
      }
   }
}
\sstroutine{
   IRC\_TRACE
}{
   Display information about a CRDD file
}{
   \sstdescription{
      This routine displays information about a CRDD file, using the
      supplied routine to display each line of text. The particular
      information displayed will depend on the type of CRDD file.
   }
   \sstinvocation{
      CALL IRC\_TRACE( IDC, ROUTNE, STATUS )
   }
   \sstarguments{
      \sstsubsection{
         IDC = INTEGER (Given)
      }{
         The IRC identifier for the astrometry structure.
      }
      \sstsubsection{
         ROUTNE = EXTERNAL (Given)
      }{
         A routine to which is passed each line of text for display.
         It should have the same argument list as MSG\_\_OUTIF (see
         SUN/104), and should be declared EXTERNAL in the calling
         routine. This routine is called with a priority of MSG\_\_NORM
         for the more commonly needed information, and MSG\_\_VERB for the
         less commonly needed information.
      }
      \sstsubsection{
         STATUS = INTEGER (Given and Returned)
      }{
         The global status.
      }
   }
}
\sstroutine{
   IRC\_TRUNC
}{
   Find the section of a scan containing data from all available
   detectors
}{
   \sstdescription{
      Each detector data stream starts and finishes at a different
      in-scan position, resulting in there being a section at each end
      of the scan which is not covered by data from all available
      detectors. This routine returns the upper and lower sample number
      limits (for each requested detector data stream) of the section
      of the CRDD file for which there is data from all the requested
      detectors.
   }
   \sstinvocation{
      CALL IRC\_TRUNC( IDC, NDETS, DETIND, SAMPLO, SAMPHI, STATUS )
   }
   \sstarguments{
      \sstsubsection{
         IDC = INTEGER (Given)
      }{
         The IRC identifier for the CRDD file.
      }
      \sstsubsection{
         NDETS = INTEGER (Given)
      }{
         The number of detector data streams to be included.
      }
      \sstsubsection{
         DETIND( NDETS ) = INTEGER (Given)
      }{
         A list of detector indices defining the detector data streams
         to be included.
      }
      \sstsubsection{
         SAMPLO( NDETS ) = REAL (Returned)
      }{
         The sample numbers defining the start of a section from each
         detector data stream for which there is corresponding data
         from all the other included detectors.
      }
      \sstsubsection{
         SAMPHI( NDETS ) = REAL (Returned)
      }{
         The sample numbers defining the end of a section from each
         detector data stream for which there is corresponding data
         from all the other included detectors.
      }
      \sstsubsection{
         STATUS = INTEGER (Given and Returned)
      }{
         The global status.
      }
   }
}
\section {Templates for IRC Routines Within the VAX LSE Editor}
The STARLSE package (see SUN/105) provides facilities for initialising the VAX
Language Sensitive Editor (LSE) to simplify the generation of Fortran
code conforming to the Starlink programming standard (see SGP/16). One of the
facilities provided by LSE is the automatic production of argument lists for
subroutine calls. Templates for all the subroutines in the IRC package can be 
made available within LSE by performing the following steps (within LSE):
\begin{enumerate}
\item Issue the LSE command GOTO FILE/READ IRC\_DIR:IRC.LSE
\item Issue the LSE command DO
\item Issue the LSE command DELETE BUFFER.
\item Move to a buffer holding a .FOR of a .GEN file in the usual way.
\item IRC subroutine templates are then available by typing in the name of an 
IRC subroutine (or an abbreviation) and expanding it (CTRL-E).
\item Help on the subroutine and its arguments can be obtained by placing the
cursor at a point in the buffer at which the subroutine name has been entered 
and pressing GOLD-PF2.
\end{enumerate}

\section {Examples of Using IRC}
\label{APP:EXAMS}
\subsection {Accessing a Single CRDD File}
This example shows the main code of a VMS program to display the coordinates of 
the reference point of a CRDD file.

\begin{quote}
\begin{tabbing} % Not clear why this is needed, since tabs are not used,
                % but the leading spaces get lost otherwise.

\verb#*  Include ADAM and IRA symbolic constants.            #\\
\verb#      INCLUDE 'SAE_PAR'                                #\\
\verb#      INCLUDE 'IRA_PAR'                                #\\
\verb#                                                       #\\
\verb#*  Declare local INTEGER variables.                    #\\
\verb#      INTEGER INDF, IDC, BAND, SOP, OBS, STATUS        #\\
\verb#                                                       #\\
\verb#*  Declare local REAL variables.                       #\\
\verb#      REAL REFRA, REFDEC, NOMSPD                       #\\
\verb#                                                       #\\
\verb#*  Declare local CHARACTER variables.                  #\\
\verb#      CHARACTER*(IRA__SZFSC) RATXT, DECTXT             #\numcir{1}\\
\verb#                                                       #\\
\verb#*  Obtain an NDF identifier for the CRDD file.         #\\
\verb#      CALL NDF_ASSOC( 'IN', 'READ', INDF, STATUS )     #\numcir{2}\\
\verb#                                                       #\\
\verb#*  Initialise the IRC system.                          #\\
\verb#      CALL IRC_INIT( STATUS )                          #\\
\verb#                                                       #\\
\verb#*  Import the CRDD file into the IRC system.           #\\
\verb#      CALL IRC_IMPRT( INDF, IDC, STATUS )              #\numcir{3}\\
\verb#                                                       #\\
\verb#*  Get the coordinates of the CRDD file reference point.#\\
\verb#      CALL IRC_INFO( IDC, BAND, REFRA, REFDEC, NOMSPD, #\numcir{4}\\
\verb#     :               SOP, OBS, STATUS )                #\\
\verb#                                                       #\\
\verb#*  Convert the coordinates into character strings.     #\\
\verb#      CALL IRA_DTOC( REFRA, REFDEC, 'EQUATORIAL(B1950)',#\\
\verb#     :               1, RATXT, DECTXT, STATUS )        #\\
\verb#                                                       #\\
\verb#*  Display the coordinates.                            #\\
\verb#      CALL NDF_MSG( 'FILE', INDF )                     #\\
\verb#      CALL MSG_OUT( 'ROUTINE_MSG1',                    #\\
\verb#     : ' The reference point of CRDD file ^FILE is:',  #\\
\verb#     :               STATUS )                          #\\
\verb#      CALL MSG_OUT( 'ROUTINE_MSG2', RATXT, STATUS )    #\\
\verb#      CALL MSG_OUT( 'ROUTINE_MSG3', DECTXT, STATUS )   #\\
\verb#                                                       #\\
\verb#*  Release the CRDD file from the IRC and NDF systems. #\\
\verb#      CALL IRC_ANNUL( IDC, STATUS )                    #\numcir{5}\\
\verb#      CALL NDF_ANNUL( INDF, STATUS )                   #\\
\verb#                                                       #\\
\verb#*  Close down the IRC system.                          #\\
\verb#      CALL IRC_CLOSE( STATUS )                         #\\

\end{tabbing}
\end{quote}

Programming notes:

\begin{enumerate}

\item The symbolic constant IRA\_\_SZFSC is used to define the length of the
character variables RATXT and DECTXT which hold the formatted sky coordinates of
the reference point. This constant is defined in the included file `IRA\_PAR'. 

\item The ADAM parameter system is used to obtain an NDF. The user may give the 
name of the CRDD file in response to the prompt for parameter `IN'. At this
point the CRDD file is known only to the NDF system, and is identified by its
``NDF identifier'', INDF.

\item The CRDD file is ``imported'' into the IRC system. The NDF identifier
tells IRC which file is to be imported, and an ``IDC identifier'', IDC, is
returned which is used to identify the CRDD file to other IRC routines.

\item IRC\_INFO is called to get the coordinate for the CRDD file reference
point (in 1950 equatorial coordinates). The IRC identifier tells IRC\_INFO which
CRDD file is to be used. 

\item The call to IRC\_ANNUL is not strictly needed since IRC\_CLOSE will
ensure that all IRC identifiers are annuled.

\end{enumerate}

\subsection {Plotting Detector Data Streams}
This example shows how plots of data value against in-scan distance may
be produced for each detector in a CRDD file, with each plot aligned so that 
an in-scan distance of zero refers to the closest approach to the CRDD file
reference point. 

\begin{quote}
\begin{tabbing} 

\verb#*  Get the bounds and size of the NDF.                 #\\
\verb#      CALL NDF_BOUND( INDF, 2, LBND, UBND, NDIM,       #\numcir{1}\\
\verb#     :                STATUS )                         #\\
\verb#      SIZE1 = UBND(1) - LBND(1) +1                     #\\
\verb#      SIZE2 = UBND(2) - LBND(2) +1                     #\\
\verb#                                                       #\\
\verb#*  Find the sample number (FSTSMP) at which the first  #\\
\verb#*  detector reaches its closest approach to the        #\\
\verb#*  reference point.                                    #\\
\verb#      FSTDET = LBND(2)                                 #\\
\verb#      CALL IRC_DCLAP( IDC, FSTDET, REFRA, REFDEC,      #\numcir{2}\\
\verb#     :                FSTSMP, FSTZFP, STATUS )         #\\
\verb#                                                       #\\
\verb#*  Loop round each detector index.                     #\\
\verb#      DO DETIN = LBND(2), UBND(2)                      #\numcir{3}\\
\verb#                                                       #\\
\verb#*  Loop round each sample in the current detector data #\\
\verb#*  stream.                                             #\\
\verb#         DO SAMPLE = LBND(1), UBND(1)                  #\\
\verb#                                                       #\\
\verb#*  Find the in-scan distance between the current sample#\\
\verb#*  and the sample which defines the closest approach of#\\
\verb#*  the first detector to the reference point.          #\\
\verb#            CALL IRC_DIST( IDC, FSTSMP, FSTDET,        #\numcir{4}\\
\verb#     :          REAL( SAMPLE ), DETIN, INSCAN, STATUS )#\\
\verb#                                                       #\\
\verb#*  Convert to arc minutes and store in an array which  #\\
\verb#*  will be used to define the X coordinate of each     #\\
\verb#*  point in the plot.                                  #\\
\verb#            XDAT( SAMPLE ) = INSCAN*IRA__RTOD*60.0     #\numcir{5}\\
\verb#                                                       #\\
\verb#         END DO                                        #\\
\verb#                                                       #\\
\verb#*  Find the detector number corresponding to the       #\\
\verb#*  current detector index.                             #\\
\verb#         DETNO = IRC_DETNO( IDC, DETIN, STATUS )       #\numcir{6}\\
\verb#                                                       #\\
\verb#*  Plot the curve.                                     #\\
\verb#         CALL PLOT( SIZE1, DATA( LBND(1), DETIN ),     #\numcir{7}\\
\verb#     :              XDAT, DETNO, STATUS )              #\\
\verb#                                                       #\\
\verb#*  Do the next detector data stream.                   #\\
\verb#      END DO                                           #\\

\end{tabbing}
\end{quote}

Programming notes:

\begin{enumerate}

\item INDF is the NDF identifier for the CRDD file obtained previously. This
call is made since neither axis is constrained to have a lower bound of 1. 

\item IRC\_DCLAP is called to enable the different plots to be aligned with
each other. The returned value FSTSMP gives the fractional sample number within 
the first detector data stream at which the closest approach to the reference 
point occurs. This sample number is defined as having an in-scan distance of 
zero in all the plots. The choice of the first detector is arbitrary, any 
detector would do.

\item The {\em detector index} value gives the row number within the NDF, not 
the IRAS detector {\em number}. Thus detector index values vary between the 
lower and upper bounds of the second dimension of the NDF.

\item IRC\_DIST gives a positive or negative in-scan distance depending on
which side of FSTSMP the current sample is located. Note, SAMPLE is an 
{\em integer} sample number, but IRC\_DIST requires {\em fractional} values, 
therefore the REAL function must be used.

\item XDAT is a REAL array which should have upper and lower bounds equal to the
those of the first dimension of the NDF. IRC\_DIST returns INSCAN in radians,
and is converted to degrees by multiplying it by the symbolic constant
IRA\_\_RTOD (Radians TO Degrees) defined in the file IRA\_PAR. The factor of
60.0 then converts the degrees value to arc-minutes. 

\item IRC\_DETNO should be declared as an integer in the calling routine.

\item The routine PLOT plots the detector data on some graphics device. More
arguments would probably be needed for this routine in practice. The simple
routine used here would have the following declarations: 

\begin{quote}
\begin{tabbing} 

\verb#      SUBROUTINE PLOT( SIZE, YDATA, XDATA, DETNO,      #\\
\verb#     :                 STATUS )                        #\\
\verb#                                                       #\\
\verb#      INTEGER SIZE, DETNO, STATUS                      #\\
\verb#      REAL YDATA( SIZE ), XDATA( SIZE )                #\\

\end{tabbing}
\end{quote}

Note, the YDATA and XDATA arrays now have a lower bound of 1 to make plotting
easier. Y values which contained the Starlink ``BAD'' value should be excluded
from the plot in some way. 

\end{enumerate}
\section {Legal Values for NDF Component UNITS}
\label{APP:UNITS}
The UNITS component of the NDF holds a character string which is used to 
determine the physical quantity which the sample values represent. Applications 
should always check the UNITS component by comparing its value with the 
IRC symbolic constants (defined in IRC\_PAR) listed below. Using symbolic 
constants rather than literal values reduces the chances of bugs being into 
introduced into software due to errors in the spelling of literal units values.

The following units systems are currently recognised:
\begin{description}
\item [IRC\_\_F] - Flux values in units of Pico-Watts ( i.e. $10^{-12}$ of a 
Watt ) per square metre.
\item [IRC\_\_J] - Flux density values in Janskys. 
\item [IRC\_\_JPS] - Surface brightness values in Janskys per steradian. 
\item [IRC\_\_MJPS] - Surface brightness values in Mega-Janskys per steradian. 
\item [IRC\_\_FPS] - Surface brightness values in Pico-Watts per square metre,
per steradian. 
\end{description}

\section {Legal Values for NDF Component LABEL}
\label{APP:LABEL}
Legal values for the LABEL component of the NDF are:
\begin{description}
\item [Pointed Observation CRDD] - The DATA array holds CRDD derived
from a Pointed Observation (PO), using the main survey array (as opposed to the 
CPC).
\item [Survey CRDD] - The DATA array holds CRDD derived from the all sky survey.
\end{description}

\section {The IRAS focal plane coordinate system}
\label{APP:FOCALP}
Positions within the focal plane are described in the IRAS Catalogs and Atlases 
Explanatory Supplement by a Cartesian coordinate system in which the two axes
are labelled $Z$ and $Y$ ($X$ is normal to the focal plane in the direction of 
the boresight). The origin of the $ZY$ system is at the focal plane position
corresponding to boresight of the telescope. The Y axis is parallel to the scan 
direction. As a point on the sky passes over the focal plane is moves in the
positive Y direction. This is the opposite direction to that in which the 
boresight moves as it passes over the sky. The difference can be thought in 
terms of where a fictitious observer is placed. An observer ``sat on the focal 
plane'' would see the stars moving in the positive Y direction. An observer
``fixed to the sky'' would see the telescope boresight moving in the negative
Y direction. IRC often uses the positive Y direction as a reference direction.

The Z axis is in the cross-scan direction. Rotation from the positive Y 
direction to the positive Z direction is in the same sense as rotation from
east to north. In fact, the Z axis always points towards the Sun.

The Explanatory Supplement contains a schematic diagram of the focal plane (fig.
 II.C.6, page II-11 ) in which the Y axis is positive to the left, and Z is
positive downwards. Table II.C.3 (pages II-12 and II-13) list the focal plane
coordinates of the detector centres. These values, together with many other IRAS
specific constants, are made available by including the file I90\_DAT.FOR, and
should always be accessed by means of the symbolic values defined by that file. 

\section {Survey Data with Boresight Pointing Information}

This appendix holds the definition of the DETAILS component of the CRDD\_INFO
structure, for the case of survey data with ``boresight'' pointing information.
The HDS type of the DETAILS component is SURVEY\_BSIGHT. IRC assumes that such
CRDD files have a constant scan speed equivalent to 3.85 arc-minutes per second,
and that each column in the NDF DATA array corresponds to a set of {\em
simultaneously obtained} detector samples. A ``scan'' in this context consists
of a set of detector samples equally spaced in time (any detector samples for
which no valid data is available are given the Starlink ``BAD'' value
(VAL\_BADR), taken from a single observation within a given SOP. The scan length
is thus restricted to being less than the total scan length of the observation, 
but is otherwise arbitrary.

\subsection{ The DETAILS structure}
The DETAILS structure contains the following components:
\begin {description}

\item [BASE\_UTCS] - (A \_DOUBLE scalar) The UTCS of detector sample number 1.
Note, this is not necessarily the {\em first} sample in the DATA array. For
instance, if sample numbers start at -100 then sample 1 is actually the 102'nd
sample. Zero UTCS is taken as 00.00 1-JAN-1981. 

\item [BASE\_MJD] - (A \_DOUBLE scalar) The Modified Julian Date 
corresponding to BASE\_UTCS. 

\item [BORE\_POSNS] - (An \_INTEGER scalar) The number of boresight positions 
stored in the following arrays. 

\item [UTCS\_OFFSET(BORE\_POSNS)] -  (A \_REAL array) The difference in UTCS 
between the associated boresight position and detector sample number 1. Note, 
single precision is sufficient since only UTCS {\em differences} are stored.
By contrast, storing {\em absolute} UTCS values require double precision.

\item [PSI(BORE\_POSNS)] -  (A \_REAL array) The clock angle $\psi$, of the
associated boresight position, in radians. See the IRAS Catalogs and Atlases
Explanatory Supplement (Exp. Supp.) Fig. III.B.7,  for a description of the
``satellite'' coordinate system, ${\psi,\theta}$. 

\item [THETA(BORE\_POSNS)] -  (A \_REAL array) The cone angle $\theta$, of the 
associated boresight position, in radians (see Exp. Supp. Fig. III.B.7).

\item [LAMBDA\_SUN(BORE\_POSNS)] -  (A \_REAL array) The ecliptic longitude
(1950) of the sun at the time of the associated boresight position, in radians. 

\item [SUPPORT\_INFO] - (A structure with type SUPPORT\_INFO) Holds items of 
support information (see section \ref {SEC:SUPP}) as listed below.

\end{description}

The SUPPORT\_INFO structure is used to store information contained in the
Boresight Pointing History Files (BPHF) distributed by IPAC. These files are
described in the IRAS SDAS Software Interface Specification, data set IR.BPHP.D
(PR51) which should be consulted for more information. The following components
are included in SUPPORT\_INFO: 

\begin{description}

\item [UNC\_PSI(BORE\_POSNS)] - (A \_REAL array) The uncertainty in $\psi$ at
the associated boresight position, in radians, given as ``SIGPSI'' in the BPHF. 

\item [UNC\_THETA(BORE\_POSNS)] - (A \_REAL array) The uncertainty in $\theta$
at the associated boresight position, in radians, given as ``SIGNU'' in the 
BPHF. SIGNU is actually the uncertainty in $\nu$, not $\theta$. But since 
$\theta = \nu + \pi/2$ the uncertainty in $\theta$ will equal that in $\nu$.

\item [GEOG\_LAT(BORE\_POSNS)] -  (A \_REAL array) The geographic latitude of 
of the boresight at the associated boresight position, in radians, given as 
``BETG'' in the BPHF.

\item [GEOG\_LONG(BORE\_POSNS)] -  (A \_REAL array) The geographic longitude of 
of the boresight at the associated boresight position, in radians, given as 
``LAMG'' in the BPHF.

\end {description}

\subsection{Satellite Coordinates}
The boresight position is often given in terms of the ``clock angle'', $\psi$
and the ``cone angle'', $\theta$, described in the Exp. Supp. chapter 3. The
satellite coordinates stored in the BPHF are currently used by IRC as the basis
for position reconstruction. The reasons for using satellite coordinates rather
than the stored elliptic coordinates are all related to the fact that, for
survey scans, $\theta$ was held constant (see Exp. Supp. paragraph III.B.8), and
the rate of change of $\psi$ with time was constant (see Exp. Supp. paragraph
C.3). The ecliptic coordinates do not share this simple behaviour. Both ecliptic
longitude and ecliptic latitude vary non-linearly with time in a complex manner.
Since boresight position is recorded less frequently than detector data, it is
necessary to interpolate between boresight positions to get the position of the
boresight at detector samples between the stored positions. It is sometimes also
necessary to extrapolate beyond the UTCS range stored in the components of
DETAILS. Both interpolation and extrapolation are easier to perform in
satellite coordinates than ecliptic coordinates. 

Detector sample positions are reconstructed as follows:
\begin{itemize}
\item The solar longitude is precessed from mean 1950 to mean of date. 
\item Least squares linear fits are made to the clock angle $\psi$, the cone
angle $\theta$ and the solar longitude (mean of date). The independent variable 
is the offset in UTCS since detector sample number 1.
\item For each detector sample, the UTCS offset since detector sample number 1 is
calculated assuming that all samples are equally spaced in time. 
\item The clock and cone angles, and the solar longitude (mean of date) at the 
moment the detector sample was taken are estimated using the linear fits 
established earlier.
\item The ecliptic coordinates $(\lambda,\beta)$ (mean of date) of the boresight
are calculated from the clock and cone angles ($\psi$ and $\theta$), and solar
longitude ($\lambda_{sun}$) using the following equations: 
\begin{eqnarray*}
\beta & = & \arcsin( \sin \theta. \cos \psi )\\
\lambda & = & \lambda_{sun} + \arctan \left(\frac{-\sin \theta.\sin \psi}{\cos 
\theta} \right)
\end{eqnarray*}
\item These ecliptic coordinates are then converted to equatorial coordinates 
(B1950).
\end{itemize}

\section {Adding Support for New CRDD File Formats}
IRC has been designed to enable the package to be modified to include support 
for any future variations in CRDD file formats (see section \ref {SEC:INDEP}).
This appendix describes the general outline of how such a modification can be 
made. It should be noted that IRC does not {\em create} CRDD files from 
scratch. An application must exist which will create the CRDD files (usually
by reading a magnetic tape archive). This application must decide how to 
assign values  to the components of the CRDD\_INFO structure (see section
\ref{SEC:CRDD_INFO}. It must also decide on the HDS {\em type} of the DETAILS 
structure (see SUN/92 for restrictions on HDS types), and on what information 
is stored in that structure (and how). The
information in DETAILS must be sufficient to allow sample positions, scan
angles, and scan speeds to be determined (see section \ref{SEC:TERMS}).

Once the structure and type of DETAILS has been defined, various IRC routines 
need to be modified. IRC\_IMPRT reads the HDS type of the DETAILS structure 
from the given CRDD file, checks it is a recognized type, and if so, stores it 
in common (in array CCM\_TYPE). Other routines then check CCM\_TYPE to decide
which lower level subroutine should be called to perform the required task. 
Such lower level routines should exist for each CRDD type. The steps to
including a new CRDD type (once the DETAILS structure is defined) are thus:

\begin {enumerate}
\item Modify IRC\_IMPRT to include the HDS type of the new DETAILS structure in 
the list of recognised types.
\item Modify the following subroutines to include the new type: IRC1\_BCLPI,
IRC1\_BPOSI, IRC1\_CLPNT, IRC1\_DCLPI, IRC1\_DISTI, IRC1\_DPOSI, IRC1\_OFFSI,
IRC\_SATSI, IRC\_TRACI, IRC\_POSMP, IRC\_SIMUL. Most of these routines just
perform a check on the value of CCM\_TYPE and call a relevant lower level
subroutine. 
\item Write the new subroutines to perform the tasks required by the routines 
modified in the previous step. The routines which process type SURVEY\_BSIGHT 
can be examined as examples of such subroutines.
\item Write applications to check that IRC functions correctly with the new CRDD 
type.
\item Modify ID1.TEX to incorporate the new CRDD type. This should include an
appendix which describes the new CRDD format, and the related DETAILS structure.
\end {enumerate}

\section {Packages Called by IRC}
IRC\_ makes calls to the following packages:
\begin {description}
\item [CHR\_] - The CHR character handling package; see SUN/40.
\item [CMP\_] - HDS; see SUN/92.
\item [DAT\_] - HDS; see SUN/92.
\item [ERR\_] - The Starlink error reporting package; see SUN/104.
\item [IRA\_] - The IRAS90 Astrometry package.
\item [MSG\_] - The Starlink message reporting package; see SUN/104.
\item [NAG] - Double precision NAG library.
\item [NDF\_] - The NDF access package; see SUN/33.
\item [PAR\_] - The ADAM parameter system; see SUN/114.
\item [PSX\_] - POSIX interface.
\item [SLA\_] - The SLA package; see SUN/67.
\end{description}

Access to these packages, together with packages called from within these 
packages, is necessary to use IRC. 

\section {IRC Error Codes}
\label {APP:ERRORS}
IRC routines can return any $STATUS$ value generated by the subroutine packages 
which it calls. In addition it can return the following IRC-specific values:

\begin{description}

\item {\bf IRC\_\_BADBN     }\\
An illegal IRAS band number was specified (outside the range 1 to 4). 

\item {\bf IRC\_\_BADDB     }\\
The CRDD file contained data from a detector which is not included in the band 
specified by the BAND component of the CRDD\_INFO structure.

\item {\bf IRC\_\_BADDI     }\\
A detector index is out of bounds. Detector indices must lie within the bounds 
of the second dimension of the NDF.

\item {\bf IRC\_\_BADEX     }\\
Unable to evaluate an accurate position for a detector sample lying outside the 
bounds of the first dimension of the DATA array.

\item {\bf IRC\_\_BADNC     }\\
An illegal coordinate index was given. Coordinates have indices of 1 or 2 (eg
in an equatorial sky coordinate system, Right Ascension has index 1 and 
Declination has index 2).

\item {\bf IRC\_\_BADDN     }\\
A detector number is out of bounds. Detector numbers must lie in the range 
1 to 62.

\item {\bf IRC\_\_BADDT     }\\
The DET\_NUMBERS component of CRDD\_INFO has the wrong size. It should have the 
same size as the second dimension of the NDF.

\item {\bf IRC\_\_BADLA     }\\
An illegal value was found for the LABEL component of the NDF.

\item {\bf IRC\_\_BADTY     }\\
The DETAILS component of CRDD\_INFO has an unrecognised HDS type.

\item {\bf IRC\_\_BADUN     }\\
An illegal value was found for the UNITS component of the NDF.

\item {\bf IRC\_\_D2BIG     }\\
The second dimension of the NDF is too big. The second dimension is restricted 
to 16 rows or less.

\item {\bf IRC\_\_INVID     }\\
An invalid IDC identifier was supplied.

\item {\bf IRC\_\_MULDE     }\\
Two or more rows of the DATA array contains data from the same detector.

\item {\bf IRC\_\_NAGER     }\\
A NAG routine has failed to run succesfully.

\item {\bf IRC\_\_NOBSD     }\\
The CRDD file has insufficient good boresight data to define the boresight
track (at least 1 good sample per 10 seconds is needed).

\item {\bf IRC\_\_NOCCM     }\\
The NDF has no CRDD\_INFO component in the IRAS extension.

\item {\bf IRC\_\_NOEXT     }\\
The NDF has no IRAS extension.

\item {\bf IRC\_\_NOLAB     }\\
A blank value was found for the LABEL component of the NDF.

\item {\bf IRC\_\_NOMOR     }\\
The IRC system is full; all IRC identifiers are in use.

\item {\bf IRC\_\_NOT2D     }\\
The NDF is not two dimensional.

\item {\bf IRC\_\_NOUNI     }\\
A blank value was found for the UNITS component of the NDF.

\item {\bf IRC\_\_NVAL      }\\
An illegal number of values has been supplied.

\item {\bf IRC\_\_ZEROS     }\\
CRDD samples have been found which claim to have the same in-scan position 
(i.e. zero scan speed). This will be caused by a programming error.

\end{description}

\section {Changes Introduced in the Current Version of this Document}
\label {SEC:CHANGES}

Changes introduced in version 14 of ID1:
\begin {enumerate}
\item New routine IRC\_SATCO
\end {enumerate}

Changes introduced in version 13 of ID1:
\begin {enumerate}
\item Description of the use of IRC on UNIX machines included.
\item It is no longer necessary to include I90\_PAR before including
IRC\_ERR.
\item The names of symbolic constants (including STATUS values) which previously
were longer than five characters (excluding the IRC\_\_ prefix) have been
truncated to five characters. The actual {\em values} associated with each name 
have not changed.
\item The VMS version of IRC is now released in the form of a sharable image 
rather than an object library. Applications linked with the sharable image have 
the advantage that they will not need to be {\em re}-linked when IRC is upgraded 
in future. Note, the object library has been removed from the IRC system.
\item The linking and development procedures have been modified to bring them 
into line with Starlink standards (see SSN/8).
\item IRC subroutine templates and help can now be made available in the STARLSE
editing environment.
\item IRC is no longer directly dependant on the ARY\_ system (although it may
still have {\em in}-direct dependencies on ARY\_, for example through the
NDF\_ system).
\item New routine IRC\_TRACE added.

\end {enumerate}

\end{document}
