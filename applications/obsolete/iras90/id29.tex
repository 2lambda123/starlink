\documentstyle[11pt]{article}
\pagestyle{myheadings}

%------------------------------------------------------------------------------
\newcommand{\stardoccategory}  {IRAS90 Document}
\newcommand{\stardocinitials}  {ID}
\newcommand{\stardocnumber}    {29.0}
\newcommand{\stardocauthors}   {David S. Berry}
\newcommand{\stardocdate}      {21th May 1993}
\newcommand{\stardoctitle}     {COLTEMP and COLCORR -- Algorithms}
%------------------------------------------------------------------------------

\newcommand{\stardocname}{\stardocinitials /\stardocnumber}
\renewcommand{\_}{{\tt\char'137}}     % re-centres the underscore
\markright{\stardocname}
\setlength{\textwidth}{160mm}
\setlength{\textheight}{230mm}
\setlength{\topmargin}{-2mm}
\setlength{\oddsidemargin}{0mm}
\setlength{\evensidemargin}{0mm}
\setlength{\parindent}{0mm}
\setlength{\parskip}{\medskipamount}
\setlength{\unitlength}{1mm}
\setlength{\unitlength}{1mm}

%------------------------------------------------------------------------------
% Add any \newcommand or \newenvironment commands here

% degrees symbol
\newcommand{\dgs}{\hbox{$^\circ$}} 
% centre an asterisk
\newcommand{\lsk}{\raisebox{-0.4ex}{\rm *}}
% A kind of list item, like description, but with an easily adjustable
% item separation.
\newcommand{\menuitem}[2]
  {{\bf #1}: \addtolength{\baselineskip}{-0.4ex}
  \parbox[t]{128mm}{#2} \addtolength{\baselineskip}{0.4ex} \\ \hspace{-5pt}}
% an environment for references
\newenvironment{refs}{\goodbreak
                      \vspace{3ex}
                      \begin{list}{}{\setlength{\topsep}{0mm}
                                     \setlength{\partopsep}{0mm}
                                     \setlength{\itemsep}{0mm}
                                     \setlength{\parsep}{0mm}
                                     \setlength{\leftmargin}{1.5em}
                                     \setlength{\itemindent}{-\leftmargin}
                                     \setlength{\labelsep}{0mm}
                                     \setlength{\labelwidth}{0mm}}
                    }{\end{list}}

\begin{document}
\thispagestyle{empty}
SCIENCE \& ENGINEERING RESEARCH COUNCIL \hfill \stardocname\\
RUTHERFORD APPLETON LABORATORY\\
{\large\bf IRAS90\\}
{\large\bf \stardoccategory\ \stardocnumber}
\begin{flushright}
\stardocauthors\\
\stardocdate
\end{flushright}
\vspace{-4mm}
\rule{\textwidth}{0.5mm}
\vspace{5mm}
\begin{center}
{\Large\bf \stardoctitle}
\end{center}
\vspace{5mm}
\setlength{\parskip}{0mm} \tableofcontents
\setlength{\parskip}{\medskipamount} \markright{\stardocname}
\vspace{1cm}

The {\small COLTEMP} application produces a colour temperature map and an 
optical depth map from two input surface brightness images in different
wavebands. Temperatures in the range 10 $K$ to 10000 $K$ can be handled.

The {\small COLCORR} application produces a colour corrected surface brightness 
image from a temperature image and an optical depth image (previously produced 
by {\small COLTEMP}).

\section{The Planck Function}

The intensity of radiation emitted by a blackbody in
$erg.sec^{-1}.cm^{-2}.ster^{-1}.Hz^{-1}$ as a function of frequency $\nu$ (in
$Hz$) and temperature $T$ (in $Kelvin$) is given by the Planck function
$B(\nu,T)$: 

\begin{equation}
\label{EQ:BB}
B( \nu, T ) = \frac{2h\nu^{3}/c^{2}}{e^{h\nu/kT}-1}
\end{equation}

where $h$ is Planck's constant ($6.626\times 10^{-27} erg.sec$), $k$ is
Boltzmann's constant ($1.381\times 10^{-16} erg.deg^{-1}$) and $c$ is the
velocity of light ($2.9979\times 10^{10} cm.sec^{-1}$). 

If the source is at a distance $D$ from a detector with area $A$, then the flux
density emitted into a solid angle of $A/D^{2}$ steradians is collected by the
detector. If the emitting area of the blackbody is $\sigma$, this flux density
(in ergs per second per Hertz per square centimetre of collecting surface) is
$f(\nu,T)$ where 

\begin{eqnarray*}
f( \nu, T ) = \frac{B( \nu, T ).\sigma.A/D^{2}}{A}
\end{eqnarray*}

which is just 

\begin{eqnarray*}
f( \nu, T ) = B( \nu, T ).\sigma/D^{2}
\end{eqnarray*}

The solid angle of the source as seen from the detector is $\sigma/D^{2}$ and so
the surface brightness of the source as seen at the detector is just equal to
the Planck function $B(\nu,T)$. To obtain the surface brightness in units of
$Jy/ster$, the Planck function is multiplied by a factor of $10^{29}$. 
If $\nu$ is specified in the more convenient units of $10^{12}$ $Hz$, then
equation \ref{EQ:BB} can be written as:

\begin{equation}
\label{EQ:BB3}
B( \nu, T ) = \frac{C_{1}.\nu^{3}}{e^{C_{2}.\nu/T}-1}
\end{equation}

where

\begin{eqnarray*}
C_{1} & = & 10^{29}.2h.10^{36}/c^{2}\\
      & = & 1.474508 \times 10^{18}
\end{eqnarray*}

and 

\begin{eqnarray*}
C_{2} & = & 10^{12}.h/k\\
      & = & 47.979725
\end{eqnarray*}

Note, the factor of $10^{29}$ in $C_{1}$ is used to convert the returned surface
brightness values into $Jy/ster$. As a numerical example, a source at 100 $K$
emits $1.9 \times 10^{19}$ $Jy/ster$ at a wavelength of 50 $\mu$m (= $5.9958
\times 10^{12}$ $Hz$). Dealing with such large numbers can introduce
computational errors, and so {\small COLTEMP} works with flux densities in units
of $10^{18}$ $Jy$ (an ``Exa-Jansky'', $EJy$). The constant $C_{1}$ then takes
the value $1.474508$. 

The Planck function as described by equation \ref{EQ:BB3} can become difficult
to compute if the term $C_{2}.\nu/T$ becomes too large (the exponential can
result in over-flow), and in these cases ($C_{2}.\nu/T>20$) the following
approximation is used (based on the fact that the $-1$ term in the denominator
is insignificant with respect to the exponential term and so can be ignored): 

\begin{equation}
B( \nu, T ) = C_{1}.\nu^{3}.e^{-C_{2}.\nu/T}
\end{equation}

\section{Greybody Spectrum}
The only source spectrum supported by {\small COLTEMP} at the moment is a
``greybody'' spectrum. This arises within a cloud in which the radiation emitted
by each elemental volume is considered to have a blackbody spectrum, but which
then suffers extinction as it passes through the rest of the dust cloud, the
extinction varying at different frequencies. The surface brightness received by
the detector from such a cloud is $f( \nu, T )$, where: 

\begin{equation}
\label{EQ:FD}
f( \nu, T ) = (1-e^{-\tau(\nu)}).B( \nu, T )
\end{equation}

where $\tau(\nu)$ is the optical depth through the cloud at frequency $\nu$.
This optical depth is modeled by the expression:

\begin{eqnarray*}
\tau(\nu) = (\nu/\nu_{c})^{\beta}
\end{eqnarray*}

where $\beta$ is the emissivity spectral index, and $\nu_{c}$ is the critical 
frequency at which the optical depth is equal to unity. 

{\small COLTEMP} assumes that the source of the extended emission is optically 
thin ($\tau \ll 1.0$) and so:
\begin{eqnarray*}
1-e^{-\tau} \approx \tau
\end{eqnarray*}
Thus 
\begin{equation}
f( \nu, T ) = (\nu/\nu_{c})^{\beta}.B( \nu, T )
\end{equation}

\section{Flux Values Measured by IRAS}
The survey detectors used by {\small IRAS} were sensitive over a wide pass-band,
and so integrated the flux density of the source over a range of frequencies. If
the spectral response of the entire {\small IRAS} system at frequency $\nu$ for 
detectors in the $i$th waveband (i=1, 2, 3, 4 for 12, 25, 60 and 100 $\mu$m) is 
$R_{i}(\nu)$ then the measured flux, $F_{i}$ is:

\begin{equation}
F_{i} = 10^{16}.\Omega.\int R_{i}(\nu).(\nu/\nu_{c})^{\beta}.B( \nu, T ).d\nu
\end{equation}

where $\Omega$ is the effective solid angle of the source. A factor of $10^{12}$
is to take account of $\nu$ being in units of $10^{12}$ $Hz$, and a factor of
$10^{4}$ converts the value of the expression from units of $EJy.Hz$ to
$pW/m^{2}$ (the usual units for flux values in {\small IRAS90}). 

The values for the spectral response curves $R_{i}(\nu)$ are taken from the
{\small IRAS} Catalogs and Atlases Explanatory Supplement, table II.C.5,
``Relative system response''. 

{\small COLTEMP} assumes that the effective solid angle of the source $\Omega$
is just equal to the pixel area. This will not be the case for sources smaller
than the pixel size, and so {\small COLTEMP} should only be used to derive
information about {\em extended} emission. If $f_{i}$ is the flux per steradian
in waveband $i$, then: 

\begin{equation}
\label{EQ:OBS}
f_{i} = 10^{16}.\int R_{i}(\nu).(\nu/\nu_{c})^{\beta}.B( \nu, T ).d\nu
\end{equation}

\section{Evaluating Colour Temperature and Optical Depth}
If we have two independent values of $f_{i}$ taken from two different wavebands,
we can in principle expect to be able to derive just two independent parameters
from the data. Equation \ref{EQ:OBS} has three unknowns $\beta$, $\nu_{c}$ and
$T$, and so a value must be assumed for one of them. Of these unknowns $\beta$
is the least likely to vary across the image, and so the user is required to
supply a value for $\beta$. The task is then to use equation \ref{EQ:OBS} to
evaluate $T$ and $\nu_{c}$, given $\beta$ and two observed flux (per steradian)
values (say $f_{i}$ and $f_{j}$). 

The observations in the two wavebands ($i$ and $j$) are described by

\begin{eqnarray*}
f_{i} = 10^{16}.\int R_{i}(\nu).(\nu/\nu_{c})^{\beta}.B( \nu, T ).d\nu\\
f_{j} = 10^{16}.\int R_{j}(\nu).(\nu/\nu_{c})^{\beta}.B( \nu, T ).d\nu
\end{eqnarray*}

Dividing the two equations gives:

\begin{equation}
\label{EQ:RAT}
\frac{f_{i}}{f_{j}} = \frac{ \int R_{i}(\nu).\nu^{\beta}.B( \nu, T ).d\nu}
{ \int R_{j}(\nu).\nu^{\beta}.B( \nu, T ).d\nu}
\end{equation}

All the constants, including the $(1.0/\nu_{c})^{\beta}$ term in the integrands,
have cancelled, resulting in the right hand side being a function of $T$ but not
$\nu_{c}$. This function can be evaluated for various values of $T$ and a look 
up table created which gives temperature in terms of flux ratio $f_{i}/f_{j}$.
This table is fitted with a cubic spline to enable intermediate flux ratios to 
be used. Having obtained $T$, the critical frequency is given by:

\begin{eqnarray*}
\nu_c^{\beta} = \frac{10^{16}}{f_{i}}.\int R_{i}(\nu).\nu^{\beta}.B( \nu, T ).d\nu
\end{eqnarray*}

and the optical depth at frequency $\nu_{0}$ is:

\begin{equation}
\label{EQ:TAUN}
\tau(\nu_{0}) = \left(\frac{\nu_{0}}{\nu_{c} }\right)^{\beta}
\end{equation}
or
\begin{equation}
\tau(\nu_{0}) = \frac{\nu_{0}^{\beta}.f_{i}}{10^{16}.\int R_{i}(\nu).\nu^{\beta}.B( \nu, T ).d\nu}
\end{equation}

The optical depth values stored in the output {\small NDF} by {\small COLTEMP} 
are scaled by $10^{16}$ to avoid computational problems such as underflow. The 
numbers thus stored represent optical depth in units of $10^{-16}$. With this 
scaling the factor of $10^{16}$ goes from the above equation giving:

\begin{equation}
\label {EQ:TAU}
\tau(\nu_{0}) = \frac{\nu_{0}^{\beta}.f_{i}}{\int R_{i}(\nu).\nu^{\beta}.B( \nu, T ).d\nu}
\end{equation}

\section{Computational Algorithms}
The top level routine {\small COLTEMP} looks after access to input and output
data files, parameters, etc, and then calls lower level routines to do the
calculations. It first calls {\small CTEMZ0} to set up the data describing the
required cubic splines, and then calls {\small CTEMZ2} to store values in the
output data files, using the input data files and the cubic splines found by
{\small CTEMZ0}. 

\subsection{Spline fitting - CTEMZ0}
This routine uses {\small NAG} routines to find the knots and co-efficients
describing two cubic splines, $S_{1}$ and $S_{2}$. The first one, $S_{1}$, gives
temperature as a function of the ratio, $r$, of the flux per steradian seen in
shorter waveband to that seen in the longer waveband:

\begin{eqnarray*}
r & = & f_{i}/f_{j}\\
T & = & S_{1}( r )
\end{eqnarray*}

(where $i<j$). The second one, $S_{2}$, gives the flux per steradian seen in the
shorter waveband as a function of temperature: 

\begin{eqnarray*}
f_{i} = S_{2}( T )
\end{eqnarray*}

The two cubic splines are continuous functions which interpolate a set of
explicitly calculated values calculated as follows. An array of 49 temperature
values is first set up, logarithmically spaced between 10 $K$ and 10000 $K$. The
flux per steradian which would be observed from a greybody at each of these
temperatures is then calculated and stored ($\nu_c$ is fixed at $10^{12}$ $Hz$ for
these calculations). This is done for each of the two wavebands and the ratio of
the two values is stored (these values correspond to the right hand side of
equation \ref{EQ:RAT}). The ratio values increase monotonically with temperature
and so it is possible to invert the relationship and use the ratio value as the
independent variable of the cubic spline, $S_{1}$. The range of ratio values
covered by the cubic spline is returned for use in routine {\small CTEMZ2}. The
second cubic spline, $S_{2}$, is then found by interpolating the flux per
steradian values from the shorter waveband using the temperature values as the
independent variable. 

\subsection{Calculation of observed model flux - CTEMZ1}
The flux per steradian at a given temperature (assuming $\nu_{c}=10^{12}$ $Hz$) 
is calculated by routine {\small CTEMZ1}. The value returned by this routine is:

\begin{eqnarray*}
f_{i} = \int R_{i}(\nu).\nu^{\beta}.B( \nu, T ).d\nu
\end{eqnarray*}
 
The integrand is evaluated at each frequency at which the spectral response 
curve $R_{i}$ is tabulated in the {\em Exp. Supp.}. In addition, two extra 
zero values are appended to the start and end of each response curve. A cubic
spline is then fitted to the integrand values, and a {\small NAG} routine is
used to evaluate the definite integral under the cubic spline. The user may 
choose to modify the spectral response curves used in two ways:

\begin{itemize}
\item Each point in the curve may be shifted by a fixed amount (specified by the 
user) in wavelength.
\item Each response value may be multiplied by a fixed factor (specified by the 
user).
\end{itemize}

These facilities can be used to investigate the effects of uncertainties in the
response curves on the calculated temperatures and optical depths. 

\subsection{Creation of output values - CTEMZ2}
Only input pixel values which are positive and not {\em bad} are used. All other
pixels produce {\em bad} output values for temperature and optical depth. The
input surface brightness values are converted to flux per steradian values and
the ratio of the shorter to longer waveband values is found. If this ratio is
within the ratio range returned by {\small CTEMZ0}, then the cubic spline
$S_{1}$ is evaluated at the ratio value to obtain the corresponding temperature.
If the ratio value is outside the allowed range then {\em bad} values are stored
for temperature and optical depth. This temperature is then used to evaluate the
observed model flux per steradian using cubic spline $S_{2}$. This is used
within equation \ref{EQ:TAU} to evaluate the optical depth at the required
frequency. 

\section{Variances}
Uncertainties in the calculated temperatures and optical depths are caused by 
two effects:

\begin{enumerate}
\item Uncertainty in the spectral response curves $R_{i}$
\item Uncertainty in the input surface brightness values.
\end{enumerate}

No attempt it made to generate output variance values based on the uncertainty
in the spectral response curves. In effect, they are considered to be completely
accurate. However, facilities are provided for the user to investigate the
effects of modifying the response curves in various ways. The curves can either
be shifted in wavelength by a fixed amount or scaled by a given factor before
being used. The changes in the calculated temperatures and optical depths caused
by doing this can be used to get some idea of the uncertainty in the
temperatures and optical depths. The {\em Exp. Supp.} describes the expected
uncertainties in the spectral response curves in paragraph C3 on page VI-28. 

If both input {\small NDF}s have {\small VARIANCE} components then the variance
values can be used to generate variances for the output temperatures and optical
depths. If $\sigma^{2}_{i}$ is the variance in $f_{i}$ (a flux density per 
steradian value from waveband $i$), then the variance in the ratio value $r$ 
($=f_{i}/f_{j}$) is given by $\sigma^{2}_{r}$ where:

\begin{eqnarray*}
\frac{\sigma^{2}_{r}}{r^{2}} = \frac{\sigma^{2}_{i}}{f_{i}^{2}} + \frac{\sigma^{2}_{j}}{f_{j}^{2}} 
\end{eqnarray*}

This assumes that the noise is Gaussian and independent, which is not the case 
with {\small IRAS} noise, but it should still give a useful estimate.

The uncertainty $\sigma_{r}$ in the ratio value can be converted into an 
uncertainty $\sigma_{T}$ in the derived temperature by using the slope of the 
cubic spline $S_{1}$ which gives temperature as a function of ratio:

\begin{eqnarray*}
\sigma_{T} = \frac{dT}{dr}.\sigma_{r}
\end{eqnarray*}

The optical depth (in units of $10^{-16}$) at frequency $\nu_{0}$ is $\tau_{0}$
where: 

\begin{eqnarray*}
\tau_{0} = \nu_{0}^{\beta}.f_{i}/S_{2}( T )
\end{eqnarray*}

where $S_{2}$ is the cubic spline giving the observed flux in waveband $i$ as a
function of temperature. The variance in $\tau_{0}$ is $\sigma^{2}_{\tau}$ 
where:

\begin{eqnarray*}
\frac{\sigma^{2}_{\tau}}{\tau^{2}} = \nu_{0}^{2\beta} .\left( 
\frac{\sigma^{2}_{i}}{f_{i}^{2}}  + \frac{\sigma^{2}_{S_{2}}}{S_{2}^{2}} 
\right)
\end{eqnarray*}

The only unknown here is the uncertainty in the observed flux $\sigma_{S_{2}}$. 
This can be derived from the uncertainty in the temperature $\sigma_{T}$ by 
using the slope of the cubic spline $S_{2}$:

\begin{eqnarray*}
\sigma_{S_{2}} = \frac{dS_{2}}{dT}.\sigma_{T}
\end{eqnarray*}

\section{ Colour Correction -- COLCORR}
A ``colour corrected surface brightness'' value is the flux density per
steradian observed at a given wavelength from a source of given spectrum,
assuming the detectors are only sensitive to a single frequency. Normal surface
brightness values (and flux density values) produced by {\small IRAS90} assume
that the source flux density is proportional to wavelength. This is not the case
for black or grey body sources, and so a correction needs to be made. 

{\small COLCORR} works by taking in temperature and optical depth images 
previously created by {\small COLTEMP}. These two images define the source 
spectrum at every point in the image, and equation \ref{EQ:FD} can be used to 
evaluate the required flux density, $f$. The operation is a lot simpler if it
is again assumed that the sources are optically thin, in which case:

\begin{eqnarray*}
f = \tau.B( \nu, T )
\end{eqnarray*}

The optical depth image produced by {\small COLTEMP} contains the optical depth
at a given frequency $\nu_{0}$. If the user requires the colour corrected
surface brightness at a different frequency, $\nu_{1}$, then the optical depth
values need to be scaled to refer to $\nu_{1}$ rather than $\nu_{0}$. Equation
\ref{EQ:TAUN} can be used to do this: 

\begin{eqnarray*}
\tau(\nu_{1}) = \tau(\nu_{0}) . \left(\frac{\nu_{1}}{\nu_{0}}\right)^{\beta}
\end{eqnarray*}

$\nu_{0}$ and $\beta$ are stored in the {\small IRAS} extension of the optical 
depth image produced by {\small COLTEMP}. The values returned for the Planck 
function $B(\nu, T)$ are in units of Exa-Janskys per steradian, and the stored 
optical depth values are in units of $10^{-16}$, so the final expression for $f$ 
is:

\begin{eqnarray*}
f & = & \tau(\nu_{0}).10^{-16}.\left(\frac{\nu_{1}}{\nu_{0}}\right)^{\beta}.
        B( \nu, T ).10^{18}\\
  & = & \tau(\nu_{0}).\left(\frac{\nu_{1}}{\nu_{0}}\right)^{\beta}.
        B( \nu, T ).10^{2}
\end{eqnarray*}
or
\begin{equation}
\label{EQ:CCOR}
  f = \tau(\nu_{0}).CC.B( \nu, T )
\end{equation}
where $CC$ is a constant:
\begin{eqnarray*}
CC  =  \left(\frac{\nu_{1}}{\nu_{0}}\right)^{\beta}.10^{2}
\end{eqnarray*}
if $f$ is required in units of $Jy/ster$, or:
\begin{eqnarray*}
CC  =  \left(\frac{\nu_{1}}{\nu_{0}}\right)^{\beta}.10^{-4}
\end{eqnarray*}
if $f$ is required in units of $MJy/ster$.

Variances, $\sigma_{f}^{2}$, can be assigned to $f$ if variances,
$\sigma_{T}^{2}$ and $\sigma_{\tau}^{2}$, are supplied for $T$ and
$\tau(\nu_{0})$. $\sigma_{f}^{2}$ is given by:

\begin{eqnarray*}
\sigma_{f}^{2} = \sigma_{\tau}^{2}.\left( \frac{\partial f}{\partial 
\tau}\right)^{2} + \sigma_{T}^{2}.\left( \frac{\partial f}{\partial T}\right)^{2}
\end{eqnarray*}

Taking the partial derivative of equation \ref{EQ:CCOR} with respect to $\tau$ 
gives:

\begin{eqnarray*}
\frac{\partial f}{\partial \tau} = CC.B(\nu, T )
\end{eqnarray*}

and taking the partial derivative of equation \ref{EQ:CCOR} with respect to
$T$ gives: 

\begin{eqnarray*}
\frac{\partial f}{\partial T} = CC.\tau.\frac{\partial B}{\partial T}
\end{eqnarray*}

Looking at equation \ref{EQ:BB3} gives:

\begin{eqnarray*}
\frac{\partial B}{\partial T} & = & \frac{-C_{1}.\nu^{3}}{(e^{C_{2}.\nu/T}-1)^{2}}.
                         e^{C_{2}.\nu/T}.\left(-\frac{C_{2}.\nu}{T^{2}}\right)\\
& = & \left(\frac{B}{\nu.T}\right)^{2}.\frac{C_{2}.e^{C_{2}.\nu/T}}{C_{1}}
\end{eqnarray*}

These equations allow the variance on the colour corrected surface brightness 
values, $\sigma_{f}^{2}$, to be calculated.

\end{document}
