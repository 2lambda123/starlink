
\documentstyle{article}

\pagestyle{myheadings}
%
% *****************************************************************************
% *  This document includes GKS graphics included using the \SPECIAL          *
% *  command. The graphics are read from files A.DAT to I.DAT, and these      *
% *  must exist before running the DVI processor (DVICAN or DVIPS).           *
% *  The graphics data files are created by programs stored with the rest     *
% *  of the IRA release. Procedures exist for building the ID2 document on    *
% *  both VMS and UNIX machines. These procedures first ensure that the       *
% *  programs needed to create the graphics have been built, and then run the *
% *  programs to create the data files (unless the data files already exist). *
% *  They then run LATEX to create the DVI file, and finally run the DVI      *
% *  processor. On VMS, set default to IRA_DIR and execute the command        *
% *                                                                           *
% *  @BUILD_ID2 [printer]                                                     *
% *                                                                           *
% *  where [printer] is either POSTSCRIPT or CANON. On UNIX, change directory *
% *  to the directory containing the IRA release, and execute the command     *
% *                                                                           *
% *  mk [target]                                                              *
% *                                                                           *
% *  where [target] is either doc_can (for a Canon printer) or doc_ps (for a  *
% *  Postscript printer. On both VMS and UNIX, these commands produce device  *
% *  dependant file (eg ID2.DVI-CAN, etc) ready for printing.                 *
% *****************************************************************************
%
%------------------------------------------------------------------------------
\newcommand{\irasdoccategory}  {IRAS90 Document}
\newcommand{\irasdocinitials}  {ID}
\newcommand{\irasdocnumber}    {2.11}
\newcommand{\irasdocauthors}   {D.S. Berry}
\newcommand{\irasdocdate}      {23rd April 1993}
\newcommand{\irasdoctitle}     {IRA: The IRAS Astrometry Subroutine Package}
%------------------------------------------------------------------------------

\newcommand{\irasdocname}{\irasdocinitials /\irasdocnumber}
\renewcommand{\_}{{\tt\char'137}}     % re-centres the underscore
\markright{\irasdocname}
\setlength{\textwidth}{160mm}
\setlength{\textheight}{240mm}
\setlength{\topmargin}{-5mm}
\setlength{\oddsidemargin}{0mm}
\setlength{\evensidemargin}{0mm}
\setlength{\parindent}{0mm}
\setlength{\parskip}{\medskipamount}
\setlength{\unitlength}{1mm}

%------------------------------------------------------------------------------
% Add any \newcommand or \newenvironment commands here
%------------------------------------------------------------------------------

\begin{document}
\thispagestyle{empty}
SCIENCE \& ENGINEERING RESEARCH COUNCIL \hfill \irasdocname\\
RUTHERFORD APPLETON LABORATORY\\
{\large\bf IRAS90\\}
{\large\bf \irasdoccategory\ \irasdocnumber}
\begin{flushright}
\irasdocauthors\\
\irasdocdate
\end{flushright}
\vspace{-4mm}
\rule{\textwidth}{0.5mm}
\vspace{5mm}
\begin{center}
{\Large\bf \irasdoctitle}
\end{center}
\vspace{5mm}
\setlength{\parskip}{0mm}
\tableofcontents
\setlength{\parskip}{\medskipamount}
\markright{\irasdocname}

\section {Introduction}
The IRA library (IRAS90 library A ) contains a set of subroutines which allow
applications running under ADAM on either VMS or UNIX platforms to access and
manipulate astrometry information specified by the application or stored within
two dimensional NDFs (see SUN/33 for information on the NDF data format). The
library forms a sub-system of the IRAS90 package, but there should be nothing to
prevent it being of use in other contexts. IRA provides astrometry facilities
for images. There is nothing IRAS-specific in this package which deals with the
relation between pixel and sky coordinates in an image.

\subsection { The Relationship with the Starlink Astrometry Package}
At the current time no standards exists within Starlink for storing or accessing
astrometric information within NDFs. However, Starlink recognise the need for
such standards and intend to address them in the near future. When such
standards exist, software which uses other methods for handling astrometry will
in general not be able to share data with software which uses the Starlink
standards. The routines within IRA were designed bearing this in mind, with the
intention that when the Starlink standard is defined the IRA routines will be
re-written to use the new standards. It should be possible to do this in such a
way as to leave the argument lists of all the IRA routines un-altered. This
would mean that software which uses IRA will automatically be compatible with
the Starlink standard once IRA has been modified. This provides an interim
solution to the problem of handling astrometry information, while at the same
time minimising the amount of effort which will be needed to upgrade the
software to use the Starlink standard, when such a standard is produced.

\subsection {Facilities Available in the Current Implementation}
The IRA subroutines were designed to meet the needs of the IRAS90 package, and
as such will probably not provide as many facilities as the final Starlink
system, which will presumably have to cater for a wide range of positional
accuracies, dimensionalities, etc. The current IRA system provides facilities
for storing information within an NDF to enable the position of each image pixel
on the sky to be determined (for a limit number of projections), and the image
pixel with any given sky position to be determined. It also provides some
facilities for manipulating sky positions in various ways (eg spherical
trigonometry, formatting and decoding of text strings holding sky positions
etc), and several routines for doing astrometry oriented graphics such as
plotting sky coordinate grids, drawing great circles, etc. The current
implementation of IRA is referred to in this document as the ``interim''
implementation to distinguish it from the version based on the Starlink standard
which will eventually be produced (the use of the word {\em interim} in this
context should not be confused with the Starlink INTERIM environment). The next
section contains more detail about the sort of information which IRA handles.

Astrometric functions that are specific to the IRAS data, such as those
relating to the relative positions of the detectors in the IRAS focal plane are
{\em not} included in this package but in the IRAS90 IRC package (see ID/1).

\section {Overview}

The IRA system is restricted to two dimensional images, in which the centre of
each image pixel represents a point on the sky. Routines exist to relate a pixel
position to a position on the sky and vice-versa. This consists of a description
of the {\em projection} used to map the sky onto the image, and vice-versa, and
a description of the {\em coordinate system} produced by the projection.

The primary purpose of IRA is the storage of information within an image to
enable later applications to transform between image pixel and sky position
coordinates. This is a ``global'' operation which relates to the entire image.
In this sense IRA is restricted to 2D images. A secondary purpose of IRA is to
provide useful routines for manipulating and displaying sky coordinates. These
routines can also be useful in processing non-image data, but the full
facilities of IRA are only available for 2D images. If an application wanted to
stack several images together into a 3D NDF (for example images taken at the
same position with the same pixel size but at several different wavelengths in
each of which the same pixels correspond to the same sky positions), then IRA
could be used as it stands so long as corresponding pixels from each plane of
the image stack have identical sky coordinates. In this sense, sky position is
independent of the third axis and is thus essentially 2D.

\subsection {Image Coordinates}
In what follows, I will refer to the two coordinate systems as ``image
coordinates'' and ``sky coordinates''.
A two dimensional image in which the
integer pixel indices range from $i_{low}$ to $i_{high}$ on the first
(``horizontal'' or ``X'') axis, and from $j_{low}$ to $j_{high}$ on the second
(``vertical'' or ``Y'') axis have image coordinates in the range $i_{low}-1$ to
$i_{high}$ and $j_{low}-1$ to $j_{high}$. Image coordinates are floating point
values in which the centre of the pixel with {\em indices }
$(1,1)$ has {\em image coordinates} $(0.5,0.5)$. Thus in the above example, image
coordinates $(i_{low}-1,j_{low}-1)$ refer to the bottom left corner of pixel
$(i_{low},j_{low})$, and image coordinates $(i_{high},j_{high})$ refers to the
top right corner of pixel $(i_{high},j_{high})$. See SSN/22 for another
description of image coordinates (referred to in that document as ``picture
coordinates'').

\subsection {Sky Coordinates}
\label {SEC:SCS}
Each point on the sky can be represented in several different sky coordinate
systems. IRA restricts sky coordinate systems to two dimensional, spherical
coordinates, representing longitude and latitude values on the celestial sphere.
The symbol ``$A$'' is used throughout IRA to refer to a longitude value, and the
symbol ``$B$'' to refer to a latitude value. All such values are given to and
returned from IRA in the form of double precision floating point variables, in
units of radians. No rules exist relating to the presence or absence of
multiples of $2.\pi$ within sky coordinate values (e.g a value of $4.5.\pi$ will
be accepted). A routine is provided (IRA\_NORM) to normalise longitude/latitude
pairs such that longitude is in the range $0$ to $2.\pi$ and latitude is in the
range $-\pi/2$ to $+\pi/2$.

The sky coordinate systems currently supported by IRA are:
\begin {itemize}
\item Ecliptic (IAU 1980)
\item Equatorial (FK4 and FK5)
\item Galactic (IAU 1958)
\end{itemize}

Ecliptic and equatorial coordinates are referred to the mean equinox of a given
epoch. This epoch is specified within IRA by appending it to the end of the name
of the sky coordinate system, in parentheses; for instance EQUATORIAL(1983.5)
(only the four most significant decimal places are used). The epoch may be
preceded by a single character, B or J, indicating if the epoch is a Besselian
epoch (B) or a Julian epoch (J). If this character is missing (as in the above
example), then the epoch is assumed to be a Besselian epoch. If no equinox is
specified in this way, then a default of B1950.0 is used.

If a Julian epoch is used to specify the reference equinox for an equatorial
coordinate system, then the equatorial coordinates are assumed to be in the
IAU 1976, FK5, Fricke system. If the equinox is specified using a Besselian
epoch, then the coordinates are assumed to be in the FK4, Bessel-Newcomb system.

IRA provides routines to convert coordinates from one system to another,
including precession from one equinox to another if required. No corrections for
nutation are applied in this process. To ensure that all such conversions can be
performed, it is also necessary to store the Julian epoch of the observations. A
single mean epoch (1983.5D0) is sufficient to describe all IRAS data. This value
is available in the symbolic constant IRA\_\_IRJEP which can be made available
by following the instructions described in section \ref {SEC:CON}.

Throughout IRA sky coordinate systems are specified wither by an ``SCS'' value,
or by a ``NAME''. A name specifies the basic system and must be one of the list
given above (eg ``EQUATORIAL''). All IRA routines will accept any unambiguous
abbreviation for such a name. A name in this sense implies nothing about any
reference equinox, and so in general a name does not specify the sky coordinate
system completely. An SCS value consists of a name with an optional equinox
specifier appended to the end, as described above. If no equinox specifier is
contained in an SCS value, the default of B1950.0 is used (if necessary).

\subsection {Projections}
\label {SEC:PARS}
Projections of the sky into an image can be described by mathematical
transformations between sky coordinates and image coordinates. Each of these
projections consist of two ``mappings''. The forward mapping converts image
coordinates to sky coordinates, and the inverse mapping converts sky coordinates
to image coordinates. Projection transformations are thus restricted to mappings
which convert a single pair of numbers, into another single pair of numbers (eg
$(X,Y)$ to $(RA,DEC)$ ).

Currently, explicit support is restricted to the following four projections:
\begin {itemize}
\item Gnomonic (or ``tangent plane'') projection.
\item Lambert normal equivalent cylindrical projection.
\item Aitoff equal area projection.
\item Orthographic projection.
\end {itemize}
The basic algorithms to implement the first three of these projections are
described in the IRAS Catalogues and Atlases Explanatory Supplement, section
D.2, but have been extended to incorporate any orientation and scale for the
image axes, and an arbitrary reference point. The image axes are restricted to
be perpendicular to each other on the sky, with rotation from the first (X) to
the second (Y) image axes being in the sense of rotation from north to east
(i.e. rotation from north to east is anti-clockwise when the image is displayed
with X increasing left to right and Y increasing bottom to top).

The four supported projection are parameterised by eight parameters, as
follows:
\begin {enumerate}
\item The sky longitude (eg RA) of a ``reference'' point. This reference point
usually has some geometric significance for the projection (eg for the Gnomonic,
or tangent plane projection, the reference point is the point on the sky at
which the tangent plane touches the celestial sphere).
\item The sky latitude (eg DEC) of the reference point.
\item The position of the reference point along the first (X) image axis.
\item The position of the reference point along the second (Y) image axis.
\item The absolute size of a pixel along the first image axis.
\item The absolute size of a pixel along the second image axis.
\item The position angle of the second (Y) image axis on the sky with respect to
north.
\item An angle through which to rotate the celestial sphere before the
projection is performed (see below).
\end {enumerate}

The rotation specified by the final parameter is about the radius vector from
the centre of the celestial sphere to the reference point, and thus corresponds
to ``tilting'' the celestial sphere rather than ``spinning'' it. The tilt is
positive in an anticlockwise direction when viewing the centre of the celestial
sphere from the reference position. For some projections (such as Gnomonic and
Orthographic) rotating the celestial sphere before projection has no effect.
This is because the projection surface for these projections is a plane, and is
therefore axisymetric about the radius passing through the reference point. For
other projections this is not true. For instance, the projection surface for the
Lambert normal equivalent cylindrical projection is a cylinder, and thus is {\em
not} axisymetric about the radius passing through the reference point. The
facility for rotating the celestial sphere prior to the projection gives another
degree of freedom in describing the projection, but for most applications the
final parameter will probably be given a value of zero.

See appendix \ref {APP:PROJ} for further information about these projections.

\subsubsection {The Interim Implementation}
IRA conceals the way in which projections are handled. This will hopefully make
it possible to re-implement IRA in terms of the official Starlink astrometry
system without changing any argument lists. In fact, the {\em interim} version
of IRA handles most projections by storing the projection name and projection
parameters in the NDF. Coordinate data are then transformed by calling an
appropriate IRA routine which performs the required projection. This is a
relatively fast way of doing things, but it restricts the projections which can
be used to those for which explicit IRA transformation routines exist.

\section {Using IRA Subroutines}
This section gives a brief outline of the IRA routines which are available to
perform some common tasks. The specific details required to use these routines
are not included here but can be found in the subroutine specifications
contained in appendix \ref {SEC:FULLSPEC}.

\subsection {Constants and Error Values}
\label {SEC:CON}
The IRA package has associated with it various symbolic constants defining such
things as the required length of various character variables, the Julian epoch
of the IRAS mission, etc. These values consist of a name of up to 5 characters
prefixed by ``IRA\_\_''  (note the {\em double} underscore), and can be made
available to an application by including the following lines at the start of the
routines which uses them:

\begin{verbatim}
      INCLUDE 'IRA_PAR'
\end{verbatim}


IRA\_PAR should be a logical name (on VMS) or symbolic links (on UNIX) pointing
to the appropriate files.

Note, Fortran source files for use on UNIX systems can be automatically
generated from those used on VMS systems, using the FORCONV program descibed
in SUN/111. This avoids the need to store and maintain two separate systems of
source files, and is highly recommended!

The values thus defined are described in the following sections, and also in the
subroutine specifications. Of particular use may be a set of multiplicative
constants for converting between various forms of representing angular values:

\begin{description}
\item [IRA\_\_RTOD] - Factor for converting radians to degrees.
\item [IRA\_\_R2AS] - Factor for converting radians to arc-seconds.
\item [IRA\_\_R2AM] - Factor for converting radians to arc-minutes.
\item [IRA\_\_R2TS] - Factor for converting radians to seconds.
\item [IRA\_\_R2TM] - Factor for converting radians to minutes.
\item [IRA\_\_R2TH] - Factor for converting radians to hours.
\item [IRA\_\_DTOR] - Factor for converting degrees to radians.
\item [IRA\_\_AS2R] - Factor for converting arc-seconds to radians.
\item [IRA\_\_AM2R] - Factor for converting arc-minutes to radians.
\item [IRA\_\_TS2R] - Factor for converting seconds to radians.
\item [IRA\_\_TM2R] - Factor for converting minutes to radians.
\item [IRA\_\_TH2R] - Factor for converting hours to radians.

\end{description}

In addition, the following multiples of $\pi$ are defined:
\begin{description}
\item [IRA\_\_PI] - $\pi$
\item [IRA\_\_PIBY2] - $0.5*\pi$
\item [IRA\_\_TWOPI] - $2.0*\pi$
\end{description}

Another set of symbolic constants is made available by the statements

\begin{verbatim}
      INCLUDE 'IRA_ERR'
\end{verbatim}

These values have the same format of those contained in IRA\_PAR, put define
various error conditions which can be generated within the IRA package.
Applications can compare the $STATUS$ argument with these values to check for
specific error conditions. These values are described in appendix \ref
{APP:ERRORS}.

\subsection{IRA Identifiers}
IRA can store astrometry information for many different pixel grids
simultaneously. To distinguish between them, IRA uses a system of
``identifiers''. An IRA identifier is an integer value which is used by IRA to
locate the astrometry information relating to a particular pixel grid. These
identifiers are generated by the IRA package and returned to the calling
application, which can then pass them to other IRA routines to specify which
astrometry information is to be used. When access is no longer needed by the
application to the astrometry information, the related IRA identifier should be
annulled using routine IRA\_ANNUL. This releases the resources used by the
astrometry information, so that it can be re-used.

Note, IRA stores astrometry information internally within common arrays. The
identifiers used by IRA identify a particular set of values stored in common,
{\em not} the HDS structures which hold the astrometry information in the NDF.
Routines IRA\_IMPRT and IRA\_READ copy the astrometry information from the HDS
object, into internal common storage. Any modification made to the astrometry
information (such as by routine IRA\_ROT, etc) will only change the internally
held values, {\em not} the values stored in the NDF. To effect permanent changes
in the NDF values, the interbnally held astrometry information should be written
out using IRA\_CREAT, IRA\_EXPRT or IRA\_WRITE. This approach means that NDF
identifiers and HDS locators can be annulled by the calling application once the
astrometry information has been copied into internal storage.

\subsection {Starting Up and Shutting Down}
The routine IRA\_INIT should be called before calling any IRA routines which has
an IRA identifier as an argument. It initialises the IRA internal common blocks
and annuls any currently active IRA identifiers. When all handling of
astrometry information has been completed, IRA\_CLOSE should be called which
ensures that all IRA identifiers have been annulled. IRA\_ACTIV can be used to
see if IRA is currently active.

\subsection {Creation of New IRA Identifiers}
IRA identifiers are used to refer to a particular set of astrometry information.
Each identifier has associated with it information describing a projection, a
sky coordinate system and an observation epoch (this information is actually
stored within common blocks internal to IRA). IRA identifiers can be created by
any of the routines IRA\_CREAT, IRA\_IMPRT and IRA\_READ. The difference between
these routines lies in the source from which they obtain the astrometry
information. If the calling application already knows the projection name,
projection parameters, sky coordinate system and observation epoch, then
IRA\_CREAT can be called. As well as storing the astrometry information
internally and returning an IRA identifier for it, this routine also has an
option for storing the astrometry information in an NDF. If the astrometry
information is to be obtained from an NDF, then IRA\_IMPRT should be called. The
calling application gives an NDF identifier and a search is made through the NDF
for suitable astrometry information. Such information is then copied into the
IRA common blocks and an IRA identifier is returned for it. IRA\_READ is similar
to IRA\_IMPRT except that the calling application supplies a locator to an HDS
object rather than an NDF identifier. The astrometry information is read from
the specified HDS object and an IRA identifier returned for it.

IRA identifiers should be annulled using IRA\_ANNUL when no longer needed.

\subsection {Storing Astrometry Information within an NDF}
Astrometry information is stored within an NDF in an ``astrometry structure''.
In the interim implementation, this consists of an HDS object of HDS type
IRAS\_ASTROMETRY which is stored as a component within an NDF extension.
IRA\_EXPRT can be called to create an astrometry structure within an NDF if an
IRA identifier is already available for the astrometry information.
Alternatively, IRA\_CREAT can be called if no IRA identifier has yet been issued
for the astrometry information. By default, these routines create astrometry
structures within the ``IRAS'' extension, and the astrometry structure itself is
assumed to have an HDS name of ``ASTROMETRY''. These names may be changed by
calling the routine IRA\_LOCAT, making it possible for IRAS to be used for NDFs
which don't contain an IRAS extension. The names set up by the call to
IRA\_LOCAT remain in force until another call to IRA\_LOCAT or a call to
IRA\_INIT.

The NDF must be open with WRITE or UPDATE access in order to create a new
astrometry structure. NB, the NDF extension which is to hold the astrometry
structure must already exist. If IRA\_CREAT or IRA\_EXPRT cannot find the
required NDF extension, it will return with $STATUS$ set to the value
IRA\_\_NOEXT.

\subsection {Reading Astrometry Information from an NDF}
To handle astrometry information stored in an existing NDF, an identifier to the
NDF must first be obtained (eg using NDF\_ASSOC, see SUN/33). This is then
passed to the routine IRA\_IMPRT to import the astrometry information from the
NDF into IRA internal common blocks. IRA\_IMPRT searches through all available
NDF extensions until a suitable astrometry structure is found, and then copies
the information from the astrometry structure into the IRA internal common
blocks. It then returns an IRA identifier for the astrometry information which
can be passed to other IRA routines to gain access to the information stored in
common. The IRA identifier should be annulled using IRA\_ANNUL when it is no
longer needed.

IRA\_FIND will locate the astrometry structure in an NDF but does not read the
astrometry information into common. This routine can be used (for instance) to
locate the astrometry structure prior to deleting it.

IRA\_SCSEP can then be called to get the sky coordinate system (including an
equinox specifier if required) generated by the projection, and the Julian epoch
of the observations.

IRA\_READ is like IRA\_IMPRT except that it reads the astrometry information
out of a specified HDS object.

IRA\_TRACE can be used to display information about an existing astrometry
information on the screen.

\subsection {Modifying Astrometry Information}
If the pixel values store in an image are moved around, the associated
astrometry information will no longer correctly describe the astrometry of the
image. The routines IRA\_MAG, IRA\_MOVE and IRA\_ROT can be used to modify
astrometry information to take account of various linear geometric
transformations which may be applied to the data. The transformations are
restricted to ones which will not alter the basic properties of the projection.
They are :

\begin{itemize}
\item Shift the reference position to a different location in image coordinates
(see routine IRA\_MOVE).
\item Magnify the image by reducing the pixel size. Separate magnification
factors can be specified for the X and Y axes (see routine IRA\_MAG).
\item Rotate the image coordinate system (see routine IRA\_ROT).
\end{itemize}

Note, these routines do not write the modified astrometry information back to
any NDF or HDS object from which it may have been originally obtained. If the
values are to be stored back in an NDF or HDS object then a call must be made to
IRA\_EXPRT or IRA\_WRITE.

\subsection {Transforming Coordinate Data}
The subroutine IRA\_TRANS can be used to transform coordinate values from image
coordinates to sky coordinates, or vice-versa, using the projection specified by
a given IRA identifier (the actual {\em name} of the projection is not
available to the application). If many data points are to be transformed, it is
faster to put the data into two arrays, and make a single call to IRA\_TRANS to
process them all, rather than making many calls to IRA\_TRANS to process a
single point each call.

\subsection {Handling Formatted Sky Coordinates}
A formatted sky coordinate value consists of a character string holding a sky
coordinate value in character form, in a format which is readily usable by a
user. A single value can refer to either a longitude or a latitude value. A pair
of formatted sky coordinate values consists of one of each. Routines exist to
convert from floating point values (in radians) to formatted strings, and
vice-versa, which allow several different formats for the strings. IRA\_CTOD and
IRA\_DTOC decode and encode a {\em pair} of floating point values, IRA\_CTOD1
and IRA\_DTOC1 handle {\em single} values. The encoding routines always ensure
that displayed longitude values are in the range zero to 360 degrees (or the
equivalent time range, zero to 24 hours) and that the displayed latitude is in
the range +90 degrees to -90 degrees.

IRA\_GETCO gets a pair of sky coordinate values from the ADAM environment
using the ADAM parameter system. Two parameters are used, one for each value.
For each, the user gives a string in a format acceptable to IRA\_CTOD.

\subsection {Mathematical Operations on Sky Coordinate Values}
Sky coordinate positions can be converted from one supported sky coordinate
system to another using IRA\_CONVT (which in turn calls various SLALIB routines,
see SUN/67). IRA\_PACON converts both sky positions {\em and} position angles.
Any required precession is performed by these routines, but nutation is ignored.

The length of the arc joining two given sky positions can be found using
IRA\_DIST. Given two sky positions and a position angle (measured from north
through east in the current sky coordinate system), IRA\_DIST2 returned two
arc-distances, which are the distances which must be moved parallel and
perpendicular to the given position angle in order to move from the first sky
position to the second.

Two routines exist for finding a sky position which is a given distance away
from a reference sky position. They differ in how the direction of movement is
defined. IRA\_OFFST moves away from the reference point, towards a specified
second point. IRA\_SHIFT moves away from the reference point at a given position
angle (measured north through east in the current sky coordinate system).

Redundant multiple of $2.\pi$ can be removed from a pair of sky coordinate
values by calling IRA\_NORM. This routine ensures that longitude values are in
the range zero to $2.\pi$, and latitude values are in the range $+\pi/2$ to
$-\pi/2$.

\subsection {Encoding and Decoding Equinox Specifiers}
Certain sky coordinate systems (eg ECLIPTIC and EQUATORIAL) are referred to the
mean equinox of a given epoch. This equinox is specified by appending the epoch
to the name of the sky coordinate system as described in section \ref {SEC:SCS}.
The routine IRA\_SETEQ can be used to do the character string manipulation
required to do this. The routine IRA\_GETEQ will do the opposite of this; given
an SCS string it will return the name of the sky coordinate system, the floating
point epoch value, and a character indicating if the epoch is Besselian or
Julian.

The routine IRA\_GTSCS will get a complete SCS string (including equinox
specifier) from the ADAM environment, using a specified ADAM parameter.

\subsection {Graphics}
Routines are provided for producing astrometry related graphics. These routines
plot within a nominated section of the current SGS zone. Routines are available
for drawing individual meridians, parallels and great circles, for plotting
formatted coordinate values, and for drawing a boundary around the region
containing valid sky coordinates. A routine is provided which packages these in
order to produce a labelled sky coordinate grid. The accuracy of plotted curves
can be specified by the calling application to give the required play-off
between accuracy and speed of execution.

\subsection {Calculating Pixel Bounds Required to Cover a Given Area of Sky}
When creating an image from scratch, it is often necessary to determine the
pixel bounds which will result in the image covering a given area of sky. The
area is usually specified by giving the sky coordinates of the centre of the
image, and the size of the image (i.e. an arc-length in radians) along both
axes. Since pixel size in general varies across an image it is not always
sufficient to divide the required size of an image by the ``nominal'' pixel size
to get the dimension of the image in pixels. The routine IRA\_XYLIM finds the
pixel bounds which result in the two image axes having the required arc-length.

\subsection {Miscellaneous Routines}
IRA\_SCNAM will return a description of each axis of the sky coordinate system,
and an abbreviation of its name. For example, when using an equatorial sky
coordinate system, the description of the first axis is ``Right Ascension'' and
the abbreviation is ``RA''.

\subsection {Demonstration Applications}
Two demonstration applications are available in IRA\_DIR which give examples of
the use of most of the IRA routines. IRA\_DEMO1 produces an NDF containing an
image of an artificial ``patchwork'' sky mapped using a requested projection,
and an astrometry structure which describes this image. The image can be
displayed etc, using KAPPA. IRA\_DEMO2 accepts an NDF as input, locates an
astrometry structure within it and uses the information to convert user supplied
coordinates from image to sky coordinates or vice-versa.

\section {Compiling and Linking with IRA}
\label{SEC:LINK}
This section describes how to compile and link applications which use IRA
subroutines, on both VMS and UNIX systems. It is assumed that the IRAS90 package
is installed as part of the Starlink Software Collection.

\subsection{VMS}
Each terminal session which is to include the compilation or linking of
applications which use the IRA package should start by issuing the commands:

\begin{verbatim}
$ ADAMSTART
$ ADAMDEV
$ IRAS90
$ IRAS90_DEV
\end{verbatim}

These commands set up logical names related to all the IRAS90
subsystems, including IRA.

To link a VMS ADAM application with the IRA package, the linker options file
IRAS90\_LINK\_ADAM should be used. For example, to compile and link an ADAM
application called PROG with the IRA library, the following commands should be
used:

\begin{verbatim}
$ ADAMSTART
$ ADAMDEV
$ IRAS90
$ IRAS90_DEV
$ FORT PROG
$ ALINK PROG, IRAS90_LINK_ADAM/OPT
\end{verbatim}

Stand-alone (i.e. non-ADAM) applications can be linked with the ``standalone''
version of IRA. This version excludes the routines described in appendix
\ref{APP:ADAM}. To do this the link options file IRAS90\_LINK should be used
instead of IRAS90\_LINK\_ADAM. Thus to compile and link a stand-alone
application with the IRA package, the following commands should be given:

\begin{verbatim}
$ IRAS90
$ IRAS90_DEV
$ FORT PROG
$ LINK PROG, IRAS90_LINK/OPT
\end{verbatim}

\subsection{UNIX}


Each terminal session which is to include the compilation or linking of
applications which use any IRAS90 sub-package should do the following:

\begin{enumerate}
\item Execute the {\bf iras90} command. This sets up an alias for the
{\bf iras90\_dev} command.

\item Execute the {\bf iras90\_dev} command.  This creates soft links
within the current directory to
all the iras90 include files, defines an environment variable IRAS90\_SRC
pointing to the IRAS90 source directory, adds the {\bf bin}
sub-directory of IRAS90\_SRC on to the end of the current value for the
environment variable PATH, and adds the {\bf lib} sub-directory of
IRAS90\_SRC on to the end of the current value for the environment
variable LD\_LIBRARY\_PATH. The command {\bf iras90\_dev star} will do
the same but will also create soft links to all starlink include files
(i.e. all files in the directory /star/include).

\item If a UNIX application accesses any include files (such as
IRA\_PAR) then they should be specified (within the source file) in upper case
without any directory
path. The I90\_PAR file (for instance) can be included using the statement

\verb+      INCLUDE 'I90_PAR'+\\

\item Soft links can be deleted (using the {\bf rm} command) when no longer
needed.
The command {\bf iras90\_dev remove} will remove all IRAS90 soft links from
the current directory. The command {\bf iras90\_dev remove star} will
remove all IRAS90 and Starlink soft links from the current directory.
Note, soft links are only accessable in the directory from which the {bf ln}
or {\bf iras90\_dev} command was issued.
\item For ADAM applications the following {\bf alink} command should be used:

\verb+% alink prog.f -L$IRAS90_SRC/lib `ira_link_adam`+\\

where {\bf prog.f} is the fortran source file for the ATASK.
Note the use of opening apostrophies (`) instead of the more usual closing
apostrophy (') in the above {\bf alink} command.

\item For a stand-alone program the following {\bf f77} command should be used:

\verb+% f77 prog.f -o prog -L$IRAS90_SRC/lib `ira_link`+\\
\end{enumerate}


\appendix
\section {Routine Descriptions}
% Command for displaying routines in routine lists:
% =================================================

\newcommand{\noteroutine}[2]{{\small \bf #1} \\
                              \hspace*{3em} {\em #2} \\[1.5ex]}

\subsection {Routine List}
\noteroutine{IRA\_ACTIV(ACTIVE)}
   {Inquire if IRA is currently active.}
\noteroutine{IRA\_ANNUL(IDA,STATUS)}
   {Annul an IRA identifier.}
\noteroutine{IRA\_CLOSE(STATUS)}
   {Close down the IRA and TRN packages.}
\noteroutine{IRA\_CONVT(NVAL,AIN,BIN,SCSIN,SCSOUT,EPOCH,AOUT,BOUT,STATUS)}
   {Convert sky coordinates from one system to another.}
\noteroutine{IRA\_CREAT(PROJ,NP,P,SCS,EPOCH,INDF,IDA,STATUS)}
   {Create an IRA identifier for specified astrometry information.}
\noteroutine{IRA\_CTOD(ATEXT,BTEXT,SCS,A,B,STATUS)}
   {Convert a pair of sky coordinate values held as formatted strings into
   double precision, floating point values.}
\noteroutine{IRA\_CTOD1(TEXT,SCS,NC,VALUE,STATUS)}
   {Convert a single sky coordinate value held in a formatted string into a
   double precision floating point value.}
\noteroutine{IRA\_DIST(A0,B0,A1,B1,DIST,STATUS)}
   {Find the arc distance between two sky positions.}
\noteroutine{IRA\_DIST2(A0,B0,ANGLE,A1,B1,PARDST,PRPDST,STATUS)}
   {Resolve the displacement between two sky positions into distances parallel
    and perpendicular to a given position angle.}
\noteroutine{IRA\_DRBND(IDA,LBND,UBND,STATUS)}
   {Draw a curve around the region containing valid sky coordinates.}
\noteroutine{IRA\_DRBRK(MAXBRK,OUT,BREAK,VBREAK,NBREAK,LENGTH,STATUS)}
   {Return information about the breaks (etc) in a plotted curve.}
\noteroutine{IRA\_DRGRD(IDA,SCS,LBND,UBND,STATUS)}
   {Draw a sky coordinate grid.}
\noteroutine{IRA\_DRGTC(IDA,A,B,ANGLE,DIST,SCS,LBND,UBND,STATUS)}
   {Draw a section of a great circle.}
\noteroutine{IRA\_DRMER(IDA,A,B,INCB,SCS,LBND,UBND,STATUS)}
   {Draw a section of a meridian.}
\noteroutine{IRA\_DROPS(ITEM,VALUE,STATUS)}
   {Get the current value of a graphics option.}
\noteroutine{IRA\_DROPT(ITEM,VALUE,STATUS)}
   {Set a new value for a graphics option.}
\noteroutine{IRA\_DRPAR(IDA,A,B,INCA,SCS,LBND,UBND,STATUS)}
   {Draw a section of a parallel.}
\noteroutine{IRA\_DRVAL(VALUE,SCS,NC,XPOS,YPOS,STYLE,ACC,CONTXT,STATUS)}
   {Draw a formatted sky coordinate value.}
\noteroutine{IRA\_DRVPO(X1,Y1,X2,Y2,H,STATUS)}
   {Return area covered by a coordinate value drawn by IRA\_DRVAL.}
\noteroutine{IRA\_DTOC(A,B,SCS,STYLE,ATEXT,BTEXT,STATUS)}
   {Format a  pair of sky coordinate values into a pair of character strings.}
\noteroutine{IRA\_DTOC1(VALUE.SCS,NC,STYLE,TEXT,STATUS)}
   {Format a  single sky coordinate value into a character string.}
\noteroutine{IRA\_FIND(INDF,THERE,XNAME,ASNAME,LOC,STATUS)}
   {Find an astrometry structure within an NDF.}
\noteroutine{IRA\_EXPRT(IDA,INDF,STATUS)}
   {Store astrometry infromation in an NDF}
\noteroutine{IRA\_GETCO(APAR,BPAR,PRMAPP,SCS,DEFLT,A,B,STATUS)}
   {Read a pair of sky coordinates from the ADAM environment.}
\noteroutine{IRA\_GETEQ(SCS,EQU,BJ,NAME,STATUS)}
   {Get the epoch of the reference equinox, and the name, from an SCS string.}
\noteroutine{IRA\_GTCO1(PARAM,PROMPT,SCS,NC,DEFLT,VALUE,STATUS)}
   {Read a single sky coordinate from the ADAM environment.}
\noteroutine{IRA\_GTSCS(PARAM,DEFLT,SCS,STATUS )}
   {Get a complete SCS value (including equinox specifier) from the ADAM
    environment.}
\noteroutine{IRA\_IMPRT(INDF,IDA,STATUS)}
   {Import astrometry information from an NDF into the IRA system.}
\noteroutine{IRA\_INIT(STATUS)}
   {Initialise the IRA identifier system.}
\noteroutine{IRA\_IPROJ(LIST,STATUS)}
   {Return a list containing the names of explicitly supported projections.}
\noteroutine{IRA\_ISCS(LIST,STATUS)}
   {Return a list containing the names of supported sky coordinate systems.}
\noteroutine{IRA\_LOCAT(XNAME,ASNAME,STATUS)}
   {Set the names of the HDS object and NDF extension used to store the
   astrometry information.}
\noteroutine{IRA\_MAG(IDA,MAGX,MAGY,XC,YC,STATUS)}
   {Modify astrometry information to take account of a magnification}
\noteroutine{IRA\_MOVE(IDA,SHIFTX,SHIFTY,STATUS)}
   {Modify astrometry information to take account of a shift of image
coordinates}
\noteroutine{IRA\_NORM(A,B,STATUS)}
   {Shift a pair of sky coordinate values into the ``first order ranges'' (e.g.
   longitude between zero and $2.\pi$ and latitude between $+\pi/2$ and
   $-\pi/2$ ).}
\noteroutine{IRA\_OFFST(A0,B0,A1,B1,DIST,A2,B2,STATUS)}
   {Find a sky position which is offset from a given position by a given
   amount in the direction of another given position.}
\noteroutine{IRA\_PACON(NVAL,AIN,BIN,PAIN,SCSIN,SCSOUT,EPOCH,AOUT,BOUT,PAOUT,STATUS)}
   {Convert sky coordinates and position angles from one system to another.}
\noteroutine{IRA\_PIXSZ(IDA,PIXSZ,STATUS)}
   {Get the nominal pixel dimensions.}
\noteroutine{IRA\_READ(LOC,IDA,STATUS)}
   {Read astrometry information from an HDS object.}
\noteroutine{IRA\_ROT(IDA,ROT,XC,YC,STATUS)}
   {Modify astrometry information to take account of a rotation}
\noteroutine{IRA\_SCNAM(SCS,NC,DESCR,LD,ABBREV,LA,STATUS)}
   {Return the name and abbreviation for an axis of a sky coordinate system.}
\noteroutine{IRA\_SCSEP(IDA,SCS,EPOCH,STATUS)}
   {Return the sky coordinate system and epoch associated with an IRA
    identifier.}
\noteroutine{IRA\_SETEQ(EQU,BJ,SCS,STATUS)}
   {Set the epoch of the reference equinox in an SCS string.}
\noteroutine{IRA\_SHIFT(A0,B0,ANGLE,DIST,A1,B1,ENDANG,STATUS)}
   {Find a sky position which is offset from a given position by a given
   amount along a given  bearing.}
\noteroutine{IRA\_TRACE(IDA,ROUTNE,STATUS )}
   {Display astrometry information.}
\noteroutine{IRA\_TRANS(NVAL,IN1,IN2,FORWRD,SCS,IDA,OUT1,OUT2,STATUS)}
   {Transform coordinate data.}
\noteroutine{IRA\_VALID(NVAL,FORWRD,SCS,IDA,IN1,IN2,OK,STATUS)}
   {Deterime if a given sky or image position corresponds to a valid position
    in the other system.}
\noteroutine{IRA\_WRITE(IDA,LOC,STATUS)}
   {Write astrometry information to an HDS object.}
\noteroutine{IRA\_XYLIM(IDA,ACEN,BCEN,XSIZE,YSIZE,LBND,UBND,STATUS)}
   {Determine the pixel bounds required for an image to cover a given area of
    sky.}

\section {Classified List of Routines}

\subsection {Access to Existing Astrometry Information}
\noteroutine{IRA\_FIND(INDF,THERE,XNAME,ASNAME,LOC,STATUS)}
   {Find an astrometry structure within an NDF.}
\noteroutine{IRA\_IMPRT(INDF,IDA,STATUS)}
   {Import astrometry information from an NDF into the IRA system.}
\noteroutine{IRA\_LOCAT(XNAME,ASNAME,STATUS)}
   {Set the names of the HDS object and NDF extension used to store the
   astrometry information.}
\noteroutine{IRA\_MAG(IDA,MAGX,MAGY,XC,YC,STATUS)}
   {Modify astrometry information to take account of a magnification}
\noteroutine{IRA\_MOVE(IDA,SHIFTX,SHIFTY,STATUS)}
   {Modify astrometry information to take account of a shift of image
coordinates}
\noteroutine{IRA\_READ(LOC,IDA,STATUS)}
   {Read astrometry information from an HDS object.}
\noteroutine{IRA\_ROT(IDA,ROT,XC,YC,STATUS)}
   {Modify astrometry information to take account of a rotation}
\noteroutine{IRA\_SCSEP(IDA,SCS,EPOCH,STATUS)}
   {Return the sky coordinate system and epoch associated with an IRA
    identifier.}
\noteroutine{IRA\_TRACE(IDA,ROUTNE,STATUS )}
   {Display astrometry information.}
\noteroutine{IRA\_TRANS(NVAL,IN1,IN2,FORWRD,SCS,IDA,OUT1,OUT2,STATUS)}
   {Transform coordinate data.}
\noteroutine{IRA\_VALID(NVAL,FORWRD,SCS,IDA,IN1,IN2,OK,STATUS)}
   {Deterime if a given sky or image position corresponds to a valid position
    in the other system.}
\noteroutine{IRA\_XYLIM(IDA,ACEN,BCEN,XSIZE,YSIZE,LBND,UBND,STATUS)}
   {Determine the pixel bounds required for an image to cover a given area of
    sky.}

\subsection {Accessing ADAM Parameters}
\label {APP:ADAM}
\noteroutine{IRA\_GETCO(APAR,BPAR,PRMAPP,SCS,DEFLT,A,B,STATUS)}
   {Read a pair of sky coordinates from the ADAM environment.}
\noteroutine{IRA\_GTCO1(PARAM,PROMPT,SCS,NC,DEFLT,VALUE,STATUS)}
   {Read a single sky coordinate from the ADAM environment.}
\noteroutine{IRA\_GTSCS(PARAM,DEFLT,SCS,STATUS )}
   {Get a complete SCS value (including equinox specifier) from the ADAM
    environment.}

\subsection {Control of IRA Identifier}
\noteroutine{IRA\_ANNUL(IDA,STATUS)}
   {Annul an IRA identifier.}
\noteroutine{IRA\_CLOSE(STATUS)}
   {Close down the IRA and TRN packages.}
\noteroutine{IRA\_CREAT(PROJ,NP,P,SCS,EPOCH,INDF,IDA,STATUS)}
   {Create an IRA identifier for specified astrometry information.}
\noteroutine{IRA\_EXPRT(IDA,INDF,STATUS)}
   {Store astrometry infromation in an NDF}
\noteroutine{IRA\_IMPRT(INDF,IDA,STATUS)}
   {Import astrometry information from an NDF into the IRA system.}
\noteroutine{IRA\_INIT(STATUS)}
   {Initialise the IRA identifier system.}
\noteroutine{IRA\_READ(LOC,IDA,STATUS)}
   {Read astrometry information from an HDS object.}
\noteroutine{IRA\_WRITE(IDA,LOC,STATUS)}
   {Write astrometry information to an HDS object.}

\subsection {Control of the IRA Package}
\noteroutine{IRA\_ACTIV(ACTIVE)}
   {Inquire if IRA is currently active.}
\noteroutine{IRA\_CLOSE(STATUS)}
   {Close down the IRA and TRN packages.}
\noteroutine{IRA\_INIT(STATUS)}
   {Initialise the IRA identifier system.}
\noteroutine{IRA\_LOCAT(XNAME,ASNAME,STATUS)}
   {Set the names of the HDS object and NDF extension used to store the
   astrometry information.}

\subsection {Creation of Astrometry Information}
\noteroutine{IRA\_CREAT(PROJ,NP,P,SCS,EPOCH,INDF,IDA,STATUS)}
   {Create an IRA identifier for specified astrometry information.}
\noteroutine{IRA\_WRITE(IDA,LOC,STATUS)}
   {Write astrometry information to an HDS object.}

\subsection {Handling Formatted Sky Coordinate Values}
\noteroutine{IRA\_CTOD(ATEXT,BTEXT,SCS,A,B,STATUS)}
   {Convert a pair of sky coordinate values held as formatted strings into
   double precision, floating point values.}
\noteroutine{IRA\_CTOD1(TEXT,SCS,NC,VALUE,STATUS)}
   {Convert a single sky coordinate value held in a formatted string into a
   double precision floating point value.}
\noteroutine{IRA\_DRVAL(VALUE,SCS,NC,XPOS,YPOS,STYLE,ACC,CONTXT,STATUS)}
   {Draw a formatted sky coordinate value.}
\noteroutine{IRA\_DRVPO(X1,Y1,X2,Y2,H,STATUS)}
   {Return area covered by a coordinate value drawn by IRA\_DRVAL.}
\noteroutine{IRA\_DTOC(A,B,SCS,STYLE,ATEXT,BTEXT,STATUS)}
   {Format a  pair of sky coordinate values into a pair of character strings.}
\noteroutine{IRA\_DTOC1(VALUE.SCS,NC,STYLE,TEXT,STATUS)}
   {Format a  single sky coordinate value into a character string.}
\noteroutine{IRA\_GETCO(APAR,BPAR,PRMAPP,SCS,DEFLT,A,B,STATUS)}
   {Read a pair of sky coordinates from the ADAM environment.}
\noteroutine{IRA\_GTCO1(PARAM,PROMPT,SCS,NC,DEFLT,VALUE,STATUS)}
   {Read a single sky coordinate from the ADAM environment.}

\subsection {Handling Named Projections}
\noteroutine{IRA\_CREAT(PROJ,NP,P,SCS,EPOCH,INDF,IDA,STATUS)}
   {Create an IRA identifier for specified astrometry information.}
\noteroutine{IRA\_IPROJ(LIST,STATUS)}
   {Return a list containing the names of explicitly supported projections.}

\subsection {Graphics}
\noteroutine{IRA\_DRBND(IDA,LBND,UBND,STATUS)}
   {Draw a curve around the region containing valid sky coordinates.}
\noteroutine{IRA\_DRBRK(MAXBRK,OUT,BREAK,VBREAK,NBREAK,LENGTH,STATUS)}
   {Return information about the breaks (etc) in a plotted curve.}
\noteroutine{IRA\_DRGRD(IDA,SCS,LBND,UBND,STATUS)}
   {Draw a sky coordinate grid.}
\noteroutine{IRA\_DRGTC(IDA,A,B,ANGLE,DIST,SCS,LBND,UBND,STATUS)}
   {Draw a section of a great circle.}
\noteroutine{IRA\_DRMER(IDA,A,B,INCB,SCS,LBND,UBND,STATUS)}
   {Draw a section of a meridian.}
\noteroutine{IRA\_DROPS(ITEM,VALUE,STATUS)}
   {Get the current value of a graphics option.}
\noteroutine{IRA\_DROPT(ITEM,VALUE,STATUS)}
   {Set a new value for a graphics option.}
\noteroutine{IRA\_DRPAR(IDA,A,B,INCA,SCS,LBND,UBND,STATUS)}
   {Draw a section of a parallel.}
\noteroutine{IRA\_DRVAL(VALUE,SCS,NC,XPOS,YPOS,STYLE,ACC,CONTXT,STATUS)}
   {Draw a formatted sky coordinate value.}
\noteroutine{IRA\_DRVPO(X1,Y1,X2,Y2,H,STATUS)}
   {Return area covered by a coordinate value drawn by IRA\_DRVAL.}

\subsection {Sky Coordinate Conversions}
\noteroutine{IRA\_CONVT(NVAL,AIN,BIN,SCSIN,SCSOUT,EPOCH,AOUT,BOUT,STATUS)}
   {Convert sky coordinates from one system to another.}
\noteroutine{IRA\_PACON(NVAL,AIN,BIN,PAIN,SCSIN,SCSOUT,EPOCH,AOUT,BOUT,PAOUT,STATUS)}
   {Convert sky coordinates and position angles from one system to another.}

\subsection {Specification of Sky Coordinate Systems}
\noteroutine{IRA\_GETEQ(SCS,EQU,BJ,NAME,STATUS)}
   {Get the epoch of the reference equinox, and the name, from an SCS string.}
\noteroutine{IRA\_GTSCS(PARAM,DEFLT,SCS,STATUS )}
   {Get a complete SCS value (including equinox specifier) from the ADAM
    environment.}
\noteroutine{IRA\_ISCS(LIST,STATUS)}
   {Return a list containing the names of supported sky coordinate systems.}
\noteroutine{IRA\_SCNAM(SCS,NC,DESCR,LD,ABBREV,LA,STATUS)}
   {Return the name and abbreviation for an axis of a sky coordinate system.}
\noteroutine{IRA\_SCSEP(IDA,SCS,EPOCH,STATUS)}
   {Return the sky coordinate system and epoch associated with an IRA
    identifier.}
\noteroutine{IRA\_SETEQ(EQU,BJ,SCS,STATUS)}
   {Set the epoch of the reference equinox in an SCS string.}

\subsection {Spherical Trigonometry}
\noteroutine{IRA\_DIST(A0,B0,A1,B1,DIST,STATUS)}
   {Find the arc distance between two sky positions.}
\noteroutine{IRA\_DIST2(A0,B0,ANGLE,A1,B1,PARDST,PRPDST,STATUS)}
   {Resolve the displacement between two sky positions into distances parallel
    and perpendicular to a given position angle.}
\noteroutine{IRA\_NORM(A,B,STATUS)}
   {Shift a pair of sky coordinate values into the ``first order ranges'' (e.g.
   longitude between zero and $2.\pi$ and latitude between $+\pi/2$ and
   $-\pi/2$ ).}
\noteroutine{IRA\_OFFST(A0,B0,A1,B1,DIST,A2,B2,STATUS)}
   {Find a sky position which is offset from a given position by a given
   amount in the direction of another given position.}
\noteroutine{IRA\_SHIFT(A0,B0,ANGLE,DIST,A1,B1,ENDANG,STATUS)}
   {Find a sky position which is offset from a given position by a given
   amount along a given  bearing.}

\subsection {Transformation Between Image and Sky Coordinates}
\noteroutine{IRA\_TRANS(NVAL,IN1,IN2,FORWRD,SCS,IDA,OUT1,OUT2,STATUS)}
   {Transform coordinate data.}
\noteroutine{IRA\_VALID(NVAL,FORWRD,SCS,IDA,IN1,IN2,OK,STATUS)}
   {Deterime if a given sky or image position corresponds to a valid position
    in the other system.}
\noteroutine{IRA\_XYLIM(IDA,ACEN,BCEN,XSIZE,YSIZE,LBND,UBND,STATUS)}
   {Determine the pixel bounds required for an image to cover a given area of
    sky.}

\section {Full Routine Specifications}
\label {SEC:FULLSPEC}

\renewcommand{\_}{{\tt\char'137}}

%+
%  Name:
%     LAYOUT.TEX

%  Purpose:
%     Define Latex commands for laying out documentation produced by PROLAT.

%  Language:
%     Latex

%  Type of Module:
%     Data file for use by the PROLAT application.

%  Description:
%     This file defines Latex commands which allow routine documentation
%     produced by the SST application PROLAT to be processed by Latex. The
%     contents of this file should be included in the source presented to
%     Latex in front of any output from PROLAT. By default, this is done
%     automatically by PROLAT.

%  Notes:
%     The definitions in this file should be included explicitly in any file
%     which requires them. The \include directive should not be used, as it
%     may not then be possible to process the resulting document with Latex
%     at a later date if changes to this definitions file become necessary.

%  Authors:
%     RFWS: R.F. Warren-Smith (STARLINK)

%  History:
%     10-SEP-1990 (RFWS):
%        Original version.
%     10-SEP-1990 (RFWS):
%        Added the implementation status section.
%     12-SEP-1990 (RFWS):
%        Added support for the usage section and adjusted various spacings.
%     10-DEC-1991 (RFWS):
%        Refer to font files in lower case for UNIX compatibility.
%     {enter_further_changes_here}

%  Bugs:
%     {note_any_bugs_here}

%-

%  Define length variables.
\newlength{\sstbannerlength}
\newlength{\sstcaptionlength}

%  Define a \tt font of the required size.
\font\ssttt=cmtt10 scaled 1095

%  Define a command to produce a routine header, including its name,
%  a purpose description and the rest of the routine's documentation.
\newcommand{\sstroutine}[3]{
   \goodbreak
   \rule{\textwidth}{0.5mm}
   \vspace{-7ex}
   \newline
   \settowidth{\sstbannerlength}{{\Large {\bf #1}}}
   \setlength{\sstcaptionlength}{\textwidth}
   \addtolength{\sstbannerlength}{0.5em}
   \addtolength{\sstcaptionlength}{-2.0\sstbannerlength}
   \addtolength{\sstcaptionlength}{-4.45pt}
   \parbox[t]{\sstbannerlength}{\flushleft{\Large {\bf #1}}}
   \parbox[t]{\sstcaptionlength}{\center{\Large #2}}
   \parbox[t]{\sstbannerlength}{\flushright{\Large {\bf #1}}}
   \begin{description}
      #3
   \end{description}
}

%  Format the description section.
\newcommand{\sstdescription}[1]{\item[Description:] #1}

%  Format the usage section.
\newcommand{\sstusage}[1]{\item[Usage:] \mbox{} \\[1.3ex] {\ssttt #1}}

%  Format the invocation section.
\newcommand{\sstinvocation}[1]{\item[Invocation:]\hspace{0.4em}{\tt #1}}

%  Format the arguments section.
\newcommand{\sstarguments}[1]{
   \item[Arguments:] \mbox{} \\
   \vspace{-3.5ex}
   \begin{description}
      #1
   \end{description}
}

%  Format the returned value section (for a function).
\newcommand{\sstreturnedvalue}[1]{
   \item[Returned Value:] \mbox{} \\
   \vspace{-3.5ex}
   \begin{description}
      #1
   \end{description}
}

%  Format the parameters section (for an application).
\newcommand{\sstparameters}[1]{
   \item[Parameters:] \mbox{} \\
   \vspace{-3.5ex}
   \begin{description}
      #1
   \end{description}
}

%  Format the examples section.
\newcommand{\sstexamples}[1]{
   \item[Examples:] \mbox{} \\
   \vspace{-3.5ex}
   \begin{description}
      #1
   \end{description}
}

%  Define the format of a subsection in a normal section.
\newcommand{\sstsubsection}[1]{\item[{#1}] \mbox{} \\}

%  Define the format of a subsection in the examples section.
\newcommand{\sstexamplesubsection}[1]{\item[{\ssttt #1}] \mbox{} \\}

%  Format the notes section.
\newcommand{\sstnotes}[1]{\item[Notes:] \mbox{} \\[1.3ex] #1}

%  Provide a general-purpose format for additional (DIY) sections.
\newcommand{\sstdiytopic}[2]{\item[{\hspace{-0.35em}#1\hspace{-0.35em}:}] \mbox{} \\[1.3ex] #2}

%  Format the implementation status section.
\newcommand{\sstimplementationstatus}[1]{
   \item[{Implementation Status:}] \mbox{} \\[1.3ex] #1}

%  Format the bugs section.
\newcommand{\sstbugs}[1]{\item[Bugs:] #1}

%  Format a list of items while in paragraph mode.
\newcommand{\sstitemlist}[1]{
  \mbox{} \\
  \vspace{-3.5ex}
  \begin{itemize}
     #1
  \end{itemize}
}

%  Define the format of an item.
\newcommand{\sstitem}{\item}

%  End of LAYOUT.TEX layout definitions.
%.

\sstroutine{
   IRA\_ACTIV
}{
   Determine if the IRA astrometry package is currently active
}{
   \sstdescription{
      This routine returns a true value if IRA is currently active
      (i.e. if IRA\_INIT has been called), and a false value otherwise.
   }
   \sstinvocation{
      CALL IRA\_ACTIV( ACTIVE )
   }
   \sstarguments{
      \sstsubsection{
         ACTIVE = LOGICAL (Returned)
      }{
         The current status of the IRA package.
      }
   }
}
\sstroutine{
   IRA\_ANNUL
}{
   Annul an IRA identifier
}{
   \sstdescription{
      This routine should be called when access to the astrometry
      information associated with an IRA identifier is no longer
      needed. It releases the resources used to store the astrometry
      information.

      This routine attempts to execute even if STATUS is bad on entry.
      However, in this case no error report will be produced if this
      routine subsequently fails.
   }
   \sstinvocation{
      CALL IRA\_ANNUL( IDA, STATUS )
   }
   \sstarguments{
      \sstsubsection{
         IDA = INTEGER ( Given )
      }{
         The IRA identifier to be annulled.
      }
      \sstsubsection{
         STATUS = INTEGER (Given and Returned)
      }{
         The global status.
      }
   }
}
\sstroutine{
   IRA\_CLOSE
}{
   Close down the IRA astrometry package
}{
   \sstdescription{
      This routine should be called once IRA facilities are no longer
      needed. It annulls any currently valid IRA identifiers, releasing
      any resources reserved by them. Once this routine has been called,
      any further use of IRA must be preceeded with a call to IRA\_INIT.

      This routine attempts to execute even if STATUS is set to a bad
      value on entry.
   }
   \sstinvocation{
      CALL IRA\_CLOSE( STATUS )
   }
   \sstarguments{
      \sstsubsection{
         STATUS = INTEGER (Given and Returned)
      }{
         The global status.
      }
   }
}
\sstroutine{
   IRA\_CONVT
}{
   Convert sky coordinates from one system to another
}{
   \sstdescription{
      This routine use SLALIB to convert a list of sky coordinates from
      one supported Sky Coordinate System (SCS) to any other supported
      system.  It is assumed that the observations were made at the
      date given by the Julian epoch supplied.  If the input and output
      coordinates are referred to different mean equinox, then
      precession is applied to convert the input coordinates to the
      output system.  No correction for nutation is included. If any of
      the input coordinate values are equal to the Starlink {\tt "}BAD{\tt "} value
      (VAL\_\_BADD) then the corresponding output values will both be set
      to the bad value.
   }
   \sstinvocation{
      CALL IRA\_CONVT( NVAL, AIN, BIN, SCSIN, SCSOUT, EPOCH, AOUT, BOUT,
                      STATUS )
   }
   \sstarguments{
      \sstsubsection{
         NVAL = INTEGER (Given)
      }{
         The number of sky coordinate pairs to be converted.
      }
      \sstsubsection{
         AIN( NVAL ) = DOUBLE PRECISION (Given)
      }{
         A list of sky longitude coordinate values to be converted, in
         radians.
      }
      \sstsubsection{
         BIN( NVAL ) = DOUBLE PRECISION (Given)
      }{
         A list of sky latitude coordinate values to be converted, in
         radians.
      }
      \sstsubsection{
         SCSIN = CHARACTER $*$ ( $*$ ) (Given)
      }{
         A string holding the name of the sky coordinate system of the
         input list. Any unambiguous abbreviation will do. An optional
         equinox specifier may be included in the name (see ID2 section
         {\tt "}Sky Coordinates{\tt "}).
      }
      \sstsubsection{
         SCSOUT = CHARACTER $*$ ( $*$ ) (Given)
      }{
         A string holding the name of the sky coordinate system
         required for the output list. Any unambiguous abbreviation
         will do.An optional equinox specifier may be included in the
         name.
      }
      \sstsubsection{
         EPOCH = DOUBLE PRECISION (Given)
      }{
         The Julian epoch at which the observations were made. When
         dealing with IRAS data, the global constant IRA\_\_IRJEP should
         be specified. This constant is a Julian epoch suitable for all
         IRAS data.
      }
      \sstsubsection{
         AOUT( NVAL ) = DOUBLE PRECISION (Returned)
      }{
         The list of converted sky longitude coordinate values, in
         radians.
      }
      \sstsubsection{
         BOUT( NVAL ) = DOUBLE PRECISION (Returned)
      }{
         The list of converted sky latitude coordinate values, in
         radians.
      }
      \sstsubsection{
         STATUS = INTEGER (Given and Returned)
      }{
         The global status.
      }
   }
}
\sstroutine{
   IRA\_CREAT
}{
   Create an identifier for specified astrometry information
}{
   \sstdescription{
      The supplied astrometry information is stored in internal common
      blocks and an {\tt "}IRA identifier{\tt "} is returned which can be passed to
      other IRA routines to refer to the stored astrometry information.
      This identifier should be annulled when it is no longer required
      by calling IRA\_ANNUL. In addition, a call to IRA\_EXPRT may
      optionally be made to store the astrometry information in an NDF
      (see argument INDF).

      The projection used is specified by the argument PROJ, and must
      be one of the supported projection types (see routine IRA\_IPROJ).
      For more information on the available projections, see the ID/2
      appendix {\tt "}Projection Equations{\tt "}. Each projection requires the
      values for several parameters to be supplied in argument P. These
      parameters have the same meaning for all projections (for further
      details see ID/2 appendix {\tt "}Projection Equations{\tt "}):

      P(1): The longitude (in the Sky Coordinate System specified by
            argument SCS) of the reference point in radians.

      P(2): The latitude (in the Sky Coordinate System specified by
            argument SCS) of the reference point in radians.

      P(3): The first image coordinate (i.e. X value) of the reference
            point. Image coordinates are fractional values in which the
            centre of the pixel (1,1) has coordinates (0.5,0.5).

      P(4): The second image coordinate (i.e. Y value) of the
            reference point.

      P(5): The size along the X image axis, of a pixel centred at the
            reference point, in radians. Actual pixel size will vary
            over the image due to the distorting effect of the
            projection. The absolute value is used.

      P(6): The size along the Y image axis, of a pixel centred at the
            reference point, in radians. The absolute value is used.

      P(7): The position angle of the Y image axis, in radians. That is,
            the angle from north to the positive direction of the Y
            image axis, measured positive in the same sense as
            rotation from north to east. (Here {\tt "}north{\tt "} and {\tt "}east{\tt "} are
            defined by the value of SCS). The X image axis is 90 degrees
            west of the Y axis.

      P(8): An angle through which the celestial sphere is to be rotated
            before doing the projection. The axis of the rotation is a
            radius passing through the reference point. The rotation is
            in an anti-clockwise sense when looking from the reference
            point towards the centre of the celestial sphere. The value
            should be in radians. Changing this angle does not change
            the orientation of the image axes with respect to north
            (which is set by p(7)).
   }
   \sstinvocation{
      CALL IRA\_CREAT( PROJ, NP, P, SCS, EPOCH, INDF, IDA, STATUS )
   }
   \sstarguments{
      \sstsubsection{
         PROJ = CHARACTER $*$ ( $*$ ) (Given)
      }{
         The projection type (see routine IRA\_IPROJ for a list of
         currently recognised values). Any unambiguous abbreviation can
         be given.
      }
      \sstsubsection{
         NP = INTEGER (Given)
      }{
         The size of array P.
      }
      \sstsubsection{
         P( NP ) = DOUBLE PRECISION (Given)
      }{
         The parameter values required by the projection.
      }
      \sstsubsection{
         SCS = CHARACTER $*$ ( $*$ ) (Given)
      }{
         The name of the Sky Coordinate System which the projection is
         to create, or an unambiguous abbreviation. See routine IRA\_ISCS
         for a list of currently recognised values.  See ID2 section
         {\tt "}Sky Coordinates{\tt "}) for general information of Sky Coordinate
         Systems.
      }
      \sstsubsection{
         EPOCH = DOUBLE PRECISION (Given)
      }{
         The Julian epoch at which the observations were made. A single
         mean epoch is sufficient to describe all IRAS observations.
         Such a value is contained in the IRA constant IRA\_\_IRJEP.
      }
      \sstsubsection{
         INDF = INTEGER (Given)
      }{
         The identifier for the NDF in which the astrometry information
         is to be stored. If an invalid NDF identifier is given (eg the
         value NDF\_\_NOID) then the astrometry information is not stored
         in an NDF (but IDA can still be used to refer to the astrometry
         information).
      }
      \sstsubsection{
         IDA = INTEGER (Returned)
      }{
         The returned IRA identifier.
      }
      \sstsubsection{
         STATUS = INTEGER (Given and Returned)
      }{
         The global status.
      }
   }
}
\sstroutine{
   IRA\_CTOD
}{
   Converts formatted sky coordinate values into double precision
   values
}{
   \sstdescription{
      The input strings are presumed to hold sky coordinate values in
      character form.  This routine reads the strings and produces
      double precision values holding the coordinate values, in
      radians.

      Each input string can consist of a set of up to three {\tt "}fields{\tt "}.
      Each field starts with a numeric value (which can have a
      fractional part) terminated by a character string. This character
      string consists of an optional single character, called the
      terminator character, followed by an arbitrary number of spaces
      The terminator character (if present) must be one of the letters
      h,d,m,s or r.  An h terminator indicates that the field value is
      in units of hours. A d terminator indicates that the field value
      is in units of degrees.  An r terminator indicates that the field
      value is in units of radians.  An m terminator indicates that the
      field value is in units of minutes.  An s terminator indicates
      that the field value is in units of seconds.  The interpretation
      of minutes and seconds depends on whether the value is a time
      value or an angle value. The longitude value for equatorial sky
      coordinate systems (RA) is expected to be a measure of time, all
      other coordinate values are expected to be a measure of angle.
      These defaults are overriden if the first field is a {\tt "}degrees{\tt "},
      {\tt "}hours{\tt "} or {\tt "}radians{\tt "} field (as indicated by the presence of a d,
      h or r terminator character).  The interpretation of fields with
      no terminator character depends on which field is being
      considered. If the first field has no terminator, it is assumed
      to be either a degrees or hours field.  If the second or third
      field has no terminator, it is assumed to be a seconds field if
      the previous field was a minutes field, and a minutes field if
      the previous value was an hours or degrees field.

      In addition, an input string may contain a single field with no
      terminator character in an {\tt "}encoded{\tt "} form. {\tt "}Encoded{\tt "} fields are
      identified by the fact that the field contains 5 or more digits
      to the left of the decimal point (including leading zeros if
      necessary).  These fields are decoded into hours or degrees as
      follows: Any fractional part is taken as the fractional part of
      the seconds field, the tens and units digits are taken as the
      integer part of the seconds field, the hundreds and thousands
      digits are taken as the minutes fields, the remaining digits are
      taken as the degrees or hours field. Thus -12345.4 would be
      interpreted as (- 1 hour 23 mins 45.4 seconds) or (- 1 degree 23
      mins 45.4 seconds). The same value could also be specified as
      \sstitemlist{

         \sstitem
         1 23 45.5, -1h 23m 45.5s (if it represents a time value), or

         \sstitem
         1d 23 45.5 (if it represents an angular value).

      }
      The supplied values must be in their first order ranges (i.e. 0
      to 2.PI for longitude values and -PI/2 to $+$PI/2 for latitude
      values). Values outside these ranges cause an error to be
      reported, and the status value IRA\_\_RANGE is returned). The
      exception to this is if the string is prefixed with a {\tt "}$*${\tt "}
      character, in which case any numeric value may be supplied. In
      this case the supplied value is returned directly (eg if the
      string {\tt "}$*$400D{\tt "} is given, the radian equivalent of 400 degrees
      will be returned, not 40 (=400-360) degrees).  If either of the
      input strings are blank the corresponding output value is set to
      the Starlink {\tt "}BAD{\tt "} value (VAL\_\_BADD).
   }
   \sstinvocation{
      CALL IRA\_CTOD( ATEXT, BTEXT, SCS, A, B, STATUS )
   }
   \sstarguments{
      \sstsubsection{
         ATEXT = CHARACTER $*$ ( $*$ ) (Given)
      }{
         The string containing the formatted version of the longitude
         value. If this string is blank, A is returned with the {\tt "}BAD{\tt "}
         value (VAL\_\_BADD), but no error is reported.
      }
      \sstsubsection{
         BTEXT = CHARACTER $*$ ( $*$ ) (Given)
      }{
         The string containing the formatted version of the latitude
         value. If this string is blank, B is returned with the {\tt "}BAD{\tt "}
         value (VAL\_\_BADD), but no error is reported.
      }
      \sstsubsection{
         SCS = CHARACTER $*$ ( $*$ ) (Given)
      }{
         The sky coordinate system (see ID2 section {\tt "}Sky Coordinates{\tt "}).
         Any unambiguous abbreviation will do.
      }
      \sstsubsection{
         A = DOUBLE PRECISION (Returned)
      }{
         The longitude value represented by the string ATEXT, in
         radians.
      }
      \sstsubsection{
         B = DOUBLE PRECISION (Returned)
      }{
         The latitude value represented by the string BTEXT, in
         radians.
      }
      \sstsubsection{
         STATUS = INTEGER (Given and Returned)
      }{
         The global status.
      }
   }
}
\sstroutine{
   IRA\_CTOD1
}{
   Converts a single formatted sky coordinate value into a double
   precision value
}{
   \sstdescription{
      The input string is presumed to hold a sky coordinate value in
      character form. If NC is 1, the string is interpreted as a
      longitude value. If NC is 2, the string is interpreted as a
      latitude value.  This routine reads the string and produces a
      double precision value holding the coordinate value in radians.
      The value is not shifted into the first order range (eg if an
      angular value equivalent to 3$*$PI is given, the value 3$*$PI will be
      returned, not 1$*$PI). If the input string is blank the output
      value is set to the Starlink {\tt "}BAD{\tt "} value (VAL\_\_BADD). Refer to
      IRA\_CTOD for details of the allowed format for the input string.
   }
   \sstinvocation{
      CALL IRA\_CTOD1( TEXT, SCS, NC, VALUE, STATUS )
   }
   \sstarguments{
      \sstsubsection{
         TEXT = CHARACTER $*$ ( $*$ ) (Given)
      }{
         The string containing the formatted version of the sky
         coordinate value. If this string is blank, VALUE is returned
         with the {\tt "}BAD{\tt "} value (VAL\_\_BADD), but no error report is
         made.
      }
      \sstsubsection{
         SCS = CHARACTER $*$ ( $*$ ) (Given)
      }{
         The sky coordinate system (see ID2 section {\tt "}Sky Coordinates{\tt "}).
         Any unambiguous abbreviation will do.
      }
      \sstsubsection{
         NC = INTEGER (Given)
      }{
         Determines which sky coordinate is to be used. If a value of 1
         is supplied, the string is interpreted as a longitude value
         (eg RA if an equatorial system is being used). If a value of 2
         is supplied, the string is interpreted as a latitude value.
         Any other value results in an error being reported.
      }
      \sstsubsection{
         VALUE = DOUBLE PRECISION (Returned)
      }{
         The numerical value of the sky coordinate represented by the
         string in TEXT. The value is in radians.
      }
      \sstsubsection{
         STATUS = INTEGER (Given and Returned)
      }{
         The global status.
      }
   }
}
\sstroutine{
   IRA\_DIST
}{
   Find the arc distance between two sky positions
}{
   \sstdescription{
      This routine returns the length of the arc joining the two given
      sky positions. If any of the input coordinate values are equal to
      the Starlink {\tt "}BAD{\tt "} value (VAL\_\_BADD) then the returned distance
      will also be equal to the BAD value.
   }
   \sstinvocation{
      CALL IRA\_DIST( A0, B0, A1, B1, DIST, STATUS )
   }
   \sstarguments{
      \sstsubsection{
         A0 = DOUBLE PRECISION (Given)
      }{
         The sky longitude of the first position, in radians.
      }
      \sstsubsection{
         B0 = DOUBLE PRECISION (Given)
      }{
         The sky latitude of the first position, in radians.
      }
      \sstsubsection{
         A1 = DOUBLE PRECISION (Given)
      }{
         The sky longitude of the second position, in radians.
      }
      \sstsubsection{
         B1 = DOUBLE PRECISION (Given)
      }{
         The sky latitude of the second position, in radians.
      }
      \sstsubsection{
         DIST = DOUBLE PRECISION (Returned)
      }{
         The length of the arc joining the two positions, in
         radians. This is always in the range 0 to $+$PI.
      }
      \sstsubsection{
         STATUS = INTEGER (Given and Returned)
      }{
         The global status.
      }
   }
}
\sstroutine{
   IRA\_DIST2
}{
   Resolve a sky position into distances parallel and perpendicular
   to a given great circle
}{
   \sstdescription{
      The great circle used is the great circle which passes through
      sky position (A0,B0) (the {\tt "}reference point{\tt "}) at the position
      angle given by ANGLE.  The two returned distances PARDST and
      PRPDST are components of the displacement from (A0,B0) to
      (A1,B1), both in radians. PARDST ({\tt "}parallel distance{\tt "}) is the
      distance which must be moved from (A0,B0) along the great circle
      (in the direction specified by the position angle, ANGLE) to
      reach the point of closest approach to (A1,B1). PRPDST
      ({\tt "}perpendicular distance{\tt "}) is then the distance from this point
      of closest approach, to (A1,B1). This second displacement will be
      an arc of a great circle which crosses the first great circle at
      right angles at the point of closest approach. Note, performing
      the shifts in the opposite order (perpendicular then parallel) is
      NOT the same. The shifts should always be thought of as being
      FIRST along the great circle specified by the given angle, AND
      THEN perpendicular to the great circle. This is because the curve
      with constant PRPDST (for varying PARDST) is not a great circle
      unless PRPDST is zero.

      PRPDST is positive if rotation from the given position angle to
      the outlying point (as seen from the reference point) is in the
      same sense as rotation from north to east.
   }
   \sstinvocation{
      CALL IRA\_DIST2( A0, B0, ANGLE, A1, B1, PARDST, PRPDST,
                      STATUS )
   }
   \sstarguments{
      \sstsubsection{
         A0 = DOUBLE PRECISION (Given)
      }{
         The sky longitude of the reference point, in radians.
      }
      \sstsubsection{
         B0 = DOUBLE PRECISION (Given)
      }{
         The sky latitude of the reference point, in radians.
      }
      \sstsubsection{
         ANGLE = DOUBLE PRECISION (Given)
      }{
         The position angle of the great circle as seen from the
         reference point. That is, the angle from north to the required
         direction, in radians. Positive angles are in the sense of
         rotation from north to east.
      }
      \sstsubsection{
         A1 = DOUBLE PRECISION (Given)
      }{
         The sky longitude of the outlying point, in radians.
      }
      \sstsubsection{
         B1 = DOUBLE PRECISION (Given)
      }{
         The sky latitude of the outlying point, in radians.
      }
      \sstsubsection{
         PARDST = DOUBLE PRECISION (Returned)
      }{
         The arc-distance (in radians) from the reference point, to the
         point of closest approach on the great circle defined by the
         reference point and the position angle . If any of the input
         coordinate values are equal to the Starlink {\tt "}BAD{\tt "} value
         (VAL\_\_BADD), then PRPDST is returned with the BAD value.
         A positive value is returned if the outlying point is in the
         direction of the given position angle.
      }
      \sstsubsection{
         PRPDST = DOUBLE PRECISION (Returned)
      }{
         The arc-distance (in radians) of the outlying point, from the
         point of closest approach on the great circle defined by the
         reference point and the position angle . If any of the input
         coordinate values are equal to the Starlink {\tt "}BAD{\tt "} value
         (VAL\_\_BADD), then PRPDST is returned with the BAD value.
         A positive value is returned if rotation from the position
         angle specified by ANGLE to the outlying point (as seen from
         the reference point) is in the same sense as rotation from
         north to east.
      }
      \sstsubsection{
         STATUS = INTEGER (Given and Returned)
      }{
         The global status.
      }
   }
}
\sstroutine{
   IRA\_DRBND
}{
   Draw a boundary around the area representing valid sky positions
}{
   \sstdescription{
      This routine draws a curve around the region of image space
      containing valid sky positions. For small images this will
      usually just result in a box being drawn coincident with the
      edges of the image. However, some projections (for instance
      Aitoff and orthographic projections) project the sky into a
      finite area of image space, resulting in some image coordinates
      not corresponding to any valid sky coordinates. Large images may
      extend beyond the valid region of image space, and in this case
      this routine draws a boundary around the used area of image
      space.

      The plotting is done within the section of the current SGS zone
      specified by LBND and UBND, and it is assumed that the world
      coordinate system corresponds to image (or pixel) coordinates.
   }
   \sstinvocation{
      CALL IRA\_DRBND( IDA, LBND, UBND, STATUS )
   }
   \sstarguments{
      \sstsubsection{
         IDA = INTEGER (Given)
      }{
         An IRA identifier for the astrometry information.
      }
      \sstsubsection{
         LBND( 2 ) = REAL (Given)
      }{
         Lower world coordinate bounds for each axis defining the
         section of the current SGS zone in which the curve is to be
         drawn.
      }
      \sstsubsection{
         UBND( 2 ) = REAL (Given)
      }{
         Upper world coordinate bounds for each axis defining the
         section of the current SGS zone in which the curve is to be
         drawn.
      }
      \sstsubsection{
         STATUS = INTEGER (Given and Returned)
      }{
         The global status.
      }
   }
   \sstnotes{
      \sstitemlist{

         \sstitem
         This routine is effected by the TOLERANCE and PEN1 options set
         up by routine IRA\_DROPT.
      }
   }
}
\sstroutine{
   IRA\_DRBRK
}{
   Return information about a plotted curve
}{
   \sstdescription{
      This routine returns various items of information about the curve
      most recently drawn using any of the routines IRA\_DRGTC, IRA\_DRMER
      and IRA\_DRPAR. If the graphics options LINES (see routine
      IRA\_DROPT) is set so that the drawing of curves is suppressed, the
      information returned by this routine refers to the curve which
      would have been drawn if LINES had been set to allow drawing of
      curves.
   }
   \sstinvocation{
      CALL IRA\_DRBRK( MAXBRK, OUT, BREAK, VBREAK, NBREAK, LENGTH,
                      STATUS )
   }
   \sstarguments{
      \sstsubsection{
         MAXBRK = INTEGER (Given)
      }{
         The size of the arguments VBREAK and BREAK. The symbolic
         value IRA\_\_MXBRK can be used. An error is reported if the
         supplied arrays are too small.
      }
      \sstsubsection{
         OUT = LOGICAL (Returned)
      }{
         If .true. then the curve was completely outside the plotting
         zone, and so no part of it could be plotted.
      }
      \sstsubsection{
         BREAK( 2, MAXBRK ) = REAL (Returned)
      }{
         Contains the world coordinates at which each break in the
         plotted curve occurred. These breaks may be caused by
         discontinuities in the transformation between sky coordinates
         and image coordinates, or simply by the fact that the curve
         went outside the plotting window. The start and end of the
         curve are also considered to be {\tt "}breaks{\tt "} even when they occur
         within the plotting window. The exception to this is if the
         start and end are coincident. In this case no break is recorded
         for either the start or the end. BREAK( 1, I ) holds the X
         world coordinate value at the I{\tt '}th break, and BREAK( 2, I )
         holds the Y world coordinate value.
      }
      \sstsubsection{
         VBREAK( 2, MAXBRK ) = REAL (Returned)
      }{
         Contains the unit vector (within the world coordinate system)
         parallel to the tangent to the curve at each break. The sense
         is such that the vector always points back along the plotted
         section of the curve.
      }
      \sstsubsection{
         NBREAK = INTEGER (Returned)
      }{
         The number of breaks for which information is returned in BREAK
         and VBREAK.
      }
      \sstsubsection{
         LENGTH = REAL (Returned)
      }{
         The plotted length of the curve (in world coordinates).
      }
      \sstsubsection{
         STATUS = INTEGER (Given and Returned)
      }{
         The global status.
      }
   }
}
\sstroutine{
   IRA\_DRGRD
}{
   Draw a sky coordinate grid
}{
   \sstdescription{
      This routine draws a complete sky coordinate grid.  The grid is
      drawn over a section of the current SGS zone, specified by UBND
      and LBND. It is assumed that the world coordinate system
      associated with the current zone correspond to image (or pixel)
      coordinates. A boundary line is drawn around the region
      containing valid sky coordinate data.

      If the graphics option LINES (see routine IRA\_DROPT) is set to a
      negative value, then lines of constant latitude and longitude are
      drawn across the plot. Otherwise, tick marks are used to indicate
      longitude and latitude.
   }
   \sstinvocation{
      CALL IRA\_DRGRD( IDA, SCS, LBND, UBND, STATUS )
   }
   \sstarguments{
      \sstsubsection{
         IDA = INTEGER (Given)
      }{
         An IRA identifier for the astrometry information.
      }
      \sstsubsection{
         SCS = CHARACTER $*$ ( $*$ ) (Given)
      }{
         The name of the sky coordinate system to display. Any unambiguous
         abbreviation will do. This need not be the same as the sky
         coordinate system stored in the astrometry structure
         identified by IDA. See ID/2 section {\tt "}Sky Coordinates{\tt "} for more
         information.
      }
      \sstsubsection{
         LBND( 2 ) = REAL (Given)
      }{
         Lower world coordinate bounds for each axis defining the
         section of the current SGS zone covered by the grid. Labels
         may be placed outside this region.
      }
      \sstsubsection{
         UBND( 2 ) = REAL (Given)
      }{
         Upper world coordinate bounds for each axis defining the
         section of the current SGS zone covered by the grid. Labels
         may be placed outside this region.
      }
      \sstsubsection{
         STATUS = INTEGER (Given and Returned)
      }{
         The global status.
      }
   }
   \sstnotes{
      \sstitemlist{

         \sstitem
         This routine is effected by the TOLERANCE, TEXT\_SIZE,
         COORD\_SIZE, LONG\_GAP, LAT\_GAP, LINES, LAT\_ACC, LONG\_ACC, PEN1,
         PEN2, PEN3 and PEN4 options set up by routine IRA\_DROPT.
      }
   }
}
\sstroutine{
   IRA\_DRGTC
}{
   Draw a section of a great circle
}{
   \sstdescription{
      This routine draws a curve representing a section of a great
      circle, starting at a given position and extending for a given
      length, at a given position angle. The curve is split into a
      number of sections and each section drawn as a straight line. The
      number of sections used depends on the curvature of the great
      circle and the value of the graphics option TOLERANCE (see
      routine IRA\_DROPT); highly curved great circles are split into
      more sections than nearly straight ones. The plotting is done
      within the section of the current SGS zone specified by LBND and
      UBND, and it is assumed that the world coordinate system within
      the zone corresponds to image (or pixel) coordinates.  Only the
      section of the great circle which lies within the zone is
      displayed.

      Various items of information about the plotted curve can be
      obtained once the plot has been produced using routine IRA\_DRBRK.
   }
   \sstinvocation{
      CALL IRA\_DRGTC( IDA, A, B, ANGLE, DIST, SCS, LBND, UBND,
                      STATUS )
   }
   \sstarguments{
      \sstsubsection{
         IDA = INTEGER (Given)
      }{
         An IRA identifier for the astrometry information.
      }
      \sstsubsection{
         A = DOUBLE PRECISION (Given)
      }{
         The sky longitude at the start of the great circle, in radians.
      }
      \sstsubsection{
         B = DOUBLE PRECISION (Given)
      }{
         The sky latitude at the start of the great circle, in radians.
      }
      \sstsubsection{
         ANGLE = DOUBLE PRECISION (Given)
      }{
         The position angle of the great circle at the given starting
         position. That is, the angle from north to the required
         direction, in radians. Positive angles are in the sense of
         rotation from north to east.
      }
      \sstsubsection{
         DIST = DOUBLE PRECISION (Given)
      }{
         The arc-length of the section of the great circle to draw, in
         radians. This can be positive or negative. The value used is
         limited internally to an absolute value no greater than 2.PI.
      }
      \sstsubsection{
         SCS = CHARACTER $*$ ( $*$ ) (Given)
      }{
         The name of the sky coordinate system to use. Any unambiguous
         abbreviation will do. This need not be the same as the sky
         coordinate system stored in the astrometry structure
         identified by IDA. See ID/2 section {\tt "}Sky Coordinates{\tt "} for more
         information.
      }
      \sstsubsection{
         LBND( 2 ) = REAL (Given)
      }{
         Lower world coordinate bounds for each axis defining the
         section of the current SGS zone in which the curve is to be
         drawn.
      }
      \sstsubsection{
         UBND( 2 ) = REAL (Given)
      }{
         Upper world coordinate bounds for each axis defining the
         section of the current SGS zone in which the curve is to be
         drawn.
      }
      \sstsubsection{
         STATUS = INTEGER (Given and Returned)
      }{
         The global status.
      }
   }
   \sstnotes{
      \sstitemlist{

         \sstitem
         This routine is effected by the TOLERANCE, LINES and PEN2
         options set up by routine IRA\_DROPT.
      }
   }
}
\sstroutine{
   IRA\_DRMER
}{
   Draw a section of a meridian
}{
   \sstdescription{
      This routine draws a curve representing a section of a meridian,
      starting at a given position and extending over a given latitude
      range. The curve is split into a number of sections and each
      section drawn as a straight line. The number of sections used
      depends on the curvature of the meridian and the value of
      graphics option TOLERANCE (see routine IRA\_DROPT); highly curved
      meridians are split into more sections than nearly straight ones.
      The plotting is done within the section of the current SGS zone
      specified by LBND and UBND, and it is assumed that the world
      coordinate system within the zone corresponds to image (or pixel)
      coordinates.  Only the section of the meridian which lies within
      the zone is displayed.

      Various items of information about the plotted curve can be
      obtained once the plot has been produced using routine IRA\_DRBRK.
   }
   \sstinvocation{
      CALL IRA\_DRMER( IDA, A, B, INCB, SCS, LBND, UBND, STATUS )
   }
   \sstarguments{
      \sstsubsection{
         IDA = INTEGER (Given)
      }{
         An IRA identifier for the astrometry information.
      }
      \sstsubsection{
         A = DOUBLE PRECISION (Given)
      }{
         The sky longitude at the start of the meridian, in radians.
      }
      \sstsubsection{
         B = DOUBLE PRECISION (Given)
      }{
         The sky latitude at the start of the meridian, in radians.
      }
      \sstsubsection{
         INCB = DOUBLE PRECISION (Given)
      }{
         The increment in sky latitude between the start and end of
         the meridian, in radians. This can be positive or negative.
         The value used is limited internally to an absolute value no
         greater than 2.PI.
      }
      \sstsubsection{
         SCS = CHARACTER $*$ ( $*$ ) (Given)
      }{
         The name of the sky coordinate system to use. Any unambiguous
         abbreviation will do. This need not be the same as the sky
         coordinate system stored in the astrometry structure
         identified by IDA. See ID/2 section {\tt "}Sky Coordinates{\tt "} for more
         information.
      }
      \sstsubsection{
         LBND( 2 ) = REAL (Given)
      }{
         Lower world coordinate bounds for each axis defining the
         section of the current SGS zone in which the curve is to be
         drawn.
      }
      \sstsubsection{
         UBND( 2 ) = REAL (Given)
      }{
         Upper world coordinate bounds for each axis defining the
         section of the current SGS zone in which the curve is to be
         drawn.
      }
      \sstsubsection{
         STATUS = INTEGER (Given and Returned)
      }{
         The global status.
      }
   }
   \sstnotes{
      \sstitemlist{

         \sstitem
         This routine is effected by the TOLERANCE, LINES and PEN2
         options set up by routine IRA\_DROPT.
      }
   }
}
\sstroutine{
   IRA\_DROPS
}{
   Get the current value of a graphics option
}{
   \sstdescription{
      This routine returns the current value of a graphics option. See
      routine IRA\_DROPT for a description of these options.
   }
   \sstinvocation{
      CALL IRA\_DROPS( ITEM, VALUE, STATUS )
   }
   \sstarguments{
      \sstsubsection{
         ITEM = CHARACTER $*$ ( $*$ ) Given)
      }{
         The name of the option (see the {\tt "}Notes:{\tt "} section below). An
         unambiguous abbreviation may be supplied. Case is ignored.
      }
      \sstsubsection{
         VALUE = DOUBLE PRECISION (Returned)
      }{
         The current value of the option.
      }
      \sstsubsection{
         STATUS = INTEGER (Given and Returned)
      }{
         The global status.
      }
   }
}
\sstroutine{
   IRA\_DROPT
}{
   Set an option for the IRA graphics routines
}{
   \sstdescription{
      This routine allows several aspects of the graphics produced by
      the routines IRA\_DRxxx (eg IRA\_DRGRD, etc) to be controlled by
      setting new values for various {\tt "}options{\tt "}. These option take
      default values (described in the {\tt "}Notes:{\tt "} section below unless a
      new value is assigned to them using this routine.
   }
   \sstinvocation{
      CALL IRA\_DROPT( ITEM, VALUE, STATUS )
   }
   \sstarguments{
      \sstsubsection{
         ITEM = CHARACTER $*$ ( $*$ ) Given)
      }{
         The name of the option (see the {\tt "}Notes:{\tt "} section below). An
         unambiguous abbreviation may be supplied. Case is ignored.
      }
      \sstsubsection{
         VALUE = DOUBLE PRECISION (Given)
      }{
         The new value for the option.
      }
      \sstsubsection{
         STATUS = INTEGER (Given and Returned)
      }{
         The global status.
      }
   }
   \sstnotes{
      The argument ITEM may take any of the following values (default
      values are shown in square brackets at the end of each
      description):

      \sstitemlist{

         \sstitem
         {\tt '}TEXT\_SIZE{\tt '}  the height of axis titles, expressed as a
         fraction of the maximum dimension of the area containing the
         plot. A negative value suppresses text labels. [0.0125]

         \sstitem
         {\tt '}COORD\_SIZE{\tt '}  the height of formatted coordinate values,
         expressed as a fraction of the maximum dimension of the SGS zone
         selected on entry to the plotting routine. A negative value
         supresses coordinate labels. [0.0125]

         \sstitem
         {\tt '}TOLERANCE{\tt '}  a measure of the accuracy required when drawing
         curves. The value should be between 0.0 (for maximum accuracy)
         and 10.0 (for minimum accuracy). Greater accuracy is bought at the
         cost of much greater processing time. The nearest integer value
         is used. [6.0]

         \sstitem
         {\tt '}LINES{\tt '}  if positive, then complete curves will be drawn. If
         negative, no curves will be drawn and tick marks may be drawn
         instead. [$+$1.0]

         \sstitem
         {\tt '}LONG\_GAP{\tt '}  the gap in longitude between meridians, in radians.
         A negative or zero value causes an internally calculated value to
         be used. [-1.0]

         \sstitem
         {\tt '}LAT\_GAP{\tt '}  the gap in latitude between meridians, in radians.
         A negative or zero value causes an internally calculated value to
         be used. [-1.0]

         \sstitem
         {\tt '}LONG\_ACC{\tt '}  specifies the accuracy to which a longitude value
         should be displayed, in radians.  The displayed value is such that
         a change of 1 in the least significant field corresponds to the
         largest value which is smaller than (or equal to) the supplied
         value. For instance, if a seconds field is not required in the
         displayed text, then LONG\_ACC could be given the radian equivalent
         of 1 minute. A negative or zero value causes an internally
         calculated value to be used. [-1.0]

         \sstitem
         {\tt '}LAT\_ACC{\tt '}  specifies the accuracy to which a latitude value
         should be displayed, in radians. See LONG\_ACC. [-1.0]

         \sstitem
         {\tt '}PEN1{\tt '}  the SGS pen to use when drawing the boundary (see
         IRA\_DRBND). The nearest integer value is used. [1.0]

         \sstitem
         {\tt '}PEN2{\tt '}  the SGS pen to use when drawing curves or ticks. The
         nearest integer value is used. [1.0]

         \sstitem
         {\tt '}PEN3{\tt '}  the SGS pen to use when drawing text labels. The
         nearest integer value is used. [1.0]

         \sstitem
         {\tt '}PEN4{\tt '}  the SGS pen to use when drawing coordinate labels. The
         nearest integer value is used. [1.0]
      }
   }
}
\sstroutine{
   IRA\_DRPAR
}{
   Draw a section of a parallel
}{
   \sstdescription{
      This routine draws a curve representing a section of a parallel,
      starting at a given position and extending over a given longitude
      range. The curve is split into a number of sections and each
      section drawn as a straight line. The number of sections used
      depends on the curvature of the parallel and the value of
      graphics options TOLERANCE (See routine IRA\_DROPT); highly curved
      parallels are split into more sections than nearly straight ones.
      The plotting is done within the section of the current SGS zone
      specified by LBND and UBND, and it is assumed that the world
      coordinate system within the zone corresponds to image (or pixel)
      coordinates.  Only the section of the parallel which lies within
      the zone is displayed.

      Various items of information about the plotted curve can be
      obtained once the plot has been produced using routine IRA\_DRBRK.
   }
   \sstinvocation{
      CALL IRA\_DRPAR( IDA, A, B, INCA, SCS, LBND, UBND, STATUS )
   }
   \sstarguments{
      \sstsubsection{
         IDA = INTEGER (Given)
      }{
         An IRA identifier for the astrometry information.
      }
      \sstsubsection{
         A = DOUBLE PRECISION (Given)
      }{
         The sky longitude at the start of the parallel, in radians.
      }
      \sstsubsection{
         B = DOUBLE PRECISION (Given)
      }{
         The sky latitude at the start of the parallel, in radians.
      }
      \sstsubsection{
         INCA = DOUBLE PRECISION (Given)
      }{
         The increment in sky longitude between the start and end of
         the parallel, in radians. This can be positive or negative.
         The value used is limited internally to an absolute value no
         greater than 2.PI.
      }
      \sstsubsection{
         SCS = CHARACTER $*$ ( $*$ ) (Given)
      }{
         The name of the sky coordinate system to use. Any unambiguous
         abbreviation will do. This need not be the same as the sky
         coordinate system stored in the astrometry structure
         identified by IDA. See ID/2 section {\tt "}Sky Coordinates{\tt "} for more
         information.
      }
      \sstsubsection{
         LBND( 2 ) = REAL (Given)
      }{
         Lower world coordinate bounds for each axis defining the
         section of the current SGS zone in which the curve is to be
         drawn.
      }
      \sstsubsection{
         UBND( 2 ) = REAL (Given)
      }{
         Upper world coordinate bounds for each axis defining the
         section of the current SGS zone in which the curve is to be
         drawn.
      }
      \sstsubsection{
         STATUS = INTEGER (Given and Returned)
      }{
         The global status.
      }
   }
   \sstnotes{
      \sstitemlist{

         \sstitem
         This routine is effected by the TOLERANCE, LINES and PEN2
         options set up by routine IRA\_DROPT.
      }
   }
}
\sstroutine{
   IRA\_DRVAL
}{
   Plot a text string holding a longitude or latitude value
}{
   \sstdescription{
      This routine plots a text string holding a formatted version of a
      longitude or latitude value, at a user specified location.
      Plotting is performed with the current SGS zone.

      A facility is available to suppress the display of leading fields
      which have not changed since the last value was displayed (see
      argument CONTXT).
   }
   \sstinvocation{
      CALL IRA\_DRVAL( VALUE, SCS, NC, XPOS, YPOS, STYLE, ACC, CONTXT,
                      STATUS )
   }
   \sstarguments{
      \sstsubsection{
         VALUE = DOUBLE PRECISION (Given)
      }{
         The longitude or latitude value, in radians.
      }
      \sstsubsection{
         SCS = CHARACTER $*$ ( $*$ ) (Given)
      }{
         The name of the sky coordinate system to use. Any unambiguous
         abbreviation will do. See ID/2 section {\tt "}Sky Coordinates{\tt "} for
         more information.
      }
      \sstsubsection{
         NC = INTEGER (Given)
      }{
         The axis index; 1 if VALUE is a longitude value and 2 if it is
         a latitude value.
      }
      \sstsubsection{
         XPOS = REAL (Given)
      }{
         The world coordinate X value at which to put the text. The
         current SGS text justification and up vector are used to
         position and orient the text.
      }
      \sstsubsection{
         YPOS = REAL (Given)
      }{
         The world coordinate Y value at which to put the text.
      }
      \sstsubsection{
         STYLE = INTEGER (Given)
      }{
         The style number for the text. The styles available are
         described in IRA\_DTOC. Note, styles 1 and 3 are not available
         and style 5 has a superscript unit symbol added at the end of
         the string. Superscripts (h, m, s, o, {\tt '} or {\tt "} ) are used to
         indicate units rather than the normal text characters h, m, s
         and d.
      }
      \sstsubsection{
         ACC = DOUBLE PRECISION (Given)
      }{
         Specifies the accuracy to which the value should be displayed,
         in radians.  The displayed text is such that a change of 1 in
         the least significant field corresponds to the largest value
         which is smaller than (or equal to) the supplied value of ACC.
         For instance, if an arc-seconds field is not required in the
         displayed text, then ACC could be given the radian equivalent
         of 1 arc-minute.
      }
      \sstsubsection{
         CONTXT = CHARACTER $*$ ( $*$ ) (Given and Returned)
      }{
         The {\tt "}context {\tt "}. This returns information about the text fields
         plotted by this routine. This value can be passed on to the
         next call of this routine in order to suppress the display of
         unchanged leading fields. If the supplied context is blank or
         invalid, then all fields are displayed and the context is
         returned holding information describing the displayed text
         fields. If the supplied context is equal to {\tt "}KEEP{\tt "} then all
         fields are displayed, and the value of CONTXT is left
         unchanged. In normal practice CONTXT should be set blank
         before the first call to this routine, and then left
         un-altered between successive calls. It should have a declared
         size equal to the symbolic constant IRA\_\_SZCTX.
      }
      \sstsubsection{
         STATUS = INTEGER (Given and Returned)
      }{
         The global status.
      }
   }
   \sstnotes{
      \sstitemlist{

         \sstitem
         This routine is effected by the COORD\_SIZE and PEN4 options
         set up by routine IRA\_DROPT.
      }
   }
}
\sstroutine{
   IRA\_DRVPO
}{
   Return information about a plotted coordinate value
}{
   \sstdescription{
      This routine returns items of information describing the area in
      which the last value plotted by IRA\_DRVAL appeared.
   }
   \sstinvocation{
      CALL IRA\_DRVPO( X1, Y1, X2, Y2, H, STATUS )
   }
   \sstarguments{
      \sstsubsection{
         X1 = REAL (Returned)
      }{
         The X coordinate at the centre of the left hand edge of the
         box in which the text was written.
      }
      \sstsubsection{
         Y1 = REAL (Returned)
      }{
         The Y coordinate at the centre of the left hand edge of the
         box in which the text was written.
      }
      \sstsubsection{
         X2 = REAL (Returned)
      }{
         The X coordinate at the centre of the right hand edge of the
         box in which the text was written.
      }
      \sstsubsection{
         Y2 = REAL (Returned)
      }{
         The Y coordinate at the centre of the right hand edge of the
         box in which the text was written.
      }
      \sstsubsection{
         H = REAL (Returned)
      }{
         The height of the box in which the text was written,
         perpendicular to the line joining (X1,Y1) and (X2,Y2).
      }
      \sstsubsection{
         STATUS = INTEGER (Given and Returned)
      }{
         The global status.
      }
   }
}
\sstroutine{
   IRA\_DTOC
}{
   Convert a pair of double precision sky coordinate values to
   character form
}{
   \sstdescription{
      This routine creates a pair of text strings containing formatted
      versions of the given sky coordinate values.  The exact format
      depends on the type of sky coordinate system in use and the value
      of STYLE (see the {\tt "}Notes{\tt "} section below). The input coordinate
      values are shifted into their first order ranges before being used
      (see IRA\_NORM).
   }
   \sstinvocation{
      CALL IRA\_DTOC( A, B, SCS, STYLE, ATEXT, BTEXT, STATUS )
   }
   \sstarguments{
      \sstsubsection{
         A = DOUBLE PRECISION  (Given)
      }{
         The value of the sky longitude to be formatted, in radians. If
         A has the Starlink {\tt "}BAD{\tt "} value (VAL\_\_BADD) then the output
         string ATEXT is set blank.
      }
      \sstsubsection{
         B = DOUBLE PRECISION  (Given)
      }{
         The value of the sky latitude to be formatted, in radians. If
         B has the {\tt "}BAD{\tt "} value then the output string BTEXT is set
         blank.
      }
      \sstsubsection{
         SCS = CHARACTER $*$ ( $*$ ) (Given)
      }{
         The sky coordinate system in use (see ID2 section {\tt "}Sky
         Coordinates{\tt "}). Any unambiguous abbreviation will do.
      }
      \sstsubsection{
         STYLE = INTEGER (Given)
      }{
         A value in the range 1 to 5 which specifies the style of
         output formatting required.  Additionally, a value of zero can
         specified which causes a default style to be used dependant on
         the value of SCS.  See the {\tt "}Notes{\tt "} section below for a
         description of the individual styles and defaults for each
         SCS.
      }
      \sstsubsection{
         ATEXT = CHARACTER $*$ ( $*$ ) (Returned)
      }{
         The string containing the formatted description of the sky
         longitude value A. The variable supplied for ATEXT should
         have a declared length equal to the value of parameter
         IRA\_\_SZFSC.
      }
      \sstsubsection{
         BTEXT = CHARACTER $*$ ( $*$ ) (Returned)
      }{
         The string containing the formatted description of the sky
         latitude value B. The variable supplied for BTEXT should
         have a declared length equal to the value of parameter
         IRA\_\_SZFSC.
      }
      \sstsubsection{
         STATUS = INTEGER (Given and Returned)
      }{
         The global status.
      }
   }
   \sstnotes{
      \sstitemlist{

         \sstitem
         SCS = {\tt "}EQUATORIAL{\tt "}
           Default is style 2 (used if argument STYLE is zero on entry).

      }
         STYLE = 1:  (a full description)

            {\tt "}RA = 12hrs 3m 0.02s{\tt "} and  {\tt "}DEC = -33deg 23m 0.00s{\tt "}

         STYLE = 2:  (a more brief form readable by IRA\_CTOD)

            {\tt "}12h 3m 0.02s{\tt "} and {\tt "}-33d 23m 0.00s{\tt "}

         STYLE = 3:  ( a very brief form readable by IRA\_CTOD)

            {\tt "}120300.00{\tt "} and {\tt "}-332300.00{\tt "} ( eg hhmmss.ss and ddmmss.ss )

         STYLE = 4:  (a brief form readable by IRA\_CTOD)

            {\tt "}12 03 0.02{\tt "} and {\tt "}-33 23 0.00{\tt "}

         STYLE = 5:  (a brief form readable by IRA\_CTOD)

            {\tt "}12.050006{\tt "} and {\tt "}-33.383333{\tt "} (eg fractional values in
                                       hours (RA) and degrees (DEC) )

      \sstitemlist{

         \sstitem
         SCS = {\tt "}GALACTIC{\tt "}
           Default is style 5 (used if argument STYLE is zero on entry).

      }
         STYLE = 1:

            {\tt "}l = 12deg 3m 0.02s{\tt "} and  {\tt "}b = -33deg 23m 0.00s{\tt "}

         STYLE = 2:

            {\tt "}12deg 3m 0.02s{\tt "} and {\tt "}-33d 23m 0.00s{\tt "}

         STYLE = 3:

            {\tt "}0120300.00{\tt "} and {\tt "}-332300.00{\tt "} (eg dddmmss.ss and ddmmss.ss )

         STYLE = 4:  (a brief form readable by IRA\_CTOD)

            {\tt "}12 03 0.02{\tt "} and {\tt "}-33 23 0.00{\tt "}

         STYLE = 5:  (a brief form readable by IRA\_CTOD)

            {\tt "}12.050006{\tt "} and {\tt "}-33.383333{\tt "} (eg fractional values in
                                         degrees )

      \sstitemlist{

         \sstitem
         SCS = {\tt "}ECLIPTIC{\tt "}
           Default is style 5 (used if argument STYLE is zero on entry).

      }
         STYLE = 1:

            {\tt "}Lambda = 12deg 3m 0.02s{\tt "} and  {\tt "}Beta = -33deg 23m 0.00s{\tt "}

         STYLE = 2:

            {\tt "}12deg 3m 0.02s{\tt "} and {\tt "}-33d 23m 0.00s{\tt "}

         STYLE = 3:

            {\tt "}0120300.00{\tt "} and {\tt "}-332300.00{\tt "} (eg dddmmss.ss and ddmmss.ss )

         STYLE = 4:  (a brief form readable by IRA\_CTOD)

            {\tt "}12 03 0.02{\tt "} and {\tt "}-33 23 0.00{\tt "}

         STYLE = 5:  (a brief form readable by IRA\_CTOD)

            {\tt "}12.050006{\tt "} and {\tt "}-33.383333{\tt "} (eg fractional values in
                                         degrees )
   }
}
\sstroutine{
   IRA\_DTOC1
}{
   Convert a single floating point sky coordinate value to
   character form
}{
   \sstdescription{
      This routine creates a text string containing a formatted version
      of the given sky coordinate value. The value is assumed to be a
      longitude value if NC is 1, and a latitude if NC is 2. The
      formats of the output string are as described in routine
      IRA\_DTOC. Longitude values are shifted into the range 0 - 2$*$PI
      before being used.  Latitude values are shifted into the range
      $+$/- PI before being used. An error is reported if a latitude
      value then has an absolute value greater than PI/2 (this differs
      from the behaviour of IRA\_NORM which always reduces the latitude
      value to $+$/- PI/2).
   }
   \sstinvocation{
      CALL IRA\_DTOC1( VALUE, SCS, NC, STYLE, TEXT, STATUS )
   }
   \sstarguments{
      \sstsubsection{
         VALUE = DOUBLE PRECISION (Given)
      }{
         The value of the sky coordinate to be formatted, in radians.
         If VALUE has the Starlink {\tt "}BAD{\tt "} value (VAL\_\_BADD) then the
         output string TEXT is set blank.
      }
      \sstsubsection{
         SCS = CHARACTER $*$ ( $*$ ) (Given)
      }{
         The sky coordinate system in use (see ID2 section {\tt "}Sky
         Coordinates{\tt "}). Any unambiguous abbreviation will do.
      }
      \sstsubsection{
         NC = INTEGER (Given)
      }{
         Determines which sky coordinate is given. If a value of 1 is
         supplied, VALUE is interpreted as a longitude value (eg RA if
         an equatorial system is being used).  If a value of 2 is
         supplied, VALUE is interpreted as a latitude value. Any other
         value causes an error to be reported.
      }
      \sstsubsection{
         STYLE = INTEGER (Given)
      }{
         A value in the range 1 to 5 which specifies the style of
         output formatting required. In addition a value of zero can be
         specified which causes a default style to be used dependant on
         the value of SCS. See routine IRA\_DTOC for a description of
         the styles and defaults.
      }
      \sstsubsection{
         TEXT = CHARACTER $*$ ( $*$ ) (Returned)
      }{
         The string containing the formatted description of the sky
         coordinate value. The variable supplied for TEXT should have a
         declared length equal to the value of parameter IRA\_\_SZFSC.
      }
      \sstsubsection{
         STATUS = INTEGER (Given and Returned)
      }{
         The global status.
      }
   }
}
\sstroutine{
   IRA\_EXPRT
}{
   Store astrometry information in an NDF
}{
   \sstdescription{
      An HDS structure is created containing the astrometry information
      identified by IDA. This {\tt "}astrometry structure{\tt "} is stored as a
      component of an extension within the NDF specified by INDF (any
      previous astrometry structure is over-written). The names of the
      NDF extension and the astrometry structure are set by a call to
      IRA\_LOCAT. If no such call is made the names of the extension and
      astrometry structure retain the values set up in IRA\_INIT ({\tt "}IRAS{\tt "}
      and {\tt "}ASTROMETRY{\tt "}).  The astrometry structure has an HDS data type
      of IRAS\_ASTROMETRY. The NDF extension must already exist before
      calling this routine.
   }
   \sstinvocation{
      CALL IRA\_EXPRT( IDA, INDF, STATUS )
   }
   \sstarguments{
      \sstsubsection{
         IDA = INTEGER (Given)
      }{
         An IRA identifier for the astrometry information.
      }
      \sstsubsection{
         INDF = INTEGER (Given)
      }{
         The identifier for the NDF in which the astrometry information
         is to be stored.
      }
      \sstsubsection{
         STATUS = INTEGER (Given and Returned)
      }{
         The global status.
      }
   }
}
\sstroutine{
   IRA\_FIND
}{
   Find an astrometry structure within an NDF
}{
   \sstdescription{
      A search is made for an astrometry structure within an NDF.  If
      one is found, THERE is returned true. If one is not found, no
      error is reported but THERE is returned false.  The name of the
      extension in which it was found is returned, together with the
      name of the astrometry structure, and a locator to the extension.
   }
   \sstinvocation{
      CALL IRA\_FIND( INDF, THERE, XNAME, ASNAME, LOC, STATUS )
   }
   \sstarguments{
      \sstsubsection{
         INDF = INTEGER (Given)
      }{
         The NDF identifier.
      }
      \sstsubsection{
         THERE = LOGICAL (Returned)
      }{
         True if an astrometry structure was found, false otherwise.
      }
      \sstsubsection{
         XNAME = CHARACTER $*$ ( $*$ ) (Returned)
      }{
         The name of the NDF extension in which the astrometry
         structure was found. Blank if none found.
      }
      \sstsubsection{
         ASNAME = CHARACTER $*$ ( $*$ ) (Returned)
      }{
         The name of a component of the NDF extension holding the
         astrometry information. Blank if none found.
      }
      \sstsubsection{
         LOC = CHARACTER $*$ ( $*$ ) (Returned)
      }{
         A locator to the extension identified by XNAME. Blank if no
         astrometry structure is found.
      }
      \sstsubsection{
         STATUS = INTEGER (Given and Returned)
      }{
         The global status.
      }
   }
}
\sstroutine{
   IRA\_GETCO
}{
   Obtain a pair of sky coordinates from the ADAM environment
}{
   \sstdescription{
      The ADAM parameters specified by arguments APAR and BPAR are used
      to acquire values for the first and second sky coordinates
      respectively, in the sky coordinate system specified by argument
      SCS. The parameters are obtained as literal character strings and
      decoded into floating point values. See routine IRA\_CTOD for a
      description of the allowed formats of the strings associated with
      these parameters. The input values of arguments A and B can
      optionally be supplied to the user as default parameter values.
      The parameter prompt strings contained in the application{\tt '}s
      interface file can be overridden by giving a non-blank value for
      argument PRMAPP. In this case, the prompts are formed by
      appending the value of PRMAPP to the coordinate descriptions
      returned by routine IRA\_SCNAM. For instance, if PRMAPP = {\tt "} of the
      field centre{\tt "}, and an equatorial sky coordinate system is in use,
      then the prompt for APAR will be {\tt "}Right Ascension of the field
      centre{\tt "}, and the prompt for BPAR will be {\tt "}Declination of the
      field centre{\tt "}. Note, the total length of the prompt strings is
      limited to 80 characters. If PRMAPP is blank, then the current
      prompt strings are used (initially equal to the values in the
      interface file).
   }
   \sstinvocation{
      CALL IRA\_GETCO( APAR, BPAR, PRMAPP, SCS, DEFLT, A, B, STATUS )
   }
   \sstarguments{
      \sstsubsection{
         CHARACTER = APAR (Given)
      }{
         The name of the ADAM parameter (type LITERAL) to use
         for the sky longitude value.
      }
      \sstsubsection{
         CHARACTER = BPAR (Given)
      }{
         The name of the ADAM parameter (type LITERAL) to use
         for the sky latitude value.
      }
      \sstsubsection{
         PRMAPP = CHARACTER $*$ ( $*$ ) (Given)
      }{
         A string to append to each axis description to form the
         parameter prompt strings. If this is blank then the current
         prompt strings are used (i.e. initially set to the values in
         the interface file).
      }
      \sstsubsection{
         SCS = CHARACTER $*$ ( $*$ ) (Given)
      }{
         The name of the sky coordinate system to use. Any unambiguous
         abbreviation will do (see ID2 section {\tt "}Sky Coordinates{\tt "}).
      }
      \sstsubsection{
         DEFLT = LOGICAL (Given)
      }{
         True if the input values of A and B are to be communicated to
         the environment as run-time defaults for the parameters
         specified by APAR and BPAR. If A or B is {\tt "}BAD{\tt "} on entry
         (i.e. equal to VAL\_\_BADD ) then no default is set up for the
         corresponding parameter.
      }
      \sstsubsection{
         A = DOUBLE PRECISION (Given and Returned)
      }{
         The value of the first sky coordinate. In radians.
      }
      \sstsubsection{
         B = DOUBLE PRECISION (Given and Returned)
      }{
         The value of the second sky coordinate. In radians.
      }
      \sstsubsection{
         STATUS = INTEGER (Given and Returned)
      }{
         The global status.
      }
   }
}
\sstroutine{
   IRA\_GETEQ
}{
   Extract the epoch of the reference equinox from a string
   specifying a Sky Coordinate System
}{
   \sstdescription{
      If, on entry, the argument SCS contains an explicit equinox
      specifier (see routine IRA\_ISCS), the epoch contained within it
      is returned in argument EQU as a double precision value, and
      argument BJ is returned equal to the character {\tt "}B{\tt "} or {\tt "}J{\tt "}
      depending on whether the epoch is Besselian or Julian. If there is
      no equinox specifier in argument SCS on entry, then the default of
      B1950 is returned.

      If the sky coordinate system specified by SCS is not referred to
      the equinox (eg GALACTIC) then EQU is returned equal to the
      Starlink {\tt "}BAD{\tt "} value VAL\_\_BADD, and BJ is returned blank.

      The argument NAME is returned holding the full (unabbreviated)
      name of the sky coordinate system without any equinox specifier.
      On exit, the argument SCS holds the full name plus an explicit
      equinox specifier (for systems which are referred to the
      equinox). Thus, if SCS contained {\tt "}EQUAT{\tt "} on entry, it would
      contain {\tt "}EQUATORIAL(B1950){\tt "} on exit.
   }
   \sstinvocation{
      CALL IRA\_GETEQ( SCS, EQU, BJ, NAME, STATUS )
   }
   \sstarguments{
      \sstsubsection{
         SCS = CHARACTER $*$ ( $*$ ) (Given and Returned)
      }{
         On entry this should contain an SCS name (or any unambiguous
         abbreviation), with or without an equinox specifier. On exit,
         it contains the full SCS name with an explicit equinox
         specifier (for those sky coordinate systems which are referred
         to the equinox). If no equinox specifier is present on entry,
         then a value of B1950 is used (if required). This variable
         should have a declared length given by the symbolic constant
         IRA\_\_SZSCS.
      }
      \sstsubsection{
         EQU = DOUBLE PRECISION (Returned)
      }{
         The epoch of the reference equinox. This is extracted
         from any explicit equinox specifier contained in SCS on entry.
         If there is no equinox specifier in SCS, a value of 1950.0
         is returned. If the sky coordinate system is not referred to
         the equinox (eg GALACTIC) the Starlink {\tt "}BAD{\tt "} value (VAL\_\_BADD)
         is returned, irrespective of any equinox specifier in SCS.
      }
      \sstsubsection{
         BJ = CHARACTER $*$ ( $*$ ) (Returned)
      }{
         Returned holding either the character B or J. Indicates if
         argument EQU gives a Besselian or Julian epoch. Returned blank
         if the sky coordinate system is not referred to the equinox.
      }
      \sstsubsection{
         NAME = CHARACTER $*$ ( $*$ ) (Returned)
      }{
         The full name of the sky coordinate system without any equinox
         specifier. This variable should have a declared length given by
         the symbolic constant IRA\_\_SZSCS.
      }
      \sstsubsection{
         STATUS = INTEGER (Given and Returned)
      }{
         The global status.
      }
   }
}
\sstroutine{
   IRA\_GTCO1
}{
   Obtain a single sky coordinate value from the ADAM environment
}{
   \sstdescription{
      The ADAM parameter specified by argument PARAM is used to acquire
      a longitude or latitude value in the requested sky coordinate
      system. Argument NC determines which is to be obtained.  The
      string is decoded into a double precision number representing the
      sky position.  See the documentation for IRA\_CTOR for a
      description of the allowed formats.  The input sky coordinate
      value can optionally be communicated to the environment as a
      dynamic default.
   }
   \sstinvocation{
      CALL IRA\_GTCO1( PARAM, PROMPT, SCS, NC, DEFLT, VALUE, STATUS )
   }
   \sstarguments{
      \sstsubsection{
         PARAM = CHARACTER $*$ ( $*$ ) (Given)
      }{
         The name of the ADAM parameter used to get the sky
         coordinate value.
      }
      \sstsubsection{
         PROMPT = CHARACTER $*$ ( $*$ ) (Given)
      }{
         A string to override the current prompt for the parameter.
         If this is blank, the prompt is left at its current value.
         The initial value for the prompt is defined in the interface
         file. Note, unlike routine IRA\_GETCO, the axis name is not
         automatically included in the prompt.
      }
      \sstsubsection{
         SCS = CHARACTER $*$ ( $*$ ) (Given)
      }{
         The sky coordinate system in use. Any unambiguous abbreviation
         will do  (see ID2 section {\tt "}Sky Coordinates{\tt "}).
      }
      \sstsubsection{
         NC = INTEGER (Given)
      }{
         Determines which sky coordinate is to be returned. If a value
         of 1 is supplied, the string obtained for the parameter is
         interpreted as a longitude value (eg RA if an equatorial
         system is being used). If a value of 2 is supplied, the string
         is interpreted as a latitude value.  Any other value causes an
         error to be reported.
      }
      \sstsubsection{
         DEFLT = LOGICAL (Given)
      }{
         If true, then the value of VALUE on entry is communicated to
         the environment as a dynamic default. If false, or if VALUE is
         {\tt "}BAD{\tt "} on entry (i.e. equal to VAL\_\_BADD), then no dynamic
         default is set up.
      }
      \sstsubsection{
         VALUE = DOUBLE PRECISION (Given and Returned)
      }{
         The sky coordinate value. On input it contains the default
         value (in radians) to use if DEFLT is true. On exit it
         contains the decoded value obtained from the environment, in
         radians.
      }
      \sstsubsection{
         STATUS = INTEGER (Given and Returned)
      }{
         The global status.
      }
   }
}
\sstroutine{
   IRA\_GTSCS
}{
   Get the full name of a Sky Coordinate System (with equinox
   specifier) from the environment
}{
   \sstdescription{
      The ADAM parameter specified by argument SCSPAR is used to get a
      character string from the environment. A check is done to make
      sure that the string obtained represents a supported Sky
      Coordinate System (SCS). The user may include an equinox
      specifier (see IRA\_ISCS) in the text string to override the
      default reference equinox of B1950. If an illegal SCS name is
      entered the user is reprompted. If DEFLT is given true, the value
      of SCS on entry is used as a default for the parameter.  The
      value of SCS is expanded (both on entry and exit) to a full SCS
      name (an abbreviation of the SCS may be supplied either by the
      calling routine or by the user instead of the full name).
   }
   \sstinvocation{
      CALL IRA\_GTSCS( SCSPAR, DEFLT, SCS, STATUS )
   }
   \sstarguments{
      \sstsubsection{
         SCSPAR = CHARACTER $*$ ( $*$ ) (Given)
      }{
         The name of the ADAM parameter to use, which should be of type
         LITERAL.
      }
      \sstsubsection{
         DEFLT = LOGICAL (Given)
      }{
         If true, then the value of argument SCS on entry is used (after
         expansion) as the run-time default for the parameter.
      }
      \sstsubsection{
         SCS = CHARACTER $*$ ( $*$ ) (Given and Returned)
      }{
         On entry, specifies the default value for the parameter. On
         exit, contains the full version of the sky coordinate system
         entered by the user. The supplied variable should have a
         declared length given by symbolic constant IRA\_\_SZSCS.
      }
      \sstsubsection{
         STATUS = INTEGER (Given and Returned)
      }{
         The global status.
      }
   }
}
\sstroutine{
   IRA\_IMPRT
}{
   Get an identifier for astrometry information stored in an NDF
}{
   \sstdescription{
      A search is made for an astrometry structure through all the
      extensions in the supplied NDF. If one is found, the astrometry
      information is copied into internal common blocks and an {\tt "}IRA
      identifier{\tt "} is returned which can be passed to other IRA routines
      to refer to the stored astrometry information.  This identifier
      should be annulled when it is no longer required by calling
      IRA\_ANNUL. An error is reported if no astrometry structure is
      found in the NDF.
   }
   \sstinvocation{
      CALL IRA\_IMPRT( INDF, IDA, STATUS )
   }
   \sstarguments{
      \sstsubsection{
         INDF = INTEGER (Given)
      }{
         The identifier for the NDF containing the astrometry
         information
      }
      \sstsubsection{
         IDA = INTEGER (Returned)
      }{
         The IRA identifier which is used by other IRA routines to
         access the astrometry information copied from the NDF.
      }
      \sstsubsection{
         STATUS = INTEGER (Given and Returned)
      }{
         The global status.
      }
   }
}
\sstroutine{
   IRA\_INIT
}{
   Initialise the IRA astrometry package
}{
   \sstdescription{
      This routine must be called before calling any other IRA routine
      which has an {\tt "}IDA{\tt "} argument. It is not necessary to call this
      routine before using routines such as IRA\_DIST which do not have
      an {\tt "}IDA{\tt "} argument. This routine annulls any currently valid IRA
      identifiers, sets the NDF extension name in which the astrometry
      structure is located to {\tt "}IRAS{\tt "}, sets the name of the astrometry
      structure to {\tt "}ASTROMETRY{\tt "}, and resets graphics options to their
      default values.
   }
   \sstinvocation{
      CALL IRA\_INIT( STATUS )
   }
   \sstarguments{
      \sstsubsection{
         STATUS = INTEGER (Given and Returned)
      }{
         The global status.
      }
   }
}
\sstroutine{
   IRA\_IPROJ
}{
   Return a list of supported projection names
}{
   \sstdescription{
      A string is returned containing the list of supported projection
      names and equivalent names. The names are separated by commas. The
      currently supported projections are:

      GNOMONIC ( or equivalently TANGENT\_PLANE )

      LAMBERT (  or equivalently CYLINDRICAL )

      AITOFF (  or equivalently ALL\_SKY )

      ORTHOGRAPHIC

      See ID/2 appendix {\tt "}Projection Equations{\tt "} for more details about
      the supported projections.
   }
   \sstinvocation{
      CALL IRA\_IPROJ( LIST, STATUS )
   }
   \sstarguments{
      \sstsubsection{
         LIST = CHARACTER $*$ ( $*$ ) (Returned)
      }{
         The list of supported projections and equivalent names. The
         character variable supplied for this argument should have a
         declared size equal to the value of parameter IRA\_\_SZPLS. If
         the supplied string is not long enough to hold all the names, a
         warning message is given, but no error status is returned. Each
         returned projection name has a maximum length given by symbolic
         constant IRA\_\_SZPRJ.
      }
      \sstsubsection{
         STATUS = INTEGER (Given and Returned)
      }{
         The global status.
      }
   }
}
\sstroutine{
   IRA\_ISCS
}{
   Return a list of supported sky coordinate systems
}{
   \sstdescription{
      A string is returned containing a list of names identifying the
      supported sky coordinate systems. The names in the output list
      are separated by commas.

      By default, Equatorial and Ecliptic coordinates are referred to
      the mean equinox of Besselian epoch 1950.0. The calling
      application can override this default by appending a string known
      as an {\tt "}equinox specifier{\tt "} to the end of the SCS name (in fact all
      IRA routines will accept any unambiguous abbreviation of the SCS
      name). An equinox specifier consists of a year with upto 4
      decimal places, preceded with the letter B or J to indicate a
      Besselian or Julian epoch, and enclosed in parentheses. The
      named coordinate system is then referred to the mean equinox of
      the epoch given in the equinox specifier. The following are
      examples of legal SCS values; EQUATORIAL(B1950), EQUAT(J2000),
      ECLIP, ECLIP(1983.2534), etc. If the date is not preceded with
      either B or J (as in the last example), a Besselian epoch is
      assumed.

      The currently supported sky coordinate systems are:

      EQUATORIAL

             The longitude axis is Right Ascension, the latitude axis
             is Declination. Other legal names can be made by appending
             an equinox specifier (eg EQUATORIAL(B1983.5) ). If no
             equinox specifier is added, the coordinates are referred
             to the mean equinox of Besselian epoch 1950.0. If the
             equinox is described by a Besselian epoch, the old FK4
             Bessel-Newcomb system is used. If a Julian epoch is used,
             the new IAU 1976, FK5, Fricke system is used.

      GALACTIC

             The longitude axis is galactic longitude and the latitude
             axis is galactic latitude, given in the IAU 1958 galactic
             coordinate system.

      ECLIPTIC

             The longitude axis is ecliptic longitude and the latitude
             axis is ecliptic latitude. Other legal names can be made
             by appending an equinox specifier (eg ECLIPTIC(B1983.5) ).
             If no equinox specifier is added, the coordinates are
             referred to the mean equinox of Besselian epoch 1950.0.

      All sky coordinate values supplied to, or returned from any IRA
      routine, are given in units of radians.
   }
   \sstinvocation{
      CALL IRA\_ISCS( LIST, STATUS )
   }
   \sstarguments{
      \sstsubsection{
         LIST = CHARACTER $*$ ( $*$ ) (Returned)
      }{
         The list of supported sky coordinate system names. The
         character variable supplied for this argument should have a
         declared size equal to the value of parameter IRA\_\_SZCLS. If
         the supplied string is not long enough to hold all the names, a
         warning message is given, but no error status is returned.
      }
      \sstsubsection{
         STATUS = INTEGER (Given and Returned)
      }{
         The global status.
      }
   }
}
\sstroutine{
   IRA\_LOCAT
}{
   Set the location for new IRA astrometry structures
}{
   \sstdescription{
      By default, IRA\_CREAT and IRA\_EXPRT store astrometry information
      in a component named {\tt "}ASTROMETRY{\tt "} within the {\tt "}IRAS{\tt "} NDF
      extension.  These names may be changed if necessary by calling
      this routine. The supplied arguments give the name of the NDF
      extension and the component name to be used by all future calls
      to IRA\_CREAT or IRA\_EXPRT. It should be ensured that the
      extension exists before calling IRA\_CREAT or IRA\_EXPRT.
   }
   \sstinvocation{
      CALL IRA\_LOCAT( XNAME, ASNAME, STATUS )
   }
   \sstarguments{
      \sstsubsection{
         XNAME = CHARACTER $*$ ( $*$ ) (Given)
      }{
         The name of NDF extension in which astrometry structures are to
         be created. If a blank value is supplied the current value is
         left unchanged.
      }
      \sstsubsection{
         ASNAME = CHARACTER $*$ ( $*$ ) (Given)
      }{
         The name of a component of the NDF extension in which to store
         astrometry information. If a blank value is supplied the
         current value is left unchanged.
      }
      \sstsubsection{
         STATUS = INTEGER (Given and Returned)
      }{
         The global status.
      }
   }
}
\sstroutine{
   IRA\_MAG
}{
   Corrects astrometry information for magnification of image
   coordinates
}{
   \sstdescription{
      This routine modifies the astrometry information identified by
      IDA to incorporate the effect of magnification of the image
      coordinates about an arbitrary centre. Such a transformation
      preserves the intrinsic properties of the original projection.

      Note, once this routine has been called, IRA\_EXPRT or IRA\_WRITE
      should be called if required to save the modified astrometry
      information in an NDF or HDS object.
   }
   \sstinvocation{
      CALL IRA\_MAG( IDA, MAGX, MAGY, XC, YC, STATUS )
   }
   \sstarguments{
      \sstsubsection{
         IDA = INTEGER (Given)
      }{
         An IRA identifier for the astrometry information to be
         modified.
      }
      \sstsubsection{
         MAGX = DOUBLE PRECISION (Given)
      }{
         The factor by which the pixel size in the X direction has been
         decreased (i.e. a value of 2.0 results is given if the pixel
         size has been halved).
      }
      \sstsubsection{
         MAGY = DOUBLE PRECISION (Given)
      }{
         The factor by which the pixel size in the Y direction has been
         decreased.
      }
      \sstsubsection{
         XC = DOUBLE PRECISION (Given)
      }{
         The X image coordinate of the centre of magnification.
      }
      \sstsubsection{
         YC = DOUBLE PRECISION (Given)
      }{
         The Y image coordinate of the centre of magnification.
      }
      \sstsubsection{
         STATUS = INTEGER (Given and Returned)
      }{
         The global status.
      }
   }
}
\sstroutine{
   IRA\_MOVE
}{
   Corrects astrometry information for shift of image coordinates
}{
   \sstdescription{
      This routine modifies the astrometry information identified by
      IDA to incorporate the effect of a shift of the origin of image
      coordinates. Such a transformation preserves the intrinsic
      properties of the original projection.

      Note, once this routine has been called, IRA\_EXPRT or IRA\_WRITE
      should be called if required to save the modified astrometry
      information in an NDF or HDS object.
   }
   \sstinvocation{
      CALL IRA\_MOVE( IDA, SHIFTX, SHIFTY, STATUS )
   }
   \sstarguments{
      \sstsubsection{
         IDA = INTEGER (Given)
      }{
         An IRA identifier for the astrometry information to be
         modified.
      }
      \sstsubsection{
         SHIFTX = DOUBLE PRECISION (Given)
      }{
         The shift in X of the origin of the image coordinates, in
         pixels.
      }
      \sstsubsection{
         SHIFTY = DOUBLE PRECISION (Given)
      }{
         The shift in Y of the origin of the image coordinates, in
         pixels.
      }
      \sstsubsection{
         STATUS = INTEGER (Given and Returned)
      }{
         The global status.
      }
   }
}
\sstroutine{
   IRA\_NORM
}{
   Convert sky coordinate values to the equivalent first order
   values
}{
   \sstdescription{
      The given latitude value is shifted into the range $+$/- PI/2 ( a
      shift of PI may be introduced in the longitude value to achieve
      this). The longitude value is then shifted into the range 0 to
      2$*$PI. If either A or B has the Starlink {\tt "}BAD{\tt "} value on entry
      (VAL\_\_BADD) then both A and B are left unchanged on exit. Latitude
      values which are within 0.01 arc-seconds of either pole are
      modified to put them exactly at the pole.
   }
   \sstinvocation{
      CALL IRA\_NORM( A, B, STATUS )
   }
   \sstarguments{
      \sstsubsection{
         A = DOUBLE PRECISION (Given and Returned)
      }{
         The longitude, in radians. On exit, the value is shifted in to
         the range 0 to 2$*$PI.
      }
      \sstsubsection{
         B = DOUBLE PRECISION (Given and Returned)
      }{
         The latitude value, in radians. On exit, the value is shifted
         in to the range -PI/2 to $+$PI/2.
      }
      \sstsubsection{
         STATUS = INTEGER (Given and Returned)
      }{
         The global status.
      }
   }
}
\sstroutine{
   IRA\_OFFST
}{
   Find a sky position which is offset towards a given point
}{
   \sstdescription{
      This routine finds the sky position which is offset away from a
      specified {\tt "}reference{\tt "} position by a given arc distance, along the
      great circle joining the reference position with a specified
      outlying sky position. If any of the input values have the
      Starlink {\tt "}BAD{\tt "} value (VAL\_\_BADD) then both output coordinate
      values will also be bad.
   }
   \sstinvocation{
      CALL IRA\_OFFST( A0, B0, A1, B1, DIST, A2, B2, STATUS )
   }
   \sstarguments{
      \sstsubsection{
         A0 = DOUBLE PRECISION (Given)
      }{
         The sky longitude of the reference position, in radians.
      }
      \sstsubsection{
         B0 = DOUBLE PRECISION (Given)
      }{
         The sky latitude of the reference position, in radians.
      }
      \sstsubsection{
         A1 = DOUBLE PRECISION (Given)
      }{
         The sky longitude of the outlying position, in radians.
      }
      \sstsubsection{
         B1 = DOUBLE PRECISION (Given)
      }{
         The sky longitude of the outlying position, in radians.
      }
      \sstsubsection{
         DIST = DOUBLE PRECISION (Given)
      }{
         The arc distance to move away from the reference position,
         towards the outlying position, in radians.
      }
      \sstsubsection{
         A2 = DOUBLE PRECISION (Returned)
      }{
         The sky longitude of the position which is the given arc
         distance away from the reference position in the direction of
         the outlying position, in radians.
      }
      \sstsubsection{
         B2 = DOUBLE PRECISION (Returned)
      }{
         The sky latitude of the position which is the given arc
         distance away from the reference position in the direction of
         the outlying position, in radians.
      }
      \sstsubsection{
         STATUS = INTEGER (Given and Returned)
      }{
         The global status.
      }
   }
}
\sstroutine{
   IRA\_PACON
}{
   Convert sky coordinates and position angles from one SCS to
   another
}{
   \sstdescription{
      This routine use SLALIB to convert a list of sky coordinates from
      one supported sky coordinate system to any other supported
      system, and also converts a position angle at each position
      (position angles are relative to {\tt "}north{\tt "}, but north changes from
      SCS to SCS, and therefore position angles also change). It is
      assumed that the observations were made at the date given by the
      Julian epoch supplied.  If any of the AIN, BIN or PAIN values are
      equal to the Starlink {\tt "}BAD{\tt "} value (VAL\_\_BADD) then the
      corresponding output AOUT, BOUT and PAOUT values will all be set
      to the bad value.
   }
   \sstinvocation{
      CALL IRA\_PACON( NVAL, AIN, BIN, PAIN, SCSIN, SCSOUT, EPOCH, AOUT,
                      BOUT, PAOUT, STATUS )
   }
   \sstarguments{
      \sstsubsection{
         NVAL = INTEGER (Given)
      }{
         The number of sky coordinate pairs to be converted.
      }
      \sstsubsection{
         AIN( NVAL ) = DOUBLE PRECISION (Given)
      }{
         A list of first sky coordinate values to be converted, in
         radians.
      }
      \sstsubsection{
         BIN( NVAL ) = DOUBLE PRECISION (Given)
      }{
         A list of second sky coordinate values to be converted, in
         radians.
      }
      \sstsubsection{
         PAIN( NVAL ) = DOUBLE PRECISION (Given)
      }{
         A list of position angles to be converted, in radians. A
         position angle is an angle from north, measured positive in
         the sense of rotation from north to east.  Conversion of
         position angle depends on the point on the celestial sphere at
         which the position angle is measured. Each position angle is
         assumed to be measured at the corresponding position given by
         AIN and BIN.
      }
      \sstsubsection{
         SCSIN = CHARACTER $*$ ( $*$ ) (Given)
      }{
         A string holding the name of the sky coordinate system of the
         input list  (see ID2 section {\tt "}Sky Coordinates{\tt "}). Any
         unambiguous abbreviation will do.
      }
      \sstsubsection{
         SCSOUT = CHARACTER $*$ ( $*$ ) (Given)
      }{
         A string holding the name of the sky coordinate system required
         for the output list. Any unambiguous abbreviation will do.
      }
      \sstsubsection{
         EPOCH = DOUBLE PRECISION (Given)
      }{
         The Julian epoch at which the observations were made. When
         dealing with IRAS data, the global constant IRA\_\_IRJEP should
         be specified. This constant is a Julian epoch suitable for all
         IRAS data.
      }
      \sstsubsection{
         AOUT( NVAL ) = DOUBLE PRECISION (Returned)
      }{
         The list of converted sky longitude values, in radians.
      }
      \sstsubsection{
         BOUT( NVAL ) = DOUBLE PRECISION (Returned)
      }{
         The list of converted sky latitude values, in radians.
      }
      \sstsubsection{
         PAOUT( NVAL ) = DOUBLE PRECISION (Given)
      }{
         The list of converted position angles, in radians.
      }
      \sstsubsection{
         STATUS = INTEGER (Given and Returned)
      }{
         The global status.
      }
   }
}
\sstroutine{
   IRA\_PIXSZ
}{
   Get nominal pixel size
}{
   \sstdescription{
      The nominal pixel dimensions stored in the astrometry information
      identified by IDA are returned.
   }
   \sstinvocation{
      CALL IRA\_PIXSZ( IDA, PIXSIZ, STATUS )
   }
   \sstarguments{
      \sstsubsection{
         IDA = INTEGER (Given)
      }{
         An IRA identifier for the astrometry information.
      }
      \sstsubsection{
         PIXSIZ( 2 ) = DOUBLE PRECISION (Returned)
      }{
         The nominal pixel dimensions, in radians.
      }
      \sstsubsection{
         STATUS = INTEGER (Given and Returned)
      }{
         The global status.
      }
   }
}
\sstroutine{
   IRA\_READ
}{
   Get an identifier for astrometry information stored in an HDS
   astrometry structure
}{
   \sstdescription{
      An attempt is made to read astrometry information from the
      supplied HDS object, assuming the object is an IRA astrometry
      structure. The astrometry information is copied into internal
      common blocks and an {\tt "}IRA identifier{\tt "} is returned which can be
      passed to other IRA routines to refer to the stored astrometry
      information.  This identifier should be annulled when it is no
      longer required by calling IRA\_ANNUL.
   }
   \sstinvocation{
      CALL IRA\_READ( LOC, IDA, STATUS )
   }
   \sstarguments{
      \sstsubsection{
         LOC = CHARACTER $*$ ( $*$ ) (Given)
      }{
         An HDS locator to an astrometry structure. The constant
         IRA\_\_HDSTY gives the HDS type required for this object.
      }
      \sstsubsection{
         IDA = INTEGER (Returned)
      }{
         The IRA identifier which is used by other IRA routines to
         access the astrometry information.
      }
      \sstsubsection{
         STATUS = INTEGER (Given and Returned)
      }{
         The global status.
      }
   }
}
\sstroutine{
   IRA\_ROT
}{
   Corrects astrometry information for rotation of image coordinates
}{
   \sstdescription{
      This routine modifies the astrometry information identified by
      IDA to incorporate the effect of a rotation of the image
      coordinates about an arbitrary centre. Such a transformation
      preserves the intrinsic properties of the original projection.

      Note, once this routine has been called, IRA\_EXPRT or IRA\_WRITE
      should be called if required to save the modified astrometry
      information in an NDF or HDS object.
   }
   \sstinvocation{
      CALL IRA\_ROT( IDA, ROT, XC, YC, STATUS )
   }
   \sstarguments{
      \sstsubsection{
         IDA = INTEGER (Given)
      }{
         An IRA identifier for the astrometry information to be
         modified.
      }
      \sstsubsection{
         ROT = DOUBLE PRECISION (Given)
      }{
         The angle through which the image coordinate frame has been
         rotated, in radians. Positive rotation is in the same sense as
         rotation from north to east.
      }
      \sstsubsection{
         XC = DOUBLE PRECISION (Given)
      }{
         The X image coordinate of the centre of rotation.
      }
      \sstsubsection{
         YC = DOUBLE PRECISION (Given)
      }{
         The Y image coordinate of the centre of rotation.
      }
      \sstsubsection{
         STATUS = INTEGER (Given and Returned)
      }{
         The global status.
      }
   }
}
\sstroutine{
   IRA\_SCNAM
}{
   Get full name and abbreviation of a sky coordinate
}{
   \sstdescription{
      A full description of the name of the longitude or latitude axis
      of the given sky coordinate system is returned, together with an
      abbreviation of the name. See the {\tt "}Notes{\tt "} section below for a
      list of the values returned.
   }
   \sstinvocation{
      CALL IRA\_SCNAM( SCS, NC, DESCR, LD, ABBREV, LA, STATUS )
   }
   \sstarguments{
      \sstsubsection{
         SCS = CHARACTER $*$ ( $*$ ) (Given)
      }{
         The sky coordinate system. Any unambiguous abbreviation will
         do.
      }
      \sstsubsection{
         NC = INTEGER (Given)
      }{
         The axis required; 1 for the longitude axis, 2 for the latitude
         axis.
      }
      \sstsubsection{
         DESCR = CHARACTER $*$ ( $*$ ) (Returned)
      }{
         A full description of the axis name (see {\tt "}Notes{\tt "} below). This
         should have a declared length of IRA\_\_SZSCD.
      }
      \sstsubsection{
         LD = INTEGER (Returned)
      }{
         No. of used characters in DESCR.
      }
      \sstsubsection{
         ABBREV = CHARACTER $*$ ( $*$ ) (Returned)
      }{
         An abbreviation for the axis name (see {\tt "}Notes{\tt "} below). This
         should have a declared length of IRA\_\_SZSCA.
      }
      \sstsubsection{
         LA = INTEGER (Returned)
      }{
         No. of used characters in ABBREV.
      }
      \sstsubsection{
         STATUS = INTEGER (Given and Returned)
      }{
         Status value.
      }
      \sstsubsection{
         Notes:
      }{
         \sstitemlist{

            \sstitem
            For equatorial coordinates, the descriptions are {\tt "}Right
            Ascension (epoch){\tt "} and {\tt "}Declination (epoch){\tt "}, and the
            abbreviations are {\tt "}RA{\tt "} and {\tt "}DEC{\tt "}.

            \sstitem
            For galactic coordinates, the descriptions are {\tt "}Galactic
            Longitude{\tt "} and {\tt "}Galactic Latitude{\tt "}, and the abbreviations are
            {\tt "}l{\tt "} and {\tt "}b{\tt "}.

            \sstitem
            For ecliptic coordinates, the descriptions are {\tt "}Ecliptic
            Longitude (epoch){\tt "} and {\tt "}Ecliptic Latitude (epoch){\tt "}, and the
            abbreviations are {\tt "}Lambda{\tt "} and {\tt "}Beta{\tt "}.
         }
      }
   }
}
\sstroutine{
   IRA\_SCSEP
}{
   Get the sky coordinate system and epoch associated with an IRA
   identifier
}{
   \sstdescription{
      The sky coordinate system and Julian epoch of observation
      identified by IDA are returned.
   }
   \sstinvocation{
      CALL IRA\_SCSEP( IDA, SCS, EPOCH, STATUS )
   }
   \sstarguments{
      \sstsubsection{
         IDA = INTEGER (Given)
      }{
         An IRA identifier for the astrometry information.
      }
      \sstsubsection{
         SCS = CHARACTER $*$ ( $*$ ) (Returned)
      }{
         The sky coordinate system to which the projection specified by
         IDA refers (see ID2 section {\tt "}Sky Coordinates{\tt "}).
      }
      \sstsubsection{
         EPOCH = DOUBLE PRECISION (Returned)
      }{
         The Julian epoch at which the observations were made.
      }
      \sstsubsection{
         STATUS = INTEGER (Given and Returned)
      }{
         The global status.
      }
   }
}
\sstroutine{
   IRA\_SETEQ
}{
   Encode the epoch of a reference equinox within an SCS name
}{
   \sstdescription{
      On entry, SCS contains the name of a Sky Coordinate System (or an
      unambiguous abbreviation), with or without an equinox specifier
      (see routine IRA\_ISCS). On exit, SCS contains the full name of
      the Sky Coordinate System with an equinox specifier appended,
      determined by arguments EQU and BJ. Any old equinox specifier is
      first removed. If the Sky Coordinate System is not referred to
      the equinox (eg GALACTIC) then no equinox specifier is included
      in SCS on exit.
   }
   \sstinvocation{
      CALL IRA\_SETEQ( EQU, BJ, SCS, STATUS )
   }
   \sstarguments{
      \sstsubsection{
         EQU = DOUBLE PRECISION (Given)
      }{
         The epoch of the reference equinox. After calling this routine,
         the sky coordinates described by SCS are referred to the mean
         equinox of the epoch given by EQU. If EQU has the Starlink
         {\tt "}BAD{\tt "} value (VAL\_\_BADD) then no equinox specifier is included
         in SCS on exit.
      }
      \sstsubsection{
         BJ = CHARACTER $*$ ( $*$ ) (Given)
      }{
         Determines if the epoch specified by argument EQU is a
         Besselian or Julian epoch. BJ should have the value {\tt "}B{\tt "} or {\tt "}J{\tt "}.
         Any other value causes an error report (except that a blank
         value causes {\tt "}B{\tt "} to be used).
      }
      \sstsubsection{
         SCS = CHARACTER $*$ ( $*$ ) (Given and Returned)
      }{
         On entry, SCS should contain an unambiguous abbreviation of a
         supported Sky Coordinate System (see routine IRA\_ISCS), with or
         without an equinox specifier. On exit, SCS contains the full
         name of the Sky Coordinate System, appended with an equinox
         specifier determined by arguments EQU and BJ. If the Sky
         Coordinate System is not one that is referred to the equinox
         (eg GALACTIC) then no equinox specifier is included in SCS on
         exit. SCS should have a declared length equal to the symbolic
         constant IRA\_\_SZSCS.
      }
      \sstsubsection{
         STATUS = INTEGER (Given and Returned)
      }{
         The global status.
      }
   }
}
\sstroutine{
   IRA\_SHIFT
}{
   Find a sky position which is offset along a given position angle
}{
   \sstdescription{
      This routine finds the sky position which is offset away from a
      specified {\tt "}reference{\tt "} position by a given arc distance, along a
      line which has a given position angle.  Note, in this context a
      {\tt "}line{\tt "} is actually a great circle on the celestial sphere.  This
      means for instance, that a line going due east or west from any
      point, for a distance of PI/2 radians, will always end on the
      equator. If any of the input values have the Starlink {\tt "}BAD{\tt "} value
      (VAL\_\_BADD) then both output coordinate values will also be bad.
      The position angle of a great circle varies along its length. The
      position angle of the requested great circle at the returned
      position is returned in argument ENDANG.
   }
   \sstinvocation{
      CALL IRA\_SHIFT( A0, B0, ANGLE, DIST, A1, B1, ENDANG, STATUS )
   }
   \sstarguments{
      \sstsubsection{
         A0 = DOUBLE PRECISION (Given)
      }{
         The sky longitude of the reference position, in radians.
      }
      \sstsubsection{
         B0 = DOUBLE PRECISION (Given)
      }{
         The sky latitude of the reference position, in radians.
      }
      \sstsubsection{
         ANGLE = DOUBLE PRECISION (Given)
      }{
         The position angle of the line (actually a great circle) going
         from the given reference position, to the required position.
         That is, the angle from north to the required direction, in
         radians. Positive angles are in the sense of rotation from
         north to east.
      }
      \sstsubsection{
         DIST = DOUBLE PRECISION (Given)
      }{
         The arc distance to move away from the reference position
         in the given direction, in radians.
      }
      \sstsubsection{
         A1 = DOUBLE PRECISION (Returned)
      }{
         The sky longitude of the required point, in radians.
      }
      \sstsubsection{
         B1 = DOUBLE PRECISION (Returned)
      }{
         The sky latitude of the required point, in radians.
      }
      \sstsubsection{
         ENDANG = DOUBLE PRECISION (Returned)
      }{
         The position angle of the line (actually great circle) as seen
         from the the required point. In general, this will not be the
         same as the value given in ANGLE, especially for large values
         of DIST, or positions close to the poles.  ENDANG is measured
         from north to the great circle, in radians.  Positive values
         are in the sense of rotation from north to east.
      }
      \sstsubsection{
         STATUS = INTEGER (Given and Returned)
      }{
         The global status.
      }
   }
}
\sstroutine{
   IRA\_TRACE
}{
   Display astrometry information
}{
   \sstdescription{
      This routine displays the astrometry information identified by
      IDA, using the supplied routine to display each line. The
      displayed information depends on the current IRA implementation,
      but will include at least a description of the sky coordinate
      system and reference equinox and probably other items as well.
   }
   \sstinvocation{
      CALL IRA\_TRACE( IDA, ROUTNE, STATUS )
   }
   \sstarguments{
      \sstsubsection{
         IDA = INTEGER (Given)
      }{
         The IRA identifier for the astrometry information.
      }
      \sstsubsection{
         ROUTNE = EXTERNAL (Given)
      }{
         A routine to which is passed each line of text for display.
         It should have the same argument list as MSG\_\_OUTIF (see
         SUN/104), and should be declared EXTERNAL in the calling
         routine. All calls to this routine are made with a priority
         of MSG\_\_NORM.
      }
      \sstsubsection{
         STATUS = INTEGER (Given and Returned)
      }{
         The global status.
      }
   }
}
\sstroutine{
   IRA\_TRANS
}{
   Transform coordinate data
}{
   \sstdescription{
      Coordinate data are transformed from sky coordinates to image
      coordinates, or vice-versa, using the projection information
      identified by IDA. The direction of the transformation is
      determined by the argument FORWRD.  If any input coordinate
      values are equal to the Starlink {\tt "}BAD{\tt "} value (VAL\_\_BADD) then
      both the output values are set to the bad value.
   }
   \sstinvocation{
      CALL IRA\_TRANS( NVAL, IN1, IN2, FORWRD, SCS, IDA,
                      OUT1, OUT2, STATUS )
   }
   \sstarguments{
      \sstsubsection{
         NVAL = INTEGER (Given)
      }{
         The number of coordinate points to be transformed.
      }
      \sstsubsection{
         IN1( NVAL ) = DOUBLE PRECISION (Given)
      }{
         If FORWRD is true, then IN1 holds values of the first image
         coordinate (X), otherwise IN1 holds values of the sky
         longitude.
      }
      \sstsubsection{
         IN2( NVAL ) = DOUBLE PRECISION (Given)
      }{
         If FORWRD is true, then IN2 holds values of the second image
         coordinate (Y), otherwise IN2 holds values of the sky
         latitude.
      }
      \sstsubsection{
         FORWRD = LOGICAL (Given)
      }{
         If true then the forward mapping is used from image coordinate
         to sky coordinate. Otherwise, the inverse mapping from sky
         coordinate to image coordinates is used.
      }
      \sstsubsection{
         SCS = CHARACTER $*$ ( $*$ ) (Given)
      }{
         The name of the sky coordinate system in which sky coordinates
         are required (if FORWRD is true), or supplied (if FORWRD is
         false). Any unambiguous abbreviation will do. This need not be
         the same as the SCS identified by IDA.  See ID2 section {\tt "}Sky
         Coordinates{\tt "} for more information on Sky Coordinate Systems.
      }
      \sstsubsection{
         IDA = INTEGER (Given)
      }{
         The IRA identifier for the astrometry information.
      }
      \sstsubsection{
         OUT1( NVAL ) = DOUBLE PRECISION (Returned)
      }{
         If FORWRD is true, then OUT1 holds values of the sky longitude
         corresponding to the image coordinates given in arrays IN1 and
         IN2. Otherwise, OUT1 holds values of the first image
         coordinate (X) corresponding to the input sky coordinates.
      }
      \sstsubsection{
         OUT2( NVAL ) = DOUBLE PRECISION (Returned)
      }{
         If FORWRD is true, then OUT2 holds values of the sky latitude
         corresponding to the image coordinates given in arrays IN1 and
         IN2. Otherwise, OUT2 holds values of the second image
         coordinate (Y) corresponding to the input sky coordinates.
      }
      \sstsubsection{
         STATUS = INTEGER (Given and Returned)
      }{
         The global status.
      }
   }
}
\sstroutine{
   IRA\_VALID
}{
   Check for valid coordinate data
}{
   \sstdescription{
      This routine can be used to check whether or not given input
      coordinates would transform to valid output coordinates if
      transformed using IRA\_TRANS. This is faster than actually doing
      the transformation using IRA\_TRANS.
   }
   \sstinvocation{
      CALL IRA\_VALID( NVAL, FORWRD, SCS, IDA, IN1, IN2, OK, STATUS )
   }
   \sstarguments{
      \sstsubsection{
         NVAL = INTEGER (Given)
      }{
         The number of coordinate points to be transformed.
      }
      \sstsubsection{
         FORWRD = LOGICAL (Given)
      }{
         If true then the forward mapping is used from image coordinate
         to sky coordinate. Otherwise, the inverse mapping from sky
         coordinate to image coordinates is used.
      }
      \sstsubsection{
         SCS = CHARACTER $*$ ( $*$ ) (Given)
      }{
         The name of the sky coordinate system for input sky coordinate
         positions. Any unambiguous abbreviation will do. This need not
         be the same as the SCS identified by IDA. If FORWRD is true,
         SCS is ignored, since if the given image coordinates transform
         to a valid position in one sky coordinate system, then they
         will transform to valid positions in all sky coordinate
         systems (the reverse is not true since some projections do not
         cover the entire sky).  See ID2 section {\tt "}Sky Coordinates{\tt "} for
         more information on Sky Coordinate Systems.
      }
      \sstsubsection{
         IDA = INTEGER (Given)
      }{
         The IRA identifier for the astrometry information.
      }
      \sstsubsection{
         IN1( NVAL ) = DOUBLE PRECISION (Given and Returned)
      }{
         If FORWRD is true, then IN1 holds values of the first image
         coordinate (X) on entry, and the corresponding U values on
         exit. Otherwise IN1 holds values of the sky longitude on entry
         and the corresponding local longitude on exit. See ID/2
         appendix D for a description of the local sky coordinate
         system, and the (U,V) coordinate system.
      }
      \sstsubsection{
         IN2( NVAL ) = DOUBLE PRECISION (Given and Returned)
      }{
         If FORWRD is true, then IN2 holds values of the second image
         coordinate (Y) on entry, and the corresponding V values on
         exit. Otherwise IN2 holds values of the sky latitude on entry
         and the corresponding local latitude on exit.
      }
      \sstsubsection{
         OK( NVAL ) = LOGICAL (Returned)
      }{
         True if the corresponding input coordinates would transform to
         valid output coordinates.
      }
      \sstsubsection{
         STATUS = INTEGER (Given and Returned)
      }{
         The global status.
      }
   }
}
\sstroutine{
   IRA\_WRITE
}{
   Write astrometry information into an HDS object
}{
   \sstdescription{
      The astrometry information identified by IDA is stored within the
      object located by argument LOC. The constant IRA\_\_HDSTY gives the
      HDS type required for the object. The object must have this type
      and be empty.
   }
   \sstinvocation{
      CALL IRA\_WRITE( IDA, LOC, STATUS )
   }
   \sstarguments{
      \sstsubsection{
         IDA = INTEGER (Given)
      }{
         An IRA identifier for the astrometry information.
      }
      \sstsubsection{
         LOC = CHARACTER $*$ ( $*$ ) (Given)
      }{
         A locator to the object to which the astrometry information is
         to be written.
      }
      \sstsubsection{
         STATUS = INTEGER (Given and Returned)
      }{
         The global status.
      }
   }
}
\sstroutine{
   IRA\_XYLIM
}{
   Find image coordinate bounds which encloses a given area of sky
}{
   \sstdescription{
      A box in image coordinates is specified by giving the sky
      coordinates corresponding to the centre of the box, and the
      arc-length of each dimension of the box, parallel to the image X
      and Y axes. The bounds of this box in image coordinates are
      returned. The effects of varying pixelk size are taken into
      account. Note, if the box extends beyond the edge of the
      projection (as may happen, for instance, if the box centre is
      placed close to the elliptical boundary of an Aitoff projection)
      then the box is truncated at the edge of the projection, but no
      error is reported. In this case, the returned bounds will not
      represent a box of the requested dimensions, but will be smaller.
   }
   \sstinvocation{
      CALL IRA\_XYLIM( IDA, ACEN, BCEN, XSIZE, YSIZE, LBND, UBND,
                      STATUS )
   }
   \sstarguments{
      \sstsubsection{
         IDA = INTEGER (Given)
      }{
         An IRA identifier for the astrometry information.
      }
      \sstsubsection{
         ACEN = DOUBLE PRECISION (Given)
      }{
         The longitude at the centre of the box.
      }
      \sstsubsection{
         BCEN = DOUBLE PRECISION (Given)
      }{
         The latitude at the centre of the box.
      }
      \sstsubsection{
         XSIZE = DOUBLE PRECISION (Given)
      }{
         The arc-length of the image X axis, in radians.
      }
      \sstsubsection{
         YSIZE = DOUBLE PRECISION (Given)
      }{
         The arc-length of the image Y axis, in radians.
      }
      \sstsubsection{
         LBND( 2 ) = DOUBLE PRECISION (Returned)
      }{
         The lower bounds of the X and Y image axes, in image
         coordinates.
      }
      \sstsubsection{
         UBND( 2 ) = DOUBLE PRECISION (Returned)
      }{
         The upper bounds of the X and Y image axes, in image
         coordinates.
      }
      \sstsubsection{
         STATUS = INTEGER (Given and Returned)
      }{
         The global status.
      }
   }
}
\section {Templates for IRA Routines Within the VAX LSE Editor}
The STARLSE package (see SUN/105) provides facilities for initialising the VAX
Language Sensitive Editor (LSE) to simplify the generation of Fortran
code conforming to the Starlink programming standard (see SGP/16). One of the
facilities provided by LSE is the automatic production of argument lists for
subroutine calls. Templates for all the subroutines in the IRA package can be
made available within LSE by performing the following steps (within LSE):
\begin{enumerate}
\item Issue the LSE command GOTO FILE/READ IRA\_DIR:IRA.LSE
\item Issue the LSE command DO
\item Issue the LSE command DELETE BUFFER
\item Move to a buffer holding a .FOR of a .GEN file in the usual way.
\item IRA subroutine templates are then available by typing in the name of an
IRA subroutine (or an abbreviation) and expanding it (CTRL-E).
\item Help on the subroutine and its arguments can be obtained by placing the
cursor at a point in the buffer at which the subroutine name has been entered
and pressing GOLD-PF2.
\end{enumerate}

\section {HDS Data Structures}
The interim implementation of IRA reads and writes externally held astrometry
information to and from HDS objects of type IRAS\_ASTROMETRY. This will be
replaced by a Starlink structure when such a structure is defined. The interim
structure contains the following components:
\begin {description}

\item [SCS] - A primitive value of type \_CHAR$*$(IRA\_\_SZSCS). This component
is mandatory and must contain the name of the sky coordinate system (including
equinox specifier) created by the forward mapping of the projection.

\item [EPOCH] - A primitive value of type \_DOUBLE. This component
is mandatory and must contain the Julian epoch of the observations described by
the astrometry structure.

\item [STATE] - A primitive value of type \_CHAR$*$(IRA\_\_SZSTA). This
component is mandatory and is used to confirm that the astrometry structure is
valid. If this component has any value other than ``DEFINED'' the astrometry
structure is considered to be invalid.

\item [PROJ\_NAME] - A primitive value of type \_CHAR$*$(IRA\_\_SZPRJ). This
component contains the name of the projection to be used when transforming
coordinate data.

\item [PROJ\_PARS] - An array of primitive values of type \_DOUBLE. The array
has a single dimension with size given by IRA\_\_MAXP, and contains values to
use for the parameters needed to defined the projection given by component
PROJ\_NAME.

\end {description}

\section {Adding New Sky Coordinate Systems and Projections}
To add new sky coordinate systems or projections to the interim implementation
of IRA, the following procedures should be followed:

\subsection {Sky Coordinate Systems}
\begin {enumerate}

\item Edit file IRA\_PAR.FOR to increase the value of IRA\_\_SZCLS by the
length of the new SCS name plus one. If necessary, also change the values of
IRA\_\_SZSCS, IRA\_\_SZFSC, IRA\_\_SZSCA, IRA\_\_SZSCD to allow for the new
name.

\item Explicit reference to the new SCS name should then be included in the
following routines; IRA1\_CHSCS, IRA1\_ICONV, IRA1\_IDTC1, IRA1\_IPACO,
IRA1\_ISCNM, IRA\_DTOC, IRA\_ISCS, IRA\_SCNAM.

\item New routines should be written to convert coordinates from each of the
supported sky coordinates systems, to the new system, and vice-versa. Calls
to these routines should be included in routines IRA1\_ICONV and IRA1\_IPACO.

\item Every routine in IRA should then be recompiled, and stored in a new object
library to replace the old object library.

\item This document (ID2) should be edited to include the new SCS name. The
SST command PROLAT should be used to regenerate the full routine specifications.

\end {enumerate}

\subsection {Projections}
\begin {enumerate}

\item Edit file IRA\_PAR.FOR to increase the value of IRA\_\_SZPLS by the
length of the new projection name plus one. If necessary, also change the values
of IRA\_\_SZPRJ and IRA\_\_MAXP to allow for the new projection.

\item A subroutine should be written to perform the forward and inverse mappings
on arrays of input coordinate data. The argument list for this routine should
match that of IRA1\_GNOM, which should be studied as an example of how such
routines should behave. It is important to make this routine as efficient as
possible, and have an accuracy of a fraction of an arc-second.

\item Subroutine IRA1\_IPRJ should then be modified to include a
call to the new subroutine.

\item A subroutine should be written to which checks if the forward and inverse
mappings would produce valid results given some input positions. This is
different to the previous subroutine in that it doesn't actually evaluate the
results of the transformation. The argument list for this routine should
match that of IRA1\_VGNOM, which should be studied as an example of how such
routines should behave. It is important to make this routine as efficient as
possible.

\item Subroutine IRA\_VALID should then be modified to include a
call to the new subroutine.

\item Explicit reference to the new projection name should then be included in
the routine IRA\_IPROJ.

\item Every routine in IRA should then be recompiled, and stored in a new object
library to replace the old object library.

\item This document (ID2) should be edited to include the new projection name.
The SST command PROLAT should be used to regenerate the full routine
specifications.

\end {enumerate}

\section {Projection Equations}
\label {APP:PROJ}

This section describes equations used to implement the forward and inverse
mappings for the Gnomonic, Lambert cylindrical, Aitoff and Orthographic
projections. The equations for the first three are derived from those contained
in the IRAS Catalogs and Atlases Explanatory Supplement (Exp. Supp.), pages X-30
to X-32, while the Orthographic projection is derived from those described in
the IPAC Report ``A User's Guide to IRAS Pointed Observation Products'',
Appendix B.

Throughout this appendix the following symbolic values will be used:

\begin{description}

\item [$(A,B)$] - the sky coordinates (longitude and latitude) of a general
point on the celestial sphere, in radians.

\item [$(X,Y)$] - the image coordinates corresponding to sky coordinates
$(A,B)$, in units of pixels.

\item [$(A_{l},B_{l})$] - the local coordinates (longitude and latitude, see
below) corresponding to sky coordinates $(A,B)$, in units of radians.

\item [$(U,V)$] - the ``$(U,V)$'' coordinates (see below) corresponding to sky
coordinates $(A,B)$, in units of radians.

\item [$(A_{0},B_{0})$] - the sky coordinate of the reference point.
These are the same as projection parameters P(1) and P(2) (see section \ref
{SEC:PARS}).

\item [$(X_{0},Y_{0})$] - the image coordinate of the reference point.
These are the same as projection parameters P(3) and P(4) (see section \ref
{SEC:PARS}).

\item [$(\Delta_{x},\Delta_{y})$] - the nominal dimensions of an image pixel
in the $X$ and $Y$ directions, as measured on the celestial sphere,in radians.
In fact, due to the distorting effect of the projection, only pixels close to
the reference point will actually have these dimensions. These are the same as
the absolute values of projection parameters P(5) and P(6).

\item [$\theta$] - The position angle of the image $Y$ axis on the celestial
sphere, measured in radians, from north through east. North is the direction of
increasing sky latitude and east is the direction of increasing sky longitude,
in the current sky coordinate system. This is the same as projection parameter
P(7). The image $X$ axis is $\pi/2$ radians to the west of the $Y$ axis.

\item [$\phi$] - The tilt of the celestial sphere prior to projection, given by
projection parameter P(8), in radians. This is positive in an anti-clockwise
direction when looking along the radius vector of the celestial sphere from the
reference point to ther centre.

\end{description}

In addition to image and sky coordinates, the IRA projection routines use two
extra coordinate systems, called the ``$(U,V)$'' coordinate system and the
``local'' coordinate system. These systems are only used internally within
IRA. Non of the user-callable routines make any reference to them. All the
projection transformations described below transform between $(U,V)$ and local
coordinates, not directly between image and sky coordinates. Conversion to and
from the various coordinate systems is performed as follows:

\begin{itemize}

\item When performing a forward mapping from image coordinates $(X,Y)$ to sky
coordinates $(A,B)$, the input image coordinates are first transformed into the
corresponding $(U,V)$ values. The forward mapping for the required projection is
then applied to these $(U,V)$ values to get the corresponding local coordinates.
Finally, these local coordinates are transformed into the corresponding sky
coordinates.

\item When performing an inverse mapping from sky coordinates $(A,B)$ to image
coordinates $(X,Y)$, the input sky coordinates are first transformed into
the corresponding local coordinate values. The inverse mapping for the required
projection is then applied to these local coordinates to get the corresponding
$(U,V)$ values. Finally, these $(U,V)$ values are transformed into the
corresponding image coordinates $(X,Y)$.

This use of local coordinates ensures that the properties of the projections are
not invalidated. For instance, The Lambert equivalent cylindrical and Aitoff
projections are both known as equal area projections, meaning that pixels have
the same size on the sky (but not the same shape) over the entire image. This is
only true if the celestial sphere is not rotated prior to projection (i.e. only
if $\phi=0$) and if the reference point is on the equator (i.e. only if
$B_{0}=0$). In order for such projections to be used for other values of $\phi$
or $B_{0}$, the sky coordinates must be transformed to a system in which the
equivalents of $\phi$ and $B_{0}$ {\em are} zero. Using the local coordinates
described above achieves this.

\end{itemize}

\subsection {The $(U,V)$ Coordinate System}
The $(U,V)$ system is a Cartesian coordinate system on the projection surface.
The origin is located at the reference point. The $V$ axis is tangential (at the
$(U,V)$ origin) to north in the local coordinate system, and the $U$ axis is
tangential to local west. Both axes are in units of radians. The transformation
between $(U,V)$ and $(X,Y)$ consists of a rotation (equal to $\theta-\phi$) to
put the image Y axis at the request position angle in sky coordinates, a shift
to put the reference point at the requested image coordinates, and a scaling to
give pixels the correct dimensions. The transformation equations are:

\begin{eqnarray*}
U & = & cos( \theta - \phi )*(X - X_{0})*\Delta_{x} +
        sin( \theta - \phi )*(Y - Y_{0})*\Delta_{y}\\
V & = & cos( \theta - \phi )*(Y - Y_{0})*\Delta_{y} -
        sin( \theta - \phi )*(X - X_{0})*\Delta_{x}
\end{eqnarray*}
and
\begin{eqnarray*}
X & = & cos( \theta - \phi )*(U/\Delta_{x}) + X_{0} -
        sin( \theta - \phi )*(V/\Delta_{y}) + Y_{0}\\
Y & = & cos( \theta - \phi )*(V/\Delta_{y}) + Y_{0} +
        sin( \theta - \phi )*(U/\Delta_{x}) + X_{0}
\end{eqnarray*}

Note, the variables $SAMPLE$ and $LINE$ referred to in the Exp. Supp.
are replaced in this appendix by $(U.scale.180/\pi)$ and $(-V.scale.180/\pi)$
where $scale$ is defined in the Exp. Supp.

\subsection {The ``Local'' Coordinate System}
The ``local'' coordinate system is a spherical coordinate system on the
celestial sphere. The local longitude is designated $A_{l}$ and the local
latitude is designated $B_{l}$. A position on the celestial sphere can be
described both by sky coordinates (eg GALACTIC etc) and by the corresponding
local coordinates. Local coordinates are defined so that the reference point has
local longitude and latitude values of zero, and so that north (in the local
system) at the reference point is at position angle $\psi$ in the sky coordinate
system. Thus the $U$ axis is always at position angle $-\pi/2$ in local
coordinates, and the $V$ axis is always at position angle zero. Also, the
reference point ( $(U,V)=(0,0)$  ) is always at $(A_{l},B_{l})=(0,0)$.

It is useful to introduce a 3-D Cartesian coordinate system $(x,y,z)$ (see
figure \ref {FIG:A}) to describe the mapping from sky coordinates to local
coordinates (lower case $(x,y,z)$ should not be confused with upper case $(X,Y)$
used to signify image coordinates). The origin of this system is at the centre
of the celestial sphere, the $x$ axis is towards the point on the celestial
sphere with zero longitude and zero latitude, the $y$ axis is towards the point
with longitude $+\pi/2$ and latitude zero, and the $z$ axis is the polar axis
towards latitude $+\pi/2$. The mapping from sky coordinates to local coordinates
is constructed as follows:
\begin{enumerate}

\item Initially the sky $(x,y,z)$ axes coincides with the local $(x,y,z)$
axes.

\item Rotate the local $(x,y,z)$ system about the local $z$ axis by an angle
$A_{0}$ (see figure \ref {FIG:B}). This results in the $x$ and $y$ axes no
longer being coincident for sky and local coordinates. The reference point then
has zero longitude in the local system. Positive rotation is anti-clockwise when
viewed from the positive end of the rotation axis.

\item Rotate the new local $(x,y,z)$ system about the local $y$ axis by an angle
$-B_{0}$ (see figure \ref {FIG:C}). This results in the $z$ axis no longer being
coincident for sky and local coordinates. The reference point now has zero
longitude and latitude in the local system.

\item Rotate the new local $(x,y,z)$ system about the local $x$ axis by an angle
$-\phi$ (see figure \ref {FIG:D}). Local north is then at a position angle
$\phi$ in the original sky coordinate system.

\end{enumerate}

\begin{figure}[htb]
\centering
\setlength{\unitlength}{1cm}
\begin{picture}(15,12)
\put(0,12){\special{include a.dat}}
\end{picture}
\caption[.]{
{\small
The $(x,y,z)$ coordinate system. The $x$ axis points towards zero longitude and
latitude, the $y$ axis towards longitude $\pi/2$ and latitude zero, and the $z$
axis towards the north pole (latitude $\pi/2$). In this figure, longitude
and latitude are measured in the current sky coordinate system. The radial
vector shows the direction of the reference point, which is defined by the sky
longitude and latitude values $A_{0}$ and $B_{0}$. When transforming from sky
coordinates to ``local'' coordinates, the $(x,y,z)$ axes defined in the two
systems are initially considered to be coincident. This is indicated here by
the absence of the ``Sky'' or ``Local'' axis prefixes used in later figures.
}}
\label {FIG:A}
\end{figure}

\begin{figure}[htb]
\centering
\setlength{\unitlength}{1cm}
\begin{picture}(15,12)
\put(0,12){\special{include b.dat}}
\end{picture}
\caption[.]{
{\small
The first step in transforming from sky to local coordinates is to rotate the
local $(x,y,z)$ system (initially coincident with the sky $(x,y,z)$ system) about
the local $z$ axis by an angle equal to the longitude, $A_{0}$, of the reference
point in the sky coordinate system. This causes the ``Sky $x$'' and ``Local
$x$'' axes to separate, as do the ``Sky $y$'' and ``Local $y$'' axes.
}}
\label {FIG:B}
\end{figure}

\begin{figure}[htb]
\centering
\setlength{\unitlength}{1cm}
\begin{picture}(15,12)
\put(0,12){\special{include c.dat}}
\end{picture}
\caption[.]{
{\small
The second step in transforming from sky to local coordinates is to rotate the
local $(x,y,z)$ system about the local $y$ axis by an angle equal to the
latitude, $B_{0}$, of the reference point in the sky coordinate system. This
causes the ``Sky $z$'' and ``Local $z$'' axes to separate. The local $x$ axis is
now coincident with the reference point. Thus, the reference point has zero
longitude and latitude in the local coordinate system.
}}
\label {FIG:C}
\end{figure}

\begin{figure}[htb]
\centering
\setlength{\unitlength}{1cm}
\begin{picture}(15,12)
\put(0,12){\special{include d.dat}}
\end{picture}
\caption[.]{
{\small
The final step in transforming from sky to local coordinates is to rotate the
local $(x,y,z)$ system about the local $x$ axis by the angle $\psi$
(supplied by the calling routine as projection parameter $P(8)$). This does not
effect the position of the reference point which still has longitude and
latitude of zero in the local system. However, local north at the reference
point (i.e. the direction of increasing local latitude, or of the great circle
joining the reference point to the local $z$ axis) is now at position angle
$\psi$ in the original sky coordinate system.
}}
\label {FIG:D}
\end{figure}

\subsection {The Gnomonic Projection}
The Gnomonic (or tangent plane) projection is a geometric projection from the
centre of the celestial sphere onto a plane which is tangent to the celestial
sphere at the reference point. The Gnomonic projection is used (for example)
for the IRAS 16.5 degree extended emission SKYPLATE images. Varying projection
parameter P(8) has no effect on the projection because the projection surface
is symetric about the radius vector of the celestial sphere passing through the
reference point.

Using local coordinates rather than sky coordinates means that the reference
point has both longitude and latitude equal to zero. The equation in the Exp.
Supp. are thus significantly simplified.

\subsubsection {The Inverse Mapping}
The inverse mapping from local coordinates $(A_{l},B_{l})$ to $(U,V)$
coordinates (called the {\em forward} transformation in the Exp. Supp.) is
defined by the following equations:

\begin{eqnarray*}
U & = & -\tan A_{l}\\
V & = & \frac{\tan B_{l}}{\cos A_{l}}
\end{eqnarray*}

\subsubsection {The Forward Mapping}
The forward mapping from $(U,V)$ coordinates to local coordinates
$(A_{l},B_{l})$ (called the {\em reverse} transformation in the Exp. Supp.) is
defined by the following equations:

\begin{eqnarray*}
A_{l} & = & \arctan \left( \frac{-U}{-V} \right)\\
B_{l} & = & \arcsin \left(\frac{V}{\sqrt{U^{2}+V^{2}+1}} \right)
\end{eqnarray*}

\subsection {The Orthographic Projection}
Like the Gnomonic projection, the Orthographic projection is a geometric
projection onto a plane which is tangent to the celestial sphere at the
reference point. The difference between the two projections lies in the point
from which the projection proceeds. The Gnomonic projection proceeds from the
centre of the celestial sphere, whereas the Orthographic projection proceeds
from a point at infinity, diametrically opposite the reference point (i.e. the
projection lines are all normal to the tangent plane). The Orthographic
projection is used (for example) for the IRAS Pointed Observation images.
Varying projection parameter P(8) has no effect on the projection because the
projection surface is symetric about the radius vector of the celestial sphere
passing through the reference point.

Again, working in local coordinates means that the reference point is always at
$(A_{l},B_{l})=(0,0)$ which results in great simplification of the equations.

\subsubsection {The Inverse Mapping}
The inverse mapping from local coordinates $(A_{l},B_{l})$ to $(U,V)$
coordinates is defined by the following equations:

\begin{eqnarray*}
U & = & -\cos B_{l}.\sin A_{l}\\
V & = & \sin B_{l}
\end{eqnarray*}

\subsubsection {The Forward Mapping}
The forward mapping from $(U,V)$ coordinates to local coordinates
$(A_{l},B_{l})$ is defined by the following equations:

\begin{eqnarray*}
A_{l} & = & \arcsin \left( \frac{-U}{\sqrt{1-V^{2}}} \right)\\
B_{l} & = & \arcsin V
\end{eqnarray*}

\subsection {The Lambert Equivalent Cylindrical Projection}
The Lambert equivalent cylindrical projection is a geometric projection from the
polar axis of the celestial sphere onto a cylinder which is tangential to the
celestial sphere at the equator. The projection proceeds normally to the polar
axis. It is used, for example, by the IRAS Galactic Plane images. It is an equal
area projection (i.e. all image pixels are the same size on the sky, but have
varying ``aspect ratio''). Varying projection parameter P(8) changes the nature
of the projection because the cylindrical projection surface is not symetric
about the radius vector of the celestial sphere passing through the reference
point.

The cylinder is tangent at the equator of the local coordinate system. In sky
coordinates the cylinder is tangent at a great circle passing through the
reference point at a position angle of $\theta+\pi/2$.

\subsubsection {The Inverse Mapping}
The inverse mapping from local coordinates $(A_{l},B_{l})$ to $(U,V)$
coordinates (called the {\em forward} transformation in the Exp. Supp.) is
defined by the following equations:
\begin{eqnarray*}
U & = & -A_{l}\\
V & = & \sin B_{l}
\end{eqnarray*}

\subsubsection {The Forward Mapping}
The forward mapping from $(U,V)$ coordinates to local coordinates
$(A_{l},B_{l})$ (called the {\em reverse} transformation in the Exp. Supp.) is
defined by the following equations:

\begin{eqnarray*}
A_{l} & = & -U\\
B_{l} & = & \arcsin V
\end{eqnarray*}

Note, if the absolute value of $U$ is greater than $\pi$ then A and B are
returned bad (i.e. equal to the value VAL\_\_BADD).

\subsection {The Aitoff Projection}
The Aitoff projection is an equal area projection used to produce
photometrically correct maps of the entire celestial sphere. Varying projection
parameter P(8) changes the nature of the projection because the projection
surface is not symetric about the radius vector of the celestial sphere passing
through the reference point. Again, the use of local coordinates mean that
Aitoff projections can be performed about any reference position and inclined at
any angle on the sky.

\subsubsection {The Inverse Mapping}
The inverse mapping from local coordinates $(A_{l},B_{l})$ to $(U,V)$
coordinates
(called the {\em forward} transformation in the Exp. Supp.) is defined by the
following equations, in which $r$ and $h$ are intermediate quantities equivalent
to the values labeled ``$\rho$'' and ``$\theta$'' in the Exp. Supp.:
\begin{eqnarray*}
r & = & \arccos (\cos B_{l}.\cos (A_{l}/2) )\\
h & = & \arcsin \left(\frac {\cos B_{l}.\sin(A_{l}/2)}{\sin r}\right)\\
U & = & -4.\sin (r/2).\sin h\\
V & = & \left\{ \begin{array}{ll}
                 +2.\sin(r/2).\cos h   & \mbox{: if $B_{l}\geq0$}\\
                 -2.\sin(r/2).\cos h   & \mbox{: otherwise}
                \end{array}
        \right.\\
\end{eqnarray*}

\subsubsection {The Forward Mapping}
The forward mapping from $(U,V)$ coordinates to local coordinates
$(A_{l},B_{l})$
( called the {\em reverse} transformation in the Exp. Supp.) is defined by
the following equations in which $\alpha$ is an intermediate quantity equivalent
to the value labeled ``$A$'' in the Exp. Supp.:

\begin{eqnarray*}
\alpha & = & \sqrt{4-(U^{2}/4)-V^{2} }\\
B_{l} & = & \arcsin (\alpha.V/2)\\
A_{l} & = & 2.\arcsin\left(\frac{-\alpha.U}{4.\cos B_{l}}\right)
\end{eqnarray*}

In above equations, values of $\alpha$ less than $\sqrt{2}$ are not allowed.
These values correspond to image pixels far from the centre of the projection.
It can be shown that the sky positions corresponding to such pixels are also
mapped by other pixels closer to the reference point (for which
$\alpha\geq\sqrt{2}$). To avoid the same part of the sky appearing twice on the
same image, the local coordinates $A_{l}$ and $B_{l}$ are both returned with the
Starlink ``BAD'' value (VAL\_\_BADD) in such cases.

\section {Packages Called by IRA}
IRA\_ makes calls to the following packages:
\begin {description}
\item [CHR\_] - The CHR character handling package; see SUN/40.
\item [CMP\_] - HDS; see SUN/92.
\item [DAT\_] - HDS; see SUN/92.
\item [ERR\_] - The Starlink error reporting package; see SUN/104.
\item [GKS\_] - The GKS graphics package; see SUN/113.
\item [MSG\_] - The Starlink message reporting package; see SUN/104.
\item [NDF\_] - The NDF access package; see SUN/33.
\item [PAR\_] - The ADAM parameter system; see SUN/114.
\item [SGS\_] - The SGS graphics package; see SUN/113.
\item [SLA\_] - The SLA package; see SUN/67.
\end{description}

Access to these packages, together with packages called from within these
packages, is necessary to use IRA.

\section {IRA Error Codes}
\label {APP:ERRORS}
IRA routines can return any $STATUS$ value generated by the subroutine packages
which it calls. In addition it can return the following IRA-specific values:

\begin{description}

\item {\bf IRA\_\_AMBPR   }\\
Ambiguous projection specified.

\item {\bf IRA\_\_AMBSC   }\\
Ambiguous sky coordinate system specified.

\item {\bf IRA\_\_BAD2V   }\\
Sky latitude value is outside the range $+\pi/2$ to $\pi/2$.

\item {\bf IRA\_\_BADAS   }\\
The astrometry structure does not have an HDS type of IRAS\_ASTROMETRY.

\item {\bf IRA\_\_BADBJ   }\\
A character specifying a Besselian or Julian epoch had a value other than B or J.

\item {\bf IRA\_\_BADEQ   }\\
An SCS string contained an illegal equinox specifier.

\item {\bf IRA\_\_BADPA   }\\
At least one of the parameter values specified for a projection had the
Starlink ``BAD'' value.

\item {\bf IRA\_\_BADNC   }\\
A value other than 1 or 2 was given for a coordinate index.

\item {\bf IRA\_\_BADPR   }\\
Unsupported projection specified.

\item {\bf IRA\_\_BADSC   }\\
Unsupported sky coordinate system specified.

\item {\bf IRA\_\_BADST   }\\
An illegal style value was specified for a formatted sky coordinate value.

\item {\bf IRA\_\_EXFLD   }\\
Too many fields in a string holding a formatted sky coordinate value.

\item {\bf IRA\_\_INVID   }\\
An invalid IRA identifier was specified.

\item {\bf IRA\_\_MIXFL   }\\
The fields in a string holding a formatted sky coordinate were in the wrong
order.

\item {\bf IRA\_\_NOAS    }\\
No astrometry structure was found.

\item {\bf IRA\_\_NOCNV   }\\
Image pixels are becoming very small, resulting in extremely large image dimensions.

\item {\bf IRA\_\_NOEXT   }\\
The NDF extension in which IRA expects to find the astrometry structure does
not exist.

\item {\bf IRA\_\_NOMOR   }\\
No more free IRA identifiers are available.

\item {\bf IRA\_\_NOPRO   }\\
No information was found in an astrometry structure describing the projection.

\item {\bf IRA\_\_ORDER   }\\
Supplied limits are in the wrong order (i.e. the upper limit is less than the
lower limit).

\item {\bf IRA\_\_RANGE   }\\
One of the fields in a formatted sky coordinate string was outside the allowed
limits for that field.

\item {\bf IRA\_\_REDFL   }\\
A redundant field was found in a string holding a formatted sky coordinate value.

\item {\bf IRA\_\_SCTER   }\\
A field with an illegal terminator character was found in a string holding a
formatted sky coordinate value.

\item {\bf IRA\_\_SING    }\\
The projection is singular.

\item {\bf IRA\_\_SMALL   }\\
The specified area of image coordinates is too small.

\item {\bf IRA\_\_TFEWP   }\\
An incorrect number of projection parameters was supplied.

\item {\bf IRA\_\_TOOSH   }\\
A supplied character variable is too short.

\item {\bf IRA\_\_UNDAS   }\\
The astrometry structure is in an undefined state.

\end{description}

\newpage
\section {Examples of the Use of IRA Graphics Routines}
\label{SEC:DREX}

This section contains some example plots produced using the routine IRA\_DRGRD.
\begin{figure}[htb]
\centering
\setlength{\unitlength}{1cm}
\begin{picture}(15,12)
\put(0,12){\special{include e.dat}}
\end{picture}
\caption[.]{
{\small
An orthographic projection in equatorial (B1950) coordinates, centred on RA = 4
hours, DEC = +50 degrees, covering a complete hemisphere, at an inclination of
30 degrees to vertical (P(8) is zero). Maximum accuracy (TOL=0) was used.
}}
\label {FIG:E}
\end{figure}

\begin{figure}[htb]
\centering
\setlength{\unitlength}{1cm}
\begin{picture}(15,12)
\put(0,12){\special{include f.dat}}
\end{picture}
\caption[.]{
{\small
An all-sky Aitoff projection in galactic coordinates.
Maximum accuracy (TOL=0) was used, and P(8) was zero.
}}
\label {FIG:F}
\end{figure}

\begin{figure}[htb]
\centering
\setlength{\unitlength}{1cm}
\begin{picture}(15,12)
\put(0,12){\special{include g.dat}}
\end{picture}
\caption[.]{
{\small
An all-sky Lambert equivalent cylindrical projection in ecliptic (B1950)
coordinates. Maximum accuracy (TOL=0) was used and P(8) was zero.
}}
\label {FIG:G}
\end{figure}

\begin{figure}[htb]
\centering
\setlength{\unitlength}{1cm}
\begin{picture}(15,12)
\put(0,12){\special{include h.dat}}
\end{picture}
\caption[.]{
{\small
A Gnomonic (tangent plane) projection in equatorial (B1950) coordinates, centred
at RA = 6 hours, DEC = +40 degrees, covering a 3 degree square region, at an
inclination of 30 degrees to vertical. Maximum accuracy (TOL=0) was used. Tick
marks are used instead of complete lines. Note the inclination of the tick marks
indicating the inclination of the meridians and parallels. P(8) was zero.
}}
\label {FIG:H}
\end{figure}

\begin{figure}[htb]
\centering
\setlength{\unitlength}{1cm}
\begin{picture}(15,12)
\put(0,12){\special{include i.dat}}
\end{picture}
\caption[.]{
{\small
A Gnomonic (tangent plane) projection in equatorial (B1950) coordinates, centred
at DEC = +90 degrees, covering a 3 degree square region. Maximum accuracy
(TOL=0) was used and P(8) was zero. Tick marks have been used instead of
complete lines. IRA\_DRGRD attempts to place longitude tick marks on the bottom
edge and latitude tick marks on the left edge. If this is not possible then
longitude or latitude labels and tick marks are placed along a central parallel
or meridian.
}}
\label {FIG:I}
\end{figure}

\clearpage

\section {Changes Introduced in the Current Version of this Document}
\label {SEC:CHANGES}

Changes introduced in version 11 of ID2:
\begin{itemize}
\item New routine IRA\_ACTIVE
\end{itemize}

Changes introduced in version 10 of ID2:
Significant changes have been made in order to remove connections between
IRA identifiers and HDS. IRA identifiers now just refer to data stored
internally within common blocks. No HDS locators are stored by IRA. Therefore,
NDFs and HDS objects can be closed safely after the astrometry infromation has
been imported without losing access to the astrometry information. These
changes have resulted in some routine being withdrawn, some new ones added, and
some old ones now have different argument lists.

Another significant change is that ``customised'' projections (previously
created using IRA\_CRETR) are no longer supported.

\begin{enumerate}
\item New routines IRA\_EXPRT, IRA\_FIND, IRA\_PIXSZ and IRA\_DROPS, added.
\item Routines IRA\_CRETR, IRA\_DELET and IRA\_PROJ removed.
\item Argument lists have changed for IRA\_WRITE and IRA\_XYLIM.
\item Routines IRA\_ROT, IRA\_MOVE and IRA\_MAG no longer have any effect on the
NDF from which the astrometry information was read. A call to IRA\_EXPRT should
be made in order to store the modified astrometry information back in the NDF.
\item The package is no longer dependant on the TRANSFORM package.
\end{enumerate}

Changes introduced in version 9 of ID2:
\begin {enumerate}
\item An eigth projection parameter has been added.
\item Routines IRA\_DROPT, IRA\_DRBRK, IRA\_DRVPO, IRA\_VALID, IRA\_MAG,
IRA\_MOVE, IRA\_ROT, IRA\_READ and IRA\_WRITE added.
\item Argument lists of all graphics routines changed (extra graphics options
have been introduced through the new routine IRA\_DROPT).
\item Abbreviations for GALACTIC and ECLIPTIC coordinate axes changed from
ELONG, ELAT, GLONG and GLAT to Lambda, Beta, l and b.
\item Description of the use of IRA on UNIX machines included.
\item STARLSE templates and help available for IRA subroutines.
\item The name IRA\_TRACE has been given to a new routine. There is no
equivalent to the previous version of IRA\_TRACE.
\item Routine IRA\_SKGRD has been removed.
\item It is no longer necessary to include I90\_PAR before including
IRA\_ERR.
\item The names of symbolic constants (both errors and parameters) which previously
were longer than five characters (excluding the IRA\_\_ prefix) have been
truncated to five characters. The actual {\em values} associated with each name
have not changed.
\item The VMS version of IRA is now released in the form of a sharable image
rather than an object library. Applications linked with the sharable image have
the advantage that they will not need to be {\em re}-linked when IRA is upgraded
in future. Note, the object library has been removed from the IRA system.
\item The linking and development procedures have been modified to bring them
into line with Starlink standards (see SSN/8).
\item IRA is no longer directly dependant on the ARY\_ system (although it may
still have {\em in}-direct dependencies on ARY\_, for example through the
NDF\_ system).
\item Sky coordinate values formatted using style 5 are now right justified
rather than left justified as before.

\end {enumerate}

Changes introduced in version 8 of ID2:
\begin {enumerate}
\item New arguments GAPLON and GAPLAT added to IRA\_DRGRD.
\end {enumerate}

Changes introduced in version 7 of ID2:
\begin {enumerate}
\item A description of useful constants has been included in section \ref
{SEC:CON}.
\item Sections \ref {SEC:DREX} and \ref {SEC:CHANGES} created.
\item Dependency of the graphics routines on the AGI database has been removed,
by introducing the arguments LBND and UBND.
\item The graphics routines now have a TOL argument which enables the
application to select the plotting accuracy (and consequently the plotting
speed).
\item The forward mapping of the Lambert Equivalent Cylindrical projection has
been modified to return bad sky coordinate values if the supplied image position
is further than $\pi$ radians away from the reference point along the X axis
(i.e. if local coordinate $U$ has an absolute value greater than $\pi$).
\end {enumerate}

\end{document}
