\documentstyle{article}
\pagestyle{myheadings}

%------------------------------------------------------------------------------
\newcommand{\irasdoccategory}  {IRAS90 Document}
\newcommand{\irasdocinitials}  {ID}
\newcommand{\irasdocnumber}    {20.1}
\newcommand{\irasdocauthors}   {D.S. Berry}
\newcommand{\irasdocdate}      {20th August 1992}
\newcommand{\irasdoctitle}     {I90: IRAS90 constants definition module}
%------------------------------------------------------------------------------

\newcommand{\irasdocname}{\irasdocinitials /\irasdocnumber}
\renewcommand{\_}{{\tt\char'137}}     % re-centres the underscore
\markright{\irasdocname}
\setlength{\textwidth}{160mm}
\setlength{\textheight}{240mm}
\setlength{\topmargin}{-5mm}
\setlength{\oddsidemargin}{0mm}
\setlength{\evensidemargin}{0mm}
\setlength{\parindent}{0mm}
\setlength{\parskip}{\medskipamount}
\setlength{\unitlength}{1mm}

%------------------------------------------------------------------------------
% Add any \newcommand or \newenvironment commands here
%------------------------------------------------------------------------------

\begin{document}
\thispagestyle{empty}
SCIENCE \& ENGINEERING RESEARCH COUNCIL \hfill \irasdocname\\
RUTHERFORD APPLETON LABORATORY\\
{\large\bf IRAS90\\}
{\large\bf \irasdoccategory\ \irasdocnumber}
\begin{flushright}
\irasdocauthors\\
\irasdocdate
\end{flushright}
\vspace{-4mm}
\rule{\textwidth}{0.5mm}
\vspace{5mm}
\begin{center}
{\Large\bf \irasdoctitle}
\end{center}
\vspace{5mm}
\setlength{\parskip}{0mm}
\tableofcontents
\setlength{\parskip}{\medskipamount}
\markright{\irasdocname}

\section {Introduction}

The I90 library contains modules which define many constant values describing
the IRAS mission and satellite, and the IRAS90 package. The symbolic names set
up by I90 which refer to these constants should always be used in preference to
the actual numerical values. For instance, if looping round all survey wavebands
(i.e. 12, 25, 60 and 100 $\mu$m bands), the following code should be used:

\small
\begin{verbatim}

      INCLUDE 'I90_DAT'

      ...

      DO I = 1, I90__BANDS
         ...
      END DO

\end{verbatim}
\normalsize

rather than

\small
\begin{verbatim}

      DO I = 1, 4
         ...
      END DO

\end{verbatim}
\normalsize

Using symbolic names guards against accidental errors creeping in. For instance,
a ``5'' may accidentally be typed instead if a ``4'' in the second example which
would go undetected by the compiler, and may give rise to some very obscure bugs
which may take a long time to track down. On the other hand, if the symbolic
name is used and accidentally mis-spelled, the compiler will issue a warning
\footnotemark saying that the mis-spelled name has not been defined, enabling
the mistake to be quickly identified and fixed.
\footnotetext{ This assumes that an IMPLICIT NONE statement has been included
in the code, which should, of course, always be done anyway.}

Using symbolic names has the additional advantage that it indicates what the
value means, whereas the literal ``4'' in the second example could be 4
detectors, 4 scans, 4 carrots, etc, etc.

\section{Declaring the Constants}

The value defined by I90 may be made available by including the following line
in all Fortran source code modules which refer to any of the constant values:

\small
\begin{verbatim}
      INCLUDE 'I90_DAT'
\end{verbatim}
\normalsize

On VMS systems I90\_DAT will be a logical name set up by the IRAS90\_DEV
command. On UNIX systems I90\_DAT will be a symbolic link to the file {\bf
i90\_dat} stored in the {\bf include} sub-directory of the main IRAS90
source directory. This symbolic link will be set up by the {\bf iras90\_dev} script.
Note, UNIX symbolic links are different to VMS logical names in that a logical
name can be accessed form any directory, whereas a symbolic link can only be
accessed from the directory in which it was defined. For this reason, the {\bf
iras90\_dev} script should be executed again whenever the current UNIX directory
is changed, in order to set up new symbolic links.

All the constants defined by I90 have names which start with ``I90\_\_''
(note the {\em double} underscore). Single valued constants are defined by
Fortran PARAMETER statements, and multi-valued constants (i.e. arrays) are
defined by a BLOCKDATA module which stores the relevant values in arrays held in
a common block. \footnotemark
\footnotetext{ The use of common arrays {\em as well as} parameter statements to
define these constants is why the include file is called I90\_DAT and not
I90\_PAR as for other libraries.}

\section{Constants Defined by I90}

This section lists the constants defined by I90, and their values.
\subsection{Single Valued Constants}
These constants are scalar values. Their values are shown in parenthesize at the
end of the description.
\begin{description}
\item [I90\_\_BANDS] (INTEGER) The number of wavebands for survey array data (4).
\item [I90\_\_CBNDS] (INTEGER) The number of wavebands for CPC data (2).
\item [I90\_\_DETS]  (INTEGER) The total number of survey detectors (62).
\item [I90\_\_FAC]   (INTEGER) The IRAS90 ADAM facility number, see SUN104 (1507).
\item [I90\_\_MAXDT] (INTEGER) The maximum number of survey detectors per
waveband (16).
\item [I90\_\_NDEAD] (INTEGER) The number of ``dead'' survey detectors (3).
\item [I90\_\_NSMAL] (INTEGER) The number of detectors with cross-scan size less
than 4 arc-minutes (12).
\item [I90\_\_SPEED] (REAL) The nominal survey scan speed in arc-minutes per
second (3.85).
\item [I90\_\_SMSIZ] (REAL) The cross scan size of the largest ``small''
detector (as specified by I90\_\_SMALL), in arc-minutes (3.47).
\end{description}

\newpage
\subsection{Multi Valued Constants}
These constants are arrays of values. The dimensions of these arrays are given
in terms of the scalar constants described above, and are shown in parenthesize
after the name of each constant.

\paragraph{INTEGER valued constants...}
\begin{itemize}
\item {\bf I90\_\_BDETS} ( I90\_\_MAXDT, I90\_\_BANDS ) - Detector numbers in each band in
cross scan order. The first array index is the cross scan position of the detector.
Position 1 is at the highest focal plane Z coordinate value, and position
I90\_\_MAXDT is at the lowest Z value. The second array index is the waveband
index.

\begin{minipage}[t]{\textwidth}
\small
\begin{verbatim}
                                detector position
         1   2   3   4   5   6   7   8   9   10  11  12  13  14  15  16
  band
   1     47, 27, 51, 23, 48, 28, 52, 24, 49, 29, 53, 25, 50, 30, 54, 26
   2     39, 19, 43, 16, 40, 20, 44, 17, 41, 21, 45, 18, 42, 22, 46,  0
   3     31, 12, 35,  8, 32, 13, 36,  9, 33, 14, 37, 10, 34, 15, 38, 11
   4     55,  4, 59,  1, 56,  5, 60,  2, 57,  6, 61,  3, 58,  7, 62,  0
\end{verbatim}
\normalsize
\end{minipage}

\item {\bf I90\_\_CWVEL} ( I90\_\_CBNDS ) - Wavelength (in microns) of each CPC waveband.
The first array index is the CPC waveband index.

\begin{minipage}[t]{\textwidth}
\small
\begin{verbatim}
     CPC waveband index
          1   2
         50, 100
\end{verbatim}
\normalsize
\end{minipage}

\item {\bf I90\_\_DBAND} ( I90\_\_DETS ) - The waveband index of each survey detector.
The first array index is the detector number.

\begin{minipage}[t]{\textwidth}
\small
\begin{verbatim}
                                 detector number
          1   2   3   4   5   6   7   8   9  10  11  12  13  14  15  16
          4,  4,  4,  4,  4,  4,  4,  3,  3,  3,  3,  3,  3,  3,  3,  2,

         17  18  19  20  21  22  23  24  25  26  27  28  29  30  31  32
          2,  2,  2,  2,  2,  2,  1,  1,  1,  1,  1,  1,  1,  1,  3,  3,

         33  34  35  36  37  38  39  40  41  42  43  44  45  46  47  48
          3,  3,  3,  3,  3,  3,  2,  2,  2,  2,  2,  2,  2,  2,  1,  1,

         49  50  51  52  53  54  55  56  57  58  59  60  61  62
          1,  1,  1,  1,  1,  1,  4,  4,  4,  4,  4,  4,  4,  4
\end{verbatim}
\normalsize
\end{minipage}

\item {\bf I90\_\_DEAD} ( I90\_\_NDEAD ) - Dead detector numbers (in arbitrary order).

\begin{minipage}[t]{\textwidth}
\small
\begin{verbatim}
            index
          1   2   3
          17, 20, 36
\end{verbatim}
\normalsize
\end{minipage}

\item {\bf I90\_\_DETGP} ( I90\_\_DETS ) - A ``group number'' is assigned to
each detector. The detectors within each waveband are divided into 3 groups
according to their cross scan sizes. Detectors in group 1 are small size,
detectors in group 2 are medium size and in 3 are full size.

\begin{minipage}[t]{\textwidth}
\small
\begin{verbatim}
                                 detector number
          1   2   3   4   5   6   7   8   9  10  11  12  13  14  15  16
          3   3   3   2   3   3   3   3   3   3   1   2   3   3   3   3

         17  18  19  20  21  22  23  24  25  26  27  28  29  30  31  32
          3   3   2   3   3   2   3   3   3   1   2   3   3   3   1   3

         33  34  35  36  37  38  39  40  41  42  43  44  45  46  47  48
          3   3   3   3   3   2   1   3   3   3   3   3   3   1   1   3

         49  50  51  52  53  54  55  56  57  58  59  60  61  62
          3   3   3   3   3   2   1   3   3   3   3   3   3   1
\end{verbatim}
\normalsize
\end{minipage}

\item {\bf I90\_\_NDETS} ( I90\_\_BANDS ) - The number of detectors in each survey
waveband. First array index is the survey waveband index.


\begin{minipage}[t]{\textwidth}
\small
\begin{verbatim}
              band
          1   2   3   4
          16, 15, 16, 15
\end{verbatim}
\normalsize
\end{minipage}

\item {\bf I90\_\_SMALL} ( I90\_\_NSMAL ) - Detector number of all detectors less than 4
arc-minutes cross scan size (in arbitrary order).

\begin{minipage}[t]{\textwidth}
\small
\begin{verbatim}
                               index
          1   2   3   4   5   6   7   8   9   10  11  12
          11, 12, 26, 27, 31, 38, 39, 46, 47, 54, 55, 62
\end{verbatim}
\normalsize
\end{minipage}

\item {\bf I90\_\_SRATE} ( I90\_\_BANDS ) - Sampling frequency (in Hz) for
each waveband.

\begin{minipage}[t]{\textwidth}
\small
\begin{verbatim}
              band
          1   2   3   4
          16, 16, 8,  4
\end{verbatim}
\normalsize
\end{minipage}

\item {\bf I90\_\_WAVEL} ( I90\_\_BANDS ) - Wavelength (in microns) of
each survey band.

\begin{minipage}[t]{\textwidth}
\small
\begin{verbatim}
              band
          1   2   3   4
          12, 25, 60, 100
\end{verbatim}
\normalsize
\end{minipage}

\item {\bf I90\_\_XDETS} ( I90\_\_DETS ) - Detector numbers in cross scan order
(high to low Z) regardless of their waveband.

\begin{minipage}[t]{\textwidth}
\small
\begin{verbatim}
                                 detector position
          1   2   3   4   5   6   7   8   9  10  11  12  13  14  15  16
         47, 31, 39, 55, 27, 12, 19,  4, 51, 35, 43, 59, 23,  8, 16,  1,

         17  18  19  20  21  22  23  24  25  26  27  28  29  30  31  32
         48, 32, 40, 56, 28, 13, 20,  5, 52, 36, 44, 60, 24,  9, 17,  2,

         33  34  35  36  37  38  39  40  41  42  43  44  45  46  47  48
         49, 33, 41, 57, 29, 14, 21,  6, 53, 37, 45, 61, 25, 10, 18,  3,

         49  50  51  52  53  54  55  56  57  58  59  60  61  62
         50, 34, 42, 58, 30, 15,  7, 22, 54, 38, 62, 46, 11, 26
\end{verbatim}
\normalsize
\end{minipage}
\end{itemize}

\paragraph{REAL valued constants...}
\begin{itemize}
\item {\bf I90\_\_BANDW} ( I90\_\_BANDS ) -  Effective detector band widths (in Hz) for
each survey waveband. Taken from the IRAS Catalogues and Atlases Explanatory
Supplement, 2nd edition (``{\em Exp. Supp}'') page X-13.

\begin{minipage}[t]{\textwidth}
\small
\begin{verbatim}
                          band
              1        2        3        4
          13.48E12, 5.16E12, 2.58E12, 1.00E12
\end{verbatim}
\normalsize
\end{minipage}

\item {\bf I90\_\_BNEFD} ( I90\_\_BANDS ) -  Mean Noise Equivalent Flux Density (in Jy)
for each survey waveband. Taken from {\em Exp. Supp.} Fig. IV.A.1.

\begin{minipage}[t]{\textwidth}
\small
\begin{verbatim}
                     band
            1      2      3      4
          0.105, 0.125, 0.170, 0.580
\end{verbatim}
\normalsize
\end{minipage}

\item {\bf I90\_\_DETDY} ( I90\_\_DETS ) -   In-scan detector mask sizes in arc-minutes.
Taken from {\em Exp. Supp.} table II.C.3. First array index is detector number.

\begin{minipage}[t]{\textwidth}
\small
\begin{verbatim}
                                detector number
           1      2      3      4      5      6      7      8      9
         3.03,  3.03,  3.03,  3.03,  3.03,  3.03,  3.03,  1.51,  1.51,

          10     11     12     13     14     15     16     17     18
         1.51,  1.51,  1.51,  1.51,  1.51,  1.51,  0.76,  0.76,  0.76,

          19     20     21     22     23     24     25     26     27
         0.76,  0.76,  0.76,  0.76,  0.76,  0.76,  0.76,  0.76,  0.76,

          28     29     30     31     32     33     34     35     36
         0.76,  0.76,  0.76,  1.51,  1.51,  1.51,  1.51,  1.51,  1.51,

          37     38     39     40     41     42     43     44     45
         1.51,  1.51,  0.76,  0.76,  0.76,  0.76,  0.76,  0.76,  0.76,

          46     47     48     49     50     51     52     53     54
         0.76,  0.76,  0.76,  0.76,  0.76,  0.76,  0.76,  0.76,  0.76,

          55     56     57     58     59     60     61     62
         3.03,  3.03,  3.03,  3.03,  3.03,  3.03,  3.03,  3.03

\end{verbatim}
\normalsize
\end{minipage}

\item {\bf I90\_\_DETDZ} ( I90\_\_DETS ) -   Cross-scan detector mask sizes in arc-minutes.
Taken from {\em Exp. Supp.}  table II.C.3. First array index is detector number.

\begin{minipage}[t]{\textwidth}
\small
\begin{verbatim}
                                detector number
           1      2      3      4      5      6      7      8      9
         5.05,  5.05,  5.05,  4.68,  5.05,  5.05,  5.05,  4.75,  4.75,

          10     11     12     13     14     15     16     17     18
         4.75,  1.30,  3.45,  4.75,  4.75,  4.75,  4.65,  4.65,  4.65,

          19     20     21     22     23     24     25     26     27
         4.48,  4.65,  4.65,  4.48,  4.45,  4.45,  4.45,  1.20,  3.33,

          28     29     30     31     32     33     34     35     36
         4.55,  4.55,  4.55,  1.28,  4.75,  4.75,  4.75,  4.75,  4.75,

          37     38     39     40     41     42     43     44     45
         4.75,  3.47,  2.33,  4.65,  4.65,  4.65,  4.65,  4.65,  4.65,

          46     47     48     49     50     51     52     53     54
         2.33,  1.18,  4.55,  4.55,  4.55,  4.55,  4.55,  4.55,  3.36,

          55     56     57     58     59     60     61     62
         2.52,  5.05,  5.05,  5.05,  5.05,  5.05,  5.05,  2.53

\end{verbatim}
\normalsize
\end{minipage}

\item {\bf I90\_\_DETY} ( I90\_\_DETS ) -    Focal plane Y value (in arc-minutes) at
detector centres. Taken from {\em Exp. Supp.} table II.C.3. First array index is detector number.

\begin{minipage}[t]{\textwidth}
\small
\begin{verbatim}
                                detector number
           1      2      3      4      5      6      7      8      9
         27.87, 27.80, 27.86, 23.83, 24.04, 23.65, 23.78, 19.64, 19.72,

          10     11     12     13     14     15     16     17     18
         19.74, 19.70, 17.20, 17.19, 17.20, 17.20, 14.01, 14.04, 14.04,

          19     20     21     22     23     24     25     26     27
         12.24, 12.27, 12.26, 12.27,  9.47,  9.46,  9.47,  9.48,  7.71,

          28     29     30     31     32     33     34     35     36
          7.71,  7.70,  7.71,  4.56,  4.59,  4.58,  4.59,  2.06,  2.06,

          37     38     39     40     41     42     43     44     45
          2.11,  2.10, -1.16, -1.16, -1.16, -1.14, -2.92, -2.92, -2.93,

          46     47     48     49     50     51     52     53     54
         -2.92, -5.67, -5.67, -5.67, -5.66, -7.42, -7.43, -7.43, -7.42,

          55      56      57      58      59      60      61      62
        -11.33, -11.42, -11.51, -11.41, -15.34, -15.49, -15.40, -15.38

\end{verbatim}
\normalsize
\end{minipage}

\item {\bf I90\_\_DETZ} ( I90\_\_DETS ) -    Focal plane Z value (in arc-minutes) at
detector centres. Taken from {\em Exp.  Supp.} table II.C.3. First array index is detector number.

\begin{minipage}[t]{\textwidth}
\small
\begin{verbatim}
                                detector number
           1       2       3       4       5       6       7       8
         8.71,   0.04,  -8.62,  12.86,   4.37,  -4.29, -12.77,   9.80,

           9      10      11      12      13      14      15      16
         1.14,  -7.53, -14.46,  13.49,   5.47,  -3.20, -11.86,   8.71,

          17      18      19      20      21      22      23      24
         0.04,  -8.62,  12.96,   4.37,  -4.29, -12.88,   9.81,   1.14,

          25      26      27      28      29      30      31      32
        -7.52, -14.50,  13.55,   5.47,  -3.19, -11.86,  14.55,   7.61,

          33      34      35      36      37      38      39      40
        -1.06,  -9.73,  11.94,   3.27,  -5.40, -13.41,  14.05,   6.55,

          41      42      43      44      45      46      47      48
        -2.12, -10.78,  10.88,   2.22,  -6.45, -13.95,  14.64,   7.65,

          49      50      51      52      53      54      55      56
        -1.02,  -9.68,  11.98,   3.32,  -5.35, -13.41,  13.95,   6.55,

          57      58      59      60      61      62
        -2.12, -10.79,  10.88,   2.21,  -6.46, -13.85

\end{verbatim}
\normalsize
\end{minipage}

\item {\bf I90\_\_DNEFD} ( I90\_\_DETS ) -   Noise Equivalent  Flux Density (in Jy)
for each detector. Taken from {\em Exp.  Supp.} Fig IV.A.1. First array index is detector number.

\begin{minipage}[t]{\textwidth}
\small
\begin{verbatim}
                                detector number
           1      2      3      4      5      6      7      8      9
         0.75,  0.45,  0.45,  0.65,  0.75,  0.45,  0.65,  0.16,  0.19,

          10     11     12     13     14     15     16     17     18
         0.16,  0.14,  0.19,  0.16,  0.16,  0.16,  0.14,  0.00,  0.14,

          19     20     21     22     23     24     25     26     27
         0.12,  0.00,  0.12,  0.14,  0.12,  0.09,  0.29,  0.12,  0.12,

          28     29     30     31     32     33     34     35     36
         0.16,  0.12,  0.12,  0.16,  0.16,  0.21,  0.19,  0.16,  0.00,

          37     38     39     40     41     42     43     44     45
         0.16,  0.16,  0.09,  0.14,  0.12,  0.24,  0.14,  0.14,  0.12,

          46     47     48     49     50     51     52     53     54
         0.09,  0.09,  0.12,  0.09,  0.09,  0.09,  0.09,  0.09,  0.09,

          55     56     57     58     59     60     61     62
         0.55,  0.65,  0.55,  0.55,  0.45,  0.75,  0.55,  0.45

\end{verbatim}
\normalsize
\end{minipage}

\item {\bf I90\_\_DY} ( I90\_\_BANDS ) -     In-scan size of detectors in each waveband, in
arc-minutes. Taken from {\em Exp. Supp.} table II.C.3.

\begin{minipage}[t]{\textwidth}
\small
\begin{verbatim}
                    band
            1     2     3     4
          0.76, 0.76, 1.51, 3.03
\end{verbatim}
\normalsize
\end{minipage}

\item {\bf I90\_\_DZ} ( I90\_\_BANDS ) -     Maximum cross-scan size of detectors in each
band, in arc-minutes. Taken from {\em Exp. Supp.} table II.C.3.

\begin{minipage}[t]{\textwidth}
\small
\begin{verbatim}
                    band
            1     2     3     4
          4.55, 4.65, 4.75, 5.05
\end{verbatim}
\normalsize
\end{minipage}

\item {\bf I90\_\_JY} ( I90\_\_BANDS ) -     Factors for converting flux values in units
of $pW.m^{-2}$ to flux density in Jy (based on I90\_\_BANDW).

\begin{minipage}[t]{\textwidth}
\small
\begin{verbatim}
                    band
            1     2      3      4
          7.42, 19.38, 38.76, 100.00
\end{verbatim}
\normalsize
\end{minipage}

\item {\bf I90\_\_OCT84} ( I90\_\_BANDS ) -  October 1984 IPAC flux calibration correction
factors.

\begin{minipage}[t]{\textwidth}
\small
\begin{verbatim}
                    band
            1     2     3     4
          0.97, 0.98, 0.93, 0.74
\end{verbatim}
\normalsize
\end{minipage}

\item {\bf I90\_\_SOLAN} ( I90\_\_DETS ) -   Effective solid angle of each detector in
units of 1.0E-7 of a steradians. Calculated by M. Moshir (see IPAC memo
701-88-54 (1)). Dead detectors have a value of zero. First array index is detector number.

\begin{minipage}[t]{\textwidth}
\small
\begin{verbatim}
                                detector number
           1      2      3      4      5      6      7      8      9
         14.1,  13.7,  12.9,  13.1,  13.2,  13.2,  13.6,  6.10,  6.04,

          10     11     12     13     14     15     16     17     18
         6.07,  1.92,  4.85,  6.10,  6.19,  6.56,  3.39,   0.0,  3.46,

          19     20     21     22     23     24     25     26     27
         3.40,   0.0,  3.48,  3.38,  3.21,  3.22,  3.17,  0.92,  2.41,

          28     29     30     31     32     33     34     35     36
         2.95,  3.25,  3.25,  1.97,  6.40,  6.32,  6.25,  6.36,   0.0,

          37     38     39     40     41     42     43     44     45
         6.37,  4.84,  1.82,  3.50,  3.44,  3.48,  3.50,  3.45,  3.46,

          46     47     48     49     50     51     52     53     54
         1.78,  0.89,  3.31,  3.22,  3.29,  3.25,  3.19,  3.24,  2.45,

          55     56     57     58     59     60     61     62
          7.9,  13.3,  14.0,  12.9,  14.0,  13.8,  14.2,   6.8

\end{verbatim}
\normalsize
\end{minipage}
\end{itemize}
\section{Linking}

Some of the constants are defined within a BLOCKDATA module which resides in an
object library (or archive), and applications which include I90\_DAT must link
with this library using the IRAS90\_LINK and IRAS90\_LINK\_ADAM options files on
VMS and the {\bf iras90\_link} and {\bf iras90\_link\_adam} scripts on UNIX.
\end{document}

