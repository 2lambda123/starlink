\documentclass[twoside,11pt]{article}
\pagestyle{myheadings}

% -----------------------------------------------------------------------------
% ? Document identification
\newcommand{\stardoccategory}  {Starlink User Note}
\newcommand{\stardocinitials}  {SUN}
\newcommand{\stardocsource}    {sun\stardocnumber}
\newcommand{\stardocnumber}    {165.4}
\newcommand{\stardocauthors}   {David S Berry, Diana Parsons}
\newcommand{\stardocdate}      {19 February 1995}
\newcommand{\stardoctitle}     {IRAS90\\[2ex]
                               Programmer's Guide}
\newcommand{\stardocabstract}  {
{\small IRAS90} is a package of applications which provides facilities
for the reduction and analysis of {\small IRAS} Survey and Pointed
Observation (PO) data (also known as ``Additional Observation'', or
``AO'' data). It does not provide facilities for handling Low
Resolution Spectrometer (LRS) data, or catalogue data.
This document gives some guide-lines for users who want to produce their own
{\small IRAS90} applications.
}
% ? End of document identification

% -----------------------------------------------------------------------------

\newcommand{\stardocname}{\stardocinitials /\stardocnumber}
\markright{\stardocname}
\setlength{\textwidth}{160mm}
\setlength{\textheight}{230mm}
\setlength{\topmargin}{-2mm}
\setlength{\oddsidemargin}{0mm}
\setlength{\evensidemargin}{0mm}
\setlength{\parindent}{0mm}
\setlength{\parskip}{\medskipamount}
\setlength{\unitlength}{1mm}

% -----------------------------------------------------------------------------
%  Hypertext definitions.
%  ======================
%  These are used by the LaTeX2HTML translator in conjunction with star2html.

%  Comment.sty: version 2.0, 19 June 1992
%  Selectively in/exclude pieces of text.
%
%  Author
%    Victor Eijkhout                                      <eijkhout@cs.utk.edu>
%    Department of Computer Science
%    University Tennessee at Knoxville
%    104 Ayres Hall
%    Knoxville, TN 37996
%    USA

%  Do not remove the %begin{latexonly} and %end{latexonly} lines (used by 
%  star2html to signify raw TeX that latex2html cannot process).
%begin{latexonly}
\makeatletter
\def\makeinnocent#1{\catcode`#1=12 }
\def\csarg#1#2{\expandafter#1\csname#2\endcsname}

\def\ThrowAwayComment#1{\begingroup
    \def\CurrentComment{#1}%
    \let\do\makeinnocent \dospecials
    \makeinnocent\^^L% and whatever other special cases
    \endlinechar`\^^M \catcode`\^^M=12 \xComment}
{\catcode`\^^M=12 \endlinechar=-1 %
 \gdef\xComment#1^^M{\def\test{#1}
      \csarg\ifx{PlainEnd\CurrentComment Test}\test
          \let\html@next\endgroup
      \else \csarg\ifx{LaLaEnd\CurrentComment Test}\test
            \edef\html@next{\endgroup\noexpand\end{\CurrentComment}}
      \else \let\html@next\xComment
      \fi \fi \html@next}
}
\makeatother

\def\includecomment
 #1{\expandafter\def\csname#1\endcsname{}%
    \expandafter\def\csname end#1\endcsname{}}
\def\excludecomment
 #1{\expandafter\def\csname#1\endcsname{\ThrowAwayComment{#1}}%
    {\escapechar=-1\relax
     \csarg\xdef{PlainEnd#1Test}{\string\\end#1}%
     \csarg\xdef{LaLaEnd#1Test}{\string\\end\string\{#1\string\}}%
    }}

%  Define environments that ignore their contents.
\excludecomment{comment}
\excludecomment{rawhtml}
\excludecomment{htmlonly}

%  Hypertext commands etc. This is a condensed version of the html.sty
%  file supplied with LaTeX2HTML by: Nikos Drakos <nikos@cbl.leeds.ac.uk> &
%  Jelle van Zeijl <jvzeijl@isou17.estec.esa.nl>. The LaTeX2HTML documentation
%  should be consulted about all commands (and the environments defined above)
%  except \xref and \xlabel which are Starlink specific.

\newcommand{\htmladdnormallinkfoot}[2]{#1\footnote{#2}}
\newcommand{\htmladdnormallink}[2]{#1}
\newcommand{\htmladdimg}[1]{}
\newenvironment{latexonly}{}{}
\newcommand{\hyperref}[4]{#2\ref{#4}#3}
\newcommand{\htmlref}[2]{#1}
\newcommand{\htmlimage}[1]{}
\newcommand{\htmladdtonavigation}[1]{}

% Define commands for HTML-only or LaTeX-only text.
\newcommand{\html}[1]{}
\newcommand{\latex}[1]{#1}

% Use latex2html 98.2.
\newcommand{\latexhtml}[2]{#1}

%  Starlink cross-references and labels.
\newcommand{\xref}[3]{#1}
\newcommand{\xlabel}[1]{}

%  LaTeX2HTML symbol.
\newcommand{\latextohtml}{{\bf LaTeX}{2}{\tt{HTML}}}

%  Define command to re-centre underscore for Latex and leave as normal
%  for HTML (severe problems with \_ in tabbing environments and \_\_
%  generally otherwise).
\newcommand{\setunderscore}{\renewcommand{\_}{{\tt\symbol{95}}}}
\latex{\setunderscore}

% -----------------------------------------------------------------------------
%  Debugging.
%  =========
%  Remove % from the following to debug links in the HTML version using Latex.

% \newcommand{\hotlink}[2]{\fbox{\begin{tabular}[t]{@{}c@{}}#1\\\hline{\footnotesize #2}\end{tabular}}}
% \renewcommand{\htmladdnormallinkfoot}[2]{\hotlink{#1}{#2}}
% \renewcommand{\htmladdnormallink}[2]{\hotlink{#1}{#2}}
% \renewcommand{\hyperref}[4]{\hotlink{#1}{\S\ref{#4}}}
% \renewcommand{\htmlref}[2]{\hotlink{#1}{\S\ref{#2}}}
% \renewcommand{\xref}[3]{\hotlink{#1}{#2 -- #3}}
%end{latexonly}
% -----------------------------------------------------------------------------
% ? Document-specific \newcommand or \newenvironment commands.

% degrees symbol
\newcommand{\dgs}{\hbox{$^\circ$}} 

% centre an asterisk
\newcommand{\lsk}{\raisebox{-0.4ex}{\rm *}}

% A kind of list item, like description, but with an easily adjustable
% item separation.
\newcommand{\menuitem}[2]
  {{\bf #1}: \addtolength{\baselineskip}{-0.4ex}
  \parbox[t]{128mm}{#2} \addtolength{\baselineskip}{0.4ex} \\ \hspace{-5pt}}

% an environment for references
\newenvironment{refs}{\goodbreak
                      \vspace{3ex}
                      \begin{list}{}{\setlength{\topsep}{0mm}
                                     \setlength{\partopsep}{0mm}
                                     \setlength{\itemsep}{0mm}
                                     \setlength{\parsep}{0mm}
                                     \setlength{\leftmargin}{1.5em}
                                     \setlength{\itemindent}{-\leftmargin}
                                     \setlength{\labelsep}{0mm}
                                     \setlength{\labelwidth}{0mm}}
                    }{\end{list}}

% ? End of document-specific commands
% -----------------------------------------------------------------------------
%  Title Page.
%  ===========
\renewcommand{\thepage}{\roman{page}}
\begin{document}
\thispagestyle{empty}

%  Latex document header.
%  ======================
\begin{latexonly}
   CCLRC / {\sc Rutherford Appleton Laboratory} \hfill {\bf \stardocname}\\
   {\large Particle Physics \& Astronomy Research Council}\\
   {\large Starlink Project\\}
   {\large \stardoccategory\ \stardocnumber}
   \begin{flushright}
   \stardocauthors\\
   \stardocdate
   \end{flushright}
   \vspace{-4mm}
   \rule{\textwidth}{0.5mm}
   \vspace{5mm}
   \begin{center}
   {\Huge\bf  \stardoctitle}
   \end{center}
   \vspace{5mm}

% ? Heading for abstract if used.
   \vspace{10mm}
   \begin{center}
      {\Large\bf Abstract}
   \end{center}
% ? End of heading for abstract.
\end{latexonly}

%  HTML documentation header.
%  ==========================
\begin{htmlonly}
   \xlabel{}
   \begin{rawhtml} <H1> \end{rawhtml}
      \stardoctitle
   \begin{rawhtml} </H1> \end{rawhtml}

% ? Add picture here if required.
% ? End of picture

   \begin{rawhtml} <P> <I> \end{rawhtml}
   \stardoccategory\ \stardocnumber \\
   \stardocauthors \\
   \stardocdate
   \begin{rawhtml} </I> </P> <H3> \end{rawhtml}
      \htmladdnormallink{CCLRC}{http://www.cclrc.ac.uk} /
      \htmladdnormallink{Rutherford Appleton Laboratory}
                        {http://www.cclrc.ac.uk/ral} \\
      \htmladdnormallink{Particle Physics \& Astronomy Research Council}
                        {http://www.pparc.ac.uk} \\
   \begin{rawhtml} </H3> <H2> \end{rawhtml}
      \htmladdnormallink{Starlink Project}{http://star-www.rl.ac.uk/}
   \begin{rawhtml} </H2> \end{rawhtml}
   \htmladdnormallink{\htmladdimg{source.gif} Retrieve hardcopy}
      {http://star-www.rl.ac.uk/cgi-bin/hcserver?\stardocsource}\\

%  HTML document table of contents. 
%  ================================
%  Add table of contents header and a navigation button to return to this 
%  point in the document (this should always go before the abstract \section). 
  \label{stardoccontents}
  \begin{rawhtml} 
    <HR>
    <H2>Contents</H2>
  \end{rawhtml}
  \htmladdtonavigation{\htmlref{\htmladdimg{contents_motif.gif}}
        {stardoccontents}}

% ? New section for abstract if used.
  \section{\xlabel{abstract}Abstract}
% ? End of new section for abstract
\end{htmlonly}

% -----------------------------------------------------------------------------
% ? Document Abstract. (if used)
%  ==================
\stardocabstract
% ? End of document abstract
% -----------------------------------------------------------------------------
% ? Latex document Table of Contents (if used).
%  ===========================================
\newpage
\begin{latexonly}
   \setlength{\parskip}{0mm}
   \tableofcontents
   \setlength{\parskip}{\medskipamount}
   \markright{\stardocname}
\end{latexonly}
% ? End of Latex document table of contents
% -----------------------------------------------------------------------------
\newpage
\renewcommand{\thepage}{\arabic{page}}
\setcounter{page}{1}

\section{Introduction to IRAS90\xlabel{introduction_to_iras90}}

{\small IRAS90} is a package of applications which provides facilities
for the reduction and analysis of {\small IRAS} Survey and Pointed
Observation (PO) data (also known as ``Additional Observation'', or
``AO'' data). It does not provide facilities for handling Low
Resolution Spectrometer (LRS) data, or catalogue data. For more
information on existing {\small IRAS90} applications see
\xref{SUN/163}{sun163}{}.  This
document gives some guide-lines for users who want to produce their own
{\small IRAS90} applications.

{\small IRAS90} contains applications that can determine positions in 
astrometric coordinates, draw grids and other functions relating to some 
standard astronomical measurement systems and some standard projections. These 
do not provide the full flexibility required by a fully functional astrometric
system, but will provide limited capabilities for images which have {\small 
IRAS90} astrometry extensions containing the relevant positional information.

The {\small IRAS90} system is written in {\small FORTRAN} and is
integrated with the Starlink {\small ADAM} environment. Anyone
intending to create their own {\small IRAS90} applications should
therefore be familiar with {\small FORTRAN} and with the {\small ADAM}
programming interface
(see \xref{SUN/101}{sun101}{}). It is recommended that programmers
conform to the Starlink programming standards described in
\xref{SGP/16}{sgp16}{} when
producing {\small IRAS90} code as this will help to produce portable,
maintainable, reliable code which is consistent with the existing
{\small IRAS90} code. Adherence to the Starlink programming standards
is particularly important if you intend to submit your applications for
inclusion in the {\small IRAS90} package. 

\section{Writing Programs for IRAS90
\xlabel{writing_programs_for_iras90}}

The {\small IRAS90} Programs try achieve the following aims

\begin{itemize}

\item The results output from the applications are correct and the programs are
bug free.

\item The applications are interoperable with other Starlink applications.

\item The applications provide a common interface to the user, not only in
using the ADAM parameter system and Starlink error handling standards, but
also in the type of provisions they make and the way in which these operate.
Examples of this are the processing of multiple NDF's where files are
processed individually, the provision of repeat facilities, and facility
to provide a logfile or reprocessable data file.

\item The applications are written in such a way that they are easy to
understand. A programmer needing to enhance, debug, or use a set of code as an
example should find a structured program, containing sufficient comment that
he can not only understand the logic, but finds it also describes the problems 
presented by the application and the way in which the code overcomes the 
problem.

\end{itemize}

We therefore recommend that the potential programmer looks at three sources of
information.

\begin{itemize}

\item The source code for other similar  {\small IRAS90} applications in
particular those with a similar interface to the user, those whose processing
follows similar lines, and those that share common input or output files.

\item Documentation, both the published {\small IRAS90} Suns and the ID
documents associated with each of the Utility Subsystem packages.

\item Seek the advice of previous {\small IRAS90} programmers, David Berry,
Diana Parsons, and other Starlink programmers.

\end{itemize}

\section{Existing Source Code and Documentation
\xlabel{existing_source_code_and_documentation}}

Existing source code is stored within the released {\small IRAS90} system. 
The majority of routines can be classified as follows:

\begin{description}

\item [A-task routines :] These are the top-level routines for each
application and include user documentation in the form of a prologue.
The prologue also contains descriptions of each parameter used. 

\item [Other application-specific routines :] These are routines which
are specific to a particular application and which are called from the
applications A-task routine (or from another application-specific
routine). In general they perform a specific operation which is of
relevance only to a single application.

\item [Interface files :] These contain the interface, or parameter
specification, for each application.

\item [Utility subsystem routines :] These are routines which perform commonly
used operations. They are not associated with any one application and
can be called by any. Details of the subsystems are given in \ref{SEC:UTIL} 
below.

\item [Include files:] The Utility subsystems have associated with them include
files which contain parameter values, predefined constants and error message
values, which should be used {\small IRAS90} applications.

\end{description}

The location of these routines depends on the operating system being
used see sections \ref{SEC:UNIXS} and \ref{SEC:VMSS} below

\subsection{IRAS90 documentation\xlabel{iras90_documentation}}

The main {\small IRAS90} documentation consists of three Sun's: 

\begin{itemize}

\item \xref{SUN/161}{sun161}{} --
IRAS90 Introductory/User Guide to Calibrated Reconstructed Data Analysis.
 
\item \xref{SUN/163}{sun163}{} --
IRAS90 -- IRAS Survey and PO Data Analysis Package -- Reference Guide

\item \xref{SUN/165}{sun165}{} -- This document

\end{itemize}

These are all released through the normal Starlink channels

In addition there are three ID documents

\begin{itemize}

\item ID 10 -- MAPCRDD mapping algorithms
 
\item ID 28 -- IRAS90 Maintenance guide

\item ID 29 -- COLTEMP and COLCORR Algorithms

\end{itemize}

And there are ID documents associated with each Utility Subsystem described in
section \ref{SEC:UTIL} below.

The structure and contents of the CRDD NDF data file is described in the ID
document associated with IRC. Notes on the Astrometry extension will be found
in the IRA ID document. Notes on the processing applied to translate other 
standard forms of IRAS image data into IRAS90 Image NDF format are contained
in the PREPARE section of \xref{SUN/163}{sun163}{}.

Anyone needing copies of the ID documents will first need to process the 
distributed files using {\small LATEX}, etc. 


\subsection{Utility Routines
\xlabel{utility_routines}\label{SEC:UTIL}}

Many general purpose utility routines have been written as part of the
{\small IRAS90} package. Use of these routines can save much effort,
and is indeed essential for many operations. The utility routines are
divided into several ``libraries'', each library containing routines
associated with a specific function. The libraries all have three
letter names beginning with ``IR'':

Each sub-system is described in a separate document distributed with
the {\small IRAS90} package. These files give a
general over-view of the purpose of the sub-system and also included
specific details about the routines contained in the sub-system. 

\begin{description}

\item [IRA :] contains routines associated with accessing astrometry
information within two dimensional images, performing spherical
trigonometry, handling of formatted sky coordinate values (eg getting
sky coordinates from the user), and generation of astrometric graphics
(eg sky coordinate grids, meridians, parallels, etc). {\small IRA}
documentation is in file {\small ID2.TEX}. The full {\small ID2}
document is a pain to prepare. A user who is content with a document in 
which some diagrams eg of projections, are missing can run {\small ID2.TEX}
through Latex three times and print the result. If the diagrams are required
we suggest contacting IRAS or Starlink staff.

\item [IRC :] contains routines used for accessing descriptive
information associated with {\small CRDD} files. {\small IRC}
documentation is in file {\small ID1.TEX}.

\item [IRI :] contains routines for controlling the descriptive
information (except astrometry which is handled by {\small IRA})
associated with {\small IRAS90} images. {\small IRI} documentation is
in file {\small ID12.TEX}.

\item [IRM :] contains a miscellaneous selection of routines, covering
a wide range of general tasks. {\small IRM} documentation is in file
{\small ID8.TEX}.

\item [IRQ :] contains routines for accessing and using quality
information within {\small IRAS90} images and {\small CRDD} files.
{\small IRA} documentation is in file {\small ID6.TEX}.

\end{description}

In addition, there is a sub-system called I90 which consists of a set
of include files, and block data files, containing definitions of a
large number of symbolic constants and array values. These contain an
assortment of data describing the {\small IRAS} satellite and mission,
and the {\small IRAS90} package. It is documented in file {\small
ID20.TEX}.

\subsection{Location of code and documentation on UNIX Systems
\xlabel{location_of_code_and_documentation_on_unix_systems}\label{SEC:UNIXS}}

All A-task routines, applications specific routines and interface files are
stored in the 
\newline
\$IRAS90\_SOURCE directory which is defined when the command iras90 is executed.

The majority of source code are stored in XXXX\_source.tar files as follows

\begin{description}

\item [apps\_source.tar :] contains A\_task routines and other application 
specific routines.

\item [data\_source.tar :] contains IRAS90 data, including the point spread 
functions for each detector, and the SPF archive file containing details of
the {\small IRAS} observation schedule. The spectral response files are 
generated by makeresp, but the files can be inspected by looking in 
\$IRAS90\_DIR.

\item [ifl\_source.tar :] contains the interface files for {\small IRAS90}
applications.

\item [iras90\_source.tar :] contains non {\small ADAM} applications eg pscont
and ffield, common files used by {\small IRAS90} applications, the tex files 
for the {\small ID} documents not associated with utility subsystems, the help
file, and various {\small IRAS90} system files.

\item [ira\_source.tar (etc.) :] There is a xxx\_source.tar file for each 
utility subsystem including {\small i90}. Note that the file names are all 
small letters not capitals. Each file contains all the source code, the
include files containing parameter values, constants,and error values, plus the
tex of the corresponding {\small ID} document.

\item [Files not included in a XXX\_source.tar file :] These include the 
tex files and postscript and canon picture files associated with the released 
SUNs.

\end{description}

The contents of an archive file can be listed using the command: 

\small
\begin{verbatim}
   % tar tvf $IRAS90_SOURCE/apps_source.tar
\end{verbatim}
\normalsize

To get the A-task source code for the application {\small DESTCRDD}, use 
the command: 

\small
\begin{verbatim}
   % tar xvf $IRAS90_SOURCE/apps_source.tar destcrdd.f 
\end{verbatim}
\normalsize

To extract the interface file for application {\small DESTCRDD}, use the
command: 

\small
\begin{verbatim}
   % tar xvf $IRAS90_SOURCE/ifl_source.tar destcrdd.ifl 
\end{verbatim}
\normalsize

\subsection{Location of code and documentation on VMS Systems
\xlabel{location_of_code_and_documentation_on_vms_systems}\label{SEC:VMSS}}

On {\small VMS} systems the source code is stored in, or in subdirectories of,
{\small IRAS90\_DIR}. This logical name is set up by running {\small IRAS90}.
The contents are as follows

\begin{itemize}

\item A-task routines are stored in the text library
{\small IRAS90\_DIR:IRAS90.TLB}, and other application specific
routines are stored in he text library {\small
IRAS90\_DIR:IRAS90\_SUB.TLB}. These correspond to the contents of 
apps\_source.tar on a Unix machine.

\item Interface files are stored in {\small IRAS90\_DIR:IRAS90\_IFL.TLB}.
This corresponds to the contents of ifl\_source.tar on a Unix machine.

\item The source code of each utility subsystem's  routines are stored
in IRx.TLB  in the subdirectory [.IRx] under the directory IRAS90\_DIR.
The corresponding include files and {\small ID} documents are stored loose
in the subdirectory. There are two exceptions. In {\small IRA} the {\small ID2}
document has a tex file and {\small ID2.TLB} and {\small ID2\_IFL.TLB} which
contain files to build diagrams. {\small I90} consists of include files
and support files only, therefore there is no {\small I90.TLB}.  These
correspond to the contents of the xxx\_source.tar files on Unix where xxx is a
utility subsystem name.

\item The contents of the VMS equivalent of data\_source.tar and 
iras90\_source.tar on Unix, are held loose in {\small IRAS90\_DIR}.

\item The released SUNs are available only as a printed copy.

\end{itemize}


To list the existing application specific routines, use the command:

\small
\begin{verbatim}
   $ LIB/LIST IRAS90_DIR:IRAS90_SUB.TLB
\end{verbatim}
\normalsize

To get the A-task source code for the application {\small DESTCRDD}, use the 
command:

\small
\begin{verbatim}
   $ LIB/EXT=DESTCRDD/OUT=DESTCRDD.FOR IRAS90_DIR:IRAS90.TLB
\end{verbatim}
\normalsize

To extract the interface file for application {\small DESTCRDD}, use the
command:

\small
\begin{verbatim}
   $ LIB/EXT=DESTCRDD/OUT=DESTCRDD.IFL IRAS90_DIR:IRAS90_IFL.TLB
\end{verbatim}
\normalsize

\section{IRAS90 Programming Conventions
\xlabel{iras90_programming_conventions}\label{SEC:CONV}}

This section contains descriptions of {\em some} of the conventions
which have been adhered to in the production of the released {\small
IRAS90} applications.  If you intend to submit your applications for
inclusion in {\small IRAS90} you should ensure that they also conforms
to these conventions:

\begin{enumerate}

\item If an application needs to read or write a list of character
strings, the facilities of the {\small GRP} package
(\xref{SUN/150}{sun150}{}) should be
used if possible.  {\small GRP} provides the user with a very versatile
system for specifying the strings, and avoids the use of explicit file
handling routines. As an example, see the way that {\small REMQUAL}
accesses the {\small QNAMES} parameter, or the way that {\small SKYPOS}
handles the files associated with parameters {\small FILE} and {\small
LOGFILE}.

\item If multiple {\small NDF}s are processed or created, then the list
of {\small NDF}s should be obtained using the facilities of the {\small
IRM} and {\small NDG} packages (see section \ref{SEC:UTIL} and appendix
\ref{APP:NDG}). As an example, see the way that {\small MAPCRDD} accesses 
its list of input {\small CRDD} files.

\item If the application processes a single input {\small NDF} to
create a single output {\small NDF}, then it should be written to allow
the user to specify a group of input {\small NDF}s, each being
processed to create a corresponding output {\small NDF}. See {\small
DESTCRDD, BACKCRDD} etc.

\item Applications should in general always check the {\small UNITS}
components of the input {\small NDF}s to ensure that the data is in a
system of units which the application knows how to process. The {\small
IRAS90} package has a set of ``standard'' units which is described in
the documentation for the {\small IRI} sub-system (see section
\ref{SEC:UTIL}). The {\small IRM} sub-system contains routines for
converting data values between system of units.

\item If appropriate, applications should allow the user to specify
which pixels or samples within the input {\small NDF}s are to be
processed, by prompting for a ``quality expression'' (these are
described in the {\small IRQ} sub-system documentation). See {\small
DESTCRDD} for an example of the use of quality expressions.

\item \label{IT:NDFLIST} If the application can produce more than one
output {\small NDF} it should create a text file holding a list of the
created output {\small NDF}s and associate this file with global
parameter {\small GLOBAL.DATA\_ARRAY}. This can be done using utility
routine {\small IRM\_LISTN}, specifying {\small ADAM} parameter
``{\small NDFLIST}''. See {\small DESTCRDD} for an example (also look
at the specification of parameters {\small OUT} and {\small NDFLIST} in
the interface file for {\small DESTCRDD}).

\item Checks for bad pixel values should always be made unless a call to 
{\small NDF\_BAD} has been made to confirm that the {\small NDF} contains 
no bad values.

\item If convenient, the bad pixel flag should be set in all output {\small 
NDF}s, using routine {\small NDF\_SBAD}.

\item The {\small PSX} library should be used for getting dynamic workspace
rather than the {\small ARY} library.

\item Application specific subroutines should have names in the form
{\em xxxxyz}, where {\em xxxx} are the first four characters from the
application name, {\em y} is a single letter (a-z), and {\em z} is a
single digit (0-9).  If there already exists an application with the
same first four letters in its name, then the preference for {\em xxxx}
is the first, fifth, sixth and seventh letters of the application name.
If this cannot be done, then any unique string can be used, but it
should indicate which application the routine is associated with. For
instance, the routines associated with {\small MAPCRDD} are named
{\small MAPCA0}, {\small MAPCA1}, {\small MAPCB0}, etc.

\item If the application processes multiple input {\small NDF}s then
the user should be told which {\small NDF} is currently being processed
using a message of the form "  Processing \verb+^+NDF..." where
\verb+^+NDF is replaced by the name of the NDF (see {\small DESTCRDD}
for instance).

\item If the application processes multiple input {\small NDF}s, then
any errors which occur while processing an input {\small NDF} should be
flushed (using routine ERR\_FLUSH) and the application should continue
to process the other input {\small NDF}s (see {\small DESTCRDD} for
instance).

\item The following names should be used for parameters associated with 
{\small NDF}s:

\begin{description}

\item [IN] - if {\small READ} access is required to the {\small NDF} (i.e. if 
the {\small NDF} is read but not modified).

\item [OUT] - if {\small WRITE} access is required to the {\small NDF} (i.e. if 
a new {\small NDF} is created).

\item [NDF] - if {\small UPDATE} access is required to the {\small NDF} (i.e. if
the {\small NDF} is read and then modified). 

\end{description}

\item All groups identified by {\small GRP} identifiers should be
deleted using routine {\small GRP\_DELET} before the application
finishes (even if the application aborts due to an error).

\item If appropriate, applications should propagate history from the
input to the output {\small NDF}s.

\item \label{IT:HIS} Applications should add history information to the
output {\small NDF}s using routine {\small IRM\_HIST}, specifying
{\small ADAM} parameter ``{\small HISTORY}''. The value supplied for
argument {\small COMMAND} should commence with the string ```IRAS90:''.
The history should contain all input and output {\small NDF} names,
together with key parameter values, etc. See {\small DESTCRDD} for an
example.

\item \label{IT:MSG_FILTER} Each application should set up a
conditional message filtering level by calling routine {\small
MSG\_IFGET}, specifying {\small ADAM} parameter {\small MSG\_FILTER}.

\item All messages displayed with routines {\small MSG\_OUTIF} and
{\small MSG\_OUT} should correspond to parameters with names of the
form {\em xxxxx\_MSGy} where {\em xxxxxx} is the name of the routine,
and {\em y} is a number which increases monotonically for each message
produced by the routine (starting at 1 for the first message).

\item All messages should start with at least two spaces.

\item All explicit assignments to the inherited status value ``{\small
STATUS}'' should be guarded against the possibility of a pre-existing
error value being over-written, as in the following code fragment:

\small
\begin{verbatim}
   *  If the pixel size is zero, report an error.
         IF( PIXSIZ .EQ. 0.0 .AND. STATUS .EQ. SAI__OK ) THEN
            STATUS = SAI__ERROR
            CALL ERR_REP( 'CTEMA0_ERR1', 
        :                 'CTEMA0: Zero pixel size encountered',
        :                 STATUS )
            GO TO 999
         END IF
\end{verbatim}
\normalsize

\item All ``{\small DO WHILE}'' loops should terminate if {\small STATUS} 
indicates an error has occurred:

\small
\begin{verbatim}
   *  Loop until all values have been done, or an error occurs.
         MORE = .TRUE.
         DO WHILE ( MORE .AND. STATUS .EQ. SAI__OK ) 
            ...
            ...
         END DO
\end{verbatim}
\normalsize

\item Any subroutine call which includes variable array sizes within
its list of arguments should be preceded with a status check (to avoid
adjustable array dimensions errors occurring if an array size of zero
is passed to the subroutine because of the error condition):

\small
\begin{verbatim}
   *  Map the data array.
         CALL NDF_MAP( INDF, 'DATA', '_REAL', 'READ', IPIN, EL, STATUS )

   *  Abort if an error has occurred.
         IF( STATUS .NE. SAI__OK ) GO TO 999

   *  Find the total data sum in the data array.
         CALL CTEMA1( EL, %VAL( IPIN ), SUM, STATUS )
\end{verbatim}
\normalsize

\item If a subroutine call uses the {\small \%VAL} construct to pass a
mapped character array to the subroutine, the final argument in the
argument list should be the length of each character string in the
array. This length should be passed using the {\small \%VAL} construct.
Particular care is needed if the argument list contains other character
values as well
(see \xref{SUN/92}{sun92}{}, section ``Mapping Character Data'').

\item If a constant value is used in an argument list for an argument
which corresponds to a double precision value, then the constant must
have the correct double precision form (i.e. must terminate with a
``D'' exponent specifier, eg 1.456D2).

\item Care should be taken that calls to {\small ERR\_ANNUL} cannot
annul the wrong error. One way to do this is to include a status check
before the subroutine call which generates the status to be annulled:

\small
\begin{verbatim}
   *  Get a value for parameter BOXSIZE.
         CALL PAR_GET0R( 'BOXSIZE', BOXSIZ, STATUS )

   *  Abort if an error has occurred.
         IF( STATUS .NE. SAI__OK ) GO TO 999

   *  Get a value for parameter SIGMA.
         CALL PAR_GET0R( 'SIGMA', SIGMA, STATUS )

   *  If a null value was supplied, annul the error and indicate that
   *  no filtering is required.
         IF( STATUS .EQ. PAR__NULL ) THEN
            CALL ERR_ANNUL( STATUS )
            FILT = .FALSE.
         ELSE
            FILT = .TRUE.
         END IF
\end{verbatim}
\normalsize

\item All error reports should correspond to parameters with names of the form
{\em xxxxx\_ERRy} where {\em xxxxxx} is the name of the routine, and {\em y} is
a number which increases monotonically for each call to {\small ERR\_REP} made
by the routine (starting at 1 for the first error). 

\item The text of error reports should start with the name of the routine in 
which the call to {\small ERR\_REP} is made, followed by a colon and a space. 
They should include as much information as possible about the conditions under 
which the error occurred:

\small
\begin{verbatim}
            CALL NDF_MSG( 'NDF', INDF )
            CALL MSG_SETI( 'C', COLUMN )
            CALL MSG_SETI( 'R', ROW )
            CALL MSG_SETR( 'D', DATA( COLUMN, ROW ) )
            CALL MSG_SETR( 'L', MAXDAT )

            CALL ERR_REP( 'CTEMA2_ERR1', 
        : 'CTEMA2: Illegal value (^D) found at column ^C, row ^R of '//
        : '^NDF. Maximum legal value is ^L.', STATUS )
\end{verbatim}
\normalsize

\item Checks for the error conditions {\small PAR\_\_NULL} and {\small
PAR\_\_ABORT} should be made and, if found, these errors should be annulled
before returning from an application. 

\item The very last thing that an application does before returning should be to
check the status and issue a contextual report if an error has occurred. 

\item Output {\small NDF}s should be deleted if an error occurs while output 
data is being calculated.

\small
\begin{verbatim}
   *  Create the output NDF.
         CALL NDF_PROP( INDF1, ' ', 'OUT', INDF2, STATUS )

   *  Map the output data array.
         CALL NDF_MAP( INDF2, 'DATA', '_REAL', 'READ', IPOUT, EL, STATUS )

   *  Abort if an error has occurred.
         IF( STATUS .NE. SAI__OK ) GO TO 999

   *  Fill the output data array with values.
         CALL CTEMC3( EL, %VAL( IPOUT ), STATUS )

   *  If an error has occurred, delete the output NDF.
    999  CONTINUE
         IF( STATUS .NE. SAI__OK ) CALL NDF_DELET( INDF2, STATUS )
\end{verbatim}
\normalsize

\item The A-task prologue should conform to the style generated by the Language 
Sensitive Editor {\small STARLSE}
(see \xref{SUN/105}{sun105}{}).

\item The first two words of the A-task prologue ``Description'' section should be ``This 
routine''.

\item The ``Description'' section should be as short as possible and give just a
general over-view of what the application does. Different ways of using the
application should not be included in the ``Description'' but should be put into
separate ``DIY topics'' sections, following the ``Examples'' section (eg see
{\small TRACECRDD} prologue). 

\item Any general information included in the A-task prologue which will need to
be duplicated in other applications should be removed from the prologue and put
into the on-line help library. The prologue should contain a reference to the
relevant topic within the help library. (This will only be of relevance to
people updating the released {\small IRAS90} system). 

\item References to the on-line help library should be made if any potentially
unfamiliar general terms or concepts are mentioned in the prologue (rather than
duplicating descriptions within the applications prologue). For example,
``Detector numbers should be given in the form of a group expression (see help
on ``Group\_Expressions'')''. 

\item The A-task prologue should contain a description of how {\em each} {\small NDF}
component is processed, particularly a description of what happens to the
{\small QUALITY} and {\small VARIANCE} components should be included.

\item The ``Usage'' and ``Examples'' sections of the standard {\small STARLSE}
prologue should be filled in. 

\item Parameter should be described in alphabetical order.

\item {\em All} parameters used by the application should be described in the 
prologue (except for {\small NDFLIST}, see item \ref{IT:NDFLIST}).

\item If a parameter is normally defaulted, the description of the parameter
should include the default value by placing it in square brackets at the end of
the parameter description. 

\item The parameter descriptions should make it clear if a scalar value, array 
or group expression is required for the parameter.

\item Information should not be duplicated in different sections of the 
prologue, but should be included once only, with relevant references in other 
sections.

\item Parameters associated with groups of {\small NDF}s should refer to a 
``group'' of {\small NDF}s (not a ``set'', or ``list'', etc).

\item If a parameter is associated with a group expression, and if modification
elements are allowed within the group expression, then a description of what the
``{\small NAME\_TOKEN}'' character (usually an asterisk) refers to should be
included in the parameter description. 

\item There should be no occurrences of the word ``{\small ADAM}'' within the 
prologue.

\item Documentation generated from the A-task prologue using {\small PROLAT}
(see \xref{SUN/110}{sun110}{}) should successfully pass through a spelling
checker such as {\small SPELL}.

\item All parameters used within the application should have entries in the 
interface file.

\item Parameters should be listed in alphabetical order in the interface file.

\item The {\small HELPLIB} field in the interface file should have the value
``{\small IRAS90\_DIR:IRAS90\_HELP}'' (on {\small VMS} systems), or 
``{\small \$IRAS90\_DIR/iras90\_help.shl}'' (on {\small UNIX} systems).

\item Each parameter should have a {\small HELPKEY} field with the value 
``\lsk''.

\item The following associations with ``global'' parameters should be used:
\begin{description}
\item [GLOBAL.DATA\_ARRAY ] should be read by parameters associated with input
{\small NDF}s, and written by ``{\small NDFLIST}'' parameters (see item
\ref{IT:NDFLIST}), and {\em not} by parameters associated with output {\small
NDF}s unless there is no {\small NDFLIST} parameter. 
\item [GLOBAL.MSG\_FILTER ] should be read and written by ``{\small 
MSG\_FILTER}'' parameters (see item \ref{IT:MSG_FILTER}).
\item [GLOBAL.QUALITY\_EXP ] should be read and written by parameters associated
with quality expression (but only used to provide the suggested default, i.e.
the {\small PPATH}, not the {\small VPATH}). 
\item [GLOBAL.IRAS90\_HISTORY] should be read and written by all ``{\small 
HISTORY} parameters'' (see item \ref{IT:HIS}).
\item [GLOBAL.SKY\_COORDS] should be read and written by all parameters 
associated with a sky coordinate system.
\end{description}

\item As many parameters as possible should be normally defaulted.

\item Positional parameters should be given positions in the same order in which
they are prompted for if not supplied on the command line. 

\item {\small IRAS} wave-bands should be obtained using the routine {\small
IRM\_GTBND} specifying the parameter ``{\small BAND}''. 

\item Status Values Returned by Utility Routines ---

The utility routines (and all other {\small IRAS90} routines) use the 
``inherited status'' error system described in 
\xref{SUN/104}{sun104}{}. Utility routines flag 
errors by reporting an error message and setting the returned status to some 
specific value indicating the type of error which has occurred. These status 
values are unique amongst {\em all} status values used by any {\small ADAM} 
software, and are calculated using the expression:

\small
\begin{verbatim}
     134250498 + 65536*<facility number> + 8*<err_num>
\end{verbatim}
\normalsize

where \verb+<facility number>+ is a number reserved by Starlink for the
{\small IRAS90} package (currently 1507), and \verb+<err_num>+ is an
error number internal to IRAS90, in the range 1 - 4095. Each utility
package has been assigned a range of error numbers as follows:

\begin{description}
\item [1 -   30] : IRA
\item [31 -   60] : IRQ
\item [61 -   90] : NDG
\item [91 -  120] : IRC
\item [121 -  200] : IRM
\item [201 -  230] : Not currently used.
\item [231 -  241] : IRQ (extra to 31 - 60)
\item [242 -  300] : IRA (extra to 1 - 30)
\item [301 -  330] : IRI
\item [331 - 4095] : Not currently used.
\end{description}


\end{enumerate}

\section{Compiling, Linking and Running
\xlabel{compiling_linking_and_running}}

\subsection{On UNIX Systems\xlabel{on_unix_systems}}

Before starting to develop {\small IRAS90} software the following commands 
should be issued:

\small
\begin{verbatim}
   % iras90
   % iras90_dev
\end{verbatim}
\normalsize

The iras90\_dev script sets up softlinks to all the {\small IRAS90} utility
subsystem and Starlink include files in your current directory. It also sets
up an environment variable {\small \$IRAS90\_SOURCE} to point to the source
directory in which the {\small IRAS90} link scripts and link libraries are 
stored, and adds it to your path.

Note, to develop {\small IRAS90} applications on {\small UNIX} systems
requires the {\small IRAS90} ``source directory'' {\small \$IRAS90\_SOURCE} 
to be present. It is possible for system managers
to delete the source directory in order to save disk space, without
stopping the released {\small IRAS90} applications from working. If the
necessary {\small IRAS90} files do not exist on your system the
\verb+iras90_dev+ script will issue a warning.


Next all the application specific routines (but not the {\small A\_task}) 
should be compiled to create corresponding object modules using the 
\verb+f77+ command. If the application name is \verb+fred+, the commands 
could be: 

\small 
\begin{verbatim}
   % f77 -O -c freda0.f
   % f77 -O -c freda1.f
     ...  
     (etc)
     ...
\end{verbatim}
\normalsize

This assumes that the application conforms to the conventions described
in section \ref{SEC:CONV} in that the names of the application specific
routines end in \verb+a0+, \verb+a1+, ... \verb+b0+, \verb+b1+, ...
etc. An archive library should then be created and the object modules
for all the application specific routines ({\em not} the A-task object
module) should be inserted into the archive. If you are not running under
Solaris the archive should finally be ``randomised'':

\small 
\begin{verbatim}
   % ar c fred.a
   % ar r fred.a freda0.o
   % ar r fred.a freda1.o
     ...  
     (etc)
     ...
   % ranlib fred.a
\end{verbatim}
\normalsize

The application \verb+fred+ can then be linked using the following command: 

\small
\begin{verbatim}
   % alink fred.f fred.a -L/star/lib -L$IRAS90_SOURCE \
     `iras90_link`
\end{verbatim}
\normalsize

Note the use of left quote marks (\verb+`+) rather than the more usual right
quote marks (\verb+'+). After completing all compilations and linkings, the
command: 

\small
\begin{verbatim}
   % iras90_dev remove
\end{verbatim}
\normalsize

should be executed in order to remove unnecessary files (actually soft
links to other files) created by the initial \verb+iras90_dev+ command.
The application can be run by just typing its name together with any
command line parameter assignments or keywords:

\small
\begin{verbatim}
   % fred in=m31 out=\*_proc
\end{verbatim}
\normalsize

If the application is to be run from a directory other than the one
containing the executable and interface files, then an alias can be set
up for it using the command:

\small
\begin{verbatim}
   % alias fred <directory path>/fred
\end{verbatim}
\normalsize

\subsection{On VMS Systems\xlabel{on_vms_systems}}

Before starting to develop {\small IRAS90} software the following commands 
should be issued:

\small
\begin{verbatim}
   $ ADAMSTART
   $ ADAMDEV
   $ IRAS90
   $ IRAS90_DEV
\end{verbatim}
\normalsize

The A-task routine and all the application specific routines should
then be compiled as usual using the {\small FORTRAN} command. An object
library should then be created and the object modules for all the
application specific routines ({\em not} the A-task object module)
should be inserted into the library. If the application name is {\small
FRED}, the commands would be:

\small 
\begin{verbatim}
   $ LIB/CRE FRED.OLB
   $ LIB FRED FREDA0
   $ LIB FRED FREDA1
     ...  
     (etc)
     ...
\end{verbatim}
\normalsize

This assumes that the application conforms to the conventions described in
section \ref{SEC:CONV} in that the names of the application specific routines
end in {\small A0, A1, ... B0, B1, ... } etc. The application FRED can then be
linked using the following command: 

\small
\begin{verbatim}
   $ ALINK FRED,FRED/LIB,IRAS90_LINK_ADAM/OPT
\end{verbatim}
\normalsize

Before running the application an interface file should be created (a
text file with the same name as the application and a file type of
{\small .IFL}). The contents of the interface file should conform to
the conventions described in section {\ref{SEC:CONV}. Again, it is a
good idea to use an existing {\small IRAS90} interface file as a basis
for a new one. The simplest way to run the application is just to use
the {\small RUN} command:

\small
\begin{verbatim}
   $ RUN FRED
\end{verbatim}
\normalsize

However, the {\small RUN} command does not allow anything to appear
after the application name on the command line, and so command line
specification of parameter values and keywords is not possible. To get
round this a ``foreign command'' must be set up to run the application.
To do this first ensure that a logical name exists pointing to the
directory containing files {\small FRED.EXE} and {\small FRED.IFL}. If
this logical name is {\small FRED\_DIR}, then the command:

\small
\begin{verbatim}
   $ FRED:==$FRED_DIR:FRED
\end{verbatim}
\normalsize

will define the foreign command {\small FRED}, so that the application can 
be run (from any directory) with command line parameter assignments:

\small
\begin{verbatim}
   $ FRED IN=M31 OUT=*_PROC
\end{verbatim}
\normalsize

To run the application from {\small ICL}, first start up {\small ICL} and then 
issue the command:

\small
\begin{verbatim}
   ICL> DEFINE FRED FRED_DIR:FRED
\end{verbatim}
\normalsize

Again, {\small FRED\_DIR} is a logical name pointing to the directory
containing {\small FRED.EXE} and {\small FRED\-.IFL}. Note, this logical
name must be a {\em job} logical name, set up prior to starting {\small
ICL} with the command:

\small
\begin{verbatim}
   $ DEFINE/JOB FRED_DIR <directory specification>
\end{verbatim}
\normalsize

\appendix
\section{The NDG Library
\xlabel{the_ndg_library}\label{APP:NDG}}

The {\small NDG} library (which is used for accessing multiple {\small NDF}s
using a single parameter) started life as an {\small IRAS90} utility library
called {\small IRG}. It will eventually be released as a Starlink subroutine
library, independent of {\small IRAS90}. Until then, it continues to be
distributed as part of the {\small IRAS90} package in the same way as the other
utility libraries. The draft documentation is in the file {\small SUN2.TEX}
which is stored in the same places as the other utility library documentation.
This documentation is rather sketchy at the moment, but hopefully it will be
sufficient, when combined with studies of existing {\small IRAS90} application,
to enable programmers to use {\small NDG}. 

\end{document}
