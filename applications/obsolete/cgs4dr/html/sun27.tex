\documentclass[a4paper]{book}
\usepackage{graphics}
\pagenumbering{roman}

%------------------------------------------------------------------------------
\newcommand{\stardoccategory}  {Starlink User Note}
\newcommand{\stardocinitials}  {SUN}
\newcommand{\stardocnumber}    {27.5}
\newcommand{\stardocauthors}   {P N Daly}
\newcommand{\stardocaddress}   {pnd$@$jach.hawaii.edu}
\newcommand{\stardocdate}      {8 April 1996}
\newcommand{\stardoctitle}     {Portable--CGS4DR \\
                                CGS4 Data Reduction}
\newcommand{\stardocversion}   {Version V1.3-0}
\newcommand{\stardocmanual}    {Users' Guide}
%------------------------------------------------------------------------------
\newcommand{\radec}     {$[\alpha,\delta\,]$}
\newcommand{\hadec}     {$[h,\delta\,]$}
\newcommand{\azel}      {$[Az,El\,]$}
\newcommand{\gal}       {$[l^{I\!I},b^{I\!I}]$}
\newcommand{\xy}        {$[x,y\,]$}
\newcommand{\xyz}       {$[x,y,z\,]$}
\newcommand{\xyzd}      {$[\dot{x},\dot{y},\dot{z}\,]$}
\newcommand{\xyzxyzd}   {$[x,y,z,\dot{x},\dot{y},\dot{z}\,]$}
\newcommand{\arcsec}    {$\hspace{-0.05em}\raisebox{-0.5ex}
                        {$^{'\hspace{-0.1em}'}$}
                        \hspace{-0.7em}.\hspace{-0.05em}$}
\newcommand{\tsec}      {\mbox{$^{\rm s}\!\!.$}}
\newcommand{\hms}[4]    {$#1^{\rm h}\,#2^{\rm m}\,#3\tsec#4$}
\newcommand{\dms}[4]    {$#1^{\circ}\,#2\raisebox{-0.5ex}{$^{'}$}\,#3\arcsec#4$}
%------------------------------------------------------------------------------
\newcommand{\stardocname}{\stardocinitials /\stardocnumber}
\markright{\stardocname}
\setlength{\textwidth}{160mm}
\setlength{\textheight}{230mm}
\setlength{\topmargin}{-2mm}
\setlength{\oddsidemargin}{0mm}
\setlength{\evensidemargin}{0mm}
\setlength{\parindent}{0mm}
\setlength{\parskip}{\medskipamount}
\setlength{\unitlength}{1mm}
\renewcommand{\_}{{\tt\char'137}}
\voffset = -1.0cm
%------------------------------------------------------------------------------
% Add any \newcommand or \newenvironment commands here
\renewcommand{\thepage}{\roman{page}}
%  Define length variables.
\newlength{\sstbannerlength}
\newlength{\sstcaptionlength}
%  Define a \tt font of the required size.
\font\ssttt=cmtt10 scaled 1095
%  Define a command to produce a routine header, including its name,
%  a purpose description and the rest of the routine's documentation.
\newcommand{\sstroutine}[3]{
   \goodbreak
   \rule{\textwidth}{0.5mm}
   \vspace{-7ex}
   \newline
   \settowidth{\sstbannerlength}{{\Large {\bf #1}}}
   \setlength{\sstcaptionlength}{\textwidth}
   \addtolength{\sstbannerlength}{0.5em}
   \addtolength{\sstcaptionlength}{-2.0\sstbannerlength}
   \addtolength{\sstcaptionlength}{-4.45pt}
   \parbox[t]{\sstbannerlength}{\flushleft{\Large {\bf #1}}}
   \parbox[t]{\sstcaptionlength}{\center{\Large #2}}
   \parbox[t]{\sstbannerlength}{\flushright{\Large {\bf #1}}}
   \begin{description}
      #3
   \end{description}
}
%  Format the description section.
\newcommand{\sstdescription}[1]{\item[Description:] #1}
%  Format the usage section.
\newcommand{\sstusage}[1]{\item[Usage:] \mbox{} \\[1.3ex] {\ssttt #1}}
%  Format the invocation section.
\newcommand{\sstinvocation}[1]{\item[Invocation:]\hspace{0.4em}{\tt #1}}
%  Format the arguments section.
\newcommand{\sstarguments}[1]{
   \item[Arguments:] \mbox{} \\
   \vspace{-3.5ex}
   \begin{description}
      #1
   \end{description}
}
%  Format the returned value section (for a function).
\newcommand{\sstreturnedvalue}[1]{
   \item[Returned Value:] \mbox{} \\
   \vspace{-3.5ex}
   \begin{description}
      #1
   \end{description}
}
%  Format the parameters section (for an application).
\newcommand{\sstparameters}[1]{
   \item[Parameters:] \mbox{} \\
   \vspace{-3.5ex}
   \begin{description}
      #1
   \end{description}
}
%  Format the examples section.
\newcommand{\sstexamples}[1]{
   \item[Examples:] \mbox{} \\
   \vspace{-3.5ex}
   \begin{description}
      #1
   \end{description}
}
%  Define the format of a subsection in a normal section.
\newcommand{\sstsubsection}[1]{\item[{#1}] \mbox{} \\}
%  Define the format of a subsection in the examples section.
\newcommand{\sstexamplesubsection}[1]{\item[{\ssttt #1}] \mbox{} \\}
%  Format the notes section.
\newcommand{\sstnotes}[1]{\item[Notes:] \mbox{} \\[1.3ex] #1}
%  Provide a general-purpose format for additional (DIY) sections.
\newcommand{\sstdiytopic}[2]{\item[{\hspace{-0.35em}#1\hspace{-0.35em}:}] \mbox{} \\[1.3ex] #2}
%  Format the implementation status section.
\newcommand{\sstimplementationstatus}[1]{
   \item[{Implementation Status:}] \mbox{} \\[1.3ex] #1}
%  Format the bugs section.
\newcommand{\sstbugs}[1]{\item[Bugs:] #1}
%  Format a list of items while in paragraph mode.
\newcommand{\sstitemlist}[1]{
  \mbox{} \\
  \vspace{-3.5ex}
  \begin{itemize}
     #1
  \end{itemize}
}
%  Define the format of an item.
\newcommand{\sstitem}{\item}
%  Define cross and tick, footnote symbols
\newcommand{\cross}{$\times$}
\newcommand{\tick}{$\surd$}
\renewcommand{\thefootnote}{\fnsymbol{footnote}}

%End of LAYOUT.TEX layout definitions.
%------------------------------------------------------------------------------

\begin{document}
\thispagestyle{empty}
CCLRC / {\sc Rutherford Appleton Laboratory} \hfill {\bf \stardocname}\\
{\large Particle Physics \& Astronomy Research Council}\\
{\large Starlink Project\\}
{\large \stardoccategory\ \stardocnumber}
\begin{flushright}
\stardocauthors\footnote[2]{{\em Present Address:}
 {\sf Joint Astronomy Centre, 660 N. A'oh\={o}k\={u} Place, University Park,
  Hilo HI 96720, USA.}} \\
{\em \stardocaddress} \\
\stardocdate
\end{flushright}
\vspace{-4mm}
\rule{\textwidth}{0.5mm}
\vspace{5mm}
\begin{center}
{\Huge\bf  \stardoctitle \\ [2.5ex]}
{\LARGE\bf \stardocversion \\ [4ex]}
{\Huge\bf  \stardocmanual}
\end{center}
\vspace{20mm}
%------------------------------------------------------------------------------
%  Introduction page

\newpage
\thispagestyle{empty}

\vspace*{\fill}

\begin {center}
\rule{80mm}{0.5mm} \\ [1ex]
{\Large\bf \stardoctitle \\ [2.5ex]
           \stardocversion} \\ [2ex]
\rule{80mm}{0.5mm}
\end{center}

\vspace*{\fill}

%  Table of contents.  Page numbering in roman.
%  -------------------------------------------
\newpage
\pagenumbering{roman}
\pagestyle{myheadings}
\markright{\stardocname}

\tableofcontents
\setlength{\parskip}{\medskipamount}

\newpage

%------------------------------------------------------------------------------
%  Acknowledgements page
\chapter*{Acknowledgements}
\markboth{Acknowledgements}{\stardocname}
\label{acknowledgements}

The following have contributed to Portable--CGS4DR or its predecessors:
Jeremy Bailey, Steven Beard, Phil Blanco, Alan Bridger, Bob Carswell, Alan 
Chipperfield, Phil Daly, Tom Geballe, Graeme Harkness, Kevin Krisciunas, 
John Lightfoot, William Lupton, Chris Mayer, Dave Mills, Matt Mountain, 
Alan Pickup, Phil Puxley, Suzanne Ramsay, Keith Shortridge, Dave Terrett 
and Gillian Wright. There may be others whom I have neglected to mention. 
My thanks to them all.
                                               
Sun is a trademark of Sun Microsystems, Inc.
Unix is a registered trademark of UNIX System Laboratories, Inc.
PostScript is a registered trademark of Adobe Systems, Inc.
{\sc pgplot} is copyright by the California Institute of Technology.

\vspace{15mm}
\begin{center}
\rule{80mm}{0.5mm}

\vspace{15mm}

{\tt http://www.jach.hawaii.edu/ukirt\_sw/}.

\vspace{15mm}

\rule{80mm}{0.5mm}

\vspace{15mm}

Reference: \ \ Daly, P. N. 1996, SUN/27, Starlink Project, CLRC.
\end{center}

\cleardoublepage
\pagenumbering{arabic}
%------------------------------------------------------------------------------
%  Part I 
\part{INTRODUCTION TO CGS4DR}
\pagestyle{myheadings}
\markboth{Introduction to Portable--CGS4DR}{\stardocname}

\chapter{Introduction}
\section{Introduction to Portable--CGS4DR}
The fourth generation {\sc cooled grating spectrometer}, CGS4, is
designed to operate on UKIRT in the 1--5 $\mu$m region of the
electromagnetic spectrum at resolutions in the range
$\lambda$/$\Delta\lambda$ $\sim$ 300--20000 (Mountain et al. 1990,
Ramsay 1993). To reduce background noise, it is maintained on the
telescope in vacuum at cryogenic temperatures. It achieved first light
at UKIRT on 4 February 1991 (Carswell et al. 1991). On 22 April 1995 a
new InSb 256 $\times$ 256 array was commissioned into the instrument.
The new array having a dark current of $<<$ 1 $e^{-}$ $s^{-1}$ and a
read noise of $\sim$ 40 $e^{-}$ per integration, is much more sensitive
than previous detectors. On any given spectroscopic night, an observer
can expect to acquire and reduce $\sim$ 100 Mb of high quality data
with CGS4.

To an observer, the system appears as two black boxes: the data
acquisition system and the data reduction system (Wright et al. 1994).
Both are under the direct control of the observer and can be
manipulated to deal with a wide variety of astronomical configurations.
The data is, therefore, both acquired and reduced in real-time at the
telescope.

This document describes the CGS4 data reduction system now called
Portable--CGS4DR. The software has been carefully designed and tested
under the Figaro (Shortridge 1993) and {\sc adam} (Lawden \& Hartley 1992)
environments under Unix.  The data files produced by Portable--CGS4DR
are readable by standard Figaro applications although not all may
handle the quality and error arrays correctly.

For best results, Portable--CGS4DR should be run on a single user
workstation having at least 32 Mb of main memory and a colour monitor.
It can, of course, run on a dumb terminal and display to any {\sc
starlink} supported graphics device.

If you think you have discovered a bug, please report it by e-mail to
{\tt ussc$@$star.rl.ac.uk}.

\section{What It Can Do}
First, it can reduce spectrographic data in an automatic way.  No
system, however, will do everything every user will ever want so some
post processing may be required in certain circumstances. The aim is to
produce publishable quality spectra at the telescope via an automated
reduction paradigm.

Briefly, Portable--CGS4DR can do the following:

\begin{itemize}
\item Allows a wide variety of data reduction configurations.
\item Interlaces oversampled data frames.
\item Reduce known {\sc bias}, {\sc dark}, {\sc flat}, {\sc arc}, {\sc object}
and {\sc sky} frames.
\item Wavelength calibrate via suitable {\sc arc} lines.
\item Flux calibrate via a suitable {\sc standard}.
\item Remove the {\sc sky}, residual sky OH-lines ($\lambda <$ 2.3 $\mu$m) 
  and thermal emission ($\lambda \geq$ 2.3 $\mu$m) from data. 
\item Add data into groups for improved signal-to-noise.
\item Extract and de-ripple a spectrum.
\item Maintains an index of reduced observations.
\item Maintains, if required, an archive of observations.
\item Plot data in a variety of ways.
\end{itemize}

The basic concept of the system is based upon re-scheduling to read a 
data reduction queue.  This queue is maintained in memory.

\section{Version History}
This is the fourth incantation of the Portable--CGS4DR software. The
functionality of each release is:

\begin{description}
\item[] {\bf V1.0--0}
 \begin{enumerate}
  \item First release of Portable-CGS4DR.
  \item Supports reduction of all frames into calibrated or uncalibrated reduced 
        observations and groups in {\sc dst} or {\sc ndf} format using {\sc stare} 
        or {\sc ndstare} observing modes.
  \item Supports calibration of {\sc arc} into reduced calibration frames.
  \item Supports observation filing, dividing by a standard source and automated
        extraction of spectra.
  \item Supports ICL command line interface.
 \end{enumerate}
\item[] {\bf V1.1--0}
 \begin{enumerate}
  \item Supports all features in V1.0-0.
  \item Supports user defined bad pixel masks.
  \item Supports 256$\times$256 {\sc alice} data in all modes (including {\sc chop}).
  \item New features include set\_/get\_polyfit, set\_/get\_file\_selection, set\_/get\_cut\_dir
        and oversampled {\sc flat} fields.
  \item Bug fixes for endgroup and plot\_graph.
  \item Support for a demo.
 \end{enumerate}
\item[] {\bf V1.2--0}
 \begin{enumerate}
  \item Supports all features in V1.1-0.
  \item Bug fixes for pause\_on\_error, ukirtfig and NBS.
  \item Supports {\sc tcladam} {\sc gui}.
 \end{enumerate}
\item[] {\bf V1.3--0}
 \begin{enumerate}
  \item Supports all features in V1.2-0.
  \item Corrects problems with bad pixel mask generation.
  \item Bug fixes to irflat, deripple, create\_window\_mask, divide\_by\_ff.
  \item Conforms to Figaro V5 structure.
  \item Removes NAG dependency by using PDA\_ routines.
  \item Essential engineering functions added (two row/col line fitting).
  \item Supports dynamic defaults in GUI.
  \item Supports on-line help in GUI.
  \item Fully documented on WWW (using client-side image maps etc).
  \item Adds default foreground/background colours to GUI.
 \end{enumerate}
\end{description}

\section{Some Basic Definitions}
Preliminary reduction of array data is common to both spectroscopy and
photometry and techniques describing the removal of instrumental
effects are noted elsewhere (Daly 1987). Before starting to use
Portable--CGS4DR, however, you should be familiar with the reduction of
spectroscopic data (Puxley 1988).  It would also be advantageous to be
familiar with Figaro data reduction (Bailey 1990, Bailey 1991).
Portable--CGS4DR software knows about the following observation types:

\begin{description}
\item[{\sc bias}]--- A short blank observation.
\item[{\sc dark}]--- Blank observation with the same on-chip exposure time as 
 another observation.
\item[{\sc flat}]--- Observation of black-body source in calibration unit, 
 the sky or the illuminated dome.
\item[{\sc object}]--- Observation of an astronomical object.
\item[{\sc sky}]--- Observation to subtract from an {\sc object} frame to 
 perform first order sky subtraction. 
\item[{\sc arc}]--- Uncalibrated observation of the arc lamp or OH-lines from
 the sky. 
\item[{\sc calibr}]--- An {\sc arc} observation converted into a wavelength 
 calibrated frame.
\item[{\sc standard}]--- A standard star observation divided by a model 
 black-body.
\end{description}

In a common data reduction sequence the following operations would be
carried out: 

\begin{enumerate}
\item Apply a bad pixel mask.
\item Subtract a {\sc bias} frame.
\item Linearise the signal.
\item Subtract a {\sc dark} frame.
\item Divide by a {\sc flat} field.
\item Interlace integrations taken at different detector positions.
\item Calibrate x-axis into wavelength.
\item Co-add observations and subtract {\sc sky} frames.
\item Divide by a {\sc standard} source.
\item Extract up to 3 spectra nodded along the slit.
\end{enumerate}

These steps are configurable to allow for a wide variety of astronomical
configurations. When observations are reduced they are recorded in an
index file of the form {\sc cgs4\_}{\em yymmdd.}{\sc index}, where {\em
yyddmm} is the UT date on which the observations were made, in the
directory {\sc \$cgs4\_index/}. The system will try to replace the date
string with an appropriate value whenever possible based upon universal
time. The contents of the index file may also be manipulated. 

\section{Files Used by the System}
When the data acquisition system obtains a raw data frame from the control
system, it writes it to an integration file whose name is of the form
iyymmdd\_oooo\_iiii where `yymmdd' is the UT date, `oooo' the observation
number and `iiii' the integration number. Integration files are,
therefore, frames taken at a single detector position and are written to
the directory {\sc \$idir/}. If a single integration is reduced, the
result is written to a reduced integration file whose name is of the form
riyymmdd\_oooo\_iiii, written to the directory {\sc \$ridir/}. 

In CGS4, however, the spectrum may be oversampled by combining several
integrations at different detector positions. A description of the
instrument configuration during an observation is held in an observation
file whose name is of the form oyymmdd\_oooo in the directory {\sc
\$odir/}.  This information is combined with the data in the integration
files to produce a reduced observation of size $x \times y*n$ elements
where {\em n} is the oversampling factor and $x \times y$ is the readout
area.  This reduced observation, whose name is of the form royymmdd\_oooo,
is held in the directory {\sc \$rodir/}.  The reduced observation file
contains the reduced data for a single observation. 

When observing astronomical objects, individual observations may be of
{\sc object} or {\sc sky}. Any number of {\sc object} and {\sc sky}
observations relating to one particular astronomical object may be
combined together into a reduced group whose name is of the form
rgyymmdd\_gggg, where `gggg' is a unique group number, which is the first
observation in the group as defined by the data acquisition system, held
in the directory {\sc \$rgdir/}. These reduced group files contain
sky-subtracted data. These files may be manipulated further by the
automatic data reduction sequence to provide enhanced sky-subtraction for
point sources and to ratio by a standard star.  Depending upon the options
selected the prefixes \_pf, \_dbs, \_spc, \_imspc may be appended to the
reduced group filenames. For example, a group called rg920530\_17 which is
divided by a standard will produce a group called rg920530\_17\_dbs. If
the same group is `polysky-ed', divided by a standard and a spectrum is
extracted, two files will be created named rg920530\_17\_pf\_dbs\_spc and
rg920530\_17\_pf\_dbs\_imspc. The Unix file clobbering mechanism should
ensure that only one version of each file exists at any one time. 

The system will clear up any temporary files upon receipt of an {\sc
endgroup} command in the data reduction queue. The data acquisition system
places such a command in the queue automatically but the data reduction
system does not. 

The data reduction system uses two other types of file. Wavelength
calibrated {\sc calibr} frames are held in files of type cayymmdd\_oooo in
the {\sc \$rodir/} directory. Standard star observations are held in files
whose names are of the form styymmdd\_gggg, held in the {\sc \$rgdir/}
directory. 

Note that the files contained in {\sc \$idir/} and {\sc \$odir/} are never
over-written whereas the files in {\sc \$ridir/}, {\sc \$rodir/} and {\sc
\$rgdir/} may be over-written by re-reducing the data. 

\section{How Does Portable--CGS4DR Select Calibration Frames?}
The files of type {\sc bias}, {\sc dark}, {\sc flat}, {\sc calibr} and
{\sc standard} are generically termed calibration frames\footnote[2]{In
this document, I have tried to use the lowercase calibrate or calibration
to refer to the technique or a generic frame and the uppercase {\sc
calibr} to refer to a specific wavelength calibration observation.}. When
such frames are required by the software to reduce another observation, a
search is made on the index file of already reduced observations. 

Normally, the user lets the software decide which calibration frames to
use. In general, it is true to say that the nearest observation in time
that matches the instrument configuration of the observed object is
selected and such frames are memory mapped to prevent unnecessary
processing thus speeding up the reduction considerably. 

If a specific observation is chosen, the software will use that
observation irrespective of any criteria mis-match. In verbose mode the
software will report any mis-matched criteria to the user but otherwise
the user may be unaware that the frame is unsuitable. Indeed, if a user
wants to know why certain calibration frames are being rejected, setting
verbose output to {\sc true} is the only way of obtaining such
information. 

\begin{table}
\begin{center}
\caption{\bf Calibration Frame Matching Criteria} \label{tab1}
\vglue 0.6cm
\begin{tabular}{|l|c|c|c|c|c|}
\hline
\ \ & \ \ & \ \ & \ \ & \ \ & \ \ \\
{\em Criteria} & {\sc bias} & {\sc dark} & {\sc flat} & {\sc calibr} & {\sc standard} \\
\ \ & \ \ & \ \ & \ \ & \ \ & \ \ \\
\hline
\ \ & \ \ & \ \ & \ \ & \ \ & \ \ \\
Detector size               & \tick   & \tick   & \tick   & \cross  & \tick  \\
Detector columns            & \cross  & \cross  & \cross  & \tick   & \cross \\
On-chip Exposure Time       & \cross  & \tick   & \cross  & \cross  & \cross \\
Observation mode (NDR)      & \cross  & \tick   & \cross  & \cross  & \cross \\
Oversampling parameters     & \cross  & \cross  & \tick   & \tick   & \tick  \\
Grating name                & \cross  & \cross  & \tick   & \cross  & \tick  \\
Grating wavelength          & \cross  & \cross  & \tick   & \cross  & \tick  \\
Slit name                   & \cross  & \cross  & \tick   & \cross  & \cross \\
Configuration index         & \cross  & \cross  & \cross  & \tick   & \cross \\
\ \ & \ \ & \ \ & \ \ & \ \ & \ \ \\
\hline
\end{tabular}
\vglue 0.3cm
{\em Key:} \cross = not checked; \tick = must match.
\end{center}
\end{table}

There are two types of calibration frame checking within Portable--CGS4DR. 
Compulsory checks are hard-wired into the code and consist of basic
integrity checks such as detector size and so forth. Optional checks are
defined only for oversampled frames and are specified by the flat\_match,
calib\_match and standard\_match parameters in the task interface file. In
both cases, these criteria are not configurable by the end user but are,
for completeness, shown in table \ref{tab1}. 

These matches are performed by evaluating and equating various {\sc fits}
items pertinent to each observation. Note that the configuration index in
the table is a time-stamp referring to when the grating was last moved and
as such provides the strictest criteria. 

\chapter{Reducing Astronomical Array Data}
\markboth{General DR Paradigms}{\stardocname}
\section{Eliminating Data with Bad Pixel Masks}
A bad pixel mask is a data frame that indicates which pixels on the
detector array are good and which bad. It may be used to mask off faulty
detectors. Normally the mask contains 0 to indicate good, and 1 to
indicate bad. It should be the same size as the area of the data array
read by the data acquisition system. Bad pixel masks are held in the {\sc
\$cgs4\_masks/} directory. 

Besides flagging faulty detectors, a bad pixel mask can also be used for
masking off un-illuminated areas at the edge of the detector array. 

A bad pixel mask can be created by the software using one of three
methods: 

\begin{enumerate}
 \item By setting all the data values lying outside a given range to bad.
 Some sample data are plotted as a histogram, and the required range is 
 selected using a cursor. This is called `thresholding'.
 \item By setting all the pixels outside a given window on the detector 
 array to bad.
 \item By thresholding an error array.
\end{enumerate}

Masks generated by any method may be combined together. 

The data reduction system can change masks by responding to a `{\sc
drmask} {\em maskname}' command entered into the data reduction queue.
There is also an option to specify no mask at all (\#). 

Bad pixel masks can be any size or shape. When a mask is created, it
assumes the same size and shape as the data from which it was created. The
size and shape of the mask created by windowing can also be specified. Of
course, allowing masks to be any size means there is a chance that the
wrong type of mask may be specified when reducing some data. The data
reduction system will check the bad pixel mask against each integration
being reduced to ensure it is the right size and will report if it is not. 

Portable--CGS4DR is shipped with several bad pixel masks specific to the
current arrays: 

\begin{description}
\item[{\sf fpa61}] --- Masks the known dead, hot and variable pixels for the 
58 $\times$ 62 InSb array in CGS4.
\item[{\sf fpa61\_lc}] --- To be used with the CGS4 long focal length camera 
 observations when the whole array is illuminated.
\item[{\sf fpa61\_75}] --- To be used with the CGS4 short focal length camera
 observations and the 75 or 150 lines per millimetre gratings.
\item[{\sf alice\_fpa42}] --- Masks the known dead, hot and variable pixels 
for the InSb 256$\times$256 array in IRCAM3.
\item[{\sf fpa46\_short}] --- Masks the known dead, hot and variable pixels 
for the InSb 256$\times$256 array in CGS4 using the short camera.
\end{description}

For the echelle, it is better to make your own windows as the illuminated area
is a strong function of the grating angle.

Note that there are problems with pixel conventions when manipulating
masks. Some packages use the {\sc starlink} pixel convention and some the Figaro
convention which differ by 0.5 in each direction. Combining masks can lead
to errors. The usual way around this problem is to use the Figaro
functions {\sc xcsub} and {\sc ycsub}\footnote[2]{To be released `real
soon now'!} to adjust to the appropriate origin. Portable--CGS4DR conforms
to the Figaro pixel convention where the data is centered on the
half-integer values {\em i.e.} the origin is at (0.5,0.5). 

\section{Reducing a Bias}
To reduce a {\sc bias}, Portable--CGS4DR reads the raw data from store in
{\sc \$idir/} and {\sc fits} header information from store in {\sc
\$odir/}, applies a bad pixel mask, writes the reduced data parameters
into the index file and stores the reduced {\sc bias} in {\sc \$rodir/} as
shown in figure \ref{latfig1}. 

\begin{figure}[htbp]
\begin{center}
\setlength{\unitlength}{0.012500in}
\begin{picture}(281,180)(19,635)
\thicklines
\put(160,735){\circle{70}}
\put( 60,815){\line( 1, 0){ 40}}
\put( 60,795){\line( 1, 0){ 40}}
\put(220,815){\line( 1, 0){ 40}}
\put(220,795){\line( 1, 0){ 40}}
\put( 20,735){\line( 1, 0){ 40}}
\put( 20,715){\line( 1, 0){ 40}}
\put(260,735){\line( 1, 0){ 40}}
\put(260,715){\line( 1, 0){ 40}}
\put(140,655){\line( 1, 0){ 40}}
\put(140,635){\line( 1, 0){ 40}}
\put( 80,795){\vector( 3,-2){ 54.231}}
\put(240,795){\vector(-3,-2){ 54.231}}
\put(160,700){\vector( 0,-1){ 45}}
\put( 65,725){\vector( 4, 1){ 58.824}}
\put(195,735){\vector( 4,-1){ 58.824}}
\put( 80,800){\makebox(0,0)[b]{\raisebox{0pt}[0pt][0pt]{\rm ODIR}}}
\put(240,800){\makebox(0,0)[b]{\raisebox{0pt}[0pt][0pt]{\rm IDIR}}}
\put( 40,720){\makebox(0,0)[b]{\raisebox{0pt}[0pt][0pt]{\rm MASKS}}}
\put(280,720){\makebox(0,0)[b]{\raisebox{0pt}[0pt][0pt]{\rm INDEX}}}
\put(160,640){\makebox(0,0)[b]{\raisebox{0pt}[0pt][0pt]{\rm RODIR}}}
\put(160,740){\makebox(0,0)[b]{\raisebox{0pt}[0pt][0pt]{\rm APPLY}}}
\put(160,720){\makebox(0,0)[b]{\raisebox{0pt}[0pt][0pt]{\rm MASK}}}
\put(140,680){\makebox(0,0)[b]{\raisebox{0pt}[0pt][0pt]{\scriptsize reduced}}}
\put(175,680){\makebox(0,0)[b]{\raisebox{0pt}[0pt][0pt]{\scriptsize bias}}}
\put(205,780){\makebox(0,0)[b]{\raisebox{0pt}[0pt][0pt]{\scriptsize raw}}}
\put(240,780){\makebox(0,0)[b]{\raisebox{0pt}[0pt][0pt]{\scriptsize bias}}}
\put( 80,780){\makebox(0,0)[b]{\raisebox{0pt}[0pt][0pt]{\scriptsize fits}}}
\put(120,780){\makebox(0,0)[b]{\raisebox{0pt}[0pt][0pt]{\scriptsize header}}}
\put(220,735){\makebox(0,0)[b]{\raisebox{0pt}[0pt][0pt]{\scriptsize bias}}}
\put(220,720){\makebox(0,0)[b]{\raisebox{0pt}[0pt][0pt]{\scriptsize info}}}
\put( 85,735){\makebox(0,0)[b]{\raisebox{0pt}[0pt][0pt]{\scriptsize b.p.m.}}}
\end{picture}
\end{center}
\caption{Reducing a {\sc bias}} \label{latfig1}
\end{figure}

Note that {\sc bias} frames are not subtracted when data is taken in {\sc
ndr} (non-destructive read) mode---even if requested---since the {\sc ndr}
algorithm has already accounted for the bias level in the data. 

\section{Linearising the Data}
After {\sc bias} subtraction, the signal from the detector array may be
corrected for non-linearity by modifying it with a suitable polynomial of
the form: 

\begin{equation}
  S(x) = Ax + Bx^{2} + Cx^{3} + Dx^{4} + Ex^{5} \label{equation_1}
\end{equation}

where $x$ is the measured signal and $S(x)$ is proportional to the light
intensity falling on the detector. These coefficients may determined
manually by examining {\sc flat} fields taken at different exposure times
as the software does not contain any automated way of doing this. Such
coefficients are specified in {\sc ascii} text in a linearization file: 

\begin{verbatim}
! +
! \$CGS4_MASKS/fpa61_linear.dat
! Linearization coefficients determined on 23-Jan-1991 by A CGS4 User
! Order given is A, B, C, D, E vertically.
! -
0.9980
0.0176
-0.0037
6.46E-5
0.0000
\end{verbatim}

Note that this means that all pixels are calibrated with the same
polynomial. The file type must be .dat and it is stored in the directory
{\sc \$cgs4\_masks/}. Note that, although every effort has been made to
avoid arithmetic errors during linearization, floating point overflows can
occur if the coefficients are set to ridiculous values. The higher order
terms, in particular, should be $<<$ 1.00 to avoid such problems. 

For the arrays used in CGS4, no measurable non-linearity has been found to
within 98\% of full well. For these reasons, this facility is normally
turned off by setting the value to \#. 

\section{Reducing a Dark}
To reduce a {\sc dark}, Portable--CGS4DR reads the raw data from store in
{\sc \$idir}/ and {\sc fits} header information from store in {\sc
\$odir/}, applies a bad pixel mask, optionally subtracts a reduced {\sc
bias} frame if in stare mode, linearises the data (if required) and writes
the reduced data parameters into the index file and stores the reduced
{\sc dark} in {\sc \$rodir/} as shown in figure \ref{latfig2}.

Note that {\sc dark} frames are not normally subtracted when the data
is reduced as {\sc object---sky} pairs.

%\begin{figure}[!b]
\begin{figure}[hpbt]
\begin{center}
\setlength{\unitlength}{0.012500in}
\begin{picture}(285,340)(15,475)
\thicklines
\put( 40,560){\makebox(0,0)[b]{\raisebox{0pt}[0pt][0pt]{\rm COEFFS}}}
\put( 15,575){\line( 1, 0){ 50}}
\put( 15,555){\line( 1, 0){ 50}}
\put( 40,720){\makebox(0,0)[b]{\raisebox{0pt}[0pt][0pt]{\rm MASKS}}}
\put( 20,735){\line( 1, 0){ 40}}
\put( 20,715){\line( 1, 0){ 40}}
\put( 40,655){\makebox(0,0)[b]{\raisebox{0pt}[0pt][0pt]{\rm RODIR}}}
\put( 20,670){\line( 1, 0){ 40}}
\put( 20,650){\line( 1, 0){ 40}}
\put( 60,815){\line( 1, 0){ 40}}
\put( 60,795){\line( 1, 0){ 40}}
\put( 65,725){\vector( 4, 1){ 58.824}}
\put( 65,660){\vector( 1, 0){ 55}}
\put( 65,565){\vector( 1, 0){ 55}}
\put( 80,795){\vector( 3,-2){ 54.231}}
\put( 80,800){\makebox(0,0)[b]{\raisebox{0pt}[0pt][0pt]{\rm ODIR}}}
\put( 80,780){\makebox(0,0)[b]{\raisebox{0pt}[0pt][0pt]{\scriptsize fits}}}
\put( 80,645){\makebox(0,0)[b]{\raisebox{0pt}[0pt][0pt]{\scriptsize bias}}}
\put( 85,550){\makebox(0,0)[b]{\raisebox{0pt}[0pt][0pt]{\scriptsize coeffs}}}
\put( 85,735){\makebox(0,0)[b]{\raisebox{0pt}[0pt][0pt]{\scriptsize b.p.m.}}}
\put( 85,665){\makebox(0,0)[b]{\raisebox{0pt}[0pt][0pt]{\scriptsize reduced}}}
\put( 90,570){\makebox(0,0)[b]{\raisebox{0pt}[0pt][0pt]{\scriptsize linear}}}
\put(120,780){\makebox(0,0)[b]{\raisebox{0pt}[0pt][0pt]{\scriptsize header}}}
\put(140,495){\line( 1, 0){ 40}}
\put(140,475){\line( 1, 0){ 40}}
\put(140,510){\makebox(0,0)[b]{\raisebox{0pt}[0pt][0pt]{\scriptsize reduced}}}
\put(160,735){\circle{70}} % apply mask
\put(160,740){\makebox(0,0)[b]{\raisebox{0pt}[0pt][0pt]{\rm APPLY}}}
\put(160,720){\makebox(0,0)[b]{\raisebox{0pt}[0pt][0pt]{\rm MASK}}}
\put(160,710){\vector( 0,-1){ 35}}
\put(160,650){\circle{70}} % -bias
\put(160,650){\makebox(0,0)[b]{\raisebox{0pt}[0pt][0pt]{\rm --BIAS}}}
\put(160,620){\vector( 0,-1){ 35}}
\put(160,560){\circle{70}} % linear
\put(160,557){\makebox(0,0)[b]{\raisebox{0pt}[0pt][0pt]{\rm LINEAR}}}
\put(160,530){\vector( 0,-1){ 35}}
\put(160,480){\makebox(0,0)[b]{\raisebox{0pt}[0pt][0pt]{\rm RODIR}}}
\put(175,510){\makebox(0,0)[b]{\raisebox{0pt}[0pt][0pt]{\scriptsize dark}}}
\put(200,560){\vector( 1, 0){ 55}}
\put(205,780){\makebox(0,0)[b]{\raisebox{0pt}[0pt][0pt]{\scriptsize raw}}}
\put(220,815){\line( 1, 0){ 40}}
\put(220,795){\line( 1, 0){ 40}}
\put(220,570){\makebox(0,0)[b]{\raisebox{0pt}[0pt][0pt]{\scriptsize dark}}}
\put(220,550){\makebox(0,0)[b]{\raisebox{0pt}[0pt][0pt]{\scriptsize info}}}
\put(225,660){\makebox(0,0)[b]{\raisebox{0pt}[0pt][0pt]{\scriptsize bias}}}
\put(225,645){\makebox(0,0)[b]{\raisebox{0pt}[0pt][0pt]{\scriptsize info}}}
\put(240,795){\vector(-3,-2){ 54.231}}
\put(240,800){\makebox(0,0)[b]{\raisebox{0pt}[0pt][0pt]{\rm IDIR}}}
\put(240,780){\makebox(0,0)[b]{\raisebox{0pt}[0pt][0pt]{\scriptsize dark}}}
\put(254,664){\vector(-4,-1){ 54.118}}
\put(260,670){\line( 1, 0){ 40}}
\put(260,650){\line( 1, 0){ 40}}
\put(260,575){\line( 1, 0){ 40}}
\put(260,555){\line( 1, 0){ 40}}
\put(280,655){\makebox(0,0)[b]{\raisebox{0pt}[0pt][0pt]{\rm INDEX}}}
\put(280,560){\makebox(0,0)[b]{\raisebox{0pt}[0pt][0pt]{\rm INDEX}}}
\end{picture}
\end{center}
\caption{Reducing a {\sc dark}} \label{latfig2}
\end{figure}

\section{Reducing a Flat}
To reduce a {\sc flat}, Portable--CGS4DR reads the raw data from store in
{\sc \$idir/} and {\sc fits} header information from store in {\sc
\$odir/}, applies a bad pixel mask, optionally subtracts a reduced {\sc
bias} frame, linearises the data (if required), optionally subtracts a
reduced {\sc dark} (with the same on-chip exposure time), normalises the
data (if required) and writes the reduced data parameters into the index
file and stores the reduced {\sc flat} in {\sc \$rodir/} as shown in figure
\ref{latfig3}.

For normalisation a 1-D spectrum is extracted from within the window
defined by the bad pixel mask and a polynomial is fitted to that spectrum.
The order of the polynomial is defined by the user and the polynomial is
grown along the slit to fill the array. Finally, the original flat field
is divided by the polynomial frame to produce a normalised flat field.
Normalisation also removes the spectrum of the black body calibration
source used to generate the flat field. 

An alternative method of normalising the data using a smoothing box is also
available.

For spectral configurations, the flat field algorithm pre-supposes that
the illumination from the black body source is uniform along the slit. 
This does mean that if a very accurate photometric comparison of spectra
at different points along the slit is required, the flat field will need
to be checked for variations in illumination and refined to remove this
variability. 

It has also been discovered that at some wavelengths, the {\sc cvf} acts
like a Fabry-Perot and produces diagonal stripes in observations. These
should be removed using oversampled {\sc flat} fields although this has
yet to be tried. An alternative strategy to remove these stripes is to
take an {\sc object} observation of the black body source at the same
oversampling factor and normalise this particular observation to create a
pseudo-{\sc flat} field. Subsequently divide your real observation by this
observation. 

Finally, if the black body has not been allowed to warm up sufficiently,
or the detector array temperature varies, or the detector is allowed to
saturate resulting in ghost images and so forth, the spectrum generated
may be too noisy. Observers should pay due care and attention to acquiring
the data. 

\begin{figure}[htbp]
\begin{center}
\setlength{\unitlength}{0.012500in}%
\begin{picture}(285,480)(15,335)
\thicklines
\put(160,735){\circle{70}}
\put(160,560){\circle{70}}
\put(160,650){\circle{70}}
\put(160,480){\circle{70}}
\put(160,400){\circle{70}}
\put( 60,815){\line( 1, 0){ 40}}
\put( 60,795){\line( 1, 0){ 40}}
\put(220,815){\line( 1, 0){ 40}}
\put(220,795){\line( 1, 0){ 40}}
\put( 20,735){\line( 1, 0){ 40}}
\put( 20,715){\line( 1, 0){ 40}}
\put( 80,795){\vector( 3,-2){ 54.231}}
\put(240,795){\vector(-3,-2){ 54.231}}
\put( 65,725){\vector( 4, 1){ 58.824}}
\put( 20,670){\line( 1, 0){ 40}}
\put( 20,650){\line( 1, 0){ 40}}
\put(260,670){\line( 1, 0){ 40}}
\put(260,650){\line( 1, 0){ 40}}
\put( 15,575){\line( 1, 0){ 50}}
\put( 15,555){\line( 1, 0){ 50}}
\put(160,710){\vector( 0,-1){ 35}}
\put(160,620){\vector( 0,-1){ 35}}
\put( 65,660){\vector( 1, 0){ 55}}
\put(254,664){\vector(-4,-1){ 54.118}}
\put( 65,565){\vector( 1, 0){ 55}}
\put(140,355){\line( 1, 0){ 40}}
\put(140,335){\line( 1, 0){ 40}}
\put(160,530){\vector( 0,-1){ 25}}
\put(160,450){\vector( 0,-1){ 25}}
\put(160,375){\vector( 0,-1){ 20}}
\put( 20,495){\line( 1, 0){ 40}}
\put( 20,475){\line( 1, 0){ 40}}
\put( 20,415){\line( 1, 0){ 40}}
\put( 20,395){\line( 1, 0){ 40}}
\put(260,415){\line( 1, 0){ 40}}
\put(260,395){\line( 1, 0){ 40}}
\put(260,475){\line( 1, 0){ 40}}
\put(260,495){\line( 1, 0){ 40}}
\put( 65,485){\vector( 1, 0){ 65}}
\put(255,480){\vector(-1, 0){ 65}}
\put( 60,405){\vector( 1, 0){ 70}}
\put(190,400){\vector( 1, 0){ 65}}
\put( 80,800){\makebox(0,0)[b]{\raisebox{0pt}[0pt][0pt]{\rm ODIR}}}
\put(240,800){\makebox(0,0)[b]{\raisebox{0pt}[0pt][0pt]{\rm IDIR}}}
\put( 40,720){\makebox(0,0)[b]{\raisebox{0pt}[0pt][0pt]{\rm MASKS}}}
\put(160,740){\makebox(0,0)[b]{\raisebox{0pt}[0pt][0pt]{\rm APPLY}}}
\put(160,720){\makebox(0,0)[b]{\raisebox{0pt}[0pt][0pt]{\rm MASK}}}
\put(205,780){\makebox(0,0)[b]{\raisebox{0pt}[0pt][0pt]{\scriptsize raw}}}
\put( 80,780){\makebox(0,0)[b]{\raisebox{0pt}[0pt][0pt]{\scriptsize fits}}}
\put(120,780){\makebox(0,0)[b]{\raisebox{0pt}[0pt][0pt]{\scriptsize header}}}
\put( 85,735){\makebox(0,0)[b]{\raisebox{0pt}[0pt][0pt]{\scriptsize b.p.m.}}}
\put(160,650){\makebox(0,0)[b]{\raisebox{0pt}[0pt][0pt]{\rm --BIAS}}}
\put(160,557){\makebox(0,0)[b]{\raisebox{0pt}[0pt][0pt]{\rm LINEAR}}}
\put( 40,655){\makebox(0,0)[b]{\raisebox{0pt}[0pt][0pt]{\rm RODIR}}}
\put( 40,560){\makebox(0,0)[b]{\raisebox{0pt}[0pt][0pt]{\rm COEFFS}}}
\put(280,655){\makebox(0,0)[b]{\raisebox{0pt}[0pt][0pt]{\rm INDEX}}}
\put( 85,665){\makebox(0,0)[b]{\raisebox{0pt}[0pt][0pt]{\scriptsize reduced}}}
\put( 80,645){\makebox(0,0)[b]{\raisebox{0pt}[0pt][0pt]{\scriptsize bias}}}
\put(225,660){\makebox(0,0)[b]{\raisebox{0pt}[0pt][0pt]{\scriptsize bias}}}
\put(225,645){\makebox(0,0)[b]{\raisebox{0pt}[0pt][0pt]{\scriptsize info}}}
\put( 90,570){\makebox(0,0)[b]{\raisebox{0pt}[0pt][0pt]{\scriptsize linear}}}
\put( 85,550){\makebox(0,0)[b]{\raisebox{0pt}[0pt][0pt]{\scriptsize coeffs}}}
\put(160,477){\makebox(0,0)[b]{\raisebox{0pt}[0pt][0pt]{\rm --DARK}}}
\put(160,397){\makebox(0,0)[b]{\raisebox{0pt}[0pt][0pt]{\rm NORM}}}
\put(160,340){\makebox(0,0)[b]{\raisebox{0pt}[0pt][0pt]{\rm RODIR}}}
\put( 40,480){\makebox(0,0)[b]{\raisebox{0pt}[0pt][0pt]{\rm RODIR}}}
\put(280,480){\makebox(0,0)[b]{\raisebox{0pt}[0pt][0pt]{\rm INDEX}}}
\put(280,400){\makebox(0,0)[b]{\raisebox{0pt}[0pt][0pt]{\rm INDEX}}}
\put( 40,400){\makebox(0,0)[b]{\raisebox{0pt}[0pt][0pt]{\rm NBS}}}
\put(240,780){\makebox(0,0)[b]{\raisebox{0pt}[0pt][0pt]{\scriptsize flat}}}
\put( 85,490){\makebox(0,0)[b]{\raisebox{0pt}[0pt][0pt]{\scriptsize reduced}}}
\put( 80,470){\makebox(0,0)[b]{\raisebox{0pt}[0pt][0pt]{\scriptsize dark}}}
\put(220,490){\makebox(0,0)[b]{\raisebox{0pt}[0pt][0pt]{\scriptsize dark}}}
\put(220,470){\makebox(0,0)[b]{\raisebox{0pt}[0pt][0pt]{\scriptsize info}}}
\put( 95,410){\makebox(0,0)[b]{\raisebox{0pt}[0pt][0pt]{\scriptsize normalisation}}}
\put( 90,385){\makebox(0,0)[b]{\raisebox{0pt}[0pt][0pt]{\scriptsize method}}}
\put(215,410){\makebox(0,0)[b]{\raisebox{0pt}[0pt][0pt]{\scriptsize flat}}}
\put(210,390){\makebox(0,0)[b]{\raisebox{0pt}[0pt][0pt]{\scriptsize info}}}
\put(140,365){\makebox(0,0)[b]{\raisebox{0pt}[0pt][0pt]{\scriptsize reduced}}}
\put(175,365){\makebox(0,0)[b]{\raisebox{0pt}[0pt][0pt]{\scriptsize flat}}}
\end{picture}
\end{center}
\caption{Reducing a {\sc flat}} \label{latfig3}
\end{figure}

\chapter{Reducing Spectroscopic Array Data}
\markboth{Spectroscopic DR Paradigms}{\stardocname}
\section{Reducing an Object, Sky or Arc}
To reduce an {\sc object}, {\sc sky} or {\sc arc}, Portable--CGS4DR reads
the raw data from store in {\sc \$idir/} and {\sc fits} header information
from store in {\sc \$odir/}, applies a bad pixel mask, optionally
subtracts a reduced {\sc bias} frame, linearises the data (if required),
optionally subtracts a reduced {\sc dark}, optionally divides by a flat
field and wavelength calibrates the data (if required), and writes the
reduced data parameters into the index file and stores the reduced {\sc
object}, {\sc sky} or {\sc arc} in {\sc \$rodir/} as shown in figure
\ref{latfig4}.

Note that if the bad pixel mask is changed between the calibration frame
and the astronomical data, the resultant quality array will contain the
logical {\sc and} of the two masks until the calibration data is
re-reduced with the new mask and the {\sc red4} task is re-initialised to
force it to memory map the new calibration frame. This has caused some
confusion in the past. 

%\begin{figure}[!htp]
\begin{figure}[htbp]
\begin{center}
\setlength{\unitlength}{0.012500in}%
\begin{picture}(285,480)(15,335)
\thicklines
\put(160,735){\circle{70}}
\put(160,560){\circle{70}}
\put(160,650){\circle{70}}
\put(160,480){\circle{70}}
\put(160,400){\circle{70}}
\put( 60,815){\line( 1, 0){ 40}}
\put( 60,795){\line( 1, 0){ 40}}
\put(220,815){\line( 1, 0){ 40}}
\put(220,795){\line( 1, 0){ 40}}
\put( 20,735){\line( 1, 0){ 40}}
\put( 20,715){\line( 1, 0){ 40}}
\put( 80,795){\vector( 3,-2){ 54.231}}
\put(240,795){\vector(-3,-2){ 54.231}}
\put( 65,725){\vector( 4, 1){ 58.824}}
\put( 20,670){\line( 1, 0){ 40}}
\put( 20,650){\line( 1, 0){ 40}}
\put(260,670){\line( 1, 0){ 40}}
\put(260,650){\line( 1, 0){ 40}}
\put( 15,575){\line( 1, 0){ 50}}
\put( 15,555){\line( 1, 0){ 50}}
\put(160,710){\vector( 0,-1){ 35}}
\put(160,620){\vector( 0,-1){ 35}}
\put( 65,660){\vector( 1, 0){ 55}}
\put(254,664){\vector(-4,-1){ 54.118}}
\put( 65,565){\vector( 1, 0){ 55}}
\put(140,355){\line( 1, 0){ 40}}
\put(140,335){\line( 1, 0){ 40}}
\put(160,530){\vector( 0,-1){ 25}}
\put(160,450){\vector( 0,-1){ 25}}
\put(160,375){\vector( 0,-1){ 20}}
\put( 20,495){\line( 1, 0){ 40}}
\put( 20,475){\line( 1, 0){ 40}}
\put( 20,415){\line( 1, 0){ 40}}
\put( 20,395){\line( 1, 0){ 40}}
\put(260,415){\line( 1, 0){ 40}}
\put(260,395){\line( 1, 0){ 40}}
\put(260,475){\line( 1, 0){ 40}}
\put(260,495){\line( 1, 0){ 40}}
\put(260,435){\line( 1, 0){ 40}}
\put(260,455){\line( 1, 0){ 40}}
\put( 65,485){\vector( 1, 0){ 65}}
\put(255,480){\vector(-1, 0){ 65}}
\put(255,440){\vector(-2,-1){ 60.0}}
\put(210,430){\makebox(0,0)[b]{\raisebox{0pt}[0pt][0pt]{\scriptsize flat info}}}
\put( 60,405){\vector( 1, 0){ 70}}
\put(190,400){\vector( 1, 0){ 65}}
\put( 80,800){\makebox(0,0)[b]{\raisebox{0pt}[0pt][0pt]{\rm ODIR}}}
\put(240,800){\makebox(0,0)[b]{\raisebox{0pt}[0pt][0pt]{\rm IDIR}}}
\put( 40,720){\makebox(0,0)[b]{\raisebox{0pt}[0pt][0pt]{\rm MASKS}}}
\put(160,740){\makebox(0,0)[b]{\raisebox{0pt}[0pt][0pt]{\rm APPLY}}}
\put(160,720){\makebox(0,0)[b]{\raisebox{0pt}[0pt][0pt]{\rm MASK}}}
\put(205,780){\makebox(0,0)[b]{\raisebox{0pt}[0pt][0pt]{\scriptsize raw}}}
\put( 80,780){\makebox(0,0)[b]{\raisebox{0pt}[0pt][0pt]{\scriptsize fits}}}
\put(120,780){\makebox(0,0)[b]{\raisebox{0pt}[0pt][0pt]{\scriptsize header}}}
\put( 85,735){\makebox(0,0)[b]{\raisebox{0pt}[0pt][0pt]{\scriptsize b.p.m.}}}
\put(160,650){\makebox(0,0)[b]{\raisebox{0pt}[0pt][0pt]{\rm --BIAS}}}
\put(160,557){\makebox(0,0)[b]{\raisebox{0pt}[0pt][0pt]{\rm LINEAR}}}
\put( 40,655){\makebox(0,0)[b]{\raisebox{0pt}[0pt][0pt]{\rm RODIR}}}
\put( 40,560){\makebox(0,0)[b]{\raisebox{0pt}[0pt][0pt]{\rm COEFFS}}}
\put(280,655){\makebox(0,0)[b]{\raisebox{0pt}[0pt][0pt]{\rm INDEX}}}
\put( 85,665){\makebox(0,0)[b]{\raisebox{0pt}[0pt][0pt]{\scriptsize reduced}}}
\put( 80,645){\makebox(0,0)[b]{\raisebox{0pt}[0pt][0pt]{\scriptsize bias}}}
\put(225,660){\makebox(0,0)[b]{\raisebox{0pt}[0pt][0pt]{\scriptsize bias}}}
\put(225,645){\makebox(0,0)[b]{\raisebox{0pt}[0pt][0pt]{\scriptsize info}}}
\put( 90,570){\makebox(0,0)[b]{\raisebox{0pt}[0pt][0pt]{\scriptsize linear}}}
\put( 85,550){\makebox(0,0)[b]{\raisebox{0pt}[0pt][0pt]{\scriptsize coeffs}}}
\put(160,477){\makebox(0,0)[b]{\raisebox{0pt}[0pt][0pt]{\rm --DARK}}}
\put(160,397){\makebox(0,0)[b]{\raisebox{0pt}[0pt][0pt]{\rm /FLAT}}}
\put(160,340){\makebox(0,0)[b]{\raisebox{0pt}[0pt][0pt]{\rm RODIR}}}
\put( 40,480){\makebox(0,0)[b]{\raisebox{0pt}[0pt][0pt]{\rm RODIR}}}
\put(280,480){\makebox(0,0)[b]{\raisebox{0pt}[0pt][0pt]{\rm INDEX}}}
\put(280,440){\makebox(0,0)[b]{\raisebox{0pt}[0pt][0pt]{\rm INDEX}}}
\put(280,400){\makebox(0,0)[b]{\raisebox{0pt}[0pt][0pt]{\rm INDEX}}}
\put( 40,400){\makebox(0,0)[b]{\raisebox{0pt}[0pt][0pt]{\rm RODIR}}}
\put(250,780){\makebox(0,0)[b]{\raisebox{0pt}[0pt][0pt]{\scriptsize obj/sky/arc}}}
\put( 85,490){\makebox(0,0)[b]{\raisebox{0pt}[0pt][0pt]{\scriptsize reduced}}}
\put( 80,470){\makebox(0,0)[b]{\raisebox{0pt}[0pt][0pt]{\scriptsize dark}}}
\put(220,490){\makebox(0,0)[b]{\raisebox{0pt}[0pt][0pt]{\scriptsize dark}}}
\put(220,470){\makebox(0,0)[b]{\raisebox{0pt}[0pt][0pt]{\scriptsize info}}}
\put( 95,410){\makebox(0,0)[b]{\raisebox{0pt}[0pt][0pt]{\scriptsize reduced}}}
\put( 90,385){\makebox(0,0)[b]{\raisebox{0pt}[0pt][0pt]{\scriptsize flat}}}
\put(230,405){\makebox(0,0)[b]{\raisebox{0pt}[0pt][0pt]{\scriptsize obj/sky/arc}}}
\put(210,390){\makebox(0,0)[b]{\raisebox{0pt}[0pt][0pt]{\scriptsize info}}}
\put(140,365){\makebox(0,0)[b]{\raisebox{0pt}[0pt][0pt]{\scriptsize reduced}}}
\put(185,365){\makebox(0,0)[b]{\raisebox{0pt}[0pt][0pt]{\scriptsize obj/sky/arc}}}
\end{picture}
\end{center}
\caption{Reducing an {\sc object}, {\sc sky} or {\sc arc}} \label{latfig4}
\end{figure}

\section{Wavelength Calibration}
Wavelength calibration can take two forms within Portable--CGS4DR {\em
i.e. viz.} estimated or calibrated. The estimated method generates a
rough scale from the grating wavelength and dispersion and is usually
accurate to better than 20\% of the resolution. The X axis label will be
set to `estimated wavelength' in microns. 

If the wavelength calibration option is set to calibrated,
Portable--CGS4DR searches the index file for a suitable {\sc calibr} frame
of the form cayymmdd\_oooo and having the same instrument configuration
and oversampling factor as the observation to be calibrated and uses that. 
Note that when such a {\sc calibr} observation is used, the X axis is
simply copied over and no attempt is made to re-bin the data. This means
that if anything other than a straight line fit is used in the {\sc arc}
function, the data will end up with an X axis whose values are not equally
spaced. 

Wavelength calibration by either method takes place when the reduced
observation file is created and before any integrations are reduced so if
you change your mind about wavelength calibration and the reduced
observation file already exists, you will have to delete it before new
instructions take effect. 

A {\sc calibr} frame is created by observing a wavelength calibration
source such as a lamp or the sky (OH-lines) and this is normally called an
{\sc arc} frame. This data is reduced in the usual way (and may be
wavelength calibrated!). CGS4 has argon, xenon and krypton lamps for this
purpose and the in-vacuum wavelengths for most infra-red emission lines
from these lamps are included with the software (as indeed are the OH
lines). They are recorded in \AA ~but may be converted to $\mu$m by the
calibrate procedure. 

Arc-lines are identified using the mouse and selected by depressing the
space bar. Note that if you know the exact value of a spectral line, you
can over-ride that chosen by the software by specifying the the `E' suffix
when inputting the wavelength value to the prompt. For example, the
wavelength 16952.3 \AA ~may be specified: 

\begin{quote}
? Enter wavelength (null to delete) - /''/ $>$ 16952.3 E
\end{quote}

After this procedure is complete, the frame is automaticall filed as a {\sc
calibr} frame. By invoking file\_calib manually, the x-axis label and units may be 
altered by the user but default to those contained in the original frame. 

Observations such as {\sc bias}, {\sc dark}, {\sc flat}, {\sc object} or 
{\sc sky} may be used for wavelength calibration but once filed as a
{\sc calibr} frame it cannot be used for any other purpose. 

\section{Sky Subtraction and the Propagation of Variance}
There are two strategies for combining observations into groups depending
upon the value of the logical flag add\_in\_pairs. If {\sc true} the
reduction routines combine reduced observations belonging to a particular
group into pairs of observations. Each pair must consist of one
observation of type {\sc object} and one of type {\sc sky}. Other
combinations are not allowed. 

If an observation is the first of the pair, it is simply remembered until
next time around. If it is the second observation, the {\sc sky} member of
the pair may be multiplied by a weighting factor indicated by the sky\_wt
parameter before being subtracted from the {\sc object} member, and the
result co-added to the group. Pairs of observations may also be weighted
according to their variance. 

The variance in the group may be propagated by combining together the
variances contained in the individual reduced observations {\em i.e.}
{\sc from\_int} according to the familiar variance rule (Bevington 1969): 

\begin{equation}
\sigma^{2}_{group} = \sum_{i=1}^{n} \sigma^{2}_{obs(i)} \label{equation_2}
\end{equation}

or, if these are poorly determined, by estimating the standard error, S,
in the variations between the pairs of reduced observations when they are
co-added into the group {\em i.e.} {\sc from\_obs} according to the
formula: 

\begin{equation}
S^2 = \frac{\sigma^2}{N} = \frac{\sum x^{2} - \frac{1}{N}(\sum x )^{2}}{N(N-1)} \label{equation_3}
\end{equation}

The latter will be necessary if the observations consist of only one
exposure per integration and one detector scan, and therefore contain no
error information themselves. 

It is important to grasp that variance weighting of errors can only be
used when the errors are already determined. It makes no sense, therefore,
to enable variance weighting and attempt to propagate errors using the
{\sc from\_obs} regime. 

Indeed, at the present time, although the data reduction provides both
methods for error propagation {\em i.e. viz.} {\sc from\_int} and {\sc
from\_obs}, the user is advised to default to {\sc from\_obs} as, aside from
the aforementioned caveat, a bug exists somewhere in the {\sc
from\_int} code that has never been traced. For completeness, however,
both are documented here. 

If add\_in\_pairs is {\sc false}, the routines apply a reduced observation
belonging to a particular group to a reduced group file.  Reduced
observations of type {\sc object} are added to the contents of the reduced
group file whilst reduced observations of type {\sc sky} are subtracted
from the contents of the reduced group file, after being optionally
multiplied by a weighting factor indicated by the sky\_wt parameter. 
Observations may also be weighted according to their variance if observing
conditions dictate. Observation types other than {\sc object} and {\sc
sky} are not allowed. 

You might well ask, then, what is the difference between add\_in\_pairs
set to {\sc false} as opposed to {\sc true} since both appear to produce
sky subtracted reduced groups? 

First, if an equal number of {\sc object} and {\sc sky} observations are
taken, the running mean should be calculated using the number of pairs of
observations, rather than the total number of observations, or the mean
signal calculated will be a factor of 2 smaller than it should be. The
signal will be calculated correctly for a series of {\sc object}
observations with no {\sc sky}s. 

Second, if there are unequal numbers of bad pixels at a given location
from the {\sc sky} and {\sc object} observations, that particular point
will not be sky-subtracted properly. 

Third, the variance weighting algorithm will work properly only for a
series of consecutive {\sc object} observations. If both {\sc object} and
{\sc sky} observations are reduced, different weights may be applied to
{\sc object} and {\sc sky}, and the data will not be properly
sky-subtracted. 

For these reasons, it is better to add such observations in pairs {\em
i.e.} with add\_in\_pairs set to {\sc true}. 

This is only part of the story, however, as CGS4 can take data in several
modes the two most common of which are {\sc chop} and {\sc stare}.  For
{\sc chop} mode sky subtraction, phase A -- B is calculated within the
acquisition electronics and groups are formed as a sum of objects. If
chopping and nodding are enabled then {\sc object---sky} pairs are formed
as for {\sc stare} mode (essentially already described above but, of
course, there should be an equal number of {\sc object} and {\sc sky}
frames taken during the acquisition phase). 

Note that when observations are combined into reduced groups, a running
average is maintained, with hooks built in for first and second
observations so that a sensible mean and variance are generated, broadly
according to the formula: 

\begin{equation}
D_{group} = \frac{n*D_{group} + (D_{object} - D_{sky})}{n+1} \label{equation_4}
\end{equation}

where $D_{item}$ is the data value for group, object and sky and $n$ is
the number of coadds. Some observers have noted that coadding many sky
subtracted frames into reduced groups ({\em i.e.} $ n > 50 $) slows the
data reduction down by a significant factor and have attributed this to
re-averaging the whole stack. That is incorrect. The probable cause of the
slow down is the many {\sc fits} items and Portable--CGS4DR specific
structures that must be included in the reduced group as part of the
reduction. Extending such structures is, unfortunately, a costly process. 

\section{Ratio-ing by a Standard Star}
Any spectra observed at the Earth's surface will differ from true due to
atmospheric effects, particularly wavelength dependent transmission,
noticeable in the infra-red due to water vapour and carbon dioxide.
The correction of these effects is to compare the source with another
having a well-behaved and understood spectral shape {\em i.e. viz.} a
standard source. 

In general, the flux density from an astronomical source,
$S_{*}(\lambda)$, and the flux density of a standard, $S_{st}(\lambda)$,
will both suffer attenuation and diminution by atmospheric transmission,
T($\lambda$), and the instrumental response, I($\lambda$). The measured
signals for each type of source will be proportional to the products of
the three terms (loosely, STI). Early type (A) stars are known to
approximate very well to black body sources\footnote[3]{Actually,
infra-red astronomers consider all standard stars as black bodies
irrespective of stellar type!} and so, for a star of known spectral type
and luminosity class, the spectrum can be re-created using: 

\begin{equation}
  Spectrum = \frac{S_{*}(\lambda) T(\lambda) I(\lambda)}{S_{st}(\lambda) T(\lambda) I(\lambda)} \times B_{T_{eff}}(\lambda)
\label{equation_5}
\end{equation}

where $B_{T_{eff}}(\lambda)$ is the black body density flux of the
standard at some effective temperature, $T_{eff}$. This term must be
proportional to the product of $S_{st}(\lambda)$ and some term which is
dependent only upon the arbitrary brightness, $k(\lambda)$, of the
`ratio-ing' source {\em i.e.} the spectral standard. Thus: 

\begin{equation}
  Spectrum \approx k(\lambda) S_{*}(\lambda) \label{equation_6}
\end{equation}

It can be seen that dividing by a standard star in this way removes
atmospheric absorption features from the spectrum. This depends upon the
path length through which both source and standard are measured. To remain
true both must be observed at, or close to, the same airmass. 

Equation \ref{equation_5} contains the key. Standard stars are observed in
exactly the same way as other astronomical objects. They are sky
subtracted into reduced group files, although a reduced observation can be
filed as a standard if desired.  Converting an observation to a standard
involves Portable--CGS4DR extracting a 1-D spectrum, generating a
normalised (model) black body, dividing the spectrum by this model black
body and growing the result along the slit. It is filed in the index file
as a frame beginning with `st' and stored in {\sc \$rgdir/}. 

When the time comes to ratio a source, Portable--CGS4DR will select a
standard based upon pre-determined criteria and divide the observation to
be calibrated by it and, hence, re-create the spectrum. The output file
has the label \_dbs appended to it and is simply the ratio of the DN/exp
in the source divided by the DN/exp in the standard {\em i.e.} no account
is taken of exposure times. 

\section{Flux Calibration}
The technique most frequently used to flux calibrate a spectrum is to
re-use the standard source which has a known brightness quoted relative to
Vega ($\alpha$ Lyrae is defined to have zero magnitude at every
wavelength). The flux density, $F_{\lambda}$, of a star at magnitude $m$
can be described thus (Puxley 1988, Mountain 1985): 

\begin{equation}
 F_{\lambda}(m) = F_{\lambda}(\alpha Lyrae) \times 10^{-0.4m} \label{equation_7}
\end{equation}

The software uses the values of flux densities for $\alpha$ Lyrae that are
listed in table \ref{tab2} and automatically accounts for
the exposure times of source and standard. 

When the flux standard is also the ratio-ing source, $k(\lambda)$ in
equation \ref{equation_6} is known and we immediately obtain a flux
calibrated spectrum.  If the flux and spectral standards are not the same
then one must determine the relative brightness of the two stars by flux
calibrating the spectral standard. 

\begin{table}
\begin{center}
\caption{\bf Flux Densities for $\alpha$ Lyrae} \label{tab2}
\vglue 0.6cm
\begin{tabular}{|c|c|c|c|c|c|}
\hline
\ \ & \ \ & \ \ & \ \ & \ \ & \ \  \\
Band & $\lambda$  & $f_{\lambda}^{*}$         & $f_{\lambda}$            & $f_{\lambda}$ & $f_{\lambda}$  \\ 
\ \  & $(\mu m)$  & $(W m^{-2} {\mu m}^{-1})$ & $(W m^{-2} {Hz}^{-1})$ & $(mJy)$         & $(ergs \ s^{-1} cm^{-2} {\mu m}^{-1})$  \\
\ \ & \ \ & \ \ & \ \ & \ \ & \ \  \\
\hline
\ \ & \ \ & \ \ & \ \ & \ \ & \ \  \\
%$V$  &  0.5556  &  3.44 $\times$ $10^{-8}$  & 3.54 $\times$ $10^{-23}$ & 3.54 $\times$ $10^{6}$ & 3.44 $\times$ $10^{-5}$ \\
$J$  &  1.2500  &  3.07 $\times$ $10^{-9}$  & 1.60 $\times$ $10^{-23}$ & 1.60 $\times$ $10^{6}$ & 3.07 $\times$ $10^{-6}$ \\
$H$  &  1.6500  &  1.12 $\times$ $10^{-9}$  & 1.02 $\times$ $10^{-23}$ & 1.02 $\times$ $10^{6}$ & 1.12 $\times$ $10^{-6}$ \\
$K$  &  2.2000  &  4.07 $\times$ $10^{-10}$ & 6.57 $\times$ $10^{-24}$ & 6.57 $\times$ $10^{5}$ & 4.07 $\times$ $10^{-7}$ \\
$L$  &  3.4500  &  7.30 $\times$ $10^{-11}$ & 2.90 $\times$ $10^{-24}$ & 2.90 $\times$ $10^{5}$ & 7.30 $\times$ $10^{-8}$ \\
$L'$ &  3.8000  &  5.24 $\times$ $10^{-11}$ & 2.52 $\times$ $10^{-24}$ & 2.52 $\times$ $10^{5}$ & 5.24 $\times$ $10^{-8}$ \\
$M$  &  4.8000  &  2.12 $\times$ $10^{-11}$ & 1.63 $\times$ $10^{-24}$ & 1.63 $\times$ $10^{5}$ & 2.12 $\times$ $10^{-8}$ \\
%\    &  7.8000  &  3.22 $\times$ $10^{-12}$ & 6.53 $\times$ $10^{-25}$ & 6.53 $\times$ $10^{4}$ & 3.22 $\times$ $10^{-9}$ \\
%\    &  8.7000  &  2.10 $\times$ $10^{-12}$ & 5.30 $\times$ $10^{-25}$ & 5.30 $\times$ $10^{4}$ & 2.10 $\times$ $10^{-9}$ \\
%\    &  9.8000  &  1.32 $\times$ $10^{-12}$ & 4.23 $\times$ $10^{-25}$ & 4.23 $\times$ $10^{4}$ & 1.32 $\times$ $10^{-9}$ \\
%$N$  &  10.100  &  1.17 $\times$ $10^{-12}$ & 3.98 $\times$ $10^{-25}$ & 3.98 $\times$ $10^{4}$ & 1.17 $\times$ $10^{-9}$ \\
%\    &  10.300  &  1.09 $\times$ $10^{-12}$ & 3.85 $\times$ $10^{-25}$ & 3.85 $\times$ $10^{4}$ & 1.09 $\times$ $10^{-9}$ \\
%\    &  11.600  &  6.81 $\times$ $10^{-13}$ & 3.05 $\times$ $10^{-25}$ & 3.05 $\times$ $10^{4}$ & 6.81 $\times$ $10^{-10}$ \\
%\    &  12.500  &  5.07 $\times$ $10^{-13}$ & 2.64 $\times$ $10^{-25}$ & 2.64 $\times$ $10^{4}$ & 5.07 $\times$ $10^{-10}$ \\
%$Q$  &  20.000  &  7.80 $\times$ $10^{-14}$ & 1.04 $\times$ $10^{-25}$ & 1.04 $\times$ $10^{4}$ & 7.80 $\times$ $10^{-11}$ \\
\ \ & \ \ & \ \ & \ \ & \ \ & \ \  \\
\hline
\end{tabular}
\vglue 0.3cm
{\em Conversion via: }
$\mid f_{\lambda}^{*} \mid$ \ \ ($W \ m^{-2} \ {\mu m}^{-1}$) $=$ 
$\frac{\mid f_{\lambda}^{*} \mid \lambda^{2}}{c}$ \ \ ($W \ m^{-2} \ {Hz}^{-1}$) $=$

$\frac{\mid f_{\lambda}^{*} \mid \lambda^{2}}{c}$ $\times$ $10^{29}$  \ \ ($mJy$) $=$
$\mid f_{\lambda}^{*} \mid$ $\times$ $10^{3}$ \ \ ($ergs \ s^{-1} \ cm^{-2} \ {\mu m}^{-1}$)

\vglue 0.1cm

where c = speed of light in microns / second.

\vglue 0.3cm
%Note: Only bands $J$, $H$, $K$, $L$, $L'$ and $M$ apply to CGS4. 
\end{center}
\end{table}

\section{Extraction of Nodded Spectra}
Portable--CGS4DR now has the ability to extract up to 3 spectra nodded
along the slit in an automated way. For one spectrum on the array, the
result is analogous to the simple Figaro {\sc extract} command. For two
spectra, the upper row is assumed to be negative and the lower positive.
For three spectra, the middle beam is assumed to be positive and is
flanked by two negative beams. Note that in the two and three beam cases,
a switch to invert the spectrum is included in case the beams are not in
the sense expected.  Note that the number of rows over which the extraction
occurs is {\em not} taken into account when combining multiple beams: just
the total extracted flux is used.

Further, for three beam extraction, two algorithms are provided {\em i.e.
viz.} for bright and faint sources. The fundamental difference is the
weighting given to the outer beams. Let the signal from the star in one
chop beam be $S$ and the error $\delta S$. Assuming that there is no beam
imbalance, the final reduced group will contain three spectra in which the
middle is the positive beam (P) and has twice the signal of the outer
negative beams (N1 and N2). 

This reduced group is formed from the difference of two observations,
namely an {\sc object} and a {\sc sky}. These individual reduced
observations will each have two spectra offset by the nodding distance.
For the {\sc object} observation, the upper beam with be negative and the
lower beam positive. For the {\sc sky} frame the situation is reversed but
the position on the array of the upper beam in the {\sc sky} matches the
lower beam in the {\sc object}. 

For the bright source algorithm, the final signal, $F_{b}$ is given by:

\begin{equation}
  F_{b} = \frac{P-(N1+N2)}{4} = \frac{2S-((-S)+(-S))}{4} = S
\label{equation_8}
\end{equation}

For the faint source algorithm, the final signal, $F_{f}$ is given by:

\begin{equation}
  F_{f} = \frac{ P-\frac{ N1+N2 }{ 2 } }{ 3 } =
   \frac{2S-\frac{ (-S)+(-S) }{ 2 }}{ 3 } = S
\label{equation_9}
\end{equation}

If $\delta F_{b}$ and $\delta F_{f}$ are the errors in the bright and faint
source algorithm, then:

\begin{equation}
  \frac{F_{b} / \delta F_{b}}{F_{f} / \delta F_{f}} = \frac{\sqrt[2]{10}}{3} 
\label{equation_10}
\end{equation}

Note that the faint source algorithm will give better cancellation of any
residual atmospheric lines on the array and a more representative error
for the case where the dominant source of noise is the infra-red
background. 

%------------------------------------------------------------------------------
%  Part II
\part{THE COMMAND LINE INTERFACE}
\pagestyle{myheadings}
\markboth{ICL Interface}{\stardocname}

\chapter{The Command Line Interface}
\section{Introduction}
This section describes the ICL command line interface to
Portable--CGS4DR now available under all Unix platforms supported by
{\sc starlink}. Significant modifications to the task structures were made
during the porting process (Daly 1995) but every attempt has been made to achieve the
same `look and feel' as the older software. 

The command line interface offers a limited subset of the previous 
functionality for two reasons:

\begin{itemize}
\item CGS4 data reduction is not well suited to a command line interface
      and some users may find the software rather `kludgey'. 
      Writing the scripts to give the full functionality required is a 
      tiresome process and little effort is available to do it.
\item Limited time was available for testing the software so only the key 
      elements of the data reduction were tested. That does not mean that the
      other aspects of CGS4 data reduction do not work merely that they have
      not been tested. If you do try the more advanced features of Portable--CGS4DR,
      please let me know how you get on.
\end{itemize}

\section{Starting the Software}
Portable--CGS4DR can be started with three optional command line parameters:

\begin{minipage}{120mm}
\begin{quote}
  \$1 is a data directory \hfill /\$\{home\}/ \\
  \$2 is a UT date \hfill /current UT-date/ \\
  \$3 is the file type {\sc dst} or {\sc ndf} \hfill /ndf/
\end{quote}
\end{minipage}

To start the software, source the {\sc starlink} login and cshrc files
and use your local data directory. {\em E.g.:}

\begin{minipage}{120mm}
\begin{quote}
  \%  source \ \ /star/etc/login \\
  \%  source \ \ /star/etc/cshrc \\
  \%  cgs4dr \ \ /scratch/pnd/19940811 \ \ 940811 \ \ ndf \\[4ex]
      {\tt Welcome to Portable-CGS4DR V1.3-0} \\[2ex]
  ICL$>$
\end{quote}
\end{minipage}

There is also a `setup' facility for defining evironmental variables without
invoking the software:

\begin{minipage}{120mm}
\begin{quote}
  \%  source \ \ /star/etc/login \\
  \%  source \ \ /star/etc/cshrc \\
  \%  cgs4dr\_setup \ \ /scratch/pnd/19940811 \ \ 940811 \ \ ndf \\[4ex]
      {\tt Setup Portable-CGS4DR V1.3-0 for /scratch/pnd/19940811} \\[2ex]
  \%
\end{quote}
\end{minipage}

\section{Changing System Defaults}
On some systems, certain defaults may be changed {\em prior} to
starting the software. Some common examples are\footnote[2]{Note that I 
use the `;:' construct for Unix in-line comments.}:

\begin{quote}
 \% setenv \ \ {\sc shell} \ \ /bin/csh \hfill ;: Invokes csh via ICL `sh' command \\
 \% setenv \ \ {\sc term} \ \ vt100     \hfill ;: Required by DEC-Alphas? \\
 \% setenv \ \ {\sc lpdest} \ \ hp3d    \hfill ;: Where `hp3d' is a PostScript printer \\
 \% setenv \ \ {\sc cgs4\_masks} \ \ /scratch/pnd/19940811 \hfill ;: User's bad pixel masks
\end{quote}

The last option allows user defined bad pixel masks {\em e.g.} 
when reducing echelle data. If the user requires system defined bad pixel 
masks as well these must be manually copied to the new directory before the 
software is invoked. For example, to use fpa46\_short and one's own masks 
enter:

\begin{quote}
 \% setenv \ \ {\sc cgs4\_masks} \ \ /scratch/pnd/19940811/masks \\
 \% cp -p \ \ {\sc \${cgs4dr\_root}}/ndf/fpa46\_short.sdf \ \ {\sc \${cgs4\_masks}}/fpa46\_short.sdf \\
 \% cgs4dr \ \ /scratch/pnd/19940811 \ \ 940811 \ \ ndf 
\end{quote}

\section{Running the Demo}
The Portable--CGS4DR demo has two parts. The first reduces several groups
and takes around 14 minutes on a SPARCstation LX. The second, which
demonstrates filing a source as a standard and dividing a reduced group by
it, takes another 2 minutes on the same machine. To load the demo files,
use the command: 

\begin{quote}
  \% cgs4dr\_demo
\end{quote}

Once the software has been loaded the demos can be run with the following
commands (and in the given order): 

\begin{quote}
  ICL$>$ cgs4dr\_demo \\
  ICL$>$ cgs4dr\_demo2
\end{quote}

\section{Set Nocheckpars and Other Defaults}
All procedures have `set nocheckpars' enabled so that they can be invoked
with or without parameters. If the required input parameters are not given
on the command line, they are either prompted for or assume sensible
defaults (as indicated in this document by /{\em default\_value}/). 

Note that, where required, full responses must be given e.g. YES not Y or y.

\section{A Typical User Session}
The procedures available in Portable--CGS4DR are described in the
following chapters but a typical user session appears below (where lines
beginning `\{' are comment lines that need not be typed in by the user). 

\begin{verbatim}
   $ cgs4dr  /scratch/pnd/19940811  940811  ndf
      .
      .
      .
   ICL> { Check the current configurations
   ICL> get_reduction_sequence
   ICL> get_file_selection
   ICL> get_ff_normalisation
   ICL> get_wavelength_calib
   ICL> get_bad_pixel_mask
   ICL> get_extract_spectrum
   ICL> get_integration_display
   ICL> get_observation_display
   ICL> get_group_display
   ICL> get_spectrum_display
      .
      .
      .
   ICL> { If necessary, set the reduction sequence etc
   ICL> set_reduction_sequence
      .
      .
      .
   ICL> set_ff_normalisation
      .
      .
      .
   ICL> { Enter observation numbers 2 through 5 to the tail of the queue
   ICL> enter_obs_range 2 5
   ICL> { Enter observation number 1 to the head of the queue
   ICL> enter_obs_range 1 1 newest
   ICL> { Check the queue
   ICL> list_queue
      .
      .
      .
   ICL> { Start the automatic reduction
   ICL> start_autoreduce
      .
      .
      .
   ICL> { Save the current config
   ICL> save_config myconfig
   ICL> { List the index file
   ICL> list_index
   ICL> { Stop automatic reduction and exit
   ICL> stop_autoreduce
   ICL> exit
\end{verbatim}

\chapter{Control \& Setup Procedures}
\markboth{Control \& Setup Procedures}{\stardocname}
\section{start\_autoreduce}
\begin{quote}
ICL$>$ start\_autoreduce \\
Use: Starts the automatic reduction of data
\end{quote}
\section{stop\_autoreduce}
\begin{quote}
ICL$>$ stop\_autoreduce \\
Use: Stops the automatic reduction of data
\end{quote}
\section{drpause}
\begin{quote}
ICL$>$ drpause \\
Use: Pauses the automatic reduction of data
\end{quote}
\section{drcontinue}
\begin{quote}
ICL$>$ drcontinue \\
Use: Resumes the automatic reduction of data
\end{quote}
\section{set\_reduction\_sequence}
\begin{quote}
ICL$>$ set\_reduction\_sequence p1 p2 p3 p4 p5 p6 p7 p8 p9 p10 p11 \\
Use: Sets the data reduction sequence \\
p1 = Subtract bias \\
p2 = Subtract dark \\
p3 = Coadd integrations \\
p4 = Archive observations \\
p5 = File observations \\
p6 = Normalise flat field \\
p7 = divide by flat field \\
p8 = Wavelength calibrate \\
p9 = Add observations into groups \\
p10 = Divide by standard \\
p11 = Extract spectrum \\
{\em E.g:} set\_reduction\_sequence YES NO YES NO YES YES YES YES YES NO YES
\end{quote}

\section{get\_reduction\_sequence}
\begin{quote}
ICL$>$ get\_reduction\_sequence \\
Use: Returns the current data reduction sequence
\end{quote}

\section{set\_file\_selection}
\begin{quote}
ICL$>$ set\_file\_selection p1 p2 p3 p4 p5 p6 p7 p8 p9 p10 \\
Use: Sets the calibration file selection parameters \\
p1 = Direction to search for BIAS frames \\ 
p2 = Direction to search for DARK frames \\ 
p3 = Direction to search for FLAT frames \\ 
p4 = Direction to search for CALIB frames \\ 
p5 = Direction to search for STANDARD frames \\ 
p6 = Name of specific BIAS observation \\
p7 = Name of specific DARK observation \\
p8 = Name of specific FLAT observation \\
p9 = Name of specific CALIB observation \\
p10 = Name of specific STANDARD observation \\
{\em E.g:} set\_file\_selection BOTH BOTH BOTH SPECIFIED BOTH '' '' '' ca950516\_5 ''
\end{quote}

\section{get\_file\_selection}
\begin{quote}
ICL$>$ get\_file\_selection \\
Use: Returns calibration frame file selection parameters
\end{quote}

\section{set\_ff\_normalisation}
\begin{quote}
ICL$>$ set\_ff\_normalisation p1 p2 p3 \\
Use: Sets the flat field normalisation parameters \\
p1 = Flat fielding method \\
p2 = Order of polynomial \\
p3 = Smoothing box \\
{\em E.g:} set\_ff\_normalisation POLYFIT 3 5
\end{quote}

\section{get\_ff\_normalisation}
\begin{quote}
ICL$>$ get\_ff\_normalisation \\
Use: Returns the current flat field normalisation parameters
\end{quote}
\section{set\_wavelength\_calib}
\begin{quote}
ICL$>$ set\_wavelength\_calib \\
Use: Sets the current wavelength calibration method \\
p1 = Wavelength calibration method \\
{\em E.g:} set\_wavelength\_calib ESTIMATED
\end{quote}
\section{get\_wavelength\_calib}
\begin{quote}
ICL$>$ get\_wavelength\_calib \\
Use: Returns the current wavelength calibration method
\end{quote}
\section{set\_bad\_pixel\_mask}
\begin{quote}
ICL$>$ set\_bad\_pixel\_mask \\
Use: Sets the bad pixel mask \\
p1 = Name of mask only \\
{\em E.g:} set\_bad\_pixel\_mask fpa61\_75
\end{quote}
\section{get\_bad\_pixel\_mask}
\begin{quote}
ICL$>$ get\_bad\_pixel\_mask \\
Use: Returns the current bad pixel mask
\end{quote}
\section{set\_polyfit}
\begin{quote}
ICL$>$ set\_polyfit p1 p2 p3 p4 p5 p6 p7 p8 p9 p10 p11 p12 \\
Use: Sets the method and rows to extract for enhanced sky subtraction \\
p1 = Type of polyfit (NONE, REDUCED\_GRP, OBJ--SKY, OBJECT) \\
p2 = Degree of polynomial to fit \\
p3 = Number of points to reject from fit \\
p4 = Use weights for fit? \\ 
p5 = Start of first sky area (-1 = ignore) \\
p6 = End of first sky area (-1 = ignore) \\
p7 = Start of second sky area (-1 = ignore) \\
p8 = End of second sky area (-1 = ignore) \\
p9 = Start of third sky area (-1 = ignore) \\
p10 = End of third sky area (-1 = ignore) \\
p11 = Start of fourth sky area (-1 = ignore) \\
p12 = End of fourth sky area (-1 = ignore) \\
{\em E.g:} set\_polyfit REDUCED\_GRP 3 0 {\sc true} 10 15 35 40 -1 -1 -1 -1
\end{quote}
\section{get\_polyfit}
\begin{quote}
ICL$>$ get\_polyfit \\
Use: Returns enhanced sky subtraction setup
\end{quote}

\section{set\_extract\_spectrum}
\begin{quote}
ICL$>$ set\_extract\_spectrum p1 p2 p3 p4 p5 p6 p7 p8 \\
Use: Sets the method and rows to extract into a 1-D spectrum and 2-D spectral image \\
p1 = Start of first row to be extracted (-1 = ignore) \\
p2 = End of first row to be extracted (-1 = ignore) \\
p3 = Start of second row to be extracted (-1 = ignore) \\
p4 = End of second row to be extracted (-1 = ignore) \\
p5 = Start of third row to be extracted (-1 = ignore) \\
p6 = End of third row to be extracted (-1 = ignore) \\
p7 = Invert spectrum upon output \\
p8 = Extraction algorithm \\
{\em E.g:} set\_extract\_spectrum 29 31 -1 -1 -1 -1 {\sc false} BRIGHT
\end{quote}
\section{get\_extract\_spectrum}
\begin{quote}
ICL$>$ get\_extract\_spectrum \\
Use: Returns automated extraction of spectra setup
\end{quote}
\section{save\_config}
\begin{quote}
ICL$>$ save\_config p1 \\
Use: Saves the cred4 and p4 configurations \\
p1 = Name of config file only \\
{\em E.g:} save\_config myconfig
\end{quote}
\section{restore\_config}
\begin{quote}
ICL$>$ restore\_config p1 \\
Use: Restores the cred4 and p4 configurations \\
p1 = Name of config file only \\
{\em E.g:} restore\_config myconfig
\end{quote}
\section{set\_integration\_display}
\label{set_int_display}
\begin{quote}
ICL$>$ set\_integration\_display p0 p1 p2 p3 p4 p5 p6 p7 p8 \\
Use: Sets the display of reduced integrations \\
p0 = Display in port 0 \\
p1 = Display in port 1 \\
p2 = Display in port 2 \\
p3 = Display in port 3 \\
p4 = Display in port 4 \\
p5 = Display in port 5 \\
p6 = Display in port 6 \\
p7 = Display in port 7 \\
p8 = Display in port 8 \\
{\em E.g:} set\_integration\_display NO NO NO NO NO NO NO NO NO
\end{quote}
\section{get\_integration\_display}
\begin{quote}
ICL$>$ get\_integration\_display \\
Use: Gets the current display mode for reduced integrations
\end{quote}
\section{set\_observation\_display}
\label{set_obs_display}
\begin{quote}
ICL$>$ set\_observation\_display p0 p1 p2 p3 p4 p5 p6 p7 p8 \\
Use: Sets the display of reduced observations \\
p0 = Display in port 0 \\
p1 = Display in port 1 \\
p2 = Display in port 2 \\
p3 = Display in port 3 \\
p4 = Display in port 4 \\
p5 = Display in port 5 \\
p6 = Display in port 6 \\
p7 = Display in port 7 \\
p8 = Display in port 8 \\
{\em E.g:} set\_observation\_display NO YES YES YES YES NO NO NO NO
\end{quote}
\section{get\_observation\_display}
\begin{quote}
ICL$>$ get\_observation\_display \\
Use: Gets the current display mode for reduced observations
\end{quote}
\section{set\_group\_display}
\label{set_grp_display}
\begin{quote}
ICL$>$ set\_group\_display p0 p1 p2 p3 p4 p5 p6 p7 p8 \\
Use: Sets the display of reduced groups \\
p0 = Display in port 0 \\
p1 = Display in port 1 \\
p2 = Display in port 2 \\
p3 = Display in port 3 \\
p4 = Display in port 4 \\
p5 = Display in port 5 \\
p6 = Display in port 6 \\
p7 = Display in port 7 \\
p8 = Display in port 8 \\
{\em E.g:} set\_group\_display NO NO NO NO NO YES YES NO NO
\end{quote}
\section{get\_group\_display}
\begin{quote}
ICL$>$ get\_group\_display \\
Use: Gets the current display mode for reduced groups
\end{quote}
\section{set\_spectrum\_display}
\label{set_spc_display}
\begin{quote}
ICL$>$ set\_spectrum\_display p0 p1 p2 p3 p4 p5 p6 p7 p8 \\
Use: Sets the display of reduced spectra \\
p0 = Display in port 0 \\
p1 = Display in port 1 \\
p2 = Display in port 2 \\
p3 = Display in port 3 \\
p4 = Display in port 4 \\
p5 = Display in port 5 \\
p6 = Display in port 6 \\
p7 = Display in port 7 \\
p8 = Display in port 8 \\
{\em E.g:} set\_spectrum\_display NO NO NO NO NO NO NO YES YES
\end{quote}
\section{get\_spectrum\_display}
\begin{quote}
ICL$>$ get\_spectrum\_display \\
Use: Gets the current display mode for reduced spectra 
\end{quote}

\chapter{Reduction Procedures}
\markboth{Reduction Procedures}{\stardocname}
\section{file\_standard}
\begin{quote}
ICL$>$ file\_standard p1 p2 p3 p4 p5 \\
Use: Files a group as a standard \\
p1 = Name of reduced group \\
p2 = Effective temperature \\
p3 = Reference wavelength \\
p4 = Start row for extraction \\
p5 = End row for extraction \\
{\em E.g:} file\_standard rg940311\_21 5000 2.2 20 40
\end{quote}
\section{file\_obs}
\begin{quote}
ICL$>$ file\_obs p1 \\
Use: Files an observation as whatever it is \\
p1 = Name of observation file \\
{\em E.g:} file\_obs o940311\_17  
\end{quote}
\section{file\_calib}
\begin{quote}
ICL$>$ file\_calib p1 \\
Use: Files an observation as a calibration \\ 
p1 = Name of calibration file \\
{\em E.g:} file\_calib o940311\_9
\end{quote}
\section{bad\_observation}
\begin{quote}
ICL$>$ bad\_observation \\
Use: Files an observation as bad \\
p1 = Name of bad observation \\
{\em E.g:} bad\_observation o940311\_5
\end{quote}
\section{list\_index}
\begin{quote}
ICL$>$ list\_index \\
Use: Lists the observation index 
\end{quote}
\section{list\_index2}
\begin{quote}
ICL$>$ list\_index2 \\
Use: Lists the observation index in an alternative format
\end{quote}
\section{calibrate}
\begin{quote}
ICL$>$ calibrate p1 p2 p3 p4 p5 p6 \\
Use: Calibrates an {\sc arc} lamp \\
p1 = Name of reduced observation (must be specified in full) \\
p2 = Start row for extraction \\
p3 = End row for extraction \\
p4 = Type of {\sc arc} lamp used \\
p5 = Order of polynomial to fit \\
p6 = Arcline half-width in pixels \\
{\em E.g:} calibrate {\sc \$rodir/}ro940311\_17 20 40 argon 2.0 1.0
\end{quote}
\section{cgs4list}
\begin{quote}
ICL$>$ cgs4list \\
Use: Lists the observation headers
\end{quote}
\chapter{Queue Procedures}
\markboth{Queue Procedures}{\stardocname}
\section{enter\_string}
\begin{quote}
ICL$>$ enter\_string p1 p2 \\
Use: Writes any string to database \\
p1 = string to write  \\
p2 = queue position /`oldest'/ \\
{\em E.g:} enter\_string 'This is a junk string' oldest
\end{quote}
\section{enter\_mask}
\begin{quote}
ICL$>$ enter\_mask p1 p2 \\
Use: Writes ``DRMASK p1'' to database \\
p1 = Name of mask only  \\
p2 = Queue position /`oldest'/ \\
{\em E.g:} enter\_mask fpa61\_75 newest
\end{quote}
\section{cancel\_mask}
\begin{quote}
ICL$>$ cancel\_mask p1 p2 \\
Use: Cancels ``DRMASK p1'' from database \\
p1 = Name of mask only  \\
p2 = Queue position /`oldest'/ \\
{\em E.g:} cancel\_mask fpa61\_75
\end{quote}
\section{enter\_config}
\begin{quote}
ICL$>$ enter\_config p1 p2 \\
Use: Writes ``DRCONFIG p1'' to database \\
p1 = Name of configuration file only \\
p2 = Queue position /`oldest'/ \\
{\em E.g:} enter\_config bias\_tests newest
\end{quote}
\section{cancel\_config}
\begin{quote}
ICL$>$ cancel\_config p1 p2 \\
Use: Cancels ``DRCONFIG p1'' from database \\
p1 = Name of configuration file only \\
p2 = Queue position /`oldest'/ \\
{\em E.g:} cancel\_config bias\_tests
\end{quote}
\section{enter\_skywt}
\begin{quote}
ICL$>$ enter\_skywt p1 p2 \\
Use: Writes ``DRSKYWT p1'' to database \\
p1 = Value of SKYWT /1.0/ \\
p2 = Queue position /`oldest'/ \\
{\em E.g:} enter\_skywt 1.5
\end{quote}
\section{cancel\_skywt}
\begin{quote}
ICL$>$ cancel\_skywt p1 p2 \\
Use: Cancels ``DRSKYWT p1'' from database \\
p1 = Value of SKYWT /1.0/ \\
p2 = Queue position /`oldest'/ \\
{\em E.g:} cancel\_skywt 1.5
\end{quote}
\section{enter\_varwt}
\begin{quote}
ICL$>$ enter\_varwt p1 p2 \\
Use: Writes ``DRVARWT p1'' to database \\
p1 = Value of VARWT /false/ \\
p2 = Queue position /`oldest'/ \\
{\em E.g:} enter\_varwt true
\end{quote}
\section{cancel\_varwt}
\begin{quote}
ICL$>$ cancel\_varwt p1 p2 \\
Use: Cancels ``DRVARWT p1'' from database \\
p1 = Value of VARWT /false/ \\
p2 = Queue position /`oldest'/ \\
{\em E.g:} cancel\_varwt true
\end{quote}
\section{enter\_obs\_range}
\begin{quote}
ICL$>$ enter\_obs\_range p1 p2 p3 \\
Use: Recursively writes ``REDUCE oyymmdd\_n'' and
``END oyymmdd\_n'' pairs 
(for n = p1 to p2) to database for pre-defined UT-date \\
p1 = Start observation number /1/ \\
p2 = End observation number /p1/ \\
p3 = Queue position /`oldest'/ \\
{\em E.g:} enter\_obs\_range 1 5 oldest 
\end{quote}
\section{cancel\_obs\_range}
\begin{quote}
ICL$>$ cancel\_obs\_range p1 p2 p3 \\
Use: Recursively cancels ``REDUCE oyymmdd\_n'' and
``END oyymmdd\_n'' pairs 
(for n = p1 to p2) from database for pre-defined UT-date \\
p1 = Start observation number /1/ \\
p2 = End observation number /p1/ \\
p3 = Queue position /`oldest'/ \\
{\em E.g:} cancel\_obs\_range 1 5
\end{quote}
\section{enter\_int\_range}
\begin{quote}
ICL$>$ enter\_int\_range p1 p2 p3 p4 \\
Use: Recursively writes ``REDUCE iyymmdd\_p1\_n'' 
(for n = p2 to p3) to database for pre-defined UT-date \\
p1 = Observation number /1/ \\
p2 = Start integration number /1/ \\
p3 = End integration number /p2/ \\
p4 = Queue position /`oldest'/ \\
{\em E.g:} enter\_int\_range 1 1 5 newest
\end{quote}
\section{cancel\_int\_range}
\begin{quote}
ICL$>$ cancel\_int\_range p1 p2 p3 p4 \\
Use: Recursively cancels ``REDUCE iyymmdd\_p1\_n'' 
(for n = p2 to p3) from database for pre-defined UT-date \\
p1 = Observation number /1/ \\
p2 = Start integration number /1/ \\
p3 = End integration number /p2/ \\
p4 = Queue position /`oldest'/ \\
{\em E.g:} cancel\_int\_range 1 1 5 newest
\end{quote}
\section{enter\_end\_range}
\begin{quote}
ICL$>$ enter\_end\_range p1 p2 p3 \\
Use: Recursively writes ``END oyymmdd\_n'' (for n = p1 to p2) 
 from database for pre-defined UT-date \\
p1 = Start observation number /1/ \\
p2 = End observation number /p1/ \\
p3 = Queue position /`oldest'/ \\
{\em E.g:} enter\_end\_range 1 3 
\end{quote}
\section{cancel\_end\_range}
\begin{quote}
ICL$>$ cancel\_end\_range p1 p2 p3 \\
Use: Recursively cancels ``END oyymmdd\_n'' (for n = p1 to p2) 
 from database for pre-defined UT-date \\
p1 = Start observation number /1/ \\
p2 = End observation number /p1/ \\
p3 = Queue position /`oldest'/ \\
{\em E.g:} cancel\_end\_range 1 3
\end{quote}
\section{enter\_endgroup}
\begin{quote}
ICL$>$ enter\_endgroup p1 p2 \\
Use: Writes ``ENDGROUP rgyymmdd\_p1'' to database for
 pre-defined UT-date \\
p1 = Group number /1/ \\
p2 = Queue position /`oldest'/ \\
{\em E.g:} enter\_endgroup 12
\end{quote}
\section{cancel\_endgroup}
\begin{quote}
ICL$>$ cancel\_endgroup p1 p2 \\
Use: Cancels ``ENDGROUP rgyymmdd\_p1'' from database for
 pre-defined UT-date \\
p1 = Group number /1/ \\
p2 = Queue position /`oldest'/ \\
{\em E.g:} cancel\_endgroup 12
\end{quote}
\section{cancel\_all\_qentries}
\begin{quote}
ICL$>$ cancel\_all\_qentries \\
Use: Deletes the whole database
\end{quote}
\section{list\_queue}
\begin{quote}
ICL$>$ list\_queue \\
Use: Lists the database in the order in which
Portable--CGS4DR will execute them
\end{quote}

\chapter{Graphics Procedures}
\markboth{Graphics Procedures}{\stardocname}
\section{lut}
\begin{quote}
ICL$>$ lut p1 \\
Use: Load a colour (look-up) table \\
p1 = Name of colour table \\
{\em E.g:} lut col19
\end{quote}
\section{plot\_image}
\begin{quote}
ICL$>$ plot\_image p1 p2 \\
Use: Plot the dataset as an image \\
p1 = Name of dataset to be displayed \\
p2 = Port number /0/ \\
{\em E.g:} plot\_image \$rodir/ro940311\_17 0
\end{quote}
\section{plot\_graph}
\begin{quote}
ICL$>$ plot\_graph p1 p2 \\
Use: Plot the dataset as a graph \\
p1 = Name of dataset to be displayed \\
p2 = Port number /0/ \\
{\em E.g:} plot\_graph \$rodir/ro940311\_17 0
\end{quote}
\section{plot\_overgraph}
\begin{quote}
ICL$>$ plot\_overgraph p1 p2 \\
Use: Plot the dataset as an image \\
p1 = Name of dataset to be displayed \\
p2 = Port number /0/ \\
{\em E.g:} plot\_overgraph \$rodir/ro940311\_17 0
\end{quote}
\section{plot\_histogram}
\begin{quote}
ICL$>$ plot\_histogram p1 p2 \\
Use: Plot the dataset as an histogram \\
p1 = Name of dataset to be displayed \\
p2 = Port number /0/ \\
{\em E.g:} plot\_histogram \$rodir/ro940311\_17 0
\end{quote}
\section{plot\_surface}
\begin{quote}
ICL$>$ plot\_surface p1 p2 \\
Use: Plot the dataset as a surface \\
p1 = Name of dataset to be displayed \\
p2 = Port number /0/ \\
{\em E.g:} plot\_surface \$rodir/ro940311\_17 0
\end{quote}
\section{plot\_contour}
\begin{quote}
ICL$>$ plot\_contour p1 p2 \\
Use: Plot the dataset as a contour map \\
p1 = Name of dataset to be displayed \\
p2 = Port number /0/ \\
{\em E.g:} plot\_contour \$rodir/ro940311\_17 0
\end{quote}
\section{cursor}
\begin{quote}
ICL$>$ cursor p1 p2 \\
Use: Enquire dataset X,Y position and data,variance,quality of point using cursor \\
p1 = Task alias \\
p2 = Port number /0/ \\
{\em E.g:} cursor (p4\_alias) 0 
\end{quote}
\section{print\_ps}
\begin{quote}
ICL$>$ print\_ps p1 p2 \\
Use: Print a {\em PostScript} file \\
p1 = Name of dataset to be displayed \\
p2 = Port number \\
{\em E.g:} set\_device0 ps\_l \\
print\_ps \$rodir/ro940311\_17 0
\end{quote}

\chapter{More on Graphics} \label{p4_chap}
\markboth{Graphics Options}{\stardocname}
\section{The Plotting Task}
P4 is the generic plotting package. The lower
layers of the software call {\sc pgplot} routines and access to the data is via
Figaro's {\sc dsa} library. Thus P4 can handle either {\sc ndf} or {\sc dst} files
and plots to any {\sc starlink} supported graphics device for which a {\sc pgplot} driver
is available.

A noticeboard is maintained in memory describing the attributes of the available 
display surfaces, plotting devices open on each surface and some aspects of the data.
This noticeboard can be manipulated directly by the user giving a wide-variety
of options to tailor the display. Indeed, some users may find the range of 
options too broad and confusing.

Although the package is quite generic and may be invoked separately from other
items of software, it still retains some intelligence regarding CGS4 data and directory structures.

P4 can plot data as an image, a graph, an overgraph (over-laid graph upon
a previous graph), a surface, a histogram or a contour map (with several
algorithms to determine contour levels). Nearly seventy colour (look-up) tables are
shipped with the software that should cater to most tastes.
P4 also generates labels and titles from the contents of the data-file or
as specified by the user.

Any number of users may run P4 concurrently. Each viewport (0 -- 8)
may have a separate plot device defined but a {\sc pgplot} restriction is that only one may be
open at any one time. Thus using multiple devices on different viewports may
slow the software down.

\section{Types of Plot and Units}
P4 has the following types of plot available:
\begin{description}
\item[{\sc image}]--- Plots an image in colour or greyscale using the full dynamic range of the data and the display device.
\item[{\sc graph}]--- Plots 1-D lines graphics as summed through several rows (or columns) of the array.
\item[{\sc overgraph}]--- A 1-D graph on the same scale as a previous graph with the overlay in a different colour.
\item[{\sc histogram}]--- Standard histogram using any number of bins. The data may be sampled with any number of steps 
  in either direction.
\item[{\sc surface}]--- A sequence of 1-D plots offset to give the impression of a surface.
\item[{\sc contour}]--- A contour map derived from linear scaling, n-Sigma about the mean or increment or decrement in magnitudes,
  common logarithms or natural logarithms. The contour levels are written to {\em filename}\_contours.lis for the plotted
  {\em filename}.
\end{description}
The data may be plotted either in axis (x,y) units or pixel (i,j) units. For example, a spectrum of a source might
be plotted  by an astronomer between, say, 2.1 -- 2.5 $\mu$m whereas an engineer might look at the same data from pixels
20 -- 50 to see if there is any noise on the detector. The natural units of P4 are axis units. 
If you want to plot in pixel units, you must set any (or all) of the istart, iend, jstart, jend values to be
non-negative. In the latter case, the labels are ignored and a default title is printed.
\section{The Viewports}
The plotting surface is divided into 9 viewports (0--8) as shown in figure \ref{latfig5}.

%\begin{figure}[h]
\begin{figure}[htbp]
\begin{center}
\begin{picture}(105,85)
\thicklines
\put(0,0){\framebox(50,38){}}
\put(0,42){\framebox(50,38){}}
\put(55,0){\framebox(50,38){}}
\put(55,42){\framebox(50,38){}}
\thinlines
\put(2,44){\framebox(46,34) {Port 0}}
\put(57,62){\framebox(22,16){Port 1}}
\put(81,62){\framebox(22,16){Port 2}}
\put(57,44){\framebox(22,16){Port 3}}
\put(81,44){\framebox(22,16){Port 4}}
\put(2,20){\framebox(46,16) {Port 5}}
\put(2,2){\framebox(46,16)  {Port 6}}
\put(57,2){\framebox(22,34) {Port 7}}
\put(81,2){\framebox(22,34) {Port 8}}
\end{picture}
\caption{The P4 Viewports} \label{latfig5}
\end{center}
\end{figure}

\section{Initialising the Display}
The CGS4 data reduction plotting task has been de-coupled from the
control task and now maintains its own noticeboard containing the
details of the plotting capability on each port. This noticeboard can
be manipulated by the user to tailor the display as required.

Note that, by default, the xwindows device is loaded the first time the
software is created. If you do not wish to do this, execute the
following commands, the first time only {\em e.g.} to set to device
postscript\_l:

\begin{quote}
 \% \ mkdir \ $\sim$/cgs4dr\_configs \\
 \% \ cp {\sc \$\{cgs4dr\_root\}}/default.p4 $\sim$/cgs4dr\_configs/default.p4 \\
 \% \ sed \ 's/xwindows/postscript\_l/' \ $\sim$/cgs4dr\_configs/default.p4 \ $>$ \ \$\$tmp \\
 \% \ mv \ \$\$tmp \ $\sim$/cgs4dr\_configs/default.p4
\end{quote}

Thereafter, whenever the software runs down cleanly, this 
$\sim$/cgs4dr\_configs/default.p4 file will be overwritten with the latest
noticeboard contents.

\section{The Set Commands}
P4 maintains a noticeboard for each port and these items are used
tailor the display in a variety of ways. The more common items can be
accessed through the set\_{\em item}n commands. The suite of commands
available are shown in table \ref{tab3}.  The set\_{\em
item} commands ({\em i.e.} without a specific port number attached)
are treated as a special case and assign the given value to {\em all} ports.

\begin{table}
\begin{center}
\begin{tabular}{||l|l|l|c|l||}
\hline
\ & \ & \ & \ & \\
Set Command & Get Command & Sets or Gets ... & Type & Default \\
$0<n<8$ & $0<n<8$ & \ & \ & \\
\ & \ & \ & \ & \\
\hline
\ & \ & \ & \ & \\
set\_device{\em n}     & get\_device{\em n}     & Device name                & \_CHAR    & xwindows \\
set\_lut{\em n}        & get\_lut{\em n}        & Colour table               & \_CHAR    & {\sc \$p4\_ct/}default  \\
set\_title{\em n}      & get\_title{\em n}      & Title                      & \_CHAR    & A\_U\_T\_O\footnotemark[2] \\
set\_type{\em n}       & get\_type{\em n}       & Display type               & \_CHAR    & IMAGE\footnotemark[3] \\
set\_plane{\em n}      & get\_plane{\em n}      & Display  plane             & \_CHAR    & DATA\footnotemark[4] \\
set\_autoscale{\em n}  & get\_autoscale{\em n}  & Autoscaling                & \_LOGICAL & {\sc true} \\
set\_erase{\em n}      & get\_erase{\em n}      & Pre-erasure of plot        & \_LOGICAL & {\sc true} \\
set\_high{\em n}       & get\_high{\em n}       & High value if autoscale=F  & \_REAL    & 1000.0 \\
set\_low{\em n}        & get\_low{\em n}        & Low value if autoscale=F   & \_REAL    & 0.0 \\
set\_whole{\em n}      & get\_whole{\em n}      & Display to whole array     & \_LOGICAL & {\sc true} \\
set\_xstart{\em n}     & get\_xstart{\em n}     & Start of X-axis if whole=F & \_REAL    & 0.0 \\
set\_xend{\em n}       & get\_xend{\em n}       & End of X-axis if whole=F   & \_REAL    & 256.0 \\
set\_ystart{\em n}     & get\_ystart{\em n}     & Start of Y-axis if whole=F & \_REAL    & 0.0 \\
set\_yend{\em n}       & get\_yend{\em n}       & End of Y-axis if whole=F   & \_REAL    & 256.0 \\
set\_istart{\em n}     & get\_istart{\em n}     & Start X-pixel for sub-plot & \_INTEGER & -1 \\
set\_iend{\em n}       & get\_iend{\em n}       & End X-pixel for sub-plot   & \_INTEGER & -1 \\
set\_jstart{\em n}     & get\_jstart{\em n}     & Start Y-pixel for sub-plot & \_INTEGER & -1 \\
set\_jend{\em n}       & get\_jend{\em n}       & End Y-pixel for sub-plot   & \_INTEGER & -1 \\
set\_errors{\em n}     & get\_errors{\em n}     & Display error bars?        & \_LOGICAL & {\sc true}\footnotemark[5] \\
set\_cut\_dir{\em n}   & get\_cut\_dir{\em n}   & Direction of cut?          & \_CHAR    & X \\
set\_cut\_start{\em n} & get\_cut\_start{\em n} & Start of cut summation     & \_REAL    & 20.0 \\
set\_cut\_end{\em n}   & get\_cut\_end{\em n}   & End of cut summation       & \_REAL    & 40.0 \\
set\_bins{\em n}       & get\_bins{\em n}       & Number of histogram bins   & \_INTEGER & 50 \\
set\_overcol{\em n}    & get\_overcol{\em n}    & Colour of overgraph line   & \_CHAR    & RED\footnotemark[6] \\
set\_fg\_colour{\em n} & get\_fg\_colour{\em n} & Foreground colour          & \_CHAR    & BLACK\footnotemark[6] \\
set\_bg\_colour{\em n} & get\_bg\_colour{\em n} & Background colour          & \_CHAR    & WHITE\footnotemark[6] \\
set\_contour{\em n}    & get\_contour{\em n}    & Contour type               & \_CHAR    & LIN\footnotemark[7] \\
\ & \ & \ & \ & \\
\hline
\end{tabular}
\caption{The Set and Get Commands for $0 < n <8$.} \label{tab3}
\end{center}
\footnotemark[2]{\scriptsize Auto-generates from axis and label units in dataset.}

\footnotemark[3]{\scriptsize Options are IMAGE | GRAPH | OVERGRAPH | HISTOGRAM | SURFACE | CONTOUR.}

\footnotemark[4]{\scriptsize Options are DATA | QUALITY | ERRORS.}

\footnotemark[5]{\scriptsize Only applies to GRAPH | OVERGRAPH | SURFACE plots.}

\footnotemark[6]{\scriptsize Options are RED | BLUE | GREEN | YELLOW | ORANGE | BLACK | WHITE.}

\footnotemark[7]{\scriptsize Options are LIN | SIG | MAGD | MAGI | LOGD | LOGI | LND |LNI.}
\end{table}

\section{The Get Commands}
In a similar fashion, there are a suite of get\_{\em item}n commands
that are the inverse of the `set' commands shown in table \ref{tab3}.
The get\_{\em item} commands ({\em i.e.} without a
specific port number attached) are treated as a special case and return
the required value from {\em all} ports.

There is an important distinction, however, inasmuch as the get\_{\em
item}n commands demand an empty variable into which to write the
result. Failure to do so, may crash Unix-ICL.  Thus the correct way to
enquire, for example, about the current plot device open port 4 would
be to use the commands:

\begin{quote}
  ICL$>$ junk = ' ' \\
  ICL$>$ get\_device4 (junk) \\
  ICL$>$ print 'Port 4 plot device = ' \& (junk) \\
  {\tt Port 4 plot device = xwindows }
\end{quote}

There is a good chance that the user will forget do this so it is recommended that
the get\_{\em item} commands be used instead.

\section{How to Enable Overgraphs}
It might appear from the other sections of this document that the 
set{\em something}\_display routines refer to specific ports. If that were true, 
there would be no way to enable overgraphs during the automated display 
of reduced data. How then do we overcome this restriction? The answer lies 
in allowing the procedure to prompt for a response and entering one of 
the following (where sans-serif type indicates an input):
\begin{quote}
ICL$>$ set\_observation\_display \\
Display in port 0?  $>$ {\tt yes} \\
Display in port 1?  $>$ {\tt yes} \\
Display in port 2?  $>$ {\tt yes} \\
Display in port 3?  $>$ {\tt yes} \\
Display in port 4?  $>$ {\tt yes} \\
Display in port 5?  $>$ {\tt overgraph port=1 cut=X spos=20 epos=40 colour=red} \\
Display in port 6?  $>$ {\tt overgraph port=1 cut=X spos=25 epos=30 colour=blue} \\
Display in port 7?  $>$ {\tt overgraph port=1 cut=X spos=30 epos=35 colour=green} \\
Display in port 8?  $>$ {\tt overgraph port=1 cut=X spos=15 epos=45 colour=yellow}
\end{quote}

In this example, port 0 -- 4 display data in whatever way they are set
up. Ports 5 -- 8, however, are used as dummies to display overgraphs in
port 1 (which has presumably been set up to display graphs in the first
place). Note the syntax of the overgraph command line. The parameters
have the following meaning:

\begin{description}
\item[{\sc port}]--- The port in which to display the overgraph.
\item[{\sc cut}]--- cut direction (x, X, y, Y, h, H, v or V).
\item[{\sc spos}]--- Start row of slice for extraction of graph data.
\item[{\sc epos}]--- End row of slice for extraction of graph data.
\item[{\sc colour}]--- The colour of the pen for the overgraph plot.
\end{description}

\section{Some Examples of Using the Task}
Once P4 has been started, a default configuration file is copied to
$\sim$/cgs4dr\_configs/default.p4 and an xwindows device is opened. An
image of a simple ramp function is also plotted using the default
colour table. Examples of using the task from this point now follow
demonstrating most of the functionality available.

Note that all ICL procedures have `set nocheckpars' enabled so that
they can be invoked with or without parameters. If the required input
parameters are not given on the command lines, they are either prompted
for. If a port number is required but not specified, the default port
selected is 0.

\subsection{Getting the Viewport Configuration}
To obtain the current settings for port 0, say, use ({\em e.g.}):

\begin{quote}
ICL$>$ get\_display 0
\end{quote}

\subsection{Plotting Error Bars}
To plot error bars on a graph of ro940311\_17 on port 5, use ({\em e.g.}):

\begin{quote}
ICL$>$ set\_errors5 true \\
ICL$>$ plot\_graph ro940311\_17 5
\end{quote}

Errors can be plotted on graphs, overgraphs and surfaces only although
the latter is not recommended as it may take some time.

\subsection{Plotting a Sub-Area of an Image in Axis Units} 
To plot a sub-area in axis units of {\sc \$kappa\_dir}/comwest on port
7, use ({\em e.g.}):

\begin{quote}
ICL$>$ set\_whole7 false \\
ICL$>$ set\_xstart7 64 \\
ICL$>$ set\_xend7 128 \\
ICL$>$ set\_ystart7 64 \\
ICL$>$ set\_yend7 192 \\
ICL$>$ plot\_image {\sc \$kappa\_dir}/comwest 7
\end{quote}

\subsection{Plotting a Sub-Area of an Image in Pixel Units}
To plot a sub-area in axis units of {\sc \$kappa\_dir/}comwest on port
7, use ({\em e.g.}):

\begin{quote}
ICL$>$ set\_whole7 false \\
ICL$>$ set\_istart7 64 \\
ICL$>$ set\_iend7 128 \\
ICL$>$ set\_jstart7 64 \\
ICL$>$ set\_jend7 192 \\
ICL$>$ plot\_image {\sc \$kappa\_dir/}comwest 7
\end{quote}

\subsection{Plotting to a Given Scale}
To plot a graph of ro940311\_17 between the limits -100 and 500 on port
8, use ({\em e.g.}):

\begin{quote}
ICL$>$ set\_autoscale8 false \\
ICL$>$ set\_low8 -100 \\
ICL$>$ set\_high8 500 \\
ICL$>$ plot\_graph ro940311\_17 8
\end{quote}
\subsection{Changing the Default Title}
To plot the standard colour ramp on port 0, with a different title, use ({\em e.g.}):
\begin{quote}
ICL$>$ set\_title0 'P4 Colour Table' \\
ICL$>$ plot\_image {\sc \$p4\_ct/}cgs4 0
\end{quote}
\subsection{Printing PostScript Graphics}
To print a PostScript file of the standard colour ramp on port 0, 
use ({\em e.g.}):
\begin{quote}
ICL$>$ set\_device0 ps\_l \\
ICL$>$ print\_ps {\sc \$p4\_ct/}cgs4 0
\end{quote}
To obtain a list of current devices, use:
\begin{quote}
 \% more /star/starlink/lib/gns/gksnames.dat
\end{quote}

\chapter{Manipulating the Noticeboards}
\markboth{Behind the Scenes with NBS}{\stardocname}
\section{How to Manipulate the Noticeboards Directly}

From time to time the user will want to do something not described in
this document. Perhaps it might be to try something that was available
under VMS. This is usually accomplished by manipulating the
noticeboards directly.  So, how is it done?

First, the noticeboards are referenced by aliases contained within the
ICL (global) variables cred4\_nb\_alias and p4\_nb\_alias. These can be
used in conjunction with the ICL defstring utility to create mnemonics
for each noticeboard item. Indeed, the files cred4.icl and p4.icl in
{\sc \$cgs4dr\_root} are full of such definitions.

Let us take an example: we wish to use a specific {\sc bias}
observation rather than let the software auto-select a suitable frame.
The required {\sc bias} is, say, ro950516\_256. This can be achieved by
defining the following mnemonics:

\begin{quote}
 ICL$>$ defstring set\_bias\_mode putnbs ((cred4\_nb\_alias)\&'.miscellaneous.bias\_mode') \\
 ICL$>$ defstring get\_bias\_mode getnbs ((cred4\_nb\_alias)\&'.miscellaneous.bias\_mode') \\
 ICL$>$ defstring set\_specified\_bias putnbs ((cred4\_nb\_alias)\&'.miscellaneous.specified\_bias') \\
 ICL$>$ defstring get\_specified\_bias getnbs ((cred4\_nb\_alias)\&'.miscellaneous.specified\_bias')
\end{quote}

The quotes, braces and ampersand are all necessary. We can use them
(rather verbosely) thus:

\begin{quote}
 ICL$>$ bias\_mode = ' ' \\
 ICL$>$ get\_bias\_mode (bias\_mode) \\
 ICL$>$ =(bias\_mode) \\
 BOTH \\
 ICL$>$ set\_bias\_mode SPECIFIED \\
 ICL$>$ set\_specified\_bias ro950516\_256 \\
 ICL$>$ get\_bias\_mode (bias\_mode) \\
 ICL$>$ =(bias\_mode) \\
 SPECIFIED \\
 ICL$>$ which\_bias = ' ' \\
 ICL$>$ get\_specified\_bias (which\_bias) \\
 ICL$>$ =(which\_bias) \\
 ro950516\_256
\end{quote}

This might seem complicated but the above definitions can be saved to a
file and loaded when needed. {\em E.g:}

\begin{quote}
 ICL$>$ \% cat mydefs.icl \\
 defstring set\_bias\_mode putnbs ((cred4\_nb\_alias)\&'.miscellaneous.bias\_mode') \\
 defstring get\_bias\_mode getnbs ((cred4\_nb\_alias)\&'.miscellaneous.bias\_mode') \\
 defstring set\_specified\_bias putnbs ((cred4\_nb\_alias)\&'.miscellaneous.specified\_bias') \\
 defstring get\_specified\_bias getnbs ((cred4\_nb\_alias)\&'.miscellaneous.specified\_bias') \\
 ICL$>$ load mydefs.icl
\end{quote}

The noticeboards are described in tables \ref{tab4}, \ref{tab5}, \ref{tab6} and \ref{tab7} respectively. 

\begin{table}
\begin{center}
\begin{tabular}{||l|l|l||}
\hline
Noticeboard Item\footnotemark[2] & Meaning & Typical Value \\
\hline
.display.int\_p0 & Display integration in port 0? & NO \\
.display.int\_p1 & Display integration in port 1? & NO \\
.display.int\_p2 & Display integration in port 2? & NO \\
.display.int\_p3 & Display integration in port 3? & NO \\
.display.int\_p4 & Display integration in port 4? & NO \\
.display.int\_p5 & Display integration in port 5? & NO \\
.display.int\_p6 & Display integration in port 6? & NO \\
.display.int\_p7 & Display integration in port 7? & NO \\
.display.int\_p8 & Display integration in port 8? & NO \\
.display.obs\_p0 & Display observation in port 0? & NO \\
.display.obs\_p1 & Display observation in port 1? & YES \\
.display.obs\_p2 & Display observation in port 2? & YES \\
.display.obs\_p3 & Display observation in port 3? & YES \\
.display.obs\_p4 & Display observation in port 4? & YES \\
.display.obs\_p5 & Display observation in port 5? & NO \\
.display.obs\_p6 & Display observation in port 6? & NO \\
.display.obs\_p7 & Display observation in port 7? & NO \\
.display.obs\_p8 & Display observation in port 8? & NO \\
.display.grp\_p0 & Display group in port 0? & NO \\
.display.grp\_p1 & Display group in port 1? & NO \\
.display.grp\_p2 & Display group in port 2? & NO \\
.display.grp\_p3 & Display group in port 3? & NO \\
.display.grp\_p4 & Display group in port 4? & NO \\
.display.grp\_p5 & Display group in port 5? & YES \\
.display.grp\_p6 & Display group in port 6? & YES \\
.display.grp\_p7 & Display group in port 7? & NO \\
.display.grp\_p8 & Display group in port 8? & NO \\
.display.spc\_p0 & Display spectrum in port 0? & YES \\
.display.spc\_p1 & Display spectrum in port 1? & NO \\
.display.spc\_p2 & Display spectrum in port 2? & NO \\
.display.spc\_p3 & Display spectrum in port 3? & NO \\
.display.spc\_p4 & Display spectrum in port 4? & NO \\
.display.spc\_p5 & Display spectrum in port 5? & NO \\
.display.spc\_p6 & Display spectrum in port 6? & NO \\
.display.spc\_p7 & Display spectrum in port 7? & NO \\
.display.spc\_p8 & Display spectrum in port 8? & NO \\
\hline
\end{tabular}
\caption{CRED4 Noticeboard Display Items} \label{tab4}
\end{center}
\footnotemark[2]{\scriptsize {\em E.g.} defstring get\_display\_intp0 getnbs ((cred4\_nb\_alias)\&'.display.int\_p0')}
\end{table}

\begin{table}
\begin{center}
\begin{tabular}{||l|l|l||}
\hline
Noticeboard Item\footnotemark[2] & Meaning & Typical Value \\
\hline
.reduction.add\_int.execute & Coadd integrations? & YES \\
.reduction.add\_obs.execute & Coadd observations into groups? & YES \\
.reduction.archive\_obs.execute & Archive observations? & NO \\
.reduction.file\_obs.execute & File observations to index? & YES \\
.reduction.divide\_by\_ff.execute & Divide by FLAT field? & YES \\
.reduction.divide\_by\_std.execute & Divide by STANDARD? & NO \\
.reduction.subtract\_bias.execute & Subtract BIAS frame? & YES \\
.reduction.subtract\_dark.execute & Subtract DARK frame? & NO \\
.reduction.normalise\_ff.execute & Normalise FLAT field? & YES \\
.reduction.normalise\_ff.method & Method for mormalising FLAT & POLYFIT \\
.reduction.normalise\_ff.order & Order for polynomial to fit & 3 \\
.reduction.normalise\_ff.boxsize & Smooth box size for fit & 5 \\
.reduction.to\_wavelength.execute & Wavelength calibrate? & YES \\
.reduction.to\_wavelength.method & Method to use for calibration & ESTIMATED \\
.reduction.autofit.execute & Automatically fit lines? & NO \\
.reduction.autofit.nrows & Number of rows to be averaged?  & 1 \\
.reduction.autofit.row1 & First row for fitting & 10 \\
.reduction.autofit.row2 & Second row for fitting & 40 \\
.reduction.autofit.xstart & Xstart for extraction and fitting & 1.0 \\
.reduction.autofit.xend & Xend for extraction and fitting & 256.0 \\
.reduction.extract\_spc.execute & Extract spectra? & NO \\
.reduction.extract\_spc.algorithm & Algorithm to use & BRIGHT \\
.reduction.extract\_spc.invert & Invert the output spectrum? & {\sc false} \\
.reduction.extract\_spc.row1s & First row for extraction & 23 \\
.reduction.extract\_spc.row1e & First row start for extraction & 25 \\
.reduction.extract\_spc.row2s & Second row start for extraction & -1 \\
.reduction.extract\_spc.row2e & Second row end for extraction & -1 \\
.reduction.extract\_spc.row3s & Third row start for extraction & -1 \\
.reduction.extract\_spc.row3e & Third row end for extraction & -1 \\
\hline
\end{tabular}
\caption{CRED4 Noticeboard Reduction Items} \label{tab5}
\end{center}
\footnotemark[2]{\scriptsize {\em E.g.} defstring get\_add\_int getnbs ((cred4\_nb\_alias)\&'.reduction.add\_int')}
\end{table}

\begin{table}
\begin{center}
\begin{tabular}{||l|l|l||}
\hline
Noticeboard Item\footnotemark[2] & Meaning & Typical Value \\
\hline
.miscellaneous.bias\_mode & How to search for BIAS & BOTH \\
.miscellaneous.dark\_mode & How to search for DARK & BOTH \\
.miscellaneous.flat\_mode & How to search for FLAT & BOTH \\
.miscellaneous.calib\_mode & How to search for CALIB & BOTH \\
.miscellaneous.standard\_mode & How to search for STD & BOTH \\
.miscellaneous.specified\_bias & Specified BIAS frame & royymmdd\_oooo \\
.miscellaneous.specified\_dark & Specified DARK frame & royymmdd\_oooo \\
.miscellaneous.specified\_flat & Specified FLAT frame & royymmdd\_oooo \\
.miscellaneous.specified\_calib & Specified CALIB frame & cayymmdd\_oooo \\
.miscellaneous.specified\_std & Specified STANDARD frame & styymmdd\_gggg \\
.miscellaneous.errors & Hot to propagate error & {\sc from\_obs} \\
.miscellaneous.lincoeffs & Name of linearisation file & \# \\
.miscellaneous.mask & Name of bad pixel mask & fpa46\_short \\
.miscellaneous.pf\_polyfit & Polysky method & NONE \\
.miscellaneous.pf\_degree & Degree of polynomial for fit & 1  \\
.miscellaneous.pf\_nreject & Number of points to reject & 0 \\
.miscellaneous.pf\_saye1 & Start of first sky area & 20 \\
.miscellaneous.pf\_says1 & End of first sky area & 25 \\
.miscellaneous.pf\_saye2 & Start of second sky area & 35 \\
.miscellaneous.pf\_says2 & End of second sky area & 40 \\
.miscellaneous.pf\_saye3 & Start of third sky area & -1 \\
.miscellaneous.pf\_says3 & End of third sky area & -1 \\
.miscellaneous.pf\_saye4 & Start of fourth sky area & -1 \\
.miscellaneous.pf\_says4 & End of fourth sky area & -1 \\
.miscellaneous.pf\_weight & Weight the fit? &  {\sc true} \\
.miscellaneous.variance\_wt & T for variance weighting & {\sc false} \\
.miscellaneous.add\_in\_pairs & Add observations in pairs? & {\sc true} \\
.miscellaneous.sky\_wt & Weighting value for sky frames & 1.0 \\
\hline
\end{tabular}
\caption{CRED4 Noticeboard Miscellaneous Items} \label{tab6}
\end{center}
\footnotemark[2]{\scriptsize {\em E.g.} defstring get\_bias\_mode getnbs ((cred4\_nb\_alias)\&'.miscellaneous.bias\_mode')}
\end{table}

\begin{table}
\begin{center}
\begin{tabular}{||l|l|l||}
\hline
Noticeboard Item\footnotemark[2] $0<n<8$ & Meaning & Typical Value \\
\hline
.port\_$n$.device\_name & Name of plotting device & xwindows \\
.port\_$n$.device\_xopt & X-axis options & BNCTSI \\
.port\_$n$.device\_yopt & X-axis options & BNCTSI \\
.port\_$n$.device\_lut & Colour table & \$P4\_CT/default \\
.port\_$n$.display\_plane & Plane of data to display & DATA \\
.port\_$n$.display\_data & Name of dataset & \$P4\_CT/cgs4 \\
.port\_$n$.display\_type & Type of display required & IMAGE \\
.port\_$n$.bg\_colour & Display device background colour & WHITE \\
.port\_$n$.fg\_colour & Display device foreground colour & BLACK \\
.port\_$n$.colour\_style & Monochrome or full colour?  & COLOUR \\
.port\_$n$.overcolour & Colour for overgraph? & RED \\
.port\_$n$.cut\_direction & Direction of cut for (OVER)GRAPHs? & X \\
.port\_$n$.contour\_type & Type of contour plot required? & LIN \\
.port\_$n$.title & Title for plot & A\_U\_T\_O \\
.port\_$n$.autoscale & T for autoscaling & {\sc true} \\
.port\_$n$.plot\_axes & T if axes are to be plotted & {\sc true} \\
.port\_$n$.plot\_errors & T if errors are to be plotted & {\sc true} \\
.port\_$n$.plot\_whole & T if whole array is to be plotted & {\sc true} \\
.port\_$n$.plot\_axes & T if axes are to be plotted & {\sc true} \\
.port\_$n$.pre\_erase\_plot & T if port is to be cleared & {\sc true} \\
.port\_$n$.histogram\_bins & Number of histogram bins & 50 \\
.port\_$n$.histogram\_smooth & Smooth box for histogramming & 3 \\
.port\_$n$.histogram\_xstep & Number of columns to overstep & 1 \\
.port\_$n$.histogram\_ystep & Number of rows to overstep & 1 \\
.port\_$n$.contour\_levels & Number of contour levels & 25 \\
.port\_$n$.istart & Start pixel number for sub-arrays & 1 \\
.port\_$n$.iend & End pixel number for sub-arrays & 256 \\
.port\_$n$.jstart & Start pixel number for sub-arrays & 1 \\
.port\_$n$.jend & End pixel number for sub-arrays & 256 \\
.port\_$n$.xstart & Xstart of sub-array & 1.0 \\
.port\_$n$.xend & Xend of sub-array & 256.0 \\
.port\_$n$.ystart & Ystart of sub-array & 1.0 \\
.port\_$n$.yend & Yend of sub-array & 256.0 \\
.port\_$n$.slice\_start & Start of slice for (OVER)GRAPH & 20.0 \\
.port\_$n$.slice\_end & End of slice for (OVER)GRAPH & 40.0 \\
.port\_$n$.char\_height & Height of characters in labels & 1.0 \\
\hline
\end{tabular}
\caption{P4 Noticeboard Modifiable Items} \label{tab7}
\end{center}
\footnotemark[2]{\scriptsize {\em E.g.} defstring get\_device0 getnbs ((p4\_nb\_alias)\&'.port\_0.device\_name')}
\end{table}

%------------------------------------------------------------------------------
%  Part III 
\part{THE Tcl/tk INTERFACE}
\pagestyle{myheadings}
\markboth{Tcl/tk Interface}{\stardocname}

\chapter{Introduction to Tcl/tk}
\section{Historical Perspective}
Tcl/tk (Ousterhout, 1994) stands for the `tool command language/toolkit'.
It was written by John Ousterhout of the University of California in
response to the continual development of ad-hoc interfaces to different
projects. What was needed was a generic command language that could be
modified to suit the user's needs {\em i.e.} it had to be {\em
extensible}. Thus tcl was born. The addition of the toolkit at a later
stage made tcl/tk a very powerful {\sc gui} prototyping environment available
with little effort and, perhaps more importantly, at no cost to the
community. 

In September 1993, the {\sc adam} V Workshop provided demonstrations of two
proposed {\sc gui}s for Starlink development: Xadam (based upon tcl/tk's expect
command) and ICLmenus (an in-house solution). {\sc gui}-builders were ruled out
at any early stage. In October 1994, at the Baltimore ADASS IV meeting, it
was clear that tcl/tk was taking the astronomical world by storm (as, for
example, had xmosaic a year or so earlier). Starlink had implemented some
extensions to tcl to allow the system to communicate with {\sc adam} tasks.
Later work done at ROE provided tcl extensions to communicate with NBS.
Together these systems formed what is now known as {\sc tcladam} (Terrett, 1995)
and it is this product that is the basis for the Portable-CGS4DR graphical
user interface. 

\section{A Few Words on {\sc gui}s}
{\sc gui} stands for graphical user interface. Personally, I prefer the term
`pictorial interface' as it gets the idea accross must quicker. In a {\sc gui}
there are graphical (pictorial) elements such as pull-down menus, action
buttons, radiobuttons, checkbuttons, slide bars, scroll bars, text panes
and the like. All of these elements are generically termed `widgets' and
they can be ordered into hierarchies called widget trees. {\sc gui}s are also
capable of providing colourful interfaces and usually require a mouse to
interact with the system. Advanced {\sc gui}s have {\em keyboard traveral} so
that a keyboard can be used for input when a mouse is not available. 

\section{The Portable-CGS4DR Style}
The Portable-CGS4DR {\sc tcladam} interface provides a window to each major
task. Each window consists of a title bar, a menu bar, a text pane and a
wigdet tree in some order. In addtion, {\sc tcladam} also provides a {\sc gwm} widget
which includes the xdisplay plotting surface. Note that the style described
below may change as users feedback their preferences. 

It may also appear different if your $\sim$/.Xdefaults file has been edited for any reason.
In particular, if you use white or near-white as a foregound colour, the Portable-CGS4DR
{\sc gui} may be unreadable. The way around this is to invoke the following command
{\em before} invoking the {\sc gui}:

\begin{quote}
 \% \ xrdb \ $<$ \ /dev/null
\end{quote}

The defaults for the Portable-CGS4DR interface are:
\begin{description}
\item[{\sf OK Button}] --- Performs some action {\em i.e.} it does something.
\item[{\sf Cancel Button}] --- Aborts some menu without action {\em i.e.} does nothing.
\item[{\sf Dismiss Button}] --- Dismisses an informational dialogue box {\em i.e.} does nothing.
\item[{\sf Background}] --- wheat.
\item[{\sf Text Pane}] --- snow.
\item[{\sf Scrollbars}] --- lightyellow (inactive), palegreen (active). 
\item[{\sf Checkbuttons}] --- Used for binary choices ({\em e.g.} autoscaling in the plotting interface). \\
 --- wheat with grey border (inactive), all grey (active). 
\item[{\sf Radiobuttons}] --- Used for multiple but restricted choices 
({\em e.g.} {\sf Yes}, {\sf No} or {\sf Ask}). \\
 --- wheat with grey border (inactive), palegreen with grey border (active).
\item[{\sf Actionbuttons}] --- Used to invoke some action
  either by communicating directly with the task ({\em e.g.} Plot) or 
  generating a dialogue box. \\
  --- pink (inactive), palegreen (active). 
for further input ({\em e.g.} flux\_calibrate).
\item[{\sf Menus}] --- wheat (inactive), palegreen (active).
\item[{\sf Menu Items}] --- wheat (inactive), palegreen (active). 
\item[{\sf Cursor}] --- green arrow (normal), orange pirate (a dialogue box is open), red wrist-watch (task busy), yellow pencil (help box open).
\item[{\sf Mouse Button 1}] Use as follows \\
  --- Single Click: Invokes action ({\em e.g.} OK or Cancel). \\
  --- Double Click: Invokes action in listbox
\item[{\sf Mouse Button 2}] Use as follows \\
  --- Single Click: Invokes dynamic default if appropriate. \\
  --- Double Click: Clears entry widget.
\item[{\sf Mouse Button 3}] --- Invokes on-line help dialogue box.
\end{description}

\subsection{The Title Bar}
The title bar identifies the software version and task name to which it applies.

\subsection{The Menu Bar}
The menu bar and the menus contained within have keyboard traversal enabled so that, for example, any option
can be invoked using the {\sc alt} key and the appropriate underscored character.

The menu bar consists of three items: {\sf File}, {\sf Options} and {\sf Help}. 

The {\sf File} menu contains just one entry: {\sf Exit}. This is used to exit Portable-CGS4DR after confirmation
via a dialogue box.

The {\sf Help} menu contains several items: {\sf Author(s)}, {\sf Tcl/tk Version} and {\sf Portable-CGS4DR Version} 
all write an information string to the text pane. The {\sf Portable-CGS4DR WWW Page} item invokes the Starlink findme script.

The {\sf Options} menu contains several items some of which are geared towards the particular task from
which the menu was invoked. The default set contains a checkbutton for toggling verbose messages on (illuminated)
and off, an action to clear the text widget and an action to communicate with the task directly via a dialogue box.

\subsection{The Text Pane}
The text pane is a large area which the tasks use to output their
messages of any type. It is not writeable by the user but may be
cleared from the {\sf Options} menu using {\sf Clear Text Widget}.

\subsection{The Widget Tree}
The widget tree is specific to each task. They are dealt with in separate chapters.

\section{Invoking Dynamic Defaults and On-line Help}
The generic mouse supplied with workstations has three buttons labelled
left-to-right as MB1, MB2 and MB3. The Portable-CGS4DR style is to use
all three buttons:

\begin{itemize}
\item MB1 - Single Click: Invoke an action or dialogue box or similar 
\item MB1 - Double Click: Invoke an action or dialogue box from a listbox widget
\item MB2 - Single Click: Invoke a dynamic default value 
\item MB2 - Double Click: Reset a text entry widget to a null string (empty) 
\item MB3 - Single Click: Invoke on-line help 
\end{itemize}

In listboxes, the primary selection of text is made using a single click with MB1.
This {\em must} be done prior to using MB3 for on-line help. Failure to do so will 
either invoke the help text associated with the current selected item (which is 
likely to be different from the one you want) or return an error: {\tt
PRIMARY selection doesn't exist or form ``STRING'' not defined}.

Dynamic defaulting has been enabled, where appropriate, using MB2. The following
scheme has been enabled: using MB2 within any particular data input widget will set the
dynamic default for just that widget whereas using MB2 on any text label within
that widget tree sets all defaults within that tree. A double click within an entry
widget clear that widget of all text.

As an example, the CRED4 setup button invokes a dialogue box with the various
data reduction sequences offerred via a series of radiobuttons. Using MB2
within any horizontal level of radiobuttons will set just that value to some
sensible default. If, however, you position the mouse over any text label
(such as `Subtract BIAS Frame?', `Subtract DARK Frame?' etc), the all of the
displayed values will be set to sensible defaults when MB2 is depressed.

\section{Starting the Software}
The Portable-CGS4DR {\sc gui} can be started with three optional command line
parameters: 

\begin{minipage}{120mm}
\begin{quote}
  \$1 is a data directory \hfill /\$\{home\}/ \\
  \$2 is a UT date \hfill /current UT-date/ \\
  \$3 is the file type {\sc dst} or {\sc ndf} \hfill /ndf/
\end{quote}
\end{minipage}

To start the software, source the {\sc starlink} login and cshrc files and
use your local data directory. {\em E.g.:}

\begin{minipage}{120mm}
\begin{quote}
  \%  source \ \ /star/etc/login \\
  \%  source \ \ /star/etc/cshrc \\
  \%  cgs4dr\_tcl \ \ /scratch/pnd/19940811 \ \ 940811 \ \ ndf
\end{quote}
\end{minipage}

If any of the command line parameters are missing, a confirmation dialogue box
is created. Items labelled in {\em red} script within this box are those that
require confirmation.

Once started, as a background job, a `traffic light' window will appear containing entries
for each of the four primary tasks. Initially, the backgrounds to these widgets are red
(meaning stop) and as the tasks are loaded and become available they turn to green (for go).
Once all four tasks are `go' the window disappears to be replaced by the 
3 full-size windows (cred4, p4\_disp and qman) and two iconic windows
(red4 and p4 control). At this point, Portable-CGS4DR using {\sc tcladam} is
ready.

There is also a `setup' facility for defining evironmental variables without
invoking the software:

\begin{minipage}{120mm}
\begin{quote}
  \%  source \ \ /star/etc/login \\
  \%  source \ \ /star/etc/cshrc \\
  \%  cgs4dr\_setup \ \ /scratch/pnd/19940811 \ \ 940811 \ \ ndf \\[4ex]
      {\tt Setup Portable-CGS4DR V1.3-0 for /scratch/pnd/19940811} \\[2ex]
  \%
\end{quote}
\end{minipage}

\subsection{Startling the Software at UKIRT}
If you are using the software at the telescope, the following startup procedure is
recommended:

\begin{minipage}{120mm}
\begin{quote}
  \%  source \ \ /star/etc/login \\
  \%  source \ \ /star/etc/cshrc \\
  \%  start\_adamnet \\
  \%  cgs4dr\_ukirt \\
  \%
\end{quote}
\end{minipage}

The alias {\it cgs4dr\_ukirt} points to the correct directory and UT-date.
When the night's observing is finished, you should remember to kill ADAMnet with:

\begin{minipage}{120mm}
\begin{quote}
  \%  stop\_adamnet \\
  \%
\end{quote}
\end{minipage}

\section{Common Problems Upon Startup}
\begin{description}
\item[{\bf \S}] {\sf /star/bin/cgs4dr/tcl/cgs4dr\_tcl: Command not found.} 

  The awish program has not been found.

  By default, awish is installed in /star/bin/awish and this is reflected
  in the first line of the file {\sc \$cgs4dr\_root}/tcl/cgs4dr\_tcl:
  \begin{verbatim}
    #!/star/bin/awish -file
  \end{verbatim}
  If awish has been installed elsewhere, this line will need editing to reflect 
  the correct location.

\item[{\bf \S}] {\sf Invalid command name ``adamtask''.}

  The {\sc adam} extensions to tcl are not readable. 

  The {\sc tcladam} software specifies a series of extensions to tcl that allow
  communications with the {\sc adam} messaging system and, hence, the tasks. These
  extensions are held in some directory in a file called adamtask.tcl (and others).
  The {\sc gui} startup file {\sc \$cgs4dr\_root}/tcl/cgs4dr\_tcl has a line thus:
  \begin{verbatim}
    set tclAdamLib /star/lib/tk/adam
  \end{verbatim}
  This needs to reflect the location of your site's copy of the {\sc tcladam} software.

\item[{\bf \S}] {\sf cgs4dr\_tcl aborting, failed creating task rendezvous file: File exists.}

  Task redezvous files were not deleted by previous shutdown.

  The task rendezvous files are located in the directory {\sc \$home}/adam and have the generic form 
  {\sf xxxx}\_{\em task}\_{\sf yyyy} where {\sf xxxx} and {\sf yyyy} are unique numbers
  and {\em task} is one of the Portable-CGS4DR tasks (cred4, red4, p4, qman and, possibly, ukirtfig).
  There are also files called cgs4dr\_tcl\_relay\_{\sf zzzz} and the like where {\sf zzzz}
  is another unique identifier. These are the task rendezvous files. If the software is, 
  for example, aborted (using \% kill -9 {\em something}) rather than shutdown, these files may not be deleted.
  Subsequent attempts to run the software will fail until these rendezvous files are
  deleted. A list of current rendezvous files can be created thus:

  \begin{verbatim}
    % ls -l ~/adam | grep 'p---'
  \end{verbatim}

  Note that if you run the software more than once, some files are created with names such as
  `cgs4dr\_tcl \#2\_xxxx'. The quotes are deliberate as the filename contains a space character
  which is completely legal under Unix. Such files must be deleted properly.

\item[{\bf \S}] {\sf grab failed: another application has grab while executing ``grab \$QmanWidgets(INTERRUPT)''}

  Tcl/tk can't grab the widget defining the interrupt.

  This occurs when another version of the queue manager task is running or a previous incantation was
  not shut down properly. The solution is to run the cgs4dr\_kill script before re-invoking the software.

\item[{\bf \S}] {\sf A dialogue box appears requesting parameter values.}

  This is a result of a file corruption probably in the ADAM parameter file.  There are two
  solutions. The easier, and preferred, solution is to shut the software down using the commands
  in \S \ref{restart}. If you wish to persevere, though, the parameter prompt box appears due to
  character items having a missing end-quote mark. You can always just add
  the end-quote (or delete the start-quote) and click on {\sf OK} for each item.

\item[{\bf \S}] {\sf Unable to allocate colour cells.}

  Software could not allocate enough colour cells.

  The number of colour cells is finite. Running too many (colour) applications (such as
  netscape, xmosaic etc) can exhaust the number of colour cells resulting in a failure
  to load the {\sc gui}. If this happens, you must delete some applications.

\item[{\bf \S}] {\sf adamnet failed to bind socket: Address already in use. \\
  adamnet: bad initialisation status = 141525083}

  This only happens at {\sc ukirt} where distributed ADAMnet communications are required
  between the VAX data acquisition and Unix data reduction systems. This happens because
  another user left ADAMnet running. The solution is to login as a privileged user and do:

  \begin{verbatim}
    % kill -9 `ps -ef | grep adamnet | grep -v grep | nawk '{print $2}'`
  \end{verbatim}
\end{description}

\section{Running the Tasks Individually}
In Portable-CGS4DR the software tasks have been de-coupled which allows each task to be
run individually. Clearly this functionality is useful for debugging. Some tasks, however,
are generic so that they can be used outside of Portable-CGS4DR. I am thinking particularly
of the plotting task which can display {\sc dst} or {\sc ndf} data. If you wish to run the
tasks separately, set up the symbolic links thus:

\begin{quote}
  \% \ ln -s \ \$(CGS4DR\_ROOT)/tcl/cgs4dr\_tcl \ p4\_tcl \\
  \% \ ln -s \ \$(CGS4DR\_ROOT)/tcl/cgs4dr\_tcl \ q4\_tcl \\
  \% \ ln -s \ \$(CGS4DR\_ROOT)/tcl/cgs4dr\_tcl \ r4\_tcl \\
  \% \ ln -s \ \$(CGS4DR\_ROOT)/tcl/cgs4dr\_tcl \ c4\_tcl 
\end{quote}

One can then invoke the plotting task, like so:

\begin{quote}
  \% \ p4\_tcl \ \&
\end{quote}

\markboth{The GUI Interfaces}{\stardocname}
\chapter{The GUI Interfaces}
\section{The Reduction Task Interface}
The RED4 interface is the simplest of its kind being a suite of commands 
contained within a listbox. An action is invoked by double clicking on a selected item using MB1
Some actions are `direct' {\em i.e} they do something with no further
input ({\em e.g.} status) whilst others display dialogue boxes for manipulation of the action
parameters ({\em e.g.} add\_two\_images). Error and informational messages are displayed in the text pane.

For a full description of each action invoke on-line help using MB3 or view the hypertext web document.

Before any such action is invoked, however, the interface checks that the task is
ready to execute commands and is not busy doing other things. If the task is busy a confirmation 
dialogue box is returned: the {\sf Cancel} button aborts the action whereas the {\sf OK} will continue 
{\em and will wait until the task becomes free}.

\section{The Queue Manager Task Interface}
The QMAN interface comprises mainly action buttons.
As before, error and informational messages are displayed in the text pane.
The various elements in the widget tree are described below.

For a full description of each action invoke on-line help using MB3 or view the hypertext web document.

\subsection{Observation Range}
The widget tree labelled {\sf Observation Range} allows the user to enter a
suitable range of observations into the data reduction queue. The {\sf First}
and {\sf Last} parameters are checked before invoking the action. The {\sf Enter}
button enters commands in the queue whilst the {\sf Remove} button removes them.
There is also a check on the total number of entries written at one time: if
more than 50 entries are requested (an arbitrary choice), the user is asked to
confirm the range of entries via a dialogue box (not shown).

\subsection{Queue Position}
The queue position can be specified using the {\sf Queue Position} radiobuttons.
Loosely, the oldest position is analagous to the `bottom' of the queue and the
newest position to the `top'. 

\subsection{List Queue}
The {\sf List Queue} action button, lists the contents of the data reduction
queue in the order in which Portable-CGS4DR will execute them. The output is written to the text pane.

\subsection{Remove All Entries}
The {\sf Remove All Entries} action button returns a dialogue box 
for confirmation of the deletion of the database. If {\sf OK}, the entire contents of the database
are deleted.

\subsection{Interrupt}
The {\sf Interrupt} action button uses widget grabs to seize control of the interface at
any particular instant. It would normally be used in the situation where a user
has inadvertently started writing a large number of entries to the queue. By clicking
on the {\sf Interrupt} button, the normal flow of {\sc tcladam} messages to the
qman task is interrupted thus preventing database overflowing and returning control to
the user.

\subsection{Qman Options}
The QMAN options menu contains the defaults plus specific options detailed below.

\subsection{Enter / Remove ...}
\subsubsection{End Group}
The widget tree labelled {\sf End Group} allows the user to enter a suitable
endgroup command into the data reduction queue. The {\sf Group Number} parameter
is checked before proceeding. The {\sf Enter} button enters the command whilst the
{\sf Remove} button removes it. Since Unix is, by default, `self-clobbering' this
functionality is largely redundant but has been retained for compatibility with the
VMS system. 

\subsubsection{End Range}
The widget tree labelled {\sf End Range} allows the user to enter a range of
END oyymmdd\_oooo commands into the queue for re-filing reduced observations.
The {\sf Observation Number} parameter is checked before proceeding. The {\sf Enter} 
button enters the command whilst the {\sf Remove} button removes it. 

\subsubsection{Integration Range}
The widget tree labelled {\sf Integration Range} allows the user to enter a range
of REDUCE iyymmdd\_oooo\_iiii commands into the queue.
The {\sf Observation Number, Start Integration, End Integration} parameters
is checked before proceeding. The {\sf Enter} button enters the command whilst the
{\sf Remove} button removes it. 

\subsubsection{String}
The widget tree labelled {\sf String} allows the user to enter a suitable
string into the data reduction queue. The {\sf String} parameter
is checked before proceeding and a radiobutton offers various string types. 
The {\sf Enter} button enters the command whilst the {\sf Remove} button removes it. 

\subsection{Set UT Date}
{\sf Set UT Date} returns a dialogue box with a single data entry field labelled {\sf UT Date} 
(which defaults to the present UT date). If {\sf OK}, the given date is used as the root
date for further data reduction commands.

\section{The Control Task Interface}
The CRED4 interface comprises mainly action buttons.
As before, error and informational messages are displayed in the text pane.
The various elements in the widget tree are described below.

For a full description of each action invoke on-line help using MB3 or view the hypertext web document.

\subsection{START / STOP}
The {\sf START} action button starts the
automated data reduction software. When invoked, this button is deleted and replaced
{\em at the same location} with a button labelled {\sf STOP}. This latter
button stops the automated data reduction software. 

\subsection{Pause}
The {\sf Pause} checkbutton toggles between pausing and resuming the
data reduction software. It does not stop the data reduction software.

\subsection{Status Label}
The data reduction status label shows the current status and is initialized to read 
{\sc stopped}. This is written on a red background. When the data reduction is started (whilst not paused), this
label changes to the string {\sc running} on a green background. If paused, the label reads {\sc paused} on an
orange background. 

\subsection{Setup}
The {\sf Setup} action button returns a dialogue box of radiobuttons that tailor the automated data
reduction sequence using the familiar {\sf Yes}, {\sf No} or {\sf Ask} switches. The action buttons within this dialogue
box are labelled {\sf OK}, {\sf Defaults} and {\sf Cancel}.
{\sf OK} and {\sf Cancel} have their usual meaning whilst {\sf Defaults} sets the database to the defaults contained
in the startup file. The defaults are not changed dynamically within the dialogue box but
are sent direct to the task noticeboard and the dialogue box disappears. To view these defaults,
therefore, the action needs to be re-invoked. The same is true for all other dialogue boxes.

\subsection{Display}
The {\sf Display} action button returns a dialogue box of
multiple-page radiobuttons. When first invoked, four radiobuttons are displayed with labels
{\sf Integrations}, {\sf Observations}, {\sf Groups} and {\sf Spectra}. Selecting any particular
value then displays a further page of radiobuttons with four choices per port: {\sf Yes}, 
{\sf No}, {\sf Ask} and {\sf Overgraph}.

If the user desires, for example, to plot images in port 1, a graph in port 2 and two overgraphs the
{\sf Observations} page would be tailored like so: click on {\sf Observations} at the page head to 
bring up the correct page. Then click on {\sf Yes} for ports 1 and 2 then
click on {\sf Overgraph} in port 3 and 4. If {\sf Overgraph} is selected a flip-out box appears
to take the appropriate command. Such text must be entered into the {\sf Command} field
manually and might be (for ports 3 and 4 for example):

\begin{verbatim}
  OVERGRAPH PORT=2 CUT=X SPOS=20 EPOS=40 COLOUR=RED
  OVERGRAPH PORT=2 CUT=X SPOS=25 EPOS=30 COLOUR=BLUE
\end{verbatim}

As before, the dialogue box action buttons include {\sf OK}, {\sf Defaults} and {\sf Cancel}.
Note here that the {\sf Defaults} button sets the defaults for {\em all four} types of display
and exits.

\subsection{Configs}
The {\sf Configs} action button returns a dialogue box that
allows the user to {\sf Save} to or {\sf Restore} from the specified configuration
file. 

\subsection{Masks}
The {\sf Masks} action button returns a dialogue box that
accepts inputs to change the bad pixel mask or linerisation file.

\subsection{Bias}
The {\sf Bias} action button returns a dialogue box that
accepts inputs to change various parameters associated with {\sc bias} reduction.

\subsection{Dark}
The {\sf Dark} action button returns a dialogue box that
accepts inputs to change various parameters associated with {\sc dark} reduction.

\subsection{Flat}
The {\sf Flat} action button returns a dialogue box that
accepts inputs to change various parameters associated with {\sc flat} reduction.

\subsection{Calibration}
The {\sf Calibration} action button returns a dialogue box that
accepts inputs to change various parameters associated with {\sc calib} reduction.

\subsection{Standard}
The {\sf Standard} action button returns a dialogue box that
accepts inputs to change various parameters associated with {\sc standard} reduction.

\subsection{Sky}
The {\sf Sky} action button returns a dialogue box that
accepts inputs to change various parameters associated with normal sky subtraction.

\subsection{Polysky}
The {\sf Polysky} action button returns a dialogue box that
accepts inputs to change various parameters associated with enhanced sky subtraction.

\subsection{Extract}
The {\sf Extract} action button returns a dialogue box that
accepts inputs to change various parameters associated with extracting a nodded spectrum.

\subsection{Cred4 Options}
The CRED4 options menu contains the defaults plus specific options detailed below.

\subsubsection{Peek Noticeboard}
Allows the user to peek at control task noticeboard values.

\subsubsection{Poke Noticeboard}
Allows the user to poke control task noticeboard values.

\subsubsection{Pause On Error}
Allows the user to tailor the behaviour when an error occurs.

\subsubsection{Load Demo \#1}
Loads the demo, if the software has been invoked correctly for this purpose.
To do so, use cgs4dr\_tcldemo and click on START when ready.

\subsubsection{Load Demo \#2}
Loads the demo, if the software has been invoked correctly for this purpose.
To do so, use cgs4dr\_tcldemo and click on START when ready.

\section{The Plotting Task Interface}
The P4 interface comprises mainly action buttons.
As before, error and informational messages are displayed in the text pane.
The various elements in the widget tree are described below. 

Loaded with the plotting task widget is a GWM display widget. This widget
has options for altering the colour values of the foreground, background,
overlay plane (if present), crosshair and individual pixel values. A further
widget provides hardcopy graphics. The {\sc clear} buttons clears the whole screen.
If you want to clear an individual port, you should use the `Send Task An Action'
menu item under options and enter `clear' for the {\sc action} and `port=1' (say)
for {\sc parameters} (sic).

For a full description of each action invoke on-line help using MB3 or view the hypertext web document.

\subsection{Data}
A dataset is specified by writing a string into the {\sf Data} field. Since
the plotting task can handle both {\sc ndf} and {\sc dst} data, any suitable
dataset can be plotted. Note that this entry widget has a binding to the plot
action upon receipt of a carriage return. That is, typing in a dataset name
and pressing {\sc return} invokes the plot action and the given dataset is
displayed.

\subsection{Port}
Ports are specified using bitmaps to represent portions of the screen 
where the data will be plotted. 
The {\sf Port} radiobuttons allow the user to toggle between different viewports.

\subsection{Plane}
The {\sf Plane} radiobuttons allow the user to specify which data plane is to be
plotted. If the requested plane does not exist, an error is returned to the text pane.

\subsection{Type}
The {\sf Type} radiobuttons allow the user to specify the type of plot to execute
in the given viewport. For {\sf Image} and {\sf Surface} plots not further information
is necessary. For the other types of plots a new flip-out widget is created {\em on the
fly} and prompts the user for start of cut, end of cut and so forth as required. Note that if
{\sf Graph} is selected, a pen colour {\em may} be selected but does not work --- that 
functionality is, at present, reserved for {\sf Overgraphs}.

\subsection{Autoscale}
The {\sf Autoscale} checkbutton allows the user to select specific scaling limits.
If `unchecked', a new flip-out widget is created {\em on the fly} to prompt the user for
high and low values. 

\subsection{Whole Array}
The {\sf Whole Array} checkbutton allows the user to select sub-areas of the array.
If `unchecked', a new flip-out widget is created {\em on the fly} to prompt the user for
(xstart,xend) and (ystart,yend) pairs.

\subsection{Plot}
The {\sf Plot} action button invokes the P4 display action to plot the given dataset
in the manner chosen. It is the most commonly used action.

\subsection{Cursor}
The {\sf Cursor} action button runs the P4 cursorval routine on the displayed dataset. 
When invoked control is passed to the GWM widget display surface and a click with
MB1 returns the X,Y co-ordinates of the point plus the data, error and quality array
values at that point.

\subsection{Set Port}
The {\sf Set Port} action button returns a dialogue box that
allows the user to tailor the display on any port and specifies a range of
options not included in the main widget trees such as device and axis attributes. If OK, the 
task noticeboard is altered to the values specified. That is to say, to reset the (autpomated)
plotting characteristics of a port, this button {\em must} be used. There is little effect
by just clicking on widgets in the p4 window.

See Chapter \ref{p4_chap} and tables
\ref{tab3}, \ref{tab7} for further information on these items.

\subsection{Configs}
The {\sf Configs} action button returns a dialogue box that
allows the user to {\sf Save} to or {\sf Restore} from the specified configuration
file. 

\subsection{P4 Options}
The P4 options menu contains the defaults plus specific options detailed below.

\subsubsection{Set Look-Up-Table}
This option invokes a listbox containing the CGS4DR colour tables. Double clicking
on any item using MB1 will load that colour table.

\subsubsection{Set Widget Colours}
This option invokes a dialogue box that allows the user to tailor the foreground,
background and overlay colours in the GWM widget.

\subsubsection{Peek Noticeboard}
Allows the user to peek at control task noticeboard values.

\subsubsection{Poke Noticeboard}
Allows the user to poke control task noticeboard values.

\subsection{Reset Autoplot}
If set (which it is by default), then the noticeboard is reset for automated plotting
after a manual plot is made to the same port. For example, if the automated reduction
is plotting images in port 1 and the user plots a graph in the same port, after the
graph is finished the noticeboard for port 1 is reset for imaging. If this toggle were
switched off, the noticeboard would not be reset and the next automated plot would would
also be made as a graph.

\chapter{Miscellaneous}
\markboth{Miscellaneous}{\stardocname}
\section{May I run the {\sc gui} More Than Once?}
No, although this is being investigated. You can run the command line version more than
once so long as it is loaded from a separate xterm.

\section{May I Run the Software in Batch?}
No. Unix does not recognize `batch' mode. The {\sc tcladam} interface, by default,
runs as a background job but the system is inherently interactive.

\section{Do I Still Need to Run the Kill Script?}
No. This was necessary under OSF/1 and an earlier version of ICL but no more. The
script has been retained, however, to start with a `clean' system when desired.

\section{How Do I Print Graphics?}
The {\sc gwm} widget is delivered with an action button labelled {\sf Print}. When
invoked, this returns a dialogue box tailoring the output to a variety of postscript
formats via an intermediate file. The file can be printed in the usual way.

\section{How Do I List the Noticeboard?}
Use the {\sf Peek Noticeboard} item in the {\sf Options} menu. This applies only to
the control and plotting tasks. The dialogue box returned defaults to the top-level
of either noticeboard. Thus selecting {\sf OK} with no further input, will list the
whole noticeboard.

\section{How Do I Reset the Tasks?}
\label{reset}
From time to time. it might be necessary to reset the tasks {\em e.g.} if you forget to
normalize a {\sc flat} field. This is done on a task-by-task basis via the {\sf Options}
menu. Click on the item {\sf Send Task an Action} and input the action name `reset'
into the data entry field. Then click on {\sf OK}.

\section{How Do I Check the Status of the Tasks?}
This is done on a task-by-task basis via the {\sf Options}
menu. Click on the item {\sf Send Task an Action} and input the action name `status'
into the data entry field. Then click on {\sf OK}.

\section{How Do I Read Old Index Files from VMS?}
Unfortunately, you can't. Under VMS the index file was maintained as an indexed sequential
access method (ISAM) file. These files were DEC Fortran compiler extensions and, as such,
were not portable between systems. 

\section{How Do I List the Available Masks and Colour Tables?}
Under Portable-CGS4DR both masks and colour tables are held in the same sub-directory.
Actually there are two sub-directories, one for {\sc dst} data and one for {\sc ndf} data.
The commands for listing {\sc ndf} data and {\sc dst} data from the shell are respectively:

\begin{quote}
 \% \ ls \ \$(CGS4DR\_ROOT)/ndf \\
 \% \ ls \ \$(CGS4DR\_ROOT)/dst
\end{quote}

Alternatively, if you are using the graphical user interface, you can select the following:

\begin{quote}
 {\sf List Masks} from the control task options menu for a list of bad pixel and window masks.\\
 {\sf Set Look--Up--Table} from the plotting task options menu for a list(box) of available colour tables.
\end{quote}

\section{How Do I Re-Normalize a {\sc flat} Field?}
If you forget to normalize a {\sc flat} field, the observation should be re-entered into the
queue at some appropriate point. The {\sc flat} field normalisation parameters should be
altered using the {\sf Reduction Params} action button from the control task interface.

Once the {\sc flat} field has been successfully re-reduced, the reduction task monolith,
RED4, {\em must} be reset to force the mapping of the new reduced observation file. To reset
any task, see section \ref{reset}.

\section{How Do I Start from Scratch?}
\label{restart}
If you encounter several problems and wish to revert to a known state, do the following:

\begin{quote}
  \% \ cgs4dr\_kill \\
  \% \ rm -f \ $\sim$/adam/* \\
  \% \ rm -f \ $\sim$/cgs4dr\_configs/*
\end{quote}

%%-----------------------------------------------------------------------------
%  References 
\chapter*{References}
\markboth{References}{\stardocname}

\begin{description}
\item[] Mountain, C. M., Robertson, D. J., Lee, T. J., \& Wade, R. 1990,
      in Instrumentation in Astronomy VII, Proc. SPIE, Vol. 1235, ed. David L. Crawford, p25--33.
\item[] Ramsay, S. K., 1993, PhD Thesis, University of Edinburgh.
\item[] Carswell, R. F. et al. 1991, Astrophysical Journal, Vol. 381, L5--8.
\item[] Wright, G. S. et al. 1993, in Infrared Detectors and Instrumentation, Proc. SPIE,
      Vol. 1946, ed. Albert M. Fowler, p547--557.
\item[] Shortridge, K. 1993, in  Astronomical Data Analysis Software
      and Systems II, A. S. P. Conf. Ser., Vol. 52, eds. R. J. Hanisch, 
      R. J. V. Brissenden \& J. Barnes, p219-223.
\item[] Lawden, M. D., \& Hartley, K. F. 1992, SG/4, Starlink Project, CCL.
\item[] Daly, P. N. 1987, MSc Thesis, University of Wales.
\item[] Puxley, P. J. 1988, PhD Thesis, University of Edinburgh.
\item[] Bailey, J. A. 1990, UON/8, Joint Astronomy Centre.
\item[] Bailey, J. A. 1991, UON/9, Joint Astronomy Centre.
\item[] Bevington, P. R. 1969, `Data Reduction and Error Analysis for the Physical Sciences',
      ISBN 0--07--005135--6, McGraw Hill Inc., NY.
\item[] Mountain, C. M. et al. 1985, Astronomy \& Astrophysics, Vol. 151, Number 2, p339--402.
\item[] Daly, P. N. 1995, in Astronomical Data Analysis Software and Systems IV, 
      A. S. P. Conf. Ser., Vol. 77, eds. R. A. Shaw, H. E. Payne \& J. J. E. Hayes, p375-378.
\item[] Ousterhout, J. K. 1994, `Tcl and the Tk Toolkit', ISBN 0--201--63337--X, Addison Wesley.
\item[] Terrett, D. L. 1995, SUN/186, Starlink Project, CCL.
\end{description}
\end{document}
