\documentstyle[11pt]{article}
\pagestyle{myheadings}

% -----------------------------------------------------------------------------
% ? Document identification
\newcommand{\stardoccategory}  {Starlink User Note}
\newcommand{\stardocinitials}  {SUN}
\newcommand{\stardocsource}    {sun208.1}
\newcommand{\stardocnumber}    {208.1}
\newcommand{\stardocauthors}   {P.\, McGale, Richard West}
\newcommand{\stardocdate}      {18 March 1996}
\newcommand{\stardoctitle}     {REPACK \\[1.5ex]
                           Handle {\sl ROSAT} Wide Field Camera survey data}
\newcommand{\stardocversion}   {Version 1.0}
\newcommand{\stardocmanual}    {User's Manual}
% ? End of document identification
% -----------------------------------------------------------------------------

\newcommand{\stardocname}{\stardocinitials /\stardocnumber}
\markright{\stardocname}
\setlength{\textwidth}{160mm}
\setlength{\textheight}{230mm}
\setlength{\topmargin}{-2mm}
\setlength{\oddsidemargin}{0mm}
\setlength{\evensidemargin}{0mm}
\setlength{\parindent}{0mm}
\setlength{\parskip}{\medskipamount}
\setlength{\unitlength}{1mm}

% -----------------------------------------------------------------------------
%  Hypertext definitions.
%  ======================
%  These are used by the LaTeX2HTML translator in conjunction with star2html.

%  Comment.sty: version 2.0, 19 June 1992
%  Selectively in/exclude pieces of text.
%
%  Author
%    Victor Eijkhout                                      <eijkhout@cs.utk.edu>
%    Department of Computer Science
%    University Tennessee at Knoxville
%    104 Ayres Hall
%    Knoxville, TN 37996
%    USA

%  Do not remove the %\begin{rawtex} and %\end{rawtex} lines (used by
%  star2html to signify raw TeX that latex2html cannot process).
%\begin{rawtex}
\makeatletter
\def\makeinnocent#1{\catcode`#1=12 }
\def\csarg#1#2{\expandafter#1\csname#2\endcsname}

\def\ThrowAwayComment#1{\begingroup
    \def\CurrentComment{#1}%
    \let\do\makeinnocent \dospecials
    \makeinnocent\^^L% and whatever other special cases
    \endlinechar`\^^M \catcode`\^^M=12 \xComment}
{\catcode`\^^M=12 \endlinechar=-1 %
 \gdef\xComment#1^^M{\def\test{#1}
      \csarg\ifx{PlainEnd\CurrentComment Test}\test
          \let\html@next\endgroup
      \else \csarg\ifx{LaLaEnd\CurrentComment Test}\test
            \edef\html@next{\endgroup\noexpand\end{\CurrentComment}}
      \else \let\html@next\xComment
      \fi \fi \html@next}
}
\makeatother

\def\includecomment
 #1{\expandafter\def\csname#1\endcsname{}%
    \expandafter\def\csname end#1\endcsname{}}
\def\excludecomment
 #1{\expandafter\def\csname#1\endcsname{\ThrowAwayComment{#1}}%
    {\escapechar=-1\relax
     \csarg\xdef{PlainEnd#1Test}{\string\\end#1}%
     \csarg\xdef{LaLaEnd#1Test}{\string\\end\string\{#1\string\}}%
    }}

%  Define environments that ignore their contents.
\excludecomment{comment}
\excludecomment{rawhtml}
\excludecomment{htmlonly}
%\end{rawtex}

%  Hypertext commands etc. This is a condensed version of the html.sty
%  file supplied with LaTeX2HTML by: Nikos Drakos <nikos@cbl.leeds.ac.uk> &
%  Jelle van Zeijl <jvzeijl@isou17.estec.esa.nl>. The LaTeX2HTML documentation
%  should be consulted about all commands (and the environments defined above)
%  except \xref and \xlabel which are Starlink specific.

\newcommand{\htmladdnormallinkfoot}[2]{#1\footnote{#2}}
\newcommand{\htmladdnormallink}[2]{#1}
\newcommand{\htmladdimg}[1]{}
\newenvironment{latexonly}{}{}
\newcommand{\hyperref}[4]{#2\ref{#4}#3}
\newcommand{\htmlref}[2]{#1}
\newcommand{\htmlimage}[1]{}
\newcommand{\htmladdtonavigation}[1]{}

%  Starlink cross-references and labels.
\newcommand{\xref}[3]{#1}
\newcommand{\xlabel}[1]{}

%  LaTeX2HTML symbol.
\newcommand{\latextohtml}{{\bf LaTeX}{2}{\tt{HTML}}}

%  Define command to re-centre underscore for Latex and leave as normal
%  for HTML (severe problems with \_ in tabbing environments and \_\_
%  generally otherwise).
\newcommand{\latex}[1]{#1}
\newcommand{\setunderscore}{\renewcommand{\_}{{\tt\symbol{95}}}}
\latex{\setunderscore}

%  Redefine the \tableofcontents command. This procrastination is necessary
%  to stop the automatic creation of a second table of contents page
%  by latex2html.
\newcommand{\latexonlytoc}[0]{\tableofcontents}

% -----------------------------------------------------------------------------
%  Debugging.
%  =========
%  Remove % on the following to debug links in the HTML version using Latex.

% \newcommand{\hotlink}[2]{\fbox{\begin{tabular}[t]{@{}c@{}}#1\\\hline{\footnotesize #2}\end{tabular}}}
% \renewcommand{\htmladdnormallinkfoot}[2]{\hotlink{#1}{#2}}
% \renewcommand{\htmladdnormallink}[2]{\hotlink{#1}{#2}}
% \renewcommand{\hyperref}[4]{\hotlink{#1}{\S\ref{#4}}}
% \renewcommand{\htmlref}[2]{\hotlink{#1}{\S\ref{#2}}}
% \renewcommand{\xref}[3]{\hotlink{#1}{#2 -- #3}}
% -----------------------------------------------------------------------------
% ? Document specific \newcommand or \newenvironment commands.

\setcounter{tocdepth}{2}
\newcommand{\etal}{et~al.\ }
\newcommand{\ro}{{\sl ROSAT }}

% ? End of document specific commands
% -----------------------------------------------------------------------------
%  Title Page.
%  ===========
\renewcommand{\thepage}{\roman{page}}
\begin{document}
\thispagestyle{empty}

%  Latex document header.
%  ======================
\begin{latexonly}
   CCLRC / {\sc Rutherford Appleton Laboratory} \hfill {\bf \stardocname}\\
   {\large Particle Physics \& Astronomy Research Council}\\
   {\large Starlink Project\\}
   {\large \stardoccategory\ \stardocnumber}
   \begin{flushright}
   \stardocauthors\\
   \stardocdate
   \end{flushright}
   \vspace{-4mm}
   \rule{\textwidth}{0.5mm}
   \vspace{5mm}
   \begin{center}
   {\Huge\bf  \stardoctitle \\ [2.5ex]}
   {\LARGE\bf \stardocversion \\ [4ex]}
   {\Huge\bf  \stardocmanual}
   \end{center}
   \vspace{5mm}

% ? Heading for abstract if used.
   \vspace{10mm}
   \begin{center}
      {\Large\bf Abstract}
   \end{center}
% ? End of heading for abstract.
\end{latexonly}

%  HTML documentation header.
%  ==========================
\begin{htmlonly}
   \xlabel{}
   \begin{rawhtml} <H1> \end{rawhtml}
      \stardoctitle\\
      \stardocversion\\
      \stardocmanual
   \begin{rawhtml} </H1> \end{rawhtml}

% ? Add picture here if required.
% ? End of picture

   \begin{rawhtml} <P> <I> \end{rawhtml}
   \stardoccategory \stardocnumber \\
   \stardocauthors \\
   \stardocdate
   \begin{rawhtml} </I> </P> <H3> \end{rawhtml}
      \htmladdnormallink{CCLRC}{http://www.cclrc.ac.uk} /
      \htmladdnormallink{Rutherford Appleton Laboratory}
                        {http://www.cclrc.ac.uk/ral} \\
      \htmladdnormallink{Particle Physics \& Astronomy Research Council}
                        {http://www.pparc.ac.uk} \\
   \begin{rawhtml} </H3> <H2> \end{rawhtml}
      \htmladdnormallink{Starlink Project}{http://www.starlink.ac.uk/}
   \begin{rawhtml} </H2> \end{rawhtml}
   \htmladdnormallink{\htmladdimg{source.gif} Retrieve hardcopy}
      {http://www.starlink.ac.uk/cgi-bin/hcserver?\stardocsource}\\

%  HTML document table of contents.
%  ================================
%  Add table of contents header and a navigation button to return to this
%  point in the document (this should always go before the abstract \section).
  \label{stardoccontents}
  \begin{rawhtml}
    <HR>
    <H2>Contents</H2>
  \end{rawhtml}
  \renewcommand{\latexonlytoc}[0]{}
  \htmladdtonavigation{\htmlref{\htmladdimg{contents_motif.gif}}
        {stardoccontents}}

% ? New section for abstract if used.
  \section{\xlabel{abstract}Abstract}
% ? End of new section for abstract
\end{htmlonly}

% -----------------------------------------------------------------------------
% ? Document Abstract. (if used)
%   ==================

During 1990 Jul to 1991 Jan, the \ro Wide Field Camera (WFC) performed
the first all--sky survey at extreme--ultraviolet (EUV) wavelengths.
The whole sky, or $\approx 96$\% of it, was imaged in two passbands, S1
and S2, covering the ranges 60--140\AA~and 110--200\AA~respectively.
An initial bright source catalogue (BSC) of 383 sources was produced by
Pounds \etal (1993, MNRAS, 260, 77).  A new list, the 2RE~Catalogue
(Pye \etal 1995, MNRAS, 274, 1165), of 479 sources has recently been
published.  The survey data \emph{i.e.}, images and `raw' photon event files
are now in the public domain.  The REPACK software package has been
developed to fully exploit these data.  This document describes the
REPACK programs.

% ? End of document abstract
% -----------------------------------------------------------------------------
% ? Latex document Table of Contents (if used).
%  ===========================================
 \newpage
 \begin{latexonly}
   \setlength{\parskip}{0mm}
   \latexonlytoc
   \setlength{\parskip}{\medskipamount}
   \markright{\stardocname}
 \end{latexonly}
% ? End of Latex document table of contents
% -----------------------------------------------------------------------------
\newpage
\renewcommand{\thepage}{\arabic{page}}
\setcounter{page}{1}

\section{\label{se_introduction}\xlabel{introduction}Introduction}

The survey data, as produced for the 2RE processing, is available to
the public through the Leicester Data Archive Service (LEDAS). Images
were sorted from the event files and screened for `high background' and
'moon in WFC field-of-view' and named after the ecliptic latitude and
longitude of the region of sky that they covered \emph{e.g.}, {\tt
im012014\_f1} and {\tt im012014\_f2} --- one for each survey passband.
They are $\approx 2.7^{\circ} \times 2.7^{\circ}$ in extent with a
resolution of $1 \times 1$ arcmin$^{2}$ per pixel.  The photon event
files are overlaid on a similar ecliptic grid system.  Some 13~000
image pairs and 13~000 event files were produced by this scheme,
covering almost the whole sky.

Images can be retrieved in various formats; HDS, FITS and GIF. The
REPACK software only operates on HDS format images. Event files,
available just in FITS format, can be sorted to ASTERIX (HDS) datasets
such as images and light-curves.

REPACK is supported under UNIX (Digital Unix, Solaris 2).

\section{\label{se_retrieving_images}\xlabel{retrieving_images}Retrieving Images and Photon Event Files}

The recommended  way of retrieving [\ro WFC survey] images and photon event
files is to first go to the LEDAS page on the World Wide Web via the URL:

\begin{quote}{\tt
\htmladdnormallink{http://ledas-www.star.le.ac.uk/}%
{http://ledas-www.star.le.ac.uk/}}
\end{quote}

The database tables held by the LEDAS can be accessed either via the
BROWSE database system, or via the WWW-based ARNIE interface. Users
should endeavour to locate the \ro WFC survey tables 1) ROSWFCSI for
the images, 2) ROSWFCSE for the raw photon event files, and 3)
ROSWFC2RE for the 2RE source list.  If the BROWSE route is taken,
supplying some search cone and then using the {\tt xp} command, image
pairs (one per waveband) can be selected and retrieved.  A similar
technique can be used to obtain photon event files.  If extraction
completes successfully users should find image and event files, of the
form {\tt im{\it nnnnnn}\_f{\it n}.sdf} and/or {\tt x{\it nnn}y{\it
nn}.fit}, in their current directory.

The LEDAS also provides easy-to-use access to the WFC survey image
database via the WFC Survey Image Browser:

\begin{quote}{\tt
\htmladdnormallink{http://ledas-www.star.le.ac.uk/WFCimage}%
{http://ledas-www.star.le.ac.uk/WFCimage}}
\end{quote}

\section{\label{se_repack_programs}\xlabel{repack_programs}The REPACK Programs}

Having obtained some \ro WFC survey data, it would be useful to have some
software to help analyse it.  The REPACK package aims to fulfill this
task.  Some of its features, such as locating images \emph{etc.}, are
adequately covered by the LEDAS-BROWSE interface (see
Section~\ref{se_retrieving_images}), but it is nonetheless important to
have a self consistent set of programs.

All REPACK programs have a {\tt `re'} prefix.  Some of them have been
wrapped up into {\tt csh} scripts, described in
Section~\ref{se_shell_scripts}, to minimise user interaction for some
of the more common operations.  The REPACK programs, presented in the
order in which they would normally be used in a data analysis session,
are as follows:

\begin{verbatim}

     re_imsrch     Names image whose centre is closest to input co-ordinates.
     re_evsrch     Names event files touched by input search cone.
     re_fitmrg     Merges a set of (FITS) event files.
     re_expos      Returns survey exposure, in seconds, at input co-ordinates.
     re_sort       Sorts (FITS) event files to images, time-series etc.
     re_imexp      Exposure corrects an (HDS) image.
     re_timexp     Exposure corrects a time-series.

\end{verbatim}

If REPACK has been properly installed at the users site then all of the
above programs will be accessible after the startup commands have been
executed:

\begin{verbatim}
    % aststart
    % re_start
\end{verbatim}

All programs are started by simply typing their name on the command line.
On-line parameters can also be specified, \emph{e.g.}:

\begin{verbatim}
    % re_sort evefil=mrgd_ev \\
\end{verbatim}

Where possible some of the program parameters supply the user with a
sensible default value.  The next section describes each of the
programs and their parameters.  Some hints are also given as to
parameter values.

\subsection{\label{ss_re_imsrch}\xlabel{re_imsrch}re\_imsrch}

This program is useful for locating the image that surveyed the
position of some source.  It returns the name of the image whose centre
is closest to the specified RA \& Dec.  The total number of counts in
each filter passband is also shown as is the RA \& Dec of the image
centre and the separation between the desired and present positions.

A similar feature is also available in BROWSE using the ROSWFCSI table
(see Section \ref{se_retrieving_images}).  Further, the appropriate
image can be retrieved.

\subsubsection{Parameters}

\begin{description}
\item[GET (Y/N)  ]
Yes if RA and DEC co-ordinates are to be input from a file.
No if from the command line.
\item[FILENAME  ] file containing one or more co-ordinate pairs.
\item[SRA  ]
The RA co-ordinate to be searched.  It may be specified in
decimal or sexagesimal degrees.
\item[SDEC  ]
The Dec co-ordinate to be searched.  It may be specified in
decimal or sexagesimal degrees.
\item[WRITE (Y/N)  ]
Yes to Write the results into a file.
\item[OUTPUT  ]
File to contain results.
\end{description}

\subsection{\label{ss_re_evsrch}\xlabel{re_evsrch}re\_evsrch}

This program is useful for locating the event files required to form an
image centred on some location.  Given a search cone it returns the
name(s) of the survey event files wholly or partially enclosed.  For
convenience these names are written to the file {\tt re\_evmrg.list}
which can be used as input to the event file merging program
\htmlref{{\tt re\_fitmrg}}{ss_re_fitmrg} (see Section~\ref{ss_re_fitmrg}).

A similar feature is also available in BROWSE using the ROSWFCSE table.
Further, the appropriate event files can be retrieved.

\subsubsection{Parameters}

\begin{description}

\item[SRA  ]
The RA of the search cone.  It may be specified in
decimal or sexagesimal degrees.
\item[SDEC  ]
The Dec of the search cone.  It may be specified in
decimal or sexagesimal degrees.
\item[RAD  ]
The radius of the search cone.  It may be specified in only decimal
degrees. A typical radius, if a time-series is to be the sorted product
for example, may be $1^{\circ}$.
\end{description}

\subsection{\label{ss_re_fitmrg}\xlabel{re_fitmrg}re\_fitmrg}

Event files need to merged before they can be sorted into images
\emph{etc.}, if it is shown by \htmlref{{\tt re\_evsrch}}{ss_re_evsrch}
that several event files are needed to map some chosen region of sky.
The program makes a copy of the first input event file and them appends
the rest updating appropriate keywords along the way.  Certain file
consistency checks such as on data format revision numbers, are also
performed.

The {\tt csh} script file \htmlref{{\tt re\_fitmrgsh}}{se_shell_scripts}
makes merging even easier --- see Section~\ref{se_shell_scripts}.

\subsubsection{Parameters}

\begin{description}

\item[LIST  ]
The list of event files to be merged.  A text file containing file
names can be used as input.  If {\tt TERMINAL} is the supplied response
then the names can be entered from the keyboard and the list terminated
with a {\tt Ctrl-D}.  The default file extension is {\tt .fit} and need
not be supplied by the user.
\item[OUT  ]
The name to be given to the new merged event file.  The default extension is
{\tt .fit}.
\item[DELFIL ]
If an output file of the same name already exists the option is given of
either overwriting it or aborting the program.
\end{description}

\subsection{\label{ss_re_expos}\xlabel{re_expos}re\_expos}

Using the \ro WFC survey sky exposure images {\tt reskyexp\_f1.fit} and
{\tt reskyexp\_f2.fit}, this program exposure corrects a source search
dataset (SSDS) produced from the ASTERIX point source searching program
PSS, or gives the exposure, in seconds, at locations specified by the
user. Both instrument vignetting and dead-time are accounted for.

\subsubsection{Parameters}

\begin{description}

\item[FILT  ]
Survey filter, S1a or S2a, for which exposure is required.  Each passband has
its own exposure map ({\tt reskyexp\_f1.fit} for S1 and {\tt reskyexp\_f2.fit}
for S2).
\item[EFFCOR  ]
 Allows a correction for degradation of detector efficiency.  Can be used
to give `at launch' values.  Count rates given in the 2RE catalogue have
had this correction applied.
\item[SLIST  ]
Name of the PSS SSDS file (default extension {\tt .sdf}).  A text file of
decimal RAs and Decs, comma   separated, can also be supplied.  If this input
file has the name {\tt posns.dat} for example, the exposure information will
be placed  in a file called {\tt posns.exp}.  If {\tt TERMINAL} is the supplied
response then RAs and Decs, in decimals, can be input directly from the
keyboard. The program then returns exposure times when it receives
a {\tt Ctrl-D}.
\item[CLIST   ]
 If a SSDS file is being processed, then an output file containing the
 newly corrected source counts can be specified.  A name is suggested by
the program.  The default extension is {\tt .sdf}.
\end{description}

\subsection{\label{ss_re_imexp}\xlabel{re_imexp}re\_imexp}

Using the sky survey exposure maps {\tt reskyexp\_f1.fit} and {\tt
reskyexp\_f2.fit}, and WFC configuration file {\tt re\_slots.fit},
this program exposure corrects a (HDS) survey image.  It automatically
figures out which passband the image is in (S1 or S2).  Instrument
vignetting and dead-time are allowed for.

\subsubsection{Parameters}

\begin{description}
\item[INP   ]
Name of (HDS) image to be exposure corrected.  The default file extension
is {\tt .sdf} and need not be supplied by the user.
\item[EFFCOR  ]
Allows a correction for degradation of detector efficiency.  Can be used
to give `at launch' values.  Count rates given in the 2RE catalogue have
had this correction applied.
\item[OUT   ]
Name of (HDS) corrected image.  An output name is suggested by the
program. The default file extension is {\tt .sdf}.
\end{description}

\subsection{\label{ss_re_sort}\xlabel{re_sort}re\_sort}

This program `sorts' (FITS) survey event files into ASTERIX (HDS)
datasets such as images and time-series.  If events from a range of
files are too be sorted to the same output file then they should first
be merged using \htmlref{{\tt re\_fitmrg}}{ss_re_fitmrg} (see
Section~\ref{ss_re_fitmrg}).

\subsubsection{Parameters}

\begin{description}
\item[EVEFIL   ]
Name of (FITS) event file to be sorted.  The default file extension is
{\tt .fit} and need not be supplied by the user.
\item[SHOW   ]
If {\tt Y} is the reply to the prompt, then some summary information on the
event file being sorted is displayed.
\item[DTYPE   ]
 Type of ASTERIX (HDS) dataset to be produced.  Options are:

\begin{verbatim}
     (I)mage   to create an ASTERIX sky coord image dataset.
     (L)in     to create a linearised detector coord image.
     (T)ime    to create a time series dataset.
     (E)vent   to create an event dataset.
\end{verbatim}

Uses of images and time-series are self evident.  Event sets can be
useful for accurate timing work.  Linear images are for specialist
technical/instrument investigations.
\item[RA]
The Right Ascension of the sort field centre. This can be entered as
decimal or sexagesimal degrees.
\item[DEC]
The Declination of the sort field centre. This can be entered as
decimal or sexagesimal degrees.
\item[DAZ]
The half azimuthal extent of the field to be sorted (degs).  For example the
survey images in the BROWSE table ROSWFCSI were created with DAZ \& DEL set to
80/60 and NXPIX \& NYPIX set to 160.  These values give images of dimension
$160 \times 160$ arcminutes with a resolution of 1 pixel per arcmin$^{2}$.
\item[DEL]
The half elevation extent of the field to be sorted (degs).
\item[NXPIX]
The number of pixels along azimuthal side of the image.
\item[NYPIX]
The number of pixels along elevation side of the image.
\item[INRAD]
In time series mode INRAD specifies the inner radius of an annulus
from which to select events. The radius should be specified in arcminutes
and may be set to zero in order to select from a circular region.
\item[OUTRAD]
In time series mode OUTRAD specifies the outer radius of an annulus
from which to select events. The radius should be specified in
arcminutes.  Typical values are 3 arcminutes for sources and 9 arcminutes
for local background estimates if INRAD is set to zero.
\item[BREJ ]
A background mask was created during the survey processing to
maximise signal-to-noise for images.  It is recommended to use this
option -- it needn't be used for time-series.
\item[MREJ ]
During the survey, the moon would occasionally appear in the WFC
filed-of-view.  This option allows the user to reject such lunar events.
\item[FILTER]
A text string definition of the filter to use. This is selected from
the second column of the following table

\begin{verbatim}
      MCF #        Filter        Material

      6             S2A          Le/Be
      8             S1A          Le/C/B4C

      (MCF = Master Calibration File)
\end{verbatim}

\item[TLO  ]
The time from which to start event sorting.   The units are absolute
MJD.  (When performing background subtraction of two concentric
time-series it is important that the two time-series be co-binned and
so TLO should be chosen to be the same for both. It is recommended to
use the {\tt csh} script \htmlref{{\tt re\_light}}{se_shell_scripts}, see
Section~\ref{se_shell_scripts}, to perform this task).
\item[THI ]
The time to end event sorting.   The units are absolute MJD.
\item[NBINS]
The number of bins into which the time extent of the sort
is to be divided. If a value '0' is given the program prompts for the
time duration for each bin (see TBIN).
\item[TBIN]
If NBIN is specified as 0 then TBIN should be set to the required
time duration of a bin (in seconds).  It is common to bin survey time-series on
the \ro spacecraft orbit of 5760 seconds --- a source could only be
observed for a maximum of $\sim 80$secs once per orbit.
\item[IRAD]
This gives the option of `stopping down' the radius of the WFC
field-of-view.  The units are degrees, the default is the full field
($\approx 5^{\circ}$ in diameter).
\item[OUTPUT]
The name of the output dataset. The default extension for the file is
{\tt .sdf } and need not be supplied by the user.
\end{description}

\subsection{\label{ss_re_timexp}\xlabel{re_timexp}re\_timexp}

This program exposure corrects a survey (HDS) time-series for
vignetting, dead-time, and point spread function as well as raw
exposure.  Essentially `on-axis' count rates are returned.   It expects
to find two files that contain information on spacecraft attitude and
instrument house-keeping rates ({\tt re\_ateres.fit} and {\tt
re\_hkrres.fit} respectively).

\subsubsection{Parameters}

\begin{description}
\item[INP   ]
Name of (HDS) time-series to be exposure corrected. The default extension is
{\tt .sdf } and need not be supplied by the user.
\item[SRC   ]
If the input time-series is for a source, rather than a background
region, then answer {\tt Y} so that a point-spread-function correction
can be applied.
\item[OUT   ]
Name of (HDS) corrected time-series. The program suggests a file name.
The default extension is {\tt .sdf}. Note that the {\tt csh} script
\htmlref{{\tt re\_light}}{se_shell_scripts}, see
Section~\ref{se_shell_scripts}, produces an exposure profile of the
corrected time-series which is useful for masking out time-bins below a
certain threshold.  Light-curves for 2RE Catalogue sources did not use
data points calculated from exposures of less than 30 seconds (see
McGale \etal, MNRAS, in press).
\end{description}

\section{\label{se_shell_scripts}\xlabel{shell_scripts}{\tt csh} Scripts}

Some of the REPACK programs have been incorporated into {\tt csh}
scripts to perform some typical procedures and so minimise user interaction.
These scripts are listed in the table below.

\begin{verbatim}
   Name        Description
   ----------------------------------------------------------------
   re_light    Produces exposure correct background subtracted time-series.
   re_spec     Produces exposure correct background subtracted spectra.
   re_fitmrgsh Merge all event files, or a selection, in the current directory.
   re_menu     Menu driven ROSAT WFC survey data analysis.
\end{verbatim}

\section{\label{se_ancilliary_files}\xlabel{ancilliary_files}Ancillary Survey Files}

In order for the REPACK software to function properly,  access to various
reservoir files is required.  These are listed in the table below.

\begin{verbatim}
File             Typ*   Description
---------------- ---    --------------------------------------------------
reskyexp_f1.fit   F     Sky exposure map for survey passband S1.
reskyexp_f2.fit   F     Sky exposure map for survey passband S2.
reexp_eff.fit     F     Sky maps to allow for change in dectector efficiency.
re_slots.fit      F     WFC configuration during survey.
re_ateres.fit     F     WFC attitude solution during survey.
re_hkrres.fit     F     WFC housekeeping rates during survey.
re_psf_f1.sdf     H     Survey point spread function for passband S1.
re_psf_f2.sdf     H     Survey point spread function for passband S2.
re_imge.dat       A     Survey image list.

* F=FITS, H=HDS, A=ASCII
\end{verbatim}

The software startup procedure should define an environment variable
{\tt re\_res} that points to the location of these files.  ASTERIX
should also have been installed at the users site if the REPACK package
is to be used --- this gives access to the \ro WFC master calibration
file.  The user is never asked by the REPACK programs to supply any of
the above filenames as they work out for themselves which ones are
required.

\section{\label{se_contact_points}\xlabel{contact_points}Points of Contact}

The REPACK package was based on software written mostly by Mike Denby
to analyse \ro WFC survey data on VAXes.  The present software was
written by Paul McGale.  Points of contact are John Pye ({\tt
pye@star.le.ac.uk}) and Jeremy Ashley ({\tt jka@star.le.ac.uk}).

\newpage

\section{\label{se_example_session}\xlabel{example_session}Example Session}

The following example shows how many of the REPACK programs and
procedures may be used.  It assumes that the user has already retrieved
survey event files from LEDAS (see Section~\ref{se_retrieving_images}) .

\begin{small}
\begin{verbatim}
% aststart                                        ! Startup ASTERIX and REPACK
% re_start

% re_evsrch                                     ! Names of relevant event files

    RE_EVSRCH, version 010595

SRA search RA (deg/hms) <REAL>=8.2
SDEC search Dec (deg/dms) <REAL>=-62.8
RAD  radius (deg) <REAL>=2.5

  # event files partially or totally enclosed:     9

  List:
  x174y16
  x175y16
  x176y16
  x174y17
  x175y17
  x176y17
  x174y18
  x175y18
  x176y18

List also stored in file re_evmrg.list.

% re_fitmrgsh                                      ! Merge present event files

  RE_FITMRGSH -- Version 010595

    Merging all x*y*.fit files in current directory.

ls: No match.

 RE_FITMRG Version 010595

    Merging            9 event files.

    Merged file is mrgd_ev.fit.

% ls                                                         ! See what we have
mrgd_ev.fit     x174y16.fit     x175y16.fit     x176y16.fit
re_evmrg.list   x174y17.fit     x175y17.fit     x176y17.fit
x174y18.fit     x175y18.fit     x176y18.fit

% re_sort                                                    ! Sort to an image

 RE_SORT Version 110595

EVEFIL Event File <CHAR>=mrgd_ev
SHOW header records = No =y

 Mode                    : S
 MDR Sequence #          : mrgd_ev
 # Azim maps             :   0
 # Elev maps             :  0
 Asc Node Ra             :    0.0
 Asc Node Dec            :    0.0
 Target                  : SURVEY
 Observer                : ROS-UKSC
 Instrument              : WFC
 Ref MJD                 : 47892.0000
 Base date               :
 Base MJD                : 48198.1993
 End MJD                 : 48212.3455
 Total Events            :    63722
 Active maps             :     9
 File Creation           : 27/04/95
 File Revision           : 1

DTYPE I,T,E or L <CHAR>=i
RA of Field centre (deg/hms) <REAL>=8.2
DEC of Field centre (deg/hms) <REAL>=-62.8
DAZ Field 1/2 width (degs) <REAL>=1.333333
DEL Field 1/2 Height (degs) <REAL>=1.333333
NXPIX Azimuth Pixels <INTEGER>=160
NYPIX Elevation Pixels <INTEGER>=160
BREJ Background rejection = Yes =
MREJ Moon in field rejection = Yes =
FILTER eg S1A <CHAR>=s1
    BASE_MJD is    48198.1993359375
TLO Start MJD = 48198.199335937 =
THI End MJD = 48212.345531322 =
IRAD Iris Radius (degs) = 2.378898 =
OUTPUT dataset <CHAR>=im_s1
    Events in image :         9000

% icl                                                    ! Display image in ICL

ICL (UNIX) Version 3.0 27/03/95

ICL> asterix

----------------------------------------------------------------------
 ASTERIX Version 1.7-0           - type ASTHELP for more information

ICL> iload im_s1 xw GCB=no
ILOAD Version 1.7-3
ICL> idisp
IDISPLAY Version 1.7-0
ICL> exit


% bsub                                    ! Create a background model for image
BSUB Version 1.3-18
INP - Input file /@twd_2_sp/ > im_s1
OUT - Background subtracted image > imb
BGND - Output background model dataset /!/ > bg_s1
PSF system options :

  ANAL           ASCA           EXOLE          PWFC           RADIAL
  RESPFILE       TABULAR        WFC            XRT_HRI        XRT_PSPC

  CSPEC(psf/model,spec[,nbin])
  POLAR(psf,rbin[,abin])        RECT(psf,xbin[,ybin])

PSF - Choose PSF to use for search /'WFC'/ > tabular
MASK - Name of a 2D dataset containing psf > $re_res/re_psf_f1
PSF read in from file $re_res/re_psf_f1

68% radius of PSF is    3.4 pixels
80% radius of PSF is    5.5 pixels
90% radius of PSF is    9.6 pixels
95% radius of PSF is   14.9 pixels

Image is 160 x 160

SURVEY image

SOURCE_THRESH - Source detection threshold /4/ >

BOX_DIM - Trial small box size /11/ >

 NON-ZERO image limits are    1 :  160 (in X) and    1 :  160 (in Y)
SLOPING_EDGES - Identify sloping edges in data /NO/ > yes

 Detected    1 sources
 source at    99.71    43.78 : Approx cts =      428.1

SM_FILT - Choose Gaussian (G) or top hat (H) smoothing function /'H'/ > g
SM_FWHM - Enter FWHM (pixels) for smoothing function /15/ >

 For background subtracted image, minimum=    -0.45 : maximum=    38.65 cts
 Mean of image=     0.335  Mean of smoothed image =     0.327
 Mean difference over image=   0.0078 : RMS=   0.5770
 Worst % deviation (data-smooth) in a box is   -88.5 % in box 212
 Average of box (data-smooth) variations =    0.34 % RMS =   16.8 %

% icl                                                ! Look at background model

ICL> asterix
ICL> iload bg_s1 xw GCB=NO
ILOAD Version 1.7-3
ICL> idisp
IDISPLAY Version 1.7-0
ICL> exit

% pss                                                     ! Source search image
PSS Version 1.7-5
INP - Dataset to be searched/tested for sources /@im_s1/ >
EXPERT - Expert mode /TRUE/ >
PSF system options :

  ANAL           ASCA           EXOLE          PWFC           RADIAL
  RESPFILE       TABULAR        WFC            XRT_HRI        XRT_PSPC

  CSPEC(psf/model,spec[,nbin])
  POLAR(psf,rbin[,abin])        RECT(psf,xbin[,ybin])

PSF - Choose PSF to use for source model /'WFC'/ > tabular
MASK - Name of a 2D dataset containing psf >  $re_res/re_psf_f1
PSF read in from file  $re_res/re_psf_f1
PSFCON - Assume constant PSF across field /YES/ >

Energy fraction    Radius

      50%         2.0   pixels
      68%         3.4   pixels
      90%         9.6   pixels
      95%        14.9   pixels

PSFPIX - Radius of PSF box in pixels /3/ > 5
X_CORR axis range is from 79.49998 to -79.49998 arcmin
Y_CORR axis range is from -79.49998 to 79.49998 arcmin
SLICE - Section of dataset to search /'*:*,*:*'/ > '65:-65,-65:65'
SOPT - Statistic option (CASH,GAUSSIAN) /'CASH'/ >
BGND - Background model /@bg_s1/ >
RESCALE - Re-scale background estimate /NO/ >
EXTEN - Fit for extension measure /NO/ >
SAMPLE - Oversampling factor for first pass /1/ >
First pass - grid spacing 1 pixels
MAP - Significance map /!/ > ims_s1
Significance varies from 0 to 41.05927
SIGMIN - Significance threshold /5/ > 4
 Src    X        Y     Signif

   1  63.65    64.87    3.351
   2 -28.63    50.58    4.083
   3 -64.04   -29.38    3.727
   4  -6.38    -1.86    3.416
   5 -18.44     9.43    3.274
   6 -58.56   -56.41    4.523
   7 -47.61   -64.72    3.238
   8  36.94    11.69    3.453
   9  36.49    24.09    3.579
  10 -19.46   -36.52   41.059
  11 -19.55   -48.27    4.024
  12 -43.19    -1.87    3.315
  13 -39.94    -2.41    3.237
Second pass
ASYMMETRIC - Asymmetric source parameter errors /NO/ >
FERL - Flux error confidence level /'1 sigma'/ >
PERL - Positional error confidence levels /'90%'/ >
 Src      RA         DEC         X        Y     Signif    Flux

   1  00 23 55.6  -63 43 12   -58.93   -56.22    4.662    18.954
   2  00 28 44.6  -61 57 16   -28.61    50.50    4.090    17.619
   3  00 29 55.3  -63 24 37   -19.33   -36.73   41.213   449.014
OUT - Source search results file (! for none) /@srclist/ > s1_src
SSUB - Source subtracted dataset /!/ >

% re_expos                                 ! Exposure correct source raw fluxes
 RE_EXPOS version 230595
FILT S1a or S2a <CHAR>=s1
    Using sky exposure map: /rosat/soft/reskyexp_f1.fit
SLIST Source list <CHAR>=s1_src
CLIST Corrected SSDS file = s1_srcc =
    Correcting            3 source locations

% ssdump hms=no                                                  ! List sources
SSDUMP Version 1.7-3
INP - Source dataset /@s1_src/ > s1_srcc
DEV - Output to /'TERMINAL'/ >
-------------------------------------------------------------------------------
-----------------------------------------------
Results file /rosat/soft/scratch/s1_srcc.sdf contains data from one source sear
ch
     N   Searched Dataset At 24-May-95 11:27:57
     1   /rosat/soft/scratch/im_s1.sdf

   Dataset was searched by PSS Version 1.7-5

 Equatorial coordinates, equinox 2000.0, units DEGREES

Positional errors are at 90% confidence, units arcmin
Flux units are count, errors are at 68.26895% confidence
Bgnd units are count/pix
-------------------------------------------------------------------------------
-----------------------------------------------

   X_CORR     Y_CORR        RA           DEC            Raw Flux          ENCPS
F      Bgnd      Perr    Signif    Cor_Flux
1  -58.9288   -56.2228     5.981595   -63.720122    19.0    +-  6.3       0.82
     0.163       1.403     4.662  1.9890E-02
2  -28.6110    50.5005     7.185864   -61.954565    17.6    +-  6.3       0.82
     0.344       0.993     4.090  2.1980E-02
3  -19.3276   -36.7263     7.480349   -63.410301    449.    +-  25.       0.82
     0.352       0.190    41.213  5.4387E-01


% pss mode=uplim                          ! Get an upper limit at some position
PSS Version 1.7-5
INP - Dataset to be searched/tested for sources /@mysrc_2_sp/ >im_s1
EXPERT - Expert mode /TRUE/ >
PSF system options :

  ANAL           ASCA           EXOLE          PWFC           RADIAL
  RESPFILE       TABULAR        WFC            XRT_HRI        XRT_PSPC

  CSPEC(psf/model,spec[,nbin])
  POLAR(psf,rbin[,abin])        RECT(psf,xbin[,ybin])

PSF - Choose PSF to use for source model /'WFC'/ > tabular
MASK - Name of a 2D dataset containing psf >  $re_res/re_psf_f1
PSF read in from file  $re_res/re_psf_f1
PSFCON - Assume constant PSF across field /YES/ >

Energy fraction    Radius

      50%         2.0   pixels
      68%         3.4   pixels
      90%         9.6   pixels
      95%        14.9   pixels

PSFPIX - Radius of PSF box in pixels /3/ > 5
BGND - Background model /@bg_s1/ >
RESCALE - Re-scale background estimate /NO/ >
PLIST - Source of RA and DECs > TERMINAL
Enter RA and DEC at prompts, ! to terminate
RA - Right ascension /!/ > 8.2
DEC - Declination > -62.8
RA - Right ascension /!/ >
1 positions read from TERMINAL
FERL - Upper limit confidence level /'1 sigma'/ > 90
 Src      RA         DEC         X        Y      Flux

   1  00 32 48.0  -62 48 00     0.00     0.00  <   11.938
OUT - Source search results file (! for none) /@srclist/ > uplim_s1


% re_light                                   ! Produce S1a and S2a light curves

   RE_LIGHT -- Version 090595

Please note that this version does not work with
background derived from an annulus!   Background is got
from a circle offset along s/c scan path.  Recommended
radii are 3am for source and 9am for background.
A typical offset value maybe 20am.
Moon times rejected. High background times NOT rejected.

   Give event file (eg mrgd_ev.fit): mrgd_ev
   Give source RA (deg\hms): 7.48
   Give source Dec (deg\hms): -63.41
   Give source name : mysrc
   Give source box radius (arcmin): 3
   Give bgnd offset (arcmin): 20
   Give bgnd box radius (arcmin): 9
   Give time bin width in secs (eg 5760): 5760

The following will be created in the default directory

mysrc_1     - Raw source time series in S1a
mysrc_1_c   - Exp corrected source
mysrc_1_b   - Raw background time series
mysrc_1_bc  - Exp corrected Background
mysrc_1_bcd - Exp corrected Background, area corrected
mysrc_1_s   - Background subtracted source flux
mysrc_1_x   - Exposure profile at the source
mysrc_1_bx  - Exposure profile at background

Files as above but for the S2a filter

   Sorting On source for S1A

 RE_SORT Version 110595
    BASE_MJD is    48198.1993359375
    Events in time-series :          279

 RE_TIMEXP Version 090595
    Tstep set to    2.000000     secs         2880 steps/bin
    Using /rosat/soft/re_hkrres.fit
    Using /rosat/soft/re_ateres.fit
    Calculating the exposure array
    Correcting raw data array

   Sorting Background for S1A

 RE_SORT Version 110595
    BASE_MJD is    48198.1993359375
    Events in time-series :           90

 RE_TIMEXP Version 090595
    Tstep set to    2.000000     secs         2880 steps/bin
    Using /rosat/soft/re_hkrres.fit
    Using /rosat/soft/re_ateres.fit
    Calculating the exposure array
    Correcting raw data array

   Subtracting Background from source

ARITHMETIC Version 1.7-1
WARNING : 191 points ignored due to bad quality
          Data copied from first input object for these points
ARITHMETIC Version 1.7-1
WARNING : 191 points ignored due to bad quality
          Data copied from first input object for these points

   Produce exposure Profile for S1a

    RE_EXPROF Version 090595
    RE_EXPROF Version 090595

   Sorting On source for S2A

 RE_SORT Version 110595
    BASE_MJD is    48198.1993359375
    Events in time-series :          587

 RE_TIMEXP Version 090595
    Tstep set to    2.000000     secs         2880 steps/bin
    Using /rosat/soft/re_hkrres.fit
    Using /rosat/soft/re_ateres.fit
    Calculating the exposure array
    Correcting raw data array

   Sorting Background for S2A

 RE_SORT Version 110595
    BASE_MJD is    48198.1993359375
    Events in time-series :          132

 RE_TIMEXP Version 090595
    Tstep set to    2.000000     secs         2880 steps/bin
    Using /rosat/soft/re_hkrres.fit
    Using /rosat/soft/re_ateres.fit
    Calculating the exposure array
    Correcting raw data array

   Subtracting Background from source

ARITHMETIC Version 1.7-1
WARNING : 193 points ignored due to bad quality
          Data copied from first input object for these points
ARITHMETIC Version 1.7-1
WARNING : 196 points ignored due to bad quality
          Data copied from first input object for these points

   Produce exposure Profile for S2a

    RE_EXPROF Version 090595
    RE_EXPROF Version 090595

% grdraw inp=mysrc_1_s  dev=xw                        ! Look at S1a light curve
GDRAW Version 1.7-5

Dataset: /rosat/soft/scratch/mysrc_1_s

% re_spec                                         ! Produce S1a and S2a spectra

   RE_SPEC -- Version 110595

Please note that this version does not work with
background derived from an annulus!   Background is got
from a circle offset along s/c scan path.  Recommended
radii are 3am for source and 9am for background.
A typical offset value maybe 20am.
Moon times rejected. High background times NOT rejected.


   Give event file (eg mrgd_ev.fit): mrgd_ev
   Give source RA (deg\hms): 7.48
   Give source Dec (deg\hms): -63.41
   Give source name : mysrc
   Give source box radius (arcmin): 3
   Give bgnd offset (arcmin): 20
   Give bgnd box radius (arcmin): 9

The following will be created in the default directory

   mysrc_1_sp     - Background subtracted source spectrum
   mysrc_1_sp.pha - XSPEC (sf format) pha file
   mysrc_1_sp.rsp - XSPEC (sf format) rsp file

Files as above but for the S2A filter

   Sorting On source for S1A

 RE_SORT Version 110595
    BASE_MJD is    48198.1993359375
    Events in time-series :          279

 RE_TIMEXP Version 090595
    Tstep set to    1.999999     secs       611116 steps/bin
    Using /rosat/soft/re_hkrres.fit
    Using /rosat/soft/re_ateres.fit
    Calculating the exposure array
    Correcting raw data array

   Sorting Background for S1A

 RE_SORT Version 110595
    BASE_MJD is    48198.1993359375
    Events in time-series :           90

 RE_TIMEXP Version 090595
    Tstep set to    1.999999     secs       611116 steps/bin
    Using /rosat/soft/re_hkrres.fit
    Using /rosat/soft/re_ateres.fit
    Calculating the exposure array
    Correcting raw data array

   Subtracting Background from source

ARITHMETIC Version 1.7-1
ARITHMETIC Version 1.7-1

   Producing spectrum for S1A

WFCSPEC Version 1.7-0

Filter: Lexan/C/B4C
        Response centred at 124 eV
AST2XSP version 1.6-3

   Sorting On source for S2A

 RE_SORT Version 110595
    BASE_MJD is    48198.1993359375
    Events in time-series :          587

 RE_TIMEXP Version 090595
    Tstep set to    1.999999     secs       611116 steps/bin
    Using /rosat/soft/re_hkrres.fit
    Using /rosat/soft/re_ateres.fit
    Calculating the exposure array
    Correcting raw data array

   Sorting Background for S2A

 RE_SORT Version 110595
    BASE_MJD is    48198.1993359375
    Events in time-series :          132

 RE_TIMEXP Version 090595
    Tstep set to    1.999999     secs       611116 steps/bin
    Using /rosat/soft/re_hkrres.fit
    Using /rosat/soft/re_ateres.fit
    Calculating the exposure array
    Correcting raw data array

   Subtracting Background from source

ARITHMETIC Version 1.7-1
ARITHMETIC Version 1.7-1

   Producing spectrum for S2A

WFCSPEC Version 1.7-0

Filter: Lexan/Be
        Response centred at 90 eV
AST2XSP version 1.6-3

\end{verbatim}
\end{small}

The above shows one way of tackling \ro WFC survey data.  Another
approach, to cut down the number of processing steps, could have been to check
the ROSWFCSI database for the image that surveyed the sky region of interest.
This could then be retrieved instead of recreating it from the event files.
Further, the 2RE catalogue could also have been searched to see if the desired
source had already been detected and if its light-curve existed.

\end{document}
